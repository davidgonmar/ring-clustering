\documentclass[conference]{IEEEtran}
\IEEEoverridecommandlockouts
% The preceding line is only needed to identify funding in the first footnote. If that is unneeded, please comment it out.
\usepackage{cite}
\usepackage{amsmath,amssymb,amsfonts}
\usepackage{algorithmic}
\usepackage{graphicx}
\usepackage{textcomp}
\usepackage{xcolor}
\usepackage{tikz}

\usetikzlibrary{calc}
\def\BibTeX{{\rm B\kern-.05em{\sc i\kern-.025em b}\kern-.08em
    T\kern-.1667em\lower.7ex\hbox{E}\kern-.125emX}}
\begin{document}

\title{Clustering Rings in Noisy Data}

\author{\IEEEauthorblockN{1\textsuperscript{st} David González Martínez}

\IEEEauthorblockA{\textit{University of Seville} \\
Seville, Spain \\
davgonmar2@alum.us.es}
}
\maketitle

\begin{abstract}
In this paper, we apply the Fuzzy K-Rings algorithm, also known as the Fuzzy C-Shells algorithm to ring clustering in noisy data.
We further propose a modification to the algorithm to make it more robust to noise, and conduct different experiments to test the performance of the algorithm.
\end{abstract}

\begin{IEEEkeywords}
Fuzzy K-Rings, Fuzzy C-Shells, Clustering, Noisy Data, Ring Clustering, Hypersphere Clustering
\end{IEEEkeywords}

\section{Introduction}
We are presented with the following problem: we have a dataset that is composed different rings and noise. Our objective is to classify the points in different clusters,
each corresponding to a different ring. The Fuzzy K-Rings algorithm, also known as the Fuzzy C-Shells algorithm, is a clustering algorithm that has been used in the past
for similar tasks. The algorithm is inspired on the Fuzzy C-Means algorithm, and was introduced, altough in different variations, in \cite{308484} and \cite{DAVE1992713}.
From now on, we'll refer to the algorithm as Fuzzy K-Rings (FKR).

\section{Fuzzy Clustering Algorithms}
In hard clustering algorithms, each data point is assigned to a single cluster. In contrast, fuzzy algorithms assign a membership degree to each data point for each cluster.
One of the most popular ones is the Fuzzy C-Means algorithm. A formal description can be found in \cite{bookpatternrecognition} and \cite{BEZDEK1984191}. It can be seen as an optimization problem,
where the objective function is to minimize the following equation:
\begin{equation}
J(U, V) = \sum_{i=1}^{n} \sum_{j=1}^{k} (u_{ij})^q d_{ij}^2
\end{equation}
where $k$ is the number of clusters, $n$ is the number of data samples, $u_{ij}$ is the membership degree of cluster $i$ to data sample $j$,
$d_{ij}$ is the eucledian distance between data point $i$ and the center of cluster $j$, and $q$ is a parameter that controls the fuzziness of the membership degrees.
That is, higher values of $q$ will make the membership degrees more 'fuzzy', and lower values will make them harder, that is, closer to regular K-Means.
$u$ is a matrix of size $n \times k$, and can be interpreted as 'how much data point $i$ belongs to cluster $j$'.
It is important to note that the following conditions must be met, as described in \cite{BEZDEK1984191}:
\begin{enumerate}
    \item $u_{ij} \in [0, 1]$
    \item $\sum_{j=1}^{k} u_{ij} = 1$
\end{enumerate}
The higher the value of $q$, the more 'fuzzy' the algorithm will be. In the limit, when $q \rightarrow 1$, the algorithm will be equivalent to K-Means.


\section{The Fuzzy K-Rings Algorithm}
The Fuzzy K-Rings algorithm is a clustering algorithm that is able to cluster data points in a ring-shaped dataset.
The algorithm is inspired on the K-Means algorithm, and described in \cite{DAVE1992713} and \cite{308484}, altough in different variations.
It is described as an optimization problem, where the objective function is to minimize the following equation:
\begin{equation}\label{eq:objective}
J_q(U, V) = \sum_{i=1}^{n} \sum_{j=1}^{k} u_{ij}^q (d_{ij} - r_i)^2
\end{equation}
where $k$ is the number of clusters, $n$ is the number of data samples, $u_{ij}$ is the membership degree of cluster $i$ to data sample $j$, $d_{ij}$ is the eucledian distance between data point $i$ and the center of cluster $j$, $r_i$ is the radius of the cluster $i$, and $q$ is a parameter that controls the fuzziness of the membership degrees.
That is, higher values of $q$ will make the membership degrees more fuzzy, and lower values will make them harder.
From now on, we'll refer to $$(d_{ij} - r_i)^2$$ as $d'_ij$.
Now, we'll describe the ways to update the different parameters in the algorithm, and then we'll describe the initialization and convergence criteria, as well as the concrete steps.

\subsection{Updating the Membership Degrees}
The membership degrees are updated using the following equation, as described in both \cite{DAVE1992713} and \cite{308484}. It's the same as the one used in the Fuzzy C-Means algorithm, but with $d_{ij}$ replaced by $d'_{ij}$.
\begin{equation}
u_{ij} = \frac{d'^2(X_j, V_i)^{\frac{-1}{q-1}}}{\sum_{k=1}^{K} d'^2(X_j, V_k)^{\frac{-1}{q-1}}}
\end{equation}

\subsection{Updating the Cluster Radii and Centers}
As mentioned, we can define the algorith as an optimization problem after fixing $U$. We can then obtain the optimal (minimum) values for the objective function
by setting the partial derivatives with respect to $r_i$ and $V_i$ to zero.
First, we have:
\begin{equation}\label{eq:d_dr}
\frac{\partial}{\partial r_i}(J_q) = \sum_{j=1}^{n} u_{ij}^q\frac{\partial}{\partial r_i} (d_{ij} - r_i)^2 = \sum_{j=1}^{n} u_{ij}^q (r_i - d_{ij}) = 0
\end{equation}

Here, we take a different approach to the one taken in \cite{308484}, and similar to the one in \cite{DAVE1992713}. It allows vectorized computations, and
automatic extension to higher dimensions, unlike \cite{308484}.

\begin{tikzpicture}\label{fig:circle}
    % Define radius
    \def\radius{3cm}
    \def\angle{45}

    % Draw the circle
    \draw (0,0) circle (\radius);

    % Define points
    \coordinate (V_i) at (0,0); % Center of the circle
    \coordinate (A) at ({\radius*cos(\angle)},0); % Point on the circle at the specified angle
    \coordinate (V'_i) at (\angle:\radius);
    \coordinate (X_j) at (3.5, 3.5);

    % Draw the triangle
    \draw (V_i) -- (A) -- (V'_i) -- cycle;
    \draw[dotted] (V'_i) -- (X_j);

    % Draw points
    \fill (V_i) circle (2pt);
    \fill (V'_i) circle (2pt);
    \fill (X_j) circle (2pt);

    \node[below] at (V_i) {$V_i$};
    \node[above right] at (V'_i) {$V'_i$};
    \node[above right] at (X_j) {$X_j$};

    % label on hypotenuse (line that connects V_i and V'_i). Not on the point, on the line
    \node[above=2pt] at ($(V_i)!0.5!(V'_i)$) {$r_i$};
    % explanation
    \node[below] at (0,-\radius-0.5) {
        \begin{tabular}{c}
            $V'_i = \frac{r_i}{d_{ij}}X_j + (1 - \frac{r_i}{d_{ij}})V_i$ \\
        \end{tabular}
    };
\end{tikzpicture}
\begin{center}
\textbf{Figure \ref{fig:circle}:} Illustration of the update of the cluster centers equation when $X_j$ is outside the circle.
\end{center}


Let $X_j$ be a data point, and $V_i$ be the center of cluster $i$. Let $d_{ij}$ be the eucledian distance between $X_j$ and $V_i$,
and $r_i$ be the radius of cluster $i$.
Let $d'_{ij}$ be the distance between $X_j$ and the circle with center $V_i$ and radius $r_i$.

Then, let the following be true:
\begin{equation}
V'_i = \frac{r_i}{d_{ij}}X_j + (1 - \frac{r_i}{d_{ij}})V_i
\end{equation}

Differentiating \eqref{eq:objective} with respect to $V_i$ and setting it to zero, we get:
\begin{equation}\label{eq:d_dV}
\frac{\partial}{\partial V_i}(J_q) = \sum_{j=1}^{n} u_{ij}^q\frac{\partial}{\partial V_i} (d'_{ij})^2 = 0
\end{equation}
Note that we can rewrite $(d'_{ij})^2$ as $(X_j - V'_i)^T(X_j - V'_i)$.
Then, we can solve:
\begin{equation}
\begin{aligned}
\frac{\partial}{\partial V_i} (d_{ij} - r_i)^2 &= \left(\frac{\partial}{\partial V_i} (|X_j - V_i| - r_i)^2\right) \\
&= -2 \left( (X_j - V_i) - \frac{r_i}{|X_j - V_i|} (X_j - V_i) \right) \\
&= -2 \left( (X_j - V_i) - \frac{r_i}{d_{ij}} (X_j - V_i) \right) \\
&= -2 \left( (1 - \frac{r_i}{d_{ij}})X_J - (1 - \frac{r_i}{d_{ij}})V_i \right)
\end{aligned}
\end{equation}
Let's define $(1 - \frac{r_i}{d_{ij}})$ as $\alpha_{ij}$.
Then, we can rewrite the equation as:
\begin{equation}
\frac{\partial}{\partial V_i} (d_{ij} - r_i)^2 = -2 \alpha_{ij} (X_j - V_i)
\end{equation}
Plugging that into \eqref{eq:d_dV}, we get:
\begin{equation}
\sum_{j=1}^{n} u_{ij}^q (d_{ij} - r_i) (X_j - V_i) = 0
\end{equation}
We now have a system of equations that we can solve for $V_i$ and $r_i$.
The solution, as noted in \cite{DAVE1992713}, is:
\begin{equation}
V_i = \frac{\sum_{j=1}^{n} u_{ij}^q X_j}{\sum_{j=1}^{n} u_{ij}^q}
\end{equation}
\begin{equation}\label{eq:r_i}
r_i = \frac{\sum_{j=1}^{n} u_{ij}^q d_{ij}}{\sum_{j=1}^{n} u_{ij}^q}
\end{equation}

\subsection{Intuition}
After having given a formal description of the algorithm, we can give an intuitive explanation of the algorithm.

First, for the membership degrees, we can see that the algorithm is trying to assign higher membership degrees to points that are closer to the ring contour.
This is done by averaging the distances of the points to the different rings, and assigning them proportionally.

As for the cluster centers, the algorithm is just using a weighted average of the points, with the weights being the membership degrees.
This is similar to the K-Means algorithm, but with the weights being the membership degrees instead of a binary value, and exactly the same as the Fuzzy C-Means algorithm.

Finally, for the radii, the algorithm is trying to assign the radii by using a weighted average of the distances of the points to the cluster centers.
This is done by using the membership degrees as weights, and the distances as the values to be averaged.


\subsection{Initializing the Parameters}
\cite{308484} proposes two initialization methods. Here, we use the second one, which is running some iterations of
the Fuzzy C-Means algorithm, and then using the membership matrix and the cluster centers to initialize the Fuzzy K-Rings algorithm.
The radii are initialized with the equation \eqref{eq:r_i}.

\subsection{Convergence Criterion}
We use the following convergence criterion:
\begin{equation}
|\hat{u_{ij}} - u_{ij}| < \epsilon \quad \forall i, j
\end{equation}
Where $\hat{u_{ij}}$ is the membership degree of the previous iteration, and $u_{ij}$ is the membership degree of the current iteration,
and $\epsilon$ is a small value, usually $10^{-3}$, given as a hyperparameter.

\subsection{Overall Steps}
The overall steps of the algorithm are as follows:
\begin{enumerate}
    \item Initialize the parameters $U$, $V$, and $R$.
    \item While the convergence criterion is not met:
    \begin{enumerate}
        \item Update the membership degrees $U$.
        \item Update the cluster radii $R$ and centers $V$ using the equations \eqref{eq:d_dr} and \eqref{eq:r_i}.
    \end{enumerate}
    \item Return the membership degrees $U$, the cluster centers $V$, and the cluster radii $R$.
    \item End.
\end{enumerate}

\subsection{Implementation}
All of the above steps can be formulated as Tensor/NDArray operations, with no need for explicit loops,
by defining the different operations as element-wise operations, tensor multiplications and exploiting
expansion and broadcasting rules as needed. Code is shown at \cite{b1}.
This allows to exploit the parallelism of modern computing frameworks, like NumPy, Eigen, or PyTorch.
The algorithm can be implemented with the following pseudocode:
\begin{verbatim}
fn fkr(X, k, q, epsilon):
    U, V, R = initialize(X, k)
    while True:
        U = update_membership(X, V, q)
        R = update_radii(X, U, q)
        V = update_centers(X, U, R, q)
        if converged(U, U_old, epsilon):
            break
    return U, V, R
\end{verbatim}
\begin{verbatim}
fn update_membership(X, V, q):
    D = distance_matrix(X, V)
    U = D ** (-1 / (q - 1))
    U = U / U.sum(axis=1)
    return u
\end{verbatim}

\section{Prepare Your Paper Before Styling}
Before you begin to format your paper, first write and save the content as a 
separate text file. Complete all content and organizational editing before 
formatting. Please note sections \ref{AA}--\ref{SCM} below for more information on 
proofreading, spelling and grammar.

Keep your text and graphic files separate until after the text has been 
formatted and styled. Do not number text heads---{\LaTeX} will do that 
for you.

\subsection{Abbreviations and Acronyms}\label{AA}
Define abbreviations and acronyms the first time they are used in the text, 
even after they have been defined in the abstract. Abbreviations such as 
IEEE, SI, MKS, CGS, ac, dc, and rms do not have to be defined. Do not use 
abbreviations in the title or heads unless they are unavoidable.

\subsection{Units}
\begin{itemize}
\item Use either SI (MKS) or CGS as primary units. (SI units are encouraged.) English units may be used as secondary units (in parentheses). An exception would be the use of English units as identifiers in trade, such as ``3.5-inch disk drive''.
\item Avoid combining SI and CGS units, such as current in amperes and magnetic field in oersteds. This often leads to confusion because equations do not balance dimensionally. If you must use mixed units, clearly state the units for each quantity that you use in an equation.
\item Do not mix complete spellings and abbreviations of units: ``Wb/m\textsuperscript{2}'' or ``webers per square meter'', not ``webers/m\textsuperscript{2}''. Spell out units when they appear in text: ``. . . a few henries'', not ``. . . a few H''.
\item Use a zero before decimal points: ``0.25'', not ``.25''. Use ``cm\textsuperscript{3}'', not ``cc''.)
\end{itemize}

\subsection{Equations}
Number equations consecutively. To make your 
equations more compact, you may use the solidus (~/~), the exp function, or 
appropriate exponents. Italicize Roman symbols for quantities and variables, 
but not Greek symbols. Use a long dash rather than a hyphen for a minus 
sign. Punctuate equations with commas or periods when they are part of a 
sentence, as in:
\begin{equation}
a+b=\gamma\label{eq}
\end{equation}

Be sure that the 
symbols in your equation have been defined before or immediately following 
the equation. Use ``\eqref{eq}'', not ``Eq.~\eqref{eq}'' or ``equation \eqref{eq}'', except at 
the beginning of a sentence: ``Equation \eqref{eq} is . . .''

\subsection{\LaTeX-Specific Advice}

Please use ``soft'' (e.g., \verb|\eqref{Eq}|) cross references instead
of ``hard'' references (e.g., \verb|(1)|). That will make it possible
to combine sections, add equations, or change the order of figures or
citations without having to go through the file line by line.

Please don't use the \verb|{eqnarray}| equation environment. Use
\verb|{align}| or \verb|{IEEEeqnarray}| instead. The \verb|{eqnarray}|
environment leaves unsightly spaces around relation symbols.

Please note that the \verb|{subequations}| environment in {\LaTeX}
will increment the main equation counter even when there are no
equation numbers displayed. If you forget that, you might write an
article in which the equation numbers skip from (17) to (20), causing
the copy editors to wonder if you've discovered a new method of
counting.

{\BibTeX} does not work by magic. It doesn't get the bibliographic
data from thin air but from .bib files. If you use {\BibTeX} to produce a
bibliography you must send the .bib files. 

{\LaTeX} can't read your mind. If you assign the same label to a
subsubsection and a table, you might find that Table I has been cross
referenced as Table IV-B3. 

{\LaTeX} does not have precognitive abilities. If you put a
\verb|\label| command before the command that updates the counter it's
supposed to be using, the label will pick up the last counter to be
cross referenced instead. In particular, a \verb|\label| command
should not go before the caption of a figure or a table.

Do not use \verb|\nonumber| inside the \verb|{array}| environment. It
will not stop equation numbers inside \verb|{array}| (there won't be
any anyway) and it might stop a wanted equation number in the
surrounding equation.

\subsection{Some Common Mistakes}\label{SCM}
\begin{itemize}
\item The word ``data'' is plural, not singular.
\item The subscript for the permeability of vacuum $\mu_{0}$, and other common scientific constants, is zero with subscript formatting, not a lowercase letter ``o''.
\item In American English, commas, semicolons, periods, question and exclamation marks are located within quotation marks only when a complete thought or name is cited, such as a title or full quotation. When quotation marks are used, instead of a bold or italic typeface, to highlight a word or phrase, punctuation should appear outside of the quotation marks. A parenthetical phrase or statement at the end of a sentence is punctuated outside of the closing parenthesis (like this). (A parenthetical sentence is punctuated within the parentheses.)
\item A graph within a graph is an ``inset'', not an ``insert''. The word alternatively is preferred to the word ``alternately'' (unless you really mean something that alternates).
\item Do not use the word ``essentially'' to mean ``approximately'' or ``effectively''.
\item In your paper title, if the words ``that uses'' can accurately replace the word ``using'', capitalize the ``u''; if not, keep using lower-cased.
\item Be aware of the different meanings of the homophones ``affect'' and ``effect'', ``complement'' and ``compliment'', ``discreet'' and ``discrete'', ``principal'' and ``principle''.
\item Do not confuse ``imply'' and ``infer''.
\item The prefix ``non'' is not a word; it should be joined to the word it modifies, usually without a hyphen.
\item There is no period after the ``et'' in the Latin abbreviation ``et al.''.
\item The abbreviation ``i.e.'' means ``that is'', and the abbreviation ``e.g.'' means ``for example''.
\end{itemize}
An excellent style manual for science writers is \cite{b7}.

\subsection{Authors and Affiliations}
\textbf{The class file is designed for, but not limited to, six authors.} A 
minimum of one author is required for all conference articles. Author names 
should be listed starting from left to right and then moving down to the 
next line. This is the author sequence that will be used in future citations 
and by indexing services. Names should not be listed in columns nor group by 
affiliation. Please keep your affiliations as succinct as possible (for 
example, do not differentiate among departments of the same organization).

\subsection{Identify the Headings}
Headings, or heads, are organizational devices that guide the reader through 
your paper. There are two types: component heads and text heads.

Component heads identify the different components of your paper and are not 
topically subordinate to each other. Examples include Acknowledgments and 
References and, for these, the correct style to use is ``Heading 5''. Use 
``figure caption'' for your Figure captions, and ``table head'' for your 
table title. Run-in heads, such as ``Abstract'', will require you to apply a 
style (in this case, italic) in addition to the style provided by the drop 
down menu to differentiate the head from the text.

Text heads organize the topics on a relational, hierarchical basis. For 
example, the paper title is the primary text head because all subsequent 
material relates and elaborates on this one topic. If there are two or more 
sub-topics, the next level head (uppercase Roman numerals) should be used 
and, conversely, if there are not at least two sub-topics, then no subheads 
should be introduced.

\subsection{Figures and Tables}
\paragraph{Positioning Figures and Tables} Place figures and tables at the top and 
bottom of columns. Avoid placing them in the middle of columns. Large 
figures and tables may span across both columns. Figure captions should be 
below the figures; table heads should appear above the tables. Insert 
figures and tables after they are cited in the text. Use the abbreviation 
``Fig.~\ref{fig}'', even at the beginning of a sentence.

\begin{table}[htbp]
\caption{Table Type Styles}
\begin{center}
\begin{tabular}{|c|c|c|c|}
\hline
\textbf{Table}&\multicolumn{3}{|c|}{\textbf{Table Column Head}} \\
\cline{2-4} 
\textbf{Head} & \textbf{\textit{Table column subhead}}& \textbf{\textit{Subhead}}& \textbf{\textit{Subhead}} \\
\hline
copy& More table copy$^{\mathrm{a}}$& &  \\
\hline
\multicolumn{4}{l}{$^{\mathrm{a}}$Sample of a Table footnote.}
\end{tabular}
\label{tab1}
\end{center}
\end{table}

\begin{figure}[htbp]
\caption{Example of a figure caption.}
\label{fig}
\end{figure}

Figure Labels: Use 8 point Times New Roman for Figure labels. Use words 
rather than symbols or abbreviations when writing Figure axis labels to 
avoid confusing the reader. As an example, write the quantity 
``Magnetization'', or ``Magnetization, M'', not just ``M''. If including 
units in the label, present them within parentheses. Do not label axes only 
with units. In the example, write ``Magnetization (A/m)'' or ``Magnetization 
\{A[m(1)]\}'', not just ``A/m''. Do not label axes with a ratio of 
quantities and units. For example, write ``Temperature (K)'', not 
``Temperature/K''.

\section*{Acknowledgment}

The preferred spelling of the word ``acknowledgment'' in America is without 
an ``e'' after the ``g''. Avoid the stilted expression ``one of us (R. B. 
G.) thanks $\ldots$''. Instead, try ``R. B. G. thanks$\ldots$''. Put sponsor 
acknowledgments in the unnumbered footnote on the first page.

\bibliographystyle{IEEEtran}
\bibliography{references.bib}

\vspace{12pt}
\color{red}
\end{document}
