\documentclass[conference]{IEEEtran}
\IEEEoverridecommandlockouts
% The preceding line is only needed to identify funding in the first footnote. If that is unneeded, please comment it out.
\usepackage{cite}
\usepackage{amsmath,amssymb,amsfonts}
\usepackage{algorithmic}
\usepackage{graphicx}
\usepackage{textcomp}
\usepackage{xcolor}
\usepackage{tikz}
\usepackage{pgfplots}
\usepackage{float}
\usepackage{stfloats} 


\usetikzlibrary{calc}
\def\BibTeX{{\rm B\kern-.05em{\sc i\kern-.025em b}\kern-.08em
    T\kern-.1667em\lower.7ex\hbox{E}\kern-.125emX}}
\begin{document}

\title{Clustering Rings in Noisy Data}

\author{\IEEEauthorblockN{1\textsuperscript{st} David González Martínez}

\IEEEauthorblockA{\textit{University of Seville} \\
Seville, Spain \\
davgonmar2@alum.us.es}
}
\maketitle

\begin{abstract}
In this paper, we apply the Fuzzy K-Rings algorithm, also known as the Fuzzy C-Shells algorithm to ring clustering in noisy data.
We further propose a modification to the algorithm to make it more robust to noise, and conduct different experiments to test the performance of the algorithm.
\end{abstract}

\begin{IEEEkeywords}
Fuzzy K-Rings, Fuzzy C-Shells, Clustering, Noisy Data, Ring Clustering, Hypersphere Clustering
\end{IEEEkeywords}

\section{Introduction}
We are presented with the following problem: we have a dataset that is composed different rings and noise. Our objective is to classify the points in different clusters,
each corresponding to a different ring. The Fuzzy K-Rings algorithm, also known as the Fuzzy C-Shells algorithm, is a clustering algorithm that has been used in the past
for similar tasks. The algorithm is inspired on the Fuzzy C-Means algorithm, and was introduced, altough in different variations, in \cite{308484} and \cite{DAVE1992713}.
From now on, we'll refer to the algorithm as Fuzzy K-Rings (FKR).
% here a plot from plots/noisy_rings.pgf
\begin{figure}[H]
    \centering
    \resizebox{0.9\linewidth}{!}{%% Creator: Matplotlib, PGF backend
%%
%% To include the figure in your LaTeX document, write
%%   \input{<filename>.pgf}
%%
%% Make sure the required packages are loaded in your preamble
%%   \usepackage{pgf}
%%
%% Also ensure that all the required font packages are loaded; for instance,
%% the lmodern package is sometimes necessary when using math font.
%%   \usepackage{lmodern}
%%
%% Figures using additional raster images can only be included by \input if
%% they are in the same directory as the main LaTeX file. For loading figures
%% from other directories you can use the `import` package
%%   \usepackage{import}
%%
%% and then include the figures with
%%   \import{<path to file>}{<filename>.pgf}
%%
%% Matplotlib used the following preamble
%%   \def\mathdefault#1{#1}
%%   \everymath=\expandafter{\the\everymath\displaystyle}
%%   
%%   \usepackage{fontspec}
%%   \setmainfont{DejaVuSerif.ttf}[Path=\detokenize{C:/Users/dagom/anaconda3/envs/pytorch/lib/site-packages/matplotlib/mpl-data/fonts/ttf/}]
%%   \setsansfont{DejaVuSans.ttf}[Path=\detokenize{C:/Users/dagom/anaconda3/envs/pytorch/lib/site-packages/matplotlib/mpl-data/fonts/ttf/}]
%%   \setmonofont{DejaVuSansMono.ttf}[Path=\detokenize{C:/Users/dagom/anaconda3/envs/pytorch/lib/site-packages/matplotlib/mpl-data/fonts/ttf/}]
%%   \makeatletter\@ifpackageloaded{underscore}{}{\usepackage[strings]{underscore}}\makeatother
%%
\begingroup%
\makeatletter%
\begin{pgfpicture}%
\pgfpathrectangle{\pgfpointorigin}{\pgfqpoint{8.000000in}{8.000000in}}%
\pgfusepath{use as bounding box, clip}%
\begin{pgfscope}%
\pgfsetbuttcap%
\pgfsetmiterjoin%
\definecolor{currentfill}{rgb}{1.000000,1.000000,1.000000}%
\pgfsetfillcolor{currentfill}%
\pgfsetlinewidth{0.000000pt}%
\definecolor{currentstroke}{rgb}{1.000000,1.000000,1.000000}%
\pgfsetstrokecolor{currentstroke}%
\pgfsetdash{}{0pt}%
\pgfpathmoveto{\pgfqpoint{0.000000in}{0.000000in}}%
\pgfpathlineto{\pgfqpoint{8.000000in}{0.000000in}}%
\pgfpathlineto{\pgfqpoint{8.000000in}{8.000000in}}%
\pgfpathlineto{\pgfqpoint{0.000000in}{8.000000in}}%
\pgfpathlineto{\pgfqpoint{0.000000in}{0.000000in}}%
\pgfpathclose%
\pgfusepath{fill}%
\end{pgfscope}%
\begin{pgfscope}%
\pgfsetbuttcap%
\pgfsetmiterjoin%
\definecolor{currentfill}{rgb}{1.000000,1.000000,1.000000}%
\pgfsetfillcolor{currentfill}%
\pgfsetlinewidth{0.000000pt}%
\definecolor{currentstroke}{rgb}{0.000000,0.000000,0.000000}%
\pgfsetstrokecolor{currentstroke}%
\pgfsetstrokeopacity{0.000000}%
\pgfsetdash{}{0pt}%
\pgfpathmoveto{\pgfqpoint{1.000000in}{0.979904in}}%
\pgfpathlineto{\pgfqpoint{7.200000in}{0.979904in}}%
\pgfpathlineto{\pgfqpoint{7.200000in}{6.940096in}}%
\pgfpathlineto{\pgfqpoint{1.000000in}{6.940096in}}%
\pgfpathlineto{\pgfqpoint{1.000000in}{0.979904in}}%
\pgfpathclose%
\pgfusepath{fill}%
\end{pgfscope}%
\begin{pgfscope}%
\pgfpathrectangle{\pgfqpoint{1.000000in}{0.979904in}}{\pgfqpoint{6.200000in}{5.960192in}}%
\pgfusepath{clip}%
\pgfsetbuttcap%
\pgfsetroundjoin%
\definecolor{currentfill}{rgb}{0.200000,0.200000,0.800000}%
\pgfsetfillcolor{currentfill}%
\pgfsetlinewidth{1.003750pt}%
\definecolor{currentstroke}{rgb}{0.200000,0.200000,0.800000}%
\pgfsetstrokecolor{currentstroke}%
\pgfsetdash{}{0pt}%
\pgfpathmoveto{\pgfqpoint{4.656257in}{5.676814in}}%
\pgfpathcurveto{\pgfqpoint{4.662081in}{5.676814in}}{\pgfqpoint{4.667667in}{5.679128in}}{\pgfqpoint{4.671785in}{5.683246in}}%
\pgfpathcurveto{\pgfqpoint{4.675903in}{5.687364in}}{\pgfqpoint{4.678217in}{5.692950in}}{\pgfqpoint{4.678217in}{5.698774in}}%
\pgfpathcurveto{\pgfqpoint{4.678217in}{5.704598in}}{\pgfqpoint{4.675903in}{5.710184in}}{\pgfqpoint{4.671785in}{5.714303in}}%
\pgfpathcurveto{\pgfqpoint{4.667667in}{5.718421in}}{\pgfqpoint{4.662081in}{5.720735in}}{\pgfqpoint{4.656257in}{5.720735in}}%
\pgfpathcurveto{\pgfqpoint{4.650433in}{5.720735in}}{\pgfqpoint{4.644847in}{5.718421in}}{\pgfqpoint{4.640729in}{5.714303in}}%
\pgfpathcurveto{\pgfqpoint{4.636610in}{5.710184in}}{\pgfqpoint{4.634297in}{5.704598in}}{\pgfqpoint{4.634297in}{5.698774in}}%
\pgfpathcurveto{\pgfqpoint{4.634297in}{5.692950in}}{\pgfqpoint{4.636610in}{5.687364in}}{\pgfqpoint{4.640729in}{5.683246in}}%
\pgfpathcurveto{\pgfqpoint{4.644847in}{5.679128in}}{\pgfqpoint{4.650433in}{5.676814in}}{\pgfqpoint{4.656257in}{5.676814in}}%
\pgfpathlineto{\pgfqpoint{4.656257in}{5.676814in}}%
\pgfpathclose%
\pgfusepath{stroke,fill}%
\end{pgfscope}%
\begin{pgfscope}%
\pgfpathrectangle{\pgfqpoint{1.000000in}{0.979904in}}{\pgfqpoint{6.200000in}{5.960192in}}%
\pgfusepath{clip}%
\pgfsetbuttcap%
\pgfsetroundjoin%
\definecolor{currentfill}{rgb}{0.200000,0.200000,0.800000}%
\pgfsetfillcolor{currentfill}%
\pgfsetlinewidth{1.003750pt}%
\definecolor{currentstroke}{rgb}{0.200000,0.200000,0.800000}%
\pgfsetstrokecolor{currentstroke}%
\pgfsetdash{}{0pt}%
\pgfpathmoveto{\pgfqpoint{4.693075in}{5.739556in}}%
\pgfpathcurveto{\pgfqpoint{4.698899in}{5.739556in}}{\pgfqpoint{4.704485in}{5.741870in}}{\pgfqpoint{4.708603in}{5.745988in}}%
\pgfpathcurveto{\pgfqpoint{4.712721in}{5.750107in}}{\pgfqpoint{4.715035in}{5.755693in}}{\pgfqpoint{4.715035in}{5.761517in}}%
\pgfpathcurveto{\pgfqpoint{4.715035in}{5.767341in}}{\pgfqpoint{4.712721in}{5.772927in}}{\pgfqpoint{4.708603in}{5.777045in}}%
\pgfpathcurveto{\pgfqpoint{4.704485in}{5.781163in}}{\pgfqpoint{4.698899in}{5.783477in}}{\pgfqpoint{4.693075in}{5.783477in}}%
\pgfpathcurveto{\pgfqpoint{4.687251in}{5.783477in}}{\pgfqpoint{4.681665in}{5.781163in}}{\pgfqpoint{4.677546in}{5.777045in}}%
\pgfpathcurveto{\pgfqpoint{4.673428in}{5.772927in}}{\pgfqpoint{4.671114in}{5.767341in}}{\pgfqpoint{4.671114in}{5.761517in}}%
\pgfpathcurveto{\pgfqpoint{4.671114in}{5.755693in}}{\pgfqpoint{4.673428in}{5.750107in}}{\pgfqpoint{4.677546in}{5.745988in}}%
\pgfpathcurveto{\pgfqpoint{4.681665in}{5.741870in}}{\pgfqpoint{4.687251in}{5.739556in}}{\pgfqpoint{4.693075in}{5.739556in}}%
\pgfpathlineto{\pgfqpoint{4.693075in}{5.739556in}}%
\pgfpathclose%
\pgfusepath{stroke,fill}%
\end{pgfscope}%
\begin{pgfscope}%
\pgfpathrectangle{\pgfqpoint{1.000000in}{0.979904in}}{\pgfqpoint{6.200000in}{5.960192in}}%
\pgfusepath{clip}%
\pgfsetbuttcap%
\pgfsetroundjoin%
\definecolor{currentfill}{rgb}{0.200000,0.200000,0.800000}%
\pgfsetfillcolor{currentfill}%
\pgfsetlinewidth{1.003750pt}%
\definecolor{currentstroke}{rgb}{0.200000,0.200000,0.800000}%
\pgfsetstrokecolor{currentstroke}%
\pgfsetdash{}{0pt}%
\pgfpathmoveto{\pgfqpoint{4.603353in}{5.791357in}}%
\pgfpathcurveto{\pgfqpoint{4.609177in}{5.791357in}}{\pgfqpoint{4.614763in}{5.793671in}}{\pgfqpoint{4.618881in}{5.797789in}}%
\pgfpathcurveto{\pgfqpoint{4.622999in}{5.801908in}}{\pgfqpoint{4.625313in}{5.807494in}}{\pgfqpoint{4.625313in}{5.813318in}}%
\pgfpathcurveto{\pgfqpoint{4.625313in}{5.819142in}}{\pgfqpoint{4.622999in}{5.824728in}}{\pgfqpoint{4.618881in}{5.828846in}}%
\pgfpathcurveto{\pgfqpoint{4.614763in}{5.832964in}}{\pgfqpoint{4.609177in}{5.835278in}}{\pgfqpoint{4.603353in}{5.835278in}}%
\pgfpathcurveto{\pgfqpoint{4.597529in}{5.835278in}}{\pgfqpoint{4.591943in}{5.832964in}}{\pgfqpoint{4.587825in}{5.828846in}}%
\pgfpathcurveto{\pgfqpoint{4.583707in}{5.824728in}}{\pgfqpoint{4.581393in}{5.819142in}}{\pgfqpoint{4.581393in}{5.813318in}}%
\pgfpathcurveto{\pgfqpoint{4.581393in}{5.807494in}}{\pgfqpoint{4.583707in}{5.801908in}}{\pgfqpoint{4.587825in}{5.797789in}}%
\pgfpathcurveto{\pgfqpoint{4.591943in}{5.793671in}}{\pgfqpoint{4.597529in}{5.791357in}}{\pgfqpoint{4.603353in}{5.791357in}}%
\pgfpathlineto{\pgfqpoint{4.603353in}{5.791357in}}%
\pgfpathclose%
\pgfusepath{stroke,fill}%
\end{pgfscope}%
\begin{pgfscope}%
\pgfpathrectangle{\pgfqpoint{1.000000in}{0.979904in}}{\pgfqpoint{6.200000in}{5.960192in}}%
\pgfusepath{clip}%
\pgfsetbuttcap%
\pgfsetroundjoin%
\definecolor{currentfill}{rgb}{0.200000,0.200000,0.800000}%
\pgfsetfillcolor{currentfill}%
\pgfsetlinewidth{1.003750pt}%
\definecolor{currentstroke}{rgb}{0.200000,0.200000,0.800000}%
\pgfsetstrokecolor{currentstroke}%
\pgfsetdash{}{0pt}%
\pgfpathmoveto{\pgfqpoint{4.597183in}{5.848612in}}%
\pgfpathcurveto{\pgfqpoint{4.603007in}{5.848612in}}{\pgfqpoint{4.608594in}{5.850926in}}{\pgfqpoint{4.612712in}{5.855044in}}%
\pgfpathcurveto{\pgfqpoint{4.616830in}{5.859162in}}{\pgfqpoint{4.619144in}{5.864748in}}{\pgfqpoint{4.619144in}{5.870572in}}%
\pgfpathcurveto{\pgfqpoint{4.619144in}{5.876396in}}{\pgfqpoint{4.616830in}{5.881982in}}{\pgfqpoint{4.612712in}{5.886100in}}%
\pgfpathcurveto{\pgfqpoint{4.608594in}{5.890218in}}{\pgfqpoint{4.603007in}{5.892532in}}{\pgfqpoint{4.597183in}{5.892532in}}%
\pgfpathcurveto{\pgfqpoint{4.591359in}{5.892532in}}{\pgfqpoint{4.585773in}{5.890218in}}{\pgfqpoint{4.581655in}{5.886100in}}%
\pgfpathcurveto{\pgfqpoint{4.577537in}{5.881982in}}{\pgfqpoint{4.575223in}{5.876396in}}{\pgfqpoint{4.575223in}{5.870572in}}%
\pgfpathcurveto{\pgfqpoint{4.575223in}{5.864748in}}{\pgfqpoint{4.577537in}{5.859162in}}{\pgfqpoint{4.581655in}{5.855044in}}%
\pgfpathcurveto{\pgfqpoint{4.585773in}{5.850926in}}{\pgfqpoint{4.591359in}{5.848612in}}{\pgfqpoint{4.597183in}{5.848612in}}%
\pgfpathlineto{\pgfqpoint{4.597183in}{5.848612in}}%
\pgfpathclose%
\pgfusepath{stroke,fill}%
\end{pgfscope}%
\begin{pgfscope}%
\pgfpathrectangle{\pgfqpoint{1.000000in}{0.979904in}}{\pgfqpoint{6.200000in}{5.960192in}}%
\pgfusepath{clip}%
\pgfsetbuttcap%
\pgfsetroundjoin%
\definecolor{currentfill}{rgb}{0.200000,0.200000,0.800000}%
\pgfsetfillcolor{currentfill}%
\pgfsetlinewidth{1.003750pt}%
\definecolor{currentstroke}{rgb}{0.200000,0.200000,0.800000}%
\pgfsetstrokecolor{currentstroke}%
\pgfsetdash{}{0pt}%
\pgfpathmoveto{\pgfqpoint{4.582120in}{5.904184in}}%
\pgfpathcurveto{\pgfqpoint{4.587943in}{5.904184in}}{\pgfqpoint{4.593530in}{5.906498in}}{\pgfqpoint{4.597648in}{5.910616in}}%
\pgfpathcurveto{\pgfqpoint{4.601766in}{5.914735in}}{\pgfqpoint{4.604080in}{5.920321in}}{\pgfqpoint{4.604080in}{5.926145in}}%
\pgfpathcurveto{\pgfqpoint{4.604080in}{5.931969in}}{\pgfqpoint{4.601766in}{5.937555in}}{\pgfqpoint{4.597648in}{5.941673in}}%
\pgfpathcurveto{\pgfqpoint{4.593530in}{5.945791in}}{\pgfqpoint{4.587943in}{5.948105in}}{\pgfqpoint{4.582120in}{5.948105in}}%
\pgfpathcurveto{\pgfqpoint{4.576296in}{5.948105in}}{\pgfqpoint{4.570709in}{5.945791in}}{\pgfqpoint{4.566591in}{5.941673in}}%
\pgfpathcurveto{\pgfqpoint{4.562473in}{5.937555in}}{\pgfqpoint{4.560159in}{5.931969in}}{\pgfqpoint{4.560159in}{5.926145in}}%
\pgfpathcurveto{\pgfqpoint{4.560159in}{5.920321in}}{\pgfqpoint{4.562473in}{5.914735in}}{\pgfqpoint{4.566591in}{5.910616in}}%
\pgfpathcurveto{\pgfqpoint{4.570709in}{5.906498in}}{\pgfqpoint{4.576296in}{5.904184in}}{\pgfqpoint{4.582120in}{5.904184in}}%
\pgfpathlineto{\pgfqpoint{4.582120in}{5.904184in}}%
\pgfpathclose%
\pgfusepath{stroke,fill}%
\end{pgfscope}%
\begin{pgfscope}%
\pgfpathrectangle{\pgfqpoint{1.000000in}{0.979904in}}{\pgfqpoint{6.200000in}{5.960192in}}%
\pgfusepath{clip}%
\pgfsetbuttcap%
\pgfsetroundjoin%
\definecolor{currentfill}{rgb}{0.200000,0.200000,0.800000}%
\pgfsetfillcolor{currentfill}%
\pgfsetlinewidth{1.003750pt}%
\definecolor{currentstroke}{rgb}{0.200000,0.200000,0.800000}%
\pgfsetstrokecolor{currentstroke}%
\pgfsetdash{}{0pt}%
\pgfpathmoveto{\pgfqpoint{4.548156in}{5.953466in}}%
\pgfpathcurveto{\pgfqpoint{4.553980in}{5.953466in}}{\pgfqpoint{4.559566in}{5.955780in}}{\pgfqpoint{4.563684in}{5.959898in}}%
\pgfpathcurveto{\pgfqpoint{4.567802in}{5.964016in}}{\pgfqpoint{4.570116in}{5.969603in}}{\pgfqpoint{4.570116in}{5.975427in}}%
\pgfpathcurveto{\pgfqpoint{4.570116in}{5.981251in}}{\pgfqpoint{4.567802in}{5.986837in}}{\pgfqpoint{4.563684in}{5.990955in}}%
\pgfpathcurveto{\pgfqpoint{4.559566in}{5.995073in}}{\pgfqpoint{4.553980in}{5.997387in}}{\pgfqpoint{4.548156in}{5.997387in}}%
\pgfpathcurveto{\pgfqpoint{4.542332in}{5.997387in}}{\pgfqpoint{4.536746in}{5.995073in}}{\pgfqpoint{4.532627in}{5.990955in}}%
\pgfpathcurveto{\pgfqpoint{4.528509in}{5.986837in}}{\pgfqpoint{4.526195in}{5.981251in}}{\pgfqpoint{4.526195in}{5.975427in}}%
\pgfpathcurveto{\pgfqpoint{4.526195in}{5.969603in}}{\pgfqpoint{4.528509in}{5.964016in}}{\pgfqpoint{4.532627in}{5.959898in}}%
\pgfpathcurveto{\pgfqpoint{4.536746in}{5.955780in}}{\pgfqpoint{4.542332in}{5.953466in}}{\pgfqpoint{4.548156in}{5.953466in}}%
\pgfpathlineto{\pgfqpoint{4.548156in}{5.953466in}}%
\pgfpathclose%
\pgfusepath{stroke,fill}%
\end{pgfscope}%
\begin{pgfscope}%
\pgfpathrectangle{\pgfqpoint{1.000000in}{0.979904in}}{\pgfqpoint{6.200000in}{5.960192in}}%
\pgfusepath{clip}%
\pgfsetbuttcap%
\pgfsetroundjoin%
\definecolor{currentfill}{rgb}{0.200000,0.200000,0.800000}%
\pgfsetfillcolor{currentfill}%
\pgfsetlinewidth{1.003750pt}%
\definecolor{currentstroke}{rgb}{0.200000,0.200000,0.800000}%
\pgfsetstrokecolor{currentstroke}%
\pgfsetdash{}{0pt}%
\pgfpathmoveto{\pgfqpoint{4.445746in}{5.973039in}}%
\pgfpathcurveto{\pgfqpoint{4.451570in}{5.973039in}}{\pgfqpoint{4.457156in}{5.975353in}}{\pgfqpoint{4.461274in}{5.979471in}}%
\pgfpathcurveto{\pgfqpoint{4.465392in}{5.983589in}}{\pgfqpoint{4.467706in}{5.989175in}}{\pgfqpoint{4.467706in}{5.994999in}}%
\pgfpathcurveto{\pgfqpoint{4.467706in}{6.000823in}}{\pgfqpoint{4.465392in}{6.006409in}}{\pgfqpoint{4.461274in}{6.010527in}}%
\pgfpathcurveto{\pgfqpoint{4.457156in}{6.014646in}}{\pgfqpoint{4.451570in}{6.016959in}}{\pgfqpoint{4.445746in}{6.016959in}}%
\pgfpathcurveto{\pgfqpoint{4.439922in}{6.016959in}}{\pgfqpoint{4.434336in}{6.014646in}}{\pgfqpoint{4.430218in}{6.010527in}}%
\pgfpathcurveto{\pgfqpoint{4.426100in}{6.006409in}}{\pgfqpoint{4.423786in}{6.000823in}}{\pgfqpoint{4.423786in}{5.994999in}}%
\pgfpathcurveto{\pgfqpoint{4.423786in}{5.989175in}}{\pgfqpoint{4.426100in}{5.983589in}}{\pgfqpoint{4.430218in}{5.979471in}}%
\pgfpathcurveto{\pgfqpoint{4.434336in}{5.975353in}}{\pgfqpoint{4.439922in}{5.973039in}}{\pgfqpoint{4.445746in}{5.973039in}}%
\pgfpathlineto{\pgfqpoint{4.445746in}{5.973039in}}%
\pgfpathclose%
\pgfusepath{stroke,fill}%
\end{pgfscope}%
\begin{pgfscope}%
\pgfpathrectangle{\pgfqpoint{1.000000in}{0.979904in}}{\pgfqpoint{6.200000in}{5.960192in}}%
\pgfusepath{clip}%
\pgfsetbuttcap%
\pgfsetroundjoin%
\definecolor{currentfill}{rgb}{0.200000,0.200000,0.800000}%
\pgfsetfillcolor{currentfill}%
\pgfsetlinewidth{1.003750pt}%
\definecolor{currentstroke}{rgb}{0.200000,0.200000,0.800000}%
\pgfsetstrokecolor{currentstroke}%
\pgfsetdash{}{0pt}%
\pgfpathmoveto{\pgfqpoint{4.596155in}{6.100619in}}%
\pgfpathcurveto{\pgfqpoint{4.601979in}{6.100619in}}{\pgfqpoint{4.607565in}{6.102933in}}{\pgfqpoint{4.611683in}{6.107051in}}%
\pgfpathcurveto{\pgfqpoint{4.615801in}{6.111169in}}{\pgfqpoint{4.618115in}{6.116755in}}{\pgfqpoint{4.618115in}{6.122579in}}%
\pgfpathcurveto{\pgfqpoint{4.618115in}{6.128403in}}{\pgfqpoint{4.615801in}{6.133989in}}{\pgfqpoint{4.611683in}{6.138108in}}%
\pgfpathcurveto{\pgfqpoint{4.607565in}{6.142226in}}{\pgfqpoint{4.601979in}{6.144540in}}{\pgfqpoint{4.596155in}{6.144540in}}%
\pgfpathcurveto{\pgfqpoint{4.590331in}{6.144540in}}{\pgfqpoint{4.584745in}{6.142226in}}{\pgfqpoint{4.580626in}{6.138108in}}%
\pgfpathcurveto{\pgfqpoint{4.576508in}{6.133989in}}{\pgfqpoint{4.574194in}{6.128403in}}{\pgfqpoint{4.574194in}{6.122579in}}%
\pgfpathcurveto{\pgfqpoint{4.574194in}{6.116755in}}{\pgfqpoint{4.576508in}{6.111169in}}{\pgfqpoint{4.580626in}{6.107051in}}%
\pgfpathcurveto{\pgfqpoint{4.584745in}{6.102933in}}{\pgfqpoint{4.590331in}{6.100619in}}{\pgfqpoint{4.596155in}{6.100619in}}%
\pgfpathlineto{\pgfqpoint{4.596155in}{6.100619in}}%
\pgfpathclose%
\pgfusepath{stroke,fill}%
\end{pgfscope}%
\begin{pgfscope}%
\pgfpathrectangle{\pgfqpoint{1.000000in}{0.979904in}}{\pgfqpoint{6.200000in}{5.960192in}}%
\pgfusepath{clip}%
\pgfsetbuttcap%
\pgfsetroundjoin%
\definecolor{currentfill}{rgb}{0.200000,0.800000,0.200000}%
\pgfsetfillcolor{currentfill}%
\pgfsetlinewidth{1.003750pt}%
\definecolor{currentstroke}{rgb}{0.200000,0.800000,0.200000}%
\pgfsetstrokecolor{currentstroke}%
\pgfsetdash{}{0pt}%
\pgfpathmoveto{\pgfqpoint{4.561274in}{6.152780in}}%
\pgfpathcurveto{\pgfqpoint{4.567098in}{6.152780in}}{\pgfqpoint{4.572684in}{6.155094in}}{\pgfqpoint{4.576802in}{6.159212in}}%
\pgfpathcurveto{\pgfqpoint{4.580920in}{6.163330in}}{\pgfqpoint{4.583234in}{6.168916in}}{\pgfqpoint{4.583234in}{6.174740in}}%
\pgfpathcurveto{\pgfqpoint{4.583234in}{6.180564in}}{\pgfqpoint{4.580920in}{6.186150in}}{\pgfqpoint{4.576802in}{6.190269in}}%
\pgfpathcurveto{\pgfqpoint{4.572684in}{6.194387in}}{\pgfqpoint{4.567098in}{6.196701in}}{\pgfqpoint{4.561274in}{6.196701in}}%
\pgfpathcurveto{\pgfqpoint{4.555450in}{6.196701in}}{\pgfqpoint{4.549864in}{6.194387in}}{\pgfqpoint{4.545745in}{6.190269in}}%
\pgfpathcurveto{\pgfqpoint{4.541627in}{6.186150in}}{\pgfqpoint{4.539313in}{6.180564in}}{\pgfqpoint{4.539313in}{6.174740in}}%
\pgfpathcurveto{\pgfqpoint{4.539313in}{6.168916in}}{\pgfqpoint{4.541627in}{6.163330in}}{\pgfqpoint{4.545745in}{6.159212in}}%
\pgfpathcurveto{\pgfqpoint{4.549864in}{6.155094in}}{\pgfqpoint{4.555450in}{6.152780in}}{\pgfqpoint{4.561274in}{6.152780in}}%
\pgfpathlineto{\pgfqpoint{4.561274in}{6.152780in}}%
\pgfpathclose%
\pgfusepath{stroke,fill}%
\end{pgfscope}%
\begin{pgfscope}%
\pgfpathrectangle{\pgfqpoint{1.000000in}{0.979904in}}{\pgfqpoint{6.200000in}{5.960192in}}%
\pgfusepath{clip}%
\pgfsetbuttcap%
\pgfsetroundjoin%
\definecolor{currentfill}{rgb}{0.200000,0.800000,0.200000}%
\pgfsetfillcolor{currentfill}%
\pgfsetlinewidth{1.003750pt}%
\definecolor{currentstroke}{rgb}{0.200000,0.800000,0.200000}%
\pgfsetstrokecolor{currentstroke}%
\pgfsetdash{}{0pt}%
\pgfpathmoveto{\pgfqpoint{4.424906in}{6.138947in}}%
\pgfpathcurveto{\pgfqpoint{4.430729in}{6.138947in}}{\pgfqpoint{4.436316in}{6.141261in}}{\pgfqpoint{4.440434in}{6.145379in}}%
\pgfpathcurveto{\pgfqpoint{4.444552in}{6.149497in}}{\pgfqpoint{4.446866in}{6.155084in}}{\pgfqpoint{4.446866in}{6.160908in}}%
\pgfpathcurveto{\pgfqpoint{4.446866in}{6.166732in}}{\pgfqpoint{4.444552in}{6.172318in}}{\pgfqpoint{4.440434in}{6.176436in}}%
\pgfpathcurveto{\pgfqpoint{4.436316in}{6.180554in}}{\pgfqpoint{4.430729in}{6.182868in}}{\pgfqpoint{4.424906in}{6.182868in}}%
\pgfpathcurveto{\pgfqpoint{4.419082in}{6.182868in}}{\pgfqpoint{4.413495in}{6.180554in}}{\pgfqpoint{4.409377in}{6.176436in}}%
\pgfpathcurveto{\pgfqpoint{4.405259in}{6.172318in}}{\pgfqpoint{4.402945in}{6.166732in}}{\pgfqpoint{4.402945in}{6.160908in}}%
\pgfpathcurveto{\pgfqpoint{4.402945in}{6.155084in}}{\pgfqpoint{4.405259in}{6.149497in}}{\pgfqpoint{4.409377in}{6.145379in}}%
\pgfpathcurveto{\pgfqpoint{4.413495in}{6.141261in}}{\pgfqpoint{4.419082in}{6.138947in}}{\pgfqpoint{4.424906in}{6.138947in}}%
\pgfpathlineto{\pgfqpoint{4.424906in}{6.138947in}}%
\pgfpathclose%
\pgfusepath{stroke,fill}%
\end{pgfscope}%
\begin{pgfscope}%
\pgfpathrectangle{\pgfqpoint{1.000000in}{0.979904in}}{\pgfqpoint{6.200000in}{5.960192in}}%
\pgfusepath{clip}%
\pgfsetbuttcap%
\pgfsetroundjoin%
\definecolor{currentfill}{rgb}{0.200000,0.200000,0.800000}%
\pgfsetfillcolor{currentfill}%
\pgfsetlinewidth{1.003750pt}%
\definecolor{currentstroke}{rgb}{0.200000,0.200000,0.800000}%
\pgfsetstrokecolor{currentstroke}%
\pgfsetdash{}{0pt}%
\pgfpathmoveto{\pgfqpoint{4.440967in}{6.218097in}}%
\pgfpathcurveto{\pgfqpoint{4.446791in}{6.218097in}}{\pgfqpoint{4.452377in}{6.220411in}}{\pgfqpoint{4.456495in}{6.224529in}}%
\pgfpathcurveto{\pgfqpoint{4.460613in}{6.228647in}}{\pgfqpoint{4.462927in}{6.234233in}}{\pgfqpoint{4.462927in}{6.240057in}}%
\pgfpathcurveto{\pgfqpoint{4.462927in}{6.245881in}}{\pgfqpoint{4.460613in}{6.251467in}}{\pgfqpoint{4.456495in}{6.255585in}}%
\pgfpathcurveto{\pgfqpoint{4.452377in}{6.259704in}}{\pgfqpoint{4.446791in}{6.262017in}}{\pgfqpoint{4.440967in}{6.262017in}}%
\pgfpathcurveto{\pgfqpoint{4.435143in}{6.262017in}}{\pgfqpoint{4.429557in}{6.259704in}}{\pgfqpoint{4.425438in}{6.255585in}}%
\pgfpathcurveto{\pgfqpoint{4.421320in}{6.251467in}}{\pgfqpoint{4.419006in}{6.245881in}}{\pgfqpoint{4.419006in}{6.240057in}}%
\pgfpathcurveto{\pgfqpoint{4.419006in}{6.234233in}}{\pgfqpoint{4.421320in}{6.228647in}}{\pgfqpoint{4.425438in}{6.224529in}}%
\pgfpathcurveto{\pgfqpoint{4.429557in}{6.220411in}}{\pgfqpoint{4.435143in}{6.218097in}}{\pgfqpoint{4.440967in}{6.218097in}}%
\pgfpathlineto{\pgfqpoint{4.440967in}{6.218097in}}%
\pgfpathclose%
\pgfusepath{stroke,fill}%
\end{pgfscope}%
\begin{pgfscope}%
\pgfpathrectangle{\pgfqpoint{1.000000in}{0.979904in}}{\pgfqpoint{6.200000in}{5.960192in}}%
\pgfusepath{clip}%
\pgfsetbuttcap%
\pgfsetroundjoin%
\definecolor{currentfill}{rgb}{0.200000,0.200000,0.800000}%
\pgfsetfillcolor{currentfill}%
\pgfsetlinewidth{1.003750pt}%
\definecolor{currentstroke}{rgb}{0.200000,0.200000,0.800000}%
\pgfsetstrokecolor{currentstroke}%
\pgfsetdash{}{0pt}%
\pgfpathmoveto{\pgfqpoint{4.392380in}{6.252913in}}%
\pgfpathcurveto{\pgfqpoint{4.398203in}{6.252913in}}{\pgfqpoint{4.403790in}{6.255226in}}{\pgfqpoint{4.407908in}{6.259345in}}%
\pgfpathcurveto{\pgfqpoint{4.412026in}{6.263463in}}{\pgfqpoint{4.414340in}{6.269049in}}{\pgfqpoint{4.414340in}{6.274873in}}%
\pgfpathcurveto{\pgfqpoint{4.414340in}{6.280697in}}{\pgfqpoint{4.412026in}{6.286283in}}{\pgfqpoint{4.407908in}{6.290401in}}%
\pgfpathcurveto{\pgfqpoint{4.403790in}{6.294519in}}{\pgfqpoint{4.398203in}{6.296833in}}{\pgfqpoint{4.392380in}{6.296833in}}%
\pgfpathcurveto{\pgfqpoint{4.386556in}{6.296833in}}{\pgfqpoint{4.380969in}{6.294519in}}{\pgfqpoint{4.376851in}{6.290401in}}%
\pgfpathcurveto{\pgfqpoint{4.372733in}{6.286283in}}{\pgfqpoint{4.370419in}{6.280697in}}{\pgfqpoint{4.370419in}{6.274873in}}%
\pgfpathcurveto{\pgfqpoint{4.370419in}{6.269049in}}{\pgfqpoint{4.372733in}{6.263463in}}{\pgfqpoint{4.376851in}{6.259345in}}%
\pgfpathcurveto{\pgfqpoint{4.380969in}{6.255226in}}{\pgfqpoint{4.386556in}{6.252913in}}{\pgfqpoint{4.392380in}{6.252913in}}%
\pgfpathlineto{\pgfqpoint{4.392380in}{6.252913in}}%
\pgfpathclose%
\pgfusepath{stroke,fill}%
\end{pgfscope}%
\begin{pgfscope}%
\pgfpathrectangle{\pgfqpoint{1.000000in}{0.979904in}}{\pgfqpoint{6.200000in}{5.960192in}}%
\pgfusepath{clip}%
\pgfsetbuttcap%
\pgfsetroundjoin%
\definecolor{currentfill}{rgb}{0.200000,0.200000,0.800000}%
\pgfsetfillcolor{currentfill}%
\pgfsetlinewidth{1.003750pt}%
\definecolor{currentstroke}{rgb}{0.200000,0.200000,0.800000}%
\pgfsetstrokecolor{currentstroke}%
\pgfsetdash{}{0pt}%
\pgfpathmoveto{\pgfqpoint{4.355959in}{6.296728in}}%
\pgfpathcurveto{\pgfqpoint{4.361783in}{6.296728in}}{\pgfqpoint{4.367369in}{6.299042in}}{\pgfqpoint{4.371487in}{6.303160in}}%
\pgfpathcurveto{\pgfqpoint{4.375605in}{6.307278in}}{\pgfqpoint{4.377919in}{6.312864in}}{\pgfqpoint{4.377919in}{6.318688in}}%
\pgfpathcurveto{\pgfqpoint{4.377919in}{6.324512in}}{\pgfqpoint{4.375605in}{6.330098in}}{\pgfqpoint{4.371487in}{6.334216in}}%
\pgfpathcurveto{\pgfqpoint{4.367369in}{6.338334in}}{\pgfqpoint{4.361783in}{6.340648in}}{\pgfqpoint{4.355959in}{6.340648in}}%
\pgfpathcurveto{\pgfqpoint{4.350135in}{6.340648in}}{\pgfqpoint{4.344549in}{6.338334in}}{\pgfqpoint{4.340431in}{6.334216in}}%
\pgfpathcurveto{\pgfqpoint{4.336312in}{6.330098in}}{\pgfqpoint{4.333999in}{6.324512in}}{\pgfqpoint{4.333999in}{6.318688in}}%
\pgfpathcurveto{\pgfqpoint{4.333999in}{6.312864in}}{\pgfqpoint{4.336312in}{6.307278in}}{\pgfqpoint{4.340431in}{6.303160in}}%
\pgfpathcurveto{\pgfqpoint{4.344549in}{6.299042in}}{\pgfqpoint{4.350135in}{6.296728in}}{\pgfqpoint{4.355959in}{6.296728in}}%
\pgfpathlineto{\pgfqpoint{4.355959in}{6.296728in}}%
\pgfpathclose%
\pgfusepath{stroke,fill}%
\end{pgfscope}%
\begin{pgfscope}%
\pgfpathrectangle{\pgfqpoint{1.000000in}{0.979904in}}{\pgfqpoint{6.200000in}{5.960192in}}%
\pgfusepath{clip}%
\pgfsetbuttcap%
\pgfsetroundjoin%
\definecolor{currentfill}{rgb}{0.200000,0.200000,0.800000}%
\pgfsetfillcolor{currentfill}%
\pgfsetlinewidth{1.003750pt}%
\definecolor{currentstroke}{rgb}{0.200000,0.200000,0.800000}%
\pgfsetstrokecolor{currentstroke}%
\pgfsetdash{}{0pt}%
\pgfpathmoveto{\pgfqpoint{4.263627in}{6.280736in}}%
\pgfpathcurveto{\pgfqpoint{4.269451in}{6.280736in}}{\pgfqpoint{4.275037in}{6.283049in}}{\pgfqpoint{4.279156in}{6.287168in}}%
\pgfpathcurveto{\pgfqpoint{4.283274in}{6.291286in}}{\pgfqpoint{4.285588in}{6.296872in}}{\pgfqpoint{4.285588in}{6.302696in}}%
\pgfpathcurveto{\pgfqpoint{4.285588in}{6.308520in}}{\pgfqpoint{4.283274in}{6.314106in}}{\pgfqpoint{4.279156in}{6.318224in}}%
\pgfpathcurveto{\pgfqpoint{4.275037in}{6.322342in}}{\pgfqpoint{4.269451in}{6.324656in}}{\pgfqpoint{4.263627in}{6.324656in}}%
\pgfpathcurveto{\pgfqpoint{4.257803in}{6.324656in}}{\pgfqpoint{4.252217in}{6.322342in}}{\pgfqpoint{4.248099in}{6.318224in}}%
\pgfpathcurveto{\pgfqpoint{4.243981in}{6.314106in}}{\pgfqpoint{4.241667in}{6.308520in}}{\pgfqpoint{4.241667in}{6.302696in}}%
\pgfpathcurveto{\pgfqpoint{4.241667in}{6.296872in}}{\pgfqpoint{4.243981in}{6.291286in}}{\pgfqpoint{4.248099in}{6.287168in}}%
\pgfpathcurveto{\pgfqpoint{4.252217in}{6.283049in}}{\pgfqpoint{4.257803in}{6.280736in}}{\pgfqpoint{4.263627in}{6.280736in}}%
\pgfpathlineto{\pgfqpoint{4.263627in}{6.280736in}}%
\pgfpathclose%
\pgfusepath{stroke,fill}%
\end{pgfscope}%
\begin{pgfscope}%
\pgfpathrectangle{\pgfqpoint{1.000000in}{0.979904in}}{\pgfqpoint{6.200000in}{5.960192in}}%
\pgfusepath{clip}%
\pgfsetbuttcap%
\pgfsetroundjoin%
\definecolor{currentfill}{rgb}{0.200000,0.200000,0.800000}%
\pgfsetfillcolor{currentfill}%
\pgfsetlinewidth{1.003750pt}%
\definecolor{currentstroke}{rgb}{0.200000,0.200000,0.800000}%
\pgfsetstrokecolor{currentstroke}%
\pgfsetdash{}{0pt}%
\pgfpathmoveto{\pgfqpoint{4.274723in}{6.377086in}}%
\pgfpathcurveto{\pgfqpoint{4.280547in}{6.377086in}}{\pgfqpoint{4.286133in}{6.379400in}}{\pgfqpoint{4.290251in}{6.383518in}}%
\pgfpathcurveto{\pgfqpoint{4.294370in}{6.387636in}}{\pgfqpoint{4.296683in}{6.393222in}}{\pgfqpoint{4.296683in}{6.399046in}}%
\pgfpathcurveto{\pgfqpoint{4.296683in}{6.404870in}}{\pgfqpoint{4.294370in}{6.410456in}}{\pgfqpoint{4.290251in}{6.414574in}}%
\pgfpathcurveto{\pgfqpoint{4.286133in}{6.418693in}}{\pgfqpoint{4.280547in}{6.421006in}}{\pgfqpoint{4.274723in}{6.421006in}}%
\pgfpathcurveto{\pgfqpoint{4.268899in}{6.421006in}}{\pgfqpoint{4.263313in}{6.418693in}}{\pgfqpoint{4.259195in}{6.414574in}}%
\pgfpathcurveto{\pgfqpoint{4.255077in}{6.410456in}}{\pgfqpoint{4.252763in}{6.404870in}}{\pgfqpoint{4.252763in}{6.399046in}}%
\pgfpathcurveto{\pgfqpoint{4.252763in}{6.393222in}}{\pgfqpoint{4.255077in}{6.387636in}}{\pgfqpoint{4.259195in}{6.383518in}}%
\pgfpathcurveto{\pgfqpoint{4.263313in}{6.379400in}}{\pgfqpoint{4.268899in}{6.377086in}}{\pgfqpoint{4.274723in}{6.377086in}}%
\pgfpathlineto{\pgfqpoint{4.274723in}{6.377086in}}%
\pgfpathclose%
\pgfusepath{stroke,fill}%
\end{pgfscope}%
\begin{pgfscope}%
\pgfpathrectangle{\pgfqpoint{1.000000in}{0.979904in}}{\pgfqpoint{6.200000in}{5.960192in}}%
\pgfusepath{clip}%
\pgfsetbuttcap%
\pgfsetroundjoin%
\definecolor{currentfill}{rgb}{0.200000,0.200000,0.800000}%
\pgfsetfillcolor{currentfill}%
\pgfsetlinewidth{1.003750pt}%
\definecolor{currentstroke}{rgb}{0.200000,0.200000,0.800000}%
\pgfsetstrokecolor{currentstroke}%
\pgfsetdash{}{0pt}%
\pgfpathmoveto{\pgfqpoint{4.266932in}{6.464796in}}%
\pgfpathcurveto{\pgfqpoint{4.272756in}{6.464796in}}{\pgfqpoint{4.278342in}{6.467110in}}{\pgfqpoint{4.282460in}{6.471228in}}%
\pgfpathcurveto{\pgfqpoint{4.286578in}{6.475347in}}{\pgfqpoint{4.288892in}{6.480933in}}{\pgfqpoint{4.288892in}{6.486757in}}%
\pgfpathcurveto{\pgfqpoint{4.288892in}{6.492581in}}{\pgfqpoint{4.286578in}{6.498167in}}{\pgfqpoint{4.282460in}{6.502285in}}%
\pgfpathcurveto{\pgfqpoint{4.278342in}{6.506403in}}{\pgfqpoint{4.272756in}{6.508717in}}{\pgfqpoint{4.266932in}{6.508717in}}%
\pgfpathcurveto{\pgfqpoint{4.261108in}{6.508717in}}{\pgfqpoint{4.255522in}{6.506403in}}{\pgfqpoint{4.251404in}{6.502285in}}%
\pgfpathcurveto{\pgfqpoint{4.247285in}{6.498167in}}{\pgfqpoint{4.244972in}{6.492581in}}{\pgfqpoint{4.244972in}{6.486757in}}%
\pgfpathcurveto{\pgfqpoint{4.244972in}{6.480933in}}{\pgfqpoint{4.247285in}{6.475347in}}{\pgfqpoint{4.251404in}{6.471228in}}%
\pgfpathcurveto{\pgfqpoint{4.255522in}{6.467110in}}{\pgfqpoint{4.261108in}{6.464796in}}{\pgfqpoint{4.266932in}{6.464796in}}%
\pgfpathlineto{\pgfqpoint{4.266932in}{6.464796in}}%
\pgfpathclose%
\pgfusepath{stroke,fill}%
\end{pgfscope}%
\begin{pgfscope}%
\pgfpathrectangle{\pgfqpoint{1.000000in}{0.979904in}}{\pgfqpoint{6.200000in}{5.960192in}}%
\pgfusepath{clip}%
\pgfsetbuttcap%
\pgfsetroundjoin%
\definecolor{currentfill}{rgb}{0.200000,0.200000,0.800000}%
\pgfsetfillcolor{currentfill}%
\pgfsetlinewidth{1.003750pt}%
\definecolor{currentstroke}{rgb}{0.200000,0.200000,0.800000}%
\pgfsetstrokecolor{currentstroke}%
\pgfsetdash{}{0pt}%
\pgfpathmoveto{\pgfqpoint{4.270701in}{6.587242in}}%
\pgfpathcurveto{\pgfqpoint{4.276525in}{6.587242in}}{\pgfqpoint{4.282111in}{6.589556in}}{\pgfqpoint{4.286230in}{6.593674in}}%
\pgfpathcurveto{\pgfqpoint{4.290348in}{6.597792in}}{\pgfqpoint{4.292662in}{6.603378in}}{\pgfqpoint{4.292662in}{6.609202in}}%
\pgfpathcurveto{\pgfqpoint{4.292662in}{6.615026in}}{\pgfqpoint{4.290348in}{6.620612in}}{\pgfqpoint{4.286230in}{6.624730in}}%
\pgfpathcurveto{\pgfqpoint{4.282111in}{6.628848in}}{\pgfqpoint{4.276525in}{6.631162in}}{\pgfqpoint{4.270701in}{6.631162in}}%
\pgfpathcurveto{\pgfqpoint{4.264877in}{6.631162in}}{\pgfqpoint{4.259291in}{6.628848in}}{\pgfqpoint{4.255173in}{6.624730in}}%
\pgfpathcurveto{\pgfqpoint{4.251055in}{6.620612in}}{\pgfqpoint{4.248741in}{6.615026in}}{\pgfqpoint{4.248741in}{6.609202in}}%
\pgfpathcurveto{\pgfqpoint{4.248741in}{6.603378in}}{\pgfqpoint{4.251055in}{6.597792in}}{\pgfqpoint{4.255173in}{6.593674in}}%
\pgfpathcurveto{\pgfqpoint{4.259291in}{6.589556in}}{\pgfqpoint{4.264877in}{6.587242in}}{\pgfqpoint{4.270701in}{6.587242in}}%
\pgfpathlineto{\pgfqpoint{4.270701in}{6.587242in}}%
\pgfpathclose%
\pgfusepath{stroke,fill}%
\end{pgfscope}%
\begin{pgfscope}%
\pgfpathrectangle{\pgfqpoint{1.000000in}{0.979904in}}{\pgfqpoint{6.200000in}{5.960192in}}%
\pgfusepath{clip}%
\pgfsetbuttcap%
\pgfsetroundjoin%
\definecolor{currentfill}{rgb}{0.200000,0.200000,0.800000}%
\pgfsetfillcolor{currentfill}%
\pgfsetlinewidth{1.003750pt}%
\definecolor{currentstroke}{rgb}{0.200000,0.200000,0.800000}%
\pgfsetstrokecolor{currentstroke}%
\pgfsetdash{}{0pt}%
\pgfpathmoveto{\pgfqpoint{4.157566in}{6.519974in}}%
\pgfpathcurveto{\pgfqpoint{4.163390in}{6.519974in}}{\pgfqpoint{4.168976in}{6.522288in}}{\pgfqpoint{4.173094in}{6.526406in}}%
\pgfpathcurveto{\pgfqpoint{4.177212in}{6.530524in}}{\pgfqpoint{4.179526in}{6.536110in}}{\pgfqpoint{4.179526in}{6.541934in}}%
\pgfpathcurveto{\pgfqpoint{4.179526in}{6.547758in}}{\pgfqpoint{4.177212in}{6.553344in}}{\pgfqpoint{4.173094in}{6.557462in}}%
\pgfpathcurveto{\pgfqpoint{4.168976in}{6.561581in}}{\pgfqpoint{4.163390in}{6.563894in}}{\pgfqpoint{4.157566in}{6.563894in}}%
\pgfpathcurveto{\pgfqpoint{4.151742in}{6.563894in}}{\pgfqpoint{4.146156in}{6.561581in}}{\pgfqpoint{4.142038in}{6.557462in}}%
\pgfpathcurveto{\pgfqpoint{4.137920in}{6.553344in}}{\pgfqpoint{4.135606in}{6.547758in}}{\pgfqpoint{4.135606in}{6.541934in}}%
\pgfpathcurveto{\pgfqpoint{4.135606in}{6.536110in}}{\pgfqpoint{4.137920in}{6.530524in}}{\pgfqpoint{4.142038in}{6.526406in}}%
\pgfpathcurveto{\pgfqpoint{4.146156in}{6.522288in}}{\pgfqpoint{4.151742in}{6.519974in}}{\pgfqpoint{4.157566in}{6.519974in}}%
\pgfpathlineto{\pgfqpoint{4.157566in}{6.519974in}}%
\pgfpathclose%
\pgfusepath{stroke,fill}%
\end{pgfscope}%
\begin{pgfscope}%
\pgfpathrectangle{\pgfqpoint{1.000000in}{0.979904in}}{\pgfqpoint{6.200000in}{5.960192in}}%
\pgfusepath{clip}%
\pgfsetbuttcap%
\pgfsetroundjoin%
\definecolor{currentfill}{rgb}{0.200000,0.200000,0.800000}%
\pgfsetfillcolor{currentfill}%
\pgfsetlinewidth{1.003750pt}%
\definecolor{currentstroke}{rgb}{0.200000,0.200000,0.800000}%
\pgfsetstrokecolor{currentstroke}%
\pgfsetdash{}{0pt}%
\pgfpathmoveto{\pgfqpoint{4.097185in}{6.533800in}}%
\pgfpathcurveto{\pgfqpoint{4.103009in}{6.533800in}}{\pgfqpoint{4.108595in}{6.536114in}}{\pgfqpoint{4.112713in}{6.540232in}}%
\pgfpathcurveto{\pgfqpoint{4.116831in}{6.544351in}}{\pgfqpoint{4.119145in}{6.549937in}}{\pgfqpoint{4.119145in}{6.555761in}}%
\pgfpathcurveto{\pgfqpoint{4.119145in}{6.561585in}}{\pgfqpoint{4.116831in}{6.567171in}}{\pgfqpoint{4.112713in}{6.571289in}}%
\pgfpathcurveto{\pgfqpoint{4.108595in}{6.575407in}}{\pgfqpoint{4.103009in}{6.577721in}}{\pgfqpoint{4.097185in}{6.577721in}}%
\pgfpathcurveto{\pgfqpoint{4.091361in}{6.577721in}}{\pgfqpoint{4.085775in}{6.575407in}}{\pgfqpoint{4.081656in}{6.571289in}}%
\pgfpathcurveto{\pgfqpoint{4.077538in}{6.567171in}}{\pgfqpoint{4.075224in}{6.561585in}}{\pgfqpoint{4.075224in}{6.555761in}}%
\pgfpathcurveto{\pgfqpoint{4.075224in}{6.549937in}}{\pgfqpoint{4.077538in}{6.544351in}}{\pgfqpoint{4.081656in}{6.540232in}}%
\pgfpathcurveto{\pgfqpoint{4.085775in}{6.536114in}}{\pgfqpoint{4.091361in}{6.533800in}}{\pgfqpoint{4.097185in}{6.533800in}}%
\pgfpathlineto{\pgfqpoint{4.097185in}{6.533800in}}%
\pgfpathclose%
\pgfusepath{stroke,fill}%
\end{pgfscope}%
\begin{pgfscope}%
\pgfpathrectangle{\pgfqpoint{1.000000in}{0.979904in}}{\pgfqpoint{6.200000in}{5.960192in}}%
\pgfusepath{clip}%
\pgfsetbuttcap%
\pgfsetroundjoin%
\definecolor{currentfill}{rgb}{0.200000,0.200000,0.800000}%
\pgfsetfillcolor{currentfill}%
\pgfsetlinewidth{1.003750pt}%
\definecolor{currentstroke}{rgb}{0.200000,0.200000,0.800000}%
\pgfsetstrokecolor{currentstroke}%
\pgfsetdash{}{0pt}%
\pgfpathmoveto{\pgfqpoint{4.029969in}{6.525293in}}%
\pgfpathcurveto{\pgfqpoint{4.035793in}{6.525293in}}{\pgfqpoint{4.041379in}{6.527607in}}{\pgfqpoint{4.045497in}{6.531725in}}%
\pgfpathcurveto{\pgfqpoint{4.049616in}{6.535844in}}{\pgfqpoint{4.051929in}{6.541430in}}{\pgfqpoint{4.051929in}{6.547254in}}%
\pgfpathcurveto{\pgfqpoint{4.051929in}{6.553078in}}{\pgfqpoint{4.049616in}{6.558664in}}{\pgfqpoint{4.045497in}{6.562782in}}%
\pgfpathcurveto{\pgfqpoint{4.041379in}{6.566900in}}{\pgfqpoint{4.035793in}{6.569214in}}{\pgfqpoint{4.029969in}{6.569214in}}%
\pgfpathcurveto{\pgfqpoint{4.024145in}{6.569214in}}{\pgfqpoint{4.018559in}{6.566900in}}{\pgfqpoint{4.014441in}{6.562782in}}%
\pgfpathcurveto{\pgfqpoint{4.010323in}{6.558664in}}{\pgfqpoint{4.008009in}{6.553078in}}{\pgfqpoint{4.008009in}{6.547254in}}%
\pgfpathcurveto{\pgfqpoint{4.008009in}{6.541430in}}{\pgfqpoint{4.010323in}{6.535844in}}{\pgfqpoint{4.014441in}{6.531725in}}%
\pgfpathcurveto{\pgfqpoint{4.018559in}{6.527607in}}{\pgfqpoint{4.024145in}{6.525293in}}{\pgfqpoint{4.029969in}{6.525293in}}%
\pgfpathlineto{\pgfqpoint{4.029969in}{6.525293in}}%
\pgfpathclose%
\pgfusepath{stroke,fill}%
\end{pgfscope}%
\begin{pgfscope}%
\pgfpathrectangle{\pgfqpoint{1.000000in}{0.979904in}}{\pgfqpoint{6.200000in}{5.960192in}}%
\pgfusepath{clip}%
\pgfsetbuttcap%
\pgfsetroundjoin%
\definecolor{currentfill}{rgb}{0.200000,0.200000,0.800000}%
\pgfsetfillcolor{currentfill}%
\pgfsetlinewidth{1.003750pt}%
\definecolor{currentstroke}{rgb}{0.200000,0.200000,0.800000}%
\pgfsetstrokecolor{currentstroke}%
\pgfsetdash{}{0pt}%
\pgfpathmoveto{\pgfqpoint{3.974166in}{6.539847in}}%
\pgfpathcurveto{\pgfqpoint{3.979990in}{6.539847in}}{\pgfqpoint{3.985577in}{6.542161in}}{\pgfqpoint{3.989695in}{6.546279in}}%
\pgfpathcurveto{\pgfqpoint{3.993813in}{6.550397in}}{\pgfqpoint{3.996127in}{6.555984in}}{\pgfqpoint{3.996127in}{6.561808in}}%
\pgfpathcurveto{\pgfqpoint{3.996127in}{6.567631in}}{\pgfqpoint{3.993813in}{6.573218in}}{\pgfqpoint{3.989695in}{6.577336in}}%
\pgfpathcurveto{\pgfqpoint{3.985577in}{6.581454in}}{\pgfqpoint{3.979990in}{6.583768in}}{\pgfqpoint{3.974166in}{6.583768in}}%
\pgfpathcurveto{\pgfqpoint{3.968343in}{6.583768in}}{\pgfqpoint{3.962756in}{6.581454in}}{\pgfqpoint{3.958638in}{6.577336in}}%
\pgfpathcurveto{\pgfqpoint{3.954520in}{6.573218in}}{\pgfqpoint{3.952206in}{6.567631in}}{\pgfqpoint{3.952206in}{6.561808in}}%
\pgfpathcurveto{\pgfqpoint{3.952206in}{6.555984in}}{\pgfqpoint{3.954520in}{6.550397in}}{\pgfqpoint{3.958638in}{6.546279in}}%
\pgfpathcurveto{\pgfqpoint{3.962756in}{6.542161in}}{\pgfqpoint{3.968343in}{6.539847in}}{\pgfqpoint{3.974166in}{6.539847in}}%
\pgfpathlineto{\pgfqpoint{3.974166in}{6.539847in}}%
\pgfpathclose%
\pgfusepath{stroke,fill}%
\end{pgfscope}%
\begin{pgfscope}%
\pgfpathrectangle{\pgfqpoint{1.000000in}{0.979904in}}{\pgfqpoint{6.200000in}{5.960192in}}%
\pgfusepath{clip}%
\pgfsetbuttcap%
\pgfsetroundjoin%
\definecolor{currentfill}{rgb}{0.200000,0.200000,0.800000}%
\pgfsetfillcolor{currentfill}%
\pgfsetlinewidth{1.003750pt}%
\definecolor{currentstroke}{rgb}{0.200000,0.200000,0.800000}%
\pgfsetstrokecolor{currentstroke}%
\pgfsetdash{}{0pt}%
\pgfpathmoveto{\pgfqpoint{3.938249in}{6.634931in}}%
\pgfpathcurveto{\pgfqpoint{3.944073in}{6.634931in}}{\pgfqpoint{3.949659in}{6.637244in}}{\pgfqpoint{3.953777in}{6.641363in}}%
\pgfpathcurveto{\pgfqpoint{3.957895in}{6.645481in}}{\pgfqpoint{3.960209in}{6.651067in}}{\pgfqpoint{3.960209in}{6.656891in}}%
\pgfpathcurveto{\pgfqpoint{3.960209in}{6.662715in}}{\pgfqpoint{3.957895in}{6.668301in}}{\pgfqpoint{3.953777in}{6.672419in}}%
\pgfpathcurveto{\pgfqpoint{3.949659in}{6.676537in}}{\pgfqpoint{3.944073in}{6.678851in}}{\pgfqpoint{3.938249in}{6.678851in}}%
\pgfpathcurveto{\pgfqpoint{3.932425in}{6.678851in}}{\pgfqpoint{3.926839in}{6.676537in}}{\pgfqpoint{3.922720in}{6.672419in}}%
\pgfpathcurveto{\pgfqpoint{3.918602in}{6.668301in}}{\pgfqpoint{3.916288in}{6.662715in}}{\pgfqpoint{3.916288in}{6.656891in}}%
\pgfpathcurveto{\pgfqpoint{3.916288in}{6.651067in}}{\pgfqpoint{3.918602in}{6.645481in}}{\pgfqpoint{3.922720in}{6.641363in}}%
\pgfpathcurveto{\pgfqpoint{3.926839in}{6.637244in}}{\pgfqpoint{3.932425in}{6.634931in}}{\pgfqpoint{3.938249in}{6.634931in}}%
\pgfpathlineto{\pgfqpoint{3.938249in}{6.634931in}}%
\pgfpathclose%
\pgfusepath{stroke,fill}%
\end{pgfscope}%
\begin{pgfscope}%
\pgfpathrectangle{\pgfqpoint{1.000000in}{0.979904in}}{\pgfqpoint{6.200000in}{5.960192in}}%
\pgfusepath{clip}%
\pgfsetbuttcap%
\pgfsetroundjoin%
\definecolor{currentfill}{rgb}{0.200000,0.200000,0.800000}%
\pgfsetfillcolor{currentfill}%
\pgfsetlinewidth{1.003750pt}%
\definecolor{currentstroke}{rgb}{0.200000,0.200000,0.800000}%
\pgfsetstrokecolor{currentstroke}%
\pgfsetdash{}{0pt}%
\pgfpathmoveto{\pgfqpoint{3.870966in}{6.613446in}}%
\pgfpathcurveto{\pgfqpoint{3.876790in}{6.613446in}}{\pgfqpoint{3.882376in}{6.615760in}}{\pgfqpoint{3.886494in}{6.619878in}}%
\pgfpathcurveto{\pgfqpoint{3.890612in}{6.623996in}}{\pgfqpoint{3.892926in}{6.629582in}}{\pgfqpoint{3.892926in}{6.635406in}}%
\pgfpathcurveto{\pgfqpoint{3.892926in}{6.641230in}}{\pgfqpoint{3.890612in}{6.646816in}}{\pgfqpoint{3.886494in}{6.650935in}}%
\pgfpathcurveto{\pgfqpoint{3.882376in}{6.655053in}}{\pgfqpoint{3.876790in}{6.657367in}}{\pgfqpoint{3.870966in}{6.657367in}}%
\pgfpathcurveto{\pgfqpoint{3.865142in}{6.657367in}}{\pgfqpoint{3.859556in}{6.655053in}}{\pgfqpoint{3.855437in}{6.650935in}}%
\pgfpathcurveto{\pgfqpoint{3.851319in}{6.646816in}}{\pgfqpoint{3.849005in}{6.641230in}}{\pgfqpoint{3.849005in}{6.635406in}}%
\pgfpathcurveto{\pgfqpoint{3.849005in}{6.629582in}}{\pgfqpoint{3.851319in}{6.623996in}}{\pgfqpoint{3.855437in}{6.619878in}}%
\pgfpathcurveto{\pgfqpoint{3.859556in}{6.615760in}}{\pgfqpoint{3.865142in}{6.613446in}}{\pgfqpoint{3.870966in}{6.613446in}}%
\pgfpathlineto{\pgfqpoint{3.870966in}{6.613446in}}%
\pgfpathclose%
\pgfusepath{stroke,fill}%
\end{pgfscope}%
\begin{pgfscope}%
\pgfpathrectangle{\pgfqpoint{1.000000in}{0.979904in}}{\pgfqpoint{6.200000in}{5.960192in}}%
\pgfusepath{clip}%
\pgfsetbuttcap%
\pgfsetroundjoin%
\definecolor{currentfill}{rgb}{0.200000,0.200000,0.800000}%
\pgfsetfillcolor{currentfill}%
\pgfsetlinewidth{1.003750pt}%
\definecolor{currentstroke}{rgb}{0.200000,0.200000,0.800000}%
\pgfsetstrokecolor{currentstroke}%
\pgfsetdash{}{0pt}%
\pgfpathmoveto{\pgfqpoint{3.805779in}{6.573176in}}%
\pgfpathcurveto{\pgfqpoint{3.811603in}{6.573176in}}{\pgfqpoint{3.817189in}{6.575490in}}{\pgfqpoint{3.821307in}{6.579608in}}%
\pgfpathcurveto{\pgfqpoint{3.825426in}{6.583726in}}{\pgfqpoint{3.827740in}{6.589312in}}{\pgfqpoint{3.827740in}{6.595136in}}%
\pgfpathcurveto{\pgfqpoint{3.827740in}{6.600960in}}{\pgfqpoint{3.825426in}{6.606546in}}{\pgfqpoint{3.821307in}{6.610664in}}%
\pgfpathcurveto{\pgfqpoint{3.817189in}{6.614783in}}{\pgfqpoint{3.811603in}{6.617096in}}{\pgfqpoint{3.805779in}{6.617096in}}%
\pgfpathcurveto{\pgfqpoint{3.799955in}{6.617096in}}{\pgfqpoint{3.794369in}{6.614783in}}{\pgfqpoint{3.790251in}{6.610664in}}%
\pgfpathcurveto{\pgfqpoint{3.786133in}{6.606546in}}{\pgfqpoint{3.783819in}{6.600960in}}{\pgfqpoint{3.783819in}{6.595136in}}%
\pgfpathcurveto{\pgfqpoint{3.783819in}{6.589312in}}{\pgfqpoint{3.786133in}{6.583726in}}{\pgfqpoint{3.790251in}{6.579608in}}%
\pgfpathcurveto{\pgfqpoint{3.794369in}{6.575490in}}{\pgfqpoint{3.799955in}{6.573176in}}{\pgfqpoint{3.805779in}{6.573176in}}%
\pgfpathlineto{\pgfqpoint{3.805779in}{6.573176in}}%
\pgfpathclose%
\pgfusepath{stroke,fill}%
\end{pgfscope}%
\begin{pgfscope}%
\pgfpathrectangle{\pgfqpoint{1.000000in}{0.979904in}}{\pgfqpoint{6.200000in}{5.960192in}}%
\pgfusepath{clip}%
\pgfsetbuttcap%
\pgfsetroundjoin%
\definecolor{currentfill}{rgb}{0.200000,0.200000,0.800000}%
\pgfsetfillcolor{currentfill}%
\pgfsetlinewidth{1.003750pt}%
\definecolor{currentstroke}{rgb}{0.200000,0.200000,0.800000}%
\pgfsetstrokecolor{currentstroke}%
\pgfsetdash{}{0pt}%
\pgfpathmoveto{\pgfqpoint{3.747753in}{6.557255in}}%
\pgfpathcurveto{\pgfqpoint{3.753577in}{6.557255in}}{\pgfqpoint{3.759163in}{6.559569in}}{\pgfqpoint{3.763281in}{6.563687in}}%
\pgfpathcurveto{\pgfqpoint{3.767399in}{6.567805in}}{\pgfqpoint{3.769713in}{6.573392in}}{\pgfqpoint{3.769713in}{6.579216in}}%
\pgfpathcurveto{\pgfqpoint{3.769713in}{6.585039in}}{\pgfqpoint{3.767399in}{6.590626in}}{\pgfqpoint{3.763281in}{6.594744in}}%
\pgfpathcurveto{\pgfqpoint{3.759163in}{6.598862in}}{\pgfqpoint{3.753577in}{6.601176in}}{\pgfqpoint{3.747753in}{6.601176in}}%
\pgfpathcurveto{\pgfqpoint{3.741929in}{6.601176in}}{\pgfqpoint{3.736343in}{6.598862in}}{\pgfqpoint{3.732224in}{6.594744in}}%
\pgfpathcurveto{\pgfqpoint{3.728106in}{6.590626in}}{\pgfqpoint{3.725792in}{6.585039in}}{\pgfqpoint{3.725792in}{6.579216in}}%
\pgfpathcurveto{\pgfqpoint{3.725792in}{6.573392in}}{\pgfqpoint{3.728106in}{6.567805in}}{\pgfqpoint{3.732224in}{6.563687in}}%
\pgfpathcurveto{\pgfqpoint{3.736343in}{6.559569in}}{\pgfqpoint{3.741929in}{6.557255in}}{\pgfqpoint{3.747753in}{6.557255in}}%
\pgfpathlineto{\pgfqpoint{3.747753in}{6.557255in}}%
\pgfpathclose%
\pgfusepath{stroke,fill}%
\end{pgfscope}%
\begin{pgfscope}%
\pgfpathrectangle{\pgfqpoint{1.000000in}{0.979904in}}{\pgfqpoint{6.200000in}{5.960192in}}%
\pgfusepath{clip}%
\pgfsetbuttcap%
\pgfsetroundjoin%
\definecolor{currentfill}{rgb}{0.200000,0.200000,0.800000}%
\pgfsetfillcolor{currentfill}%
\pgfsetlinewidth{1.003750pt}%
\definecolor{currentstroke}{rgb}{0.200000,0.200000,0.800000}%
\pgfsetstrokecolor{currentstroke}%
\pgfsetdash{}{0pt}%
\pgfpathmoveto{\pgfqpoint{3.691103in}{6.603755in}}%
\pgfpathcurveto{\pgfqpoint{3.696927in}{6.603755in}}{\pgfqpoint{3.702514in}{6.606069in}}{\pgfqpoint{3.706632in}{6.610187in}}%
\pgfpathcurveto{\pgfqpoint{3.710750in}{6.614305in}}{\pgfqpoint{3.713064in}{6.619891in}}{\pgfqpoint{3.713064in}{6.625715in}}%
\pgfpathcurveto{\pgfqpoint{3.713064in}{6.631539in}}{\pgfqpoint{3.710750in}{6.637125in}}{\pgfqpoint{3.706632in}{6.641243in}}%
\pgfpathcurveto{\pgfqpoint{3.702514in}{6.645362in}}{\pgfqpoint{3.696927in}{6.647675in}}{\pgfqpoint{3.691103in}{6.647675in}}%
\pgfpathcurveto{\pgfqpoint{3.685280in}{6.647675in}}{\pgfqpoint{3.679693in}{6.645362in}}{\pgfqpoint{3.675575in}{6.641243in}}%
\pgfpathcurveto{\pgfqpoint{3.671457in}{6.637125in}}{\pgfqpoint{3.669143in}{6.631539in}}{\pgfqpoint{3.669143in}{6.625715in}}%
\pgfpathcurveto{\pgfqpoint{3.669143in}{6.619891in}}{\pgfqpoint{3.671457in}{6.614305in}}{\pgfqpoint{3.675575in}{6.610187in}}%
\pgfpathcurveto{\pgfqpoint{3.679693in}{6.606069in}}{\pgfqpoint{3.685280in}{6.603755in}}{\pgfqpoint{3.691103in}{6.603755in}}%
\pgfpathlineto{\pgfqpoint{3.691103in}{6.603755in}}%
\pgfpathclose%
\pgfusepath{stroke,fill}%
\end{pgfscope}%
\begin{pgfscope}%
\pgfpathrectangle{\pgfqpoint{1.000000in}{0.979904in}}{\pgfqpoint{6.200000in}{5.960192in}}%
\pgfusepath{clip}%
\pgfsetbuttcap%
\pgfsetroundjoin%
\definecolor{currentfill}{rgb}{0.200000,0.200000,0.800000}%
\pgfsetfillcolor{currentfill}%
\pgfsetlinewidth{1.003750pt}%
\definecolor{currentstroke}{rgb}{0.200000,0.200000,0.800000}%
\pgfsetstrokecolor{currentstroke}%
\pgfsetdash{}{0pt}%
\pgfpathmoveto{\pgfqpoint{3.633907in}{6.581286in}}%
\pgfpathcurveto{\pgfqpoint{3.639731in}{6.581286in}}{\pgfqpoint{3.645317in}{6.583600in}}{\pgfqpoint{3.649435in}{6.587718in}}%
\pgfpathcurveto{\pgfqpoint{3.653553in}{6.591837in}}{\pgfqpoint{3.655867in}{6.597423in}}{\pgfqpoint{3.655867in}{6.603247in}}%
\pgfpathcurveto{\pgfqpoint{3.655867in}{6.609071in}}{\pgfqpoint{3.653553in}{6.614657in}}{\pgfqpoint{3.649435in}{6.618775in}}%
\pgfpathcurveto{\pgfqpoint{3.645317in}{6.622893in}}{\pgfqpoint{3.639731in}{6.625207in}}{\pgfqpoint{3.633907in}{6.625207in}}%
\pgfpathcurveto{\pgfqpoint{3.628083in}{6.625207in}}{\pgfqpoint{3.622496in}{6.622893in}}{\pgfqpoint{3.618378in}{6.618775in}}%
\pgfpathcurveto{\pgfqpoint{3.614260in}{6.614657in}}{\pgfqpoint{3.611946in}{6.609071in}}{\pgfqpoint{3.611946in}{6.603247in}}%
\pgfpathcurveto{\pgfqpoint{3.611946in}{6.597423in}}{\pgfqpoint{3.614260in}{6.591837in}}{\pgfqpoint{3.618378in}{6.587718in}}%
\pgfpathcurveto{\pgfqpoint{3.622496in}{6.583600in}}{\pgfqpoint{3.628083in}{6.581286in}}{\pgfqpoint{3.633907in}{6.581286in}}%
\pgfpathlineto{\pgfqpoint{3.633907in}{6.581286in}}%
\pgfpathclose%
\pgfusepath{stroke,fill}%
\end{pgfscope}%
\begin{pgfscope}%
\pgfpathrectangle{\pgfqpoint{1.000000in}{0.979904in}}{\pgfqpoint{6.200000in}{5.960192in}}%
\pgfusepath{clip}%
\pgfsetbuttcap%
\pgfsetroundjoin%
\definecolor{currentfill}{rgb}{0.200000,0.200000,0.800000}%
\pgfsetfillcolor{currentfill}%
\pgfsetlinewidth{1.003750pt}%
\definecolor{currentstroke}{rgb}{0.200000,0.200000,0.800000}%
\pgfsetstrokecolor{currentstroke}%
\pgfsetdash{}{0pt}%
\pgfpathmoveto{\pgfqpoint{3.572698in}{6.602640in}}%
\pgfpathcurveto{\pgfqpoint{3.578522in}{6.602640in}}{\pgfqpoint{3.584109in}{6.604954in}}{\pgfqpoint{3.588227in}{6.609072in}}%
\pgfpathcurveto{\pgfqpoint{3.592345in}{6.613190in}}{\pgfqpoint{3.594659in}{6.618776in}}{\pgfqpoint{3.594659in}{6.624600in}}%
\pgfpathcurveto{\pgfqpoint{3.594659in}{6.630424in}}{\pgfqpoint{3.592345in}{6.636010in}}{\pgfqpoint{3.588227in}{6.640129in}}%
\pgfpathcurveto{\pgfqpoint{3.584109in}{6.644247in}}{\pgfqpoint{3.578522in}{6.646561in}}{\pgfqpoint{3.572698in}{6.646561in}}%
\pgfpathcurveto{\pgfqpoint{3.566875in}{6.646561in}}{\pgfqpoint{3.561288in}{6.644247in}}{\pgfqpoint{3.557170in}{6.640129in}}%
\pgfpathcurveto{\pgfqpoint{3.553052in}{6.636010in}}{\pgfqpoint{3.550738in}{6.630424in}}{\pgfqpoint{3.550738in}{6.624600in}}%
\pgfpathcurveto{\pgfqpoint{3.550738in}{6.618776in}}{\pgfqpoint{3.553052in}{6.613190in}}{\pgfqpoint{3.557170in}{6.609072in}}%
\pgfpathcurveto{\pgfqpoint{3.561288in}{6.604954in}}{\pgfqpoint{3.566875in}{6.602640in}}{\pgfqpoint{3.572698in}{6.602640in}}%
\pgfpathlineto{\pgfqpoint{3.572698in}{6.602640in}}%
\pgfpathclose%
\pgfusepath{stroke,fill}%
\end{pgfscope}%
\begin{pgfscope}%
\pgfpathrectangle{\pgfqpoint{1.000000in}{0.979904in}}{\pgfqpoint{6.200000in}{5.960192in}}%
\pgfusepath{clip}%
\pgfsetbuttcap%
\pgfsetroundjoin%
\definecolor{currentfill}{rgb}{0.200000,0.200000,0.800000}%
\pgfsetfillcolor{currentfill}%
\pgfsetlinewidth{1.003750pt}%
\definecolor{currentstroke}{rgb}{0.200000,0.200000,0.800000}%
\pgfsetstrokecolor{currentstroke}%
\pgfsetdash{}{0pt}%
\pgfpathmoveto{\pgfqpoint{3.516680in}{6.580705in}}%
\pgfpathcurveto{\pgfqpoint{3.522504in}{6.580705in}}{\pgfqpoint{3.528090in}{6.583019in}}{\pgfqpoint{3.532208in}{6.587137in}}%
\pgfpathcurveto{\pgfqpoint{3.536327in}{6.591255in}}{\pgfqpoint{3.538640in}{6.596841in}}{\pgfqpoint{3.538640in}{6.602665in}}%
\pgfpathcurveto{\pgfqpoint{3.538640in}{6.608489in}}{\pgfqpoint{3.536327in}{6.614075in}}{\pgfqpoint{3.532208in}{6.618193in}}%
\pgfpathcurveto{\pgfqpoint{3.528090in}{6.622311in}}{\pgfqpoint{3.522504in}{6.624625in}}{\pgfqpoint{3.516680in}{6.624625in}}%
\pgfpathcurveto{\pgfqpoint{3.510856in}{6.624625in}}{\pgfqpoint{3.505270in}{6.622311in}}{\pgfqpoint{3.501152in}{6.618193in}}%
\pgfpathcurveto{\pgfqpoint{3.497034in}{6.614075in}}{\pgfqpoint{3.494720in}{6.608489in}}{\pgfqpoint{3.494720in}{6.602665in}}%
\pgfpathcurveto{\pgfqpoint{3.494720in}{6.596841in}}{\pgfqpoint{3.497034in}{6.591255in}}{\pgfqpoint{3.501152in}{6.587137in}}%
\pgfpathcurveto{\pgfqpoint{3.505270in}{6.583019in}}{\pgfqpoint{3.510856in}{6.580705in}}{\pgfqpoint{3.516680in}{6.580705in}}%
\pgfpathlineto{\pgfqpoint{3.516680in}{6.580705in}}%
\pgfpathclose%
\pgfusepath{stroke,fill}%
\end{pgfscope}%
\begin{pgfscope}%
\pgfpathrectangle{\pgfqpoint{1.000000in}{0.979904in}}{\pgfqpoint{6.200000in}{5.960192in}}%
\pgfusepath{clip}%
\pgfsetbuttcap%
\pgfsetroundjoin%
\definecolor{currentfill}{rgb}{0.200000,0.200000,0.800000}%
\pgfsetfillcolor{currentfill}%
\pgfsetlinewidth{1.003750pt}%
\definecolor{currentstroke}{rgb}{0.200000,0.200000,0.800000}%
\pgfsetstrokecolor{currentstroke}%
\pgfsetdash{}{0pt}%
\pgfpathmoveto{\pgfqpoint{3.450984in}{6.598534in}}%
\pgfpathcurveto{\pgfqpoint{3.456808in}{6.598534in}}{\pgfqpoint{3.462394in}{6.600848in}}{\pgfqpoint{3.466512in}{6.604966in}}%
\pgfpathcurveto{\pgfqpoint{3.470630in}{6.609084in}}{\pgfqpoint{3.472944in}{6.614670in}}{\pgfqpoint{3.472944in}{6.620494in}}%
\pgfpathcurveto{\pgfqpoint{3.472944in}{6.626318in}}{\pgfqpoint{3.470630in}{6.631904in}}{\pgfqpoint{3.466512in}{6.636022in}}%
\pgfpathcurveto{\pgfqpoint{3.462394in}{6.640140in}}{\pgfqpoint{3.456808in}{6.642454in}}{\pgfqpoint{3.450984in}{6.642454in}}%
\pgfpathcurveto{\pgfqpoint{3.445160in}{6.642454in}}{\pgfqpoint{3.439574in}{6.640140in}}{\pgfqpoint{3.435456in}{6.636022in}}%
\pgfpathcurveto{\pgfqpoint{3.431338in}{6.631904in}}{\pgfqpoint{3.429024in}{6.626318in}}{\pgfqpoint{3.429024in}{6.620494in}}%
\pgfpathcurveto{\pgfqpoint{3.429024in}{6.614670in}}{\pgfqpoint{3.431338in}{6.609084in}}{\pgfqpoint{3.435456in}{6.604966in}}%
\pgfpathcurveto{\pgfqpoint{3.439574in}{6.600848in}}{\pgfqpoint{3.445160in}{6.598534in}}{\pgfqpoint{3.450984in}{6.598534in}}%
\pgfpathlineto{\pgfqpoint{3.450984in}{6.598534in}}%
\pgfpathclose%
\pgfusepath{stroke,fill}%
\end{pgfscope}%
\begin{pgfscope}%
\pgfpathrectangle{\pgfqpoint{1.000000in}{0.979904in}}{\pgfqpoint{6.200000in}{5.960192in}}%
\pgfusepath{clip}%
\pgfsetbuttcap%
\pgfsetroundjoin%
\definecolor{currentfill}{rgb}{0.200000,0.200000,0.800000}%
\pgfsetfillcolor{currentfill}%
\pgfsetlinewidth{1.003750pt}%
\definecolor{currentstroke}{rgb}{0.200000,0.200000,0.800000}%
\pgfsetstrokecolor{currentstroke}%
\pgfsetdash{}{0pt}%
\pgfpathmoveto{\pgfqpoint{3.421748in}{6.497563in}}%
\pgfpathcurveto{\pgfqpoint{3.427572in}{6.497563in}}{\pgfqpoint{3.433158in}{6.499877in}}{\pgfqpoint{3.437276in}{6.503995in}}%
\pgfpathcurveto{\pgfqpoint{3.441395in}{6.508113in}}{\pgfqpoint{3.443708in}{6.513699in}}{\pgfqpoint{3.443708in}{6.519523in}}%
\pgfpathcurveto{\pgfqpoint{3.443708in}{6.525347in}}{\pgfqpoint{3.441395in}{6.530933in}}{\pgfqpoint{3.437276in}{6.535052in}}%
\pgfpathcurveto{\pgfqpoint{3.433158in}{6.539170in}}{\pgfqpoint{3.427572in}{6.541484in}}{\pgfqpoint{3.421748in}{6.541484in}}%
\pgfpathcurveto{\pgfqpoint{3.415924in}{6.541484in}}{\pgfqpoint{3.410338in}{6.539170in}}{\pgfqpoint{3.406220in}{6.535052in}}%
\pgfpathcurveto{\pgfqpoint{3.402102in}{6.530933in}}{\pgfqpoint{3.399788in}{6.525347in}}{\pgfqpoint{3.399788in}{6.519523in}}%
\pgfpathcurveto{\pgfqpoint{3.399788in}{6.513699in}}{\pgfqpoint{3.402102in}{6.508113in}}{\pgfqpoint{3.406220in}{6.503995in}}%
\pgfpathcurveto{\pgfqpoint{3.410338in}{6.499877in}}{\pgfqpoint{3.415924in}{6.497563in}}{\pgfqpoint{3.421748in}{6.497563in}}%
\pgfpathlineto{\pgfqpoint{3.421748in}{6.497563in}}%
\pgfpathclose%
\pgfusepath{stroke,fill}%
\end{pgfscope}%
\begin{pgfscope}%
\pgfpathrectangle{\pgfqpoint{1.000000in}{0.979904in}}{\pgfqpoint{6.200000in}{5.960192in}}%
\pgfusepath{clip}%
\pgfsetbuttcap%
\pgfsetroundjoin%
\definecolor{currentfill}{rgb}{0.200000,0.200000,0.800000}%
\pgfsetfillcolor{currentfill}%
\pgfsetlinewidth{1.003750pt}%
\definecolor{currentstroke}{rgb}{0.200000,0.200000,0.800000}%
\pgfsetstrokecolor{currentstroke}%
\pgfsetdash{}{0pt}%
\pgfpathmoveto{\pgfqpoint{3.321948in}{6.593244in}}%
\pgfpathcurveto{\pgfqpoint{3.327772in}{6.593244in}}{\pgfqpoint{3.333359in}{6.595558in}}{\pgfqpoint{3.337477in}{6.599676in}}%
\pgfpathcurveto{\pgfqpoint{3.341595in}{6.603794in}}{\pgfqpoint{3.343909in}{6.609381in}}{\pgfqpoint{3.343909in}{6.615204in}}%
\pgfpathcurveto{\pgfqpoint{3.343909in}{6.621028in}}{\pgfqpoint{3.341595in}{6.626615in}}{\pgfqpoint{3.337477in}{6.630733in}}%
\pgfpathcurveto{\pgfqpoint{3.333359in}{6.634851in}}{\pgfqpoint{3.327772in}{6.637165in}}{\pgfqpoint{3.321948in}{6.637165in}}%
\pgfpathcurveto{\pgfqpoint{3.316125in}{6.637165in}}{\pgfqpoint{3.310538in}{6.634851in}}{\pgfqpoint{3.306420in}{6.630733in}}%
\pgfpathcurveto{\pgfqpoint{3.302302in}{6.626615in}}{\pgfqpoint{3.299988in}{6.621028in}}{\pgfqpoint{3.299988in}{6.615204in}}%
\pgfpathcurveto{\pgfqpoint{3.299988in}{6.609381in}}{\pgfqpoint{3.302302in}{6.603794in}}{\pgfqpoint{3.306420in}{6.599676in}}%
\pgfpathcurveto{\pgfqpoint{3.310538in}{6.595558in}}{\pgfqpoint{3.316125in}{6.593244in}}{\pgfqpoint{3.321948in}{6.593244in}}%
\pgfpathlineto{\pgfqpoint{3.321948in}{6.593244in}}%
\pgfpathclose%
\pgfusepath{stroke,fill}%
\end{pgfscope}%
\begin{pgfscope}%
\pgfpathrectangle{\pgfqpoint{1.000000in}{0.979904in}}{\pgfqpoint{6.200000in}{5.960192in}}%
\pgfusepath{clip}%
\pgfsetbuttcap%
\pgfsetroundjoin%
\definecolor{currentfill}{rgb}{0.200000,0.200000,0.800000}%
\pgfsetfillcolor{currentfill}%
\pgfsetlinewidth{1.003750pt}%
\definecolor{currentstroke}{rgb}{0.200000,0.200000,0.800000}%
\pgfsetstrokecolor{currentstroke}%
\pgfsetdash{}{0pt}%
\pgfpathmoveto{\pgfqpoint{3.224867in}{6.647218in}}%
\pgfpathcurveto{\pgfqpoint{3.230691in}{6.647218in}}{\pgfqpoint{3.236277in}{6.649532in}}{\pgfqpoint{3.240395in}{6.653650in}}%
\pgfpathcurveto{\pgfqpoint{3.244513in}{6.657768in}}{\pgfqpoint{3.246827in}{6.663354in}}{\pgfqpoint{3.246827in}{6.669178in}}%
\pgfpathcurveto{\pgfqpoint{3.246827in}{6.675002in}}{\pgfqpoint{3.244513in}{6.680588in}}{\pgfqpoint{3.240395in}{6.684707in}}%
\pgfpathcurveto{\pgfqpoint{3.236277in}{6.688825in}}{\pgfqpoint{3.230691in}{6.691139in}}{\pgfqpoint{3.224867in}{6.691139in}}%
\pgfpathcurveto{\pgfqpoint{3.219043in}{6.691139in}}{\pgfqpoint{3.213457in}{6.688825in}}{\pgfqpoint{3.209339in}{6.684707in}}%
\pgfpathcurveto{\pgfqpoint{3.205220in}{6.680588in}}{\pgfqpoint{3.202907in}{6.675002in}}{\pgfqpoint{3.202907in}{6.669178in}}%
\pgfpathcurveto{\pgfqpoint{3.202907in}{6.663354in}}{\pgfqpoint{3.205220in}{6.657768in}}{\pgfqpoint{3.209339in}{6.653650in}}%
\pgfpathcurveto{\pgfqpoint{3.213457in}{6.649532in}}{\pgfqpoint{3.219043in}{6.647218in}}{\pgfqpoint{3.224867in}{6.647218in}}%
\pgfpathlineto{\pgfqpoint{3.224867in}{6.647218in}}%
\pgfpathclose%
\pgfusepath{stroke,fill}%
\end{pgfscope}%
\begin{pgfscope}%
\pgfpathrectangle{\pgfqpoint{1.000000in}{0.979904in}}{\pgfqpoint{6.200000in}{5.960192in}}%
\pgfusepath{clip}%
\pgfsetbuttcap%
\pgfsetroundjoin%
\definecolor{currentfill}{rgb}{0.200000,0.200000,0.800000}%
\pgfsetfillcolor{currentfill}%
\pgfsetlinewidth{1.003750pt}%
\definecolor{currentstroke}{rgb}{0.200000,0.200000,0.800000}%
\pgfsetstrokecolor{currentstroke}%
\pgfsetdash{}{0pt}%
\pgfpathmoveto{\pgfqpoint{3.206327in}{6.541948in}}%
\pgfpathcurveto{\pgfqpoint{3.212151in}{6.541948in}}{\pgfqpoint{3.217737in}{6.544262in}}{\pgfqpoint{3.221855in}{6.548380in}}%
\pgfpathcurveto{\pgfqpoint{3.225973in}{6.552498in}}{\pgfqpoint{3.228287in}{6.558084in}}{\pgfqpoint{3.228287in}{6.563908in}}%
\pgfpathcurveto{\pgfqpoint{3.228287in}{6.569732in}}{\pgfqpoint{3.225973in}{6.575318in}}{\pgfqpoint{3.221855in}{6.579436in}}%
\pgfpathcurveto{\pgfqpoint{3.217737in}{6.583554in}}{\pgfqpoint{3.212151in}{6.585868in}}{\pgfqpoint{3.206327in}{6.585868in}}%
\pgfpathcurveto{\pgfqpoint{3.200503in}{6.585868in}}{\pgfqpoint{3.194917in}{6.583554in}}{\pgfqpoint{3.190799in}{6.579436in}}%
\pgfpathcurveto{\pgfqpoint{3.186681in}{6.575318in}}{\pgfqpoint{3.184367in}{6.569732in}}{\pgfqpoint{3.184367in}{6.563908in}}%
\pgfpathcurveto{\pgfqpoint{3.184367in}{6.558084in}}{\pgfqpoint{3.186681in}{6.552498in}}{\pgfqpoint{3.190799in}{6.548380in}}%
\pgfpathcurveto{\pgfqpoint{3.194917in}{6.544262in}}{\pgfqpoint{3.200503in}{6.541948in}}{\pgfqpoint{3.206327in}{6.541948in}}%
\pgfpathlineto{\pgfqpoint{3.206327in}{6.541948in}}%
\pgfpathclose%
\pgfusepath{stroke,fill}%
\end{pgfscope}%
\begin{pgfscope}%
\pgfpathrectangle{\pgfqpoint{1.000000in}{0.979904in}}{\pgfqpoint{6.200000in}{5.960192in}}%
\pgfusepath{clip}%
\pgfsetbuttcap%
\pgfsetroundjoin%
\definecolor{currentfill}{rgb}{0.200000,0.200000,0.800000}%
\pgfsetfillcolor{currentfill}%
\pgfsetlinewidth{1.003750pt}%
\definecolor{currentstroke}{rgb}{0.200000,0.200000,0.800000}%
\pgfsetstrokecolor{currentstroke}%
\pgfsetdash{}{0pt}%
\pgfpathmoveto{\pgfqpoint{3.181117in}{6.465457in}}%
\pgfpathcurveto{\pgfqpoint{3.186941in}{6.465457in}}{\pgfqpoint{3.192527in}{6.467771in}}{\pgfqpoint{3.196645in}{6.471889in}}%
\pgfpathcurveto{\pgfqpoint{3.200763in}{6.476008in}}{\pgfqpoint{3.203077in}{6.481594in}}{\pgfqpoint{3.203077in}{6.487418in}}%
\pgfpathcurveto{\pgfqpoint{3.203077in}{6.493242in}}{\pgfqpoint{3.200763in}{6.498828in}}{\pgfqpoint{3.196645in}{6.502946in}}%
\pgfpathcurveto{\pgfqpoint{3.192527in}{6.507064in}}{\pgfqpoint{3.186941in}{6.509378in}}{\pgfqpoint{3.181117in}{6.509378in}}%
\pgfpathcurveto{\pgfqpoint{3.175293in}{6.509378in}}{\pgfqpoint{3.169707in}{6.507064in}}{\pgfqpoint{3.165589in}{6.502946in}}%
\pgfpathcurveto{\pgfqpoint{3.161471in}{6.498828in}}{\pgfqpoint{3.159157in}{6.493242in}}{\pgfqpoint{3.159157in}{6.487418in}}%
\pgfpathcurveto{\pgfqpoint{3.159157in}{6.481594in}}{\pgfqpoint{3.161471in}{6.476008in}}{\pgfqpoint{3.165589in}{6.471889in}}%
\pgfpathcurveto{\pgfqpoint{3.169707in}{6.467771in}}{\pgfqpoint{3.175293in}{6.465457in}}{\pgfqpoint{3.181117in}{6.465457in}}%
\pgfpathlineto{\pgfqpoint{3.181117in}{6.465457in}}%
\pgfpathclose%
\pgfusepath{stroke,fill}%
\end{pgfscope}%
\begin{pgfscope}%
\pgfpathrectangle{\pgfqpoint{1.000000in}{0.979904in}}{\pgfqpoint{6.200000in}{5.960192in}}%
\pgfusepath{clip}%
\pgfsetbuttcap%
\pgfsetroundjoin%
\definecolor{currentfill}{rgb}{0.200000,0.200000,0.800000}%
\pgfsetfillcolor{currentfill}%
\pgfsetlinewidth{1.003750pt}%
\definecolor{currentstroke}{rgb}{0.200000,0.200000,0.800000}%
\pgfsetstrokecolor{currentstroke}%
\pgfsetdash{}{0pt}%
\pgfpathmoveto{\pgfqpoint{3.110847in}{6.458589in}}%
\pgfpathcurveto{\pgfqpoint{3.116671in}{6.458589in}}{\pgfqpoint{3.122257in}{6.460903in}}{\pgfqpoint{3.126376in}{6.465021in}}%
\pgfpathcurveto{\pgfqpoint{3.130494in}{6.469139in}}{\pgfqpoint{3.132808in}{6.474725in}}{\pgfqpoint{3.132808in}{6.480549in}}%
\pgfpathcurveto{\pgfqpoint{3.132808in}{6.486373in}}{\pgfqpoint{3.130494in}{6.491959in}}{\pgfqpoint{3.126376in}{6.496077in}}%
\pgfpathcurveto{\pgfqpoint{3.122257in}{6.500195in}}{\pgfqpoint{3.116671in}{6.502509in}}{\pgfqpoint{3.110847in}{6.502509in}}%
\pgfpathcurveto{\pgfqpoint{3.105023in}{6.502509in}}{\pgfqpoint{3.099437in}{6.500195in}}{\pgfqpoint{3.095319in}{6.496077in}}%
\pgfpathcurveto{\pgfqpoint{3.091201in}{6.491959in}}{\pgfqpoint{3.088887in}{6.486373in}}{\pgfqpoint{3.088887in}{6.480549in}}%
\pgfpathcurveto{\pgfqpoint{3.088887in}{6.474725in}}{\pgfqpoint{3.091201in}{6.469139in}}{\pgfqpoint{3.095319in}{6.465021in}}%
\pgfpathcurveto{\pgfqpoint{3.099437in}{6.460903in}}{\pgfqpoint{3.105023in}{6.458589in}}{\pgfqpoint{3.110847in}{6.458589in}}%
\pgfpathlineto{\pgfqpoint{3.110847in}{6.458589in}}%
\pgfpathclose%
\pgfusepath{stroke,fill}%
\end{pgfscope}%
\begin{pgfscope}%
\pgfpathrectangle{\pgfqpoint{1.000000in}{0.979904in}}{\pgfqpoint{6.200000in}{5.960192in}}%
\pgfusepath{clip}%
\pgfsetbuttcap%
\pgfsetroundjoin%
\definecolor{currentfill}{rgb}{0.200000,0.200000,0.800000}%
\pgfsetfillcolor{currentfill}%
\pgfsetlinewidth{1.003750pt}%
\definecolor{currentstroke}{rgb}{0.200000,0.200000,0.800000}%
\pgfsetstrokecolor{currentstroke}%
\pgfsetdash{}{0pt}%
\pgfpathmoveto{\pgfqpoint{3.128159in}{6.343462in}}%
\pgfpathcurveto{\pgfqpoint{3.133983in}{6.343462in}}{\pgfqpoint{3.139569in}{6.345775in}}{\pgfqpoint{3.143687in}{6.349894in}}%
\pgfpathcurveto{\pgfqpoint{3.147805in}{6.354012in}}{\pgfqpoint{3.150119in}{6.359598in}}{\pgfqpoint{3.150119in}{6.365422in}}%
\pgfpathcurveto{\pgfqpoint{3.150119in}{6.371246in}}{\pgfqpoint{3.147805in}{6.376832in}}{\pgfqpoint{3.143687in}{6.380950in}}%
\pgfpathcurveto{\pgfqpoint{3.139569in}{6.385068in}}{\pgfqpoint{3.133983in}{6.387382in}}{\pgfqpoint{3.128159in}{6.387382in}}%
\pgfpathcurveto{\pgfqpoint{3.122335in}{6.387382in}}{\pgfqpoint{3.116749in}{6.385068in}}{\pgfqpoint{3.112631in}{6.380950in}}%
\pgfpathcurveto{\pgfqpoint{3.108512in}{6.376832in}}{\pgfqpoint{3.106199in}{6.371246in}}{\pgfqpoint{3.106199in}{6.365422in}}%
\pgfpathcurveto{\pgfqpoint{3.106199in}{6.359598in}}{\pgfqpoint{3.108512in}{6.354012in}}{\pgfqpoint{3.112631in}{6.349894in}}%
\pgfpathcurveto{\pgfqpoint{3.116749in}{6.345775in}}{\pgfqpoint{3.122335in}{6.343462in}}{\pgfqpoint{3.128159in}{6.343462in}}%
\pgfpathlineto{\pgfqpoint{3.128159in}{6.343462in}}%
\pgfpathclose%
\pgfusepath{stroke,fill}%
\end{pgfscope}%
\begin{pgfscope}%
\pgfpathrectangle{\pgfqpoint{1.000000in}{0.979904in}}{\pgfqpoint{6.200000in}{5.960192in}}%
\pgfusepath{clip}%
\pgfsetbuttcap%
\pgfsetroundjoin%
\definecolor{currentfill}{rgb}{0.200000,0.200000,0.800000}%
\pgfsetfillcolor{currentfill}%
\pgfsetlinewidth{1.003750pt}%
\definecolor{currentstroke}{rgb}{0.200000,0.200000,0.800000}%
\pgfsetstrokecolor{currentstroke}%
\pgfsetdash{}{0pt}%
\pgfpathmoveto{\pgfqpoint{3.108411in}{6.283770in}}%
\pgfpathcurveto{\pgfqpoint{3.114235in}{6.283770in}}{\pgfqpoint{3.119821in}{6.286084in}}{\pgfqpoint{3.123940in}{6.290202in}}%
\pgfpathcurveto{\pgfqpoint{3.128058in}{6.294320in}}{\pgfqpoint{3.130372in}{6.299906in}}{\pgfqpoint{3.130372in}{6.305730in}}%
\pgfpathcurveto{\pgfqpoint{3.130372in}{6.311554in}}{\pgfqpoint{3.128058in}{6.317140in}}{\pgfqpoint{3.123940in}{6.321258in}}%
\pgfpathcurveto{\pgfqpoint{3.119821in}{6.325376in}}{\pgfqpoint{3.114235in}{6.327690in}}{\pgfqpoint{3.108411in}{6.327690in}}%
\pgfpathcurveto{\pgfqpoint{3.102587in}{6.327690in}}{\pgfqpoint{3.097001in}{6.325376in}}{\pgfqpoint{3.092883in}{6.321258in}}%
\pgfpathcurveto{\pgfqpoint{3.088765in}{6.317140in}}{\pgfqpoint{3.086451in}{6.311554in}}{\pgfqpoint{3.086451in}{6.305730in}}%
\pgfpathcurveto{\pgfqpoint{3.086451in}{6.299906in}}{\pgfqpoint{3.088765in}{6.294320in}}{\pgfqpoint{3.092883in}{6.290202in}}%
\pgfpathcurveto{\pgfqpoint{3.097001in}{6.286084in}}{\pgfqpoint{3.102587in}{6.283770in}}{\pgfqpoint{3.108411in}{6.283770in}}%
\pgfpathlineto{\pgfqpoint{3.108411in}{6.283770in}}%
\pgfpathclose%
\pgfusepath{stroke,fill}%
\end{pgfscope}%
\begin{pgfscope}%
\pgfpathrectangle{\pgfqpoint{1.000000in}{0.979904in}}{\pgfqpoint{6.200000in}{5.960192in}}%
\pgfusepath{clip}%
\pgfsetbuttcap%
\pgfsetroundjoin%
\definecolor{currentfill}{rgb}{0.200000,0.200000,0.800000}%
\pgfsetfillcolor{currentfill}%
\pgfsetlinewidth{1.003750pt}%
\definecolor{currentstroke}{rgb}{0.200000,0.200000,0.800000}%
\pgfsetstrokecolor{currentstroke}%
\pgfsetdash{}{0pt}%
\pgfpathmoveto{\pgfqpoint{2.977920in}{6.328040in}}%
\pgfpathcurveto{\pgfqpoint{2.983744in}{6.328040in}}{\pgfqpoint{2.989330in}{6.330354in}}{\pgfqpoint{2.993448in}{6.334472in}}%
\pgfpathcurveto{\pgfqpoint{2.997566in}{6.338591in}}{\pgfqpoint{2.999880in}{6.344177in}}{\pgfqpoint{2.999880in}{6.350001in}}%
\pgfpathcurveto{\pgfqpoint{2.999880in}{6.355825in}}{\pgfqpoint{2.997566in}{6.361411in}}{\pgfqpoint{2.993448in}{6.365529in}}%
\pgfpathcurveto{\pgfqpoint{2.989330in}{6.369647in}}{\pgfqpoint{2.983744in}{6.371961in}}{\pgfqpoint{2.977920in}{6.371961in}}%
\pgfpathcurveto{\pgfqpoint{2.972096in}{6.371961in}}{\pgfqpoint{2.966510in}{6.369647in}}{\pgfqpoint{2.962392in}{6.365529in}}%
\pgfpathcurveto{\pgfqpoint{2.958274in}{6.361411in}}{\pgfqpoint{2.955960in}{6.355825in}}{\pgfqpoint{2.955960in}{6.350001in}}%
\pgfpathcurveto{\pgfqpoint{2.955960in}{6.344177in}}{\pgfqpoint{2.958274in}{6.338591in}}{\pgfqpoint{2.962392in}{6.334472in}}%
\pgfpathcurveto{\pgfqpoint{2.966510in}{6.330354in}}{\pgfqpoint{2.972096in}{6.328040in}}{\pgfqpoint{2.977920in}{6.328040in}}%
\pgfpathlineto{\pgfqpoint{2.977920in}{6.328040in}}%
\pgfpathclose%
\pgfusepath{stroke,fill}%
\end{pgfscope}%
\begin{pgfscope}%
\pgfpathrectangle{\pgfqpoint{1.000000in}{0.979904in}}{\pgfqpoint{6.200000in}{5.960192in}}%
\pgfusepath{clip}%
\pgfsetbuttcap%
\pgfsetroundjoin%
\definecolor{currentfill}{rgb}{0.200000,0.200000,0.800000}%
\pgfsetfillcolor{currentfill}%
\pgfsetlinewidth{1.003750pt}%
\definecolor{currentstroke}{rgb}{0.200000,0.200000,0.800000}%
\pgfsetstrokecolor{currentstroke}%
\pgfsetdash{}{0pt}%
\pgfpathmoveto{\pgfqpoint{2.992793in}{6.237538in}}%
\pgfpathcurveto{\pgfqpoint{2.998617in}{6.237538in}}{\pgfqpoint{3.004203in}{6.239852in}}{\pgfqpoint{3.008321in}{6.243970in}}%
\pgfpathcurveto{\pgfqpoint{3.012439in}{6.248089in}}{\pgfqpoint{3.014753in}{6.253675in}}{\pgfqpoint{3.014753in}{6.259499in}}%
\pgfpathcurveto{\pgfqpoint{3.014753in}{6.265323in}}{\pgfqpoint{3.012439in}{6.270909in}}{\pgfqpoint{3.008321in}{6.275027in}}%
\pgfpathcurveto{\pgfqpoint{3.004203in}{6.279145in}}{\pgfqpoint{2.998617in}{6.281459in}}{\pgfqpoint{2.992793in}{6.281459in}}%
\pgfpathcurveto{\pgfqpoint{2.986969in}{6.281459in}}{\pgfqpoint{2.981383in}{6.279145in}}{\pgfqpoint{2.977265in}{6.275027in}}%
\pgfpathcurveto{\pgfqpoint{2.973147in}{6.270909in}}{\pgfqpoint{2.970833in}{6.265323in}}{\pgfqpoint{2.970833in}{6.259499in}}%
\pgfpathcurveto{\pgfqpoint{2.970833in}{6.253675in}}{\pgfqpoint{2.973147in}{6.248089in}}{\pgfqpoint{2.977265in}{6.243970in}}%
\pgfpathcurveto{\pgfqpoint{2.981383in}{6.239852in}}{\pgfqpoint{2.986969in}{6.237538in}}{\pgfqpoint{2.992793in}{6.237538in}}%
\pgfpathlineto{\pgfqpoint{2.992793in}{6.237538in}}%
\pgfpathclose%
\pgfusepath{stroke,fill}%
\end{pgfscope}%
\begin{pgfscope}%
\pgfpathrectangle{\pgfqpoint{1.000000in}{0.979904in}}{\pgfqpoint{6.200000in}{5.960192in}}%
\pgfusepath{clip}%
\pgfsetbuttcap%
\pgfsetroundjoin%
\definecolor{currentfill}{rgb}{0.200000,0.200000,0.800000}%
\pgfsetfillcolor{currentfill}%
\pgfsetlinewidth{1.003750pt}%
\definecolor{currentstroke}{rgb}{0.200000,0.200000,0.800000}%
\pgfsetstrokecolor{currentstroke}%
\pgfsetdash{}{0pt}%
\pgfpathmoveto{\pgfqpoint{2.994000in}{6.166861in}}%
\pgfpathcurveto{\pgfqpoint{2.999823in}{6.166861in}}{\pgfqpoint{3.005410in}{6.169175in}}{\pgfqpoint{3.009528in}{6.173293in}}%
\pgfpathcurveto{\pgfqpoint{3.013646in}{6.177411in}}{\pgfqpoint{3.015960in}{6.182998in}}{\pgfqpoint{3.015960in}{6.188821in}}%
\pgfpathcurveto{\pgfqpoint{3.015960in}{6.194645in}}{\pgfqpoint{3.013646in}{6.200232in}}{\pgfqpoint{3.009528in}{6.204350in}}%
\pgfpathcurveto{\pgfqpoint{3.005410in}{6.208468in}}{\pgfqpoint{2.999823in}{6.210782in}}{\pgfqpoint{2.994000in}{6.210782in}}%
\pgfpathcurveto{\pgfqpoint{2.988176in}{6.210782in}}{\pgfqpoint{2.982589in}{6.208468in}}{\pgfqpoint{2.978471in}{6.204350in}}%
\pgfpathcurveto{\pgfqpoint{2.974353in}{6.200232in}}{\pgfqpoint{2.972039in}{6.194645in}}{\pgfqpoint{2.972039in}{6.188821in}}%
\pgfpathcurveto{\pgfqpoint{2.972039in}{6.182998in}}{\pgfqpoint{2.974353in}{6.177411in}}{\pgfqpoint{2.978471in}{6.173293in}}%
\pgfpathcurveto{\pgfqpoint{2.982589in}{6.169175in}}{\pgfqpoint{2.988176in}{6.166861in}}{\pgfqpoint{2.994000in}{6.166861in}}%
\pgfpathlineto{\pgfqpoint{2.994000in}{6.166861in}}%
\pgfpathclose%
\pgfusepath{stroke,fill}%
\end{pgfscope}%
\begin{pgfscope}%
\pgfpathrectangle{\pgfqpoint{1.000000in}{0.979904in}}{\pgfqpoint{6.200000in}{5.960192in}}%
\pgfusepath{clip}%
\pgfsetbuttcap%
\pgfsetroundjoin%
\definecolor{currentfill}{rgb}{0.200000,0.200000,0.800000}%
\pgfsetfillcolor{currentfill}%
\pgfsetlinewidth{1.003750pt}%
\definecolor{currentstroke}{rgb}{0.200000,0.200000,0.800000}%
\pgfsetstrokecolor{currentstroke}%
\pgfsetdash{}{0pt}%
\pgfpathmoveto{\pgfqpoint{2.943135in}{6.133431in}}%
\pgfpathcurveto{\pgfqpoint{2.948959in}{6.133431in}}{\pgfqpoint{2.954545in}{6.135745in}}{\pgfqpoint{2.958663in}{6.139863in}}%
\pgfpathcurveto{\pgfqpoint{2.962781in}{6.143982in}}{\pgfqpoint{2.965095in}{6.149568in}}{\pgfqpoint{2.965095in}{6.155392in}}%
\pgfpathcurveto{\pgfqpoint{2.965095in}{6.161216in}}{\pgfqpoint{2.962781in}{6.166802in}}{\pgfqpoint{2.958663in}{6.170920in}}%
\pgfpathcurveto{\pgfqpoint{2.954545in}{6.175038in}}{\pgfqpoint{2.948959in}{6.177352in}}{\pgfqpoint{2.943135in}{6.177352in}}%
\pgfpathcurveto{\pgfqpoint{2.937311in}{6.177352in}}{\pgfqpoint{2.931725in}{6.175038in}}{\pgfqpoint{2.927607in}{6.170920in}}%
\pgfpathcurveto{\pgfqpoint{2.923488in}{6.166802in}}{\pgfqpoint{2.921175in}{6.161216in}}{\pgfqpoint{2.921175in}{6.155392in}}%
\pgfpathcurveto{\pgfqpoint{2.921175in}{6.149568in}}{\pgfqpoint{2.923488in}{6.143982in}}{\pgfqpoint{2.927607in}{6.139863in}}%
\pgfpathcurveto{\pgfqpoint{2.931725in}{6.135745in}}{\pgfqpoint{2.937311in}{6.133431in}}{\pgfqpoint{2.943135in}{6.133431in}}%
\pgfpathlineto{\pgfqpoint{2.943135in}{6.133431in}}%
\pgfpathclose%
\pgfusepath{stroke,fill}%
\end{pgfscope}%
\begin{pgfscope}%
\pgfpathrectangle{\pgfqpoint{1.000000in}{0.979904in}}{\pgfqpoint{6.200000in}{5.960192in}}%
\pgfusepath{clip}%
\pgfsetbuttcap%
\pgfsetroundjoin%
\definecolor{currentfill}{rgb}{0.200000,0.200000,0.800000}%
\pgfsetfillcolor{currentfill}%
\pgfsetlinewidth{1.003750pt}%
\definecolor{currentstroke}{rgb}{0.200000,0.200000,0.800000}%
\pgfsetstrokecolor{currentstroke}%
\pgfsetdash{}{0pt}%
\pgfpathmoveto{\pgfqpoint{2.927498in}{6.078063in}}%
\pgfpathcurveto{\pgfqpoint{2.933322in}{6.078063in}}{\pgfqpoint{2.938908in}{6.080377in}}{\pgfqpoint{2.943026in}{6.084495in}}%
\pgfpathcurveto{\pgfqpoint{2.947144in}{6.088613in}}{\pgfqpoint{2.949458in}{6.094199in}}{\pgfqpoint{2.949458in}{6.100023in}}%
\pgfpathcurveto{\pgfqpoint{2.949458in}{6.105847in}}{\pgfqpoint{2.947144in}{6.111433in}}{\pgfqpoint{2.943026in}{6.115552in}}%
\pgfpathcurveto{\pgfqpoint{2.938908in}{6.119670in}}{\pgfqpoint{2.933322in}{6.121984in}}{\pgfqpoint{2.927498in}{6.121984in}}%
\pgfpathcurveto{\pgfqpoint{2.921674in}{6.121984in}}{\pgfqpoint{2.916088in}{6.119670in}}{\pgfqpoint{2.911969in}{6.115552in}}%
\pgfpathcurveto{\pgfqpoint{2.907851in}{6.111433in}}{\pgfqpoint{2.905537in}{6.105847in}}{\pgfqpoint{2.905537in}{6.100023in}}%
\pgfpathcurveto{\pgfqpoint{2.905537in}{6.094199in}}{\pgfqpoint{2.907851in}{6.088613in}}{\pgfqpoint{2.911969in}{6.084495in}}%
\pgfpathcurveto{\pgfqpoint{2.916088in}{6.080377in}}{\pgfqpoint{2.921674in}{6.078063in}}{\pgfqpoint{2.927498in}{6.078063in}}%
\pgfpathlineto{\pgfqpoint{2.927498in}{6.078063in}}%
\pgfpathclose%
\pgfusepath{stroke,fill}%
\end{pgfscope}%
\begin{pgfscope}%
\pgfpathrectangle{\pgfqpoint{1.000000in}{0.979904in}}{\pgfqpoint{6.200000in}{5.960192in}}%
\pgfusepath{clip}%
\pgfsetbuttcap%
\pgfsetroundjoin%
\definecolor{currentfill}{rgb}{0.200000,0.200000,0.800000}%
\pgfsetfillcolor{currentfill}%
\pgfsetlinewidth{1.003750pt}%
\definecolor{currentstroke}{rgb}{0.200000,0.200000,0.800000}%
\pgfsetstrokecolor{currentstroke}%
\pgfsetdash{}{0pt}%
\pgfpathmoveto{\pgfqpoint{2.842131in}{6.054800in}}%
\pgfpathcurveto{\pgfqpoint{2.847955in}{6.054800in}}{\pgfqpoint{2.853541in}{6.057114in}}{\pgfqpoint{2.857659in}{6.061232in}}%
\pgfpathcurveto{\pgfqpoint{2.861777in}{6.065350in}}{\pgfqpoint{2.864091in}{6.070937in}}{\pgfqpoint{2.864091in}{6.076760in}}%
\pgfpathcurveto{\pgfqpoint{2.864091in}{6.082584in}}{\pgfqpoint{2.861777in}{6.088171in}}{\pgfqpoint{2.857659in}{6.092289in}}%
\pgfpathcurveto{\pgfqpoint{2.853541in}{6.096407in}}{\pgfqpoint{2.847955in}{6.098721in}}{\pgfqpoint{2.842131in}{6.098721in}}%
\pgfpathcurveto{\pgfqpoint{2.836307in}{6.098721in}}{\pgfqpoint{2.830721in}{6.096407in}}{\pgfqpoint{2.826602in}{6.092289in}}%
\pgfpathcurveto{\pgfqpoint{2.822484in}{6.088171in}}{\pgfqpoint{2.820170in}{6.082584in}}{\pgfqpoint{2.820170in}{6.076760in}}%
\pgfpathcurveto{\pgfqpoint{2.820170in}{6.070937in}}{\pgfqpoint{2.822484in}{6.065350in}}{\pgfqpoint{2.826602in}{6.061232in}}%
\pgfpathcurveto{\pgfqpoint{2.830721in}{6.057114in}}{\pgfqpoint{2.836307in}{6.054800in}}{\pgfqpoint{2.842131in}{6.054800in}}%
\pgfpathlineto{\pgfqpoint{2.842131in}{6.054800in}}%
\pgfpathclose%
\pgfusepath{stroke,fill}%
\end{pgfscope}%
\begin{pgfscope}%
\pgfpathrectangle{\pgfqpoint{1.000000in}{0.979904in}}{\pgfqpoint{6.200000in}{5.960192in}}%
\pgfusepath{clip}%
\pgfsetbuttcap%
\pgfsetroundjoin%
\definecolor{currentfill}{rgb}{0.200000,0.200000,0.800000}%
\pgfsetfillcolor{currentfill}%
\pgfsetlinewidth{1.003750pt}%
\definecolor{currentstroke}{rgb}{0.200000,0.200000,0.800000}%
\pgfsetstrokecolor{currentstroke}%
\pgfsetdash{}{0pt}%
\pgfpathmoveto{\pgfqpoint{2.797576in}{6.007385in}}%
\pgfpathcurveto{\pgfqpoint{2.803399in}{6.007385in}}{\pgfqpoint{2.808986in}{6.009699in}}{\pgfqpoint{2.813104in}{6.013817in}}%
\pgfpathcurveto{\pgfqpoint{2.817222in}{6.017935in}}{\pgfqpoint{2.819536in}{6.023521in}}{\pgfqpoint{2.819536in}{6.029345in}}%
\pgfpathcurveto{\pgfqpoint{2.819536in}{6.035169in}}{\pgfqpoint{2.817222in}{6.040756in}}{\pgfqpoint{2.813104in}{6.044874in}}%
\pgfpathcurveto{\pgfqpoint{2.808986in}{6.048992in}}{\pgfqpoint{2.803399in}{6.051306in}}{\pgfqpoint{2.797576in}{6.051306in}}%
\pgfpathcurveto{\pgfqpoint{2.791752in}{6.051306in}}{\pgfqpoint{2.786165in}{6.048992in}}{\pgfqpoint{2.782047in}{6.044874in}}%
\pgfpathcurveto{\pgfqpoint{2.777929in}{6.040756in}}{\pgfqpoint{2.775615in}{6.035169in}}{\pgfqpoint{2.775615in}{6.029345in}}%
\pgfpathcurveto{\pgfqpoint{2.775615in}{6.023521in}}{\pgfqpoint{2.777929in}{6.017935in}}{\pgfqpoint{2.782047in}{6.013817in}}%
\pgfpathcurveto{\pgfqpoint{2.786165in}{6.009699in}}{\pgfqpoint{2.791752in}{6.007385in}}{\pgfqpoint{2.797576in}{6.007385in}}%
\pgfpathlineto{\pgfqpoint{2.797576in}{6.007385in}}%
\pgfpathclose%
\pgfusepath{stroke,fill}%
\end{pgfscope}%
\begin{pgfscope}%
\pgfpathrectangle{\pgfqpoint{1.000000in}{0.979904in}}{\pgfqpoint{6.200000in}{5.960192in}}%
\pgfusepath{clip}%
\pgfsetbuttcap%
\pgfsetroundjoin%
\definecolor{currentfill}{rgb}{0.200000,0.200000,0.800000}%
\pgfsetfillcolor{currentfill}%
\pgfsetlinewidth{1.003750pt}%
\definecolor{currentstroke}{rgb}{0.200000,0.200000,0.800000}%
\pgfsetstrokecolor{currentstroke}%
\pgfsetdash{}{0pt}%
\pgfpathmoveto{\pgfqpoint{2.762879in}{5.953684in}}%
\pgfpathcurveto{\pgfqpoint{2.768703in}{5.953684in}}{\pgfqpoint{2.774289in}{5.955998in}}{\pgfqpoint{2.778408in}{5.960116in}}%
\pgfpathcurveto{\pgfqpoint{2.782526in}{5.964234in}}{\pgfqpoint{2.784840in}{5.969820in}}{\pgfqpoint{2.784840in}{5.975644in}}%
\pgfpathcurveto{\pgfqpoint{2.784840in}{5.981468in}}{\pgfqpoint{2.782526in}{5.987054in}}{\pgfqpoint{2.778408in}{5.991173in}}%
\pgfpathcurveto{\pgfqpoint{2.774289in}{5.995291in}}{\pgfqpoint{2.768703in}{5.997605in}}{\pgfqpoint{2.762879in}{5.997605in}}%
\pgfpathcurveto{\pgfqpoint{2.757055in}{5.997605in}}{\pgfqpoint{2.751469in}{5.995291in}}{\pgfqpoint{2.747351in}{5.991173in}}%
\pgfpathcurveto{\pgfqpoint{2.743233in}{5.987054in}}{\pgfqpoint{2.740919in}{5.981468in}}{\pgfqpoint{2.740919in}{5.975644in}}%
\pgfpathcurveto{\pgfqpoint{2.740919in}{5.969820in}}{\pgfqpoint{2.743233in}{5.964234in}}{\pgfqpoint{2.747351in}{5.960116in}}%
\pgfpathcurveto{\pgfqpoint{2.751469in}{5.955998in}}{\pgfqpoint{2.757055in}{5.953684in}}{\pgfqpoint{2.762879in}{5.953684in}}%
\pgfpathlineto{\pgfqpoint{2.762879in}{5.953684in}}%
\pgfpathclose%
\pgfusepath{stroke,fill}%
\end{pgfscope}%
\begin{pgfscope}%
\pgfpathrectangle{\pgfqpoint{1.000000in}{0.979904in}}{\pgfqpoint{6.200000in}{5.960192in}}%
\pgfusepath{clip}%
\pgfsetbuttcap%
\pgfsetroundjoin%
\definecolor{currentfill}{rgb}{0.200000,0.200000,0.800000}%
\pgfsetfillcolor{currentfill}%
\pgfsetlinewidth{1.003750pt}%
\definecolor{currentstroke}{rgb}{0.200000,0.200000,0.800000}%
\pgfsetstrokecolor{currentstroke}%
\pgfsetdash{}{0pt}%
\pgfpathmoveto{\pgfqpoint{2.746877in}{5.893399in}}%
\pgfpathcurveto{\pgfqpoint{2.752701in}{5.893399in}}{\pgfqpoint{2.758287in}{5.895713in}}{\pgfqpoint{2.762405in}{5.899831in}}%
\pgfpathcurveto{\pgfqpoint{2.766523in}{5.903949in}}{\pgfqpoint{2.768837in}{5.909535in}}{\pgfqpoint{2.768837in}{5.915359in}}%
\pgfpathcurveto{\pgfqpoint{2.768837in}{5.921183in}}{\pgfqpoint{2.766523in}{5.926770in}}{\pgfqpoint{2.762405in}{5.930888in}}%
\pgfpathcurveto{\pgfqpoint{2.758287in}{5.935006in}}{\pgfqpoint{2.752701in}{5.937320in}}{\pgfqpoint{2.746877in}{5.937320in}}%
\pgfpathcurveto{\pgfqpoint{2.741053in}{5.937320in}}{\pgfqpoint{2.735467in}{5.935006in}}{\pgfqpoint{2.731349in}{5.930888in}}%
\pgfpathcurveto{\pgfqpoint{2.727231in}{5.926770in}}{\pgfqpoint{2.724917in}{5.921183in}}{\pgfqpoint{2.724917in}{5.915359in}}%
\pgfpathcurveto{\pgfqpoint{2.724917in}{5.909535in}}{\pgfqpoint{2.727231in}{5.903949in}}{\pgfqpoint{2.731349in}{5.899831in}}%
\pgfpathcurveto{\pgfqpoint{2.735467in}{5.895713in}}{\pgfqpoint{2.741053in}{5.893399in}}{\pgfqpoint{2.746877in}{5.893399in}}%
\pgfpathlineto{\pgfqpoint{2.746877in}{5.893399in}}%
\pgfpathclose%
\pgfusepath{stroke,fill}%
\end{pgfscope}%
\begin{pgfscope}%
\pgfpathrectangle{\pgfqpoint{1.000000in}{0.979904in}}{\pgfqpoint{6.200000in}{5.960192in}}%
\pgfusepath{clip}%
\pgfsetbuttcap%
\pgfsetroundjoin%
\definecolor{currentfill}{rgb}{0.200000,0.200000,0.800000}%
\pgfsetfillcolor{currentfill}%
\pgfsetlinewidth{1.003750pt}%
\definecolor{currentstroke}{rgb}{0.200000,0.200000,0.800000}%
\pgfsetstrokecolor{currentstroke}%
\pgfsetdash{}{0pt}%
\pgfpathmoveto{\pgfqpoint{2.742371in}{5.830976in}}%
\pgfpathcurveto{\pgfqpoint{2.748195in}{5.830976in}}{\pgfqpoint{2.753781in}{5.833289in}}{\pgfqpoint{2.757899in}{5.837408in}}%
\pgfpathcurveto{\pgfqpoint{2.762017in}{5.841526in}}{\pgfqpoint{2.764331in}{5.847112in}}{\pgfqpoint{2.764331in}{5.852936in}}%
\pgfpathcurveto{\pgfqpoint{2.764331in}{5.858760in}}{\pgfqpoint{2.762017in}{5.864346in}}{\pgfqpoint{2.757899in}{5.868464in}}%
\pgfpathcurveto{\pgfqpoint{2.753781in}{5.872582in}}{\pgfqpoint{2.748195in}{5.874896in}}{\pgfqpoint{2.742371in}{5.874896in}}%
\pgfpathcurveto{\pgfqpoint{2.736547in}{5.874896in}}{\pgfqpoint{2.730961in}{5.872582in}}{\pgfqpoint{2.726842in}{5.868464in}}%
\pgfpathcurveto{\pgfqpoint{2.722724in}{5.864346in}}{\pgfqpoint{2.720410in}{5.858760in}}{\pgfqpoint{2.720410in}{5.852936in}}%
\pgfpathcurveto{\pgfqpoint{2.720410in}{5.847112in}}{\pgfqpoint{2.722724in}{5.841526in}}{\pgfqpoint{2.726842in}{5.837408in}}%
\pgfpathcurveto{\pgfqpoint{2.730961in}{5.833289in}}{\pgfqpoint{2.736547in}{5.830976in}}{\pgfqpoint{2.742371in}{5.830976in}}%
\pgfpathlineto{\pgfqpoint{2.742371in}{5.830976in}}%
\pgfpathclose%
\pgfusepath{stroke,fill}%
\end{pgfscope}%
\begin{pgfscope}%
\pgfpathrectangle{\pgfqpoint{1.000000in}{0.979904in}}{\pgfqpoint{6.200000in}{5.960192in}}%
\pgfusepath{clip}%
\pgfsetbuttcap%
\pgfsetroundjoin%
\definecolor{currentfill}{rgb}{0.200000,0.200000,0.800000}%
\pgfsetfillcolor{currentfill}%
\pgfsetlinewidth{1.003750pt}%
\definecolor{currentstroke}{rgb}{0.200000,0.200000,0.800000}%
\pgfsetstrokecolor{currentstroke}%
\pgfsetdash{}{0pt}%
\pgfpathmoveto{\pgfqpoint{2.785351in}{5.764707in}}%
\pgfpathcurveto{\pgfqpoint{2.791175in}{5.764707in}}{\pgfqpoint{2.796761in}{5.767021in}}{\pgfqpoint{2.800879in}{5.771139in}}%
\pgfpathcurveto{\pgfqpoint{2.804997in}{5.775258in}}{\pgfqpoint{2.807311in}{5.780844in}}{\pgfqpoint{2.807311in}{5.786668in}}%
\pgfpathcurveto{\pgfqpoint{2.807311in}{5.792492in}}{\pgfqpoint{2.804997in}{5.798078in}}{\pgfqpoint{2.800879in}{5.802196in}}%
\pgfpathcurveto{\pgfqpoint{2.796761in}{5.806314in}}{\pgfqpoint{2.791175in}{5.808628in}}{\pgfqpoint{2.785351in}{5.808628in}}%
\pgfpathcurveto{\pgfqpoint{2.779527in}{5.808628in}}{\pgfqpoint{2.773941in}{5.806314in}}{\pgfqpoint{2.769823in}{5.802196in}}%
\pgfpathcurveto{\pgfqpoint{2.765704in}{5.798078in}}{\pgfqpoint{2.763391in}{5.792492in}}{\pgfqpoint{2.763391in}{5.786668in}}%
\pgfpathcurveto{\pgfqpoint{2.763391in}{5.780844in}}{\pgfqpoint{2.765704in}{5.775258in}}{\pgfqpoint{2.769823in}{5.771139in}}%
\pgfpathcurveto{\pgfqpoint{2.773941in}{5.767021in}}{\pgfqpoint{2.779527in}{5.764707in}}{\pgfqpoint{2.785351in}{5.764707in}}%
\pgfpathlineto{\pgfqpoint{2.785351in}{5.764707in}}%
\pgfpathclose%
\pgfusepath{stroke,fill}%
\end{pgfscope}%
\begin{pgfscope}%
\pgfpathrectangle{\pgfqpoint{1.000000in}{0.979904in}}{\pgfqpoint{6.200000in}{5.960192in}}%
\pgfusepath{clip}%
\pgfsetbuttcap%
\pgfsetroundjoin%
\definecolor{currentfill}{rgb}{0.200000,0.200000,0.800000}%
\pgfsetfillcolor{currentfill}%
\pgfsetlinewidth{1.003750pt}%
\definecolor{currentstroke}{rgb}{0.200000,0.200000,0.800000}%
\pgfsetstrokecolor{currentstroke}%
\pgfsetdash{}{0pt}%
\pgfpathmoveto{\pgfqpoint{2.803706in}{5.705450in}}%
\pgfpathcurveto{\pgfqpoint{2.809530in}{5.705450in}}{\pgfqpoint{2.815117in}{5.707764in}}{\pgfqpoint{2.819235in}{5.711882in}}%
\pgfpathcurveto{\pgfqpoint{2.823353in}{5.716001in}}{\pgfqpoint{2.825667in}{5.721587in}}{\pgfqpoint{2.825667in}{5.727411in}}%
\pgfpathcurveto{\pgfqpoint{2.825667in}{5.733235in}}{\pgfqpoint{2.823353in}{5.738821in}}{\pgfqpoint{2.819235in}{5.742939in}}%
\pgfpathcurveto{\pgfqpoint{2.815117in}{5.747057in}}{\pgfqpoint{2.809530in}{5.749371in}}{\pgfqpoint{2.803706in}{5.749371in}}%
\pgfpathcurveto{\pgfqpoint{2.797883in}{5.749371in}}{\pgfqpoint{2.792296in}{5.747057in}}{\pgfqpoint{2.788178in}{5.742939in}}%
\pgfpathcurveto{\pgfqpoint{2.784060in}{5.738821in}}{\pgfqpoint{2.781746in}{5.733235in}}{\pgfqpoint{2.781746in}{5.727411in}}%
\pgfpathcurveto{\pgfqpoint{2.781746in}{5.721587in}}{\pgfqpoint{2.784060in}{5.716001in}}{\pgfqpoint{2.788178in}{5.711882in}}%
\pgfpathcurveto{\pgfqpoint{2.792296in}{5.707764in}}{\pgfqpoint{2.797883in}{5.705450in}}{\pgfqpoint{2.803706in}{5.705450in}}%
\pgfpathlineto{\pgfqpoint{2.803706in}{5.705450in}}%
\pgfpathclose%
\pgfusepath{stroke,fill}%
\end{pgfscope}%
\begin{pgfscope}%
\pgfpathrectangle{\pgfqpoint{1.000000in}{0.979904in}}{\pgfqpoint{6.200000in}{5.960192in}}%
\pgfusepath{clip}%
\pgfsetbuttcap%
\pgfsetroundjoin%
\definecolor{currentfill}{rgb}{0.200000,0.200000,0.800000}%
\pgfsetfillcolor{currentfill}%
\pgfsetlinewidth{1.003750pt}%
\definecolor{currentstroke}{rgb}{0.200000,0.200000,0.800000}%
\pgfsetstrokecolor{currentstroke}%
\pgfsetdash{}{0pt}%
\pgfpathmoveto{\pgfqpoint{2.850621in}{5.649667in}}%
\pgfpathcurveto{\pgfqpoint{2.856445in}{5.649667in}}{\pgfqpoint{2.862031in}{5.651981in}}{\pgfqpoint{2.866149in}{5.656099in}}%
\pgfpathcurveto{\pgfqpoint{2.870267in}{5.660217in}}{\pgfqpoint{2.872581in}{5.665803in}}{\pgfqpoint{2.872581in}{5.671627in}}%
\pgfpathcurveto{\pgfqpoint{2.872581in}{5.677451in}}{\pgfqpoint{2.870267in}{5.683037in}}{\pgfqpoint{2.866149in}{5.687155in}}%
\pgfpathcurveto{\pgfqpoint{2.862031in}{5.691274in}}{\pgfqpoint{2.856445in}{5.693587in}}{\pgfqpoint{2.850621in}{5.693587in}}%
\pgfpathcurveto{\pgfqpoint{2.844797in}{5.693587in}}{\pgfqpoint{2.839211in}{5.691274in}}{\pgfqpoint{2.835093in}{5.687155in}}%
\pgfpathcurveto{\pgfqpoint{2.830974in}{5.683037in}}{\pgfqpoint{2.828661in}{5.677451in}}{\pgfqpoint{2.828661in}{5.671627in}}%
\pgfpathcurveto{\pgfqpoint{2.828661in}{5.665803in}}{\pgfqpoint{2.830974in}{5.660217in}}{\pgfqpoint{2.835093in}{5.656099in}}%
\pgfpathcurveto{\pgfqpoint{2.839211in}{5.651981in}}{\pgfqpoint{2.844797in}{5.649667in}}{\pgfqpoint{2.850621in}{5.649667in}}%
\pgfpathlineto{\pgfqpoint{2.850621in}{5.649667in}}%
\pgfpathclose%
\pgfusepath{stroke,fill}%
\end{pgfscope}%
\begin{pgfscope}%
\pgfpathrectangle{\pgfqpoint{1.000000in}{0.979904in}}{\pgfqpoint{6.200000in}{5.960192in}}%
\pgfusepath{clip}%
\pgfsetbuttcap%
\pgfsetroundjoin%
\definecolor{currentfill}{rgb}{0.200000,0.200000,0.800000}%
\pgfsetfillcolor{currentfill}%
\pgfsetlinewidth{1.003750pt}%
\definecolor{currentstroke}{rgb}{0.200000,0.200000,0.800000}%
\pgfsetstrokecolor{currentstroke}%
\pgfsetdash{}{0pt}%
\pgfpathmoveto{\pgfqpoint{2.716330in}{5.582330in}}%
\pgfpathcurveto{\pgfqpoint{2.722154in}{5.582330in}}{\pgfqpoint{2.727740in}{5.584644in}}{\pgfqpoint{2.731858in}{5.588762in}}%
\pgfpathcurveto{\pgfqpoint{2.735976in}{5.592880in}}{\pgfqpoint{2.738290in}{5.598466in}}{\pgfqpoint{2.738290in}{5.604290in}}%
\pgfpathcurveto{\pgfqpoint{2.738290in}{5.610114in}}{\pgfqpoint{2.735976in}{5.615700in}}{\pgfqpoint{2.731858in}{5.619818in}}%
\pgfpathcurveto{\pgfqpoint{2.727740in}{5.623937in}}{\pgfqpoint{2.722154in}{5.626250in}}{\pgfqpoint{2.716330in}{5.626250in}}%
\pgfpathcurveto{\pgfqpoint{2.710506in}{5.626250in}}{\pgfqpoint{2.704920in}{5.623937in}}{\pgfqpoint{2.700801in}{5.619818in}}%
\pgfpathcurveto{\pgfqpoint{2.696683in}{5.615700in}}{\pgfqpoint{2.694369in}{5.610114in}}{\pgfqpoint{2.694369in}{5.604290in}}%
\pgfpathcurveto{\pgfqpoint{2.694369in}{5.598466in}}{\pgfqpoint{2.696683in}{5.592880in}}{\pgfqpoint{2.700801in}{5.588762in}}%
\pgfpathcurveto{\pgfqpoint{2.704920in}{5.584644in}}{\pgfqpoint{2.710506in}{5.582330in}}{\pgfqpoint{2.716330in}{5.582330in}}%
\pgfpathlineto{\pgfqpoint{2.716330in}{5.582330in}}%
\pgfpathclose%
\pgfusepath{stroke,fill}%
\end{pgfscope}%
\begin{pgfscope}%
\pgfpathrectangle{\pgfqpoint{1.000000in}{0.979904in}}{\pgfqpoint{6.200000in}{5.960192in}}%
\pgfusepath{clip}%
\pgfsetbuttcap%
\pgfsetroundjoin%
\definecolor{currentfill}{rgb}{0.200000,0.200000,0.800000}%
\pgfsetfillcolor{currentfill}%
\pgfsetlinewidth{1.003750pt}%
\definecolor{currentstroke}{rgb}{0.200000,0.200000,0.800000}%
\pgfsetstrokecolor{currentstroke}%
\pgfsetdash{}{0pt}%
\pgfpathmoveto{\pgfqpoint{2.791237in}{5.530472in}}%
\pgfpathcurveto{\pgfqpoint{2.797060in}{5.530472in}}{\pgfqpoint{2.802647in}{5.532785in}}{\pgfqpoint{2.806765in}{5.536904in}}%
\pgfpathcurveto{\pgfqpoint{2.810883in}{5.541022in}}{\pgfqpoint{2.813197in}{5.546608in}}{\pgfqpoint{2.813197in}{5.552432in}}%
\pgfpathcurveto{\pgfqpoint{2.813197in}{5.558256in}}{\pgfqpoint{2.810883in}{5.563842in}}{\pgfqpoint{2.806765in}{5.567960in}}%
\pgfpathcurveto{\pgfqpoint{2.802647in}{5.572078in}}{\pgfqpoint{2.797060in}{5.574392in}}{\pgfqpoint{2.791237in}{5.574392in}}%
\pgfpathcurveto{\pgfqpoint{2.785413in}{5.574392in}}{\pgfqpoint{2.779826in}{5.572078in}}{\pgfqpoint{2.775708in}{5.567960in}}%
\pgfpathcurveto{\pgfqpoint{2.771590in}{5.563842in}}{\pgfqpoint{2.769276in}{5.558256in}}{\pgfqpoint{2.769276in}{5.552432in}}%
\pgfpathcurveto{\pgfqpoint{2.769276in}{5.546608in}}{\pgfqpoint{2.771590in}{5.541022in}}{\pgfqpoint{2.775708in}{5.536904in}}%
\pgfpathcurveto{\pgfqpoint{2.779826in}{5.532785in}}{\pgfqpoint{2.785413in}{5.530472in}}{\pgfqpoint{2.791237in}{5.530472in}}%
\pgfpathlineto{\pgfqpoint{2.791237in}{5.530472in}}%
\pgfpathclose%
\pgfusepath{stroke,fill}%
\end{pgfscope}%
\begin{pgfscope}%
\pgfpathrectangle{\pgfqpoint{1.000000in}{0.979904in}}{\pgfqpoint{6.200000in}{5.960192in}}%
\pgfusepath{clip}%
\pgfsetbuttcap%
\pgfsetroundjoin%
\definecolor{currentfill}{rgb}{0.200000,0.200000,0.800000}%
\pgfsetfillcolor{currentfill}%
\pgfsetlinewidth{1.003750pt}%
\definecolor{currentstroke}{rgb}{0.200000,0.200000,0.800000}%
\pgfsetstrokecolor{currentstroke}%
\pgfsetdash{}{0pt}%
\pgfpathmoveto{\pgfqpoint{2.814400in}{5.475480in}}%
\pgfpathcurveto{\pgfqpoint{2.820224in}{5.475480in}}{\pgfqpoint{2.825810in}{5.477794in}}{\pgfqpoint{2.829928in}{5.481912in}}%
\pgfpathcurveto{\pgfqpoint{2.834046in}{5.486030in}}{\pgfqpoint{2.836360in}{5.491616in}}{\pgfqpoint{2.836360in}{5.497440in}}%
\pgfpathcurveto{\pgfqpoint{2.836360in}{5.503264in}}{\pgfqpoint{2.834046in}{5.508850in}}{\pgfqpoint{2.829928in}{5.512968in}}%
\pgfpathcurveto{\pgfqpoint{2.825810in}{5.517086in}}{\pgfqpoint{2.820224in}{5.519400in}}{\pgfqpoint{2.814400in}{5.519400in}}%
\pgfpathcurveto{\pgfqpoint{2.808576in}{5.519400in}}{\pgfqpoint{2.802990in}{5.517086in}}{\pgfqpoint{2.798872in}{5.512968in}}%
\pgfpathcurveto{\pgfqpoint{2.794754in}{5.508850in}}{\pgfqpoint{2.792440in}{5.503264in}}{\pgfqpoint{2.792440in}{5.497440in}}%
\pgfpathcurveto{\pgfqpoint{2.792440in}{5.491616in}}{\pgfqpoint{2.794754in}{5.486030in}}{\pgfqpoint{2.798872in}{5.481912in}}%
\pgfpathcurveto{\pgfqpoint{2.802990in}{5.477794in}}{\pgfqpoint{2.808576in}{5.475480in}}{\pgfqpoint{2.814400in}{5.475480in}}%
\pgfpathlineto{\pgfqpoint{2.814400in}{5.475480in}}%
\pgfpathclose%
\pgfusepath{stroke,fill}%
\end{pgfscope}%
\begin{pgfscope}%
\pgfpathrectangle{\pgfqpoint{1.000000in}{0.979904in}}{\pgfqpoint{6.200000in}{5.960192in}}%
\pgfusepath{clip}%
\pgfsetbuttcap%
\pgfsetroundjoin%
\definecolor{currentfill}{rgb}{0.200000,0.200000,0.800000}%
\pgfsetfillcolor{currentfill}%
\pgfsetlinewidth{1.003750pt}%
\definecolor{currentstroke}{rgb}{0.200000,0.200000,0.800000}%
\pgfsetstrokecolor{currentstroke}%
\pgfsetdash{}{0pt}%
\pgfpathmoveto{\pgfqpoint{2.748016in}{5.395580in}}%
\pgfpathcurveto{\pgfqpoint{2.753840in}{5.395580in}}{\pgfqpoint{2.759426in}{5.397894in}}{\pgfqpoint{2.763544in}{5.402012in}}%
\pgfpathcurveto{\pgfqpoint{2.767662in}{5.406130in}}{\pgfqpoint{2.769976in}{5.411716in}}{\pgfqpoint{2.769976in}{5.417540in}}%
\pgfpathcurveto{\pgfqpoint{2.769976in}{5.423364in}}{\pgfqpoint{2.767662in}{5.428950in}}{\pgfqpoint{2.763544in}{5.433068in}}%
\pgfpathcurveto{\pgfqpoint{2.759426in}{5.437186in}}{\pgfqpoint{2.753840in}{5.439500in}}{\pgfqpoint{2.748016in}{5.439500in}}%
\pgfpathcurveto{\pgfqpoint{2.742192in}{5.439500in}}{\pgfqpoint{2.736606in}{5.437186in}}{\pgfqpoint{2.732488in}{5.433068in}}%
\pgfpathcurveto{\pgfqpoint{2.728370in}{5.428950in}}{\pgfqpoint{2.726056in}{5.423364in}}{\pgfqpoint{2.726056in}{5.417540in}}%
\pgfpathcurveto{\pgfqpoint{2.726056in}{5.411716in}}{\pgfqpoint{2.728370in}{5.406130in}}{\pgfqpoint{2.732488in}{5.402012in}}%
\pgfpathcurveto{\pgfqpoint{2.736606in}{5.397894in}}{\pgfqpoint{2.742192in}{5.395580in}}{\pgfqpoint{2.748016in}{5.395580in}}%
\pgfpathlineto{\pgfqpoint{2.748016in}{5.395580in}}%
\pgfpathclose%
\pgfusepath{stroke,fill}%
\end{pgfscope}%
\begin{pgfscope}%
\pgfpathrectangle{\pgfqpoint{1.000000in}{0.979904in}}{\pgfqpoint{6.200000in}{5.960192in}}%
\pgfusepath{clip}%
\pgfsetbuttcap%
\pgfsetroundjoin%
\definecolor{currentfill}{rgb}{0.200000,0.200000,0.800000}%
\pgfsetfillcolor{currentfill}%
\pgfsetlinewidth{1.003750pt}%
\definecolor{currentstroke}{rgb}{0.200000,0.200000,0.800000}%
\pgfsetstrokecolor{currentstroke}%
\pgfsetdash{}{0pt}%
\pgfpathmoveto{\pgfqpoint{2.871380in}{5.373106in}}%
\pgfpathcurveto{\pgfqpoint{2.877204in}{5.373106in}}{\pgfqpoint{2.882790in}{5.375420in}}{\pgfqpoint{2.886909in}{5.379538in}}%
\pgfpathcurveto{\pgfqpoint{2.891027in}{5.383656in}}{\pgfqpoint{2.893341in}{5.389242in}}{\pgfqpoint{2.893341in}{5.395066in}}%
\pgfpathcurveto{\pgfqpoint{2.893341in}{5.400890in}}{\pgfqpoint{2.891027in}{5.406476in}}{\pgfqpoint{2.886909in}{5.410594in}}%
\pgfpathcurveto{\pgfqpoint{2.882790in}{5.414712in}}{\pgfqpoint{2.877204in}{5.417026in}}{\pgfqpoint{2.871380in}{5.417026in}}%
\pgfpathcurveto{\pgfqpoint{2.865556in}{5.417026in}}{\pgfqpoint{2.859970in}{5.414712in}}{\pgfqpoint{2.855852in}{5.410594in}}%
\pgfpathcurveto{\pgfqpoint{2.851734in}{5.406476in}}{\pgfqpoint{2.849420in}{5.400890in}}{\pgfqpoint{2.849420in}{5.395066in}}%
\pgfpathcurveto{\pgfqpoint{2.849420in}{5.389242in}}{\pgfqpoint{2.851734in}{5.383656in}}{\pgfqpoint{2.855852in}{5.379538in}}%
\pgfpathcurveto{\pgfqpoint{2.859970in}{5.375420in}}{\pgfqpoint{2.865556in}{5.373106in}}{\pgfqpoint{2.871380in}{5.373106in}}%
\pgfpathlineto{\pgfqpoint{2.871380in}{5.373106in}}%
\pgfpathclose%
\pgfusepath{stroke,fill}%
\end{pgfscope}%
\begin{pgfscope}%
\pgfpathrectangle{\pgfqpoint{1.000000in}{0.979904in}}{\pgfqpoint{6.200000in}{5.960192in}}%
\pgfusepath{clip}%
\pgfsetbuttcap%
\pgfsetroundjoin%
\definecolor{currentfill}{rgb}{0.200000,0.200000,0.800000}%
\pgfsetfillcolor{currentfill}%
\pgfsetlinewidth{1.003750pt}%
\definecolor{currentstroke}{rgb}{0.200000,0.200000,0.800000}%
\pgfsetstrokecolor{currentstroke}%
\pgfsetdash{}{0pt}%
\pgfpathmoveto{\pgfqpoint{2.808089in}{5.283930in}}%
\pgfpathcurveto{\pgfqpoint{2.813913in}{5.283930in}}{\pgfqpoint{2.819499in}{5.286244in}}{\pgfqpoint{2.823617in}{5.290362in}}%
\pgfpathcurveto{\pgfqpoint{2.827735in}{5.294480in}}{\pgfqpoint{2.830049in}{5.300066in}}{\pgfqpoint{2.830049in}{5.305890in}}%
\pgfpathcurveto{\pgfqpoint{2.830049in}{5.311714in}}{\pgfqpoint{2.827735in}{5.317300in}}{\pgfqpoint{2.823617in}{5.321418in}}%
\pgfpathcurveto{\pgfqpoint{2.819499in}{5.325536in}}{\pgfqpoint{2.813913in}{5.327850in}}{\pgfqpoint{2.808089in}{5.327850in}}%
\pgfpathcurveto{\pgfqpoint{2.802265in}{5.327850in}}{\pgfqpoint{2.796679in}{5.325536in}}{\pgfqpoint{2.792561in}{5.321418in}}%
\pgfpathcurveto{\pgfqpoint{2.788443in}{5.317300in}}{\pgfqpoint{2.786129in}{5.311714in}}{\pgfqpoint{2.786129in}{5.305890in}}%
\pgfpathcurveto{\pgfqpoint{2.786129in}{5.300066in}}{\pgfqpoint{2.788443in}{5.294480in}}{\pgfqpoint{2.792561in}{5.290362in}}%
\pgfpathcurveto{\pgfqpoint{2.796679in}{5.286244in}}{\pgfqpoint{2.802265in}{5.283930in}}{\pgfqpoint{2.808089in}{5.283930in}}%
\pgfpathlineto{\pgfqpoint{2.808089in}{5.283930in}}%
\pgfpathclose%
\pgfusepath{stroke,fill}%
\end{pgfscope}%
\begin{pgfscope}%
\pgfpathrectangle{\pgfqpoint{1.000000in}{0.979904in}}{\pgfqpoint{6.200000in}{5.960192in}}%
\pgfusepath{clip}%
\pgfsetbuttcap%
\pgfsetroundjoin%
\definecolor{currentfill}{rgb}{0.200000,0.200000,0.800000}%
\pgfsetfillcolor{currentfill}%
\pgfsetlinewidth{1.003750pt}%
\definecolor{currentstroke}{rgb}{0.200000,0.200000,0.800000}%
\pgfsetstrokecolor{currentstroke}%
\pgfsetdash{}{0pt}%
\pgfpathmoveto{\pgfqpoint{2.860197in}{5.240869in}}%
\pgfpathcurveto{\pgfqpoint{2.866021in}{5.240869in}}{\pgfqpoint{2.871607in}{5.243183in}}{\pgfqpoint{2.875725in}{5.247301in}}%
\pgfpathcurveto{\pgfqpoint{2.879843in}{5.251419in}}{\pgfqpoint{2.882157in}{5.257005in}}{\pgfqpoint{2.882157in}{5.262829in}}%
\pgfpathcurveto{\pgfqpoint{2.882157in}{5.268653in}}{\pgfqpoint{2.879843in}{5.274240in}}{\pgfqpoint{2.875725in}{5.278358in}}%
\pgfpathcurveto{\pgfqpoint{2.871607in}{5.282476in}}{\pgfqpoint{2.866021in}{5.284790in}}{\pgfqpoint{2.860197in}{5.284790in}}%
\pgfpathcurveto{\pgfqpoint{2.854373in}{5.284790in}}{\pgfqpoint{2.848787in}{5.282476in}}{\pgfqpoint{2.844669in}{5.278358in}}%
\pgfpathcurveto{\pgfqpoint{2.840551in}{5.274240in}}{\pgfqpoint{2.838237in}{5.268653in}}{\pgfqpoint{2.838237in}{5.262829in}}%
\pgfpathcurveto{\pgfqpoint{2.838237in}{5.257005in}}{\pgfqpoint{2.840551in}{5.251419in}}{\pgfqpoint{2.844669in}{5.247301in}}%
\pgfpathcurveto{\pgfqpoint{2.848787in}{5.243183in}}{\pgfqpoint{2.854373in}{5.240869in}}{\pgfqpoint{2.860197in}{5.240869in}}%
\pgfpathlineto{\pgfqpoint{2.860197in}{5.240869in}}%
\pgfpathclose%
\pgfusepath{stroke,fill}%
\end{pgfscope}%
\begin{pgfscope}%
\pgfpathrectangle{\pgfqpoint{1.000000in}{0.979904in}}{\pgfqpoint{6.200000in}{5.960192in}}%
\pgfusepath{clip}%
\pgfsetbuttcap%
\pgfsetroundjoin%
\definecolor{currentfill}{rgb}{0.200000,0.200000,0.800000}%
\pgfsetfillcolor{currentfill}%
\pgfsetlinewidth{1.003750pt}%
\definecolor{currentstroke}{rgb}{0.200000,0.200000,0.800000}%
\pgfsetstrokecolor{currentstroke}%
\pgfsetdash{}{0pt}%
\pgfpathmoveto{\pgfqpoint{2.950293in}{5.224482in}}%
\pgfpathcurveto{\pgfqpoint{2.956117in}{5.224482in}}{\pgfqpoint{2.961703in}{5.226796in}}{\pgfqpoint{2.965821in}{5.230914in}}%
\pgfpathcurveto{\pgfqpoint{2.969939in}{5.235032in}}{\pgfqpoint{2.972253in}{5.240618in}}{\pgfqpoint{2.972253in}{5.246442in}}%
\pgfpathcurveto{\pgfqpoint{2.972253in}{5.252266in}}{\pgfqpoint{2.969939in}{5.257852in}}{\pgfqpoint{2.965821in}{5.261971in}}%
\pgfpathcurveto{\pgfqpoint{2.961703in}{5.266089in}}{\pgfqpoint{2.956117in}{5.268403in}}{\pgfqpoint{2.950293in}{5.268403in}}%
\pgfpathcurveto{\pgfqpoint{2.944469in}{5.268403in}}{\pgfqpoint{2.938883in}{5.266089in}}{\pgfqpoint{2.934764in}{5.261971in}}%
\pgfpathcurveto{\pgfqpoint{2.930646in}{5.257852in}}{\pgfqpoint{2.928332in}{5.252266in}}{\pgfqpoint{2.928332in}{5.246442in}}%
\pgfpathcurveto{\pgfqpoint{2.928332in}{5.240618in}}{\pgfqpoint{2.930646in}{5.235032in}}{\pgfqpoint{2.934764in}{5.230914in}}%
\pgfpathcurveto{\pgfqpoint{2.938883in}{5.226796in}}{\pgfqpoint{2.944469in}{5.224482in}}{\pgfqpoint{2.950293in}{5.224482in}}%
\pgfpathlineto{\pgfqpoint{2.950293in}{5.224482in}}%
\pgfpathclose%
\pgfusepath{stroke,fill}%
\end{pgfscope}%
\begin{pgfscope}%
\pgfpathrectangle{\pgfqpoint{1.000000in}{0.979904in}}{\pgfqpoint{6.200000in}{5.960192in}}%
\pgfusepath{clip}%
\pgfsetbuttcap%
\pgfsetroundjoin%
\definecolor{currentfill}{rgb}{0.200000,0.200000,0.800000}%
\pgfsetfillcolor{currentfill}%
\pgfsetlinewidth{1.003750pt}%
\definecolor{currentstroke}{rgb}{0.200000,0.200000,0.800000}%
\pgfsetstrokecolor{currentstroke}%
\pgfsetdash{}{0pt}%
\pgfpathmoveto{\pgfqpoint{2.938460in}{5.148531in}}%
\pgfpathcurveto{\pgfqpoint{2.944284in}{5.148531in}}{\pgfqpoint{2.949870in}{5.150845in}}{\pgfqpoint{2.953989in}{5.154963in}}%
\pgfpathcurveto{\pgfqpoint{2.958107in}{5.159081in}}{\pgfqpoint{2.960421in}{5.164667in}}{\pgfqpoint{2.960421in}{5.170491in}}%
\pgfpathcurveto{\pgfqpoint{2.960421in}{5.176315in}}{\pgfqpoint{2.958107in}{5.181901in}}{\pgfqpoint{2.953989in}{5.186020in}}%
\pgfpathcurveto{\pgfqpoint{2.949870in}{5.190138in}}{\pgfqpoint{2.944284in}{5.192452in}}{\pgfqpoint{2.938460in}{5.192452in}}%
\pgfpathcurveto{\pgfqpoint{2.932636in}{5.192452in}}{\pgfqpoint{2.927050in}{5.190138in}}{\pgfqpoint{2.922932in}{5.186020in}}%
\pgfpathcurveto{\pgfqpoint{2.918814in}{5.181901in}}{\pgfqpoint{2.916500in}{5.176315in}}{\pgfqpoint{2.916500in}{5.170491in}}%
\pgfpathcurveto{\pgfqpoint{2.916500in}{5.164667in}}{\pgfqpoint{2.918814in}{5.159081in}}{\pgfqpoint{2.922932in}{5.154963in}}%
\pgfpathcurveto{\pgfqpoint{2.927050in}{5.150845in}}{\pgfqpoint{2.932636in}{5.148531in}}{\pgfqpoint{2.938460in}{5.148531in}}%
\pgfpathlineto{\pgfqpoint{2.938460in}{5.148531in}}%
\pgfpathclose%
\pgfusepath{stroke,fill}%
\end{pgfscope}%
\begin{pgfscope}%
\pgfpathrectangle{\pgfqpoint{1.000000in}{0.979904in}}{\pgfqpoint{6.200000in}{5.960192in}}%
\pgfusepath{clip}%
\pgfsetbuttcap%
\pgfsetroundjoin%
\definecolor{currentfill}{rgb}{0.200000,0.200000,0.800000}%
\pgfsetfillcolor{currentfill}%
\pgfsetlinewidth{1.003750pt}%
\definecolor{currentstroke}{rgb}{0.200000,0.200000,0.800000}%
\pgfsetstrokecolor{currentstroke}%
\pgfsetdash{}{0pt}%
\pgfpathmoveto{\pgfqpoint{2.929772in}{5.066529in}}%
\pgfpathcurveto{\pgfqpoint{2.935596in}{5.066529in}}{\pgfqpoint{2.941182in}{5.068843in}}{\pgfqpoint{2.945300in}{5.072961in}}%
\pgfpathcurveto{\pgfqpoint{2.949418in}{5.077079in}}{\pgfqpoint{2.951732in}{5.082666in}}{\pgfqpoint{2.951732in}{5.088490in}}%
\pgfpathcurveto{\pgfqpoint{2.951732in}{5.094313in}}{\pgfqpoint{2.949418in}{5.099900in}}{\pgfqpoint{2.945300in}{5.104018in}}%
\pgfpathcurveto{\pgfqpoint{2.941182in}{5.108136in}}{\pgfqpoint{2.935596in}{5.110450in}}{\pgfqpoint{2.929772in}{5.110450in}}%
\pgfpathcurveto{\pgfqpoint{2.923948in}{5.110450in}}{\pgfqpoint{2.918362in}{5.108136in}}{\pgfqpoint{2.914244in}{5.104018in}}%
\pgfpathcurveto{\pgfqpoint{2.910125in}{5.099900in}}{\pgfqpoint{2.907812in}{5.094313in}}{\pgfqpoint{2.907812in}{5.088490in}}%
\pgfpathcurveto{\pgfqpoint{2.907812in}{5.082666in}}{\pgfqpoint{2.910125in}{5.077079in}}{\pgfqpoint{2.914244in}{5.072961in}}%
\pgfpathcurveto{\pgfqpoint{2.918362in}{5.068843in}}{\pgfqpoint{2.923948in}{5.066529in}}{\pgfqpoint{2.929772in}{5.066529in}}%
\pgfpathlineto{\pgfqpoint{2.929772in}{5.066529in}}%
\pgfpathclose%
\pgfusepath{stroke,fill}%
\end{pgfscope}%
\begin{pgfscope}%
\pgfpathrectangle{\pgfqpoint{1.000000in}{0.979904in}}{\pgfqpoint{6.200000in}{5.960192in}}%
\pgfusepath{clip}%
\pgfsetbuttcap%
\pgfsetroundjoin%
\definecolor{currentfill}{rgb}{0.200000,0.200000,0.800000}%
\pgfsetfillcolor{currentfill}%
\pgfsetlinewidth{1.003750pt}%
\definecolor{currentstroke}{rgb}{0.200000,0.200000,0.800000}%
\pgfsetstrokecolor{currentstroke}%
\pgfsetdash{}{0pt}%
\pgfpathmoveto{\pgfqpoint{2.977054in}{5.024813in}}%
\pgfpathcurveto{\pgfqpoint{2.982878in}{5.024813in}}{\pgfqpoint{2.988464in}{5.027127in}}{\pgfqpoint{2.992583in}{5.031245in}}%
\pgfpathcurveto{\pgfqpoint{2.996701in}{5.035363in}}{\pgfqpoint{2.999015in}{5.040949in}}{\pgfqpoint{2.999015in}{5.046773in}}%
\pgfpathcurveto{\pgfqpoint{2.999015in}{5.052597in}}{\pgfqpoint{2.996701in}{5.058183in}}{\pgfqpoint{2.992583in}{5.062302in}}%
\pgfpathcurveto{\pgfqpoint{2.988464in}{5.066420in}}{\pgfqpoint{2.982878in}{5.068734in}}{\pgfqpoint{2.977054in}{5.068734in}}%
\pgfpathcurveto{\pgfqpoint{2.971230in}{5.068734in}}{\pgfqpoint{2.965644in}{5.066420in}}{\pgfqpoint{2.961526in}{5.062302in}}%
\pgfpathcurveto{\pgfqpoint{2.957408in}{5.058183in}}{\pgfqpoint{2.955094in}{5.052597in}}{\pgfqpoint{2.955094in}{5.046773in}}%
\pgfpathcurveto{\pgfqpoint{2.955094in}{5.040949in}}{\pgfqpoint{2.957408in}{5.035363in}}{\pgfqpoint{2.961526in}{5.031245in}}%
\pgfpathcurveto{\pgfqpoint{2.965644in}{5.027127in}}{\pgfqpoint{2.971230in}{5.024813in}}{\pgfqpoint{2.977054in}{5.024813in}}%
\pgfpathlineto{\pgfqpoint{2.977054in}{5.024813in}}%
\pgfpathclose%
\pgfusepath{stroke,fill}%
\end{pgfscope}%
\begin{pgfscope}%
\pgfpathrectangle{\pgfqpoint{1.000000in}{0.979904in}}{\pgfqpoint{6.200000in}{5.960192in}}%
\pgfusepath{clip}%
\pgfsetbuttcap%
\pgfsetroundjoin%
\definecolor{currentfill}{rgb}{0.200000,0.200000,0.800000}%
\pgfsetfillcolor{currentfill}%
\pgfsetlinewidth{1.003750pt}%
\definecolor{currentstroke}{rgb}{0.200000,0.200000,0.800000}%
\pgfsetstrokecolor{currentstroke}%
\pgfsetdash{}{0pt}%
\pgfpathmoveto{\pgfqpoint{2.993827in}{4.953442in}}%
\pgfpathcurveto{\pgfqpoint{2.999651in}{4.953442in}}{\pgfqpoint{3.005238in}{4.955756in}}{\pgfqpoint{3.009356in}{4.959874in}}%
\pgfpathcurveto{\pgfqpoint{3.013474in}{4.963992in}}{\pgfqpoint{3.015788in}{4.969578in}}{\pgfqpoint{3.015788in}{4.975402in}}%
\pgfpathcurveto{\pgfqpoint{3.015788in}{4.981226in}}{\pgfqpoint{3.013474in}{4.986812in}}{\pgfqpoint{3.009356in}{4.990930in}}%
\pgfpathcurveto{\pgfqpoint{3.005238in}{4.995049in}}{\pgfqpoint{2.999651in}{4.997363in}}{\pgfqpoint{2.993827in}{4.997363in}}%
\pgfpathcurveto{\pgfqpoint{2.988004in}{4.997363in}}{\pgfqpoint{2.982417in}{4.995049in}}{\pgfqpoint{2.978299in}{4.990930in}}%
\pgfpathcurveto{\pgfqpoint{2.974181in}{4.986812in}}{\pgfqpoint{2.971867in}{4.981226in}}{\pgfqpoint{2.971867in}{4.975402in}}%
\pgfpathcurveto{\pgfqpoint{2.971867in}{4.969578in}}{\pgfqpoint{2.974181in}{4.963992in}}{\pgfqpoint{2.978299in}{4.959874in}}%
\pgfpathcurveto{\pgfqpoint{2.982417in}{4.955756in}}{\pgfqpoint{2.988004in}{4.953442in}}{\pgfqpoint{2.993827in}{4.953442in}}%
\pgfpathlineto{\pgfqpoint{2.993827in}{4.953442in}}%
\pgfpathclose%
\pgfusepath{stroke,fill}%
\end{pgfscope}%
\begin{pgfscope}%
\pgfpathrectangle{\pgfqpoint{1.000000in}{0.979904in}}{\pgfqpoint{6.200000in}{5.960192in}}%
\pgfusepath{clip}%
\pgfsetbuttcap%
\pgfsetroundjoin%
\definecolor{currentfill}{rgb}{0.200000,0.200000,0.800000}%
\pgfsetfillcolor{currentfill}%
\pgfsetlinewidth{1.003750pt}%
\definecolor{currentstroke}{rgb}{0.200000,0.200000,0.800000}%
\pgfsetstrokecolor{currentstroke}%
\pgfsetdash{}{0pt}%
\pgfpathmoveto{\pgfqpoint{3.081348in}{4.956145in}}%
\pgfpathcurveto{\pgfqpoint{3.087172in}{4.956145in}}{\pgfqpoint{3.092759in}{4.958458in}}{\pgfqpoint{3.096877in}{4.962577in}}%
\pgfpathcurveto{\pgfqpoint{3.100995in}{4.966695in}}{\pgfqpoint{3.103309in}{4.972281in}}{\pgfqpoint{3.103309in}{4.978105in}}%
\pgfpathcurveto{\pgfqpoint{3.103309in}{4.983929in}}{\pgfqpoint{3.100995in}{4.989515in}}{\pgfqpoint{3.096877in}{4.993633in}}%
\pgfpathcurveto{\pgfqpoint{3.092759in}{4.997751in}}{\pgfqpoint{3.087172in}{5.000065in}}{\pgfqpoint{3.081348in}{5.000065in}}%
\pgfpathcurveto{\pgfqpoint{3.075525in}{5.000065in}}{\pgfqpoint{3.069938in}{4.997751in}}{\pgfqpoint{3.065820in}{4.993633in}}%
\pgfpathcurveto{\pgfqpoint{3.061702in}{4.989515in}}{\pgfqpoint{3.059388in}{4.983929in}}{\pgfqpoint{3.059388in}{4.978105in}}%
\pgfpathcurveto{\pgfqpoint{3.059388in}{4.972281in}}{\pgfqpoint{3.061702in}{4.966695in}}{\pgfqpoint{3.065820in}{4.962577in}}%
\pgfpathcurveto{\pgfqpoint{3.069938in}{4.958458in}}{\pgfqpoint{3.075525in}{4.956145in}}{\pgfqpoint{3.081348in}{4.956145in}}%
\pgfpathlineto{\pgfqpoint{3.081348in}{4.956145in}}%
\pgfpathclose%
\pgfusepath{stroke,fill}%
\end{pgfscope}%
\begin{pgfscope}%
\pgfpathrectangle{\pgfqpoint{1.000000in}{0.979904in}}{\pgfqpoint{6.200000in}{5.960192in}}%
\pgfusepath{clip}%
\pgfsetbuttcap%
\pgfsetroundjoin%
\definecolor{currentfill}{rgb}{0.200000,0.200000,0.800000}%
\pgfsetfillcolor{currentfill}%
\pgfsetlinewidth{1.003750pt}%
\definecolor{currentstroke}{rgb}{0.200000,0.200000,0.800000}%
\pgfsetstrokecolor{currentstroke}%
\pgfsetdash{}{0pt}%
\pgfpathmoveto{\pgfqpoint{3.030822in}{4.789888in}}%
\pgfpathcurveto{\pgfqpoint{3.036646in}{4.789888in}}{\pgfqpoint{3.042232in}{4.792202in}}{\pgfqpoint{3.046351in}{4.796320in}}%
\pgfpathcurveto{\pgfqpoint{3.050469in}{4.800438in}}{\pgfqpoint{3.052783in}{4.806024in}}{\pgfqpoint{3.052783in}{4.811848in}}%
\pgfpathcurveto{\pgfqpoint{3.052783in}{4.817672in}}{\pgfqpoint{3.050469in}{4.823258in}}{\pgfqpoint{3.046351in}{4.827376in}}%
\pgfpathcurveto{\pgfqpoint{3.042232in}{4.831494in}}{\pgfqpoint{3.036646in}{4.833808in}}{\pgfqpoint{3.030822in}{4.833808in}}%
\pgfpathcurveto{\pgfqpoint{3.024998in}{4.833808in}}{\pgfqpoint{3.019412in}{4.831494in}}{\pgfqpoint{3.015294in}{4.827376in}}%
\pgfpathcurveto{\pgfqpoint{3.011176in}{4.823258in}}{\pgfqpoint{3.008862in}{4.817672in}}{\pgfqpoint{3.008862in}{4.811848in}}%
\pgfpathcurveto{\pgfqpoint{3.008862in}{4.806024in}}{\pgfqpoint{3.011176in}{4.800438in}}{\pgfqpoint{3.015294in}{4.796320in}}%
\pgfpathcurveto{\pgfqpoint{3.019412in}{4.792202in}}{\pgfqpoint{3.024998in}{4.789888in}}{\pgfqpoint{3.030822in}{4.789888in}}%
\pgfpathlineto{\pgfqpoint{3.030822in}{4.789888in}}%
\pgfpathclose%
\pgfusepath{stroke,fill}%
\end{pgfscope}%
\begin{pgfscope}%
\pgfpathrectangle{\pgfqpoint{1.000000in}{0.979904in}}{\pgfqpoint{6.200000in}{5.960192in}}%
\pgfusepath{clip}%
\pgfsetbuttcap%
\pgfsetroundjoin%
\definecolor{currentfill}{rgb}{0.200000,0.200000,0.800000}%
\pgfsetfillcolor{currentfill}%
\pgfsetlinewidth{1.003750pt}%
\definecolor{currentstroke}{rgb}{0.200000,0.200000,0.800000}%
\pgfsetstrokecolor{currentstroke}%
\pgfsetdash{}{0pt}%
\pgfpathmoveto{\pgfqpoint{3.214592in}{4.938485in}}%
\pgfpathcurveto{\pgfqpoint{3.220416in}{4.938485in}}{\pgfqpoint{3.226002in}{4.940799in}}{\pgfqpoint{3.230120in}{4.944917in}}%
\pgfpathcurveto{\pgfqpoint{3.234238in}{4.949036in}}{\pgfqpoint{3.236552in}{4.954622in}}{\pgfqpoint{3.236552in}{4.960446in}}%
\pgfpathcurveto{\pgfqpoint{3.236552in}{4.966270in}}{\pgfqpoint{3.234238in}{4.971856in}}{\pgfqpoint{3.230120in}{4.975974in}}%
\pgfpathcurveto{\pgfqpoint{3.226002in}{4.980092in}}{\pgfqpoint{3.220416in}{4.982406in}}{\pgfqpoint{3.214592in}{4.982406in}}%
\pgfpathcurveto{\pgfqpoint{3.208768in}{4.982406in}}{\pgfqpoint{3.203182in}{4.980092in}}{\pgfqpoint{3.199064in}{4.975974in}}%
\pgfpathcurveto{\pgfqpoint{3.194946in}{4.971856in}}{\pgfqpoint{3.192632in}{4.966270in}}{\pgfqpoint{3.192632in}{4.960446in}}%
\pgfpathcurveto{\pgfqpoint{3.192632in}{4.954622in}}{\pgfqpoint{3.194946in}{4.949036in}}{\pgfqpoint{3.199064in}{4.944917in}}%
\pgfpathcurveto{\pgfqpoint{3.203182in}{4.940799in}}{\pgfqpoint{3.208768in}{4.938485in}}{\pgfqpoint{3.214592in}{4.938485in}}%
\pgfpathlineto{\pgfqpoint{3.214592in}{4.938485in}}%
\pgfpathclose%
\pgfusepath{stroke,fill}%
\end{pgfscope}%
\begin{pgfscope}%
\pgfpathrectangle{\pgfqpoint{1.000000in}{0.979904in}}{\pgfqpoint{6.200000in}{5.960192in}}%
\pgfusepath{clip}%
\pgfsetbuttcap%
\pgfsetroundjoin%
\definecolor{currentfill}{rgb}{0.200000,0.800000,0.200000}%
\pgfsetfillcolor{currentfill}%
\pgfsetlinewidth{1.003750pt}%
\definecolor{currentstroke}{rgb}{0.200000,0.800000,0.200000}%
\pgfsetstrokecolor{currentstroke}%
\pgfsetdash{}{0pt}%
\pgfpathmoveto{\pgfqpoint{3.248568in}{4.884843in}}%
\pgfpathcurveto{\pgfqpoint{3.254391in}{4.884843in}}{\pgfqpoint{3.259978in}{4.887157in}}{\pgfqpoint{3.264096in}{4.891275in}}%
\pgfpathcurveto{\pgfqpoint{3.268214in}{4.895393in}}{\pgfqpoint{3.270528in}{4.900979in}}{\pgfqpoint{3.270528in}{4.906803in}}%
\pgfpathcurveto{\pgfqpoint{3.270528in}{4.912627in}}{\pgfqpoint{3.268214in}{4.918213in}}{\pgfqpoint{3.264096in}{4.922332in}}%
\pgfpathcurveto{\pgfqpoint{3.259978in}{4.926450in}}{\pgfqpoint{3.254391in}{4.928764in}}{\pgfqpoint{3.248568in}{4.928764in}}%
\pgfpathcurveto{\pgfqpoint{3.242744in}{4.928764in}}{\pgfqpoint{3.237157in}{4.926450in}}{\pgfqpoint{3.233039in}{4.922332in}}%
\pgfpathcurveto{\pgfqpoint{3.228921in}{4.918213in}}{\pgfqpoint{3.226607in}{4.912627in}}{\pgfqpoint{3.226607in}{4.906803in}}%
\pgfpathcurveto{\pgfqpoint{3.226607in}{4.900979in}}{\pgfqpoint{3.228921in}{4.895393in}}{\pgfqpoint{3.233039in}{4.891275in}}%
\pgfpathcurveto{\pgfqpoint{3.237157in}{4.887157in}}{\pgfqpoint{3.242744in}{4.884843in}}{\pgfqpoint{3.248568in}{4.884843in}}%
\pgfpathlineto{\pgfqpoint{3.248568in}{4.884843in}}%
\pgfpathclose%
\pgfusepath{stroke,fill}%
\end{pgfscope}%
\begin{pgfscope}%
\pgfpathrectangle{\pgfqpoint{1.000000in}{0.979904in}}{\pgfqpoint{6.200000in}{5.960192in}}%
\pgfusepath{clip}%
\pgfsetbuttcap%
\pgfsetroundjoin%
\definecolor{currentfill}{rgb}{0.200000,0.200000,0.800000}%
\pgfsetfillcolor{currentfill}%
\pgfsetlinewidth{1.003750pt}%
\definecolor{currentstroke}{rgb}{0.200000,0.200000,0.800000}%
\pgfsetstrokecolor{currentstroke}%
\pgfsetdash{}{0pt}%
\pgfpathmoveto{\pgfqpoint{3.291760in}{4.841380in}}%
\pgfpathcurveto{\pgfqpoint{3.297584in}{4.841380in}}{\pgfqpoint{3.303170in}{4.843694in}}{\pgfqpoint{3.307288in}{4.847812in}}%
\pgfpathcurveto{\pgfqpoint{3.311406in}{4.851930in}}{\pgfqpoint{3.313720in}{4.857516in}}{\pgfqpoint{3.313720in}{4.863340in}}%
\pgfpathcurveto{\pgfqpoint{3.313720in}{4.869164in}}{\pgfqpoint{3.311406in}{4.874750in}}{\pgfqpoint{3.307288in}{4.878869in}}%
\pgfpathcurveto{\pgfqpoint{3.303170in}{4.882987in}}{\pgfqpoint{3.297584in}{4.885301in}}{\pgfqpoint{3.291760in}{4.885301in}}%
\pgfpathcurveto{\pgfqpoint{3.285936in}{4.885301in}}{\pgfqpoint{3.280350in}{4.882987in}}{\pgfqpoint{3.276232in}{4.878869in}}%
\pgfpathcurveto{\pgfqpoint{3.272113in}{4.874750in}}{\pgfqpoint{3.269800in}{4.869164in}}{\pgfqpoint{3.269800in}{4.863340in}}%
\pgfpathcurveto{\pgfqpoint{3.269800in}{4.857516in}}{\pgfqpoint{3.272113in}{4.851930in}}{\pgfqpoint{3.276232in}{4.847812in}}%
\pgfpathcurveto{\pgfqpoint{3.280350in}{4.843694in}}{\pgfqpoint{3.285936in}{4.841380in}}{\pgfqpoint{3.291760in}{4.841380in}}%
\pgfpathlineto{\pgfqpoint{3.291760in}{4.841380in}}%
\pgfpathclose%
\pgfusepath{stroke,fill}%
\end{pgfscope}%
\begin{pgfscope}%
\pgfpathrectangle{\pgfqpoint{1.000000in}{0.979904in}}{\pgfqpoint{6.200000in}{5.960192in}}%
\pgfusepath{clip}%
\pgfsetbuttcap%
\pgfsetroundjoin%
\definecolor{currentfill}{rgb}{0.200000,0.200000,0.800000}%
\pgfsetfillcolor{currentfill}%
\pgfsetlinewidth{1.003750pt}%
\definecolor{currentstroke}{rgb}{0.200000,0.200000,0.800000}%
\pgfsetstrokecolor{currentstroke}%
\pgfsetdash{}{0pt}%
\pgfpathmoveto{\pgfqpoint{3.373414in}{4.883252in}}%
\pgfpathcurveto{\pgfqpoint{3.379238in}{4.883252in}}{\pgfqpoint{3.384824in}{4.885566in}}{\pgfqpoint{3.388942in}{4.889684in}}%
\pgfpathcurveto{\pgfqpoint{3.393061in}{4.893802in}}{\pgfqpoint{3.395374in}{4.899389in}}{\pgfqpoint{3.395374in}{4.905212in}}%
\pgfpathcurveto{\pgfqpoint{3.395374in}{4.911036in}}{\pgfqpoint{3.393061in}{4.916623in}}{\pgfqpoint{3.388942in}{4.920741in}}%
\pgfpathcurveto{\pgfqpoint{3.384824in}{4.924859in}}{\pgfqpoint{3.379238in}{4.927173in}}{\pgfqpoint{3.373414in}{4.927173in}}%
\pgfpathcurveto{\pgfqpoint{3.367590in}{4.927173in}}{\pgfqpoint{3.362004in}{4.924859in}}{\pgfqpoint{3.357886in}{4.920741in}}%
\pgfpathcurveto{\pgfqpoint{3.353768in}{4.916623in}}{\pgfqpoint{3.351454in}{4.911036in}}{\pgfqpoint{3.351454in}{4.905212in}}%
\pgfpathcurveto{\pgfqpoint{3.351454in}{4.899389in}}{\pgfqpoint{3.353768in}{4.893802in}}{\pgfqpoint{3.357886in}{4.889684in}}%
\pgfpathcurveto{\pgfqpoint{3.362004in}{4.885566in}}{\pgfqpoint{3.367590in}{4.883252in}}{\pgfqpoint{3.373414in}{4.883252in}}%
\pgfpathlineto{\pgfqpoint{3.373414in}{4.883252in}}%
\pgfpathclose%
\pgfusepath{stroke,fill}%
\end{pgfscope}%
\begin{pgfscope}%
\pgfpathrectangle{\pgfqpoint{1.000000in}{0.979904in}}{\pgfqpoint{6.200000in}{5.960192in}}%
\pgfusepath{clip}%
\pgfsetbuttcap%
\pgfsetroundjoin%
\definecolor{currentfill}{rgb}{0.200000,0.200000,0.800000}%
\pgfsetfillcolor{currentfill}%
\pgfsetlinewidth{1.003750pt}%
\definecolor{currentstroke}{rgb}{0.200000,0.200000,0.800000}%
\pgfsetstrokecolor{currentstroke}%
\pgfsetdash{}{0pt}%
\pgfpathmoveto{\pgfqpoint{3.381704in}{4.740365in}}%
\pgfpathcurveto{\pgfqpoint{3.387528in}{4.740365in}}{\pgfqpoint{3.393114in}{4.742679in}}{\pgfqpoint{3.397232in}{4.746797in}}%
\pgfpathcurveto{\pgfqpoint{3.401350in}{4.750915in}}{\pgfqpoint{3.403664in}{4.756501in}}{\pgfqpoint{3.403664in}{4.762325in}}%
\pgfpathcurveto{\pgfqpoint{3.403664in}{4.768149in}}{\pgfqpoint{3.401350in}{4.773735in}}{\pgfqpoint{3.397232in}{4.777853in}}%
\pgfpathcurveto{\pgfqpoint{3.393114in}{4.781971in}}{\pgfqpoint{3.387528in}{4.784285in}}{\pgfqpoint{3.381704in}{4.784285in}}%
\pgfpathcurveto{\pgfqpoint{3.375880in}{4.784285in}}{\pgfqpoint{3.370294in}{4.781971in}}{\pgfqpoint{3.366176in}{4.777853in}}%
\pgfpathcurveto{\pgfqpoint{3.362057in}{4.773735in}}{\pgfqpoint{3.359744in}{4.768149in}}{\pgfqpoint{3.359744in}{4.762325in}}%
\pgfpathcurveto{\pgfqpoint{3.359744in}{4.756501in}}{\pgfqpoint{3.362057in}{4.750915in}}{\pgfqpoint{3.366176in}{4.746797in}}%
\pgfpathcurveto{\pgfqpoint{3.370294in}{4.742679in}}{\pgfqpoint{3.375880in}{4.740365in}}{\pgfqpoint{3.381704in}{4.740365in}}%
\pgfpathlineto{\pgfqpoint{3.381704in}{4.740365in}}%
\pgfpathclose%
\pgfusepath{stroke,fill}%
\end{pgfscope}%
\begin{pgfscope}%
\pgfpathrectangle{\pgfqpoint{1.000000in}{0.979904in}}{\pgfqpoint{6.200000in}{5.960192in}}%
\pgfusepath{clip}%
\pgfsetbuttcap%
\pgfsetroundjoin%
\definecolor{currentfill}{rgb}{0.200000,0.200000,0.800000}%
\pgfsetfillcolor{currentfill}%
\pgfsetlinewidth{1.003750pt}%
\definecolor{currentstroke}{rgb}{0.200000,0.200000,0.800000}%
\pgfsetstrokecolor{currentstroke}%
\pgfsetdash{}{0pt}%
\pgfpathmoveto{\pgfqpoint{3.443714in}{4.728800in}}%
\pgfpathcurveto{\pgfqpoint{3.449538in}{4.728800in}}{\pgfqpoint{3.455124in}{4.731114in}}{\pgfqpoint{3.459243in}{4.735232in}}%
\pgfpathcurveto{\pgfqpoint{3.463361in}{4.739350in}}{\pgfqpoint{3.465675in}{4.744936in}}{\pgfqpoint{3.465675in}{4.750760in}}%
\pgfpathcurveto{\pgfqpoint{3.465675in}{4.756584in}}{\pgfqpoint{3.463361in}{4.762170in}}{\pgfqpoint{3.459243in}{4.766289in}}%
\pgfpathcurveto{\pgfqpoint{3.455124in}{4.770407in}}{\pgfqpoint{3.449538in}{4.772721in}}{\pgfqpoint{3.443714in}{4.772721in}}%
\pgfpathcurveto{\pgfqpoint{3.437890in}{4.772721in}}{\pgfqpoint{3.432304in}{4.770407in}}{\pgfqpoint{3.428186in}{4.766289in}}%
\pgfpathcurveto{\pgfqpoint{3.424068in}{4.762170in}}{\pgfqpoint{3.421754in}{4.756584in}}{\pgfqpoint{3.421754in}{4.750760in}}%
\pgfpathcurveto{\pgfqpoint{3.421754in}{4.744936in}}{\pgfqpoint{3.424068in}{4.739350in}}{\pgfqpoint{3.428186in}{4.735232in}}%
\pgfpathcurveto{\pgfqpoint{3.432304in}{4.731114in}}{\pgfqpoint{3.437890in}{4.728800in}}{\pgfqpoint{3.443714in}{4.728800in}}%
\pgfpathlineto{\pgfqpoint{3.443714in}{4.728800in}}%
\pgfpathclose%
\pgfusepath{stroke,fill}%
\end{pgfscope}%
\begin{pgfscope}%
\pgfpathrectangle{\pgfqpoint{1.000000in}{0.979904in}}{\pgfqpoint{6.200000in}{5.960192in}}%
\pgfusepath{clip}%
\pgfsetbuttcap%
\pgfsetroundjoin%
\definecolor{currentfill}{rgb}{0.200000,0.200000,0.800000}%
\pgfsetfillcolor{currentfill}%
\pgfsetlinewidth{1.003750pt}%
\definecolor{currentstroke}{rgb}{0.200000,0.200000,0.800000}%
\pgfsetstrokecolor{currentstroke}%
\pgfsetdash{}{0pt}%
\pgfpathmoveto{\pgfqpoint{3.513980in}{4.760021in}}%
\pgfpathcurveto{\pgfqpoint{3.519804in}{4.760021in}}{\pgfqpoint{3.525391in}{4.762335in}}{\pgfqpoint{3.529509in}{4.766453in}}%
\pgfpathcurveto{\pgfqpoint{3.533627in}{4.770571in}}{\pgfqpoint{3.535941in}{4.776157in}}{\pgfqpoint{3.535941in}{4.781981in}}%
\pgfpathcurveto{\pgfqpoint{3.535941in}{4.787805in}}{\pgfqpoint{3.533627in}{4.793391in}}{\pgfqpoint{3.529509in}{4.797509in}}%
\pgfpathcurveto{\pgfqpoint{3.525391in}{4.801628in}}{\pgfqpoint{3.519804in}{4.803941in}}{\pgfqpoint{3.513980in}{4.803941in}}%
\pgfpathcurveto{\pgfqpoint{3.508157in}{4.803941in}}{\pgfqpoint{3.502570in}{4.801628in}}{\pgfqpoint{3.498452in}{4.797509in}}%
\pgfpathcurveto{\pgfqpoint{3.494334in}{4.793391in}}{\pgfqpoint{3.492020in}{4.787805in}}{\pgfqpoint{3.492020in}{4.781981in}}%
\pgfpathcurveto{\pgfqpoint{3.492020in}{4.776157in}}{\pgfqpoint{3.494334in}{4.770571in}}{\pgfqpoint{3.498452in}{4.766453in}}%
\pgfpathcurveto{\pgfqpoint{3.502570in}{4.762335in}}{\pgfqpoint{3.508157in}{4.760021in}}{\pgfqpoint{3.513980in}{4.760021in}}%
\pgfpathlineto{\pgfqpoint{3.513980in}{4.760021in}}%
\pgfpathclose%
\pgfusepath{stroke,fill}%
\end{pgfscope}%
\begin{pgfscope}%
\pgfpathrectangle{\pgfqpoint{1.000000in}{0.979904in}}{\pgfqpoint{6.200000in}{5.960192in}}%
\pgfusepath{clip}%
\pgfsetbuttcap%
\pgfsetroundjoin%
\definecolor{currentfill}{rgb}{0.200000,0.200000,0.800000}%
\pgfsetfillcolor{currentfill}%
\pgfsetlinewidth{1.003750pt}%
\definecolor{currentstroke}{rgb}{0.200000,0.200000,0.800000}%
\pgfsetstrokecolor{currentstroke}%
\pgfsetdash{}{0pt}%
\pgfpathmoveto{\pgfqpoint{3.576692in}{4.778764in}}%
\pgfpathcurveto{\pgfqpoint{3.582516in}{4.778764in}}{\pgfqpoint{3.588102in}{4.781077in}}{\pgfqpoint{3.592220in}{4.785196in}}%
\pgfpathcurveto{\pgfqpoint{3.596338in}{4.789314in}}{\pgfqpoint{3.598652in}{4.794900in}}{\pgfqpoint{3.598652in}{4.800724in}}%
\pgfpathcurveto{\pgfqpoint{3.598652in}{4.806548in}}{\pgfqpoint{3.596338in}{4.812134in}}{\pgfqpoint{3.592220in}{4.816252in}}%
\pgfpathcurveto{\pgfqpoint{3.588102in}{4.820370in}}{\pgfqpoint{3.582516in}{4.822684in}}{\pgfqpoint{3.576692in}{4.822684in}}%
\pgfpathcurveto{\pgfqpoint{3.570868in}{4.822684in}}{\pgfqpoint{3.565282in}{4.820370in}}{\pgfqpoint{3.561164in}{4.816252in}}%
\pgfpathcurveto{\pgfqpoint{3.557046in}{4.812134in}}{\pgfqpoint{3.554732in}{4.806548in}}{\pgfqpoint{3.554732in}{4.800724in}}%
\pgfpathcurveto{\pgfqpoint{3.554732in}{4.794900in}}{\pgfqpoint{3.557046in}{4.789314in}}{\pgfqpoint{3.561164in}{4.785196in}}%
\pgfpathcurveto{\pgfqpoint{3.565282in}{4.781077in}}{\pgfqpoint{3.570868in}{4.778764in}}{\pgfqpoint{3.576692in}{4.778764in}}%
\pgfpathlineto{\pgfqpoint{3.576692in}{4.778764in}}%
\pgfpathclose%
\pgfusepath{stroke,fill}%
\end{pgfscope}%
\begin{pgfscope}%
\pgfpathrectangle{\pgfqpoint{1.000000in}{0.979904in}}{\pgfqpoint{6.200000in}{5.960192in}}%
\pgfusepath{clip}%
\pgfsetbuttcap%
\pgfsetroundjoin%
\definecolor{currentfill}{rgb}{0.200000,0.200000,0.800000}%
\pgfsetfillcolor{currentfill}%
\pgfsetlinewidth{1.003750pt}%
\definecolor{currentstroke}{rgb}{0.200000,0.200000,0.800000}%
\pgfsetstrokecolor{currentstroke}%
\pgfsetdash{}{0pt}%
\pgfpathmoveto{\pgfqpoint{3.637949in}{4.823190in}}%
\pgfpathcurveto{\pgfqpoint{3.643773in}{4.823190in}}{\pgfqpoint{3.649359in}{4.825504in}}{\pgfqpoint{3.653477in}{4.829622in}}%
\pgfpathcurveto{\pgfqpoint{3.657595in}{4.833740in}}{\pgfqpoint{3.659909in}{4.839327in}}{\pgfqpoint{3.659909in}{4.845150in}}%
\pgfpathcurveto{\pgfqpoint{3.659909in}{4.850974in}}{\pgfqpoint{3.657595in}{4.856561in}}{\pgfqpoint{3.653477in}{4.860679in}}%
\pgfpathcurveto{\pgfqpoint{3.649359in}{4.864797in}}{\pgfqpoint{3.643773in}{4.867111in}}{\pgfqpoint{3.637949in}{4.867111in}}%
\pgfpathcurveto{\pgfqpoint{3.632125in}{4.867111in}}{\pgfqpoint{3.626539in}{4.864797in}}{\pgfqpoint{3.622421in}{4.860679in}}%
\pgfpathcurveto{\pgfqpoint{3.618303in}{4.856561in}}{\pgfqpoint{3.615989in}{4.850974in}}{\pgfqpoint{3.615989in}{4.845150in}}%
\pgfpathcurveto{\pgfqpoint{3.615989in}{4.839327in}}{\pgfqpoint{3.618303in}{4.833740in}}{\pgfqpoint{3.622421in}{4.829622in}}%
\pgfpathcurveto{\pgfqpoint{3.626539in}{4.825504in}}{\pgfqpoint{3.632125in}{4.823190in}}{\pgfqpoint{3.637949in}{4.823190in}}%
\pgfpathlineto{\pgfqpoint{3.637949in}{4.823190in}}%
\pgfpathclose%
\pgfusepath{stroke,fill}%
\end{pgfscope}%
\begin{pgfscope}%
\pgfpathrectangle{\pgfqpoint{1.000000in}{0.979904in}}{\pgfqpoint{6.200000in}{5.960192in}}%
\pgfusepath{clip}%
\pgfsetbuttcap%
\pgfsetroundjoin%
\definecolor{currentfill}{rgb}{0.200000,0.200000,0.800000}%
\pgfsetfillcolor{currentfill}%
\pgfsetlinewidth{1.003750pt}%
\definecolor{currentstroke}{rgb}{0.200000,0.200000,0.800000}%
\pgfsetstrokecolor{currentstroke}%
\pgfsetdash{}{0pt}%
\pgfpathmoveto{\pgfqpoint{3.691632in}{4.783162in}}%
\pgfpathcurveto{\pgfqpoint{3.697456in}{4.783162in}}{\pgfqpoint{3.703042in}{4.785476in}}{\pgfqpoint{3.707160in}{4.789594in}}%
\pgfpathcurveto{\pgfqpoint{3.711278in}{4.793712in}}{\pgfqpoint{3.713592in}{4.799299in}}{\pgfqpoint{3.713592in}{4.805123in}}%
\pgfpathcurveto{\pgfqpoint{3.713592in}{4.810946in}}{\pgfqpoint{3.711278in}{4.816533in}}{\pgfqpoint{3.707160in}{4.820651in}}%
\pgfpathcurveto{\pgfqpoint{3.703042in}{4.824769in}}{\pgfqpoint{3.697456in}{4.827083in}}{\pgfqpoint{3.691632in}{4.827083in}}%
\pgfpathcurveto{\pgfqpoint{3.685808in}{4.827083in}}{\pgfqpoint{3.680222in}{4.824769in}}{\pgfqpoint{3.676103in}{4.820651in}}%
\pgfpathcurveto{\pgfqpoint{3.671985in}{4.816533in}}{\pgfqpoint{3.669671in}{4.810946in}}{\pgfqpoint{3.669671in}{4.805123in}}%
\pgfpathcurveto{\pgfqpoint{3.669671in}{4.799299in}}{\pgfqpoint{3.671985in}{4.793712in}}{\pgfqpoint{3.676103in}{4.789594in}}%
\pgfpathcurveto{\pgfqpoint{3.680222in}{4.785476in}}{\pgfqpoint{3.685808in}{4.783162in}}{\pgfqpoint{3.691632in}{4.783162in}}%
\pgfpathlineto{\pgfqpoint{3.691632in}{4.783162in}}%
\pgfpathclose%
\pgfusepath{stroke,fill}%
\end{pgfscope}%
\begin{pgfscope}%
\pgfpathrectangle{\pgfqpoint{1.000000in}{0.979904in}}{\pgfqpoint{6.200000in}{5.960192in}}%
\pgfusepath{clip}%
\pgfsetbuttcap%
\pgfsetroundjoin%
\definecolor{currentfill}{rgb}{0.200000,0.200000,0.800000}%
\pgfsetfillcolor{currentfill}%
\pgfsetlinewidth{1.003750pt}%
\definecolor{currentstroke}{rgb}{0.200000,0.200000,0.800000}%
\pgfsetstrokecolor{currentstroke}%
\pgfsetdash{}{0pt}%
\pgfpathmoveto{\pgfqpoint{3.752209in}{4.702820in}}%
\pgfpathcurveto{\pgfqpoint{3.758033in}{4.702820in}}{\pgfqpoint{3.763619in}{4.705134in}}{\pgfqpoint{3.767737in}{4.709252in}}%
\pgfpathcurveto{\pgfqpoint{3.771856in}{4.713370in}}{\pgfqpoint{3.774169in}{4.718956in}}{\pgfqpoint{3.774169in}{4.724780in}}%
\pgfpathcurveto{\pgfqpoint{3.774169in}{4.730604in}}{\pgfqpoint{3.771856in}{4.736190in}}{\pgfqpoint{3.767737in}{4.740308in}}%
\pgfpathcurveto{\pgfqpoint{3.763619in}{4.744427in}}{\pgfqpoint{3.758033in}{4.746741in}}{\pgfqpoint{3.752209in}{4.746741in}}%
\pgfpathcurveto{\pgfqpoint{3.746385in}{4.746741in}}{\pgfqpoint{3.740799in}{4.744427in}}{\pgfqpoint{3.736681in}{4.740308in}}%
\pgfpathcurveto{\pgfqpoint{3.732563in}{4.736190in}}{\pgfqpoint{3.730249in}{4.730604in}}{\pgfqpoint{3.730249in}{4.724780in}}%
\pgfpathcurveto{\pgfqpoint{3.730249in}{4.718956in}}{\pgfqpoint{3.732563in}{4.713370in}}{\pgfqpoint{3.736681in}{4.709252in}}%
\pgfpathcurveto{\pgfqpoint{3.740799in}{4.705134in}}{\pgfqpoint{3.746385in}{4.702820in}}{\pgfqpoint{3.752209in}{4.702820in}}%
\pgfpathlineto{\pgfqpoint{3.752209in}{4.702820in}}%
\pgfpathclose%
\pgfusepath{stroke,fill}%
\end{pgfscope}%
\begin{pgfscope}%
\pgfpathrectangle{\pgfqpoint{1.000000in}{0.979904in}}{\pgfqpoint{6.200000in}{5.960192in}}%
\pgfusepath{clip}%
\pgfsetbuttcap%
\pgfsetroundjoin%
\definecolor{currentfill}{rgb}{0.200000,0.200000,0.800000}%
\pgfsetfillcolor{currentfill}%
\pgfsetlinewidth{1.003750pt}%
\definecolor{currentstroke}{rgb}{0.200000,0.200000,0.800000}%
\pgfsetstrokecolor{currentstroke}%
\pgfsetdash{}{0pt}%
\pgfpathmoveto{\pgfqpoint{3.808269in}{4.758127in}}%
\pgfpathcurveto{\pgfqpoint{3.814093in}{4.758127in}}{\pgfqpoint{3.819679in}{4.760440in}}{\pgfqpoint{3.823797in}{4.764559in}}%
\pgfpathcurveto{\pgfqpoint{3.827916in}{4.768677in}}{\pgfqpoint{3.830229in}{4.774263in}}{\pgfqpoint{3.830229in}{4.780087in}}%
\pgfpathcurveto{\pgfqpoint{3.830229in}{4.785911in}}{\pgfqpoint{3.827916in}{4.791497in}}{\pgfqpoint{3.823797in}{4.795615in}}%
\pgfpathcurveto{\pgfqpoint{3.819679in}{4.799733in}}{\pgfqpoint{3.814093in}{4.802047in}}{\pgfqpoint{3.808269in}{4.802047in}}%
\pgfpathcurveto{\pgfqpoint{3.802445in}{4.802047in}}{\pgfqpoint{3.796859in}{4.799733in}}{\pgfqpoint{3.792741in}{4.795615in}}%
\pgfpathcurveto{\pgfqpoint{3.788623in}{4.791497in}}{\pgfqpoint{3.786309in}{4.785911in}}{\pgfqpoint{3.786309in}{4.780087in}}%
\pgfpathcurveto{\pgfqpoint{3.786309in}{4.774263in}}{\pgfqpoint{3.788623in}{4.768677in}}{\pgfqpoint{3.792741in}{4.764559in}}%
\pgfpathcurveto{\pgfqpoint{3.796859in}{4.760440in}}{\pgfqpoint{3.802445in}{4.758127in}}{\pgfqpoint{3.808269in}{4.758127in}}%
\pgfpathlineto{\pgfqpoint{3.808269in}{4.758127in}}%
\pgfpathclose%
\pgfusepath{stroke,fill}%
\end{pgfscope}%
\begin{pgfscope}%
\pgfpathrectangle{\pgfqpoint{1.000000in}{0.979904in}}{\pgfqpoint{6.200000in}{5.960192in}}%
\pgfusepath{clip}%
\pgfsetbuttcap%
\pgfsetroundjoin%
\definecolor{currentfill}{rgb}{0.200000,0.200000,0.800000}%
\pgfsetfillcolor{currentfill}%
\pgfsetlinewidth{1.003750pt}%
\definecolor{currentstroke}{rgb}{0.200000,0.200000,0.800000}%
\pgfsetstrokecolor{currentstroke}%
\pgfsetdash{}{0pt}%
\pgfpathmoveto{\pgfqpoint{3.864351in}{4.777693in}}%
\pgfpathcurveto{\pgfqpoint{3.870175in}{4.777693in}}{\pgfqpoint{3.875762in}{4.780007in}}{\pgfqpoint{3.879880in}{4.784125in}}%
\pgfpathcurveto{\pgfqpoint{3.883998in}{4.788243in}}{\pgfqpoint{3.886312in}{4.793829in}}{\pgfqpoint{3.886312in}{4.799653in}}%
\pgfpathcurveto{\pgfqpoint{3.886312in}{4.805477in}}{\pgfqpoint{3.883998in}{4.811063in}}{\pgfqpoint{3.879880in}{4.815181in}}%
\pgfpathcurveto{\pgfqpoint{3.875762in}{4.819300in}}{\pgfqpoint{3.870175in}{4.821614in}}{\pgfqpoint{3.864351in}{4.821614in}}%
\pgfpathcurveto{\pgfqpoint{3.858528in}{4.821614in}}{\pgfqpoint{3.852941in}{4.819300in}}{\pgfqpoint{3.848823in}{4.815181in}}%
\pgfpathcurveto{\pgfqpoint{3.844705in}{4.811063in}}{\pgfqpoint{3.842391in}{4.805477in}}{\pgfqpoint{3.842391in}{4.799653in}}%
\pgfpathcurveto{\pgfqpoint{3.842391in}{4.793829in}}{\pgfqpoint{3.844705in}{4.788243in}}{\pgfqpoint{3.848823in}{4.784125in}}%
\pgfpathcurveto{\pgfqpoint{3.852941in}{4.780007in}}{\pgfqpoint{3.858528in}{4.777693in}}{\pgfqpoint{3.864351in}{4.777693in}}%
\pgfpathlineto{\pgfqpoint{3.864351in}{4.777693in}}%
\pgfpathclose%
\pgfusepath{stroke,fill}%
\end{pgfscope}%
\begin{pgfscope}%
\pgfpathrectangle{\pgfqpoint{1.000000in}{0.979904in}}{\pgfqpoint{6.200000in}{5.960192in}}%
\pgfusepath{clip}%
\pgfsetbuttcap%
\pgfsetroundjoin%
\definecolor{currentfill}{rgb}{0.200000,0.200000,0.800000}%
\pgfsetfillcolor{currentfill}%
\pgfsetlinewidth{1.003750pt}%
\definecolor{currentstroke}{rgb}{0.200000,0.200000,0.800000}%
\pgfsetstrokecolor{currentstroke}%
\pgfsetdash{}{0pt}%
\pgfpathmoveto{\pgfqpoint{3.932670in}{4.741693in}}%
\pgfpathcurveto{\pgfqpoint{3.938494in}{4.741693in}}{\pgfqpoint{3.944080in}{4.744007in}}{\pgfqpoint{3.948198in}{4.748125in}}%
\pgfpathcurveto{\pgfqpoint{3.952316in}{4.752244in}}{\pgfqpoint{3.954630in}{4.757830in}}{\pgfqpoint{3.954630in}{4.763654in}}%
\pgfpathcurveto{\pgfqpoint{3.954630in}{4.769478in}}{\pgfqpoint{3.952316in}{4.775064in}}{\pgfqpoint{3.948198in}{4.779182in}}%
\pgfpathcurveto{\pgfqpoint{3.944080in}{4.783300in}}{\pgfqpoint{3.938494in}{4.785614in}}{\pgfqpoint{3.932670in}{4.785614in}}%
\pgfpathcurveto{\pgfqpoint{3.926846in}{4.785614in}}{\pgfqpoint{3.921260in}{4.783300in}}{\pgfqpoint{3.917142in}{4.779182in}}%
\pgfpathcurveto{\pgfqpoint{3.913024in}{4.775064in}}{\pgfqpoint{3.910710in}{4.769478in}}{\pgfqpoint{3.910710in}{4.763654in}}%
\pgfpathcurveto{\pgfqpoint{3.910710in}{4.757830in}}{\pgfqpoint{3.913024in}{4.752244in}}{\pgfqpoint{3.917142in}{4.748125in}}%
\pgfpathcurveto{\pgfqpoint{3.921260in}{4.744007in}}{\pgfqpoint{3.926846in}{4.741693in}}{\pgfqpoint{3.932670in}{4.741693in}}%
\pgfpathlineto{\pgfqpoint{3.932670in}{4.741693in}}%
\pgfpathclose%
\pgfusepath{stroke,fill}%
\end{pgfscope}%
\begin{pgfscope}%
\pgfpathrectangle{\pgfqpoint{1.000000in}{0.979904in}}{\pgfqpoint{6.200000in}{5.960192in}}%
\pgfusepath{clip}%
\pgfsetbuttcap%
\pgfsetroundjoin%
\definecolor{currentfill}{rgb}{0.200000,0.200000,0.800000}%
\pgfsetfillcolor{currentfill}%
\pgfsetlinewidth{1.003750pt}%
\definecolor{currentstroke}{rgb}{0.200000,0.200000,0.800000}%
\pgfsetstrokecolor{currentstroke}%
\pgfsetdash{}{0pt}%
\pgfpathmoveto{\pgfqpoint{4.001868in}{4.724691in}}%
\pgfpathcurveto{\pgfqpoint{4.007692in}{4.724691in}}{\pgfqpoint{4.013278in}{4.727005in}}{\pgfqpoint{4.017397in}{4.731123in}}%
\pgfpathcurveto{\pgfqpoint{4.021515in}{4.735241in}}{\pgfqpoint{4.023829in}{4.740828in}}{\pgfqpoint{4.023829in}{4.746652in}}%
\pgfpathcurveto{\pgfqpoint{4.023829in}{4.752476in}}{\pgfqpoint{4.021515in}{4.758062in}}{\pgfqpoint{4.017397in}{4.762180in}}%
\pgfpathcurveto{\pgfqpoint{4.013278in}{4.766298in}}{\pgfqpoint{4.007692in}{4.768612in}}{\pgfqpoint{4.001868in}{4.768612in}}%
\pgfpathcurveto{\pgfqpoint{3.996044in}{4.768612in}}{\pgfqpoint{3.990458in}{4.766298in}}{\pgfqpoint{3.986340in}{4.762180in}}%
\pgfpathcurveto{\pgfqpoint{3.982222in}{4.758062in}}{\pgfqpoint{3.979908in}{4.752476in}}{\pgfqpoint{3.979908in}{4.746652in}}%
\pgfpathcurveto{\pgfqpoint{3.979908in}{4.740828in}}{\pgfqpoint{3.982222in}{4.735241in}}{\pgfqpoint{3.986340in}{4.731123in}}%
\pgfpathcurveto{\pgfqpoint{3.990458in}{4.727005in}}{\pgfqpoint{3.996044in}{4.724691in}}{\pgfqpoint{4.001868in}{4.724691in}}%
\pgfpathlineto{\pgfqpoint{4.001868in}{4.724691in}}%
\pgfpathclose%
\pgfusepath{stroke,fill}%
\end{pgfscope}%
\begin{pgfscope}%
\pgfpathrectangle{\pgfqpoint{1.000000in}{0.979904in}}{\pgfqpoint{6.200000in}{5.960192in}}%
\pgfusepath{clip}%
\pgfsetbuttcap%
\pgfsetroundjoin%
\definecolor{currentfill}{rgb}{0.200000,0.200000,0.800000}%
\pgfsetfillcolor{currentfill}%
\pgfsetlinewidth{1.003750pt}%
\definecolor{currentstroke}{rgb}{0.200000,0.200000,0.800000}%
\pgfsetstrokecolor{currentstroke}%
\pgfsetdash{}{0pt}%
\pgfpathmoveto{\pgfqpoint{4.044598in}{4.790045in}}%
\pgfpathcurveto{\pgfqpoint{4.050422in}{4.790045in}}{\pgfqpoint{4.056008in}{4.792359in}}{\pgfqpoint{4.060126in}{4.796477in}}%
\pgfpathcurveto{\pgfqpoint{4.064244in}{4.800595in}}{\pgfqpoint{4.066558in}{4.806181in}}{\pgfqpoint{4.066558in}{4.812005in}}%
\pgfpathcurveto{\pgfqpoint{4.066558in}{4.817829in}}{\pgfqpoint{4.064244in}{4.823415in}}{\pgfqpoint{4.060126in}{4.827533in}}%
\pgfpathcurveto{\pgfqpoint{4.056008in}{4.831651in}}{\pgfqpoint{4.050422in}{4.833965in}}{\pgfqpoint{4.044598in}{4.833965in}}%
\pgfpathcurveto{\pgfqpoint{4.038774in}{4.833965in}}{\pgfqpoint{4.033188in}{4.831651in}}{\pgfqpoint{4.029069in}{4.827533in}}%
\pgfpathcurveto{\pgfqpoint{4.024951in}{4.823415in}}{\pgfqpoint{4.022637in}{4.817829in}}{\pgfqpoint{4.022637in}{4.812005in}}%
\pgfpathcurveto{\pgfqpoint{4.022637in}{4.806181in}}{\pgfqpoint{4.024951in}{4.800595in}}{\pgfqpoint{4.029069in}{4.796477in}}%
\pgfpathcurveto{\pgfqpoint{4.033188in}{4.792359in}}{\pgfqpoint{4.038774in}{4.790045in}}{\pgfqpoint{4.044598in}{4.790045in}}%
\pgfpathlineto{\pgfqpoint{4.044598in}{4.790045in}}%
\pgfpathclose%
\pgfusepath{stroke,fill}%
\end{pgfscope}%
\begin{pgfscope}%
\pgfpathrectangle{\pgfqpoint{1.000000in}{0.979904in}}{\pgfqpoint{6.200000in}{5.960192in}}%
\pgfusepath{clip}%
\pgfsetbuttcap%
\pgfsetroundjoin%
\definecolor{currentfill}{rgb}{0.200000,0.200000,0.800000}%
\pgfsetfillcolor{currentfill}%
\pgfsetlinewidth{1.003750pt}%
\definecolor{currentstroke}{rgb}{0.200000,0.200000,0.800000}%
\pgfsetstrokecolor{currentstroke}%
\pgfsetdash{}{0pt}%
\pgfpathmoveto{\pgfqpoint{4.101423in}{4.810547in}}%
\pgfpathcurveto{\pgfqpoint{4.107247in}{4.810547in}}{\pgfqpoint{4.112833in}{4.812861in}}{\pgfqpoint{4.116951in}{4.816979in}}%
\pgfpathcurveto{\pgfqpoint{4.121069in}{4.821097in}}{\pgfqpoint{4.123383in}{4.826684in}}{\pgfqpoint{4.123383in}{4.832507in}}%
\pgfpathcurveto{\pgfqpoint{4.123383in}{4.838331in}}{\pgfqpoint{4.121069in}{4.843918in}}{\pgfqpoint{4.116951in}{4.848036in}}%
\pgfpathcurveto{\pgfqpoint{4.112833in}{4.852154in}}{\pgfqpoint{4.107247in}{4.854468in}}{\pgfqpoint{4.101423in}{4.854468in}}%
\pgfpathcurveto{\pgfqpoint{4.095599in}{4.854468in}}{\pgfqpoint{4.090013in}{4.852154in}}{\pgfqpoint{4.085895in}{4.848036in}}%
\pgfpathcurveto{\pgfqpoint{4.081777in}{4.843918in}}{\pgfqpoint{4.079463in}{4.838331in}}{\pgfqpoint{4.079463in}{4.832507in}}%
\pgfpathcurveto{\pgfqpoint{4.079463in}{4.826684in}}{\pgfqpoint{4.081777in}{4.821097in}}{\pgfqpoint{4.085895in}{4.816979in}}%
\pgfpathcurveto{\pgfqpoint{4.090013in}{4.812861in}}{\pgfqpoint{4.095599in}{4.810547in}}{\pgfqpoint{4.101423in}{4.810547in}}%
\pgfpathlineto{\pgfqpoint{4.101423in}{4.810547in}}%
\pgfpathclose%
\pgfusepath{stroke,fill}%
\end{pgfscope}%
\begin{pgfscope}%
\pgfpathrectangle{\pgfqpoint{1.000000in}{0.979904in}}{\pgfqpoint{6.200000in}{5.960192in}}%
\pgfusepath{clip}%
\pgfsetbuttcap%
\pgfsetroundjoin%
\definecolor{currentfill}{rgb}{0.200000,0.200000,0.800000}%
\pgfsetfillcolor{currentfill}%
\pgfsetlinewidth{1.003750pt}%
\definecolor{currentstroke}{rgb}{0.200000,0.200000,0.800000}%
\pgfsetstrokecolor{currentstroke}%
\pgfsetdash{}{0pt}%
\pgfpathmoveto{\pgfqpoint{4.182063in}{4.787933in}}%
\pgfpathcurveto{\pgfqpoint{4.187887in}{4.787933in}}{\pgfqpoint{4.193473in}{4.790247in}}{\pgfqpoint{4.197591in}{4.794365in}}%
\pgfpathcurveto{\pgfqpoint{4.201709in}{4.798484in}}{\pgfqpoint{4.204023in}{4.804070in}}{\pgfqpoint{4.204023in}{4.809894in}}%
\pgfpathcurveto{\pgfqpoint{4.204023in}{4.815718in}}{\pgfqpoint{4.201709in}{4.821304in}}{\pgfqpoint{4.197591in}{4.825422in}}%
\pgfpathcurveto{\pgfqpoint{4.193473in}{4.829540in}}{\pgfqpoint{4.187887in}{4.831854in}}{\pgfqpoint{4.182063in}{4.831854in}}%
\pgfpathcurveto{\pgfqpoint{4.176239in}{4.831854in}}{\pgfqpoint{4.170653in}{4.829540in}}{\pgfqpoint{4.166534in}{4.825422in}}%
\pgfpathcurveto{\pgfqpoint{4.162416in}{4.821304in}}{\pgfqpoint{4.160102in}{4.815718in}}{\pgfqpoint{4.160102in}{4.809894in}}%
\pgfpathcurveto{\pgfqpoint{4.160102in}{4.804070in}}{\pgfqpoint{4.162416in}{4.798484in}}{\pgfqpoint{4.166534in}{4.794365in}}%
\pgfpathcurveto{\pgfqpoint{4.170653in}{4.790247in}}{\pgfqpoint{4.176239in}{4.787933in}}{\pgfqpoint{4.182063in}{4.787933in}}%
\pgfpathlineto{\pgfqpoint{4.182063in}{4.787933in}}%
\pgfpathclose%
\pgfusepath{stroke,fill}%
\end{pgfscope}%
\begin{pgfscope}%
\pgfpathrectangle{\pgfqpoint{1.000000in}{0.979904in}}{\pgfqpoint{6.200000in}{5.960192in}}%
\pgfusepath{clip}%
\pgfsetbuttcap%
\pgfsetroundjoin%
\definecolor{currentfill}{rgb}{0.200000,0.200000,0.800000}%
\pgfsetfillcolor{currentfill}%
\pgfsetlinewidth{1.003750pt}%
\definecolor{currentstroke}{rgb}{0.200000,0.200000,0.800000}%
\pgfsetstrokecolor{currentstroke}%
\pgfsetdash{}{0pt}%
\pgfpathmoveto{\pgfqpoint{4.204016in}{4.873863in}}%
\pgfpathcurveto{\pgfqpoint{4.209840in}{4.873863in}}{\pgfqpoint{4.215426in}{4.876177in}}{\pgfqpoint{4.219544in}{4.880295in}}%
\pgfpathcurveto{\pgfqpoint{4.223662in}{4.884413in}}{\pgfqpoint{4.225976in}{4.889999in}}{\pgfqpoint{4.225976in}{4.895823in}}%
\pgfpathcurveto{\pgfqpoint{4.225976in}{4.901647in}}{\pgfqpoint{4.223662in}{4.907233in}}{\pgfqpoint{4.219544in}{4.911351in}}%
\pgfpathcurveto{\pgfqpoint{4.215426in}{4.915469in}}{\pgfqpoint{4.209840in}{4.917783in}}{\pgfqpoint{4.204016in}{4.917783in}}%
\pgfpathcurveto{\pgfqpoint{4.198192in}{4.917783in}}{\pgfqpoint{4.192606in}{4.915469in}}{\pgfqpoint{4.188488in}{4.911351in}}%
\pgfpathcurveto{\pgfqpoint{4.184370in}{4.907233in}}{\pgfqpoint{4.182056in}{4.901647in}}{\pgfqpoint{4.182056in}{4.895823in}}%
\pgfpathcurveto{\pgfqpoint{4.182056in}{4.889999in}}{\pgfqpoint{4.184370in}{4.884413in}}{\pgfqpoint{4.188488in}{4.880295in}}%
\pgfpathcurveto{\pgfqpoint{4.192606in}{4.876177in}}{\pgfqpoint{4.198192in}{4.873863in}}{\pgfqpoint{4.204016in}{4.873863in}}%
\pgfpathlineto{\pgfqpoint{4.204016in}{4.873863in}}%
\pgfpathclose%
\pgfusepath{stroke,fill}%
\end{pgfscope}%
\begin{pgfscope}%
\pgfpathrectangle{\pgfqpoint{1.000000in}{0.979904in}}{\pgfqpoint{6.200000in}{5.960192in}}%
\pgfusepath{clip}%
\pgfsetbuttcap%
\pgfsetroundjoin%
\definecolor{currentfill}{rgb}{0.200000,0.200000,0.800000}%
\pgfsetfillcolor{currentfill}%
\pgfsetlinewidth{1.003750pt}%
\definecolor{currentstroke}{rgb}{0.200000,0.200000,0.800000}%
\pgfsetstrokecolor{currentstroke}%
\pgfsetdash{}{0pt}%
\pgfpathmoveto{\pgfqpoint{4.239018in}{4.928032in}}%
\pgfpathcurveto{\pgfqpoint{4.244842in}{4.928032in}}{\pgfqpoint{4.250428in}{4.930345in}}{\pgfqpoint{4.254546in}{4.934464in}}%
\pgfpathcurveto{\pgfqpoint{4.258664in}{4.938582in}}{\pgfqpoint{4.260978in}{4.944168in}}{\pgfqpoint{4.260978in}{4.949992in}}%
\pgfpathcurveto{\pgfqpoint{4.260978in}{4.955816in}}{\pgfqpoint{4.258664in}{4.961402in}}{\pgfqpoint{4.254546in}{4.965520in}}%
\pgfpathcurveto{\pgfqpoint{4.250428in}{4.969638in}}{\pgfqpoint{4.244842in}{4.971952in}}{\pgfqpoint{4.239018in}{4.971952in}}%
\pgfpathcurveto{\pgfqpoint{4.233194in}{4.971952in}}{\pgfqpoint{4.227608in}{4.969638in}}{\pgfqpoint{4.223490in}{4.965520in}}%
\pgfpathcurveto{\pgfqpoint{4.219371in}{4.961402in}}{\pgfqpoint{4.217058in}{4.955816in}}{\pgfqpoint{4.217058in}{4.949992in}}%
\pgfpathcurveto{\pgfqpoint{4.217058in}{4.944168in}}{\pgfqpoint{4.219371in}{4.938582in}}{\pgfqpoint{4.223490in}{4.934464in}}%
\pgfpathcurveto{\pgfqpoint{4.227608in}{4.930345in}}{\pgfqpoint{4.233194in}{4.928032in}}{\pgfqpoint{4.239018in}{4.928032in}}%
\pgfpathlineto{\pgfqpoint{4.239018in}{4.928032in}}%
\pgfpathclose%
\pgfusepath{stroke,fill}%
\end{pgfscope}%
\begin{pgfscope}%
\pgfpathrectangle{\pgfqpoint{1.000000in}{0.979904in}}{\pgfqpoint{6.200000in}{5.960192in}}%
\pgfusepath{clip}%
\pgfsetbuttcap%
\pgfsetroundjoin%
\definecolor{currentfill}{rgb}{0.200000,0.200000,0.800000}%
\pgfsetfillcolor{currentfill}%
\pgfsetlinewidth{1.003750pt}%
\definecolor{currentstroke}{rgb}{0.200000,0.200000,0.800000}%
\pgfsetstrokecolor{currentstroke}%
\pgfsetdash{}{0pt}%
\pgfpathmoveto{\pgfqpoint{4.318302in}{4.922901in}}%
\pgfpathcurveto{\pgfqpoint{4.324126in}{4.922901in}}{\pgfqpoint{4.329712in}{4.925215in}}{\pgfqpoint{4.333830in}{4.929333in}}%
\pgfpathcurveto{\pgfqpoint{4.337948in}{4.933452in}}{\pgfqpoint{4.340262in}{4.939038in}}{\pgfqpoint{4.340262in}{4.944862in}}%
\pgfpathcurveto{\pgfqpoint{4.340262in}{4.950686in}}{\pgfqpoint{4.337948in}{4.956272in}}{\pgfqpoint{4.333830in}{4.960390in}}%
\pgfpathcurveto{\pgfqpoint{4.329712in}{4.964508in}}{\pgfqpoint{4.324126in}{4.966822in}}{\pgfqpoint{4.318302in}{4.966822in}}%
\pgfpathcurveto{\pgfqpoint{4.312478in}{4.966822in}}{\pgfqpoint{4.306892in}{4.964508in}}{\pgfqpoint{4.302773in}{4.960390in}}%
\pgfpathcurveto{\pgfqpoint{4.298655in}{4.956272in}}{\pgfqpoint{4.296341in}{4.950686in}}{\pgfqpoint{4.296341in}{4.944862in}}%
\pgfpathcurveto{\pgfqpoint{4.296341in}{4.939038in}}{\pgfqpoint{4.298655in}{4.933452in}}{\pgfqpoint{4.302773in}{4.929333in}}%
\pgfpathcurveto{\pgfqpoint{4.306892in}{4.925215in}}{\pgfqpoint{4.312478in}{4.922901in}}{\pgfqpoint{4.318302in}{4.922901in}}%
\pgfpathlineto{\pgfqpoint{4.318302in}{4.922901in}}%
\pgfpathclose%
\pgfusepath{stroke,fill}%
\end{pgfscope}%
\begin{pgfscope}%
\pgfpathrectangle{\pgfqpoint{1.000000in}{0.979904in}}{\pgfqpoint{6.200000in}{5.960192in}}%
\pgfusepath{clip}%
\pgfsetbuttcap%
\pgfsetroundjoin%
\definecolor{currentfill}{rgb}{0.200000,0.200000,0.800000}%
\pgfsetfillcolor{currentfill}%
\pgfsetlinewidth{1.003750pt}%
\definecolor{currentstroke}{rgb}{0.200000,0.200000,0.800000}%
\pgfsetstrokecolor{currentstroke}%
\pgfsetdash{}{0pt}%
\pgfpathmoveto{\pgfqpoint{4.331414in}{4.999503in}}%
\pgfpathcurveto{\pgfqpoint{4.337238in}{4.999503in}}{\pgfqpoint{4.342824in}{5.001817in}}{\pgfqpoint{4.346942in}{5.005935in}}%
\pgfpathcurveto{\pgfqpoint{4.351060in}{5.010053in}}{\pgfqpoint{4.353374in}{5.015639in}}{\pgfqpoint{4.353374in}{5.021463in}}%
\pgfpathcurveto{\pgfqpoint{4.353374in}{5.027287in}}{\pgfqpoint{4.351060in}{5.032873in}}{\pgfqpoint{4.346942in}{5.036992in}}%
\pgfpathcurveto{\pgfqpoint{4.342824in}{5.041110in}}{\pgfqpoint{4.337238in}{5.043424in}}{\pgfqpoint{4.331414in}{5.043424in}}%
\pgfpathcurveto{\pgfqpoint{4.325590in}{5.043424in}}{\pgfqpoint{4.320004in}{5.041110in}}{\pgfqpoint{4.315886in}{5.036992in}}%
\pgfpathcurveto{\pgfqpoint{4.311768in}{5.032873in}}{\pgfqpoint{4.309454in}{5.027287in}}{\pgfqpoint{4.309454in}{5.021463in}}%
\pgfpathcurveto{\pgfqpoint{4.309454in}{5.015639in}}{\pgfqpoint{4.311768in}{5.010053in}}{\pgfqpoint{4.315886in}{5.005935in}}%
\pgfpathcurveto{\pgfqpoint{4.320004in}{5.001817in}}{\pgfqpoint{4.325590in}{4.999503in}}{\pgfqpoint{4.331414in}{4.999503in}}%
\pgfpathlineto{\pgfqpoint{4.331414in}{4.999503in}}%
\pgfpathclose%
\pgfusepath{stroke,fill}%
\end{pgfscope}%
\begin{pgfscope}%
\pgfpathrectangle{\pgfqpoint{1.000000in}{0.979904in}}{\pgfqpoint{6.200000in}{5.960192in}}%
\pgfusepath{clip}%
\pgfsetbuttcap%
\pgfsetroundjoin%
\definecolor{currentfill}{rgb}{0.200000,0.200000,0.800000}%
\pgfsetfillcolor{currentfill}%
\pgfsetlinewidth{1.003750pt}%
\definecolor{currentstroke}{rgb}{0.200000,0.200000,0.800000}%
\pgfsetstrokecolor{currentstroke}%
\pgfsetdash{}{0pt}%
\pgfpathmoveto{\pgfqpoint{4.368842in}{5.044616in}}%
\pgfpathcurveto{\pgfqpoint{4.374666in}{5.044616in}}{\pgfqpoint{4.380252in}{5.046930in}}{\pgfqpoint{4.384371in}{5.051048in}}%
\pgfpathcurveto{\pgfqpoint{4.388489in}{5.055166in}}{\pgfqpoint{4.390803in}{5.060752in}}{\pgfqpoint{4.390803in}{5.066576in}}%
\pgfpathcurveto{\pgfqpoint{4.390803in}{5.072400in}}{\pgfqpoint{4.388489in}{5.077986in}}{\pgfqpoint{4.384371in}{5.082105in}}%
\pgfpathcurveto{\pgfqpoint{4.380252in}{5.086223in}}{\pgfqpoint{4.374666in}{5.088537in}}{\pgfqpoint{4.368842in}{5.088537in}}%
\pgfpathcurveto{\pgfqpoint{4.363018in}{5.088537in}}{\pgfqpoint{4.357432in}{5.086223in}}{\pgfqpoint{4.353314in}{5.082105in}}%
\pgfpathcurveto{\pgfqpoint{4.349196in}{5.077986in}}{\pgfqpoint{4.346882in}{5.072400in}}{\pgfqpoint{4.346882in}{5.066576in}}%
\pgfpathcurveto{\pgfqpoint{4.346882in}{5.060752in}}{\pgfqpoint{4.349196in}{5.055166in}}{\pgfqpoint{4.353314in}{5.051048in}}%
\pgfpathcurveto{\pgfqpoint{4.357432in}{5.046930in}}{\pgfqpoint{4.363018in}{5.044616in}}{\pgfqpoint{4.368842in}{5.044616in}}%
\pgfpathlineto{\pgfqpoint{4.368842in}{5.044616in}}%
\pgfpathclose%
\pgfusepath{stroke,fill}%
\end{pgfscope}%
\begin{pgfscope}%
\pgfpathrectangle{\pgfqpoint{1.000000in}{0.979904in}}{\pgfqpoint{6.200000in}{5.960192in}}%
\pgfusepath{clip}%
\pgfsetbuttcap%
\pgfsetroundjoin%
\definecolor{currentfill}{rgb}{0.200000,0.200000,0.800000}%
\pgfsetfillcolor{currentfill}%
\pgfsetlinewidth{1.003750pt}%
\definecolor{currentstroke}{rgb}{0.200000,0.200000,0.800000}%
\pgfsetstrokecolor{currentstroke}%
\pgfsetdash{}{0pt}%
\pgfpathmoveto{\pgfqpoint{4.474323in}{5.031957in}}%
\pgfpathcurveto{\pgfqpoint{4.480147in}{5.031957in}}{\pgfqpoint{4.485733in}{5.034271in}}{\pgfqpoint{4.489851in}{5.038389in}}%
\pgfpathcurveto{\pgfqpoint{4.493969in}{5.042507in}}{\pgfqpoint{4.496283in}{5.048093in}}{\pgfqpoint{4.496283in}{5.053917in}}%
\pgfpathcurveto{\pgfqpoint{4.496283in}{5.059741in}}{\pgfqpoint{4.493969in}{5.065327in}}{\pgfqpoint{4.489851in}{5.069445in}}%
\pgfpathcurveto{\pgfqpoint{4.485733in}{5.073563in}}{\pgfqpoint{4.480147in}{5.075877in}}{\pgfqpoint{4.474323in}{5.075877in}}%
\pgfpathcurveto{\pgfqpoint{4.468499in}{5.075877in}}{\pgfqpoint{4.462913in}{5.073563in}}{\pgfqpoint{4.458795in}{5.069445in}}%
\pgfpathcurveto{\pgfqpoint{4.454677in}{5.065327in}}{\pgfqpoint{4.452363in}{5.059741in}}{\pgfqpoint{4.452363in}{5.053917in}}%
\pgfpathcurveto{\pgfqpoint{4.452363in}{5.048093in}}{\pgfqpoint{4.454677in}{5.042507in}}{\pgfqpoint{4.458795in}{5.038389in}}%
\pgfpathcurveto{\pgfqpoint{4.462913in}{5.034271in}}{\pgfqpoint{4.468499in}{5.031957in}}{\pgfqpoint{4.474323in}{5.031957in}}%
\pgfpathlineto{\pgfqpoint{4.474323in}{5.031957in}}%
\pgfpathclose%
\pgfusepath{stroke,fill}%
\end{pgfscope}%
\begin{pgfscope}%
\pgfpathrectangle{\pgfqpoint{1.000000in}{0.979904in}}{\pgfqpoint{6.200000in}{5.960192in}}%
\pgfusepath{clip}%
\pgfsetbuttcap%
\pgfsetroundjoin%
\definecolor{currentfill}{rgb}{0.200000,0.200000,0.800000}%
\pgfsetfillcolor{currentfill}%
\pgfsetlinewidth{1.003750pt}%
\definecolor{currentstroke}{rgb}{0.200000,0.200000,0.800000}%
\pgfsetstrokecolor{currentstroke}%
\pgfsetdash{}{0pt}%
\pgfpathmoveto{\pgfqpoint{4.446781in}{5.131250in}}%
\pgfpathcurveto{\pgfqpoint{4.452605in}{5.131250in}}{\pgfqpoint{4.458191in}{5.133564in}}{\pgfqpoint{4.462310in}{5.137682in}}%
\pgfpathcurveto{\pgfqpoint{4.466428in}{5.141800in}}{\pgfqpoint{4.468742in}{5.147386in}}{\pgfqpoint{4.468742in}{5.153210in}}%
\pgfpathcurveto{\pgfqpoint{4.468742in}{5.159034in}}{\pgfqpoint{4.466428in}{5.164620in}}{\pgfqpoint{4.462310in}{5.168739in}}%
\pgfpathcurveto{\pgfqpoint{4.458191in}{5.172857in}}{\pgfqpoint{4.452605in}{5.175171in}}{\pgfqpoint{4.446781in}{5.175171in}}%
\pgfpathcurveto{\pgfqpoint{4.440957in}{5.175171in}}{\pgfqpoint{4.435371in}{5.172857in}}{\pgfqpoint{4.431253in}{5.168739in}}%
\pgfpathcurveto{\pgfqpoint{4.427135in}{5.164620in}}{\pgfqpoint{4.424821in}{5.159034in}}{\pgfqpoint{4.424821in}{5.153210in}}%
\pgfpathcurveto{\pgfqpoint{4.424821in}{5.147386in}}{\pgfqpoint{4.427135in}{5.141800in}}{\pgfqpoint{4.431253in}{5.137682in}}%
\pgfpathcurveto{\pgfqpoint{4.435371in}{5.133564in}}{\pgfqpoint{4.440957in}{5.131250in}}{\pgfqpoint{4.446781in}{5.131250in}}%
\pgfpathlineto{\pgfqpoint{4.446781in}{5.131250in}}%
\pgfpathclose%
\pgfusepath{stroke,fill}%
\end{pgfscope}%
\begin{pgfscope}%
\pgfpathrectangle{\pgfqpoint{1.000000in}{0.979904in}}{\pgfqpoint{6.200000in}{5.960192in}}%
\pgfusepath{clip}%
\pgfsetbuttcap%
\pgfsetroundjoin%
\definecolor{currentfill}{rgb}{0.200000,0.200000,0.800000}%
\pgfsetfillcolor{currentfill}%
\pgfsetlinewidth{1.003750pt}%
\definecolor{currentstroke}{rgb}{0.200000,0.200000,0.800000}%
\pgfsetstrokecolor{currentstroke}%
\pgfsetdash{}{0pt}%
\pgfpathmoveto{\pgfqpoint{4.374411in}{5.247132in}}%
\pgfpathcurveto{\pgfqpoint{4.380235in}{5.247132in}}{\pgfqpoint{4.385821in}{5.249446in}}{\pgfqpoint{4.389939in}{5.253564in}}%
\pgfpathcurveto{\pgfqpoint{4.394057in}{5.257682in}}{\pgfqpoint{4.396371in}{5.263268in}}{\pgfqpoint{4.396371in}{5.269092in}}%
\pgfpathcurveto{\pgfqpoint{4.396371in}{5.274916in}}{\pgfqpoint{4.394057in}{5.280502in}}{\pgfqpoint{4.389939in}{5.284620in}}%
\pgfpathcurveto{\pgfqpoint{4.385821in}{5.288738in}}{\pgfqpoint{4.380235in}{5.291052in}}{\pgfqpoint{4.374411in}{5.291052in}}%
\pgfpathcurveto{\pgfqpoint{4.368587in}{5.291052in}}{\pgfqpoint{4.363001in}{5.288738in}}{\pgfqpoint{4.358883in}{5.284620in}}%
\pgfpathcurveto{\pgfqpoint{4.354765in}{5.280502in}}{\pgfqpoint{4.352451in}{5.274916in}}{\pgfqpoint{4.352451in}{5.269092in}}%
\pgfpathcurveto{\pgfqpoint{4.352451in}{5.263268in}}{\pgfqpoint{4.354765in}{5.257682in}}{\pgfqpoint{4.358883in}{5.253564in}}%
\pgfpathcurveto{\pgfqpoint{4.363001in}{5.249446in}}{\pgfqpoint{4.368587in}{5.247132in}}{\pgfqpoint{4.374411in}{5.247132in}}%
\pgfpathlineto{\pgfqpoint{4.374411in}{5.247132in}}%
\pgfpathclose%
\pgfusepath{stroke,fill}%
\end{pgfscope}%
\begin{pgfscope}%
\pgfpathrectangle{\pgfqpoint{1.000000in}{0.979904in}}{\pgfqpoint{6.200000in}{5.960192in}}%
\pgfusepath{clip}%
\pgfsetbuttcap%
\pgfsetroundjoin%
\definecolor{currentfill}{rgb}{0.200000,0.200000,0.800000}%
\pgfsetfillcolor{currentfill}%
\pgfsetlinewidth{1.003750pt}%
\definecolor{currentstroke}{rgb}{0.200000,0.200000,0.800000}%
\pgfsetstrokecolor{currentstroke}%
\pgfsetdash{}{0pt}%
\pgfpathmoveto{\pgfqpoint{4.539440in}{5.212996in}}%
\pgfpathcurveto{\pgfqpoint{4.545264in}{5.212996in}}{\pgfqpoint{4.550850in}{5.215310in}}{\pgfqpoint{4.554968in}{5.219428in}}%
\pgfpathcurveto{\pgfqpoint{4.559086in}{5.223546in}}{\pgfqpoint{4.561400in}{5.229132in}}{\pgfqpoint{4.561400in}{5.234956in}}%
\pgfpathcurveto{\pgfqpoint{4.561400in}{5.240780in}}{\pgfqpoint{4.559086in}{5.246366in}}{\pgfqpoint{4.554968in}{5.250485in}}%
\pgfpathcurveto{\pgfqpoint{4.550850in}{5.254603in}}{\pgfqpoint{4.545264in}{5.256917in}}{\pgfqpoint{4.539440in}{5.256917in}}%
\pgfpathcurveto{\pgfqpoint{4.533616in}{5.256917in}}{\pgfqpoint{4.528030in}{5.254603in}}{\pgfqpoint{4.523912in}{5.250485in}}%
\pgfpathcurveto{\pgfqpoint{4.519793in}{5.246366in}}{\pgfqpoint{4.517480in}{5.240780in}}{\pgfqpoint{4.517480in}{5.234956in}}%
\pgfpathcurveto{\pgfqpoint{4.517480in}{5.229132in}}{\pgfqpoint{4.519793in}{5.223546in}}{\pgfqpoint{4.523912in}{5.219428in}}%
\pgfpathcurveto{\pgfqpoint{4.528030in}{5.215310in}}{\pgfqpoint{4.533616in}{5.212996in}}{\pgfqpoint{4.539440in}{5.212996in}}%
\pgfpathlineto{\pgfqpoint{4.539440in}{5.212996in}}%
\pgfpathclose%
\pgfusepath{stroke,fill}%
\end{pgfscope}%
\begin{pgfscope}%
\pgfpathrectangle{\pgfqpoint{1.000000in}{0.979904in}}{\pgfqpoint{6.200000in}{5.960192in}}%
\pgfusepath{clip}%
\pgfsetbuttcap%
\pgfsetroundjoin%
\definecolor{currentfill}{rgb}{0.200000,0.200000,0.800000}%
\pgfsetfillcolor{currentfill}%
\pgfsetlinewidth{1.003750pt}%
\definecolor{currentstroke}{rgb}{0.200000,0.200000,0.800000}%
\pgfsetstrokecolor{currentstroke}%
\pgfsetdash{}{0pt}%
\pgfpathmoveto{\pgfqpoint{4.531678in}{5.283700in}}%
\pgfpathcurveto{\pgfqpoint{4.537502in}{5.283700in}}{\pgfqpoint{4.543088in}{5.286014in}}{\pgfqpoint{4.547207in}{5.290132in}}%
\pgfpathcurveto{\pgfqpoint{4.551325in}{5.294250in}}{\pgfqpoint{4.553639in}{5.299836in}}{\pgfqpoint{4.553639in}{5.305660in}}%
\pgfpathcurveto{\pgfqpoint{4.553639in}{5.311484in}}{\pgfqpoint{4.551325in}{5.317070in}}{\pgfqpoint{4.547207in}{5.321189in}}%
\pgfpathcurveto{\pgfqpoint{4.543088in}{5.325307in}}{\pgfqpoint{4.537502in}{5.327621in}}{\pgfqpoint{4.531678in}{5.327621in}}%
\pgfpathcurveto{\pgfqpoint{4.525854in}{5.327621in}}{\pgfqpoint{4.520268in}{5.325307in}}{\pgfqpoint{4.516150in}{5.321189in}}%
\pgfpathcurveto{\pgfqpoint{4.512032in}{5.317070in}}{\pgfqpoint{4.509718in}{5.311484in}}{\pgfqpoint{4.509718in}{5.305660in}}%
\pgfpathcurveto{\pgfqpoint{4.509718in}{5.299836in}}{\pgfqpoint{4.512032in}{5.294250in}}{\pgfqpoint{4.516150in}{5.290132in}}%
\pgfpathcurveto{\pgfqpoint{4.520268in}{5.286014in}}{\pgfqpoint{4.525854in}{5.283700in}}{\pgfqpoint{4.531678in}{5.283700in}}%
\pgfpathlineto{\pgfqpoint{4.531678in}{5.283700in}}%
\pgfpathclose%
\pgfusepath{stroke,fill}%
\end{pgfscope}%
\begin{pgfscope}%
\pgfpathrectangle{\pgfqpoint{1.000000in}{0.979904in}}{\pgfqpoint{6.200000in}{5.960192in}}%
\pgfusepath{clip}%
\pgfsetbuttcap%
\pgfsetroundjoin%
\definecolor{currentfill}{rgb}{0.200000,0.200000,0.800000}%
\pgfsetfillcolor{currentfill}%
\pgfsetlinewidth{1.003750pt}%
\definecolor{currentstroke}{rgb}{0.200000,0.200000,0.800000}%
\pgfsetstrokecolor{currentstroke}%
\pgfsetdash{}{0pt}%
\pgfpathmoveto{\pgfqpoint{4.560188in}{5.334773in}}%
\pgfpathcurveto{\pgfqpoint{4.566012in}{5.334773in}}{\pgfqpoint{4.571598in}{5.337087in}}{\pgfqpoint{4.575717in}{5.341205in}}%
\pgfpathcurveto{\pgfqpoint{4.579835in}{5.345324in}}{\pgfqpoint{4.582149in}{5.350910in}}{\pgfqpoint{4.582149in}{5.356734in}}%
\pgfpathcurveto{\pgfqpoint{4.582149in}{5.362558in}}{\pgfqpoint{4.579835in}{5.368144in}}{\pgfqpoint{4.575717in}{5.372262in}}%
\pgfpathcurveto{\pgfqpoint{4.571598in}{5.376380in}}{\pgfqpoint{4.566012in}{5.378694in}}{\pgfqpoint{4.560188in}{5.378694in}}%
\pgfpathcurveto{\pgfqpoint{4.554364in}{5.378694in}}{\pgfqpoint{4.548778in}{5.376380in}}{\pgfqpoint{4.544660in}{5.372262in}}%
\pgfpathcurveto{\pgfqpoint{4.540542in}{5.368144in}}{\pgfqpoint{4.538228in}{5.362558in}}{\pgfqpoint{4.538228in}{5.356734in}}%
\pgfpathcurveto{\pgfqpoint{4.538228in}{5.350910in}}{\pgfqpoint{4.540542in}{5.345324in}}{\pgfqpoint{4.544660in}{5.341205in}}%
\pgfpathcurveto{\pgfqpoint{4.548778in}{5.337087in}}{\pgfqpoint{4.554364in}{5.334773in}}{\pgfqpoint{4.560188in}{5.334773in}}%
\pgfpathlineto{\pgfqpoint{4.560188in}{5.334773in}}%
\pgfpathclose%
\pgfusepath{stroke,fill}%
\end{pgfscope}%
\begin{pgfscope}%
\pgfpathrectangle{\pgfqpoint{1.000000in}{0.979904in}}{\pgfqpoint{6.200000in}{5.960192in}}%
\pgfusepath{clip}%
\pgfsetbuttcap%
\pgfsetroundjoin%
\definecolor{currentfill}{rgb}{0.200000,0.200000,0.800000}%
\pgfsetfillcolor{currentfill}%
\pgfsetlinewidth{1.003750pt}%
\definecolor{currentstroke}{rgb}{0.200000,0.200000,0.800000}%
\pgfsetstrokecolor{currentstroke}%
\pgfsetdash{}{0pt}%
\pgfpathmoveto{\pgfqpoint{4.617568in}{5.377365in}}%
\pgfpathcurveto{\pgfqpoint{4.623392in}{5.377365in}}{\pgfqpoint{4.628978in}{5.379678in}}{\pgfqpoint{4.633096in}{5.383797in}}%
\pgfpathcurveto{\pgfqpoint{4.637214in}{5.387915in}}{\pgfqpoint{4.639528in}{5.393501in}}{\pgfqpoint{4.639528in}{5.399325in}}%
\pgfpathcurveto{\pgfqpoint{4.639528in}{5.405149in}}{\pgfqpoint{4.637214in}{5.410735in}}{\pgfqpoint{4.633096in}{5.414853in}}%
\pgfpathcurveto{\pgfqpoint{4.628978in}{5.418971in}}{\pgfqpoint{4.623392in}{5.421285in}}{\pgfqpoint{4.617568in}{5.421285in}}%
\pgfpathcurveto{\pgfqpoint{4.611744in}{5.421285in}}{\pgfqpoint{4.606158in}{5.418971in}}{\pgfqpoint{4.602040in}{5.414853in}}%
\pgfpathcurveto{\pgfqpoint{4.597921in}{5.410735in}}{\pgfqpoint{4.595608in}{5.405149in}}{\pgfqpoint{4.595608in}{5.399325in}}%
\pgfpathcurveto{\pgfqpoint{4.595608in}{5.393501in}}{\pgfqpoint{4.597921in}{5.387915in}}{\pgfqpoint{4.602040in}{5.383797in}}%
\pgfpathcurveto{\pgfqpoint{4.606158in}{5.379678in}}{\pgfqpoint{4.611744in}{5.377365in}}{\pgfqpoint{4.617568in}{5.377365in}}%
\pgfpathlineto{\pgfqpoint{4.617568in}{5.377365in}}%
\pgfpathclose%
\pgfusepath{stroke,fill}%
\end{pgfscope}%
\begin{pgfscope}%
\pgfpathrectangle{\pgfqpoint{1.000000in}{0.979904in}}{\pgfqpoint{6.200000in}{5.960192in}}%
\pgfusepath{clip}%
\pgfsetbuttcap%
\pgfsetroundjoin%
\definecolor{currentfill}{rgb}{0.200000,0.200000,0.800000}%
\pgfsetfillcolor{currentfill}%
\pgfsetlinewidth{1.003750pt}%
\definecolor{currentstroke}{rgb}{0.200000,0.200000,0.800000}%
\pgfsetstrokecolor{currentstroke}%
\pgfsetdash{}{0pt}%
\pgfpathmoveto{\pgfqpoint{4.602935in}{5.444043in}}%
\pgfpathcurveto{\pgfqpoint{4.608759in}{5.444043in}}{\pgfqpoint{4.614345in}{5.446357in}}{\pgfqpoint{4.618463in}{5.450475in}}%
\pgfpathcurveto{\pgfqpoint{4.622582in}{5.454593in}}{\pgfqpoint{4.624895in}{5.460179in}}{\pgfqpoint{4.624895in}{5.466003in}}%
\pgfpathcurveto{\pgfqpoint{4.624895in}{5.471827in}}{\pgfqpoint{4.622582in}{5.477413in}}{\pgfqpoint{4.618463in}{5.481531in}}%
\pgfpathcurveto{\pgfqpoint{4.614345in}{5.485649in}}{\pgfqpoint{4.608759in}{5.487963in}}{\pgfqpoint{4.602935in}{5.487963in}}%
\pgfpathcurveto{\pgfqpoint{4.597111in}{5.487963in}}{\pgfqpoint{4.591525in}{5.485649in}}{\pgfqpoint{4.587407in}{5.481531in}}%
\pgfpathcurveto{\pgfqpoint{4.583289in}{5.477413in}}{\pgfqpoint{4.580975in}{5.471827in}}{\pgfqpoint{4.580975in}{5.466003in}}%
\pgfpathcurveto{\pgfqpoint{4.580975in}{5.460179in}}{\pgfqpoint{4.583289in}{5.454593in}}{\pgfqpoint{4.587407in}{5.450475in}}%
\pgfpathcurveto{\pgfqpoint{4.591525in}{5.446357in}}{\pgfqpoint{4.597111in}{5.444043in}}{\pgfqpoint{4.602935in}{5.444043in}}%
\pgfpathlineto{\pgfqpoint{4.602935in}{5.444043in}}%
\pgfpathclose%
\pgfusepath{stroke,fill}%
\end{pgfscope}%
\begin{pgfscope}%
\pgfpathrectangle{\pgfqpoint{1.000000in}{0.979904in}}{\pgfqpoint{6.200000in}{5.960192in}}%
\pgfusepath{clip}%
\pgfsetbuttcap%
\pgfsetroundjoin%
\definecolor{currentfill}{rgb}{0.200000,0.200000,0.800000}%
\pgfsetfillcolor{currentfill}%
\pgfsetlinewidth{1.003750pt}%
\definecolor{currentstroke}{rgb}{0.200000,0.200000,0.800000}%
\pgfsetstrokecolor{currentstroke}%
\pgfsetdash{}{0pt}%
\pgfpathmoveto{\pgfqpoint{4.635741in}{5.497585in}}%
\pgfpathcurveto{\pgfqpoint{4.641565in}{5.497585in}}{\pgfqpoint{4.647152in}{5.499899in}}{\pgfqpoint{4.651270in}{5.504017in}}%
\pgfpathcurveto{\pgfqpoint{4.655388in}{5.508135in}}{\pgfqpoint{4.657702in}{5.513721in}}{\pgfqpoint{4.657702in}{5.519545in}}%
\pgfpathcurveto{\pgfqpoint{4.657702in}{5.525369in}}{\pgfqpoint{4.655388in}{5.530955in}}{\pgfqpoint{4.651270in}{5.535073in}}%
\pgfpathcurveto{\pgfqpoint{4.647152in}{5.539192in}}{\pgfqpoint{4.641565in}{5.541505in}}{\pgfqpoint{4.635741in}{5.541505in}}%
\pgfpathcurveto{\pgfqpoint{4.629918in}{5.541505in}}{\pgfqpoint{4.624331in}{5.539192in}}{\pgfqpoint{4.620213in}{5.535073in}}%
\pgfpathcurveto{\pgfqpoint{4.616095in}{5.530955in}}{\pgfqpoint{4.613781in}{5.525369in}}{\pgfqpoint{4.613781in}{5.519545in}}%
\pgfpathcurveto{\pgfqpoint{4.613781in}{5.513721in}}{\pgfqpoint{4.616095in}{5.508135in}}{\pgfqpoint{4.620213in}{5.504017in}}%
\pgfpathcurveto{\pgfqpoint{4.624331in}{5.499899in}}{\pgfqpoint{4.629918in}{5.497585in}}{\pgfqpoint{4.635741in}{5.497585in}}%
\pgfpathlineto{\pgfqpoint{4.635741in}{5.497585in}}%
\pgfpathclose%
\pgfusepath{stroke,fill}%
\end{pgfscope}%
\begin{pgfscope}%
\pgfpathrectangle{\pgfqpoint{1.000000in}{0.979904in}}{\pgfqpoint{6.200000in}{5.960192in}}%
\pgfusepath{clip}%
\pgfsetbuttcap%
\pgfsetroundjoin%
\definecolor{currentfill}{rgb}{0.200000,0.200000,0.800000}%
\pgfsetfillcolor{currentfill}%
\pgfsetlinewidth{1.003750pt}%
\definecolor{currentstroke}{rgb}{0.200000,0.200000,0.800000}%
\pgfsetstrokecolor{currentstroke}%
\pgfsetdash{}{0pt}%
\pgfpathmoveto{\pgfqpoint{4.527490in}{5.571952in}}%
\pgfpathcurveto{\pgfqpoint{4.533314in}{5.571952in}}{\pgfqpoint{4.538901in}{5.574266in}}{\pgfqpoint{4.543019in}{5.578384in}}%
\pgfpathcurveto{\pgfqpoint{4.547137in}{5.582502in}}{\pgfqpoint{4.549451in}{5.588088in}}{\pgfqpoint{4.549451in}{5.593912in}}%
\pgfpathcurveto{\pgfqpoint{4.549451in}{5.599736in}}{\pgfqpoint{4.547137in}{5.605323in}}{\pgfqpoint{4.543019in}{5.609441in}}%
\pgfpathcurveto{\pgfqpoint{4.538901in}{5.613559in}}{\pgfqpoint{4.533314in}{5.615873in}}{\pgfqpoint{4.527490in}{5.615873in}}%
\pgfpathcurveto{\pgfqpoint{4.521666in}{5.615873in}}{\pgfqpoint{4.516080in}{5.613559in}}{\pgfqpoint{4.511962in}{5.609441in}}%
\pgfpathcurveto{\pgfqpoint{4.507844in}{5.605323in}}{\pgfqpoint{4.505530in}{5.599736in}}{\pgfqpoint{4.505530in}{5.593912in}}%
\pgfpathcurveto{\pgfqpoint{4.505530in}{5.588088in}}{\pgfqpoint{4.507844in}{5.582502in}}{\pgfqpoint{4.511962in}{5.578384in}}%
\pgfpathcurveto{\pgfqpoint{4.516080in}{5.574266in}}{\pgfqpoint{4.521666in}{5.571952in}}{\pgfqpoint{4.527490in}{5.571952in}}%
\pgfpathlineto{\pgfqpoint{4.527490in}{5.571952in}}%
\pgfpathclose%
\pgfusepath{stroke,fill}%
\end{pgfscope}%
\begin{pgfscope}%
\pgfpathrectangle{\pgfqpoint{1.000000in}{0.979904in}}{\pgfqpoint{6.200000in}{5.960192in}}%
\pgfusepath{clip}%
\pgfsetbuttcap%
\pgfsetroundjoin%
\definecolor{currentfill}{rgb}{0.200000,0.200000,0.800000}%
\pgfsetfillcolor{currentfill}%
\pgfsetlinewidth{1.003750pt}%
\definecolor{currentstroke}{rgb}{0.200000,0.200000,0.800000}%
\pgfsetstrokecolor{currentstroke}%
\pgfsetdash{}{0pt}%
\pgfpathmoveto{\pgfqpoint{4.589633in}{5.620646in}}%
\pgfpathcurveto{\pgfqpoint{4.595457in}{5.620646in}}{\pgfqpoint{4.601044in}{5.622959in}}{\pgfqpoint{4.605162in}{5.627078in}}%
\pgfpathcurveto{\pgfqpoint{4.609280in}{5.631196in}}{\pgfqpoint{4.611594in}{5.636782in}}{\pgfqpoint{4.611594in}{5.642606in}}%
\pgfpathcurveto{\pgfqpoint{4.611594in}{5.648430in}}{\pgfqpoint{4.609280in}{5.654016in}}{\pgfqpoint{4.605162in}{5.658134in}}%
\pgfpathcurveto{\pgfqpoint{4.601044in}{5.662252in}}{\pgfqpoint{4.595457in}{5.664566in}}{\pgfqpoint{4.589633in}{5.664566in}}%
\pgfpathcurveto{\pgfqpoint{4.583809in}{5.664566in}}{\pgfqpoint{4.578223in}{5.662252in}}{\pgfqpoint{4.574105in}{5.658134in}}%
\pgfpathcurveto{\pgfqpoint{4.569987in}{5.654016in}}{\pgfqpoint{4.567673in}{5.648430in}}{\pgfqpoint{4.567673in}{5.642606in}}%
\pgfpathcurveto{\pgfqpoint{4.567673in}{5.636782in}}{\pgfqpoint{4.569987in}{5.631196in}}{\pgfqpoint{4.574105in}{5.627078in}}%
\pgfpathcurveto{\pgfqpoint{4.578223in}{5.622959in}}{\pgfqpoint{4.583809in}{5.620646in}}{\pgfqpoint{4.589633in}{5.620646in}}%
\pgfpathlineto{\pgfqpoint{4.589633in}{5.620646in}}%
\pgfpathclose%
\pgfusepath{stroke,fill}%
\end{pgfscope}%
\begin{pgfscope}%
\pgfpathrectangle{\pgfqpoint{1.000000in}{0.979904in}}{\pgfqpoint{6.200000in}{5.960192in}}%
\pgfusepath{clip}%
\pgfsetbuttcap%
\pgfsetroundjoin%
\definecolor{currentfill}{rgb}{0.200000,0.200000,0.800000}%
\pgfsetfillcolor{currentfill}%
\pgfsetlinewidth{1.003750pt}%
\definecolor{currentstroke}{rgb}{0.200000,0.200000,0.800000}%
\pgfsetstrokecolor{currentstroke}%
\pgfsetdash{}{0pt}%
\pgfpathmoveto{\pgfqpoint{4.715941in}{5.676814in}}%
\pgfpathcurveto{\pgfqpoint{4.721765in}{5.676814in}}{\pgfqpoint{4.727351in}{5.679128in}}{\pgfqpoint{4.731469in}{5.683246in}}%
\pgfpathcurveto{\pgfqpoint{4.735588in}{5.687364in}}{\pgfqpoint{4.737901in}{5.692950in}}{\pgfqpoint{4.737901in}{5.698774in}}%
\pgfpathcurveto{\pgfqpoint{4.737901in}{5.704598in}}{\pgfqpoint{4.735588in}{5.710184in}}{\pgfqpoint{4.731469in}{5.714303in}}%
\pgfpathcurveto{\pgfqpoint{4.727351in}{5.718421in}}{\pgfqpoint{4.721765in}{5.720735in}}{\pgfqpoint{4.715941in}{5.720735in}}%
\pgfpathcurveto{\pgfqpoint{4.710117in}{5.720735in}}{\pgfqpoint{4.704531in}{5.718421in}}{\pgfqpoint{4.700413in}{5.714303in}}%
\pgfpathcurveto{\pgfqpoint{4.696295in}{5.710184in}}{\pgfqpoint{4.693981in}{5.704598in}}{\pgfqpoint{4.693981in}{5.698774in}}%
\pgfpathcurveto{\pgfqpoint{4.693981in}{5.692950in}}{\pgfqpoint{4.696295in}{5.687364in}}{\pgfqpoint{4.700413in}{5.683246in}}%
\pgfpathcurveto{\pgfqpoint{4.704531in}{5.679128in}}{\pgfqpoint{4.710117in}{5.676814in}}{\pgfqpoint{4.715941in}{5.676814in}}%
\pgfpathlineto{\pgfqpoint{4.715941in}{5.676814in}}%
\pgfpathclose%
\pgfusepath{stroke,fill}%
\end{pgfscope}%
\begin{pgfscope}%
\pgfpathrectangle{\pgfqpoint{1.000000in}{0.979904in}}{\pgfqpoint{6.200000in}{5.960192in}}%
\pgfusepath{clip}%
\pgfsetbuttcap%
\pgfsetroundjoin%
\definecolor{currentfill}{rgb}{0.200000,0.800000,0.200000}%
\pgfsetfillcolor{currentfill}%
\pgfsetlinewidth{1.003750pt}%
\definecolor{currentstroke}{rgb}{0.200000,0.800000,0.200000}%
\pgfsetstrokecolor{currentstroke}%
\pgfsetdash{}{0pt}%
\pgfpathmoveto{\pgfqpoint{6.611965in}{4.431554in}}%
\pgfpathcurveto{\pgfqpoint{6.617789in}{4.431554in}}{\pgfqpoint{6.623375in}{4.433868in}}{\pgfqpoint{6.627493in}{4.437986in}}%
\pgfpathcurveto{\pgfqpoint{6.631612in}{4.442104in}}{\pgfqpoint{6.633925in}{4.447690in}}{\pgfqpoint{6.633925in}{4.453514in}}%
\pgfpathcurveto{\pgfqpoint{6.633925in}{4.459338in}}{\pgfqpoint{6.631612in}{4.464924in}}{\pgfqpoint{6.627493in}{4.469042in}}%
\pgfpathcurveto{\pgfqpoint{6.623375in}{4.473160in}}{\pgfqpoint{6.617789in}{4.475474in}}{\pgfqpoint{6.611965in}{4.475474in}}%
\pgfpathcurveto{\pgfqpoint{6.606141in}{4.475474in}}{\pgfqpoint{6.600555in}{4.473160in}}{\pgfqpoint{6.596437in}{4.469042in}}%
\pgfpathcurveto{\pgfqpoint{6.592319in}{4.464924in}}{\pgfqpoint{6.590005in}{4.459338in}}{\pgfqpoint{6.590005in}{4.453514in}}%
\pgfpathcurveto{\pgfqpoint{6.590005in}{4.447690in}}{\pgfqpoint{6.592319in}{4.442104in}}{\pgfqpoint{6.596437in}{4.437986in}}%
\pgfpathcurveto{\pgfqpoint{6.600555in}{4.433868in}}{\pgfqpoint{6.606141in}{4.431554in}}{\pgfqpoint{6.611965in}{4.431554in}}%
\pgfpathlineto{\pgfqpoint{6.611965in}{4.431554in}}%
\pgfpathclose%
\pgfusepath{stroke,fill}%
\end{pgfscope}%
\begin{pgfscope}%
\pgfpathrectangle{\pgfqpoint{1.000000in}{0.979904in}}{\pgfqpoint{6.200000in}{5.960192in}}%
\pgfusepath{clip}%
\pgfsetbuttcap%
\pgfsetroundjoin%
\definecolor{currentfill}{rgb}{0.200000,0.800000,0.200000}%
\pgfsetfillcolor{currentfill}%
\pgfsetlinewidth{1.003750pt}%
\definecolor{currentstroke}{rgb}{0.200000,0.800000,0.200000}%
\pgfsetstrokecolor{currentstroke}%
\pgfsetdash{}{0pt}%
\pgfpathmoveto{\pgfqpoint{6.708854in}{4.542051in}}%
\pgfpathcurveto{\pgfqpoint{6.714678in}{4.542051in}}{\pgfqpoint{6.720265in}{4.544364in}}{\pgfqpoint{6.724383in}{4.548483in}}%
\pgfpathcurveto{\pgfqpoint{6.728501in}{4.552601in}}{\pgfqpoint{6.730815in}{4.558187in}}{\pgfqpoint{6.730815in}{4.564011in}}%
\pgfpathcurveto{\pgfqpoint{6.730815in}{4.569835in}}{\pgfqpoint{6.728501in}{4.575421in}}{\pgfqpoint{6.724383in}{4.579539in}}%
\pgfpathcurveto{\pgfqpoint{6.720265in}{4.583657in}}{\pgfqpoint{6.714678in}{4.585971in}}{\pgfqpoint{6.708854in}{4.585971in}}%
\pgfpathcurveto{\pgfqpoint{6.703030in}{4.585971in}}{\pgfqpoint{6.697444in}{4.583657in}}{\pgfqpoint{6.693326in}{4.579539in}}%
\pgfpathcurveto{\pgfqpoint{6.689208in}{4.575421in}}{\pgfqpoint{6.686894in}{4.569835in}}{\pgfqpoint{6.686894in}{4.564011in}}%
\pgfpathcurveto{\pgfqpoint{6.686894in}{4.558187in}}{\pgfqpoint{6.689208in}{4.552601in}}{\pgfqpoint{6.693326in}{4.548483in}}%
\pgfpathcurveto{\pgfqpoint{6.697444in}{4.544364in}}{\pgfqpoint{6.703030in}{4.542051in}}{\pgfqpoint{6.708854in}{4.542051in}}%
\pgfpathlineto{\pgfqpoint{6.708854in}{4.542051in}}%
\pgfpathclose%
\pgfusepath{stroke,fill}%
\end{pgfscope}%
\begin{pgfscope}%
\pgfpathrectangle{\pgfqpoint{1.000000in}{0.979904in}}{\pgfqpoint{6.200000in}{5.960192in}}%
\pgfusepath{clip}%
\pgfsetbuttcap%
\pgfsetroundjoin%
\definecolor{currentfill}{rgb}{0.200000,0.800000,0.200000}%
\pgfsetfillcolor{currentfill}%
\pgfsetlinewidth{1.003750pt}%
\definecolor{currentstroke}{rgb}{0.200000,0.800000,0.200000}%
\pgfsetstrokecolor{currentstroke}%
\pgfsetdash{}{0pt}%
\pgfpathmoveto{\pgfqpoint{6.702911in}{4.652685in}}%
\pgfpathcurveto{\pgfqpoint{6.708735in}{4.652685in}}{\pgfqpoint{6.714321in}{4.654999in}}{\pgfqpoint{6.718439in}{4.659117in}}%
\pgfpathcurveto{\pgfqpoint{6.722557in}{4.663235in}}{\pgfqpoint{6.724871in}{4.668821in}}{\pgfqpoint{6.724871in}{4.674645in}}%
\pgfpathcurveto{\pgfqpoint{6.724871in}{4.680469in}}{\pgfqpoint{6.722557in}{4.686056in}}{\pgfqpoint{6.718439in}{4.690174in}}%
\pgfpathcurveto{\pgfqpoint{6.714321in}{4.694292in}}{\pgfqpoint{6.708735in}{4.696606in}}{\pgfqpoint{6.702911in}{4.696606in}}%
\pgfpathcurveto{\pgfqpoint{6.697087in}{4.696606in}}{\pgfqpoint{6.691501in}{4.694292in}}{\pgfqpoint{6.687383in}{4.690174in}}%
\pgfpathcurveto{\pgfqpoint{6.683265in}{4.686056in}}{\pgfqpoint{6.680951in}{4.680469in}}{\pgfqpoint{6.680951in}{4.674645in}}%
\pgfpathcurveto{\pgfqpoint{6.680951in}{4.668821in}}{\pgfqpoint{6.683265in}{4.663235in}}{\pgfqpoint{6.687383in}{4.659117in}}%
\pgfpathcurveto{\pgfqpoint{6.691501in}{4.654999in}}{\pgfqpoint{6.697087in}{4.652685in}}{\pgfqpoint{6.702911in}{4.652685in}}%
\pgfpathlineto{\pgfqpoint{6.702911in}{4.652685in}}%
\pgfpathclose%
\pgfusepath{stroke,fill}%
\end{pgfscope}%
\begin{pgfscope}%
\pgfpathrectangle{\pgfqpoint{1.000000in}{0.979904in}}{\pgfqpoint{6.200000in}{5.960192in}}%
\pgfusepath{clip}%
\pgfsetbuttcap%
\pgfsetroundjoin%
\definecolor{currentfill}{rgb}{0.200000,0.800000,0.200000}%
\pgfsetfillcolor{currentfill}%
\pgfsetlinewidth{1.003750pt}%
\definecolor{currentstroke}{rgb}{0.200000,0.800000,0.200000}%
\pgfsetstrokecolor{currentstroke}%
\pgfsetdash{}{0pt}%
\pgfpathmoveto{\pgfqpoint{6.810589in}{4.786266in}}%
\pgfpathcurveto{\pgfqpoint{6.816413in}{4.786266in}}{\pgfqpoint{6.821999in}{4.788580in}}{\pgfqpoint{6.826117in}{4.792698in}}%
\pgfpathcurveto{\pgfqpoint{6.830236in}{4.796816in}}{\pgfqpoint{6.832549in}{4.802403in}}{\pgfqpoint{6.832549in}{4.808226in}}%
\pgfpathcurveto{\pgfqpoint{6.832549in}{4.814050in}}{\pgfqpoint{6.830236in}{4.819637in}}{\pgfqpoint{6.826117in}{4.823755in}}%
\pgfpathcurveto{\pgfqpoint{6.821999in}{4.827873in}}{\pgfqpoint{6.816413in}{4.830187in}}{\pgfqpoint{6.810589in}{4.830187in}}%
\pgfpathcurveto{\pgfqpoint{6.804765in}{4.830187in}}{\pgfqpoint{6.799179in}{4.827873in}}{\pgfqpoint{6.795061in}{4.823755in}}%
\pgfpathcurveto{\pgfqpoint{6.790943in}{4.819637in}}{\pgfqpoint{6.788629in}{4.814050in}}{\pgfqpoint{6.788629in}{4.808226in}}%
\pgfpathcurveto{\pgfqpoint{6.788629in}{4.802403in}}{\pgfqpoint{6.790943in}{4.796816in}}{\pgfqpoint{6.795061in}{4.792698in}}%
\pgfpathcurveto{\pgfqpoint{6.799179in}{4.788580in}}{\pgfqpoint{6.804765in}{4.786266in}}{\pgfqpoint{6.810589in}{4.786266in}}%
\pgfpathlineto{\pgfqpoint{6.810589in}{4.786266in}}%
\pgfpathclose%
\pgfusepath{stroke,fill}%
\end{pgfscope}%
\begin{pgfscope}%
\pgfpathrectangle{\pgfqpoint{1.000000in}{0.979904in}}{\pgfqpoint{6.200000in}{5.960192in}}%
\pgfusepath{clip}%
\pgfsetbuttcap%
\pgfsetroundjoin%
\definecolor{currentfill}{rgb}{0.200000,0.800000,0.200000}%
\pgfsetfillcolor{currentfill}%
\pgfsetlinewidth{1.003750pt}%
\definecolor{currentstroke}{rgb}{0.200000,0.800000,0.200000}%
\pgfsetstrokecolor{currentstroke}%
\pgfsetdash{}{0pt}%
\pgfpathmoveto{\pgfqpoint{6.648005in}{4.866893in}}%
\pgfpathcurveto{\pgfqpoint{6.653829in}{4.866893in}}{\pgfqpoint{6.659415in}{4.869207in}}{\pgfqpoint{6.663533in}{4.873325in}}%
\pgfpathcurveto{\pgfqpoint{6.667651in}{4.877443in}}{\pgfqpoint{6.669965in}{4.883029in}}{\pgfqpoint{6.669965in}{4.888853in}}%
\pgfpathcurveto{\pgfqpoint{6.669965in}{4.894677in}}{\pgfqpoint{6.667651in}{4.900263in}}{\pgfqpoint{6.663533in}{4.904381in}}%
\pgfpathcurveto{\pgfqpoint{6.659415in}{4.908499in}}{\pgfqpoint{6.653829in}{4.910813in}}{\pgfqpoint{6.648005in}{4.910813in}}%
\pgfpathcurveto{\pgfqpoint{6.642181in}{4.910813in}}{\pgfqpoint{6.636595in}{4.908499in}}{\pgfqpoint{6.632477in}{4.904381in}}%
\pgfpathcurveto{\pgfqpoint{6.628359in}{4.900263in}}{\pgfqpoint{6.626045in}{4.894677in}}{\pgfqpoint{6.626045in}{4.888853in}}%
\pgfpathcurveto{\pgfqpoint{6.626045in}{4.883029in}}{\pgfqpoint{6.628359in}{4.877443in}}{\pgfqpoint{6.632477in}{4.873325in}}%
\pgfpathcurveto{\pgfqpoint{6.636595in}{4.869207in}}{\pgfqpoint{6.642181in}{4.866893in}}{\pgfqpoint{6.648005in}{4.866893in}}%
\pgfpathlineto{\pgfqpoint{6.648005in}{4.866893in}}%
\pgfpathclose%
\pgfusepath{stroke,fill}%
\end{pgfscope}%
\begin{pgfscope}%
\pgfpathrectangle{\pgfqpoint{1.000000in}{0.979904in}}{\pgfqpoint{6.200000in}{5.960192in}}%
\pgfusepath{clip}%
\pgfsetbuttcap%
\pgfsetroundjoin%
\definecolor{currentfill}{rgb}{0.200000,0.800000,0.200000}%
\pgfsetfillcolor{currentfill}%
\pgfsetlinewidth{1.003750pt}%
\definecolor{currentstroke}{rgb}{0.200000,0.800000,0.200000}%
\pgfsetstrokecolor{currentstroke}%
\pgfsetdash{}{0pt}%
\pgfpathmoveto{\pgfqpoint{6.641913in}{4.980608in}}%
\pgfpathcurveto{\pgfqpoint{6.647737in}{4.980608in}}{\pgfqpoint{6.653323in}{4.982922in}}{\pgfqpoint{6.657441in}{4.987040in}}%
\pgfpathcurveto{\pgfqpoint{6.661559in}{4.991158in}}{\pgfqpoint{6.663873in}{4.996745in}}{\pgfqpoint{6.663873in}{5.002569in}}%
\pgfpathcurveto{\pgfqpoint{6.663873in}{5.008393in}}{\pgfqpoint{6.661559in}{5.013979in}}{\pgfqpoint{6.657441in}{5.018097in}}%
\pgfpathcurveto{\pgfqpoint{6.653323in}{5.022215in}}{\pgfqpoint{6.647737in}{5.024529in}}{\pgfqpoint{6.641913in}{5.024529in}}%
\pgfpathcurveto{\pgfqpoint{6.636089in}{5.024529in}}{\pgfqpoint{6.630503in}{5.022215in}}{\pgfqpoint{6.626385in}{5.018097in}}%
\pgfpathcurveto{\pgfqpoint{6.622266in}{5.013979in}}{\pgfqpoint{6.619953in}{5.008393in}}{\pgfqpoint{6.619953in}{5.002569in}}%
\pgfpathcurveto{\pgfqpoint{6.619953in}{4.996745in}}{\pgfqpoint{6.622266in}{4.991158in}}{\pgfqpoint{6.626385in}{4.987040in}}%
\pgfpathcurveto{\pgfqpoint{6.630503in}{4.982922in}}{\pgfqpoint{6.636089in}{4.980608in}}{\pgfqpoint{6.641913in}{4.980608in}}%
\pgfpathlineto{\pgfqpoint{6.641913in}{4.980608in}}%
\pgfpathclose%
\pgfusepath{stroke,fill}%
\end{pgfscope}%
\begin{pgfscope}%
\pgfpathrectangle{\pgfqpoint{1.000000in}{0.979904in}}{\pgfqpoint{6.200000in}{5.960192in}}%
\pgfusepath{clip}%
\pgfsetbuttcap%
\pgfsetroundjoin%
\definecolor{currentfill}{rgb}{0.200000,0.800000,0.200000}%
\pgfsetfillcolor{currentfill}%
\pgfsetlinewidth{1.003750pt}%
\definecolor{currentstroke}{rgb}{0.200000,0.800000,0.200000}%
\pgfsetstrokecolor{currentstroke}%
\pgfsetdash{}{0pt}%
\pgfpathmoveto{\pgfqpoint{6.587161in}{5.078901in}}%
\pgfpathcurveto{\pgfqpoint{6.592985in}{5.078901in}}{\pgfqpoint{6.598571in}{5.081215in}}{\pgfqpoint{6.602689in}{5.085333in}}%
\pgfpathcurveto{\pgfqpoint{6.606807in}{5.089451in}}{\pgfqpoint{6.609121in}{5.095037in}}{\pgfqpoint{6.609121in}{5.100861in}}%
\pgfpathcurveto{\pgfqpoint{6.609121in}{5.106685in}}{\pgfqpoint{6.606807in}{5.112271in}}{\pgfqpoint{6.602689in}{5.116389in}}%
\pgfpathcurveto{\pgfqpoint{6.598571in}{5.120508in}}{\pgfqpoint{6.592985in}{5.122821in}}{\pgfqpoint{6.587161in}{5.122821in}}%
\pgfpathcurveto{\pgfqpoint{6.581337in}{5.122821in}}{\pgfqpoint{6.575751in}{5.120508in}}{\pgfqpoint{6.571633in}{5.116389in}}%
\pgfpathcurveto{\pgfqpoint{6.567515in}{5.112271in}}{\pgfqpoint{6.565201in}{5.106685in}}{\pgfqpoint{6.565201in}{5.100861in}}%
\pgfpathcurveto{\pgfqpoint{6.565201in}{5.095037in}}{\pgfqpoint{6.567515in}{5.089451in}}{\pgfqpoint{6.571633in}{5.085333in}}%
\pgfpathcurveto{\pgfqpoint{6.575751in}{5.081215in}}{\pgfqpoint{6.581337in}{5.078901in}}{\pgfqpoint{6.587161in}{5.078901in}}%
\pgfpathlineto{\pgfqpoint{6.587161in}{5.078901in}}%
\pgfpathclose%
\pgfusepath{stroke,fill}%
\end{pgfscope}%
\begin{pgfscope}%
\pgfpathrectangle{\pgfqpoint{1.000000in}{0.979904in}}{\pgfqpoint{6.200000in}{5.960192in}}%
\pgfusepath{clip}%
\pgfsetbuttcap%
\pgfsetroundjoin%
\definecolor{currentfill}{rgb}{0.200000,0.800000,0.200000}%
\pgfsetfillcolor{currentfill}%
\pgfsetlinewidth{1.003750pt}%
\definecolor{currentstroke}{rgb}{0.200000,0.800000,0.200000}%
\pgfsetstrokecolor{currentstroke}%
\pgfsetdash{}{0pt}%
\pgfpathmoveto{\pgfqpoint{6.488524in}{5.154295in}}%
\pgfpathcurveto{\pgfqpoint{6.494348in}{5.154295in}}{\pgfqpoint{6.499934in}{5.156609in}}{\pgfqpoint{6.504052in}{5.160727in}}%
\pgfpathcurveto{\pgfqpoint{6.508171in}{5.164846in}}{\pgfqpoint{6.510484in}{5.170432in}}{\pgfqpoint{6.510484in}{5.176256in}}%
\pgfpathcurveto{\pgfqpoint{6.510484in}{5.182080in}}{\pgfqpoint{6.508171in}{5.187666in}}{\pgfqpoint{6.504052in}{5.191784in}}%
\pgfpathcurveto{\pgfqpoint{6.499934in}{5.195902in}}{\pgfqpoint{6.494348in}{5.198216in}}{\pgfqpoint{6.488524in}{5.198216in}}%
\pgfpathcurveto{\pgfqpoint{6.482700in}{5.198216in}}{\pgfqpoint{6.477114in}{5.195902in}}{\pgfqpoint{6.472996in}{5.191784in}}%
\pgfpathcurveto{\pgfqpoint{6.468878in}{5.187666in}}{\pgfqpoint{6.466564in}{5.182080in}}{\pgfqpoint{6.466564in}{5.176256in}}%
\pgfpathcurveto{\pgfqpoint{6.466564in}{5.170432in}}{\pgfqpoint{6.468878in}{5.164846in}}{\pgfqpoint{6.472996in}{5.160727in}}%
\pgfpathcurveto{\pgfqpoint{6.477114in}{5.156609in}}{\pgfqpoint{6.482700in}{5.154295in}}{\pgfqpoint{6.488524in}{5.154295in}}%
\pgfpathlineto{\pgfqpoint{6.488524in}{5.154295in}}%
\pgfpathclose%
\pgfusepath{stroke,fill}%
\end{pgfscope}%
\begin{pgfscope}%
\pgfpathrectangle{\pgfqpoint{1.000000in}{0.979904in}}{\pgfqpoint{6.200000in}{5.960192in}}%
\pgfusepath{clip}%
\pgfsetbuttcap%
\pgfsetroundjoin%
\definecolor{currentfill}{rgb}{0.200000,0.800000,0.200000}%
\pgfsetfillcolor{currentfill}%
\pgfsetlinewidth{1.003750pt}%
\definecolor{currentstroke}{rgb}{0.200000,0.800000,0.200000}%
\pgfsetstrokecolor{currentstroke}%
\pgfsetdash{}{0pt}%
\pgfpathmoveto{\pgfqpoint{6.546797in}{5.308768in}}%
\pgfpathcurveto{\pgfqpoint{6.552621in}{5.308768in}}{\pgfqpoint{6.558208in}{5.311082in}}{\pgfqpoint{6.562326in}{5.315200in}}%
\pgfpathcurveto{\pgfqpoint{6.566444in}{5.319318in}}{\pgfqpoint{6.568758in}{5.324904in}}{\pgfqpoint{6.568758in}{5.330728in}}%
\pgfpathcurveto{\pgfqpoint{6.568758in}{5.336552in}}{\pgfqpoint{6.566444in}{5.342138in}}{\pgfqpoint{6.562326in}{5.346256in}}%
\pgfpathcurveto{\pgfqpoint{6.558208in}{5.350375in}}{\pgfqpoint{6.552621in}{5.352688in}}{\pgfqpoint{6.546797in}{5.352688in}}%
\pgfpathcurveto{\pgfqpoint{6.540974in}{5.352688in}}{\pgfqpoint{6.535387in}{5.350375in}}{\pgfqpoint{6.531269in}{5.346256in}}%
\pgfpathcurveto{\pgfqpoint{6.527151in}{5.342138in}}{\pgfqpoint{6.524837in}{5.336552in}}{\pgfqpoint{6.524837in}{5.330728in}}%
\pgfpathcurveto{\pgfqpoint{6.524837in}{5.324904in}}{\pgfqpoint{6.527151in}{5.319318in}}{\pgfqpoint{6.531269in}{5.315200in}}%
\pgfpathcurveto{\pgfqpoint{6.535387in}{5.311082in}}{\pgfqpoint{6.540974in}{5.308768in}}{\pgfqpoint{6.546797in}{5.308768in}}%
\pgfpathlineto{\pgfqpoint{6.546797in}{5.308768in}}%
\pgfpathclose%
\pgfusepath{stroke,fill}%
\end{pgfscope}%
\begin{pgfscope}%
\pgfpathrectangle{\pgfqpoint{1.000000in}{0.979904in}}{\pgfqpoint{6.200000in}{5.960192in}}%
\pgfusepath{clip}%
\pgfsetbuttcap%
\pgfsetroundjoin%
\definecolor{currentfill}{rgb}{0.200000,0.800000,0.200000}%
\pgfsetfillcolor{currentfill}%
\pgfsetlinewidth{1.003750pt}%
\definecolor{currentstroke}{rgb}{0.200000,0.800000,0.200000}%
\pgfsetstrokecolor{currentstroke}%
\pgfsetdash{}{0pt}%
\pgfpathmoveto{\pgfqpoint{6.470540in}{5.395785in}}%
\pgfpathcurveto{\pgfqpoint{6.476364in}{5.395785in}}{\pgfqpoint{6.481950in}{5.398099in}}{\pgfqpoint{6.486068in}{5.402217in}}%
\pgfpathcurveto{\pgfqpoint{6.490186in}{5.406336in}}{\pgfqpoint{6.492500in}{5.411922in}}{\pgfqpoint{6.492500in}{5.417746in}}%
\pgfpathcurveto{\pgfqpoint{6.492500in}{5.423570in}}{\pgfqpoint{6.490186in}{5.429156in}}{\pgfqpoint{6.486068in}{5.433274in}}%
\pgfpathcurveto{\pgfqpoint{6.481950in}{5.437392in}}{\pgfqpoint{6.476364in}{5.439706in}}{\pgfqpoint{6.470540in}{5.439706in}}%
\pgfpathcurveto{\pgfqpoint{6.464716in}{5.439706in}}{\pgfqpoint{6.459130in}{5.437392in}}{\pgfqpoint{6.455012in}{5.433274in}}%
\pgfpathcurveto{\pgfqpoint{6.450893in}{5.429156in}}{\pgfqpoint{6.448580in}{5.423570in}}{\pgfqpoint{6.448580in}{5.417746in}}%
\pgfpathcurveto{\pgfqpoint{6.448580in}{5.411922in}}{\pgfqpoint{6.450893in}{5.406336in}}{\pgfqpoint{6.455012in}{5.402217in}}%
\pgfpathcurveto{\pgfqpoint{6.459130in}{5.398099in}}{\pgfqpoint{6.464716in}{5.395785in}}{\pgfqpoint{6.470540in}{5.395785in}}%
\pgfpathlineto{\pgfqpoint{6.470540in}{5.395785in}}%
\pgfpathclose%
\pgfusepath{stroke,fill}%
\end{pgfscope}%
\begin{pgfscope}%
\pgfpathrectangle{\pgfqpoint{1.000000in}{0.979904in}}{\pgfqpoint{6.200000in}{5.960192in}}%
\pgfusepath{clip}%
\pgfsetbuttcap%
\pgfsetroundjoin%
\definecolor{currentfill}{rgb}{0.200000,0.800000,0.200000}%
\pgfsetfillcolor{currentfill}%
\pgfsetlinewidth{1.003750pt}%
\definecolor{currentstroke}{rgb}{0.200000,0.800000,0.200000}%
\pgfsetstrokecolor{currentstroke}%
\pgfsetdash{}{0pt}%
\pgfpathmoveto{\pgfqpoint{6.408034in}{5.490234in}}%
\pgfpathcurveto{\pgfqpoint{6.413858in}{5.490234in}}{\pgfqpoint{6.419444in}{5.492548in}}{\pgfqpoint{6.423562in}{5.496666in}}%
\pgfpathcurveto{\pgfqpoint{6.427681in}{5.500784in}}{\pgfqpoint{6.429994in}{5.506370in}}{\pgfqpoint{6.429994in}{5.512194in}}%
\pgfpathcurveto{\pgfqpoint{6.429994in}{5.518018in}}{\pgfqpoint{6.427681in}{5.523604in}}{\pgfqpoint{6.423562in}{5.527722in}}%
\pgfpathcurveto{\pgfqpoint{6.419444in}{5.531840in}}{\pgfqpoint{6.413858in}{5.534154in}}{\pgfqpoint{6.408034in}{5.534154in}}%
\pgfpathcurveto{\pgfqpoint{6.402210in}{5.534154in}}{\pgfqpoint{6.396624in}{5.531840in}}{\pgfqpoint{6.392506in}{5.527722in}}%
\pgfpathcurveto{\pgfqpoint{6.388388in}{5.523604in}}{\pgfqpoint{6.386074in}{5.518018in}}{\pgfqpoint{6.386074in}{5.512194in}}%
\pgfpathcurveto{\pgfqpoint{6.386074in}{5.506370in}}{\pgfqpoint{6.388388in}{5.500784in}}{\pgfqpoint{6.392506in}{5.496666in}}%
\pgfpathcurveto{\pgfqpoint{6.396624in}{5.492548in}}{\pgfqpoint{6.402210in}{5.490234in}}{\pgfqpoint{6.408034in}{5.490234in}}%
\pgfpathlineto{\pgfqpoint{6.408034in}{5.490234in}}%
\pgfpathclose%
\pgfusepath{stroke,fill}%
\end{pgfscope}%
\begin{pgfscope}%
\pgfpathrectangle{\pgfqpoint{1.000000in}{0.979904in}}{\pgfqpoint{6.200000in}{5.960192in}}%
\pgfusepath{clip}%
\pgfsetbuttcap%
\pgfsetroundjoin%
\definecolor{currentfill}{rgb}{0.200000,0.800000,0.200000}%
\pgfsetfillcolor{currentfill}%
\pgfsetlinewidth{1.003750pt}%
\definecolor{currentstroke}{rgb}{0.200000,0.800000,0.200000}%
\pgfsetstrokecolor{currentstroke}%
\pgfsetdash{}{0pt}%
\pgfpathmoveto{\pgfqpoint{6.268882in}{5.521306in}}%
\pgfpathcurveto{\pgfqpoint{6.274706in}{5.521306in}}{\pgfqpoint{6.280292in}{5.523620in}}{\pgfqpoint{6.284410in}{5.527738in}}%
\pgfpathcurveto{\pgfqpoint{6.288529in}{5.531856in}}{\pgfqpoint{6.290843in}{5.537442in}}{\pgfqpoint{6.290843in}{5.543266in}}%
\pgfpathcurveto{\pgfqpoint{6.290843in}{5.549090in}}{\pgfqpoint{6.288529in}{5.554676in}}{\pgfqpoint{6.284410in}{5.558795in}}%
\pgfpathcurveto{\pgfqpoint{6.280292in}{5.562913in}}{\pgfqpoint{6.274706in}{5.565227in}}{\pgfqpoint{6.268882in}{5.565227in}}%
\pgfpathcurveto{\pgfqpoint{6.263058in}{5.565227in}}{\pgfqpoint{6.257472in}{5.562913in}}{\pgfqpoint{6.253354in}{5.558795in}}%
\pgfpathcurveto{\pgfqpoint{6.249236in}{5.554676in}}{\pgfqpoint{6.246922in}{5.549090in}}{\pgfqpoint{6.246922in}{5.543266in}}%
\pgfpathcurveto{\pgfqpoint{6.246922in}{5.537442in}}{\pgfqpoint{6.249236in}{5.531856in}}{\pgfqpoint{6.253354in}{5.527738in}}%
\pgfpathcurveto{\pgfqpoint{6.257472in}{5.523620in}}{\pgfqpoint{6.263058in}{5.521306in}}{\pgfqpoint{6.268882in}{5.521306in}}%
\pgfpathlineto{\pgfqpoint{6.268882in}{5.521306in}}%
\pgfpathclose%
\pgfusepath{stroke,fill}%
\end{pgfscope}%
\begin{pgfscope}%
\pgfpathrectangle{\pgfqpoint{1.000000in}{0.979904in}}{\pgfqpoint{6.200000in}{5.960192in}}%
\pgfusepath{clip}%
\pgfsetbuttcap%
\pgfsetroundjoin%
\definecolor{currentfill}{rgb}{0.200000,0.800000,0.200000}%
\pgfsetfillcolor{currentfill}%
\pgfsetlinewidth{1.003750pt}%
\definecolor{currentstroke}{rgb}{0.200000,0.800000,0.200000}%
\pgfsetstrokecolor{currentstroke}%
\pgfsetdash{}{0pt}%
\pgfpathmoveto{\pgfqpoint{6.285936in}{5.686138in}}%
\pgfpathcurveto{\pgfqpoint{6.291760in}{5.686138in}}{\pgfqpoint{6.297347in}{5.688452in}}{\pgfqpoint{6.301465in}{5.692570in}}%
\pgfpathcurveto{\pgfqpoint{6.305583in}{5.696688in}}{\pgfqpoint{6.307897in}{5.702275in}}{\pgfqpoint{6.307897in}{5.708099in}}%
\pgfpathcurveto{\pgfqpoint{6.307897in}{5.713923in}}{\pgfqpoint{6.305583in}{5.719509in}}{\pgfqpoint{6.301465in}{5.723627in}}%
\pgfpathcurveto{\pgfqpoint{6.297347in}{5.727745in}}{\pgfqpoint{6.291760in}{5.730059in}}{\pgfqpoint{6.285936in}{5.730059in}}%
\pgfpathcurveto{\pgfqpoint{6.280112in}{5.730059in}}{\pgfqpoint{6.274526in}{5.727745in}}{\pgfqpoint{6.270408in}{5.723627in}}%
\pgfpathcurveto{\pgfqpoint{6.266290in}{5.719509in}}{\pgfqpoint{6.263976in}{5.713923in}}{\pgfqpoint{6.263976in}{5.708099in}}%
\pgfpathcurveto{\pgfqpoint{6.263976in}{5.702275in}}{\pgfqpoint{6.266290in}{5.696688in}}{\pgfqpoint{6.270408in}{5.692570in}}%
\pgfpathcurveto{\pgfqpoint{6.274526in}{5.688452in}}{\pgfqpoint{6.280112in}{5.686138in}}{\pgfqpoint{6.285936in}{5.686138in}}%
\pgfpathlineto{\pgfqpoint{6.285936in}{5.686138in}}%
\pgfpathclose%
\pgfusepath{stroke,fill}%
\end{pgfscope}%
\begin{pgfscope}%
\pgfpathrectangle{\pgfqpoint{1.000000in}{0.979904in}}{\pgfqpoint{6.200000in}{5.960192in}}%
\pgfusepath{clip}%
\pgfsetbuttcap%
\pgfsetroundjoin%
\definecolor{currentfill}{rgb}{0.200000,0.800000,0.200000}%
\pgfsetfillcolor{currentfill}%
\pgfsetlinewidth{1.003750pt}%
\definecolor{currentstroke}{rgb}{0.200000,0.800000,0.200000}%
\pgfsetstrokecolor{currentstroke}%
\pgfsetdash{}{0pt}%
\pgfpathmoveto{\pgfqpoint{6.102751in}{5.657753in}}%
\pgfpathcurveto{\pgfqpoint{6.108575in}{5.657753in}}{\pgfqpoint{6.114161in}{5.660067in}}{\pgfqpoint{6.118279in}{5.664185in}}%
\pgfpathcurveto{\pgfqpoint{6.122397in}{5.668303in}}{\pgfqpoint{6.124711in}{5.673889in}}{\pgfqpoint{6.124711in}{5.679713in}}%
\pgfpathcurveto{\pgfqpoint{6.124711in}{5.685537in}}{\pgfqpoint{6.122397in}{5.691123in}}{\pgfqpoint{6.118279in}{5.695241in}}%
\pgfpathcurveto{\pgfqpoint{6.114161in}{5.699359in}}{\pgfqpoint{6.108575in}{5.701673in}}{\pgfqpoint{6.102751in}{5.701673in}}%
\pgfpathcurveto{\pgfqpoint{6.096927in}{5.701673in}}{\pgfqpoint{6.091341in}{5.699359in}}{\pgfqpoint{6.087223in}{5.695241in}}%
\pgfpathcurveto{\pgfqpoint{6.083105in}{5.691123in}}{\pgfqpoint{6.080791in}{5.685537in}}{\pgfqpoint{6.080791in}{5.679713in}}%
\pgfpathcurveto{\pgfqpoint{6.080791in}{5.673889in}}{\pgfqpoint{6.083105in}{5.668303in}}{\pgfqpoint{6.087223in}{5.664185in}}%
\pgfpathcurveto{\pgfqpoint{6.091341in}{5.660067in}}{\pgfqpoint{6.096927in}{5.657753in}}{\pgfqpoint{6.102751in}{5.657753in}}%
\pgfpathlineto{\pgfqpoint{6.102751in}{5.657753in}}%
\pgfpathclose%
\pgfusepath{stroke,fill}%
\end{pgfscope}%
\begin{pgfscope}%
\pgfpathrectangle{\pgfqpoint{1.000000in}{0.979904in}}{\pgfqpoint{6.200000in}{5.960192in}}%
\pgfusepath{clip}%
\pgfsetbuttcap%
\pgfsetroundjoin%
\definecolor{currentfill}{rgb}{0.200000,0.800000,0.200000}%
\pgfsetfillcolor{currentfill}%
\pgfsetlinewidth{1.003750pt}%
\definecolor{currentstroke}{rgb}{0.200000,0.800000,0.200000}%
\pgfsetstrokecolor{currentstroke}%
\pgfsetdash{}{0pt}%
\pgfpathmoveto{\pgfqpoint{6.089233in}{5.809011in}}%
\pgfpathcurveto{\pgfqpoint{6.095057in}{5.809011in}}{\pgfqpoint{6.100643in}{5.811325in}}{\pgfqpoint{6.104761in}{5.815443in}}%
\pgfpathcurveto{\pgfqpoint{6.108879in}{5.819562in}}{\pgfqpoint{6.111193in}{5.825148in}}{\pgfqpoint{6.111193in}{5.830972in}}%
\pgfpathcurveto{\pgfqpoint{6.111193in}{5.836796in}}{\pgfqpoint{6.108879in}{5.842382in}}{\pgfqpoint{6.104761in}{5.846500in}}%
\pgfpathcurveto{\pgfqpoint{6.100643in}{5.850618in}}{\pgfqpoint{6.095057in}{5.852932in}}{\pgfqpoint{6.089233in}{5.852932in}}%
\pgfpathcurveto{\pgfqpoint{6.083409in}{5.852932in}}{\pgfqpoint{6.077823in}{5.850618in}}{\pgfqpoint{6.073704in}{5.846500in}}%
\pgfpathcurveto{\pgfqpoint{6.069586in}{5.842382in}}{\pgfqpoint{6.067272in}{5.836796in}}{\pgfqpoint{6.067272in}{5.830972in}}%
\pgfpathcurveto{\pgfqpoint{6.067272in}{5.825148in}}{\pgfqpoint{6.069586in}{5.819562in}}{\pgfqpoint{6.073704in}{5.815443in}}%
\pgfpathcurveto{\pgfqpoint{6.077823in}{5.811325in}}{\pgfqpoint{6.083409in}{5.809011in}}{\pgfqpoint{6.089233in}{5.809011in}}%
\pgfpathlineto{\pgfqpoint{6.089233in}{5.809011in}}%
\pgfpathclose%
\pgfusepath{stroke,fill}%
\end{pgfscope}%
\begin{pgfscope}%
\pgfpathrectangle{\pgfqpoint{1.000000in}{0.979904in}}{\pgfqpoint{6.200000in}{5.960192in}}%
\pgfusepath{clip}%
\pgfsetbuttcap%
\pgfsetroundjoin%
\definecolor{currentfill}{rgb}{0.200000,0.800000,0.200000}%
\pgfsetfillcolor{currentfill}%
\pgfsetlinewidth{1.003750pt}%
\definecolor{currentstroke}{rgb}{0.200000,0.800000,0.200000}%
\pgfsetstrokecolor{currentstroke}%
\pgfsetdash{}{0pt}%
\pgfpathmoveto{\pgfqpoint{6.048968in}{5.946519in}}%
\pgfpathcurveto{\pgfqpoint{6.054792in}{5.946519in}}{\pgfqpoint{6.060378in}{5.948833in}}{\pgfqpoint{6.064496in}{5.952951in}}%
\pgfpathcurveto{\pgfqpoint{6.068614in}{5.957069in}}{\pgfqpoint{6.070928in}{5.962656in}}{\pgfqpoint{6.070928in}{5.968479in}}%
\pgfpathcurveto{\pgfqpoint{6.070928in}{5.974303in}}{\pgfqpoint{6.068614in}{5.979890in}}{\pgfqpoint{6.064496in}{5.984008in}}%
\pgfpathcurveto{\pgfqpoint{6.060378in}{5.988126in}}{\pgfqpoint{6.054792in}{5.990440in}}{\pgfqpoint{6.048968in}{5.990440in}}%
\pgfpathcurveto{\pgfqpoint{6.043144in}{5.990440in}}{\pgfqpoint{6.037558in}{5.988126in}}{\pgfqpoint{6.033440in}{5.984008in}}%
\pgfpathcurveto{\pgfqpoint{6.029322in}{5.979890in}}{\pgfqpoint{6.027008in}{5.974303in}}{\pgfqpoint{6.027008in}{5.968479in}}%
\pgfpathcurveto{\pgfqpoint{6.027008in}{5.962656in}}{\pgfqpoint{6.029322in}{5.957069in}}{\pgfqpoint{6.033440in}{5.952951in}}%
\pgfpathcurveto{\pgfqpoint{6.037558in}{5.948833in}}{\pgfqpoint{6.043144in}{5.946519in}}{\pgfqpoint{6.048968in}{5.946519in}}%
\pgfpathlineto{\pgfqpoint{6.048968in}{5.946519in}}%
\pgfpathclose%
\pgfusepath{stroke,fill}%
\end{pgfscope}%
\begin{pgfscope}%
\pgfpathrectangle{\pgfqpoint{1.000000in}{0.979904in}}{\pgfqpoint{6.200000in}{5.960192in}}%
\pgfusepath{clip}%
\pgfsetbuttcap%
\pgfsetroundjoin%
\definecolor{currentfill}{rgb}{0.200000,0.800000,0.200000}%
\pgfsetfillcolor{currentfill}%
\pgfsetlinewidth{1.003750pt}%
\definecolor{currentstroke}{rgb}{0.200000,0.800000,0.200000}%
\pgfsetstrokecolor{currentstroke}%
\pgfsetdash{}{0pt}%
\pgfpathmoveto{\pgfqpoint{5.861730in}{5.868480in}}%
\pgfpathcurveto{\pgfqpoint{5.867554in}{5.868480in}}{\pgfqpoint{5.873140in}{5.870794in}}{\pgfqpoint{5.877258in}{5.874912in}}%
\pgfpathcurveto{\pgfqpoint{5.881376in}{5.879031in}}{\pgfqpoint{5.883690in}{5.884617in}}{\pgfqpoint{5.883690in}{5.890441in}}%
\pgfpathcurveto{\pgfqpoint{5.883690in}{5.896265in}}{\pgfqpoint{5.881376in}{5.901851in}}{\pgfqpoint{5.877258in}{5.905969in}}%
\pgfpathcurveto{\pgfqpoint{5.873140in}{5.910087in}}{\pgfqpoint{5.867554in}{5.912401in}}{\pgfqpoint{5.861730in}{5.912401in}}%
\pgfpathcurveto{\pgfqpoint{5.855906in}{5.912401in}}{\pgfqpoint{5.850320in}{5.910087in}}{\pgfqpoint{5.846201in}{5.905969in}}%
\pgfpathcurveto{\pgfqpoint{5.842083in}{5.901851in}}{\pgfqpoint{5.839769in}{5.896265in}}{\pgfqpoint{5.839769in}{5.890441in}}%
\pgfpathcurveto{\pgfqpoint{5.839769in}{5.884617in}}{\pgfqpoint{5.842083in}{5.879031in}}{\pgfqpoint{5.846201in}{5.874912in}}%
\pgfpathcurveto{\pgfqpoint{5.850320in}{5.870794in}}{\pgfqpoint{5.855906in}{5.868480in}}{\pgfqpoint{5.861730in}{5.868480in}}%
\pgfpathlineto{\pgfqpoint{5.861730in}{5.868480in}}%
\pgfpathclose%
\pgfusepath{stroke,fill}%
\end{pgfscope}%
\begin{pgfscope}%
\pgfpathrectangle{\pgfqpoint{1.000000in}{0.979904in}}{\pgfqpoint{6.200000in}{5.960192in}}%
\pgfusepath{clip}%
\pgfsetbuttcap%
\pgfsetroundjoin%
\definecolor{currentfill}{rgb}{0.200000,0.800000,0.200000}%
\pgfsetfillcolor{currentfill}%
\pgfsetlinewidth{1.003750pt}%
\definecolor{currentstroke}{rgb}{0.200000,0.800000,0.200000}%
\pgfsetstrokecolor{currentstroke}%
\pgfsetdash{}{0pt}%
\pgfpathmoveto{\pgfqpoint{5.779429in}{5.941974in}}%
\pgfpathcurveto{\pgfqpoint{5.785253in}{5.941974in}}{\pgfqpoint{5.790839in}{5.944288in}}{\pgfqpoint{5.794958in}{5.948406in}}%
\pgfpathcurveto{\pgfqpoint{5.799076in}{5.952524in}}{\pgfqpoint{5.801390in}{5.958111in}}{\pgfqpoint{5.801390in}{5.963935in}}%
\pgfpathcurveto{\pgfqpoint{5.801390in}{5.969758in}}{\pgfqpoint{5.799076in}{5.975345in}}{\pgfqpoint{5.794958in}{5.979463in}}%
\pgfpathcurveto{\pgfqpoint{5.790839in}{5.983581in}}{\pgfqpoint{5.785253in}{5.985895in}}{\pgfqpoint{5.779429in}{5.985895in}}%
\pgfpathcurveto{\pgfqpoint{5.773605in}{5.985895in}}{\pgfqpoint{5.768019in}{5.983581in}}{\pgfqpoint{5.763901in}{5.979463in}}%
\pgfpathcurveto{\pgfqpoint{5.759783in}{5.975345in}}{\pgfqpoint{5.757469in}{5.969758in}}{\pgfqpoint{5.757469in}{5.963935in}}%
\pgfpathcurveto{\pgfqpoint{5.757469in}{5.958111in}}{\pgfqpoint{5.759783in}{5.952524in}}{\pgfqpoint{5.763901in}{5.948406in}}%
\pgfpathcurveto{\pgfqpoint{5.768019in}{5.944288in}}{\pgfqpoint{5.773605in}{5.941974in}}{\pgfqpoint{5.779429in}{5.941974in}}%
\pgfpathlineto{\pgfqpoint{5.779429in}{5.941974in}}%
\pgfpathclose%
\pgfusepath{stroke,fill}%
\end{pgfscope}%
\begin{pgfscope}%
\pgfpathrectangle{\pgfqpoint{1.000000in}{0.979904in}}{\pgfqpoint{6.200000in}{5.960192in}}%
\pgfusepath{clip}%
\pgfsetbuttcap%
\pgfsetroundjoin%
\definecolor{currentfill}{rgb}{0.200000,0.800000,0.200000}%
\pgfsetfillcolor{currentfill}%
\pgfsetlinewidth{1.003750pt}%
\definecolor{currentstroke}{rgb}{0.200000,0.800000,0.200000}%
\pgfsetstrokecolor{currentstroke}%
\pgfsetdash{}{0pt}%
\pgfpathmoveto{\pgfqpoint{5.696681in}{6.022400in}}%
\pgfpathcurveto{\pgfqpoint{5.702505in}{6.022400in}}{\pgfqpoint{5.708091in}{6.024714in}}{\pgfqpoint{5.712209in}{6.028832in}}%
\pgfpathcurveto{\pgfqpoint{5.716327in}{6.032950in}}{\pgfqpoint{5.718641in}{6.038536in}}{\pgfqpoint{5.718641in}{6.044360in}}%
\pgfpathcurveto{\pgfqpoint{5.718641in}{6.050184in}}{\pgfqpoint{5.716327in}{6.055770in}}{\pgfqpoint{5.712209in}{6.059889in}}%
\pgfpathcurveto{\pgfqpoint{5.708091in}{6.064007in}}{\pgfqpoint{5.702505in}{6.066321in}}{\pgfqpoint{5.696681in}{6.066321in}}%
\pgfpathcurveto{\pgfqpoint{5.690857in}{6.066321in}}{\pgfqpoint{5.685271in}{6.064007in}}{\pgfqpoint{5.681153in}{6.059889in}}%
\pgfpathcurveto{\pgfqpoint{5.677035in}{6.055770in}}{\pgfqpoint{5.674721in}{6.050184in}}{\pgfqpoint{5.674721in}{6.044360in}}%
\pgfpathcurveto{\pgfqpoint{5.674721in}{6.038536in}}{\pgfqpoint{5.677035in}{6.032950in}}{\pgfqpoint{5.681153in}{6.028832in}}%
\pgfpathcurveto{\pgfqpoint{5.685271in}{6.024714in}}{\pgfqpoint{5.690857in}{6.022400in}}{\pgfqpoint{5.696681in}{6.022400in}}%
\pgfpathlineto{\pgfqpoint{5.696681in}{6.022400in}}%
\pgfpathclose%
\pgfusepath{stroke,fill}%
\end{pgfscope}%
\begin{pgfscope}%
\pgfpathrectangle{\pgfqpoint{1.000000in}{0.979904in}}{\pgfqpoint{6.200000in}{5.960192in}}%
\pgfusepath{clip}%
\pgfsetbuttcap%
\pgfsetroundjoin%
\definecolor{currentfill}{rgb}{0.200000,0.800000,0.200000}%
\pgfsetfillcolor{currentfill}%
\pgfsetlinewidth{1.003750pt}%
\definecolor{currentstroke}{rgb}{0.200000,0.800000,0.200000}%
\pgfsetstrokecolor{currentstroke}%
\pgfsetdash{}{0pt}%
\pgfpathmoveto{\pgfqpoint{5.582900in}{6.035383in}}%
\pgfpathcurveto{\pgfqpoint{5.588724in}{6.035383in}}{\pgfqpoint{5.594310in}{6.037697in}}{\pgfqpoint{5.598429in}{6.041815in}}%
\pgfpathcurveto{\pgfqpoint{5.602547in}{6.045933in}}{\pgfqpoint{5.604861in}{6.051519in}}{\pgfqpoint{5.604861in}{6.057343in}}%
\pgfpathcurveto{\pgfqpoint{5.604861in}{6.063167in}}{\pgfqpoint{5.602547in}{6.068753in}}{\pgfqpoint{5.598429in}{6.072871in}}%
\pgfpathcurveto{\pgfqpoint{5.594310in}{6.076990in}}{\pgfqpoint{5.588724in}{6.079303in}}{\pgfqpoint{5.582900in}{6.079303in}}%
\pgfpathcurveto{\pgfqpoint{5.577076in}{6.079303in}}{\pgfqpoint{5.571490in}{6.076990in}}{\pgfqpoint{5.567372in}{6.072871in}}%
\pgfpathcurveto{\pgfqpoint{5.563254in}{6.068753in}}{\pgfqpoint{5.560940in}{6.063167in}}{\pgfqpoint{5.560940in}{6.057343in}}%
\pgfpathcurveto{\pgfqpoint{5.560940in}{6.051519in}}{\pgfqpoint{5.563254in}{6.045933in}}{\pgfqpoint{5.567372in}{6.041815in}}%
\pgfpathcurveto{\pgfqpoint{5.571490in}{6.037697in}}{\pgfqpoint{5.577076in}{6.035383in}}{\pgfqpoint{5.582900in}{6.035383in}}%
\pgfpathlineto{\pgfqpoint{5.582900in}{6.035383in}}%
\pgfpathclose%
\pgfusepath{stroke,fill}%
\end{pgfscope}%
\begin{pgfscope}%
\pgfpathrectangle{\pgfqpoint{1.000000in}{0.979904in}}{\pgfqpoint{6.200000in}{5.960192in}}%
\pgfusepath{clip}%
\pgfsetbuttcap%
\pgfsetroundjoin%
\definecolor{currentfill}{rgb}{0.200000,0.800000,0.200000}%
\pgfsetfillcolor{currentfill}%
\pgfsetlinewidth{1.003750pt}%
\definecolor{currentstroke}{rgb}{0.200000,0.800000,0.200000}%
\pgfsetstrokecolor{currentstroke}%
\pgfsetdash{}{0pt}%
\pgfpathmoveto{\pgfqpoint{5.463449in}{6.017961in}}%
\pgfpathcurveto{\pgfqpoint{5.469272in}{6.017961in}}{\pgfqpoint{5.474859in}{6.020275in}}{\pgfqpoint{5.478977in}{6.024393in}}%
\pgfpathcurveto{\pgfqpoint{5.483095in}{6.028511in}}{\pgfqpoint{5.485409in}{6.034098in}}{\pgfqpoint{5.485409in}{6.039922in}}%
\pgfpathcurveto{\pgfqpoint{5.485409in}{6.045745in}}{\pgfqpoint{5.483095in}{6.051332in}}{\pgfqpoint{5.478977in}{6.055450in}}%
\pgfpathcurveto{\pgfqpoint{5.474859in}{6.059568in}}{\pgfqpoint{5.469272in}{6.061882in}}{\pgfqpoint{5.463449in}{6.061882in}}%
\pgfpathcurveto{\pgfqpoint{5.457625in}{6.061882in}}{\pgfqpoint{5.452038in}{6.059568in}}{\pgfqpoint{5.447920in}{6.055450in}}%
\pgfpathcurveto{\pgfqpoint{5.443802in}{6.051332in}}{\pgfqpoint{5.441488in}{6.045745in}}{\pgfqpoint{5.441488in}{6.039922in}}%
\pgfpathcurveto{\pgfqpoint{5.441488in}{6.034098in}}{\pgfqpoint{5.443802in}{6.028511in}}{\pgfqpoint{5.447920in}{6.024393in}}%
\pgfpathcurveto{\pgfqpoint{5.452038in}{6.020275in}}{\pgfqpoint{5.457625in}{6.017961in}}{\pgfqpoint{5.463449in}{6.017961in}}%
\pgfpathlineto{\pgfqpoint{5.463449in}{6.017961in}}%
\pgfpathclose%
\pgfusepath{stroke,fill}%
\end{pgfscope}%
\begin{pgfscope}%
\pgfpathrectangle{\pgfqpoint{1.000000in}{0.979904in}}{\pgfqpoint{6.200000in}{5.960192in}}%
\pgfusepath{clip}%
\pgfsetbuttcap%
\pgfsetroundjoin%
\definecolor{currentfill}{rgb}{0.200000,0.800000,0.200000}%
\pgfsetfillcolor{currentfill}%
\pgfsetlinewidth{1.003750pt}%
\definecolor{currentstroke}{rgb}{0.200000,0.800000,0.200000}%
\pgfsetstrokecolor{currentstroke}%
\pgfsetdash{}{0pt}%
\pgfpathmoveto{\pgfqpoint{5.382093in}{6.129542in}}%
\pgfpathcurveto{\pgfqpoint{5.387917in}{6.129542in}}{\pgfqpoint{5.393504in}{6.131856in}}{\pgfqpoint{5.397622in}{6.135974in}}%
\pgfpathcurveto{\pgfqpoint{5.401740in}{6.140092in}}{\pgfqpoint{5.404054in}{6.145679in}}{\pgfqpoint{5.404054in}{6.151503in}}%
\pgfpathcurveto{\pgfqpoint{5.404054in}{6.157326in}}{\pgfqpoint{5.401740in}{6.162913in}}{\pgfqpoint{5.397622in}{6.167031in}}%
\pgfpathcurveto{\pgfqpoint{5.393504in}{6.171149in}}{\pgfqpoint{5.387917in}{6.173463in}}{\pgfqpoint{5.382093in}{6.173463in}}%
\pgfpathcurveto{\pgfqpoint{5.376270in}{6.173463in}}{\pgfqpoint{5.370683in}{6.171149in}}{\pgfqpoint{5.366565in}{6.167031in}}%
\pgfpathcurveto{\pgfqpoint{5.362447in}{6.162913in}}{\pgfqpoint{5.360133in}{6.157326in}}{\pgfqpoint{5.360133in}{6.151503in}}%
\pgfpathcurveto{\pgfqpoint{5.360133in}{6.145679in}}{\pgfqpoint{5.362447in}{6.140092in}}{\pgfqpoint{5.366565in}{6.135974in}}%
\pgfpathcurveto{\pgfqpoint{5.370683in}{6.131856in}}{\pgfqpoint{5.376270in}{6.129542in}}{\pgfqpoint{5.382093in}{6.129542in}}%
\pgfpathlineto{\pgfqpoint{5.382093in}{6.129542in}}%
\pgfpathclose%
\pgfusepath{stroke,fill}%
\end{pgfscope}%
\begin{pgfscope}%
\pgfpathrectangle{\pgfqpoint{1.000000in}{0.979904in}}{\pgfqpoint{6.200000in}{5.960192in}}%
\pgfusepath{clip}%
\pgfsetbuttcap%
\pgfsetroundjoin%
\definecolor{currentfill}{rgb}{0.200000,0.800000,0.200000}%
\pgfsetfillcolor{currentfill}%
\pgfsetlinewidth{1.003750pt}%
\definecolor{currentstroke}{rgb}{0.200000,0.800000,0.200000}%
\pgfsetstrokecolor{currentstroke}%
\pgfsetdash{}{0pt}%
\pgfpathmoveto{\pgfqpoint{5.263152in}{6.093161in}}%
\pgfpathcurveto{\pgfqpoint{5.268976in}{6.093161in}}{\pgfqpoint{5.274562in}{6.095475in}}{\pgfqpoint{5.278680in}{6.099593in}}%
\pgfpathcurveto{\pgfqpoint{5.282799in}{6.103711in}}{\pgfqpoint{5.285112in}{6.109298in}}{\pgfqpoint{5.285112in}{6.115122in}}%
\pgfpathcurveto{\pgfqpoint{5.285112in}{6.120945in}}{\pgfqpoint{5.282799in}{6.126532in}}{\pgfqpoint{5.278680in}{6.130650in}}%
\pgfpathcurveto{\pgfqpoint{5.274562in}{6.134768in}}{\pgfqpoint{5.268976in}{6.137082in}}{\pgfqpoint{5.263152in}{6.137082in}}%
\pgfpathcurveto{\pgfqpoint{5.257328in}{6.137082in}}{\pgfqpoint{5.251742in}{6.134768in}}{\pgfqpoint{5.247624in}{6.130650in}}%
\pgfpathcurveto{\pgfqpoint{5.243506in}{6.126532in}}{\pgfqpoint{5.241192in}{6.120945in}}{\pgfqpoint{5.241192in}{6.115122in}}%
\pgfpathcurveto{\pgfqpoint{5.241192in}{6.109298in}}{\pgfqpoint{5.243506in}{6.103711in}}{\pgfqpoint{5.247624in}{6.099593in}}%
\pgfpathcurveto{\pgfqpoint{5.251742in}{6.095475in}}{\pgfqpoint{5.257328in}{6.093161in}}{\pgfqpoint{5.263152in}{6.093161in}}%
\pgfpathlineto{\pgfqpoint{5.263152in}{6.093161in}}%
\pgfpathclose%
\pgfusepath{stroke,fill}%
\end{pgfscope}%
\begin{pgfscope}%
\pgfpathrectangle{\pgfqpoint{1.000000in}{0.979904in}}{\pgfqpoint{6.200000in}{5.960192in}}%
\pgfusepath{clip}%
\pgfsetbuttcap%
\pgfsetroundjoin%
\definecolor{currentfill}{rgb}{0.200000,0.800000,0.200000}%
\pgfsetfillcolor{currentfill}%
\pgfsetlinewidth{1.003750pt}%
\definecolor{currentstroke}{rgb}{0.200000,0.800000,0.200000}%
\pgfsetstrokecolor{currentstroke}%
\pgfsetdash{}{0pt}%
\pgfpathmoveto{\pgfqpoint{5.166209in}{6.189386in}}%
\pgfpathcurveto{\pgfqpoint{5.172033in}{6.189386in}}{\pgfqpoint{5.177619in}{6.191700in}}{\pgfqpoint{5.181737in}{6.195818in}}%
\pgfpathcurveto{\pgfqpoint{5.185855in}{6.199936in}}{\pgfqpoint{5.188169in}{6.205522in}}{\pgfqpoint{5.188169in}{6.211346in}}%
\pgfpathcurveto{\pgfqpoint{5.188169in}{6.217170in}}{\pgfqpoint{5.185855in}{6.222756in}}{\pgfqpoint{5.181737in}{6.226874in}}%
\pgfpathcurveto{\pgfqpoint{5.177619in}{6.230992in}}{\pgfqpoint{5.172033in}{6.233306in}}{\pgfqpoint{5.166209in}{6.233306in}}%
\pgfpathcurveto{\pgfqpoint{5.160385in}{6.233306in}}{\pgfqpoint{5.154799in}{6.230992in}}{\pgfqpoint{5.150681in}{6.226874in}}%
\pgfpathcurveto{\pgfqpoint{5.146562in}{6.222756in}}{\pgfqpoint{5.144248in}{6.217170in}}{\pgfqpoint{5.144248in}{6.211346in}}%
\pgfpathcurveto{\pgfqpoint{5.144248in}{6.205522in}}{\pgfqpoint{5.146562in}{6.199936in}}{\pgfqpoint{5.150681in}{6.195818in}}%
\pgfpathcurveto{\pgfqpoint{5.154799in}{6.191700in}}{\pgfqpoint{5.160385in}{6.189386in}}{\pgfqpoint{5.166209in}{6.189386in}}%
\pgfpathlineto{\pgfqpoint{5.166209in}{6.189386in}}%
\pgfpathclose%
\pgfusepath{stroke,fill}%
\end{pgfscope}%
\begin{pgfscope}%
\pgfpathrectangle{\pgfqpoint{1.000000in}{0.979904in}}{\pgfqpoint{6.200000in}{5.960192in}}%
\pgfusepath{clip}%
\pgfsetbuttcap%
\pgfsetroundjoin%
\definecolor{currentfill}{rgb}{0.200000,0.800000,0.200000}%
\pgfsetfillcolor{currentfill}%
\pgfsetlinewidth{1.003750pt}%
\definecolor{currentstroke}{rgb}{0.200000,0.800000,0.200000}%
\pgfsetstrokecolor{currentstroke}%
\pgfsetdash{}{0pt}%
\pgfpathmoveto{\pgfqpoint{5.050809in}{6.124455in}}%
\pgfpathcurveto{\pgfqpoint{5.056633in}{6.124455in}}{\pgfqpoint{5.062219in}{6.126769in}}{\pgfqpoint{5.066337in}{6.130887in}}%
\pgfpathcurveto{\pgfqpoint{5.070455in}{6.135005in}}{\pgfqpoint{5.072769in}{6.140592in}}{\pgfqpoint{5.072769in}{6.146416in}}%
\pgfpathcurveto{\pgfqpoint{5.072769in}{6.152239in}}{\pgfqpoint{5.070455in}{6.157826in}}{\pgfqpoint{5.066337in}{6.161944in}}%
\pgfpathcurveto{\pgfqpoint{5.062219in}{6.166062in}}{\pgfqpoint{5.056633in}{6.168376in}}{\pgfqpoint{5.050809in}{6.168376in}}%
\pgfpathcurveto{\pgfqpoint{5.044985in}{6.168376in}}{\pgfqpoint{5.039399in}{6.166062in}}{\pgfqpoint{5.035280in}{6.161944in}}%
\pgfpathcurveto{\pgfqpoint{5.031162in}{6.157826in}}{\pgfqpoint{5.028848in}{6.152239in}}{\pgfqpoint{5.028848in}{6.146416in}}%
\pgfpathcurveto{\pgfqpoint{5.028848in}{6.140592in}}{\pgfqpoint{5.031162in}{6.135005in}}{\pgfqpoint{5.035280in}{6.130887in}}%
\pgfpathcurveto{\pgfqpoint{5.039399in}{6.126769in}}{\pgfqpoint{5.044985in}{6.124455in}}{\pgfqpoint{5.050809in}{6.124455in}}%
\pgfpathlineto{\pgfqpoint{5.050809in}{6.124455in}}%
\pgfpathclose%
\pgfusepath{stroke,fill}%
\end{pgfscope}%
\begin{pgfscope}%
\pgfpathrectangle{\pgfqpoint{1.000000in}{0.979904in}}{\pgfqpoint{6.200000in}{5.960192in}}%
\pgfusepath{clip}%
\pgfsetbuttcap%
\pgfsetroundjoin%
\definecolor{currentfill}{rgb}{0.200000,0.800000,0.200000}%
\pgfsetfillcolor{currentfill}%
\pgfsetlinewidth{1.003750pt}%
\definecolor{currentstroke}{rgb}{0.200000,0.800000,0.200000}%
\pgfsetstrokecolor{currentstroke}%
\pgfsetdash{}{0pt}%
\pgfpathmoveto{\pgfqpoint{4.941197in}{6.257161in}}%
\pgfpathcurveto{\pgfqpoint{4.947021in}{6.257161in}}{\pgfqpoint{4.952607in}{6.259475in}}{\pgfqpoint{4.956725in}{6.263593in}}%
\pgfpathcurveto{\pgfqpoint{4.960844in}{6.267711in}}{\pgfqpoint{4.963157in}{6.273297in}}{\pgfqpoint{4.963157in}{6.279121in}}%
\pgfpathcurveto{\pgfqpoint{4.963157in}{6.284945in}}{\pgfqpoint{4.960844in}{6.290531in}}{\pgfqpoint{4.956725in}{6.294650in}}%
\pgfpathcurveto{\pgfqpoint{4.952607in}{6.298768in}}{\pgfqpoint{4.947021in}{6.301082in}}{\pgfqpoint{4.941197in}{6.301082in}}%
\pgfpathcurveto{\pgfqpoint{4.935373in}{6.301082in}}{\pgfqpoint{4.929787in}{6.298768in}}{\pgfqpoint{4.925669in}{6.294650in}}%
\pgfpathcurveto{\pgfqpoint{4.921551in}{6.290531in}}{\pgfqpoint{4.919237in}{6.284945in}}{\pgfqpoint{4.919237in}{6.279121in}}%
\pgfpathcurveto{\pgfqpoint{4.919237in}{6.273297in}}{\pgfqpoint{4.921551in}{6.267711in}}{\pgfqpoint{4.925669in}{6.263593in}}%
\pgfpathcurveto{\pgfqpoint{4.929787in}{6.259475in}}{\pgfqpoint{4.935373in}{6.257161in}}{\pgfqpoint{4.941197in}{6.257161in}}%
\pgfpathlineto{\pgfqpoint{4.941197in}{6.257161in}}%
\pgfpathclose%
\pgfusepath{stroke,fill}%
\end{pgfscope}%
\begin{pgfscope}%
\pgfpathrectangle{\pgfqpoint{1.000000in}{0.979904in}}{\pgfqpoint{6.200000in}{5.960192in}}%
\pgfusepath{clip}%
\pgfsetbuttcap%
\pgfsetroundjoin%
\definecolor{currentfill}{rgb}{0.200000,0.800000,0.200000}%
\pgfsetfillcolor{currentfill}%
\pgfsetlinewidth{1.003750pt}%
\definecolor{currentstroke}{rgb}{0.200000,0.800000,0.200000}%
\pgfsetstrokecolor{currentstroke}%
\pgfsetdash{}{0pt}%
\pgfpathmoveto{\pgfqpoint{4.835279in}{6.128246in}}%
\pgfpathcurveto{\pgfqpoint{4.841103in}{6.128246in}}{\pgfqpoint{4.846689in}{6.130560in}}{\pgfqpoint{4.850807in}{6.134678in}}%
\pgfpathcurveto{\pgfqpoint{4.854925in}{6.138797in}}{\pgfqpoint{4.857239in}{6.144383in}}{\pgfqpoint{4.857239in}{6.150207in}}%
\pgfpathcurveto{\pgfqpoint{4.857239in}{6.156031in}}{\pgfqpoint{4.854925in}{6.161617in}}{\pgfqpoint{4.850807in}{6.165735in}}%
\pgfpathcurveto{\pgfqpoint{4.846689in}{6.169853in}}{\pgfqpoint{4.841103in}{6.172167in}}{\pgfqpoint{4.835279in}{6.172167in}}%
\pgfpathcurveto{\pgfqpoint{4.829455in}{6.172167in}}{\pgfqpoint{4.823869in}{6.169853in}}{\pgfqpoint{4.819751in}{6.165735in}}%
\pgfpathcurveto{\pgfqpoint{4.815632in}{6.161617in}}{\pgfqpoint{4.813319in}{6.156031in}}{\pgfqpoint{4.813319in}{6.150207in}}%
\pgfpathcurveto{\pgfqpoint{4.813319in}{6.144383in}}{\pgfqpoint{4.815632in}{6.138797in}}{\pgfqpoint{4.819751in}{6.134678in}}%
\pgfpathcurveto{\pgfqpoint{4.823869in}{6.130560in}}{\pgfqpoint{4.829455in}{6.128246in}}{\pgfqpoint{4.835279in}{6.128246in}}%
\pgfpathlineto{\pgfqpoint{4.835279in}{6.128246in}}%
\pgfpathclose%
\pgfusepath{stroke,fill}%
\end{pgfscope}%
\begin{pgfscope}%
\pgfpathrectangle{\pgfqpoint{1.000000in}{0.979904in}}{\pgfqpoint{6.200000in}{5.960192in}}%
\pgfusepath{clip}%
\pgfsetbuttcap%
\pgfsetroundjoin%
\definecolor{currentfill}{rgb}{0.200000,0.800000,0.200000}%
\pgfsetfillcolor{currentfill}%
\pgfsetlinewidth{1.003750pt}%
\definecolor{currentstroke}{rgb}{0.200000,0.800000,0.200000}%
\pgfsetstrokecolor{currentstroke}%
\pgfsetdash{}{0pt}%
\pgfpathmoveto{\pgfqpoint{4.724457in}{6.140496in}}%
\pgfpathcurveto{\pgfqpoint{4.730281in}{6.140496in}}{\pgfqpoint{4.735867in}{6.142809in}}{\pgfqpoint{4.739985in}{6.146928in}}%
\pgfpathcurveto{\pgfqpoint{4.744104in}{6.151046in}}{\pgfqpoint{4.746417in}{6.156632in}}{\pgfqpoint{4.746417in}{6.162456in}}%
\pgfpathcurveto{\pgfqpoint{4.746417in}{6.168280in}}{\pgfqpoint{4.744104in}{6.173866in}}{\pgfqpoint{4.739985in}{6.177984in}}%
\pgfpathcurveto{\pgfqpoint{4.735867in}{6.182102in}}{\pgfqpoint{4.730281in}{6.184416in}}{\pgfqpoint{4.724457in}{6.184416in}}%
\pgfpathcurveto{\pgfqpoint{4.718633in}{6.184416in}}{\pgfqpoint{4.713047in}{6.182102in}}{\pgfqpoint{4.708929in}{6.177984in}}%
\pgfpathcurveto{\pgfqpoint{4.704811in}{6.173866in}}{\pgfqpoint{4.702497in}{6.168280in}}{\pgfqpoint{4.702497in}{6.162456in}}%
\pgfpathcurveto{\pgfqpoint{4.702497in}{6.156632in}}{\pgfqpoint{4.704811in}{6.151046in}}{\pgfqpoint{4.708929in}{6.146928in}}%
\pgfpathcurveto{\pgfqpoint{4.713047in}{6.142809in}}{\pgfqpoint{4.718633in}{6.140496in}}{\pgfqpoint{4.724457in}{6.140496in}}%
\pgfpathlineto{\pgfqpoint{4.724457in}{6.140496in}}%
\pgfpathclose%
\pgfusepath{stroke,fill}%
\end{pgfscope}%
\begin{pgfscope}%
\pgfpathrectangle{\pgfqpoint{1.000000in}{0.979904in}}{\pgfqpoint{6.200000in}{5.960192in}}%
\pgfusepath{clip}%
\pgfsetbuttcap%
\pgfsetroundjoin%
\definecolor{currentfill}{rgb}{0.200000,0.800000,0.200000}%
\pgfsetfillcolor{currentfill}%
\pgfsetlinewidth{1.003750pt}%
\definecolor{currentstroke}{rgb}{0.200000,0.800000,0.200000}%
\pgfsetstrokecolor{currentstroke}%
\pgfsetdash{}{0pt}%
\pgfpathmoveto{\pgfqpoint{4.604224in}{6.180444in}}%
\pgfpathcurveto{\pgfqpoint{4.610048in}{6.180444in}}{\pgfqpoint{4.615635in}{6.182758in}}{\pgfqpoint{4.619753in}{6.186876in}}%
\pgfpathcurveto{\pgfqpoint{4.623871in}{6.190994in}}{\pgfqpoint{4.626185in}{6.196580in}}{\pgfqpoint{4.626185in}{6.202404in}}%
\pgfpathcurveto{\pgfqpoint{4.626185in}{6.208228in}}{\pgfqpoint{4.623871in}{6.213814in}}{\pgfqpoint{4.619753in}{6.217932in}}%
\pgfpathcurveto{\pgfqpoint{4.615635in}{6.222051in}}{\pgfqpoint{4.610048in}{6.224364in}}{\pgfqpoint{4.604224in}{6.224364in}}%
\pgfpathcurveto{\pgfqpoint{4.598400in}{6.224364in}}{\pgfqpoint{4.592814in}{6.222051in}}{\pgfqpoint{4.588696in}{6.217932in}}%
\pgfpathcurveto{\pgfqpoint{4.584578in}{6.213814in}}{\pgfqpoint{4.582264in}{6.208228in}}{\pgfqpoint{4.582264in}{6.202404in}}%
\pgfpathcurveto{\pgfqpoint{4.582264in}{6.196580in}}{\pgfqpoint{4.584578in}{6.190994in}}{\pgfqpoint{4.588696in}{6.186876in}}%
\pgfpathcurveto{\pgfqpoint{4.592814in}{6.182758in}}{\pgfqpoint{4.598400in}{6.180444in}}{\pgfqpoint{4.604224in}{6.180444in}}%
\pgfpathlineto{\pgfqpoint{4.604224in}{6.180444in}}%
\pgfpathclose%
\pgfusepath{stroke,fill}%
\end{pgfscope}%
\begin{pgfscope}%
\pgfpathrectangle{\pgfqpoint{1.000000in}{0.979904in}}{\pgfqpoint{6.200000in}{5.960192in}}%
\pgfusepath{clip}%
\pgfsetbuttcap%
\pgfsetroundjoin%
\definecolor{currentfill}{rgb}{0.200000,0.200000,0.800000}%
\pgfsetfillcolor{currentfill}%
\pgfsetlinewidth{1.003750pt}%
\definecolor{currentstroke}{rgb}{0.200000,0.200000,0.800000}%
\pgfsetstrokecolor{currentstroke}%
\pgfsetdash{}{0pt}%
\pgfpathmoveto{\pgfqpoint{4.522944in}{6.049167in}}%
\pgfpathcurveto{\pgfqpoint{4.528768in}{6.049167in}}{\pgfqpoint{4.534354in}{6.051481in}}{\pgfqpoint{4.538472in}{6.055599in}}%
\pgfpathcurveto{\pgfqpoint{4.542590in}{6.059718in}}{\pgfqpoint{4.544904in}{6.065304in}}{\pgfqpoint{4.544904in}{6.071128in}}%
\pgfpathcurveto{\pgfqpoint{4.544904in}{6.076952in}}{\pgfqpoint{4.542590in}{6.082538in}}{\pgfqpoint{4.538472in}{6.086656in}}%
\pgfpathcurveto{\pgfqpoint{4.534354in}{6.090774in}}{\pgfqpoint{4.528768in}{6.093088in}}{\pgfqpoint{4.522944in}{6.093088in}}%
\pgfpathcurveto{\pgfqpoint{4.517120in}{6.093088in}}{\pgfqpoint{4.511533in}{6.090774in}}{\pgfqpoint{4.507415in}{6.086656in}}%
\pgfpathcurveto{\pgfqpoint{4.503297in}{6.082538in}}{\pgfqpoint{4.500983in}{6.076952in}}{\pgfqpoint{4.500983in}{6.071128in}}%
\pgfpathcurveto{\pgfqpoint{4.500983in}{6.065304in}}{\pgfqpoint{4.503297in}{6.059718in}}{\pgfqpoint{4.507415in}{6.055599in}}%
\pgfpathcurveto{\pgfqpoint{4.511533in}{6.051481in}}{\pgfqpoint{4.517120in}{6.049167in}}{\pgfqpoint{4.522944in}{6.049167in}}%
\pgfpathlineto{\pgfqpoint{4.522944in}{6.049167in}}%
\pgfpathclose%
\pgfusepath{stroke,fill}%
\end{pgfscope}%
\begin{pgfscope}%
\pgfpathrectangle{\pgfqpoint{1.000000in}{0.979904in}}{\pgfqpoint{6.200000in}{5.960192in}}%
\pgfusepath{clip}%
\pgfsetbuttcap%
\pgfsetroundjoin%
\definecolor{currentfill}{rgb}{0.200000,0.800000,0.200000}%
\pgfsetfillcolor{currentfill}%
\pgfsetlinewidth{1.003750pt}%
\definecolor{currentstroke}{rgb}{0.200000,0.800000,0.200000}%
\pgfsetstrokecolor{currentstroke}%
\pgfsetdash{}{0pt}%
\pgfpathmoveto{\pgfqpoint{4.395942in}{6.090664in}}%
\pgfpathcurveto{\pgfqpoint{4.401766in}{6.090664in}}{\pgfqpoint{4.407352in}{6.092978in}}{\pgfqpoint{4.411470in}{6.097096in}}%
\pgfpathcurveto{\pgfqpoint{4.415589in}{6.101214in}}{\pgfqpoint{4.417902in}{6.106801in}}{\pgfqpoint{4.417902in}{6.112625in}}%
\pgfpathcurveto{\pgfqpoint{4.417902in}{6.118449in}}{\pgfqpoint{4.415589in}{6.124035in}}{\pgfqpoint{4.411470in}{6.128153in}}%
\pgfpathcurveto{\pgfqpoint{4.407352in}{6.132271in}}{\pgfqpoint{4.401766in}{6.134585in}}{\pgfqpoint{4.395942in}{6.134585in}}%
\pgfpathcurveto{\pgfqpoint{4.390118in}{6.134585in}}{\pgfqpoint{4.384532in}{6.132271in}}{\pgfqpoint{4.380414in}{6.128153in}}%
\pgfpathcurveto{\pgfqpoint{4.376296in}{6.124035in}}{\pgfqpoint{4.373982in}{6.118449in}}{\pgfqpoint{4.373982in}{6.112625in}}%
\pgfpathcurveto{\pgfqpoint{4.373982in}{6.106801in}}{\pgfqpoint{4.376296in}{6.101214in}}{\pgfqpoint{4.380414in}{6.097096in}}%
\pgfpathcurveto{\pgfqpoint{4.384532in}{6.092978in}}{\pgfqpoint{4.390118in}{6.090664in}}{\pgfqpoint{4.395942in}{6.090664in}}%
\pgfpathlineto{\pgfqpoint{4.395942in}{6.090664in}}%
\pgfpathclose%
\pgfusepath{stroke,fill}%
\end{pgfscope}%
\begin{pgfscope}%
\pgfpathrectangle{\pgfqpoint{1.000000in}{0.979904in}}{\pgfqpoint{6.200000in}{5.960192in}}%
\pgfusepath{clip}%
\pgfsetbuttcap%
\pgfsetroundjoin%
\definecolor{currentfill}{rgb}{0.200000,0.800000,0.200000}%
\pgfsetfillcolor{currentfill}%
\pgfsetlinewidth{1.003750pt}%
\definecolor{currentstroke}{rgb}{0.200000,0.800000,0.200000}%
\pgfsetstrokecolor{currentstroke}%
\pgfsetdash{}{0pt}%
\pgfpathmoveto{\pgfqpoint{4.269739in}{6.103741in}}%
\pgfpathcurveto{\pgfqpoint{4.275563in}{6.103741in}}{\pgfqpoint{4.281149in}{6.106055in}}{\pgfqpoint{4.285268in}{6.110173in}}%
\pgfpathcurveto{\pgfqpoint{4.289386in}{6.114291in}}{\pgfqpoint{4.291700in}{6.119877in}}{\pgfqpoint{4.291700in}{6.125701in}}%
\pgfpathcurveto{\pgfqpoint{4.291700in}{6.131525in}}{\pgfqpoint{4.289386in}{6.137111in}}{\pgfqpoint{4.285268in}{6.141229in}}%
\pgfpathcurveto{\pgfqpoint{4.281149in}{6.145348in}}{\pgfqpoint{4.275563in}{6.147661in}}{\pgfqpoint{4.269739in}{6.147661in}}%
\pgfpathcurveto{\pgfqpoint{4.263915in}{6.147661in}}{\pgfqpoint{4.258329in}{6.145348in}}{\pgfqpoint{4.254211in}{6.141229in}}%
\pgfpathcurveto{\pgfqpoint{4.250093in}{6.137111in}}{\pgfqpoint{4.247779in}{6.131525in}}{\pgfqpoint{4.247779in}{6.125701in}}%
\pgfpathcurveto{\pgfqpoint{4.247779in}{6.119877in}}{\pgfqpoint{4.250093in}{6.114291in}}{\pgfqpoint{4.254211in}{6.110173in}}%
\pgfpathcurveto{\pgfqpoint{4.258329in}{6.106055in}}{\pgfqpoint{4.263915in}{6.103741in}}{\pgfqpoint{4.269739in}{6.103741in}}%
\pgfpathlineto{\pgfqpoint{4.269739in}{6.103741in}}%
\pgfpathclose%
\pgfusepath{stroke,fill}%
\end{pgfscope}%
\begin{pgfscope}%
\pgfpathrectangle{\pgfqpoint{1.000000in}{0.979904in}}{\pgfqpoint{6.200000in}{5.960192in}}%
\pgfusepath{clip}%
\pgfsetbuttcap%
\pgfsetroundjoin%
\definecolor{currentfill}{rgb}{0.200000,0.800000,0.200000}%
\pgfsetfillcolor{currentfill}%
\pgfsetlinewidth{1.003750pt}%
\definecolor{currentstroke}{rgb}{0.200000,0.800000,0.200000}%
\pgfsetstrokecolor{currentstroke}%
\pgfsetdash{}{0pt}%
\pgfpathmoveto{\pgfqpoint{4.233763in}{5.917394in}}%
\pgfpathcurveto{\pgfqpoint{4.239587in}{5.917394in}}{\pgfqpoint{4.245174in}{5.919708in}}{\pgfqpoint{4.249292in}{5.923826in}}%
\pgfpathcurveto{\pgfqpoint{4.253410in}{5.927944in}}{\pgfqpoint{4.255724in}{5.933530in}}{\pgfqpoint{4.255724in}{5.939354in}}%
\pgfpathcurveto{\pgfqpoint{4.255724in}{5.945178in}}{\pgfqpoint{4.253410in}{5.950764in}}{\pgfqpoint{4.249292in}{5.954882in}}%
\pgfpathcurveto{\pgfqpoint{4.245174in}{5.959000in}}{\pgfqpoint{4.239587in}{5.961314in}}{\pgfqpoint{4.233763in}{5.961314in}}%
\pgfpathcurveto{\pgfqpoint{4.227940in}{5.961314in}}{\pgfqpoint{4.222353in}{5.959000in}}{\pgfqpoint{4.218235in}{5.954882in}}%
\pgfpathcurveto{\pgfqpoint{4.214117in}{5.950764in}}{\pgfqpoint{4.211803in}{5.945178in}}{\pgfqpoint{4.211803in}{5.939354in}}%
\pgfpathcurveto{\pgfqpoint{4.211803in}{5.933530in}}{\pgfqpoint{4.214117in}{5.927944in}}{\pgfqpoint{4.218235in}{5.923826in}}%
\pgfpathcurveto{\pgfqpoint{4.222353in}{5.919708in}}{\pgfqpoint{4.227940in}{5.917394in}}{\pgfqpoint{4.233763in}{5.917394in}}%
\pgfpathlineto{\pgfqpoint{4.233763in}{5.917394in}}%
\pgfpathclose%
\pgfusepath{stroke,fill}%
\end{pgfscope}%
\begin{pgfscope}%
\pgfpathrectangle{\pgfqpoint{1.000000in}{0.979904in}}{\pgfqpoint{6.200000in}{5.960192in}}%
\pgfusepath{clip}%
\pgfsetbuttcap%
\pgfsetroundjoin%
\definecolor{currentfill}{rgb}{0.200000,0.800000,0.200000}%
\pgfsetfillcolor{currentfill}%
\pgfsetlinewidth{1.003750pt}%
\definecolor{currentstroke}{rgb}{0.200000,0.800000,0.200000}%
\pgfsetstrokecolor{currentstroke}%
\pgfsetdash{}{0pt}%
\pgfpathmoveto{\pgfqpoint{4.093466in}{5.950042in}}%
\pgfpathcurveto{\pgfqpoint{4.099290in}{5.950042in}}{\pgfqpoint{4.104877in}{5.952356in}}{\pgfqpoint{4.108995in}{5.956474in}}%
\pgfpathcurveto{\pgfqpoint{4.113113in}{5.960592in}}{\pgfqpoint{4.115427in}{5.966178in}}{\pgfqpoint{4.115427in}{5.972002in}}%
\pgfpathcurveto{\pgfqpoint{4.115427in}{5.977826in}}{\pgfqpoint{4.113113in}{5.983412in}}{\pgfqpoint{4.108995in}{5.987530in}}%
\pgfpathcurveto{\pgfqpoint{4.104877in}{5.991648in}}{\pgfqpoint{4.099290in}{5.993962in}}{\pgfqpoint{4.093466in}{5.993962in}}%
\pgfpathcurveto{\pgfqpoint{4.087643in}{5.993962in}}{\pgfqpoint{4.082056in}{5.991648in}}{\pgfqpoint{4.077938in}{5.987530in}}%
\pgfpathcurveto{\pgfqpoint{4.073820in}{5.983412in}}{\pgfqpoint{4.071506in}{5.977826in}}{\pgfqpoint{4.071506in}{5.972002in}}%
\pgfpathcurveto{\pgfqpoint{4.071506in}{5.966178in}}{\pgfqpoint{4.073820in}{5.960592in}}{\pgfqpoint{4.077938in}{5.956474in}}%
\pgfpathcurveto{\pgfqpoint{4.082056in}{5.952356in}}{\pgfqpoint{4.087643in}{5.950042in}}{\pgfqpoint{4.093466in}{5.950042in}}%
\pgfpathlineto{\pgfqpoint{4.093466in}{5.950042in}}%
\pgfpathclose%
\pgfusepath{stroke,fill}%
\end{pgfscope}%
\begin{pgfscope}%
\pgfpathrectangle{\pgfqpoint{1.000000in}{0.979904in}}{\pgfqpoint{6.200000in}{5.960192in}}%
\pgfusepath{clip}%
\pgfsetbuttcap%
\pgfsetroundjoin%
\definecolor{currentfill}{rgb}{0.200000,0.800000,0.200000}%
\pgfsetfillcolor{currentfill}%
\pgfsetlinewidth{1.003750pt}%
\definecolor{currentstroke}{rgb}{0.200000,0.800000,0.200000}%
\pgfsetstrokecolor{currentstroke}%
\pgfsetdash{}{0pt}%
\pgfpathmoveto{\pgfqpoint{3.996712in}{5.894705in}}%
\pgfpathcurveto{\pgfqpoint{4.002535in}{5.894705in}}{\pgfqpoint{4.008122in}{5.897019in}}{\pgfqpoint{4.012240in}{5.901137in}}%
\pgfpathcurveto{\pgfqpoint{4.016358in}{5.905255in}}{\pgfqpoint{4.018672in}{5.910841in}}{\pgfqpoint{4.018672in}{5.916665in}}%
\pgfpathcurveto{\pgfqpoint{4.018672in}{5.922489in}}{\pgfqpoint{4.016358in}{5.928075in}}{\pgfqpoint{4.012240in}{5.932193in}}%
\pgfpathcurveto{\pgfqpoint{4.008122in}{5.936311in}}{\pgfqpoint{4.002535in}{5.938625in}}{\pgfqpoint{3.996712in}{5.938625in}}%
\pgfpathcurveto{\pgfqpoint{3.990888in}{5.938625in}}{\pgfqpoint{3.985301in}{5.936311in}}{\pgfqpoint{3.981183in}{5.932193in}}%
\pgfpathcurveto{\pgfqpoint{3.977065in}{5.928075in}}{\pgfqpoint{3.974751in}{5.922489in}}{\pgfqpoint{3.974751in}{5.916665in}}%
\pgfpathcurveto{\pgfqpoint{3.974751in}{5.910841in}}{\pgfqpoint{3.977065in}{5.905255in}}{\pgfqpoint{3.981183in}{5.901137in}}%
\pgfpathcurveto{\pgfqpoint{3.985301in}{5.897019in}}{\pgfqpoint{3.990888in}{5.894705in}}{\pgfqpoint{3.996712in}{5.894705in}}%
\pgfpathlineto{\pgfqpoint{3.996712in}{5.894705in}}%
\pgfpathclose%
\pgfusepath{stroke,fill}%
\end{pgfscope}%
\begin{pgfscope}%
\pgfpathrectangle{\pgfqpoint{1.000000in}{0.979904in}}{\pgfqpoint{6.200000in}{5.960192in}}%
\pgfusepath{clip}%
\pgfsetbuttcap%
\pgfsetroundjoin%
\definecolor{currentfill}{rgb}{0.200000,0.800000,0.200000}%
\pgfsetfillcolor{currentfill}%
\pgfsetlinewidth{1.003750pt}%
\definecolor{currentstroke}{rgb}{0.200000,0.800000,0.200000}%
\pgfsetstrokecolor{currentstroke}%
\pgfsetdash{}{0pt}%
\pgfpathmoveto{\pgfqpoint{3.889102in}{5.852055in}}%
\pgfpathcurveto{\pgfqpoint{3.894926in}{5.852055in}}{\pgfqpoint{3.900512in}{5.854369in}}{\pgfqpoint{3.904630in}{5.858487in}}%
\pgfpathcurveto{\pgfqpoint{3.908748in}{5.862605in}}{\pgfqpoint{3.911062in}{5.868192in}}{\pgfqpoint{3.911062in}{5.874015in}}%
\pgfpathcurveto{\pgfqpoint{3.911062in}{5.879839in}}{\pgfqpoint{3.908748in}{5.885426in}}{\pgfqpoint{3.904630in}{5.889544in}}%
\pgfpathcurveto{\pgfqpoint{3.900512in}{5.893662in}}{\pgfqpoint{3.894926in}{5.895976in}}{\pgfqpoint{3.889102in}{5.895976in}}%
\pgfpathcurveto{\pgfqpoint{3.883278in}{5.895976in}}{\pgfqpoint{3.877692in}{5.893662in}}{\pgfqpoint{3.873574in}{5.889544in}}%
\pgfpathcurveto{\pgfqpoint{3.869455in}{5.885426in}}{\pgfqpoint{3.867142in}{5.879839in}}{\pgfqpoint{3.867142in}{5.874015in}}%
\pgfpathcurveto{\pgfqpoint{3.867142in}{5.868192in}}{\pgfqpoint{3.869455in}{5.862605in}}{\pgfqpoint{3.873574in}{5.858487in}}%
\pgfpathcurveto{\pgfqpoint{3.877692in}{5.854369in}}{\pgfqpoint{3.883278in}{5.852055in}}{\pgfqpoint{3.889102in}{5.852055in}}%
\pgfpathlineto{\pgfqpoint{3.889102in}{5.852055in}}%
\pgfpathclose%
\pgfusepath{stroke,fill}%
\end{pgfscope}%
\begin{pgfscope}%
\pgfpathrectangle{\pgfqpoint{1.000000in}{0.979904in}}{\pgfqpoint{6.200000in}{5.960192in}}%
\pgfusepath{clip}%
\pgfsetbuttcap%
\pgfsetroundjoin%
\definecolor{currentfill}{rgb}{0.200000,0.800000,0.200000}%
\pgfsetfillcolor{currentfill}%
\pgfsetlinewidth{1.003750pt}%
\definecolor{currentstroke}{rgb}{0.200000,0.800000,0.200000}%
\pgfsetstrokecolor{currentstroke}%
\pgfsetdash{}{0pt}%
\pgfpathmoveto{\pgfqpoint{3.871326in}{5.699682in}}%
\pgfpathcurveto{\pgfqpoint{3.877150in}{5.699682in}}{\pgfqpoint{3.882736in}{5.701996in}}{\pgfqpoint{3.886855in}{5.706114in}}%
\pgfpathcurveto{\pgfqpoint{3.890973in}{5.710232in}}{\pgfqpoint{3.893287in}{5.715818in}}{\pgfqpoint{3.893287in}{5.721642in}}%
\pgfpathcurveto{\pgfqpoint{3.893287in}{5.727466in}}{\pgfqpoint{3.890973in}{5.733052in}}{\pgfqpoint{3.886855in}{5.737171in}}%
\pgfpathcurveto{\pgfqpoint{3.882736in}{5.741289in}}{\pgfqpoint{3.877150in}{5.743603in}}{\pgfqpoint{3.871326in}{5.743603in}}%
\pgfpathcurveto{\pgfqpoint{3.865502in}{5.743603in}}{\pgfqpoint{3.859916in}{5.741289in}}{\pgfqpoint{3.855798in}{5.737171in}}%
\pgfpathcurveto{\pgfqpoint{3.851680in}{5.733052in}}{\pgfqpoint{3.849366in}{5.727466in}}{\pgfqpoint{3.849366in}{5.721642in}}%
\pgfpathcurveto{\pgfqpoint{3.849366in}{5.715818in}}{\pgfqpoint{3.851680in}{5.710232in}}{\pgfqpoint{3.855798in}{5.706114in}}%
\pgfpathcurveto{\pgfqpoint{3.859916in}{5.701996in}}{\pgfqpoint{3.865502in}{5.699682in}}{\pgfqpoint{3.871326in}{5.699682in}}%
\pgfpathlineto{\pgfqpoint{3.871326in}{5.699682in}}%
\pgfpathclose%
\pgfusepath{stroke,fill}%
\end{pgfscope}%
\begin{pgfscope}%
\pgfpathrectangle{\pgfqpoint{1.000000in}{0.979904in}}{\pgfqpoint{6.200000in}{5.960192in}}%
\pgfusepath{clip}%
\pgfsetbuttcap%
\pgfsetroundjoin%
\definecolor{currentfill}{rgb}{0.200000,0.800000,0.200000}%
\pgfsetfillcolor{currentfill}%
\pgfsetlinewidth{1.003750pt}%
\definecolor{currentstroke}{rgb}{0.200000,0.800000,0.200000}%
\pgfsetstrokecolor{currentstroke}%
\pgfsetdash{}{0pt}%
\pgfpathmoveto{\pgfqpoint{3.796217in}{5.624279in}}%
\pgfpathcurveto{\pgfqpoint{3.802041in}{5.624279in}}{\pgfqpoint{3.807627in}{5.626592in}}{\pgfqpoint{3.811745in}{5.630711in}}%
\pgfpathcurveto{\pgfqpoint{3.815863in}{5.634829in}}{\pgfqpoint{3.818177in}{5.640415in}}{\pgfqpoint{3.818177in}{5.646239in}}%
\pgfpathcurveto{\pgfqpoint{3.818177in}{5.652063in}}{\pgfqpoint{3.815863in}{5.657649in}}{\pgfqpoint{3.811745in}{5.661767in}}%
\pgfpathcurveto{\pgfqpoint{3.807627in}{5.665885in}}{\pgfqpoint{3.802041in}{5.668199in}}{\pgfqpoint{3.796217in}{5.668199in}}%
\pgfpathcurveto{\pgfqpoint{3.790393in}{5.668199in}}{\pgfqpoint{3.784807in}{5.665885in}}{\pgfqpoint{3.780689in}{5.661767in}}%
\pgfpathcurveto{\pgfqpoint{3.776571in}{5.657649in}}{\pgfqpoint{3.774257in}{5.652063in}}{\pgfqpoint{3.774257in}{5.646239in}}%
\pgfpathcurveto{\pgfqpoint{3.774257in}{5.640415in}}{\pgfqpoint{3.776571in}{5.634829in}}{\pgfqpoint{3.780689in}{5.630711in}}%
\pgfpathcurveto{\pgfqpoint{3.784807in}{5.626592in}}{\pgfqpoint{3.790393in}{5.624279in}}{\pgfqpoint{3.796217in}{5.624279in}}%
\pgfpathlineto{\pgfqpoint{3.796217in}{5.624279in}}%
\pgfpathclose%
\pgfusepath{stroke,fill}%
\end{pgfscope}%
\begin{pgfscope}%
\pgfpathrectangle{\pgfqpoint{1.000000in}{0.979904in}}{\pgfqpoint{6.200000in}{5.960192in}}%
\pgfusepath{clip}%
\pgfsetbuttcap%
\pgfsetroundjoin%
\definecolor{currentfill}{rgb}{0.200000,0.800000,0.200000}%
\pgfsetfillcolor{currentfill}%
\pgfsetlinewidth{1.003750pt}%
\definecolor{currentstroke}{rgb}{0.200000,0.800000,0.200000}%
\pgfsetstrokecolor{currentstroke}%
\pgfsetdash{}{0pt}%
\pgfpathmoveto{\pgfqpoint{3.650082in}{5.612598in}}%
\pgfpathcurveto{\pgfqpoint{3.655906in}{5.612598in}}{\pgfqpoint{3.661493in}{5.614912in}}{\pgfqpoint{3.665611in}{5.619030in}}%
\pgfpathcurveto{\pgfqpoint{3.669729in}{5.623149in}}{\pgfqpoint{3.672043in}{5.628735in}}{\pgfqpoint{3.672043in}{5.634559in}}%
\pgfpathcurveto{\pgfqpoint{3.672043in}{5.640383in}}{\pgfqpoint{3.669729in}{5.645969in}}{\pgfqpoint{3.665611in}{5.650087in}}%
\pgfpathcurveto{\pgfqpoint{3.661493in}{5.654205in}}{\pgfqpoint{3.655906in}{5.656519in}}{\pgfqpoint{3.650082in}{5.656519in}}%
\pgfpathcurveto{\pgfqpoint{3.644259in}{5.656519in}}{\pgfqpoint{3.638672in}{5.654205in}}{\pgfqpoint{3.634554in}{5.650087in}}%
\pgfpathcurveto{\pgfqpoint{3.630436in}{5.645969in}}{\pgfqpoint{3.628122in}{5.640383in}}{\pgfqpoint{3.628122in}{5.634559in}}%
\pgfpathcurveto{\pgfqpoint{3.628122in}{5.628735in}}{\pgfqpoint{3.630436in}{5.623149in}}{\pgfqpoint{3.634554in}{5.619030in}}%
\pgfpathcurveto{\pgfqpoint{3.638672in}{5.614912in}}{\pgfqpoint{3.644259in}{5.612598in}}{\pgfqpoint{3.650082in}{5.612598in}}%
\pgfpathlineto{\pgfqpoint{3.650082in}{5.612598in}}%
\pgfpathclose%
\pgfusepath{stroke,fill}%
\end{pgfscope}%
\begin{pgfscope}%
\pgfpathrectangle{\pgfqpoint{1.000000in}{0.979904in}}{\pgfqpoint{6.200000in}{5.960192in}}%
\pgfusepath{clip}%
\pgfsetbuttcap%
\pgfsetroundjoin%
\definecolor{currentfill}{rgb}{0.200000,0.800000,0.200000}%
\pgfsetfillcolor{currentfill}%
\pgfsetlinewidth{1.003750pt}%
\definecolor{currentstroke}{rgb}{0.200000,0.800000,0.200000}%
\pgfsetstrokecolor{currentstroke}%
\pgfsetdash{}{0pt}%
\pgfpathmoveto{\pgfqpoint{3.587216in}{5.519118in}}%
\pgfpathcurveto{\pgfqpoint{3.593040in}{5.519118in}}{\pgfqpoint{3.598626in}{5.521432in}}{\pgfqpoint{3.602744in}{5.525550in}}%
\pgfpathcurveto{\pgfqpoint{3.606862in}{5.529668in}}{\pgfqpoint{3.609176in}{5.535254in}}{\pgfqpoint{3.609176in}{5.541078in}}%
\pgfpathcurveto{\pgfqpoint{3.609176in}{5.546902in}}{\pgfqpoint{3.606862in}{5.552488in}}{\pgfqpoint{3.602744in}{5.556606in}}%
\pgfpathcurveto{\pgfqpoint{3.598626in}{5.560724in}}{\pgfqpoint{3.593040in}{5.563038in}}{\pgfqpoint{3.587216in}{5.563038in}}%
\pgfpathcurveto{\pgfqpoint{3.581392in}{5.563038in}}{\pgfqpoint{3.575806in}{5.560724in}}{\pgfqpoint{3.571687in}{5.556606in}}%
\pgfpathcurveto{\pgfqpoint{3.567569in}{5.552488in}}{\pgfqpoint{3.565255in}{5.546902in}}{\pgfqpoint{3.565255in}{5.541078in}}%
\pgfpathcurveto{\pgfqpoint{3.565255in}{5.535254in}}{\pgfqpoint{3.567569in}{5.529668in}}{\pgfqpoint{3.571687in}{5.525550in}}%
\pgfpathcurveto{\pgfqpoint{3.575806in}{5.521432in}}{\pgfqpoint{3.581392in}{5.519118in}}{\pgfqpoint{3.587216in}{5.519118in}}%
\pgfpathlineto{\pgfqpoint{3.587216in}{5.519118in}}%
\pgfpathclose%
\pgfusepath{stroke,fill}%
\end{pgfscope}%
\begin{pgfscope}%
\pgfpathrectangle{\pgfqpoint{1.000000in}{0.979904in}}{\pgfqpoint{6.200000in}{5.960192in}}%
\pgfusepath{clip}%
\pgfsetbuttcap%
\pgfsetroundjoin%
\definecolor{currentfill}{rgb}{0.200000,0.800000,0.200000}%
\pgfsetfillcolor{currentfill}%
\pgfsetlinewidth{1.003750pt}%
\definecolor{currentstroke}{rgb}{0.200000,0.800000,0.200000}%
\pgfsetstrokecolor{currentstroke}%
\pgfsetdash{}{0pt}%
\pgfpathmoveto{\pgfqpoint{3.523054in}{5.427817in}}%
\pgfpathcurveto{\pgfqpoint{3.528878in}{5.427817in}}{\pgfqpoint{3.534464in}{5.430131in}}{\pgfqpoint{3.538582in}{5.434249in}}%
\pgfpathcurveto{\pgfqpoint{3.542701in}{5.438367in}}{\pgfqpoint{3.545014in}{5.443954in}}{\pgfqpoint{3.545014in}{5.449778in}}%
\pgfpathcurveto{\pgfqpoint{3.545014in}{5.455601in}}{\pgfqpoint{3.542701in}{5.461188in}}{\pgfqpoint{3.538582in}{5.465306in}}%
\pgfpathcurveto{\pgfqpoint{3.534464in}{5.469424in}}{\pgfqpoint{3.528878in}{5.471738in}}{\pgfqpoint{3.523054in}{5.471738in}}%
\pgfpathcurveto{\pgfqpoint{3.517230in}{5.471738in}}{\pgfqpoint{3.511644in}{5.469424in}}{\pgfqpoint{3.507526in}{5.465306in}}%
\pgfpathcurveto{\pgfqpoint{3.503408in}{5.461188in}}{\pgfqpoint{3.501094in}{5.455601in}}{\pgfqpoint{3.501094in}{5.449778in}}%
\pgfpathcurveto{\pgfqpoint{3.501094in}{5.443954in}}{\pgfqpoint{3.503408in}{5.438367in}}{\pgfqpoint{3.507526in}{5.434249in}}%
\pgfpathcurveto{\pgfqpoint{3.511644in}{5.430131in}}{\pgfqpoint{3.517230in}{5.427817in}}{\pgfqpoint{3.523054in}{5.427817in}}%
\pgfpathlineto{\pgfqpoint{3.523054in}{5.427817in}}%
\pgfpathclose%
\pgfusepath{stroke,fill}%
\end{pgfscope}%
\begin{pgfscope}%
\pgfpathrectangle{\pgfqpoint{1.000000in}{0.979904in}}{\pgfqpoint{6.200000in}{5.960192in}}%
\pgfusepath{clip}%
\pgfsetbuttcap%
\pgfsetroundjoin%
\definecolor{currentfill}{rgb}{0.200000,0.800000,0.200000}%
\pgfsetfillcolor{currentfill}%
\pgfsetlinewidth{1.003750pt}%
\definecolor{currentstroke}{rgb}{0.200000,0.800000,0.200000}%
\pgfsetstrokecolor{currentstroke}%
\pgfsetdash{}{0pt}%
\pgfpathmoveto{\pgfqpoint{3.458413in}{5.336645in}}%
\pgfpathcurveto{\pgfqpoint{3.464237in}{5.336645in}}{\pgfqpoint{3.469823in}{5.338959in}}{\pgfqpoint{3.473941in}{5.343077in}}%
\pgfpathcurveto{\pgfqpoint{3.478059in}{5.347195in}}{\pgfqpoint{3.480373in}{5.352781in}}{\pgfqpoint{3.480373in}{5.358605in}}%
\pgfpathcurveto{\pgfqpoint{3.480373in}{5.364429in}}{\pgfqpoint{3.478059in}{5.370016in}}{\pgfqpoint{3.473941in}{5.374134in}}%
\pgfpathcurveto{\pgfqpoint{3.469823in}{5.378252in}}{\pgfqpoint{3.464237in}{5.380566in}}{\pgfqpoint{3.458413in}{5.380566in}}%
\pgfpathcurveto{\pgfqpoint{3.452589in}{5.380566in}}{\pgfqpoint{3.447003in}{5.378252in}}{\pgfqpoint{3.442884in}{5.374134in}}%
\pgfpathcurveto{\pgfqpoint{3.438766in}{5.370016in}}{\pgfqpoint{3.436452in}{5.364429in}}{\pgfqpoint{3.436452in}{5.358605in}}%
\pgfpathcurveto{\pgfqpoint{3.436452in}{5.352781in}}{\pgfqpoint{3.438766in}{5.347195in}}{\pgfqpoint{3.442884in}{5.343077in}}%
\pgfpathcurveto{\pgfqpoint{3.447003in}{5.338959in}}{\pgfqpoint{3.452589in}{5.336645in}}{\pgfqpoint{3.458413in}{5.336645in}}%
\pgfpathlineto{\pgfqpoint{3.458413in}{5.336645in}}%
\pgfpathclose%
\pgfusepath{stroke,fill}%
\end{pgfscope}%
\begin{pgfscope}%
\pgfpathrectangle{\pgfqpoint{1.000000in}{0.979904in}}{\pgfqpoint{6.200000in}{5.960192in}}%
\pgfusepath{clip}%
\pgfsetbuttcap%
\pgfsetroundjoin%
\definecolor{currentfill}{rgb}{0.200000,0.800000,0.200000}%
\pgfsetfillcolor{currentfill}%
\pgfsetlinewidth{1.003750pt}%
\definecolor{currentstroke}{rgb}{0.200000,0.800000,0.200000}%
\pgfsetstrokecolor{currentstroke}%
\pgfsetdash{}{0pt}%
\pgfpathmoveto{\pgfqpoint{3.452459in}{5.213986in}}%
\pgfpathcurveto{\pgfqpoint{3.458283in}{5.213986in}}{\pgfqpoint{3.463869in}{5.216300in}}{\pgfqpoint{3.467987in}{5.220418in}}%
\pgfpathcurveto{\pgfqpoint{3.472105in}{5.224536in}}{\pgfqpoint{3.474419in}{5.230122in}}{\pgfqpoint{3.474419in}{5.235946in}}%
\pgfpathcurveto{\pgfqpoint{3.474419in}{5.241770in}}{\pgfqpoint{3.472105in}{5.247356in}}{\pgfqpoint{3.467987in}{5.251475in}}%
\pgfpathcurveto{\pgfqpoint{3.463869in}{5.255593in}}{\pgfqpoint{3.458283in}{5.257907in}}{\pgfqpoint{3.452459in}{5.257907in}}%
\pgfpathcurveto{\pgfqpoint{3.446635in}{5.257907in}}{\pgfqpoint{3.441049in}{5.255593in}}{\pgfqpoint{3.436931in}{5.251475in}}%
\pgfpathcurveto{\pgfqpoint{3.432813in}{5.247356in}}{\pgfqpoint{3.430499in}{5.241770in}}{\pgfqpoint{3.430499in}{5.235946in}}%
\pgfpathcurveto{\pgfqpoint{3.430499in}{5.230122in}}{\pgfqpoint{3.432813in}{5.224536in}}{\pgfqpoint{3.436931in}{5.220418in}}%
\pgfpathcurveto{\pgfqpoint{3.441049in}{5.216300in}}{\pgfqpoint{3.446635in}{5.213986in}}{\pgfqpoint{3.452459in}{5.213986in}}%
\pgfpathlineto{\pgfqpoint{3.452459in}{5.213986in}}%
\pgfpathclose%
\pgfusepath{stroke,fill}%
\end{pgfscope}%
\begin{pgfscope}%
\pgfpathrectangle{\pgfqpoint{1.000000in}{0.979904in}}{\pgfqpoint{6.200000in}{5.960192in}}%
\pgfusepath{clip}%
\pgfsetbuttcap%
\pgfsetroundjoin%
\definecolor{currentfill}{rgb}{0.200000,0.800000,0.200000}%
\pgfsetfillcolor{currentfill}%
\pgfsetlinewidth{1.003750pt}%
\definecolor{currentstroke}{rgb}{0.200000,0.800000,0.200000}%
\pgfsetstrokecolor{currentstroke}%
\pgfsetdash{}{0pt}%
\pgfpathmoveto{\pgfqpoint{3.361548in}{5.135558in}}%
\pgfpathcurveto{\pgfqpoint{3.367372in}{5.135558in}}{\pgfqpoint{3.372958in}{5.137872in}}{\pgfqpoint{3.377076in}{5.141990in}}%
\pgfpathcurveto{\pgfqpoint{3.381194in}{5.146108in}}{\pgfqpoint{3.383508in}{5.151694in}}{\pgfqpoint{3.383508in}{5.157518in}}%
\pgfpathcurveto{\pgfqpoint{3.383508in}{5.163342in}}{\pgfqpoint{3.381194in}{5.168928in}}{\pgfqpoint{3.377076in}{5.173046in}}%
\pgfpathcurveto{\pgfqpoint{3.372958in}{5.177164in}}{\pgfqpoint{3.367372in}{5.179478in}}{\pgfqpoint{3.361548in}{5.179478in}}%
\pgfpathcurveto{\pgfqpoint{3.355724in}{5.179478in}}{\pgfqpoint{3.350138in}{5.177164in}}{\pgfqpoint{3.346019in}{5.173046in}}%
\pgfpathcurveto{\pgfqpoint{3.341901in}{5.168928in}}{\pgfqpoint{3.339587in}{5.163342in}}{\pgfqpoint{3.339587in}{5.157518in}}%
\pgfpathcurveto{\pgfqpoint{3.339587in}{5.151694in}}{\pgfqpoint{3.341901in}{5.146108in}}{\pgfqpoint{3.346019in}{5.141990in}}%
\pgfpathcurveto{\pgfqpoint{3.350138in}{5.137872in}}{\pgfqpoint{3.355724in}{5.135558in}}{\pgfqpoint{3.361548in}{5.135558in}}%
\pgfpathlineto{\pgfqpoint{3.361548in}{5.135558in}}%
\pgfpathclose%
\pgfusepath{stroke,fill}%
\end{pgfscope}%
\begin{pgfscope}%
\pgfpathrectangle{\pgfqpoint{1.000000in}{0.979904in}}{\pgfqpoint{6.200000in}{5.960192in}}%
\pgfusepath{clip}%
\pgfsetbuttcap%
\pgfsetroundjoin%
\definecolor{currentfill}{rgb}{0.200000,0.800000,0.200000}%
\pgfsetfillcolor{currentfill}%
\pgfsetlinewidth{1.003750pt}%
\definecolor{currentstroke}{rgb}{0.200000,0.800000,0.200000}%
\pgfsetstrokecolor{currentstroke}%
\pgfsetdash{}{0pt}%
\pgfpathmoveto{\pgfqpoint{3.317103in}{5.033219in}}%
\pgfpathcurveto{\pgfqpoint{3.322927in}{5.033219in}}{\pgfqpoint{3.328514in}{5.035533in}}{\pgfqpoint{3.332632in}{5.039651in}}%
\pgfpathcurveto{\pgfqpoint{3.336750in}{5.043769in}}{\pgfqpoint{3.339064in}{5.049356in}}{\pgfqpoint{3.339064in}{5.055179in}}%
\pgfpathcurveto{\pgfqpoint{3.339064in}{5.061003in}}{\pgfqpoint{3.336750in}{5.066590in}}{\pgfqpoint{3.332632in}{5.070708in}}%
\pgfpathcurveto{\pgfqpoint{3.328514in}{5.074826in}}{\pgfqpoint{3.322927in}{5.077140in}}{\pgfqpoint{3.317103in}{5.077140in}}%
\pgfpathcurveto{\pgfqpoint{3.311280in}{5.077140in}}{\pgfqpoint{3.305693in}{5.074826in}}{\pgfqpoint{3.301575in}{5.070708in}}%
\pgfpathcurveto{\pgfqpoint{3.297457in}{5.066590in}}{\pgfqpoint{3.295143in}{5.061003in}}{\pgfqpoint{3.295143in}{5.055179in}}%
\pgfpathcurveto{\pgfqpoint{3.295143in}{5.049356in}}{\pgfqpoint{3.297457in}{5.043769in}}{\pgfqpoint{3.301575in}{5.039651in}}%
\pgfpathcurveto{\pgfqpoint{3.305693in}{5.035533in}}{\pgfqpoint{3.311280in}{5.033219in}}{\pgfqpoint{3.317103in}{5.033219in}}%
\pgfpathlineto{\pgfqpoint{3.317103in}{5.033219in}}%
\pgfpathclose%
\pgfusepath{stroke,fill}%
\end{pgfscope}%
\begin{pgfscope}%
\pgfpathrectangle{\pgfqpoint{1.000000in}{0.979904in}}{\pgfqpoint{6.200000in}{5.960192in}}%
\pgfusepath{clip}%
\pgfsetbuttcap%
\pgfsetroundjoin%
\definecolor{currentfill}{rgb}{0.200000,0.800000,0.200000}%
\pgfsetfillcolor{currentfill}%
\pgfsetlinewidth{1.003750pt}%
\definecolor{currentstroke}{rgb}{0.200000,0.800000,0.200000}%
\pgfsetstrokecolor{currentstroke}%
\pgfsetdash{}{0pt}%
\pgfpathmoveto{\pgfqpoint{3.333557in}{4.912105in}}%
\pgfpathcurveto{\pgfqpoint{3.339381in}{4.912105in}}{\pgfqpoint{3.344967in}{4.914419in}}{\pgfqpoint{3.349085in}{4.918537in}}%
\pgfpathcurveto{\pgfqpoint{3.353203in}{4.922656in}}{\pgfqpoint{3.355517in}{4.928242in}}{\pgfqpoint{3.355517in}{4.934066in}}%
\pgfpathcurveto{\pgfqpoint{3.355517in}{4.939890in}}{\pgfqpoint{3.353203in}{4.945476in}}{\pgfqpoint{3.349085in}{4.949594in}}%
\pgfpathcurveto{\pgfqpoint{3.344967in}{4.953712in}}{\pgfqpoint{3.339381in}{4.956026in}}{\pgfqpoint{3.333557in}{4.956026in}}%
\pgfpathcurveto{\pgfqpoint{3.327733in}{4.956026in}}{\pgfqpoint{3.322147in}{4.953712in}}{\pgfqpoint{3.318029in}{4.949594in}}%
\pgfpathcurveto{\pgfqpoint{3.313911in}{4.945476in}}{\pgfqpoint{3.311597in}{4.939890in}}{\pgfqpoint{3.311597in}{4.934066in}}%
\pgfpathcurveto{\pgfqpoint{3.311597in}{4.928242in}}{\pgfqpoint{3.313911in}{4.922656in}}{\pgfqpoint{3.318029in}{4.918537in}}%
\pgfpathcurveto{\pgfqpoint{3.322147in}{4.914419in}}{\pgfqpoint{3.327733in}{4.912105in}}{\pgfqpoint{3.333557in}{4.912105in}}%
\pgfpathlineto{\pgfqpoint{3.333557in}{4.912105in}}%
\pgfpathclose%
\pgfusepath{stroke,fill}%
\end{pgfscope}%
\begin{pgfscope}%
\pgfpathrectangle{\pgfqpoint{1.000000in}{0.979904in}}{\pgfqpoint{6.200000in}{5.960192in}}%
\pgfusepath{clip}%
\pgfsetbuttcap%
\pgfsetroundjoin%
\definecolor{currentfill}{rgb}{0.200000,0.200000,0.800000}%
\pgfsetfillcolor{currentfill}%
\pgfsetlinewidth{1.003750pt}%
\definecolor{currentstroke}{rgb}{0.200000,0.200000,0.800000}%
\pgfsetstrokecolor{currentstroke}%
\pgfsetdash{}{0pt}%
\pgfpathmoveto{\pgfqpoint{3.172844in}{4.837497in}}%
\pgfpathcurveto{\pgfqpoint{3.178668in}{4.837497in}}{\pgfqpoint{3.184254in}{4.839811in}}{\pgfqpoint{3.188372in}{4.843929in}}%
\pgfpathcurveto{\pgfqpoint{3.192491in}{4.848047in}}{\pgfqpoint{3.194804in}{4.853633in}}{\pgfqpoint{3.194804in}{4.859457in}}%
\pgfpathcurveto{\pgfqpoint{3.194804in}{4.865281in}}{\pgfqpoint{3.192491in}{4.870867in}}{\pgfqpoint{3.188372in}{4.874985in}}%
\pgfpathcurveto{\pgfqpoint{3.184254in}{4.879103in}}{\pgfqpoint{3.178668in}{4.881417in}}{\pgfqpoint{3.172844in}{4.881417in}}%
\pgfpathcurveto{\pgfqpoint{3.167020in}{4.881417in}}{\pgfqpoint{3.161434in}{4.879103in}}{\pgfqpoint{3.157316in}{4.874985in}}%
\pgfpathcurveto{\pgfqpoint{3.153198in}{4.870867in}}{\pgfqpoint{3.150884in}{4.865281in}}{\pgfqpoint{3.150884in}{4.859457in}}%
\pgfpathcurveto{\pgfqpoint{3.150884in}{4.853633in}}{\pgfqpoint{3.153198in}{4.848047in}}{\pgfqpoint{3.157316in}{4.843929in}}%
\pgfpathcurveto{\pgfqpoint{3.161434in}{4.839811in}}{\pgfqpoint{3.167020in}{4.837497in}}{\pgfqpoint{3.172844in}{4.837497in}}%
\pgfpathlineto{\pgfqpoint{3.172844in}{4.837497in}}%
\pgfpathclose%
\pgfusepath{stroke,fill}%
\end{pgfscope}%
\begin{pgfscope}%
\pgfpathrectangle{\pgfqpoint{1.000000in}{0.979904in}}{\pgfqpoint{6.200000in}{5.960192in}}%
\pgfusepath{clip}%
\pgfsetbuttcap%
\pgfsetroundjoin%
\definecolor{currentfill}{rgb}{0.200000,0.800000,0.200000}%
\pgfsetfillcolor{currentfill}%
\pgfsetlinewidth{1.003750pt}%
\definecolor{currentstroke}{rgb}{0.200000,0.800000,0.200000}%
\pgfsetstrokecolor{currentstroke}%
\pgfsetdash{}{0pt}%
\pgfpathmoveto{\pgfqpoint{3.223640in}{4.711018in}}%
\pgfpathcurveto{\pgfqpoint{3.229464in}{4.711018in}}{\pgfqpoint{3.235051in}{4.713331in}}{\pgfqpoint{3.239169in}{4.717450in}}%
\pgfpathcurveto{\pgfqpoint{3.243287in}{4.721568in}}{\pgfqpoint{3.245601in}{4.727154in}}{\pgfqpoint{3.245601in}{4.732978in}}%
\pgfpathcurveto{\pgfqpoint{3.245601in}{4.738802in}}{\pgfqpoint{3.243287in}{4.744388in}}{\pgfqpoint{3.239169in}{4.748506in}}%
\pgfpathcurveto{\pgfqpoint{3.235051in}{4.752624in}}{\pgfqpoint{3.229464in}{4.754938in}}{\pgfqpoint{3.223640in}{4.754938in}}%
\pgfpathcurveto{\pgfqpoint{3.217816in}{4.754938in}}{\pgfqpoint{3.212230in}{4.752624in}}{\pgfqpoint{3.208112in}{4.748506in}}%
\pgfpathcurveto{\pgfqpoint{3.203994in}{4.744388in}}{\pgfqpoint{3.201680in}{4.738802in}}{\pgfqpoint{3.201680in}{4.732978in}}%
\pgfpathcurveto{\pgfqpoint{3.201680in}{4.727154in}}{\pgfqpoint{3.203994in}{4.721568in}}{\pgfqpoint{3.208112in}{4.717450in}}%
\pgfpathcurveto{\pgfqpoint{3.212230in}{4.713331in}}{\pgfqpoint{3.217816in}{4.711018in}}{\pgfqpoint{3.223640in}{4.711018in}}%
\pgfpathlineto{\pgfqpoint{3.223640in}{4.711018in}}%
\pgfpathclose%
\pgfusepath{stroke,fill}%
\end{pgfscope}%
\begin{pgfscope}%
\pgfpathrectangle{\pgfqpoint{1.000000in}{0.979904in}}{\pgfqpoint{6.200000in}{5.960192in}}%
\pgfusepath{clip}%
\pgfsetbuttcap%
\pgfsetroundjoin%
\definecolor{currentfill}{rgb}{0.200000,0.800000,0.200000}%
\pgfsetfillcolor{currentfill}%
\pgfsetlinewidth{1.003750pt}%
\definecolor{currentstroke}{rgb}{0.200000,0.800000,0.200000}%
\pgfsetstrokecolor{currentstroke}%
\pgfsetdash{}{0pt}%
\pgfpathmoveto{\pgfqpoint{3.297379in}{4.591285in}}%
\pgfpathcurveto{\pgfqpoint{3.303203in}{4.591285in}}{\pgfqpoint{3.308789in}{4.593599in}}{\pgfqpoint{3.312907in}{4.597717in}}%
\pgfpathcurveto{\pgfqpoint{3.317025in}{4.601836in}}{\pgfqpoint{3.319339in}{4.607422in}}{\pgfqpoint{3.319339in}{4.613246in}}%
\pgfpathcurveto{\pgfqpoint{3.319339in}{4.619070in}}{\pgfqpoint{3.317025in}{4.624656in}}{\pgfqpoint{3.312907in}{4.628774in}}%
\pgfpathcurveto{\pgfqpoint{3.308789in}{4.632892in}}{\pgfqpoint{3.303203in}{4.635206in}}{\pgfqpoint{3.297379in}{4.635206in}}%
\pgfpathcurveto{\pgfqpoint{3.291555in}{4.635206in}}{\pgfqpoint{3.285968in}{4.632892in}}{\pgfqpoint{3.281850in}{4.628774in}}%
\pgfpathcurveto{\pgfqpoint{3.277732in}{4.624656in}}{\pgfqpoint{3.275418in}{4.619070in}}{\pgfqpoint{3.275418in}{4.613246in}}%
\pgfpathcurveto{\pgfqpoint{3.275418in}{4.607422in}}{\pgfqpoint{3.277732in}{4.601836in}}{\pgfqpoint{3.281850in}{4.597717in}}%
\pgfpathcurveto{\pgfqpoint{3.285968in}{4.593599in}}{\pgfqpoint{3.291555in}{4.591285in}}{\pgfqpoint{3.297379in}{4.591285in}}%
\pgfpathlineto{\pgfqpoint{3.297379in}{4.591285in}}%
\pgfpathclose%
\pgfusepath{stroke,fill}%
\end{pgfscope}%
\begin{pgfscope}%
\pgfpathrectangle{\pgfqpoint{1.000000in}{0.979904in}}{\pgfqpoint{6.200000in}{5.960192in}}%
\pgfusepath{clip}%
\pgfsetbuttcap%
\pgfsetroundjoin%
\definecolor{currentfill}{rgb}{0.200000,0.800000,0.200000}%
\pgfsetfillcolor{currentfill}%
\pgfsetlinewidth{1.003750pt}%
\definecolor{currentstroke}{rgb}{0.200000,0.800000,0.200000}%
\pgfsetstrokecolor{currentstroke}%
\pgfsetdash{}{0pt}%
\pgfpathmoveto{\pgfqpoint{3.192624in}{4.487980in}}%
\pgfpathcurveto{\pgfqpoint{3.198448in}{4.487980in}}{\pgfqpoint{3.204034in}{4.490294in}}{\pgfqpoint{3.208152in}{4.494412in}}%
\pgfpathcurveto{\pgfqpoint{3.212270in}{4.498530in}}{\pgfqpoint{3.214584in}{4.504116in}}{\pgfqpoint{3.214584in}{4.509940in}}%
\pgfpathcurveto{\pgfqpoint{3.214584in}{4.515764in}}{\pgfqpoint{3.212270in}{4.521350in}}{\pgfqpoint{3.208152in}{4.525468in}}%
\pgfpathcurveto{\pgfqpoint{3.204034in}{4.529586in}}{\pgfqpoint{3.198448in}{4.531900in}}{\pgfqpoint{3.192624in}{4.531900in}}%
\pgfpathcurveto{\pgfqpoint{3.186800in}{4.531900in}}{\pgfqpoint{3.181214in}{4.529586in}}{\pgfqpoint{3.177096in}{4.525468in}}%
\pgfpathcurveto{\pgfqpoint{3.172977in}{4.521350in}}{\pgfqpoint{3.170664in}{4.515764in}}{\pgfqpoint{3.170664in}{4.509940in}}%
\pgfpathcurveto{\pgfqpoint{3.170664in}{4.504116in}}{\pgfqpoint{3.172977in}{4.498530in}}{\pgfqpoint{3.177096in}{4.494412in}}%
\pgfpathcurveto{\pgfqpoint{3.181214in}{4.490294in}}{\pgfqpoint{3.186800in}{4.487980in}}{\pgfqpoint{3.192624in}{4.487980in}}%
\pgfpathlineto{\pgfqpoint{3.192624in}{4.487980in}}%
\pgfpathclose%
\pgfusepath{stroke,fill}%
\end{pgfscope}%
\begin{pgfscope}%
\pgfpathrectangle{\pgfqpoint{1.000000in}{0.979904in}}{\pgfqpoint{6.200000in}{5.960192in}}%
\pgfusepath{clip}%
\pgfsetbuttcap%
\pgfsetroundjoin%
\definecolor{currentfill}{rgb}{0.800000,0.200000,0.200000}%
\pgfsetfillcolor{currentfill}%
\pgfsetlinewidth{1.003750pt}%
\definecolor{currentstroke}{rgb}{0.800000,0.200000,0.200000}%
\pgfsetstrokecolor{currentstroke}%
\pgfsetdash{}{0pt}%
\pgfpathmoveto{\pgfqpoint{3.279128in}{4.377874in}}%
\pgfpathcurveto{\pgfqpoint{3.284952in}{4.377874in}}{\pgfqpoint{3.290538in}{4.380187in}}{\pgfqpoint{3.294656in}{4.384306in}}%
\pgfpathcurveto{\pgfqpoint{3.298775in}{4.388424in}}{\pgfqpoint{3.301088in}{4.394010in}}{\pgfqpoint{3.301088in}{4.399834in}}%
\pgfpathcurveto{\pgfqpoint{3.301088in}{4.405658in}}{\pgfqpoint{3.298775in}{4.411244in}}{\pgfqpoint{3.294656in}{4.415362in}}%
\pgfpathcurveto{\pgfqpoint{3.290538in}{4.419480in}}{\pgfqpoint{3.284952in}{4.421794in}}{\pgfqpoint{3.279128in}{4.421794in}}%
\pgfpathcurveto{\pgfqpoint{3.273304in}{4.421794in}}{\pgfqpoint{3.267718in}{4.419480in}}{\pgfqpoint{3.263600in}{4.415362in}}%
\pgfpathcurveto{\pgfqpoint{3.259482in}{4.411244in}}{\pgfqpoint{3.257168in}{4.405658in}}{\pgfqpoint{3.257168in}{4.399834in}}%
\pgfpathcurveto{\pgfqpoint{3.257168in}{4.394010in}}{\pgfqpoint{3.259482in}{4.388424in}}{\pgfqpoint{3.263600in}{4.384306in}}%
\pgfpathcurveto{\pgfqpoint{3.267718in}{4.380187in}}{\pgfqpoint{3.273304in}{4.377874in}}{\pgfqpoint{3.279128in}{4.377874in}}%
\pgfpathlineto{\pgfqpoint{3.279128in}{4.377874in}}%
\pgfpathclose%
\pgfusepath{stroke,fill}%
\end{pgfscope}%
\begin{pgfscope}%
\pgfpathrectangle{\pgfqpoint{1.000000in}{0.979904in}}{\pgfqpoint{6.200000in}{5.960192in}}%
\pgfusepath{clip}%
\pgfsetbuttcap%
\pgfsetroundjoin%
\definecolor{currentfill}{rgb}{0.200000,0.800000,0.200000}%
\pgfsetfillcolor{currentfill}%
\pgfsetlinewidth{1.003750pt}%
\definecolor{currentstroke}{rgb}{0.200000,0.800000,0.200000}%
\pgfsetstrokecolor{currentstroke}%
\pgfsetdash{}{0pt}%
\pgfpathmoveto{\pgfqpoint{3.192887in}{4.261844in}}%
\pgfpathcurveto{\pgfqpoint{3.198711in}{4.261844in}}{\pgfqpoint{3.204297in}{4.264158in}}{\pgfqpoint{3.208415in}{4.268276in}}%
\pgfpathcurveto{\pgfqpoint{3.212534in}{4.272394in}}{\pgfqpoint{3.214847in}{4.277980in}}{\pgfqpoint{3.214847in}{4.283804in}}%
\pgfpathcurveto{\pgfqpoint{3.214847in}{4.289628in}}{\pgfqpoint{3.212534in}{4.295214in}}{\pgfqpoint{3.208415in}{4.299333in}}%
\pgfpathcurveto{\pgfqpoint{3.204297in}{4.303451in}}{\pgfqpoint{3.198711in}{4.305765in}}{\pgfqpoint{3.192887in}{4.305765in}}%
\pgfpathcurveto{\pgfqpoint{3.187063in}{4.305765in}}{\pgfqpoint{3.181477in}{4.303451in}}{\pgfqpoint{3.177359in}{4.299333in}}%
\pgfpathcurveto{\pgfqpoint{3.173241in}{4.295214in}}{\pgfqpoint{3.170927in}{4.289628in}}{\pgfqpoint{3.170927in}{4.283804in}}%
\pgfpathcurveto{\pgfqpoint{3.170927in}{4.277980in}}{\pgfqpoint{3.173241in}{4.272394in}}{\pgfqpoint{3.177359in}{4.268276in}}%
\pgfpathcurveto{\pgfqpoint{3.181477in}{4.264158in}}{\pgfqpoint{3.187063in}{4.261844in}}{\pgfqpoint{3.192887in}{4.261844in}}%
\pgfpathlineto{\pgfqpoint{3.192887in}{4.261844in}}%
\pgfpathclose%
\pgfusepath{stroke,fill}%
\end{pgfscope}%
\begin{pgfscope}%
\pgfpathrectangle{\pgfqpoint{1.000000in}{0.979904in}}{\pgfqpoint{6.200000in}{5.960192in}}%
\pgfusepath{clip}%
\pgfsetbuttcap%
\pgfsetroundjoin%
\definecolor{currentfill}{rgb}{0.200000,0.800000,0.200000}%
\pgfsetfillcolor{currentfill}%
\pgfsetlinewidth{1.003750pt}%
\definecolor{currentstroke}{rgb}{0.200000,0.800000,0.200000}%
\pgfsetstrokecolor{currentstroke}%
\pgfsetdash{}{0pt}%
\pgfpathmoveto{\pgfqpoint{3.187766in}{4.146349in}}%
\pgfpathcurveto{\pgfqpoint{3.193589in}{4.146349in}}{\pgfqpoint{3.199176in}{4.148663in}}{\pgfqpoint{3.203294in}{4.152781in}}%
\pgfpathcurveto{\pgfqpoint{3.207412in}{4.156900in}}{\pgfqpoint{3.209726in}{4.162486in}}{\pgfqpoint{3.209726in}{4.168310in}}%
\pgfpathcurveto{\pgfqpoint{3.209726in}{4.174134in}}{\pgfqpoint{3.207412in}{4.179720in}}{\pgfqpoint{3.203294in}{4.183838in}}%
\pgfpathcurveto{\pgfqpoint{3.199176in}{4.187956in}}{\pgfqpoint{3.193589in}{4.190270in}}{\pgfqpoint{3.187766in}{4.190270in}}%
\pgfpathcurveto{\pgfqpoint{3.181942in}{4.190270in}}{\pgfqpoint{3.176355in}{4.187956in}}{\pgfqpoint{3.172237in}{4.183838in}}%
\pgfpathcurveto{\pgfqpoint{3.168119in}{4.179720in}}{\pgfqpoint{3.165805in}{4.174134in}}{\pgfqpoint{3.165805in}{4.168310in}}%
\pgfpathcurveto{\pgfqpoint{3.165805in}{4.162486in}}{\pgfqpoint{3.168119in}{4.156900in}}{\pgfqpoint{3.172237in}{4.152781in}}%
\pgfpathcurveto{\pgfqpoint{3.176355in}{4.148663in}}{\pgfqpoint{3.181942in}{4.146349in}}{\pgfqpoint{3.187766in}{4.146349in}}%
\pgfpathlineto{\pgfqpoint{3.187766in}{4.146349in}}%
\pgfpathclose%
\pgfusepath{stroke,fill}%
\end{pgfscope}%
\begin{pgfscope}%
\pgfpathrectangle{\pgfqpoint{1.000000in}{0.979904in}}{\pgfqpoint{6.200000in}{5.960192in}}%
\pgfusepath{clip}%
\pgfsetbuttcap%
\pgfsetroundjoin%
\definecolor{currentfill}{rgb}{0.200000,0.800000,0.200000}%
\pgfsetfillcolor{currentfill}%
\pgfsetlinewidth{1.003750pt}%
\definecolor{currentstroke}{rgb}{0.200000,0.800000,0.200000}%
\pgfsetstrokecolor{currentstroke}%
\pgfsetdash{}{0pt}%
\pgfpathmoveto{\pgfqpoint{3.311877in}{4.057012in}}%
\pgfpathcurveto{\pgfqpoint{3.317700in}{4.057012in}}{\pgfqpoint{3.323287in}{4.059326in}}{\pgfqpoint{3.327405in}{4.063445in}}%
\pgfpathcurveto{\pgfqpoint{3.331523in}{4.067563in}}{\pgfqpoint{3.333837in}{4.073149in}}{\pgfqpoint{3.333837in}{4.078973in}}%
\pgfpathcurveto{\pgfqpoint{3.333837in}{4.084797in}}{\pgfqpoint{3.331523in}{4.090383in}}{\pgfqpoint{3.327405in}{4.094501in}}%
\pgfpathcurveto{\pgfqpoint{3.323287in}{4.098619in}}{\pgfqpoint{3.317700in}{4.100933in}}{\pgfqpoint{3.311877in}{4.100933in}}%
\pgfpathcurveto{\pgfqpoint{3.306053in}{4.100933in}}{\pgfqpoint{3.300466in}{4.098619in}}{\pgfqpoint{3.296348in}{4.094501in}}%
\pgfpathcurveto{\pgfqpoint{3.292230in}{4.090383in}}{\pgfqpoint{3.289916in}{4.084797in}}{\pgfqpoint{3.289916in}{4.078973in}}%
\pgfpathcurveto{\pgfqpoint{3.289916in}{4.073149in}}{\pgfqpoint{3.292230in}{4.067563in}}{\pgfqpoint{3.296348in}{4.063445in}}%
\pgfpathcurveto{\pgfqpoint{3.300466in}{4.059326in}}{\pgfqpoint{3.306053in}{4.057012in}}{\pgfqpoint{3.311877in}{4.057012in}}%
\pgfpathlineto{\pgfqpoint{3.311877in}{4.057012in}}%
\pgfpathclose%
\pgfusepath{stroke,fill}%
\end{pgfscope}%
\begin{pgfscope}%
\pgfpathrectangle{\pgfqpoint{1.000000in}{0.979904in}}{\pgfqpoint{6.200000in}{5.960192in}}%
\pgfusepath{clip}%
\pgfsetbuttcap%
\pgfsetroundjoin%
\definecolor{currentfill}{rgb}{0.200000,0.800000,0.200000}%
\pgfsetfillcolor{currentfill}%
\pgfsetlinewidth{1.003750pt}%
\definecolor{currentstroke}{rgb}{0.200000,0.800000,0.200000}%
\pgfsetstrokecolor{currentstroke}%
\pgfsetdash{}{0pt}%
\pgfpathmoveto{\pgfqpoint{3.248148in}{3.925924in}}%
\pgfpathcurveto{\pgfqpoint{3.253972in}{3.925924in}}{\pgfqpoint{3.259558in}{3.928237in}}{\pgfqpoint{3.263676in}{3.932356in}}%
\pgfpathcurveto{\pgfqpoint{3.267795in}{3.936474in}}{\pgfqpoint{3.270108in}{3.942060in}}{\pgfqpoint{3.270108in}{3.947884in}}%
\pgfpathcurveto{\pgfqpoint{3.270108in}{3.953708in}}{\pgfqpoint{3.267795in}{3.959294in}}{\pgfqpoint{3.263676in}{3.963412in}}%
\pgfpathcurveto{\pgfqpoint{3.259558in}{3.967530in}}{\pgfqpoint{3.253972in}{3.969844in}}{\pgfqpoint{3.248148in}{3.969844in}}%
\pgfpathcurveto{\pgfqpoint{3.242324in}{3.969844in}}{\pgfqpoint{3.236738in}{3.967530in}}{\pgfqpoint{3.232620in}{3.963412in}}%
\pgfpathcurveto{\pgfqpoint{3.228502in}{3.959294in}}{\pgfqpoint{3.226188in}{3.953708in}}{\pgfqpoint{3.226188in}{3.947884in}}%
\pgfpathcurveto{\pgfqpoint{3.226188in}{3.942060in}}{\pgfqpoint{3.228502in}{3.936474in}}{\pgfqpoint{3.232620in}{3.932356in}}%
\pgfpathcurveto{\pgfqpoint{3.236738in}{3.928237in}}{\pgfqpoint{3.242324in}{3.925924in}}{\pgfqpoint{3.248148in}{3.925924in}}%
\pgfpathlineto{\pgfqpoint{3.248148in}{3.925924in}}%
\pgfpathclose%
\pgfusepath{stroke,fill}%
\end{pgfscope}%
\begin{pgfscope}%
\pgfpathrectangle{\pgfqpoint{1.000000in}{0.979904in}}{\pgfqpoint{6.200000in}{5.960192in}}%
\pgfusepath{clip}%
\pgfsetbuttcap%
\pgfsetroundjoin%
\definecolor{currentfill}{rgb}{0.200000,0.800000,0.200000}%
\pgfsetfillcolor{currentfill}%
\pgfsetlinewidth{1.003750pt}%
\definecolor{currentstroke}{rgb}{0.200000,0.800000,0.200000}%
\pgfsetstrokecolor{currentstroke}%
\pgfsetdash{}{0pt}%
\pgfpathmoveto{\pgfqpoint{3.311135in}{3.827716in}}%
\pgfpathcurveto{\pgfqpoint{3.316959in}{3.827716in}}{\pgfqpoint{3.322545in}{3.830030in}}{\pgfqpoint{3.326663in}{3.834148in}}%
\pgfpathcurveto{\pgfqpoint{3.330782in}{3.838266in}}{\pgfqpoint{3.333095in}{3.843852in}}{\pgfqpoint{3.333095in}{3.849676in}}%
\pgfpathcurveto{\pgfqpoint{3.333095in}{3.855500in}}{\pgfqpoint{3.330782in}{3.861086in}}{\pgfqpoint{3.326663in}{3.865204in}}%
\pgfpathcurveto{\pgfqpoint{3.322545in}{3.869323in}}{\pgfqpoint{3.316959in}{3.871636in}}{\pgfqpoint{3.311135in}{3.871636in}}%
\pgfpathcurveto{\pgfqpoint{3.305311in}{3.871636in}}{\pgfqpoint{3.299725in}{3.869323in}}{\pgfqpoint{3.295607in}{3.865204in}}%
\pgfpathcurveto{\pgfqpoint{3.291489in}{3.861086in}}{\pgfqpoint{3.289175in}{3.855500in}}{\pgfqpoint{3.289175in}{3.849676in}}%
\pgfpathcurveto{\pgfqpoint{3.289175in}{3.843852in}}{\pgfqpoint{3.291489in}{3.838266in}}{\pgfqpoint{3.295607in}{3.834148in}}%
\pgfpathcurveto{\pgfqpoint{3.299725in}{3.830030in}}{\pgfqpoint{3.305311in}{3.827716in}}{\pgfqpoint{3.311135in}{3.827716in}}%
\pgfpathlineto{\pgfqpoint{3.311135in}{3.827716in}}%
\pgfpathclose%
\pgfusepath{stroke,fill}%
\end{pgfscope}%
\begin{pgfscope}%
\pgfpathrectangle{\pgfqpoint{1.000000in}{0.979904in}}{\pgfqpoint{6.200000in}{5.960192in}}%
\pgfusepath{clip}%
\pgfsetbuttcap%
\pgfsetroundjoin%
\definecolor{currentfill}{rgb}{0.200000,0.800000,0.200000}%
\pgfsetfillcolor{currentfill}%
\pgfsetlinewidth{1.003750pt}%
\definecolor{currentstroke}{rgb}{0.200000,0.800000,0.200000}%
\pgfsetstrokecolor{currentstroke}%
\pgfsetdash{}{0pt}%
\pgfpathmoveto{\pgfqpoint{3.332880in}{3.715003in}}%
\pgfpathcurveto{\pgfqpoint{3.338704in}{3.715003in}}{\pgfqpoint{3.344290in}{3.717317in}}{\pgfqpoint{3.348408in}{3.721435in}}%
\pgfpathcurveto{\pgfqpoint{3.352526in}{3.725553in}}{\pgfqpoint{3.354840in}{3.731139in}}{\pgfqpoint{3.354840in}{3.736963in}}%
\pgfpathcurveto{\pgfqpoint{3.354840in}{3.742787in}}{\pgfqpoint{3.352526in}{3.748374in}}{\pgfqpoint{3.348408in}{3.752492in}}%
\pgfpathcurveto{\pgfqpoint{3.344290in}{3.756610in}}{\pgfqpoint{3.338704in}{3.758924in}}{\pgfqpoint{3.332880in}{3.758924in}}%
\pgfpathcurveto{\pgfqpoint{3.327056in}{3.758924in}}{\pgfqpoint{3.321470in}{3.756610in}}{\pgfqpoint{3.317352in}{3.752492in}}%
\pgfpathcurveto{\pgfqpoint{3.313233in}{3.748374in}}{\pgfqpoint{3.310920in}{3.742787in}}{\pgfqpoint{3.310920in}{3.736963in}}%
\pgfpathcurveto{\pgfqpoint{3.310920in}{3.731139in}}{\pgfqpoint{3.313233in}{3.725553in}}{\pgfqpoint{3.317352in}{3.721435in}}%
\pgfpathcurveto{\pgfqpoint{3.321470in}{3.717317in}}{\pgfqpoint{3.327056in}{3.715003in}}{\pgfqpoint{3.332880in}{3.715003in}}%
\pgfpathlineto{\pgfqpoint{3.332880in}{3.715003in}}%
\pgfpathclose%
\pgfusepath{stroke,fill}%
\end{pgfscope}%
\begin{pgfscope}%
\pgfpathrectangle{\pgfqpoint{1.000000in}{0.979904in}}{\pgfqpoint{6.200000in}{5.960192in}}%
\pgfusepath{clip}%
\pgfsetbuttcap%
\pgfsetroundjoin%
\definecolor{currentfill}{rgb}{0.200000,0.800000,0.200000}%
\pgfsetfillcolor{currentfill}%
\pgfsetlinewidth{1.003750pt}%
\definecolor{currentstroke}{rgb}{0.200000,0.800000,0.200000}%
\pgfsetstrokecolor{currentstroke}%
\pgfsetdash{}{0pt}%
\pgfpathmoveto{\pgfqpoint{3.330940in}{3.586474in}}%
\pgfpathcurveto{\pgfqpoint{3.336764in}{3.586474in}}{\pgfqpoint{3.342350in}{3.588788in}}{\pgfqpoint{3.346468in}{3.592906in}}%
\pgfpathcurveto{\pgfqpoint{3.350586in}{3.597024in}}{\pgfqpoint{3.352900in}{3.602610in}}{\pgfqpoint{3.352900in}{3.608434in}}%
\pgfpathcurveto{\pgfqpoint{3.352900in}{3.614258in}}{\pgfqpoint{3.350586in}{3.619844in}}{\pgfqpoint{3.346468in}{3.623962in}}%
\pgfpathcurveto{\pgfqpoint{3.342350in}{3.628080in}}{\pgfqpoint{3.336764in}{3.630394in}}{\pgfqpoint{3.330940in}{3.630394in}}%
\pgfpathcurveto{\pgfqpoint{3.325116in}{3.630394in}}{\pgfqpoint{3.319530in}{3.628080in}}{\pgfqpoint{3.315411in}{3.623962in}}%
\pgfpathcurveto{\pgfqpoint{3.311293in}{3.619844in}}{\pgfqpoint{3.308979in}{3.614258in}}{\pgfqpoint{3.308979in}{3.608434in}}%
\pgfpathcurveto{\pgfqpoint{3.308979in}{3.602610in}}{\pgfqpoint{3.311293in}{3.597024in}}{\pgfqpoint{3.315411in}{3.592906in}}%
\pgfpathcurveto{\pgfqpoint{3.319530in}{3.588788in}}{\pgfqpoint{3.325116in}{3.586474in}}{\pgfqpoint{3.330940in}{3.586474in}}%
\pgfpathlineto{\pgfqpoint{3.330940in}{3.586474in}}%
\pgfpathclose%
\pgfusepath{stroke,fill}%
\end{pgfscope}%
\begin{pgfscope}%
\pgfpathrectangle{\pgfqpoint{1.000000in}{0.979904in}}{\pgfqpoint{6.200000in}{5.960192in}}%
\pgfusepath{clip}%
\pgfsetbuttcap%
\pgfsetroundjoin%
\definecolor{currentfill}{rgb}{0.200000,0.800000,0.200000}%
\pgfsetfillcolor{currentfill}%
\pgfsetlinewidth{1.003750pt}%
\definecolor{currentstroke}{rgb}{0.200000,0.800000,0.200000}%
\pgfsetstrokecolor{currentstroke}%
\pgfsetdash{}{0pt}%
\pgfpathmoveto{\pgfqpoint{3.485354in}{3.542592in}}%
\pgfpathcurveto{\pgfqpoint{3.491178in}{3.542592in}}{\pgfqpoint{3.496764in}{3.544906in}}{\pgfqpoint{3.500882in}{3.549024in}}%
\pgfpathcurveto{\pgfqpoint{3.505000in}{3.553142in}}{\pgfqpoint{3.507314in}{3.558728in}}{\pgfqpoint{3.507314in}{3.564552in}}%
\pgfpathcurveto{\pgfqpoint{3.507314in}{3.570376in}}{\pgfqpoint{3.505000in}{3.575962in}}{\pgfqpoint{3.500882in}{3.580081in}}%
\pgfpathcurveto{\pgfqpoint{3.496764in}{3.584199in}}{\pgfqpoint{3.491178in}{3.586513in}}{\pgfqpoint{3.485354in}{3.586513in}}%
\pgfpathcurveto{\pgfqpoint{3.479530in}{3.586513in}}{\pgfqpoint{3.473944in}{3.584199in}}{\pgfqpoint{3.469826in}{3.580081in}}%
\pgfpathcurveto{\pgfqpoint{3.465708in}{3.575962in}}{\pgfqpoint{3.463394in}{3.570376in}}{\pgfqpoint{3.463394in}{3.564552in}}%
\pgfpathcurveto{\pgfqpoint{3.463394in}{3.558728in}}{\pgfqpoint{3.465708in}{3.553142in}}{\pgfqpoint{3.469826in}{3.549024in}}%
\pgfpathcurveto{\pgfqpoint{3.473944in}{3.544906in}}{\pgfqpoint{3.479530in}{3.542592in}}{\pgfqpoint{3.485354in}{3.542592in}}%
\pgfpathlineto{\pgfqpoint{3.485354in}{3.542592in}}%
\pgfpathclose%
\pgfusepath{stroke,fill}%
\end{pgfscope}%
\begin{pgfscope}%
\pgfpathrectangle{\pgfqpoint{1.000000in}{0.979904in}}{\pgfqpoint{6.200000in}{5.960192in}}%
\pgfusepath{clip}%
\pgfsetbuttcap%
\pgfsetroundjoin%
\definecolor{currentfill}{rgb}{0.200000,0.800000,0.200000}%
\pgfsetfillcolor{currentfill}%
\pgfsetlinewidth{1.003750pt}%
\definecolor{currentstroke}{rgb}{0.200000,0.800000,0.200000}%
\pgfsetstrokecolor{currentstroke}%
\pgfsetdash{}{0pt}%
\pgfpathmoveto{\pgfqpoint{3.566940in}{3.465504in}}%
\pgfpathcurveto{\pgfqpoint{3.572764in}{3.465504in}}{\pgfqpoint{3.578351in}{3.467817in}}{\pgfqpoint{3.582469in}{3.471936in}}%
\pgfpathcurveto{\pgfqpoint{3.586587in}{3.476054in}}{\pgfqpoint{3.588901in}{3.481640in}}{\pgfqpoint{3.588901in}{3.487464in}}%
\pgfpathcurveto{\pgfqpoint{3.588901in}{3.493288in}}{\pgfqpoint{3.586587in}{3.498874in}}{\pgfqpoint{3.582469in}{3.502992in}}%
\pgfpathcurveto{\pgfqpoint{3.578351in}{3.507110in}}{\pgfqpoint{3.572764in}{3.509424in}}{\pgfqpoint{3.566940in}{3.509424in}}%
\pgfpathcurveto{\pgfqpoint{3.561117in}{3.509424in}}{\pgfqpoint{3.555530in}{3.507110in}}{\pgfqpoint{3.551412in}{3.502992in}}%
\pgfpathcurveto{\pgfqpoint{3.547294in}{3.498874in}}{\pgfqpoint{3.544980in}{3.493288in}}{\pgfqpoint{3.544980in}{3.487464in}}%
\pgfpathcurveto{\pgfqpoint{3.544980in}{3.481640in}}{\pgfqpoint{3.547294in}{3.476054in}}{\pgfqpoint{3.551412in}{3.471936in}}%
\pgfpathcurveto{\pgfqpoint{3.555530in}{3.467817in}}{\pgfqpoint{3.561117in}{3.465504in}}{\pgfqpoint{3.566940in}{3.465504in}}%
\pgfpathlineto{\pgfqpoint{3.566940in}{3.465504in}}%
\pgfpathclose%
\pgfusepath{stroke,fill}%
\end{pgfscope}%
\begin{pgfscope}%
\pgfpathrectangle{\pgfqpoint{1.000000in}{0.979904in}}{\pgfqpoint{6.200000in}{5.960192in}}%
\pgfusepath{clip}%
\pgfsetbuttcap%
\pgfsetroundjoin%
\definecolor{currentfill}{rgb}{0.200000,0.800000,0.200000}%
\pgfsetfillcolor{currentfill}%
\pgfsetlinewidth{1.003750pt}%
\definecolor{currentstroke}{rgb}{0.200000,0.800000,0.200000}%
\pgfsetstrokecolor{currentstroke}%
\pgfsetdash{}{0pt}%
\pgfpathmoveto{\pgfqpoint{3.636699in}{3.382904in}}%
\pgfpathcurveto{\pgfqpoint{3.642523in}{3.382904in}}{\pgfqpoint{3.648109in}{3.385218in}}{\pgfqpoint{3.652227in}{3.389336in}}%
\pgfpathcurveto{\pgfqpoint{3.656345in}{3.393454in}}{\pgfqpoint{3.658659in}{3.399040in}}{\pgfqpoint{3.658659in}{3.404864in}}%
\pgfpathcurveto{\pgfqpoint{3.658659in}{3.410688in}}{\pgfqpoint{3.656345in}{3.416274in}}{\pgfqpoint{3.652227in}{3.420392in}}%
\pgfpathcurveto{\pgfqpoint{3.648109in}{3.424510in}}{\pgfqpoint{3.642523in}{3.426824in}}{\pgfqpoint{3.636699in}{3.426824in}}%
\pgfpathcurveto{\pgfqpoint{3.630875in}{3.426824in}}{\pgfqpoint{3.625289in}{3.424510in}}{\pgfqpoint{3.621171in}{3.420392in}}%
\pgfpathcurveto{\pgfqpoint{3.617053in}{3.416274in}}{\pgfqpoint{3.614739in}{3.410688in}}{\pgfqpoint{3.614739in}{3.404864in}}%
\pgfpathcurveto{\pgfqpoint{3.614739in}{3.399040in}}{\pgfqpoint{3.617053in}{3.393454in}}{\pgfqpoint{3.621171in}{3.389336in}}%
\pgfpathcurveto{\pgfqpoint{3.625289in}{3.385218in}}{\pgfqpoint{3.630875in}{3.382904in}}{\pgfqpoint{3.636699in}{3.382904in}}%
\pgfpathlineto{\pgfqpoint{3.636699in}{3.382904in}}%
\pgfpathclose%
\pgfusepath{stroke,fill}%
\end{pgfscope}%
\begin{pgfscope}%
\pgfpathrectangle{\pgfqpoint{1.000000in}{0.979904in}}{\pgfqpoint{6.200000in}{5.960192in}}%
\pgfusepath{clip}%
\pgfsetbuttcap%
\pgfsetroundjoin%
\definecolor{currentfill}{rgb}{0.200000,0.800000,0.200000}%
\pgfsetfillcolor{currentfill}%
\pgfsetlinewidth{1.003750pt}%
\definecolor{currentstroke}{rgb}{0.200000,0.800000,0.200000}%
\pgfsetstrokecolor{currentstroke}%
\pgfsetdash{}{0pt}%
\pgfpathmoveto{\pgfqpoint{3.702979in}{3.297834in}}%
\pgfpathcurveto{\pgfqpoint{3.708803in}{3.297834in}}{\pgfqpoint{3.714389in}{3.300148in}}{\pgfqpoint{3.718507in}{3.304266in}}%
\pgfpathcurveto{\pgfqpoint{3.722625in}{3.308384in}}{\pgfqpoint{3.724939in}{3.313970in}}{\pgfqpoint{3.724939in}{3.319794in}}%
\pgfpathcurveto{\pgfqpoint{3.724939in}{3.325618in}}{\pgfqpoint{3.722625in}{3.331204in}}{\pgfqpoint{3.718507in}{3.335322in}}%
\pgfpathcurveto{\pgfqpoint{3.714389in}{3.339441in}}{\pgfqpoint{3.708803in}{3.341754in}}{\pgfqpoint{3.702979in}{3.341754in}}%
\pgfpathcurveto{\pgfqpoint{3.697155in}{3.341754in}}{\pgfqpoint{3.691569in}{3.339441in}}{\pgfqpoint{3.687451in}{3.335322in}}%
\pgfpathcurveto{\pgfqpoint{3.683332in}{3.331204in}}{\pgfqpoint{3.681019in}{3.325618in}}{\pgfqpoint{3.681019in}{3.319794in}}%
\pgfpathcurveto{\pgfqpoint{3.681019in}{3.313970in}}{\pgfqpoint{3.683332in}{3.308384in}}{\pgfqpoint{3.687451in}{3.304266in}}%
\pgfpathcurveto{\pgfqpoint{3.691569in}{3.300148in}}{\pgfqpoint{3.697155in}{3.297834in}}{\pgfqpoint{3.702979in}{3.297834in}}%
\pgfpathlineto{\pgfqpoint{3.702979in}{3.297834in}}%
\pgfpathclose%
\pgfusepath{stroke,fill}%
\end{pgfscope}%
\begin{pgfscope}%
\pgfpathrectangle{\pgfqpoint{1.000000in}{0.979904in}}{\pgfqpoint{6.200000in}{5.960192in}}%
\pgfusepath{clip}%
\pgfsetbuttcap%
\pgfsetroundjoin%
\definecolor{currentfill}{rgb}{0.200000,0.800000,0.200000}%
\pgfsetfillcolor{currentfill}%
\pgfsetlinewidth{1.003750pt}%
\definecolor{currentstroke}{rgb}{0.200000,0.800000,0.200000}%
\pgfsetstrokecolor{currentstroke}%
\pgfsetdash{}{0pt}%
\pgfpathmoveto{\pgfqpoint{3.749942in}{3.191814in}}%
\pgfpathcurveto{\pgfqpoint{3.755766in}{3.191814in}}{\pgfqpoint{3.761352in}{3.194127in}}{\pgfqpoint{3.765470in}{3.198246in}}%
\pgfpathcurveto{\pgfqpoint{3.769588in}{3.202364in}}{\pgfqpoint{3.771902in}{3.207950in}}{\pgfqpoint{3.771902in}{3.213774in}}%
\pgfpathcurveto{\pgfqpoint{3.771902in}{3.219598in}}{\pgfqpoint{3.769588in}{3.225184in}}{\pgfqpoint{3.765470in}{3.229302in}}%
\pgfpathcurveto{\pgfqpoint{3.761352in}{3.233420in}}{\pgfqpoint{3.755766in}{3.235734in}}{\pgfqpoint{3.749942in}{3.235734in}}%
\pgfpathcurveto{\pgfqpoint{3.744118in}{3.235734in}}{\pgfqpoint{3.738532in}{3.233420in}}{\pgfqpoint{3.734414in}{3.229302in}}%
\pgfpathcurveto{\pgfqpoint{3.730295in}{3.225184in}}{\pgfqpoint{3.727982in}{3.219598in}}{\pgfqpoint{3.727982in}{3.213774in}}%
\pgfpathcurveto{\pgfqpoint{3.727982in}{3.207950in}}{\pgfqpoint{3.730295in}{3.202364in}}{\pgfqpoint{3.734414in}{3.198246in}}%
\pgfpathcurveto{\pgfqpoint{3.738532in}{3.194127in}}{\pgfqpoint{3.744118in}{3.191814in}}{\pgfqpoint{3.749942in}{3.191814in}}%
\pgfpathlineto{\pgfqpoint{3.749942in}{3.191814in}}%
\pgfpathclose%
\pgfusepath{stroke,fill}%
\end{pgfscope}%
\begin{pgfscope}%
\pgfpathrectangle{\pgfqpoint{1.000000in}{0.979904in}}{\pgfqpoint{6.200000in}{5.960192in}}%
\pgfusepath{clip}%
\pgfsetbuttcap%
\pgfsetroundjoin%
\definecolor{currentfill}{rgb}{0.200000,0.800000,0.200000}%
\pgfsetfillcolor{currentfill}%
\pgfsetlinewidth{1.003750pt}%
\definecolor{currentstroke}{rgb}{0.200000,0.800000,0.200000}%
\pgfsetstrokecolor{currentstroke}%
\pgfsetdash{}{0pt}%
\pgfpathmoveto{\pgfqpoint{3.816496in}{3.100147in}}%
\pgfpathcurveto{\pgfqpoint{3.822320in}{3.100147in}}{\pgfqpoint{3.827906in}{3.102461in}}{\pgfqpoint{3.832024in}{3.106579in}}%
\pgfpathcurveto{\pgfqpoint{3.836142in}{3.110697in}}{\pgfqpoint{3.838456in}{3.116284in}}{\pgfqpoint{3.838456in}{3.122107in}}%
\pgfpathcurveto{\pgfqpoint{3.838456in}{3.127931in}}{\pgfqpoint{3.836142in}{3.133518in}}{\pgfqpoint{3.832024in}{3.137636in}}%
\pgfpathcurveto{\pgfqpoint{3.827906in}{3.141754in}}{\pgfqpoint{3.822320in}{3.144068in}}{\pgfqpoint{3.816496in}{3.144068in}}%
\pgfpathcurveto{\pgfqpoint{3.810672in}{3.144068in}}{\pgfqpoint{3.805085in}{3.141754in}}{\pgfqpoint{3.800967in}{3.137636in}}%
\pgfpathcurveto{\pgfqpoint{3.796849in}{3.133518in}}{\pgfqpoint{3.794535in}{3.127931in}}{\pgfqpoint{3.794535in}{3.122107in}}%
\pgfpathcurveto{\pgfqpoint{3.794535in}{3.116284in}}{\pgfqpoint{3.796849in}{3.110697in}}{\pgfqpoint{3.800967in}{3.106579in}}%
\pgfpathcurveto{\pgfqpoint{3.805085in}{3.102461in}}{\pgfqpoint{3.810672in}{3.100147in}}{\pgfqpoint{3.816496in}{3.100147in}}%
\pgfpathlineto{\pgfqpoint{3.816496in}{3.100147in}}%
\pgfpathclose%
\pgfusepath{stroke,fill}%
\end{pgfscope}%
\begin{pgfscope}%
\pgfpathrectangle{\pgfqpoint{1.000000in}{0.979904in}}{\pgfqpoint{6.200000in}{5.960192in}}%
\pgfusepath{clip}%
\pgfsetbuttcap%
\pgfsetroundjoin%
\definecolor{currentfill}{rgb}{0.200000,0.800000,0.200000}%
\pgfsetfillcolor{currentfill}%
\pgfsetlinewidth{1.003750pt}%
\definecolor{currentstroke}{rgb}{0.200000,0.800000,0.200000}%
\pgfsetstrokecolor{currentstroke}%
\pgfsetdash{}{0pt}%
\pgfpathmoveto{\pgfqpoint{3.905333in}{3.032379in}}%
\pgfpathcurveto{\pgfqpoint{3.911157in}{3.032379in}}{\pgfqpoint{3.916743in}{3.034693in}}{\pgfqpoint{3.920861in}{3.038811in}}%
\pgfpathcurveto{\pgfqpoint{3.924979in}{3.042929in}}{\pgfqpoint{3.927293in}{3.048515in}}{\pgfqpoint{3.927293in}{3.054339in}}%
\pgfpathcurveto{\pgfqpoint{3.927293in}{3.060163in}}{\pgfqpoint{3.924979in}{3.065749in}}{\pgfqpoint{3.920861in}{3.069868in}}%
\pgfpathcurveto{\pgfqpoint{3.916743in}{3.073986in}}{\pgfqpoint{3.911157in}{3.076300in}}{\pgfqpoint{3.905333in}{3.076300in}}%
\pgfpathcurveto{\pgfqpoint{3.899509in}{3.076300in}}{\pgfqpoint{3.893922in}{3.073986in}}{\pgfqpoint{3.889804in}{3.069868in}}%
\pgfpathcurveto{\pgfqpoint{3.885686in}{3.065749in}}{\pgfqpoint{3.883372in}{3.060163in}}{\pgfqpoint{3.883372in}{3.054339in}}%
\pgfpathcurveto{\pgfqpoint{3.883372in}{3.048515in}}{\pgfqpoint{3.885686in}{3.042929in}}{\pgfqpoint{3.889804in}{3.038811in}}%
\pgfpathcurveto{\pgfqpoint{3.893922in}{3.034693in}}{\pgfqpoint{3.899509in}{3.032379in}}{\pgfqpoint{3.905333in}{3.032379in}}%
\pgfpathlineto{\pgfqpoint{3.905333in}{3.032379in}}%
\pgfpathclose%
\pgfusepath{stroke,fill}%
\end{pgfscope}%
\begin{pgfscope}%
\pgfpathrectangle{\pgfqpoint{1.000000in}{0.979904in}}{\pgfqpoint{6.200000in}{5.960192in}}%
\pgfusepath{clip}%
\pgfsetbuttcap%
\pgfsetroundjoin%
\definecolor{currentfill}{rgb}{0.200000,0.800000,0.200000}%
\pgfsetfillcolor{currentfill}%
\pgfsetlinewidth{1.003750pt}%
\definecolor{currentstroke}{rgb}{0.200000,0.800000,0.200000}%
\pgfsetstrokecolor{currentstroke}%
\pgfsetdash{}{0pt}%
\pgfpathmoveto{\pgfqpoint{3.980039in}{2.943343in}}%
\pgfpathcurveto{\pgfqpoint{3.985863in}{2.943343in}}{\pgfqpoint{3.991449in}{2.945657in}}{\pgfqpoint{3.995567in}{2.949775in}}%
\pgfpathcurveto{\pgfqpoint{3.999685in}{2.953893in}}{\pgfqpoint{4.001999in}{2.959479in}}{\pgfqpoint{4.001999in}{2.965303in}}%
\pgfpathcurveto{\pgfqpoint{4.001999in}{2.971127in}}{\pgfqpoint{3.999685in}{2.976713in}}{\pgfqpoint{3.995567in}{2.980832in}}%
\pgfpathcurveto{\pgfqpoint{3.991449in}{2.984950in}}{\pgfqpoint{3.985863in}{2.987264in}}{\pgfqpoint{3.980039in}{2.987264in}}%
\pgfpathcurveto{\pgfqpoint{3.974215in}{2.987264in}}{\pgfqpoint{3.968629in}{2.984950in}}{\pgfqpoint{3.964511in}{2.980832in}}%
\pgfpathcurveto{\pgfqpoint{3.960393in}{2.976713in}}{\pgfqpoint{3.958079in}{2.971127in}}{\pgfqpoint{3.958079in}{2.965303in}}%
\pgfpathcurveto{\pgfqpoint{3.958079in}{2.959479in}}{\pgfqpoint{3.960393in}{2.953893in}}{\pgfqpoint{3.964511in}{2.949775in}}%
\pgfpathcurveto{\pgfqpoint{3.968629in}{2.945657in}}{\pgfqpoint{3.974215in}{2.943343in}}{\pgfqpoint{3.980039in}{2.943343in}}%
\pgfpathlineto{\pgfqpoint{3.980039in}{2.943343in}}%
\pgfpathclose%
\pgfusepath{stroke,fill}%
\end{pgfscope}%
\begin{pgfscope}%
\pgfpathrectangle{\pgfqpoint{1.000000in}{0.979904in}}{\pgfqpoint{6.200000in}{5.960192in}}%
\pgfusepath{clip}%
\pgfsetbuttcap%
\pgfsetroundjoin%
\definecolor{currentfill}{rgb}{0.200000,0.800000,0.200000}%
\pgfsetfillcolor{currentfill}%
\pgfsetlinewidth{1.003750pt}%
\definecolor{currentstroke}{rgb}{0.200000,0.800000,0.200000}%
\pgfsetstrokecolor{currentstroke}%
\pgfsetdash{}{0pt}%
\pgfpathmoveto{\pgfqpoint{4.098020in}{2.920952in}}%
\pgfpathcurveto{\pgfqpoint{4.103843in}{2.920952in}}{\pgfqpoint{4.109430in}{2.923266in}}{\pgfqpoint{4.113548in}{2.927384in}}%
\pgfpathcurveto{\pgfqpoint{4.117666in}{2.931502in}}{\pgfqpoint{4.119980in}{2.937088in}}{\pgfqpoint{4.119980in}{2.942912in}}%
\pgfpathcurveto{\pgfqpoint{4.119980in}{2.948736in}}{\pgfqpoint{4.117666in}{2.954322in}}{\pgfqpoint{4.113548in}{2.958440in}}%
\pgfpathcurveto{\pgfqpoint{4.109430in}{2.962559in}}{\pgfqpoint{4.103843in}{2.964872in}}{\pgfqpoint{4.098020in}{2.964872in}}%
\pgfpathcurveto{\pgfqpoint{4.092196in}{2.964872in}}{\pgfqpoint{4.086609in}{2.962559in}}{\pgfqpoint{4.082491in}{2.958440in}}%
\pgfpathcurveto{\pgfqpoint{4.078373in}{2.954322in}}{\pgfqpoint{4.076059in}{2.948736in}}{\pgfqpoint{4.076059in}{2.942912in}}%
\pgfpathcurveto{\pgfqpoint{4.076059in}{2.937088in}}{\pgfqpoint{4.078373in}{2.931502in}}{\pgfqpoint{4.082491in}{2.927384in}}%
\pgfpathcurveto{\pgfqpoint{4.086609in}{2.923266in}}{\pgfqpoint{4.092196in}{2.920952in}}{\pgfqpoint{4.098020in}{2.920952in}}%
\pgfpathlineto{\pgfqpoint{4.098020in}{2.920952in}}%
\pgfpathclose%
\pgfusepath{stroke,fill}%
\end{pgfscope}%
\begin{pgfscope}%
\pgfpathrectangle{\pgfqpoint{1.000000in}{0.979904in}}{\pgfqpoint{6.200000in}{5.960192in}}%
\pgfusepath{clip}%
\pgfsetbuttcap%
\pgfsetroundjoin%
\definecolor{currentfill}{rgb}{0.200000,0.800000,0.200000}%
\pgfsetfillcolor{currentfill}%
\pgfsetlinewidth{1.003750pt}%
\definecolor{currentstroke}{rgb}{0.200000,0.800000,0.200000}%
\pgfsetstrokecolor{currentstroke}%
\pgfsetdash{}{0pt}%
\pgfpathmoveto{\pgfqpoint{4.161644in}{2.800198in}}%
\pgfpathcurveto{\pgfqpoint{4.167468in}{2.800198in}}{\pgfqpoint{4.173054in}{2.802511in}}{\pgfqpoint{4.177172in}{2.806630in}}%
\pgfpathcurveto{\pgfqpoint{4.181290in}{2.810748in}}{\pgfqpoint{4.183604in}{2.816334in}}{\pgfqpoint{4.183604in}{2.822158in}}%
\pgfpathcurveto{\pgfqpoint{4.183604in}{2.827982in}}{\pgfqpoint{4.181290in}{2.833568in}}{\pgfqpoint{4.177172in}{2.837686in}}%
\pgfpathcurveto{\pgfqpoint{4.173054in}{2.841804in}}{\pgfqpoint{4.167468in}{2.844118in}}{\pgfqpoint{4.161644in}{2.844118in}}%
\pgfpathcurveto{\pgfqpoint{4.155820in}{2.844118in}}{\pgfqpoint{4.150234in}{2.841804in}}{\pgfqpoint{4.146116in}{2.837686in}}%
\pgfpathcurveto{\pgfqpoint{4.141997in}{2.833568in}}{\pgfqpoint{4.139684in}{2.827982in}}{\pgfqpoint{4.139684in}{2.822158in}}%
\pgfpathcurveto{\pgfqpoint{4.139684in}{2.816334in}}{\pgfqpoint{4.141997in}{2.810748in}}{\pgfqpoint{4.146116in}{2.806630in}}%
\pgfpathcurveto{\pgfqpoint{4.150234in}{2.802511in}}{\pgfqpoint{4.155820in}{2.800198in}}{\pgfqpoint{4.161644in}{2.800198in}}%
\pgfpathlineto{\pgfqpoint{4.161644in}{2.800198in}}%
\pgfpathclose%
\pgfusepath{stroke,fill}%
\end{pgfscope}%
\begin{pgfscope}%
\pgfpathrectangle{\pgfqpoint{1.000000in}{0.979904in}}{\pgfqpoint{6.200000in}{5.960192in}}%
\pgfusepath{clip}%
\pgfsetbuttcap%
\pgfsetroundjoin%
\definecolor{currentfill}{rgb}{0.200000,0.800000,0.200000}%
\pgfsetfillcolor{currentfill}%
\pgfsetlinewidth{1.003750pt}%
\definecolor{currentstroke}{rgb}{0.200000,0.800000,0.200000}%
\pgfsetstrokecolor{currentstroke}%
\pgfsetdash{}{0pt}%
\pgfpathmoveto{\pgfqpoint{4.301958in}{2.836286in}}%
\pgfpathcurveto{\pgfqpoint{4.307782in}{2.836286in}}{\pgfqpoint{4.313369in}{2.838599in}}{\pgfqpoint{4.317487in}{2.842718in}}%
\pgfpathcurveto{\pgfqpoint{4.321605in}{2.846836in}}{\pgfqpoint{4.323919in}{2.852422in}}{\pgfqpoint{4.323919in}{2.858246in}}%
\pgfpathcurveto{\pgfqpoint{4.323919in}{2.864070in}}{\pgfqpoint{4.321605in}{2.869656in}}{\pgfqpoint{4.317487in}{2.873774in}}%
\pgfpathcurveto{\pgfqpoint{4.313369in}{2.877892in}}{\pgfqpoint{4.307782in}{2.880206in}}{\pgfqpoint{4.301958in}{2.880206in}}%
\pgfpathcurveto{\pgfqpoint{4.296134in}{2.880206in}}{\pgfqpoint{4.290548in}{2.877892in}}{\pgfqpoint{4.286430in}{2.873774in}}%
\pgfpathcurveto{\pgfqpoint{4.282312in}{2.869656in}}{\pgfqpoint{4.279998in}{2.864070in}}{\pgfqpoint{4.279998in}{2.858246in}}%
\pgfpathcurveto{\pgfqpoint{4.279998in}{2.852422in}}{\pgfqpoint{4.282312in}{2.846836in}}{\pgfqpoint{4.286430in}{2.842718in}}%
\pgfpathcurveto{\pgfqpoint{4.290548in}{2.838599in}}{\pgfqpoint{4.296134in}{2.836286in}}{\pgfqpoint{4.301958in}{2.836286in}}%
\pgfpathlineto{\pgfqpoint{4.301958in}{2.836286in}}%
\pgfpathclose%
\pgfusepath{stroke,fill}%
\end{pgfscope}%
\begin{pgfscope}%
\pgfpathrectangle{\pgfqpoint{1.000000in}{0.979904in}}{\pgfqpoint{6.200000in}{5.960192in}}%
\pgfusepath{clip}%
\pgfsetbuttcap%
\pgfsetroundjoin%
\definecolor{currentfill}{rgb}{0.800000,0.200000,0.200000}%
\pgfsetfillcolor{currentfill}%
\pgfsetlinewidth{1.003750pt}%
\definecolor{currentstroke}{rgb}{0.800000,0.200000,0.200000}%
\pgfsetstrokecolor{currentstroke}%
\pgfsetdash{}{0pt}%
\pgfpathmoveto{\pgfqpoint{4.352686in}{2.647462in}}%
\pgfpathcurveto{\pgfqpoint{4.358510in}{2.647462in}}{\pgfqpoint{4.364096in}{2.649776in}}{\pgfqpoint{4.368214in}{2.653894in}}%
\pgfpathcurveto{\pgfqpoint{4.372332in}{2.658012in}}{\pgfqpoint{4.374646in}{2.663598in}}{\pgfqpoint{4.374646in}{2.669422in}}%
\pgfpathcurveto{\pgfqpoint{4.374646in}{2.675246in}}{\pgfqpoint{4.372332in}{2.680832in}}{\pgfqpoint{4.368214in}{2.684951in}}%
\pgfpathcurveto{\pgfqpoint{4.364096in}{2.689069in}}{\pgfqpoint{4.358510in}{2.691383in}}{\pgfqpoint{4.352686in}{2.691383in}}%
\pgfpathcurveto{\pgfqpoint{4.346862in}{2.691383in}}{\pgfqpoint{4.341276in}{2.689069in}}{\pgfqpoint{4.337158in}{2.684951in}}%
\pgfpathcurveto{\pgfqpoint{4.333039in}{2.680832in}}{\pgfqpoint{4.330726in}{2.675246in}}{\pgfqpoint{4.330726in}{2.669422in}}%
\pgfpathcurveto{\pgfqpoint{4.330726in}{2.663598in}}{\pgfqpoint{4.333039in}{2.658012in}}{\pgfqpoint{4.337158in}{2.653894in}}%
\pgfpathcurveto{\pgfqpoint{4.341276in}{2.649776in}}{\pgfqpoint{4.346862in}{2.647462in}}{\pgfqpoint{4.352686in}{2.647462in}}%
\pgfpathlineto{\pgfqpoint{4.352686in}{2.647462in}}%
\pgfpathclose%
\pgfusepath{stroke,fill}%
\end{pgfscope}%
\begin{pgfscope}%
\pgfpathrectangle{\pgfqpoint{1.000000in}{0.979904in}}{\pgfqpoint{6.200000in}{5.960192in}}%
\pgfusepath{clip}%
\pgfsetbuttcap%
\pgfsetroundjoin%
\definecolor{currentfill}{rgb}{0.800000,0.200000,0.200000}%
\pgfsetfillcolor{currentfill}%
\pgfsetlinewidth{1.003750pt}%
\definecolor{currentstroke}{rgb}{0.800000,0.200000,0.200000}%
\pgfsetstrokecolor{currentstroke}%
\pgfsetdash{}{0pt}%
\pgfpathmoveto{\pgfqpoint{4.496696in}{2.719000in}}%
\pgfpathcurveto{\pgfqpoint{4.502519in}{2.719000in}}{\pgfqpoint{4.508106in}{2.721314in}}{\pgfqpoint{4.512224in}{2.725432in}}%
\pgfpathcurveto{\pgfqpoint{4.516342in}{2.729550in}}{\pgfqpoint{4.518656in}{2.735137in}}{\pgfqpoint{4.518656in}{2.740960in}}%
\pgfpathcurveto{\pgfqpoint{4.518656in}{2.746784in}}{\pgfqpoint{4.516342in}{2.752371in}}{\pgfqpoint{4.512224in}{2.756489in}}%
\pgfpathcurveto{\pgfqpoint{4.508106in}{2.760607in}}{\pgfqpoint{4.502519in}{2.762921in}}{\pgfqpoint{4.496696in}{2.762921in}}%
\pgfpathcurveto{\pgfqpoint{4.490872in}{2.762921in}}{\pgfqpoint{4.485285in}{2.760607in}}{\pgfqpoint{4.481167in}{2.756489in}}%
\pgfpathcurveto{\pgfqpoint{4.477049in}{2.752371in}}{\pgfqpoint{4.474735in}{2.746784in}}{\pgfqpoint{4.474735in}{2.740960in}}%
\pgfpathcurveto{\pgfqpoint{4.474735in}{2.735137in}}{\pgfqpoint{4.477049in}{2.729550in}}{\pgfqpoint{4.481167in}{2.725432in}}%
\pgfpathcurveto{\pgfqpoint{4.485285in}{2.721314in}}{\pgfqpoint{4.490872in}{2.719000in}}{\pgfqpoint{4.496696in}{2.719000in}}%
\pgfpathlineto{\pgfqpoint{4.496696in}{2.719000in}}%
\pgfpathclose%
\pgfusepath{stroke,fill}%
\end{pgfscope}%
\begin{pgfscope}%
\pgfpathrectangle{\pgfqpoint{1.000000in}{0.979904in}}{\pgfqpoint{6.200000in}{5.960192in}}%
\pgfusepath{clip}%
\pgfsetbuttcap%
\pgfsetroundjoin%
\definecolor{currentfill}{rgb}{0.200000,0.800000,0.200000}%
\pgfsetfillcolor{currentfill}%
\pgfsetlinewidth{1.003750pt}%
\definecolor{currentstroke}{rgb}{0.200000,0.800000,0.200000}%
\pgfsetstrokecolor{currentstroke}%
\pgfsetdash{}{0pt}%
\pgfpathmoveto{\pgfqpoint{4.622379in}{2.769425in}}%
\pgfpathcurveto{\pgfqpoint{4.628203in}{2.769425in}}{\pgfqpoint{4.633789in}{2.771739in}}{\pgfqpoint{4.637907in}{2.775857in}}%
\pgfpathcurveto{\pgfqpoint{4.642025in}{2.779975in}}{\pgfqpoint{4.644339in}{2.785561in}}{\pgfqpoint{4.644339in}{2.791385in}}%
\pgfpathcurveto{\pgfqpoint{4.644339in}{2.797209in}}{\pgfqpoint{4.642025in}{2.802795in}}{\pgfqpoint{4.637907in}{2.806913in}}%
\pgfpathcurveto{\pgfqpoint{4.633789in}{2.811031in}}{\pgfqpoint{4.628203in}{2.813345in}}{\pgfqpoint{4.622379in}{2.813345in}}%
\pgfpathcurveto{\pgfqpoint{4.616555in}{2.813345in}}{\pgfqpoint{4.610968in}{2.811031in}}{\pgfqpoint{4.606850in}{2.806913in}}%
\pgfpathcurveto{\pgfqpoint{4.602732in}{2.802795in}}{\pgfqpoint{4.600418in}{2.797209in}}{\pgfqpoint{4.600418in}{2.791385in}}%
\pgfpathcurveto{\pgfqpoint{4.600418in}{2.785561in}}{\pgfqpoint{4.602732in}{2.779975in}}{\pgfqpoint{4.606850in}{2.775857in}}%
\pgfpathcurveto{\pgfqpoint{4.610968in}{2.771739in}}{\pgfqpoint{4.616555in}{2.769425in}}{\pgfqpoint{4.622379in}{2.769425in}}%
\pgfpathlineto{\pgfqpoint{4.622379in}{2.769425in}}%
\pgfpathclose%
\pgfusepath{stroke,fill}%
\end{pgfscope}%
\begin{pgfscope}%
\pgfpathrectangle{\pgfqpoint{1.000000in}{0.979904in}}{\pgfqpoint{6.200000in}{5.960192in}}%
\pgfusepath{clip}%
\pgfsetbuttcap%
\pgfsetroundjoin%
\definecolor{currentfill}{rgb}{0.200000,0.800000,0.200000}%
\pgfsetfillcolor{currentfill}%
\pgfsetlinewidth{1.003750pt}%
\definecolor{currentstroke}{rgb}{0.200000,0.800000,0.200000}%
\pgfsetstrokecolor{currentstroke}%
\pgfsetdash{}{0pt}%
\pgfpathmoveto{\pgfqpoint{4.730111in}{2.761938in}}%
\pgfpathcurveto{\pgfqpoint{4.735935in}{2.761938in}}{\pgfqpoint{4.741521in}{2.764252in}}{\pgfqpoint{4.745640in}{2.768370in}}%
\pgfpathcurveto{\pgfqpoint{4.749758in}{2.772488in}}{\pgfqpoint{4.752072in}{2.778074in}}{\pgfqpoint{4.752072in}{2.783898in}}%
\pgfpathcurveto{\pgfqpoint{4.752072in}{2.789722in}}{\pgfqpoint{4.749758in}{2.795308in}}{\pgfqpoint{4.745640in}{2.799426in}}%
\pgfpathcurveto{\pgfqpoint{4.741521in}{2.803544in}}{\pgfqpoint{4.735935in}{2.805858in}}{\pgfqpoint{4.730111in}{2.805858in}}%
\pgfpathcurveto{\pgfqpoint{4.724287in}{2.805858in}}{\pgfqpoint{4.718701in}{2.803544in}}{\pgfqpoint{4.714583in}{2.799426in}}%
\pgfpathcurveto{\pgfqpoint{4.710465in}{2.795308in}}{\pgfqpoint{4.708151in}{2.789722in}}{\pgfqpoint{4.708151in}{2.783898in}}%
\pgfpathcurveto{\pgfqpoint{4.708151in}{2.778074in}}{\pgfqpoint{4.710465in}{2.772488in}}{\pgfqpoint{4.714583in}{2.768370in}}%
\pgfpathcurveto{\pgfqpoint{4.718701in}{2.764252in}}{\pgfqpoint{4.724287in}{2.761938in}}{\pgfqpoint{4.730111in}{2.761938in}}%
\pgfpathlineto{\pgfqpoint{4.730111in}{2.761938in}}%
\pgfpathclose%
\pgfusepath{stroke,fill}%
\end{pgfscope}%
\begin{pgfscope}%
\pgfpathrectangle{\pgfqpoint{1.000000in}{0.979904in}}{\pgfqpoint{6.200000in}{5.960192in}}%
\pgfusepath{clip}%
\pgfsetbuttcap%
\pgfsetroundjoin%
\definecolor{currentfill}{rgb}{0.200000,0.800000,0.200000}%
\pgfsetfillcolor{currentfill}%
\pgfsetlinewidth{1.003750pt}%
\definecolor{currentstroke}{rgb}{0.200000,0.800000,0.200000}%
\pgfsetstrokecolor{currentstroke}%
\pgfsetdash{}{0pt}%
\pgfpathmoveto{\pgfqpoint{4.829957in}{2.667917in}}%
\pgfpathcurveto{\pgfqpoint{4.835781in}{2.667917in}}{\pgfqpoint{4.841367in}{2.670231in}}{\pgfqpoint{4.845485in}{2.674349in}}%
\pgfpathcurveto{\pgfqpoint{4.849603in}{2.678467in}}{\pgfqpoint{4.851917in}{2.684053in}}{\pgfqpoint{4.851917in}{2.689877in}}%
\pgfpathcurveto{\pgfqpoint{4.851917in}{2.695701in}}{\pgfqpoint{4.849603in}{2.701287in}}{\pgfqpoint{4.845485in}{2.705405in}}%
\pgfpathcurveto{\pgfqpoint{4.841367in}{2.709524in}}{\pgfqpoint{4.835781in}{2.711837in}}{\pgfqpoint{4.829957in}{2.711837in}}%
\pgfpathcurveto{\pgfqpoint{4.824133in}{2.711837in}}{\pgfqpoint{4.818547in}{2.709524in}}{\pgfqpoint{4.814429in}{2.705405in}}%
\pgfpathcurveto{\pgfqpoint{4.810310in}{2.701287in}}{\pgfqpoint{4.807997in}{2.695701in}}{\pgfqpoint{4.807997in}{2.689877in}}%
\pgfpathcurveto{\pgfqpoint{4.807997in}{2.684053in}}{\pgfqpoint{4.810310in}{2.678467in}}{\pgfqpoint{4.814429in}{2.674349in}}%
\pgfpathcurveto{\pgfqpoint{4.818547in}{2.670231in}}{\pgfqpoint{4.824133in}{2.667917in}}{\pgfqpoint{4.829957in}{2.667917in}}%
\pgfpathlineto{\pgfqpoint{4.829957in}{2.667917in}}%
\pgfpathclose%
\pgfusepath{stroke,fill}%
\end{pgfscope}%
\begin{pgfscope}%
\pgfpathrectangle{\pgfqpoint{1.000000in}{0.979904in}}{\pgfqpoint{6.200000in}{5.960192in}}%
\pgfusepath{clip}%
\pgfsetbuttcap%
\pgfsetroundjoin%
\definecolor{currentfill}{rgb}{0.200000,0.800000,0.200000}%
\pgfsetfillcolor{currentfill}%
\pgfsetlinewidth{1.003750pt}%
\definecolor{currentstroke}{rgb}{0.200000,0.800000,0.200000}%
\pgfsetstrokecolor{currentstroke}%
\pgfsetdash{}{0pt}%
\pgfpathmoveto{\pgfqpoint{4.942690in}{2.700025in}}%
\pgfpathcurveto{\pgfqpoint{4.948514in}{2.700025in}}{\pgfqpoint{4.954100in}{2.702339in}}{\pgfqpoint{4.958218in}{2.706457in}}%
\pgfpathcurveto{\pgfqpoint{4.962336in}{2.710575in}}{\pgfqpoint{4.964650in}{2.716161in}}{\pgfqpoint{4.964650in}{2.721985in}}%
\pgfpathcurveto{\pgfqpoint{4.964650in}{2.727809in}}{\pgfqpoint{4.962336in}{2.733395in}}{\pgfqpoint{4.958218in}{2.737513in}}%
\pgfpathcurveto{\pgfqpoint{4.954100in}{2.741632in}}{\pgfqpoint{4.948514in}{2.743945in}}{\pgfqpoint{4.942690in}{2.743945in}}%
\pgfpathcurveto{\pgfqpoint{4.936866in}{2.743945in}}{\pgfqpoint{4.931280in}{2.741632in}}{\pgfqpoint{4.927162in}{2.737513in}}%
\pgfpathcurveto{\pgfqpoint{4.923044in}{2.733395in}}{\pgfqpoint{4.920730in}{2.727809in}}{\pgfqpoint{4.920730in}{2.721985in}}%
\pgfpathcurveto{\pgfqpoint{4.920730in}{2.716161in}}{\pgfqpoint{4.923044in}{2.710575in}}{\pgfqpoint{4.927162in}{2.706457in}}%
\pgfpathcurveto{\pgfqpoint{4.931280in}{2.702339in}}{\pgfqpoint{4.936866in}{2.700025in}}{\pgfqpoint{4.942690in}{2.700025in}}%
\pgfpathlineto{\pgfqpoint{4.942690in}{2.700025in}}%
\pgfpathclose%
\pgfusepath{stroke,fill}%
\end{pgfscope}%
\begin{pgfscope}%
\pgfpathrectangle{\pgfqpoint{1.000000in}{0.979904in}}{\pgfqpoint{6.200000in}{5.960192in}}%
\pgfusepath{clip}%
\pgfsetbuttcap%
\pgfsetroundjoin%
\definecolor{currentfill}{rgb}{0.200000,0.800000,0.200000}%
\pgfsetfillcolor{currentfill}%
\pgfsetlinewidth{1.003750pt}%
\definecolor{currentstroke}{rgb}{0.200000,0.800000,0.200000}%
\pgfsetstrokecolor{currentstroke}%
\pgfsetdash{}{0pt}%
\pgfpathmoveto{\pgfqpoint{5.054932in}{2.652086in}}%
\pgfpathcurveto{\pgfqpoint{5.060756in}{2.652086in}}{\pgfqpoint{5.066342in}{2.654400in}}{\pgfqpoint{5.070461in}{2.658518in}}%
\pgfpathcurveto{\pgfqpoint{5.074579in}{2.662636in}}{\pgfqpoint{5.076893in}{2.668222in}}{\pgfqpoint{5.076893in}{2.674046in}}%
\pgfpathcurveto{\pgfqpoint{5.076893in}{2.679870in}}{\pgfqpoint{5.074579in}{2.685457in}}{\pgfqpoint{5.070461in}{2.689575in}}%
\pgfpathcurveto{\pgfqpoint{5.066342in}{2.693693in}}{\pgfqpoint{5.060756in}{2.696007in}}{\pgfqpoint{5.054932in}{2.696007in}}%
\pgfpathcurveto{\pgfqpoint{5.049108in}{2.696007in}}{\pgfqpoint{5.043522in}{2.693693in}}{\pgfqpoint{5.039404in}{2.689575in}}%
\pgfpathcurveto{\pgfqpoint{5.035286in}{2.685457in}}{\pgfqpoint{5.032972in}{2.679870in}}{\pgfqpoint{5.032972in}{2.674046in}}%
\pgfpathcurveto{\pgfqpoint{5.032972in}{2.668222in}}{\pgfqpoint{5.035286in}{2.662636in}}{\pgfqpoint{5.039404in}{2.658518in}}%
\pgfpathcurveto{\pgfqpoint{5.043522in}{2.654400in}}{\pgfqpoint{5.049108in}{2.652086in}}{\pgfqpoint{5.054932in}{2.652086in}}%
\pgfpathlineto{\pgfqpoint{5.054932in}{2.652086in}}%
\pgfpathclose%
\pgfusepath{stroke,fill}%
\end{pgfscope}%
\begin{pgfscope}%
\pgfpathrectangle{\pgfqpoint{1.000000in}{0.979904in}}{\pgfqpoint{6.200000in}{5.960192in}}%
\pgfusepath{clip}%
\pgfsetbuttcap%
\pgfsetroundjoin%
\definecolor{currentfill}{rgb}{0.200000,0.800000,0.200000}%
\pgfsetfillcolor{currentfill}%
\pgfsetlinewidth{1.003750pt}%
\definecolor{currentstroke}{rgb}{0.200000,0.800000,0.200000}%
\pgfsetstrokecolor{currentstroke}%
\pgfsetdash{}{0pt}%
\pgfpathmoveto{\pgfqpoint{5.166207in}{2.673740in}}%
\pgfpathcurveto{\pgfqpoint{5.172031in}{2.673740in}}{\pgfqpoint{5.177617in}{2.676054in}}{\pgfqpoint{5.181735in}{2.680172in}}%
\pgfpathcurveto{\pgfqpoint{5.185853in}{2.684291in}}{\pgfqpoint{5.188167in}{2.689877in}}{\pgfqpoint{5.188167in}{2.695701in}}%
\pgfpathcurveto{\pgfqpoint{5.188167in}{2.701525in}}{\pgfqpoint{5.185853in}{2.707111in}}{\pgfqpoint{5.181735in}{2.711229in}}%
\pgfpathcurveto{\pgfqpoint{5.177617in}{2.715347in}}{\pgfqpoint{5.172031in}{2.717661in}}{\pgfqpoint{5.166207in}{2.717661in}}%
\pgfpathcurveto{\pgfqpoint{5.160383in}{2.717661in}}{\pgfqpoint{5.154797in}{2.715347in}}{\pgfqpoint{5.150678in}{2.711229in}}%
\pgfpathcurveto{\pgfqpoint{5.146560in}{2.707111in}}{\pgfqpoint{5.144246in}{2.701525in}}{\pgfqpoint{5.144246in}{2.695701in}}%
\pgfpathcurveto{\pgfqpoint{5.144246in}{2.689877in}}{\pgfqpoint{5.146560in}{2.684291in}}{\pgfqpoint{5.150678in}{2.680172in}}%
\pgfpathcurveto{\pgfqpoint{5.154797in}{2.676054in}}{\pgfqpoint{5.160383in}{2.673740in}}{\pgfqpoint{5.166207in}{2.673740in}}%
\pgfpathlineto{\pgfqpoint{5.166207in}{2.673740in}}%
\pgfpathclose%
\pgfusepath{stroke,fill}%
\end{pgfscope}%
\begin{pgfscope}%
\pgfpathrectangle{\pgfqpoint{1.000000in}{0.979904in}}{\pgfqpoint{6.200000in}{5.960192in}}%
\pgfusepath{clip}%
\pgfsetbuttcap%
\pgfsetroundjoin%
\definecolor{currentfill}{rgb}{0.200000,0.800000,0.200000}%
\pgfsetfillcolor{currentfill}%
\pgfsetlinewidth{1.003750pt}%
\definecolor{currentstroke}{rgb}{0.200000,0.800000,0.200000}%
\pgfsetstrokecolor{currentstroke}%
\pgfsetdash{}{0pt}%
\pgfpathmoveto{\pgfqpoint{5.272268in}{2.718245in}}%
\pgfpathcurveto{\pgfqpoint{5.278092in}{2.718245in}}{\pgfqpoint{5.283678in}{2.720559in}}{\pgfqpoint{5.287797in}{2.724677in}}%
\pgfpathcurveto{\pgfqpoint{5.291915in}{2.728796in}}{\pgfqpoint{5.294229in}{2.734382in}}{\pgfqpoint{5.294229in}{2.740206in}}%
\pgfpathcurveto{\pgfqpoint{5.294229in}{2.746030in}}{\pgfqpoint{5.291915in}{2.751616in}}{\pgfqpoint{5.287797in}{2.755734in}}%
\pgfpathcurveto{\pgfqpoint{5.283678in}{2.759852in}}{\pgfqpoint{5.278092in}{2.762166in}}{\pgfqpoint{5.272268in}{2.762166in}}%
\pgfpathcurveto{\pgfqpoint{5.266444in}{2.762166in}}{\pgfqpoint{5.260858in}{2.759852in}}{\pgfqpoint{5.256740in}{2.755734in}}%
\pgfpathcurveto{\pgfqpoint{5.252622in}{2.751616in}}{\pgfqpoint{5.250308in}{2.746030in}}{\pgfqpoint{5.250308in}{2.740206in}}%
\pgfpathcurveto{\pgfqpoint{5.250308in}{2.734382in}}{\pgfqpoint{5.252622in}{2.728796in}}{\pgfqpoint{5.256740in}{2.724677in}}%
\pgfpathcurveto{\pgfqpoint{5.260858in}{2.720559in}}{\pgfqpoint{5.266444in}{2.718245in}}{\pgfqpoint{5.272268in}{2.718245in}}%
\pgfpathlineto{\pgfqpoint{5.272268in}{2.718245in}}%
\pgfpathclose%
\pgfusepath{stroke,fill}%
\end{pgfscope}%
\begin{pgfscope}%
\pgfpathrectangle{\pgfqpoint{1.000000in}{0.979904in}}{\pgfqpoint{6.200000in}{5.960192in}}%
\pgfusepath{clip}%
\pgfsetbuttcap%
\pgfsetroundjoin%
\definecolor{currentfill}{rgb}{0.200000,0.800000,0.200000}%
\pgfsetfillcolor{currentfill}%
\pgfsetlinewidth{1.003750pt}%
\definecolor{currentstroke}{rgb}{0.200000,0.800000,0.200000}%
\pgfsetstrokecolor{currentstroke}%
\pgfsetdash{}{0pt}%
\pgfpathmoveto{\pgfqpoint{5.370644in}{2.780762in}}%
\pgfpathcurveto{\pgfqpoint{5.376468in}{2.780762in}}{\pgfqpoint{5.382054in}{2.783076in}}{\pgfqpoint{5.386172in}{2.787194in}}%
\pgfpathcurveto{\pgfqpoint{5.390290in}{2.791312in}}{\pgfqpoint{5.392604in}{2.796898in}}{\pgfqpoint{5.392604in}{2.802722in}}%
\pgfpathcurveto{\pgfqpoint{5.392604in}{2.808546in}}{\pgfqpoint{5.390290in}{2.814132in}}{\pgfqpoint{5.386172in}{2.818250in}}%
\pgfpathcurveto{\pgfqpoint{5.382054in}{2.822368in}}{\pgfqpoint{5.376468in}{2.824682in}}{\pgfqpoint{5.370644in}{2.824682in}}%
\pgfpathcurveto{\pgfqpoint{5.364820in}{2.824682in}}{\pgfqpoint{5.359234in}{2.822368in}}{\pgfqpoint{5.355115in}{2.818250in}}%
\pgfpathcurveto{\pgfqpoint{5.350997in}{2.814132in}}{\pgfqpoint{5.348683in}{2.808546in}}{\pgfqpoint{5.348683in}{2.802722in}}%
\pgfpathcurveto{\pgfqpoint{5.348683in}{2.796898in}}{\pgfqpoint{5.350997in}{2.791312in}}{\pgfqpoint{5.355115in}{2.787194in}}%
\pgfpathcurveto{\pgfqpoint{5.359234in}{2.783076in}}{\pgfqpoint{5.364820in}{2.780762in}}{\pgfqpoint{5.370644in}{2.780762in}}%
\pgfpathlineto{\pgfqpoint{5.370644in}{2.780762in}}%
\pgfpathclose%
\pgfusepath{stroke,fill}%
\end{pgfscope}%
\begin{pgfscope}%
\pgfpathrectangle{\pgfqpoint{1.000000in}{0.979904in}}{\pgfqpoint{6.200000in}{5.960192in}}%
\pgfusepath{clip}%
\pgfsetbuttcap%
\pgfsetroundjoin%
\definecolor{currentfill}{rgb}{0.200000,0.800000,0.200000}%
\pgfsetfillcolor{currentfill}%
\pgfsetlinewidth{1.003750pt}%
\definecolor{currentstroke}{rgb}{0.200000,0.800000,0.200000}%
\pgfsetstrokecolor{currentstroke}%
\pgfsetdash{}{0pt}%
\pgfpathmoveto{\pgfqpoint{5.464013in}{2.843331in}}%
\pgfpathcurveto{\pgfqpoint{5.469837in}{2.843331in}}{\pgfqpoint{5.475423in}{2.845645in}}{\pgfqpoint{5.479541in}{2.849763in}}%
\pgfpathcurveto{\pgfqpoint{5.483659in}{2.853882in}}{\pgfqpoint{5.485973in}{2.859468in}}{\pgfqpoint{5.485973in}{2.865292in}}%
\pgfpathcurveto{\pgfqpoint{5.485973in}{2.871116in}}{\pgfqpoint{5.483659in}{2.876702in}}{\pgfqpoint{5.479541in}{2.880820in}}%
\pgfpathcurveto{\pgfqpoint{5.475423in}{2.884938in}}{\pgfqpoint{5.469837in}{2.887252in}}{\pgfqpoint{5.464013in}{2.887252in}}%
\pgfpathcurveto{\pgfqpoint{5.458189in}{2.887252in}}{\pgfqpoint{5.452603in}{2.884938in}}{\pgfqpoint{5.448485in}{2.880820in}}%
\pgfpathcurveto{\pgfqpoint{5.444366in}{2.876702in}}{\pgfqpoint{5.442053in}{2.871116in}}{\pgfqpoint{5.442053in}{2.865292in}}%
\pgfpathcurveto{\pgfqpoint{5.442053in}{2.859468in}}{\pgfqpoint{5.444366in}{2.853882in}}{\pgfqpoint{5.448485in}{2.849763in}}%
\pgfpathcurveto{\pgfqpoint{5.452603in}{2.845645in}}{\pgfqpoint{5.458189in}{2.843331in}}{\pgfqpoint{5.464013in}{2.843331in}}%
\pgfpathlineto{\pgfqpoint{5.464013in}{2.843331in}}%
\pgfpathclose%
\pgfusepath{stroke,fill}%
\end{pgfscope}%
\begin{pgfscope}%
\pgfpathrectangle{\pgfqpoint{1.000000in}{0.979904in}}{\pgfqpoint{6.200000in}{5.960192in}}%
\pgfusepath{clip}%
\pgfsetbuttcap%
\pgfsetroundjoin%
\definecolor{currentfill}{rgb}{0.200000,0.800000,0.200000}%
\pgfsetfillcolor{currentfill}%
\pgfsetlinewidth{1.003750pt}%
\definecolor{currentstroke}{rgb}{0.200000,0.800000,0.200000}%
\pgfsetstrokecolor{currentstroke}%
\pgfsetdash{}{0pt}%
\pgfpathmoveto{\pgfqpoint{5.583979in}{2.824901in}}%
\pgfpathcurveto{\pgfqpoint{5.589803in}{2.824901in}}{\pgfqpoint{5.595389in}{2.827215in}}{\pgfqpoint{5.599507in}{2.831333in}}%
\pgfpathcurveto{\pgfqpoint{5.603625in}{2.835451in}}{\pgfqpoint{5.605939in}{2.841038in}}{\pgfqpoint{5.605939in}{2.846862in}}%
\pgfpathcurveto{\pgfqpoint{5.605939in}{2.852685in}}{\pgfqpoint{5.603625in}{2.858272in}}{\pgfqpoint{5.599507in}{2.862390in}}%
\pgfpathcurveto{\pgfqpoint{5.595389in}{2.866508in}}{\pgfqpoint{5.589803in}{2.868822in}}{\pgfqpoint{5.583979in}{2.868822in}}%
\pgfpathcurveto{\pgfqpoint{5.578155in}{2.868822in}}{\pgfqpoint{5.572569in}{2.866508in}}{\pgfqpoint{5.568451in}{2.862390in}}%
\pgfpathcurveto{\pgfqpoint{5.564332in}{2.858272in}}{\pgfqpoint{5.562019in}{2.852685in}}{\pgfqpoint{5.562019in}{2.846862in}}%
\pgfpathcurveto{\pgfqpoint{5.562019in}{2.841038in}}{\pgfqpoint{5.564332in}{2.835451in}}{\pgfqpoint{5.568451in}{2.831333in}}%
\pgfpathcurveto{\pgfqpoint{5.572569in}{2.827215in}}{\pgfqpoint{5.578155in}{2.824901in}}{\pgfqpoint{5.583979in}{2.824901in}}%
\pgfpathlineto{\pgfqpoint{5.583979in}{2.824901in}}%
\pgfpathclose%
\pgfusepath{stroke,fill}%
\end{pgfscope}%
\begin{pgfscope}%
\pgfpathrectangle{\pgfqpoint{1.000000in}{0.979904in}}{\pgfqpoint{6.200000in}{5.960192in}}%
\pgfusepath{clip}%
\pgfsetbuttcap%
\pgfsetroundjoin%
\definecolor{currentfill}{rgb}{0.200000,0.800000,0.200000}%
\pgfsetfillcolor{currentfill}%
\pgfsetlinewidth{1.003750pt}%
\definecolor{currentstroke}{rgb}{0.200000,0.800000,0.200000}%
\pgfsetstrokecolor{currentstroke}%
\pgfsetdash{}{0pt}%
\pgfpathmoveto{\pgfqpoint{5.713360in}{2.804185in}}%
\pgfpathcurveto{\pgfqpoint{5.719184in}{2.804185in}}{\pgfqpoint{5.724770in}{2.806499in}}{\pgfqpoint{5.728888in}{2.810617in}}%
\pgfpathcurveto{\pgfqpoint{5.733007in}{2.814735in}}{\pgfqpoint{5.735320in}{2.820321in}}{\pgfqpoint{5.735320in}{2.826145in}}%
\pgfpathcurveto{\pgfqpoint{5.735320in}{2.831969in}}{\pgfqpoint{5.733007in}{2.837556in}}{\pgfqpoint{5.728888in}{2.841674in}}%
\pgfpathcurveto{\pgfqpoint{5.724770in}{2.845792in}}{\pgfqpoint{5.719184in}{2.848106in}}{\pgfqpoint{5.713360in}{2.848106in}}%
\pgfpathcurveto{\pgfqpoint{5.707536in}{2.848106in}}{\pgfqpoint{5.701950in}{2.845792in}}{\pgfqpoint{5.697832in}{2.841674in}}%
\pgfpathcurveto{\pgfqpoint{5.693714in}{2.837556in}}{\pgfqpoint{5.691400in}{2.831969in}}{\pgfqpoint{5.691400in}{2.826145in}}%
\pgfpathcurveto{\pgfqpoint{5.691400in}{2.820321in}}{\pgfqpoint{5.693714in}{2.814735in}}{\pgfqpoint{5.697832in}{2.810617in}}%
\pgfpathcurveto{\pgfqpoint{5.701950in}{2.806499in}}{\pgfqpoint{5.707536in}{2.804185in}}{\pgfqpoint{5.713360in}{2.804185in}}%
\pgfpathlineto{\pgfqpoint{5.713360in}{2.804185in}}%
\pgfpathclose%
\pgfusepath{stroke,fill}%
\end{pgfscope}%
\begin{pgfscope}%
\pgfpathrectangle{\pgfqpoint{1.000000in}{0.979904in}}{\pgfqpoint{6.200000in}{5.960192in}}%
\pgfusepath{clip}%
\pgfsetbuttcap%
\pgfsetroundjoin%
\definecolor{currentfill}{rgb}{0.200000,0.800000,0.200000}%
\pgfsetfillcolor{currentfill}%
\pgfsetlinewidth{1.003750pt}%
\definecolor{currentstroke}{rgb}{0.200000,0.800000,0.200000}%
\pgfsetstrokecolor{currentstroke}%
\pgfsetdash{}{0pt}%
\pgfpathmoveto{\pgfqpoint{5.798984in}{2.884637in}}%
\pgfpathcurveto{\pgfqpoint{5.804808in}{2.884637in}}{\pgfqpoint{5.810394in}{2.886950in}}{\pgfqpoint{5.814512in}{2.891069in}}%
\pgfpathcurveto{\pgfqpoint{5.818630in}{2.895187in}}{\pgfqpoint{5.820944in}{2.900773in}}{\pgfqpoint{5.820944in}{2.906597in}}%
\pgfpathcurveto{\pgfqpoint{5.820944in}{2.912421in}}{\pgfqpoint{5.818630in}{2.918007in}}{\pgfqpoint{5.814512in}{2.922125in}}%
\pgfpathcurveto{\pgfqpoint{5.810394in}{2.926243in}}{\pgfqpoint{5.804808in}{2.928557in}}{\pgfqpoint{5.798984in}{2.928557in}}%
\pgfpathcurveto{\pgfqpoint{5.793160in}{2.928557in}}{\pgfqpoint{5.787574in}{2.926243in}}{\pgfqpoint{5.783455in}{2.922125in}}%
\pgfpathcurveto{\pgfqpoint{5.779337in}{2.918007in}}{\pgfqpoint{5.777023in}{2.912421in}}{\pgfqpoint{5.777023in}{2.906597in}}%
\pgfpathcurveto{\pgfqpoint{5.777023in}{2.900773in}}{\pgfqpoint{5.779337in}{2.895187in}}{\pgfqpoint{5.783455in}{2.891069in}}%
\pgfpathcurveto{\pgfqpoint{5.787574in}{2.886950in}}{\pgfqpoint{5.793160in}{2.884637in}}{\pgfqpoint{5.798984in}{2.884637in}}%
\pgfpathlineto{\pgfqpoint{5.798984in}{2.884637in}}%
\pgfpathclose%
\pgfusepath{stroke,fill}%
\end{pgfscope}%
\begin{pgfscope}%
\pgfpathrectangle{\pgfqpoint{1.000000in}{0.979904in}}{\pgfqpoint{6.200000in}{5.960192in}}%
\pgfusepath{clip}%
\pgfsetbuttcap%
\pgfsetroundjoin%
\definecolor{currentfill}{rgb}{0.200000,0.800000,0.200000}%
\pgfsetfillcolor{currentfill}%
\pgfsetlinewidth{1.003750pt}%
\definecolor{currentstroke}{rgb}{0.200000,0.800000,0.200000}%
\pgfsetstrokecolor{currentstroke}%
\pgfsetdash{}{0pt}%
\pgfpathmoveto{\pgfqpoint{5.854591in}{3.006132in}}%
\pgfpathcurveto{\pgfqpoint{5.860415in}{3.006132in}}{\pgfqpoint{5.866001in}{3.008446in}}{\pgfqpoint{5.870119in}{3.012564in}}%
\pgfpathcurveto{\pgfqpoint{5.874237in}{3.016683in}}{\pgfqpoint{5.876551in}{3.022269in}}{\pgfqpoint{5.876551in}{3.028093in}}%
\pgfpathcurveto{\pgfqpoint{5.876551in}{3.033917in}}{\pgfqpoint{5.874237in}{3.039503in}}{\pgfqpoint{5.870119in}{3.043621in}}%
\pgfpathcurveto{\pgfqpoint{5.866001in}{3.047739in}}{\pgfqpoint{5.860415in}{3.050053in}}{\pgfqpoint{5.854591in}{3.050053in}}%
\pgfpathcurveto{\pgfqpoint{5.848767in}{3.050053in}}{\pgfqpoint{5.843181in}{3.047739in}}{\pgfqpoint{5.839063in}{3.043621in}}%
\pgfpathcurveto{\pgfqpoint{5.834945in}{3.039503in}}{\pgfqpoint{5.832631in}{3.033917in}}{\pgfqpoint{5.832631in}{3.028093in}}%
\pgfpathcurveto{\pgfqpoint{5.832631in}{3.022269in}}{\pgfqpoint{5.834945in}{3.016683in}}{\pgfqpoint{5.839063in}{3.012564in}}%
\pgfpathcurveto{\pgfqpoint{5.843181in}{3.008446in}}{\pgfqpoint{5.848767in}{3.006132in}}{\pgfqpoint{5.854591in}{3.006132in}}%
\pgfpathlineto{\pgfqpoint{5.854591in}{3.006132in}}%
\pgfpathclose%
\pgfusepath{stroke,fill}%
\end{pgfscope}%
\begin{pgfscope}%
\pgfpathrectangle{\pgfqpoint{1.000000in}{0.979904in}}{\pgfqpoint{6.200000in}{5.960192in}}%
\pgfusepath{clip}%
\pgfsetbuttcap%
\pgfsetroundjoin%
\definecolor{currentfill}{rgb}{0.200000,0.800000,0.200000}%
\pgfsetfillcolor{currentfill}%
\pgfsetlinewidth{1.003750pt}%
\definecolor{currentstroke}{rgb}{0.200000,0.800000,0.200000}%
\pgfsetstrokecolor{currentstroke}%
\pgfsetdash{}{0pt}%
\pgfpathmoveto{\pgfqpoint{5.986885in}{3.003772in}}%
\pgfpathcurveto{\pgfqpoint{5.992709in}{3.003772in}}{\pgfqpoint{5.998295in}{3.006086in}}{\pgfqpoint{6.002413in}{3.010204in}}%
\pgfpathcurveto{\pgfqpoint{6.006531in}{3.014322in}}{\pgfqpoint{6.008845in}{3.019908in}}{\pgfqpoint{6.008845in}{3.025732in}}%
\pgfpathcurveto{\pgfqpoint{6.008845in}{3.031556in}}{\pgfqpoint{6.006531in}{3.037142in}}{\pgfqpoint{6.002413in}{3.041261in}}%
\pgfpathcurveto{\pgfqpoint{5.998295in}{3.045379in}}{\pgfqpoint{5.992709in}{3.047693in}}{\pgfqpoint{5.986885in}{3.047693in}}%
\pgfpathcurveto{\pgfqpoint{5.981061in}{3.047693in}}{\pgfqpoint{5.975474in}{3.045379in}}{\pgfqpoint{5.971356in}{3.041261in}}%
\pgfpathcurveto{\pgfqpoint{5.967238in}{3.037142in}}{\pgfqpoint{5.964924in}{3.031556in}}{\pgfqpoint{5.964924in}{3.025732in}}%
\pgfpathcurveto{\pgfqpoint{5.964924in}{3.019908in}}{\pgfqpoint{5.967238in}{3.014322in}}{\pgfqpoint{5.971356in}{3.010204in}}%
\pgfpathcurveto{\pgfqpoint{5.975474in}{3.006086in}}{\pgfqpoint{5.981061in}{3.003772in}}{\pgfqpoint{5.986885in}{3.003772in}}%
\pgfpathlineto{\pgfqpoint{5.986885in}{3.003772in}}%
\pgfpathclose%
\pgfusepath{stroke,fill}%
\end{pgfscope}%
\begin{pgfscope}%
\pgfpathrectangle{\pgfqpoint{1.000000in}{0.979904in}}{\pgfqpoint{6.200000in}{5.960192in}}%
\pgfusepath{clip}%
\pgfsetbuttcap%
\pgfsetroundjoin%
\definecolor{currentfill}{rgb}{0.200000,0.800000,0.200000}%
\pgfsetfillcolor{currentfill}%
\pgfsetlinewidth{1.003750pt}%
\definecolor{currentstroke}{rgb}{0.200000,0.800000,0.200000}%
\pgfsetstrokecolor{currentstroke}%
\pgfsetdash{}{0pt}%
\pgfpathmoveto{\pgfqpoint{6.082490in}{3.062395in}}%
\pgfpathcurveto{\pgfqpoint{6.088314in}{3.062395in}}{\pgfqpoint{6.093900in}{3.064709in}}{\pgfqpoint{6.098018in}{3.068827in}}%
\pgfpathcurveto{\pgfqpoint{6.102137in}{3.072945in}}{\pgfqpoint{6.104450in}{3.078532in}}{\pgfqpoint{6.104450in}{3.084355in}}%
\pgfpathcurveto{\pgfqpoint{6.104450in}{3.090179in}}{\pgfqpoint{6.102137in}{3.095766in}}{\pgfqpoint{6.098018in}{3.099884in}}%
\pgfpathcurveto{\pgfqpoint{6.093900in}{3.104002in}}{\pgfqpoint{6.088314in}{3.106316in}}{\pgfqpoint{6.082490in}{3.106316in}}%
\pgfpathcurveto{\pgfqpoint{6.076666in}{3.106316in}}{\pgfqpoint{6.071080in}{3.104002in}}{\pgfqpoint{6.066962in}{3.099884in}}%
\pgfpathcurveto{\pgfqpoint{6.062844in}{3.095766in}}{\pgfqpoint{6.060530in}{3.090179in}}{\pgfqpoint{6.060530in}{3.084355in}}%
\pgfpathcurveto{\pgfqpoint{6.060530in}{3.078532in}}{\pgfqpoint{6.062844in}{3.072945in}}{\pgfqpoint{6.066962in}{3.068827in}}%
\pgfpathcurveto{\pgfqpoint{6.071080in}{3.064709in}}{\pgfqpoint{6.076666in}{3.062395in}}{\pgfqpoint{6.082490in}{3.062395in}}%
\pgfpathlineto{\pgfqpoint{6.082490in}{3.062395in}}%
\pgfpathclose%
\pgfusepath{stroke,fill}%
\end{pgfscope}%
\begin{pgfscope}%
\pgfpathrectangle{\pgfqpoint{1.000000in}{0.979904in}}{\pgfqpoint{6.200000in}{5.960192in}}%
\pgfusepath{clip}%
\pgfsetbuttcap%
\pgfsetroundjoin%
\definecolor{currentfill}{rgb}{0.200000,0.800000,0.200000}%
\pgfsetfillcolor{currentfill}%
\pgfsetlinewidth{1.003750pt}%
\definecolor{currentstroke}{rgb}{0.200000,0.800000,0.200000}%
\pgfsetstrokecolor{currentstroke}%
\pgfsetdash{}{0pt}%
\pgfpathmoveto{\pgfqpoint{6.121398in}{3.185166in}}%
\pgfpathcurveto{\pgfqpoint{6.127222in}{3.185166in}}{\pgfqpoint{6.132809in}{3.187480in}}{\pgfqpoint{6.136927in}{3.191598in}}%
\pgfpathcurveto{\pgfqpoint{6.141045in}{3.195716in}}{\pgfqpoint{6.143359in}{3.201302in}}{\pgfqpoint{6.143359in}{3.207126in}}%
\pgfpathcurveto{\pgfqpoint{6.143359in}{3.212950in}}{\pgfqpoint{6.141045in}{3.218536in}}{\pgfqpoint{6.136927in}{3.222654in}}%
\pgfpathcurveto{\pgfqpoint{6.132809in}{3.226772in}}{\pgfqpoint{6.127222in}{3.229086in}}{\pgfqpoint{6.121398in}{3.229086in}}%
\pgfpathcurveto{\pgfqpoint{6.115575in}{3.229086in}}{\pgfqpoint{6.109988in}{3.226772in}}{\pgfqpoint{6.105870in}{3.222654in}}%
\pgfpathcurveto{\pgfqpoint{6.101752in}{3.218536in}}{\pgfqpoint{6.099438in}{3.212950in}}{\pgfqpoint{6.099438in}{3.207126in}}%
\pgfpathcurveto{\pgfqpoint{6.099438in}{3.201302in}}{\pgfqpoint{6.101752in}{3.195716in}}{\pgfqpoint{6.105870in}{3.191598in}}%
\pgfpathcurveto{\pgfqpoint{6.109988in}{3.187480in}}{\pgfqpoint{6.115575in}{3.185166in}}{\pgfqpoint{6.121398in}{3.185166in}}%
\pgfpathlineto{\pgfqpoint{6.121398in}{3.185166in}}%
\pgfpathclose%
\pgfusepath{stroke,fill}%
\end{pgfscope}%
\begin{pgfscope}%
\pgfpathrectangle{\pgfqpoint{1.000000in}{0.979904in}}{\pgfqpoint{6.200000in}{5.960192in}}%
\pgfusepath{clip}%
\pgfsetbuttcap%
\pgfsetroundjoin%
\definecolor{currentfill}{rgb}{0.200000,0.800000,0.200000}%
\pgfsetfillcolor{currentfill}%
\pgfsetlinewidth{1.003750pt}%
\definecolor{currentstroke}{rgb}{0.200000,0.800000,0.200000}%
\pgfsetstrokecolor{currentstroke}%
\pgfsetdash{}{0pt}%
\pgfpathmoveto{\pgfqpoint{6.237974in}{3.222702in}}%
\pgfpathcurveto{\pgfqpoint{6.243797in}{3.222702in}}{\pgfqpoint{6.249384in}{3.225015in}}{\pgfqpoint{6.253502in}{3.229134in}}%
\pgfpathcurveto{\pgfqpoint{6.257620in}{3.233252in}}{\pgfqpoint{6.259934in}{3.238838in}}{\pgfqpoint{6.259934in}{3.244662in}}%
\pgfpathcurveto{\pgfqpoint{6.259934in}{3.250486in}}{\pgfqpoint{6.257620in}{3.256072in}}{\pgfqpoint{6.253502in}{3.260190in}}%
\pgfpathcurveto{\pgfqpoint{6.249384in}{3.264308in}}{\pgfqpoint{6.243797in}{3.266622in}}{\pgfqpoint{6.237974in}{3.266622in}}%
\pgfpathcurveto{\pgfqpoint{6.232150in}{3.266622in}}{\pgfqpoint{6.226563in}{3.264308in}}{\pgfqpoint{6.222445in}{3.260190in}}%
\pgfpathcurveto{\pgfqpoint{6.218327in}{3.256072in}}{\pgfqpoint{6.216013in}{3.250486in}}{\pgfqpoint{6.216013in}{3.244662in}}%
\pgfpathcurveto{\pgfqpoint{6.216013in}{3.238838in}}{\pgfqpoint{6.218327in}{3.233252in}}{\pgfqpoint{6.222445in}{3.229134in}}%
\pgfpathcurveto{\pgfqpoint{6.226563in}{3.225015in}}{\pgfqpoint{6.232150in}{3.222702in}}{\pgfqpoint{6.237974in}{3.222702in}}%
\pgfpathlineto{\pgfqpoint{6.237974in}{3.222702in}}%
\pgfpathclose%
\pgfusepath{stroke,fill}%
\end{pgfscope}%
\begin{pgfscope}%
\pgfpathrectangle{\pgfqpoint{1.000000in}{0.979904in}}{\pgfqpoint{6.200000in}{5.960192in}}%
\pgfusepath{clip}%
\pgfsetbuttcap%
\pgfsetroundjoin%
\definecolor{currentfill}{rgb}{0.200000,0.800000,0.200000}%
\pgfsetfillcolor{currentfill}%
\pgfsetlinewidth{1.003750pt}%
\definecolor{currentstroke}{rgb}{0.200000,0.800000,0.200000}%
\pgfsetstrokecolor{currentstroke}%
\pgfsetdash{}{0pt}%
\pgfpathmoveto{\pgfqpoint{6.308209in}{3.308802in}}%
\pgfpathcurveto{\pgfqpoint{6.314033in}{3.308802in}}{\pgfqpoint{6.319619in}{3.311116in}}{\pgfqpoint{6.323737in}{3.315234in}}%
\pgfpathcurveto{\pgfqpoint{6.327855in}{3.319352in}}{\pgfqpoint{6.330169in}{3.324939in}}{\pgfqpoint{6.330169in}{3.330762in}}%
\pgfpathcurveto{\pgfqpoint{6.330169in}{3.336586in}}{\pgfqpoint{6.327855in}{3.342173in}}{\pgfqpoint{6.323737in}{3.346291in}}%
\pgfpathcurveto{\pgfqpoint{6.319619in}{3.350409in}}{\pgfqpoint{6.314033in}{3.352723in}}{\pgfqpoint{6.308209in}{3.352723in}}%
\pgfpathcurveto{\pgfqpoint{6.302385in}{3.352723in}}{\pgfqpoint{6.296799in}{3.350409in}}{\pgfqpoint{6.292681in}{3.346291in}}%
\pgfpathcurveto{\pgfqpoint{6.288563in}{3.342173in}}{\pgfqpoint{6.286249in}{3.336586in}}{\pgfqpoint{6.286249in}{3.330762in}}%
\pgfpathcurveto{\pgfqpoint{6.286249in}{3.324939in}}{\pgfqpoint{6.288563in}{3.319352in}}{\pgfqpoint{6.292681in}{3.315234in}}%
\pgfpathcurveto{\pgfqpoint{6.296799in}{3.311116in}}{\pgfqpoint{6.302385in}{3.308802in}}{\pgfqpoint{6.308209in}{3.308802in}}%
\pgfpathlineto{\pgfqpoint{6.308209in}{3.308802in}}%
\pgfpathclose%
\pgfusepath{stroke,fill}%
\end{pgfscope}%
\begin{pgfscope}%
\pgfpathrectangle{\pgfqpoint{1.000000in}{0.979904in}}{\pgfqpoint{6.200000in}{5.960192in}}%
\pgfusepath{clip}%
\pgfsetbuttcap%
\pgfsetroundjoin%
\definecolor{currentfill}{rgb}{0.200000,0.800000,0.200000}%
\pgfsetfillcolor{currentfill}%
\pgfsetlinewidth{1.003750pt}%
\definecolor{currentstroke}{rgb}{0.200000,0.800000,0.200000}%
\pgfsetstrokecolor{currentstroke}%
\pgfsetdash{}{0pt}%
\pgfpathmoveto{\pgfqpoint{6.325404in}{3.433713in}}%
\pgfpathcurveto{\pgfqpoint{6.331228in}{3.433713in}}{\pgfqpoint{6.336814in}{3.436027in}}{\pgfqpoint{6.340932in}{3.440145in}}%
\pgfpathcurveto{\pgfqpoint{6.345050in}{3.444263in}}{\pgfqpoint{6.347364in}{3.449849in}}{\pgfqpoint{6.347364in}{3.455673in}}%
\pgfpathcurveto{\pgfqpoint{6.347364in}{3.461497in}}{\pgfqpoint{6.345050in}{3.467083in}}{\pgfqpoint{6.340932in}{3.471201in}}%
\pgfpathcurveto{\pgfqpoint{6.336814in}{3.475320in}}{\pgfqpoint{6.331228in}{3.477633in}}{\pgfqpoint{6.325404in}{3.477633in}}%
\pgfpathcurveto{\pgfqpoint{6.319580in}{3.477633in}}{\pgfqpoint{6.313994in}{3.475320in}}{\pgfqpoint{6.309876in}{3.471201in}}%
\pgfpathcurveto{\pgfqpoint{6.305758in}{3.467083in}}{\pgfqpoint{6.303444in}{3.461497in}}{\pgfqpoint{6.303444in}{3.455673in}}%
\pgfpathcurveto{\pgfqpoint{6.303444in}{3.449849in}}{\pgfqpoint{6.305758in}{3.444263in}}{\pgfqpoint{6.309876in}{3.440145in}}%
\pgfpathcurveto{\pgfqpoint{6.313994in}{3.436027in}}{\pgfqpoint{6.319580in}{3.433713in}}{\pgfqpoint{6.325404in}{3.433713in}}%
\pgfpathlineto{\pgfqpoint{6.325404in}{3.433713in}}%
\pgfpathclose%
\pgfusepath{stroke,fill}%
\end{pgfscope}%
\begin{pgfscope}%
\pgfpathrectangle{\pgfqpoint{1.000000in}{0.979904in}}{\pgfqpoint{6.200000in}{5.960192in}}%
\pgfusepath{clip}%
\pgfsetbuttcap%
\pgfsetroundjoin%
\definecolor{currentfill}{rgb}{0.200000,0.800000,0.200000}%
\pgfsetfillcolor{currentfill}%
\pgfsetlinewidth{1.003750pt}%
\definecolor{currentstroke}{rgb}{0.200000,0.800000,0.200000}%
\pgfsetstrokecolor{currentstroke}%
\pgfsetdash{}{0pt}%
\pgfpathmoveto{\pgfqpoint{6.452947in}{3.478628in}}%
\pgfpathcurveto{\pgfqpoint{6.458771in}{3.478628in}}{\pgfqpoint{6.464358in}{3.480942in}}{\pgfqpoint{6.468476in}{3.485060in}}%
\pgfpathcurveto{\pgfqpoint{6.472594in}{3.489178in}}{\pgfqpoint{6.474908in}{3.494764in}}{\pgfqpoint{6.474908in}{3.500588in}}%
\pgfpathcurveto{\pgfqpoint{6.474908in}{3.506412in}}{\pgfqpoint{6.472594in}{3.511998in}}{\pgfqpoint{6.468476in}{3.516116in}}%
\pgfpathcurveto{\pgfqpoint{6.464358in}{3.520234in}}{\pgfqpoint{6.458771in}{3.522548in}}{\pgfqpoint{6.452947in}{3.522548in}}%
\pgfpathcurveto{\pgfqpoint{6.447124in}{3.522548in}}{\pgfqpoint{6.441537in}{3.520234in}}{\pgfqpoint{6.437419in}{3.516116in}}%
\pgfpathcurveto{\pgfqpoint{6.433301in}{3.511998in}}{\pgfqpoint{6.430987in}{3.506412in}}{\pgfqpoint{6.430987in}{3.500588in}}%
\pgfpathcurveto{\pgfqpoint{6.430987in}{3.494764in}}{\pgfqpoint{6.433301in}{3.489178in}}{\pgfqpoint{6.437419in}{3.485060in}}%
\pgfpathcurveto{\pgfqpoint{6.441537in}{3.480942in}}{\pgfqpoint{6.447124in}{3.478628in}}{\pgfqpoint{6.452947in}{3.478628in}}%
\pgfpathlineto{\pgfqpoint{6.452947in}{3.478628in}}%
\pgfpathclose%
\pgfusepath{stroke,fill}%
\end{pgfscope}%
\begin{pgfscope}%
\pgfpathrectangle{\pgfqpoint{1.000000in}{0.979904in}}{\pgfqpoint{6.200000in}{5.960192in}}%
\pgfusepath{clip}%
\pgfsetbuttcap%
\pgfsetroundjoin%
\definecolor{currentfill}{rgb}{0.200000,0.800000,0.200000}%
\pgfsetfillcolor{currentfill}%
\pgfsetlinewidth{1.003750pt}%
\definecolor{currentstroke}{rgb}{0.200000,0.800000,0.200000}%
\pgfsetstrokecolor{currentstroke}%
\pgfsetdash{}{0pt}%
\pgfpathmoveto{\pgfqpoint{6.519812in}{3.569354in}}%
\pgfpathcurveto{\pgfqpoint{6.525636in}{3.569354in}}{\pgfqpoint{6.531222in}{3.571668in}}{\pgfqpoint{6.535340in}{3.575786in}}%
\pgfpathcurveto{\pgfqpoint{6.539459in}{3.579904in}}{\pgfqpoint{6.541772in}{3.585490in}}{\pgfqpoint{6.541772in}{3.591314in}}%
\pgfpathcurveto{\pgfqpoint{6.541772in}{3.597138in}}{\pgfqpoint{6.539459in}{3.602724in}}{\pgfqpoint{6.535340in}{3.606842in}}%
\pgfpathcurveto{\pgfqpoint{6.531222in}{3.610960in}}{\pgfqpoint{6.525636in}{3.613274in}}{\pgfqpoint{6.519812in}{3.613274in}}%
\pgfpathcurveto{\pgfqpoint{6.513988in}{3.613274in}}{\pgfqpoint{6.508402in}{3.610960in}}{\pgfqpoint{6.504284in}{3.606842in}}%
\pgfpathcurveto{\pgfqpoint{6.500166in}{3.602724in}}{\pgfqpoint{6.497852in}{3.597138in}}{\pgfqpoint{6.497852in}{3.591314in}}%
\pgfpathcurveto{\pgfqpoint{6.497852in}{3.585490in}}{\pgfqpoint{6.500166in}{3.579904in}}{\pgfqpoint{6.504284in}{3.575786in}}%
\pgfpathcurveto{\pgfqpoint{6.508402in}{3.571668in}}{\pgfqpoint{6.513988in}{3.569354in}}{\pgfqpoint{6.519812in}{3.569354in}}%
\pgfpathlineto{\pgfqpoint{6.519812in}{3.569354in}}%
\pgfpathclose%
\pgfusepath{stroke,fill}%
\end{pgfscope}%
\begin{pgfscope}%
\pgfpathrectangle{\pgfqpoint{1.000000in}{0.979904in}}{\pgfqpoint{6.200000in}{5.960192in}}%
\pgfusepath{clip}%
\pgfsetbuttcap%
\pgfsetroundjoin%
\definecolor{currentfill}{rgb}{0.200000,0.800000,0.200000}%
\pgfsetfillcolor{currentfill}%
\pgfsetlinewidth{1.003750pt}%
\definecolor{currentstroke}{rgb}{0.200000,0.800000,0.200000}%
\pgfsetstrokecolor{currentstroke}%
\pgfsetdash{}{0pt}%
\pgfpathmoveto{\pgfqpoint{6.596396in}{3.657465in}}%
\pgfpathcurveto{\pgfqpoint{6.602220in}{3.657465in}}{\pgfqpoint{6.607806in}{3.659779in}}{\pgfqpoint{6.611924in}{3.663897in}}%
\pgfpathcurveto{\pgfqpoint{6.616043in}{3.668015in}}{\pgfqpoint{6.618356in}{3.673601in}}{\pgfqpoint{6.618356in}{3.679425in}}%
\pgfpathcurveto{\pgfqpoint{6.618356in}{3.685249in}}{\pgfqpoint{6.616043in}{3.690835in}}{\pgfqpoint{6.611924in}{3.694953in}}%
\pgfpathcurveto{\pgfqpoint{6.607806in}{3.699071in}}{\pgfqpoint{6.602220in}{3.701385in}}{\pgfqpoint{6.596396in}{3.701385in}}%
\pgfpathcurveto{\pgfqpoint{6.590572in}{3.701385in}}{\pgfqpoint{6.584986in}{3.699071in}}{\pgfqpoint{6.580868in}{3.694953in}}%
\pgfpathcurveto{\pgfqpoint{6.576750in}{3.690835in}}{\pgfqpoint{6.574436in}{3.685249in}}{\pgfqpoint{6.574436in}{3.679425in}}%
\pgfpathcurveto{\pgfqpoint{6.574436in}{3.673601in}}{\pgfqpoint{6.576750in}{3.668015in}}{\pgfqpoint{6.580868in}{3.663897in}}%
\pgfpathcurveto{\pgfqpoint{6.584986in}{3.659779in}}{\pgfqpoint{6.590572in}{3.657465in}}{\pgfqpoint{6.596396in}{3.657465in}}%
\pgfpathlineto{\pgfqpoint{6.596396in}{3.657465in}}%
\pgfpathclose%
\pgfusepath{stroke,fill}%
\end{pgfscope}%
\begin{pgfscope}%
\pgfpathrectangle{\pgfqpoint{1.000000in}{0.979904in}}{\pgfqpoint{6.200000in}{5.960192in}}%
\pgfusepath{clip}%
\pgfsetbuttcap%
\pgfsetroundjoin%
\definecolor{currentfill}{rgb}{0.200000,0.800000,0.200000}%
\pgfsetfillcolor{currentfill}%
\pgfsetlinewidth{1.003750pt}%
\definecolor{currentstroke}{rgb}{0.200000,0.800000,0.200000}%
\pgfsetstrokecolor{currentstroke}%
\pgfsetdash{}{0pt}%
\pgfpathmoveto{\pgfqpoint{6.640777in}{3.762742in}}%
\pgfpathcurveto{\pgfqpoint{6.646601in}{3.762742in}}{\pgfqpoint{6.652187in}{3.765056in}}{\pgfqpoint{6.656305in}{3.769174in}}%
\pgfpathcurveto{\pgfqpoint{6.660423in}{3.773292in}}{\pgfqpoint{6.662737in}{3.778878in}}{\pgfqpoint{6.662737in}{3.784702in}}%
\pgfpathcurveto{\pgfqpoint{6.662737in}{3.790526in}}{\pgfqpoint{6.660423in}{3.796112in}}{\pgfqpoint{6.656305in}{3.800230in}}%
\pgfpathcurveto{\pgfqpoint{6.652187in}{3.804348in}}{\pgfqpoint{6.646601in}{3.806662in}}{\pgfqpoint{6.640777in}{3.806662in}}%
\pgfpathcurveto{\pgfqpoint{6.634953in}{3.806662in}}{\pgfqpoint{6.629367in}{3.804348in}}{\pgfqpoint{6.625249in}{3.800230in}}%
\pgfpathcurveto{\pgfqpoint{6.621131in}{3.796112in}}{\pgfqpoint{6.618817in}{3.790526in}}{\pgfqpoint{6.618817in}{3.784702in}}%
\pgfpathcurveto{\pgfqpoint{6.618817in}{3.778878in}}{\pgfqpoint{6.621131in}{3.773292in}}{\pgfqpoint{6.625249in}{3.769174in}}%
\pgfpathcurveto{\pgfqpoint{6.629367in}{3.765056in}}{\pgfqpoint{6.634953in}{3.762742in}}{\pgfqpoint{6.640777in}{3.762742in}}%
\pgfpathlineto{\pgfqpoint{6.640777in}{3.762742in}}%
\pgfpathclose%
\pgfusepath{stroke,fill}%
\end{pgfscope}%
\begin{pgfscope}%
\pgfpathrectangle{\pgfqpoint{1.000000in}{0.979904in}}{\pgfqpoint{6.200000in}{5.960192in}}%
\pgfusepath{clip}%
\pgfsetbuttcap%
\pgfsetroundjoin%
\definecolor{currentfill}{rgb}{0.200000,0.800000,0.200000}%
\pgfsetfillcolor{currentfill}%
\pgfsetlinewidth{1.003750pt}%
\definecolor{currentstroke}{rgb}{0.200000,0.800000,0.200000}%
\pgfsetstrokecolor{currentstroke}%
\pgfsetdash{}{0pt}%
\pgfpathmoveto{\pgfqpoint{6.557268in}{3.910299in}}%
\pgfpathcurveto{\pgfqpoint{6.563092in}{3.910299in}}{\pgfqpoint{6.568678in}{3.912613in}}{\pgfqpoint{6.572796in}{3.916731in}}%
\pgfpathcurveto{\pgfqpoint{6.576914in}{3.920849in}}{\pgfqpoint{6.579228in}{3.926436in}}{\pgfqpoint{6.579228in}{3.932259in}}%
\pgfpathcurveto{\pgfqpoint{6.579228in}{3.938083in}}{\pgfqpoint{6.576914in}{3.943670in}}{\pgfqpoint{6.572796in}{3.947788in}}%
\pgfpathcurveto{\pgfqpoint{6.568678in}{3.951906in}}{\pgfqpoint{6.563092in}{3.954220in}}{\pgfqpoint{6.557268in}{3.954220in}}%
\pgfpathcurveto{\pgfqpoint{6.551444in}{3.954220in}}{\pgfqpoint{6.545857in}{3.951906in}}{\pgfqpoint{6.541739in}{3.947788in}}%
\pgfpathcurveto{\pgfqpoint{6.537621in}{3.943670in}}{\pgfqpoint{6.535307in}{3.938083in}}{\pgfqpoint{6.535307in}{3.932259in}}%
\pgfpathcurveto{\pgfqpoint{6.535307in}{3.926436in}}{\pgfqpoint{6.537621in}{3.920849in}}{\pgfqpoint{6.541739in}{3.916731in}}%
\pgfpathcurveto{\pgfqpoint{6.545857in}{3.912613in}}{\pgfqpoint{6.551444in}{3.910299in}}{\pgfqpoint{6.557268in}{3.910299in}}%
\pgfpathlineto{\pgfqpoint{6.557268in}{3.910299in}}%
\pgfpathclose%
\pgfusepath{stroke,fill}%
\end{pgfscope}%
\begin{pgfscope}%
\pgfpathrectangle{\pgfqpoint{1.000000in}{0.979904in}}{\pgfqpoint{6.200000in}{5.960192in}}%
\pgfusepath{clip}%
\pgfsetbuttcap%
\pgfsetroundjoin%
\definecolor{currentfill}{rgb}{0.200000,0.800000,0.200000}%
\pgfsetfillcolor{currentfill}%
\pgfsetlinewidth{1.003750pt}%
\definecolor{currentstroke}{rgb}{0.200000,0.800000,0.200000}%
\pgfsetstrokecolor{currentstroke}%
\pgfsetdash{}{0pt}%
\pgfpathmoveto{\pgfqpoint{6.609721in}{4.006148in}}%
\pgfpathcurveto{\pgfqpoint{6.615545in}{4.006148in}}{\pgfqpoint{6.621131in}{4.008462in}}{\pgfqpoint{6.625249in}{4.012580in}}%
\pgfpathcurveto{\pgfqpoint{6.629367in}{4.016698in}}{\pgfqpoint{6.631681in}{4.022284in}}{\pgfqpoint{6.631681in}{4.028108in}}%
\pgfpathcurveto{\pgfqpoint{6.631681in}{4.033932in}}{\pgfqpoint{6.629367in}{4.039518in}}{\pgfqpoint{6.625249in}{4.043636in}}%
\pgfpathcurveto{\pgfqpoint{6.621131in}{4.047755in}}{\pgfqpoint{6.615545in}{4.050068in}}{\pgfqpoint{6.609721in}{4.050068in}}%
\pgfpathcurveto{\pgfqpoint{6.603897in}{4.050068in}}{\pgfqpoint{6.598311in}{4.047755in}}{\pgfqpoint{6.594193in}{4.043636in}}%
\pgfpathcurveto{\pgfqpoint{6.590075in}{4.039518in}}{\pgfqpoint{6.587761in}{4.033932in}}{\pgfqpoint{6.587761in}{4.028108in}}%
\pgfpathcurveto{\pgfqpoint{6.587761in}{4.022284in}}{\pgfqpoint{6.590075in}{4.016698in}}{\pgfqpoint{6.594193in}{4.012580in}}%
\pgfpathcurveto{\pgfqpoint{6.598311in}{4.008462in}}{\pgfqpoint{6.603897in}{4.006148in}}{\pgfqpoint{6.609721in}{4.006148in}}%
\pgfpathlineto{\pgfqpoint{6.609721in}{4.006148in}}%
\pgfpathclose%
\pgfusepath{stroke,fill}%
\end{pgfscope}%
\begin{pgfscope}%
\pgfpathrectangle{\pgfqpoint{1.000000in}{0.979904in}}{\pgfqpoint{6.200000in}{5.960192in}}%
\pgfusepath{clip}%
\pgfsetbuttcap%
\pgfsetroundjoin%
\definecolor{currentfill}{rgb}{0.200000,0.800000,0.200000}%
\pgfsetfillcolor{currentfill}%
\pgfsetlinewidth{1.003750pt}%
\definecolor{currentstroke}{rgb}{0.200000,0.800000,0.200000}%
\pgfsetstrokecolor{currentstroke}%
\pgfsetdash{}{0pt}%
\pgfpathmoveto{\pgfqpoint{6.709092in}{4.096403in}}%
\pgfpathcurveto{\pgfqpoint{6.714916in}{4.096403in}}{\pgfqpoint{6.720502in}{4.098717in}}{\pgfqpoint{6.724620in}{4.102835in}}%
\pgfpathcurveto{\pgfqpoint{6.728738in}{4.106953in}}{\pgfqpoint{6.731052in}{4.112539in}}{\pgfqpoint{6.731052in}{4.118363in}}%
\pgfpathcurveto{\pgfqpoint{6.731052in}{4.124187in}}{\pgfqpoint{6.728738in}{4.129774in}}{\pgfqpoint{6.724620in}{4.133892in}}%
\pgfpathcurveto{\pgfqpoint{6.720502in}{4.138010in}}{\pgfqpoint{6.714916in}{4.140324in}}{\pgfqpoint{6.709092in}{4.140324in}}%
\pgfpathcurveto{\pgfqpoint{6.703268in}{4.140324in}}{\pgfqpoint{6.697682in}{4.138010in}}{\pgfqpoint{6.693564in}{4.133892in}}%
\pgfpathcurveto{\pgfqpoint{6.689445in}{4.129774in}}{\pgfqpoint{6.687132in}{4.124187in}}{\pgfqpoint{6.687132in}{4.118363in}}%
\pgfpathcurveto{\pgfqpoint{6.687132in}{4.112539in}}{\pgfqpoint{6.689445in}{4.106953in}}{\pgfqpoint{6.693564in}{4.102835in}}%
\pgfpathcurveto{\pgfqpoint{6.697682in}{4.098717in}}{\pgfqpoint{6.703268in}{4.096403in}}{\pgfqpoint{6.709092in}{4.096403in}}%
\pgfpathlineto{\pgfqpoint{6.709092in}{4.096403in}}%
\pgfpathclose%
\pgfusepath{stroke,fill}%
\end{pgfscope}%
\begin{pgfscope}%
\pgfpathrectangle{\pgfqpoint{1.000000in}{0.979904in}}{\pgfqpoint{6.200000in}{5.960192in}}%
\pgfusepath{clip}%
\pgfsetbuttcap%
\pgfsetroundjoin%
\definecolor{currentfill}{rgb}{0.200000,0.800000,0.200000}%
\pgfsetfillcolor{currentfill}%
\pgfsetlinewidth{1.003750pt}%
\definecolor{currentstroke}{rgb}{0.200000,0.800000,0.200000}%
\pgfsetstrokecolor{currentstroke}%
\pgfsetdash{}{0pt}%
\pgfpathmoveto{\pgfqpoint{6.726782in}{4.207376in}}%
\pgfpathcurveto{\pgfqpoint{6.732606in}{4.207376in}}{\pgfqpoint{6.738192in}{4.209690in}}{\pgfqpoint{6.742310in}{4.213808in}}%
\pgfpathcurveto{\pgfqpoint{6.746428in}{4.217926in}}{\pgfqpoint{6.748742in}{4.223512in}}{\pgfqpoint{6.748742in}{4.229336in}}%
\pgfpathcurveto{\pgfqpoint{6.748742in}{4.235160in}}{\pgfqpoint{6.746428in}{4.240746in}}{\pgfqpoint{6.742310in}{4.244864in}}%
\pgfpathcurveto{\pgfqpoint{6.738192in}{4.248982in}}{\pgfqpoint{6.732606in}{4.251296in}}{\pgfqpoint{6.726782in}{4.251296in}}%
\pgfpathcurveto{\pgfqpoint{6.720958in}{4.251296in}}{\pgfqpoint{6.715371in}{4.248982in}}{\pgfqpoint{6.711253in}{4.244864in}}%
\pgfpathcurveto{\pgfqpoint{6.707135in}{4.240746in}}{\pgfqpoint{6.704821in}{4.235160in}}{\pgfqpoint{6.704821in}{4.229336in}}%
\pgfpathcurveto{\pgfqpoint{6.704821in}{4.223512in}}{\pgfqpoint{6.707135in}{4.217926in}}{\pgfqpoint{6.711253in}{4.213808in}}%
\pgfpathcurveto{\pgfqpoint{6.715371in}{4.209690in}}{\pgfqpoint{6.720958in}{4.207376in}}{\pgfqpoint{6.726782in}{4.207376in}}%
\pgfpathlineto{\pgfqpoint{6.726782in}{4.207376in}}%
\pgfpathclose%
\pgfusepath{stroke,fill}%
\end{pgfscope}%
\begin{pgfscope}%
\pgfpathrectangle{\pgfqpoint{1.000000in}{0.979904in}}{\pgfqpoint{6.200000in}{5.960192in}}%
\pgfusepath{clip}%
\pgfsetbuttcap%
\pgfsetroundjoin%
\definecolor{currentfill}{rgb}{0.200000,0.800000,0.200000}%
\pgfsetfillcolor{currentfill}%
\pgfsetlinewidth{1.003750pt}%
\definecolor{currentstroke}{rgb}{0.200000,0.800000,0.200000}%
\pgfsetstrokecolor{currentstroke}%
\pgfsetdash{}{0pt}%
\pgfpathmoveto{\pgfqpoint{6.737530in}{4.319234in}}%
\pgfpathcurveto{\pgfqpoint{6.743354in}{4.319234in}}{\pgfqpoint{6.748940in}{4.321548in}}{\pgfqpoint{6.753058in}{4.325666in}}%
\pgfpathcurveto{\pgfqpoint{6.757176in}{4.329785in}}{\pgfqpoint{6.759490in}{4.335371in}}{\pgfqpoint{6.759490in}{4.341195in}}%
\pgfpathcurveto{\pgfqpoint{6.759490in}{4.347019in}}{\pgfqpoint{6.757176in}{4.352605in}}{\pgfqpoint{6.753058in}{4.356723in}}%
\pgfpathcurveto{\pgfqpoint{6.748940in}{4.360841in}}{\pgfqpoint{6.743354in}{4.363155in}}{\pgfqpoint{6.737530in}{4.363155in}}%
\pgfpathcurveto{\pgfqpoint{6.731706in}{4.363155in}}{\pgfqpoint{6.726120in}{4.360841in}}{\pgfqpoint{6.722002in}{4.356723in}}%
\pgfpathcurveto{\pgfqpoint{6.717884in}{4.352605in}}{\pgfqpoint{6.715570in}{4.347019in}}{\pgfqpoint{6.715570in}{4.341195in}}%
\pgfpathcurveto{\pgfqpoint{6.715570in}{4.335371in}}{\pgfqpoint{6.717884in}{4.329785in}}{\pgfqpoint{6.722002in}{4.325666in}}%
\pgfpathcurveto{\pgfqpoint{6.726120in}{4.321548in}}{\pgfqpoint{6.731706in}{4.319234in}}{\pgfqpoint{6.737530in}{4.319234in}}%
\pgfpathlineto{\pgfqpoint{6.737530in}{4.319234in}}%
\pgfpathclose%
\pgfusepath{stroke,fill}%
\end{pgfscope}%
\begin{pgfscope}%
\pgfpathrectangle{\pgfqpoint{1.000000in}{0.979904in}}{\pgfqpoint{6.200000in}{5.960192in}}%
\pgfusepath{clip}%
\pgfsetbuttcap%
\pgfsetroundjoin%
\definecolor{currentfill}{rgb}{0.200000,0.800000,0.200000}%
\pgfsetfillcolor{currentfill}%
\pgfsetlinewidth{1.003750pt}%
\definecolor{currentstroke}{rgb}{0.200000,0.800000,0.200000}%
\pgfsetstrokecolor{currentstroke}%
\pgfsetdash{}{0pt}%
\pgfpathmoveto{\pgfqpoint{6.918182in}{4.431554in}}%
\pgfpathcurveto{\pgfqpoint{6.924006in}{4.431554in}}{\pgfqpoint{6.929592in}{4.433868in}}{\pgfqpoint{6.933710in}{4.437986in}}%
\pgfpathcurveto{\pgfqpoint{6.937828in}{4.442104in}}{\pgfqpoint{6.940142in}{4.447690in}}{\pgfqpoint{6.940142in}{4.453514in}}%
\pgfpathcurveto{\pgfqpoint{6.940142in}{4.459338in}}{\pgfqpoint{6.937828in}{4.464924in}}{\pgfqpoint{6.933710in}{4.469042in}}%
\pgfpathcurveto{\pgfqpoint{6.929592in}{4.473160in}}{\pgfqpoint{6.924006in}{4.475474in}}{\pgfqpoint{6.918182in}{4.475474in}}%
\pgfpathcurveto{\pgfqpoint{6.912358in}{4.475474in}}{\pgfqpoint{6.906772in}{4.473160in}}{\pgfqpoint{6.902654in}{4.469042in}}%
\pgfpathcurveto{\pgfqpoint{6.898535in}{4.464924in}}{\pgfqpoint{6.896222in}{4.459338in}}{\pgfqpoint{6.896222in}{4.453514in}}%
\pgfpathcurveto{\pgfqpoint{6.896222in}{4.447690in}}{\pgfqpoint{6.898535in}{4.442104in}}{\pgfqpoint{6.902654in}{4.437986in}}%
\pgfpathcurveto{\pgfqpoint{6.906772in}{4.433868in}}{\pgfqpoint{6.912358in}{4.431554in}}{\pgfqpoint{6.918182in}{4.431554in}}%
\pgfpathlineto{\pgfqpoint{6.918182in}{4.431554in}}%
\pgfpathclose%
\pgfusepath{stroke,fill}%
\end{pgfscope}%
\begin{pgfscope}%
\pgfpathrectangle{\pgfqpoint{1.000000in}{0.979904in}}{\pgfqpoint{6.200000in}{5.960192in}}%
\pgfusepath{clip}%
\pgfsetbuttcap%
\pgfsetroundjoin%
\definecolor{currentfill}{rgb}{0.800000,0.200000,0.200000}%
\pgfsetfillcolor{currentfill}%
\pgfsetlinewidth{1.003750pt}%
\definecolor{currentstroke}{rgb}{0.800000,0.200000,0.200000}%
\pgfsetstrokecolor{currentstroke}%
\pgfsetdash{}{0pt}%
\pgfpathmoveto{\pgfqpoint{4.593682in}{2.865802in}}%
\pgfpathcurveto{\pgfqpoint{4.599506in}{2.865802in}}{\pgfqpoint{4.605092in}{2.868116in}}{\pgfqpoint{4.609210in}{2.872234in}}%
\pgfpathcurveto{\pgfqpoint{4.613328in}{2.876352in}}{\pgfqpoint{4.615642in}{2.881939in}}{\pgfqpoint{4.615642in}{2.887762in}}%
\pgfpathcurveto{\pgfqpoint{4.615642in}{2.893586in}}{\pgfqpoint{4.613328in}{2.899173in}}{\pgfqpoint{4.609210in}{2.903291in}}%
\pgfpathcurveto{\pgfqpoint{4.605092in}{2.907409in}}{\pgfqpoint{4.599506in}{2.909723in}}{\pgfqpoint{4.593682in}{2.909723in}}%
\pgfpathcurveto{\pgfqpoint{4.587858in}{2.909723in}}{\pgfqpoint{4.582272in}{2.907409in}}{\pgfqpoint{4.578154in}{2.903291in}}%
\pgfpathcurveto{\pgfqpoint{4.574036in}{2.899173in}}{\pgfqpoint{4.571722in}{2.893586in}}{\pgfqpoint{4.571722in}{2.887762in}}%
\pgfpathcurveto{\pgfqpoint{4.571722in}{2.881939in}}{\pgfqpoint{4.574036in}{2.876352in}}{\pgfqpoint{4.578154in}{2.872234in}}%
\pgfpathcurveto{\pgfqpoint{4.582272in}{2.868116in}}{\pgfqpoint{4.587858in}{2.865802in}}{\pgfqpoint{4.593682in}{2.865802in}}%
\pgfpathlineto{\pgfqpoint{4.593682in}{2.865802in}}%
\pgfpathclose%
\pgfusepath{stroke,fill}%
\end{pgfscope}%
\begin{pgfscope}%
\pgfpathrectangle{\pgfqpoint{1.000000in}{0.979904in}}{\pgfqpoint{6.200000in}{5.960192in}}%
\pgfusepath{clip}%
\pgfsetbuttcap%
\pgfsetroundjoin%
\definecolor{currentfill}{rgb}{0.800000,0.200000,0.200000}%
\pgfsetfillcolor{currentfill}%
\pgfsetlinewidth{1.003750pt}%
\definecolor{currentstroke}{rgb}{0.800000,0.200000,0.200000}%
\pgfsetstrokecolor{currentstroke}%
\pgfsetdash{}{0pt}%
\pgfpathmoveto{\pgfqpoint{4.572193in}{2.969984in}}%
\pgfpathcurveto{\pgfqpoint{4.578017in}{2.969984in}}{\pgfqpoint{4.583603in}{2.972298in}}{\pgfqpoint{4.587722in}{2.976416in}}%
\pgfpathcurveto{\pgfqpoint{4.591840in}{2.980534in}}{\pgfqpoint{4.594154in}{2.986120in}}{\pgfqpoint{4.594154in}{2.991944in}}%
\pgfpathcurveto{\pgfqpoint{4.594154in}{2.997768in}}{\pgfqpoint{4.591840in}{3.003354in}}{\pgfqpoint{4.587722in}{3.007472in}}%
\pgfpathcurveto{\pgfqpoint{4.583603in}{3.011590in}}{\pgfqpoint{4.578017in}{3.013904in}}{\pgfqpoint{4.572193in}{3.013904in}}%
\pgfpathcurveto{\pgfqpoint{4.566369in}{3.013904in}}{\pgfqpoint{4.560783in}{3.011590in}}{\pgfqpoint{4.556665in}{3.007472in}}%
\pgfpathcurveto{\pgfqpoint{4.552547in}{3.003354in}}{\pgfqpoint{4.550233in}{2.997768in}}{\pgfqpoint{4.550233in}{2.991944in}}%
\pgfpathcurveto{\pgfqpoint{4.550233in}{2.986120in}}{\pgfqpoint{4.552547in}{2.980534in}}{\pgfqpoint{4.556665in}{2.976416in}}%
\pgfpathcurveto{\pgfqpoint{4.560783in}{2.972298in}}{\pgfqpoint{4.566369in}{2.969984in}}{\pgfqpoint{4.572193in}{2.969984in}}%
\pgfpathlineto{\pgfqpoint{4.572193in}{2.969984in}}%
\pgfpathclose%
\pgfusepath{stroke,fill}%
\end{pgfscope}%
\begin{pgfscope}%
\pgfpathrectangle{\pgfqpoint{1.000000in}{0.979904in}}{\pgfqpoint{6.200000in}{5.960192in}}%
\pgfusepath{clip}%
\pgfsetbuttcap%
\pgfsetroundjoin%
\definecolor{currentfill}{rgb}{0.800000,0.200000,0.200000}%
\pgfsetfillcolor{currentfill}%
\pgfsetlinewidth{1.003750pt}%
\definecolor{currentstroke}{rgb}{0.800000,0.200000,0.200000}%
\pgfsetstrokecolor{currentstroke}%
\pgfsetdash{}{0pt}%
\pgfpathmoveto{\pgfqpoint{4.522502in}{3.068669in}}%
\pgfpathcurveto{\pgfqpoint{4.528326in}{3.068669in}}{\pgfqpoint{4.533912in}{3.070983in}}{\pgfqpoint{4.538030in}{3.075101in}}%
\pgfpathcurveto{\pgfqpoint{4.542148in}{3.079219in}}{\pgfqpoint{4.544462in}{3.084805in}}{\pgfqpoint{4.544462in}{3.090629in}}%
\pgfpathcurveto{\pgfqpoint{4.544462in}{3.096453in}}{\pgfqpoint{4.542148in}{3.102039in}}{\pgfqpoint{4.538030in}{3.106157in}}%
\pgfpathcurveto{\pgfqpoint{4.533912in}{3.110275in}}{\pgfqpoint{4.528326in}{3.112589in}}{\pgfqpoint{4.522502in}{3.112589in}}%
\pgfpathcurveto{\pgfqpoint{4.516678in}{3.112589in}}{\pgfqpoint{4.511092in}{3.110275in}}{\pgfqpoint{4.506974in}{3.106157in}}%
\pgfpathcurveto{\pgfqpoint{4.502856in}{3.102039in}}{\pgfqpoint{4.500542in}{3.096453in}}{\pgfqpoint{4.500542in}{3.090629in}}%
\pgfpathcurveto{\pgfqpoint{4.500542in}{3.084805in}}{\pgfqpoint{4.502856in}{3.079219in}}{\pgfqpoint{4.506974in}{3.075101in}}%
\pgfpathcurveto{\pgfqpoint{4.511092in}{3.070983in}}{\pgfqpoint{4.516678in}{3.068669in}}{\pgfqpoint{4.522502in}{3.068669in}}%
\pgfpathlineto{\pgfqpoint{4.522502in}{3.068669in}}%
\pgfpathclose%
\pgfusepath{stroke,fill}%
\end{pgfscope}%
\begin{pgfscope}%
\pgfpathrectangle{\pgfqpoint{1.000000in}{0.979904in}}{\pgfqpoint{6.200000in}{5.960192in}}%
\pgfusepath{clip}%
\pgfsetbuttcap%
\pgfsetroundjoin%
\definecolor{currentfill}{rgb}{0.800000,0.200000,0.200000}%
\pgfsetfillcolor{currentfill}%
\pgfsetlinewidth{1.003750pt}%
\definecolor{currentstroke}{rgb}{0.800000,0.200000,0.200000}%
\pgfsetstrokecolor{currentstroke}%
\pgfsetdash{}{0pt}%
\pgfpathmoveto{\pgfqpoint{4.503512in}{3.168517in}}%
\pgfpathcurveto{\pgfqpoint{4.509336in}{3.168517in}}{\pgfqpoint{4.514922in}{3.170831in}}{\pgfqpoint{4.519040in}{3.174949in}}%
\pgfpathcurveto{\pgfqpoint{4.523158in}{3.179067in}}{\pgfqpoint{4.525472in}{3.184653in}}{\pgfqpoint{4.525472in}{3.190477in}}%
\pgfpathcurveto{\pgfqpoint{4.525472in}{3.196301in}}{\pgfqpoint{4.523158in}{3.201888in}}{\pgfqpoint{4.519040in}{3.206006in}}%
\pgfpathcurveto{\pgfqpoint{4.514922in}{3.210124in}}{\pgfqpoint{4.509336in}{3.212438in}}{\pgfqpoint{4.503512in}{3.212438in}}%
\pgfpathcurveto{\pgfqpoint{4.497688in}{3.212438in}}{\pgfqpoint{4.492101in}{3.210124in}}{\pgfqpoint{4.487983in}{3.206006in}}%
\pgfpathcurveto{\pgfqpoint{4.483865in}{3.201888in}}{\pgfqpoint{4.481551in}{3.196301in}}{\pgfqpoint{4.481551in}{3.190477in}}%
\pgfpathcurveto{\pgfqpoint{4.481551in}{3.184653in}}{\pgfqpoint{4.483865in}{3.179067in}}{\pgfqpoint{4.487983in}{3.174949in}}%
\pgfpathcurveto{\pgfqpoint{4.492101in}{3.170831in}}{\pgfqpoint{4.497688in}{3.168517in}}{\pgfqpoint{4.503512in}{3.168517in}}%
\pgfpathlineto{\pgfqpoint{4.503512in}{3.168517in}}%
\pgfpathclose%
\pgfusepath{stroke,fill}%
\end{pgfscope}%
\begin{pgfscope}%
\pgfpathrectangle{\pgfqpoint{1.000000in}{0.979904in}}{\pgfqpoint{6.200000in}{5.960192in}}%
\pgfusepath{clip}%
\pgfsetbuttcap%
\pgfsetroundjoin%
\definecolor{currentfill}{rgb}{0.800000,0.200000,0.200000}%
\pgfsetfillcolor{currentfill}%
\pgfsetlinewidth{1.003750pt}%
\definecolor{currentstroke}{rgb}{0.800000,0.200000,0.200000}%
\pgfsetstrokecolor{currentstroke}%
\pgfsetdash{}{0pt}%
\pgfpathmoveto{\pgfqpoint{4.412911in}{3.249818in}}%
\pgfpathcurveto{\pgfqpoint{4.418735in}{3.249818in}}{\pgfqpoint{4.424321in}{3.252132in}}{\pgfqpoint{4.428439in}{3.256250in}}%
\pgfpathcurveto{\pgfqpoint{4.432557in}{3.260368in}}{\pgfqpoint{4.434871in}{3.265954in}}{\pgfqpoint{4.434871in}{3.271778in}}%
\pgfpathcurveto{\pgfqpoint{4.434871in}{3.277602in}}{\pgfqpoint{4.432557in}{3.283188in}}{\pgfqpoint{4.428439in}{3.287306in}}%
\pgfpathcurveto{\pgfqpoint{4.424321in}{3.291424in}}{\pgfqpoint{4.418735in}{3.293738in}}{\pgfqpoint{4.412911in}{3.293738in}}%
\pgfpathcurveto{\pgfqpoint{4.407087in}{3.293738in}}{\pgfqpoint{4.401501in}{3.291424in}}{\pgfqpoint{4.397383in}{3.287306in}}%
\pgfpathcurveto{\pgfqpoint{4.393264in}{3.283188in}}{\pgfqpoint{4.390950in}{3.277602in}}{\pgfqpoint{4.390950in}{3.271778in}}%
\pgfpathcurveto{\pgfqpoint{4.390950in}{3.265954in}}{\pgfqpoint{4.393264in}{3.260368in}}{\pgfqpoint{4.397383in}{3.256250in}}%
\pgfpathcurveto{\pgfqpoint{4.401501in}{3.252132in}}{\pgfqpoint{4.407087in}{3.249818in}}{\pgfqpoint{4.412911in}{3.249818in}}%
\pgfpathlineto{\pgfqpoint{4.412911in}{3.249818in}}%
\pgfpathclose%
\pgfusepath{stroke,fill}%
\end{pgfscope}%
\begin{pgfscope}%
\pgfpathrectangle{\pgfqpoint{1.000000in}{0.979904in}}{\pgfqpoint{6.200000in}{5.960192in}}%
\pgfusepath{clip}%
\pgfsetbuttcap%
\pgfsetroundjoin%
\definecolor{currentfill}{rgb}{0.800000,0.200000,0.200000}%
\pgfsetfillcolor{currentfill}%
\pgfsetlinewidth{1.003750pt}%
\definecolor{currentstroke}{rgb}{0.800000,0.200000,0.200000}%
\pgfsetstrokecolor{currentstroke}%
\pgfsetdash{}{0pt}%
\pgfpathmoveto{\pgfqpoint{4.368023in}{3.337150in}}%
\pgfpathcurveto{\pgfqpoint{4.373847in}{3.337150in}}{\pgfqpoint{4.379433in}{3.339463in}}{\pgfqpoint{4.383551in}{3.343582in}}%
\pgfpathcurveto{\pgfqpoint{4.387670in}{3.347700in}}{\pgfqpoint{4.389983in}{3.353286in}}{\pgfqpoint{4.389983in}{3.359110in}}%
\pgfpathcurveto{\pgfqpoint{4.389983in}{3.364934in}}{\pgfqpoint{4.387670in}{3.370520in}}{\pgfqpoint{4.383551in}{3.374638in}}%
\pgfpathcurveto{\pgfqpoint{4.379433in}{3.378756in}}{\pgfqpoint{4.373847in}{3.381070in}}{\pgfqpoint{4.368023in}{3.381070in}}%
\pgfpathcurveto{\pgfqpoint{4.362199in}{3.381070in}}{\pgfqpoint{4.356613in}{3.378756in}}{\pgfqpoint{4.352495in}{3.374638in}}%
\pgfpathcurveto{\pgfqpoint{4.348377in}{3.370520in}}{\pgfqpoint{4.346063in}{3.364934in}}{\pgfqpoint{4.346063in}{3.359110in}}%
\pgfpathcurveto{\pgfqpoint{4.346063in}{3.353286in}}{\pgfqpoint{4.348377in}{3.347700in}}{\pgfqpoint{4.352495in}{3.343582in}}%
\pgfpathcurveto{\pgfqpoint{4.356613in}{3.339463in}}{\pgfqpoint{4.362199in}{3.337150in}}{\pgfqpoint{4.368023in}{3.337150in}}%
\pgfpathlineto{\pgfqpoint{4.368023in}{3.337150in}}%
\pgfpathclose%
\pgfusepath{stroke,fill}%
\end{pgfscope}%
\begin{pgfscope}%
\pgfpathrectangle{\pgfqpoint{1.000000in}{0.979904in}}{\pgfqpoint{6.200000in}{5.960192in}}%
\pgfusepath{clip}%
\pgfsetbuttcap%
\pgfsetroundjoin%
\definecolor{currentfill}{rgb}{0.800000,0.200000,0.200000}%
\pgfsetfillcolor{currentfill}%
\pgfsetlinewidth{1.003750pt}%
\definecolor{currentstroke}{rgb}{0.800000,0.200000,0.200000}%
\pgfsetstrokecolor{currentstroke}%
\pgfsetdash{}{0pt}%
\pgfpathmoveto{\pgfqpoint{4.417289in}{3.460071in}}%
\pgfpathcurveto{\pgfqpoint{4.423113in}{3.460071in}}{\pgfqpoint{4.428700in}{3.462385in}}{\pgfqpoint{4.432818in}{3.466503in}}%
\pgfpathcurveto{\pgfqpoint{4.436936in}{3.470621in}}{\pgfqpoint{4.439250in}{3.476207in}}{\pgfqpoint{4.439250in}{3.482031in}}%
\pgfpathcurveto{\pgfqpoint{4.439250in}{3.487855in}}{\pgfqpoint{4.436936in}{3.493441in}}{\pgfqpoint{4.432818in}{3.497560in}}%
\pgfpathcurveto{\pgfqpoint{4.428700in}{3.501678in}}{\pgfqpoint{4.423113in}{3.503992in}}{\pgfqpoint{4.417289in}{3.503992in}}%
\pgfpathcurveto{\pgfqpoint{4.411465in}{3.503992in}}{\pgfqpoint{4.405879in}{3.501678in}}{\pgfqpoint{4.401761in}{3.497560in}}%
\pgfpathcurveto{\pgfqpoint{4.397643in}{3.493441in}}{\pgfqpoint{4.395329in}{3.487855in}}{\pgfqpoint{4.395329in}{3.482031in}}%
\pgfpathcurveto{\pgfqpoint{4.395329in}{3.476207in}}{\pgfqpoint{4.397643in}{3.470621in}}{\pgfqpoint{4.401761in}{3.466503in}}%
\pgfpathcurveto{\pgfqpoint{4.405879in}{3.462385in}}{\pgfqpoint{4.411465in}{3.460071in}}{\pgfqpoint{4.417289in}{3.460071in}}%
\pgfpathlineto{\pgfqpoint{4.417289in}{3.460071in}}%
\pgfpathclose%
\pgfusepath{stroke,fill}%
\end{pgfscope}%
\begin{pgfscope}%
\pgfpathrectangle{\pgfqpoint{1.000000in}{0.979904in}}{\pgfqpoint{6.200000in}{5.960192in}}%
\pgfusepath{clip}%
\pgfsetbuttcap%
\pgfsetroundjoin%
\definecolor{currentfill}{rgb}{0.800000,0.200000,0.200000}%
\pgfsetfillcolor{currentfill}%
\pgfsetlinewidth{1.003750pt}%
\definecolor{currentstroke}{rgb}{0.800000,0.200000,0.200000}%
\pgfsetstrokecolor{currentstroke}%
\pgfsetdash{}{0pt}%
\pgfpathmoveto{\pgfqpoint{4.327700in}{3.529741in}}%
\pgfpathcurveto{\pgfqpoint{4.333524in}{3.529741in}}{\pgfqpoint{4.339110in}{3.532054in}}{\pgfqpoint{4.343228in}{3.536173in}}%
\pgfpathcurveto{\pgfqpoint{4.347346in}{3.540291in}}{\pgfqpoint{4.349660in}{3.545877in}}{\pgfqpoint{4.349660in}{3.551701in}}%
\pgfpathcurveto{\pgfqpoint{4.349660in}{3.557525in}}{\pgfqpoint{4.347346in}{3.563111in}}{\pgfqpoint{4.343228in}{3.567229in}}%
\pgfpathcurveto{\pgfqpoint{4.339110in}{3.571347in}}{\pgfqpoint{4.333524in}{3.573661in}}{\pgfqpoint{4.327700in}{3.573661in}}%
\pgfpathcurveto{\pgfqpoint{4.321876in}{3.573661in}}{\pgfqpoint{4.316290in}{3.571347in}}{\pgfqpoint{4.312172in}{3.567229in}}%
\pgfpathcurveto{\pgfqpoint{4.308054in}{3.563111in}}{\pgfqpoint{4.305740in}{3.557525in}}{\pgfqpoint{4.305740in}{3.551701in}}%
\pgfpathcurveto{\pgfqpoint{4.305740in}{3.545877in}}{\pgfqpoint{4.308054in}{3.540291in}}{\pgfqpoint{4.312172in}{3.536173in}}%
\pgfpathcurveto{\pgfqpoint{4.316290in}{3.532054in}}{\pgfqpoint{4.321876in}{3.529741in}}{\pgfqpoint{4.327700in}{3.529741in}}%
\pgfpathlineto{\pgfqpoint{4.327700in}{3.529741in}}%
\pgfpathclose%
\pgfusepath{stroke,fill}%
\end{pgfscope}%
\begin{pgfscope}%
\pgfpathrectangle{\pgfqpoint{1.000000in}{0.979904in}}{\pgfqpoint{6.200000in}{5.960192in}}%
\pgfusepath{clip}%
\pgfsetbuttcap%
\pgfsetroundjoin%
\definecolor{currentfill}{rgb}{0.800000,0.200000,0.200000}%
\pgfsetfillcolor{currentfill}%
\pgfsetlinewidth{1.003750pt}%
\definecolor{currentstroke}{rgb}{0.800000,0.200000,0.200000}%
\pgfsetstrokecolor{currentstroke}%
\pgfsetdash{}{0pt}%
\pgfpathmoveto{\pgfqpoint{4.320628in}{3.637926in}}%
\pgfpathcurveto{\pgfqpoint{4.326452in}{3.637926in}}{\pgfqpoint{4.332038in}{3.640240in}}{\pgfqpoint{4.336157in}{3.644358in}}%
\pgfpathcurveto{\pgfqpoint{4.340275in}{3.648476in}}{\pgfqpoint{4.342589in}{3.654062in}}{\pgfqpoint{4.342589in}{3.659886in}}%
\pgfpathcurveto{\pgfqpoint{4.342589in}{3.665710in}}{\pgfqpoint{4.340275in}{3.671296in}}{\pgfqpoint{4.336157in}{3.675415in}}%
\pgfpathcurveto{\pgfqpoint{4.332038in}{3.679533in}}{\pgfqpoint{4.326452in}{3.681847in}}{\pgfqpoint{4.320628in}{3.681847in}}%
\pgfpathcurveto{\pgfqpoint{4.314804in}{3.681847in}}{\pgfqpoint{4.309218in}{3.679533in}}{\pgfqpoint{4.305100in}{3.675415in}}%
\pgfpathcurveto{\pgfqpoint{4.300982in}{3.671296in}}{\pgfqpoint{4.298668in}{3.665710in}}{\pgfqpoint{4.298668in}{3.659886in}}%
\pgfpathcurveto{\pgfqpoint{4.298668in}{3.654062in}}{\pgfqpoint{4.300982in}{3.648476in}}{\pgfqpoint{4.305100in}{3.644358in}}%
\pgfpathcurveto{\pgfqpoint{4.309218in}{3.640240in}}{\pgfqpoint{4.314804in}{3.637926in}}{\pgfqpoint{4.320628in}{3.637926in}}%
\pgfpathlineto{\pgfqpoint{4.320628in}{3.637926in}}%
\pgfpathclose%
\pgfusepath{stroke,fill}%
\end{pgfscope}%
\begin{pgfscope}%
\pgfpathrectangle{\pgfqpoint{1.000000in}{0.979904in}}{\pgfqpoint{6.200000in}{5.960192in}}%
\pgfusepath{clip}%
\pgfsetbuttcap%
\pgfsetroundjoin%
\definecolor{currentfill}{rgb}{0.800000,0.200000,0.200000}%
\pgfsetfillcolor{currentfill}%
\pgfsetlinewidth{1.003750pt}%
\definecolor{currentstroke}{rgb}{0.800000,0.200000,0.200000}%
\pgfsetstrokecolor{currentstroke}%
\pgfsetdash{}{0pt}%
\pgfpathmoveto{\pgfqpoint{4.246768in}{3.710188in}}%
\pgfpathcurveto{\pgfqpoint{4.252592in}{3.710188in}}{\pgfqpoint{4.258178in}{3.712502in}}{\pgfqpoint{4.262296in}{3.716621in}}%
\pgfpathcurveto{\pgfqpoint{4.266415in}{3.720739in}}{\pgfqpoint{4.268728in}{3.726325in}}{\pgfqpoint{4.268728in}{3.732149in}}%
\pgfpathcurveto{\pgfqpoint{4.268728in}{3.737973in}}{\pgfqpoint{4.266415in}{3.743559in}}{\pgfqpoint{4.262296in}{3.747677in}}%
\pgfpathcurveto{\pgfqpoint{4.258178in}{3.751795in}}{\pgfqpoint{4.252592in}{3.754109in}}{\pgfqpoint{4.246768in}{3.754109in}}%
\pgfpathcurveto{\pgfqpoint{4.240944in}{3.754109in}}{\pgfqpoint{4.235358in}{3.751795in}}{\pgfqpoint{4.231240in}{3.747677in}}%
\pgfpathcurveto{\pgfqpoint{4.227122in}{3.743559in}}{\pgfqpoint{4.224808in}{3.737973in}}{\pgfqpoint{4.224808in}{3.732149in}}%
\pgfpathcurveto{\pgfqpoint{4.224808in}{3.726325in}}{\pgfqpoint{4.227122in}{3.720739in}}{\pgfqpoint{4.231240in}{3.716621in}}%
\pgfpathcurveto{\pgfqpoint{4.235358in}{3.712502in}}{\pgfqpoint{4.240944in}{3.710188in}}{\pgfqpoint{4.246768in}{3.710188in}}%
\pgfpathlineto{\pgfqpoint{4.246768in}{3.710188in}}%
\pgfpathclose%
\pgfusepath{stroke,fill}%
\end{pgfscope}%
\begin{pgfscope}%
\pgfpathrectangle{\pgfqpoint{1.000000in}{0.979904in}}{\pgfqpoint{6.200000in}{5.960192in}}%
\pgfusepath{clip}%
\pgfsetbuttcap%
\pgfsetroundjoin%
\definecolor{currentfill}{rgb}{0.800000,0.200000,0.200000}%
\pgfsetfillcolor{currentfill}%
\pgfsetlinewidth{1.003750pt}%
\definecolor{currentstroke}{rgb}{0.800000,0.200000,0.200000}%
\pgfsetstrokecolor{currentstroke}%
\pgfsetdash{}{0pt}%
\pgfpathmoveto{\pgfqpoint{4.136999in}{3.752378in}}%
\pgfpathcurveto{\pgfqpoint{4.142823in}{3.752378in}}{\pgfqpoint{4.148409in}{3.754692in}}{\pgfqpoint{4.152527in}{3.758810in}}%
\pgfpathcurveto{\pgfqpoint{4.156645in}{3.762928in}}{\pgfqpoint{4.158959in}{3.768514in}}{\pgfqpoint{4.158959in}{3.774338in}}%
\pgfpathcurveto{\pgfqpoint{4.158959in}{3.780162in}}{\pgfqpoint{4.156645in}{3.785748in}}{\pgfqpoint{4.152527in}{3.789866in}}%
\pgfpathcurveto{\pgfqpoint{4.148409in}{3.793984in}}{\pgfqpoint{4.142823in}{3.796298in}}{\pgfqpoint{4.136999in}{3.796298in}}%
\pgfpathcurveto{\pgfqpoint{4.131175in}{3.796298in}}{\pgfqpoint{4.125589in}{3.793984in}}{\pgfqpoint{4.121470in}{3.789866in}}%
\pgfpathcurveto{\pgfqpoint{4.117352in}{3.785748in}}{\pgfqpoint{4.115038in}{3.780162in}}{\pgfqpoint{4.115038in}{3.774338in}}%
\pgfpathcurveto{\pgfqpoint{4.115038in}{3.768514in}}{\pgfqpoint{4.117352in}{3.762928in}}{\pgfqpoint{4.121470in}{3.758810in}}%
\pgfpathcurveto{\pgfqpoint{4.125589in}{3.754692in}}{\pgfqpoint{4.131175in}{3.752378in}}{\pgfqpoint{4.136999in}{3.752378in}}%
\pgfpathlineto{\pgfqpoint{4.136999in}{3.752378in}}%
\pgfpathclose%
\pgfusepath{stroke,fill}%
\end{pgfscope}%
\begin{pgfscope}%
\pgfpathrectangle{\pgfqpoint{1.000000in}{0.979904in}}{\pgfqpoint{6.200000in}{5.960192in}}%
\pgfusepath{clip}%
\pgfsetbuttcap%
\pgfsetroundjoin%
\definecolor{currentfill}{rgb}{0.800000,0.200000,0.200000}%
\pgfsetfillcolor{currentfill}%
\pgfsetlinewidth{1.003750pt}%
\definecolor{currentstroke}{rgb}{0.800000,0.200000,0.200000}%
\pgfsetstrokecolor{currentstroke}%
\pgfsetdash{}{0pt}%
\pgfpathmoveto{\pgfqpoint{4.165558in}{3.900144in}}%
\pgfpathcurveto{\pgfqpoint{4.171382in}{3.900144in}}{\pgfqpoint{4.176968in}{3.902458in}}{\pgfqpoint{4.181086in}{3.906576in}}%
\pgfpathcurveto{\pgfqpoint{4.185204in}{3.910694in}}{\pgfqpoint{4.187518in}{3.916280in}}{\pgfqpoint{4.187518in}{3.922104in}}%
\pgfpathcurveto{\pgfqpoint{4.187518in}{3.927928in}}{\pgfqpoint{4.185204in}{3.933514in}}{\pgfqpoint{4.181086in}{3.937632in}}%
\pgfpathcurveto{\pgfqpoint{4.176968in}{3.941750in}}{\pgfqpoint{4.171382in}{3.944064in}}{\pgfqpoint{4.165558in}{3.944064in}}%
\pgfpathcurveto{\pgfqpoint{4.159734in}{3.944064in}}{\pgfqpoint{4.154148in}{3.941750in}}{\pgfqpoint{4.150029in}{3.937632in}}%
\pgfpathcurveto{\pgfqpoint{4.145911in}{3.933514in}}{\pgfqpoint{4.143597in}{3.927928in}}{\pgfqpoint{4.143597in}{3.922104in}}%
\pgfpathcurveto{\pgfqpoint{4.143597in}{3.916280in}}{\pgfqpoint{4.145911in}{3.910694in}}{\pgfqpoint{4.150029in}{3.906576in}}%
\pgfpathcurveto{\pgfqpoint{4.154148in}{3.902458in}}{\pgfqpoint{4.159734in}{3.900144in}}{\pgfqpoint{4.165558in}{3.900144in}}%
\pgfpathlineto{\pgfqpoint{4.165558in}{3.900144in}}%
\pgfpathclose%
\pgfusepath{stroke,fill}%
\end{pgfscope}%
\begin{pgfscope}%
\pgfpathrectangle{\pgfqpoint{1.000000in}{0.979904in}}{\pgfqpoint{6.200000in}{5.960192in}}%
\pgfusepath{clip}%
\pgfsetbuttcap%
\pgfsetroundjoin%
\definecolor{currentfill}{rgb}{0.800000,0.200000,0.200000}%
\pgfsetfillcolor{currentfill}%
\pgfsetlinewidth{1.003750pt}%
\definecolor{currentstroke}{rgb}{0.800000,0.200000,0.200000}%
\pgfsetstrokecolor{currentstroke}%
\pgfsetdash{}{0pt}%
\pgfpathmoveto{\pgfqpoint{4.067836in}{3.947983in}}%
\pgfpathcurveto{\pgfqpoint{4.073660in}{3.947983in}}{\pgfqpoint{4.079246in}{3.950297in}}{\pgfqpoint{4.083364in}{3.954415in}}%
\pgfpathcurveto{\pgfqpoint{4.087483in}{3.958533in}}{\pgfqpoint{4.089796in}{3.964119in}}{\pgfqpoint{4.089796in}{3.969943in}}%
\pgfpathcurveto{\pgfqpoint{4.089796in}{3.975767in}}{\pgfqpoint{4.087483in}{3.981354in}}{\pgfqpoint{4.083364in}{3.985472in}}%
\pgfpathcurveto{\pgfqpoint{4.079246in}{3.989590in}}{\pgfqpoint{4.073660in}{3.991904in}}{\pgfqpoint{4.067836in}{3.991904in}}%
\pgfpathcurveto{\pgfqpoint{4.062012in}{3.991904in}}{\pgfqpoint{4.056426in}{3.989590in}}{\pgfqpoint{4.052308in}{3.985472in}}%
\pgfpathcurveto{\pgfqpoint{4.048190in}{3.981354in}}{\pgfqpoint{4.045876in}{3.975767in}}{\pgfqpoint{4.045876in}{3.969943in}}%
\pgfpathcurveto{\pgfqpoint{4.045876in}{3.964119in}}{\pgfqpoint{4.048190in}{3.958533in}}{\pgfqpoint{4.052308in}{3.954415in}}%
\pgfpathcurveto{\pgfqpoint{4.056426in}{3.950297in}}{\pgfqpoint{4.062012in}{3.947983in}}{\pgfqpoint{4.067836in}{3.947983in}}%
\pgfpathlineto{\pgfqpoint{4.067836in}{3.947983in}}%
\pgfpathclose%
\pgfusepath{stroke,fill}%
\end{pgfscope}%
\begin{pgfscope}%
\pgfpathrectangle{\pgfqpoint{1.000000in}{0.979904in}}{\pgfqpoint{6.200000in}{5.960192in}}%
\pgfusepath{clip}%
\pgfsetbuttcap%
\pgfsetroundjoin%
\definecolor{currentfill}{rgb}{0.800000,0.200000,0.200000}%
\pgfsetfillcolor{currentfill}%
\pgfsetlinewidth{1.003750pt}%
\definecolor{currentstroke}{rgb}{0.800000,0.200000,0.200000}%
\pgfsetstrokecolor{currentstroke}%
\pgfsetdash{}{0pt}%
\pgfpathmoveto{\pgfqpoint{4.014725in}{4.037070in}}%
\pgfpathcurveto{\pgfqpoint{4.020549in}{4.037070in}}{\pgfqpoint{4.026135in}{4.039384in}}{\pgfqpoint{4.030253in}{4.043502in}}%
\pgfpathcurveto{\pgfqpoint{4.034371in}{4.047620in}}{\pgfqpoint{4.036685in}{4.053206in}}{\pgfqpoint{4.036685in}{4.059030in}}%
\pgfpathcurveto{\pgfqpoint{4.036685in}{4.064854in}}{\pgfqpoint{4.034371in}{4.070440in}}{\pgfqpoint{4.030253in}{4.074559in}}%
\pgfpathcurveto{\pgfqpoint{4.026135in}{4.078677in}}{\pgfqpoint{4.020549in}{4.080991in}}{\pgfqpoint{4.014725in}{4.080991in}}%
\pgfpathcurveto{\pgfqpoint{4.008901in}{4.080991in}}{\pgfqpoint{4.003315in}{4.078677in}}{\pgfqpoint{3.999197in}{4.074559in}}%
\pgfpathcurveto{\pgfqpoint{3.995079in}{4.070440in}}{\pgfqpoint{3.992765in}{4.064854in}}{\pgfqpoint{3.992765in}{4.059030in}}%
\pgfpathcurveto{\pgfqpoint{3.992765in}{4.053206in}}{\pgfqpoint{3.995079in}{4.047620in}}{\pgfqpoint{3.999197in}{4.043502in}}%
\pgfpathcurveto{\pgfqpoint{4.003315in}{4.039384in}}{\pgfqpoint{4.008901in}{4.037070in}}{\pgfqpoint{4.014725in}{4.037070in}}%
\pgfpathlineto{\pgfqpoint{4.014725in}{4.037070in}}%
\pgfpathclose%
\pgfusepath{stroke,fill}%
\end{pgfscope}%
\begin{pgfscope}%
\pgfpathrectangle{\pgfqpoint{1.000000in}{0.979904in}}{\pgfqpoint{6.200000in}{5.960192in}}%
\pgfusepath{clip}%
\pgfsetbuttcap%
\pgfsetroundjoin%
\definecolor{currentfill}{rgb}{0.800000,0.200000,0.200000}%
\pgfsetfillcolor{currentfill}%
\pgfsetlinewidth{1.003750pt}%
\definecolor{currentstroke}{rgb}{0.800000,0.200000,0.200000}%
\pgfsetstrokecolor{currentstroke}%
\pgfsetdash{}{0pt}%
\pgfpathmoveto{\pgfqpoint{3.905698in}{4.063246in}}%
\pgfpathcurveto{\pgfqpoint{3.911522in}{4.063246in}}{\pgfqpoint{3.917108in}{4.065560in}}{\pgfqpoint{3.921226in}{4.069678in}}%
\pgfpathcurveto{\pgfqpoint{3.925345in}{4.073796in}}{\pgfqpoint{3.927658in}{4.079382in}}{\pgfqpoint{3.927658in}{4.085206in}}%
\pgfpathcurveto{\pgfqpoint{3.927658in}{4.091030in}}{\pgfqpoint{3.925345in}{4.096616in}}{\pgfqpoint{3.921226in}{4.100734in}}%
\pgfpathcurveto{\pgfqpoint{3.917108in}{4.104853in}}{\pgfqpoint{3.911522in}{4.107166in}}{\pgfqpoint{3.905698in}{4.107166in}}%
\pgfpathcurveto{\pgfqpoint{3.899874in}{4.107166in}}{\pgfqpoint{3.894288in}{4.104853in}}{\pgfqpoint{3.890170in}{4.100734in}}%
\pgfpathcurveto{\pgfqpoint{3.886052in}{4.096616in}}{\pgfqpoint{3.883738in}{4.091030in}}{\pgfqpoint{3.883738in}{4.085206in}}%
\pgfpathcurveto{\pgfqpoint{3.883738in}{4.079382in}}{\pgfqpoint{3.886052in}{4.073796in}}{\pgfqpoint{3.890170in}{4.069678in}}%
\pgfpathcurveto{\pgfqpoint{3.894288in}{4.065560in}}{\pgfqpoint{3.899874in}{4.063246in}}{\pgfqpoint{3.905698in}{4.063246in}}%
\pgfpathlineto{\pgfqpoint{3.905698in}{4.063246in}}%
\pgfpathclose%
\pgfusepath{stroke,fill}%
\end{pgfscope}%
\begin{pgfscope}%
\pgfpathrectangle{\pgfqpoint{1.000000in}{0.979904in}}{\pgfqpoint{6.200000in}{5.960192in}}%
\pgfusepath{clip}%
\pgfsetbuttcap%
\pgfsetroundjoin%
\definecolor{currentfill}{rgb}{0.800000,0.200000,0.200000}%
\pgfsetfillcolor{currentfill}%
\pgfsetlinewidth{1.003750pt}%
\definecolor{currentstroke}{rgb}{0.800000,0.200000,0.200000}%
\pgfsetstrokecolor{currentstroke}%
\pgfsetdash{}{0pt}%
\pgfpathmoveto{\pgfqpoint{3.844975in}{4.146664in}}%
\pgfpathcurveto{\pgfqpoint{3.850799in}{4.146664in}}{\pgfqpoint{3.856385in}{4.148978in}}{\pgfqpoint{3.860503in}{4.153096in}}%
\pgfpathcurveto{\pgfqpoint{3.864622in}{4.157214in}}{\pgfqpoint{3.866935in}{4.162801in}}{\pgfqpoint{3.866935in}{4.168625in}}%
\pgfpathcurveto{\pgfqpoint{3.866935in}{4.174448in}}{\pgfqpoint{3.864622in}{4.180035in}}{\pgfqpoint{3.860503in}{4.184153in}}%
\pgfpathcurveto{\pgfqpoint{3.856385in}{4.188271in}}{\pgfqpoint{3.850799in}{4.190585in}}{\pgfqpoint{3.844975in}{4.190585in}}%
\pgfpathcurveto{\pgfqpoint{3.839151in}{4.190585in}}{\pgfqpoint{3.833565in}{4.188271in}}{\pgfqpoint{3.829447in}{4.184153in}}%
\pgfpathcurveto{\pgfqpoint{3.825329in}{4.180035in}}{\pgfqpoint{3.823015in}{4.174448in}}{\pgfqpoint{3.823015in}{4.168625in}}%
\pgfpathcurveto{\pgfqpoint{3.823015in}{4.162801in}}{\pgfqpoint{3.825329in}{4.157214in}}{\pgfqpoint{3.829447in}{4.153096in}}%
\pgfpathcurveto{\pgfqpoint{3.833565in}{4.148978in}}{\pgfqpoint{3.839151in}{4.146664in}}{\pgfqpoint{3.844975in}{4.146664in}}%
\pgfpathlineto{\pgfqpoint{3.844975in}{4.146664in}}%
\pgfpathclose%
\pgfusepath{stroke,fill}%
\end{pgfscope}%
\begin{pgfscope}%
\pgfpathrectangle{\pgfqpoint{1.000000in}{0.979904in}}{\pgfqpoint{6.200000in}{5.960192in}}%
\pgfusepath{clip}%
\pgfsetbuttcap%
\pgfsetroundjoin%
\definecolor{currentfill}{rgb}{0.800000,0.200000,0.200000}%
\pgfsetfillcolor{currentfill}%
\pgfsetlinewidth{1.003750pt}%
\definecolor{currentstroke}{rgb}{0.800000,0.200000,0.200000}%
\pgfsetstrokecolor{currentstroke}%
\pgfsetdash{}{0pt}%
\pgfpathmoveto{\pgfqpoint{3.814916in}{4.287377in}}%
\pgfpathcurveto{\pgfqpoint{3.820740in}{4.287377in}}{\pgfqpoint{3.826326in}{4.289690in}}{\pgfqpoint{3.830444in}{4.293809in}}%
\pgfpathcurveto{\pgfqpoint{3.834562in}{4.297927in}}{\pgfqpoint{3.836876in}{4.303513in}}{\pgfqpoint{3.836876in}{4.309337in}}%
\pgfpathcurveto{\pgfqpoint{3.836876in}{4.315161in}}{\pgfqpoint{3.834562in}{4.320747in}}{\pgfqpoint{3.830444in}{4.324865in}}%
\pgfpathcurveto{\pgfqpoint{3.826326in}{4.328983in}}{\pgfqpoint{3.820740in}{4.331297in}}{\pgfqpoint{3.814916in}{4.331297in}}%
\pgfpathcurveto{\pgfqpoint{3.809092in}{4.331297in}}{\pgfqpoint{3.803505in}{4.328983in}}{\pgfqpoint{3.799387in}{4.324865in}}%
\pgfpathcurveto{\pgfqpoint{3.795269in}{4.320747in}}{\pgfqpoint{3.792955in}{4.315161in}}{\pgfqpoint{3.792955in}{4.309337in}}%
\pgfpathcurveto{\pgfqpoint{3.792955in}{4.303513in}}{\pgfqpoint{3.795269in}{4.297927in}}{\pgfqpoint{3.799387in}{4.293809in}}%
\pgfpathcurveto{\pgfqpoint{3.803505in}{4.289690in}}{\pgfqpoint{3.809092in}{4.287377in}}{\pgfqpoint{3.814916in}{4.287377in}}%
\pgfpathlineto{\pgfqpoint{3.814916in}{4.287377in}}%
\pgfpathclose%
\pgfusepath{stroke,fill}%
\end{pgfscope}%
\begin{pgfscope}%
\pgfpathrectangle{\pgfqpoint{1.000000in}{0.979904in}}{\pgfqpoint{6.200000in}{5.960192in}}%
\pgfusepath{clip}%
\pgfsetbuttcap%
\pgfsetroundjoin%
\definecolor{currentfill}{rgb}{0.800000,0.200000,0.200000}%
\pgfsetfillcolor{currentfill}%
\pgfsetlinewidth{1.003750pt}%
\definecolor{currentstroke}{rgb}{0.800000,0.200000,0.200000}%
\pgfsetstrokecolor{currentstroke}%
\pgfsetdash{}{0pt}%
\pgfpathmoveto{\pgfqpoint{3.687100in}{4.273494in}}%
\pgfpathcurveto{\pgfqpoint{3.692924in}{4.273494in}}{\pgfqpoint{3.698511in}{4.275808in}}{\pgfqpoint{3.702629in}{4.279927in}}%
\pgfpathcurveto{\pgfqpoint{3.706747in}{4.284045in}}{\pgfqpoint{3.709061in}{4.289631in}}{\pgfqpoint{3.709061in}{4.295455in}}%
\pgfpathcurveto{\pgfqpoint{3.709061in}{4.301279in}}{\pgfqpoint{3.706747in}{4.306865in}}{\pgfqpoint{3.702629in}{4.310983in}}%
\pgfpathcurveto{\pgfqpoint{3.698511in}{4.315101in}}{\pgfqpoint{3.692924in}{4.317415in}}{\pgfqpoint{3.687100in}{4.317415in}}%
\pgfpathcurveto{\pgfqpoint{3.681277in}{4.317415in}}{\pgfqpoint{3.675690in}{4.315101in}}{\pgfqpoint{3.671572in}{4.310983in}}%
\pgfpathcurveto{\pgfqpoint{3.667454in}{4.306865in}}{\pgfqpoint{3.665140in}{4.301279in}}{\pgfqpoint{3.665140in}{4.295455in}}%
\pgfpathcurveto{\pgfqpoint{3.665140in}{4.289631in}}{\pgfqpoint{3.667454in}{4.284045in}}{\pgfqpoint{3.671572in}{4.279927in}}%
\pgfpathcurveto{\pgfqpoint{3.675690in}{4.275808in}}{\pgfqpoint{3.681277in}{4.273494in}}{\pgfqpoint{3.687100in}{4.273494in}}%
\pgfpathlineto{\pgfqpoint{3.687100in}{4.273494in}}%
\pgfpathclose%
\pgfusepath{stroke,fill}%
\end{pgfscope}%
\begin{pgfscope}%
\pgfpathrectangle{\pgfqpoint{1.000000in}{0.979904in}}{\pgfqpoint{6.200000in}{5.960192in}}%
\pgfusepath{clip}%
\pgfsetbuttcap%
\pgfsetroundjoin%
\definecolor{currentfill}{rgb}{0.800000,0.200000,0.200000}%
\pgfsetfillcolor{currentfill}%
\pgfsetlinewidth{1.003750pt}%
\definecolor{currentstroke}{rgb}{0.800000,0.200000,0.200000}%
\pgfsetstrokecolor{currentstroke}%
\pgfsetdash{}{0pt}%
\pgfpathmoveto{\pgfqpoint{3.605580in}{4.338816in}}%
\pgfpathcurveto{\pgfqpoint{3.611404in}{4.338816in}}{\pgfqpoint{3.616990in}{4.341130in}}{\pgfqpoint{3.621108in}{4.345248in}}%
\pgfpathcurveto{\pgfqpoint{3.625227in}{4.349366in}}{\pgfqpoint{3.627540in}{4.354952in}}{\pgfqpoint{3.627540in}{4.360776in}}%
\pgfpathcurveto{\pgfqpoint{3.627540in}{4.366600in}}{\pgfqpoint{3.625227in}{4.372186in}}{\pgfqpoint{3.621108in}{4.376304in}}%
\pgfpathcurveto{\pgfqpoint{3.616990in}{4.380422in}}{\pgfqpoint{3.611404in}{4.382736in}}{\pgfqpoint{3.605580in}{4.382736in}}%
\pgfpathcurveto{\pgfqpoint{3.599756in}{4.382736in}}{\pgfqpoint{3.594170in}{4.380422in}}{\pgfqpoint{3.590052in}{4.376304in}}%
\pgfpathcurveto{\pgfqpoint{3.585934in}{4.372186in}}{\pgfqpoint{3.583620in}{4.366600in}}{\pgfqpoint{3.583620in}{4.360776in}}%
\pgfpathcurveto{\pgfqpoint{3.583620in}{4.354952in}}{\pgfqpoint{3.585934in}{4.349366in}}{\pgfqpoint{3.590052in}{4.345248in}}%
\pgfpathcurveto{\pgfqpoint{3.594170in}{4.341130in}}{\pgfqpoint{3.599756in}{4.338816in}}{\pgfqpoint{3.605580in}{4.338816in}}%
\pgfpathlineto{\pgfqpoint{3.605580in}{4.338816in}}%
\pgfpathclose%
\pgfusepath{stroke,fill}%
\end{pgfscope}%
\begin{pgfscope}%
\pgfpathrectangle{\pgfqpoint{1.000000in}{0.979904in}}{\pgfqpoint{6.200000in}{5.960192in}}%
\pgfusepath{clip}%
\pgfsetbuttcap%
\pgfsetroundjoin%
\definecolor{currentfill}{rgb}{0.800000,0.200000,0.200000}%
\pgfsetfillcolor{currentfill}%
\pgfsetlinewidth{1.003750pt}%
\definecolor{currentstroke}{rgb}{0.800000,0.200000,0.200000}%
\pgfsetstrokecolor{currentstroke}%
\pgfsetdash{}{0pt}%
\pgfpathmoveto{\pgfqpoint{3.476331in}{4.288289in}}%
\pgfpathcurveto{\pgfqpoint{3.482155in}{4.288289in}}{\pgfqpoint{3.487741in}{4.290603in}}{\pgfqpoint{3.491859in}{4.294721in}}%
\pgfpathcurveto{\pgfqpoint{3.495977in}{4.298839in}}{\pgfqpoint{3.498291in}{4.304425in}}{\pgfqpoint{3.498291in}{4.310249in}}%
\pgfpathcurveto{\pgfqpoint{3.498291in}{4.316073in}}{\pgfqpoint{3.495977in}{4.321659in}}{\pgfqpoint{3.491859in}{4.325778in}}%
\pgfpathcurveto{\pgfqpoint{3.487741in}{4.329896in}}{\pgfqpoint{3.482155in}{4.332210in}}{\pgfqpoint{3.476331in}{4.332210in}}%
\pgfpathcurveto{\pgfqpoint{3.470507in}{4.332210in}}{\pgfqpoint{3.464921in}{4.329896in}}{\pgfqpoint{3.460803in}{4.325778in}}%
\pgfpathcurveto{\pgfqpoint{3.456685in}{4.321659in}}{\pgfqpoint{3.454371in}{4.316073in}}{\pgfqpoint{3.454371in}{4.310249in}}%
\pgfpathcurveto{\pgfqpoint{3.454371in}{4.304425in}}{\pgfqpoint{3.456685in}{4.298839in}}{\pgfqpoint{3.460803in}{4.294721in}}%
\pgfpathcurveto{\pgfqpoint{3.464921in}{4.290603in}}{\pgfqpoint{3.470507in}{4.288289in}}{\pgfqpoint{3.476331in}{4.288289in}}%
\pgfpathlineto{\pgfqpoint{3.476331in}{4.288289in}}%
\pgfpathclose%
\pgfusepath{stroke,fill}%
\end{pgfscope}%
\begin{pgfscope}%
\pgfpathrectangle{\pgfqpoint{1.000000in}{0.979904in}}{\pgfqpoint{6.200000in}{5.960192in}}%
\pgfusepath{clip}%
\pgfsetbuttcap%
\pgfsetroundjoin%
\definecolor{currentfill}{rgb}{0.800000,0.200000,0.200000}%
\pgfsetfillcolor{currentfill}%
\pgfsetlinewidth{1.003750pt}%
\definecolor{currentstroke}{rgb}{0.800000,0.200000,0.200000}%
\pgfsetstrokecolor{currentstroke}%
\pgfsetdash{}{0pt}%
\pgfpathmoveto{\pgfqpoint{3.442674in}{4.505318in}}%
\pgfpathcurveto{\pgfqpoint{3.448498in}{4.505318in}}{\pgfqpoint{3.454084in}{4.507632in}}{\pgfqpoint{3.458202in}{4.511750in}}%
\pgfpathcurveto{\pgfqpoint{3.462320in}{4.515868in}}{\pgfqpoint{3.464634in}{4.521455in}}{\pgfqpoint{3.464634in}{4.527279in}}%
\pgfpathcurveto{\pgfqpoint{3.464634in}{4.533103in}}{\pgfqpoint{3.462320in}{4.538689in}}{\pgfqpoint{3.458202in}{4.542807in}}%
\pgfpathcurveto{\pgfqpoint{3.454084in}{4.546925in}}{\pgfqpoint{3.448498in}{4.549239in}}{\pgfqpoint{3.442674in}{4.549239in}}%
\pgfpathcurveto{\pgfqpoint{3.436850in}{4.549239in}}{\pgfqpoint{3.431264in}{4.546925in}}{\pgfqpoint{3.427146in}{4.542807in}}%
\pgfpathcurveto{\pgfqpoint{3.423028in}{4.538689in}}{\pgfqpoint{3.420714in}{4.533103in}}{\pgfqpoint{3.420714in}{4.527279in}}%
\pgfpathcurveto{\pgfqpoint{3.420714in}{4.521455in}}{\pgfqpoint{3.423028in}{4.515868in}}{\pgfqpoint{3.427146in}{4.511750in}}%
\pgfpathcurveto{\pgfqpoint{3.431264in}{4.507632in}}{\pgfqpoint{3.436850in}{4.505318in}}{\pgfqpoint{3.442674in}{4.505318in}}%
\pgfpathlineto{\pgfqpoint{3.442674in}{4.505318in}}%
\pgfpathclose%
\pgfusepath{stroke,fill}%
\end{pgfscope}%
\begin{pgfscope}%
\pgfpathrectangle{\pgfqpoint{1.000000in}{0.979904in}}{\pgfqpoint{6.200000in}{5.960192in}}%
\pgfusepath{clip}%
\pgfsetbuttcap%
\pgfsetroundjoin%
\definecolor{currentfill}{rgb}{0.200000,0.800000,0.200000}%
\pgfsetfillcolor{currentfill}%
\pgfsetlinewidth{1.003750pt}%
\definecolor{currentstroke}{rgb}{0.200000,0.800000,0.200000}%
\pgfsetstrokecolor{currentstroke}%
\pgfsetdash{}{0pt}%
\pgfpathmoveto{\pgfqpoint{3.330838in}{4.506218in}}%
\pgfpathcurveto{\pgfqpoint{3.336662in}{4.506218in}}{\pgfqpoint{3.342248in}{4.508532in}}{\pgfqpoint{3.346366in}{4.512650in}}%
\pgfpathcurveto{\pgfqpoint{3.350484in}{4.516768in}}{\pgfqpoint{3.352798in}{4.522354in}}{\pgfqpoint{3.352798in}{4.528178in}}%
\pgfpathcurveto{\pgfqpoint{3.352798in}{4.534002in}}{\pgfqpoint{3.350484in}{4.539588in}}{\pgfqpoint{3.346366in}{4.543706in}}%
\pgfpathcurveto{\pgfqpoint{3.342248in}{4.547825in}}{\pgfqpoint{3.336662in}{4.550138in}}{\pgfqpoint{3.330838in}{4.550138in}}%
\pgfpathcurveto{\pgfqpoint{3.325014in}{4.550138in}}{\pgfqpoint{3.319428in}{4.547825in}}{\pgfqpoint{3.315310in}{4.543706in}}%
\pgfpathcurveto{\pgfqpoint{3.311192in}{4.539588in}}{\pgfqpoint{3.308878in}{4.534002in}}{\pgfqpoint{3.308878in}{4.528178in}}%
\pgfpathcurveto{\pgfqpoint{3.308878in}{4.522354in}}{\pgfqpoint{3.311192in}{4.516768in}}{\pgfqpoint{3.315310in}{4.512650in}}%
\pgfpathcurveto{\pgfqpoint{3.319428in}{4.508532in}}{\pgfqpoint{3.325014in}{4.506218in}}{\pgfqpoint{3.330838in}{4.506218in}}%
\pgfpathlineto{\pgfqpoint{3.330838in}{4.506218in}}%
\pgfpathclose%
\pgfusepath{stroke,fill}%
\end{pgfscope}%
\begin{pgfscope}%
\pgfpathrectangle{\pgfqpoint{1.000000in}{0.979904in}}{\pgfqpoint{6.200000in}{5.960192in}}%
\pgfusepath{clip}%
\pgfsetbuttcap%
\pgfsetroundjoin%
\definecolor{currentfill}{rgb}{0.200000,0.800000,0.200000}%
\pgfsetfillcolor{currentfill}%
\pgfsetlinewidth{1.003750pt}%
\definecolor{currentstroke}{rgb}{0.200000,0.800000,0.200000}%
\pgfsetstrokecolor{currentstroke}%
\pgfsetdash{}{0pt}%
\pgfpathmoveto{\pgfqpoint{3.221573in}{4.503076in}}%
\pgfpathcurveto{\pgfqpoint{3.227397in}{4.503076in}}{\pgfqpoint{3.232983in}{4.505390in}}{\pgfqpoint{3.237101in}{4.509508in}}%
\pgfpathcurveto{\pgfqpoint{3.241219in}{4.513626in}}{\pgfqpoint{3.243533in}{4.519213in}}{\pgfqpoint{3.243533in}{4.525037in}}%
\pgfpathcurveto{\pgfqpoint{3.243533in}{4.530860in}}{\pgfqpoint{3.241219in}{4.536447in}}{\pgfqpoint{3.237101in}{4.540565in}}%
\pgfpathcurveto{\pgfqpoint{3.232983in}{4.544683in}}{\pgfqpoint{3.227397in}{4.546997in}}{\pgfqpoint{3.221573in}{4.546997in}}%
\pgfpathcurveto{\pgfqpoint{3.215749in}{4.546997in}}{\pgfqpoint{3.210163in}{4.544683in}}{\pgfqpoint{3.206045in}{4.540565in}}%
\pgfpathcurveto{\pgfqpoint{3.201927in}{4.536447in}}{\pgfqpoint{3.199613in}{4.530860in}}{\pgfqpoint{3.199613in}{4.525037in}}%
\pgfpathcurveto{\pgfqpoint{3.199613in}{4.519213in}}{\pgfqpoint{3.201927in}{4.513626in}}{\pgfqpoint{3.206045in}{4.509508in}}%
\pgfpathcurveto{\pgfqpoint{3.210163in}{4.505390in}}{\pgfqpoint{3.215749in}{4.503076in}}{\pgfqpoint{3.221573in}{4.503076in}}%
\pgfpathlineto{\pgfqpoint{3.221573in}{4.503076in}}%
\pgfpathclose%
\pgfusepath{stroke,fill}%
\end{pgfscope}%
\begin{pgfscope}%
\pgfpathrectangle{\pgfqpoint{1.000000in}{0.979904in}}{\pgfqpoint{6.200000in}{5.960192in}}%
\pgfusepath{clip}%
\pgfsetbuttcap%
\pgfsetroundjoin%
\definecolor{currentfill}{rgb}{0.800000,0.200000,0.200000}%
\pgfsetfillcolor{currentfill}%
\pgfsetlinewidth{1.003750pt}%
\definecolor{currentstroke}{rgb}{0.800000,0.200000,0.200000}%
\pgfsetstrokecolor{currentstroke}%
\pgfsetdash{}{0pt}%
\pgfpathmoveto{\pgfqpoint{3.114005in}{4.489897in}}%
\pgfpathcurveto{\pgfqpoint{3.119829in}{4.489897in}}{\pgfqpoint{3.125415in}{4.492211in}}{\pgfqpoint{3.129534in}{4.496329in}}%
\pgfpathcurveto{\pgfqpoint{3.133652in}{4.500447in}}{\pgfqpoint{3.135966in}{4.506033in}}{\pgfqpoint{3.135966in}{4.511857in}}%
\pgfpathcurveto{\pgfqpoint{3.135966in}{4.517681in}}{\pgfqpoint{3.133652in}{4.523267in}}{\pgfqpoint{3.129534in}{4.527386in}}%
\pgfpathcurveto{\pgfqpoint{3.125415in}{4.531504in}}{\pgfqpoint{3.119829in}{4.533818in}}{\pgfqpoint{3.114005in}{4.533818in}}%
\pgfpathcurveto{\pgfqpoint{3.108181in}{4.533818in}}{\pgfqpoint{3.102595in}{4.531504in}}{\pgfqpoint{3.098477in}{4.527386in}}%
\pgfpathcurveto{\pgfqpoint{3.094359in}{4.523267in}}{\pgfqpoint{3.092045in}{4.517681in}}{\pgfqpoint{3.092045in}{4.511857in}}%
\pgfpathcurveto{\pgfqpoint{3.092045in}{4.506033in}}{\pgfqpoint{3.094359in}{4.500447in}}{\pgfqpoint{3.098477in}{4.496329in}}%
\pgfpathcurveto{\pgfqpoint{3.102595in}{4.492211in}}{\pgfqpoint{3.108181in}{4.489897in}}{\pgfqpoint{3.114005in}{4.489897in}}%
\pgfpathlineto{\pgfqpoint{3.114005in}{4.489897in}}%
\pgfpathclose%
\pgfusepath{stroke,fill}%
\end{pgfscope}%
\begin{pgfscope}%
\pgfpathrectangle{\pgfqpoint{1.000000in}{0.979904in}}{\pgfqpoint{6.200000in}{5.960192in}}%
\pgfusepath{clip}%
\pgfsetbuttcap%
\pgfsetroundjoin%
\definecolor{currentfill}{rgb}{0.800000,0.200000,0.200000}%
\pgfsetfillcolor{currentfill}%
\pgfsetlinewidth{1.003750pt}%
\definecolor{currentstroke}{rgb}{0.800000,0.200000,0.200000}%
\pgfsetstrokecolor{currentstroke}%
\pgfsetdash{}{0pt}%
\pgfpathmoveto{\pgfqpoint{3.008580in}{4.454996in}}%
\pgfpathcurveto{\pgfqpoint{3.014404in}{4.454996in}}{\pgfqpoint{3.019990in}{4.457310in}}{\pgfqpoint{3.024108in}{4.461428in}}%
\pgfpathcurveto{\pgfqpoint{3.028227in}{4.465546in}}{\pgfqpoint{3.030540in}{4.471132in}}{\pgfqpoint{3.030540in}{4.476956in}}%
\pgfpathcurveto{\pgfqpoint{3.030540in}{4.482780in}}{\pgfqpoint{3.028227in}{4.488366in}}{\pgfqpoint{3.024108in}{4.492484in}}%
\pgfpathcurveto{\pgfqpoint{3.019990in}{4.496603in}}{\pgfqpoint{3.014404in}{4.498916in}}{\pgfqpoint{3.008580in}{4.498916in}}%
\pgfpathcurveto{\pgfqpoint{3.002756in}{4.498916in}}{\pgfqpoint{2.997170in}{4.496603in}}{\pgfqpoint{2.993052in}{4.492484in}}%
\pgfpathcurveto{\pgfqpoint{2.988934in}{4.488366in}}{\pgfqpoint{2.986620in}{4.482780in}}{\pgfqpoint{2.986620in}{4.476956in}}%
\pgfpathcurveto{\pgfqpoint{2.986620in}{4.471132in}}{\pgfqpoint{2.988934in}{4.465546in}}{\pgfqpoint{2.993052in}{4.461428in}}%
\pgfpathcurveto{\pgfqpoint{2.997170in}{4.457310in}}{\pgfqpoint{3.002756in}{4.454996in}}{\pgfqpoint{3.008580in}{4.454996in}}%
\pgfpathlineto{\pgfqpoint{3.008580in}{4.454996in}}%
\pgfpathclose%
\pgfusepath{stroke,fill}%
\end{pgfscope}%
\begin{pgfscope}%
\pgfpathrectangle{\pgfqpoint{1.000000in}{0.979904in}}{\pgfqpoint{6.200000in}{5.960192in}}%
\pgfusepath{clip}%
\pgfsetbuttcap%
\pgfsetroundjoin%
\definecolor{currentfill}{rgb}{0.800000,0.200000,0.200000}%
\pgfsetfillcolor{currentfill}%
\pgfsetlinewidth{1.003750pt}%
\definecolor{currentstroke}{rgb}{0.800000,0.200000,0.200000}%
\pgfsetstrokecolor{currentstroke}%
\pgfsetdash{}{0pt}%
\pgfpathmoveto{\pgfqpoint{2.907275in}{4.479246in}}%
\pgfpathcurveto{\pgfqpoint{2.913099in}{4.479246in}}{\pgfqpoint{2.918686in}{4.481560in}}{\pgfqpoint{2.922804in}{4.485678in}}%
\pgfpathcurveto{\pgfqpoint{2.926922in}{4.489797in}}{\pgfqpoint{2.929236in}{4.495383in}}{\pgfqpoint{2.929236in}{4.501207in}}%
\pgfpathcurveto{\pgfqpoint{2.929236in}{4.507031in}}{\pgfqpoint{2.926922in}{4.512617in}}{\pgfqpoint{2.922804in}{4.516735in}}%
\pgfpathcurveto{\pgfqpoint{2.918686in}{4.520853in}}{\pgfqpoint{2.913099in}{4.523167in}}{\pgfqpoint{2.907275in}{4.523167in}}%
\pgfpathcurveto{\pgfqpoint{2.901452in}{4.523167in}}{\pgfqpoint{2.895865in}{4.520853in}}{\pgfqpoint{2.891747in}{4.516735in}}%
\pgfpathcurveto{\pgfqpoint{2.887629in}{4.512617in}}{\pgfqpoint{2.885315in}{4.507031in}}{\pgfqpoint{2.885315in}{4.501207in}}%
\pgfpathcurveto{\pgfqpoint{2.885315in}{4.495383in}}{\pgfqpoint{2.887629in}{4.489797in}}{\pgfqpoint{2.891747in}{4.485678in}}%
\pgfpathcurveto{\pgfqpoint{2.895865in}{4.481560in}}{\pgfqpoint{2.901452in}{4.479246in}}{\pgfqpoint{2.907275in}{4.479246in}}%
\pgfpathlineto{\pgfqpoint{2.907275in}{4.479246in}}%
\pgfpathclose%
\pgfusepath{stroke,fill}%
\end{pgfscope}%
\begin{pgfscope}%
\pgfpathrectangle{\pgfqpoint{1.000000in}{0.979904in}}{\pgfqpoint{6.200000in}{5.960192in}}%
\pgfusepath{clip}%
\pgfsetbuttcap%
\pgfsetroundjoin%
\definecolor{currentfill}{rgb}{0.800000,0.200000,0.200000}%
\pgfsetfillcolor{currentfill}%
\pgfsetlinewidth{1.003750pt}%
\definecolor{currentstroke}{rgb}{0.800000,0.200000,0.200000}%
\pgfsetstrokecolor{currentstroke}%
\pgfsetdash{}{0pt}%
\pgfpathmoveto{\pgfqpoint{2.802448in}{4.506423in}}%
\pgfpathcurveto{\pgfqpoint{2.808272in}{4.506423in}}{\pgfqpoint{2.813858in}{4.508736in}}{\pgfqpoint{2.817976in}{4.512855in}}%
\pgfpathcurveto{\pgfqpoint{2.822095in}{4.516973in}}{\pgfqpoint{2.824408in}{4.522559in}}{\pgfqpoint{2.824408in}{4.528383in}}%
\pgfpathcurveto{\pgfqpoint{2.824408in}{4.534207in}}{\pgfqpoint{2.822095in}{4.539793in}}{\pgfqpoint{2.817976in}{4.543911in}}%
\pgfpathcurveto{\pgfqpoint{2.813858in}{4.548029in}}{\pgfqpoint{2.808272in}{4.550343in}}{\pgfqpoint{2.802448in}{4.550343in}}%
\pgfpathcurveto{\pgfqpoint{2.796624in}{4.550343in}}{\pgfqpoint{2.791038in}{4.548029in}}{\pgfqpoint{2.786920in}{4.543911in}}%
\pgfpathcurveto{\pgfqpoint{2.782802in}{4.539793in}}{\pgfqpoint{2.780488in}{4.534207in}}{\pgfqpoint{2.780488in}{4.528383in}}%
\pgfpathcurveto{\pgfqpoint{2.780488in}{4.522559in}}{\pgfqpoint{2.782802in}{4.516973in}}{\pgfqpoint{2.786920in}{4.512855in}}%
\pgfpathcurveto{\pgfqpoint{2.791038in}{4.508736in}}{\pgfqpoint{2.796624in}{4.506423in}}{\pgfqpoint{2.802448in}{4.506423in}}%
\pgfpathlineto{\pgfqpoint{2.802448in}{4.506423in}}%
\pgfpathclose%
\pgfusepath{stroke,fill}%
\end{pgfscope}%
\begin{pgfscope}%
\pgfpathrectangle{\pgfqpoint{1.000000in}{0.979904in}}{\pgfqpoint{6.200000in}{5.960192in}}%
\pgfusepath{clip}%
\pgfsetbuttcap%
\pgfsetroundjoin%
\definecolor{currentfill}{rgb}{0.800000,0.200000,0.200000}%
\pgfsetfillcolor{currentfill}%
\pgfsetlinewidth{1.003750pt}%
\definecolor{currentstroke}{rgb}{0.800000,0.200000,0.200000}%
\pgfsetstrokecolor{currentstroke}%
\pgfsetdash{}{0pt}%
\pgfpathmoveto{\pgfqpoint{2.711316in}{4.406798in}}%
\pgfpathcurveto{\pgfqpoint{2.717140in}{4.406798in}}{\pgfqpoint{2.722726in}{4.409112in}}{\pgfqpoint{2.726844in}{4.413230in}}%
\pgfpathcurveto{\pgfqpoint{2.730962in}{4.417348in}}{\pgfqpoint{2.733276in}{4.422935in}}{\pgfqpoint{2.733276in}{4.428759in}}%
\pgfpathcurveto{\pgfqpoint{2.733276in}{4.434582in}}{\pgfqpoint{2.730962in}{4.440169in}}{\pgfqpoint{2.726844in}{4.444287in}}%
\pgfpathcurveto{\pgfqpoint{2.722726in}{4.448405in}}{\pgfqpoint{2.717140in}{4.450719in}}{\pgfqpoint{2.711316in}{4.450719in}}%
\pgfpathcurveto{\pgfqpoint{2.705492in}{4.450719in}}{\pgfqpoint{2.699906in}{4.448405in}}{\pgfqpoint{2.695787in}{4.444287in}}%
\pgfpathcurveto{\pgfqpoint{2.691669in}{4.440169in}}{\pgfqpoint{2.689355in}{4.434582in}}{\pgfqpoint{2.689355in}{4.428759in}}%
\pgfpathcurveto{\pgfqpoint{2.689355in}{4.422935in}}{\pgfqpoint{2.691669in}{4.417348in}}{\pgfqpoint{2.695787in}{4.413230in}}%
\pgfpathcurveto{\pgfqpoint{2.699906in}{4.409112in}}{\pgfqpoint{2.705492in}{4.406798in}}{\pgfqpoint{2.711316in}{4.406798in}}%
\pgfpathlineto{\pgfqpoint{2.711316in}{4.406798in}}%
\pgfpathclose%
\pgfusepath{stroke,fill}%
\end{pgfscope}%
\begin{pgfscope}%
\pgfpathrectangle{\pgfqpoint{1.000000in}{0.979904in}}{\pgfqpoint{6.200000in}{5.960192in}}%
\pgfusepath{clip}%
\pgfsetbuttcap%
\pgfsetroundjoin%
\definecolor{currentfill}{rgb}{0.800000,0.200000,0.200000}%
\pgfsetfillcolor{currentfill}%
\pgfsetlinewidth{1.003750pt}%
\definecolor{currentstroke}{rgb}{0.800000,0.200000,0.200000}%
\pgfsetstrokecolor{currentstroke}%
\pgfsetdash{}{0pt}%
\pgfpathmoveto{\pgfqpoint{2.612026in}{4.399203in}}%
\pgfpathcurveto{\pgfqpoint{2.617850in}{4.399203in}}{\pgfqpoint{2.623436in}{4.401516in}}{\pgfqpoint{2.627554in}{4.405635in}}%
\pgfpathcurveto{\pgfqpoint{2.631672in}{4.409753in}}{\pgfqpoint{2.633986in}{4.415339in}}{\pgfqpoint{2.633986in}{4.421163in}}%
\pgfpathcurveto{\pgfqpoint{2.633986in}{4.426987in}}{\pgfqpoint{2.631672in}{4.432573in}}{\pgfqpoint{2.627554in}{4.436691in}}%
\pgfpathcurveto{\pgfqpoint{2.623436in}{4.440809in}}{\pgfqpoint{2.617850in}{4.443123in}}{\pgfqpoint{2.612026in}{4.443123in}}%
\pgfpathcurveto{\pgfqpoint{2.606202in}{4.443123in}}{\pgfqpoint{2.600615in}{4.440809in}}{\pgfqpoint{2.596497in}{4.436691in}}%
\pgfpathcurveto{\pgfqpoint{2.592379in}{4.432573in}}{\pgfqpoint{2.590065in}{4.426987in}}{\pgfqpoint{2.590065in}{4.421163in}}%
\pgfpathcurveto{\pgfqpoint{2.590065in}{4.415339in}}{\pgfqpoint{2.592379in}{4.409753in}}{\pgfqpoint{2.596497in}{4.405635in}}%
\pgfpathcurveto{\pgfqpoint{2.600615in}{4.401516in}}{\pgfqpoint{2.606202in}{4.399203in}}{\pgfqpoint{2.612026in}{4.399203in}}%
\pgfpathlineto{\pgfqpoint{2.612026in}{4.399203in}}%
\pgfpathclose%
\pgfusepath{stroke,fill}%
\end{pgfscope}%
\begin{pgfscope}%
\pgfpathrectangle{\pgfqpoint{1.000000in}{0.979904in}}{\pgfqpoint{6.200000in}{5.960192in}}%
\pgfusepath{clip}%
\pgfsetbuttcap%
\pgfsetroundjoin%
\definecolor{currentfill}{rgb}{0.800000,0.200000,0.200000}%
\pgfsetfillcolor{currentfill}%
\pgfsetlinewidth{1.003750pt}%
\definecolor{currentstroke}{rgb}{0.800000,0.200000,0.200000}%
\pgfsetstrokecolor{currentstroke}%
\pgfsetdash{}{0pt}%
\pgfpathmoveto{\pgfqpoint{2.532731in}{4.313140in}}%
\pgfpathcurveto{\pgfqpoint{2.538555in}{4.313140in}}{\pgfqpoint{2.544142in}{4.315454in}}{\pgfqpoint{2.548260in}{4.319572in}}%
\pgfpathcurveto{\pgfqpoint{2.552378in}{4.323690in}}{\pgfqpoint{2.554692in}{4.329276in}}{\pgfqpoint{2.554692in}{4.335100in}}%
\pgfpathcurveto{\pgfqpoint{2.554692in}{4.340924in}}{\pgfqpoint{2.552378in}{4.346510in}}{\pgfqpoint{2.548260in}{4.350629in}}%
\pgfpathcurveto{\pgfqpoint{2.544142in}{4.354747in}}{\pgfqpoint{2.538555in}{4.357061in}}{\pgfqpoint{2.532731in}{4.357061in}}%
\pgfpathcurveto{\pgfqpoint{2.526908in}{4.357061in}}{\pgfqpoint{2.521321in}{4.354747in}}{\pgfqpoint{2.517203in}{4.350629in}}%
\pgfpathcurveto{\pgfqpoint{2.513085in}{4.346510in}}{\pgfqpoint{2.510771in}{4.340924in}}{\pgfqpoint{2.510771in}{4.335100in}}%
\pgfpathcurveto{\pgfqpoint{2.510771in}{4.329276in}}{\pgfqpoint{2.513085in}{4.323690in}}{\pgfqpoint{2.517203in}{4.319572in}}%
\pgfpathcurveto{\pgfqpoint{2.521321in}{4.315454in}}{\pgfqpoint{2.526908in}{4.313140in}}{\pgfqpoint{2.532731in}{4.313140in}}%
\pgfpathlineto{\pgfqpoint{2.532731in}{4.313140in}}%
\pgfpathclose%
\pgfusepath{stroke,fill}%
\end{pgfscope}%
\begin{pgfscope}%
\pgfpathrectangle{\pgfqpoint{1.000000in}{0.979904in}}{\pgfqpoint{6.200000in}{5.960192in}}%
\pgfusepath{clip}%
\pgfsetbuttcap%
\pgfsetroundjoin%
\definecolor{currentfill}{rgb}{0.800000,0.200000,0.200000}%
\pgfsetfillcolor{currentfill}%
\pgfsetlinewidth{1.003750pt}%
\definecolor{currentstroke}{rgb}{0.800000,0.200000,0.200000}%
\pgfsetstrokecolor{currentstroke}%
\pgfsetdash{}{0pt}%
\pgfpathmoveto{\pgfqpoint{2.426170in}{4.329837in}}%
\pgfpathcurveto{\pgfqpoint{2.431994in}{4.329837in}}{\pgfqpoint{2.437580in}{4.332151in}}{\pgfqpoint{2.441698in}{4.336269in}}%
\pgfpathcurveto{\pgfqpoint{2.445816in}{4.340387in}}{\pgfqpoint{2.448130in}{4.345974in}}{\pgfqpoint{2.448130in}{4.351798in}}%
\pgfpathcurveto{\pgfqpoint{2.448130in}{4.357621in}}{\pgfqpoint{2.445816in}{4.363208in}}{\pgfqpoint{2.441698in}{4.367326in}}%
\pgfpathcurveto{\pgfqpoint{2.437580in}{4.371444in}}{\pgfqpoint{2.431994in}{4.373758in}}{\pgfqpoint{2.426170in}{4.373758in}}%
\pgfpathcurveto{\pgfqpoint{2.420346in}{4.373758in}}{\pgfqpoint{2.414760in}{4.371444in}}{\pgfqpoint{2.410642in}{4.367326in}}%
\pgfpathcurveto{\pgfqpoint{2.406524in}{4.363208in}}{\pgfqpoint{2.404210in}{4.357621in}}{\pgfqpoint{2.404210in}{4.351798in}}%
\pgfpathcurveto{\pgfqpoint{2.404210in}{4.345974in}}{\pgfqpoint{2.406524in}{4.340387in}}{\pgfqpoint{2.410642in}{4.336269in}}%
\pgfpathcurveto{\pgfqpoint{2.414760in}{4.332151in}}{\pgfqpoint{2.420346in}{4.329837in}}{\pgfqpoint{2.426170in}{4.329837in}}%
\pgfpathlineto{\pgfqpoint{2.426170in}{4.329837in}}%
\pgfpathclose%
\pgfusepath{stroke,fill}%
\end{pgfscope}%
\begin{pgfscope}%
\pgfpathrectangle{\pgfqpoint{1.000000in}{0.979904in}}{\pgfqpoint{6.200000in}{5.960192in}}%
\pgfusepath{clip}%
\pgfsetbuttcap%
\pgfsetroundjoin%
\definecolor{currentfill}{rgb}{0.800000,0.200000,0.200000}%
\pgfsetfillcolor{currentfill}%
\pgfsetlinewidth{1.003750pt}%
\definecolor{currentstroke}{rgb}{0.800000,0.200000,0.200000}%
\pgfsetstrokecolor{currentstroke}%
\pgfsetdash{}{0pt}%
\pgfpathmoveto{\pgfqpoint{2.284461in}{4.413822in}}%
\pgfpathcurveto{\pgfqpoint{2.290285in}{4.413822in}}{\pgfqpoint{2.295871in}{4.416136in}}{\pgfqpoint{2.299989in}{4.420254in}}%
\pgfpathcurveto{\pgfqpoint{2.304107in}{4.424372in}}{\pgfqpoint{2.306421in}{4.429958in}}{\pgfqpoint{2.306421in}{4.435782in}}%
\pgfpathcurveto{\pgfqpoint{2.306421in}{4.441606in}}{\pgfqpoint{2.304107in}{4.447192in}}{\pgfqpoint{2.299989in}{4.451311in}}%
\pgfpathcurveto{\pgfqpoint{2.295871in}{4.455429in}}{\pgfqpoint{2.290285in}{4.457743in}}{\pgfqpoint{2.284461in}{4.457743in}}%
\pgfpathcurveto{\pgfqpoint{2.278637in}{4.457743in}}{\pgfqpoint{2.273051in}{4.455429in}}{\pgfqpoint{2.268933in}{4.451311in}}%
\pgfpathcurveto{\pgfqpoint{2.264814in}{4.447192in}}{\pgfqpoint{2.262501in}{4.441606in}}{\pgfqpoint{2.262501in}{4.435782in}}%
\pgfpathcurveto{\pgfqpoint{2.262501in}{4.429958in}}{\pgfqpoint{2.264814in}{4.424372in}}{\pgfqpoint{2.268933in}{4.420254in}}%
\pgfpathcurveto{\pgfqpoint{2.273051in}{4.416136in}}{\pgfqpoint{2.278637in}{4.413822in}}{\pgfqpoint{2.284461in}{4.413822in}}%
\pgfpathlineto{\pgfqpoint{2.284461in}{4.413822in}}%
\pgfpathclose%
\pgfusepath{stroke,fill}%
\end{pgfscope}%
\begin{pgfscope}%
\pgfpathrectangle{\pgfqpoint{1.000000in}{0.979904in}}{\pgfqpoint{6.200000in}{5.960192in}}%
\pgfusepath{clip}%
\pgfsetbuttcap%
\pgfsetroundjoin%
\definecolor{currentfill}{rgb}{0.800000,0.200000,0.200000}%
\pgfsetfillcolor{currentfill}%
\pgfsetlinewidth{1.003750pt}%
\definecolor{currentstroke}{rgb}{0.800000,0.200000,0.200000}%
\pgfsetstrokecolor{currentstroke}%
\pgfsetdash{}{0pt}%
\pgfpathmoveto{\pgfqpoint{2.234601in}{4.274715in}}%
\pgfpathcurveto{\pgfqpoint{2.240425in}{4.274715in}}{\pgfqpoint{2.246011in}{4.277029in}}{\pgfqpoint{2.250130in}{4.281147in}}%
\pgfpathcurveto{\pgfqpoint{2.254248in}{4.285265in}}{\pgfqpoint{2.256562in}{4.290851in}}{\pgfqpoint{2.256562in}{4.296675in}}%
\pgfpathcurveto{\pgfqpoint{2.256562in}{4.302499in}}{\pgfqpoint{2.254248in}{4.308085in}}{\pgfqpoint{2.250130in}{4.312204in}}%
\pgfpathcurveto{\pgfqpoint{2.246011in}{4.316322in}}{\pgfqpoint{2.240425in}{4.318636in}}{\pgfqpoint{2.234601in}{4.318636in}}%
\pgfpathcurveto{\pgfqpoint{2.228777in}{4.318636in}}{\pgfqpoint{2.223191in}{4.316322in}}{\pgfqpoint{2.219073in}{4.312204in}}%
\pgfpathcurveto{\pgfqpoint{2.214955in}{4.308085in}}{\pgfqpoint{2.212641in}{4.302499in}}{\pgfqpoint{2.212641in}{4.296675in}}%
\pgfpathcurveto{\pgfqpoint{2.212641in}{4.290851in}}{\pgfqpoint{2.214955in}{4.285265in}}{\pgfqpoint{2.219073in}{4.281147in}}%
\pgfpathcurveto{\pgfqpoint{2.223191in}{4.277029in}}{\pgfqpoint{2.228777in}{4.274715in}}{\pgfqpoint{2.234601in}{4.274715in}}%
\pgfpathlineto{\pgfqpoint{2.234601in}{4.274715in}}%
\pgfpathclose%
\pgfusepath{stroke,fill}%
\end{pgfscope}%
\begin{pgfscope}%
\pgfpathrectangle{\pgfqpoint{1.000000in}{0.979904in}}{\pgfqpoint{6.200000in}{5.960192in}}%
\pgfusepath{clip}%
\pgfsetbuttcap%
\pgfsetroundjoin%
\definecolor{currentfill}{rgb}{0.800000,0.200000,0.200000}%
\pgfsetfillcolor{currentfill}%
\pgfsetlinewidth{1.003750pt}%
\definecolor{currentstroke}{rgb}{0.800000,0.200000,0.200000}%
\pgfsetstrokecolor{currentstroke}%
\pgfsetdash{}{0pt}%
\pgfpathmoveto{\pgfqpoint{2.146213in}{4.228345in}}%
\pgfpathcurveto{\pgfqpoint{2.152037in}{4.228345in}}{\pgfqpoint{2.157623in}{4.230658in}}{\pgfqpoint{2.161742in}{4.234777in}}%
\pgfpathcurveto{\pgfqpoint{2.165860in}{4.238895in}}{\pgfqpoint{2.168174in}{4.244481in}}{\pgfqpoint{2.168174in}{4.250305in}}%
\pgfpathcurveto{\pgfqpoint{2.168174in}{4.256129in}}{\pgfqpoint{2.165860in}{4.261715in}}{\pgfqpoint{2.161742in}{4.265833in}}%
\pgfpathcurveto{\pgfqpoint{2.157623in}{4.269951in}}{\pgfqpoint{2.152037in}{4.272265in}}{\pgfqpoint{2.146213in}{4.272265in}}%
\pgfpathcurveto{\pgfqpoint{2.140389in}{4.272265in}}{\pgfqpoint{2.134803in}{4.269951in}}{\pgfqpoint{2.130685in}{4.265833in}}%
\pgfpathcurveto{\pgfqpoint{2.126567in}{4.261715in}}{\pgfqpoint{2.124253in}{4.256129in}}{\pgfqpoint{2.124253in}{4.250305in}}%
\pgfpathcurveto{\pgfqpoint{2.124253in}{4.244481in}}{\pgfqpoint{2.126567in}{4.238895in}}{\pgfqpoint{2.130685in}{4.234777in}}%
\pgfpathcurveto{\pgfqpoint{2.134803in}{4.230658in}}{\pgfqpoint{2.140389in}{4.228345in}}{\pgfqpoint{2.146213in}{4.228345in}}%
\pgfpathlineto{\pgfqpoint{2.146213in}{4.228345in}}%
\pgfpathclose%
\pgfusepath{stroke,fill}%
\end{pgfscope}%
\begin{pgfscope}%
\pgfpathrectangle{\pgfqpoint{1.000000in}{0.979904in}}{\pgfqpoint{6.200000in}{5.960192in}}%
\pgfusepath{clip}%
\pgfsetbuttcap%
\pgfsetroundjoin%
\definecolor{currentfill}{rgb}{0.800000,0.200000,0.200000}%
\pgfsetfillcolor{currentfill}%
\pgfsetlinewidth{1.003750pt}%
\definecolor{currentstroke}{rgb}{0.800000,0.200000,0.200000}%
\pgfsetstrokecolor{currentstroke}%
\pgfsetdash{}{0pt}%
\pgfpathmoveto{\pgfqpoint{2.042699in}{4.203785in}}%
\pgfpathcurveto{\pgfqpoint{2.048523in}{4.203785in}}{\pgfqpoint{2.054109in}{4.206099in}}{\pgfqpoint{2.058227in}{4.210217in}}%
\pgfpathcurveto{\pgfqpoint{2.062346in}{4.214335in}}{\pgfqpoint{2.064659in}{4.219922in}}{\pgfqpoint{2.064659in}{4.225745in}}%
\pgfpathcurveto{\pgfqpoint{2.064659in}{4.231569in}}{\pgfqpoint{2.062346in}{4.237156in}}{\pgfqpoint{2.058227in}{4.241274in}}%
\pgfpathcurveto{\pgfqpoint{2.054109in}{4.245392in}}{\pgfqpoint{2.048523in}{4.247706in}}{\pgfqpoint{2.042699in}{4.247706in}}%
\pgfpathcurveto{\pgfqpoint{2.036875in}{4.247706in}}{\pgfqpoint{2.031289in}{4.245392in}}{\pgfqpoint{2.027171in}{4.241274in}}%
\pgfpathcurveto{\pgfqpoint{2.023053in}{4.237156in}}{\pgfqpoint{2.020739in}{4.231569in}}{\pgfqpoint{2.020739in}{4.225745in}}%
\pgfpathcurveto{\pgfqpoint{2.020739in}{4.219922in}}{\pgfqpoint{2.023053in}{4.214335in}}{\pgfqpoint{2.027171in}{4.210217in}}%
\pgfpathcurveto{\pgfqpoint{2.031289in}{4.206099in}}{\pgfqpoint{2.036875in}{4.203785in}}{\pgfqpoint{2.042699in}{4.203785in}}%
\pgfpathlineto{\pgfqpoint{2.042699in}{4.203785in}}%
\pgfpathclose%
\pgfusepath{stroke,fill}%
\end{pgfscope}%
\begin{pgfscope}%
\pgfpathrectangle{\pgfqpoint{1.000000in}{0.979904in}}{\pgfqpoint{6.200000in}{5.960192in}}%
\pgfusepath{clip}%
\pgfsetbuttcap%
\pgfsetroundjoin%
\definecolor{currentfill}{rgb}{0.800000,0.200000,0.200000}%
\pgfsetfillcolor{currentfill}%
\pgfsetlinewidth{1.003750pt}%
\definecolor{currentstroke}{rgb}{0.800000,0.200000,0.200000}%
\pgfsetstrokecolor{currentstroke}%
\pgfsetdash{}{0pt}%
\pgfpathmoveto{\pgfqpoint{2.021645in}{4.063148in}}%
\pgfpathcurveto{\pgfqpoint{2.027469in}{4.063148in}}{\pgfqpoint{2.033055in}{4.065461in}}{\pgfqpoint{2.037173in}{4.069580in}}%
\pgfpathcurveto{\pgfqpoint{2.041292in}{4.073698in}}{\pgfqpoint{2.043605in}{4.079284in}}{\pgfqpoint{2.043605in}{4.085108in}}%
\pgfpathcurveto{\pgfqpoint{2.043605in}{4.090932in}}{\pgfqpoint{2.041292in}{4.096518in}}{\pgfqpoint{2.037173in}{4.100636in}}%
\pgfpathcurveto{\pgfqpoint{2.033055in}{4.104754in}}{\pgfqpoint{2.027469in}{4.107068in}}{\pgfqpoint{2.021645in}{4.107068in}}%
\pgfpathcurveto{\pgfqpoint{2.015821in}{4.107068in}}{\pgfqpoint{2.010235in}{4.104754in}}{\pgfqpoint{2.006117in}{4.100636in}}%
\pgfpathcurveto{\pgfqpoint{2.001999in}{4.096518in}}{\pgfqpoint{1.999685in}{4.090932in}}{\pgfqpoint{1.999685in}{4.085108in}}%
\pgfpathcurveto{\pgfqpoint{1.999685in}{4.079284in}}{\pgfqpoint{2.001999in}{4.073698in}}{\pgfqpoint{2.006117in}{4.069580in}}%
\pgfpathcurveto{\pgfqpoint{2.010235in}{4.065461in}}{\pgfqpoint{2.015821in}{4.063148in}}{\pgfqpoint{2.021645in}{4.063148in}}%
\pgfpathlineto{\pgfqpoint{2.021645in}{4.063148in}}%
\pgfpathclose%
\pgfusepath{stroke,fill}%
\end{pgfscope}%
\begin{pgfscope}%
\pgfpathrectangle{\pgfqpoint{1.000000in}{0.979904in}}{\pgfqpoint{6.200000in}{5.960192in}}%
\pgfusepath{clip}%
\pgfsetbuttcap%
\pgfsetroundjoin%
\definecolor{currentfill}{rgb}{0.800000,0.200000,0.200000}%
\pgfsetfillcolor{currentfill}%
\pgfsetlinewidth{1.003750pt}%
\definecolor{currentstroke}{rgb}{0.800000,0.200000,0.200000}%
\pgfsetstrokecolor{currentstroke}%
\pgfsetdash{}{0pt}%
\pgfpathmoveto{\pgfqpoint{1.858676in}{4.105497in}}%
\pgfpathcurveto{\pgfqpoint{1.864500in}{4.105497in}}{\pgfqpoint{1.870086in}{4.107811in}}{\pgfqpoint{1.874204in}{4.111929in}}%
\pgfpathcurveto{\pgfqpoint{1.878322in}{4.116047in}}{\pgfqpoint{1.880636in}{4.121633in}}{\pgfqpoint{1.880636in}{4.127457in}}%
\pgfpathcurveto{\pgfqpoint{1.880636in}{4.133281in}}{\pgfqpoint{1.878322in}{4.138867in}}{\pgfqpoint{1.874204in}{4.142985in}}%
\pgfpathcurveto{\pgfqpoint{1.870086in}{4.147103in}}{\pgfqpoint{1.864500in}{4.149417in}}{\pgfqpoint{1.858676in}{4.149417in}}%
\pgfpathcurveto{\pgfqpoint{1.852852in}{4.149417in}}{\pgfqpoint{1.847266in}{4.147103in}}{\pgfqpoint{1.843148in}{4.142985in}}%
\pgfpathcurveto{\pgfqpoint{1.839030in}{4.138867in}}{\pgfqpoint{1.836716in}{4.133281in}}{\pgfqpoint{1.836716in}{4.127457in}}%
\pgfpathcurveto{\pgfqpoint{1.836716in}{4.121633in}}{\pgfqpoint{1.839030in}{4.116047in}}{\pgfqpoint{1.843148in}{4.111929in}}%
\pgfpathcurveto{\pgfqpoint{1.847266in}{4.107811in}}{\pgfqpoint{1.852852in}{4.105497in}}{\pgfqpoint{1.858676in}{4.105497in}}%
\pgfpathlineto{\pgfqpoint{1.858676in}{4.105497in}}%
\pgfpathclose%
\pgfusepath{stroke,fill}%
\end{pgfscope}%
\begin{pgfscope}%
\pgfpathrectangle{\pgfqpoint{1.000000in}{0.979904in}}{\pgfqpoint{6.200000in}{5.960192in}}%
\pgfusepath{clip}%
\pgfsetbuttcap%
\pgfsetroundjoin%
\definecolor{currentfill}{rgb}{0.800000,0.200000,0.200000}%
\pgfsetfillcolor{currentfill}%
\pgfsetlinewidth{1.003750pt}%
\definecolor{currentstroke}{rgb}{0.800000,0.200000,0.200000}%
\pgfsetstrokecolor{currentstroke}%
\pgfsetdash{}{0pt}%
\pgfpathmoveto{\pgfqpoint{1.770265in}{4.047010in}}%
\pgfpathcurveto{\pgfqpoint{1.776088in}{4.047010in}}{\pgfqpoint{1.781675in}{4.049324in}}{\pgfqpoint{1.785793in}{4.053442in}}%
\pgfpathcurveto{\pgfqpoint{1.789911in}{4.057560in}}{\pgfqpoint{1.792225in}{4.063146in}}{\pgfqpoint{1.792225in}{4.068970in}}%
\pgfpathcurveto{\pgfqpoint{1.792225in}{4.074794in}}{\pgfqpoint{1.789911in}{4.080380in}}{\pgfqpoint{1.785793in}{4.084498in}}%
\pgfpathcurveto{\pgfqpoint{1.781675in}{4.088617in}}{\pgfqpoint{1.776088in}{4.090930in}}{\pgfqpoint{1.770265in}{4.090930in}}%
\pgfpathcurveto{\pgfqpoint{1.764441in}{4.090930in}}{\pgfqpoint{1.758854in}{4.088617in}}{\pgfqpoint{1.754736in}{4.084498in}}%
\pgfpathcurveto{\pgfqpoint{1.750618in}{4.080380in}}{\pgfqpoint{1.748304in}{4.074794in}}{\pgfqpoint{1.748304in}{4.068970in}}%
\pgfpathcurveto{\pgfqpoint{1.748304in}{4.063146in}}{\pgfqpoint{1.750618in}{4.057560in}}{\pgfqpoint{1.754736in}{4.053442in}}%
\pgfpathcurveto{\pgfqpoint{1.758854in}{4.049324in}}{\pgfqpoint{1.764441in}{4.047010in}}{\pgfqpoint{1.770265in}{4.047010in}}%
\pgfpathlineto{\pgfqpoint{1.770265in}{4.047010in}}%
\pgfpathclose%
\pgfusepath{stroke,fill}%
\end{pgfscope}%
\begin{pgfscope}%
\pgfpathrectangle{\pgfqpoint{1.000000in}{0.979904in}}{\pgfqpoint{6.200000in}{5.960192in}}%
\pgfusepath{clip}%
\pgfsetbuttcap%
\pgfsetroundjoin%
\definecolor{currentfill}{rgb}{0.800000,0.200000,0.200000}%
\pgfsetfillcolor{currentfill}%
\pgfsetlinewidth{1.003750pt}%
\definecolor{currentstroke}{rgb}{0.800000,0.200000,0.200000}%
\pgfsetstrokecolor{currentstroke}%
\pgfsetdash{}{0pt}%
\pgfpathmoveto{\pgfqpoint{1.732845in}{3.939441in}}%
\pgfpathcurveto{\pgfqpoint{1.738669in}{3.939441in}}{\pgfqpoint{1.744255in}{3.941754in}}{\pgfqpoint{1.748373in}{3.945873in}}%
\pgfpathcurveto{\pgfqpoint{1.752491in}{3.949991in}}{\pgfqpoint{1.754805in}{3.955577in}}{\pgfqpoint{1.754805in}{3.961401in}}%
\pgfpathcurveto{\pgfqpoint{1.754805in}{3.967225in}}{\pgfqpoint{1.752491in}{3.972811in}}{\pgfqpoint{1.748373in}{3.976929in}}%
\pgfpathcurveto{\pgfqpoint{1.744255in}{3.981047in}}{\pgfqpoint{1.738669in}{3.983361in}}{\pgfqpoint{1.732845in}{3.983361in}}%
\pgfpathcurveto{\pgfqpoint{1.727021in}{3.983361in}}{\pgfqpoint{1.721435in}{3.981047in}}{\pgfqpoint{1.717317in}{3.976929in}}%
\pgfpathcurveto{\pgfqpoint{1.713199in}{3.972811in}}{\pgfqpoint{1.710885in}{3.967225in}}{\pgfqpoint{1.710885in}{3.961401in}}%
\pgfpathcurveto{\pgfqpoint{1.710885in}{3.955577in}}{\pgfqpoint{1.713199in}{3.949991in}}{\pgfqpoint{1.717317in}{3.945873in}}%
\pgfpathcurveto{\pgfqpoint{1.721435in}{3.941754in}}{\pgfqpoint{1.727021in}{3.939441in}}{\pgfqpoint{1.732845in}{3.939441in}}%
\pgfpathlineto{\pgfqpoint{1.732845in}{3.939441in}}%
\pgfpathclose%
\pgfusepath{stroke,fill}%
\end{pgfscope}%
\begin{pgfscope}%
\pgfpathrectangle{\pgfqpoint{1.000000in}{0.979904in}}{\pgfqpoint{6.200000in}{5.960192in}}%
\pgfusepath{clip}%
\pgfsetbuttcap%
\pgfsetroundjoin%
\definecolor{currentfill}{rgb}{0.800000,0.200000,0.200000}%
\pgfsetfillcolor{currentfill}%
\pgfsetlinewidth{1.003750pt}%
\definecolor{currentstroke}{rgb}{0.800000,0.200000,0.200000}%
\pgfsetstrokecolor{currentstroke}%
\pgfsetdash{}{0pt}%
\pgfpathmoveto{\pgfqpoint{1.614032in}{3.902954in}}%
\pgfpathcurveto{\pgfqpoint{1.619856in}{3.902954in}}{\pgfqpoint{1.625442in}{3.905268in}}{\pgfqpoint{1.629560in}{3.909386in}}%
\pgfpathcurveto{\pgfqpoint{1.633678in}{3.913504in}}{\pgfqpoint{1.635992in}{3.919090in}}{\pgfqpoint{1.635992in}{3.924914in}}%
\pgfpathcurveto{\pgfqpoint{1.635992in}{3.930738in}}{\pgfqpoint{1.633678in}{3.936324in}}{\pgfqpoint{1.629560in}{3.940442in}}%
\pgfpathcurveto{\pgfqpoint{1.625442in}{3.944561in}}{\pgfqpoint{1.619856in}{3.946874in}}{\pgfqpoint{1.614032in}{3.946874in}}%
\pgfpathcurveto{\pgfqpoint{1.608208in}{3.946874in}}{\pgfqpoint{1.602622in}{3.944561in}}{\pgfqpoint{1.598504in}{3.940442in}}%
\pgfpathcurveto{\pgfqpoint{1.594385in}{3.936324in}}{\pgfqpoint{1.592072in}{3.930738in}}{\pgfqpoint{1.592072in}{3.924914in}}%
\pgfpathcurveto{\pgfqpoint{1.592072in}{3.919090in}}{\pgfqpoint{1.594385in}{3.913504in}}{\pgfqpoint{1.598504in}{3.909386in}}%
\pgfpathcurveto{\pgfqpoint{1.602622in}{3.905268in}}{\pgfqpoint{1.608208in}{3.902954in}}{\pgfqpoint{1.614032in}{3.902954in}}%
\pgfpathlineto{\pgfqpoint{1.614032in}{3.902954in}}%
\pgfpathclose%
\pgfusepath{stroke,fill}%
\end{pgfscope}%
\begin{pgfscope}%
\pgfpathrectangle{\pgfqpoint{1.000000in}{0.979904in}}{\pgfqpoint{6.200000in}{5.960192in}}%
\pgfusepath{clip}%
\pgfsetbuttcap%
\pgfsetroundjoin%
\definecolor{currentfill}{rgb}{0.800000,0.200000,0.200000}%
\pgfsetfillcolor{currentfill}%
\pgfsetlinewidth{1.003750pt}%
\definecolor{currentstroke}{rgb}{0.800000,0.200000,0.200000}%
\pgfsetstrokecolor{currentstroke}%
\pgfsetdash{}{0pt}%
\pgfpathmoveto{\pgfqpoint{1.681058in}{3.727617in}}%
\pgfpathcurveto{\pgfqpoint{1.686882in}{3.727617in}}{\pgfqpoint{1.692468in}{3.729931in}}{\pgfqpoint{1.696586in}{3.734049in}}%
\pgfpathcurveto{\pgfqpoint{1.700704in}{3.738167in}}{\pgfqpoint{1.703018in}{3.743753in}}{\pgfqpoint{1.703018in}{3.749577in}}%
\pgfpathcurveto{\pgfqpoint{1.703018in}{3.755401in}}{\pgfqpoint{1.700704in}{3.760987in}}{\pgfqpoint{1.696586in}{3.765105in}}%
\pgfpathcurveto{\pgfqpoint{1.692468in}{3.769224in}}{\pgfqpoint{1.686882in}{3.771538in}}{\pgfqpoint{1.681058in}{3.771538in}}%
\pgfpathcurveto{\pgfqpoint{1.675234in}{3.771538in}}{\pgfqpoint{1.669648in}{3.769224in}}{\pgfqpoint{1.665530in}{3.765105in}}%
\pgfpathcurveto{\pgfqpoint{1.661412in}{3.760987in}}{\pgfqpoint{1.659098in}{3.755401in}}{\pgfqpoint{1.659098in}{3.749577in}}%
\pgfpathcurveto{\pgfqpoint{1.659098in}{3.743753in}}{\pgfqpoint{1.661412in}{3.738167in}}{\pgfqpoint{1.665530in}{3.734049in}}%
\pgfpathcurveto{\pgfqpoint{1.669648in}{3.729931in}}{\pgfqpoint{1.675234in}{3.727617in}}{\pgfqpoint{1.681058in}{3.727617in}}%
\pgfpathlineto{\pgfqpoint{1.681058in}{3.727617in}}%
\pgfpathclose%
\pgfusepath{stroke,fill}%
\end{pgfscope}%
\begin{pgfscope}%
\pgfpathrectangle{\pgfqpoint{1.000000in}{0.979904in}}{\pgfqpoint{6.200000in}{5.960192in}}%
\pgfusepath{clip}%
\pgfsetbuttcap%
\pgfsetroundjoin%
\definecolor{currentfill}{rgb}{0.800000,0.200000,0.200000}%
\pgfsetfillcolor{currentfill}%
\pgfsetlinewidth{1.003750pt}%
\definecolor{currentstroke}{rgb}{0.800000,0.200000,0.200000}%
\pgfsetstrokecolor{currentstroke}%
\pgfsetdash{}{0pt}%
\pgfpathmoveto{\pgfqpoint{1.484590in}{3.732897in}}%
\pgfpathcurveto{\pgfqpoint{1.490414in}{3.732897in}}{\pgfqpoint{1.496000in}{3.735211in}}{\pgfqpoint{1.500118in}{3.739329in}}%
\pgfpathcurveto{\pgfqpoint{1.504236in}{3.743447in}}{\pgfqpoint{1.506550in}{3.749033in}}{\pgfqpoint{1.506550in}{3.754857in}}%
\pgfpathcurveto{\pgfqpoint{1.506550in}{3.760681in}}{\pgfqpoint{1.504236in}{3.766267in}}{\pgfqpoint{1.500118in}{3.770385in}}%
\pgfpathcurveto{\pgfqpoint{1.496000in}{3.774503in}}{\pgfqpoint{1.490414in}{3.776817in}}{\pgfqpoint{1.484590in}{3.776817in}}%
\pgfpathcurveto{\pgfqpoint{1.478766in}{3.776817in}}{\pgfqpoint{1.473180in}{3.774503in}}{\pgfqpoint{1.469062in}{3.770385in}}%
\pgfpathcurveto{\pgfqpoint{1.464943in}{3.766267in}}{\pgfqpoint{1.462630in}{3.760681in}}{\pgfqpoint{1.462630in}{3.754857in}}%
\pgfpathcurveto{\pgfqpoint{1.462630in}{3.749033in}}{\pgfqpoint{1.464943in}{3.743447in}}{\pgfqpoint{1.469062in}{3.739329in}}%
\pgfpathcurveto{\pgfqpoint{1.473180in}{3.735211in}}{\pgfqpoint{1.478766in}{3.732897in}}{\pgfqpoint{1.484590in}{3.732897in}}%
\pgfpathlineto{\pgfqpoint{1.484590in}{3.732897in}}%
\pgfpathclose%
\pgfusepath{stroke,fill}%
\end{pgfscope}%
\begin{pgfscope}%
\pgfpathrectangle{\pgfqpoint{1.000000in}{0.979904in}}{\pgfqpoint{6.200000in}{5.960192in}}%
\pgfusepath{clip}%
\pgfsetbuttcap%
\pgfsetroundjoin%
\definecolor{currentfill}{rgb}{0.800000,0.200000,0.200000}%
\pgfsetfillcolor{currentfill}%
\pgfsetlinewidth{1.003750pt}%
\definecolor{currentstroke}{rgb}{0.800000,0.200000,0.200000}%
\pgfsetstrokecolor{currentstroke}%
\pgfsetdash{}{0pt}%
\pgfpathmoveto{\pgfqpoint{1.553919in}{3.576705in}}%
\pgfpathcurveto{\pgfqpoint{1.559743in}{3.576705in}}{\pgfqpoint{1.565329in}{3.579019in}}{\pgfqpoint{1.569447in}{3.583137in}}%
\pgfpathcurveto{\pgfqpoint{1.573565in}{3.587255in}}{\pgfqpoint{1.575879in}{3.592841in}}{\pgfqpoint{1.575879in}{3.598665in}}%
\pgfpathcurveto{\pgfqpoint{1.575879in}{3.604489in}}{\pgfqpoint{1.573565in}{3.610075in}}{\pgfqpoint{1.569447in}{3.614193in}}%
\pgfpathcurveto{\pgfqpoint{1.565329in}{3.618311in}}{\pgfqpoint{1.559743in}{3.620625in}}{\pgfqpoint{1.553919in}{3.620625in}}%
\pgfpathcurveto{\pgfqpoint{1.548095in}{3.620625in}}{\pgfqpoint{1.542509in}{3.618311in}}{\pgfqpoint{1.538391in}{3.614193in}}%
\pgfpathcurveto{\pgfqpoint{1.534273in}{3.610075in}}{\pgfqpoint{1.531959in}{3.604489in}}{\pgfqpoint{1.531959in}{3.598665in}}%
\pgfpathcurveto{\pgfqpoint{1.531959in}{3.592841in}}{\pgfqpoint{1.534273in}{3.587255in}}{\pgfqpoint{1.538391in}{3.583137in}}%
\pgfpathcurveto{\pgfqpoint{1.542509in}{3.579019in}}{\pgfqpoint{1.548095in}{3.576705in}}{\pgfqpoint{1.553919in}{3.576705in}}%
\pgfpathlineto{\pgfqpoint{1.553919in}{3.576705in}}%
\pgfpathclose%
\pgfusepath{stroke,fill}%
\end{pgfscope}%
\begin{pgfscope}%
\pgfpathrectangle{\pgfqpoint{1.000000in}{0.979904in}}{\pgfqpoint{6.200000in}{5.960192in}}%
\pgfusepath{clip}%
\pgfsetbuttcap%
\pgfsetroundjoin%
\definecolor{currentfill}{rgb}{0.800000,0.200000,0.200000}%
\pgfsetfillcolor{currentfill}%
\pgfsetlinewidth{1.003750pt}%
\definecolor{currentstroke}{rgb}{0.800000,0.200000,0.200000}%
\pgfsetstrokecolor{currentstroke}%
\pgfsetdash{}{0pt}%
\pgfpathmoveto{\pgfqpoint{1.538300in}{3.476132in}}%
\pgfpathcurveto{\pgfqpoint{1.544124in}{3.476132in}}{\pgfqpoint{1.549710in}{3.478446in}}{\pgfqpoint{1.553829in}{3.482564in}}%
\pgfpathcurveto{\pgfqpoint{1.557947in}{3.486683in}}{\pgfqpoint{1.560261in}{3.492269in}}{\pgfqpoint{1.560261in}{3.498093in}}%
\pgfpathcurveto{\pgfqpoint{1.560261in}{3.503917in}}{\pgfqpoint{1.557947in}{3.509503in}}{\pgfqpoint{1.553829in}{3.513621in}}%
\pgfpathcurveto{\pgfqpoint{1.549710in}{3.517739in}}{\pgfqpoint{1.544124in}{3.520053in}}{\pgfqpoint{1.538300in}{3.520053in}}%
\pgfpathcurveto{\pgfqpoint{1.532476in}{3.520053in}}{\pgfqpoint{1.526890in}{3.517739in}}{\pgfqpoint{1.522772in}{3.513621in}}%
\pgfpathcurveto{\pgfqpoint{1.518654in}{3.509503in}}{\pgfqpoint{1.516340in}{3.503917in}}{\pgfqpoint{1.516340in}{3.498093in}}%
\pgfpathcurveto{\pgfqpoint{1.516340in}{3.492269in}}{\pgfqpoint{1.518654in}{3.486683in}}{\pgfqpoint{1.522772in}{3.482564in}}%
\pgfpathcurveto{\pgfqpoint{1.526890in}{3.478446in}}{\pgfqpoint{1.532476in}{3.476132in}}{\pgfqpoint{1.538300in}{3.476132in}}%
\pgfpathlineto{\pgfqpoint{1.538300in}{3.476132in}}%
\pgfpathclose%
\pgfusepath{stroke,fill}%
\end{pgfscope}%
\begin{pgfscope}%
\pgfpathrectangle{\pgfqpoint{1.000000in}{0.979904in}}{\pgfqpoint{6.200000in}{5.960192in}}%
\pgfusepath{clip}%
\pgfsetbuttcap%
\pgfsetroundjoin%
\definecolor{currentfill}{rgb}{0.800000,0.200000,0.200000}%
\pgfsetfillcolor{currentfill}%
\pgfsetlinewidth{1.003750pt}%
\definecolor{currentstroke}{rgb}{0.800000,0.200000,0.200000}%
\pgfsetstrokecolor{currentstroke}%
\pgfsetdash{}{0pt}%
\pgfpathmoveto{\pgfqpoint{1.450421in}{3.405372in}}%
\pgfpathcurveto{\pgfqpoint{1.456245in}{3.405372in}}{\pgfqpoint{1.461831in}{3.407686in}}{\pgfqpoint{1.465949in}{3.411804in}}%
\pgfpathcurveto{\pgfqpoint{1.470068in}{3.415922in}}{\pgfqpoint{1.472381in}{3.421509in}}{\pgfqpoint{1.472381in}{3.427332in}}%
\pgfpathcurveto{\pgfqpoint{1.472381in}{3.433156in}}{\pgfqpoint{1.470068in}{3.438743in}}{\pgfqpoint{1.465949in}{3.442861in}}%
\pgfpathcurveto{\pgfqpoint{1.461831in}{3.446979in}}{\pgfqpoint{1.456245in}{3.449293in}}{\pgfqpoint{1.450421in}{3.449293in}}%
\pgfpathcurveto{\pgfqpoint{1.444597in}{3.449293in}}{\pgfqpoint{1.439011in}{3.446979in}}{\pgfqpoint{1.434893in}{3.442861in}}%
\pgfpathcurveto{\pgfqpoint{1.430775in}{3.438743in}}{\pgfqpoint{1.428461in}{3.433156in}}{\pgfqpoint{1.428461in}{3.427332in}}%
\pgfpathcurveto{\pgfqpoint{1.428461in}{3.421509in}}{\pgfqpoint{1.430775in}{3.415922in}}{\pgfqpoint{1.434893in}{3.411804in}}%
\pgfpathcurveto{\pgfqpoint{1.439011in}{3.407686in}}{\pgfqpoint{1.444597in}{3.405372in}}{\pgfqpoint{1.450421in}{3.405372in}}%
\pgfpathlineto{\pgfqpoint{1.450421in}{3.405372in}}%
\pgfpathclose%
\pgfusepath{stroke,fill}%
\end{pgfscope}%
\begin{pgfscope}%
\pgfpathrectangle{\pgfqpoint{1.000000in}{0.979904in}}{\pgfqpoint{6.200000in}{5.960192in}}%
\pgfusepath{clip}%
\pgfsetbuttcap%
\pgfsetroundjoin%
\definecolor{currentfill}{rgb}{0.800000,0.200000,0.200000}%
\pgfsetfillcolor{currentfill}%
\pgfsetlinewidth{1.003750pt}%
\definecolor{currentstroke}{rgb}{0.800000,0.200000,0.200000}%
\pgfsetstrokecolor{currentstroke}%
\pgfsetdash{}{0pt}%
\pgfpathmoveto{\pgfqpoint{1.311800in}{3.341794in}}%
\pgfpathcurveto{\pgfqpoint{1.317624in}{3.341794in}}{\pgfqpoint{1.323210in}{3.344107in}}{\pgfqpoint{1.327328in}{3.348226in}}%
\pgfpathcurveto{\pgfqpoint{1.331446in}{3.352344in}}{\pgfqpoint{1.333760in}{3.357930in}}{\pgfqpoint{1.333760in}{3.363754in}}%
\pgfpathcurveto{\pgfqpoint{1.333760in}{3.369578in}}{\pgfqpoint{1.331446in}{3.375164in}}{\pgfqpoint{1.327328in}{3.379282in}}%
\pgfpathcurveto{\pgfqpoint{1.323210in}{3.383400in}}{\pgfqpoint{1.317624in}{3.385714in}}{\pgfqpoint{1.311800in}{3.385714in}}%
\pgfpathcurveto{\pgfqpoint{1.305976in}{3.385714in}}{\pgfqpoint{1.300390in}{3.383400in}}{\pgfqpoint{1.296272in}{3.379282in}}%
\pgfpathcurveto{\pgfqpoint{1.292154in}{3.375164in}}{\pgfqpoint{1.289840in}{3.369578in}}{\pgfqpoint{1.289840in}{3.363754in}}%
\pgfpathcurveto{\pgfqpoint{1.289840in}{3.357930in}}{\pgfqpoint{1.292154in}{3.352344in}}{\pgfqpoint{1.296272in}{3.348226in}}%
\pgfpathcurveto{\pgfqpoint{1.300390in}{3.344107in}}{\pgfqpoint{1.305976in}{3.341794in}}{\pgfqpoint{1.311800in}{3.341794in}}%
\pgfpathlineto{\pgfqpoint{1.311800in}{3.341794in}}%
\pgfpathclose%
\pgfusepath{stroke,fill}%
\end{pgfscope}%
\begin{pgfscope}%
\pgfpathrectangle{\pgfqpoint{1.000000in}{0.979904in}}{\pgfqpoint{6.200000in}{5.960192in}}%
\pgfusepath{clip}%
\pgfsetbuttcap%
\pgfsetroundjoin%
\definecolor{currentfill}{rgb}{0.800000,0.200000,0.200000}%
\pgfsetfillcolor{currentfill}%
\pgfsetlinewidth{1.003750pt}%
\definecolor{currentstroke}{rgb}{0.800000,0.200000,0.200000}%
\pgfsetstrokecolor{currentstroke}%
\pgfsetdash{}{0pt}%
\pgfpathmoveto{\pgfqpoint{1.382763in}{3.215911in}}%
\pgfpathcurveto{\pgfqpoint{1.388587in}{3.215911in}}{\pgfqpoint{1.394173in}{3.218225in}}{\pgfqpoint{1.398291in}{3.222343in}}%
\pgfpathcurveto{\pgfqpoint{1.402409in}{3.226461in}}{\pgfqpoint{1.404723in}{3.232047in}}{\pgfqpoint{1.404723in}{3.237871in}}%
\pgfpathcurveto{\pgfqpoint{1.404723in}{3.243695in}}{\pgfqpoint{1.402409in}{3.249281in}}{\pgfqpoint{1.398291in}{3.253400in}}%
\pgfpathcurveto{\pgfqpoint{1.394173in}{3.257518in}}{\pgfqpoint{1.388587in}{3.259832in}}{\pgfqpoint{1.382763in}{3.259832in}}%
\pgfpathcurveto{\pgfqpoint{1.376939in}{3.259832in}}{\pgfqpoint{1.371353in}{3.257518in}}{\pgfqpoint{1.367235in}{3.253400in}}%
\pgfpathcurveto{\pgfqpoint{1.363116in}{3.249281in}}{\pgfqpoint{1.360803in}{3.243695in}}{\pgfqpoint{1.360803in}{3.237871in}}%
\pgfpathcurveto{\pgfqpoint{1.360803in}{3.232047in}}{\pgfqpoint{1.363116in}{3.226461in}}{\pgfqpoint{1.367235in}{3.222343in}}%
\pgfpathcurveto{\pgfqpoint{1.371353in}{3.218225in}}{\pgfqpoint{1.376939in}{3.215911in}}{\pgfqpoint{1.382763in}{3.215911in}}%
\pgfpathlineto{\pgfqpoint{1.382763in}{3.215911in}}%
\pgfpathclose%
\pgfusepath{stroke,fill}%
\end{pgfscope}%
\begin{pgfscope}%
\pgfpathrectangle{\pgfqpoint{1.000000in}{0.979904in}}{\pgfqpoint{6.200000in}{5.960192in}}%
\pgfusepath{clip}%
\pgfsetbuttcap%
\pgfsetroundjoin%
\definecolor{currentfill}{rgb}{0.800000,0.200000,0.200000}%
\pgfsetfillcolor{currentfill}%
\pgfsetlinewidth{1.003750pt}%
\definecolor{currentstroke}{rgb}{0.800000,0.200000,0.200000}%
\pgfsetstrokecolor{currentstroke}%
\pgfsetdash{}{0pt}%
\pgfpathmoveto{\pgfqpoint{1.411518in}{3.109237in}}%
\pgfpathcurveto{\pgfqpoint{1.417342in}{3.109237in}}{\pgfqpoint{1.422929in}{3.111551in}}{\pgfqpoint{1.427047in}{3.115669in}}%
\pgfpathcurveto{\pgfqpoint{1.431165in}{3.119787in}}{\pgfqpoint{1.433479in}{3.125373in}}{\pgfqpoint{1.433479in}{3.131197in}}%
\pgfpathcurveto{\pgfqpoint{1.433479in}{3.137021in}}{\pgfqpoint{1.431165in}{3.142607in}}{\pgfqpoint{1.427047in}{3.146725in}}%
\pgfpathcurveto{\pgfqpoint{1.422929in}{3.150844in}}{\pgfqpoint{1.417342in}{3.153157in}}{\pgfqpoint{1.411518in}{3.153157in}}%
\pgfpathcurveto{\pgfqpoint{1.405695in}{3.153157in}}{\pgfqpoint{1.400108in}{3.150844in}}{\pgfqpoint{1.395990in}{3.146725in}}%
\pgfpathcurveto{\pgfqpoint{1.391872in}{3.142607in}}{\pgfqpoint{1.389558in}{3.137021in}}{\pgfqpoint{1.389558in}{3.131197in}}%
\pgfpathcurveto{\pgfqpoint{1.389558in}{3.125373in}}{\pgfqpoint{1.391872in}{3.119787in}}{\pgfqpoint{1.395990in}{3.115669in}}%
\pgfpathcurveto{\pgfqpoint{1.400108in}{3.111551in}}{\pgfqpoint{1.405695in}{3.109237in}}{\pgfqpoint{1.411518in}{3.109237in}}%
\pgfpathlineto{\pgfqpoint{1.411518in}{3.109237in}}%
\pgfpathclose%
\pgfusepath{stroke,fill}%
\end{pgfscope}%
\begin{pgfscope}%
\pgfpathrectangle{\pgfqpoint{1.000000in}{0.979904in}}{\pgfqpoint{6.200000in}{5.960192in}}%
\pgfusepath{clip}%
\pgfsetbuttcap%
\pgfsetroundjoin%
\definecolor{currentfill}{rgb}{0.800000,0.200000,0.200000}%
\pgfsetfillcolor{currentfill}%
\pgfsetlinewidth{1.003750pt}%
\definecolor{currentstroke}{rgb}{0.800000,0.200000,0.200000}%
\pgfsetstrokecolor{currentstroke}%
\pgfsetdash{}{0pt}%
\pgfpathmoveto{\pgfqpoint{1.418280in}{3.010429in}}%
\pgfpathcurveto{\pgfqpoint{1.424104in}{3.010429in}}{\pgfqpoint{1.429690in}{3.012743in}}{\pgfqpoint{1.433808in}{3.016861in}}%
\pgfpathcurveto{\pgfqpoint{1.437926in}{3.020979in}}{\pgfqpoint{1.440240in}{3.026565in}}{\pgfqpoint{1.440240in}{3.032389in}}%
\pgfpathcurveto{\pgfqpoint{1.440240in}{3.038213in}}{\pgfqpoint{1.437926in}{3.043799in}}{\pgfqpoint{1.433808in}{3.047917in}}%
\pgfpathcurveto{\pgfqpoint{1.429690in}{3.052035in}}{\pgfqpoint{1.424104in}{3.054349in}}{\pgfqpoint{1.418280in}{3.054349in}}%
\pgfpathcurveto{\pgfqpoint{1.412456in}{3.054349in}}{\pgfqpoint{1.406870in}{3.052035in}}{\pgfqpoint{1.402752in}{3.047917in}}%
\pgfpathcurveto{\pgfqpoint{1.398634in}{3.043799in}}{\pgfqpoint{1.396320in}{3.038213in}}{\pgfqpoint{1.396320in}{3.032389in}}%
\pgfpathcurveto{\pgfqpoint{1.396320in}{3.026565in}}{\pgfqpoint{1.398634in}{3.020979in}}{\pgfqpoint{1.402752in}{3.016861in}}%
\pgfpathcurveto{\pgfqpoint{1.406870in}{3.012743in}}{\pgfqpoint{1.412456in}{3.010429in}}{\pgfqpoint{1.418280in}{3.010429in}}%
\pgfpathlineto{\pgfqpoint{1.418280in}{3.010429in}}%
\pgfpathclose%
\pgfusepath{stroke,fill}%
\end{pgfscope}%
\begin{pgfscope}%
\pgfpathrectangle{\pgfqpoint{1.000000in}{0.979904in}}{\pgfqpoint{6.200000in}{5.960192in}}%
\pgfusepath{clip}%
\pgfsetbuttcap%
\pgfsetroundjoin%
\definecolor{currentfill}{rgb}{0.800000,0.200000,0.200000}%
\pgfsetfillcolor{currentfill}%
\pgfsetlinewidth{1.003750pt}%
\definecolor{currentstroke}{rgb}{0.800000,0.200000,0.200000}%
\pgfsetstrokecolor{currentstroke}%
\pgfsetdash{}{0pt}%
\pgfpathmoveto{\pgfqpoint{1.349060in}{2.916079in}}%
\pgfpathcurveto{\pgfqpoint{1.354884in}{2.916079in}}{\pgfqpoint{1.360470in}{2.918393in}}{\pgfqpoint{1.364589in}{2.922511in}}%
\pgfpathcurveto{\pgfqpoint{1.368707in}{2.926629in}}{\pgfqpoint{1.371021in}{2.932215in}}{\pgfqpoint{1.371021in}{2.938039in}}%
\pgfpathcurveto{\pgfqpoint{1.371021in}{2.943863in}}{\pgfqpoint{1.368707in}{2.949449in}}{\pgfqpoint{1.364589in}{2.953567in}}%
\pgfpathcurveto{\pgfqpoint{1.360470in}{2.957685in}}{\pgfqpoint{1.354884in}{2.959999in}}{\pgfqpoint{1.349060in}{2.959999in}}%
\pgfpathcurveto{\pgfqpoint{1.343236in}{2.959999in}}{\pgfqpoint{1.337650in}{2.957685in}}{\pgfqpoint{1.333532in}{2.953567in}}%
\pgfpathcurveto{\pgfqpoint{1.329414in}{2.949449in}}{\pgfqpoint{1.327100in}{2.943863in}}{\pgfqpoint{1.327100in}{2.938039in}}%
\pgfpathcurveto{\pgfqpoint{1.327100in}{2.932215in}}{\pgfqpoint{1.329414in}{2.926629in}}{\pgfqpoint{1.333532in}{2.922511in}}%
\pgfpathcurveto{\pgfqpoint{1.337650in}{2.918393in}}{\pgfqpoint{1.343236in}{2.916079in}}{\pgfqpoint{1.349060in}{2.916079in}}%
\pgfpathlineto{\pgfqpoint{1.349060in}{2.916079in}}%
\pgfpathclose%
\pgfusepath{stroke,fill}%
\end{pgfscope}%
\begin{pgfscope}%
\pgfpathrectangle{\pgfqpoint{1.000000in}{0.979904in}}{\pgfqpoint{6.200000in}{5.960192in}}%
\pgfusepath{clip}%
\pgfsetbuttcap%
\pgfsetroundjoin%
\definecolor{currentfill}{rgb}{0.800000,0.200000,0.200000}%
\pgfsetfillcolor{currentfill}%
\pgfsetlinewidth{1.003750pt}%
\definecolor{currentstroke}{rgb}{0.800000,0.200000,0.200000}%
\pgfsetstrokecolor{currentstroke}%
\pgfsetdash{}{0pt}%
\pgfpathmoveto{\pgfqpoint{1.449279in}{2.818707in}}%
\pgfpathcurveto{\pgfqpoint{1.455102in}{2.818707in}}{\pgfqpoint{1.460689in}{2.821021in}}{\pgfqpoint{1.464807in}{2.825139in}}%
\pgfpathcurveto{\pgfqpoint{1.468925in}{2.829257in}}{\pgfqpoint{1.471239in}{2.834843in}}{\pgfqpoint{1.471239in}{2.840667in}}%
\pgfpathcurveto{\pgfqpoint{1.471239in}{2.846491in}}{\pgfqpoint{1.468925in}{2.852077in}}{\pgfqpoint{1.464807in}{2.856196in}}%
\pgfpathcurveto{\pgfqpoint{1.460689in}{2.860314in}}{\pgfqpoint{1.455102in}{2.862628in}}{\pgfqpoint{1.449279in}{2.862628in}}%
\pgfpathcurveto{\pgfqpoint{1.443455in}{2.862628in}}{\pgfqpoint{1.437868in}{2.860314in}}{\pgfqpoint{1.433750in}{2.856196in}}%
\pgfpathcurveto{\pgfqpoint{1.429632in}{2.852077in}}{\pgfqpoint{1.427318in}{2.846491in}}{\pgfqpoint{1.427318in}{2.840667in}}%
\pgfpathcurveto{\pgfqpoint{1.427318in}{2.834843in}}{\pgfqpoint{1.429632in}{2.829257in}}{\pgfqpoint{1.433750in}{2.825139in}}%
\pgfpathcurveto{\pgfqpoint{1.437868in}{2.821021in}}{\pgfqpoint{1.443455in}{2.818707in}}{\pgfqpoint{1.449279in}{2.818707in}}%
\pgfpathlineto{\pgfqpoint{1.449279in}{2.818707in}}%
\pgfpathclose%
\pgfusepath{stroke,fill}%
\end{pgfscope}%
\begin{pgfscope}%
\pgfpathrectangle{\pgfqpoint{1.000000in}{0.979904in}}{\pgfqpoint{6.200000in}{5.960192in}}%
\pgfusepath{clip}%
\pgfsetbuttcap%
\pgfsetroundjoin%
\definecolor{currentfill}{rgb}{0.800000,0.200000,0.200000}%
\pgfsetfillcolor{currentfill}%
\pgfsetlinewidth{1.003750pt}%
\definecolor{currentstroke}{rgb}{0.800000,0.200000,0.200000}%
\pgfsetstrokecolor{currentstroke}%
\pgfsetdash{}{0pt}%
\pgfpathmoveto{\pgfqpoint{1.418557in}{2.721202in}}%
\pgfpathcurveto{\pgfqpoint{1.424381in}{2.721202in}}{\pgfqpoint{1.429967in}{2.723516in}}{\pgfqpoint{1.434086in}{2.727634in}}%
\pgfpathcurveto{\pgfqpoint{1.438204in}{2.731752in}}{\pgfqpoint{1.440518in}{2.737339in}}{\pgfqpoint{1.440518in}{2.743162in}}%
\pgfpathcurveto{\pgfqpoint{1.440518in}{2.748986in}}{\pgfqpoint{1.438204in}{2.754573in}}{\pgfqpoint{1.434086in}{2.758691in}}%
\pgfpathcurveto{\pgfqpoint{1.429967in}{2.762809in}}{\pgfqpoint{1.424381in}{2.765123in}}{\pgfqpoint{1.418557in}{2.765123in}}%
\pgfpathcurveto{\pgfqpoint{1.412733in}{2.765123in}}{\pgfqpoint{1.407147in}{2.762809in}}{\pgfqpoint{1.403029in}{2.758691in}}%
\pgfpathcurveto{\pgfqpoint{1.398911in}{2.754573in}}{\pgfqpoint{1.396597in}{2.748986in}}{\pgfqpoint{1.396597in}{2.743162in}}%
\pgfpathcurveto{\pgfqpoint{1.396597in}{2.737339in}}{\pgfqpoint{1.398911in}{2.731752in}}{\pgfqpoint{1.403029in}{2.727634in}}%
\pgfpathcurveto{\pgfqpoint{1.407147in}{2.723516in}}{\pgfqpoint{1.412733in}{2.721202in}}{\pgfqpoint{1.418557in}{2.721202in}}%
\pgfpathlineto{\pgfqpoint{1.418557in}{2.721202in}}%
\pgfpathclose%
\pgfusepath{stroke,fill}%
\end{pgfscope}%
\begin{pgfscope}%
\pgfpathrectangle{\pgfqpoint{1.000000in}{0.979904in}}{\pgfqpoint{6.200000in}{5.960192in}}%
\pgfusepath{clip}%
\pgfsetbuttcap%
\pgfsetroundjoin%
\definecolor{currentfill}{rgb}{0.800000,0.200000,0.200000}%
\pgfsetfillcolor{currentfill}%
\pgfsetlinewidth{1.003750pt}%
\definecolor{currentstroke}{rgb}{0.800000,0.200000,0.200000}%
\pgfsetstrokecolor{currentstroke}%
\pgfsetdash{}{0pt}%
\pgfpathmoveto{\pgfqpoint{1.281818in}{2.601614in}}%
\pgfpathcurveto{\pgfqpoint{1.287642in}{2.601614in}}{\pgfqpoint{1.293228in}{2.603928in}}{\pgfqpoint{1.297346in}{2.608046in}}%
\pgfpathcurveto{\pgfqpoint{1.301465in}{2.612164in}}{\pgfqpoint{1.303778in}{2.617750in}}{\pgfqpoint{1.303778in}{2.623574in}}%
\pgfpathcurveto{\pgfqpoint{1.303778in}{2.629398in}}{\pgfqpoint{1.301465in}{2.634984in}}{\pgfqpoint{1.297346in}{2.639103in}}%
\pgfpathcurveto{\pgfqpoint{1.293228in}{2.643221in}}{\pgfqpoint{1.287642in}{2.645535in}}{\pgfqpoint{1.281818in}{2.645535in}}%
\pgfpathcurveto{\pgfqpoint{1.275994in}{2.645535in}}{\pgfqpoint{1.270408in}{2.643221in}}{\pgfqpoint{1.266290in}{2.639103in}}%
\pgfpathcurveto{\pgfqpoint{1.262172in}{2.634984in}}{\pgfqpoint{1.259858in}{2.629398in}}{\pgfqpoint{1.259858in}{2.623574in}}%
\pgfpathcurveto{\pgfqpoint{1.259858in}{2.617750in}}{\pgfqpoint{1.262172in}{2.612164in}}{\pgfqpoint{1.266290in}{2.608046in}}%
\pgfpathcurveto{\pgfqpoint{1.270408in}{2.603928in}}{\pgfqpoint{1.275994in}{2.601614in}}{\pgfqpoint{1.281818in}{2.601614in}}%
\pgfpathlineto{\pgfqpoint{1.281818in}{2.601614in}}%
\pgfpathclose%
\pgfusepath{stroke,fill}%
\end{pgfscope}%
\begin{pgfscope}%
\pgfpathrectangle{\pgfqpoint{1.000000in}{0.979904in}}{\pgfqpoint{6.200000in}{5.960192in}}%
\pgfusepath{clip}%
\pgfsetbuttcap%
\pgfsetroundjoin%
\definecolor{currentfill}{rgb}{0.800000,0.200000,0.200000}%
\pgfsetfillcolor{currentfill}%
\pgfsetlinewidth{1.003750pt}%
\definecolor{currentstroke}{rgb}{0.800000,0.200000,0.200000}%
\pgfsetstrokecolor{currentstroke}%
\pgfsetdash{}{0pt}%
\pgfpathmoveto{\pgfqpoint{1.313274in}{2.499999in}}%
\pgfpathcurveto{\pgfqpoint{1.319098in}{2.499999in}}{\pgfqpoint{1.324684in}{2.502312in}}{\pgfqpoint{1.328802in}{2.506431in}}%
\pgfpathcurveto{\pgfqpoint{1.332920in}{2.510549in}}{\pgfqpoint{1.335234in}{2.516135in}}{\pgfqpoint{1.335234in}{2.521959in}}%
\pgfpathcurveto{\pgfqpoint{1.335234in}{2.527783in}}{\pgfqpoint{1.332920in}{2.533369in}}{\pgfqpoint{1.328802in}{2.537487in}}%
\pgfpathcurveto{\pgfqpoint{1.324684in}{2.541605in}}{\pgfqpoint{1.319098in}{2.543919in}}{\pgfqpoint{1.313274in}{2.543919in}}%
\pgfpathcurveto{\pgfqpoint{1.307450in}{2.543919in}}{\pgfqpoint{1.301864in}{2.541605in}}{\pgfqpoint{1.297745in}{2.537487in}}%
\pgfpathcurveto{\pgfqpoint{1.293627in}{2.533369in}}{\pgfqpoint{1.291313in}{2.527783in}}{\pgfqpoint{1.291313in}{2.521959in}}%
\pgfpathcurveto{\pgfqpoint{1.291313in}{2.516135in}}{\pgfqpoint{1.293627in}{2.510549in}}{\pgfqpoint{1.297745in}{2.506431in}}%
\pgfpathcurveto{\pgfqpoint{1.301864in}{2.502312in}}{\pgfqpoint{1.307450in}{2.499999in}}{\pgfqpoint{1.313274in}{2.499999in}}%
\pgfpathlineto{\pgfqpoint{1.313274in}{2.499999in}}%
\pgfpathclose%
\pgfusepath{stroke,fill}%
\end{pgfscope}%
\begin{pgfscope}%
\pgfpathrectangle{\pgfqpoint{1.000000in}{0.979904in}}{\pgfqpoint{6.200000in}{5.960192in}}%
\pgfusepath{clip}%
\pgfsetbuttcap%
\pgfsetroundjoin%
\definecolor{currentfill}{rgb}{0.800000,0.200000,0.200000}%
\pgfsetfillcolor{currentfill}%
\pgfsetlinewidth{1.003750pt}%
\definecolor{currentstroke}{rgb}{0.800000,0.200000,0.200000}%
\pgfsetstrokecolor{currentstroke}%
\pgfsetdash{}{0pt}%
\pgfpathmoveto{\pgfqpoint{1.413636in}{2.419713in}}%
\pgfpathcurveto{\pgfqpoint{1.419460in}{2.419713in}}{\pgfqpoint{1.425046in}{2.422027in}}{\pgfqpoint{1.429165in}{2.426145in}}%
\pgfpathcurveto{\pgfqpoint{1.433283in}{2.430263in}}{\pgfqpoint{1.435597in}{2.435849in}}{\pgfqpoint{1.435597in}{2.441673in}}%
\pgfpathcurveto{\pgfqpoint{1.435597in}{2.447497in}}{\pgfqpoint{1.433283in}{2.453083in}}{\pgfqpoint{1.429165in}{2.457201in}}%
\pgfpathcurveto{\pgfqpoint{1.425046in}{2.461319in}}{\pgfqpoint{1.419460in}{2.463633in}}{\pgfqpoint{1.413636in}{2.463633in}}%
\pgfpathcurveto{\pgfqpoint{1.407812in}{2.463633in}}{\pgfqpoint{1.402226in}{2.461319in}}{\pgfqpoint{1.398108in}{2.457201in}}%
\pgfpathcurveto{\pgfqpoint{1.393990in}{2.453083in}}{\pgfqpoint{1.391676in}{2.447497in}}{\pgfqpoint{1.391676in}{2.441673in}}%
\pgfpathcurveto{\pgfqpoint{1.391676in}{2.435849in}}{\pgfqpoint{1.393990in}{2.430263in}}{\pgfqpoint{1.398108in}{2.426145in}}%
\pgfpathcurveto{\pgfqpoint{1.402226in}{2.422027in}}{\pgfqpoint{1.407812in}{2.419713in}}{\pgfqpoint{1.413636in}{2.419713in}}%
\pgfpathlineto{\pgfqpoint{1.413636in}{2.419713in}}%
\pgfpathclose%
\pgfusepath{stroke,fill}%
\end{pgfscope}%
\begin{pgfscope}%
\pgfpathrectangle{\pgfqpoint{1.000000in}{0.979904in}}{\pgfqpoint{6.200000in}{5.960192in}}%
\pgfusepath{clip}%
\pgfsetbuttcap%
\pgfsetroundjoin%
\definecolor{currentfill}{rgb}{0.800000,0.200000,0.200000}%
\pgfsetfillcolor{currentfill}%
\pgfsetlinewidth{1.003750pt}%
\definecolor{currentstroke}{rgb}{0.800000,0.200000,0.200000}%
\pgfsetstrokecolor{currentstroke}%
\pgfsetdash{}{0pt}%
\pgfpathmoveto{\pgfqpoint{1.461802in}{2.330375in}}%
\pgfpathcurveto{\pgfqpoint{1.467626in}{2.330375in}}{\pgfqpoint{1.473212in}{2.332688in}}{\pgfqpoint{1.477330in}{2.336807in}}%
\pgfpathcurveto{\pgfqpoint{1.481449in}{2.340925in}}{\pgfqpoint{1.483762in}{2.346511in}}{\pgfqpoint{1.483762in}{2.352335in}}%
\pgfpathcurveto{\pgfqpoint{1.483762in}{2.358159in}}{\pgfqpoint{1.481449in}{2.363745in}}{\pgfqpoint{1.477330in}{2.367863in}}%
\pgfpathcurveto{\pgfqpoint{1.473212in}{2.371981in}}{\pgfqpoint{1.467626in}{2.374295in}}{\pgfqpoint{1.461802in}{2.374295in}}%
\pgfpathcurveto{\pgfqpoint{1.455978in}{2.374295in}}{\pgfqpoint{1.450392in}{2.371981in}}{\pgfqpoint{1.446274in}{2.367863in}}%
\pgfpathcurveto{\pgfqpoint{1.442156in}{2.363745in}}{\pgfqpoint{1.439842in}{2.358159in}}{\pgfqpoint{1.439842in}{2.352335in}}%
\pgfpathcurveto{\pgfqpoint{1.439842in}{2.346511in}}{\pgfqpoint{1.442156in}{2.340925in}}{\pgfqpoint{1.446274in}{2.336807in}}%
\pgfpathcurveto{\pgfqpoint{1.450392in}{2.332688in}}{\pgfqpoint{1.455978in}{2.330375in}}{\pgfqpoint{1.461802in}{2.330375in}}%
\pgfpathlineto{\pgfqpoint{1.461802in}{2.330375in}}%
\pgfpathclose%
\pgfusepath{stroke,fill}%
\end{pgfscope}%
\begin{pgfscope}%
\pgfpathrectangle{\pgfqpoint{1.000000in}{0.979904in}}{\pgfqpoint{6.200000in}{5.960192in}}%
\pgfusepath{clip}%
\pgfsetbuttcap%
\pgfsetroundjoin%
\definecolor{currentfill}{rgb}{0.800000,0.200000,0.200000}%
\pgfsetfillcolor{currentfill}%
\pgfsetlinewidth{1.003750pt}%
\definecolor{currentstroke}{rgb}{0.800000,0.200000,0.200000}%
\pgfsetstrokecolor{currentstroke}%
\pgfsetdash{}{0pt}%
\pgfpathmoveto{\pgfqpoint{1.473245in}{2.227001in}}%
\pgfpathcurveto{\pgfqpoint{1.479069in}{2.227001in}}{\pgfqpoint{1.484655in}{2.229315in}}{\pgfqpoint{1.488773in}{2.233433in}}%
\pgfpathcurveto{\pgfqpoint{1.492891in}{2.237551in}}{\pgfqpoint{1.495205in}{2.243137in}}{\pgfqpoint{1.495205in}{2.248961in}}%
\pgfpathcurveto{\pgfqpoint{1.495205in}{2.254785in}}{\pgfqpoint{1.492891in}{2.260371in}}{\pgfqpoint{1.488773in}{2.264490in}}%
\pgfpathcurveto{\pgfqpoint{1.484655in}{2.268608in}}{\pgfqpoint{1.479069in}{2.270922in}}{\pgfqpoint{1.473245in}{2.270922in}}%
\pgfpathcurveto{\pgfqpoint{1.467421in}{2.270922in}}{\pgfqpoint{1.461835in}{2.268608in}}{\pgfqpoint{1.457717in}{2.264490in}}%
\pgfpathcurveto{\pgfqpoint{1.453599in}{2.260371in}}{\pgfqpoint{1.451285in}{2.254785in}}{\pgfqpoint{1.451285in}{2.248961in}}%
\pgfpathcurveto{\pgfqpoint{1.451285in}{2.243137in}}{\pgfqpoint{1.453599in}{2.237551in}}{\pgfqpoint{1.457717in}{2.233433in}}%
\pgfpathcurveto{\pgfqpoint{1.461835in}{2.229315in}}{\pgfqpoint{1.467421in}{2.227001in}}{\pgfqpoint{1.473245in}{2.227001in}}%
\pgfpathlineto{\pgfqpoint{1.473245in}{2.227001in}}%
\pgfpathclose%
\pgfusepath{stroke,fill}%
\end{pgfscope}%
\begin{pgfscope}%
\pgfpathrectangle{\pgfqpoint{1.000000in}{0.979904in}}{\pgfqpoint{6.200000in}{5.960192in}}%
\pgfusepath{clip}%
\pgfsetbuttcap%
\pgfsetroundjoin%
\definecolor{currentfill}{rgb}{0.800000,0.200000,0.200000}%
\pgfsetfillcolor{currentfill}%
\pgfsetlinewidth{1.003750pt}%
\definecolor{currentstroke}{rgb}{0.800000,0.200000,0.200000}%
\pgfsetstrokecolor{currentstroke}%
\pgfsetdash{}{0pt}%
\pgfpathmoveto{\pgfqpoint{1.408730in}{2.080050in}}%
\pgfpathcurveto{\pgfqpoint{1.414554in}{2.080050in}}{\pgfqpoint{1.420140in}{2.082363in}}{\pgfqpoint{1.424258in}{2.086482in}}%
\pgfpathcurveto{\pgfqpoint{1.428376in}{2.090600in}}{\pgfqpoint{1.430690in}{2.096186in}}{\pgfqpoint{1.430690in}{2.102010in}}%
\pgfpathcurveto{\pgfqpoint{1.430690in}{2.107834in}}{\pgfqpoint{1.428376in}{2.113420in}}{\pgfqpoint{1.424258in}{2.117538in}}%
\pgfpathcurveto{\pgfqpoint{1.420140in}{2.121656in}}{\pgfqpoint{1.414554in}{2.123970in}}{\pgfqpoint{1.408730in}{2.123970in}}%
\pgfpathcurveto{\pgfqpoint{1.402906in}{2.123970in}}{\pgfqpoint{1.397320in}{2.121656in}}{\pgfqpoint{1.393202in}{2.117538in}}%
\pgfpathcurveto{\pgfqpoint{1.389084in}{2.113420in}}{\pgfqpoint{1.386770in}{2.107834in}}{\pgfqpoint{1.386770in}{2.102010in}}%
\pgfpathcurveto{\pgfqpoint{1.386770in}{2.096186in}}{\pgfqpoint{1.389084in}{2.090600in}}{\pgfqpoint{1.393202in}{2.086482in}}%
\pgfpathcurveto{\pgfqpoint{1.397320in}{2.082363in}}{\pgfqpoint{1.402906in}{2.080050in}}{\pgfqpoint{1.408730in}{2.080050in}}%
\pgfpathlineto{\pgfqpoint{1.408730in}{2.080050in}}%
\pgfpathclose%
\pgfusepath{stroke,fill}%
\end{pgfscope}%
\begin{pgfscope}%
\pgfpathrectangle{\pgfqpoint{1.000000in}{0.979904in}}{\pgfqpoint{6.200000in}{5.960192in}}%
\pgfusepath{clip}%
\pgfsetbuttcap%
\pgfsetroundjoin%
\definecolor{currentfill}{rgb}{0.800000,0.200000,0.200000}%
\pgfsetfillcolor{currentfill}%
\pgfsetlinewidth{1.003750pt}%
\definecolor{currentstroke}{rgb}{0.800000,0.200000,0.200000}%
\pgfsetstrokecolor{currentstroke}%
\pgfsetdash{}{0pt}%
\pgfpathmoveto{\pgfqpoint{1.550084in}{2.037919in}}%
\pgfpathcurveto{\pgfqpoint{1.555908in}{2.037919in}}{\pgfqpoint{1.561494in}{2.040233in}}{\pgfqpoint{1.565612in}{2.044351in}}%
\pgfpathcurveto{\pgfqpoint{1.569730in}{2.048469in}}{\pgfqpoint{1.572044in}{2.054056in}}{\pgfqpoint{1.572044in}{2.059879in}}%
\pgfpathcurveto{\pgfqpoint{1.572044in}{2.065703in}}{\pgfqpoint{1.569730in}{2.071290in}}{\pgfqpoint{1.565612in}{2.075408in}}%
\pgfpathcurveto{\pgfqpoint{1.561494in}{2.079526in}}{\pgfqpoint{1.555908in}{2.081840in}}{\pgfqpoint{1.550084in}{2.081840in}}%
\pgfpathcurveto{\pgfqpoint{1.544260in}{2.081840in}}{\pgfqpoint{1.538674in}{2.079526in}}{\pgfqpoint{1.534555in}{2.075408in}}%
\pgfpathcurveto{\pgfqpoint{1.530437in}{2.071290in}}{\pgfqpoint{1.528123in}{2.065703in}}{\pgfqpoint{1.528123in}{2.059879in}}%
\pgfpathcurveto{\pgfqpoint{1.528123in}{2.054056in}}{\pgfqpoint{1.530437in}{2.048469in}}{\pgfqpoint{1.534555in}{2.044351in}}%
\pgfpathcurveto{\pgfqpoint{1.538674in}{2.040233in}}{\pgfqpoint{1.544260in}{2.037919in}}{\pgfqpoint{1.550084in}{2.037919in}}%
\pgfpathlineto{\pgfqpoint{1.550084in}{2.037919in}}%
\pgfpathclose%
\pgfusepath{stroke,fill}%
\end{pgfscope}%
\begin{pgfscope}%
\pgfpathrectangle{\pgfqpoint{1.000000in}{0.979904in}}{\pgfqpoint{6.200000in}{5.960192in}}%
\pgfusepath{clip}%
\pgfsetbuttcap%
\pgfsetroundjoin%
\definecolor{currentfill}{rgb}{0.800000,0.200000,0.200000}%
\pgfsetfillcolor{currentfill}%
\pgfsetlinewidth{1.003750pt}%
\definecolor{currentstroke}{rgb}{0.800000,0.200000,0.200000}%
\pgfsetstrokecolor{currentstroke}%
\pgfsetdash{}{0pt}%
\pgfpathmoveto{\pgfqpoint{1.583316in}{1.936697in}}%
\pgfpathcurveto{\pgfqpoint{1.589140in}{1.936697in}}{\pgfqpoint{1.594727in}{1.939011in}}{\pgfqpoint{1.598845in}{1.943129in}}%
\pgfpathcurveto{\pgfqpoint{1.602963in}{1.947248in}}{\pgfqpoint{1.605277in}{1.952834in}}{\pgfqpoint{1.605277in}{1.958658in}}%
\pgfpathcurveto{\pgfqpoint{1.605277in}{1.964482in}}{\pgfqpoint{1.602963in}{1.970068in}}{\pgfqpoint{1.598845in}{1.974186in}}%
\pgfpathcurveto{\pgfqpoint{1.594727in}{1.978304in}}{\pgfqpoint{1.589140in}{1.980618in}}{\pgfqpoint{1.583316in}{1.980618in}}%
\pgfpathcurveto{\pgfqpoint{1.577492in}{1.980618in}}{\pgfqpoint{1.571906in}{1.978304in}}{\pgfqpoint{1.567788in}{1.974186in}}%
\pgfpathcurveto{\pgfqpoint{1.563670in}{1.970068in}}{\pgfqpoint{1.561356in}{1.964482in}}{\pgfqpoint{1.561356in}{1.958658in}}%
\pgfpathcurveto{\pgfqpoint{1.561356in}{1.952834in}}{\pgfqpoint{1.563670in}{1.947248in}}{\pgfqpoint{1.567788in}{1.943129in}}%
\pgfpathcurveto{\pgfqpoint{1.571906in}{1.939011in}}{\pgfqpoint{1.577492in}{1.936697in}}{\pgfqpoint{1.583316in}{1.936697in}}%
\pgfpathlineto{\pgfqpoint{1.583316in}{1.936697in}}%
\pgfpathclose%
\pgfusepath{stroke,fill}%
\end{pgfscope}%
\begin{pgfscope}%
\pgfpathrectangle{\pgfqpoint{1.000000in}{0.979904in}}{\pgfqpoint{6.200000in}{5.960192in}}%
\pgfusepath{clip}%
\pgfsetbuttcap%
\pgfsetroundjoin%
\definecolor{currentfill}{rgb}{0.800000,0.200000,0.200000}%
\pgfsetfillcolor{currentfill}%
\pgfsetlinewidth{1.003750pt}%
\definecolor{currentstroke}{rgb}{0.800000,0.200000,0.200000}%
\pgfsetstrokecolor{currentstroke}%
\pgfsetdash{}{0pt}%
\pgfpathmoveto{\pgfqpoint{1.691346in}{1.889451in}}%
\pgfpathcurveto{\pgfqpoint{1.697169in}{1.889451in}}{\pgfqpoint{1.702756in}{1.891765in}}{\pgfqpoint{1.706874in}{1.895883in}}%
\pgfpathcurveto{\pgfqpoint{1.710992in}{1.900001in}}{\pgfqpoint{1.713306in}{1.905587in}}{\pgfqpoint{1.713306in}{1.911411in}}%
\pgfpathcurveto{\pgfqpoint{1.713306in}{1.917235in}}{\pgfqpoint{1.710992in}{1.922821in}}{\pgfqpoint{1.706874in}{1.926939in}}%
\pgfpathcurveto{\pgfqpoint{1.702756in}{1.931057in}}{\pgfqpoint{1.697169in}{1.933371in}}{\pgfqpoint{1.691346in}{1.933371in}}%
\pgfpathcurveto{\pgfqpoint{1.685522in}{1.933371in}}{\pgfqpoint{1.679935in}{1.931057in}}{\pgfqpoint{1.675817in}{1.926939in}}%
\pgfpathcurveto{\pgfqpoint{1.671699in}{1.922821in}}{\pgfqpoint{1.669385in}{1.917235in}}{\pgfqpoint{1.669385in}{1.911411in}}%
\pgfpathcurveto{\pgfqpoint{1.669385in}{1.905587in}}{\pgfqpoint{1.671699in}{1.900001in}}{\pgfqpoint{1.675817in}{1.895883in}}%
\pgfpathcurveto{\pgfqpoint{1.679935in}{1.891765in}}{\pgfqpoint{1.685522in}{1.889451in}}{\pgfqpoint{1.691346in}{1.889451in}}%
\pgfpathlineto{\pgfqpoint{1.691346in}{1.889451in}}%
\pgfpathclose%
\pgfusepath{stroke,fill}%
\end{pgfscope}%
\begin{pgfscope}%
\pgfpathrectangle{\pgfqpoint{1.000000in}{0.979904in}}{\pgfqpoint{6.200000in}{5.960192in}}%
\pgfusepath{clip}%
\pgfsetbuttcap%
\pgfsetroundjoin%
\definecolor{currentfill}{rgb}{0.800000,0.200000,0.200000}%
\pgfsetfillcolor{currentfill}%
\pgfsetlinewidth{1.003750pt}%
\definecolor{currentstroke}{rgb}{0.800000,0.200000,0.200000}%
\pgfsetstrokecolor{currentstroke}%
\pgfsetdash{}{0pt}%
\pgfpathmoveto{\pgfqpoint{1.745356in}{1.803357in}}%
\pgfpathcurveto{\pgfqpoint{1.751180in}{1.803357in}}{\pgfqpoint{1.756766in}{1.805671in}}{\pgfqpoint{1.760884in}{1.809789in}}%
\pgfpathcurveto{\pgfqpoint{1.765002in}{1.813907in}}{\pgfqpoint{1.767316in}{1.819493in}}{\pgfqpoint{1.767316in}{1.825317in}}%
\pgfpathcurveto{\pgfqpoint{1.767316in}{1.831141in}}{\pgfqpoint{1.765002in}{1.836727in}}{\pgfqpoint{1.760884in}{1.840846in}}%
\pgfpathcurveto{\pgfqpoint{1.756766in}{1.844964in}}{\pgfqpoint{1.751180in}{1.847278in}}{\pgfqpoint{1.745356in}{1.847278in}}%
\pgfpathcurveto{\pgfqpoint{1.739532in}{1.847278in}}{\pgfqpoint{1.733946in}{1.844964in}}{\pgfqpoint{1.729828in}{1.840846in}}%
\pgfpathcurveto{\pgfqpoint{1.725709in}{1.836727in}}{\pgfqpoint{1.723396in}{1.831141in}}{\pgfqpoint{1.723396in}{1.825317in}}%
\pgfpathcurveto{\pgfqpoint{1.723396in}{1.819493in}}{\pgfqpoint{1.725709in}{1.813907in}}{\pgfqpoint{1.729828in}{1.809789in}}%
\pgfpathcurveto{\pgfqpoint{1.733946in}{1.805671in}}{\pgfqpoint{1.739532in}{1.803357in}}{\pgfqpoint{1.745356in}{1.803357in}}%
\pgfpathlineto{\pgfqpoint{1.745356in}{1.803357in}}%
\pgfpathclose%
\pgfusepath{stroke,fill}%
\end{pgfscope}%
\begin{pgfscope}%
\pgfpathrectangle{\pgfqpoint{1.000000in}{0.979904in}}{\pgfqpoint{6.200000in}{5.960192in}}%
\pgfusepath{clip}%
\pgfsetbuttcap%
\pgfsetroundjoin%
\definecolor{currentfill}{rgb}{0.800000,0.200000,0.200000}%
\pgfsetfillcolor{currentfill}%
\pgfsetlinewidth{1.003750pt}%
\definecolor{currentstroke}{rgb}{0.800000,0.200000,0.200000}%
\pgfsetstrokecolor{currentstroke}%
\pgfsetdash{}{0pt}%
\pgfpathmoveto{\pgfqpoint{1.793312in}{1.708010in}}%
\pgfpathcurveto{\pgfqpoint{1.799135in}{1.708010in}}{\pgfqpoint{1.804722in}{1.710324in}}{\pgfqpoint{1.808840in}{1.714442in}}%
\pgfpathcurveto{\pgfqpoint{1.812958in}{1.718560in}}{\pgfqpoint{1.815272in}{1.724147in}}{\pgfqpoint{1.815272in}{1.729970in}}%
\pgfpathcurveto{\pgfqpoint{1.815272in}{1.735794in}}{\pgfqpoint{1.812958in}{1.741381in}}{\pgfqpoint{1.808840in}{1.745499in}}%
\pgfpathcurveto{\pgfqpoint{1.804722in}{1.749617in}}{\pgfqpoint{1.799135in}{1.751931in}}{\pgfqpoint{1.793312in}{1.751931in}}%
\pgfpathcurveto{\pgfqpoint{1.787488in}{1.751931in}}{\pgfqpoint{1.781901in}{1.749617in}}{\pgfqpoint{1.777783in}{1.745499in}}%
\pgfpathcurveto{\pgfqpoint{1.773665in}{1.741381in}}{\pgfqpoint{1.771351in}{1.735794in}}{\pgfqpoint{1.771351in}{1.729970in}}%
\pgfpathcurveto{\pgfqpoint{1.771351in}{1.724147in}}{\pgfqpoint{1.773665in}{1.718560in}}{\pgfqpoint{1.777783in}{1.714442in}}%
\pgfpathcurveto{\pgfqpoint{1.781901in}{1.710324in}}{\pgfqpoint{1.787488in}{1.708010in}}{\pgfqpoint{1.793312in}{1.708010in}}%
\pgfpathlineto{\pgfqpoint{1.793312in}{1.708010in}}%
\pgfpathclose%
\pgfusepath{stroke,fill}%
\end{pgfscope}%
\begin{pgfscope}%
\pgfpathrectangle{\pgfqpoint{1.000000in}{0.979904in}}{\pgfqpoint{6.200000in}{5.960192in}}%
\pgfusepath{clip}%
\pgfsetbuttcap%
\pgfsetroundjoin%
\definecolor{currentfill}{rgb}{0.800000,0.200000,0.200000}%
\pgfsetfillcolor{currentfill}%
\pgfsetlinewidth{1.003750pt}%
\definecolor{currentstroke}{rgb}{0.800000,0.200000,0.200000}%
\pgfsetstrokecolor{currentstroke}%
\pgfsetdash{}{0pt}%
\pgfpathmoveto{\pgfqpoint{1.887142in}{1.658959in}}%
\pgfpathcurveto{\pgfqpoint{1.892966in}{1.658959in}}{\pgfqpoint{1.898552in}{1.661273in}}{\pgfqpoint{1.902670in}{1.665391in}}%
\pgfpathcurveto{\pgfqpoint{1.906788in}{1.669509in}}{\pgfqpoint{1.909102in}{1.675095in}}{\pgfqpoint{1.909102in}{1.680919in}}%
\pgfpathcurveto{\pgfqpoint{1.909102in}{1.686743in}}{\pgfqpoint{1.906788in}{1.692329in}}{\pgfqpoint{1.902670in}{1.696448in}}%
\pgfpathcurveto{\pgfqpoint{1.898552in}{1.700566in}}{\pgfqpoint{1.892966in}{1.702880in}}{\pgfqpoint{1.887142in}{1.702880in}}%
\pgfpathcurveto{\pgfqpoint{1.881318in}{1.702880in}}{\pgfqpoint{1.875732in}{1.700566in}}{\pgfqpoint{1.871614in}{1.696448in}}%
\pgfpathcurveto{\pgfqpoint{1.867496in}{1.692329in}}{\pgfqpoint{1.865182in}{1.686743in}}{\pgfqpoint{1.865182in}{1.680919in}}%
\pgfpathcurveto{\pgfqpoint{1.865182in}{1.675095in}}{\pgfqpoint{1.867496in}{1.669509in}}{\pgfqpoint{1.871614in}{1.665391in}}%
\pgfpathcurveto{\pgfqpoint{1.875732in}{1.661273in}}{\pgfqpoint{1.881318in}{1.658959in}}{\pgfqpoint{1.887142in}{1.658959in}}%
\pgfpathlineto{\pgfqpoint{1.887142in}{1.658959in}}%
\pgfpathclose%
\pgfusepath{stroke,fill}%
\end{pgfscope}%
\begin{pgfscope}%
\pgfpathrectangle{\pgfqpoint{1.000000in}{0.979904in}}{\pgfqpoint{6.200000in}{5.960192in}}%
\pgfusepath{clip}%
\pgfsetbuttcap%
\pgfsetroundjoin%
\definecolor{currentfill}{rgb}{0.800000,0.200000,0.200000}%
\pgfsetfillcolor{currentfill}%
\pgfsetlinewidth{1.003750pt}%
\definecolor{currentstroke}{rgb}{0.800000,0.200000,0.200000}%
\pgfsetstrokecolor{currentstroke}%
\pgfsetdash{}{0pt}%
\pgfpathmoveto{\pgfqpoint{2.030807in}{1.680496in}}%
\pgfpathcurveto{\pgfqpoint{2.036631in}{1.680496in}}{\pgfqpoint{2.042218in}{1.682810in}}{\pgfqpoint{2.046336in}{1.686928in}}%
\pgfpathcurveto{\pgfqpoint{2.050454in}{1.691046in}}{\pgfqpoint{2.052768in}{1.696632in}}{\pgfqpoint{2.052768in}{1.702456in}}%
\pgfpathcurveto{\pgfqpoint{2.052768in}{1.708280in}}{\pgfqpoint{2.050454in}{1.713866in}}{\pgfqpoint{2.046336in}{1.717985in}}%
\pgfpathcurveto{\pgfqpoint{2.042218in}{1.722103in}}{\pgfqpoint{2.036631in}{1.724417in}}{\pgfqpoint{2.030807in}{1.724417in}}%
\pgfpathcurveto{\pgfqpoint{2.024984in}{1.724417in}}{\pgfqpoint{2.019397in}{1.722103in}}{\pgfqpoint{2.015279in}{1.717985in}}%
\pgfpathcurveto{\pgfqpoint{2.011161in}{1.713866in}}{\pgfqpoint{2.008847in}{1.708280in}}{\pgfqpoint{2.008847in}{1.702456in}}%
\pgfpathcurveto{\pgfqpoint{2.008847in}{1.696632in}}{\pgfqpoint{2.011161in}{1.691046in}}{\pgfqpoint{2.015279in}{1.686928in}}%
\pgfpathcurveto{\pgfqpoint{2.019397in}{1.682810in}}{\pgfqpoint{2.024984in}{1.680496in}}{\pgfqpoint{2.030807in}{1.680496in}}%
\pgfpathlineto{\pgfqpoint{2.030807in}{1.680496in}}%
\pgfpathclose%
\pgfusepath{stroke,fill}%
\end{pgfscope}%
\begin{pgfscope}%
\pgfpathrectangle{\pgfqpoint{1.000000in}{0.979904in}}{\pgfqpoint{6.200000in}{5.960192in}}%
\pgfusepath{clip}%
\pgfsetbuttcap%
\pgfsetroundjoin%
\definecolor{currentfill}{rgb}{0.800000,0.200000,0.200000}%
\pgfsetfillcolor{currentfill}%
\pgfsetlinewidth{1.003750pt}%
\definecolor{currentstroke}{rgb}{0.800000,0.200000,0.200000}%
\pgfsetstrokecolor{currentstroke}%
\pgfsetdash{}{0pt}%
\pgfpathmoveto{\pgfqpoint{2.014404in}{1.485290in}}%
\pgfpathcurveto{\pgfqpoint{2.020228in}{1.485290in}}{\pgfqpoint{2.025814in}{1.487604in}}{\pgfqpoint{2.029932in}{1.491722in}}%
\pgfpathcurveto{\pgfqpoint{2.034050in}{1.495840in}}{\pgfqpoint{2.036364in}{1.501427in}}{\pgfqpoint{2.036364in}{1.507251in}}%
\pgfpathcurveto{\pgfqpoint{2.036364in}{1.513074in}}{\pgfqpoint{2.034050in}{1.518661in}}{\pgfqpoint{2.029932in}{1.522779in}}%
\pgfpathcurveto{\pgfqpoint{2.025814in}{1.526897in}}{\pgfqpoint{2.020228in}{1.529211in}}{\pgfqpoint{2.014404in}{1.529211in}}%
\pgfpathcurveto{\pgfqpoint{2.008580in}{1.529211in}}{\pgfqpoint{2.002994in}{1.526897in}}{\pgfqpoint{1.998876in}{1.522779in}}%
\pgfpathcurveto{\pgfqpoint{1.994758in}{1.518661in}}{\pgfqpoint{1.992444in}{1.513074in}}{\pgfqpoint{1.992444in}{1.507251in}}%
\pgfpathcurveto{\pgfqpoint{1.992444in}{1.501427in}}{\pgfqpoint{1.994758in}{1.495840in}}{\pgfqpoint{1.998876in}{1.491722in}}%
\pgfpathcurveto{\pgfqpoint{2.002994in}{1.487604in}}{\pgfqpoint{2.008580in}{1.485290in}}{\pgfqpoint{2.014404in}{1.485290in}}%
\pgfpathlineto{\pgfqpoint{2.014404in}{1.485290in}}%
\pgfpathclose%
\pgfusepath{stroke,fill}%
\end{pgfscope}%
\begin{pgfscope}%
\pgfpathrectangle{\pgfqpoint{1.000000in}{0.979904in}}{\pgfqpoint{6.200000in}{5.960192in}}%
\pgfusepath{clip}%
\pgfsetbuttcap%
\pgfsetroundjoin%
\definecolor{currentfill}{rgb}{0.800000,0.200000,0.200000}%
\pgfsetfillcolor{currentfill}%
\pgfsetlinewidth{1.003750pt}%
\definecolor{currentstroke}{rgb}{0.800000,0.200000,0.200000}%
\pgfsetstrokecolor{currentstroke}%
\pgfsetdash{}{0pt}%
\pgfpathmoveto{\pgfqpoint{2.154100in}{1.516920in}}%
\pgfpathcurveto{\pgfqpoint{2.159924in}{1.516920in}}{\pgfqpoint{2.165510in}{1.519234in}}{\pgfqpoint{2.169628in}{1.523352in}}%
\pgfpathcurveto{\pgfqpoint{2.173746in}{1.527470in}}{\pgfqpoint{2.176060in}{1.533056in}}{\pgfqpoint{2.176060in}{1.538880in}}%
\pgfpathcurveto{\pgfqpoint{2.176060in}{1.544704in}}{\pgfqpoint{2.173746in}{1.550290in}}{\pgfqpoint{2.169628in}{1.554408in}}%
\pgfpathcurveto{\pgfqpoint{2.165510in}{1.558526in}}{\pgfqpoint{2.159924in}{1.560840in}}{\pgfqpoint{2.154100in}{1.560840in}}%
\pgfpathcurveto{\pgfqpoint{2.148276in}{1.560840in}}{\pgfqpoint{2.142690in}{1.558526in}}{\pgfqpoint{2.138572in}{1.554408in}}%
\pgfpathcurveto{\pgfqpoint{2.134453in}{1.550290in}}{\pgfqpoint{2.132140in}{1.544704in}}{\pgfqpoint{2.132140in}{1.538880in}}%
\pgfpathcurveto{\pgfqpoint{2.132140in}{1.533056in}}{\pgfqpoint{2.134453in}{1.527470in}}{\pgfqpoint{2.138572in}{1.523352in}}%
\pgfpathcurveto{\pgfqpoint{2.142690in}{1.519234in}}{\pgfqpoint{2.148276in}{1.516920in}}{\pgfqpoint{2.154100in}{1.516920in}}%
\pgfpathlineto{\pgfqpoint{2.154100in}{1.516920in}}%
\pgfpathclose%
\pgfusepath{stroke,fill}%
\end{pgfscope}%
\begin{pgfscope}%
\pgfpathrectangle{\pgfqpoint{1.000000in}{0.979904in}}{\pgfqpoint{6.200000in}{5.960192in}}%
\pgfusepath{clip}%
\pgfsetbuttcap%
\pgfsetroundjoin%
\definecolor{currentfill}{rgb}{0.800000,0.200000,0.200000}%
\pgfsetfillcolor{currentfill}%
\pgfsetlinewidth{1.003750pt}%
\definecolor{currentstroke}{rgb}{0.800000,0.200000,0.200000}%
\pgfsetstrokecolor{currentstroke}%
\pgfsetdash{}{0pt}%
\pgfpathmoveto{\pgfqpoint{2.176982in}{1.340631in}}%
\pgfpathcurveto{\pgfqpoint{2.182806in}{1.340631in}}{\pgfqpoint{2.188392in}{1.342945in}}{\pgfqpoint{2.192510in}{1.347063in}}%
\pgfpathcurveto{\pgfqpoint{2.196628in}{1.351181in}}{\pgfqpoint{2.198942in}{1.356767in}}{\pgfqpoint{2.198942in}{1.362591in}}%
\pgfpathcurveto{\pgfqpoint{2.198942in}{1.368415in}}{\pgfqpoint{2.196628in}{1.374001in}}{\pgfqpoint{2.192510in}{1.378120in}}%
\pgfpathcurveto{\pgfqpoint{2.188392in}{1.382238in}}{\pgfqpoint{2.182806in}{1.384552in}}{\pgfqpoint{2.176982in}{1.384552in}}%
\pgfpathcurveto{\pgfqpoint{2.171158in}{1.384552in}}{\pgfqpoint{2.165572in}{1.382238in}}{\pgfqpoint{2.161454in}{1.378120in}}%
\pgfpathcurveto{\pgfqpoint{2.157336in}{1.374001in}}{\pgfqpoint{2.155022in}{1.368415in}}{\pgfqpoint{2.155022in}{1.362591in}}%
\pgfpathcurveto{\pgfqpoint{2.155022in}{1.356767in}}{\pgfqpoint{2.157336in}{1.351181in}}{\pgfqpoint{2.161454in}{1.347063in}}%
\pgfpathcurveto{\pgfqpoint{2.165572in}{1.342945in}}{\pgfqpoint{2.171158in}{1.340631in}}{\pgfqpoint{2.176982in}{1.340631in}}%
\pgfpathlineto{\pgfqpoint{2.176982in}{1.340631in}}%
\pgfpathclose%
\pgfusepath{stroke,fill}%
\end{pgfscope}%
\begin{pgfscope}%
\pgfpathrectangle{\pgfqpoint{1.000000in}{0.979904in}}{\pgfqpoint{6.200000in}{5.960192in}}%
\pgfusepath{clip}%
\pgfsetbuttcap%
\pgfsetroundjoin%
\definecolor{currentfill}{rgb}{0.800000,0.200000,0.200000}%
\pgfsetfillcolor{currentfill}%
\pgfsetlinewidth{1.003750pt}%
\definecolor{currentstroke}{rgb}{0.800000,0.200000,0.200000}%
\pgfsetstrokecolor{currentstroke}%
\pgfsetdash{}{0pt}%
\pgfpathmoveto{\pgfqpoint{2.304435in}{1.365469in}}%
\pgfpathcurveto{\pgfqpoint{2.310259in}{1.365469in}}{\pgfqpoint{2.315845in}{1.367783in}}{\pgfqpoint{2.319964in}{1.371901in}}%
\pgfpathcurveto{\pgfqpoint{2.324082in}{1.376019in}}{\pgfqpoint{2.326396in}{1.381606in}}{\pgfqpoint{2.326396in}{1.387429in}}%
\pgfpathcurveto{\pgfqpoint{2.326396in}{1.393253in}}{\pgfqpoint{2.324082in}{1.398840in}}{\pgfqpoint{2.319964in}{1.402958in}}%
\pgfpathcurveto{\pgfqpoint{2.315845in}{1.407076in}}{\pgfqpoint{2.310259in}{1.409390in}}{\pgfqpoint{2.304435in}{1.409390in}}%
\pgfpathcurveto{\pgfqpoint{2.298611in}{1.409390in}}{\pgfqpoint{2.293025in}{1.407076in}}{\pgfqpoint{2.288907in}{1.402958in}}%
\pgfpathcurveto{\pgfqpoint{2.284789in}{1.398840in}}{\pgfqpoint{2.282475in}{1.393253in}}{\pgfqpoint{2.282475in}{1.387429in}}%
\pgfpathcurveto{\pgfqpoint{2.282475in}{1.381606in}}{\pgfqpoint{2.284789in}{1.376019in}}{\pgfqpoint{2.288907in}{1.371901in}}%
\pgfpathcurveto{\pgfqpoint{2.293025in}{1.367783in}}{\pgfqpoint{2.298611in}{1.365469in}}{\pgfqpoint{2.304435in}{1.365469in}}%
\pgfpathlineto{\pgfqpoint{2.304435in}{1.365469in}}%
\pgfpathclose%
\pgfusepath{stroke,fill}%
\end{pgfscope}%
\begin{pgfscope}%
\pgfpathrectangle{\pgfqpoint{1.000000in}{0.979904in}}{\pgfqpoint{6.200000in}{5.960192in}}%
\pgfusepath{clip}%
\pgfsetbuttcap%
\pgfsetroundjoin%
\definecolor{currentfill}{rgb}{0.800000,0.200000,0.200000}%
\pgfsetfillcolor{currentfill}%
\pgfsetlinewidth{1.003750pt}%
\definecolor{currentstroke}{rgb}{0.800000,0.200000,0.200000}%
\pgfsetstrokecolor{currentstroke}%
\pgfsetdash{}{0pt}%
\pgfpathmoveto{\pgfqpoint{2.411545in}{1.359511in}}%
\pgfpathcurveto{\pgfqpoint{2.417369in}{1.359511in}}{\pgfqpoint{2.422955in}{1.361825in}}{\pgfqpoint{2.427073in}{1.365943in}}%
\pgfpathcurveto{\pgfqpoint{2.431191in}{1.370061in}}{\pgfqpoint{2.433505in}{1.375647in}}{\pgfqpoint{2.433505in}{1.381471in}}%
\pgfpathcurveto{\pgfqpoint{2.433505in}{1.387295in}}{\pgfqpoint{2.431191in}{1.392881in}}{\pgfqpoint{2.427073in}{1.396999in}}%
\pgfpathcurveto{\pgfqpoint{2.422955in}{1.401117in}}{\pgfqpoint{2.417369in}{1.403431in}}{\pgfqpoint{2.411545in}{1.403431in}}%
\pgfpathcurveto{\pgfqpoint{2.405721in}{1.403431in}}{\pgfqpoint{2.400135in}{1.401117in}}{\pgfqpoint{2.396017in}{1.396999in}}%
\pgfpathcurveto{\pgfqpoint{2.391899in}{1.392881in}}{\pgfqpoint{2.389585in}{1.387295in}}{\pgfqpoint{2.389585in}{1.381471in}}%
\pgfpathcurveto{\pgfqpoint{2.389585in}{1.375647in}}{\pgfqpoint{2.391899in}{1.370061in}}{\pgfqpoint{2.396017in}{1.365943in}}%
\pgfpathcurveto{\pgfqpoint{2.400135in}{1.361825in}}{\pgfqpoint{2.405721in}{1.359511in}}{\pgfqpoint{2.411545in}{1.359511in}}%
\pgfpathlineto{\pgfqpoint{2.411545in}{1.359511in}}%
\pgfpathclose%
\pgfusepath{stroke,fill}%
\end{pgfscope}%
\begin{pgfscope}%
\pgfpathrectangle{\pgfqpoint{1.000000in}{0.979904in}}{\pgfqpoint{6.200000in}{5.960192in}}%
\pgfusepath{clip}%
\pgfsetbuttcap%
\pgfsetroundjoin%
\definecolor{currentfill}{rgb}{0.800000,0.200000,0.200000}%
\pgfsetfillcolor{currentfill}%
\pgfsetlinewidth{1.003750pt}%
\definecolor{currentstroke}{rgb}{0.800000,0.200000,0.200000}%
\pgfsetstrokecolor{currentstroke}%
\pgfsetdash{}{0pt}%
\pgfpathmoveto{\pgfqpoint{2.502922in}{1.310641in}}%
\pgfpathcurveto{\pgfqpoint{2.508746in}{1.310641in}}{\pgfqpoint{2.514332in}{1.312955in}}{\pgfqpoint{2.518450in}{1.317073in}}%
\pgfpathcurveto{\pgfqpoint{2.522568in}{1.321192in}}{\pgfqpoint{2.524882in}{1.326778in}}{\pgfqpoint{2.524882in}{1.332602in}}%
\pgfpathcurveto{\pgfqpoint{2.524882in}{1.338426in}}{\pgfqpoint{2.522568in}{1.344012in}}{\pgfqpoint{2.518450in}{1.348130in}}%
\pgfpathcurveto{\pgfqpoint{2.514332in}{1.352248in}}{\pgfqpoint{2.508746in}{1.354562in}}{\pgfqpoint{2.502922in}{1.354562in}}%
\pgfpathcurveto{\pgfqpoint{2.497098in}{1.354562in}}{\pgfqpoint{2.491511in}{1.352248in}}{\pgfqpoint{2.487393in}{1.348130in}}%
\pgfpathcurveto{\pgfqpoint{2.483275in}{1.344012in}}{\pgfqpoint{2.480961in}{1.338426in}}{\pgfqpoint{2.480961in}{1.332602in}}%
\pgfpathcurveto{\pgfqpoint{2.480961in}{1.326778in}}{\pgfqpoint{2.483275in}{1.321192in}}{\pgfqpoint{2.487393in}{1.317073in}}%
\pgfpathcurveto{\pgfqpoint{2.491511in}{1.312955in}}{\pgfqpoint{2.497098in}{1.310641in}}{\pgfqpoint{2.502922in}{1.310641in}}%
\pgfpathlineto{\pgfqpoint{2.502922in}{1.310641in}}%
\pgfpathclose%
\pgfusepath{stroke,fill}%
\end{pgfscope}%
\begin{pgfscope}%
\pgfpathrectangle{\pgfqpoint{1.000000in}{0.979904in}}{\pgfqpoint{6.200000in}{5.960192in}}%
\pgfusepath{clip}%
\pgfsetbuttcap%
\pgfsetroundjoin%
\definecolor{currentfill}{rgb}{0.800000,0.200000,0.200000}%
\pgfsetfillcolor{currentfill}%
\pgfsetlinewidth{1.003750pt}%
\definecolor{currentstroke}{rgb}{0.800000,0.200000,0.200000}%
\pgfsetstrokecolor{currentstroke}%
\pgfsetdash{}{0pt}%
\pgfpathmoveto{\pgfqpoint{2.608375in}{1.314953in}}%
\pgfpathcurveto{\pgfqpoint{2.614198in}{1.314953in}}{\pgfqpoint{2.619785in}{1.317267in}}{\pgfqpoint{2.623903in}{1.321385in}}%
\pgfpathcurveto{\pgfqpoint{2.628021in}{1.325503in}}{\pgfqpoint{2.630335in}{1.331089in}}{\pgfqpoint{2.630335in}{1.336913in}}%
\pgfpathcurveto{\pgfqpoint{2.630335in}{1.342737in}}{\pgfqpoint{2.628021in}{1.348323in}}{\pgfqpoint{2.623903in}{1.352441in}}%
\pgfpathcurveto{\pgfqpoint{2.619785in}{1.356559in}}{\pgfqpoint{2.614198in}{1.358873in}}{\pgfqpoint{2.608375in}{1.358873in}}%
\pgfpathcurveto{\pgfqpoint{2.602551in}{1.358873in}}{\pgfqpoint{2.596964in}{1.356559in}}{\pgfqpoint{2.592846in}{1.352441in}}%
\pgfpathcurveto{\pgfqpoint{2.588728in}{1.348323in}}{\pgfqpoint{2.586414in}{1.342737in}}{\pgfqpoint{2.586414in}{1.336913in}}%
\pgfpathcurveto{\pgfqpoint{2.586414in}{1.331089in}}{\pgfqpoint{2.588728in}{1.325503in}}{\pgfqpoint{2.592846in}{1.321385in}}%
\pgfpathcurveto{\pgfqpoint{2.596964in}{1.317267in}}{\pgfqpoint{2.602551in}{1.314953in}}{\pgfqpoint{2.608375in}{1.314953in}}%
\pgfpathlineto{\pgfqpoint{2.608375in}{1.314953in}}%
\pgfpathclose%
\pgfusepath{stroke,fill}%
\end{pgfscope}%
\begin{pgfscope}%
\pgfpathrectangle{\pgfqpoint{1.000000in}{0.979904in}}{\pgfqpoint{6.200000in}{5.960192in}}%
\pgfusepath{clip}%
\pgfsetbuttcap%
\pgfsetroundjoin%
\definecolor{currentfill}{rgb}{0.800000,0.200000,0.200000}%
\pgfsetfillcolor{currentfill}%
\pgfsetlinewidth{1.003750pt}%
\definecolor{currentstroke}{rgb}{0.800000,0.200000,0.200000}%
\pgfsetstrokecolor{currentstroke}%
\pgfsetdash{}{0pt}%
\pgfpathmoveto{\pgfqpoint{2.710242in}{1.317342in}}%
\pgfpathcurveto{\pgfqpoint{2.716066in}{1.317342in}}{\pgfqpoint{2.721653in}{1.319656in}}{\pgfqpoint{2.725771in}{1.323774in}}%
\pgfpathcurveto{\pgfqpoint{2.729889in}{1.327892in}}{\pgfqpoint{2.732203in}{1.333478in}}{\pgfqpoint{2.732203in}{1.339302in}}%
\pgfpathcurveto{\pgfqpoint{2.732203in}{1.345126in}}{\pgfqpoint{2.729889in}{1.350712in}}{\pgfqpoint{2.725771in}{1.354830in}}%
\pgfpathcurveto{\pgfqpoint{2.721653in}{1.358948in}}{\pgfqpoint{2.716066in}{1.361262in}}{\pgfqpoint{2.710242in}{1.361262in}}%
\pgfpathcurveto{\pgfqpoint{2.704419in}{1.361262in}}{\pgfqpoint{2.698832in}{1.358948in}}{\pgfqpoint{2.694714in}{1.354830in}}%
\pgfpathcurveto{\pgfqpoint{2.690596in}{1.350712in}}{\pgfqpoint{2.688282in}{1.345126in}}{\pgfqpoint{2.688282in}{1.339302in}}%
\pgfpathcurveto{\pgfqpoint{2.688282in}{1.333478in}}{\pgfqpoint{2.690596in}{1.327892in}}{\pgfqpoint{2.694714in}{1.323774in}}%
\pgfpathcurveto{\pgfqpoint{2.698832in}{1.319656in}}{\pgfqpoint{2.704419in}{1.317342in}}{\pgfqpoint{2.710242in}{1.317342in}}%
\pgfpathlineto{\pgfqpoint{2.710242in}{1.317342in}}%
\pgfpathclose%
\pgfusepath{stroke,fill}%
\end{pgfscope}%
\begin{pgfscope}%
\pgfpathrectangle{\pgfqpoint{1.000000in}{0.979904in}}{\pgfqpoint{6.200000in}{5.960192in}}%
\pgfusepath{clip}%
\pgfsetbuttcap%
\pgfsetroundjoin%
\definecolor{currentfill}{rgb}{0.800000,0.200000,0.200000}%
\pgfsetfillcolor{currentfill}%
\pgfsetlinewidth{1.003750pt}%
\definecolor{currentstroke}{rgb}{0.800000,0.200000,0.200000}%
\pgfsetstrokecolor{currentstroke}%
\pgfsetdash{}{0pt}%
\pgfpathmoveto{\pgfqpoint{2.805982in}{1.269632in}}%
\pgfpathcurveto{\pgfqpoint{2.811806in}{1.269632in}}{\pgfqpoint{2.817392in}{1.271946in}}{\pgfqpoint{2.821510in}{1.276064in}}%
\pgfpathcurveto{\pgfqpoint{2.825628in}{1.280182in}}{\pgfqpoint{2.827942in}{1.285768in}}{\pgfqpoint{2.827942in}{1.291592in}}%
\pgfpathcurveto{\pgfqpoint{2.827942in}{1.297416in}}{\pgfqpoint{2.825628in}{1.303002in}}{\pgfqpoint{2.821510in}{1.307120in}}%
\pgfpathcurveto{\pgfqpoint{2.817392in}{1.311238in}}{\pgfqpoint{2.811806in}{1.313552in}}{\pgfqpoint{2.805982in}{1.313552in}}%
\pgfpathcurveto{\pgfqpoint{2.800158in}{1.313552in}}{\pgfqpoint{2.794572in}{1.311238in}}{\pgfqpoint{2.790454in}{1.307120in}}%
\pgfpathcurveto{\pgfqpoint{2.786336in}{1.303002in}}{\pgfqpoint{2.784022in}{1.297416in}}{\pgfqpoint{2.784022in}{1.291592in}}%
\pgfpathcurveto{\pgfqpoint{2.784022in}{1.285768in}}{\pgfqpoint{2.786336in}{1.280182in}}{\pgfqpoint{2.790454in}{1.276064in}}%
\pgfpathcurveto{\pgfqpoint{2.794572in}{1.271946in}}{\pgfqpoint{2.800158in}{1.269632in}}{\pgfqpoint{2.805982in}{1.269632in}}%
\pgfpathlineto{\pgfqpoint{2.805982in}{1.269632in}}%
\pgfpathclose%
\pgfusepath{stroke,fill}%
\end{pgfscope}%
\begin{pgfscope}%
\pgfpathrectangle{\pgfqpoint{1.000000in}{0.979904in}}{\pgfqpoint{6.200000in}{5.960192in}}%
\pgfusepath{clip}%
\pgfsetbuttcap%
\pgfsetroundjoin%
\definecolor{currentfill}{rgb}{0.800000,0.200000,0.200000}%
\pgfsetfillcolor{currentfill}%
\pgfsetlinewidth{1.003750pt}%
\definecolor{currentstroke}{rgb}{0.800000,0.200000,0.200000}%
\pgfsetstrokecolor{currentstroke}%
\pgfsetdash{}{0pt}%
\pgfpathmoveto{\pgfqpoint{2.906903in}{1.228861in}}%
\pgfpathcurveto{\pgfqpoint{2.912727in}{1.228861in}}{\pgfqpoint{2.918313in}{1.231175in}}{\pgfqpoint{2.922431in}{1.235293in}}%
\pgfpathcurveto{\pgfqpoint{2.926549in}{1.239412in}}{\pgfqpoint{2.928863in}{1.244998in}}{\pgfqpoint{2.928863in}{1.250822in}}%
\pgfpathcurveto{\pgfqpoint{2.928863in}{1.256646in}}{\pgfqpoint{2.926549in}{1.262232in}}{\pgfqpoint{2.922431in}{1.266350in}}%
\pgfpathcurveto{\pgfqpoint{2.918313in}{1.270468in}}{\pgfqpoint{2.912727in}{1.272782in}}{\pgfqpoint{2.906903in}{1.272782in}}%
\pgfpathcurveto{\pgfqpoint{2.901079in}{1.272782in}}{\pgfqpoint{2.895492in}{1.270468in}}{\pgfqpoint{2.891374in}{1.266350in}}%
\pgfpathcurveto{\pgfqpoint{2.887256in}{1.262232in}}{\pgfqpoint{2.884942in}{1.256646in}}{\pgfqpoint{2.884942in}{1.250822in}}%
\pgfpathcurveto{\pgfqpoint{2.884942in}{1.244998in}}{\pgfqpoint{2.887256in}{1.239412in}}{\pgfqpoint{2.891374in}{1.235293in}}%
\pgfpathcurveto{\pgfqpoint{2.895492in}{1.231175in}}{\pgfqpoint{2.901079in}{1.228861in}}{\pgfqpoint{2.906903in}{1.228861in}}%
\pgfpathlineto{\pgfqpoint{2.906903in}{1.228861in}}%
\pgfpathclose%
\pgfusepath{stroke,fill}%
\end{pgfscope}%
\begin{pgfscope}%
\pgfpathrectangle{\pgfqpoint{1.000000in}{0.979904in}}{\pgfqpoint{6.200000in}{5.960192in}}%
\pgfusepath{clip}%
\pgfsetbuttcap%
\pgfsetroundjoin%
\definecolor{currentfill}{rgb}{0.800000,0.200000,0.200000}%
\pgfsetfillcolor{currentfill}%
\pgfsetlinewidth{1.003750pt}%
\definecolor{currentstroke}{rgb}{0.800000,0.200000,0.200000}%
\pgfsetstrokecolor{currentstroke}%
\pgfsetdash{}{0pt}%
\pgfpathmoveto{\pgfqpoint{3.007369in}{1.302030in}}%
\pgfpathcurveto{\pgfqpoint{3.013193in}{1.302030in}}{\pgfqpoint{3.018779in}{1.304343in}}{\pgfqpoint{3.022897in}{1.308462in}}%
\pgfpathcurveto{\pgfqpoint{3.027016in}{1.312580in}}{\pgfqpoint{3.029329in}{1.318166in}}{\pgfqpoint{3.029329in}{1.323990in}}%
\pgfpathcurveto{\pgfqpoint{3.029329in}{1.329814in}}{\pgfqpoint{3.027016in}{1.335400in}}{\pgfqpoint{3.022897in}{1.339518in}}%
\pgfpathcurveto{\pgfqpoint{3.018779in}{1.343636in}}{\pgfqpoint{3.013193in}{1.345950in}}{\pgfqpoint{3.007369in}{1.345950in}}%
\pgfpathcurveto{\pgfqpoint{3.001545in}{1.345950in}}{\pgfqpoint{2.995959in}{1.343636in}}{\pgfqpoint{2.991841in}{1.339518in}}%
\pgfpathcurveto{\pgfqpoint{2.987723in}{1.335400in}}{\pgfqpoint{2.985409in}{1.329814in}}{\pgfqpoint{2.985409in}{1.323990in}}%
\pgfpathcurveto{\pgfqpoint{2.985409in}{1.318166in}}{\pgfqpoint{2.987723in}{1.312580in}}{\pgfqpoint{2.991841in}{1.308462in}}%
\pgfpathcurveto{\pgfqpoint{2.995959in}{1.304343in}}{\pgfqpoint{3.001545in}{1.302030in}}{\pgfqpoint{3.007369in}{1.302030in}}%
\pgfpathlineto{\pgfqpoint{3.007369in}{1.302030in}}%
\pgfpathclose%
\pgfusepath{stroke,fill}%
\end{pgfscope}%
\begin{pgfscope}%
\pgfpathrectangle{\pgfqpoint{1.000000in}{0.979904in}}{\pgfqpoint{6.200000in}{5.960192in}}%
\pgfusepath{clip}%
\pgfsetbuttcap%
\pgfsetroundjoin%
\definecolor{currentfill}{rgb}{0.800000,0.200000,0.200000}%
\pgfsetfillcolor{currentfill}%
\pgfsetlinewidth{1.003750pt}%
\definecolor{currentstroke}{rgb}{0.800000,0.200000,0.200000}%
\pgfsetstrokecolor{currentstroke}%
\pgfsetdash{}{0pt}%
\pgfpathmoveto{\pgfqpoint{3.105472in}{1.318218in}}%
\pgfpathcurveto{\pgfqpoint{3.111296in}{1.318218in}}{\pgfqpoint{3.116882in}{1.320532in}}{\pgfqpoint{3.121001in}{1.324650in}}%
\pgfpathcurveto{\pgfqpoint{3.125119in}{1.328769in}}{\pgfqpoint{3.127433in}{1.334355in}}{\pgfqpoint{3.127433in}{1.340179in}}%
\pgfpathcurveto{\pgfqpoint{3.127433in}{1.346003in}}{\pgfqpoint{3.125119in}{1.351589in}}{\pgfqpoint{3.121001in}{1.355707in}}%
\pgfpathcurveto{\pgfqpoint{3.116882in}{1.359825in}}{\pgfqpoint{3.111296in}{1.362139in}}{\pgfqpoint{3.105472in}{1.362139in}}%
\pgfpathcurveto{\pgfqpoint{3.099648in}{1.362139in}}{\pgfqpoint{3.094062in}{1.359825in}}{\pgfqpoint{3.089944in}{1.355707in}}%
\pgfpathcurveto{\pgfqpoint{3.085826in}{1.351589in}}{\pgfqpoint{3.083512in}{1.346003in}}{\pgfqpoint{3.083512in}{1.340179in}}%
\pgfpathcurveto{\pgfqpoint{3.083512in}{1.334355in}}{\pgfqpoint{3.085826in}{1.328769in}}{\pgfqpoint{3.089944in}{1.324650in}}%
\pgfpathcurveto{\pgfqpoint{3.094062in}{1.320532in}}{\pgfqpoint{3.099648in}{1.318218in}}{\pgfqpoint{3.105472in}{1.318218in}}%
\pgfpathlineto{\pgfqpoint{3.105472in}{1.318218in}}%
\pgfpathclose%
\pgfusepath{stroke,fill}%
\end{pgfscope}%
\begin{pgfscope}%
\pgfpathrectangle{\pgfqpoint{1.000000in}{0.979904in}}{\pgfqpoint{6.200000in}{5.960192in}}%
\pgfusepath{clip}%
\pgfsetbuttcap%
\pgfsetroundjoin%
\definecolor{currentfill}{rgb}{0.800000,0.200000,0.200000}%
\pgfsetfillcolor{currentfill}%
\pgfsetlinewidth{1.003750pt}%
\definecolor{currentstroke}{rgb}{0.800000,0.200000,0.200000}%
\pgfsetstrokecolor{currentstroke}%
\pgfsetdash{}{0pt}%
\pgfpathmoveto{\pgfqpoint{3.205736in}{1.318343in}}%
\pgfpathcurveto{\pgfqpoint{3.211560in}{1.318343in}}{\pgfqpoint{3.217146in}{1.320657in}}{\pgfqpoint{3.221265in}{1.324775in}}%
\pgfpathcurveto{\pgfqpoint{3.225383in}{1.328893in}}{\pgfqpoint{3.227697in}{1.334479in}}{\pgfqpoint{3.227697in}{1.340303in}}%
\pgfpathcurveto{\pgfqpoint{3.227697in}{1.346127in}}{\pgfqpoint{3.225383in}{1.351713in}}{\pgfqpoint{3.221265in}{1.355832in}}%
\pgfpathcurveto{\pgfqpoint{3.217146in}{1.359950in}}{\pgfqpoint{3.211560in}{1.362264in}}{\pgfqpoint{3.205736in}{1.362264in}}%
\pgfpathcurveto{\pgfqpoint{3.199912in}{1.362264in}}{\pgfqpoint{3.194326in}{1.359950in}}{\pgfqpoint{3.190208in}{1.355832in}}%
\pgfpathcurveto{\pgfqpoint{3.186090in}{1.351713in}}{\pgfqpoint{3.183776in}{1.346127in}}{\pgfqpoint{3.183776in}{1.340303in}}%
\pgfpathcurveto{\pgfqpoint{3.183776in}{1.334479in}}{\pgfqpoint{3.186090in}{1.328893in}}{\pgfqpoint{3.190208in}{1.324775in}}%
\pgfpathcurveto{\pgfqpoint{3.194326in}{1.320657in}}{\pgfqpoint{3.199912in}{1.318343in}}{\pgfqpoint{3.205736in}{1.318343in}}%
\pgfpathlineto{\pgfqpoint{3.205736in}{1.318343in}}%
\pgfpathclose%
\pgfusepath{stroke,fill}%
\end{pgfscope}%
\begin{pgfscope}%
\pgfpathrectangle{\pgfqpoint{1.000000in}{0.979904in}}{\pgfqpoint{6.200000in}{5.960192in}}%
\pgfusepath{clip}%
\pgfsetbuttcap%
\pgfsetroundjoin%
\definecolor{currentfill}{rgb}{0.800000,0.200000,0.200000}%
\pgfsetfillcolor{currentfill}%
\pgfsetlinewidth{1.003750pt}%
\definecolor{currentstroke}{rgb}{0.800000,0.200000,0.200000}%
\pgfsetstrokecolor{currentstroke}%
\pgfsetdash{}{0pt}%
\pgfpathmoveto{\pgfqpoint{3.319688in}{1.271348in}}%
\pgfpathcurveto{\pgfqpoint{3.325512in}{1.271348in}}{\pgfqpoint{3.331098in}{1.273662in}}{\pgfqpoint{3.335216in}{1.277780in}}%
\pgfpathcurveto{\pgfqpoint{3.339334in}{1.281898in}}{\pgfqpoint{3.341648in}{1.287485in}}{\pgfqpoint{3.341648in}{1.293309in}}%
\pgfpathcurveto{\pgfqpoint{3.341648in}{1.299132in}}{\pgfqpoint{3.339334in}{1.304719in}}{\pgfqpoint{3.335216in}{1.308837in}}%
\pgfpathcurveto{\pgfqpoint{3.331098in}{1.312955in}}{\pgfqpoint{3.325512in}{1.315269in}}{\pgfqpoint{3.319688in}{1.315269in}}%
\pgfpathcurveto{\pgfqpoint{3.313864in}{1.315269in}}{\pgfqpoint{3.308278in}{1.312955in}}{\pgfqpoint{3.304160in}{1.308837in}}%
\pgfpathcurveto{\pgfqpoint{3.300041in}{1.304719in}}{\pgfqpoint{3.297728in}{1.299132in}}{\pgfqpoint{3.297728in}{1.293309in}}%
\pgfpathcurveto{\pgfqpoint{3.297728in}{1.287485in}}{\pgfqpoint{3.300041in}{1.281898in}}{\pgfqpoint{3.304160in}{1.277780in}}%
\pgfpathcurveto{\pgfqpoint{3.308278in}{1.273662in}}{\pgfqpoint{3.313864in}{1.271348in}}{\pgfqpoint{3.319688in}{1.271348in}}%
\pgfpathlineto{\pgfqpoint{3.319688in}{1.271348in}}%
\pgfpathclose%
\pgfusepath{stroke,fill}%
\end{pgfscope}%
\begin{pgfscope}%
\pgfpathrectangle{\pgfqpoint{1.000000in}{0.979904in}}{\pgfqpoint{6.200000in}{5.960192in}}%
\pgfusepath{clip}%
\pgfsetbuttcap%
\pgfsetroundjoin%
\definecolor{currentfill}{rgb}{0.800000,0.200000,0.200000}%
\pgfsetfillcolor{currentfill}%
\pgfsetlinewidth{1.003750pt}%
\definecolor{currentstroke}{rgb}{0.800000,0.200000,0.200000}%
\pgfsetstrokecolor{currentstroke}%
\pgfsetdash{}{0pt}%
\pgfpathmoveto{\pgfqpoint{3.427921in}{1.273732in}}%
\pgfpathcurveto{\pgfqpoint{3.433745in}{1.273732in}}{\pgfqpoint{3.439331in}{1.276046in}}{\pgfqpoint{3.443449in}{1.280164in}}%
\pgfpathcurveto{\pgfqpoint{3.447567in}{1.284282in}}{\pgfqpoint{3.449881in}{1.289869in}}{\pgfqpoint{3.449881in}{1.295692in}}%
\pgfpathcurveto{\pgfqpoint{3.449881in}{1.301516in}}{\pgfqpoint{3.447567in}{1.307103in}}{\pgfqpoint{3.443449in}{1.311221in}}%
\pgfpathcurveto{\pgfqpoint{3.439331in}{1.315339in}}{\pgfqpoint{3.433745in}{1.317653in}}{\pgfqpoint{3.427921in}{1.317653in}}%
\pgfpathcurveto{\pgfqpoint{3.422097in}{1.317653in}}{\pgfqpoint{3.416511in}{1.315339in}}{\pgfqpoint{3.412393in}{1.311221in}}%
\pgfpathcurveto{\pgfqpoint{3.408275in}{1.307103in}}{\pgfqpoint{3.405961in}{1.301516in}}{\pgfqpoint{3.405961in}{1.295692in}}%
\pgfpathcurveto{\pgfqpoint{3.405961in}{1.289869in}}{\pgfqpoint{3.408275in}{1.284282in}}{\pgfqpoint{3.412393in}{1.280164in}}%
\pgfpathcurveto{\pgfqpoint{3.416511in}{1.276046in}}{\pgfqpoint{3.422097in}{1.273732in}}{\pgfqpoint{3.427921in}{1.273732in}}%
\pgfpathlineto{\pgfqpoint{3.427921in}{1.273732in}}%
\pgfpathclose%
\pgfusepath{stroke,fill}%
\end{pgfscope}%
\begin{pgfscope}%
\pgfpathrectangle{\pgfqpoint{1.000000in}{0.979904in}}{\pgfqpoint{6.200000in}{5.960192in}}%
\pgfusepath{clip}%
\pgfsetbuttcap%
\pgfsetroundjoin%
\definecolor{currentfill}{rgb}{0.800000,0.200000,0.200000}%
\pgfsetfillcolor{currentfill}%
\pgfsetlinewidth{1.003750pt}%
\definecolor{currentstroke}{rgb}{0.800000,0.200000,0.200000}%
\pgfsetstrokecolor{currentstroke}%
\pgfsetdash{}{0pt}%
\pgfpathmoveto{\pgfqpoint{3.507250in}{1.362385in}}%
\pgfpathcurveto{\pgfqpoint{3.513074in}{1.362385in}}{\pgfqpoint{3.518660in}{1.364699in}}{\pgfqpoint{3.522778in}{1.368817in}}%
\pgfpathcurveto{\pgfqpoint{3.526896in}{1.372935in}}{\pgfqpoint{3.529210in}{1.378521in}}{\pgfqpoint{3.529210in}{1.384345in}}%
\pgfpathcurveto{\pgfqpoint{3.529210in}{1.390169in}}{\pgfqpoint{3.526896in}{1.395755in}}{\pgfqpoint{3.522778in}{1.399874in}}%
\pgfpathcurveto{\pgfqpoint{3.518660in}{1.403992in}}{\pgfqpoint{3.513074in}{1.406306in}}{\pgfqpoint{3.507250in}{1.406306in}}%
\pgfpathcurveto{\pgfqpoint{3.501426in}{1.406306in}}{\pgfqpoint{3.495840in}{1.403992in}}{\pgfqpoint{3.491722in}{1.399874in}}%
\pgfpathcurveto{\pgfqpoint{3.487604in}{1.395755in}}{\pgfqpoint{3.485290in}{1.390169in}}{\pgfqpoint{3.485290in}{1.384345in}}%
\pgfpathcurveto{\pgfqpoint{3.485290in}{1.378521in}}{\pgfqpoint{3.487604in}{1.372935in}}{\pgfqpoint{3.491722in}{1.368817in}}%
\pgfpathcurveto{\pgfqpoint{3.495840in}{1.364699in}}{\pgfqpoint{3.501426in}{1.362385in}}{\pgfqpoint{3.507250in}{1.362385in}}%
\pgfpathlineto{\pgfqpoint{3.507250in}{1.362385in}}%
\pgfpathclose%
\pgfusepath{stroke,fill}%
\end{pgfscope}%
\begin{pgfscope}%
\pgfpathrectangle{\pgfqpoint{1.000000in}{0.979904in}}{\pgfqpoint{6.200000in}{5.960192in}}%
\pgfusepath{clip}%
\pgfsetbuttcap%
\pgfsetroundjoin%
\definecolor{currentfill}{rgb}{0.800000,0.200000,0.200000}%
\pgfsetfillcolor{currentfill}%
\pgfsetlinewidth{1.003750pt}%
\definecolor{currentstroke}{rgb}{0.800000,0.200000,0.200000}%
\pgfsetstrokecolor{currentstroke}%
\pgfsetdash{}{0pt}%
\pgfpathmoveto{\pgfqpoint{3.606280in}{1.391257in}}%
\pgfpathcurveto{\pgfqpoint{3.612104in}{1.391257in}}{\pgfqpoint{3.617690in}{1.393571in}}{\pgfqpoint{3.621808in}{1.397689in}}%
\pgfpathcurveto{\pgfqpoint{3.625926in}{1.401807in}}{\pgfqpoint{3.628240in}{1.407393in}}{\pgfqpoint{3.628240in}{1.413217in}}%
\pgfpathcurveto{\pgfqpoint{3.628240in}{1.419041in}}{\pgfqpoint{3.625926in}{1.424627in}}{\pgfqpoint{3.621808in}{1.428746in}}%
\pgfpathcurveto{\pgfqpoint{3.617690in}{1.432864in}}{\pgfqpoint{3.612104in}{1.435178in}}{\pgfqpoint{3.606280in}{1.435178in}}%
\pgfpathcurveto{\pgfqpoint{3.600456in}{1.435178in}}{\pgfqpoint{3.594870in}{1.432864in}}{\pgfqpoint{3.590751in}{1.428746in}}%
\pgfpathcurveto{\pgfqpoint{3.586633in}{1.424627in}}{\pgfqpoint{3.584319in}{1.419041in}}{\pgfqpoint{3.584319in}{1.413217in}}%
\pgfpathcurveto{\pgfqpoint{3.584319in}{1.407393in}}{\pgfqpoint{3.586633in}{1.401807in}}{\pgfqpoint{3.590751in}{1.397689in}}%
\pgfpathcurveto{\pgfqpoint{3.594870in}{1.393571in}}{\pgfqpoint{3.600456in}{1.391257in}}{\pgfqpoint{3.606280in}{1.391257in}}%
\pgfpathlineto{\pgfqpoint{3.606280in}{1.391257in}}%
\pgfpathclose%
\pgfusepath{stroke,fill}%
\end{pgfscope}%
\begin{pgfscope}%
\pgfpathrectangle{\pgfqpoint{1.000000in}{0.979904in}}{\pgfqpoint{6.200000in}{5.960192in}}%
\pgfusepath{clip}%
\pgfsetbuttcap%
\pgfsetroundjoin%
\definecolor{currentfill}{rgb}{0.800000,0.200000,0.200000}%
\pgfsetfillcolor{currentfill}%
\pgfsetlinewidth{1.003750pt}%
\definecolor{currentstroke}{rgb}{0.800000,0.200000,0.200000}%
\pgfsetstrokecolor{currentstroke}%
\pgfsetdash{}{0pt}%
\pgfpathmoveto{\pgfqpoint{3.668101in}{1.493571in}}%
\pgfpathcurveto{\pgfqpoint{3.673925in}{1.493571in}}{\pgfqpoint{3.679511in}{1.495885in}}{\pgfqpoint{3.683629in}{1.500003in}}%
\pgfpathcurveto{\pgfqpoint{3.687747in}{1.504121in}}{\pgfqpoint{3.690061in}{1.509707in}}{\pgfqpoint{3.690061in}{1.515531in}}%
\pgfpathcurveto{\pgfqpoint{3.690061in}{1.521355in}}{\pgfqpoint{3.687747in}{1.526941in}}{\pgfqpoint{3.683629in}{1.531060in}}%
\pgfpathcurveto{\pgfqpoint{3.679511in}{1.535178in}}{\pgfqpoint{3.673925in}{1.537492in}}{\pgfqpoint{3.668101in}{1.537492in}}%
\pgfpathcurveto{\pgfqpoint{3.662277in}{1.537492in}}{\pgfqpoint{3.656691in}{1.535178in}}{\pgfqpoint{3.652573in}{1.531060in}}%
\pgfpathcurveto{\pgfqpoint{3.648454in}{1.526941in}}{\pgfqpoint{3.646141in}{1.521355in}}{\pgfqpoint{3.646141in}{1.515531in}}%
\pgfpathcurveto{\pgfqpoint{3.646141in}{1.509707in}}{\pgfqpoint{3.648454in}{1.504121in}}{\pgfqpoint{3.652573in}{1.500003in}}%
\pgfpathcurveto{\pgfqpoint{3.656691in}{1.495885in}}{\pgfqpoint{3.662277in}{1.493571in}}{\pgfqpoint{3.668101in}{1.493571in}}%
\pgfpathlineto{\pgfqpoint{3.668101in}{1.493571in}}%
\pgfpathclose%
\pgfusepath{stroke,fill}%
\end{pgfscope}%
\begin{pgfscope}%
\pgfpathrectangle{\pgfqpoint{1.000000in}{0.979904in}}{\pgfqpoint{6.200000in}{5.960192in}}%
\pgfusepath{clip}%
\pgfsetbuttcap%
\pgfsetroundjoin%
\definecolor{currentfill}{rgb}{0.800000,0.200000,0.200000}%
\pgfsetfillcolor{currentfill}%
\pgfsetlinewidth{1.003750pt}%
\definecolor{currentstroke}{rgb}{0.800000,0.200000,0.200000}%
\pgfsetstrokecolor{currentstroke}%
\pgfsetdash{}{0pt}%
\pgfpathmoveto{\pgfqpoint{3.795793in}{1.475048in}}%
\pgfpathcurveto{\pgfqpoint{3.801617in}{1.475048in}}{\pgfqpoint{3.807203in}{1.477362in}}{\pgfqpoint{3.811321in}{1.481480in}}%
\pgfpathcurveto{\pgfqpoint{3.815439in}{1.485598in}}{\pgfqpoint{3.817753in}{1.491185in}}{\pgfqpoint{3.817753in}{1.497008in}}%
\pgfpathcurveto{\pgfqpoint{3.817753in}{1.502832in}}{\pgfqpoint{3.815439in}{1.508419in}}{\pgfqpoint{3.811321in}{1.512537in}}%
\pgfpathcurveto{\pgfqpoint{3.807203in}{1.516655in}}{\pgfqpoint{3.801617in}{1.518969in}}{\pgfqpoint{3.795793in}{1.518969in}}%
\pgfpathcurveto{\pgfqpoint{3.789969in}{1.518969in}}{\pgfqpoint{3.784383in}{1.516655in}}{\pgfqpoint{3.780264in}{1.512537in}}%
\pgfpathcurveto{\pgfqpoint{3.776146in}{1.508419in}}{\pgfqpoint{3.773832in}{1.502832in}}{\pgfqpoint{3.773832in}{1.497008in}}%
\pgfpathcurveto{\pgfqpoint{3.773832in}{1.491185in}}{\pgfqpoint{3.776146in}{1.485598in}}{\pgfqpoint{3.780264in}{1.481480in}}%
\pgfpathcurveto{\pgfqpoint{3.784383in}{1.477362in}}{\pgfqpoint{3.789969in}{1.475048in}}{\pgfqpoint{3.795793in}{1.475048in}}%
\pgfpathlineto{\pgfqpoint{3.795793in}{1.475048in}}%
\pgfpathclose%
\pgfusepath{stroke,fill}%
\end{pgfscope}%
\begin{pgfscope}%
\pgfpathrectangle{\pgfqpoint{1.000000in}{0.979904in}}{\pgfqpoint{6.200000in}{5.960192in}}%
\pgfusepath{clip}%
\pgfsetbuttcap%
\pgfsetroundjoin%
\definecolor{currentfill}{rgb}{0.800000,0.200000,0.200000}%
\pgfsetfillcolor{currentfill}%
\pgfsetlinewidth{1.003750pt}%
\definecolor{currentstroke}{rgb}{0.800000,0.200000,0.200000}%
\pgfsetstrokecolor{currentstroke}%
\pgfsetdash{}{0pt}%
\pgfpathmoveto{\pgfqpoint{3.872260in}{1.546625in}}%
\pgfpathcurveto{\pgfqpoint{3.878084in}{1.546625in}}{\pgfqpoint{3.883670in}{1.548938in}}{\pgfqpoint{3.887788in}{1.553057in}}%
\pgfpathcurveto{\pgfqpoint{3.891906in}{1.557175in}}{\pgfqpoint{3.894220in}{1.562761in}}{\pgfqpoint{3.894220in}{1.568585in}}%
\pgfpathcurveto{\pgfqpoint{3.894220in}{1.574409in}}{\pgfqpoint{3.891906in}{1.579995in}}{\pgfqpoint{3.887788in}{1.584113in}}%
\pgfpathcurveto{\pgfqpoint{3.883670in}{1.588231in}}{\pgfqpoint{3.878084in}{1.590545in}}{\pgfqpoint{3.872260in}{1.590545in}}%
\pgfpathcurveto{\pgfqpoint{3.866436in}{1.590545in}}{\pgfqpoint{3.860850in}{1.588231in}}{\pgfqpoint{3.856731in}{1.584113in}}%
\pgfpathcurveto{\pgfqpoint{3.852613in}{1.579995in}}{\pgfqpoint{3.850299in}{1.574409in}}{\pgfqpoint{3.850299in}{1.568585in}}%
\pgfpathcurveto{\pgfqpoint{3.850299in}{1.562761in}}{\pgfqpoint{3.852613in}{1.557175in}}{\pgfqpoint{3.856731in}{1.553057in}}%
\pgfpathcurveto{\pgfqpoint{3.860850in}{1.548938in}}{\pgfqpoint{3.866436in}{1.546625in}}{\pgfqpoint{3.872260in}{1.546625in}}%
\pgfpathlineto{\pgfqpoint{3.872260in}{1.546625in}}%
\pgfpathclose%
\pgfusepath{stroke,fill}%
\end{pgfscope}%
\begin{pgfscope}%
\pgfpathrectangle{\pgfqpoint{1.000000in}{0.979904in}}{\pgfqpoint{6.200000in}{5.960192in}}%
\pgfusepath{clip}%
\pgfsetbuttcap%
\pgfsetroundjoin%
\definecolor{currentfill}{rgb}{0.800000,0.200000,0.200000}%
\pgfsetfillcolor{currentfill}%
\pgfsetlinewidth{1.003750pt}%
\definecolor{currentstroke}{rgb}{0.800000,0.200000,0.200000}%
\pgfsetstrokecolor{currentstroke}%
\pgfsetdash{}{0pt}%
\pgfpathmoveto{\pgfqpoint{3.911820in}{1.660823in}}%
\pgfpathcurveto{\pgfqpoint{3.917644in}{1.660823in}}{\pgfqpoint{3.923230in}{1.663137in}}{\pgfqpoint{3.927348in}{1.667255in}}%
\pgfpathcurveto{\pgfqpoint{3.931467in}{1.671373in}}{\pgfqpoint{3.933780in}{1.676959in}}{\pgfqpoint{3.933780in}{1.682783in}}%
\pgfpathcurveto{\pgfqpoint{3.933780in}{1.688607in}}{\pgfqpoint{3.931467in}{1.694193in}}{\pgfqpoint{3.927348in}{1.698312in}}%
\pgfpathcurveto{\pgfqpoint{3.923230in}{1.702430in}}{\pgfqpoint{3.917644in}{1.704744in}}{\pgfqpoint{3.911820in}{1.704744in}}%
\pgfpathcurveto{\pgfqpoint{3.905996in}{1.704744in}}{\pgfqpoint{3.900410in}{1.702430in}}{\pgfqpoint{3.896292in}{1.698312in}}%
\pgfpathcurveto{\pgfqpoint{3.892174in}{1.694193in}}{\pgfqpoint{3.889860in}{1.688607in}}{\pgfqpoint{3.889860in}{1.682783in}}%
\pgfpathcurveto{\pgfqpoint{3.889860in}{1.676959in}}{\pgfqpoint{3.892174in}{1.671373in}}{\pgfqpoint{3.896292in}{1.667255in}}%
\pgfpathcurveto{\pgfqpoint{3.900410in}{1.663137in}}{\pgfqpoint{3.905996in}{1.660823in}}{\pgfqpoint{3.911820in}{1.660823in}}%
\pgfpathlineto{\pgfqpoint{3.911820in}{1.660823in}}%
\pgfpathclose%
\pgfusepath{stroke,fill}%
\end{pgfscope}%
\begin{pgfscope}%
\pgfpathrectangle{\pgfqpoint{1.000000in}{0.979904in}}{\pgfqpoint{6.200000in}{5.960192in}}%
\pgfusepath{clip}%
\pgfsetbuttcap%
\pgfsetroundjoin%
\definecolor{currentfill}{rgb}{0.800000,0.200000,0.200000}%
\pgfsetfillcolor{currentfill}%
\pgfsetlinewidth{1.003750pt}%
\definecolor{currentstroke}{rgb}{0.800000,0.200000,0.200000}%
\pgfsetstrokecolor{currentstroke}%
\pgfsetdash{}{0pt}%
\pgfpathmoveto{\pgfqpoint{3.947483in}{1.767335in}}%
\pgfpathcurveto{\pgfqpoint{3.953307in}{1.767335in}}{\pgfqpoint{3.958893in}{1.769649in}}{\pgfqpoint{3.963011in}{1.773767in}}%
\pgfpathcurveto{\pgfqpoint{3.967129in}{1.777885in}}{\pgfqpoint{3.969443in}{1.783471in}}{\pgfqpoint{3.969443in}{1.789295in}}%
\pgfpathcurveto{\pgfqpoint{3.969443in}{1.795119in}}{\pgfqpoint{3.967129in}{1.800705in}}{\pgfqpoint{3.963011in}{1.804823in}}%
\pgfpathcurveto{\pgfqpoint{3.958893in}{1.808941in}}{\pgfqpoint{3.953307in}{1.811255in}}{\pgfqpoint{3.947483in}{1.811255in}}%
\pgfpathcurveto{\pgfqpoint{3.941659in}{1.811255in}}{\pgfqpoint{3.936073in}{1.808941in}}{\pgfqpoint{3.931954in}{1.804823in}}%
\pgfpathcurveto{\pgfqpoint{3.927836in}{1.800705in}}{\pgfqpoint{3.925522in}{1.795119in}}{\pgfqpoint{3.925522in}{1.789295in}}%
\pgfpathcurveto{\pgfqpoint{3.925522in}{1.783471in}}{\pgfqpoint{3.927836in}{1.777885in}}{\pgfqpoint{3.931954in}{1.773767in}}%
\pgfpathcurveto{\pgfqpoint{3.936073in}{1.769649in}}{\pgfqpoint{3.941659in}{1.767335in}}{\pgfqpoint{3.947483in}{1.767335in}}%
\pgfpathlineto{\pgfqpoint{3.947483in}{1.767335in}}%
\pgfpathclose%
\pgfusepath{stroke,fill}%
\end{pgfscope}%
\begin{pgfscope}%
\pgfpathrectangle{\pgfqpoint{1.000000in}{0.979904in}}{\pgfqpoint{6.200000in}{5.960192in}}%
\pgfusepath{clip}%
\pgfsetbuttcap%
\pgfsetroundjoin%
\definecolor{currentfill}{rgb}{0.800000,0.200000,0.200000}%
\pgfsetfillcolor{currentfill}%
\pgfsetlinewidth{1.003750pt}%
\definecolor{currentstroke}{rgb}{0.800000,0.200000,0.200000}%
\pgfsetstrokecolor{currentstroke}%
\pgfsetdash{}{0pt}%
\pgfpathmoveto{\pgfqpoint{4.036885in}{1.813133in}}%
\pgfpathcurveto{\pgfqpoint{4.042709in}{1.813133in}}{\pgfqpoint{4.048295in}{1.815447in}}{\pgfqpoint{4.052413in}{1.819565in}}%
\pgfpathcurveto{\pgfqpoint{4.056532in}{1.823683in}}{\pgfqpoint{4.058845in}{1.829269in}}{\pgfqpoint{4.058845in}{1.835093in}}%
\pgfpathcurveto{\pgfqpoint{4.058845in}{1.840917in}}{\pgfqpoint{4.056532in}{1.846503in}}{\pgfqpoint{4.052413in}{1.850622in}}%
\pgfpathcurveto{\pgfqpoint{4.048295in}{1.854740in}}{\pgfqpoint{4.042709in}{1.857054in}}{\pgfqpoint{4.036885in}{1.857054in}}%
\pgfpathcurveto{\pgfqpoint{4.031061in}{1.857054in}}{\pgfqpoint{4.025475in}{1.854740in}}{\pgfqpoint{4.021357in}{1.850622in}}%
\pgfpathcurveto{\pgfqpoint{4.017239in}{1.846503in}}{\pgfqpoint{4.014925in}{1.840917in}}{\pgfqpoint{4.014925in}{1.835093in}}%
\pgfpathcurveto{\pgfqpoint{4.014925in}{1.829269in}}{\pgfqpoint{4.017239in}{1.823683in}}{\pgfqpoint{4.021357in}{1.819565in}}%
\pgfpathcurveto{\pgfqpoint{4.025475in}{1.815447in}}{\pgfqpoint{4.031061in}{1.813133in}}{\pgfqpoint{4.036885in}{1.813133in}}%
\pgfpathlineto{\pgfqpoint{4.036885in}{1.813133in}}%
\pgfpathclose%
\pgfusepath{stroke,fill}%
\end{pgfscope}%
\begin{pgfscope}%
\pgfpathrectangle{\pgfqpoint{1.000000in}{0.979904in}}{\pgfqpoint{6.200000in}{5.960192in}}%
\pgfusepath{clip}%
\pgfsetbuttcap%
\pgfsetroundjoin%
\definecolor{currentfill}{rgb}{0.800000,0.200000,0.200000}%
\pgfsetfillcolor{currentfill}%
\pgfsetlinewidth{1.003750pt}%
\definecolor{currentstroke}{rgb}{0.800000,0.200000,0.200000}%
\pgfsetstrokecolor{currentstroke}%
\pgfsetdash{}{0pt}%
\pgfpathmoveto{\pgfqpoint{4.123837in}{1.866469in}}%
\pgfpathcurveto{\pgfqpoint{4.129660in}{1.866469in}}{\pgfqpoint{4.135247in}{1.868783in}}{\pgfqpoint{4.139365in}{1.872901in}}%
\pgfpathcurveto{\pgfqpoint{4.143483in}{1.877019in}}{\pgfqpoint{4.145797in}{1.882605in}}{\pgfqpoint{4.145797in}{1.888429in}}%
\pgfpathcurveto{\pgfqpoint{4.145797in}{1.894253in}}{\pgfqpoint{4.143483in}{1.899839in}}{\pgfqpoint{4.139365in}{1.903957in}}%
\pgfpathcurveto{\pgfqpoint{4.135247in}{1.908076in}}{\pgfqpoint{4.129660in}{1.910389in}}{\pgfqpoint{4.123837in}{1.910389in}}%
\pgfpathcurveto{\pgfqpoint{4.118013in}{1.910389in}}{\pgfqpoint{4.112426in}{1.908076in}}{\pgfqpoint{4.108308in}{1.903957in}}%
\pgfpathcurveto{\pgfqpoint{4.104190in}{1.899839in}}{\pgfqpoint{4.101876in}{1.894253in}}{\pgfqpoint{4.101876in}{1.888429in}}%
\pgfpathcurveto{\pgfqpoint{4.101876in}{1.882605in}}{\pgfqpoint{4.104190in}{1.877019in}}{\pgfqpoint{4.108308in}{1.872901in}}%
\pgfpathcurveto{\pgfqpoint{4.112426in}{1.868783in}}{\pgfqpoint{4.118013in}{1.866469in}}{\pgfqpoint{4.123837in}{1.866469in}}%
\pgfpathlineto{\pgfqpoint{4.123837in}{1.866469in}}%
\pgfpathclose%
\pgfusepath{stroke,fill}%
\end{pgfscope}%
\begin{pgfscope}%
\pgfpathrectangle{\pgfqpoint{1.000000in}{0.979904in}}{\pgfqpoint{6.200000in}{5.960192in}}%
\pgfusepath{clip}%
\pgfsetbuttcap%
\pgfsetroundjoin%
\definecolor{currentfill}{rgb}{0.800000,0.200000,0.200000}%
\pgfsetfillcolor{currentfill}%
\pgfsetlinewidth{1.003750pt}%
\definecolor{currentstroke}{rgb}{0.800000,0.200000,0.200000}%
\pgfsetstrokecolor{currentstroke}%
\pgfsetdash{}{0pt}%
\pgfpathmoveto{\pgfqpoint{4.263056in}{1.886413in}}%
\pgfpathcurveto{\pgfqpoint{4.268880in}{1.886413in}}{\pgfqpoint{4.274466in}{1.888727in}}{\pgfqpoint{4.278584in}{1.892845in}}%
\pgfpathcurveto{\pgfqpoint{4.282702in}{1.896963in}}{\pgfqpoint{4.285016in}{1.902549in}}{\pgfqpoint{4.285016in}{1.908373in}}%
\pgfpathcurveto{\pgfqpoint{4.285016in}{1.914197in}}{\pgfqpoint{4.282702in}{1.919783in}}{\pgfqpoint{4.278584in}{1.923901in}}%
\pgfpathcurveto{\pgfqpoint{4.274466in}{1.928019in}}{\pgfqpoint{4.268880in}{1.930333in}}{\pgfqpoint{4.263056in}{1.930333in}}%
\pgfpathcurveto{\pgfqpoint{4.257232in}{1.930333in}}{\pgfqpoint{4.251646in}{1.928019in}}{\pgfqpoint{4.247528in}{1.923901in}}%
\pgfpathcurveto{\pgfqpoint{4.243409in}{1.919783in}}{\pgfqpoint{4.241095in}{1.914197in}}{\pgfqpoint{4.241095in}{1.908373in}}%
\pgfpathcurveto{\pgfqpoint{4.241095in}{1.902549in}}{\pgfqpoint{4.243409in}{1.896963in}}{\pgfqpoint{4.247528in}{1.892845in}}%
\pgfpathcurveto{\pgfqpoint{4.251646in}{1.888727in}}{\pgfqpoint{4.257232in}{1.886413in}}{\pgfqpoint{4.263056in}{1.886413in}}%
\pgfpathlineto{\pgfqpoint{4.263056in}{1.886413in}}%
\pgfpathclose%
\pgfusepath{stroke,fill}%
\end{pgfscope}%
\begin{pgfscope}%
\pgfpathrectangle{\pgfqpoint{1.000000in}{0.979904in}}{\pgfqpoint{6.200000in}{5.960192in}}%
\pgfusepath{clip}%
\pgfsetbuttcap%
\pgfsetroundjoin%
\definecolor{currentfill}{rgb}{0.800000,0.200000,0.200000}%
\pgfsetfillcolor{currentfill}%
\pgfsetlinewidth{1.003750pt}%
\definecolor{currentstroke}{rgb}{0.800000,0.200000,0.200000}%
\pgfsetstrokecolor{currentstroke}%
\pgfsetdash{}{0pt}%
\pgfpathmoveto{\pgfqpoint{4.339943in}{1.961536in}}%
\pgfpathcurveto{\pgfqpoint{4.345767in}{1.961536in}}{\pgfqpoint{4.351353in}{1.963850in}}{\pgfqpoint{4.355471in}{1.967968in}}%
\pgfpathcurveto{\pgfqpoint{4.359589in}{1.972086in}}{\pgfqpoint{4.361903in}{1.977673in}}{\pgfqpoint{4.361903in}{1.983496in}}%
\pgfpathcurveto{\pgfqpoint{4.361903in}{1.989320in}}{\pgfqpoint{4.359589in}{1.994907in}}{\pgfqpoint{4.355471in}{1.999025in}}%
\pgfpathcurveto{\pgfqpoint{4.351353in}{2.003143in}}{\pgfqpoint{4.345767in}{2.005457in}}{\pgfqpoint{4.339943in}{2.005457in}}%
\pgfpathcurveto{\pgfqpoint{4.334119in}{2.005457in}}{\pgfqpoint{4.328533in}{2.003143in}}{\pgfqpoint{4.324415in}{1.999025in}}%
\pgfpathcurveto{\pgfqpoint{4.320296in}{1.994907in}}{\pgfqpoint{4.317983in}{1.989320in}}{\pgfqpoint{4.317983in}{1.983496in}}%
\pgfpathcurveto{\pgfqpoint{4.317983in}{1.977673in}}{\pgfqpoint{4.320296in}{1.972086in}}{\pgfqpoint{4.324415in}{1.967968in}}%
\pgfpathcurveto{\pgfqpoint{4.328533in}{1.963850in}}{\pgfqpoint{4.334119in}{1.961536in}}{\pgfqpoint{4.339943in}{1.961536in}}%
\pgfpathlineto{\pgfqpoint{4.339943in}{1.961536in}}%
\pgfpathclose%
\pgfusepath{stroke,fill}%
\end{pgfscope}%
\begin{pgfscope}%
\pgfpathrectangle{\pgfqpoint{1.000000in}{0.979904in}}{\pgfqpoint{6.200000in}{5.960192in}}%
\pgfusepath{clip}%
\pgfsetbuttcap%
\pgfsetroundjoin%
\definecolor{currentfill}{rgb}{0.800000,0.200000,0.200000}%
\pgfsetfillcolor{currentfill}%
\pgfsetlinewidth{1.003750pt}%
\definecolor{currentstroke}{rgb}{0.800000,0.200000,0.200000}%
\pgfsetstrokecolor{currentstroke}%
\pgfsetdash{}{0pt}%
\pgfpathmoveto{\pgfqpoint{4.302088in}{2.103994in}}%
\pgfpathcurveto{\pgfqpoint{4.307912in}{2.103994in}}{\pgfqpoint{4.313499in}{2.106308in}}{\pgfqpoint{4.317617in}{2.110426in}}%
\pgfpathcurveto{\pgfqpoint{4.321735in}{2.114544in}}{\pgfqpoint{4.324049in}{2.120130in}}{\pgfqpoint{4.324049in}{2.125954in}}%
\pgfpathcurveto{\pgfqpoint{4.324049in}{2.131778in}}{\pgfqpoint{4.321735in}{2.137364in}}{\pgfqpoint{4.317617in}{2.141482in}}%
\pgfpathcurveto{\pgfqpoint{4.313499in}{2.145600in}}{\pgfqpoint{4.307912in}{2.147914in}}{\pgfqpoint{4.302088in}{2.147914in}}%
\pgfpathcurveto{\pgfqpoint{4.296264in}{2.147914in}}{\pgfqpoint{4.290678in}{2.145600in}}{\pgfqpoint{4.286560in}{2.141482in}}%
\pgfpathcurveto{\pgfqpoint{4.282442in}{2.137364in}}{\pgfqpoint{4.280128in}{2.131778in}}{\pgfqpoint{4.280128in}{2.125954in}}%
\pgfpathcurveto{\pgfqpoint{4.280128in}{2.120130in}}{\pgfqpoint{4.282442in}{2.114544in}}{\pgfqpoint{4.286560in}{2.110426in}}%
\pgfpathcurveto{\pgfqpoint{4.290678in}{2.106308in}}{\pgfqpoint{4.296264in}{2.103994in}}{\pgfqpoint{4.302088in}{2.103994in}}%
\pgfpathlineto{\pgfqpoint{4.302088in}{2.103994in}}%
\pgfpathclose%
\pgfusepath{stroke,fill}%
\end{pgfscope}%
\begin{pgfscope}%
\pgfpathrectangle{\pgfqpoint{1.000000in}{0.979904in}}{\pgfqpoint{6.200000in}{5.960192in}}%
\pgfusepath{clip}%
\pgfsetbuttcap%
\pgfsetroundjoin%
\definecolor{currentfill}{rgb}{0.800000,0.200000,0.200000}%
\pgfsetfillcolor{currentfill}%
\pgfsetlinewidth{1.003750pt}%
\definecolor{currentstroke}{rgb}{0.800000,0.200000,0.200000}%
\pgfsetstrokecolor{currentstroke}%
\pgfsetdash{}{0pt}%
\pgfpathmoveto{\pgfqpoint{4.349163in}{2.191648in}}%
\pgfpathcurveto{\pgfqpoint{4.354987in}{2.191648in}}{\pgfqpoint{4.360573in}{2.193961in}}{\pgfqpoint{4.364691in}{2.198080in}}%
\pgfpathcurveto{\pgfqpoint{4.368809in}{2.202198in}}{\pgfqpoint{4.371123in}{2.207784in}}{\pgfqpoint{4.371123in}{2.213608in}}%
\pgfpathcurveto{\pgfqpoint{4.371123in}{2.219432in}}{\pgfqpoint{4.368809in}{2.225018in}}{\pgfqpoint{4.364691in}{2.229136in}}%
\pgfpathcurveto{\pgfqpoint{4.360573in}{2.233254in}}{\pgfqpoint{4.354987in}{2.235568in}}{\pgfqpoint{4.349163in}{2.235568in}}%
\pgfpathcurveto{\pgfqpoint{4.343339in}{2.235568in}}{\pgfqpoint{4.337752in}{2.233254in}}{\pgfqpoint{4.333634in}{2.229136in}}%
\pgfpathcurveto{\pgfqpoint{4.329516in}{2.225018in}}{\pgfqpoint{4.327202in}{2.219432in}}{\pgfqpoint{4.327202in}{2.213608in}}%
\pgfpathcurveto{\pgfqpoint{4.327202in}{2.207784in}}{\pgfqpoint{4.329516in}{2.202198in}}{\pgfqpoint{4.333634in}{2.198080in}}%
\pgfpathcurveto{\pgfqpoint{4.337752in}{2.193961in}}{\pgfqpoint{4.343339in}{2.191648in}}{\pgfqpoint{4.349163in}{2.191648in}}%
\pgfpathlineto{\pgfqpoint{4.349163in}{2.191648in}}%
\pgfpathclose%
\pgfusepath{stroke,fill}%
\end{pgfscope}%
\begin{pgfscope}%
\pgfpathrectangle{\pgfqpoint{1.000000in}{0.979904in}}{\pgfqpoint{6.200000in}{5.960192in}}%
\pgfusepath{clip}%
\pgfsetbuttcap%
\pgfsetroundjoin%
\definecolor{currentfill}{rgb}{0.800000,0.200000,0.200000}%
\pgfsetfillcolor{currentfill}%
\pgfsetlinewidth{1.003750pt}%
\definecolor{currentstroke}{rgb}{0.800000,0.200000,0.200000}%
\pgfsetstrokecolor{currentstroke}%
\pgfsetdash{}{0pt}%
\pgfpathmoveto{\pgfqpoint{4.422048in}{2.269628in}}%
\pgfpathcurveto{\pgfqpoint{4.427871in}{2.269628in}}{\pgfqpoint{4.433458in}{2.271942in}}{\pgfqpoint{4.437576in}{2.276060in}}%
\pgfpathcurveto{\pgfqpoint{4.441694in}{2.280179in}}{\pgfqpoint{4.444008in}{2.285765in}}{\pgfqpoint{4.444008in}{2.291589in}}%
\pgfpathcurveto{\pgfqpoint{4.444008in}{2.297413in}}{\pgfqpoint{4.441694in}{2.302999in}}{\pgfqpoint{4.437576in}{2.307117in}}%
\pgfpathcurveto{\pgfqpoint{4.433458in}{2.311235in}}{\pgfqpoint{4.427871in}{2.313549in}}{\pgfqpoint{4.422048in}{2.313549in}}%
\pgfpathcurveto{\pgfqpoint{4.416224in}{2.313549in}}{\pgfqpoint{4.410637in}{2.311235in}}{\pgfqpoint{4.406519in}{2.307117in}}%
\pgfpathcurveto{\pgfqpoint{4.402401in}{2.302999in}}{\pgfqpoint{4.400087in}{2.297413in}}{\pgfqpoint{4.400087in}{2.291589in}}%
\pgfpathcurveto{\pgfqpoint{4.400087in}{2.285765in}}{\pgfqpoint{4.402401in}{2.280179in}}{\pgfqpoint{4.406519in}{2.276060in}}%
\pgfpathcurveto{\pgfqpoint{4.410637in}{2.271942in}}{\pgfqpoint{4.416224in}{2.269628in}}{\pgfqpoint{4.422048in}{2.269628in}}%
\pgfpathlineto{\pgfqpoint{4.422048in}{2.269628in}}%
\pgfpathclose%
\pgfusepath{stroke,fill}%
\end{pgfscope}%
\begin{pgfscope}%
\pgfpathrectangle{\pgfqpoint{1.000000in}{0.979904in}}{\pgfqpoint{6.200000in}{5.960192in}}%
\pgfusepath{clip}%
\pgfsetbuttcap%
\pgfsetroundjoin%
\definecolor{currentfill}{rgb}{0.800000,0.200000,0.200000}%
\pgfsetfillcolor{currentfill}%
\pgfsetlinewidth{1.003750pt}%
\definecolor{currentstroke}{rgb}{0.800000,0.200000,0.200000}%
\pgfsetstrokecolor{currentstroke}%
\pgfsetdash{}{0pt}%
\pgfpathmoveto{\pgfqpoint{4.444286in}{2.369408in}}%
\pgfpathcurveto{\pgfqpoint{4.450109in}{2.369408in}}{\pgfqpoint{4.455696in}{2.371722in}}{\pgfqpoint{4.459814in}{2.375840in}}%
\pgfpathcurveto{\pgfqpoint{4.463932in}{2.379958in}}{\pgfqpoint{4.466246in}{2.385544in}}{\pgfqpoint{4.466246in}{2.391368in}}%
\pgfpathcurveto{\pgfqpoint{4.466246in}{2.397192in}}{\pgfqpoint{4.463932in}{2.402778in}}{\pgfqpoint{4.459814in}{2.406896in}}%
\pgfpathcurveto{\pgfqpoint{4.455696in}{2.411015in}}{\pgfqpoint{4.450109in}{2.413328in}}{\pgfqpoint{4.444286in}{2.413328in}}%
\pgfpathcurveto{\pgfqpoint{4.438462in}{2.413328in}}{\pgfqpoint{4.432875in}{2.411015in}}{\pgfqpoint{4.428757in}{2.406896in}}%
\pgfpathcurveto{\pgfqpoint{4.424639in}{2.402778in}}{\pgfqpoint{4.422325in}{2.397192in}}{\pgfqpoint{4.422325in}{2.391368in}}%
\pgfpathcurveto{\pgfqpoint{4.422325in}{2.385544in}}{\pgfqpoint{4.424639in}{2.379958in}}{\pgfqpoint{4.428757in}{2.375840in}}%
\pgfpathcurveto{\pgfqpoint{4.432875in}{2.371722in}}{\pgfqpoint{4.438462in}{2.369408in}}{\pgfqpoint{4.444286in}{2.369408in}}%
\pgfpathlineto{\pgfqpoint{4.444286in}{2.369408in}}%
\pgfpathclose%
\pgfusepath{stroke,fill}%
\end{pgfscope}%
\begin{pgfscope}%
\pgfpathrectangle{\pgfqpoint{1.000000in}{0.979904in}}{\pgfqpoint{6.200000in}{5.960192in}}%
\pgfusepath{clip}%
\pgfsetbuttcap%
\pgfsetroundjoin%
\definecolor{currentfill}{rgb}{0.800000,0.200000,0.200000}%
\pgfsetfillcolor{currentfill}%
\pgfsetlinewidth{1.003750pt}%
\definecolor{currentstroke}{rgb}{0.800000,0.200000,0.200000}%
\pgfsetstrokecolor{currentstroke}%
\pgfsetdash{}{0pt}%
\pgfpathmoveto{\pgfqpoint{4.519911in}{2.454024in}}%
\pgfpathcurveto{\pgfqpoint{4.525735in}{2.454024in}}{\pgfqpoint{4.531321in}{2.456338in}}{\pgfqpoint{4.535439in}{2.460456in}}%
\pgfpathcurveto{\pgfqpoint{4.539558in}{2.464574in}}{\pgfqpoint{4.541871in}{2.470160in}}{\pgfqpoint{4.541871in}{2.475984in}}%
\pgfpathcurveto{\pgfqpoint{4.541871in}{2.481808in}}{\pgfqpoint{4.539558in}{2.487394in}}{\pgfqpoint{4.535439in}{2.491512in}}%
\pgfpathcurveto{\pgfqpoint{4.531321in}{2.495631in}}{\pgfqpoint{4.525735in}{2.497944in}}{\pgfqpoint{4.519911in}{2.497944in}}%
\pgfpathcurveto{\pgfqpoint{4.514087in}{2.497944in}}{\pgfqpoint{4.508501in}{2.495631in}}{\pgfqpoint{4.504383in}{2.491512in}}%
\pgfpathcurveto{\pgfqpoint{4.500265in}{2.487394in}}{\pgfqpoint{4.497951in}{2.481808in}}{\pgfqpoint{4.497951in}{2.475984in}}%
\pgfpathcurveto{\pgfqpoint{4.497951in}{2.470160in}}{\pgfqpoint{4.500265in}{2.464574in}}{\pgfqpoint{4.504383in}{2.460456in}}%
\pgfpathcurveto{\pgfqpoint{4.508501in}{2.456338in}}{\pgfqpoint{4.514087in}{2.454024in}}{\pgfqpoint{4.519911in}{2.454024in}}%
\pgfpathlineto{\pgfqpoint{4.519911in}{2.454024in}}%
\pgfpathclose%
\pgfusepath{stroke,fill}%
\end{pgfscope}%
\begin{pgfscope}%
\pgfpathrectangle{\pgfqpoint{1.000000in}{0.979904in}}{\pgfqpoint{6.200000in}{5.960192in}}%
\pgfusepath{clip}%
\pgfsetbuttcap%
\pgfsetroundjoin%
\definecolor{currentfill}{rgb}{0.800000,0.200000,0.200000}%
\pgfsetfillcolor{currentfill}%
\pgfsetlinewidth{1.003750pt}%
\definecolor{currentstroke}{rgb}{0.800000,0.200000,0.200000}%
\pgfsetstrokecolor{currentstroke}%
\pgfsetdash{}{0pt}%
\pgfpathmoveto{\pgfqpoint{4.467076in}{2.570110in}}%
\pgfpathcurveto{\pgfqpoint{4.472900in}{2.570110in}}{\pgfqpoint{4.478486in}{2.572424in}}{\pgfqpoint{4.482604in}{2.576542in}}%
\pgfpathcurveto{\pgfqpoint{4.486722in}{2.580660in}}{\pgfqpoint{4.489036in}{2.586246in}}{\pgfqpoint{4.489036in}{2.592070in}}%
\pgfpathcurveto{\pgfqpoint{4.489036in}{2.597894in}}{\pgfqpoint{4.486722in}{2.603480in}}{\pgfqpoint{4.482604in}{2.607598in}}%
\pgfpathcurveto{\pgfqpoint{4.478486in}{2.611716in}}{\pgfqpoint{4.472900in}{2.614030in}}{\pgfqpoint{4.467076in}{2.614030in}}%
\pgfpathcurveto{\pgfqpoint{4.461252in}{2.614030in}}{\pgfqpoint{4.455666in}{2.611716in}}{\pgfqpoint{4.451548in}{2.607598in}}%
\pgfpathcurveto{\pgfqpoint{4.447430in}{2.603480in}}{\pgfqpoint{4.445116in}{2.597894in}}{\pgfqpoint{4.445116in}{2.592070in}}%
\pgfpathcurveto{\pgfqpoint{4.445116in}{2.586246in}}{\pgfqpoint{4.447430in}{2.580660in}}{\pgfqpoint{4.451548in}{2.576542in}}%
\pgfpathcurveto{\pgfqpoint{4.455666in}{2.572424in}}{\pgfqpoint{4.461252in}{2.570110in}}{\pgfqpoint{4.467076in}{2.570110in}}%
\pgfpathlineto{\pgfqpoint{4.467076in}{2.570110in}}%
\pgfpathclose%
\pgfusepath{stroke,fill}%
\end{pgfscope}%
\begin{pgfscope}%
\pgfpathrectangle{\pgfqpoint{1.000000in}{0.979904in}}{\pgfqpoint{6.200000in}{5.960192in}}%
\pgfusepath{clip}%
\pgfsetbuttcap%
\pgfsetroundjoin%
\definecolor{currentfill}{rgb}{0.800000,0.200000,0.200000}%
\pgfsetfillcolor{currentfill}%
\pgfsetlinewidth{1.003750pt}%
\definecolor{currentstroke}{rgb}{0.800000,0.200000,0.200000}%
\pgfsetstrokecolor{currentstroke}%
\pgfsetdash{}{0pt}%
\pgfpathmoveto{\pgfqpoint{4.560230in}{2.658121in}}%
\pgfpathcurveto{\pgfqpoint{4.566054in}{2.658121in}}{\pgfqpoint{4.571640in}{2.660435in}}{\pgfqpoint{4.575758in}{2.664553in}}%
\pgfpathcurveto{\pgfqpoint{4.579876in}{2.668671in}}{\pgfqpoint{4.582190in}{2.674257in}}{\pgfqpoint{4.582190in}{2.680081in}}%
\pgfpathcurveto{\pgfqpoint{4.582190in}{2.685905in}}{\pgfqpoint{4.579876in}{2.691491in}}{\pgfqpoint{4.575758in}{2.695609in}}%
\pgfpathcurveto{\pgfqpoint{4.571640in}{2.699728in}}{\pgfqpoint{4.566054in}{2.702041in}}{\pgfqpoint{4.560230in}{2.702041in}}%
\pgfpathcurveto{\pgfqpoint{4.554406in}{2.702041in}}{\pgfqpoint{4.548819in}{2.699728in}}{\pgfqpoint{4.544701in}{2.695609in}}%
\pgfpathcurveto{\pgfqpoint{4.540583in}{2.691491in}}{\pgfqpoint{4.538269in}{2.685905in}}{\pgfqpoint{4.538269in}{2.680081in}}%
\pgfpathcurveto{\pgfqpoint{4.538269in}{2.674257in}}{\pgfqpoint{4.540583in}{2.668671in}}{\pgfqpoint{4.544701in}{2.664553in}}%
\pgfpathcurveto{\pgfqpoint{4.548819in}{2.660435in}}{\pgfqpoint{4.554406in}{2.658121in}}{\pgfqpoint{4.560230in}{2.658121in}}%
\pgfpathlineto{\pgfqpoint{4.560230in}{2.658121in}}%
\pgfpathclose%
\pgfusepath{stroke,fill}%
\end{pgfscope}%
\begin{pgfscope}%
\pgfpathrectangle{\pgfqpoint{1.000000in}{0.979904in}}{\pgfqpoint{6.200000in}{5.960192in}}%
\pgfusepath{clip}%
\pgfsetbuttcap%
\pgfsetroundjoin%
\definecolor{currentfill}{rgb}{0.200000,0.800000,0.200000}%
\pgfsetfillcolor{currentfill}%
\pgfsetlinewidth{1.003750pt}%
\definecolor{currentstroke}{rgb}{0.200000,0.800000,0.200000}%
\pgfsetstrokecolor{currentstroke}%
\pgfsetdash{}{0pt}%
\pgfpathmoveto{\pgfqpoint{4.507201in}{2.765751in}}%
\pgfpathcurveto{\pgfqpoint{4.513025in}{2.765751in}}{\pgfqpoint{4.518611in}{2.768065in}}{\pgfqpoint{4.522730in}{2.772183in}}%
\pgfpathcurveto{\pgfqpoint{4.526848in}{2.776301in}}{\pgfqpoint{4.529162in}{2.781887in}}{\pgfqpoint{4.529162in}{2.787711in}}%
\pgfpathcurveto{\pgfqpoint{4.529162in}{2.793535in}}{\pgfqpoint{4.526848in}{2.799121in}}{\pgfqpoint{4.522730in}{2.803240in}}%
\pgfpathcurveto{\pgfqpoint{4.518611in}{2.807358in}}{\pgfqpoint{4.513025in}{2.809672in}}{\pgfqpoint{4.507201in}{2.809672in}}%
\pgfpathcurveto{\pgfqpoint{4.501377in}{2.809672in}}{\pgfqpoint{4.495791in}{2.807358in}}{\pgfqpoint{4.491673in}{2.803240in}}%
\pgfpathcurveto{\pgfqpoint{4.487555in}{2.799121in}}{\pgfqpoint{4.485241in}{2.793535in}}{\pgfqpoint{4.485241in}{2.787711in}}%
\pgfpathcurveto{\pgfqpoint{4.485241in}{2.781887in}}{\pgfqpoint{4.487555in}{2.776301in}}{\pgfqpoint{4.491673in}{2.772183in}}%
\pgfpathcurveto{\pgfqpoint{4.495791in}{2.768065in}}{\pgfqpoint{4.501377in}{2.765751in}}{\pgfqpoint{4.507201in}{2.765751in}}%
\pgfpathlineto{\pgfqpoint{4.507201in}{2.765751in}}%
\pgfpathclose%
\pgfusepath{stroke,fill}%
\end{pgfscope}%
\begin{pgfscope}%
\pgfpathrectangle{\pgfqpoint{1.000000in}{0.979904in}}{\pgfqpoint{6.200000in}{5.960192in}}%
\pgfusepath{clip}%
\pgfsetbuttcap%
\pgfsetroundjoin%
\definecolor{currentfill}{rgb}{0.800000,0.200000,0.200000}%
\pgfsetfillcolor{currentfill}%
\pgfsetlinewidth{1.003750pt}%
\definecolor{currentstroke}{rgb}{0.800000,0.200000,0.200000}%
\pgfsetstrokecolor{currentstroke}%
\pgfsetdash{}{0pt}%
\pgfpathmoveto{\pgfqpoint{4.568379in}{2.865802in}}%
\pgfpathcurveto{\pgfqpoint{4.574203in}{2.865802in}}{\pgfqpoint{4.579789in}{2.868116in}}{\pgfqpoint{4.583907in}{2.872234in}}%
\pgfpathcurveto{\pgfqpoint{4.588025in}{2.876352in}}{\pgfqpoint{4.590339in}{2.881939in}}{\pgfqpoint{4.590339in}{2.887762in}}%
\pgfpathcurveto{\pgfqpoint{4.590339in}{2.893586in}}{\pgfqpoint{4.588025in}{2.899173in}}{\pgfqpoint{4.583907in}{2.903291in}}%
\pgfpathcurveto{\pgfqpoint{4.579789in}{2.907409in}}{\pgfqpoint{4.574203in}{2.909723in}}{\pgfqpoint{4.568379in}{2.909723in}}%
\pgfpathcurveto{\pgfqpoint{4.562555in}{2.909723in}}{\pgfqpoint{4.556969in}{2.907409in}}{\pgfqpoint{4.552851in}{2.903291in}}%
\pgfpathcurveto{\pgfqpoint{4.548732in}{2.899173in}}{\pgfqpoint{4.546419in}{2.893586in}}{\pgfqpoint{4.546419in}{2.887762in}}%
\pgfpathcurveto{\pgfqpoint{4.546419in}{2.881939in}}{\pgfqpoint{4.548732in}{2.876352in}}{\pgfqpoint{4.552851in}{2.872234in}}%
\pgfpathcurveto{\pgfqpoint{4.556969in}{2.868116in}}{\pgfqpoint{4.562555in}{2.865802in}}{\pgfqpoint{4.568379in}{2.865802in}}%
\pgfpathlineto{\pgfqpoint{4.568379in}{2.865802in}}%
\pgfpathclose%
\pgfusepath{stroke,fill}%
\end{pgfscope}%
\begin{pgfscope}%
\pgfpathrectangle{\pgfqpoint{1.000000in}{0.979904in}}{\pgfqpoint{6.200000in}{5.960192in}}%
\pgfusepath{clip}%
\pgfsetbuttcap%
\pgfsetmiterjoin%
\pgfsetlinewidth{1.003750pt}%
\definecolor{currentstroke}{rgb}{0.800000,0.200000,0.200000}%
\pgfsetstrokecolor{currentstroke}%
\pgfsetdash{}{0pt}%
\pgfpathmoveto{\pgfqpoint{2.949351in}{1.277768in}}%
\pgfpathcurveto{\pgfqpoint{3.371917in}{1.277768in}}{\pgfqpoint{3.777233in}{1.445655in}}{\pgfqpoint{4.076032in}{1.744454in}}%
\pgfpathcurveto{\pgfqpoint{4.374831in}{2.043254in}}{\pgfqpoint{4.542719in}{2.448569in}}{\pgfqpoint{4.542719in}{2.871135in}}%
\pgfpathcurveto{\pgfqpoint{4.542719in}{3.293701in}}{\pgfqpoint{4.374831in}{3.699017in}}{\pgfqpoint{4.076032in}{3.997816in}}%
\pgfpathcurveto{\pgfqpoint{3.777233in}{4.296615in}}{\pgfqpoint{3.371917in}{4.464503in}}{\pgfqpoint{2.949351in}{4.464503in}}%
\pgfpathcurveto{\pgfqpoint{2.526785in}{4.464503in}}{\pgfqpoint{2.121470in}{4.296615in}}{\pgfqpoint{1.822670in}{3.997816in}}%
\pgfpathcurveto{\pgfqpoint{1.523871in}{3.699017in}}{\pgfqpoint{1.355984in}{3.293701in}}{\pgfqpoint{1.355984in}{2.871135in}}%
\pgfpathcurveto{\pgfqpoint{1.355984in}{2.448569in}}{\pgfqpoint{1.523871in}{2.043254in}}{\pgfqpoint{1.822670in}{1.744454in}}%
\pgfpathcurveto{\pgfqpoint{2.121470in}{1.445655in}}{\pgfqpoint{2.526785in}{1.277768in}}{\pgfqpoint{2.949351in}{1.277768in}}%
\pgfpathlineto{\pgfqpoint{2.949351in}{1.277768in}}%
\pgfpathclose%
\pgfusepath{stroke}%
\end{pgfscope}%
\begin{pgfscope}%
\pgfpathrectangle{\pgfqpoint{1.000000in}{0.979904in}}{\pgfqpoint{6.200000in}{5.960192in}}%
\pgfusepath{clip}%
\pgfsetbuttcap%
\pgfsetroundjoin%
\definecolor{currentfill}{rgb}{0.000000,0.000000,0.000000}%
\pgfsetfillcolor{currentfill}%
\pgfsetlinewidth{1.003750pt}%
\definecolor{currentstroke}{rgb}{0.000000,0.000000,0.000000}%
\pgfsetstrokecolor{currentstroke}%
\pgfsetdash{}{0pt}%
\pgfsys@defobject{currentmarker}{\pgfqpoint{-0.021960in}{-0.021960in}}{\pgfqpoint{0.021960in}{0.021960in}}{%
\pgfpathmoveto{\pgfqpoint{0.000000in}{-0.021960in}}%
\pgfpathcurveto{\pgfqpoint{0.005824in}{-0.021960in}}{\pgfqpoint{0.011410in}{-0.019646in}}{\pgfqpoint{0.015528in}{-0.015528in}}%
\pgfpathcurveto{\pgfqpoint{0.019646in}{-0.011410in}}{\pgfqpoint{0.021960in}{-0.005824in}}{\pgfqpoint{0.021960in}{0.000000in}}%
\pgfpathcurveto{\pgfqpoint{0.021960in}{0.005824in}}{\pgfqpoint{0.019646in}{0.011410in}}{\pgfqpoint{0.015528in}{0.015528in}}%
\pgfpathcurveto{\pgfqpoint{0.011410in}{0.019646in}}{\pgfqpoint{0.005824in}{0.021960in}}{\pgfqpoint{0.000000in}{0.021960in}}%
\pgfpathcurveto{\pgfqpoint{-0.005824in}{0.021960in}}{\pgfqpoint{-0.011410in}{0.019646in}}{\pgfqpoint{-0.015528in}{0.015528in}}%
\pgfpathcurveto{\pgfqpoint{-0.019646in}{0.011410in}}{\pgfqpoint{-0.021960in}{0.005824in}}{\pgfqpoint{-0.021960in}{0.000000in}}%
\pgfpathcurveto{\pgfqpoint{-0.021960in}{-0.005824in}}{\pgfqpoint{-0.019646in}{-0.011410in}}{\pgfqpoint{-0.015528in}{-0.015528in}}%
\pgfpathcurveto{\pgfqpoint{-0.011410in}{-0.019646in}}{\pgfqpoint{-0.005824in}{-0.021960in}}{\pgfqpoint{0.000000in}{-0.021960in}}%
\pgfpathlineto{\pgfqpoint{0.000000in}{-0.021960in}}%
\pgfpathclose%
\pgfusepath{stroke,fill}%
}%
\begin{pgfscope}%
\pgfsys@transformshift{2.949351in}{2.871135in}%
\pgfsys@useobject{currentmarker}{}%
\end{pgfscope}%
\end{pgfscope}%
\begin{pgfscope}%
\pgfpathrectangle{\pgfqpoint{1.000000in}{0.979904in}}{\pgfqpoint{6.200000in}{5.960192in}}%
\pgfusepath{clip}%
\pgfsetbuttcap%
\pgfsetmiterjoin%
\pgfsetlinewidth{1.003750pt}%
\definecolor{currentstroke}{rgb}{0.200000,0.800000,0.200000}%
\pgfsetstrokecolor{currentstroke}%
\pgfsetdash{}{0pt}%
\pgfpathmoveto{\pgfqpoint{4.972638in}{2.743016in}}%
\pgfpathcurveto{\pgfqpoint{5.435940in}{2.743016in}}{\pgfqpoint{5.880329in}{2.927088in}}{\pgfqpoint{6.207933in}{3.254692in}}%
\pgfpathcurveto{\pgfqpoint{6.535537in}{3.582296in}}{\pgfqpoint{6.719609in}{4.026685in}}{\pgfqpoint{6.719609in}{4.489987in}}%
\pgfpathcurveto{\pgfqpoint{6.719609in}{4.953289in}}{\pgfqpoint{6.535537in}{5.397678in}}{\pgfqpoint{6.207933in}{5.725282in}}%
\pgfpathcurveto{\pgfqpoint{5.880329in}{6.052886in}}{\pgfqpoint{5.435940in}{6.236958in}}{\pgfqpoint{4.972638in}{6.236958in}}%
\pgfpathcurveto{\pgfqpoint{4.509336in}{6.236958in}}{\pgfqpoint{4.064947in}{6.052886in}}{\pgfqpoint{3.737343in}{5.725282in}}%
\pgfpathcurveto{\pgfqpoint{3.409739in}{5.397678in}}{\pgfqpoint{3.225667in}{4.953289in}}{\pgfqpoint{3.225667in}{4.489987in}}%
\pgfpathcurveto{\pgfqpoint{3.225667in}{4.026685in}}{\pgfqpoint{3.409739in}{3.582296in}}{\pgfqpoint{3.737343in}{3.254692in}}%
\pgfpathcurveto{\pgfqpoint{4.064947in}{2.927088in}}{\pgfqpoint{4.509336in}{2.743016in}}{\pgfqpoint{4.972638in}{2.743016in}}%
\pgfpathlineto{\pgfqpoint{4.972638in}{2.743016in}}%
\pgfpathclose%
\pgfusepath{stroke}%
\end{pgfscope}%
\begin{pgfscope}%
\pgfpathrectangle{\pgfqpoint{1.000000in}{0.979904in}}{\pgfqpoint{6.200000in}{5.960192in}}%
\pgfusepath{clip}%
\pgfsetbuttcap%
\pgfsetroundjoin%
\definecolor{currentfill}{rgb}{0.000000,0.000000,0.000000}%
\pgfsetfillcolor{currentfill}%
\pgfsetlinewidth{1.003750pt}%
\definecolor{currentstroke}{rgb}{0.000000,0.000000,0.000000}%
\pgfsetstrokecolor{currentstroke}%
\pgfsetdash{}{0pt}%
\pgfsys@defobject{currentmarker}{\pgfqpoint{-0.021960in}{-0.021960in}}{\pgfqpoint{0.021960in}{0.021960in}}{%
\pgfpathmoveto{\pgfqpoint{0.000000in}{-0.021960in}}%
\pgfpathcurveto{\pgfqpoint{0.005824in}{-0.021960in}}{\pgfqpoint{0.011410in}{-0.019646in}}{\pgfqpoint{0.015528in}{-0.015528in}}%
\pgfpathcurveto{\pgfqpoint{0.019646in}{-0.011410in}}{\pgfqpoint{0.021960in}{-0.005824in}}{\pgfqpoint{0.021960in}{0.000000in}}%
\pgfpathcurveto{\pgfqpoint{0.021960in}{0.005824in}}{\pgfqpoint{0.019646in}{0.011410in}}{\pgfqpoint{0.015528in}{0.015528in}}%
\pgfpathcurveto{\pgfqpoint{0.011410in}{0.019646in}}{\pgfqpoint{0.005824in}{0.021960in}}{\pgfqpoint{0.000000in}{0.021960in}}%
\pgfpathcurveto{\pgfqpoint{-0.005824in}{0.021960in}}{\pgfqpoint{-0.011410in}{0.019646in}}{\pgfqpoint{-0.015528in}{0.015528in}}%
\pgfpathcurveto{\pgfqpoint{-0.019646in}{0.011410in}}{\pgfqpoint{-0.021960in}{0.005824in}}{\pgfqpoint{-0.021960in}{0.000000in}}%
\pgfpathcurveto{\pgfqpoint{-0.021960in}{-0.005824in}}{\pgfqpoint{-0.019646in}{-0.011410in}}{\pgfqpoint{-0.015528in}{-0.015528in}}%
\pgfpathcurveto{\pgfqpoint{-0.011410in}{-0.019646in}}{\pgfqpoint{-0.005824in}{-0.021960in}}{\pgfqpoint{0.000000in}{-0.021960in}}%
\pgfpathlineto{\pgfqpoint{0.000000in}{-0.021960in}}%
\pgfpathclose%
\pgfusepath{stroke,fill}%
}%
\begin{pgfscope}%
\pgfsys@transformshift{4.972638in}{4.489987in}%
\pgfsys@useobject{currentmarker}{}%
\end{pgfscope}%
\end{pgfscope}%
\begin{pgfscope}%
\pgfpathrectangle{\pgfqpoint{1.000000in}{0.979904in}}{\pgfqpoint{6.200000in}{5.960192in}}%
\pgfusepath{clip}%
\pgfsetbuttcap%
\pgfsetmiterjoin%
\pgfsetlinewidth{1.003750pt}%
\definecolor{currentstroke}{rgb}{0.200000,0.200000,0.800000}%
\pgfsetstrokecolor{currentstroke}%
\pgfsetdash{}{0pt}%
\pgfpathmoveto{\pgfqpoint{3.699003in}{4.756614in}}%
\pgfpathcurveto{\pgfqpoint{3.947152in}{4.756614in}}{\pgfqpoint{4.185171in}{4.855204in}}{\pgfqpoint{4.360638in}{5.030672in}}%
\pgfpathcurveto{\pgfqpoint{4.536106in}{5.206140in}}{\pgfqpoint{4.634697in}{5.444159in}}{\pgfqpoint{4.634697in}{5.692307in}}%
\pgfpathcurveto{\pgfqpoint{4.634697in}{5.940456in}}{\pgfqpoint{4.536106in}{6.178475in}}{\pgfqpoint{4.360638in}{6.353943in}}%
\pgfpathcurveto{\pgfqpoint{4.185171in}{6.529411in}}{\pgfqpoint{3.947152in}{6.628001in}}{\pgfqpoint{3.699003in}{6.628001in}}%
\pgfpathcurveto{\pgfqpoint{3.450854in}{6.628001in}}{\pgfqpoint{3.212835in}{6.529411in}}{\pgfqpoint{3.037367in}{6.353943in}}%
\pgfpathcurveto{\pgfqpoint{2.861900in}{6.178475in}}{\pgfqpoint{2.763309in}{5.940456in}}{\pgfqpoint{2.763309in}{5.692307in}}%
\pgfpathcurveto{\pgfqpoint{2.763309in}{5.444159in}}{\pgfqpoint{2.861900in}{5.206140in}}{\pgfqpoint{3.037367in}{5.030672in}}%
\pgfpathcurveto{\pgfqpoint{3.212835in}{4.855204in}}{\pgfqpoint{3.450854in}{4.756614in}}{\pgfqpoint{3.699003in}{4.756614in}}%
\pgfpathlineto{\pgfqpoint{3.699003in}{4.756614in}}%
\pgfpathclose%
\pgfusepath{stroke}%
\end{pgfscope}%
\begin{pgfscope}%
\pgfpathrectangle{\pgfqpoint{1.000000in}{0.979904in}}{\pgfqpoint{6.200000in}{5.960192in}}%
\pgfusepath{clip}%
\pgfsetbuttcap%
\pgfsetroundjoin%
\definecolor{currentfill}{rgb}{0.000000,0.000000,0.000000}%
\pgfsetfillcolor{currentfill}%
\pgfsetlinewidth{1.003750pt}%
\definecolor{currentstroke}{rgb}{0.000000,0.000000,0.000000}%
\pgfsetstrokecolor{currentstroke}%
\pgfsetdash{}{0pt}%
\pgfsys@defobject{currentmarker}{\pgfqpoint{-0.021960in}{-0.021960in}}{\pgfqpoint{0.021960in}{0.021960in}}{%
\pgfpathmoveto{\pgfqpoint{0.000000in}{-0.021960in}}%
\pgfpathcurveto{\pgfqpoint{0.005824in}{-0.021960in}}{\pgfqpoint{0.011410in}{-0.019646in}}{\pgfqpoint{0.015528in}{-0.015528in}}%
\pgfpathcurveto{\pgfqpoint{0.019646in}{-0.011410in}}{\pgfqpoint{0.021960in}{-0.005824in}}{\pgfqpoint{0.021960in}{0.000000in}}%
\pgfpathcurveto{\pgfqpoint{0.021960in}{0.005824in}}{\pgfqpoint{0.019646in}{0.011410in}}{\pgfqpoint{0.015528in}{0.015528in}}%
\pgfpathcurveto{\pgfqpoint{0.011410in}{0.019646in}}{\pgfqpoint{0.005824in}{0.021960in}}{\pgfqpoint{0.000000in}{0.021960in}}%
\pgfpathcurveto{\pgfqpoint{-0.005824in}{0.021960in}}{\pgfqpoint{-0.011410in}{0.019646in}}{\pgfqpoint{-0.015528in}{0.015528in}}%
\pgfpathcurveto{\pgfqpoint{-0.019646in}{0.011410in}}{\pgfqpoint{-0.021960in}{0.005824in}}{\pgfqpoint{-0.021960in}{0.000000in}}%
\pgfpathcurveto{\pgfqpoint{-0.021960in}{-0.005824in}}{\pgfqpoint{-0.019646in}{-0.011410in}}{\pgfqpoint{-0.015528in}{-0.015528in}}%
\pgfpathcurveto{\pgfqpoint{-0.011410in}{-0.019646in}}{\pgfqpoint{-0.005824in}{-0.021960in}}{\pgfqpoint{0.000000in}{-0.021960in}}%
\pgfpathlineto{\pgfqpoint{0.000000in}{-0.021960in}}%
\pgfpathclose%
\pgfusepath{stroke,fill}%
}%
\begin{pgfscope}%
\pgfsys@transformshift{3.699003in}{5.692307in}%
\pgfsys@useobject{currentmarker}{}%
\end{pgfscope}%
\end{pgfscope}%
\begin{pgfscope}%
\pgfsetbuttcap%
\pgfsetroundjoin%
\definecolor{currentfill}{rgb}{0.000000,0.000000,0.000000}%
\pgfsetfillcolor{currentfill}%
\pgfsetlinewidth{0.803000pt}%
\definecolor{currentstroke}{rgb}{0.000000,0.000000,0.000000}%
\pgfsetstrokecolor{currentstroke}%
\pgfsetdash{}{0pt}%
\pgfsys@defobject{currentmarker}{\pgfqpoint{0.000000in}{-0.048611in}}{\pgfqpoint{0.000000in}{0.000000in}}{%
\pgfpathmoveto{\pgfqpoint{0.000000in}{0.000000in}}%
\pgfpathlineto{\pgfqpoint{0.000000in}{-0.048611in}}%
\pgfusepath{stroke,fill}%
}%
\begin{pgfscope}%
\pgfsys@transformshift{1.496745in}{0.979904in}%
\pgfsys@useobject{currentmarker}{}%
\end{pgfscope}%
\end{pgfscope}%
\begin{pgfscope}%
\definecolor{textcolor}{rgb}{0.000000,0.000000,0.000000}%
\pgfsetstrokecolor{textcolor}%
\pgfsetfillcolor{textcolor}%
\pgftext[x=1.496745in,y=0.882682in,,top]{\color{textcolor}{\sffamily\fontsize{10.000000}{12.000000}\selectfont\catcode`\^=\active\def^{\ifmmode\sp\else\^{}\fi}\catcode`\%=\active\def%{\%}\ensuremath{-}150}}%
\end{pgfscope}%
\begin{pgfscope}%
\pgfsetbuttcap%
\pgfsetroundjoin%
\definecolor{currentfill}{rgb}{0.000000,0.000000,0.000000}%
\pgfsetfillcolor{currentfill}%
\pgfsetlinewidth{0.803000pt}%
\definecolor{currentstroke}{rgb}{0.000000,0.000000,0.000000}%
\pgfsetstrokecolor{currentstroke}%
\pgfsetdash{}{0pt}%
\pgfsys@defobject{currentmarker}{\pgfqpoint{0.000000in}{-0.048611in}}{\pgfqpoint{0.000000in}{0.000000in}}{%
\pgfpathmoveto{\pgfqpoint{0.000000in}{0.000000in}}%
\pgfpathlineto{\pgfqpoint{0.000000in}{-0.048611in}}%
\pgfusepath{stroke,fill}%
}%
\begin{pgfscope}%
\pgfsys@transformshift{2.381022in}{0.979904in}%
\pgfsys@useobject{currentmarker}{}%
\end{pgfscope}%
\end{pgfscope}%
\begin{pgfscope}%
\definecolor{textcolor}{rgb}{0.000000,0.000000,0.000000}%
\pgfsetstrokecolor{textcolor}%
\pgfsetfillcolor{textcolor}%
\pgftext[x=2.381022in,y=0.882682in,,top]{\color{textcolor}{\sffamily\fontsize{10.000000}{12.000000}\selectfont\catcode`\^=\active\def^{\ifmmode\sp\else\^{}\fi}\catcode`\%=\active\def%{\%}\ensuremath{-}100}}%
\end{pgfscope}%
\begin{pgfscope}%
\pgfsetbuttcap%
\pgfsetroundjoin%
\definecolor{currentfill}{rgb}{0.000000,0.000000,0.000000}%
\pgfsetfillcolor{currentfill}%
\pgfsetlinewidth{0.803000pt}%
\definecolor{currentstroke}{rgb}{0.000000,0.000000,0.000000}%
\pgfsetstrokecolor{currentstroke}%
\pgfsetdash{}{0pt}%
\pgfsys@defobject{currentmarker}{\pgfqpoint{0.000000in}{-0.048611in}}{\pgfqpoint{0.000000in}{0.000000in}}{%
\pgfpathmoveto{\pgfqpoint{0.000000in}{0.000000in}}%
\pgfpathlineto{\pgfqpoint{0.000000in}{-0.048611in}}%
\pgfusepath{stroke,fill}%
}%
\begin{pgfscope}%
\pgfsys@transformshift{3.265300in}{0.979904in}%
\pgfsys@useobject{currentmarker}{}%
\end{pgfscope}%
\end{pgfscope}%
\begin{pgfscope}%
\definecolor{textcolor}{rgb}{0.000000,0.000000,0.000000}%
\pgfsetstrokecolor{textcolor}%
\pgfsetfillcolor{textcolor}%
\pgftext[x=3.265300in,y=0.882682in,,top]{\color{textcolor}{\sffamily\fontsize{10.000000}{12.000000}\selectfont\catcode`\^=\active\def^{\ifmmode\sp\else\^{}\fi}\catcode`\%=\active\def%{\%}\ensuremath{-}50}}%
\end{pgfscope}%
\begin{pgfscope}%
\pgfsetbuttcap%
\pgfsetroundjoin%
\definecolor{currentfill}{rgb}{0.000000,0.000000,0.000000}%
\pgfsetfillcolor{currentfill}%
\pgfsetlinewidth{0.803000pt}%
\definecolor{currentstroke}{rgb}{0.000000,0.000000,0.000000}%
\pgfsetstrokecolor{currentstroke}%
\pgfsetdash{}{0pt}%
\pgfsys@defobject{currentmarker}{\pgfqpoint{0.000000in}{-0.048611in}}{\pgfqpoint{0.000000in}{0.000000in}}{%
\pgfpathmoveto{\pgfqpoint{0.000000in}{0.000000in}}%
\pgfpathlineto{\pgfqpoint{0.000000in}{-0.048611in}}%
\pgfusepath{stroke,fill}%
}%
\begin{pgfscope}%
\pgfsys@transformshift{4.149578in}{0.979904in}%
\pgfsys@useobject{currentmarker}{}%
\end{pgfscope}%
\end{pgfscope}%
\begin{pgfscope}%
\definecolor{textcolor}{rgb}{0.000000,0.000000,0.000000}%
\pgfsetstrokecolor{textcolor}%
\pgfsetfillcolor{textcolor}%
\pgftext[x=4.149578in,y=0.882682in,,top]{\color{textcolor}{\sffamily\fontsize{10.000000}{12.000000}\selectfont\catcode`\^=\active\def^{\ifmmode\sp\else\^{}\fi}\catcode`\%=\active\def%{\%}0}}%
\end{pgfscope}%
\begin{pgfscope}%
\pgfsetbuttcap%
\pgfsetroundjoin%
\definecolor{currentfill}{rgb}{0.000000,0.000000,0.000000}%
\pgfsetfillcolor{currentfill}%
\pgfsetlinewidth{0.803000pt}%
\definecolor{currentstroke}{rgb}{0.000000,0.000000,0.000000}%
\pgfsetstrokecolor{currentstroke}%
\pgfsetdash{}{0pt}%
\pgfsys@defobject{currentmarker}{\pgfqpoint{0.000000in}{-0.048611in}}{\pgfqpoint{0.000000in}{0.000000in}}{%
\pgfpathmoveto{\pgfqpoint{0.000000in}{0.000000in}}%
\pgfpathlineto{\pgfqpoint{0.000000in}{-0.048611in}}%
\pgfusepath{stroke,fill}%
}%
\begin{pgfscope}%
\pgfsys@transformshift{5.033855in}{0.979904in}%
\pgfsys@useobject{currentmarker}{}%
\end{pgfscope}%
\end{pgfscope}%
\begin{pgfscope}%
\definecolor{textcolor}{rgb}{0.000000,0.000000,0.000000}%
\pgfsetstrokecolor{textcolor}%
\pgfsetfillcolor{textcolor}%
\pgftext[x=5.033855in,y=0.882682in,,top]{\color{textcolor}{\sffamily\fontsize{10.000000}{12.000000}\selectfont\catcode`\^=\active\def^{\ifmmode\sp\else\^{}\fi}\catcode`\%=\active\def%{\%}50}}%
\end{pgfscope}%
\begin{pgfscope}%
\pgfsetbuttcap%
\pgfsetroundjoin%
\definecolor{currentfill}{rgb}{0.000000,0.000000,0.000000}%
\pgfsetfillcolor{currentfill}%
\pgfsetlinewidth{0.803000pt}%
\definecolor{currentstroke}{rgb}{0.000000,0.000000,0.000000}%
\pgfsetstrokecolor{currentstroke}%
\pgfsetdash{}{0pt}%
\pgfsys@defobject{currentmarker}{\pgfqpoint{0.000000in}{-0.048611in}}{\pgfqpoint{0.000000in}{0.000000in}}{%
\pgfpathmoveto{\pgfqpoint{0.000000in}{0.000000in}}%
\pgfpathlineto{\pgfqpoint{0.000000in}{-0.048611in}}%
\pgfusepath{stroke,fill}%
}%
\begin{pgfscope}%
\pgfsys@transformshift{5.918133in}{0.979904in}%
\pgfsys@useobject{currentmarker}{}%
\end{pgfscope}%
\end{pgfscope}%
\begin{pgfscope}%
\definecolor{textcolor}{rgb}{0.000000,0.000000,0.000000}%
\pgfsetstrokecolor{textcolor}%
\pgfsetfillcolor{textcolor}%
\pgftext[x=5.918133in,y=0.882682in,,top]{\color{textcolor}{\sffamily\fontsize{10.000000}{12.000000}\selectfont\catcode`\^=\active\def^{\ifmmode\sp\else\^{}\fi}\catcode`\%=\active\def%{\%}100}}%
\end{pgfscope}%
\begin{pgfscope}%
\pgfsetbuttcap%
\pgfsetroundjoin%
\definecolor{currentfill}{rgb}{0.000000,0.000000,0.000000}%
\pgfsetfillcolor{currentfill}%
\pgfsetlinewidth{0.803000pt}%
\definecolor{currentstroke}{rgb}{0.000000,0.000000,0.000000}%
\pgfsetstrokecolor{currentstroke}%
\pgfsetdash{}{0pt}%
\pgfsys@defobject{currentmarker}{\pgfqpoint{0.000000in}{-0.048611in}}{\pgfqpoint{0.000000in}{0.000000in}}{%
\pgfpathmoveto{\pgfqpoint{0.000000in}{0.000000in}}%
\pgfpathlineto{\pgfqpoint{0.000000in}{-0.048611in}}%
\pgfusepath{stroke,fill}%
}%
\begin{pgfscope}%
\pgfsys@transformshift{6.802411in}{0.979904in}%
\pgfsys@useobject{currentmarker}{}%
\end{pgfscope}%
\end{pgfscope}%
\begin{pgfscope}%
\definecolor{textcolor}{rgb}{0.000000,0.000000,0.000000}%
\pgfsetstrokecolor{textcolor}%
\pgfsetfillcolor{textcolor}%
\pgftext[x=6.802411in,y=0.882682in,,top]{\color{textcolor}{\sffamily\fontsize{10.000000}{12.000000}\selectfont\catcode`\^=\active\def^{\ifmmode\sp\else\^{}\fi}\catcode`\%=\active\def%{\%}150}}%
\end{pgfscope}%
\begin{pgfscope}%
\pgfsetbuttcap%
\pgfsetroundjoin%
\definecolor{currentfill}{rgb}{0.000000,0.000000,0.000000}%
\pgfsetfillcolor{currentfill}%
\pgfsetlinewidth{0.803000pt}%
\definecolor{currentstroke}{rgb}{0.000000,0.000000,0.000000}%
\pgfsetstrokecolor{currentstroke}%
\pgfsetdash{}{0pt}%
\pgfsys@defobject{currentmarker}{\pgfqpoint{-0.048611in}{0.000000in}}{\pgfqpoint{-0.000000in}{0.000000in}}{%
\pgfpathmoveto{\pgfqpoint{-0.000000in}{0.000000in}}%
\pgfpathlineto{\pgfqpoint{-0.048611in}{0.000000in}}%
\pgfusepath{stroke,fill}%
}%
\begin{pgfscope}%
\pgfsys@transformshift{1.000000in}{1.451715in}%
\pgfsys@useobject{currentmarker}{}%
\end{pgfscope}%
\end{pgfscope}%
\begin{pgfscope}%
\definecolor{textcolor}{rgb}{0.000000,0.000000,0.000000}%
\pgfsetstrokecolor{textcolor}%
\pgfsetfillcolor{textcolor}%
\pgftext[x=0.529657in, y=1.398953in, left, base]{\color{textcolor}{\sffamily\fontsize{10.000000}{12.000000}\selectfont\catcode`\^=\active\def^{\ifmmode\sp\else\^{}\fi}\catcode`\%=\active\def%{\%}\ensuremath{-}150}}%
\end{pgfscope}%
\begin{pgfscope}%
\pgfsetbuttcap%
\pgfsetroundjoin%
\definecolor{currentfill}{rgb}{0.000000,0.000000,0.000000}%
\pgfsetfillcolor{currentfill}%
\pgfsetlinewidth{0.803000pt}%
\definecolor{currentstroke}{rgb}{0.000000,0.000000,0.000000}%
\pgfsetstrokecolor{currentstroke}%
\pgfsetdash{}{0pt}%
\pgfsys@defobject{currentmarker}{\pgfqpoint{-0.048611in}{0.000000in}}{\pgfqpoint{-0.000000in}{0.000000in}}{%
\pgfpathmoveto{\pgfqpoint{-0.000000in}{0.000000in}}%
\pgfpathlineto{\pgfqpoint{-0.048611in}{0.000000in}}%
\pgfusepath{stroke,fill}%
}%
\begin{pgfscope}%
\pgfsys@transformshift{1.000000in}{2.335993in}%
\pgfsys@useobject{currentmarker}{}%
\end{pgfscope}%
\end{pgfscope}%
\begin{pgfscope}%
\definecolor{textcolor}{rgb}{0.000000,0.000000,0.000000}%
\pgfsetstrokecolor{textcolor}%
\pgfsetfillcolor{textcolor}%
\pgftext[x=0.529657in, y=2.283231in, left, base]{\color{textcolor}{\sffamily\fontsize{10.000000}{12.000000}\selectfont\catcode`\^=\active\def^{\ifmmode\sp\else\^{}\fi}\catcode`\%=\active\def%{\%}\ensuremath{-}100}}%
\end{pgfscope}%
\begin{pgfscope}%
\pgfsetbuttcap%
\pgfsetroundjoin%
\definecolor{currentfill}{rgb}{0.000000,0.000000,0.000000}%
\pgfsetfillcolor{currentfill}%
\pgfsetlinewidth{0.803000pt}%
\definecolor{currentstroke}{rgb}{0.000000,0.000000,0.000000}%
\pgfsetstrokecolor{currentstroke}%
\pgfsetdash{}{0pt}%
\pgfsys@defobject{currentmarker}{\pgfqpoint{-0.048611in}{0.000000in}}{\pgfqpoint{-0.000000in}{0.000000in}}{%
\pgfpathmoveto{\pgfqpoint{-0.000000in}{0.000000in}}%
\pgfpathlineto{\pgfqpoint{-0.048611in}{0.000000in}}%
\pgfusepath{stroke,fill}%
}%
\begin{pgfscope}%
\pgfsys@transformshift{1.000000in}{3.220270in}%
\pgfsys@useobject{currentmarker}{}%
\end{pgfscope}%
\end{pgfscope}%
\begin{pgfscope}%
\definecolor{textcolor}{rgb}{0.000000,0.000000,0.000000}%
\pgfsetstrokecolor{textcolor}%
\pgfsetfillcolor{textcolor}%
\pgftext[x=0.618022in, y=3.167509in, left, base]{\color{textcolor}{\sffamily\fontsize{10.000000}{12.000000}\selectfont\catcode`\^=\active\def^{\ifmmode\sp\else\^{}\fi}\catcode`\%=\active\def%{\%}\ensuremath{-}50}}%
\end{pgfscope}%
\begin{pgfscope}%
\pgfsetbuttcap%
\pgfsetroundjoin%
\definecolor{currentfill}{rgb}{0.000000,0.000000,0.000000}%
\pgfsetfillcolor{currentfill}%
\pgfsetlinewidth{0.803000pt}%
\definecolor{currentstroke}{rgb}{0.000000,0.000000,0.000000}%
\pgfsetstrokecolor{currentstroke}%
\pgfsetdash{}{0pt}%
\pgfsys@defobject{currentmarker}{\pgfqpoint{-0.048611in}{0.000000in}}{\pgfqpoint{-0.000000in}{0.000000in}}{%
\pgfpathmoveto{\pgfqpoint{-0.000000in}{0.000000in}}%
\pgfpathlineto{\pgfqpoint{-0.048611in}{0.000000in}}%
\pgfusepath{stroke,fill}%
}%
\begin{pgfscope}%
\pgfsys@transformshift{1.000000in}{4.104548in}%
\pgfsys@useobject{currentmarker}{}%
\end{pgfscope}%
\end{pgfscope}%
\begin{pgfscope}%
\definecolor{textcolor}{rgb}{0.000000,0.000000,0.000000}%
\pgfsetstrokecolor{textcolor}%
\pgfsetfillcolor{textcolor}%
\pgftext[x=0.814412in, y=4.051786in, left, base]{\color{textcolor}{\sffamily\fontsize{10.000000}{12.000000}\selectfont\catcode`\^=\active\def^{\ifmmode\sp\else\^{}\fi}\catcode`\%=\active\def%{\%}0}}%
\end{pgfscope}%
\begin{pgfscope}%
\pgfsetbuttcap%
\pgfsetroundjoin%
\definecolor{currentfill}{rgb}{0.000000,0.000000,0.000000}%
\pgfsetfillcolor{currentfill}%
\pgfsetlinewidth{0.803000pt}%
\definecolor{currentstroke}{rgb}{0.000000,0.000000,0.000000}%
\pgfsetstrokecolor{currentstroke}%
\pgfsetdash{}{0pt}%
\pgfsys@defobject{currentmarker}{\pgfqpoint{-0.048611in}{0.000000in}}{\pgfqpoint{-0.000000in}{0.000000in}}{%
\pgfpathmoveto{\pgfqpoint{-0.000000in}{0.000000in}}%
\pgfpathlineto{\pgfqpoint{-0.048611in}{0.000000in}}%
\pgfusepath{stroke,fill}%
}%
\begin{pgfscope}%
\pgfsys@transformshift{1.000000in}{4.988826in}%
\pgfsys@useobject{currentmarker}{}%
\end{pgfscope}%
\end{pgfscope}%
\begin{pgfscope}%
\definecolor{textcolor}{rgb}{0.000000,0.000000,0.000000}%
\pgfsetstrokecolor{textcolor}%
\pgfsetfillcolor{textcolor}%
\pgftext[x=0.726047in, y=4.936064in, left, base]{\color{textcolor}{\sffamily\fontsize{10.000000}{12.000000}\selectfont\catcode`\^=\active\def^{\ifmmode\sp\else\^{}\fi}\catcode`\%=\active\def%{\%}50}}%
\end{pgfscope}%
\begin{pgfscope}%
\pgfsetbuttcap%
\pgfsetroundjoin%
\definecolor{currentfill}{rgb}{0.000000,0.000000,0.000000}%
\pgfsetfillcolor{currentfill}%
\pgfsetlinewidth{0.803000pt}%
\definecolor{currentstroke}{rgb}{0.000000,0.000000,0.000000}%
\pgfsetstrokecolor{currentstroke}%
\pgfsetdash{}{0pt}%
\pgfsys@defobject{currentmarker}{\pgfqpoint{-0.048611in}{0.000000in}}{\pgfqpoint{-0.000000in}{0.000000in}}{%
\pgfpathmoveto{\pgfqpoint{-0.000000in}{0.000000in}}%
\pgfpathlineto{\pgfqpoint{-0.048611in}{0.000000in}}%
\pgfusepath{stroke,fill}%
}%
\begin{pgfscope}%
\pgfsys@transformshift{1.000000in}{5.873103in}%
\pgfsys@useobject{currentmarker}{}%
\end{pgfscope}%
\end{pgfscope}%
\begin{pgfscope}%
\definecolor{textcolor}{rgb}{0.000000,0.000000,0.000000}%
\pgfsetstrokecolor{textcolor}%
\pgfsetfillcolor{textcolor}%
\pgftext[x=0.637682in, y=5.820342in, left, base]{\color{textcolor}{\sffamily\fontsize{10.000000}{12.000000}\selectfont\catcode`\^=\active\def^{\ifmmode\sp\else\^{}\fi}\catcode`\%=\active\def%{\%}100}}%
\end{pgfscope}%
\begin{pgfscope}%
\pgfsetbuttcap%
\pgfsetroundjoin%
\definecolor{currentfill}{rgb}{0.000000,0.000000,0.000000}%
\pgfsetfillcolor{currentfill}%
\pgfsetlinewidth{0.803000pt}%
\definecolor{currentstroke}{rgb}{0.000000,0.000000,0.000000}%
\pgfsetstrokecolor{currentstroke}%
\pgfsetdash{}{0pt}%
\pgfsys@defobject{currentmarker}{\pgfqpoint{-0.048611in}{0.000000in}}{\pgfqpoint{-0.000000in}{0.000000in}}{%
\pgfpathmoveto{\pgfqpoint{-0.000000in}{0.000000in}}%
\pgfpathlineto{\pgfqpoint{-0.048611in}{0.000000in}}%
\pgfusepath{stroke,fill}%
}%
\begin{pgfscope}%
\pgfsys@transformshift{1.000000in}{6.757381in}%
\pgfsys@useobject{currentmarker}{}%
\end{pgfscope}%
\end{pgfscope}%
\begin{pgfscope}%
\definecolor{textcolor}{rgb}{0.000000,0.000000,0.000000}%
\pgfsetstrokecolor{textcolor}%
\pgfsetfillcolor{textcolor}%
\pgftext[x=0.637682in, y=6.704619in, left, base]{\color{textcolor}{\sffamily\fontsize{10.000000}{12.000000}\selectfont\catcode`\^=\active\def^{\ifmmode\sp\else\^{}\fi}\catcode`\%=\active\def%{\%}150}}%
\end{pgfscope}%
\begin{pgfscope}%
\pgfsetrectcap%
\pgfsetmiterjoin%
\pgfsetlinewidth{0.803000pt}%
\definecolor{currentstroke}{rgb}{0.000000,0.000000,0.000000}%
\pgfsetstrokecolor{currentstroke}%
\pgfsetdash{}{0pt}%
\pgfpathmoveto{\pgfqpoint{1.000000in}{0.979904in}}%
\pgfpathlineto{\pgfqpoint{1.000000in}{6.940096in}}%
\pgfusepath{stroke}%
\end{pgfscope}%
\begin{pgfscope}%
\pgfsetrectcap%
\pgfsetmiterjoin%
\pgfsetlinewidth{0.803000pt}%
\definecolor{currentstroke}{rgb}{0.000000,0.000000,0.000000}%
\pgfsetstrokecolor{currentstroke}%
\pgfsetdash{}{0pt}%
\pgfpathmoveto{\pgfqpoint{7.200000in}{0.979904in}}%
\pgfpathlineto{\pgfqpoint{7.200000in}{6.940096in}}%
\pgfusepath{stroke}%
\end{pgfscope}%
\begin{pgfscope}%
\pgfsetrectcap%
\pgfsetmiterjoin%
\pgfsetlinewidth{0.803000pt}%
\definecolor{currentstroke}{rgb}{0.000000,0.000000,0.000000}%
\pgfsetstrokecolor{currentstroke}%
\pgfsetdash{}{0pt}%
\pgfpathmoveto{\pgfqpoint{1.000000in}{0.979904in}}%
\pgfpathlineto{\pgfqpoint{7.200000in}{0.979904in}}%
\pgfusepath{stroke}%
\end{pgfscope}%
\begin{pgfscope}%
\pgfsetrectcap%
\pgfsetmiterjoin%
\pgfsetlinewidth{0.803000pt}%
\definecolor{currentstroke}{rgb}{0.000000,0.000000,0.000000}%
\pgfsetstrokecolor{currentstroke}%
\pgfsetdash{}{0pt}%
\pgfpathmoveto{\pgfqpoint{1.000000in}{6.940096in}}%
\pgfpathlineto{\pgfqpoint{7.200000in}{6.940096in}}%
\pgfusepath{stroke}%
\end{pgfscope}%
\end{pgfpicture}%
\makeatother%
\endgroup%
}
    \label{fig:noisy_rings}
    \caption{Example of a dataset with different rings and noise, and their correct classification.}
\end{figure}


\section{Fuzzy Clustering Algorithms}
In hard clustering algorithms, each data point is assigned to a single cluster. In contrast, fuzzy algorithms assign a membership degree to each data point for each cluster.
One of the most popular ones is the Fuzzy C-Means algorithm. A formal description can be found in \cite{bookpatternrecognition} and \cite{BEZDEK1984191}. It can be seen as an optimization problem,
where the objective function is to minimize the following equation:
\begin{equation}
J(U, V) = \sum_{i=1}^{n} \sum_{j=1}^{k} (u_{ij})^q d_{ij}^2
\end{equation}
where $k$ is the number of clusters, $n$ is the number of data samples, $u_{ij}$ is the membership degree of cluster $i$ to data sample $j$,
$d_{ij}$ is the eucledian distance between data point $i$ and the center of cluster $j$, and $q$ is a parameter that controls the fuzziness of the membership degrees.
That is, higher values of $q$ will make the membership degrees more 'fuzzy', and lower values will make them harder, that is, closer to regular K-Means.
$u$ is a matrix of size $n \times k$, and can be interpreted as 'how much data point $i$ belongs to cluster $j$'.
It is important to note that the following conditions must be met, as described in \cite{BEZDEK1984191}:
\begin{enumerate}
    \item $u_{ij} \in [0, 1]$
    \item $\sum_{j=1}^{k} u_{ij} = 1$
\end{enumerate}
The higher the value of $q$, the more 'fuzzy' the algorithm will be. In the limit, when $q \rightarrow 1$, the algorithm will be equivalent to K-Means.


\section{The Fuzzy K-Rings Algorithm}
The Fuzzy K-Rings algorithm is a clustering algorithm that is able to cluster data points in a ring-shaped dataset.
The algorithm is inspired on the K-Means algorithm, and described in \cite{DAVE1992713} and \cite{308484}, altough in different variations.
It is described as an optimization problem, where the objective function is to minimize the following equation:
\begin{equation}\label{eq:objective}
J_q(U, V) = \sum_{i=1}^{n} \sum_{j=1}^{k} u_{ij}^q (d_{ij} - r_i)^2
\end{equation}
where $k$ is the number of clusters, $n$ is the number of data samples, $u_{ij}$ is the membership degree of cluster $i$ to data sample $j$, $d_{ij}$ is the eucledian distance between data point $i$ and the center of cluster $j$, $r_i$ is the radius of the cluster $i$, and $q$ is a parameter that controls the fuzziness of the membership degrees.
That is, higher values of $q$ will make the membership degrees more fuzzy, and lower values will make them harder.
From now on, we'll refer to $$(d_{ij} - r_i)^2$$ as $d'_ij$.
Now, we'll describe the ways to update the different parameters in the algorithm, and then we'll describe the initialization and convergence criteria, as well as the concrete steps.

\subsection{Updating the Membership Degrees}
The membership degrees are updated using the following equation, as described in both \cite{DAVE1992713} and \cite{308484}. It's the same as the one used in the Fuzzy C-Means algorithm, but with $d_{ij}$ replaced by $d'_{ij}$.
\begin{equation}
u_{ij} = \frac{d'^2(X_j, V_i)^{\frac{-1}{q-1}}}{\sum_{k=1}^{K} d'^2(X_j, V_k)^{\frac{-1}{q-1}}}
\end{equation}

\subsection{Updating the Cluster Radii and Centers}
As mentioned, we can define the algorith as an optimization problem after fixing $U$. We can then obtain the optimal (minimum) values for the objective function
by setting the partial derivatives with respect to $r_i$ and $V_i$ to zero.
First, we have:
\begin{equation}\label{eq:d_dr}
\frac{\partial}{\partial r_i}(J_q) = \sum_{j=1}^{n} u_{ij}^q\frac{\partial}{\partial r_i} (d_{ij} - r_i)^2 = \sum_{j=1}^{n} u_{ij}^q (r_i - d_{ij}) = 0
\end{equation}

Here, we take a different approach to the one taken in \cite{308484}, and similar to the one in \cite{DAVE1992713}. It allows vectorized computations, and
automatic extension to higher dimensions, unlike \cite{308484}.

\begin{tikzpicture}\label{fig:circle}
    % Define radius
    \def\radius{3cm}
    \def\angle{45}

    % Draw the circle
    \draw (0,0) circle (\radius);

    % Define points
    \coordinate (V_i) at (0,0); % Center of the circle
    \coordinate (A) at ({\radius*cos(\angle)},0); % Point on the circle at the specified angle
    \coordinate (V'_i) at (\angle:\radius);
    \coordinate (X_j) at (3.5, 3.5);

    % Draw the triangle
    \draw (V_i) -- (A) -- (V'_i) -- cycle;
    \draw[dotted] (V'_i) -- (X_j);

    % Draw points
    \fill (V_i) circle (2pt);
    \fill (V'_i) circle (2pt);
    \fill (X_j) circle (2pt);

    \node[below] at (V_i) {$V_i$};
    \node[above right] at (V'_i) {$V'_i$};
    \node[above right] at (X_j) {$X_j$};

    % label on hypotenuse (line that connects V_i and V'_i). Not on the point, on the line
    \node[above=2pt] at ($(V_i)!0.5!(V'_i)$) {$r_i$};
    % explanation
    \node[below] at (0,-\radius-0.5) {
        \begin{tabular}{c}
            $V'_i = \frac{r_i}{d_{ij}}X_j + (1 - \frac{r_i}{d_{ij}})V_i$ \\
        \end{tabular}
    };
\end{tikzpicture}
\begin{center}
\textbf{Figure \ref{fig:circle}:} Illustration of the update of the cluster centers equation when $X_j$ is outside the circle.
\end{center}


Let $X_j$ be a data point, and $V_i$ be the center of cluster $i$. Let $d_{ij}$ be the eucledian distance between $X_j$ and $V_i$,
and $r_i$ be the radius of cluster $i$.
Let $d'_{ij}$ be the distance between $X_j$ and the circle with center $V_i$ and radius $r_i$.

Then, let the following be true:
\begin{equation}
V'_i = \frac{r_i}{d_{ij}}X_j + (1 - \frac{r_i}{d_{ij}})V_i
\end{equation}

Differentiating \eqref{eq:objective} with respect to $V_i$ and setting it to zero, we get:
\begin{equation}\label{eq:d_dV}
\frac{\partial}{\partial V_i}(J_q) = \sum_{j=1}^{n} u_{ij}^q\frac{\partial}{\partial V_i} (d'_{ij})^2 = 0
\end{equation}
Note that we can rewrite $(d'_{ij})^2$ as $(X_j - V'_i)^T(X_j - V'_i)$.
Then, we can solve:
\begin{equation}
\begin{aligned}
\frac{\partial}{\partial V_i} (d_{ij} - r_i)^2 &= \left(\frac{\partial}{\partial V_i} (|X_j - V_i| - r_i)^2\right) \\
&= -2 \left( (X_j - V_i) - \frac{r_i}{|X_j - V_i|} (X_j - V_i) \right) \\
&= -2 \left( (X_j - V_i) - \frac{r_i}{d_{ij}} (X_j - V_i) \right) \\
&= -2 \left( (1 - \frac{r_i}{d_{ij}})X_J - (1 - \frac{r_i}{d_{ij}})V_i \right)
\end{aligned}
\end{equation}
Let's define $(1 - \frac{r_i}{d_{ij}})$ as $\alpha_{ij}$.
Then, we can rewrite the equation as:
\begin{equation}
\frac{\partial}{\partial V_i} (d_{ij} - r_i)^2 = -2 \alpha_{ij} (X_j - V_i)
\end{equation}
Plugging that into \eqref{eq:d_dV}, we get:
\begin{equation}
\sum_{j=1}^{n} u_{ij}^q (d_{ij} - r_i) (X_j - V_i) = 0
\end{equation}
We now have a system of equations that we can solve for $V_i$ and $r_i$ to obtain the critical points of the objective function. It is important that since
the equations are coupled, they must be solved together, since it yields better results. \cite{DAVE1992713} mentions (and cites the proof) that, indeed, the critical
points are minima. The experimental results also back this up.
One solution, as noted in \cite{DAVE1992713}, is:
\begin{equation}
V_i = \frac{\sum_{j=1}^{n} u_{ij}^q X_j}{\sum_{j=1}^{n} u_{ij}^q}
\end{equation}
\begin{equation}\label{eq:r_i}
r_i = \frac{\sum_{j=1}^{n} u_{ij}^q d_{ij}}{\sum_{j=1}^{n} u_{ij}^q}
\end{equation}
On the other hand, \cite{308484} proposes a different solution, which is to solve the equations separately. In my experiments, I found that solution not to work very
well in practice, and the one proposed by \cite{DAVE1992713} to work better.
It is also easily checkable that the solution proposed by \cite{DAVE1992713} is a critical point by substituting it into the equations.


\subsection{Intuition}
After having given a formal description of the algorithm, we can give an intuitive explanation of the algorithm.

First, for the membership degrees, we can see that the algorithm is trying to assign higher membership degrees to points that are closer to the ring contour.
This is done by averaging the distances of the points to the different rings, and assigning them proportionally.

As for the cluster centers, the algorithm is just using a weighted average of the points, with the weights being the membership degrees.
This is similar to the K-Means algorithm, but with the weights being the membership degrees instead of a binary value, and exactly the same as the Fuzzy C-Means algorithm.

Finally, for the radii, the algorithm is trying to assign the radii by using a weighted average of the distances of the points to the cluster centers.
This is done by using the membership degrees as weights, and the distances as the values to be averaged.


\subsection{Initializing the Parameters and Nature of the Data}
\cite{308484} proposes two initialization methods, depending on the nature of the data. We adopt both:
\begin{itemize}
    \item Concentric datasets: In this case, since all rings share the same center, but have different radii, the procedure is as follows.
    For the centers, we simply compute the baricenter of the dataset:
    \begin{equation}
        V_i = \frac{1}{n} \sum_{j=1}^{n} X_j
    \end{equation}
    Then, for the radii we define max and min as follows:
    \begin{equation}
        r_{\text{max}} = \max_{j} d(X_j, V_i)
    \end{equation}
    \begin{equation}
        r_{\text{min}} = \min_{j} d(X_j, V_i)
    \end{equation}
    They denote the maximum and minimum distance of a point to the center. Then, we can initialize the radii as sampling from a uniform distribution:
    \begin{equation}
        r_i = r_{\text{min}} + (r_{\text{max}} - r_{\text{min}}) \cdot \text{rand()}
    \end{equation}
    Which will generate a random radius between the minimum and maximum distance of a point to the center.
    \item Non-concentric datasets: In this case, the rings do not share the same center. The rings can also interlock, and their radii are usually different.
    As described in \cite{308484}, we first run the Fuzzy C-Means algorithm on the dataset. After that, we directly use the membership degrees and centers as our initial
    stato. As for the radius, we obtain it with equation \eqref{eq:r_i}.
\end{itemize}

\subsection{Convergence Criterion}
We use the following convergence criterion:
\begin{equation}
|\hat{u_{ij}} - u_{ij}| < \epsilon \quad \forall i, j
\end{equation}
Where $\hat{u_{ij}}$ is the membership degree of the previous iteration, and $u_{ij}$ is the membership degree of the current iteration,
and $\epsilon$ is a small value, usually $10^{-3}$, given as a hyperparameter.
That is, after each update, we check for the difference between the membership degrees of the current and previous iteration, and if the difference is smaller than $\epsilon$,
we break the loop. We do not stop it completely, but we will get to that later.

\subsection{Background noise detection}
In the case of noisy data, the algorithm can be sensitive to noise. To mitigate this, we propose an aditional step.
\begin{figure}[H]
    \centering
    \resizebox{0.9\linewidth}{!}{%% Creator: Matplotlib, PGF backend
%%
%% To include the figure in your LaTeX document, write
%%   \input{<filename>.pgf}
%%
%% Make sure the required packages are loaded in your preamble
%%   \usepackage{pgf}
%%
%% Also ensure that all the required font packages are loaded; for instance,
%% the lmodern package is sometimes necessary when using math font.
%%   \usepackage{lmodern}
%%
%% Figures using additional raster images can only be included by \input if
%% they are in the same directory as the main LaTeX file. For loading figures
%% from other directories you can use the `import` package
%%   \usepackage{import}
%%
%% and then include the figures with
%%   \import{<path to file>}{<filename>.pgf}
%%
%% Matplotlib used the following preamble
%%   \def\mathdefault#1{#1}
%%   \everymath=\expandafter{\the\everymath\displaystyle}
%%   
%%   \usepackage{fontspec}
%%   \setmainfont{DejaVuSerif.ttf}[Path=\detokenize{C:/Users/dagom/anaconda3/envs/pytorch/lib/site-packages/matplotlib/mpl-data/fonts/ttf/}]
%%   \setsansfont{DejaVuSans.ttf}[Path=\detokenize{C:/Users/dagom/anaconda3/envs/pytorch/lib/site-packages/matplotlib/mpl-data/fonts/ttf/}]
%%   \setmonofont{DejaVuSansMono.ttf}[Path=\detokenize{C:/Users/dagom/anaconda3/envs/pytorch/lib/site-packages/matplotlib/mpl-data/fonts/ttf/}]
%%   \makeatletter\@ifpackageloaded{underscore}{}{\usepackage[strings]{underscore}}\makeatother
%%
\begingroup%
\makeatletter%
\begin{pgfpicture}%
\pgfpathrectangle{\pgfpointorigin}{\pgfqpoint{8.000000in}{8.000000in}}%
\pgfusepath{use as bounding box, clip}%
\begin{pgfscope}%
\pgfsetbuttcap%
\pgfsetmiterjoin%
\definecolor{currentfill}{rgb}{1.000000,1.000000,1.000000}%
\pgfsetfillcolor{currentfill}%
\pgfsetlinewidth{0.000000pt}%
\definecolor{currentstroke}{rgb}{1.000000,1.000000,1.000000}%
\pgfsetstrokecolor{currentstroke}%
\pgfsetdash{}{0pt}%
\pgfpathmoveto{\pgfqpoint{0.000000in}{0.000000in}}%
\pgfpathlineto{\pgfqpoint{8.000000in}{0.000000in}}%
\pgfpathlineto{\pgfqpoint{8.000000in}{8.000000in}}%
\pgfpathlineto{\pgfqpoint{0.000000in}{8.000000in}}%
\pgfpathlineto{\pgfqpoint{0.000000in}{0.000000in}}%
\pgfpathclose%
\pgfusepath{fill}%
\end{pgfscope}%
\begin{pgfscope}%
\pgfsetbuttcap%
\pgfsetmiterjoin%
\definecolor{currentfill}{rgb}{1.000000,1.000000,1.000000}%
\pgfsetfillcolor{currentfill}%
\pgfsetlinewidth{0.000000pt}%
\definecolor{currentstroke}{rgb}{0.000000,0.000000,0.000000}%
\pgfsetstrokecolor{currentstroke}%
\pgfsetstrokeopacity{0.000000}%
\pgfsetdash{}{0pt}%
\pgfpathmoveto{\pgfqpoint{1.542338in}{0.880000in}}%
\pgfpathlineto{\pgfqpoint{6.657662in}{0.880000in}}%
\pgfpathlineto{\pgfqpoint{6.657662in}{7.040000in}}%
\pgfpathlineto{\pgfqpoint{1.542338in}{7.040000in}}%
\pgfpathlineto{\pgfqpoint{1.542338in}{0.880000in}}%
\pgfpathclose%
\pgfusepath{fill}%
\end{pgfscope}%
\begin{pgfscope}%
\pgfpathrectangle{\pgfqpoint{1.542338in}{0.880000in}}{\pgfqpoint{5.115323in}{6.160000in}}%
\pgfusepath{clip}%
\pgfsetbuttcap%
\pgfsetroundjoin%
\definecolor{currentfill}{rgb}{0.800000,0.200000,0.200000}%
\pgfsetfillcolor{currentfill}%
\pgfsetlinewidth{1.003750pt}%
\definecolor{currentstroke}{rgb}{0.800000,0.200000,0.200000}%
\pgfsetstrokecolor{currentstroke}%
\pgfsetdash{}{0pt}%
\pgfpathmoveto{\pgfqpoint{6.425147in}{2.153901in}}%
\pgfpathcurveto{\pgfqpoint{6.430971in}{2.153901in}}{\pgfqpoint{6.436557in}{2.156215in}}{\pgfqpoint{6.440675in}{2.160333in}}%
\pgfpathcurveto{\pgfqpoint{6.444793in}{2.164451in}}{\pgfqpoint{6.447107in}{2.170037in}}{\pgfqpoint{6.447107in}{2.175861in}}%
\pgfpathcurveto{\pgfqpoint{6.447107in}{2.181685in}}{\pgfqpoint{6.444793in}{2.187272in}}{\pgfqpoint{6.440675in}{2.191390in}}%
\pgfpathcurveto{\pgfqpoint{6.436557in}{2.195508in}}{\pgfqpoint{6.430971in}{2.197822in}}{\pgfqpoint{6.425147in}{2.197822in}}%
\pgfpathcurveto{\pgfqpoint{6.419323in}{2.197822in}}{\pgfqpoint{6.413737in}{2.195508in}}{\pgfqpoint{6.409619in}{2.191390in}}%
\pgfpathcurveto{\pgfqpoint{6.405501in}{2.187272in}}{\pgfqpoint{6.403187in}{2.181685in}}{\pgfqpoint{6.403187in}{2.175861in}}%
\pgfpathcurveto{\pgfqpoint{6.403187in}{2.170037in}}{\pgfqpoint{6.405501in}{2.164451in}}{\pgfqpoint{6.409619in}{2.160333in}}%
\pgfpathcurveto{\pgfqpoint{6.413737in}{2.156215in}}{\pgfqpoint{6.419323in}{2.153901in}}{\pgfqpoint{6.425147in}{2.153901in}}%
\pgfpathlineto{\pgfqpoint{6.425147in}{2.153901in}}%
\pgfpathclose%
\pgfusepath{stroke,fill}%
\end{pgfscope}%
\begin{pgfscope}%
\pgfpathrectangle{\pgfqpoint{1.542338in}{0.880000in}}{\pgfqpoint{5.115323in}{6.160000in}}%
\pgfusepath{clip}%
\pgfsetbuttcap%
\pgfsetroundjoin%
\definecolor{currentfill}{rgb}{0.800000,0.200000,0.200000}%
\pgfsetfillcolor{currentfill}%
\pgfsetlinewidth{1.003750pt}%
\definecolor{currentstroke}{rgb}{0.800000,0.200000,0.200000}%
\pgfsetstrokecolor{currentstroke}%
\pgfsetdash{}{0pt}%
\pgfpathmoveto{\pgfqpoint{6.413756in}{2.283604in}}%
\pgfpathcurveto{\pgfqpoint{6.419580in}{2.283604in}}{\pgfqpoint{6.425166in}{2.285918in}}{\pgfqpoint{6.429284in}{2.290036in}}%
\pgfpathcurveto{\pgfqpoint{6.433402in}{2.294154in}}{\pgfqpoint{6.435716in}{2.299740in}}{\pgfqpoint{6.435716in}{2.305564in}}%
\pgfpathcurveto{\pgfqpoint{6.435716in}{2.311388in}}{\pgfqpoint{6.433402in}{2.316975in}}{\pgfqpoint{6.429284in}{2.321093in}}%
\pgfpathcurveto{\pgfqpoint{6.425166in}{2.325211in}}{\pgfqpoint{6.419580in}{2.327525in}}{\pgfqpoint{6.413756in}{2.327525in}}%
\pgfpathcurveto{\pgfqpoint{6.407932in}{2.327525in}}{\pgfqpoint{6.402346in}{2.325211in}}{\pgfqpoint{6.398228in}{2.321093in}}%
\pgfpathcurveto{\pgfqpoint{6.394110in}{2.316975in}}{\pgfqpoint{6.391796in}{2.311388in}}{\pgfqpoint{6.391796in}{2.305564in}}%
\pgfpathcurveto{\pgfqpoint{6.391796in}{2.299740in}}{\pgfqpoint{6.394110in}{2.294154in}}{\pgfqpoint{6.398228in}{2.290036in}}%
\pgfpathcurveto{\pgfqpoint{6.402346in}{2.285918in}}{\pgfqpoint{6.407932in}{2.283604in}}{\pgfqpoint{6.413756in}{2.283604in}}%
\pgfpathlineto{\pgfqpoint{6.413756in}{2.283604in}}%
\pgfpathclose%
\pgfusepath{stroke,fill}%
\end{pgfscope}%
\begin{pgfscope}%
\pgfpathrectangle{\pgfqpoint{1.542338in}{0.880000in}}{\pgfqpoint{5.115323in}{6.160000in}}%
\pgfusepath{clip}%
\pgfsetbuttcap%
\pgfsetroundjoin%
\definecolor{currentfill}{rgb}{0.800000,0.200000,0.200000}%
\pgfsetfillcolor{currentfill}%
\pgfsetlinewidth{1.003750pt}%
\definecolor{currentstroke}{rgb}{0.800000,0.200000,0.200000}%
\pgfsetstrokecolor{currentstroke}%
\pgfsetdash{}{0pt}%
\pgfpathmoveto{\pgfqpoint{6.390083in}{2.411485in}}%
\pgfpathcurveto{\pgfqpoint{6.395907in}{2.411485in}}{\pgfqpoint{6.401493in}{2.413799in}}{\pgfqpoint{6.405611in}{2.417917in}}%
\pgfpathcurveto{\pgfqpoint{6.409729in}{2.422035in}}{\pgfqpoint{6.412043in}{2.427621in}}{\pgfqpoint{6.412043in}{2.433445in}}%
\pgfpathcurveto{\pgfqpoint{6.412043in}{2.439269in}}{\pgfqpoint{6.409729in}{2.444855in}}{\pgfqpoint{6.405611in}{2.448973in}}%
\pgfpathcurveto{\pgfqpoint{6.401493in}{2.453091in}}{\pgfqpoint{6.395907in}{2.455405in}}{\pgfqpoint{6.390083in}{2.455405in}}%
\pgfpathcurveto{\pgfqpoint{6.384259in}{2.455405in}}{\pgfqpoint{6.378673in}{2.453091in}}{\pgfqpoint{6.374555in}{2.448973in}}%
\pgfpathcurveto{\pgfqpoint{6.370437in}{2.444855in}}{\pgfqpoint{6.368123in}{2.439269in}}{\pgfqpoint{6.368123in}{2.433445in}}%
\pgfpathcurveto{\pgfqpoint{6.368123in}{2.427621in}}{\pgfqpoint{6.370437in}{2.422035in}}{\pgfqpoint{6.374555in}{2.417917in}}%
\pgfpathcurveto{\pgfqpoint{6.378673in}{2.413799in}}{\pgfqpoint{6.384259in}{2.411485in}}{\pgfqpoint{6.390083in}{2.411485in}}%
\pgfpathlineto{\pgfqpoint{6.390083in}{2.411485in}}%
\pgfpathclose%
\pgfusepath{stroke,fill}%
\end{pgfscope}%
\begin{pgfscope}%
\pgfpathrectangle{\pgfqpoint{1.542338in}{0.880000in}}{\pgfqpoint{5.115323in}{6.160000in}}%
\pgfusepath{clip}%
\pgfsetbuttcap%
\pgfsetroundjoin%
\definecolor{currentfill}{rgb}{0.800000,0.200000,0.200000}%
\pgfsetfillcolor{currentfill}%
\pgfsetlinewidth{1.003750pt}%
\definecolor{currentstroke}{rgb}{0.800000,0.200000,0.200000}%
\pgfsetstrokecolor{currentstroke}%
\pgfsetdash{}{0pt}%
\pgfpathmoveto{\pgfqpoint{6.351541in}{2.535977in}}%
\pgfpathcurveto{\pgfqpoint{6.357364in}{2.535977in}}{\pgfqpoint{6.362951in}{2.538291in}}{\pgfqpoint{6.367069in}{2.542409in}}%
\pgfpathcurveto{\pgfqpoint{6.371187in}{2.546527in}}{\pgfqpoint{6.373501in}{2.552114in}}{\pgfqpoint{6.373501in}{2.557937in}}%
\pgfpathcurveto{\pgfqpoint{6.373501in}{2.563761in}}{\pgfqpoint{6.371187in}{2.569348in}}{\pgfqpoint{6.367069in}{2.573466in}}%
\pgfpathcurveto{\pgfqpoint{6.362951in}{2.577584in}}{\pgfqpoint{6.357364in}{2.579898in}}{\pgfqpoint{6.351541in}{2.579898in}}%
\pgfpathcurveto{\pgfqpoint{6.345717in}{2.579898in}}{\pgfqpoint{6.340130in}{2.577584in}}{\pgfqpoint{6.336012in}{2.573466in}}%
\pgfpathcurveto{\pgfqpoint{6.331894in}{2.569348in}}{\pgfqpoint{6.329580in}{2.563761in}}{\pgfqpoint{6.329580in}{2.557937in}}%
\pgfpathcurveto{\pgfqpoint{6.329580in}{2.552114in}}{\pgfqpoint{6.331894in}{2.546527in}}{\pgfqpoint{6.336012in}{2.542409in}}%
\pgfpathcurveto{\pgfqpoint{6.340130in}{2.538291in}}{\pgfqpoint{6.345717in}{2.535977in}}{\pgfqpoint{6.351541in}{2.535977in}}%
\pgfpathlineto{\pgfqpoint{6.351541in}{2.535977in}}%
\pgfpathclose%
\pgfusepath{stroke,fill}%
\end{pgfscope}%
\begin{pgfscope}%
\pgfpathrectangle{\pgfqpoint{1.542338in}{0.880000in}}{\pgfqpoint{5.115323in}{6.160000in}}%
\pgfusepath{clip}%
\pgfsetbuttcap%
\pgfsetroundjoin%
\definecolor{currentfill}{rgb}{0.800000,0.200000,0.200000}%
\pgfsetfillcolor{currentfill}%
\pgfsetlinewidth{1.003750pt}%
\definecolor{currentstroke}{rgb}{0.800000,0.200000,0.200000}%
\pgfsetstrokecolor{currentstroke}%
\pgfsetdash{}{0pt}%
\pgfpathmoveto{\pgfqpoint{6.295859in}{2.654044in}}%
\pgfpathcurveto{\pgfqpoint{6.301683in}{2.654044in}}{\pgfqpoint{6.307269in}{2.656358in}}{\pgfqpoint{6.311387in}{2.660476in}}%
\pgfpathcurveto{\pgfqpoint{6.315505in}{2.664594in}}{\pgfqpoint{6.317819in}{2.670180in}}{\pgfqpoint{6.317819in}{2.676004in}}%
\pgfpathcurveto{\pgfqpoint{6.317819in}{2.681828in}}{\pgfqpoint{6.315505in}{2.687414in}}{\pgfqpoint{6.311387in}{2.691532in}}%
\pgfpathcurveto{\pgfqpoint{6.307269in}{2.695650in}}{\pgfqpoint{6.301683in}{2.697964in}}{\pgfqpoint{6.295859in}{2.697964in}}%
\pgfpathcurveto{\pgfqpoint{6.290035in}{2.697964in}}{\pgfqpoint{6.284449in}{2.695650in}}{\pgfqpoint{6.280331in}{2.691532in}}%
\pgfpathcurveto{\pgfqpoint{6.276213in}{2.687414in}}{\pgfqpoint{6.273899in}{2.681828in}}{\pgfqpoint{6.273899in}{2.676004in}}%
\pgfpathcurveto{\pgfqpoint{6.273899in}{2.670180in}}{\pgfqpoint{6.276213in}{2.664594in}}{\pgfqpoint{6.280331in}{2.660476in}}%
\pgfpathcurveto{\pgfqpoint{6.284449in}{2.656358in}}{\pgfqpoint{6.290035in}{2.654044in}}{\pgfqpoint{6.295859in}{2.654044in}}%
\pgfpathlineto{\pgfqpoint{6.295859in}{2.654044in}}%
\pgfpathclose%
\pgfusepath{stroke,fill}%
\end{pgfscope}%
\begin{pgfscope}%
\pgfpathrectangle{\pgfqpoint{1.542338in}{0.880000in}}{\pgfqpoint{5.115323in}{6.160000in}}%
\pgfusepath{clip}%
\pgfsetbuttcap%
\pgfsetroundjoin%
\definecolor{currentfill}{rgb}{0.800000,0.200000,0.200000}%
\pgfsetfillcolor{currentfill}%
\pgfsetlinewidth{1.003750pt}%
\definecolor{currentstroke}{rgb}{0.800000,0.200000,0.200000}%
\pgfsetstrokecolor{currentstroke}%
\pgfsetdash{}{0pt}%
\pgfpathmoveto{\pgfqpoint{6.227647in}{2.765765in}}%
\pgfpathcurveto{\pgfqpoint{6.233471in}{2.765765in}}{\pgfqpoint{6.239057in}{2.768079in}}{\pgfqpoint{6.243175in}{2.772197in}}%
\pgfpathcurveto{\pgfqpoint{6.247294in}{2.776315in}}{\pgfqpoint{6.249607in}{2.781901in}}{\pgfqpoint{6.249607in}{2.787725in}}%
\pgfpathcurveto{\pgfqpoint{6.249607in}{2.793549in}}{\pgfqpoint{6.247294in}{2.799135in}}{\pgfqpoint{6.243175in}{2.803254in}}%
\pgfpathcurveto{\pgfqpoint{6.239057in}{2.807372in}}{\pgfqpoint{6.233471in}{2.809686in}}{\pgfqpoint{6.227647in}{2.809686in}}%
\pgfpathcurveto{\pgfqpoint{6.221823in}{2.809686in}}{\pgfqpoint{6.216237in}{2.807372in}}{\pgfqpoint{6.212119in}{2.803254in}}%
\pgfpathcurveto{\pgfqpoint{6.208001in}{2.799135in}}{\pgfqpoint{6.205687in}{2.793549in}}{\pgfqpoint{6.205687in}{2.787725in}}%
\pgfpathcurveto{\pgfqpoint{6.205687in}{2.781901in}}{\pgfqpoint{6.208001in}{2.776315in}}{\pgfqpoint{6.212119in}{2.772197in}}%
\pgfpathcurveto{\pgfqpoint{6.216237in}{2.768079in}}{\pgfqpoint{6.221823in}{2.765765in}}{\pgfqpoint{6.227647in}{2.765765in}}%
\pgfpathlineto{\pgfqpoint{6.227647in}{2.765765in}}%
\pgfpathclose%
\pgfusepath{stroke,fill}%
\end{pgfscope}%
\begin{pgfscope}%
\pgfpathrectangle{\pgfqpoint{1.542338in}{0.880000in}}{\pgfqpoint{5.115323in}{6.160000in}}%
\pgfusepath{clip}%
\pgfsetbuttcap%
\pgfsetroundjoin%
\definecolor{currentfill}{rgb}{0.800000,0.200000,0.200000}%
\pgfsetfillcolor{currentfill}%
\pgfsetlinewidth{1.003750pt}%
\definecolor{currentstroke}{rgb}{0.800000,0.200000,0.200000}%
\pgfsetstrokecolor{currentstroke}%
\pgfsetdash{}{0pt}%
\pgfpathmoveto{\pgfqpoint{6.133648in}{2.856840in}}%
\pgfpathcurveto{\pgfqpoint{6.139472in}{2.856840in}}{\pgfqpoint{6.145058in}{2.859154in}}{\pgfqpoint{6.149176in}{2.863272in}}%
\pgfpathcurveto{\pgfqpoint{6.153294in}{2.867390in}}{\pgfqpoint{6.155608in}{2.872977in}}{\pgfqpoint{6.155608in}{2.878801in}}%
\pgfpathcurveto{\pgfqpoint{6.155608in}{2.884625in}}{\pgfqpoint{6.153294in}{2.890211in}}{\pgfqpoint{6.149176in}{2.894329in}}%
\pgfpathcurveto{\pgfqpoint{6.145058in}{2.898447in}}{\pgfqpoint{6.139472in}{2.900761in}}{\pgfqpoint{6.133648in}{2.900761in}}%
\pgfpathcurveto{\pgfqpoint{6.127824in}{2.900761in}}{\pgfqpoint{6.122238in}{2.898447in}}{\pgfqpoint{6.118119in}{2.894329in}}%
\pgfpathcurveto{\pgfqpoint{6.114001in}{2.890211in}}{\pgfqpoint{6.111687in}{2.884625in}}{\pgfqpoint{6.111687in}{2.878801in}}%
\pgfpathcurveto{\pgfqpoint{6.111687in}{2.872977in}}{\pgfqpoint{6.114001in}{2.867390in}}{\pgfqpoint{6.118119in}{2.863272in}}%
\pgfpathcurveto{\pgfqpoint{6.122238in}{2.859154in}}{\pgfqpoint{6.127824in}{2.856840in}}{\pgfqpoint{6.133648in}{2.856840in}}%
\pgfpathlineto{\pgfqpoint{6.133648in}{2.856840in}}%
\pgfpathclose%
\pgfusepath{stroke,fill}%
\end{pgfscope}%
\begin{pgfscope}%
\pgfpathrectangle{\pgfqpoint{1.542338in}{0.880000in}}{\pgfqpoint{5.115323in}{6.160000in}}%
\pgfusepath{clip}%
\pgfsetbuttcap%
\pgfsetroundjoin%
\definecolor{currentfill}{rgb}{0.800000,0.200000,0.200000}%
\pgfsetfillcolor{currentfill}%
\pgfsetlinewidth{1.003750pt}%
\definecolor{currentstroke}{rgb}{0.800000,0.200000,0.200000}%
\pgfsetstrokecolor{currentstroke}%
\pgfsetdash{}{0pt}%
\pgfpathmoveto{\pgfqpoint{6.042527in}{2.949817in}}%
\pgfpathcurveto{\pgfqpoint{6.048351in}{2.949817in}}{\pgfqpoint{6.053937in}{2.952131in}}{\pgfqpoint{6.058055in}{2.956249in}}%
\pgfpathcurveto{\pgfqpoint{6.062174in}{2.960367in}}{\pgfqpoint{6.064487in}{2.965954in}}{\pgfqpoint{6.064487in}{2.971778in}}%
\pgfpathcurveto{\pgfqpoint{6.064487in}{2.977601in}}{\pgfqpoint{6.062174in}{2.983188in}}{\pgfqpoint{6.058055in}{2.987306in}}%
\pgfpathcurveto{\pgfqpoint{6.053937in}{2.991424in}}{\pgfqpoint{6.048351in}{2.993738in}}{\pgfqpoint{6.042527in}{2.993738in}}%
\pgfpathcurveto{\pgfqpoint{6.036703in}{2.993738in}}{\pgfqpoint{6.031117in}{2.991424in}}{\pgfqpoint{6.026999in}{2.987306in}}%
\pgfpathcurveto{\pgfqpoint{6.022881in}{2.983188in}}{\pgfqpoint{6.020567in}{2.977601in}}{\pgfqpoint{6.020567in}{2.971778in}}%
\pgfpathcurveto{\pgfqpoint{6.020567in}{2.965954in}}{\pgfqpoint{6.022881in}{2.960367in}}{\pgfqpoint{6.026999in}{2.956249in}}%
\pgfpathcurveto{\pgfqpoint{6.031117in}{2.952131in}}{\pgfqpoint{6.036703in}{2.949817in}}{\pgfqpoint{6.042527in}{2.949817in}}%
\pgfpathlineto{\pgfqpoint{6.042527in}{2.949817in}}%
\pgfpathclose%
\pgfusepath{stroke,fill}%
\end{pgfscope}%
\begin{pgfscope}%
\pgfpathrectangle{\pgfqpoint{1.542338in}{0.880000in}}{\pgfqpoint{5.115323in}{6.160000in}}%
\pgfusepath{clip}%
\pgfsetbuttcap%
\pgfsetroundjoin%
\definecolor{currentfill}{rgb}{0.800000,0.200000,0.200000}%
\pgfsetfillcolor{currentfill}%
\pgfsetlinewidth{1.003750pt}%
\definecolor{currentstroke}{rgb}{0.800000,0.200000,0.200000}%
\pgfsetstrokecolor{currentstroke}%
\pgfsetdash{}{0pt}%
\pgfpathmoveto{\pgfqpoint{5.935374in}{3.024181in}}%
\pgfpathcurveto{\pgfqpoint{5.941198in}{3.024181in}}{\pgfqpoint{5.946784in}{3.026495in}}{\pgfqpoint{5.950902in}{3.030613in}}%
\pgfpathcurveto{\pgfqpoint{5.955021in}{3.034731in}}{\pgfqpoint{5.957334in}{3.040318in}}{\pgfqpoint{5.957334in}{3.046142in}}%
\pgfpathcurveto{\pgfqpoint{5.957334in}{3.051965in}}{\pgfqpoint{5.955021in}{3.057552in}}{\pgfqpoint{5.950902in}{3.061670in}}%
\pgfpathcurveto{\pgfqpoint{5.946784in}{3.065788in}}{\pgfqpoint{5.941198in}{3.068102in}}{\pgfqpoint{5.935374in}{3.068102in}}%
\pgfpathcurveto{\pgfqpoint{5.929550in}{3.068102in}}{\pgfqpoint{5.923964in}{3.065788in}}{\pgfqpoint{5.919846in}{3.061670in}}%
\pgfpathcurveto{\pgfqpoint{5.915728in}{3.057552in}}{\pgfqpoint{5.913414in}{3.051965in}}{\pgfqpoint{5.913414in}{3.046142in}}%
\pgfpathcurveto{\pgfqpoint{5.913414in}{3.040318in}}{\pgfqpoint{5.915728in}{3.034731in}}{\pgfqpoint{5.919846in}{3.030613in}}%
\pgfpathcurveto{\pgfqpoint{5.923964in}{3.026495in}}{\pgfqpoint{5.929550in}{3.024181in}}{\pgfqpoint{5.935374in}{3.024181in}}%
\pgfpathlineto{\pgfqpoint{5.935374in}{3.024181in}}%
\pgfpathclose%
\pgfusepath{stroke,fill}%
\end{pgfscope}%
\begin{pgfscope}%
\pgfpathrectangle{\pgfqpoint{1.542338in}{0.880000in}}{\pgfqpoint{5.115323in}{6.160000in}}%
\pgfusepath{clip}%
\pgfsetbuttcap%
\pgfsetroundjoin%
\definecolor{currentfill}{rgb}{0.800000,0.200000,0.200000}%
\pgfsetfillcolor{currentfill}%
\pgfsetlinewidth{1.003750pt}%
\definecolor{currentstroke}{rgb}{0.800000,0.200000,0.200000}%
\pgfsetstrokecolor{currentstroke}%
\pgfsetdash{}{0pt}%
\pgfpathmoveto{\pgfqpoint{5.819344in}{3.083575in}}%
\pgfpathcurveto{\pgfqpoint{5.825168in}{3.083575in}}{\pgfqpoint{5.830754in}{3.085889in}}{\pgfqpoint{5.834872in}{3.090007in}}%
\pgfpathcurveto{\pgfqpoint{5.838991in}{3.094125in}}{\pgfqpoint{5.841304in}{3.099711in}}{\pgfqpoint{5.841304in}{3.105535in}}%
\pgfpathcurveto{\pgfqpoint{5.841304in}{3.111359in}}{\pgfqpoint{5.838991in}{3.116945in}}{\pgfqpoint{5.834872in}{3.121063in}}%
\pgfpathcurveto{\pgfqpoint{5.830754in}{3.125181in}}{\pgfqpoint{5.825168in}{3.127495in}}{\pgfqpoint{5.819344in}{3.127495in}}%
\pgfpathcurveto{\pgfqpoint{5.813520in}{3.127495in}}{\pgfqpoint{5.807934in}{3.125181in}}{\pgfqpoint{5.803816in}{3.121063in}}%
\pgfpathcurveto{\pgfqpoint{5.799698in}{3.116945in}}{\pgfqpoint{5.797384in}{3.111359in}}{\pgfqpoint{5.797384in}{3.105535in}}%
\pgfpathcurveto{\pgfqpoint{5.797384in}{3.099711in}}{\pgfqpoint{5.799698in}{3.094125in}}{\pgfqpoint{5.803816in}{3.090007in}}%
\pgfpathcurveto{\pgfqpoint{5.807934in}{3.085889in}}{\pgfqpoint{5.813520in}{3.083575in}}{\pgfqpoint{5.819344in}{3.083575in}}%
\pgfpathlineto{\pgfqpoint{5.819344in}{3.083575in}}%
\pgfpathclose%
\pgfusepath{stroke,fill}%
\end{pgfscope}%
\begin{pgfscope}%
\pgfpathrectangle{\pgfqpoint{1.542338in}{0.880000in}}{\pgfqpoint{5.115323in}{6.160000in}}%
\pgfusepath{clip}%
\pgfsetbuttcap%
\pgfsetroundjoin%
\definecolor{currentfill}{rgb}{0.800000,0.200000,0.200000}%
\pgfsetfillcolor{currentfill}%
\pgfsetlinewidth{1.003750pt}%
\definecolor{currentstroke}{rgb}{0.800000,0.200000,0.200000}%
\pgfsetstrokecolor{currentstroke}%
\pgfsetdash{}{0pt}%
\pgfpathmoveto{\pgfqpoint{5.697564in}{3.130193in}}%
\pgfpathcurveto{\pgfqpoint{5.703388in}{3.130193in}}{\pgfqpoint{5.708974in}{3.132507in}}{\pgfqpoint{5.713092in}{3.136625in}}%
\pgfpathcurveto{\pgfqpoint{5.717210in}{3.140744in}}{\pgfqpoint{5.719524in}{3.146330in}}{\pgfqpoint{5.719524in}{3.152154in}}%
\pgfpathcurveto{\pgfqpoint{5.719524in}{3.157978in}}{\pgfqpoint{5.717210in}{3.163564in}}{\pgfqpoint{5.713092in}{3.167682in}}%
\pgfpathcurveto{\pgfqpoint{5.708974in}{3.171800in}}{\pgfqpoint{5.703388in}{3.174114in}}{\pgfqpoint{5.697564in}{3.174114in}}%
\pgfpathcurveto{\pgfqpoint{5.691740in}{3.174114in}}{\pgfqpoint{5.686154in}{3.171800in}}{\pgfqpoint{5.682035in}{3.167682in}}%
\pgfpathcurveto{\pgfqpoint{5.677917in}{3.163564in}}{\pgfqpoint{5.675603in}{3.157978in}}{\pgfqpoint{5.675603in}{3.152154in}}%
\pgfpathcurveto{\pgfqpoint{5.675603in}{3.146330in}}{\pgfqpoint{5.677917in}{3.140744in}}{\pgfqpoint{5.682035in}{3.136625in}}%
\pgfpathcurveto{\pgfqpoint{5.686154in}{3.132507in}}{\pgfqpoint{5.691740in}{3.130193in}}{\pgfqpoint{5.697564in}{3.130193in}}%
\pgfpathlineto{\pgfqpoint{5.697564in}{3.130193in}}%
\pgfpathclose%
\pgfusepath{stroke,fill}%
\end{pgfscope}%
\begin{pgfscope}%
\pgfpathrectangle{\pgfqpoint{1.542338in}{0.880000in}}{\pgfqpoint{5.115323in}{6.160000in}}%
\pgfusepath{clip}%
\pgfsetbuttcap%
\pgfsetroundjoin%
\definecolor{currentfill}{rgb}{0.800000,0.200000,0.200000}%
\pgfsetfillcolor{currentfill}%
\pgfsetlinewidth{1.003750pt}%
\definecolor{currentstroke}{rgb}{0.800000,0.200000,0.200000}%
\pgfsetstrokecolor{currentstroke}%
\pgfsetdash{}{0pt}%
\pgfpathmoveto{\pgfqpoint{5.569964in}{3.156913in}}%
\pgfpathcurveto{\pgfqpoint{5.575788in}{3.156913in}}{\pgfqpoint{5.581375in}{3.159227in}}{\pgfqpoint{5.585493in}{3.163345in}}%
\pgfpathcurveto{\pgfqpoint{5.589611in}{3.167463in}}{\pgfqpoint{5.591925in}{3.173050in}}{\pgfqpoint{5.591925in}{3.178874in}}%
\pgfpathcurveto{\pgfqpoint{5.591925in}{3.184697in}}{\pgfqpoint{5.589611in}{3.190284in}}{\pgfqpoint{5.585493in}{3.194402in}}%
\pgfpathcurveto{\pgfqpoint{5.581375in}{3.198520in}}{\pgfqpoint{5.575788in}{3.200834in}}{\pgfqpoint{5.569964in}{3.200834in}}%
\pgfpathcurveto{\pgfqpoint{5.564141in}{3.200834in}}{\pgfqpoint{5.558554in}{3.198520in}}{\pgfqpoint{5.554436in}{3.194402in}}%
\pgfpathcurveto{\pgfqpoint{5.550318in}{3.190284in}}{\pgfqpoint{5.548004in}{3.184697in}}{\pgfqpoint{5.548004in}{3.178874in}}%
\pgfpathcurveto{\pgfqpoint{5.548004in}{3.173050in}}{\pgfqpoint{5.550318in}{3.167463in}}{\pgfqpoint{5.554436in}{3.163345in}}%
\pgfpathcurveto{\pgfqpoint{5.558554in}{3.159227in}}{\pgfqpoint{5.564141in}{3.156913in}}{\pgfqpoint{5.569964in}{3.156913in}}%
\pgfpathlineto{\pgfqpoint{5.569964in}{3.156913in}}%
\pgfpathclose%
\pgfusepath{stroke,fill}%
\end{pgfscope}%
\begin{pgfscope}%
\pgfpathrectangle{\pgfqpoint{1.542338in}{0.880000in}}{\pgfqpoint{5.115323in}{6.160000in}}%
\pgfusepath{clip}%
\pgfsetbuttcap%
\pgfsetroundjoin%
\definecolor{currentfill}{rgb}{0.800000,0.200000,0.200000}%
\pgfsetfillcolor{currentfill}%
\pgfsetlinewidth{1.003750pt}%
\definecolor{currentstroke}{rgb}{0.800000,0.200000,0.200000}%
\pgfsetstrokecolor{currentstroke}%
\pgfsetdash{}{0pt}%
\pgfpathmoveto{\pgfqpoint{5.440272in}{3.166328in}}%
\pgfpathcurveto{\pgfqpoint{5.446096in}{3.166328in}}{\pgfqpoint{5.451682in}{3.168642in}}{\pgfqpoint{5.455800in}{3.172760in}}%
\pgfpathcurveto{\pgfqpoint{5.459918in}{3.176879in}}{\pgfqpoint{5.462232in}{3.182465in}}{\pgfqpoint{5.462232in}{3.188289in}}%
\pgfpathcurveto{\pgfqpoint{5.462232in}{3.194113in}}{\pgfqpoint{5.459918in}{3.199699in}}{\pgfqpoint{5.455800in}{3.203817in}}%
\pgfpathcurveto{\pgfqpoint{5.451682in}{3.207935in}}{\pgfqpoint{5.446096in}{3.210249in}}{\pgfqpoint{5.440272in}{3.210249in}}%
\pgfpathcurveto{\pgfqpoint{5.434448in}{3.210249in}}{\pgfqpoint{5.428862in}{3.207935in}}{\pgfqpoint{5.424744in}{3.203817in}}%
\pgfpathcurveto{\pgfqpoint{5.420625in}{3.199699in}}{\pgfqpoint{5.418312in}{3.194113in}}{\pgfqpoint{5.418312in}{3.188289in}}%
\pgfpathcurveto{\pgfqpoint{5.418312in}{3.182465in}}{\pgfqpoint{5.420625in}{3.176879in}}{\pgfqpoint{5.424744in}{3.172760in}}%
\pgfpathcurveto{\pgfqpoint{5.428862in}{3.168642in}}{\pgfqpoint{5.434448in}{3.166328in}}{\pgfqpoint{5.440272in}{3.166328in}}%
\pgfpathlineto{\pgfqpoint{5.440272in}{3.166328in}}%
\pgfpathclose%
\pgfusepath{stroke,fill}%
\end{pgfscope}%
\begin{pgfscope}%
\pgfpathrectangle{\pgfqpoint{1.542338in}{0.880000in}}{\pgfqpoint{5.115323in}{6.160000in}}%
\pgfusepath{clip}%
\pgfsetbuttcap%
\pgfsetroundjoin%
\definecolor{currentfill}{rgb}{0.800000,0.200000,0.200000}%
\pgfsetfillcolor{currentfill}%
\pgfsetlinewidth{1.003750pt}%
\definecolor{currentstroke}{rgb}{0.800000,0.200000,0.200000}%
\pgfsetstrokecolor{currentstroke}%
\pgfsetdash{}{0pt}%
\pgfpathmoveto{\pgfqpoint{5.310072in}{3.167012in}}%
\pgfpathcurveto{\pgfqpoint{5.315896in}{3.167012in}}{\pgfqpoint{5.321482in}{3.169326in}}{\pgfqpoint{5.325600in}{3.173444in}}%
\pgfpathcurveto{\pgfqpoint{5.329718in}{3.177563in}}{\pgfqpoint{5.332032in}{3.183149in}}{\pgfqpoint{5.332032in}{3.188973in}}%
\pgfpathcurveto{\pgfqpoint{5.332032in}{3.194797in}}{\pgfqpoint{5.329718in}{3.200383in}}{\pgfqpoint{5.325600in}{3.204501in}}%
\pgfpathcurveto{\pgfqpoint{5.321482in}{3.208619in}}{\pgfqpoint{5.315896in}{3.210933in}}{\pgfqpoint{5.310072in}{3.210933in}}%
\pgfpathcurveto{\pgfqpoint{5.304248in}{3.210933in}}{\pgfqpoint{5.298661in}{3.208619in}}{\pgfqpoint{5.294543in}{3.204501in}}%
\pgfpathcurveto{\pgfqpoint{5.290425in}{3.200383in}}{\pgfqpoint{5.288111in}{3.194797in}}{\pgfqpoint{5.288111in}{3.188973in}}%
\pgfpathcurveto{\pgfqpoint{5.288111in}{3.183149in}}{\pgfqpoint{5.290425in}{3.177563in}}{\pgfqpoint{5.294543in}{3.173444in}}%
\pgfpathcurveto{\pgfqpoint{5.298661in}{3.169326in}}{\pgfqpoint{5.304248in}{3.167012in}}{\pgfqpoint{5.310072in}{3.167012in}}%
\pgfpathlineto{\pgfqpoint{5.310072in}{3.167012in}}%
\pgfpathclose%
\pgfusepath{stroke,fill}%
\end{pgfscope}%
\begin{pgfscope}%
\pgfpathrectangle{\pgfqpoint{1.542338in}{0.880000in}}{\pgfqpoint{5.115323in}{6.160000in}}%
\pgfusepath{clip}%
\pgfsetbuttcap%
\pgfsetroundjoin%
\definecolor{currentfill}{rgb}{0.800000,0.200000,0.200000}%
\pgfsetfillcolor{currentfill}%
\pgfsetlinewidth{1.003750pt}%
\definecolor{currentstroke}{rgb}{0.800000,0.200000,0.200000}%
\pgfsetstrokecolor{currentstroke}%
\pgfsetdash{}{0pt}%
\pgfpathmoveto{\pgfqpoint{5.180931in}{3.147902in}}%
\pgfpathcurveto{\pgfqpoint{5.186755in}{3.147902in}}{\pgfqpoint{5.192341in}{3.150216in}}{\pgfqpoint{5.196459in}{3.154334in}}%
\pgfpathcurveto{\pgfqpoint{5.200577in}{3.158452in}}{\pgfqpoint{5.202891in}{3.164039in}}{\pgfqpoint{5.202891in}{3.169863in}}%
\pgfpathcurveto{\pgfqpoint{5.202891in}{3.175686in}}{\pgfqpoint{5.200577in}{3.181273in}}{\pgfqpoint{5.196459in}{3.185391in}}%
\pgfpathcurveto{\pgfqpoint{5.192341in}{3.189509in}}{\pgfqpoint{5.186755in}{3.191823in}}{\pgfqpoint{5.180931in}{3.191823in}}%
\pgfpathcurveto{\pgfqpoint{5.175107in}{3.191823in}}{\pgfqpoint{5.169521in}{3.189509in}}{\pgfqpoint{5.165403in}{3.185391in}}%
\pgfpathcurveto{\pgfqpoint{5.161285in}{3.181273in}}{\pgfqpoint{5.158971in}{3.175686in}}{\pgfqpoint{5.158971in}{3.169863in}}%
\pgfpathcurveto{\pgfqpoint{5.158971in}{3.164039in}}{\pgfqpoint{5.161285in}{3.158452in}}{\pgfqpoint{5.165403in}{3.154334in}}%
\pgfpathcurveto{\pgfqpoint{5.169521in}{3.150216in}}{\pgfqpoint{5.175107in}{3.147902in}}{\pgfqpoint{5.180931in}{3.147902in}}%
\pgfpathlineto{\pgfqpoint{5.180931in}{3.147902in}}%
\pgfpathclose%
\pgfusepath{stroke,fill}%
\end{pgfscope}%
\begin{pgfscope}%
\pgfpathrectangle{\pgfqpoint{1.542338in}{0.880000in}}{\pgfqpoint{5.115323in}{6.160000in}}%
\pgfusepath{clip}%
\pgfsetbuttcap%
\pgfsetroundjoin%
\definecolor{currentfill}{rgb}{0.800000,0.200000,0.200000}%
\pgfsetfillcolor{currentfill}%
\pgfsetlinewidth{1.003750pt}%
\definecolor{currentstroke}{rgb}{0.800000,0.200000,0.200000}%
\pgfsetstrokecolor{currentstroke}%
\pgfsetdash{}{0pt}%
\pgfpathmoveto{\pgfqpoint{5.056649in}{3.108105in}}%
\pgfpathcurveto{\pgfqpoint{5.062473in}{3.108105in}}{\pgfqpoint{5.068059in}{3.110419in}}{\pgfqpoint{5.072178in}{3.114537in}}%
\pgfpathcurveto{\pgfqpoint{5.076296in}{3.118655in}}{\pgfqpoint{5.078610in}{3.124242in}}{\pgfqpoint{5.078610in}{3.130066in}}%
\pgfpathcurveto{\pgfqpoint{5.078610in}{3.135889in}}{\pgfqpoint{5.076296in}{3.141476in}}{\pgfqpoint{5.072178in}{3.145594in}}%
\pgfpathcurveto{\pgfqpoint{5.068059in}{3.149712in}}{\pgfqpoint{5.062473in}{3.152026in}}{\pgfqpoint{5.056649in}{3.152026in}}%
\pgfpathcurveto{\pgfqpoint{5.050825in}{3.152026in}}{\pgfqpoint{5.045239in}{3.149712in}}{\pgfqpoint{5.041121in}{3.145594in}}%
\pgfpathcurveto{\pgfqpoint{5.037003in}{3.141476in}}{\pgfqpoint{5.034689in}{3.135889in}}{\pgfqpoint{5.034689in}{3.130066in}}%
\pgfpathcurveto{\pgfqpoint{5.034689in}{3.124242in}}{\pgfqpoint{5.037003in}{3.118655in}}{\pgfqpoint{5.041121in}{3.114537in}}%
\pgfpathcurveto{\pgfqpoint{5.045239in}{3.110419in}}{\pgfqpoint{5.050825in}{3.108105in}}{\pgfqpoint{5.056649in}{3.108105in}}%
\pgfpathlineto{\pgfqpoint{5.056649in}{3.108105in}}%
\pgfpathclose%
\pgfusepath{stroke,fill}%
\end{pgfscope}%
\begin{pgfscope}%
\pgfpathrectangle{\pgfqpoint{1.542338in}{0.880000in}}{\pgfqpoint{5.115323in}{6.160000in}}%
\pgfusepath{clip}%
\pgfsetbuttcap%
\pgfsetroundjoin%
\definecolor{currentfill}{rgb}{0.800000,0.200000,0.200000}%
\pgfsetfillcolor{currentfill}%
\pgfsetlinewidth{1.003750pt}%
\definecolor{currentstroke}{rgb}{0.800000,0.200000,0.200000}%
\pgfsetstrokecolor{currentstroke}%
\pgfsetdash{}{0pt}%
\pgfpathmoveto{\pgfqpoint{4.935456in}{3.059305in}}%
\pgfpathcurveto{\pgfqpoint{4.941280in}{3.059305in}}{\pgfqpoint{4.946867in}{3.061619in}}{\pgfqpoint{4.950985in}{3.065737in}}%
\pgfpathcurveto{\pgfqpoint{4.955103in}{3.069855in}}{\pgfqpoint{4.957417in}{3.075442in}}{\pgfqpoint{4.957417in}{3.081265in}}%
\pgfpathcurveto{\pgfqpoint{4.957417in}{3.087089in}}{\pgfqpoint{4.955103in}{3.092676in}}{\pgfqpoint{4.950985in}{3.096794in}}%
\pgfpathcurveto{\pgfqpoint{4.946867in}{3.100912in}}{\pgfqpoint{4.941280in}{3.103226in}}{\pgfqpoint{4.935456in}{3.103226in}}%
\pgfpathcurveto{\pgfqpoint{4.929633in}{3.103226in}}{\pgfqpoint{4.924046in}{3.100912in}}{\pgfqpoint{4.919928in}{3.096794in}}%
\pgfpathcurveto{\pgfqpoint{4.915810in}{3.092676in}}{\pgfqpoint{4.913496in}{3.087089in}}{\pgfqpoint{4.913496in}{3.081265in}}%
\pgfpathcurveto{\pgfqpoint{4.913496in}{3.075442in}}{\pgfqpoint{4.915810in}{3.069855in}}{\pgfqpoint{4.919928in}{3.065737in}}%
\pgfpathcurveto{\pgfqpoint{4.924046in}{3.061619in}}{\pgfqpoint{4.929633in}{3.059305in}}{\pgfqpoint{4.935456in}{3.059305in}}%
\pgfpathlineto{\pgfqpoint{4.935456in}{3.059305in}}%
\pgfpathclose%
\pgfusepath{stroke,fill}%
\end{pgfscope}%
\begin{pgfscope}%
\pgfpathrectangle{\pgfqpoint{1.542338in}{0.880000in}}{\pgfqpoint{5.115323in}{6.160000in}}%
\pgfusepath{clip}%
\pgfsetbuttcap%
\pgfsetroundjoin%
\definecolor{currentfill}{rgb}{0.800000,0.200000,0.200000}%
\pgfsetfillcolor{currentfill}%
\pgfsetlinewidth{1.003750pt}%
\definecolor{currentstroke}{rgb}{0.800000,0.200000,0.200000}%
\pgfsetstrokecolor{currentstroke}%
\pgfsetdash{}{0pt}%
\pgfpathmoveto{\pgfqpoint{4.826532in}{2.987200in}}%
\pgfpathcurveto{\pgfqpoint{4.832356in}{2.987200in}}{\pgfqpoint{4.837942in}{2.989514in}}{\pgfqpoint{4.842060in}{2.993632in}}%
\pgfpathcurveto{\pgfqpoint{4.846179in}{2.997750in}}{\pgfqpoint{4.848492in}{3.003336in}}{\pgfqpoint{4.848492in}{3.009160in}}%
\pgfpathcurveto{\pgfqpoint{4.848492in}{3.014984in}}{\pgfqpoint{4.846179in}{3.020570in}}{\pgfqpoint{4.842060in}{3.024688in}}%
\pgfpathcurveto{\pgfqpoint{4.837942in}{3.028806in}}{\pgfqpoint{4.832356in}{3.031120in}}{\pgfqpoint{4.826532in}{3.031120in}}%
\pgfpathcurveto{\pgfqpoint{4.820708in}{3.031120in}}{\pgfqpoint{4.815122in}{3.028806in}}{\pgfqpoint{4.811004in}{3.024688in}}%
\pgfpathcurveto{\pgfqpoint{4.806886in}{3.020570in}}{\pgfqpoint{4.804572in}{3.014984in}}{\pgfqpoint{4.804572in}{3.009160in}}%
\pgfpathcurveto{\pgfqpoint{4.804572in}{3.003336in}}{\pgfqpoint{4.806886in}{2.997750in}}{\pgfqpoint{4.811004in}{2.993632in}}%
\pgfpathcurveto{\pgfqpoint{4.815122in}{2.989514in}}{\pgfqpoint{4.820708in}{2.987200in}}{\pgfqpoint{4.826532in}{2.987200in}}%
\pgfpathlineto{\pgfqpoint{4.826532in}{2.987200in}}%
\pgfpathclose%
\pgfusepath{stroke,fill}%
\end{pgfscope}%
\begin{pgfscope}%
\pgfpathrectangle{\pgfqpoint{1.542338in}{0.880000in}}{\pgfqpoint{5.115323in}{6.160000in}}%
\pgfusepath{clip}%
\pgfsetbuttcap%
\pgfsetroundjoin%
\definecolor{currentfill}{rgb}{0.800000,0.200000,0.200000}%
\pgfsetfillcolor{currentfill}%
\pgfsetlinewidth{1.003750pt}%
\definecolor{currentstroke}{rgb}{0.800000,0.200000,0.200000}%
\pgfsetstrokecolor{currentstroke}%
\pgfsetdash{}{0pt}%
\pgfpathmoveto{\pgfqpoint{4.722861in}{2.908101in}}%
\pgfpathcurveto{\pgfqpoint{4.728685in}{2.908101in}}{\pgfqpoint{4.734271in}{2.910415in}}{\pgfqpoint{4.738389in}{2.914533in}}%
\pgfpathcurveto{\pgfqpoint{4.742507in}{2.918651in}}{\pgfqpoint{4.744821in}{2.924237in}}{\pgfqpoint{4.744821in}{2.930061in}}%
\pgfpathcurveto{\pgfqpoint{4.744821in}{2.935885in}}{\pgfqpoint{4.742507in}{2.941471in}}{\pgfqpoint{4.738389in}{2.945590in}}%
\pgfpathcurveto{\pgfqpoint{4.734271in}{2.949708in}}{\pgfqpoint{4.728685in}{2.952022in}}{\pgfqpoint{4.722861in}{2.952022in}}%
\pgfpathcurveto{\pgfqpoint{4.717037in}{2.952022in}}{\pgfqpoint{4.711451in}{2.949708in}}{\pgfqpoint{4.707333in}{2.945590in}}%
\pgfpathcurveto{\pgfqpoint{4.703214in}{2.941471in}}{\pgfqpoint{4.700901in}{2.935885in}}{\pgfqpoint{4.700901in}{2.930061in}}%
\pgfpathcurveto{\pgfqpoint{4.700901in}{2.924237in}}{\pgfqpoint{4.703214in}{2.918651in}}{\pgfqpoint{4.707333in}{2.914533in}}%
\pgfpathcurveto{\pgfqpoint{4.711451in}{2.910415in}}{\pgfqpoint{4.717037in}{2.908101in}}{\pgfqpoint{4.722861in}{2.908101in}}%
\pgfpathlineto{\pgfqpoint{4.722861in}{2.908101in}}%
\pgfpathclose%
\pgfusepath{stroke,fill}%
\end{pgfscope}%
\begin{pgfscope}%
\pgfpathrectangle{\pgfqpoint{1.542338in}{0.880000in}}{\pgfqpoint{5.115323in}{6.160000in}}%
\pgfusepath{clip}%
\pgfsetbuttcap%
\pgfsetroundjoin%
\definecolor{currentfill}{rgb}{0.800000,0.200000,0.200000}%
\pgfsetfillcolor{currentfill}%
\pgfsetlinewidth{1.003750pt}%
\definecolor{currentstroke}{rgb}{0.800000,0.200000,0.200000}%
\pgfsetstrokecolor{currentstroke}%
\pgfsetdash{}{0pt}%
\pgfpathmoveto{\pgfqpoint{4.638282in}{2.809006in}}%
\pgfpathcurveto{\pgfqpoint{4.644106in}{2.809006in}}{\pgfqpoint{4.649692in}{2.811320in}}{\pgfqpoint{4.653810in}{2.815438in}}%
\pgfpathcurveto{\pgfqpoint{4.657929in}{2.819556in}}{\pgfqpoint{4.660242in}{2.825142in}}{\pgfqpoint{4.660242in}{2.830966in}}%
\pgfpathcurveto{\pgfqpoint{4.660242in}{2.836790in}}{\pgfqpoint{4.657929in}{2.842376in}}{\pgfqpoint{4.653810in}{2.846494in}}%
\pgfpathcurveto{\pgfqpoint{4.649692in}{2.850612in}}{\pgfqpoint{4.644106in}{2.852926in}}{\pgfqpoint{4.638282in}{2.852926in}}%
\pgfpathcurveto{\pgfqpoint{4.632458in}{2.852926in}}{\pgfqpoint{4.626872in}{2.850612in}}{\pgfqpoint{4.622754in}{2.846494in}}%
\pgfpathcurveto{\pgfqpoint{4.618636in}{2.842376in}}{\pgfqpoint{4.616322in}{2.836790in}}{\pgfqpoint{4.616322in}{2.830966in}}%
\pgfpathcurveto{\pgfqpoint{4.616322in}{2.825142in}}{\pgfqpoint{4.618636in}{2.819556in}}{\pgfqpoint{4.622754in}{2.815438in}}%
\pgfpathcurveto{\pgfqpoint{4.626872in}{2.811320in}}{\pgfqpoint{4.632458in}{2.809006in}}{\pgfqpoint{4.638282in}{2.809006in}}%
\pgfpathlineto{\pgfqpoint{4.638282in}{2.809006in}}%
\pgfpathclose%
\pgfusepath{stroke,fill}%
\end{pgfscope}%
\begin{pgfscope}%
\pgfpathrectangle{\pgfqpoint{1.542338in}{0.880000in}}{\pgfqpoint{5.115323in}{6.160000in}}%
\pgfusepath{clip}%
\pgfsetbuttcap%
\pgfsetroundjoin%
\definecolor{currentfill}{rgb}{0.800000,0.200000,0.200000}%
\pgfsetfillcolor{currentfill}%
\pgfsetlinewidth{1.003750pt}%
\definecolor{currentstroke}{rgb}{0.800000,0.200000,0.200000}%
\pgfsetstrokecolor{currentstroke}%
\pgfsetdash{}{0pt}%
\pgfpathmoveto{\pgfqpoint{4.547961in}{2.713598in}}%
\pgfpathcurveto{\pgfqpoint{4.553785in}{2.713598in}}{\pgfqpoint{4.559371in}{2.715912in}}{\pgfqpoint{4.563489in}{2.720030in}}%
\pgfpathcurveto{\pgfqpoint{4.567607in}{2.724148in}}{\pgfqpoint{4.569921in}{2.729734in}}{\pgfqpoint{4.569921in}{2.735558in}}%
\pgfpathcurveto{\pgfqpoint{4.569921in}{2.741382in}}{\pgfqpoint{4.567607in}{2.746968in}}{\pgfqpoint{4.563489in}{2.751086in}}%
\pgfpathcurveto{\pgfqpoint{4.559371in}{2.755204in}}{\pgfqpoint{4.553785in}{2.757518in}}{\pgfqpoint{4.547961in}{2.757518in}}%
\pgfpathcurveto{\pgfqpoint{4.542137in}{2.757518in}}{\pgfqpoint{4.536551in}{2.755204in}}{\pgfqpoint{4.532433in}{2.751086in}}%
\pgfpathcurveto{\pgfqpoint{4.528314in}{2.746968in}}{\pgfqpoint{4.526001in}{2.741382in}}{\pgfqpoint{4.526001in}{2.735558in}}%
\pgfpathcurveto{\pgfqpoint{4.526001in}{2.729734in}}{\pgfqpoint{4.528314in}{2.724148in}}{\pgfqpoint{4.532433in}{2.720030in}}%
\pgfpathcurveto{\pgfqpoint{4.536551in}{2.715912in}}{\pgfqpoint{4.542137in}{2.713598in}}{\pgfqpoint{4.547961in}{2.713598in}}%
\pgfpathlineto{\pgfqpoint{4.547961in}{2.713598in}}%
\pgfpathclose%
\pgfusepath{stroke,fill}%
\end{pgfscope}%
\begin{pgfscope}%
\pgfpathrectangle{\pgfqpoint{1.542338in}{0.880000in}}{\pgfqpoint{5.115323in}{6.160000in}}%
\pgfusepath{clip}%
\pgfsetbuttcap%
\pgfsetroundjoin%
\definecolor{currentfill}{rgb}{0.800000,0.200000,0.200000}%
\pgfsetfillcolor{currentfill}%
\pgfsetlinewidth{1.003750pt}%
\definecolor{currentstroke}{rgb}{0.800000,0.200000,0.200000}%
\pgfsetstrokecolor{currentstroke}%
\pgfsetdash{}{0pt}%
\pgfpathmoveto{\pgfqpoint{4.492381in}{2.594746in}}%
\pgfpathcurveto{\pgfqpoint{4.498205in}{2.594746in}}{\pgfqpoint{4.503792in}{2.597060in}}{\pgfqpoint{4.507910in}{2.601178in}}%
\pgfpathcurveto{\pgfqpoint{4.512028in}{2.605296in}}{\pgfqpoint{4.514342in}{2.610882in}}{\pgfqpoint{4.514342in}{2.616706in}}%
\pgfpathcurveto{\pgfqpoint{4.514342in}{2.622530in}}{\pgfqpoint{4.512028in}{2.628116in}}{\pgfqpoint{4.507910in}{2.632235in}}%
\pgfpathcurveto{\pgfqpoint{4.503792in}{2.636353in}}{\pgfqpoint{4.498205in}{2.638667in}}{\pgfqpoint{4.492381in}{2.638667in}}%
\pgfpathcurveto{\pgfqpoint{4.486557in}{2.638667in}}{\pgfqpoint{4.480971in}{2.636353in}}{\pgfqpoint{4.476853in}{2.632235in}}%
\pgfpathcurveto{\pgfqpoint{4.472735in}{2.628116in}}{\pgfqpoint{4.470421in}{2.622530in}}{\pgfqpoint{4.470421in}{2.616706in}}%
\pgfpathcurveto{\pgfqpoint{4.470421in}{2.610882in}}{\pgfqpoint{4.472735in}{2.605296in}}{\pgfqpoint{4.476853in}{2.601178in}}%
\pgfpathcurveto{\pgfqpoint{4.480971in}{2.597060in}}{\pgfqpoint{4.486557in}{2.594746in}}{\pgfqpoint{4.492381in}{2.594746in}}%
\pgfpathlineto{\pgfqpoint{4.492381in}{2.594746in}}%
\pgfpathclose%
\pgfusepath{stroke,fill}%
\end{pgfscope}%
\begin{pgfscope}%
\pgfpathrectangle{\pgfqpoint{1.542338in}{0.880000in}}{\pgfqpoint{5.115323in}{6.160000in}}%
\pgfusepath{clip}%
\pgfsetbuttcap%
\pgfsetroundjoin%
\definecolor{currentfill}{rgb}{0.800000,0.200000,0.200000}%
\pgfsetfillcolor{currentfill}%
\pgfsetlinewidth{1.003750pt}%
\definecolor{currentstroke}{rgb}{0.800000,0.200000,0.200000}%
\pgfsetstrokecolor{currentstroke}%
\pgfsetdash{}{0pt}%
\pgfpathmoveto{\pgfqpoint{4.444516in}{2.473735in}}%
\pgfpathcurveto{\pgfqpoint{4.450340in}{2.473735in}}{\pgfqpoint{4.455926in}{2.476049in}}{\pgfqpoint{4.460044in}{2.480167in}}%
\pgfpathcurveto{\pgfqpoint{4.464162in}{2.484285in}}{\pgfqpoint{4.466476in}{2.489872in}}{\pgfqpoint{4.466476in}{2.495695in}}%
\pgfpathcurveto{\pgfqpoint{4.466476in}{2.501519in}}{\pgfqpoint{4.464162in}{2.507106in}}{\pgfqpoint{4.460044in}{2.511224in}}%
\pgfpathcurveto{\pgfqpoint{4.455926in}{2.515342in}}{\pgfqpoint{4.450340in}{2.517656in}}{\pgfqpoint{4.444516in}{2.517656in}}%
\pgfpathcurveto{\pgfqpoint{4.438692in}{2.517656in}}{\pgfqpoint{4.433106in}{2.515342in}}{\pgfqpoint{4.428988in}{2.511224in}}%
\pgfpathcurveto{\pgfqpoint{4.424870in}{2.507106in}}{\pgfqpoint{4.422556in}{2.501519in}}{\pgfqpoint{4.422556in}{2.495695in}}%
\pgfpathcurveto{\pgfqpoint{4.422556in}{2.489872in}}{\pgfqpoint{4.424870in}{2.484285in}}{\pgfqpoint{4.428988in}{2.480167in}}%
\pgfpathcurveto{\pgfqpoint{4.433106in}{2.476049in}}{\pgfqpoint{4.438692in}{2.473735in}}{\pgfqpoint{4.444516in}{2.473735in}}%
\pgfpathlineto{\pgfqpoint{4.444516in}{2.473735in}}%
\pgfpathclose%
\pgfusepath{stroke,fill}%
\end{pgfscope}%
\begin{pgfscope}%
\pgfpathrectangle{\pgfqpoint{1.542338in}{0.880000in}}{\pgfqpoint{5.115323in}{6.160000in}}%
\pgfusepath{clip}%
\pgfsetbuttcap%
\pgfsetroundjoin%
\definecolor{currentfill}{rgb}{0.800000,0.200000,0.200000}%
\pgfsetfillcolor{currentfill}%
\pgfsetlinewidth{1.003750pt}%
\definecolor{currentstroke}{rgb}{0.800000,0.200000,0.200000}%
\pgfsetstrokecolor{currentstroke}%
\pgfsetdash{}{0pt}%
\pgfpathmoveto{\pgfqpoint{4.410649in}{2.348097in}}%
\pgfpathcurveto{\pgfqpoint{4.416473in}{2.348097in}}{\pgfqpoint{4.422059in}{2.350411in}}{\pgfqpoint{4.426177in}{2.354529in}}%
\pgfpathcurveto{\pgfqpoint{4.430295in}{2.358648in}}{\pgfqpoint{4.432609in}{2.364234in}}{\pgfqpoint{4.432609in}{2.370058in}}%
\pgfpathcurveto{\pgfqpoint{4.432609in}{2.375882in}}{\pgfqpoint{4.430295in}{2.381468in}}{\pgfqpoint{4.426177in}{2.385586in}}%
\pgfpathcurveto{\pgfqpoint{4.422059in}{2.389704in}}{\pgfqpoint{4.416473in}{2.392018in}}{\pgfqpoint{4.410649in}{2.392018in}}%
\pgfpathcurveto{\pgfqpoint{4.404825in}{2.392018in}}{\pgfqpoint{4.399238in}{2.389704in}}{\pgfqpoint{4.395120in}{2.385586in}}%
\pgfpathcurveto{\pgfqpoint{4.391002in}{2.381468in}}{\pgfqpoint{4.388688in}{2.375882in}}{\pgfqpoint{4.388688in}{2.370058in}}%
\pgfpathcurveto{\pgfqpoint{4.388688in}{2.364234in}}{\pgfqpoint{4.391002in}{2.358648in}}{\pgfqpoint{4.395120in}{2.354529in}}%
\pgfpathcurveto{\pgfqpoint{4.399238in}{2.350411in}}{\pgfqpoint{4.404825in}{2.348097in}}{\pgfqpoint{4.410649in}{2.348097in}}%
\pgfpathlineto{\pgfqpoint{4.410649in}{2.348097in}}%
\pgfpathclose%
\pgfusepath{stroke,fill}%
\end{pgfscope}%
\begin{pgfscope}%
\pgfpathrectangle{\pgfqpoint{1.542338in}{0.880000in}}{\pgfqpoint{5.115323in}{6.160000in}}%
\pgfusepath{clip}%
\pgfsetbuttcap%
\pgfsetroundjoin%
\definecolor{currentfill}{rgb}{0.800000,0.200000,0.200000}%
\pgfsetfillcolor{currentfill}%
\pgfsetlinewidth{1.003750pt}%
\definecolor{currentstroke}{rgb}{0.800000,0.200000,0.200000}%
\pgfsetstrokecolor{currentstroke}%
\pgfsetdash{}{0pt}%
\pgfpathmoveto{\pgfqpoint{4.396738in}{2.218814in}}%
\pgfpathcurveto{\pgfqpoint{4.402562in}{2.218814in}}{\pgfqpoint{4.408148in}{2.221128in}}{\pgfqpoint{4.412267in}{2.225246in}}%
\pgfpathcurveto{\pgfqpoint{4.416385in}{2.229364in}}{\pgfqpoint{4.418699in}{2.234950in}}{\pgfqpoint{4.418699in}{2.240774in}}%
\pgfpathcurveto{\pgfqpoint{4.418699in}{2.246598in}}{\pgfqpoint{4.416385in}{2.252184in}}{\pgfqpoint{4.412267in}{2.256302in}}%
\pgfpathcurveto{\pgfqpoint{4.408148in}{2.260420in}}{\pgfqpoint{4.402562in}{2.262734in}}{\pgfqpoint{4.396738in}{2.262734in}}%
\pgfpathcurveto{\pgfqpoint{4.390914in}{2.262734in}}{\pgfqpoint{4.385328in}{2.260420in}}{\pgfqpoint{4.381210in}{2.256302in}}%
\pgfpathcurveto{\pgfqpoint{4.377092in}{2.252184in}}{\pgfqpoint{4.374778in}{2.246598in}}{\pgfqpoint{4.374778in}{2.240774in}}%
\pgfpathcurveto{\pgfqpoint{4.374778in}{2.234950in}}{\pgfqpoint{4.377092in}{2.229364in}}{\pgfqpoint{4.381210in}{2.225246in}}%
\pgfpathcurveto{\pgfqpoint{4.385328in}{2.221128in}}{\pgfqpoint{4.390914in}{2.218814in}}{\pgfqpoint{4.396738in}{2.218814in}}%
\pgfpathlineto{\pgfqpoint{4.396738in}{2.218814in}}%
\pgfpathclose%
\pgfusepath{stroke,fill}%
\end{pgfscope}%
\begin{pgfscope}%
\pgfpathrectangle{\pgfqpoint{1.542338in}{0.880000in}}{\pgfqpoint{5.115323in}{6.160000in}}%
\pgfusepath{clip}%
\pgfsetbuttcap%
\pgfsetroundjoin%
\definecolor{currentfill}{rgb}{0.800000,0.200000,0.200000}%
\pgfsetfillcolor{currentfill}%
\pgfsetlinewidth{1.003750pt}%
\definecolor{currentstroke}{rgb}{0.800000,0.200000,0.200000}%
\pgfsetstrokecolor{currentstroke}%
\pgfsetdash{}{0pt}%
\pgfpathmoveto{\pgfqpoint{4.397619in}{2.089045in}}%
\pgfpathcurveto{\pgfqpoint{4.403443in}{2.089045in}}{\pgfqpoint{4.409029in}{2.091359in}}{\pgfqpoint{4.413147in}{2.095477in}}%
\pgfpathcurveto{\pgfqpoint{4.417265in}{2.099595in}}{\pgfqpoint{4.419579in}{2.105181in}}{\pgfqpoint{4.419579in}{2.111005in}}%
\pgfpathcurveto{\pgfqpoint{4.419579in}{2.116829in}}{\pgfqpoint{4.417265in}{2.122415in}}{\pgfqpoint{4.413147in}{2.126534in}}%
\pgfpathcurveto{\pgfqpoint{4.409029in}{2.130652in}}{\pgfqpoint{4.403443in}{2.132966in}}{\pgfqpoint{4.397619in}{2.132966in}}%
\pgfpathcurveto{\pgfqpoint{4.391795in}{2.132966in}}{\pgfqpoint{4.386208in}{2.130652in}}{\pgfqpoint{4.382090in}{2.126534in}}%
\pgfpathcurveto{\pgfqpoint{4.377972in}{2.122415in}}{\pgfqpoint{4.375658in}{2.116829in}}{\pgfqpoint{4.375658in}{2.111005in}}%
\pgfpathcurveto{\pgfqpoint{4.375658in}{2.105181in}}{\pgfqpoint{4.377972in}{2.099595in}}{\pgfqpoint{4.382090in}{2.095477in}}%
\pgfpathcurveto{\pgfqpoint{4.386208in}{2.091359in}}{\pgfqpoint{4.391795in}{2.089045in}}{\pgfqpoint{4.397619in}{2.089045in}}%
\pgfpathlineto{\pgfqpoint{4.397619in}{2.089045in}}%
\pgfpathclose%
\pgfusepath{stroke,fill}%
\end{pgfscope}%
\begin{pgfscope}%
\pgfpathrectangle{\pgfqpoint{1.542338in}{0.880000in}}{\pgfqpoint{5.115323in}{6.160000in}}%
\pgfusepath{clip}%
\pgfsetbuttcap%
\pgfsetroundjoin%
\definecolor{currentfill}{rgb}{0.800000,0.200000,0.200000}%
\pgfsetfillcolor{currentfill}%
\pgfsetlinewidth{1.003750pt}%
\definecolor{currentstroke}{rgb}{0.800000,0.200000,0.200000}%
\pgfsetstrokecolor{currentstroke}%
\pgfsetdash{}{0pt}%
\pgfpathmoveto{\pgfqpoint{4.409440in}{1.959469in}}%
\pgfpathcurveto{\pgfqpoint{4.415264in}{1.959469in}}{\pgfqpoint{4.420850in}{1.961783in}}{\pgfqpoint{4.424968in}{1.965901in}}%
\pgfpathcurveto{\pgfqpoint{4.429086in}{1.970020in}}{\pgfqpoint{4.431400in}{1.975606in}}{\pgfqpoint{4.431400in}{1.981430in}}%
\pgfpathcurveto{\pgfqpoint{4.431400in}{1.987254in}}{\pgfqpoint{4.429086in}{1.992840in}}{\pgfqpoint{4.424968in}{1.996958in}}%
\pgfpathcurveto{\pgfqpoint{4.420850in}{2.001076in}}{\pgfqpoint{4.415264in}{2.003390in}}{\pgfqpoint{4.409440in}{2.003390in}}%
\pgfpathcurveto{\pgfqpoint{4.403616in}{2.003390in}}{\pgfqpoint{4.398030in}{2.001076in}}{\pgfqpoint{4.393911in}{1.996958in}}%
\pgfpathcurveto{\pgfqpoint{4.389793in}{1.992840in}}{\pgfqpoint{4.387479in}{1.987254in}}{\pgfqpoint{4.387479in}{1.981430in}}%
\pgfpathcurveto{\pgfqpoint{4.387479in}{1.975606in}}{\pgfqpoint{4.389793in}{1.970020in}}{\pgfqpoint{4.393911in}{1.965901in}}%
\pgfpathcurveto{\pgfqpoint{4.398030in}{1.961783in}}{\pgfqpoint{4.403616in}{1.959469in}}{\pgfqpoint{4.409440in}{1.959469in}}%
\pgfpathlineto{\pgfqpoint{4.409440in}{1.959469in}}%
\pgfpathclose%
\pgfusepath{stroke,fill}%
\end{pgfscope}%
\begin{pgfscope}%
\pgfpathrectangle{\pgfqpoint{1.542338in}{0.880000in}}{\pgfqpoint{5.115323in}{6.160000in}}%
\pgfusepath{clip}%
\pgfsetbuttcap%
\pgfsetroundjoin%
\definecolor{currentfill}{rgb}{0.800000,0.200000,0.200000}%
\pgfsetfillcolor{currentfill}%
\pgfsetlinewidth{1.003750pt}%
\definecolor{currentstroke}{rgb}{0.800000,0.200000,0.200000}%
\pgfsetstrokecolor{currentstroke}%
\pgfsetdash{}{0pt}%
\pgfpathmoveto{\pgfqpoint{4.442221in}{1.833305in}}%
\pgfpathcurveto{\pgfqpoint{4.448045in}{1.833305in}}{\pgfqpoint{4.453631in}{1.835619in}}{\pgfqpoint{4.457749in}{1.839737in}}%
\pgfpathcurveto{\pgfqpoint{4.461868in}{1.843855in}}{\pgfqpoint{4.464181in}{1.849442in}}{\pgfqpoint{4.464181in}{1.855265in}}%
\pgfpathcurveto{\pgfqpoint{4.464181in}{1.861089in}}{\pgfqpoint{4.461868in}{1.866676in}}{\pgfqpoint{4.457749in}{1.870794in}}%
\pgfpathcurveto{\pgfqpoint{4.453631in}{1.874912in}}{\pgfqpoint{4.448045in}{1.877226in}}{\pgfqpoint{4.442221in}{1.877226in}}%
\pgfpathcurveto{\pgfqpoint{4.436397in}{1.877226in}}{\pgfqpoint{4.430811in}{1.874912in}}{\pgfqpoint{4.426693in}{1.870794in}}%
\pgfpathcurveto{\pgfqpoint{4.422575in}{1.866676in}}{\pgfqpoint{4.420261in}{1.861089in}}{\pgfqpoint{4.420261in}{1.855265in}}%
\pgfpathcurveto{\pgfqpoint{4.420261in}{1.849442in}}{\pgfqpoint{4.422575in}{1.843855in}}{\pgfqpoint{4.426693in}{1.839737in}}%
\pgfpathcurveto{\pgfqpoint{4.430811in}{1.835619in}}{\pgfqpoint{4.436397in}{1.833305in}}{\pgfqpoint{4.442221in}{1.833305in}}%
\pgfpathlineto{\pgfqpoint{4.442221in}{1.833305in}}%
\pgfpathclose%
\pgfusepath{stroke,fill}%
\end{pgfscope}%
\begin{pgfscope}%
\pgfpathrectangle{\pgfqpoint{1.542338in}{0.880000in}}{\pgfqpoint{5.115323in}{6.160000in}}%
\pgfusepath{clip}%
\pgfsetbuttcap%
\pgfsetroundjoin%
\definecolor{currentfill}{rgb}{0.800000,0.200000,0.200000}%
\pgfsetfillcolor{currentfill}%
\pgfsetlinewidth{1.003750pt}%
\definecolor{currentstroke}{rgb}{0.800000,0.200000,0.200000}%
\pgfsetstrokecolor{currentstroke}%
\pgfsetdash{}{0pt}%
\pgfpathmoveto{\pgfqpoint{4.490749in}{1.712270in}}%
\pgfpathcurveto{\pgfqpoint{4.496573in}{1.712270in}}{\pgfqpoint{4.502159in}{1.714584in}}{\pgfqpoint{4.506277in}{1.718702in}}%
\pgfpathcurveto{\pgfqpoint{4.510395in}{1.722820in}}{\pgfqpoint{4.512709in}{1.728406in}}{\pgfqpoint{4.512709in}{1.734230in}}%
\pgfpathcurveto{\pgfqpoint{4.512709in}{1.740054in}}{\pgfqpoint{4.510395in}{1.745640in}}{\pgfqpoint{4.506277in}{1.749758in}}%
\pgfpathcurveto{\pgfqpoint{4.502159in}{1.753877in}}{\pgfqpoint{4.496573in}{1.756190in}}{\pgfqpoint{4.490749in}{1.756190in}}%
\pgfpathcurveto{\pgfqpoint{4.484925in}{1.756190in}}{\pgfqpoint{4.479338in}{1.753877in}}{\pgfqpoint{4.475220in}{1.749758in}}%
\pgfpathcurveto{\pgfqpoint{4.471102in}{1.745640in}}{\pgfqpoint{4.468788in}{1.740054in}}{\pgfqpoint{4.468788in}{1.734230in}}%
\pgfpathcurveto{\pgfqpoint{4.468788in}{1.728406in}}{\pgfqpoint{4.471102in}{1.722820in}}{\pgfqpoint{4.475220in}{1.718702in}}%
\pgfpathcurveto{\pgfqpoint{4.479338in}{1.714584in}}{\pgfqpoint{4.484925in}{1.712270in}}{\pgfqpoint{4.490749in}{1.712270in}}%
\pgfpathlineto{\pgfqpoint{4.490749in}{1.712270in}}%
\pgfpathclose%
\pgfusepath{stroke,fill}%
\end{pgfscope}%
\begin{pgfscope}%
\pgfpathrectangle{\pgfqpoint{1.542338in}{0.880000in}}{\pgfqpoint{5.115323in}{6.160000in}}%
\pgfusepath{clip}%
\pgfsetbuttcap%
\pgfsetroundjoin%
\definecolor{currentfill}{rgb}{0.800000,0.200000,0.200000}%
\pgfsetfillcolor{currentfill}%
\pgfsetlinewidth{1.003750pt}%
\definecolor{currentstroke}{rgb}{0.800000,0.200000,0.200000}%
\pgfsetstrokecolor{currentstroke}%
\pgfsetdash{}{0pt}%
\pgfpathmoveto{\pgfqpoint{4.555620in}{1.599190in}}%
\pgfpathcurveto{\pgfqpoint{4.561443in}{1.599190in}}{\pgfqpoint{4.567030in}{1.601504in}}{\pgfqpoint{4.571148in}{1.605622in}}%
\pgfpathcurveto{\pgfqpoint{4.575266in}{1.609740in}}{\pgfqpoint{4.577580in}{1.615326in}}{\pgfqpoint{4.577580in}{1.621150in}}%
\pgfpathcurveto{\pgfqpoint{4.577580in}{1.626974in}}{\pgfqpoint{4.575266in}{1.632560in}}{\pgfqpoint{4.571148in}{1.636678in}}%
\pgfpathcurveto{\pgfqpoint{4.567030in}{1.640796in}}{\pgfqpoint{4.561443in}{1.643110in}}{\pgfqpoint{4.555620in}{1.643110in}}%
\pgfpathcurveto{\pgfqpoint{4.549796in}{1.643110in}}{\pgfqpoint{4.544209in}{1.640796in}}{\pgfqpoint{4.540091in}{1.636678in}}%
\pgfpathcurveto{\pgfqpoint{4.535973in}{1.632560in}}{\pgfqpoint{4.533659in}{1.626974in}}{\pgfqpoint{4.533659in}{1.621150in}}%
\pgfpathcurveto{\pgfqpoint{4.533659in}{1.615326in}}{\pgfqpoint{4.535973in}{1.609740in}}{\pgfqpoint{4.540091in}{1.605622in}}%
\pgfpathcurveto{\pgfqpoint{4.544209in}{1.601504in}}{\pgfqpoint{4.549796in}{1.599190in}}{\pgfqpoint{4.555620in}{1.599190in}}%
\pgfpathlineto{\pgfqpoint{4.555620in}{1.599190in}}%
\pgfpathclose%
\pgfusepath{stroke,fill}%
\end{pgfscope}%
\begin{pgfscope}%
\pgfpathrectangle{\pgfqpoint{1.542338in}{0.880000in}}{\pgfqpoint{5.115323in}{6.160000in}}%
\pgfusepath{clip}%
\pgfsetbuttcap%
\pgfsetroundjoin%
\definecolor{currentfill}{rgb}{0.800000,0.200000,0.200000}%
\pgfsetfillcolor{currentfill}%
\pgfsetlinewidth{1.003750pt}%
\definecolor{currentstroke}{rgb}{0.800000,0.200000,0.200000}%
\pgfsetstrokecolor{currentstroke}%
\pgfsetdash{}{0pt}%
\pgfpathmoveto{\pgfqpoint{4.633701in}{1.494897in}}%
\pgfpathcurveto{\pgfqpoint{4.639525in}{1.494897in}}{\pgfqpoint{4.645111in}{1.497211in}}{\pgfqpoint{4.649229in}{1.501329in}}%
\pgfpathcurveto{\pgfqpoint{4.653348in}{1.505447in}}{\pgfqpoint{4.655661in}{1.511033in}}{\pgfqpoint{4.655661in}{1.516857in}}%
\pgfpathcurveto{\pgfqpoint{4.655661in}{1.522681in}}{\pgfqpoint{4.653348in}{1.528267in}}{\pgfqpoint{4.649229in}{1.532385in}}%
\pgfpathcurveto{\pgfqpoint{4.645111in}{1.536503in}}{\pgfqpoint{4.639525in}{1.538817in}}{\pgfqpoint{4.633701in}{1.538817in}}%
\pgfpathcurveto{\pgfqpoint{4.627877in}{1.538817in}}{\pgfqpoint{4.622291in}{1.536503in}}{\pgfqpoint{4.618173in}{1.532385in}}%
\pgfpathcurveto{\pgfqpoint{4.614055in}{1.528267in}}{\pgfqpoint{4.611741in}{1.522681in}}{\pgfqpoint{4.611741in}{1.516857in}}%
\pgfpathcurveto{\pgfqpoint{4.611741in}{1.511033in}}{\pgfqpoint{4.614055in}{1.505447in}}{\pgfqpoint{4.618173in}{1.501329in}}%
\pgfpathcurveto{\pgfqpoint{4.622291in}{1.497211in}}{\pgfqpoint{4.627877in}{1.494897in}}{\pgfqpoint{4.633701in}{1.494897in}}%
\pgfpathlineto{\pgfqpoint{4.633701in}{1.494897in}}%
\pgfpathclose%
\pgfusepath{stroke,fill}%
\end{pgfscope}%
\begin{pgfscope}%
\pgfpathrectangle{\pgfqpoint{1.542338in}{0.880000in}}{\pgfqpoint{5.115323in}{6.160000in}}%
\pgfusepath{clip}%
\pgfsetbuttcap%
\pgfsetroundjoin%
\definecolor{currentfill}{rgb}{0.800000,0.200000,0.200000}%
\pgfsetfillcolor{currentfill}%
\pgfsetlinewidth{1.003750pt}%
\definecolor{currentstroke}{rgb}{0.800000,0.200000,0.200000}%
\pgfsetstrokecolor{currentstroke}%
\pgfsetdash{}{0pt}%
\pgfpathmoveto{\pgfqpoint{4.727351in}{1.404645in}}%
\pgfpathcurveto{\pgfqpoint{4.733175in}{1.404645in}}{\pgfqpoint{4.738761in}{1.406959in}}{\pgfqpoint{4.742879in}{1.411077in}}%
\pgfpathcurveto{\pgfqpoint{4.746997in}{1.415195in}}{\pgfqpoint{4.749311in}{1.420782in}}{\pgfqpoint{4.749311in}{1.426606in}}%
\pgfpathcurveto{\pgfqpoint{4.749311in}{1.432429in}}{\pgfqpoint{4.746997in}{1.438016in}}{\pgfqpoint{4.742879in}{1.442134in}}%
\pgfpathcurveto{\pgfqpoint{4.738761in}{1.446252in}}{\pgfqpoint{4.733175in}{1.448566in}}{\pgfqpoint{4.727351in}{1.448566in}}%
\pgfpathcurveto{\pgfqpoint{4.721527in}{1.448566in}}{\pgfqpoint{4.715941in}{1.446252in}}{\pgfqpoint{4.711823in}{1.442134in}}%
\pgfpathcurveto{\pgfqpoint{4.707704in}{1.438016in}}{\pgfqpoint{4.705391in}{1.432429in}}{\pgfqpoint{4.705391in}{1.426606in}}%
\pgfpathcurveto{\pgfqpoint{4.705391in}{1.420782in}}{\pgfqpoint{4.707704in}{1.415195in}}{\pgfqpoint{4.711823in}{1.411077in}}%
\pgfpathcurveto{\pgfqpoint{4.715941in}{1.406959in}}{\pgfqpoint{4.721527in}{1.404645in}}{\pgfqpoint{4.727351in}{1.404645in}}%
\pgfpathlineto{\pgfqpoint{4.727351in}{1.404645in}}%
\pgfpathclose%
\pgfusepath{stroke,fill}%
\end{pgfscope}%
\begin{pgfscope}%
\pgfpathrectangle{\pgfqpoint{1.542338in}{0.880000in}}{\pgfqpoint{5.115323in}{6.160000in}}%
\pgfusepath{clip}%
\pgfsetbuttcap%
\pgfsetroundjoin%
\definecolor{currentfill}{rgb}{0.800000,0.200000,0.200000}%
\pgfsetfillcolor{currentfill}%
\pgfsetlinewidth{1.003750pt}%
\definecolor{currentstroke}{rgb}{0.800000,0.200000,0.200000}%
\pgfsetstrokecolor{currentstroke}%
\pgfsetdash{}{0pt}%
\pgfpathmoveto{\pgfqpoint{4.827106in}{1.321426in}}%
\pgfpathcurveto{\pgfqpoint{4.832930in}{1.321426in}}{\pgfqpoint{4.838516in}{1.323739in}}{\pgfqpoint{4.842635in}{1.327858in}}%
\pgfpathcurveto{\pgfqpoint{4.846753in}{1.331976in}}{\pgfqpoint{4.849067in}{1.337562in}}{\pgfqpoint{4.849067in}{1.343386in}}%
\pgfpathcurveto{\pgfqpoint{4.849067in}{1.349210in}}{\pgfqpoint{4.846753in}{1.354796in}}{\pgfqpoint{4.842635in}{1.358914in}}%
\pgfpathcurveto{\pgfqpoint{4.838516in}{1.363032in}}{\pgfqpoint{4.832930in}{1.365346in}}{\pgfqpoint{4.827106in}{1.365346in}}%
\pgfpathcurveto{\pgfqpoint{4.821282in}{1.365346in}}{\pgfqpoint{4.815696in}{1.363032in}}{\pgfqpoint{4.811578in}{1.358914in}}%
\pgfpathcurveto{\pgfqpoint{4.807460in}{1.354796in}}{\pgfqpoint{4.805146in}{1.349210in}}{\pgfqpoint{4.805146in}{1.343386in}}%
\pgfpathcurveto{\pgfqpoint{4.805146in}{1.337562in}}{\pgfqpoint{4.807460in}{1.331976in}}{\pgfqpoint{4.811578in}{1.327858in}}%
\pgfpathcurveto{\pgfqpoint{4.815696in}{1.323739in}}{\pgfqpoint{4.821282in}{1.321426in}}{\pgfqpoint{4.827106in}{1.321426in}}%
\pgfpathlineto{\pgfqpoint{4.827106in}{1.321426in}}%
\pgfpathclose%
\pgfusepath{stroke,fill}%
\end{pgfscope}%
\begin{pgfscope}%
\pgfpathrectangle{\pgfqpoint{1.542338in}{0.880000in}}{\pgfqpoint{5.115323in}{6.160000in}}%
\pgfusepath{clip}%
\pgfsetbuttcap%
\pgfsetroundjoin%
\definecolor{currentfill}{rgb}{0.800000,0.200000,0.200000}%
\pgfsetfillcolor{currentfill}%
\pgfsetlinewidth{1.003750pt}%
\definecolor{currentstroke}{rgb}{0.800000,0.200000,0.200000}%
\pgfsetstrokecolor{currentstroke}%
\pgfsetdash{}{0pt}%
\pgfpathmoveto{\pgfqpoint{4.937124in}{1.251693in}}%
\pgfpathcurveto{\pgfqpoint{4.942948in}{1.251693in}}{\pgfqpoint{4.948534in}{1.254007in}}{\pgfqpoint{4.952652in}{1.258125in}}%
\pgfpathcurveto{\pgfqpoint{4.956770in}{1.262244in}}{\pgfqpoint{4.959084in}{1.267830in}}{\pgfqpoint{4.959084in}{1.273654in}}%
\pgfpathcurveto{\pgfqpoint{4.959084in}{1.279478in}}{\pgfqpoint{4.956770in}{1.285064in}}{\pgfqpoint{4.952652in}{1.289182in}}%
\pgfpathcurveto{\pgfqpoint{4.948534in}{1.293300in}}{\pgfqpoint{4.942948in}{1.295614in}}{\pgfqpoint{4.937124in}{1.295614in}}%
\pgfpathcurveto{\pgfqpoint{4.931300in}{1.295614in}}{\pgfqpoint{4.925714in}{1.293300in}}{\pgfqpoint{4.921596in}{1.289182in}}%
\pgfpathcurveto{\pgfqpoint{4.917478in}{1.285064in}}{\pgfqpoint{4.915164in}{1.279478in}}{\pgfqpoint{4.915164in}{1.273654in}}%
\pgfpathcurveto{\pgfqpoint{4.915164in}{1.267830in}}{\pgfqpoint{4.917478in}{1.262244in}}{\pgfqpoint{4.921596in}{1.258125in}}%
\pgfpathcurveto{\pgfqpoint{4.925714in}{1.254007in}}{\pgfqpoint{4.931300in}{1.251693in}}{\pgfqpoint{4.937124in}{1.251693in}}%
\pgfpathlineto{\pgfqpoint{4.937124in}{1.251693in}}%
\pgfpathclose%
\pgfusepath{stroke,fill}%
\end{pgfscope}%
\begin{pgfscope}%
\pgfpathrectangle{\pgfqpoint{1.542338in}{0.880000in}}{\pgfqpoint{5.115323in}{6.160000in}}%
\pgfusepath{clip}%
\pgfsetbuttcap%
\pgfsetroundjoin%
\definecolor{currentfill}{rgb}{0.800000,0.200000,0.200000}%
\pgfsetfillcolor{currentfill}%
\pgfsetlinewidth{1.003750pt}%
\definecolor{currentstroke}{rgb}{0.800000,0.200000,0.200000}%
\pgfsetstrokecolor{currentstroke}%
\pgfsetdash{}{0pt}%
\pgfpathmoveto{\pgfqpoint{5.057331in}{1.201548in}}%
\pgfpathcurveto{\pgfqpoint{5.063155in}{1.201548in}}{\pgfqpoint{5.068741in}{1.203862in}}{\pgfqpoint{5.072859in}{1.207980in}}%
\pgfpathcurveto{\pgfqpoint{5.076977in}{1.212099in}}{\pgfqpoint{5.079291in}{1.217685in}}{\pgfqpoint{5.079291in}{1.223509in}}%
\pgfpathcurveto{\pgfqpoint{5.079291in}{1.229333in}}{\pgfqpoint{5.076977in}{1.234919in}}{\pgfqpoint{5.072859in}{1.239037in}}%
\pgfpathcurveto{\pgfqpoint{5.068741in}{1.243155in}}{\pgfqpoint{5.063155in}{1.245469in}}{\pgfqpoint{5.057331in}{1.245469in}}%
\pgfpathcurveto{\pgfqpoint{5.051507in}{1.245469in}}{\pgfqpoint{5.045921in}{1.243155in}}{\pgfqpoint{5.041802in}{1.239037in}}%
\pgfpathcurveto{\pgfqpoint{5.037684in}{1.234919in}}{\pgfqpoint{5.035370in}{1.229333in}}{\pgfqpoint{5.035370in}{1.223509in}}%
\pgfpathcurveto{\pgfqpoint{5.035370in}{1.217685in}}{\pgfqpoint{5.037684in}{1.212099in}}{\pgfqpoint{5.041802in}{1.207980in}}%
\pgfpathcurveto{\pgfqpoint{5.045921in}{1.203862in}}{\pgfqpoint{5.051507in}{1.201548in}}{\pgfqpoint{5.057331in}{1.201548in}}%
\pgfpathlineto{\pgfqpoint{5.057331in}{1.201548in}}%
\pgfpathclose%
\pgfusepath{stroke,fill}%
\end{pgfscope}%
\begin{pgfscope}%
\pgfpathrectangle{\pgfqpoint{1.542338in}{0.880000in}}{\pgfqpoint{5.115323in}{6.160000in}}%
\pgfusepath{clip}%
\pgfsetbuttcap%
\pgfsetroundjoin%
\definecolor{currentfill}{rgb}{0.800000,0.200000,0.200000}%
\pgfsetfillcolor{currentfill}%
\pgfsetlinewidth{1.003750pt}%
\definecolor{currentstroke}{rgb}{0.800000,0.200000,0.200000}%
\pgfsetstrokecolor{currentstroke}%
\pgfsetdash{}{0pt}%
\pgfpathmoveto{\pgfqpoint{5.181250in}{1.161298in}}%
\pgfpathcurveto{\pgfqpoint{5.187074in}{1.161298in}}{\pgfqpoint{5.192660in}{1.163611in}}{\pgfqpoint{5.196778in}{1.167730in}}%
\pgfpathcurveto{\pgfqpoint{5.200896in}{1.171848in}}{\pgfqpoint{5.203210in}{1.177434in}}{\pgfqpoint{5.203210in}{1.183258in}}%
\pgfpathcurveto{\pgfqpoint{5.203210in}{1.189082in}}{\pgfqpoint{5.200896in}{1.194668in}}{\pgfqpoint{5.196778in}{1.198786in}}%
\pgfpathcurveto{\pgfqpoint{5.192660in}{1.202904in}}{\pgfqpoint{5.187074in}{1.205218in}}{\pgfqpoint{5.181250in}{1.205218in}}%
\pgfpathcurveto{\pgfqpoint{5.175426in}{1.205218in}}{\pgfqpoint{5.169840in}{1.202904in}}{\pgfqpoint{5.165722in}{1.198786in}}%
\pgfpathcurveto{\pgfqpoint{5.161604in}{1.194668in}}{\pgfqpoint{5.159290in}{1.189082in}}{\pgfqpoint{5.159290in}{1.183258in}}%
\pgfpathcurveto{\pgfqpoint{5.159290in}{1.177434in}}{\pgfqpoint{5.161604in}{1.171848in}}{\pgfqpoint{5.165722in}{1.167730in}}%
\pgfpathcurveto{\pgfqpoint{5.169840in}{1.163611in}}{\pgfqpoint{5.175426in}{1.161298in}}{\pgfqpoint{5.181250in}{1.161298in}}%
\pgfpathlineto{\pgfqpoint{5.181250in}{1.161298in}}%
\pgfpathclose%
\pgfusepath{stroke,fill}%
\end{pgfscope}%
\begin{pgfscope}%
\pgfpathrectangle{\pgfqpoint{1.542338in}{0.880000in}}{\pgfqpoint{5.115323in}{6.160000in}}%
\pgfusepath{clip}%
\pgfsetbuttcap%
\pgfsetroundjoin%
\definecolor{currentfill}{rgb}{0.800000,0.200000,0.200000}%
\pgfsetfillcolor{currentfill}%
\pgfsetlinewidth{1.003750pt}%
\definecolor{currentstroke}{rgb}{0.800000,0.200000,0.200000}%
\pgfsetstrokecolor{currentstroke}%
\pgfsetdash{}{0pt}%
\pgfpathmoveto{\pgfqpoint{5.310206in}{1.142183in}}%
\pgfpathcurveto{\pgfqpoint{5.316030in}{1.142183in}}{\pgfqpoint{5.321616in}{1.144497in}}{\pgfqpoint{5.325734in}{1.148615in}}%
\pgfpathcurveto{\pgfqpoint{5.329852in}{1.152733in}}{\pgfqpoint{5.332166in}{1.158320in}}{\pgfqpoint{5.332166in}{1.164144in}}%
\pgfpathcurveto{\pgfqpoint{5.332166in}{1.169968in}}{\pgfqpoint{5.329852in}{1.175554in}}{\pgfqpoint{5.325734in}{1.179672in}}%
\pgfpathcurveto{\pgfqpoint{5.321616in}{1.183790in}}{\pgfqpoint{5.316030in}{1.186104in}}{\pgfqpoint{5.310206in}{1.186104in}}%
\pgfpathcurveto{\pgfqpoint{5.304382in}{1.186104in}}{\pgfqpoint{5.298796in}{1.183790in}}{\pgfqpoint{5.294678in}{1.179672in}}%
\pgfpathcurveto{\pgfqpoint{5.290560in}{1.175554in}}{\pgfqpoint{5.288246in}{1.169968in}}{\pgfqpoint{5.288246in}{1.164144in}}%
\pgfpathcurveto{\pgfqpoint{5.288246in}{1.158320in}}{\pgfqpoint{5.290560in}{1.152733in}}{\pgfqpoint{5.294678in}{1.148615in}}%
\pgfpathcurveto{\pgfqpoint{5.298796in}{1.144497in}}{\pgfqpoint{5.304382in}{1.142183in}}{\pgfqpoint{5.310206in}{1.142183in}}%
\pgfpathlineto{\pgfqpoint{5.310206in}{1.142183in}}%
\pgfpathclose%
\pgfusepath{stroke,fill}%
\end{pgfscope}%
\begin{pgfscope}%
\pgfpathrectangle{\pgfqpoint{1.542338in}{0.880000in}}{\pgfqpoint{5.115323in}{6.160000in}}%
\pgfusepath{clip}%
\pgfsetbuttcap%
\pgfsetroundjoin%
\definecolor{currentfill}{rgb}{0.800000,0.200000,0.200000}%
\pgfsetfillcolor{currentfill}%
\pgfsetlinewidth{1.003750pt}%
\definecolor{currentstroke}{rgb}{0.800000,0.200000,0.200000}%
\pgfsetstrokecolor{currentstroke}%
\pgfsetdash{}{0pt}%
\pgfpathmoveto{\pgfqpoint{5.440382in}{1.138040in}}%
\pgfpathcurveto{\pgfqpoint{5.446206in}{1.138040in}}{\pgfqpoint{5.451792in}{1.140354in}}{\pgfqpoint{5.455910in}{1.144472in}}%
\pgfpathcurveto{\pgfqpoint{5.460028in}{1.148590in}}{\pgfqpoint{5.462342in}{1.154176in}}{\pgfqpoint{5.462342in}{1.160000in}}%
\pgfpathcurveto{\pgfqpoint{5.462342in}{1.165824in}}{\pgfqpoint{5.460028in}{1.171410in}}{\pgfqpoint{5.455910in}{1.175528in}}%
\pgfpathcurveto{\pgfqpoint{5.451792in}{1.179646in}}{\pgfqpoint{5.446206in}{1.181960in}}{\pgfqpoint{5.440382in}{1.181960in}}%
\pgfpathcurveto{\pgfqpoint{5.434558in}{1.181960in}}{\pgfqpoint{5.428972in}{1.179646in}}{\pgfqpoint{5.424854in}{1.175528in}}%
\pgfpathcurveto{\pgfqpoint{5.420736in}{1.171410in}}{\pgfqpoint{5.418422in}{1.165824in}}{\pgfqpoint{5.418422in}{1.160000in}}%
\pgfpathcurveto{\pgfqpoint{5.418422in}{1.154176in}}{\pgfqpoint{5.420736in}{1.148590in}}{\pgfqpoint{5.424854in}{1.144472in}}%
\pgfpathcurveto{\pgfqpoint{5.428972in}{1.140354in}}{\pgfqpoint{5.434558in}{1.138040in}}{\pgfqpoint{5.440382in}{1.138040in}}%
\pgfpathlineto{\pgfqpoint{5.440382in}{1.138040in}}%
\pgfpathclose%
\pgfusepath{stroke,fill}%
\end{pgfscope}%
\begin{pgfscope}%
\pgfpathrectangle{\pgfqpoint{1.542338in}{0.880000in}}{\pgfqpoint{5.115323in}{6.160000in}}%
\pgfusepath{clip}%
\pgfsetbuttcap%
\pgfsetroundjoin%
\definecolor{currentfill}{rgb}{0.800000,0.200000,0.200000}%
\pgfsetfillcolor{currentfill}%
\pgfsetlinewidth{1.003750pt}%
\definecolor{currentstroke}{rgb}{0.800000,0.200000,0.200000}%
\pgfsetstrokecolor{currentstroke}%
\pgfsetdash{}{0pt}%
\pgfpathmoveto{\pgfqpoint{5.570244in}{1.149158in}}%
\pgfpathcurveto{\pgfqpoint{5.576068in}{1.149158in}}{\pgfqpoint{5.581654in}{1.151472in}}{\pgfqpoint{5.585773in}{1.155590in}}%
\pgfpathcurveto{\pgfqpoint{5.589891in}{1.159708in}}{\pgfqpoint{5.592205in}{1.165294in}}{\pgfqpoint{5.592205in}{1.171118in}}%
\pgfpathcurveto{\pgfqpoint{5.592205in}{1.176942in}}{\pgfqpoint{5.589891in}{1.182528in}}{\pgfqpoint{5.585773in}{1.186646in}}%
\pgfpathcurveto{\pgfqpoint{5.581654in}{1.190765in}}{\pgfqpoint{5.576068in}{1.193079in}}{\pgfqpoint{5.570244in}{1.193079in}}%
\pgfpathcurveto{\pgfqpoint{5.564420in}{1.193079in}}{\pgfqpoint{5.558834in}{1.190765in}}{\pgfqpoint{5.554716in}{1.186646in}}%
\pgfpathcurveto{\pgfqpoint{5.550598in}{1.182528in}}{\pgfqpoint{5.548284in}{1.176942in}}{\pgfqpoint{5.548284in}{1.171118in}}%
\pgfpathcurveto{\pgfqpoint{5.548284in}{1.165294in}}{\pgfqpoint{5.550598in}{1.159708in}}{\pgfqpoint{5.554716in}{1.155590in}}%
\pgfpathcurveto{\pgfqpoint{5.558834in}{1.151472in}}{\pgfqpoint{5.564420in}{1.149158in}}{\pgfqpoint{5.570244in}{1.149158in}}%
\pgfpathlineto{\pgfqpoint{5.570244in}{1.149158in}}%
\pgfpathclose%
\pgfusepath{stroke,fill}%
\end{pgfscope}%
\begin{pgfscope}%
\pgfpathrectangle{\pgfqpoint{1.542338in}{0.880000in}}{\pgfqpoint{5.115323in}{6.160000in}}%
\pgfusepath{clip}%
\pgfsetbuttcap%
\pgfsetroundjoin%
\definecolor{currentfill}{rgb}{0.800000,0.200000,0.200000}%
\pgfsetfillcolor{currentfill}%
\pgfsetlinewidth{1.003750pt}%
\definecolor{currentstroke}{rgb}{0.800000,0.200000,0.200000}%
\pgfsetstrokecolor{currentstroke}%
\pgfsetdash{}{0pt}%
\pgfpathmoveto{\pgfqpoint{5.697712in}{1.177110in}}%
\pgfpathcurveto{\pgfqpoint{5.703536in}{1.177110in}}{\pgfqpoint{5.709122in}{1.179424in}}{\pgfqpoint{5.713240in}{1.183542in}}%
\pgfpathcurveto{\pgfqpoint{5.717358in}{1.187660in}}{\pgfqpoint{5.719672in}{1.193246in}}{\pgfqpoint{5.719672in}{1.199070in}}%
\pgfpathcurveto{\pgfqpoint{5.719672in}{1.204894in}}{\pgfqpoint{5.717358in}{1.210480in}}{\pgfqpoint{5.713240in}{1.214598in}}%
\pgfpathcurveto{\pgfqpoint{5.709122in}{1.218716in}}{\pgfqpoint{5.703536in}{1.221030in}}{\pgfqpoint{5.697712in}{1.221030in}}%
\pgfpathcurveto{\pgfqpoint{5.691888in}{1.221030in}}{\pgfqpoint{5.686302in}{1.218716in}}{\pgfqpoint{5.682184in}{1.214598in}}%
\pgfpathcurveto{\pgfqpoint{5.678065in}{1.210480in}}{\pgfqpoint{5.675752in}{1.204894in}}{\pgfqpoint{5.675752in}{1.199070in}}%
\pgfpathcurveto{\pgfqpoint{5.675752in}{1.193246in}}{\pgfqpoint{5.678065in}{1.187660in}}{\pgfqpoint{5.682184in}{1.183542in}}%
\pgfpathcurveto{\pgfqpoint{5.686302in}{1.179424in}}{\pgfqpoint{5.691888in}{1.177110in}}{\pgfqpoint{5.697712in}{1.177110in}}%
\pgfpathlineto{\pgfqpoint{5.697712in}{1.177110in}}%
\pgfpathclose%
\pgfusepath{stroke,fill}%
\end{pgfscope}%
\begin{pgfscope}%
\pgfpathrectangle{\pgfqpoint{1.542338in}{0.880000in}}{\pgfqpoint{5.115323in}{6.160000in}}%
\pgfusepath{clip}%
\pgfsetbuttcap%
\pgfsetroundjoin%
\definecolor{currentfill}{rgb}{0.800000,0.200000,0.200000}%
\pgfsetfillcolor{currentfill}%
\pgfsetlinewidth{1.003750pt}%
\definecolor{currentstroke}{rgb}{0.800000,0.200000,0.200000}%
\pgfsetstrokecolor{currentstroke}%
\pgfsetdash{}{0pt}%
\pgfpathmoveto{\pgfqpoint{5.820603in}{1.221383in}}%
\pgfpathcurveto{\pgfqpoint{5.826427in}{1.221383in}}{\pgfqpoint{5.832013in}{1.223697in}}{\pgfqpoint{5.836131in}{1.227816in}}%
\pgfpathcurveto{\pgfqpoint{5.840250in}{1.231934in}}{\pgfqpoint{5.842563in}{1.237520in}}{\pgfqpoint{5.842563in}{1.243344in}}%
\pgfpathcurveto{\pgfqpoint{5.842563in}{1.249168in}}{\pgfqpoint{5.840250in}{1.254754in}}{\pgfqpoint{5.836131in}{1.258872in}}%
\pgfpathcurveto{\pgfqpoint{5.832013in}{1.262990in}}{\pgfqpoint{5.826427in}{1.265304in}}{\pgfqpoint{5.820603in}{1.265304in}}%
\pgfpathcurveto{\pgfqpoint{5.814779in}{1.265304in}}{\pgfqpoint{5.809193in}{1.262990in}}{\pgfqpoint{5.805075in}{1.258872in}}%
\pgfpathcurveto{\pgfqpoint{5.800957in}{1.254754in}}{\pgfqpoint{5.798643in}{1.249168in}}{\pgfqpoint{5.798643in}{1.243344in}}%
\pgfpathcurveto{\pgfqpoint{5.798643in}{1.237520in}}{\pgfqpoint{5.800957in}{1.231934in}}{\pgfqpoint{5.805075in}{1.227816in}}%
\pgfpathcurveto{\pgfqpoint{5.809193in}{1.223697in}}{\pgfqpoint{5.814779in}{1.221383in}}{\pgfqpoint{5.820603in}{1.221383in}}%
\pgfpathlineto{\pgfqpoint{5.820603in}{1.221383in}}%
\pgfpathclose%
\pgfusepath{stroke,fill}%
\end{pgfscope}%
\begin{pgfscope}%
\pgfpathrectangle{\pgfqpoint{1.542338in}{0.880000in}}{\pgfqpoint{5.115323in}{6.160000in}}%
\pgfusepath{clip}%
\pgfsetbuttcap%
\pgfsetroundjoin%
\definecolor{currentfill}{rgb}{0.800000,0.200000,0.200000}%
\pgfsetfillcolor{currentfill}%
\pgfsetlinewidth{1.003750pt}%
\definecolor{currentstroke}{rgb}{0.800000,0.200000,0.200000}%
\pgfsetstrokecolor{currentstroke}%
\pgfsetdash{}{0pt}%
\pgfpathmoveto{\pgfqpoint{5.939115in}{1.277450in}}%
\pgfpathcurveto{\pgfqpoint{5.944939in}{1.277450in}}{\pgfqpoint{5.950525in}{1.279764in}}{\pgfqpoint{5.954643in}{1.283882in}}%
\pgfpathcurveto{\pgfqpoint{5.958762in}{1.288000in}}{\pgfqpoint{5.961075in}{1.293586in}}{\pgfqpoint{5.961075in}{1.299410in}}%
\pgfpathcurveto{\pgfqpoint{5.961075in}{1.305234in}}{\pgfqpoint{5.958762in}{1.310820in}}{\pgfqpoint{5.954643in}{1.314938in}}%
\pgfpathcurveto{\pgfqpoint{5.950525in}{1.319057in}}{\pgfqpoint{5.944939in}{1.321370in}}{\pgfqpoint{5.939115in}{1.321370in}}%
\pgfpathcurveto{\pgfqpoint{5.933291in}{1.321370in}}{\pgfqpoint{5.927705in}{1.319057in}}{\pgfqpoint{5.923587in}{1.314938in}}%
\pgfpathcurveto{\pgfqpoint{5.919469in}{1.310820in}}{\pgfqpoint{5.917155in}{1.305234in}}{\pgfqpoint{5.917155in}{1.299410in}}%
\pgfpathcurveto{\pgfqpoint{5.917155in}{1.293586in}}{\pgfqpoint{5.919469in}{1.288000in}}{\pgfqpoint{5.923587in}{1.283882in}}%
\pgfpathcurveto{\pgfqpoint{5.927705in}{1.279764in}}{\pgfqpoint{5.933291in}{1.277450in}}{\pgfqpoint{5.939115in}{1.277450in}}%
\pgfpathlineto{\pgfqpoint{5.939115in}{1.277450in}}%
\pgfpathclose%
\pgfusepath{stroke,fill}%
\end{pgfscope}%
\begin{pgfscope}%
\pgfpathrectangle{\pgfqpoint{1.542338in}{0.880000in}}{\pgfqpoint{5.115323in}{6.160000in}}%
\pgfusepath{clip}%
\pgfsetbuttcap%
\pgfsetroundjoin%
\definecolor{currentfill}{rgb}{0.800000,0.200000,0.200000}%
\pgfsetfillcolor{currentfill}%
\pgfsetlinewidth{1.003750pt}%
\definecolor{currentstroke}{rgb}{0.800000,0.200000,0.200000}%
\pgfsetstrokecolor{currentstroke}%
\pgfsetdash{}{0pt}%
\pgfpathmoveto{\pgfqpoint{6.044030in}{1.356101in}}%
\pgfpathcurveto{\pgfqpoint{6.049854in}{1.356101in}}{\pgfqpoint{6.055440in}{1.358415in}}{\pgfqpoint{6.059558in}{1.362533in}}%
\pgfpathcurveto{\pgfqpoint{6.063676in}{1.366651in}}{\pgfqpoint{6.065990in}{1.372237in}}{\pgfqpoint{6.065990in}{1.378061in}}%
\pgfpathcurveto{\pgfqpoint{6.065990in}{1.383885in}}{\pgfqpoint{6.063676in}{1.389471in}}{\pgfqpoint{6.059558in}{1.393589in}}%
\pgfpathcurveto{\pgfqpoint{6.055440in}{1.397707in}}{\pgfqpoint{6.049854in}{1.400021in}}{\pgfqpoint{6.044030in}{1.400021in}}%
\pgfpathcurveto{\pgfqpoint{6.038206in}{1.400021in}}{\pgfqpoint{6.032620in}{1.397707in}}{\pgfqpoint{6.028501in}{1.393589in}}%
\pgfpathcurveto{\pgfqpoint{6.024383in}{1.389471in}}{\pgfqpoint{6.022069in}{1.383885in}}{\pgfqpoint{6.022069in}{1.378061in}}%
\pgfpathcurveto{\pgfqpoint{6.022069in}{1.372237in}}{\pgfqpoint{6.024383in}{1.366651in}}{\pgfqpoint{6.028501in}{1.362533in}}%
\pgfpathcurveto{\pgfqpoint{6.032620in}{1.358415in}}{\pgfqpoint{6.038206in}{1.356101in}}{\pgfqpoint{6.044030in}{1.356101in}}%
\pgfpathlineto{\pgfqpoint{6.044030in}{1.356101in}}%
\pgfpathclose%
\pgfusepath{stroke,fill}%
\end{pgfscope}%
\begin{pgfscope}%
\pgfpathrectangle{\pgfqpoint{1.542338in}{0.880000in}}{\pgfqpoint{5.115323in}{6.160000in}}%
\pgfusepath{clip}%
\pgfsetbuttcap%
\pgfsetroundjoin%
\definecolor{currentfill}{rgb}{0.800000,0.200000,0.200000}%
\pgfsetfillcolor{currentfill}%
\pgfsetlinewidth{1.003750pt}%
\definecolor{currentstroke}{rgb}{0.800000,0.200000,0.200000}%
\pgfsetstrokecolor{currentstroke}%
\pgfsetdash{}{0pt}%
\pgfpathmoveto{\pgfqpoint{6.134067in}{1.450556in}}%
\pgfpathcurveto{\pgfqpoint{6.139891in}{1.450556in}}{\pgfqpoint{6.145477in}{1.452870in}}{\pgfqpoint{6.149595in}{1.456988in}}%
\pgfpathcurveto{\pgfqpoint{6.153713in}{1.461106in}}{\pgfqpoint{6.156027in}{1.466693in}}{\pgfqpoint{6.156027in}{1.472516in}}%
\pgfpathcurveto{\pgfqpoint{6.156027in}{1.478340in}}{\pgfqpoint{6.153713in}{1.483927in}}{\pgfqpoint{6.149595in}{1.488045in}}%
\pgfpathcurveto{\pgfqpoint{6.145477in}{1.492163in}}{\pgfqpoint{6.139891in}{1.494477in}}{\pgfqpoint{6.134067in}{1.494477in}}%
\pgfpathcurveto{\pgfqpoint{6.128243in}{1.494477in}}{\pgfqpoint{6.122657in}{1.492163in}}{\pgfqpoint{6.118538in}{1.488045in}}%
\pgfpathcurveto{\pgfqpoint{6.114420in}{1.483927in}}{\pgfqpoint{6.112106in}{1.478340in}}{\pgfqpoint{6.112106in}{1.472516in}}%
\pgfpathcurveto{\pgfqpoint{6.112106in}{1.466693in}}{\pgfqpoint{6.114420in}{1.461106in}}{\pgfqpoint{6.118538in}{1.456988in}}%
\pgfpathcurveto{\pgfqpoint{6.122657in}{1.452870in}}{\pgfqpoint{6.128243in}{1.450556in}}{\pgfqpoint{6.134067in}{1.450556in}}%
\pgfpathlineto{\pgfqpoint{6.134067in}{1.450556in}}%
\pgfpathclose%
\pgfusepath{stroke,fill}%
\end{pgfscope}%
\begin{pgfscope}%
\pgfpathrectangle{\pgfqpoint{1.542338in}{0.880000in}}{\pgfqpoint{5.115323in}{6.160000in}}%
\pgfusepath{clip}%
\pgfsetbuttcap%
\pgfsetroundjoin%
\definecolor{currentfill}{rgb}{0.800000,0.200000,0.200000}%
\pgfsetfillcolor{currentfill}%
\pgfsetlinewidth{1.003750pt}%
\definecolor{currentstroke}{rgb}{0.800000,0.200000,0.200000}%
\pgfsetstrokecolor{currentstroke}%
\pgfsetdash{}{0pt}%
\pgfpathmoveto{\pgfqpoint{6.219322in}{1.548251in}}%
\pgfpathcurveto{\pgfqpoint{6.225146in}{1.548251in}}{\pgfqpoint{6.230732in}{1.550565in}}{\pgfqpoint{6.234850in}{1.554683in}}%
\pgfpathcurveto{\pgfqpoint{6.238968in}{1.558801in}}{\pgfqpoint{6.241282in}{1.564387in}}{\pgfqpoint{6.241282in}{1.570211in}}%
\pgfpathcurveto{\pgfqpoint{6.241282in}{1.576035in}}{\pgfqpoint{6.238968in}{1.581621in}}{\pgfqpoint{6.234850in}{1.585739in}}%
\pgfpathcurveto{\pgfqpoint{6.230732in}{1.589857in}}{\pgfqpoint{6.225146in}{1.592171in}}{\pgfqpoint{6.219322in}{1.592171in}}%
\pgfpathcurveto{\pgfqpoint{6.213498in}{1.592171in}}{\pgfqpoint{6.207912in}{1.589857in}}{\pgfqpoint{6.203793in}{1.585739in}}%
\pgfpathcurveto{\pgfqpoint{6.199675in}{1.581621in}}{\pgfqpoint{6.197361in}{1.576035in}}{\pgfqpoint{6.197361in}{1.570211in}}%
\pgfpathcurveto{\pgfqpoint{6.197361in}{1.564387in}}{\pgfqpoint{6.199675in}{1.558801in}}{\pgfqpoint{6.203793in}{1.554683in}}%
\pgfpathcurveto{\pgfqpoint{6.207912in}{1.550565in}}{\pgfqpoint{6.213498in}{1.548251in}}{\pgfqpoint{6.219322in}{1.548251in}}%
\pgfpathlineto{\pgfqpoint{6.219322in}{1.548251in}}%
\pgfpathclose%
\pgfusepath{stroke,fill}%
\end{pgfscope}%
\begin{pgfscope}%
\pgfpathrectangle{\pgfqpoint{1.542338in}{0.880000in}}{\pgfqpoint{5.115323in}{6.160000in}}%
\pgfusepath{clip}%
\pgfsetbuttcap%
\pgfsetroundjoin%
\definecolor{currentfill}{rgb}{0.800000,0.200000,0.200000}%
\pgfsetfillcolor{currentfill}%
\pgfsetlinewidth{1.003750pt}%
\definecolor{currentstroke}{rgb}{0.800000,0.200000,0.200000}%
\pgfsetstrokecolor{currentstroke}%
\pgfsetdash{}{0pt}%
\pgfpathmoveto{\pgfqpoint{6.292939in}{1.655403in}}%
\pgfpathcurveto{\pgfqpoint{6.298762in}{1.655403in}}{\pgfqpoint{6.304349in}{1.657717in}}{\pgfqpoint{6.308467in}{1.661835in}}%
\pgfpathcurveto{\pgfqpoint{6.312585in}{1.665954in}}{\pgfqpoint{6.314899in}{1.671540in}}{\pgfqpoint{6.314899in}{1.677364in}}%
\pgfpathcurveto{\pgfqpoint{6.314899in}{1.683188in}}{\pgfqpoint{6.312585in}{1.688774in}}{\pgfqpoint{6.308467in}{1.692892in}}%
\pgfpathcurveto{\pgfqpoint{6.304349in}{1.697010in}}{\pgfqpoint{6.298762in}{1.699324in}}{\pgfqpoint{6.292939in}{1.699324in}}%
\pgfpathcurveto{\pgfqpoint{6.287115in}{1.699324in}}{\pgfqpoint{6.281528in}{1.697010in}}{\pgfqpoint{6.277410in}{1.692892in}}%
\pgfpathcurveto{\pgfqpoint{6.273292in}{1.688774in}}{\pgfqpoint{6.270978in}{1.683188in}}{\pgfqpoint{6.270978in}{1.677364in}}%
\pgfpathcurveto{\pgfqpoint{6.270978in}{1.671540in}}{\pgfqpoint{6.273292in}{1.665954in}}{\pgfqpoint{6.277410in}{1.661835in}}%
\pgfpathcurveto{\pgfqpoint{6.281528in}{1.657717in}}{\pgfqpoint{6.287115in}{1.655403in}}{\pgfqpoint{6.292939in}{1.655403in}}%
\pgfpathlineto{\pgfqpoint{6.292939in}{1.655403in}}%
\pgfpathclose%
\pgfusepath{stroke,fill}%
\end{pgfscope}%
\begin{pgfscope}%
\pgfpathrectangle{\pgfqpoint{1.542338in}{0.880000in}}{\pgfqpoint{5.115323in}{6.160000in}}%
\pgfusepath{clip}%
\pgfsetbuttcap%
\pgfsetroundjoin%
\definecolor{currentfill}{rgb}{0.800000,0.200000,0.200000}%
\pgfsetfillcolor{currentfill}%
\pgfsetlinewidth{1.003750pt}%
\definecolor{currentstroke}{rgb}{0.800000,0.200000,0.200000}%
\pgfsetstrokecolor{currentstroke}%
\pgfsetdash{}{0pt}%
\pgfpathmoveto{\pgfqpoint{6.347601in}{1.773420in}}%
\pgfpathcurveto{\pgfqpoint{6.353425in}{1.773420in}}{\pgfqpoint{6.359011in}{1.775734in}}{\pgfqpoint{6.363129in}{1.779852in}}%
\pgfpathcurveto{\pgfqpoint{6.367248in}{1.783970in}}{\pgfqpoint{6.369561in}{1.789556in}}{\pgfqpoint{6.369561in}{1.795380in}}%
\pgfpathcurveto{\pgfqpoint{6.369561in}{1.801204in}}{\pgfqpoint{6.367248in}{1.806790in}}{\pgfqpoint{6.363129in}{1.810908in}}%
\pgfpathcurveto{\pgfqpoint{6.359011in}{1.815027in}}{\pgfqpoint{6.353425in}{1.817340in}}{\pgfqpoint{6.347601in}{1.817340in}}%
\pgfpathcurveto{\pgfqpoint{6.341777in}{1.817340in}}{\pgfqpoint{6.336191in}{1.815027in}}{\pgfqpoint{6.332073in}{1.810908in}}%
\pgfpathcurveto{\pgfqpoint{6.327955in}{1.806790in}}{\pgfqpoint{6.325641in}{1.801204in}}{\pgfqpoint{6.325641in}{1.795380in}}%
\pgfpathcurveto{\pgfqpoint{6.325641in}{1.789556in}}{\pgfqpoint{6.327955in}{1.783970in}}{\pgfqpoint{6.332073in}{1.779852in}}%
\pgfpathcurveto{\pgfqpoint{6.336191in}{1.775734in}}{\pgfqpoint{6.341777in}{1.773420in}}{\pgfqpoint{6.347601in}{1.773420in}}%
\pgfpathlineto{\pgfqpoint{6.347601in}{1.773420in}}%
\pgfpathclose%
\pgfusepath{stroke,fill}%
\end{pgfscope}%
\begin{pgfscope}%
\pgfpathrectangle{\pgfqpoint{1.542338in}{0.880000in}}{\pgfqpoint{5.115323in}{6.160000in}}%
\pgfusepath{clip}%
\pgfsetbuttcap%
\pgfsetroundjoin%
\definecolor{currentfill}{rgb}{0.800000,0.200000,0.200000}%
\pgfsetfillcolor{currentfill}%
\pgfsetlinewidth{1.003750pt}%
\definecolor{currentstroke}{rgb}{0.800000,0.200000,0.200000}%
\pgfsetstrokecolor{currentstroke}%
\pgfsetdash{}{0pt}%
\pgfpathmoveto{\pgfqpoint{6.393019in}{1.895548in}}%
\pgfpathcurveto{\pgfqpoint{6.398843in}{1.895548in}}{\pgfqpoint{6.404429in}{1.897861in}}{\pgfqpoint{6.408547in}{1.901980in}}%
\pgfpathcurveto{\pgfqpoint{6.412666in}{1.906098in}}{\pgfqpoint{6.414979in}{1.911684in}}{\pgfqpoint{6.414979in}{1.917508in}}%
\pgfpathcurveto{\pgfqpoint{6.414979in}{1.923332in}}{\pgfqpoint{6.412666in}{1.928918in}}{\pgfqpoint{6.408547in}{1.933036in}}%
\pgfpathcurveto{\pgfqpoint{6.404429in}{1.937154in}}{\pgfqpoint{6.398843in}{1.939468in}}{\pgfqpoint{6.393019in}{1.939468in}}%
\pgfpathcurveto{\pgfqpoint{6.387195in}{1.939468in}}{\pgfqpoint{6.381609in}{1.937154in}}{\pgfqpoint{6.377491in}{1.933036in}}%
\pgfpathcurveto{\pgfqpoint{6.373373in}{1.928918in}}{\pgfqpoint{6.371059in}{1.923332in}}{\pgfqpoint{6.371059in}{1.917508in}}%
\pgfpathcurveto{\pgfqpoint{6.371059in}{1.911684in}}{\pgfqpoint{6.373373in}{1.906098in}}{\pgfqpoint{6.377491in}{1.901980in}}%
\pgfpathcurveto{\pgfqpoint{6.381609in}{1.897861in}}{\pgfqpoint{6.387195in}{1.895548in}}{\pgfqpoint{6.393019in}{1.895548in}}%
\pgfpathlineto{\pgfqpoint{6.393019in}{1.895548in}}%
\pgfpathclose%
\pgfusepath{stroke,fill}%
\end{pgfscope}%
\begin{pgfscope}%
\pgfpathrectangle{\pgfqpoint{1.542338in}{0.880000in}}{\pgfqpoint{5.115323in}{6.160000in}}%
\pgfusepath{clip}%
\pgfsetbuttcap%
\pgfsetroundjoin%
\definecolor{currentfill}{rgb}{0.800000,0.200000,0.200000}%
\pgfsetfillcolor{currentfill}%
\pgfsetlinewidth{1.003750pt}%
\definecolor{currentstroke}{rgb}{0.800000,0.200000,0.200000}%
\pgfsetstrokecolor{currentstroke}%
\pgfsetdash{}{0pt}%
\pgfpathmoveto{\pgfqpoint{6.418391in}{2.023601in}}%
\pgfpathcurveto{\pgfqpoint{6.424214in}{2.023601in}}{\pgfqpoint{6.429801in}{2.025914in}}{\pgfqpoint{6.433919in}{2.030033in}}%
\pgfpathcurveto{\pgfqpoint{6.438037in}{2.034151in}}{\pgfqpoint{6.440351in}{2.039737in}}{\pgfqpoint{6.440351in}{2.045561in}}%
\pgfpathcurveto{\pgfqpoint{6.440351in}{2.051385in}}{\pgfqpoint{6.438037in}{2.056971in}}{\pgfqpoint{6.433919in}{2.061089in}}%
\pgfpathcurveto{\pgfqpoint{6.429801in}{2.065207in}}{\pgfqpoint{6.424214in}{2.067521in}}{\pgfqpoint{6.418391in}{2.067521in}}%
\pgfpathcurveto{\pgfqpoint{6.412567in}{2.067521in}}{\pgfqpoint{6.406980in}{2.065207in}}{\pgfqpoint{6.402862in}{2.061089in}}%
\pgfpathcurveto{\pgfqpoint{6.398744in}{2.056971in}}{\pgfqpoint{6.396430in}{2.051385in}}{\pgfqpoint{6.396430in}{2.045561in}}%
\pgfpathcurveto{\pgfqpoint{6.396430in}{2.039737in}}{\pgfqpoint{6.398744in}{2.034151in}}{\pgfqpoint{6.402862in}{2.030033in}}%
\pgfpathcurveto{\pgfqpoint{6.406980in}{2.025914in}}{\pgfqpoint{6.412567in}{2.023601in}}{\pgfqpoint{6.418391in}{2.023601in}}%
\pgfpathlineto{\pgfqpoint{6.418391in}{2.023601in}}%
\pgfpathclose%
\pgfusepath{stroke,fill}%
\end{pgfscope}%
\begin{pgfscope}%
\pgfpathrectangle{\pgfqpoint{1.542338in}{0.880000in}}{\pgfqpoint{5.115323in}{6.160000in}}%
\pgfusepath{clip}%
\pgfsetbuttcap%
\pgfsetroundjoin%
\definecolor{currentfill}{rgb}{0.800000,0.200000,0.200000}%
\pgfsetfillcolor{currentfill}%
\pgfsetlinewidth{1.003750pt}%
\definecolor{currentstroke}{rgb}{0.800000,0.200000,0.200000}%
\pgfsetstrokecolor{currentstroke}%
\pgfsetdash{}{0pt}%
\pgfpathmoveto{\pgfqpoint{6.422822in}{2.153901in}}%
\pgfpathcurveto{\pgfqpoint{6.428645in}{2.153901in}}{\pgfqpoint{6.434232in}{2.156215in}}{\pgfqpoint{6.438350in}{2.160333in}}%
\pgfpathcurveto{\pgfqpoint{6.442468in}{2.164451in}}{\pgfqpoint{6.444782in}{2.170037in}}{\pgfqpoint{6.444782in}{2.175861in}}%
\pgfpathcurveto{\pgfqpoint{6.444782in}{2.181685in}}{\pgfqpoint{6.442468in}{2.187272in}}{\pgfqpoint{6.438350in}{2.191390in}}%
\pgfpathcurveto{\pgfqpoint{6.434232in}{2.195508in}}{\pgfqpoint{6.428645in}{2.197822in}}{\pgfqpoint{6.422822in}{2.197822in}}%
\pgfpathcurveto{\pgfqpoint{6.416998in}{2.197822in}}{\pgfqpoint{6.411411in}{2.195508in}}{\pgfqpoint{6.407293in}{2.191390in}}%
\pgfpathcurveto{\pgfqpoint{6.403175in}{2.187272in}}{\pgfqpoint{6.400861in}{2.181685in}}{\pgfqpoint{6.400861in}{2.175861in}}%
\pgfpathcurveto{\pgfqpoint{6.400861in}{2.170037in}}{\pgfqpoint{6.403175in}{2.164451in}}{\pgfqpoint{6.407293in}{2.160333in}}%
\pgfpathcurveto{\pgfqpoint{6.411411in}{2.156215in}}{\pgfqpoint{6.416998in}{2.153901in}}{\pgfqpoint{6.422822in}{2.153901in}}%
\pgfpathlineto{\pgfqpoint{6.422822in}{2.153901in}}%
\pgfpathclose%
\pgfusepath{stroke,fill}%
\end{pgfscope}%
\begin{pgfscope}%
\pgfpathrectangle{\pgfqpoint{1.542338in}{0.880000in}}{\pgfqpoint{5.115323in}{6.160000in}}%
\pgfusepath{clip}%
\pgfsetbuttcap%
\pgfsetroundjoin%
\definecolor{currentfill}{rgb}{0.200000,0.200000,0.800000}%
\pgfsetfillcolor{currentfill}%
\pgfsetlinewidth{1.003750pt}%
\definecolor{currentstroke}{rgb}{0.200000,0.200000,0.800000}%
\pgfsetstrokecolor{currentstroke}%
\pgfsetdash{}{0pt}%
\pgfpathmoveto{\pgfqpoint{4.165962in}{2.531321in}}%
\pgfpathcurveto{\pgfqpoint{4.171786in}{2.531321in}}{\pgfqpoint{4.177372in}{2.533635in}}{\pgfqpoint{4.181490in}{2.537753in}}%
\pgfpathcurveto{\pgfqpoint{4.185608in}{2.541872in}}{\pgfqpoint{4.187922in}{2.547458in}}{\pgfqpoint{4.187922in}{2.553282in}}%
\pgfpathcurveto{\pgfqpoint{4.187922in}{2.559106in}}{\pgfqpoint{4.185608in}{2.564692in}}{\pgfqpoint{4.181490in}{2.568810in}}%
\pgfpathcurveto{\pgfqpoint{4.177372in}{2.572928in}}{\pgfqpoint{4.171786in}{2.575242in}}{\pgfqpoint{4.165962in}{2.575242in}}%
\pgfpathcurveto{\pgfqpoint{4.160138in}{2.575242in}}{\pgfqpoint{4.154552in}{2.572928in}}{\pgfqpoint{4.150434in}{2.568810in}}%
\pgfpathcurveto{\pgfqpoint{4.146315in}{2.564692in}}{\pgfqpoint{4.144002in}{2.559106in}}{\pgfqpoint{4.144002in}{2.553282in}}%
\pgfpathcurveto{\pgfqpoint{4.144002in}{2.547458in}}{\pgfqpoint{4.146315in}{2.541872in}}{\pgfqpoint{4.150434in}{2.537753in}}%
\pgfpathcurveto{\pgfqpoint{4.154552in}{2.533635in}}{\pgfqpoint{4.160138in}{2.531321in}}{\pgfqpoint{4.165962in}{2.531321in}}%
\pgfpathlineto{\pgfqpoint{4.165962in}{2.531321in}}%
\pgfpathclose%
\pgfusepath{stroke,fill}%
\end{pgfscope}%
\begin{pgfscope}%
\pgfpathrectangle{\pgfqpoint{1.542338in}{0.880000in}}{\pgfqpoint{5.115323in}{6.160000in}}%
\pgfusepath{clip}%
\pgfsetbuttcap%
\pgfsetroundjoin%
\definecolor{currentfill}{rgb}{0.200000,0.200000,0.800000}%
\pgfsetfillcolor{currentfill}%
\pgfsetlinewidth{1.003750pt}%
\definecolor{currentstroke}{rgb}{0.200000,0.200000,0.800000}%
\pgfsetstrokecolor{currentstroke}%
\pgfsetdash{}{0pt}%
\pgfpathmoveto{\pgfqpoint{4.168069in}{2.685262in}}%
\pgfpathcurveto{\pgfqpoint{4.173893in}{2.685262in}}{\pgfqpoint{4.179480in}{2.687576in}}{\pgfqpoint{4.183598in}{2.691694in}}%
\pgfpathcurveto{\pgfqpoint{4.187716in}{2.695812in}}{\pgfqpoint{4.190030in}{2.701398in}}{\pgfqpoint{4.190030in}{2.707222in}}%
\pgfpathcurveto{\pgfqpoint{4.190030in}{2.713046in}}{\pgfqpoint{4.187716in}{2.718632in}}{\pgfqpoint{4.183598in}{2.722750in}}%
\pgfpathcurveto{\pgfqpoint{4.179480in}{2.726868in}}{\pgfqpoint{4.173893in}{2.729182in}}{\pgfqpoint{4.168069in}{2.729182in}}%
\pgfpathcurveto{\pgfqpoint{4.162246in}{2.729182in}}{\pgfqpoint{4.156659in}{2.726868in}}{\pgfqpoint{4.152541in}{2.722750in}}%
\pgfpathcurveto{\pgfqpoint{4.148423in}{2.718632in}}{\pgfqpoint{4.146109in}{2.713046in}}{\pgfqpoint{4.146109in}{2.707222in}}%
\pgfpathcurveto{\pgfqpoint{4.146109in}{2.701398in}}{\pgfqpoint{4.148423in}{2.695812in}}{\pgfqpoint{4.152541in}{2.691694in}}%
\pgfpathcurveto{\pgfqpoint{4.156659in}{2.687576in}}{\pgfqpoint{4.162246in}{2.685262in}}{\pgfqpoint{4.168069in}{2.685262in}}%
\pgfpathlineto{\pgfqpoint{4.168069in}{2.685262in}}%
\pgfpathclose%
\pgfusepath{stroke,fill}%
\end{pgfscope}%
\begin{pgfscope}%
\pgfpathrectangle{\pgfqpoint{1.542338in}{0.880000in}}{\pgfqpoint{5.115323in}{6.160000in}}%
\pgfusepath{clip}%
\pgfsetbuttcap%
\pgfsetroundjoin%
\definecolor{currentfill}{rgb}{0.200000,0.200000,0.800000}%
\pgfsetfillcolor{currentfill}%
\pgfsetlinewidth{1.003750pt}%
\definecolor{currentstroke}{rgb}{0.200000,0.200000,0.800000}%
\pgfsetstrokecolor{currentstroke}%
\pgfsetdash{}{0pt}%
\pgfpathmoveto{\pgfqpoint{4.137677in}{2.836437in}}%
\pgfpathcurveto{\pgfqpoint{4.143501in}{2.836437in}}{\pgfqpoint{4.149087in}{2.838751in}}{\pgfqpoint{4.153205in}{2.842869in}}%
\pgfpathcurveto{\pgfqpoint{4.157323in}{2.846987in}}{\pgfqpoint{4.159637in}{2.852574in}}{\pgfqpoint{4.159637in}{2.858398in}}%
\pgfpathcurveto{\pgfqpoint{4.159637in}{2.864221in}}{\pgfqpoint{4.157323in}{2.869808in}}{\pgfqpoint{4.153205in}{2.873926in}}%
\pgfpathcurveto{\pgfqpoint{4.149087in}{2.878044in}}{\pgfqpoint{4.143501in}{2.880358in}}{\pgfqpoint{4.137677in}{2.880358in}}%
\pgfpathcurveto{\pgfqpoint{4.131853in}{2.880358in}}{\pgfqpoint{4.126267in}{2.878044in}}{\pgfqpoint{4.122149in}{2.873926in}}%
\pgfpathcurveto{\pgfqpoint{4.118030in}{2.869808in}}{\pgfqpoint{4.115717in}{2.864221in}}{\pgfqpoint{4.115717in}{2.858398in}}%
\pgfpathcurveto{\pgfqpoint{4.115717in}{2.852574in}}{\pgfqpoint{4.118030in}{2.846987in}}{\pgfqpoint{4.122149in}{2.842869in}}%
\pgfpathcurveto{\pgfqpoint{4.126267in}{2.838751in}}{\pgfqpoint{4.131853in}{2.836437in}}{\pgfqpoint{4.137677in}{2.836437in}}%
\pgfpathlineto{\pgfqpoint{4.137677in}{2.836437in}}%
\pgfpathclose%
\pgfusepath{stroke,fill}%
\end{pgfscope}%
\begin{pgfscope}%
\pgfpathrectangle{\pgfqpoint{1.542338in}{0.880000in}}{\pgfqpoint{5.115323in}{6.160000in}}%
\pgfusepath{clip}%
\pgfsetbuttcap%
\pgfsetroundjoin%
\definecolor{currentfill}{rgb}{0.200000,0.200000,0.800000}%
\pgfsetfillcolor{currentfill}%
\pgfsetlinewidth{1.003750pt}%
\definecolor{currentstroke}{rgb}{0.200000,0.200000,0.800000}%
\pgfsetstrokecolor{currentstroke}%
\pgfsetdash{}{0pt}%
\pgfpathmoveto{\pgfqpoint{4.079875in}{2.978985in}}%
\pgfpathcurveto{\pgfqpoint{4.085699in}{2.978985in}}{\pgfqpoint{4.091285in}{2.981299in}}{\pgfqpoint{4.095403in}{2.985417in}}%
\pgfpathcurveto{\pgfqpoint{4.099521in}{2.989535in}}{\pgfqpoint{4.101835in}{2.995121in}}{\pgfqpoint{4.101835in}{3.000945in}}%
\pgfpathcurveto{\pgfqpoint{4.101835in}{3.006769in}}{\pgfqpoint{4.099521in}{3.012355in}}{\pgfqpoint{4.095403in}{3.016473in}}%
\pgfpathcurveto{\pgfqpoint{4.091285in}{3.020591in}}{\pgfqpoint{4.085699in}{3.022905in}}{\pgfqpoint{4.079875in}{3.022905in}}%
\pgfpathcurveto{\pgfqpoint{4.074051in}{3.022905in}}{\pgfqpoint{4.068465in}{3.020591in}}{\pgfqpoint{4.064347in}{3.016473in}}%
\pgfpathcurveto{\pgfqpoint{4.060229in}{3.012355in}}{\pgfqpoint{4.057915in}{3.006769in}}{\pgfqpoint{4.057915in}{3.000945in}}%
\pgfpathcurveto{\pgfqpoint{4.057915in}{2.995121in}}{\pgfqpoint{4.060229in}{2.989535in}}{\pgfqpoint{4.064347in}{2.985417in}}%
\pgfpathcurveto{\pgfqpoint{4.068465in}{2.981299in}}{\pgfqpoint{4.074051in}{2.978985in}}{\pgfqpoint{4.079875in}{2.978985in}}%
\pgfpathlineto{\pgfqpoint{4.079875in}{2.978985in}}%
\pgfpathclose%
\pgfusepath{stroke,fill}%
\end{pgfscope}%
\begin{pgfscope}%
\pgfpathrectangle{\pgfqpoint{1.542338in}{0.880000in}}{\pgfqpoint{5.115323in}{6.160000in}}%
\pgfusepath{clip}%
\pgfsetbuttcap%
\pgfsetroundjoin%
\definecolor{currentfill}{rgb}{0.200000,0.200000,0.800000}%
\pgfsetfillcolor{currentfill}%
\pgfsetlinewidth{1.003750pt}%
\definecolor{currentstroke}{rgb}{0.200000,0.200000,0.800000}%
\pgfsetstrokecolor{currentstroke}%
\pgfsetdash{}{0pt}%
\pgfpathmoveto{\pgfqpoint{4.016867in}{3.118576in}}%
\pgfpathcurveto{\pgfqpoint{4.022691in}{3.118576in}}{\pgfqpoint{4.028277in}{3.120890in}}{\pgfqpoint{4.032396in}{3.125008in}}%
\pgfpathcurveto{\pgfqpoint{4.036514in}{3.129126in}}{\pgfqpoint{4.038828in}{3.134712in}}{\pgfqpoint{4.038828in}{3.140536in}}%
\pgfpathcurveto{\pgfqpoint{4.038828in}{3.146360in}}{\pgfqpoint{4.036514in}{3.151946in}}{\pgfqpoint{4.032396in}{3.156064in}}%
\pgfpathcurveto{\pgfqpoint{4.028277in}{3.160182in}}{\pgfqpoint{4.022691in}{3.162496in}}{\pgfqpoint{4.016867in}{3.162496in}}%
\pgfpathcurveto{\pgfqpoint{4.011043in}{3.162496in}}{\pgfqpoint{4.005457in}{3.160182in}}{\pgfqpoint{4.001339in}{3.156064in}}%
\pgfpathcurveto{\pgfqpoint{3.997221in}{3.151946in}}{\pgfqpoint{3.994907in}{3.146360in}}{\pgfqpoint{3.994907in}{3.140536in}}%
\pgfpathcurveto{\pgfqpoint{3.994907in}{3.134712in}}{\pgfqpoint{3.997221in}{3.129126in}}{\pgfqpoint{4.001339in}{3.125008in}}%
\pgfpathcurveto{\pgfqpoint{4.005457in}{3.120890in}}{\pgfqpoint{4.011043in}{3.118576in}}{\pgfqpoint{4.016867in}{3.118576in}}%
\pgfpathlineto{\pgfqpoint{4.016867in}{3.118576in}}%
\pgfpathclose%
\pgfusepath{stroke,fill}%
\end{pgfscope}%
\begin{pgfscope}%
\pgfpathrectangle{\pgfqpoint{1.542338in}{0.880000in}}{\pgfqpoint{5.115323in}{6.160000in}}%
\pgfusepath{clip}%
\pgfsetbuttcap%
\pgfsetroundjoin%
\definecolor{currentfill}{rgb}{0.200000,0.200000,0.800000}%
\pgfsetfillcolor{currentfill}%
\pgfsetlinewidth{1.003750pt}%
\definecolor{currentstroke}{rgb}{0.200000,0.200000,0.800000}%
\pgfsetstrokecolor{currentstroke}%
\pgfsetdash{}{0pt}%
\pgfpathmoveto{\pgfqpoint{3.938480in}{3.251029in}}%
\pgfpathcurveto{\pgfqpoint{3.944304in}{3.251029in}}{\pgfqpoint{3.949890in}{3.253343in}}{\pgfqpoint{3.954008in}{3.257461in}}%
\pgfpathcurveto{\pgfqpoint{3.958126in}{3.261579in}}{\pgfqpoint{3.960440in}{3.267165in}}{\pgfqpoint{3.960440in}{3.272989in}}%
\pgfpathcurveto{\pgfqpoint{3.960440in}{3.278813in}}{\pgfqpoint{3.958126in}{3.284399in}}{\pgfqpoint{3.954008in}{3.288517in}}%
\pgfpathcurveto{\pgfqpoint{3.949890in}{3.292635in}}{\pgfqpoint{3.944304in}{3.294949in}}{\pgfqpoint{3.938480in}{3.294949in}}%
\pgfpathcurveto{\pgfqpoint{3.932656in}{3.294949in}}{\pgfqpoint{3.927070in}{3.292635in}}{\pgfqpoint{3.922952in}{3.288517in}}%
\pgfpathcurveto{\pgfqpoint{3.918833in}{3.284399in}}{\pgfqpoint{3.916520in}{3.278813in}}{\pgfqpoint{3.916520in}{3.272989in}}%
\pgfpathcurveto{\pgfqpoint{3.916520in}{3.267165in}}{\pgfqpoint{3.918833in}{3.261579in}}{\pgfqpoint{3.922952in}{3.257461in}}%
\pgfpathcurveto{\pgfqpoint{3.927070in}{3.253343in}}{\pgfqpoint{3.932656in}{3.251029in}}{\pgfqpoint{3.938480in}{3.251029in}}%
\pgfpathlineto{\pgfqpoint{3.938480in}{3.251029in}}%
\pgfpathclose%
\pgfusepath{stroke,fill}%
\end{pgfscope}%
\begin{pgfscope}%
\pgfpathrectangle{\pgfqpoint{1.542338in}{0.880000in}}{\pgfqpoint{5.115323in}{6.160000in}}%
\pgfusepath{clip}%
\pgfsetbuttcap%
\pgfsetroundjoin%
\definecolor{currentfill}{rgb}{0.200000,0.200000,0.800000}%
\pgfsetfillcolor{currentfill}%
\pgfsetlinewidth{1.003750pt}%
\definecolor{currentstroke}{rgb}{0.200000,0.200000,0.800000}%
\pgfsetstrokecolor{currentstroke}%
\pgfsetdash{}{0pt}%
\pgfpathmoveto{\pgfqpoint{3.837150in}{3.367103in}}%
\pgfpathcurveto{\pgfqpoint{3.842974in}{3.367103in}}{\pgfqpoint{3.848560in}{3.369416in}}{\pgfqpoint{3.852679in}{3.373535in}}%
\pgfpathcurveto{\pgfqpoint{3.856797in}{3.377653in}}{\pgfqpoint{3.859111in}{3.383239in}}{\pgfqpoint{3.859111in}{3.389063in}}%
\pgfpathcurveto{\pgfqpoint{3.859111in}{3.394887in}}{\pgfqpoint{3.856797in}{3.400473in}}{\pgfqpoint{3.852679in}{3.404591in}}%
\pgfpathcurveto{\pgfqpoint{3.848560in}{3.408709in}}{\pgfqpoint{3.842974in}{3.411023in}}{\pgfqpoint{3.837150in}{3.411023in}}%
\pgfpathcurveto{\pgfqpoint{3.831326in}{3.411023in}}{\pgfqpoint{3.825740in}{3.408709in}}{\pgfqpoint{3.821622in}{3.404591in}}%
\pgfpathcurveto{\pgfqpoint{3.817504in}{3.400473in}}{\pgfqpoint{3.815190in}{3.394887in}}{\pgfqpoint{3.815190in}{3.389063in}}%
\pgfpathcurveto{\pgfqpoint{3.815190in}{3.383239in}}{\pgfqpoint{3.817504in}{3.377653in}}{\pgfqpoint{3.821622in}{3.373535in}}%
\pgfpathcurveto{\pgfqpoint{3.825740in}{3.369416in}}{\pgfqpoint{3.831326in}{3.367103in}}{\pgfqpoint{3.837150in}{3.367103in}}%
\pgfpathlineto{\pgfqpoint{3.837150in}{3.367103in}}%
\pgfpathclose%
\pgfusepath{stroke,fill}%
\end{pgfscope}%
\begin{pgfscope}%
\pgfpathrectangle{\pgfqpoint{1.542338in}{0.880000in}}{\pgfqpoint{5.115323in}{6.160000in}}%
\pgfusepath{clip}%
\pgfsetbuttcap%
\pgfsetroundjoin%
\definecolor{currentfill}{rgb}{0.200000,0.200000,0.800000}%
\pgfsetfillcolor{currentfill}%
\pgfsetlinewidth{1.003750pt}%
\definecolor{currentstroke}{rgb}{0.200000,0.200000,0.800000}%
\pgfsetstrokecolor{currentstroke}%
\pgfsetdash{}{0pt}%
\pgfpathmoveto{\pgfqpoint{3.719259in}{3.465674in}}%
\pgfpathcurveto{\pgfqpoint{3.725083in}{3.465674in}}{\pgfqpoint{3.730669in}{3.467988in}}{\pgfqpoint{3.734787in}{3.472106in}}%
\pgfpathcurveto{\pgfqpoint{3.738905in}{3.476224in}}{\pgfqpoint{3.741219in}{3.481810in}}{\pgfqpoint{3.741219in}{3.487634in}}%
\pgfpathcurveto{\pgfqpoint{3.741219in}{3.493458in}}{\pgfqpoint{3.738905in}{3.499044in}}{\pgfqpoint{3.734787in}{3.503162in}}%
\pgfpathcurveto{\pgfqpoint{3.730669in}{3.507281in}}{\pgfqpoint{3.725083in}{3.509594in}}{\pgfqpoint{3.719259in}{3.509594in}}%
\pgfpathcurveto{\pgfqpoint{3.713435in}{3.509594in}}{\pgfqpoint{3.707849in}{3.507281in}}{\pgfqpoint{3.703731in}{3.503162in}}%
\pgfpathcurveto{\pgfqpoint{3.699613in}{3.499044in}}{\pgfqpoint{3.697299in}{3.493458in}}{\pgfqpoint{3.697299in}{3.487634in}}%
\pgfpathcurveto{\pgfqpoint{3.697299in}{3.481810in}}{\pgfqpoint{3.699613in}{3.476224in}}{\pgfqpoint{3.703731in}{3.472106in}}%
\pgfpathcurveto{\pgfqpoint{3.707849in}{3.467988in}}{\pgfqpoint{3.713435in}{3.465674in}}{\pgfqpoint{3.719259in}{3.465674in}}%
\pgfpathlineto{\pgfqpoint{3.719259in}{3.465674in}}%
\pgfpathclose%
\pgfusepath{stroke,fill}%
\end{pgfscope}%
\begin{pgfscope}%
\pgfpathrectangle{\pgfqpoint{1.542338in}{0.880000in}}{\pgfqpoint{5.115323in}{6.160000in}}%
\pgfusepath{clip}%
\pgfsetbuttcap%
\pgfsetroundjoin%
\definecolor{currentfill}{rgb}{0.200000,0.200000,0.800000}%
\pgfsetfillcolor{currentfill}%
\pgfsetlinewidth{1.003750pt}%
\definecolor{currentstroke}{rgb}{0.200000,0.200000,0.800000}%
\pgfsetstrokecolor{currentstroke}%
\pgfsetdash{}{0pt}%
\pgfpathmoveto{\pgfqpoint{3.596055in}{3.557238in}}%
\pgfpathcurveto{\pgfqpoint{3.601879in}{3.557238in}}{\pgfqpoint{3.607465in}{3.559552in}}{\pgfqpoint{3.611583in}{3.563670in}}%
\pgfpathcurveto{\pgfqpoint{3.615701in}{3.567789in}}{\pgfqpoint{3.618015in}{3.573375in}}{\pgfqpoint{3.618015in}{3.579199in}}%
\pgfpathcurveto{\pgfqpoint{3.618015in}{3.585023in}}{\pgfqpoint{3.615701in}{3.590609in}}{\pgfqpoint{3.611583in}{3.594727in}}%
\pgfpathcurveto{\pgfqpoint{3.607465in}{3.598845in}}{\pgfqpoint{3.601879in}{3.601159in}}{\pgfqpoint{3.596055in}{3.601159in}}%
\pgfpathcurveto{\pgfqpoint{3.590231in}{3.601159in}}{\pgfqpoint{3.584644in}{3.598845in}}{\pgfqpoint{3.580526in}{3.594727in}}%
\pgfpathcurveto{\pgfqpoint{3.576408in}{3.590609in}}{\pgfqpoint{3.574094in}{3.585023in}}{\pgfqpoint{3.574094in}{3.579199in}}%
\pgfpathcurveto{\pgfqpoint{3.574094in}{3.573375in}}{\pgfqpoint{3.576408in}{3.567789in}}{\pgfqpoint{3.580526in}{3.563670in}}%
\pgfpathcurveto{\pgfqpoint{3.584644in}{3.559552in}}{\pgfqpoint{3.590231in}{3.557238in}}{\pgfqpoint{3.596055in}{3.557238in}}%
\pgfpathlineto{\pgfqpoint{3.596055in}{3.557238in}}%
\pgfpathclose%
\pgfusepath{stroke,fill}%
\end{pgfscope}%
\begin{pgfscope}%
\pgfpathrectangle{\pgfqpoint{1.542338in}{0.880000in}}{\pgfqpoint{5.115323in}{6.160000in}}%
\pgfusepath{clip}%
\pgfsetbuttcap%
\pgfsetroundjoin%
\definecolor{currentfill}{rgb}{0.200000,0.200000,0.800000}%
\pgfsetfillcolor{currentfill}%
\pgfsetlinewidth{1.003750pt}%
\definecolor{currentstroke}{rgb}{0.200000,0.200000,0.800000}%
\pgfsetstrokecolor{currentstroke}%
\pgfsetdash{}{0pt}%
\pgfpathmoveto{\pgfqpoint{3.463304in}{3.636358in}}%
\pgfpathcurveto{\pgfqpoint{3.469128in}{3.636358in}}{\pgfqpoint{3.474714in}{3.638671in}}{\pgfqpoint{3.478833in}{3.642790in}}%
\pgfpathcurveto{\pgfqpoint{3.482951in}{3.646908in}}{\pgfqpoint{3.485265in}{3.652494in}}{\pgfqpoint{3.485265in}{3.658318in}}%
\pgfpathcurveto{\pgfqpoint{3.485265in}{3.664142in}}{\pgfqpoint{3.482951in}{3.669728in}}{\pgfqpoint{3.478833in}{3.673846in}}%
\pgfpathcurveto{\pgfqpoint{3.474714in}{3.677964in}}{\pgfqpoint{3.469128in}{3.680278in}}{\pgfqpoint{3.463304in}{3.680278in}}%
\pgfpathcurveto{\pgfqpoint{3.457480in}{3.680278in}}{\pgfqpoint{3.451894in}{3.677964in}}{\pgfqpoint{3.447776in}{3.673846in}}%
\pgfpathcurveto{\pgfqpoint{3.443658in}{3.669728in}}{\pgfqpoint{3.441344in}{3.664142in}}{\pgfqpoint{3.441344in}{3.658318in}}%
\pgfpathcurveto{\pgfqpoint{3.441344in}{3.652494in}}{\pgfqpoint{3.443658in}{3.646908in}}{\pgfqpoint{3.447776in}{3.642790in}}%
\pgfpathcurveto{\pgfqpoint{3.451894in}{3.638671in}}{\pgfqpoint{3.457480in}{3.636358in}}{\pgfqpoint{3.463304in}{3.636358in}}%
\pgfpathlineto{\pgfqpoint{3.463304in}{3.636358in}}%
\pgfpathclose%
\pgfusepath{stroke,fill}%
\end{pgfscope}%
\begin{pgfscope}%
\pgfpathrectangle{\pgfqpoint{1.542338in}{0.880000in}}{\pgfqpoint{5.115323in}{6.160000in}}%
\pgfusepath{clip}%
\pgfsetbuttcap%
\pgfsetroundjoin%
\definecolor{currentfill}{rgb}{0.200000,0.200000,0.800000}%
\pgfsetfillcolor{currentfill}%
\pgfsetlinewidth{1.003750pt}%
\definecolor{currentstroke}{rgb}{0.200000,0.200000,0.800000}%
\pgfsetstrokecolor{currentstroke}%
\pgfsetdash{}{0pt}%
\pgfpathmoveto{\pgfqpoint{3.314676in}{3.678709in}}%
\pgfpathcurveto{\pgfqpoint{3.320500in}{3.678709in}}{\pgfqpoint{3.326086in}{3.681022in}}{\pgfqpoint{3.330204in}{3.685141in}}%
\pgfpathcurveto{\pgfqpoint{3.334322in}{3.689259in}}{\pgfqpoint{3.336636in}{3.694845in}}{\pgfqpoint{3.336636in}{3.700669in}}%
\pgfpathcurveto{\pgfqpoint{3.336636in}{3.706493in}}{\pgfqpoint{3.334322in}{3.712079in}}{\pgfqpoint{3.330204in}{3.716197in}}%
\pgfpathcurveto{\pgfqpoint{3.326086in}{3.720315in}}{\pgfqpoint{3.320500in}{3.722629in}}{\pgfqpoint{3.314676in}{3.722629in}}%
\pgfpathcurveto{\pgfqpoint{3.308852in}{3.722629in}}{\pgfqpoint{3.303266in}{3.720315in}}{\pgfqpoint{3.299148in}{3.716197in}}%
\pgfpathcurveto{\pgfqpoint{3.295030in}{3.712079in}}{\pgfqpoint{3.292716in}{3.706493in}}{\pgfqpoint{3.292716in}{3.700669in}}%
\pgfpathcurveto{\pgfqpoint{3.292716in}{3.694845in}}{\pgfqpoint{3.295030in}{3.689259in}}{\pgfqpoint{3.299148in}{3.685141in}}%
\pgfpathcurveto{\pgfqpoint{3.303266in}{3.681022in}}{\pgfqpoint{3.308852in}{3.678709in}}{\pgfqpoint{3.314676in}{3.678709in}}%
\pgfpathlineto{\pgfqpoint{3.314676in}{3.678709in}}%
\pgfpathclose%
\pgfusepath{stroke,fill}%
\end{pgfscope}%
\begin{pgfscope}%
\pgfpathrectangle{\pgfqpoint{1.542338in}{0.880000in}}{\pgfqpoint{5.115323in}{6.160000in}}%
\pgfusepath{clip}%
\pgfsetbuttcap%
\pgfsetroundjoin%
\definecolor{currentfill}{rgb}{0.200000,0.200000,0.800000}%
\pgfsetfillcolor{currentfill}%
\pgfsetlinewidth{1.003750pt}%
\definecolor{currentstroke}{rgb}{0.200000,0.200000,0.800000}%
\pgfsetstrokecolor{currentstroke}%
\pgfsetdash{}{0pt}%
\pgfpathmoveto{\pgfqpoint{3.164683in}{3.709912in}}%
\pgfpathcurveto{\pgfqpoint{3.170507in}{3.709912in}}{\pgfqpoint{3.176093in}{3.712226in}}{\pgfqpoint{3.180211in}{3.716344in}}%
\pgfpathcurveto{\pgfqpoint{3.184329in}{3.720462in}}{\pgfqpoint{3.186643in}{3.726048in}}{\pgfqpoint{3.186643in}{3.731872in}}%
\pgfpathcurveto{\pgfqpoint{3.186643in}{3.737696in}}{\pgfqpoint{3.184329in}{3.743282in}}{\pgfqpoint{3.180211in}{3.747400in}}%
\pgfpathcurveto{\pgfqpoint{3.176093in}{3.751518in}}{\pgfqpoint{3.170507in}{3.753832in}}{\pgfqpoint{3.164683in}{3.753832in}}%
\pgfpathcurveto{\pgfqpoint{3.158859in}{3.753832in}}{\pgfqpoint{3.153273in}{3.751518in}}{\pgfqpoint{3.149155in}{3.747400in}}%
\pgfpathcurveto{\pgfqpoint{3.145037in}{3.743282in}}{\pgfqpoint{3.142723in}{3.737696in}}{\pgfqpoint{3.142723in}{3.731872in}}%
\pgfpathcurveto{\pgfqpoint{3.142723in}{3.726048in}}{\pgfqpoint{3.145037in}{3.720462in}}{\pgfqpoint{3.149155in}{3.716344in}}%
\pgfpathcurveto{\pgfqpoint{3.153273in}{3.712226in}}{\pgfqpoint{3.158859in}{3.709912in}}{\pgfqpoint{3.164683in}{3.709912in}}%
\pgfpathlineto{\pgfqpoint{3.164683in}{3.709912in}}%
\pgfpathclose%
\pgfusepath{stroke,fill}%
\end{pgfscope}%
\begin{pgfscope}%
\pgfpathrectangle{\pgfqpoint{1.542338in}{0.880000in}}{\pgfqpoint{5.115323in}{6.160000in}}%
\pgfusepath{clip}%
\pgfsetbuttcap%
\pgfsetroundjoin%
\definecolor{currentfill}{rgb}{0.200000,0.200000,0.800000}%
\pgfsetfillcolor{currentfill}%
\pgfsetlinewidth{1.003750pt}%
\definecolor{currentstroke}{rgb}{0.200000,0.200000,0.800000}%
\pgfsetstrokecolor{currentstroke}%
\pgfsetdash{}{0pt}%
\pgfpathmoveto{\pgfqpoint{3.012507in}{3.727832in}}%
\pgfpathcurveto{\pgfqpoint{3.018331in}{3.727832in}}{\pgfqpoint{3.023917in}{3.730146in}}{\pgfqpoint{3.028036in}{3.734264in}}%
\pgfpathcurveto{\pgfqpoint{3.032154in}{3.738382in}}{\pgfqpoint{3.034468in}{3.743968in}}{\pgfqpoint{3.034468in}{3.749792in}}%
\pgfpathcurveto{\pgfqpoint{3.034468in}{3.755616in}}{\pgfqpoint{3.032154in}{3.761202in}}{\pgfqpoint{3.028036in}{3.765321in}}%
\pgfpathcurveto{\pgfqpoint{3.023917in}{3.769439in}}{\pgfqpoint{3.018331in}{3.771753in}}{\pgfqpoint{3.012507in}{3.771753in}}%
\pgfpathcurveto{\pgfqpoint{3.006683in}{3.771753in}}{\pgfqpoint{3.001097in}{3.769439in}}{\pgfqpoint{2.996979in}{3.765321in}}%
\pgfpathcurveto{\pgfqpoint{2.992861in}{3.761202in}}{\pgfqpoint{2.990547in}{3.755616in}}{\pgfqpoint{2.990547in}{3.749792in}}%
\pgfpathcurveto{\pgfqpoint{2.990547in}{3.743968in}}{\pgfqpoint{2.992861in}{3.738382in}}{\pgfqpoint{2.996979in}{3.734264in}}%
\pgfpathcurveto{\pgfqpoint{3.001097in}{3.730146in}}{\pgfqpoint{3.006683in}{3.727832in}}{\pgfqpoint{3.012507in}{3.727832in}}%
\pgfpathlineto{\pgfqpoint{3.012507in}{3.727832in}}%
\pgfpathclose%
\pgfusepath{stroke,fill}%
\end{pgfscope}%
\begin{pgfscope}%
\pgfpathrectangle{\pgfqpoint{1.542338in}{0.880000in}}{\pgfqpoint{5.115323in}{6.160000in}}%
\pgfusepath{clip}%
\pgfsetbuttcap%
\pgfsetroundjoin%
\definecolor{currentfill}{rgb}{0.200000,0.200000,0.800000}%
\pgfsetfillcolor{currentfill}%
\pgfsetlinewidth{1.003750pt}%
\definecolor{currentstroke}{rgb}{0.200000,0.200000,0.800000}%
\pgfsetstrokecolor{currentstroke}%
\pgfsetdash{}{0pt}%
\pgfpathmoveto{\pgfqpoint{2.859157in}{3.723212in}}%
\pgfpathcurveto{\pgfqpoint{2.864981in}{3.723212in}}{\pgfqpoint{2.870568in}{3.725525in}}{\pgfqpoint{2.874686in}{3.729644in}}%
\pgfpathcurveto{\pgfqpoint{2.878804in}{3.733762in}}{\pgfqpoint{2.881118in}{3.739348in}}{\pgfqpoint{2.881118in}{3.745172in}}%
\pgfpathcurveto{\pgfqpoint{2.881118in}{3.750996in}}{\pgfqpoint{2.878804in}{3.756582in}}{\pgfqpoint{2.874686in}{3.760700in}}%
\pgfpathcurveto{\pgfqpoint{2.870568in}{3.764818in}}{\pgfqpoint{2.864981in}{3.767132in}}{\pgfqpoint{2.859157in}{3.767132in}}%
\pgfpathcurveto{\pgfqpoint{2.853333in}{3.767132in}}{\pgfqpoint{2.847747in}{3.764818in}}{\pgfqpoint{2.843629in}{3.760700in}}%
\pgfpathcurveto{\pgfqpoint{2.839511in}{3.756582in}}{\pgfqpoint{2.837197in}{3.750996in}}{\pgfqpoint{2.837197in}{3.745172in}}%
\pgfpathcurveto{\pgfqpoint{2.837197in}{3.739348in}}{\pgfqpoint{2.839511in}{3.733762in}}{\pgfqpoint{2.843629in}{3.729644in}}%
\pgfpathcurveto{\pgfqpoint{2.847747in}{3.725525in}}{\pgfqpoint{2.853333in}{3.723212in}}{\pgfqpoint{2.859157in}{3.723212in}}%
\pgfpathlineto{\pgfqpoint{2.859157in}{3.723212in}}%
\pgfpathclose%
\pgfusepath{stroke,fill}%
\end{pgfscope}%
\begin{pgfscope}%
\pgfpathrectangle{\pgfqpoint{1.542338in}{0.880000in}}{\pgfqpoint{5.115323in}{6.160000in}}%
\pgfusepath{clip}%
\pgfsetbuttcap%
\pgfsetroundjoin%
\definecolor{currentfill}{rgb}{0.200000,0.200000,0.800000}%
\pgfsetfillcolor{currentfill}%
\pgfsetlinewidth{1.003750pt}%
\definecolor{currentstroke}{rgb}{0.200000,0.200000,0.800000}%
\pgfsetstrokecolor{currentstroke}%
\pgfsetdash{}{0pt}%
\pgfpathmoveto{\pgfqpoint{2.708375in}{3.695705in}}%
\pgfpathcurveto{\pgfqpoint{2.714199in}{3.695705in}}{\pgfqpoint{2.719785in}{3.698019in}}{\pgfqpoint{2.723903in}{3.702137in}}%
\pgfpathcurveto{\pgfqpoint{2.728021in}{3.706255in}}{\pgfqpoint{2.730335in}{3.711841in}}{\pgfqpoint{2.730335in}{3.717665in}}%
\pgfpathcurveto{\pgfqpoint{2.730335in}{3.723489in}}{\pgfqpoint{2.728021in}{3.729075in}}{\pgfqpoint{2.723903in}{3.733193in}}%
\pgfpathcurveto{\pgfqpoint{2.719785in}{3.737312in}}{\pgfqpoint{2.714199in}{3.739626in}}{\pgfqpoint{2.708375in}{3.739626in}}%
\pgfpathcurveto{\pgfqpoint{2.702551in}{3.739626in}}{\pgfqpoint{2.696965in}{3.737312in}}{\pgfqpoint{2.692846in}{3.733193in}}%
\pgfpathcurveto{\pgfqpoint{2.688728in}{3.729075in}}{\pgfqpoint{2.686414in}{3.723489in}}{\pgfqpoint{2.686414in}{3.717665in}}%
\pgfpathcurveto{\pgfqpoint{2.686414in}{3.711841in}}{\pgfqpoint{2.688728in}{3.706255in}}{\pgfqpoint{2.692846in}{3.702137in}}%
\pgfpathcurveto{\pgfqpoint{2.696965in}{3.698019in}}{\pgfqpoint{2.702551in}{3.695705in}}{\pgfqpoint{2.708375in}{3.695705in}}%
\pgfpathlineto{\pgfqpoint{2.708375in}{3.695705in}}%
\pgfpathclose%
\pgfusepath{stroke,fill}%
\end{pgfscope}%
\begin{pgfscope}%
\pgfpathrectangle{\pgfqpoint{1.542338in}{0.880000in}}{\pgfqpoint{5.115323in}{6.160000in}}%
\pgfusepath{clip}%
\pgfsetbuttcap%
\pgfsetroundjoin%
\definecolor{currentfill}{rgb}{0.200000,0.200000,0.800000}%
\pgfsetfillcolor{currentfill}%
\pgfsetlinewidth{1.003750pt}%
\definecolor{currentstroke}{rgb}{0.200000,0.200000,0.800000}%
\pgfsetstrokecolor{currentstroke}%
\pgfsetdash{}{0pt}%
\pgfpathmoveto{\pgfqpoint{2.560319in}{3.655800in}}%
\pgfpathcurveto{\pgfqpoint{2.566143in}{3.655800in}}{\pgfqpoint{2.571729in}{3.658114in}}{\pgfqpoint{2.575847in}{3.662232in}}%
\pgfpathcurveto{\pgfqpoint{2.579965in}{3.666350in}}{\pgfqpoint{2.582279in}{3.671937in}}{\pgfqpoint{2.582279in}{3.677761in}}%
\pgfpathcurveto{\pgfqpoint{2.582279in}{3.683585in}}{\pgfqpoint{2.579965in}{3.689171in}}{\pgfqpoint{2.575847in}{3.693289in}}%
\pgfpathcurveto{\pgfqpoint{2.571729in}{3.697407in}}{\pgfqpoint{2.566143in}{3.699721in}}{\pgfqpoint{2.560319in}{3.699721in}}%
\pgfpathcurveto{\pgfqpoint{2.554495in}{3.699721in}}{\pgfqpoint{2.548909in}{3.697407in}}{\pgfqpoint{2.544791in}{3.693289in}}%
\pgfpathcurveto{\pgfqpoint{2.540673in}{3.689171in}}{\pgfqpoint{2.538359in}{3.683585in}}{\pgfqpoint{2.538359in}{3.677761in}}%
\pgfpathcurveto{\pgfqpoint{2.538359in}{3.671937in}}{\pgfqpoint{2.540673in}{3.666350in}}{\pgfqpoint{2.544791in}{3.662232in}}%
\pgfpathcurveto{\pgfqpoint{2.548909in}{3.658114in}}{\pgfqpoint{2.554495in}{3.655800in}}{\pgfqpoint{2.560319in}{3.655800in}}%
\pgfpathlineto{\pgfqpoint{2.560319in}{3.655800in}}%
\pgfpathclose%
\pgfusepath{stroke,fill}%
\end{pgfscope}%
\begin{pgfscope}%
\pgfpathrectangle{\pgfqpoint{1.542338in}{0.880000in}}{\pgfqpoint{5.115323in}{6.160000in}}%
\pgfusepath{clip}%
\pgfsetbuttcap%
\pgfsetroundjoin%
\definecolor{currentfill}{rgb}{0.200000,0.200000,0.800000}%
\pgfsetfillcolor{currentfill}%
\pgfsetlinewidth{1.003750pt}%
\definecolor{currentstroke}{rgb}{0.200000,0.200000,0.800000}%
\pgfsetstrokecolor{currentstroke}%
\pgfsetdash{}{0pt}%
\pgfpathmoveto{\pgfqpoint{2.423070in}{3.587614in}}%
\pgfpathcurveto{\pgfqpoint{2.428894in}{3.587614in}}{\pgfqpoint{2.434481in}{3.589928in}}{\pgfqpoint{2.438599in}{3.594046in}}%
\pgfpathcurveto{\pgfqpoint{2.442717in}{3.598164in}}{\pgfqpoint{2.445031in}{3.603750in}}{\pgfqpoint{2.445031in}{3.609574in}}%
\pgfpathcurveto{\pgfqpoint{2.445031in}{3.615398in}}{\pgfqpoint{2.442717in}{3.620984in}}{\pgfqpoint{2.438599in}{3.625102in}}%
\pgfpathcurveto{\pgfqpoint{2.434481in}{3.629221in}}{\pgfqpoint{2.428894in}{3.631534in}}{\pgfqpoint{2.423070in}{3.631534in}}%
\pgfpathcurveto{\pgfqpoint{2.417247in}{3.631534in}}{\pgfqpoint{2.411660in}{3.629221in}}{\pgfqpoint{2.407542in}{3.625102in}}%
\pgfpathcurveto{\pgfqpoint{2.403424in}{3.620984in}}{\pgfqpoint{2.401110in}{3.615398in}}{\pgfqpoint{2.401110in}{3.609574in}}%
\pgfpathcurveto{\pgfqpoint{2.401110in}{3.603750in}}{\pgfqpoint{2.403424in}{3.598164in}}{\pgfqpoint{2.407542in}{3.594046in}}%
\pgfpathcurveto{\pgfqpoint{2.411660in}{3.589928in}}{\pgfqpoint{2.417247in}{3.587614in}}{\pgfqpoint{2.423070in}{3.587614in}}%
\pgfpathlineto{\pgfqpoint{2.423070in}{3.587614in}}%
\pgfpathclose%
\pgfusepath{stroke,fill}%
\end{pgfscope}%
\begin{pgfscope}%
\pgfpathrectangle{\pgfqpoint{1.542338in}{0.880000in}}{\pgfqpoint{5.115323in}{6.160000in}}%
\pgfusepath{clip}%
\pgfsetbuttcap%
\pgfsetroundjoin%
\definecolor{currentfill}{rgb}{0.200000,0.200000,0.800000}%
\pgfsetfillcolor{currentfill}%
\pgfsetlinewidth{1.003750pt}%
\definecolor{currentstroke}{rgb}{0.200000,0.200000,0.800000}%
\pgfsetstrokecolor{currentstroke}%
\pgfsetdash{}{0pt}%
\pgfpathmoveto{\pgfqpoint{2.291167in}{3.510412in}}%
\pgfpathcurveto{\pgfqpoint{2.296990in}{3.510412in}}{\pgfqpoint{2.302577in}{3.512726in}}{\pgfqpoint{2.306695in}{3.516844in}}%
\pgfpathcurveto{\pgfqpoint{2.310813in}{3.520962in}}{\pgfqpoint{2.313127in}{3.526548in}}{\pgfqpoint{2.313127in}{3.532372in}}%
\pgfpathcurveto{\pgfqpoint{2.313127in}{3.538196in}}{\pgfqpoint{2.310813in}{3.543782in}}{\pgfqpoint{2.306695in}{3.547900in}}%
\pgfpathcurveto{\pgfqpoint{2.302577in}{3.552018in}}{\pgfqpoint{2.296990in}{3.554332in}}{\pgfqpoint{2.291167in}{3.554332in}}%
\pgfpathcurveto{\pgfqpoint{2.285343in}{3.554332in}}{\pgfqpoint{2.279756in}{3.552018in}}{\pgfqpoint{2.275638in}{3.547900in}}%
\pgfpathcurveto{\pgfqpoint{2.271520in}{3.543782in}}{\pgfqpoint{2.269206in}{3.538196in}}{\pgfqpoint{2.269206in}{3.532372in}}%
\pgfpathcurveto{\pgfqpoint{2.269206in}{3.526548in}}{\pgfqpoint{2.271520in}{3.520962in}}{\pgfqpoint{2.275638in}{3.516844in}}%
\pgfpathcurveto{\pgfqpoint{2.279756in}{3.512726in}}{\pgfqpoint{2.285343in}{3.510412in}}{\pgfqpoint{2.291167in}{3.510412in}}%
\pgfpathlineto{\pgfqpoint{2.291167in}{3.510412in}}%
\pgfpathclose%
\pgfusepath{stroke,fill}%
\end{pgfscope}%
\begin{pgfscope}%
\pgfpathrectangle{\pgfqpoint{1.542338in}{0.880000in}}{\pgfqpoint{5.115323in}{6.160000in}}%
\pgfusepath{clip}%
\pgfsetbuttcap%
\pgfsetroundjoin%
\definecolor{currentfill}{rgb}{0.200000,0.200000,0.800000}%
\pgfsetfillcolor{currentfill}%
\pgfsetlinewidth{1.003750pt}%
\definecolor{currentstroke}{rgb}{0.200000,0.200000,0.800000}%
\pgfsetstrokecolor{currentstroke}%
\pgfsetdash{}{0pt}%
\pgfpathmoveto{\pgfqpoint{2.167728in}{3.419268in}}%
\pgfpathcurveto{\pgfqpoint{2.173552in}{3.419268in}}{\pgfqpoint{2.179138in}{3.421582in}}{\pgfqpoint{2.183256in}{3.425700in}}%
\pgfpathcurveto{\pgfqpoint{2.187374in}{3.429818in}}{\pgfqpoint{2.189688in}{3.435405in}}{\pgfqpoint{2.189688in}{3.441228in}}%
\pgfpathcurveto{\pgfqpoint{2.189688in}{3.447052in}}{\pgfqpoint{2.187374in}{3.452639in}}{\pgfqpoint{2.183256in}{3.456757in}}%
\pgfpathcurveto{\pgfqpoint{2.179138in}{3.460875in}}{\pgfqpoint{2.173552in}{3.463189in}}{\pgfqpoint{2.167728in}{3.463189in}}%
\pgfpathcurveto{\pgfqpoint{2.161904in}{3.463189in}}{\pgfqpoint{2.156318in}{3.460875in}}{\pgfqpoint{2.152200in}{3.456757in}}%
\pgfpathcurveto{\pgfqpoint{2.148081in}{3.452639in}}{\pgfqpoint{2.145768in}{3.447052in}}{\pgfqpoint{2.145768in}{3.441228in}}%
\pgfpathcurveto{\pgfqpoint{2.145768in}{3.435405in}}{\pgfqpoint{2.148081in}{3.429818in}}{\pgfqpoint{2.152200in}{3.425700in}}%
\pgfpathcurveto{\pgfqpoint{2.156318in}{3.421582in}}{\pgfqpoint{2.161904in}{3.419268in}}{\pgfqpoint{2.167728in}{3.419268in}}%
\pgfpathlineto{\pgfqpoint{2.167728in}{3.419268in}}%
\pgfpathclose%
\pgfusepath{stroke,fill}%
\end{pgfscope}%
\begin{pgfscope}%
\pgfpathrectangle{\pgfqpoint{1.542338in}{0.880000in}}{\pgfqpoint{5.115323in}{6.160000in}}%
\pgfusepath{clip}%
\pgfsetbuttcap%
\pgfsetroundjoin%
\definecolor{currentfill}{rgb}{0.200000,0.200000,0.800000}%
\pgfsetfillcolor{currentfill}%
\pgfsetlinewidth{1.003750pt}%
\definecolor{currentstroke}{rgb}{0.200000,0.200000,0.800000}%
\pgfsetstrokecolor{currentstroke}%
\pgfsetdash{}{0pt}%
\pgfpathmoveto{\pgfqpoint{2.068482in}{3.302317in}}%
\pgfpathcurveto{\pgfqpoint{2.074306in}{3.302317in}}{\pgfqpoint{2.079892in}{3.304631in}}{\pgfqpoint{2.084011in}{3.308749in}}%
\pgfpathcurveto{\pgfqpoint{2.088129in}{3.312867in}}{\pgfqpoint{2.090443in}{3.318453in}}{\pgfqpoint{2.090443in}{3.324277in}}%
\pgfpathcurveto{\pgfqpoint{2.090443in}{3.330101in}}{\pgfqpoint{2.088129in}{3.335687in}}{\pgfqpoint{2.084011in}{3.339806in}}%
\pgfpathcurveto{\pgfqpoint{2.079892in}{3.343924in}}{\pgfqpoint{2.074306in}{3.346238in}}{\pgfqpoint{2.068482in}{3.346238in}}%
\pgfpathcurveto{\pgfqpoint{2.062658in}{3.346238in}}{\pgfqpoint{2.057072in}{3.343924in}}{\pgfqpoint{2.052954in}{3.339806in}}%
\pgfpathcurveto{\pgfqpoint{2.048836in}{3.335687in}}{\pgfqpoint{2.046522in}{3.330101in}}{\pgfqpoint{2.046522in}{3.324277in}}%
\pgfpathcurveto{\pgfqpoint{2.046522in}{3.318453in}}{\pgfqpoint{2.048836in}{3.312867in}}{\pgfqpoint{2.052954in}{3.308749in}}%
\pgfpathcurveto{\pgfqpoint{2.057072in}{3.304631in}}{\pgfqpoint{2.062658in}{3.302317in}}{\pgfqpoint{2.068482in}{3.302317in}}%
\pgfpathlineto{\pgfqpoint{2.068482in}{3.302317in}}%
\pgfpathclose%
\pgfusepath{stroke,fill}%
\end{pgfscope}%
\begin{pgfscope}%
\pgfpathrectangle{\pgfqpoint{1.542338in}{0.880000in}}{\pgfqpoint{5.115323in}{6.160000in}}%
\pgfusepath{clip}%
\pgfsetbuttcap%
\pgfsetroundjoin%
\definecolor{currentfill}{rgb}{0.200000,0.200000,0.800000}%
\pgfsetfillcolor{currentfill}%
\pgfsetlinewidth{1.003750pt}%
\definecolor{currentstroke}{rgb}{0.200000,0.200000,0.800000}%
\pgfsetstrokecolor{currentstroke}%
\pgfsetdash{}{0pt}%
\pgfpathmoveto{\pgfqpoint{1.968382in}{3.185996in}}%
\pgfpathcurveto{\pgfqpoint{1.974206in}{3.185996in}}{\pgfqpoint{1.979792in}{3.188309in}}{\pgfqpoint{1.983910in}{3.192428in}}%
\pgfpathcurveto{\pgfqpoint{1.988029in}{3.196546in}}{\pgfqpoint{1.990342in}{3.202132in}}{\pgfqpoint{1.990342in}{3.207956in}}%
\pgfpathcurveto{\pgfqpoint{1.990342in}{3.213780in}}{\pgfqpoint{1.988029in}{3.219366in}}{\pgfqpoint{1.983910in}{3.223484in}}%
\pgfpathcurveto{\pgfqpoint{1.979792in}{3.227602in}}{\pgfqpoint{1.974206in}{3.229916in}}{\pgfqpoint{1.968382in}{3.229916in}}%
\pgfpathcurveto{\pgfqpoint{1.962558in}{3.229916in}}{\pgfqpoint{1.956972in}{3.227602in}}{\pgfqpoint{1.952854in}{3.223484in}}%
\pgfpathcurveto{\pgfqpoint{1.948736in}{3.219366in}}{\pgfqpoint{1.946422in}{3.213780in}}{\pgfqpoint{1.946422in}{3.207956in}}%
\pgfpathcurveto{\pgfqpoint{1.946422in}{3.202132in}}{\pgfqpoint{1.948736in}{3.196546in}}{\pgfqpoint{1.952854in}{3.192428in}}%
\pgfpathcurveto{\pgfqpoint{1.956972in}{3.188309in}}{\pgfqpoint{1.962558in}{3.185996in}}{\pgfqpoint{1.968382in}{3.185996in}}%
\pgfpathlineto{\pgfqpoint{1.968382in}{3.185996in}}%
\pgfpathclose%
\pgfusepath{stroke,fill}%
\end{pgfscope}%
\begin{pgfscope}%
\pgfpathrectangle{\pgfqpoint{1.542338in}{0.880000in}}{\pgfqpoint{5.115323in}{6.160000in}}%
\pgfusepath{clip}%
\pgfsetbuttcap%
\pgfsetroundjoin%
\definecolor{currentfill}{rgb}{0.200000,0.200000,0.800000}%
\pgfsetfillcolor{currentfill}%
\pgfsetlinewidth{1.003750pt}%
\definecolor{currentstroke}{rgb}{0.200000,0.200000,0.800000}%
\pgfsetstrokecolor{currentstroke}%
\pgfsetdash{}{0pt}%
\pgfpathmoveto{\pgfqpoint{1.903465in}{3.046930in}}%
\pgfpathcurveto{\pgfqpoint{1.909289in}{3.046930in}}{\pgfqpoint{1.914875in}{3.049244in}}{\pgfqpoint{1.918993in}{3.053362in}}%
\pgfpathcurveto{\pgfqpoint{1.923112in}{3.057480in}}{\pgfqpoint{1.925425in}{3.063066in}}{\pgfqpoint{1.925425in}{3.068890in}}%
\pgfpathcurveto{\pgfqpoint{1.925425in}{3.074714in}}{\pgfqpoint{1.923112in}{3.080300in}}{\pgfqpoint{1.918993in}{3.084419in}}%
\pgfpathcurveto{\pgfqpoint{1.914875in}{3.088537in}}{\pgfqpoint{1.909289in}{3.090851in}}{\pgfqpoint{1.903465in}{3.090851in}}%
\pgfpathcurveto{\pgfqpoint{1.897641in}{3.090851in}}{\pgfqpoint{1.892055in}{3.088537in}}{\pgfqpoint{1.887937in}{3.084419in}}%
\pgfpathcurveto{\pgfqpoint{1.883819in}{3.080300in}}{\pgfqpoint{1.881505in}{3.074714in}}{\pgfqpoint{1.881505in}{3.068890in}}%
\pgfpathcurveto{\pgfqpoint{1.881505in}{3.063066in}}{\pgfqpoint{1.883819in}{3.057480in}}{\pgfqpoint{1.887937in}{3.053362in}}%
\pgfpathcurveto{\pgfqpoint{1.892055in}{3.049244in}}{\pgfqpoint{1.897641in}{3.046930in}}{\pgfqpoint{1.903465in}{3.046930in}}%
\pgfpathlineto{\pgfqpoint{1.903465in}{3.046930in}}%
\pgfpathclose%
\pgfusepath{stroke,fill}%
\end{pgfscope}%
\begin{pgfscope}%
\pgfpathrectangle{\pgfqpoint{1.542338in}{0.880000in}}{\pgfqpoint{5.115323in}{6.160000in}}%
\pgfusepath{clip}%
\pgfsetbuttcap%
\pgfsetroundjoin%
\definecolor{currentfill}{rgb}{0.200000,0.200000,0.800000}%
\pgfsetfillcolor{currentfill}%
\pgfsetlinewidth{1.003750pt}%
\definecolor{currentstroke}{rgb}{0.200000,0.200000,0.800000}%
\pgfsetstrokecolor{currentstroke}%
\pgfsetdash{}{0pt}%
\pgfpathmoveto{\pgfqpoint{1.845252in}{2.906137in}}%
\pgfpathcurveto{\pgfqpoint{1.851075in}{2.906137in}}{\pgfqpoint{1.856662in}{2.908451in}}{\pgfqpoint{1.860780in}{2.912569in}}%
\pgfpathcurveto{\pgfqpoint{1.864898in}{2.916687in}}{\pgfqpoint{1.867212in}{2.922274in}}{\pgfqpoint{1.867212in}{2.928098in}}%
\pgfpathcurveto{\pgfqpoint{1.867212in}{2.933922in}}{\pgfqpoint{1.864898in}{2.939508in}}{\pgfqpoint{1.860780in}{2.943626in}}%
\pgfpathcurveto{\pgfqpoint{1.856662in}{2.947744in}}{\pgfqpoint{1.851075in}{2.950058in}}{\pgfqpoint{1.845252in}{2.950058in}}%
\pgfpathcurveto{\pgfqpoint{1.839428in}{2.950058in}}{\pgfqpoint{1.833841in}{2.947744in}}{\pgfqpoint{1.829723in}{2.943626in}}%
\pgfpathcurveto{\pgfqpoint{1.825605in}{2.939508in}}{\pgfqpoint{1.823291in}{2.933922in}}{\pgfqpoint{1.823291in}{2.928098in}}%
\pgfpathcurveto{\pgfqpoint{1.823291in}{2.922274in}}{\pgfqpoint{1.825605in}{2.916687in}}{\pgfqpoint{1.829723in}{2.912569in}}%
\pgfpathcurveto{\pgfqpoint{1.833841in}{2.908451in}}{\pgfqpoint{1.839428in}{2.906137in}}{\pgfqpoint{1.845252in}{2.906137in}}%
\pgfpathlineto{\pgfqpoint{1.845252in}{2.906137in}}%
\pgfpathclose%
\pgfusepath{stroke,fill}%
\end{pgfscope}%
\begin{pgfscope}%
\pgfpathrectangle{\pgfqpoint{1.542338in}{0.880000in}}{\pgfqpoint{5.115323in}{6.160000in}}%
\pgfusepath{clip}%
\pgfsetbuttcap%
\pgfsetroundjoin%
\definecolor{currentfill}{rgb}{0.200000,0.200000,0.800000}%
\pgfsetfillcolor{currentfill}%
\pgfsetlinewidth{1.003750pt}%
\definecolor{currentstroke}{rgb}{0.200000,0.200000,0.800000}%
\pgfsetstrokecolor{currentstroke}%
\pgfsetdash{}{0pt}%
\pgfpathmoveto{\pgfqpoint{1.801733in}{2.759647in}}%
\pgfpathcurveto{\pgfqpoint{1.807557in}{2.759647in}}{\pgfqpoint{1.813144in}{2.761961in}}{\pgfqpoint{1.817262in}{2.766079in}}%
\pgfpathcurveto{\pgfqpoint{1.821380in}{2.770197in}}{\pgfqpoint{1.823694in}{2.775783in}}{\pgfqpoint{1.823694in}{2.781607in}}%
\pgfpathcurveto{\pgfqpoint{1.823694in}{2.787431in}}{\pgfqpoint{1.821380in}{2.793017in}}{\pgfqpoint{1.817262in}{2.797136in}}%
\pgfpathcurveto{\pgfqpoint{1.813144in}{2.801254in}}{\pgfqpoint{1.807557in}{2.803568in}}{\pgfqpoint{1.801733in}{2.803568in}}%
\pgfpathcurveto{\pgfqpoint{1.795909in}{2.803568in}}{\pgfqpoint{1.790323in}{2.801254in}}{\pgfqpoint{1.786205in}{2.797136in}}%
\pgfpathcurveto{\pgfqpoint{1.782087in}{2.793017in}}{\pgfqpoint{1.779773in}{2.787431in}}{\pgfqpoint{1.779773in}{2.781607in}}%
\pgfpathcurveto{\pgfqpoint{1.779773in}{2.775783in}}{\pgfqpoint{1.782087in}{2.770197in}}{\pgfqpoint{1.786205in}{2.766079in}}%
\pgfpathcurveto{\pgfqpoint{1.790323in}{2.761961in}}{\pgfqpoint{1.795909in}{2.759647in}}{\pgfqpoint{1.801733in}{2.759647in}}%
\pgfpathlineto{\pgfqpoint{1.801733in}{2.759647in}}%
\pgfpathclose%
\pgfusepath{stroke,fill}%
\end{pgfscope}%
\begin{pgfscope}%
\pgfpathrectangle{\pgfqpoint{1.542338in}{0.880000in}}{\pgfqpoint{5.115323in}{6.160000in}}%
\pgfusepath{clip}%
\pgfsetbuttcap%
\pgfsetroundjoin%
\definecolor{currentfill}{rgb}{0.200000,0.200000,0.800000}%
\pgfsetfillcolor{currentfill}%
\pgfsetlinewidth{1.003750pt}%
\definecolor{currentstroke}{rgb}{0.200000,0.200000,0.800000}%
\pgfsetstrokecolor{currentstroke}%
\pgfsetdash{}{0pt}%
\pgfpathmoveto{\pgfqpoint{1.779013in}{2.608051in}}%
\pgfpathcurveto{\pgfqpoint{1.784837in}{2.608051in}}{\pgfqpoint{1.790423in}{2.610365in}}{\pgfqpoint{1.794541in}{2.614483in}}%
\pgfpathcurveto{\pgfqpoint{1.798659in}{2.618601in}}{\pgfqpoint{1.800973in}{2.624187in}}{\pgfqpoint{1.800973in}{2.630011in}}%
\pgfpathcurveto{\pgfqpoint{1.800973in}{2.635835in}}{\pgfqpoint{1.798659in}{2.641421in}}{\pgfqpoint{1.794541in}{2.645539in}}%
\pgfpathcurveto{\pgfqpoint{1.790423in}{2.649658in}}{\pgfqpoint{1.784837in}{2.651971in}}{\pgfqpoint{1.779013in}{2.651971in}}%
\pgfpathcurveto{\pgfqpoint{1.773189in}{2.651971in}}{\pgfqpoint{1.767603in}{2.649658in}}{\pgfqpoint{1.763485in}{2.645539in}}%
\pgfpathcurveto{\pgfqpoint{1.759367in}{2.641421in}}{\pgfqpoint{1.757053in}{2.635835in}}{\pgfqpoint{1.757053in}{2.630011in}}%
\pgfpathcurveto{\pgfqpoint{1.757053in}{2.624187in}}{\pgfqpoint{1.759367in}{2.618601in}}{\pgfqpoint{1.763485in}{2.614483in}}%
\pgfpathcurveto{\pgfqpoint{1.767603in}{2.610365in}}{\pgfqpoint{1.773189in}{2.608051in}}{\pgfqpoint{1.779013in}{2.608051in}}%
\pgfpathlineto{\pgfqpoint{1.779013in}{2.608051in}}%
\pgfpathclose%
\pgfusepath{stroke,fill}%
\end{pgfscope}%
\begin{pgfscope}%
\pgfpathrectangle{\pgfqpoint{1.542338in}{0.880000in}}{\pgfqpoint{5.115323in}{6.160000in}}%
\pgfusepath{clip}%
\pgfsetbuttcap%
\pgfsetroundjoin%
\definecolor{currentfill}{rgb}{0.200000,0.200000,0.800000}%
\pgfsetfillcolor{currentfill}%
\pgfsetlinewidth{1.003750pt}%
\definecolor{currentstroke}{rgb}{0.200000,0.200000,0.800000}%
\pgfsetstrokecolor{currentstroke}%
\pgfsetdash{}{0pt}%
\pgfpathmoveto{\pgfqpoint{1.774853in}{2.454325in}}%
\pgfpathcurveto{\pgfqpoint{1.780677in}{2.454325in}}{\pgfqpoint{1.786263in}{2.456639in}}{\pgfqpoint{1.790381in}{2.460757in}}%
\pgfpathcurveto{\pgfqpoint{1.794499in}{2.464875in}}{\pgfqpoint{1.796813in}{2.470461in}}{\pgfqpoint{1.796813in}{2.476285in}}%
\pgfpathcurveto{\pgfqpoint{1.796813in}{2.482109in}}{\pgfqpoint{1.794499in}{2.487695in}}{\pgfqpoint{1.790381in}{2.491813in}}%
\pgfpathcurveto{\pgfqpoint{1.786263in}{2.495931in}}{\pgfqpoint{1.780677in}{2.498245in}}{\pgfqpoint{1.774853in}{2.498245in}}%
\pgfpathcurveto{\pgfqpoint{1.769029in}{2.498245in}}{\pgfqpoint{1.763443in}{2.495931in}}{\pgfqpoint{1.759325in}{2.491813in}}%
\pgfpathcurveto{\pgfqpoint{1.755207in}{2.487695in}}{\pgfqpoint{1.752893in}{2.482109in}}{\pgfqpoint{1.752893in}{2.476285in}}%
\pgfpathcurveto{\pgfqpoint{1.752893in}{2.470461in}}{\pgfqpoint{1.755207in}{2.464875in}}{\pgfqpoint{1.759325in}{2.460757in}}%
\pgfpathcurveto{\pgfqpoint{1.763443in}{2.456639in}}{\pgfqpoint{1.769029in}{2.454325in}}{\pgfqpoint{1.774853in}{2.454325in}}%
\pgfpathlineto{\pgfqpoint{1.774853in}{2.454325in}}%
\pgfpathclose%
\pgfusepath{stroke,fill}%
\end{pgfscope}%
\begin{pgfscope}%
\pgfpathrectangle{\pgfqpoint{1.542338in}{0.880000in}}{\pgfqpoint{5.115323in}{6.160000in}}%
\pgfusepath{clip}%
\pgfsetbuttcap%
\pgfsetroundjoin%
\definecolor{currentfill}{rgb}{0.200000,0.200000,0.800000}%
\pgfsetfillcolor{currentfill}%
\pgfsetlinewidth{1.003750pt}%
\definecolor{currentstroke}{rgb}{0.200000,0.200000,0.800000}%
\pgfsetstrokecolor{currentstroke}%
\pgfsetdash{}{0pt}%
\pgfpathmoveto{\pgfqpoint{1.799540in}{2.302569in}}%
\pgfpathcurveto{\pgfqpoint{1.805364in}{2.302569in}}{\pgfqpoint{1.810950in}{2.304882in}}{\pgfqpoint{1.815068in}{2.309001in}}%
\pgfpathcurveto{\pgfqpoint{1.819187in}{2.313119in}}{\pgfqpoint{1.821500in}{2.318705in}}{\pgfqpoint{1.821500in}{2.324529in}}%
\pgfpathcurveto{\pgfqpoint{1.821500in}{2.330353in}}{\pgfqpoint{1.819187in}{2.335939in}}{\pgfqpoint{1.815068in}{2.340057in}}%
\pgfpathcurveto{\pgfqpoint{1.810950in}{2.344175in}}{\pgfqpoint{1.805364in}{2.346489in}}{\pgfqpoint{1.799540in}{2.346489in}}%
\pgfpathcurveto{\pgfqpoint{1.793716in}{2.346489in}}{\pgfqpoint{1.788130in}{2.344175in}}{\pgfqpoint{1.784012in}{2.340057in}}%
\pgfpathcurveto{\pgfqpoint{1.779894in}{2.335939in}}{\pgfqpoint{1.777580in}{2.330353in}}{\pgfqpoint{1.777580in}{2.324529in}}%
\pgfpathcurveto{\pgfqpoint{1.777580in}{2.318705in}}{\pgfqpoint{1.779894in}{2.313119in}}{\pgfqpoint{1.784012in}{2.309001in}}%
\pgfpathcurveto{\pgfqpoint{1.788130in}{2.304882in}}{\pgfqpoint{1.793716in}{2.302569in}}{\pgfqpoint{1.799540in}{2.302569in}}%
\pgfpathlineto{\pgfqpoint{1.799540in}{2.302569in}}%
\pgfpathclose%
\pgfusepath{stroke,fill}%
\end{pgfscope}%
\begin{pgfscope}%
\pgfpathrectangle{\pgfqpoint{1.542338in}{0.880000in}}{\pgfqpoint{5.115323in}{6.160000in}}%
\pgfusepath{clip}%
\pgfsetbuttcap%
\pgfsetroundjoin%
\definecolor{currentfill}{rgb}{0.200000,0.200000,0.800000}%
\pgfsetfillcolor{currentfill}%
\pgfsetlinewidth{1.003750pt}%
\definecolor{currentstroke}{rgb}{0.200000,0.200000,0.800000}%
\pgfsetstrokecolor{currentstroke}%
\pgfsetdash{}{0pt}%
\pgfpathmoveto{\pgfqpoint{1.839595in}{2.154627in}}%
\pgfpathcurveto{\pgfqpoint{1.845419in}{2.154627in}}{\pgfqpoint{1.851005in}{2.156941in}}{\pgfqpoint{1.855123in}{2.161059in}}%
\pgfpathcurveto{\pgfqpoint{1.859241in}{2.165177in}}{\pgfqpoint{1.861555in}{2.170764in}}{\pgfqpoint{1.861555in}{2.176587in}}%
\pgfpathcurveto{\pgfqpoint{1.861555in}{2.182411in}}{\pgfqpoint{1.859241in}{2.187998in}}{\pgfqpoint{1.855123in}{2.192116in}}%
\pgfpathcurveto{\pgfqpoint{1.851005in}{2.196234in}}{\pgfqpoint{1.845419in}{2.198548in}}{\pgfqpoint{1.839595in}{2.198548in}}%
\pgfpathcurveto{\pgfqpoint{1.833771in}{2.198548in}}{\pgfqpoint{1.828185in}{2.196234in}}{\pgfqpoint{1.824066in}{2.192116in}}%
\pgfpathcurveto{\pgfqpoint{1.819948in}{2.187998in}}{\pgfqpoint{1.817634in}{2.182411in}}{\pgfqpoint{1.817634in}{2.176587in}}%
\pgfpathcurveto{\pgfqpoint{1.817634in}{2.170764in}}{\pgfqpoint{1.819948in}{2.165177in}}{\pgfqpoint{1.824066in}{2.161059in}}%
\pgfpathcurveto{\pgfqpoint{1.828185in}{2.156941in}}{\pgfqpoint{1.833771in}{2.154627in}}{\pgfqpoint{1.839595in}{2.154627in}}%
\pgfpathlineto{\pgfqpoint{1.839595in}{2.154627in}}%
\pgfpathclose%
\pgfusepath{stroke,fill}%
\end{pgfscope}%
\begin{pgfscope}%
\pgfpathrectangle{\pgfqpoint{1.542338in}{0.880000in}}{\pgfqpoint{5.115323in}{6.160000in}}%
\pgfusepath{clip}%
\pgfsetbuttcap%
\pgfsetroundjoin%
\definecolor{currentfill}{rgb}{0.200000,0.200000,0.800000}%
\pgfsetfillcolor{currentfill}%
\pgfsetlinewidth{1.003750pt}%
\definecolor{currentstroke}{rgb}{0.200000,0.200000,0.800000}%
\pgfsetstrokecolor{currentstroke}%
\pgfsetdash{}{0pt}%
\pgfpathmoveto{\pgfqpoint{1.894092in}{2.011199in}}%
\pgfpathcurveto{\pgfqpoint{1.899916in}{2.011199in}}{\pgfqpoint{1.905502in}{2.013513in}}{\pgfqpoint{1.909620in}{2.017631in}}%
\pgfpathcurveto{\pgfqpoint{1.913738in}{2.021749in}}{\pgfqpoint{1.916052in}{2.027335in}}{\pgfqpoint{1.916052in}{2.033159in}}%
\pgfpathcurveto{\pgfqpoint{1.916052in}{2.038983in}}{\pgfqpoint{1.913738in}{2.044569in}}{\pgfqpoint{1.909620in}{2.048687in}}%
\pgfpathcurveto{\pgfqpoint{1.905502in}{2.052805in}}{\pgfqpoint{1.899916in}{2.055119in}}{\pgfqpoint{1.894092in}{2.055119in}}%
\pgfpathcurveto{\pgfqpoint{1.888268in}{2.055119in}}{\pgfqpoint{1.882682in}{2.052805in}}{\pgfqpoint{1.878564in}{2.048687in}}%
\pgfpathcurveto{\pgfqpoint{1.874446in}{2.044569in}}{\pgfqpoint{1.872132in}{2.038983in}}{\pgfqpoint{1.872132in}{2.033159in}}%
\pgfpathcurveto{\pgfqpoint{1.872132in}{2.027335in}}{\pgfqpoint{1.874446in}{2.021749in}}{\pgfqpoint{1.878564in}{2.017631in}}%
\pgfpathcurveto{\pgfqpoint{1.882682in}{2.013513in}}{\pgfqpoint{1.888268in}{2.011199in}}{\pgfqpoint{1.894092in}{2.011199in}}%
\pgfpathlineto{\pgfqpoint{1.894092in}{2.011199in}}%
\pgfpathclose%
\pgfusepath{stroke,fill}%
\end{pgfscope}%
\begin{pgfscope}%
\pgfpathrectangle{\pgfqpoint{1.542338in}{0.880000in}}{\pgfqpoint{5.115323in}{6.160000in}}%
\pgfusepath{clip}%
\pgfsetbuttcap%
\pgfsetroundjoin%
\definecolor{currentfill}{rgb}{0.200000,0.200000,0.800000}%
\pgfsetfillcolor{currentfill}%
\pgfsetlinewidth{1.003750pt}%
\definecolor{currentstroke}{rgb}{0.200000,0.200000,0.800000}%
\pgfsetstrokecolor{currentstroke}%
\pgfsetdash{}{0pt}%
\pgfpathmoveto{\pgfqpoint{1.974351in}{1.880532in}}%
\pgfpathcurveto{\pgfqpoint{1.980175in}{1.880532in}}{\pgfqpoint{1.985761in}{1.882846in}}{\pgfqpoint{1.989879in}{1.886964in}}%
\pgfpathcurveto{\pgfqpoint{1.993997in}{1.891082in}}{\pgfqpoint{1.996311in}{1.896669in}}{\pgfqpoint{1.996311in}{1.902493in}}%
\pgfpathcurveto{\pgfqpoint{1.996311in}{1.908317in}}{\pgfqpoint{1.993997in}{1.913903in}}{\pgfqpoint{1.989879in}{1.918021in}}%
\pgfpathcurveto{\pgfqpoint{1.985761in}{1.922139in}}{\pgfqpoint{1.980175in}{1.924453in}}{\pgfqpoint{1.974351in}{1.924453in}}%
\pgfpathcurveto{\pgfqpoint{1.968527in}{1.924453in}}{\pgfqpoint{1.962941in}{1.922139in}}{\pgfqpoint{1.958823in}{1.918021in}}%
\pgfpathcurveto{\pgfqpoint{1.954705in}{1.913903in}}{\pgfqpoint{1.952391in}{1.908317in}}{\pgfqpoint{1.952391in}{1.902493in}}%
\pgfpathcurveto{\pgfqpoint{1.952391in}{1.896669in}}{\pgfqpoint{1.954705in}{1.891082in}}{\pgfqpoint{1.958823in}{1.886964in}}%
\pgfpathcurveto{\pgfqpoint{1.962941in}{1.882846in}}{\pgfqpoint{1.968527in}{1.880532in}}{\pgfqpoint{1.974351in}{1.880532in}}%
\pgfpathlineto{\pgfqpoint{1.974351in}{1.880532in}}%
\pgfpathclose%
\pgfusepath{stroke,fill}%
\end{pgfscope}%
\begin{pgfscope}%
\pgfpathrectangle{\pgfqpoint{1.542338in}{0.880000in}}{\pgfqpoint{5.115323in}{6.160000in}}%
\pgfusepath{clip}%
\pgfsetbuttcap%
\pgfsetroundjoin%
\definecolor{currentfill}{rgb}{0.200000,0.200000,0.800000}%
\pgfsetfillcolor{currentfill}%
\pgfsetlinewidth{1.003750pt}%
\definecolor{currentstroke}{rgb}{0.200000,0.200000,0.800000}%
\pgfsetstrokecolor{currentstroke}%
\pgfsetdash{}{0pt}%
\pgfpathmoveto{\pgfqpoint{2.062692in}{1.755396in}}%
\pgfpathcurveto{\pgfqpoint{2.068516in}{1.755396in}}{\pgfqpoint{2.074102in}{1.757710in}}{\pgfqpoint{2.078220in}{1.761828in}}%
\pgfpathcurveto{\pgfqpoint{2.082338in}{1.765946in}}{\pgfqpoint{2.084652in}{1.771533in}}{\pgfqpoint{2.084652in}{1.777357in}}%
\pgfpathcurveto{\pgfqpoint{2.084652in}{1.783180in}}{\pgfqpoint{2.082338in}{1.788767in}}{\pgfqpoint{2.078220in}{1.792885in}}%
\pgfpathcurveto{\pgfqpoint{2.074102in}{1.797003in}}{\pgfqpoint{2.068516in}{1.799317in}}{\pgfqpoint{2.062692in}{1.799317in}}%
\pgfpathcurveto{\pgfqpoint{2.056868in}{1.799317in}}{\pgfqpoint{2.051282in}{1.797003in}}{\pgfqpoint{2.047164in}{1.792885in}}%
\pgfpathcurveto{\pgfqpoint{2.043046in}{1.788767in}}{\pgfqpoint{2.040732in}{1.783180in}}{\pgfqpoint{2.040732in}{1.777357in}}%
\pgfpathcurveto{\pgfqpoint{2.040732in}{1.771533in}}{\pgfqpoint{2.043046in}{1.765946in}}{\pgfqpoint{2.047164in}{1.761828in}}%
\pgfpathcurveto{\pgfqpoint{2.051282in}{1.757710in}}{\pgfqpoint{2.056868in}{1.755396in}}{\pgfqpoint{2.062692in}{1.755396in}}%
\pgfpathlineto{\pgfqpoint{2.062692in}{1.755396in}}%
\pgfpathclose%
\pgfusepath{stroke,fill}%
\end{pgfscope}%
\begin{pgfscope}%
\pgfpathrectangle{\pgfqpoint{1.542338in}{0.880000in}}{\pgfqpoint{5.115323in}{6.160000in}}%
\pgfusepath{clip}%
\pgfsetbuttcap%
\pgfsetroundjoin%
\definecolor{currentfill}{rgb}{0.200000,0.200000,0.800000}%
\pgfsetfillcolor{currentfill}%
\pgfsetlinewidth{1.003750pt}%
\definecolor{currentstroke}{rgb}{0.200000,0.200000,0.800000}%
\pgfsetstrokecolor{currentstroke}%
\pgfsetdash{}{0pt}%
\pgfpathmoveto{\pgfqpoint{2.173011in}{1.649192in}}%
\pgfpathcurveto{\pgfqpoint{2.178835in}{1.649192in}}{\pgfqpoint{2.184421in}{1.651506in}}{\pgfqpoint{2.188540in}{1.655624in}}%
\pgfpathcurveto{\pgfqpoint{2.192658in}{1.659742in}}{\pgfqpoint{2.194972in}{1.665329in}}{\pgfqpoint{2.194972in}{1.671153in}}%
\pgfpathcurveto{\pgfqpoint{2.194972in}{1.676977in}}{\pgfqpoint{2.192658in}{1.682563in}}{\pgfqpoint{2.188540in}{1.686681in}}%
\pgfpathcurveto{\pgfqpoint{2.184421in}{1.690799in}}{\pgfqpoint{2.178835in}{1.693113in}}{\pgfqpoint{2.173011in}{1.693113in}}%
\pgfpathcurveto{\pgfqpoint{2.167187in}{1.693113in}}{\pgfqpoint{2.161601in}{1.690799in}}{\pgfqpoint{2.157483in}{1.686681in}}%
\pgfpathcurveto{\pgfqpoint{2.153365in}{1.682563in}}{\pgfqpoint{2.151051in}{1.676977in}}{\pgfqpoint{2.151051in}{1.671153in}}%
\pgfpathcurveto{\pgfqpoint{2.151051in}{1.665329in}}{\pgfqpoint{2.153365in}{1.659742in}}{\pgfqpoint{2.157483in}{1.655624in}}%
\pgfpathcurveto{\pgfqpoint{2.161601in}{1.651506in}}{\pgfqpoint{2.167187in}{1.649192in}}{\pgfqpoint{2.173011in}{1.649192in}}%
\pgfpathlineto{\pgfqpoint{2.173011in}{1.649192in}}%
\pgfpathclose%
\pgfusepath{stroke,fill}%
\end{pgfscope}%
\begin{pgfscope}%
\pgfpathrectangle{\pgfqpoint{1.542338in}{0.880000in}}{\pgfqpoint{5.115323in}{6.160000in}}%
\pgfusepath{clip}%
\pgfsetbuttcap%
\pgfsetroundjoin%
\definecolor{currentfill}{rgb}{0.200000,0.200000,0.800000}%
\pgfsetfillcolor{currentfill}%
\pgfsetlinewidth{1.003750pt}%
\definecolor{currentstroke}{rgb}{0.200000,0.200000,0.800000}%
\pgfsetstrokecolor{currentstroke}%
\pgfsetdash{}{0pt}%
\pgfpathmoveto{\pgfqpoint{2.289448in}{1.549768in}}%
\pgfpathcurveto{\pgfqpoint{2.295272in}{1.549768in}}{\pgfqpoint{2.300858in}{1.552082in}}{\pgfqpoint{2.304977in}{1.556200in}}%
\pgfpathcurveto{\pgfqpoint{2.309095in}{1.560318in}}{\pgfqpoint{2.311409in}{1.565904in}}{\pgfqpoint{2.311409in}{1.571728in}}%
\pgfpathcurveto{\pgfqpoint{2.311409in}{1.577552in}}{\pgfqpoint{2.309095in}{1.583138in}}{\pgfqpoint{2.304977in}{1.587256in}}%
\pgfpathcurveto{\pgfqpoint{2.300858in}{1.591374in}}{\pgfqpoint{2.295272in}{1.593688in}}{\pgfqpoint{2.289448in}{1.593688in}}%
\pgfpathcurveto{\pgfqpoint{2.283624in}{1.593688in}}{\pgfqpoint{2.278038in}{1.591374in}}{\pgfqpoint{2.273920in}{1.587256in}}%
\pgfpathcurveto{\pgfqpoint{2.269802in}{1.583138in}}{\pgfqpoint{2.267488in}{1.577552in}}{\pgfqpoint{2.267488in}{1.571728in}}%
\pgfpathcurveto{\pgfqpoint{2.267488in}{1.565904in}}{\pgfqpoint{2.269802in}{1.560318in}}{\pgfqpoint{2.273920in}{1.556200in}}%
\pgfpathcurveto{\pgfqpoint{2.278038in}{1.552082in}}{\pgfqpoint{2.283624in}{1.549768in}}{\pgfqpoint{2.289448in}{1.549768in}}%
\pgfpathlineto{\pgfqpoint{2.289448in}{1.549768in}}%
\pgfpathclose%
\pgfusepath{stroke,fill}%
\end{pgfscope}%
\begin{pgfscope}%
\pgfpathrectangle{\pgfqpoint{1.542338in}{0.880000in}}{\pgfqpoint{5.115323in}{6.160000in}}%
\pgfusepath{clip}%
\pgfsetbuttcap%
\pgfsetroundjoin%
\definecolor{currentfill}{rgb}{0.200000,0.200000,0.800000}%
\pgfsetfillcolor{currentfill}%
\pgfsetlinewidth{1.003750pt}%
\definecolor{currentstroke}{rgb}{0.200000,0.200000,0.800000}%
\pgfsetstrokecolor{currentstroke}%
\pgfsetdash{}{0pt}%
\pgfpathmoveto{\pgfqpoint{2.420982in}{1.471026in}}%
\pgfpathcurveto{\pgfqpoint{2.426806in}{1.471026in}}{\pgfqpoint{2.432392in}{1.473340in}}{\pgfqpoint{2.436511in}{1.477458in}}%
\pgfpathcurveto{\pgfqpoint{2.440629in}{1.481576in}}{\pgfqpoint{2.442943in}{1.487163in}}{\pgfqpoint{2.442943in}{1.492986in}}%
\pgfpathcurveto{\pgfqpoint{2.442943in}{1.498810in}}{\pgfqpoint{2.440629in}{1.504397in}}{\pgfqpoint{2.436511in}{1.508515in}}%
\pgfpathcurveto{\pgfqpoint{2.432392in}{1.512633in}}{\pgfqpoint{2.426806in}{1.514947in}}{\pgfqpoint{2.420982in}{1.514947in}}%
\pgfpathcurveto{\pgfqpoint{2.415158in}{1.514947in}}{\pgfqpoint{2.409572in}{1.512633in}}{\pgfqpoint{2.405454in}{1.508515in}}%
\pgfpathcurveto{\pgfqpoint{2.401336in}{1.504397in}}{\pgfqpoint{2.399022in}{1.498810in}}{\pgfqpoint{2.399022in}{1.492986in}}%
\pgfpathcurveto{\pgfqpoint{2.399022in}{1.487163in}}{\pgfqpoint{2.401336in}{1.481576in}}{\pgfqpoint{2.405454in}{1.477458in}}%
\pgfpathcurveto{\pgfqpoint{2.409572in}{1.473340in}}{\pgfqpoint{2.415158in}{1.471026in}}{\pgfqpoint{2.420982in}{1.471026in}}%
\pgfpathlineto{\pgfqpoint{2.420982in}{1.471026in}}%
\pgfpathclose%
\pgfusepath{stroke,fill}%
\end{pgfscope}%
\begin{pgfscope}%
\pgfpathrectangle{\pgfqpoint{1.542338in}{0.880000in}}{\pgfqpoint{5.115323in}{6.160000in}}%
\pgfusepath{clip}%
\pgfsetbuttcap%
\pgfsetroundjoin%
\definecolor{currentfill}{rgb}{0.200000,0.200000,0.800000}%
\pgfsetfillcolor{currentfill}%
\pgfsetlinewidth{1.003750pt}%
\definecolor{currentstroke}{rgb}{0.200000,0.200000,0.800000}%
\pgfsetstrokecolor{currentstroke}%
\pgfsetdash{}{0pt}%
\pgfpathmoveto{\pgfqpoint{2.562046in}{1.411536in}}%
\pgfpathcurveto{\pgfqpoint{2.567870in}{1.411536in}}{\pgfqpoint{2.573456in}{1.413850in}}{\pgfqpoint{2.577575in}{1.417968in}}%
\pgfpathcurveto{\pgfqpoint{2.581693in}{1.422086in}}{\pgfqpoint{2.584007in}{1.427672in}}{\pgfqpoint{2.584007in}{1.433496in}}%
\pgfpathcurveto{\pgfqpoint{2.584007in}{1.439320in}}{\pgfqpoint{2.581693in}{1.444907in}}{\pgfqpoint{2.577575in}{1.449025in}}%
\pgfpathcurveto{\pgfqpoint{2.573456in}{1.453143in}}{\pgfqpoint{2.567870in}{1.455457in}}{\pgfqpoint{2.562046in}{1.455457in}}%
\pgfpathcurveto{\pgfqpoint{2.556222in}{1.455457in}}{\pgfqpoint{2.550636in}{1.453143in}}{\pgfqpoint{2.546518in}{1.449025in}}%
\pgfpathcurveto{\pgfqpoint{2.542400in}{1.444907in}}{\pgfqpoint{2.540086in}{1.439320in}}{\pgfqpoint{2.540086in}{1.433496in}}%
\pgfpathcurveto{\pgfqpoint{2.540086in}{1.427672in}}{\pgfqpoint{2.542400in}{1.422086in}}{\pgfqpoint{2.546518in}{1.417968in}}%
\pgfpathcurveto{\pgfqpoint{2.550636in}{1.413850in}}{\pgfqpoint{2.556222in}{1.411536in}}{\pgfqpoint{2.562046in}{1.411536in}}%
\pgfpathlineto{\pgfqpoint{2.562046in}{1.411536in}}%
\pgfpathclose%
\pgfusepath{stroke,fill}%
\end{pgfscope}%
\begin{pgfscope}%
\pgfpathrectangle{\pgfqpoint{1.542338in}{0.880000in}}{\pgfqpoint{5.115323in}{6.160000in}}%
\pgfusepath{clip}%
\pgfsetbuttcap%
\pgfsetroundjoin%
\definecolor{currentfill}{rgb}{0.200000,0.200000,0.800000}%
\pgfsetfillcolor{currentfill}%
\pgfsetlinewidth{1.003750pt}%
\definecolor{currentstroke}{rgb}{0.200000,0.200000,0.800000}%
\pgfsetstrokecolor{currentstroke}%
\pgfsetdash{}{0pt}%
\pgfpathmoveto{\pgfqpoint{2.707759in}{1.364239in}}%
\pgfpathcurveto{\pgfqpoint{2.713583in}{1.364239in}}{\pgfqpoint{2.719169in}{1.366553in}}{\pgfqpoint{2.723287in}{1.370671in}}%
\pgfpathcurveto{\pgfqpoint{2.727405in}{1.374789in}}{\pgfqpoint{2.729719in}{1.380375in}}{\pgfqpoint{2.729719in}{1.386199in}}%
\pgfpathcurveto{\pgfqpoint{2.729719in}{1.392023in}}{\pgfqpoint{2.727405in}{1.397609in}}{\pgfqpoint{2.723287in}{1.401727in}}%
\pgfpathcurveto{\pgfqpoint{2.719169in}{1.405846in}}{\pgfqpoint{2.713583in}{1.408159in}}{\pgfqpoint{2.707759in}{1.408159in}}%
\pgfpathcurveto{\pgfqpoint{2.701935in}{1.408159in}}{\pgfqpoint{2.696349in}{1.405846in}}{\pgfqpoint{2.692230in}{1.401727in}}%
\pgfpathcurveto{\pgfqpoint{2.688112in}{1.397609in}}{\pgfqpoint{2.685798in}{1.392023in}}{\pgfqpoint{2.685798in}{1.386199in}}%
\pgfpathcurveto{\pgfqpoint{2.685798in}{1.380375in}}{\pgfqpoint{2.688112in}{1.374789in}}{\pgfqpoint{2.692230in}{1.370671in}}%
\pgfpathcurveto{\pgfqpoint{2.696349in}{1.366553in}}{\pgfqpoint{2.701935in}{1.364239in}}{\pgfqpoint{2.707759in}{1.364239in}}%
\pgfpathlineto{\pgfqpoint{2.707759in}{1.364239in}}%
\pgfpathclose%
\pgfusepath{stroke,fill}%
\end{pgfscope}%
\begin{pgfscope}%
\pgfpathrectangle{\pgfqpoint{1.542338in}{0.880000in}}{\pgfqpoint{5.115323in}{6.160000in}}%
\pgfusepath{clip}%
\pgfsetbuttcap%
\pgfsetroundjoin%
\definecolor{currentfill}{rgb}{0.200000,0.200000,0.800000}%
\pgfsetfillcolor{currentfill}%
\pgfsetlinewidth{1.003750pt}%
\definecolor{currentstroke}{rgb}{0.200000,0.200000,0.800000}%
\pgfsetstrokecolor{currentstroke}%
\pgfsetdash{}{0pt}%
\pgfpathmoveto{\pgfqpoint{2.859191in}{1.339783in}}%
\pgfpathcurveto{\pgfqpoint{2.865015in}{1.339783in}}{\pgfqpoint{2.870601in}{1.342097in}}{\pgfqpoint{2.874720in}{1.346215in}}%
\pgfpathcurveto{\pgfqpoint{2.878838in}{1.350333in}}{\pgfqpoint{2.881152in}{1.355919in}}{\pgfqpoint{2.881152in}{1.361743in}}%
\pgfpathcurveto{\pgfqpoint{2.881152in}{1.367567in}}{\pgfqpoint{2.878838in}{1.373153in}}{\pgfqpoint{2.874720in}{1.377271in}}%
\pgfpathcurveto{\pgfqpoint{2.870601in}{1.381389in}}{\pgfqpoint{2.865015in}{1.383703in}}{\pgfqpoint{2.859191in}{1.383703in}}%
\pgfpathcurveto{\pgfqpoint{2.853367in}{1.383703in}}{\pgfqpoint{2.847781in}{1.381389in}}{\pgfqpoint{2.843663in}{1.377271in}}%
\pgfpathcurveto{\pgfqpoint{2.839545in}{1.373153in}}{\pgfqpoint{2.837231in}{1.367567in}}{\pgfqpoint{2.837231in}{1.361743in}}%
\pgfpathcurveto{\pgfqpoint{2.837231in}{1.355919in}}{\pgfqpoint{2.839545in}{1.350333in}}{\pgfqpoint{2.843663in}{1.346215in}}%
\pgfpathcurveto{\pgfqpoint{2.847781in}{1.342097in}}{\pgfqpoint{2.853367in}{1.339783in}}{\pgfqpoint{2.859191in}{1.339783in}}%
\pgfpathlineto{\pgfqpoint{2.859191in}{1.339783in}}%
\pgfpathclose%
\pgfusepath{stroke,fill}%
\end{pgfscope}%
\begin{pgfscope}%
\pgfpathrectangle{\pgfqpoint{1.542338in}{0.880000in}}{\pgfqpoint{5.115323in}{6.160000in}}%
\pgfusepath{clip}%
\pgfsetbuttcap%
\pgfsetroundjoin%
\definecolor{currentfill}{rgb}{0.200000,0.200000,0.800000}%
\pgfsetfillcolor{currentfill}%
\pgfsetlinewidth{1.003750pt}%
\definecolor{currentstroke}{rgb}{0.200000,0.200000,0.800000}%
\pgfsetstrokecolor{currentstroke}%
\pgfsetdash{}{0pt}%
\pgfpathmoveto{\pgfqpoint{3.012636in}{1.330786in}}%
\pgfpathcurveto{\pgfqpoint{3.018460in}{1.330786in}}{\pgfqpoint{3.024047in}{1.333100in}}{\pgfqpoint{3.028165in}{1.337218in}}%
\pgfpathcurveto{\pgfqpoint{3.032283in}{1.341336in}}{\pgfqpoint{3.034597in}{1.346922in}}{\pgfqpoint{3.034597in}{1.352746in}}%
\pgfpathcurveto{\pgfqpoint{3.034597in}{1.358570in}}{\pgfqpoint{3.032283in}{1.364156in}}{\pgfqpoint{3.028165in}{1.368275in}}%
\pgfpathcurveto{\pgfqpoint{3.024047in}{1.372393in}}{\pgfqpoint{3.018460in}{1.374707in}}{\pgfqpoint{3.012636in}{1.374707in}}%
\pgfpathcurveto{\pgfqpoint{3.006813in}{1.374707in}}{\pgfqpoint{3.001226in}{1.372393in}}{\pgfqpoint{2.997108in}{1.368275in}}%
\pgfpathcurveto{\pgfqpoint{2.992990in}{1.364156in}}{\pgfqpoint{2.990676in}{1.358570in}}{\pgfqpoint{2.990676in}{1.352746in}}%
\pgfpathcurveto{\pgfqpoint{2.990676in}{1.346922in}}{\pgfqpoint{2.992990in}{1.341336in}}{\pgfqpoint{2.997108in}{1.337218in}}%
\pgfpathcurveto{\pgfqpoint{3.001226in}{1.333100in}}{\pgfqpoint{3.006813in}{1.330786in}}{\pgfqpoint{3.012636in}{1.330786in}}%
\pgfpathlineto{\pgfqpoint{3.012636in}{1.330786in}}%
\pgfpathclose%
\pgfusepath{stroke,fill}%
\end{pgfscope}%
\begin{pgfscope}%
\pgfpathrectangle{\pgfqpoint{1.542338in}{0.880000in}}{\pgfqpoint{5.115323in}{6.160000in}}%
\pgfusepath{clip}%
\pgfsetbuttcap%
\pgfsetroundjoin%
\definecolor{currentfill}{rgb}{0.200000,0.200000,0.800000}%
\pgfsetfillcolor{currentfill}%
\pgfsetlinewidth{1.003750pt}%
\definecolor{currentstroke}{rgb}{0.200000,0.200000,0.800000}%
\pgfsetstrokecolor{currentstroke}%
\pgfsetdash{}{0pt}%
\pgfpathmoveto{\pgfqpoint{3.164060in}{1.356587in}}%
\pgfpathcurveto{\pgfqpoint{3.169883in}{1.356587in}}{\pgfqpoint{3.175470in}{1.358901in}}{\pgfqpoint{3.179588in}{1.363019in}}%
\pgfpathcurveto{\pgfqpoint{3.183706in}{1.367137in}}{\pgfqpoint{3.186020in}{1.372723in}}{\pgfqpoint{3.186020in}{1.378547in}}%
\pgfpathcurveto{\pgfqpoint{3.186020in}{1.384371in}}{\pgfqpoint{3.183706in}{1.389957in}}{\pgfqpoint{3.179588in}{1.394075in}}%
\pgfpathcurveto{\pgfqpoint{3.175470in}{1.398194in}}{\pgfqpoint{3.169883in}{1.400507in}}{\pgfqpoint{3.164060in}{1.400507in}}%
\pgfpathcurveto{\pgfqpoint{3.158236in}{1.400507in}}{\pgfqpoint{3.152649in}{1.398194in}}{\pgfqpoint{3.148531in}{1.394075in}}%
\pgfpathcurveto{\pgfqpoint{3.144413in}{1.389957in}}{\pgfqpoint{3.142099in}{1.384371in}}{\pgfqpoint{3.142099in}{1.378547in}}%
\pgfpathcurveto{\pgfqpoint{3.142099in}{1.372723in}}{\pgfqpoint{3.144413in}{1.367137in}}{\pgfqpoint{3.148531in}{1.363019in}}%
\pgfpathcurveto{\pgfqpoint{3.152649in}{1.358901in}}{\pgfqpoint{3.158236in}{1.356587in}}{\pgfqpoint{3.164060in}{1.356587in}}%
\pgfpathlineto{\pgfqpoint{3.164060in}{1.356587in}}%
\pgfpathclose%
\pgfusepath{stroke,fill}%
\end{pgfscope}%
\begin{pgfscope}%
\pgfpathrectangle{\pgfqpoint{1.542338in}{0.880000in}}{\pgfqpoint{5.115323in}{6.160000in}}%
\pgfusepath{clip}%
\pgfsetbuttcap%
\pgfsetroundjoin%
\definecolor{currentfill}{rgb}{0.200000,0.200000,0.800000}%
\pgfsetfillcolor{currentfill}%
\pgfsetlinewidth{1.003750pt}%
\definecolor{currentstroke}{rgb}{0.200000,0.200000,0.800000}%
\pgfsetstrokecolor{currentstroke}%
\pgfsetdash{}{0pt}%
\pgfpathmoveto{\pgfqpoint{3.314469in}{1.384632in}}%
\pgfpathcurveto{\pgfqpoint{3.320293in}{1.384632in}}{\pgfqpoint{3.325879in}{1.386945in}}{\pgfqpoint{3.329997in}{1.391064in}}%
\pgfpathcurveto{\pgfqpoint{3.334116in}{1.395182in}}{\pgfqpoint{3.336429in}{1.400768in}}{\pgfqpoint{3.336429in}{1.406592in}}%
\pgfpathcurveto{\pgfqpoint{3.336429in}{1.412416in}}{\pgfqpoint{3.334116in}{1.418002in}}{\pgfqpoint{3.329997in}{1.422120in}}%
\pgfpathcurveto{\pgfqpoint{3.325879in}{1.426238in}}{\pgfqpoint{3.320293in}{1.428552in}}{\pgfqpoint{3.314469in}{1.428552in}}%
\pgfpathcurveto{\pgfqpoint{3.308645in}{1.428552in}}{\pgfqpoint{3.303059in}{1.426238in}}{\pgfqpoint{3.298941in}{1.422120in}}%
\pgfpathcurveto{\pgfqpoint{3.294823in}{1.418002in}}{\pgfqpoint{3.292509in}{1.412416in}}{\pgfqpoint{3.292509in}{1.406592in}}%
\pgfpathcurveto{\pgfqpoint{3.292509in}{1.400768in}}{\pgfqpoint{3.294823in}{1.395182in}}{\pgfqpoint{3.298941in}{1.391064in}}%
\pgfpathcurveto{\pgfqpoint{3.303059in}{1.386945in}}{\pgfqpoint{3.308645in}{1.384632in}}{\pgfqpoint{3.314469in}{1.384632in}}%
\pgfpathlineto{\pgfqpoint{3.314469in}{1.384632in}}%
\pgfpathclose%
\pgfusepath{stroke,fill}%
\end{pgfscope}%
\begin{pgfscope}%
\pgfpathrectangle{\pgfqpoint{1.542338in}{0.880000in}}{\pgfqpoint{5.115323in}{6.160000in}}%
\pgfusepath{clip}%
\pgfsetbuttcap%
\pgfsetroundjoin%
\definecolor{currentfill}{rgb}{0.200000,0.200000,0.800000}%
\pgfsetfillcolor{currentfill}%
\pgfsetlinewidth{1.003750pt}%
\definecolor{currentstroke}{rgb}{0.200000,0.200000,0.800000}%
\pgfsetstrokecolor{currentstroke}%
\pgfsetdash{}{0pt}%
\pgfpathmoveto{\pgfqpoint{3.457199in}{1.440078in}}%
\pgfpathcurveto{\pgfqpoint{3.463023in}{1.440078in}}{\pgfqpoint{3.468609in}{1.442392in}}{\pgfqpoint{3.472727in}{1.446510in}}%
\pgfpathcurveto{\pgfqpoint{3.476845in}{1.450628in}}{\pgfqpoint{3.479159in}{1.456214in}}{\pgfqpoint{3.479159in}{1.462038in}}%
\pgfpathcurveto{\pgfqpoint{3.479159in}{1.467862in}}{\pgfqpoint{3.476845in}{1.473448in}}{\pgfqpoint{3.472727in}{1.477567in}}%
\pgfpathcurveto{\pgfqpoint{3.468609in}{1.481685in}}{\pgfqpoint{3.463023in}{1.483999in}}{\pgfqpoint{3.457199in}{1.483999in}}%
\pgfpathcurveto{\pgfqpoint{3.451375in}{1.483999in}}{\pgfqpoint{3.445789in}{1.481685in}}{\pgfqpoint{3.441670in}{1.477567in}}%
\pgfpathcurveto{\pgfqpoint{3.437552in}{1.473448in}}{\pgfqpoint{3.435238in}{1.467862in}}{\pgfqpoint{3.435238in}{1.462038in}}%
\pgfpathcurveto{\pgfqpoint{3.435238in}{1.456214in}}{\pgfqpoint{3.437552in}{1.450628in}}{\pgfqpoint{3.441670in}{1.446510in}}%
\pgfpathcurveto{\pgfqpoint{3.445789in}{1.442392in}}{\pgfqpoint{3.451375in}{1.440078in}}{\pgfqpoint{3.457199in}{1.440078in}}%
\pgfpathlineto{\pgfqpoint{3.457199in}{1.440078in}}%
\pgfpathclose%
\pgfusepath{stroke,fill}%
\end{pgfscope}%
\begin{pgfscope}%
\pgfpathrectangle{\pgfqpoint{1.542338in}{0.880000in}}{\pgfqpoint{5.115323in}{6.160000in}}%
\pgfusepath{clip}%
\pgfsetbuttcap%
\pgfsetroundjoin%
\definecolor{currentfill}{rgb}{0.200000,0.200000,0.800000}%
\pgfsetfillcolor{currentfill}%
\pgfsetlinewidth{1.003750pt}%
\definecolor{currentstroke}{rgb}{0.200000,0.200000,0.800000}%
\pgfsetstrokecolor{currentstroke}%
\pgfsetdash{}{0pt}%
\pgfpathmoveto{\pgfqpoint{3.596170in}{1.505214in}}%
\pgfpathcurveto{\pgfqpoint{3.601994in}{1.505214in}}{\pgfqpoint{3.607580in}{1.507528in}}{\pgfqpoint{3.611698in}{1.511646in}}%
\pgfpathcurveto{\pgfqpoint{3.615816in}{1.515764in}}{\pgfqpoint{3.618130in}{1.521350in}}{\pgfqpoint{3.618130in}{1.527174in}}%
\pgfpathcurveto{\pgfqpoint{3.618130in}{1.532998in}}{\pgfqpoint{3.615816in}{1.538584in}}{\pgfqpoint{3.611698in}{1.542703in}}%
\pgfpathcurveto{\pgfqpoint{3.607580in}{1.546821in}}{\pgfqpoint{3.601994in}{1.549135in}}{\pgfqpoint{3.596170in}{1.549135in}}%
\pgfpathcurveto{\pgfqpoint{3.590346in}{1.549135in}}{\pgfqpoint{3.584760in}{1.546821in}}{\pgfqpoint{3.580642in}{1.542703in}}%
\pgfpathcurveto{\pgfqpoint{3.576524in}{1.538584in}}{\pgfqpoint{3.574210in}{1.532998in}}{\pgfqpoint{3.574210in}{1.527174in}}%
\pgfpathcurveto{\pgfqpoint{3.574210in}{1.521350in}}{\pgfqpoint{3.576524in}{1.515764in}}{\pgfqpoint{3.580642in}{1.511646in}}%
\pgfpathcurveto{\pgfqpoint{3.584760in}{1.507528in}}{\pgfqpoint{3.590346in}{1.505214in}}{\pgfqpoint{3.596170in}{1.505214in}}%
\pgfpathlineto{\pgfqpoint{3.596170in}{1.505214in}}%
\pgfpathclose%
\pgfusepath{stroke,fill}%
\end{pgfscope}%
\begin{pgfscope}%
\pgfpathrectangle{\pgfqpoint{1.542338in}{0.880000in}}{\pgfqpoint{5.115323in}{6.160000in}}%
\pgfusepath{clip}%
\pgfsetbuttcap%
\pgfsetroundjoin%
\definecolor{currentfill}{rgb}{0.200000,0.200000,0.800000}%
\pgfsetfillcolor{currentfill}%
\pgfsetlinewidth{1.003750pt}%
\definecolor{currentstroke}{rgb}{0.200000,0.200000,0.800000}%
\pgfsetstrokecolor{currentstroke}%
\pgfsetdash{}{0pt}%
\pgfpathmoveto{\pgfqpoint{3.721989in}{1.593546in}}%
\pgfpathcurveto{\pgfqpoint{3.727813in}{1.593546in}}{\pgfqpoint{3.733399in}{1.595860in}}{\pgfqpoint{3.737517in}{1.599978in}}%
\pgfpathcurveto{\pgfqpoint{3.741635in}{1.604096in}}{\pgfqpoint{3.743949in}{1.609682in}}{\pgfqpoint{3.743949in}{1.615506in}}%
\pgfpathcurveto{\pgfqpoint{3.743949in}{1.621330in}}{\pgfqpoint{3.741635in}{1.626916in}}{\pgfqpoint{3.737517in}{1.631034in}}%
\pgfpathcurveto{\pgfqpoint{3.733399in}{1.635152in}}{\pgfqpoint{3.727813in}{1.637466in}}{\pgfqpoint{3.721989in}{1.637466in}}%
\pgfpathcurveto{\pgfqpoint{3.716165in}{1.637466in}}{\pgfqpoint{3.710579in}{1.635152in}}{\pgfqpoint{3.706461in}{1.631034in}}%
\pgfpathcurveto{\pgfqpoint{3.702342in}{1.626916in}}{\pgfqpoint{3.700029in}{1.621330in}}{\pgfqpoint{3.700029in}{1.615506in}}%
\pgfpathcurveto{\pgfqpoint{3.700029in}{1.609682in}}{\pgfqpoint{3.702342in}{1.604096in}}{\pgfqpoint{3.706461in}{1.599978in}}%
\pgfpathcurveto{\pgfqpoint{3.710579in}{1.595860in}}{\pgfqpoint{3.716165in}{1.593546in}}{\pgfqpoint{3.721989in}{1.593546in}}%
\pgfpathlineto{\pgfqpoint{3.721989in}{1.593546in}}%
\pgfpathclose%
\pgfusepath{stroke,fill}%
\end{pgfscope}%
\begin{pgfscope}%
\pgfpathrectangle{\pgfqpoint{1.542338in}{0.880000in}}{\pgfqpoint{5.115323in}{6.160000in}}%
\pgfusepath{clip}%
\pgfsetbuttcap%
\pgfsetroundjoin%
\definecolor{currentfill}{rgb}{0.200000,0.200000,0.800000}%
\pgfsetfillcolor{currentfill}%
\pgfsetlinewidth{1.003750pt}%
\definecolor{currentstroke}{rgb}{0.200000,0.200000,0.800000}%
\pgfsetstrokecolor{currentstroke}%
\pgfsetdash{}{0pt}%
\pgfpathmoveto{\pgfqpoint{3.835060in}{1.697565in}}%
\pgfpathcurveto{\pgfqpoint{3.840883in}{1.697565in}}{\pgfqpoint{3.846470in}{1.699879in}}{\pgfqpoint{3.850588in}{1.703997in}}%
\pgfpathcurveto{\pgfqpoint{3.854706in}{1.708115in}}{\pgfqpoint{3.857020in}{1.713701in}}{\pgfqpoint{3.857020in}{1.719525in}}%
\pgfpathcurveto{\pgfqpoint{3.857020in}{1.725349in}}{\pgfqpoint{3.854706in}{1.730935in}}{\pgfqpoint{3.850588in}{1.735053in}}%
\pgfpathcurveto{\pgfqpoint{3.846470in}{1.739172in}}{\pgfqpoint{3.840883in}{1.741485in}}{\pgfqpoint{3.835060in}{1.741485in}}%
\pgfpathcurveto{\pgfqpoint{3.829236in}{1.741485in}}{\pgfqpoint{3.823649in}{1.739172in}}{\pgfqpoint{3.819531in}{1.735053in}}%
\pgfpathcurveto{\pgfqpoint{3.815413in}{1.730935in}}{\pgfqpoint{3.813099in}{1.725349in}}{\pgfqpoint{3.813099in}{1.719525in}}%
\pgfpathcurveto{\pgfqpoint{3.813099in}{1.713701in}}{\pgfqpoint{3.815413in}{1.708115in}}{\pgfqpoint{3.819531in}{1.703997in}}%
\pgfpathcurveto{\pgfqpoint{3.823649in}{1.699879in}}{\pgfqpoint{3.829236in}{1.697565in}}{\pgfqpoint{3.835060in}{1.697565in}}%
\pgfpathlineto{\pgfqpoint{3.835060in}{1.697565in}}%
\pgfpathclose%
\pgfusepath{stroke,fill}%
\end{pgfscope}%
\begin{pgfscope}%
\pgfpathrectangle{\pgfqpoint{1.542338in}{0.880000in}}{\pgfqpoint{5.115323in}{6.160000in}}%
\pgfusepath{clip}%
\pgfsetbuttcap%
\pgfsetroundjoin%
\definecolor{currentfill}{rgb}{0.200000,0.200000,0.800000}%
\pgfsetfillcolor{currentfill}%
\pgfsetlinewidth{1.003750pt}%
\definecolor{currentstroke}{rgb}{0.200000,0.200000,0.800000}%
\pgfsetstrokecolor{currentstroke}%
\pgfsetdash{}{0pt}%
\pgfpathmoveto{\pgfqpoint{3.930711in}{1.817412in}}%
\pgfpathcurveto{\pgfqpoint{3.936535in}{1.817412in}}{\pgfqpoint{3.942122in}{1.819726in}}{\pgfqpoint{3.946240in}{1.823844in}}%
\pgfpathcurveto{\pgfqpoint{3.950358in}{1.827962in}}{\pgfqpoint{3.952672in}{1.833548in}}{\pgfqpoint{3.952672in}{1.839372in}}%
\pgfpathcurveto{\pgfqpoint{3.952672in}{1.845196in}}{\pgfqpoint{3.950358in}{1.850782in}}{\pgfqpoint{3.946240in}{1.854900in}}%
\pgfpathcurveto{\pgfqpoint{3.942122in}{1.859018in}}{\pgfqpoint{3.936535in}{1.861332in}}{\pgfqpoint{3.930711in}{1.861332in}}%
\pgfpathcurveto{\pgfqpoint{3.924888in}{1.861332in}}{\pgfqpoint{3.919301in}{1.859018in}}{\pgfqpoint{3.915183in}{1.854900in}}%
\pgfpathcurveto{\pgfqpoint{3.911065in}{1.850782in}}{\pgfqpoint{3.908751in}{1.845196in}}{\pgfqpoint{3.908751in}{1.839372in}}%
\pgfpathcurveto{\pgfqpoint{3.908751in}{1.833548in}}{\pgfqpoint{3.911065in}{1.827962in}}{\pgfqpoint{3.915183in}{1.823844in}}%
\pgfpathcurveto{\pgfqpoint{3.919301in}{1.819726in}}{\pgfqpoint{3.924888in}{1.817412in}}{\pgfqpoint{3.930711in}{1.817412in}}%
\pgfpathlineto{\pgfqpoint{3.930711in}{1.817412in}}%
\pgfpathclose%
\pgfusepath{stroke,fill}%
\end{pgfscope}%
\begin{pgfscope}%
\pgfpathrectangle{\pgfqpoint{1.542338in}{0.880000in}}{\pgfqpoint{5.115323in}{6.160000in}}%
\pgfusepath{clip}%
\pgfsetbuttcap%
\pgfsetroundjoin%
\definecolor{currentfill}{rgb}{0.200000,0.200000,0.800000}%
\pgfsetfillcolor{currentfill}%
\pgfsetlinewidth{1.003750pt}%
\definecolor{currentstroke}{rgb}{0.200000,0.200000,0.800000}%
\pgfsetstrokecolor{currentstroke}%
\pgfsetdash{}{0pt}%
\pgfpathmoveto{\pgfqpoint{4.019710in}{1.942466in}}%
\pgfpathcurveto{\pgfqpoint{4.025534in}{1.942466in}}{\pgfqpoint{4.031120in}{1.944780in}}{\pgfqpoint{4.035238in}{1.948898in}}%
\pgfpathcurveto{\pgfqpoint{4.039356in}{1.953016in}}{\pgfqpoint{4.041670in}{1.958602in}}{\pgfqpoint{4.041670in}{1.964426in}}%
\pgfpathcurveto{\pgfqpoint{4.041670in}{1.970250in}}{\pgfqpoint{4.039356in}{1.975836in}}{\pgfqpoint{4.035238in}{1.979955in}}%
\pgfpathcurveto{\pgfqpoint{4.031120in}{1.984073in}}{\pgfqpoint{4.025534in}{1.986387in}}{\pgfqpoint{4.019710in}{1.986387in}}%
\pgfpathcurveto{\pgfqpoint{4.013886in}{1.986387in}}{\pgfqpoint{4.008300in}{1.984073in}}{\pgfqpoint{4.004182in}{1.979955in}}%
\pgfpathcurveto{\pgfqpoint{4.000063in}{1.975836in}}{\pgfqpoint{3.997750in}{1.970250in}}{\pgfqpoint{3.997750in}{1.964426in}}%
\pgfpathcurveto{\pgfqpoint{3.997750in}{1.958602in}}{\pgfqpoint{4.000063in}{1.953016in}}{\pgfqpoint{4.004182in}{1.948898in}}%
\pgfpathcurveto{\pgfqpoint{4.008300in}{1.944780in}}{\pgfqpoint{4.013886in}{1.942466in}}{\pgfqpoint{4.019710in}{1.942466in}}%
\pgfpathlineto{\pgfqpoint{4.019710in}{1.942466in}}%
\pgfpathclose%
\pgfusepath{stroke,fill}%
\end{pgfscope}%
\begin{pgfscope}%
\pgfpathrectangle{\pgfqpoint{1.542338in}{0.880000in}}{\pgfqpoint{5.115323in}{6.160000in}}%
\pgfusepath{clip}%
\pgfsetbuttcap%
\pgfsetroundjoin%
\definecolor{currentfill}{rgb}{0.200000,0.200000,0.800000}%
\pgfsetfillcolor{currentfill}%
\pgfsetlinewidth{1.003750pt}%
\definecolor{currentstroke}{rgb}{0.200000,0.200000,0.800000}%
\pgfsetstrokecolor{currentstroke}%
\pgfsetdash{}{0pt}%
\pgfpathmoveto{\pgfqpoint{4.077438in}{2.084644in}}%
\pgfpathcurveto{\pgfqpoint{4.083262in}{2.084644in}}{\pgfqpoint{4.088848in}{2.086958in}}{\pgfqpoint{4.092967in}{2.091076in}}%
\pgfpathcurveto{\pgfqpoint{4.097085in}{2.095195in}}{\pgfqpoint{4.099399in}{2.100781in}}{\pgfqpoint{4.099399in}{2.106605in}}%
\pgfpathcurveto{\pgfqpoint{4.099399in}{2.112429in}}{\pgfqpoint{4.097085in}{2.118015in}}{\pgfqpoint{4.092967in}{2.122133in}}%
\pgfpathcurveto{\pgfqpoint{4.088848in}{2.126251in}}{\pgfqpoint{4.083262in}{2.128565in}}{\pgfqpoint{4.077438in}{2.128565in}}%
\pgfpathcurveto{\pgfqpoint{4.071614in}{2.128565in}}{\pgfqpoint{4.066028in}{2.126251in}}{\pgfqpoint{4.061910in}{2.122133in}}%
\pgfpathcurveto{\pgfqpoint{4.057792in}{2.118015in}}{\pgfqpoint{4.055478in}{2.112429in}}{\pgfqpoint{4.055478in}{2.106605in}}%
\pgfpathcurveto{\pgfqpoint{4.055478in}{2.100781in}}{\pgfqpoint{4.057792in}{2.095195in}}{\pgfqpoint{4.061910in}{2.091076in}}%
\pgfpathcurveto{\pgfqpoint{4.066028in}{2.086958in}}{\pgfqpoint{4.071614in}{2.084644in}}{\pgfqpoint{4.077438in}{2.084644in}}%
\pgfpathlineto{\pgfqpoint{4.077438in}{2.084644in}}%
\pgfpathclose%
\pgfusepath{stroke,fill}%
\end{pgfscope}%
\begin{pgfscope}%
\pgfpathrectangle{\pgfqpoint{1.542338in}{0.880000in}}{\pgfqpoint{5.115323in}{6.160000in}}%
\pgfusepath{clip}%
\pgfsetbuttcap%
\pgfsetroundjoin%
\definecolor{currentfill}{rgb}{0.200000,0.200000,0.800000}%
\pgfsetfillcolor{currentfill}%
\pgfsetlinewidth{1.003750pt}%
\definecolor{currentstroke}{rgb}{0.200000,0.200000,0.800000}%
\pgfsetstrokecolor{currentstroke}%
\pgfsetdash{}{0pt}%
\pgfpathmoveto{\pgfqpoint{4.124703in}{2.229607in}}%
\pgfpathcurveto{\pgfqpoint{4.130527in}{2.229607in}}{\pgfqpoint{4.136113in}{2.231921in}}{\pgfqpoint{4.140232in}{2.236040in}}%
\pgfpathcurveto{\pgfqpoint{4.144350in}{2.240158in}}{\pgfqpoint{4.146664in}{2.245744in}}{\pgfqpoint{4.146664in}{2.251568in}}%
\pgfpathcurveto{\pgfqpoint{4.146664in}{2.257392in}}{\pgfqpoint{4.144350in}{2.262978in}}{\pgfqpoint{4.140232in}{2.267096in}}%
\pgfpathcurveto{\pgfqpoint{4.136113in}{2.271214in}}{\pgfqpoint{4.130527in}{2.273528in}}{\pgfqpoint{4.124703in}{2.273528in}}%
\pgfpathcurveto{\pgfqpoint{4.118879in}{2.273528in}}{\pgfqpoint{4.113293in}{2.271214in}}{\pgfqpoint{4.109175in}{2.267096in}}%
\pgfpathcurveto{\pgfqpoint{4.105057in}{2.262978in}}{\pgfqpoint{4.102743in}{2.257392in}}{\pgfqpoint{4.102743in}{2.251568in}}%
\pgfpathcurveto{\pgfqpoint{4.102743in}{2.245744in}}{\pgfqpoint{4.105057in}{2.240158in}}{\pgfqpoint{4.109175in}{2.236040in}}%
\pgfpathcurveto{\pgfqpoint{4.113293in}{2.231921in}}{\pgfqpoint{4.118879in}{2.229607in}}{\pgfqpoint{4.124703in}{2.229607in}}%
\pgfpathlineto{\pgfqpoint{4.124703in}{2.229607in}}%
\pgfpathclose%
\pgfusepath{stroke,fill}%
\end{pgfscope}%
\begin{pgfscope}%
\pgfpathrectangle{\pgfqpoint{1.542338in}{0.880000in}}{\pgfqpoint{5.115323in}{6.160000in}}%
\pgfusepath{clip}%
\pgfsetbuttcap%
\pgfsetroundjoin%
\definecolor{currentfill}{rgb}{0.200000,0.200000,0.800000}%
\pgfsetfillcolor{currentfill}%
\pgfsetlinewidth{1.003750pt}%
\definecolor{currentstroke}{rgb}{0.200000,0.200000,0.800000}%
\pgfsetstrokecolor{currentstroke}%
\pgfsetdash{}{0pt}%
\pgfpathmoveto{\pgfqpoint{4.155669in}{2.378980in}}%
\pgfpathcurveto{\pgfqpoint{4.161493in}{2.378980in}}{\pgfqpoint{4.167079in}{2.381294in}}{\pgfqpoint{4.171197in}{2.385412in}}%
\pgfpathcurveto{\pgfqpoint{4.175315in}{2.389530in}}{\pgfqpoint{4.177629in}{2.395116in}}{\pgfqpoint{4.177629in}{2.400940in}}%
\pgfpathcurveto{\pgfqpoint{4.177629in}{2.406764in}}{\pgfqpoint{4.175315in}{2.412350in}}{\pgfqpoint{4.171197in}{2.416468in}}%
\pgfpathcurveto{\pgfqpoint{4.167079in}{2.420586in}}{\pgfqpoint{4.161493in}{2.422900in}}{\pgfqpoint{4.155669in}{2.422900in}}%
\pgfpathcurveto{\pgfqpoint{4.149845in}{2.422900in}}{\pgfqpoint{4.144259in}{2.420586in}}{\pgfqpoint{4.140140in}{2.416468in}}%
\pgfpathcurveto{\pgfqpoint{4.136022in}{2.412350in}}{\pgfqpoint{4.133708in}{2.406764in}}{\pgfqpoint{4.133708in}{2.400940in}}%
\pgfpathcurveto{\pgfqpoint{4.133708in}{2.395116in}}{\pgfqpoint{4.136022in}{2.389530in}}{\pgfqpoint{4.140140in}{2.385412in}}%
\pgfpathcurveto{\pgfqpoint{4.144259in}{2.381294in}}{\pgfqpoint{4.149845in}{2.378980in}}{\pgfqpoint{4.155669in}{2.378980in}}%
\pgfpathlineto{\pgfqpoint{4.155669in}{2.378980in}}%
\pgfpathclose%
\pgfusepath{stroke,fill}%
\end{pgfscope}%
\begin{pgfscope}%
\pgfpathrectangle{\pgfqpoint{1.542338in}{0.880000in}}{\pgfqpoint{5.115323in}{6.160000in}}%
\pgfusepath{clip}%
\pgfsetbuttcap%
\pgfsetroundjoin%
\definecolor{currentfill}{rgb}{0.200000,0.200000,0.800000}%
\pgfsetfillcolor{currentfill}%
\pgfsetlinewidth{1.003750pt}%
\definecolor{currentstroke}{rgb}{0.200000,0.200000,0.800000}%
\pgfsetstrokecolor{currentstroke}%
\pgfsetdash{}{0pt}%
\pgfpathmoveto{\pgfqpoint{4.170437in}{2.531321in}}%
\pgfpathcurveto{\pgfqpoint{4.176261in}{2.531321in}}{\pgfqpoint{4.181847in}{2.533635in}}{\pgfqpoint{4.185965in}{2.537753in}}%
\pgfpathcurveto{\pgfqpoint{4.190083in}{2.541872in}}{\pgfqpoint{4.192397in}{2.547458in}}{\pgfqpoint{4.192397in}{2.553282in}}%
\pgfpathcurveto{\pgfqpoint{4.192397in}{2.559106in}}{\pgfqpoint{4.190083in}{2.564692in}}{\pgfqpoint{4.185965in}{2.568810in}}%
\pgfpathcurveto{\pgfqpoint{4.181847in}{2.572928in}}{\pgfqpoint{4.176261in}{2.575242in}}{\pgfqpoint{4.170437in}{2.575242in}}%
\pgfpathcurveto{\pgfqpoint{4.164613in}{2.575242in}}{\pgfqpoint{4.159027in}{2.572928in}}{\pgfqpoint{4.154909in}{2.568810in}}%
\pgfpathcurveto{\pgfqpoint{4.150790in}{2.564692in}}{\pgfqpoint{4.148477in}{2.559106in}}{\pgfqpoint{4.148477in}{2.553282in}}%
\pgfpathcurveto{\pgfqpoint{4.148477in}{2.547458in}}{\pgfqpoint{4.150790in}{2.541872in}}{\pgfqpoint{4.154909in}{2.537753in}}%
\pgfpathcurveto{\pgfqpoint{4.159027in}{2.533635in}}{\pgfqpoint{4.164613in}{2.531321in}}{\pgfqpoint{4.170437in}{2.531321in}}%
\pgfpathlineto{\pgfqpoint{4.170437in}{2.531321in}}%
\pgfpathclose%
\pgfusepath{stroke,fill}%
\end{pgfscope}%
\begin{pgfscope}%
\pgfpathrectangle{\pgfqpoint{1.542338in}{0.880000in}}{\pgfqpoint{5.115323in}{6.160000in}}%
\pgfusepath{clip}%
\pgfsetbuttcap%
\pgfsetroundjoin%
\definecolor{currentfill}{rgb}{0.200000,0.800000,0.200000}%
\pgfsetfillcolor{currentfill}%
\pgfsetlinewidth{1.003750pt}%
\definecolor{currentstroke}{rgb}{0.200000,0.800000,0.200000}%
\pgfsetstrokecolor{currentstroke}%
\pgfsetdash{}{0pt}%
\pgfpathmoveto{\pgfqpoint{5.439255in}{5.791555in}}%
\pgfpathcurveto{\pgfqpoint{5.445079in}{5.791555in}}{\pgfqpoint{5.450665in}{5.793869in}}{\pgfqpoint{5.454783in}{5.797987in}}%
\pgfpathcurveto{\pgfqpoint{5.458901in}{5.802105in}}{\pgfqpoint{5.461215in}{5.807691in}}{\pgfqpoint{5.461215in}{5.813515in}}%
\pgfpathcurveto{\pgfqpoint{5.461215in}{5.819339in}}{\pgfqpoint{5.458901in}{5.824925in}}{\pgfqpoint{5.454783in}{5.829043in}}%
\pgfpathcurveto{\pgfqpoint{5.450665in}{5.833162in}}{\pgfqpoint{5.445079in}{5.835475in}}{\pgfqpoint{5.439255in}{5.835475in}}%
\pgfpathcurveto{\pgfqpoint{5.433431in}{5.835475in}}{\pgfqpoint{5.427845in}{5.833162in}}{\pgfqpoint{5.423726in}{5.829043in}}%
\pgfpathcurveto{\pgfqpoint{5.419608in}{5.824925in}}{\pgfqpoint{5.417294in}{5.819339in}}{\pgfqpoint{5.417294in}{5.813515in}}%
\pgfpathcurveto{\pgfqpoint{5.417294in}{5.807691in}}{\pgfqpoint{5.419608in}{5.802105in}}{\pgfqpoint{5.423726in}{5.797987in}}%
\pgfpathcurveto{\pgfqpoint{5.427845in}{5.793869in}}{\pgfqpoint{5.433431in}{5.791555in}}{\pgfqpoint{5.439255in}{5.791555in}}%
\pgfpathlineto{\pgfqpoint{5.439255in}{5.791555in}}%
\pgfpathclose%
\pgfusepath{stroke,fill}%
\end{pgfscope}%
\begin{pgfscope}%
\pgfpathrectangle{\pgfqpoint{1.542338in}{0.880000in}}{\pgfqpoint{5.115323in}{6.160000in}}%
\pgfusepath{clip}%
\pgfsetbuttcap%
\pgfsetroundjoin%
\definecolor{currentfill}{rgb}{0.200000,0.800000,0.200000}%
\pgfsetfillcolor{currentfill}%
\pgfsetlinewidth{1.003750pt}%
\definecolor{currentstroke}{rgb}{0.200000,0.800000,0.200000}%
\pgfsetstrokecolor{currentstroke}%
\pgfsetdash{}{0pt}%
\pgfpathmoveto{\pgfqpoint{5.423202in}{5.913116in}}%
\pgfpathcurveto{\pgfqpoint{5.429026in}{5.913116in}}{\pgfqpoint{5.434613in}{5.915430in}}{\pgfqpoint{5.438731in}{5.919548in}}%
\pgfpathcurveto{\pgfqpoint{5.442849in}{5.923666in}}{\pgfqpoint{5.445163in}{5.929253in}}{\pgfqpoint{5.445163in}{5.935076in}}%
\pgfpathcurveto{\pgfqpoint{5.445163in}{5.940900in}}{\pgfqpoint{5.442849in}{5.946487in}}{\pgfqpoint{5.438731in}{5.950605in}}%
\pgfpathcurveto{\pgfqpoint{5.434613in}{5.954723in}}{\pgfqpoint{5.429026in}{5.957037in}}{\pgfqpoint{5.423202in}{5.957037in}}%
\pgfpathcurveto{\pgfqpoint{5.417378in}{5.957037in}}{\pgfqpoint{5.411792in}{5.954723in}}{\pgfqpoint{5.407674in}{5.950605in}}%
\pgfpathcurveto{\pgfqpoint{5.403556in}{5.946487in}}{\pgfqpoint{5.401242in}{5.940900in}}{\pgfqpoint{5.401242in}{5.935076in}}%
\pgfpathcurveto{\pgfqpoint{5.401242in}{5.929253in}}{\pgfqpoint{5.403556in}{5.923666in}}{\pgfqpoint{5.407674in}{5.919548in}}%
\pgfpathcurveto{\pgfqpoint{5.411792in}{5.915430in}}{\pgfqpoint{5.417378in}{5.913116in}}{\pgfqpoint{5.423202in}{5.913116in}}%
\pgfpathlineto{\pgfqpoint{5.423202in}{5.913116in}}%
\pgfpathclose%
\pgfusepath{stroke,fill}%
\end{pgfscope}%
\begin{pgfscope}%
\pgfpathrectangle{\pgfqpoint{1.542338in}{0.880000in}}{\pgfqpoint{5.115323in}{6.160000in}}%
\pgfusepath{clip}%
\pgfsetbuttcap%
\pgfsetroundjoin%
\definecolor{currentfill}{rgb}{0.200000,0.800000,0.200000}%
\pgfsetfillcolor{currentfill}%
\pgfsetlinewidth{1.003750pt}%
\definecolor{currentstroke}{rgb}{0.200000,0.800000,0.200000}%
\pgfsetstrokecolor{currentstroke}%
\pgfsetdash{}{0pt}%
\pgfpathmoveto{\pgfqpoint{5.396086in}{6.031677in}}%
\pgfpathcurveto{\pgfqpoint{5.401910in}{6.031677in}}{\pgfqpoint{5.407497in}{6.033991in}}{\pgfqpoint{5.411615in}{6.038109in}}%
\pgfpathcurveto{\pgfqpoint{5.415733in}{6.042227in}}{\pgfqpoint{5.418047in}{6.047813in}}{\pgfqpoint{5.418047in}{6.053637in}}%
\pgfpathcurveto{\pgfqpoint{5.418047in}{6.059461in}}{\pgfqpoint{5.415733in}{6.065047in}}{\pgfqpoint{5.411615in}{6.069165in}}%
\pgfpathcurveto{\pgfqpoint{5.407497in}{6.073284in}}{\pgfqpoint{5.401910in}{6.075597in}}{\pgfqpoint{5.396086in}{6.075597in}}%
\pgfpathcurveto{\pgfqpoint{5.390262in}{6.075597in}}{\pgfqpoint{5.384676in}{6.073284in}}{\pgfqpoint{5.380558in}{6.069165in}}%
\pgfpathcurveto{\pgfqpoint{5.376440in}{6.065047in}}{\pgfqpoint{5.374126in}{6.059461in}}{\pgfqpoint{5.374126in}{6.053637in}}%
\pgfpathcurveto{\pgfqpoint{5.374126in}{6.047813in}}{\pgfqpoint{5.376440in}{6.042227in}}{\pgfqpoint{5.380558in}{6.038109in}}%
\pgfpathcurveto{\pgfqpoint{5.384676in}{6.033991in}}{\pgfqpoint{5.390262in}{6.031677in}}{\pgfqpoint{5.396086in}{6.031677in}}%
\pgfpathlineto{\pgfqpoint{5.396086in}{6.031677in}}%
\pgfpathclose%
\pgfusepath{stroke,fill}%
\end{pgfscope}%
\begin{pgfscope}%
\pgfpathrectangle{\pgfqpoint{1.542338in}{0.880000in}}{\pgfqpoint{5.115323in}{6.160000in}}%
\pgfusepath{clip}%
\pgfsetbuttcap%
\pgfsetroundjoin%
\definecolor{currentfill}{rgb}{0.200000,0.800000,0.200000}%
\pgfsetfillcolor{currentfill}%
\pgfsetlinewidth{1.003750pt}%
\definecolor{currentstroke}{rgb}{0.200000,0.800000,0.200000}%
\pgfsetstrokecolor{currentstroke}%
\pgfsetdash{}{0pt}%
\pgfpathmoveto{\pgfqpoint{5.362552in}{6.148700in}}%
\pgfpathcurveto{\pgfqpoint{5.368376in}{6.148700in}}{\pgfqpoint{5.373962in}{6.151014in}}{\pgfqpoint{5.378080in}{6.155132in}}%
\pgfpathcurveto{\pgfqpoint{5.382198in}{6.159250in}}{\pgfqpoint{5.384512in}{6.164836in}}{\pgfqpoint{5.384512in}{6.170660in}}%
\pgfpathcurveto{\pgfqpoint{5.384512in}{6.176484in}}{\pgfqpoint{5.382198in}{6.182070in}}{\pgfqpoint{5.378080in}{6.186188in}}%
\pgfpathcurveto{\pgfqpoint{5.373962in}{6.190306in}}{\pgfqpoint{5.368376in}{6.192620in}}{\pgfqpoint{5.362552in}{6.192620in}}%
\pgfpathcurveto{\pgfqpoint{5.356728in}{6.192620in}}{\pgfqpoint{5.351142in}{6.190306in}}{\pgfqpoint{5.347023in}{6.186188in}}%
\pgfpathcurveto{\pgfqpoint{5.342905in}{6.182070in}}{\pgfqpoint{5.340591in}{6.176484in}}{\pgfqpoint{5.340591in}{6.170660in}}%
\pgfpathcurveto{\pgfqpoint{5.340591in}{6.164836in}}{\pgfqpoint{5.342905in}{6.159250in}}{\pgfqpoint{5.347023in}{6.155132in}}%
\pgfpathcurveto{\pgfqpoint{5.351142in}{6.151014in}}{\pgfqpoint{5.356728in}{6.148700in}}{\pgfqpoint{5.362552in}{6.148700in}}%
\pgfpathlineto{\pgfqpoint{5.362552in}{6.148700in}}%
\pgfpathclose%
\pgfusepath{stroke,fill}%
\end{pgfscope}%
\begin{pgfscope}%
\pgfpathrectangle{\pgfqpoint{1.542338in}{0.880000in}}{\pgfqpoint{5.115323in}{6.160000in}}%
\pgfusepath{clip}%
\pgfsetbuttcap%
\pgfsetroundjoin%
\definecolor{currentfill}{rgb}{0.200000,0.800000,0.200000}%
\pgfsetfillcolor{currentfill}%
\pgfsetlinewidth{1.003750pt}%
\definecolor{currentstroke}{rgb}{0.200000,0.800000,0.200000}%
\pgfsetstrokecolor{currentstroke}%
\pgfsetdash{}{0pt}%
\pgfpathmoveto{\pgfqpoint{5.307444in}{6.257339in}}%
\pgfpathcurveto{\pgfqpoint{5.313268in}{6.257339in}}{\pgfqpoint{5.318854in}{6.259653in}}{\pgfqpoint{5.322972in}{6.263771in}}%
\pgfpathcurveto{\pgfqpoint{5.327091in}{6.267889in}}{\pgfqpoint{5.329404in}{6.273476in}}{\pgfqpoint{5.329404in}{6.279299in}}%
\pgfpathcurveto{\pgfqpoint{5.329404in}{6.285123in}}{\pgfqpoint{5.327091in}{6.290710in}}{\pgfqpoint{5.322972in}{6.294828in}}%
\pgfpathcurveto{\pgfqpoint{5.318854in}{6.298946in}}{\pgfqpoint{5.313268in}{6.301260in}}{\pgfqpoint{5.307444in}{6.301260in}}%
\pgfpathcurveto{\pgfqpoint{5.301620in}{6.301260in}}{\pgfqpoint{5.296034in}{6.298946in}}{\pgfqpoint{5.291916in}{6.294828in}}%
\pgfpathcurveto{\pgfqpoint{5.287798in}{6.290710in}}{\pgfqpoint{5.285484in}{6.285123in}}{\pgfqpoint{5.285484in}{6.279299in}}%
\pgfpathcurveto{\pgfqpoint{5.285484in}{6.273476in}}{\pgfqpoint{5.287798in}{6.267889in}}{\pgfqpoint{5.291916in}{6.263771in}}%
\pgfpathcurveto{\pgfqpoint{5.296034in}{6.259653in}}{\pgfqpoint{5.301620in}{6.257339in}}{\pgfqpoint{5.307444in}{6.257339in}}%
\pgfpathlineto{\pgfqpoint{5.307444in}{6.257339in}}%
\pgfpathclose%
\pgfusepath{stroke,fill}%
\end{pgfscope}%
\begin{pgfscope}%
\pgfpathrectangle{\pgfqpoint{1.542338in}{0.880000in}}{\pgfqpoint{5.115323in}{6.160000in}}%
\pgfusepath{clip}%
\pgfsetbuttcap%
\pgfsetroundjoin%
\definecolor{currentfill}{rgb}{0.200000,0.800000,0.200000}%
\pgfsetfillcolor{currentfill}%
\pgfsetlinewidth{1.003750pt}%
\definecolor{currentstroke}{rgb}{0.200000,0.800000,0.200000}%
\pgfsetstrokecolor{currentstroke}%
\pgfsetdash{}{0pt}%
\pgfpathmoveto{\pgfqpoint{5.235079in}{6.354789in}}%
\pgfpathcurveto{\pgfqpoint{5.240903in}{6.354789in}}{\pgfqpoint{5.246489in}{6.357103in}}{\pgfqpoint{5.250608in}{6.361221in}}%
\pgfpathcurveto{\pgfqpoint{5.254726in}{6.365339in}}{\pgfqpoint{5.257040in}{6.370925in}}{\pgfqpoint{5.257040in}{6.376749in}}%
\pgfpathcurveto{\pgfqpoint{5.257040in}{6.382573in}}{\pgfqpoint{5.254726in}{6.388159in}}{\pgfqpoint{5.250608in}{6.392277in}}%
\pgfpathcurveto{\pgfqpoint{5.246489in}{6.396396in}}{\pgfqpoint{5.240903in}{6.398710in}}{\pgfqpoint{5.235079in}{6.398710in}}%
\pgfpathcurveto{\pgfqpoint{5.229255in}{6.398710in}}{\pgfqpoint{5.223669in}{6.396396in}}{\pgfqpoint{5.219551in}{6.392277in}}%
\pgfpathcurveto{\pgfqpoint{5.215433in}{6.388159in}}{\pgfqpoint{5.213119in}{6.382573in}}{\pgfqpoint{5.213119in}{6.376749in}}%
\pgfpathcurveto{\pgfqpoint{5.213119in}{6.370925in}}{\pgfqpoint{5.215433in}{6.365339in}}{\pgfqpoint{5.219551in}{6.361221in}}%
\pgfpathcurveto{\pgfqpoint{5.223669in}{6.357103in}}{\pgfqpoint{5.229255in}{6.354789in}}{\pgfqpoint{5.235079in}{6.354789in}}%
\pgfpathlineto{\pgfqpoint{5.235079in}{6.354789in}}%
\pgfpathclose%
\pgfusepath{stroke,fill}%
\end{pgfscope}%
\begin{pgfscope}%
\pgfpathrectangle{\pgfqpoint{1.542338in}{0.880000in}}{\pgfqpoint{5.115323in}{6.160000in}}%
\pgfusepath{clip}%
\pgfsetbuttcap%
\pgfsetroundjoin%
\definecolor{currentfill}{rgb}{0.200000,0.800000,0.200000}%
\pgfsetfillcolor{currentfill}%
\pgfsetlinewidth{1.003750pt}%
\definecolor{currentstroke}{rgb}{0.200000,0.800000,0.200000}%
\pgfsetstrokecolor{currentstroke}%
\pgfsetdash{}{0pt}%
\pgfpathmoveto{\pgfqpoint{5.158410in}{6.448174in}}%
\pgfpathcurveto{\pgfqpoint{5.164234in}{6.448174in}}{\pgfqpoint{5.169821in}{6.450488in}}{\pgfqpoint{5.173939in}{6.454606in}}%
\pgfpathcurveto{\pgfqpoint{5.178057in}{6.458724in}}{\pgfqpoint{5.180371in}{6.464311in}}{\pgfqpoint{5.180371in}{6.470135in}}%
\pgfpathcurveto{\pgfqpoint{5.180371in}{6.475958in}}{\pgfqpoint{5.178057in}{6.481545in}}{\pgfqpoint{5.173939in}{6.485663in}}%
\pgfpathcurveto{\pgfqpoint{5.169821in}{6.489781in}}{\pgfqpoint{5.164234in}{6.492095in}}{\pgfqpoint{5.158410in}{6.492095in}}%
\pgfpathcurveto{\pgfqpoint{5.152586in}{6.492095in}}{\pgfqpoint{5.147000in}{6.489781in}}{\pgfqpoint{5.142882in}{6.485663in}}%
\pgfpathcurveto{\pgfqpoint{5.138764in}{6.481545in}}{\pgfqpoint{5.136450in}{6.475958in}}{\pgfqpoint{5.136450in}{6.470135in}}%
\pgfpathcurveto{\pgfqpoint{5.136450in}{6.464311in}}{\pgfqpoint{5.138764in}{6.458724in}}{\pgfqpoint{5.142882in}{6.454606in}}%
\pgfpathcurveto{\pgfqpoint{5.147000in}{6.450488in}}{\pgfqpoint{5.152586in}{6.448174in}}{\pgfqpoint{5.158410in}{6.448174in}}%
\pgfpathlineto{\pgfqpoint{5.158410in}{6.448174in}}%
\pgfpathclose%
\pgfusepath{stroke,fill}%
\end{pgfscope}%
\begin{pgfscope}%
\pgfpathrectangle{\pgfqpoint{1.542338in}{0.880000in}}{\pgfqpoint{5.115323in}{6.160000in}}%
\pgfusepath{clip}%
\pgfsetbuttcap%
\pgfsetroundjoin%
\definecolor{currentfill}{rgb}{0.200000,0.800000,0.200000}%
\pgfsetfillcolor{currentfill}%
\pgfsetlinewidth{1.003750pt}%
\definecolor{currentstroke}{rgb}{0.200000,0.800000,0.200000}%
\pgfsetstrokecolor{currentstroke}%
\pgfsetdash{}{0pt}%
\pgfpathmoveto{\pgfqpoint{5.073019in}{6.534680in}}%
\pgfpathcurveto{\pgfqpoint{5.078843in}{6.534680in}}{\pgfqpoint{5.084430in}{6.536994in}}{\pgfqpoint{5.088548in}{6.541112in}}%
\pgfpathcurveto{\pgfqpoint{5.092666in}{6.545230in}}{\pgfqpoint{5.094980in}{6.550816in}}{\pgfqpoint{5.094980in}{6.556640in}}%
\pgfpathcurveto{\pgfqpoint{5.094980in}{6.562464in}}{\pgfqpoint{5.092666in}{6.568050in}}{\pgfqpoint{5.088548in}{6.572168in}}%
\pgfpathcurveto{\pgfqpoint{5.084430in}{6.576287in}}{\pgfqpoint{5.078843in}{6.578600in}}{\pgfqpoint{5.073019in}{6.578600in}}%
\pgfpathcurveto{\pgfqpoint{5.067195in}{6.578600in}}{\pgfqpoint{5.061609in}{6.576287in}}{\pgfqpoint{5.057491in}{6.572168in}}%
\pgfpathcurveto{\pgfqpoint{5.053373in}{6.568050in}}{\pgfqpoint{5.051059in}{6.562464in}}{\pgfqpoint{5.051059in}{6.556640in}}%
\pgfpathcurveto{\pgfqpoint{5.051059in}{6.550816in}}{\pgfqpoint{5.053373in}{6.545230in}}{\pgfqpoint{5.057491in}{6.541112in}}%
\pgfpathcurveto{\pgfqpoint{5.061609in}{6.536994in}}{\pgfqpoint{5.067195in}{6.534680in}}{\pgfqpoint{5.073019in}{6.534680in}}%
\pgfpathlineto{\pgfqpoint{5.073019in}{6.534680in}}%
\pgfpathclose%
\pgfusepath{stroke,fill}%
\end{pgfscope}%
\begin{pgfscope}%
\pgfpathrectangle{\pgfqpoint{1.542338in}{0.880000in}}{\pgfqpoint{5.115323in}{6.160000in}}%
\pgfusepath{clip}%
\pgfsetbuttcap%
\pgfsetroundjoin%
\definecolor{currentfill}{rgb}{0.200000,0.800000,0.200000}%
\pgfsetfillcolor{currentfill}%
\pgfsetlinewidth{1.003750pt}%
\definecolor{currentstroke}{rgb}{0.200000,0.800000,0.200000}%
\pgfsetstrokecolor{currentstroke}%
\pgfsetdash{}{0pt}%
\pgfpathmoveto{\pgfqpoint{4.968984in}{6.597530in}}%
\pgfpathcurveto{\pgfqpoint{4.974808in}{6.597530in}}{\pgfqpoint{4.980394in}{6.599844in}}{\pgfqpoint{4.984512in}{6.603962in}}%
\pgfpathcurveto{\pgfqpoint{4.988630in}{6.608080in}}{\pgfqpoint{4.990944in}{6.613667in}}{\pgfqpoint{4.990944in}{6.619491in}}%
\pgfpathcurveto{\pgfqpoint{4.990944in}{6.625314in}}{\pgfqpoint{4.988630in}{6.630901in}}{\pgfqpoint{4.984512in}{6.635019in}}%
\pgfpathcurveto{\pgfqpoint{4.980394in}{6.639137in}}{\pgfqpoint{4.974808in}{6.641451in}}{\pgfqpoint{4.968984in}{6.641451in}}%
\pgfpathcurveto{\pgfqpoint{4.963160in}{6.641451in}}{\pgfqpoint{4.957574in}{6.639137in}}{\pgfqpoint{4.953456in}{6.635019in}}%
\pgfpathcurveto{\pgfqpoint{4.949338in}{6.630901in}}{\pgfqpoint{4.947024in}{6.625314in}}{\pgfqpoint{4.947024in}{6.619491in}}%
\pgfpathcurveto{\pgfqpoint{4.947024in}{6.613667in}}{\pgfqpoint{4.949338in}{6.608080in}}{\pgfqpoint{4.953456in}{6.603962in}}%
\pgfpathcurveto{\pgfqpoint{4.957574in}{6.599844in}}{\pgfqpoint{4.963160in}{6.597530in}}{\pgfqpoint{4.968984in}{6.597530in}}%
\pgfpathlineto{\pgfqpoint{4.968984in}{6.597530in}}%
\pgfpathclose%
\pgfusepath{stroke,fill}%
\end{pgfscope}%
\begin{pgfscope}%
\pgfpathrectangle{\pgfqpoint{1.542338in}{0.880000in}}{\pgfqpoint{5.115323in}{6.160000in}}%
\pgfusepath{clip}%
\pgfsetbuttcap%
\pgfsetroundjoin%
\definecolor{currentfill}{rgb}{0.200000,0.800000,0.200000}%
\pgfsetfillcolor{currentfill}%
\pgfsetlinewidth{1.003750pt}%
\definecolor{currentstroke}{rgb}{0.200000,0.800000,0.200000}%
\pgfsetstrokecolor{currentstroke}%
\pgfsetdash{}{0pt}%
\pgfpathmoveto{\pgfqpoint{4.864110in}{6.658369in}}%
\pgfpathcurveto{\pgfqpoint{4.869934in}{6.658369in}}{\pgfqpoint{4.875520in}{6.660683in}}{\pgfqpoint{4.879638in}{6.664801in}}%
\pgfpathcurveto{\pgfqpoint{4.883756in}{6.668920in}}{\pgfqpoint{4.886070in}{6.674506in}}{\pgfqpoint{4.886070in}{6.680330in}}%
\pgfpathcurveto{\pgfqpoint{4.886070in}{6.686154in}}{\pgfqpoint{4.883756in}{6.691740in}}{\pgfqpoint{4.879638in}{6.695858in}}%
\pgfpathcurveto{\pgfqpoint{4.875520in}{6.699976in}}{\pgfqpoint{4.869934in}{6.702290in}}{\pgfqpoint{4.864110in}{6.702290in}}%
\pgfpathcurveto{\pgfqpoint{4.858286in}{6.702290in}}{\pgfqpoint{4.852700in}{6.699976in}}{\pgfqpoint{4.848582in}{6.695858in}}%
\pgfpathcurveto{\pgfqpoint{4.844464in}{6.691740in}}{\pgfqpoint{4.842150in}{6.686154in}}{\pgfqpoint{4.842150in}{6.680330in}}%
\pgfpathcurveto{\pgfqpoint{4.842150in}{6.674506in}}{\pgfqpoint{4.844464in}{6.668920in}}{\pgfqpoint{4.848582in}{6.664801in}}%
\pgfpathcurveto{\pgfqpoint{4.852700in}{6.660683in}}{\pgfqpoint{4.858286in}{6.658369in}}{\pgfqpoint{4.864110in}{6.658369in}}%
\pgfpathlineto{\pgfqpoint{4.864110in}{6.658369in}}%
\pgfpathclose%
\pgfusepath{stroke,fill}%
\end{pgfscope}%
\begin{pgfscope}%
\pgfpathrectangle{\pgfqpoint{1.542338in}{0.880000in}}{\pgfqpoint{5.115323in}{6.160000in}}%
\pgfusepath{clip}%
\pgfsetbuttcap%
\pgfsetroundjoin%
\definecolor{currentfill}{rgb}{0.200000,0.800000,0.200000}%
\pgfsetfillcolor{currentfill}%
\pgfsetlinewidth{1.003750pt}%
\definecolor{currentstroke}{rgb}{0.200000,0.800000,0.200000}%
\pgfsetstrokecolor{currentstroke}%
\pgfsetdash{}{0pt}%
\pgfpathmoveto{\pgfqpoint{4.750433in}{6.701396in}}%
\pgfpathcurveto{\pgfqpoint{4.756257in}{6.701396in}}{\pgfqpoint{4.761843in}{6.703710in}}{\pgfqpoint{4.765961in}{6.707828in}}%
\pgfpathcurveto{\pgfqpoint{4.770080in}{6.711947in}}{\pgfqpoint{4.772393in}{6.717533in}}{\pgfqpoint{4.772393in}{6.723357in}}%
\pgfpathcurveto{\pgfqpoint{4.772393in}{6.729181in}}{\pgfqpoint{4.770080in}{6.734767in}}{\pgfqpoint{4.765961in}{6.738885in}}%
\pgfpathcurveto{\pgfqpoint{4.761843in}{6.743003in}}{\pgfqpoint{4.756257in}{6.745317in}}{\pgfqpoint{4.750433in}{6.745317in}}%
\pgfpathcurveto{\pgfqpoint{4.744609in}{6.745317in}}{\pgfqpoint{4.739023in}{6.743003in}}{\pgfqpoint{4.734905in}{6.738885in}}%
\pgfpathcurveto{\pgfqpoint{4.730787in}{6.734767in}}{\pgfqpoint{4.728473in}{6.729181in}}{\pgfqpoint{4.728473in}{6.723357in}}%
\pgfpathcurveto{\pgfqpoint{4.728473in}{6.717533in}}{\pgfqpoint{4.730787in}{6.711947in}}{\pgfqpoint{4.734905in}{6.707828in}}%
\pgfpathcurveto{\pgfqpoint{4.739023in}{6.703710in}}{\pgfqpoint{4.744609in}{6.701396in}}{\pgfqpoint{4.750433in}{6.701396in}}%
\pgfpathlineto{\pgfqpoint{4.750433in}{6.701396in}}%
\pgfpathclose%
\pgfusepath{stroke,fill}%
\end{pgfscope}%
\begin{pgfscope}%
\pgfpathrectangle{\pgfqpoint{1.542338in}{0.880000in}}{\pgfqpoint{5.115323in}{6.160000in}}%
\pgfusepath{clip}%
\pgfsetbuttcap%
\pgfsetroundjoin%
\definecolor{currentfill}{rgb}{0.200000,0.800000,0.200000}%
\pgfsetfillcolor{currentfill}%
\pgfsetlinewidth{1.003750pt}%
\definecolor{currentstroke}{rgb}{0.200000,0.800000,0.200000}%
\pgfsetstrokecolor{currentstroke}%
\pgfsetdash{}{0pt}%
\pgfpathmoveto{\pgfqpoint{4.631269in}{6.724751in}}%
\pgfpathcurveto{\pgfqpoint{4.637093in}{6.724751in}}{\pgfqpoint{4.642679in}{6.727065in}}{\pgfqpoint{4.646797in}{6.731183in}}%
\pgfpathcurveto{\pgfqpoint{4.650915in}{6.735301in}}{\pgfqpoint{4.653229in}{6.740888in}}{\pgfqpoint{4.653229in}{6.746711in}}%
\pgfpathcurveto{\pgfqpoint{4.653229in}{6.752535in}}{\pgfqpoint{4.650915in}{6.758122in}}{\pgfqpoint{4.646797in}{6.762240in}}%
\pgfpathcurveto{\pgfqpoint{4.642679in}{6.766358in}}{\pgfqpoint{4.637093in}{6.768672in}}{\pgfqpoint{4.631269in}{6.768672in}}%
\pgfpathcurveto{\pgfqpoint{4.625445in}{6.768672in}}{\pgfqpoint{4.619859in}{6.766358in}}{\pgfqpoint{4.615741in}{6.762240in}}%
\pgfpathcurveto{\pgfqpoint{4.611623in}{6.758122in}}{\pgfqpoint{4.609309in}{6.752535in}}{\pgfqpoint{4.609309in}{6.746711in}}%
\pgfpathcurveto{\pgfqpoint{4.609309in}{6.740888in}}{\pgfqpoint{4.611623in}{6.735301in}}{\pgfqpoint{4.615741in}{6.731183in}}%
\pgfpathcurveto{\pgfqpoint{4.619859in}{6.727065in}}{\pgfqpoint{4.625445in}{6.724751in}}{\pgfqpoint{4.631269in}{6.724751in}}%
\pgfpathlineto{\pgfqpoint{4.631269in}{6.724751in}}%
\pgfpathclose%
\pgfusepath{stroke,fill}%
\end{pgfscope}%
\begin{pgfscope}%
\pgfpathrectangle{\pgfqpoint{1.542338in}{0.880000in}}{\pgfqpoint{5.115323in}{6.160000in}}%
\pgfusepath{clip}%
\pgfsetbuttcap%
\pgfsetroundjoin%
\definecolor{currentfill}{rgb}{0.200000,0.800000,0.200000}%
\pgfsetfillcolor{currentfill}%
\pgfsetlinewidth{1.003750pt}%
\definecolor{currentstroke}{rgb}{0.200000,0.800000,0.200000}%
\pgfsetstrokecolor{currentstroke}%
\pgfsetdash{}{0pt}%
\pgfpathmoveto{\pgfqpoint{4.510749in}{6.738040in}}%
\pgfpathcurveto{\pgfqpoint{4.516573in}{6.738040in}}{\pgfqpoint{4.522159in}{6.740354in}}{\pgfqpoint{4.526277in}{6.744472in}}%
\pgfpathcurveto{\pgfqpoint{4.530395in}{6.748590in}}{\pgfqpoint{4.532709in}{6.754176in}}{\pgfqpoint{4.532709in}{6.760000in}}%
\pgfpathcurveto{\pgfqpoint{4.532709in}{6.765824in}}{\pgfqpoint{4.530395in}{6.771410in}}{\pgfqpoint{4.526277in}{6.775528in}}%
\pgfpathcurveto{\pgfqpoint{4.522159in}{6.779646in}}{\pgfqpoint{4.516573in}{6.781960in}}{\pgfqpoint{4.510749in}{6.781960in}}%
\pgfpathcurveto{\pgfqpoint{4.504925in}{6.781960in}}{\pgfqpoint{4.499339in}{6.779646in}}{\pgfqpoint{4.495221in}{6.775528in}}%
\pgfpathcurveto{\pgfqpoint{4.491103in}{6.771410in}}{\pgfqpoint{4.488789in}{6.765824in}}{\pgfqpoint{4.488789in}{6.760000in}}%
\pgfpathcurveto{\pgfqpoint{4.488789in}{6.754176in}}{\pgfqpoint{4.491103in}{6.748590in}}{\pgfqpoint{4.495221in}{6.744472in}}%
\pgfpathcurveto{\pgfqpoint{4.499339in}{6.740354in}}{\pgfqpoint{4.504925in}{6.738040in}}{\pgfqpoint{4.510749in}{6.738040in}}%
\pgfpathlineto{\pgfqpoint{4.510749in}{6.738040in}}%
\pgfpathclose%
\pgfusepath{stroke,fill}%
\end{pgfscope}%
\begin{pgfscope}%
\pgfpathrectangle{\pgfqpoint{1.542338in}{0.880000in}}{\pgfqpoint{5.115323in}{6.160000in}}%
\pgfusepath{clip}%
\pgfsetbuttcap%
\pgfsetroundjoin%
\definecolor{currentfill}{rgb}{0.200000,0.800000,0.200000}%
\pgfsetfillcolor{currentfill}%
\pgfsetlinewidth{1.003750pt}%
\definecolor{currentstroke}{rgb}{0.200000,0.800000,0.200000}%
\pgfsetstrokecolor{currentstroke}%
\pgfsetdash{}{0pt}%
\pgfpathmoveto{\pgfqpoint{4.389204in}{6.736865in}}%
\pgfpathcurveto{\pgfqpoint{4.395028in}{6.736865in}}{\pgfqpoint{4.400614in}{6.739179in}}{\pgfqpoint{4.404732in}{6.743297in}}%
\pgfpathcurveto{\pgfqpoint{4.408850in}{6.747416in}}{\pgfqpoint{4.411164in}{6.753002in}}{\pgfqpoint{4.411164in}{6.758826in}}%
\pgfpathcurveto{\pgfqpoint{4.411164in}{6.764650in}}{\pgfqpoint{4.408850in}{6.770236in}}{\pgfqpoint{4.404732in}{6.774354in}}%
\pgfpathcurveto{\pgfqpoint{4.400614in}{6.778472in}}{\pgfqpoint{4.395028in}{6.780786in}}{\pgfqpoint{4.389204in}{6.780786in}}%
\pgfpathcurveto{\pgfqpoint{4.383380in}{6.780786in}}{\pgfqpoint{4.377794in}{6.778472in}}{\pgfqpoint{4.373676in}{6.774354in}}%
\pgfpathcurveto{\pgfqpoint{4.369558in}{6.770236in}}{\pgfqpoint{4.367244in}{6.764650in}}{\pgfqpoint{4.367244in}{6.758826in}}%
\pgfpathcurveto{\pgfqpoint{4.367244in}{6.753002in}}{\pgfqpoint{4.369558in}{6.747416in}}{\pgfqpoint{4.373676in}{6.743297in}}%
\pgfpathcurveto{\pgfqpoint{4.377794in}{6.739179in}}{\pgfqpoint{4.383380in}{6.736865in}}{\pgfqpoint{4.389204in}{6.736865in}}%
\pgfpathlineto{\pgfqpoint{4.389204in}{6.736865in}}%
\pgfpathclose%
\pgfusepath{stroke,fill}%
\end{pgfscope}%
\begin{pgfscope}%
\pgfpathrectangle{\pgfqpoint{1.542338in}{0.880000in}}{\pgfqpoint{5.115323in}{6.160000in}}%
\pgfusepath{clip}%
\pgfsetbuttcap%
\pgfsetroundjoin%
\definecolor{currentfill}{rgb}{0.200000,0.800000,0.200000}%
\pgfsetfillcolor{currentfill}%
\pgfsetlinewidth{1.003750pt}%
\definecolor{currentstroke}{rgb}{0.200000,0.800000,0.200000}%
\pgfsetstrokecolor{currentstroke}%
\pgfsetdash{}{0pt}%
\pgfpathmoveto{\pgfqpoint{4.269112in}{6.717256in}}%
\pgfpathcurveto{\pgfqpoint{4.274936in}{6.717256in}}{\pgfqpoint{4.280522in}{6.719570in}}{\pgfqpoint{4.284640in}{6.723688in}}%
\pgfpathcurveto{\pgfqpoint{4.288758in}{6.727806in}}{\pgfqpoint{4.291072in}{6.733393in}}{\pgfqpoint{4.291072in}{6.739217in}}%
\pgfpathcurveto{\pgfqpoint{4.291072in}{6.745040in}}{\pgfqpoint{4.288758in}{6.750627in}}{\pgfqpoint{4.284640in}{6.754745in}}%
\pgfpathcurveto{\pgfqpoint{4.280522in}{6.758863in}}{\pgfqpoint{4.274936in}{6.761177in}}{\pgfqpoint{4.269112in}{6.761177in}}%
\pgfpathcurveto{\pgfqpoint{4.263288in}{6.761177in}}{\pgfqpoint{4.257702in}{6.758863in}}{\pgfqpoint{4.253583in}{6.754745in}}%
\pgfpathcurveto{\pgfqpoint{4.249465in}{6.750627in}}{\pgfqpoint{4.247151in}{6.745040in}}{\pgfqpoint{4.247151in}{6.739217in}}%
\pgfpathcurveto{\pgfqpoint{4.247151in}{6.733393in}}{\pgfqpoint{4.249465in}{6.727806in}}{\pgfqpoint{4.253583in}{6.723688in}}%
\pgfpathcurveto{\pgfqpoint{4.257702in}{6.719570in}}{\pgfqpoint{4.263288in}{6.717256in}}{\pgfqpoint{4.269112in}{6.717256in}}%
\pgfpathlineto{\pgfqpoint{4.269112in}{6.717256in}}%
\pgfpathclose%
\pgfusepath{stroke,fill}%
\end{pgfscope}%
\begin{pgfscope}%
\pgfpathrectangle{\pgfqpoint{1.542338in}{0.880000in}}{\pgfqpoint{5.115323in}{6.160000in}}%
\pgfusepath{clip}%
\pgfsetbuttcap%
\pgfsetroundjoin%
\definecolor{currentfill}{rgb}{0.200000,0.800000,0.200000}%
\pgfsetfillcolor{currentfill}%
\pgfsetlinewidth{1.003750pt}%
\definecolor{currentstroke}{rgb}{0.200000,0.800000,0.200000}%
\pgfsetstrokecolor{currentstroke}%
\pgfsetdash{}{0pt}%
\pgfpathmoveto{\pgfqpoint{4.152936in}{6.681372in}}%
\pgfpathcurveto{\pgfqpoint{4.158760in}{6.681372in}}{\pgfqpoint{4.164346in}{6.683686in}}{\pgfqpoint{4.168464in}{6.687804in}}%
\pgfpathcurveto{\pgfqpoint{4.172582in}{6.691922in}}{\pgfqpoint{4.174896in}{6.697509in}}{\pgfqpoint{4.174896in}{6.703332in}}%
\pgfpathcurveto{\pgfqpoint{4.174896in}{6.709156in}}{\pgfqpoint{4.172582in}{6.714743in}}{\pgfqpoint{4.168464in}{6.718861in}}%
\pgfpathcurveto{\pgfqpoint{4.164346in}{6.722979in}}{\pgfqpoint{4.158760in}{6.725293in}}{\pgfqpoint{4.152936in}{6.725293in}}%
\pgfpathcurveto{\pgfqpoint{4.147112in}{6.725293in}}{\pgfqpoint{4.141526in}{6.722979in}}{\pgfqpoint{4.137408in}{6.718861in}}%
\pgfpathcurveto{\pgfqpoint{4.133290in}{6.714743in}}{\pgfqpoint{4.130976in}{6.709156in}}{\pgfqpoint{4.130976in}{6.703332in}}%
\pgfpathcurveto{\pgfqpoint{4.130976in}{6.697509in}}{\pgfqpoint{4.133290in}{6.691922in}}{\pgfqpoint{4.137408in}{6.687804in}}%
\pgfpathcurveto{\pgfqpoint{4.141526in}{6.683686in}}{\pgfqpoint{4.147112in}{6.681372in}}{\pgfqpoint{4.152936in}{6.681372in}}%
\pgfpathlineto{\pgfqpoint{4.152936in}{6.681372in}}%
\pgfpathclose%
\pgfusepath{stroke,fill}%
\end{pgfscope}%
\begin{pgfscope}%
\pgfpathrectangle{\pgfqpoint{1.542338in}{0.880000in}}{\pgfqpoint{5.115323in}{6.160000in}}%
\pgfusepath{clip}%
\pgfsetbuttcap%
\pgfsetroundjoin%
\definecolor{currentfill}{rgb}{0.200000,0.800000,0.200000}%
\pgfsetfillcolor{currentfill}%
\pgfsetlinewidth{1.003750pt}%
\definecolor{currentstroke}{rgb}{0.200000,0.800000,0.200000}%
\pgfsetstrokecolor{currentstroke}%
\pgfsetdash{}{0pt}%
\pgfpathmoveto{\pgfqpoint{4.042693in}{6.630551in}}%
\pgfpathcurveto{\pgfqpoint{4.048517in}{6.630551in}}{\pgfqpoint{4.054104in}{6.632865in}}{\pgfqpoint{4.058222in}{6.636983in}}%
\pgfpathcurveto{\pgfqpoint{4.062340in}{6.641101in}}{\pgfqpoint{4.064654in}{6.646687in}}{\pgfqpoint{4.064654in}{6.652511in}}%
\pgfpathcurveto{\pgfqpoint{4.064654in}{6.658335in}}{\pgfqpoint{4.062340in}{6.663921in}}{\pgfqpoint{4.058222in}{6.668039in}}%
\pgfpathcurveto{\pgfqpoint{4.054104in}{6.672157in}}{\pgfqpoint{4.048517in}{6.674471in}}{\pgfqpoint{4.042693in}{6.674471in}}%
\pgfpathcurveto{\pgfqpoint{4.036869in}{6.674471in}}{\pgfqpoint{4.031283in}{6.672157in}}{\pgfqpoint{4.027165in}{6.668039in}}%
\pgfpathcurveto{\pgfqpoint{4.023047in}{6.663921in}}{\pgfqpoint{4.020733in}{6.658335in}}{\pgfqpoint{4.020733in}{6.652511in}}%
\pgfpathcurveto{\pgfqpoint{4.020733in}{6.646687in}}{\pgfqpoint{4.023047in}{6.641101in}}{\pgfqpoint{4.027165in}{6.636983in}}%
\pgfpathcurveto{\pgfqpoint{4.031283in}{6.632865in}}{\pgfqpoint{4.036869in}{6.630551in}}{\pgfqpoint{4.042693in}{6.630551in}}%
\pgfpathlineto{\pgfqpoint{4.042693in}{6.630551in}}%
\pgfpathclose%
\pgfusepath{stroke,fill}%
\end{pgfscope}%
\begin{pgfscope}%
\pgfpathrectangle{\pgfqpoint{1.542338in}{0.880000in}}{\pgfqpoint{5.115323in}{6.160000in}}%
\pgfusepath{clip}%
\pgfsetbuttcap%
\pgfsetroundjoin%
\definecolor{currentfill}{rgb}{0.200000,0.800000,0.200000}%
\pgfsetfillcolor{currentfill}%
\pgfsetlinewidth{1.003750pt}%
\definecolor{currentstroke}{rgb}{0.200000,0.800000,0.200000}%
\pgfsetstrokecolor{currentstroke}%
\pgfsetdash{}{0pt}%
\pgfpathmoveto{\pgfqpoint{3.937620in}{6.569667in}}%
\pgfpathcurveto{\pgfqpoint{3.943444in}{6.569667in}}{\pgfqpoint{3.949030in}{6.571981in}}{\pgfqpoint{3.953148in}{6.576099in}}%
\pgfpathcurveto{\pgfqpoint{3.957266in}{6.580217in}}{\pgfqpoint{3.959580in}{6.585803in}}{\pgfqpoint{3.959580in}{6.591627in}}%
\pgfpathcurveto{\pgfqpoint{3.959580in}{6.597451in}}{\pgfqpoint{3.957266in}{6.603037in}}{\pgfqpoint{3.953148in}{6.607155in}}%
\pgfpathcurveto{\pgfqpoint{3.949030in}{6.611274in}}{\pgfqpoint{3.943444in}{6.613587in}}{\pgfqpoint{3.937620in}{6.613587in}}%
\pgfpathcurveto{\pgfqpoint{3.931796in}{6.613587in}}{\pgfqpoint{3.926210in}{6.611274in}}{\pgfqpoint{3.922092in}{6.607155in}}%
\pgfpathcurveto{\pgfqpoint{3.917973in}{6.603037in}}{\pgfqpoint{3.915660in}{6.597451in}}{\pgfqpoint{3.915660in}{6.591627in}}%
\pgfpathcurveto{\pgfqpoint{3.915660in}{6.585803in}}{\pgfqpoint{3.917973in}{6.580217in}}{\pgfqpoint{3.922092in}{6.576099in}}%
\pgfpathcurveto{\pgfqpoint{3.926210in}{6.571981in}}{\pgfqpoint{3.931796in}{6.569667in}}{\pgfqpoint{3.937620in}{6.569667in}}%
\pgfpathlineto{\pgfqpoint{3.937620in}{6.569667in}}%
\pgfpathclose%
\pgfusepath{stroke,fill}%
\end{pgfscope}%
\begin{pgfscope}%
\pgfpathrectangle{\pgfqpoint{1.542338in}{0.880000in}}{\pgfqpoint{5.115323in}{6.160000in}}%
\pgfusepath{clip}%
\pgfsetbuttcap%
\pgfsetroundjoin%
\definecolor{currentfill}{rgb}{0.200000,0.800000,0.200000}%
\pgfsetfillcolor{currentfill}%
\pgfsetlinewidth{1.003750pt}%
\definecolor{currentstroke}{rgb}{0.200000,0.800000,0.200000}%
\pgfsetstrokecolor{currentstroke}%
\pgfsetdash{}{0pt}%
\pgfpathmoveto{\pgfqpoint{3.849614in}{6.486117in}}%
\pgfpathcurveto{\pgfqpoint{3.855438in}{6.486117in}}{\pgfqpoint{3.861024in}{6.488431in}}{\pgfqpoint{3.865143in}{6.492549in}}%
\pgfpathcurveto{\pgfqpoint{3.869261in}{6.496667in}}{\pgfqpoint{3.871575in}{6.502253in}}{\pgfqpoint{3.871575in}{6.508077in}}%
\pgfpathcurveto{\pgfqpoint{3.871575in}{6.513901in}}{\pgfqpoint{3.869261in}{6.519487in}}{\pgfqpoint{3.865143in}{6.523605in}}%
\pgfpathcurveto{\pgfqpoint{3.861024in}{6.527723in}}{\pgfqpoint{3.855438in}{6.530037in}}{\pgfqpoint{3.849614in}{6.530037in}}%
\pgfpathcurveto{\pgfqpoint{3.843790in}{6.530037in}}{\pgfqpoint{3.838204in}{6.527723in}}{\pgfqpoint{3.834086in}{6.523605in}}%
\pgfpathcurveto{\pgfqpoint{3.829968in}{6.519487in}}{\pgfqpoint{3.827654in}{6.513901in}}{\pgfqpoint{3.827654in}{6.508077in}}%
\pgfpathcurveto{\pgfqpoint{3.827654in}{6.502253in}}{\pgfqpoint{3.829968in}{6.496667in}}{\pgfqpoint{3.834086in}{6.492549in}}%
\pgfpathcurveto{\pgfqpoint{3.838204in}{6.488431in}}{\pgfqpoint{3.843790in}{6.486117in}}{\pgfqpoint{3.849614in}{6.486117in}}%
\pgfpathlineto{\pgfqpoint{3.849614in}{6.486117in}}%
\pgfpathclose%
\pgfusepath{stroke,fill}%
\end{pgfscope}%
\begin{pgfscope}%
\pgfpathrectangle{\pgfqpoint{1.542338in}{0.880000in}}{\pgfqpoint{5.115323in}{6.160000in}}%
\pgfusepath{clip}%
\pgfsetbuttcap%
\pgfsetroundjoin%
\definecolor{currentfill}{rgb}{0.200000,0.800000,0.200000}%
\pgfsetfillcolor{currentfill}%
\pgfsetlinewidth{1.003750pt}%
\definecolor{currentstroke}{rgb}{0.200000,0.800000,0.200000}%
\pgfsetstrokecolor{currentstroke}%
\pgfsetdash{}{0pt}%
\pgfpathmoveto{\pgfqpoint{3.755285in}{6.408852in}}%
\pgfpathcurveto{\pgfqpoint{3.761109in}{6.408852in}}{\pgfqpoint{3.766695in}{6.411166in}}{\pgfqpoint{3.770813in}{6.415284in}}%
\pgfpathcurveto{\pgfqpoint{3.774931in}{6.419402in}}{\pgfqpoint{3.777245in}{6.424988in}}{\pgfqpoint{3.777245in}{6.430812in}}%
\pgfpathcurveto{\pgfqpoint{3.777245in}{6.436636in}}{\pgfqpoint{3.774931in}{6.442222in}}{\pgfqpoint{3.770813in}{6.446341in}}%
\pgfpathcurveto{\pgfqpoint{3.766695in}{6.450459in}}{\pgfqpoint{3.761109in}{6.452773in}}{\pgfqpoint{3.755285in}{6.452773in}}%
\pgfpathcurveto{\pgfqpoint{3.749461in}{6.452773in}}{\pgfqpoint{3.743875in}{6.450459in}}{\pgfqpoint{3.739756in}{6.446341in}}%
\pgfpathcurveto{\pgfqpoint{3.735638in}{6.442222in}}{\pgfqpoint{3.733324in}{6.436636in}}{\pgfqpoint{3.733324in}{6.430812in}}%
\pgfpathcurveto{\pgfqpoint{3.733324in}{6.424988in}}{\pgfqpoint{3.735638in}{6.419402in}}{\pgfqpoint{3.739756in}{6.415284in}}%
\pgfpathcurveto{\pgfqpoint{3.743875in}{6.411166in}}{\pgfqpoint{3.749461in}{6.408852in}}{\pgfqpoint{3.755285in}{6.408852in}}%
\pgfpathlineto{\pgfqpoint{3.755285in}{6.408852in}}%
\pgfpathclose%
\pgfusepath{stroke,fill}%
\end{pgfscope}%
\begin{pgfscope}%
\pgfpathrectangle{\pgfqpoint{1.542338in}{0.880000in}}{\pgfqpoint{5.115323in}{6.160000in}}%
\pgfusepath{clip}%
\pgfsetbuttcap%
\pgfsetroundjoin%
\definecolor{currentfill}{rgb}{0.200000,0.800000,0.200000}%
\pgfsetfillcolor{currentfill}%
\pgfsetlinewidth{1.003750pt}%
\definecolor{currentstroke}{rgb}{0.200000,0.800000,0.200000}%
\pgfsetstrokecolor{currentstroke}%
\pgfsetdash{}{0pt}%
\pgfpathmoveto{\pgfqpoint{3.687710in}{6.307537in}}%
\pgfpathcurveto{\pgfqpoint{3.693534in}{6.307537in}}{\pgfqpoint{3.699120in}{6.309851in}}{\pgfqpoint{3.703238in}{6.313969in}}%
\pgfpathcurveto{\pgfqpoint{3.707356in}{6.318087in}}{\pgfqpoint{3.709670in}{6.323673in}}{\pgfqpoint{3.709670in}{6.329497in}}%
\pgfpathcurveto{\pgfqpoint{3.709670in}{6.335321in}}{\pgfqpoint{3.707356in}{6.340907in}}{\pgfqpoint{3.703238in}{6.345025in}}%
\pgfpathcurveto{\pgfqpoint{3.699120in}{6.349144in}}{\pgfqpoint{3.693534in}{6.351457in}}{\pgfqpoint{3.687710in}{6.351457in}}%
\pgfpathcurveto{\pgfqpoint{3.681886in}{6.351457in}}{\pgfqpoint{3.676300in}{6.349144in}}{\pgfqpoint{3.672182in}{6.345025in}}%
\pgfpathcurveto{\pgfqpoint{3.668064in}{6.340907in}}{\pgfqpoint{3.665750in}{6.335321in}}{\pgfqpoint{3.665750in}{6.329497in}}%
\pgfpathcurveto{\pgfqpoint{3.665750in}{6.323673in}}{\pgfqpoint{3.668064in}{6.318087in}}{\pgfqpoint{3.672182in}{6.313969in}}%
\pgfpathcurveto{\pgfqpoint{3.676300in}{6.309851in}}{\pgfqpoint{3.681886in}{6.307537in}}{\pgfqpoint{3.687710in}{6.307537in}}%
\pgfpathlineto{\pgfqpoint{3.687710in}{6.307537in}}%
\pgfpathclose%
\pgfusepath{stroke,fill}%
\end{pgfscope}%
\begin{pgfscope}%
\pgfpathrectangle{\pgfqpoint{1.542338in}{0.880000in}}{\pgfqpoint{5.115323in}{6.160000in}}%
\pgfusepath{clip}%
\pgfsetbuttcap%
\pgfsetroundjoin%
\definecolor{currentfill}{rgb}{0.200000,0.800000,0.200000}%
\pgfsetfillcolor{currentfill}%
\pgfsetlinewidth{1.003750pt}%
\definecolor{currentstroke}{rgb}{0.200000,0.800000,0.200000}%
\pgfsetstrokecolor{currentstroke}%
\pgfsetdash{}{0pt}%
\pgfpathmoveto{\pgfqpoint{3.630415in}{6.200885in}}%
\pgfpathcurveto{\pgfqpoint{3.636238in}{6.200885in}}{\pgfqpoint{3.641825in}{6.203199in}}{\pgfqpoint{3.645943in}{6.207317in}}%
\pgfpathcurveto{\pgfqpoint{3.650061in}{6.211435in}}{\pgfqpoint{3.652375in}{6.217021in}}{\pgfqpoint{3.652375in}{6.222845in}}%
\pgfpathcurveto{\pgfqpoint{3.652375in}{6.228669in}}{\pgfqpoint{3.650061in}{6.234255in}}{\pgfqpoint{3.645943in}{6.238373in}}%
\pgfpathcurveto{\pgfqpoint{3.641825in}{6.242492in}}{\pgfqpoint{3.636238in}{6.244805in}}{\pgfqpoint{3.630415in}{6.244805in}}%
\pgfpathcurveto{\pgfqpoint{3.624591in}{6.244805in}}{\pgfqpoint{3.619004in}{6.242492in}}{\pgfqpoint{3.614886in}{6.238373in}}%
\pgfpathcurveto{\pgfqpoint{3.610768in}{6.234255in}}{\pgfqpoint{3.608454in}{6.228669in}}{\pgfqpoint{3.608454in}{6.222845in}}%
\pgfpathcurveto{\pgfqpoint{3.608454in}{6.217021in}}{\pgfqpoint{3.610768in}{6.211435in}}{\pgfqpoint{3.614886in}{6.207317in}}%
\pgfpathcurveto{\pgfqpoint{3.619004in}{6.203199in}}{\pgfqpoint{3.624591in}{6.200885in}}{\pgfqpoint{3.630415in}{6.200885in}}%
\pgfpathlineto{\pgfqpoint{3.630415in}{6.200885in}}%
\pgfpathclose%
\pgfusepath{stroke,fill}%
\end{pgfscope}%
\begin{pgfscope}%
\pgfpathrectangle{\pgfqpoint{1.542338in}{0.880000in}}{\pgfqpoint{5.115323in}{6.160000in}}%
\pgfusepath{clip}%
\pgfsetbuttcap%
\pgfsetroundjoin%
\definecolor{currentfill}{rgb}{0.200000,0.800000,0.200000}%
\pgfsetfillcolor{currentfill}%
\pgfsetlinewidth{1.003750pt}%
\definecolor{currentstroke}{rgb}{0.200000,0.800000,0.200000}%
\pgfsetstrokecolor{currentstroke}%
\pgfsetdash{}{0pt}%
\pgfpathmoveto{\pgfqpoint{3.579965in}{6.090519in}}%
\pgfpathcurveto{\pgfqpoint{3.585789in}{6.090519in}}{\pgfqpoint{3.591375in}{6.092833in}}{\pgfqpoint{3.595493in}{6.096951in}}%
\pgfpathcurveto{\pgfqpoint{3.599611in}{6.101069in}}{\pgfqpoint{3.601925in}{6.106655in}}{\pgfqpoint{3.601925in}{6.112479in}}%
\pgfpathcurveto{\pgfqpoint{3.601925in}{6.118303in}}{\pgfqpoint{3.599611in}{6.123889in}}{\pgfqpoint{3.595493in}{6.128007in}}%
\pgfpathcurveto{\pgfqpoint{3.591375in}{6.132126in}}{\pgfqpoint{3.585789in}{6.134440in}}{\pgfqpoint{3.579965in}{6.134440in}}%
\pgfpathcurveto{\pgfqpoint{3.574141in}{6.134440in}}{\pgfqpoint{3.568555in}{6.132126in}}{\pgfqpoint{3.564437in}{6.128007in}}%
\pgfpathcurveto{\pgfqpoint{3.560318in}{6.123889in}}{\pgfqpoint{3.558005in}{6.118303in}}{\pgfqpoint{3.558005in}{6.112479in}}%
\pgfpathcurveto{\pgfqpoint{3.558005in}{6.106655in}}{\pgfqpoint{3.560318in}{6.101069in}}{\pgfqpoint{3.564437in}{6.096951in}}%
\pgfpathcurveto{\pgfqpoint{3.568555in}{6.092833in}}{\pgfqpoint{3.574141in}{6.090519in}}{\pgfqpoint{3.579965in}{6.090519in}}%
\pgfpathlineto{\pgfqpoint{3.579965in}{6.090519in}}%
\pgfpathclose%
\pgfusepath{stroke,fill}%
\end{pgfscope}%
\begin{pgfscope}%
\pgfpathrectangle{\pgfqpoint{1.542338in}{0.880000in}}{\pgfqpoint{5.115323in}{6.160000in}}%
\pgfusepath{clip}%
\pgfsetbuttcap%
\pgfsetroundjoin%
\definecolor{currentfill}{rgb}{0.200000,0.800000,0.200000}%
\pgfsetfillcolor{currentfill}%
\pgfsetlinewidth{1.003750pt}%
\definecolor{currentstroke}{rgb}{0.200000,0.800000,0.200000}%
\pgfsetstrokecolor{currentstroke}%
\pgfsetdash{}{0pt}%
\pgfpathmoveto{\pgfqpoint{3.549237in}{5.972898in}}%
\pgfpathcurveto{\pgfqpoint{3.555061in}{5.972898in}}{\pgfqpoint{3.560647in}{5.975212in}}{\pgfqpoint{3.564765in}{5.979330in}}%
\pgfpathcurveto{\pgfqpoint{3.568884in}{5.983448in}}{\pgfqpoint{3.571197in}{5.989035in}}{\pgfqpoint{3.571197in}{5.994859in}}%
\pgfpathcurveto{\pgfqpoint{3.571197in}{6.000682in}}{\pgfqpoint{3.568884in}{6.006269in}}{\pgfqpoint{3.564765in}{6.010387in}}%
\pgfpathcurveto{\pgfqpoint{3.560647in}{6.014505in}}{\pgfqpoint{3.555061in}{6.016819in}}{\pgfqpoint{3.549237in}{6.016819in}}%
\pgfpathcurveto{\pgfqpoint{3.543413in}{6.016819in}}{\pgfqpoint{3.537827in}{6.014505in}}{\pgfqpoint{3.533709in}{6.010387in}}%
\pgfpathcurveto{\pgfqpoint{3.529591in}{6.006269in}}{\pgfqpoint{3.527277in}{6.000682in}}{\pgfqpoint{3.527277in}{5.994859in}}%
\pgfpathcurveto{\pgfqpoint{3.527277in}{5.989035in}}{\pgfqpoint{3.529591in}{5.983448in}}{\pgfqpoint{3.533709in}{5.979330in}}%
\pgfpathcurveto{\pgfqpoint{3.537827in}{5.975212in}}{\pgfqpoint{3.543413in}{5.972898in}}{\pgfqpoint{3.549237in}{5.972898in}}%
\pgfpathlineto{\pgfqpoint{3.549237in}{5.972898in}}%
\pgfpathclose%
\pgfusepath{stroke,fill}%
\end{pgfscope}%
\begin{pgfscope}%
\pgfpathrectangle{\pgfqpoint{1.542338in}{0.880000in}}{\pgfqpoint{5.115323in}{6.160000in}}%
\pgfusepath{clip}%
\pgfsetbuttcap%
\pgfsetroundjoin%
\definecolor{currentfill}{rgb}{0.200000,0.800000,0.200000}%
\pgfsetfillcolor{currentfill}%
\pgfsetlinewidth{1.003750pt}%
\definecolor{currentstroke}{rgb}{0.200000,0.800000,0.200000}%
\pgfsetstrokecolor{currentstroke}%
\pgfsetdash{}{0pt}%
\pgfpathmoveto{\pgfqpoint{3.539893in}{5.851937in}}%
\pgfpathcurveto{\pgfqpoint{3.545717in}{5.851937in}}{\pgfqpoint{3.551303in}{5.854251in}}{\pgfqpoint{3.555422in}{5.858369in}}%
\pgfpathcurveto{\pgfqpoint{3.559540in}{5.862487in}}{\pgfqpoint{3.561854in}{5.868074in}}{\pgfqpoint{3.561854in}{5.873898in}}%
\pgfpathcurveto{\pgfqpoint{3.561854in}{5.879721in}}{\pgfqpoint{3.559540in}{5.885308in}}{\pgfqpoint{3.555422in}{5.889426in}}%
\pgfpathcurveto{\pgfqpoint{3.551303in}{5.893544in}}{\pgfqpoint{3.545717in}{5.895858in}}{\pgfqpoint{3.539893in}{5.895858in}}%
\pgfpathcurveto{\pgfqpoint{3.534069in}{5.895858in}}{\pgfqpoint{3.528483in}{5.893544in}}{\pgfqpoint{3.524365in}{5.889426in}}%
\pgfpathcurveto{\pgfqpoint{3.520247in}{5.885308in}}{\pgfqpoint{3.517933in}{5.879721in}}{\pgfqpoint{3.517933in}{5.873898in}}%
\pgfpathcurveto{\pgfqpoint{3.517933in}{5.868074in}}{\pgfqpoint{3.520247in}{5.862487in}}{\pgfqpoint{3.524365in}{5.858369in}}%
\pgfpathcurveto{\pgfqpoint{3.528483in}{5.854251in}}{\pgfqpoint{3.534069in}{5.851937in}}{\pgfqpoint{3.539893in}{5.851937in}}%
\pgfpathlineto{\pgfqpoint{3.539893in}{5.851937in}}%
\pgfpathclose%
\pgfusepath{stroke,fill}%
\end{pgfscope}%
\begin{pgfscope}%
\pgfpathrectangle{\pgfqpoint{1.542338in}{0.880000in}}{\pgfqpoint{5.115323in}{6.160000in}}%
\pgfusepath{clip}%
\pgfsetbuttcap%
\pgfsetroundjoin%
\definecolor{currentfill}{rgb}{0.200000,0.800000,0.200000}%
\pgfsetfillcolor{currentfill}%
\pgfsetlinewidth{1.003750pt}%
\definecolor{currentstroke}{rgb}{0.200000,0.800000,0.200000}%
\pgfsetstrokecolor{currentstroke}%
\pgfsetdash{}{0pt}%
\pgfpathmoveto{\pgfqpoint{3.535711in}{5.730904in}}%
\pgfpathcurveto{\pgfqpoint{3.541535in}{5.730904in}}{\pgfqpoint{3.547121in}{5.733218in}}{\pgfqpoint{3.551239in}{5.737336in}}%
\pgfpathcurveto{\pgfqpoint{3.555357in}{5.741454in}}{\pgfqpoint{3.557671in}{5.747040in}}{\pgfqpoint{3.557671in}{5.752864in}}%
\pgfpathcurveto{\pgfqpoint{3.557671in}{5.758688in}}{\pgfqpoint{3.555357in}{5.764274in}}{\pgfqpoint{3.551239in}{5.768393in}}%
\pgfpathcurveto{\pgfqpoint{3.547121in}{5.772511in}}{\pgfqpoint{3.541535in}{5.774825in}}{\pgfqpoint{3.535711in}{5.774825in}}%
\pgfpathcurveto{\pgfqpoint{3.529887in}{5.774825in}}{\pgfqpoint{3.524301in}{5.772511in}}{\pgfqpoint{3.520183in}{5.768393in}}%
\pgfpathcurveto{\pgfqpoint{3.516065in}{5.764274in}}{\pgfqpoint{3.513751in}{5.758688in}}{\pgfqpoint{3.513751in}{5.752864in}}%
\pgfpathcurveto{\pgfqpoint{3.513751in}{5.747040in}}{\pgfqpoint{3.516065in}{5.741454in}}{\pgfqpoint{3.520183in}{5.737336in}}%
\pgfpathcurveto{\pgfqpoint{3.524301in}{5.733218in}}{\pgfqpoint{3.529887in}{5.730904in}}{\pgfqpoint{3.535711in}{5.730904in}}%
\pgfpathlineto{\pgfqpoint{3.535711in}{5.730904in}}%
\pgfpathclose%
\pgfusepath{stroke,fill}%
\end{pgfscope}%
\begin{pgfscope}%
\pgfpathrectangle{\pgfqpoint{1.542338in}{0.880000in}}{\pgfqpoint{5.115323in}{6.160000in}}%
\pgfusepath{clip}%
\pgfsetbuttcap%
\pgfsetroundjoin%
\definecolor{currentfill}{rgb}{0.200000,0.800000,0.200000}%
\pgfsetfillcolor{currentfill}%
\pgfsetlinewidth{1.003750pt}%
\definecolor{currentstroke}{rgb}{0.200000,0.800000,0.200000}%
\pgfsetstrokecolor{currentstroke}%
\pgfsetdash{}{0pt}%
\pgfpathmoveto{\pgfqpoint{3.551342in}{5.610621in}}%
\pgfpathcurveto{\pgfqpoint{3.557166in}{5.610621in}}{\pgfqpoint{3.562752in}{5.612935in}}{\pgfqpoint{3.566870in}{5.617053in}}%
\pgfpathcurveto{\pgfqpoint{3.570988in}{5.621172in}}{\pgfqpoint{3.573302in}{5.626758in}}{\pgfqpoint{3.573302in}{5.632582in}}%
\pgfpathcurveto{\pgfqpoint{3.573302in}{5.638406in}}{\pgfqpoint{3.570988in}{5.643992in}}{\pgfqpoint{3.566870in}{5.648110in}}%
\pgfpathcurveto{\pgfqpoint{3.562752in}{5.652228in}}{\pgfqpoint{3.557166in}{5.654542in}}{\pgfqpoint{3.551342in}{5.654542in}}%
\pgfpathcurveto{\pgfqpoint{3.545518in}{5.654542in}}{\pgfqpoint{3.539932in}{5.652228in}}{\pgfqpoint{3.535813in}{5.648110in}}%
\pgfpathcurveto{\pgfqpoint{3.531695in}{5.643992in}}{\pgfqpoint{3.529381in}{5.638406in}}{\pgfqpoint{3.529381in}{5.632582in}}%
\pgfpathcurveto{\pgfqpoint{3.529381in}{5.626758in}}{\pgfqpoint{3.531695in}{5.621172in}}{\pgfqpoint{3.535813in}{5.617053in}}%
\pgfpathcurveto{\pgfqpoint{3.539932in}{5.612935in}}{\pgfqpoint{3.545518in}{5.610621in}}{\pgfqpoint{3.551342in}{5.610621in}}%
\pgfpathlineto{\pgfqpoint{3.551342in}{5.610621in}}%
\pgfpathclose%
\pgfusepath{stroke,fill}%
\end{pgfscope}%
\begin{pgfscope}%
\pgfpathrectangle{\pgfqpoint{1.542338in}{0.880000in}}{\pgfqpoint{5.115323in}{6.160000in}}%
\pgfusepath{clip}%
\pgfsetbuttcap%
\pgfsetroundjoin%
\definecolor{currentfill}{rgb}{0.200000,0.800000,0.200000}%
\pgfsetfillcolor{currentfill}%
\pgfsetlinewidth{1.003750pt}%
\definecolor{currentstroke}{rgb}{0.200000,0.800000,0.200000}%
\pgfsetstrokecolor{currentstroke}%
\pgfsetdash{}{0pt}%
\pgfpathmoveto{\pgfqpoint{3.582355in}{5.493384in}}%
\pgfpathcurveto{\pgfqpoint{3.588179in}{5.493384in}}{\pgfqpoint{3.593765in}{5.495698in}}{\pgfqpoint{3.597883in}{5.499816in}}%
\pgfpathcurveto{\pgfqpoint{3.602002in}{5.503935in}}{\pgfqpoint{3.604315in}{5.509521in}}{\pgfqpoint{3.604315in}{5.515345in}}%
\pgfpathcurveto{\pgfqpoint{3.604315in}{5.521169in}}{\pgfqpoint{3.602002in}{5.526755in}}{\pgfqpoint{3.597883in}{5.530873in}}%
\pgfpathcurveto{\pgfqpoint{3.593765in}{5.534991in}}{\pgfqpoint{3.588179in}{5.537305in}}{\pgfqpoint{3.582355in}{5.537305in}}%
\pgfpathcurveto{\pgfqpoint{3.576531in}{5.537305in}}{\pgfqpoint{3.570945in}{5.534991in}}{\pgfqpoint{3.566827in}{5.530873in}}%
\pgfpathcurveto{\pgfqpoint{3.562709in}{5.526755in}}{\pgfqpoint{3.560395in}{5.521169in}}{\pgfqpoint{3.560395in}{5.515345in}}%
\pgfpathcurveto{\pgfqpoint{3.560395in}{5.509521in}}{\pgfqpoint{3.562709in}{5.503935in}}{\pgfqpoint{3.566827in}{5.499816in}}%
\pgfpathcurveto{\pgfqpoint{3.570945in}{5.495698in}}{\pgfqpoint{3.576531in}{5.493384in}}{\pgfqpoint{3.582355in}{5.493384in}}%
\pgfpathlineto{\pgfqpoint{3.582355in}{5.493384in}}%
\pgfpathclose%
\pgfusepath{stroke,fill}%
\end{pgfscope}%
\begin{pgfscope}%
\pgfpathrectangle{\pgfqpoint{1.542338in}{0.880000in}}{\pgfqpoint{5.115323in}{6.160000in}}%
\pgfusepath{clip}%
\pgfsetbuttcap%
\pgfsetroundjoin%
\definecolor{currentfill}{rgb}{0.200000,0.800000,0.200000}%
\pgfsetfillcolor{currentfill}%
\pgfsetlinewidth{1.003750pt}%
\definecolor{currentstroke}{rgb}{0.200000,0.800000,0.200000}%
\pgfsetstrokecolor{currentstroke}%
\pgfsetdash{}{0pt}%
\pgfpathmoveto{\pgfqpoint{3.621882in}{5.378116in}}%
\pgfpathcurveto{\pgfqpoint{3.627706in}{5.378116in}}{\pgfqpoint{3.633293in}{5.380430in}}{\pgfqpoint{3.637411in}{5.384548in}}%
\pgfpathcurveto{\pgfqpoint{3.641529in}{5.388666in}}{\pgfqpoint{3.643843in}{5.394252in}}{\pgfqpoint{3.643843in}{5.400076in}}%
\pgfpathcurveto{\pgfqpoint{3.643843in}{5.405900in}}{\pgfqpoint{3.641529in}{5.411486in}}{\pgfqpoint{3.637411in}{5.415605in}}%
\pgfpathcurveto{\pgfqpoint{3.633293in}{5.419723in}}{\pgfqpoint{3.627706in}{5.422037in}}{\pgfqpoint{3.621882in}{5.422037in}}%
\pgfpathcurveto{\pgfqpoint{3.616058in}{5.422037in}}{\pgfqpoint{3.610472in}{5.419723in}}{\pgfqpoint{3.606354in}{5.415605in}}%
\pgfpathcurveto{\pgfqpoint{3.602236in}{5.411486in}}{\pgfqpoint{3.599922in}{5.405900in}}{\pgfqpoint{3.599922in}{5.400076in}}%
\pgfpathcurveto{\pgfqpoint{3.599922in}{5.394252in}}{\pgfqpoint{3.602236in}{5.388666in}}{\pgfqpoint{3.606354in}{5.384548in}}%
\pgfpathcurveto{\pgfqpoint{3.610472in}{5.380430in}}{\pgfqpoint{3.616058in}{5.378116in}}{\pgfqpoint{3.621882in}{5.378116in}}%
\pgfpathlineto{\pgfqpoint{3.621882in}{5.378116in}}%
\pgfpathclose%
\pgfusepath{stroke,fill}%
\end{pgfscope}%
\begin{pgfscope}%
\pgfpathrectangle{\pgfqpoint{1.542338in}{0.880000in}}{\pgfqpoint{5.115323in}{6.160000in}}%
\pgfusepath{clip}%
\pgfsetbuttcap%
\pgfsetroundjoin%
\definecolor{currentfill}{rgb}{0.200000,0.800000,0.200000}%
\pgfsetfillcolor{currentfill}%
\pgfsetlinewidth{1.003750pt}%
\definecolor{currentstroke}{rgb}{0.200000,0.800000,0.200000}%
\pgfsetstrokecolor{currentstroke}%
\pgfsetdash{}{0pt}%
\pgfpathmoveto{\pgfqpoint{3.687332in}{5.275327in}}%
\pgfpathcurveto{\pgfqpoint{3.693156in}{5.275327in}}{\pgfqpoint{3.698742in}{5.277641in}}{\pgfqpoint{3.702860in}{5.281759in}}%
\pgfpathcurveto{\pgfqpoint{3.706978in}{5.285877in}}{\pgfqpoint{3.709292in}{5.291463in}}{\pgfqpoint{3.709292in}{5.297287in}}%
\pgfpathcurveto{\pgfqpoint{3.709292in}{5.303111in}}{\pgfqpoint{3.706978in}{5.308697in}}{\pgfqpoint{3.702860in}{5.312815in}}%
\pgfpathcurveto{\pgfqpoint{3.698742in}{5.316933in}}{\pgfqpoint{3.693156in}{5.319247in}}{\pgfqpoint{3.687332in}{5.319247in}}%
\pgfpathcurveto{\pgfqpoint{3.681508in}{5.319247in}}{\pgfqpoint{3.675922in}{5.316933in}}{\pgfqpoint{3.671804in}{5.312815in}}%
\pgfpathcurveto{\pgfqpoint{3.667686in}{5.308697in}}{\pgfqpoint{3.665372in}{5.303111in}}{\pgfqpoint{3.665372in}{5.297287in}}%
\pgfpathcurveto{\pgfqpoint{3.665372in}{5.291463in}}{\pgfqpoint{3.667686in}{5.285877in}}{\pgfqpoint{3.671804in}{5.281759in}}%
\pgfpathcurveto{\pgfqpoint{3.675922in}{5.277641in}}{\pgfqpoint{3.681508in}{5.275327in}}{\pgfqpoint{3.687332in}{5.275327in}}%
\pgfpathlineto{\pgfqpoint{3.687332in}{5.275327in}}%
\pgfpathclose%
\pgfusepath{stroke,fill}%
\end{pgfscope}%
\begin{pgfscope}%
\pgfpathrectangle{\pgfqpoint{1.542338in}{0.880000in}}{\pgfqpoint{5.115323in}{6.160000in}}%
\pgfusepath{clip}%
\pgfsetbuttcap%
\pgfsetroundjoin%
\definecolor{currentfill}{rgb}{0.200000,0.800000,0.200000}%
\pgfsetfillcolor{currentfill}%
\pgfsetlinewidth{1.003750pt}%
\definecolor{currentstroke}{rgb}{0.200000,0.800000,0.200000}%
\pgfsetstrokecolor{currentstroke}%
\pgfsetdash{}{0pt}%
\pgfpathmoveto{\pgfqpoint{3.760192in}{5.178436in}}%
\pgfpathcurveto{\pgfqpoint{3.766016in}{5.178436in}}{\pgfqpoint{3.771602in}{5.180750in}}{\pgfqpoint{3.775720in}{5.184868in}}%
\pgfpathcurveto{\pgfqpoint{3.779839in}{5.188986in}}{\pgfqpoint{3.782152in}{5.194572in}}{\pgfqpoint{3.782152in}{5.200396in}}%
\pgfpathcurveto{\pgfqpoint{3.782152in}{5.206220in}}{\pgfqpoint{3.779839in}{5.211806in}}{\pgfqpoint{3.775720in}{5.215924in}}%
\pgfpathcurveto{\pgfqpoint{3.771602in}{5.220042in}}{\pgfqpoint{3.766016in}{5.222356in}}{\pgfqpoint{3.760192in}{5.222356in}}%
\pgfpathcurveto{\pgfqpoint{3.754368in}{5.222356in}}{\pgfqpoint{3.748782in}{5.220042in}}{\pgfqpoint{3.744664in}{5.215924in}}%
\pgfpathcurveto{\pgfqpoint{3.740546in}{5.211806in}}{\pgfqpoint{3.738232in}{5.206220in}}{\pgfqpoint{3.738232in}{5.200396in}}%
\pgfpathcurveto{\pgfqpoint{3.738232in}{5.194572in}}{\pgfqpoint{3.740546in}{5.188986in}}{\pgfqpoint{3.744664in}{5.184868in}}%
\pgfpathcurveto{\pgfqpoint{3.748782in}{5.180750in}}{\pgfqpoint{3.754368in}{5.178436in}}{\pgfqpoint{3.760192in}{5.178436in}}%
\pgfpathlineto{\pgfqpoint{3.760192in}{5.178436in}}%
\pgfpathclose%
\pgfusepath{stroke,fill}%
\end{pgfscope}%
\begin{pgfscope}%
\pgfpathrectangle{\pgfqpoint{1.542338in}{0.880000in}}{\pgfqpoint{5.115323in}{6.160000in}}%
\pgfusepath{clip}%
\pgfsetbuttcap%
\pgfsetroundjoin%
\definecolor{currentfill}{rgb}{0.200000,0.800000,0.200000}%
\pgfsetfillcolor{currentfill}%
\pgfsetlinewidth{1.003750pt}%
\definecolor{currentstroke}{rgb}{0.200000,0.800000,0.200000}%
\pgfsetstrokecolor{currentstroke}%
\pgfsetdash{}{0pt}%
\pgfpathmoveto{\pgfqpoint{3.845995in}{5.093007in}}%
\pgfpathcurveto{\pgfqpoint{3.851819in}{5.093007in}}{\pgfqpoint{3.857405in}{5.095321in}}{\pgfqpoint{3.861523in}{5.099439in}}%
\pgfpathcurveto{\pgfqpoint{3.865641in}{5.103558in}}{\pgfqpoint{3.867955in}{5.109144in}}{\pgfqpoint{3.867955in}{5.114968in}}%
\pgfpathcurveto{\pgfqpoint{3.867955in}{5.120792in}}{\pgfqpoint{3.865641in}{5.126378in}}{\pgfqpoint{3.861523in}{5.130496in}}%
\pgfpathcurveto{\pgfqpoint{3.857405in}{5.134614in}}{\pgfqpoint{3.851819in}{5.136928in}}{\pgfqpoint{3.845995in}{5.136928in}}%
\pgfpathcurveto{\pgfqpoint{3.840171in}{5.136928in}}{\pgfqpoint{3.834585in}{5.134614in}}{\pgfqpoint{3.830466in}{5.130496in}}%
\pgfpathcurveto{\pgfqpoint{3.826348in}{5.126378in}}{\pgfqpoint{3.824034in}{5.120792in}}{\pgfqpoint{3.824034in}{5.114968in}}%
\pgfpathcurveto{\pgfqpoint{3.824034in}{5.109144in}}{\pgfqpoint{3.826348in}{5.103558in}}{\pgfqpoint{3.830466in}{5.099439in}}%
\pgfpathcurveto{\pgfqpoint{3.834585in}{5.095321in}}{\pgfqpoint{3.840171in}{5.093007in}}{\pgfqpoint{3.845995in}{5.093007in}}%
\pgfpathlineto{\pgfqpoint{3.845995in}{5.093007in}}%
\pgfpathclose%
\pgfusepath{stroke,fill}%
\end{pgfscope}%
\begin{pgfscope}%
\pgfpathrectangle{\pgfqpoint{1.542338in}{0.880000in}}{\pgfqpoint{5.115323in}{6.160000in}}%
\pgfusepath{clip}%
\pgfsetbuttcap%
\pgfsetroundjoin%
\definecolor{currentfill}{rgb}{0.200000,0.800000,0.200000}%
\pgfsetfillcolor{currentfill}%
\pgfsetlinewidth{1.003750pt}%
\definecolor{currentstroke}{rgb}{0.200000,0.800000,0.200000}%
\pgfsetstrokecolor{currentstroke}%
\pgfsetdash{}{0pt}%
\pgfpathmoveto{\pgfqpoint{3.940863in}{5.018093in}}%
\pgfpathcurveto{\pgfqpoint{3.946687in}{5.018093in}}{\pgfqpoint{3.952273in}{5.020406in}}{\pgfqpoint{3.956391in}{5.024525in}}%
\pgfpathcurveto{\pgfqpoint{3.960510in}{5.028643in}}{\pgfqpoint{3.962823in}{5.034229in}}{\pgfqpoint{3.962823in}{5.040053in}}%
\pgfpathcurveto{\pgfqpoint{3.962823in}{5.045877in}}{\pgfqpoint{3.960510in}{5.051463in}}{\pgfqpoint{3.956391in}{5.055581in}}%
\pgfpathcurveto{\pgfqpoint{3.952273in}{5.059699in}}{\pgfqpoint{3.946687in}{5.062013in}}{\pgfqpoint{3.940863in}{5.062013in}}%
\pgfpathcurveto{\pgfqpoint{3.935039in}{5.062013in}}{\pgfqpoint{3.929453in}{5.059699in}}{\pgfqpoint{3.925335in}{5.055581in}}%
\pgfpathcurveto{\pgfqpoint{3.921217in}{5.051463in}}{\pgfqpoint{3.918903in}{5.045877in}}{\pgfqpoint{3.918903in}{5.040053in}}%
\pgfpathcurveto{\pgfqpoint{3.918903in}{5.034229in}}{\pgfqpoint{3.921217in}{5.028643in}}{\pgfqpoint{3.925335in}{5.024525in}}%
\pgfpathcurveto{\pgfqpoint{3.929453in}{5.020406in}}{\pgfqpoint{3.935039in}{5.018093in}}{\pgfqpoint{3.940863in}{5.018093in}}%
\pgfpathlineto{\pgfqpoint{3.940863in}{5.018093in}}%
\pgfpathclose%
\pgfusepath{stroke,fill}%
\end{pgfscope}%
\begin{pgfscope}%
\pgfpathrectangle{\pgfqpoint{1.542338in}{0.880000in}}{\pgfqpoint{5.115323in}{6.160000in}}%
\pgfusepath{clip}%
\pgfsetbuttcap%
\pgfsetroundjoin%
\definecolor{currentfill}{rgb}{0.200000,0.800000,0.200000}%
\pgfsetfillcolor{currentfill}%
\pgfsetlinewidth{1.003750pt}%
\definecolor{currentstroke}{rgb}{0.200000,0.800000,0.200000}%
\pgfsetstrokecolor{currentstroke}%
\pgfsetdash{}{0pt}%
\pgfpathmoveto{\pgfqpoint{4.041941in}{4.951118in}}%
\pgfpathcurveto{\pgfqpoint{4.047765in}{4.951118in}}{\pgfqpoint{4.053352in}{4.953431in}}{\pgfqpoint{4.057470in}{4.957550in}}%
\pgfpathcurveto{\pgfqpoint{4.061588in}{4.961668in}}{\pgfqpoint{4.063902in}{4.967254in}}{\pgfqpoint{4.063902in}{4.973078in}}%
\pgfpathcurveto{\pgfqpoint{4.063902in}{4.978902in}}{\pgfqpoint{4.061588in}{4.984488in}}{\pgfqpoint{4.057470in}{4.988606in}}%
\pgfpathcurveto{\pgfqpoint{4.053352in}{4.992724in}}{\pgfqpoint{4.047765in}{4.995038in}}{\pgfqpoint{4.041941in}{4.995038in}}%
\pgfpathcurveto{\pgfqpoint{4.036117in}{4.995038in}}{\pgfqpoint{4.030531in}{4.992724in}}{\pgfqpoint{4.026413in}{4.988606in}}%
\pgfpathcurveto{\pgfqpoint{4.022295in}{4.984488in}}{\pgfqpoint{4.019981in}{4.978902in}}{\pgfqpoint{4.019981in}{4.973078in}}%
\pgfpathcurveto{\pgfqpoint{4.019981in}{4.967254in}}{\pgfqpoint{4.022295in}{4.961668in}}{\pgfqpoint{4.026413in}{4.957550in}}%
\pgfpathcurveto{\pgfqpoint{4.030531in}{4.953431in}}{\pgfqpoint{4.036117in}{4.951118in}}{\pgfqpoint{4.041941in}{4.951118in}}%
\pgfpathlineto{\pgfqpoint{4.041941in}{4.951118in}}%
\pgfpathclose%
\pgfusepath{stroke,fill}%
\end{pgfscope}%
\begin{pgfscope}%
\pgfpathrectangle{\pgfqpoint{1.542338in}{0.880000in}}{\pgfqpoint{5.115323in}{6.160000in}}%
\pgfusepath{clip}%
\pgfsetbuttcap%
\pgfsetroundjoin%
\definecolor{currentfill}{rgb}{0.200000,0.800000,0.200000}%
\pgfsetfillcolor{currentfill}%
\pgfsetlinewidth{1.003750pt}%
\definecolor{currentstroke}{rgb}{0.200000,0.800000,0.200000}%
\pgfsetstrokecolor{currentstroke}%
\pgfsetdash{}{0pt}%
\pgfpathmoveto{\pgfqpoint{4.154209in}{4.905196in}}%
\pgfpathcurveto{\pgfqpoint{4.160033in}{4.905196in}}{\pgfqpoint{4.165619in}{4.907510in}}{\pgfqpoint{4.169737in}{4.911628in}}%
\pgfpathcurveto{\pgfqpoint{4.173855in}{4.915746in}}{\pgfqpoint{4.176169in}{4.921332in}}{\pgfqpoint{4.176169in}{4.927156in}}%
\pgfpathcurveto{\pgfqpoint{4.176169in}{4.932980in}}{\pgfqpoint{4.173855in}{4.938566in}}{\pgfqpoint{4.169737in}{4.942685in}}%
\pgfpathcurveto{\pgfqpoint{4.165619in}{4.946803in}}{\pgfqpoint{4.160033in}{4.949117in}}{\pgfqpoint{4.154209in}{4.949117in}}%
\pgfpathcurveto{\pgfqpoint{4.148385in}{4.949117in}}{\pgfqpoint{4.142799in}{4.946803in}}{\pgfqpoint{4.138681in}{4.942685in}}%
\pgfpathcurveto{\pgfqpoint{4.134562in}{4.938566in}}{\pgfqpoint{4.132249in}{4.932980in}}{\pgfqpoint{4.132249in}{4.927156in}}%
\pgfpathcurveto{\pgfqpoint{4.132249in}{4.921332in}}{\pgfqpoint{4.134562in}{4.915746in}}{\pgfqpoint{4.138681in}{4.911628in}}%
\pgfpathcurveto{\pgfqpoint{4.142799in}{4.907510in}}{\pgfqpoint{4.148385in}{4.905196in}}{\pgfqpoint{4.154209in}{4.905196in}}%
\pgfpathlineto{\pgfqpoint{4.154209in}{4.905196in}}%
\pgfpathclose%
\pgfusepath{stroke,fill}%
\end{pgfscope}%
\begin{pgfscope}%
\pgfpathrectangle{\pgfqpoint{1.542338in}{0.880000in}}{\pgfqpoint{5.115323in}{6.160000in}}%
\pgfusepath{clip}%
\pgfsetbuttcap%
\pgfsetroundjoin%
\definecolor{currentfill}{rgb}{0.200000,0.800000,0.200000}%
\pgfsetfillcolor{currentfill}%
\pgfsetlinewidth{1.003750pt}%
\definecolor{currentstroke}{rgb}{0.200000,0.800000,0.200000}%
\pgfsetstrokecolor{currentstroke}%
\pgfsetdash{}{0pt}%
\pgfpathmoveto{\pgfqpoint{4.269980in}{4.869658in}}%
\pgfpathcurveto{\pgfqpoint{4.275804in}{4.869658in}}{\pgfqpoint{4.281390in}{4.871972in}}{\pgfqpoint{4.285508in}{4.876090in}}%
\pgfpathcurveto{\pgfqpoint{4.289626in}{4.880208in}}{\pgfqpoint{4.291940in}{4.885794in}}{\pgfqpoint{4.291940in}{4.891618in}}%
\pgfpathcurveto{\pgfqpoint{4.291940in}{4.897442in}}{\pgfqpoint{4.289626in}{4.903028in}}{\pgfqpoint{4.285508in}{4.907146in}}%
\pgfpathcurveto{\pgfqpoint{4.281390in}{4.911264in}}{\pgfqpoint{4.275804in}{4.913578in}}{\pgfqpoint{4.269980in}{4.913578in}}%
\pgfpathcurveto{\pgfqpoint{4.264156in}{4.913578in}}{\pgfqpoint{4.258570in}{4.911264in}}{\pgfqpoint{4.254452in}{4.907146in}}%
\pgfpathcurveto{\pgfqpoint{4.250334in}{4.903028in}}{\pgfqpoint{4.248020in}{4.897442in}}{\pgfqpoint{4.248020in}{4.891618in}}%
\pgfpathcurveto{\pgfqpoint{4.248020in}{4.885794in}}{\pgfqpoint{4.250334in}{4.880208in}}{\pgfqpoint{4.254452in}{4.876090in}}%
\pgfpathcurveto{\pgfqpoint{4.258570in}{4.871972in}}{\pgfqpoint{4.264156in}{4.869658in}}{\pgfqpoint{4.269980in}{4.869658in}}%
\pgfpathlineto{\pgfqpoint{4.269980in}{4.869658in}}%
\pgfpathclose%
\pgfusepath{stroke,fill}%
\end{pgfscope}%
\begin{pgfscope}%
\pgfpathrectangle{\pgfqpoint{1.542338in}{0.880000in}}{\pgfqpoint{5.115323in}{6.160000in}}%
\pgfusepath{clip}%
\pgfsetbuttcap%
\pgfsetroundjoin%
\definecolor{currentfill}{rgb}{0.200000,0.800000,0.200000}%
\pgfsetfillcolor{currentfill}%
\pgfsetlinewidth{1.003750pt}%
\definecolor{currentstroke}{rgb}{0.200000,0.800000,0.200000}%
\pgfsetstrokecolor{currentstroke}%
\pgfsetdash{}{0pt}%
\pgfpathmoveto{\pgfqpoint{4.389448in}{4.848776in}}%
\pgfpathcurveto{\pgfqpoint{4.395272in}{4.848776in}}{\pgfqpoint{4.400858in}{4.851090in}}{\pgfqpoint{4.404977in}{4.855208in}}%
\pgfpathcurveto{\pgfqpoint{4.409095in}{4.859326in}}{\pgfqpoint{4.411409in}{4.864913in}}{\pgfqpoint{4.411409in}{4.870737in}}%
\pgfpathcurveto{\pgfqpoint{4.411409in}{4.876560in}}{\pgfqpoint{4.409095in}{4.882147in}}{\pgfqpoint{4.404977in}{4.886265in}}%
\pgfpathcurveto{\pgfqpoint{4.400858in}{4.890383in}}{\pgfqpoint{4.395272in}{4.892697in}}{\pgfqpoint{4.389448in}{4.892697in}}%
\pgfpathcurveto{\pgfqpoint{4.383624in}{4.892697in}}{\pgfqpoint{4.378038in}{4.890383in}}{\pgfqpoint{4.373920in}{4.886265in}}%
\pgfpathcurveto{\pgfqpoint{4.369802in}{4.882147in}}{\pgfqpoint{4.367488in}{4.876560in}}{\pgfqpoint{4.367488in}{4.870737in}}%
\pgfpathcurveto{\pgfqpoint{4.367488in}{4.864913in}}{\pgfqpoint{4.369802in}{4.859326in}}{\pgfqpoint{4.373920in}{4.855208in}}%
\pgfpathcurveto{\pgfqpoint{4.378038in}{4.851090in}}{\pgfqpoint{4.383624in}{4.848776in}}{\pgfqpoint{4.389448in}{4.848776in}}%
\pgfpathlineto{\pgfqpoint{4.389448in}{4.848776in}}%
\pgfpathclose%
\pgfusepath{stroke,fill}%
\end{pgfscope}%
\begin{pgfscope}%
\pgfpathrectangle{\pgfqpoint{1.542338in}{0.880000in}}{\pgfqpoint{5.115323in}{6.160000in}}%
\pgfusepath{clip}%
\pgfsetbuttcap%
\pgfsetroundjoin%
\definecolor{currentfill}{rgb}{0.200000,0.800000,0.200000}%
\pgfsetfillcolor{currentfill}%
\pgfsetlinewidth{1.003750pt}%
\definecolor{currentstroke}{rgb}{0.200000,0.800000,0.200000}%
\pgfsetstrokecolor{currentstroke}%
\pgfsetdash{}{0pt}%
\pgfpathmoveto{\pgfqpoint{4.510738in}{4.845426in}}%
\pgfpathcurveto{\pgfqpoint{4.516561in}{4.845426in}}{\pgfqpoint{4.522148in}{4.847740in}}{\pgfqpoint{4.526266in}{4.851858in}}%
\pgfpathcurveto{\pgfqpoint{4.530384in}{4.855976in}}{\pgfqpoint{4.532698in}{4.861562in}}{\pgfqpoint{4.532698in}{4.867386in}}%
\pgfpathcurveto{\pgfqpoint{4.532698in}{4.873210in}}{\pgfqpoint{4.530384in}{4.878796in}}{\pgfqpoint{4.526266in}{4.882914in}}%
\pgfpathcurveto{\pgfqpoint{4.522148in}{4.887032in}}{\pgfqpoint{4.516561in}{4.889346in}}{\pgfqpoint{4.510738in}{4.889346in}}%
\pgfpathcurveto{\pgfqpoint{4.504914in}{4.889346in}}{\pgfqpoint{4.499327in}{4.887032in}}{\pgfqpoint{4.495209in}{4.882914in}}%
\pgfpathcurveto{\pgfqpoint{4.491091in}{4.878796in}}{\pgfqpoint{4.488777in}{4.873210in}}{\pgfqpoint{4.488777in}{4.867386in}}%
\pgfpathcurveto{\pgfqpoint{4.488777in}{4.861562in}}{\pgfqpoint{4.491091in}{4.855976in}}{\pgfqpoint{4.495209in}{4.851858in}}%
\pgfpathcurveto{\pgfqpoint{4.499327in}{4.847740in}}{\pgfqpoint{4.504914in}{4.845426in}}{\pgfqpoint{4.510738in}{4.845426in}}%
\pgfpathlineto{\pgfqpoint{4.510738in}{4.845426in}}%
\pgfpathclose%
\pgfusepath{stroke,fill}%
\end{pgfscope}%
\begin{pgfscope}%
\pgfpathrectangle{\pgfqpoint{1.542338in}{0.880000in}}{\pgfqpoint{5.115323in}{6.160000in}}%
\pgfusepath{clip}%
\pgfsetbuttcap%
\pgfsetroundjoin%
\definecolor{currentfill}{rgb}{0.200000,0.800000,0.200000}%
\pgfsetfillcolor{currentfill}%
\pgfsetlinewidth{1.003750pt}%
\definecolor{currentstroke}{rgb}{0.200000,0.800000,0.200000}%
\pgfsetstrokecolor{currentstroke}%
\pgfsetdash{}{0pt}%
\pgfpathmoveto{\pgfqpoint{4.631522in}{4.856791in}}%
\pgfpathcurveto{\pgfqpoint{4.637346in}{4.856791in}}{\pgfqpoint{4.642933in}{4.859105in}}{\pgfqpoint{4.647051in}{4.863223in}}%
\pgfpathcurveto{\pgfqpoint{4.651169in}{4.867341in}}{\pgfqpoint{4.653483in}{4.872927in}}{\pgfqpoint{4.653483in}{4.878751in}}%
\pgfpathcurveto{\pgfqpoint{4.653483in}{4.884575in}}{\pgfqpoint{4.651169in}{4.890161in}}{\pgfqpoint{4.647051in}{4.894279in}}%
\pgfpathcurveto{\pgfqpoint{4.642933in}{4.898398in}}{\pgfqpoint{4.637346in}{4.900711in}}{\pgfqpoint{4.631522in}{4.900711in}}%
\pgfpathcurveto{\pgfqpoint{4.625698in}{4.900711in}}{\pgfqpoint{4.620112in}{4.898398in}}{\pgfqpoint{4.615994in}{4.894279in}}%
\pgfpathcurveto{\pgfqpoint{4.611876in}{4.890161in}}{\pgfqpoint{4.609562in}{4.884575in}}{\pgfqpoint{4.609562in}{4.878751in}}%
\pgfpathcurveto{\pgfqpoint{4.609562in}{4.872927in}}{\pgfqpoint{4.611876in}{4.867341in}}{\pgfqpoint{4.615994in}{4.863223in}}%
\pgfpathcurveto{\pgfqpoint{4.620112in}{4.859105in}}{\pgfqpoint{4.625698in}{4.856791in}}{\pgfqpoint{4.631522in}{4.856791in}}%
\pgfpathlineto{\pgfqpoint{4.631522in}{4.856791in}}%
\pgfpathclose%
\pgfusepath{stroke,fill}%
\end{pgfscope}%
\begin{pgfscope}%
\pgfpathrectangle{\pgfqpoint{1.542338in}{0.880000in}}{\pgfqpoint{5.115323in}{6.160000in}}%
\pgfusepath{clip}%
\pgfsetbuttcap%
\pgfsetroundjoin%
\definecolor{currentfill}{rgb}{0.200000,0.800000,0.200000}%
\pgfsetfillcolor{currentfill}%
\pgfsetlinewidth{1.003750pt}%
\definecolor{currentstroke}{rgb}{0.200000,0.800000,0.200000}%
\pgfsetstrokecolor{currentstroke}%
\pgfsetdash{}{0pt}%
\pgfpathmoveto{\pgfqpoint{4.749812in}{4.883805in}}%
\pgfpathcurveto{\pgfqpoint{4.755636in}{4.883805in}}{\pgfqpoint{4.761223in}{4.886119in}}{\pgfqpoint{4.765341in}{4.890237in}}%
\pgfpathcurveto{\pgfqpoint{4.769459in}{4.894355in}}{\pgfqpoint{4.771773in}{4.899941in}}{\pgfqpoint{4.771773in}{4.905765in}}%
\pgfpathcurveto{\pgfqpoint{4.771773in}{4.911589in}}{\pgfqpoint{4.769459in}{4.917175in}}{\pgfqpoint{4.765341in}{4.921293in}}%
\pgfpathcurveto{\pgfqpoint{4.761223in}{4.925411in}}{\pgfqpoint{4.755636in}{4.927725in}}{\pgfqpoint{4.749812in}{4.927725in}}%
\pgfpathcurveto{\pgfqpoint{4.743989in}{4.927725in}}{\pgfqpoint{4.738402in}{4.925411in}}{\pgfqpoint{4.734284in}{4.921293in}}%
\pgfpathcurveto{\pgfqpoint{4.730166in}{4.917175in}}{\pgfqpoint{4.727852in}{4.911589in}}{\pgfqpoint{4.727852in}{4.905765in}}%
\pgfpathcurveto{\pgfqpoint{4.727852in}{4.899941in}}{\pgfqpoint{4.730166in}{4.894355in}}{\pgfqpoint{4.734284in}{4.890237in}}%
\pgfpathcurveto{\pgfqpoint{4.738402in}{4.886119in}}{\pgfqpoint{4.743989in}{4.883805in}}{\pgfqpoint{4.749812in}{4.883805in}}%
\pgfpathlineto{\pgfqpoint{4.749812in}{4.883805in}}%
\pgfpathclose%
\pgfusepath{stroke,fill}%
\end{pgfscope}%
\begin{pgfscope}%
\pgfpathrectangle{\pgfqpoint{1.542338in}{0.880000in}}{\pgfqpoint{5.115323in}{6.160000in}}%
\pgfusepath{clip}%
\pgfsetbuttcap%
\pgfsetroundjoin%
\definecolor{currentfill}{rgb}{0.200000,0.800000,0.200000}%
\pgfsetfillcolor{currentfill}%
\pgfsetlinewidth{1.003750pt}%
\definecolor{currentstroke}{rgb}{0.200000,0.800000,0.200000}%
\pgfsetstrokecolor{currentstroke}%
\pgfsetdash{}{0pt}%
\pgfpathmoveto{\pgfqpoint{4.864483in}{4.923899in}}%
\pgfpathcurveto{\pgfqpoint{4.870307in}{4.923899in}}{\pgfqpoint{4.875893in}{4.926213in}}{\pgfqpoint{4.880011in}{4.930331in}}%
\pgfpathcurveto{\pgfqpoint{4.884129in}{4.934449in}}{\pgfqpoint{4.886443in}{4.940035in}}{\pgfqpoint{4.886443in}{4.945859in}}%
\pgfpathcurveto{\pgfqpoint{4.886443in}{4.951683in}}{\pgfqpoint{4.884129in}{4.957269in}}{\pgfqpoint{4.880011in}{4.961387in}}%
\pgfpathcurveto{\pgfqpoint{4.875893in}{4.965505in}}{\pgfqpoint{4.870307in}{4.967819in}}{\pgfqpoint{4.864483in}{4.967819in}}%
\pgfpathcurveto{\pgfqpoint{4.858659in}{4.967819in}}{\pgfqpoint{4.853073in}{4.965505in}}{\pgfqpoint{4.848954in}{4.961387in}}%
\pgfpathcurveto{\pgfqpoint{4.844836in}{4.957269in}}{\pgfqpoint{4.842522in}{4.951683in}}{\pgfqpoint{4.842522in}{4.945859in}}%
\pgfpathcurveto{\pgfqpoint{4.842522in}{4.940035in}}{\pgfqpoint{4.844836in}{4.934449in}}{\pgfqpoint{4.848954in}{4.930331in}}%
\pgfpathcurveto{\pgfqpoint{4.853073in}{4.926213in}}{\pgfqpoint{4.858659in}{4.923899in}}{\pgfqpoint{4.864483in}{4.923899in}}%
\pgfpathlineto{\pgfqpoint{4.864483in}{4.923899in}}%
\pgfpathclose%
\pgfusepath{stroke,fill}%
\end{pgfscope}%
\begin{pgfscope}%
\pgfpathrectangle{\pgfqpoint{1.542338in}{0.880000in}}{\pgfqpoint{5.115323in}{6.160000in}}%
\pgfusepath{clip}%
\pgfsetbuttcap%
\pgfsetroundjoin%
\definecolor{currentfill}{rgb}{0.200000,0.800000,0.200000}%
\pgfsetfillcolor{currentfill}%
\pgfsetlinewidth{1.003750pt}%
\definecolor{currentstroke}{rgb}{0.200000,0.800000,0.200000}%
\pgfsetstrokecolor{currentstroke}%
\pgfsetdash{}{0pt}%
\pgfpathmoveto{\pgfqpoint{4.976479in}{4.973216in}}%
\pgfpathcurveto{\pgfqpoint{4.982303in}{4.973216in}}{\pgfqpoint{4.987889in}{4.975530in}}{\pgfqpoint{4.992007in}{4.979648in}}%
\pgfpathcurveto{\pgfqpoint{4.996125in}{4.983766in}}{\pgfqpoint{4.998439in}{4.989352in}}{\pgfqpoint{4.998439in}{4.995176in}}%
\pgfpathcurveto{\pgfqpoint{4.998439in}{5.001000in}}{\pgfqpoint{4.996125in}{5.006586in}}{\pgfqpoint{4.992007in}{5.010704in}}%
\pgfpathcurveto{\pgfqpoint{4.987889in}{5.014822in}}{\pgfqpoint{4.982303in}{5.017136in}}{\pgfqpoint{4.976479in}{5.017136in}}%
\pgfpathcurveto{\pgfqpoint{4.970655in}{5.017136in}}{\pgfqpoint{4.965069in}{5.014822in}}{\pgfqpoint{4.960951in}{5.010704in}}%
\pgfpathcurveto{\pgfqpoint{4.956833in}{5.006586in}}{\pgfqpoint{4.954519in}{5.001000in}}{\pgfqpoint{4.954519in}{4.995176in}}%
\pgfpathcurveto{\pgfqpoint{4.954519in}{4.989352in}}{\pgfqpoint{4.956833in}{4.983766in}}{\pgfqpoint{4.960951in}{4.979648in}}%
\pgfpathcurveto{\pgfqpoint{4.965069in}{4.975530in}}{\pgfqpoint{4.970655in}{4.973216in}}{\pgfqpoint{4.976479in}{4.973216in}}%
\pgfpathlineto{\pgfqpoint{4.976479in}{4.973216in}}%
\pgfpathclose%
\pgfusepath{stroke,fill}%
\end{pgfscope}%
\begin{pgfscope}%
\pgfpathrectangle{\pgfqpoint{1.542338in}{0.880000in}}{\pgfqpoint{5.115323in}{6.160000in}}%
\pgfusepath{clip}%
\pgfsetbuttcap%
\pgfsetroundjoin%
\definecolor{currentfill}{rgb}{0.200000,0.800000,0.200000}%
\pgfsetfillcolor{currentfill}%
\pgfsetlinewidth{1.003750pt}%
\definecolor{currentstroke}{rgb}{0.200000,0.800000,0.200000}%
\pgfsetstrokecolor{currentstroke}%
\pgfsetdash{}{0pt}%
\pgfpathmoveto{\pgfqpoint{5.073520in}{5.047803in}}%
\pgfpathcurveto{\pgfqpoint{5.079344in}{5.047803in}}{\pgfqpoint{5.084930in}{5.050117in}}{\pgfqpoint{5.089048in}{5.054235in}}%
\pgfpathcurveto{\pgfqpoint{5.093166in}{5.058353in}}{\pgfqpoint{5.095480in}{5.063939in}}{\pgfqpoint{5.095480in}{5.069763in}}%
\pgfpathcurveto{\pgfqpoint{5.095480in}{5.075587in}}{\pgfqpoint{5.093166in}{5.081173in}}{\pgfqpoint{5.089048in}{5.085291in}}%
\pgfpathcurveto{\pgfqpoint{5.084930in}{5.089409in}}{\pgfqpoint{5.079344in}{5.091723in}}{\pgfqpoint{5.073520in}{5.091723in}}%
\pgfpathcurveto{\pgfqpoint{5.067696in}{5.091723in}}{\pgfqpoint{5.062110in}{5.089409in}}{\pgfqpoint{5.057991in}{5.085291in}}%
\pgfpathcurveto{\pgfqpoint{5.053873in}{5.081173in}}{\pgfqpoint{5.051559in}{5.075587in}}{\pgfqpoint{5.051559in}{5.069763in}}%
\pgfpathcurveto{\pgfqpoint{5.051559in}{5.063939in}}{\pgfqpoint{5.053873in}{5.058353in}}{\pgfqpoint{5.057991in}{5.054235in}}%
\pgfpathcurveto{\pgfqpoint{5.062110in}{5.050117in}}{\pgfqpoint{5.067696in}{5.047803in}}{\pgfqpoint{5.073520in}{5.047803in}}%
\pgfpathlineto{\pgfqpoint{5.073520in}{5.047803in}}%
\pgfpathclose%
\pgfusepath{stroke,fill}%
\end{pgfscope}%
\begin{pgfscope}%
\pgfpathrectangle{\pgfqpoint{1.542338in}{0.880000in}}{\pgfqpoint{5.115323in}{6.160000in}}%
\pgfusepath{clip}%
\pgfsetbuttcap%
\pgfsetroundjoin%
\definecolor{currentfill}{rgb}{0.200000,0.800000,0.200000}%
\pgfsetfillcolor{currentfill}%
\pgfsetlinewidth{1.003750pt}%
\definecolor{currentstroke}{rgb}{0.200000,0.800000,0.200000}%
\pgfsetstrokecolor{currentstroke}%
\pgfsetdash{}{0pt}%
\pgfpathmoveto{\pgfqpoint{5.159169in}{5.134201in}}%
\pgfpathcurveto{\pgfqpoint{5.164993in}{5.134201in}}{\pgfqpoint{5.170579in}{5.136515in}}{\pgfqpoint{5.174697in}{5.140633in}}%
\pgfpathcurveto{\pgfqpoint{5.178816in}{5.144751in}}{\pgfqpoint{5.181130in}{5.150337in}}{\pgfqpoint{5.181130in}{5.156161in}}%
\pgfpathcurveto{\pgfqpoint{5.181130in}{5.161985in}}{\pgfqpoint{5.178816in}{5.167571in}}{\pgfqpoint{5.174697in}{5.171689in}}%
\pgfpathcurveto{\pgfqpoint{5.170579in}{5.175807in}}{\pgfqpoint{5.164993in}{5.178121in}}{\pgfqpoint{5.159169in}{5.178121in}}%
\pgfpathcurveto{\pgfqpoint{5.153345in}{5.178121in}}{\pgfqpoint{5.147759in}{5.175807in}}{\pgfqpoint{5.143641in}{5.171689in}}%
\pgfpathcurveto{\pgfqpoint{5.139523in}{5.167571in}}{\pgfqpoint{5.137209in}{5.161985in}}{\pgfqpoint{5.137209in}{5.156161in}}%
\pgfpathcurveto{\pgfqpoint{5.137209in}{5.150337in}}{\pgfqpoint{5.139523in}{5.144751in}}{\pgfqpoint{5.143641in}{5.140633in}}%
\pgfpathcurveto{\pgfqpoint{5.147759in}{5.136515in}}{\pgfqpoint{5.153345in}{5.134201in}}{\pgfqpoint{5.159169in}{5.134201in}}%
\pgfpathlineto{\pgfqpoint{5.159169in}{5.134201in}}%
\pgfpathclose%
\pgfusepath{stroke,fill}%
\end{pgfscope}%
\begin{pgfscope}%
\pgfpathrectangle{\pgfqpoint{1.542338in}{0.880000in}}{\pgfqpoint{5.115323in}{6.160000in}}%
\pgfusepath{clip}%
\pgfsetbuttcap%
\pgfsetroundjoin%
\definecolor{currentfill}{rgb}{0.200000,0.800000,0.200000}%
\pgfsetfillcolor{currentfill}%
\pgfsetlinewidth{1.003750pt}%
\definecolor{currentstroke}{rgb}{0.200000,0.800000,0.200000}%
\pgfsetstrokecolor{currentstroke}%
\pgfsetdash{}{0pt}%
\pgfpathmoveto{\pgfqpoint{5.242733in}{5.222609in}}%
\pgfpathcurveto{\pgfqpoint{5.248557in}{5.222609in}}{\pgfqpoint{5.254143in}{5.224922in}}{\pgfqpoint{5.258261in}{5.229041in}}%
\pgfpathcurveto{\pgfqpoint{5.262380in}{5.233159in}}{\pgfqpoint{5.264693in}{5.238745in}}{\pgfqpoint{5.264693in}{5.244569in}}%
\pgfpathcurveto{\pgfqpoint{5.264693in}{5.250393in}}{\pgfqpoint{5.262380in}{5.255979in}}{\pgfqpoint{5.258261in}{5.260097in}}%
\pgfpathcurveto{\pgfqpoint{5.254143in}{5.264215in}}{\pgfqpoint{5.248557in}{5.266529in}}{\pgfqpoint{5.242733in}{5.266529in}}%
\pgfpathcurveto{\pgfqpoint{5.236909in}{5.266529in}}{\pgfqpoint{5.231323in}{5.264215in}}{\pgfqpoint{5.227205in}{5.260097in}}%
\pgfpathcurveto{\pgfqpoint{5.223087in}{5.255979in}}{\pgfqpoint{5.220773in}{5.250393in}}{\pgfqpoint{5.220773in}{5.244569in}}%
\pgfpathcurveto{\pgfqpoint{5.220773in}{5.238745in}}{\pgfqpoint{5.223087in}{5.233159in}}{\pgfqpoint{5.227205in}{5.229041in}}%
\pgfpathcurveto{\pgfqpoint{5.231323in}{5.224922in}}{\pgfqpoint{5.236909in}{5.222609in}}{\pgfqpoint{5.242733in}{5.222609in}}%
\pgfpathlineto{\pgfqpoint{5.242733in}{5.222609in}}%
\pgfpathclose%
\pgfusepath{stroke,fill}%
\end{pgfscope}%
\begin{pgfscope}%
\pgfpathrectangle{\pgfqpoint{1.542338in}{0.880000in}}{\pgfqpoint{5.115323in}{6.160000in}}%
\pgfusepath{clip}%
\pgfsetbuttcap%
\pgfsetroundjoin%
\definecolor{currentfill}{rgb}{0.200000,0.800000,0.200000}%
\pgfsetfillcolor{currentfill}%
\pgfsetlinewidth{1.003750pt}%
\definecolor{currentstroke}{rgb}{0.200000,0.800000,0.200000}%
\pgfsetstrokecolor{currentstroke}%
\pgfsetdash{}{0pt}%
\pgfpathmoveto{\pgfqpoint{5.301292in}{5.329235in}}%
\pgfpathcurveto{\pgfqpoint{5.307116in}{5.329235in}}{\pgfqpoint{5.312702in}{5.331549in}}{\pgfqpoint{5.316820in}{5.335667in}}%
\pgfpathcurveto{\pgfqpoint{5.320938in}{5.339786in}}{\pgfqpoint{5.323252in}{5.345372in}}{\pgfqpoint{5.323252in}{5.351196in}}%
\pgfpathcurveto{\pgfqpoint{5.323252in}{5.357020in}}{\pgfqpoint{5.320938in}{5.362606in}}{\pgfqpoint{5.316820in}{5.366724in}}%
\pgfpathcurveto{\pgfqpoint{5.312702in}{5.370842in}}{\pgfqpoint{5.307116in}{5.373156in}}{\pgfqpoint{5.301292in}{5.373156in}}%
\pgfpathcurveto{\pgfqpoint{5.295468in}{5.373156in}}{\pgfqpoint{5.289882in}{5.370842in}}{\pgfqpoint{5.285764in}{5.366724in}}%
\pgfpathcurveto{\pgfqpoint{5.281646in}{5.362606in}}{\pgfqpoint{5.279332in}{5.357020in}}{\pgfqpoint{5.279332in}{5.351196in}}%
\pgfpathcurveto{\pgfqpoint{5.279332in}{5.345372in}}{\pgfqpoint{5.281646in}{5.339786in}}{\pgfqpoint{5.285764in}{5.335667in}}%
\pgfpathcurveto{\pgfqpoint{5.289882in}{5.331549in}}{\pgfqpoint{5.295468in}{5.329235in}}{\pgfqpoint{5.301292in}{5.329235in}}%
\pgfpathlineto{\pgfqpoint{5.301292in}{5.329235in}}%
\pgfpathclose%
\pgfusepath{stroke,fill}%
\end{pgfscope}%
\begin{pgfscope}%
\pgfpathrectangle{\pgfqpoint{1.542338in}{0.880000in}}{\pgfqpoint{5.115323in}{6.160000in}}%
\pgfusepath{clip}%
\pgfsetbuttcap%
\pgfsetroundjoin%
\definecolor{currentfill}{rgb}{0.200000,0.800000,0.200000}%
\pgfsetfillcolor{currentfill}%
\pgfsetlinewidth{1.003750pt}%
\definecolor{currentstroke}{rgb}{0.200000,0.800000,0.200000}%
\pgfsetstrokecolor{currentstroke}%
\pgfsetdash{}{0pt}%
\pgfpathmoveto{\pgfqpoint{5.355487in}{5.437270in}}%
\pgfpathcurveto{\pgfqpoint{5.361311in}{5.437270in}}{\pgfqpoint{5.366897in}{5.439584in}}{\pgfqpoint{5.371015in}{5.443702in}}%
\pgfpathcurveto{\pgfqpoint{5.375133in}{5.447820in}}{\pgfqpoint{5.377447in}{5.453407in}}{\pgfqpoint{5.377447in}{5.459231in}}%
\pgfpathcurveto{\pgfqpoint{5.377447in}{5.465055in}}{\pgfqpoint{5.375133in}{5.470641in}}{\pgfqpoint{5.371015in}{5.474759in}}%
\pgfpathcurveto{\pgfqpoint{5.366897in}{5.478877in}}{\pgfqpoint{5.361311in}{5.481191in}}{\pgfqpoint{5.355487in}{5.481191in}}%
\pgfpathcurveto{\pgfqpoint{5.349663in}{5.481191in}}{\pgfqpoint{5.344077in}{5.478877in}}{\pgfqpoint{5.339959in}{5.474759in}}%
\pgfpathcurveto{\pgfqpoint{5.335840in}{5.470641in}}{\pgfqpoint{5.333526in}{5.465055in}}{\pgfqpoint{5.333526in}{5.459231in}}%
\pgfpathcurveto{\pgfqpoint{5.333526in}{5.453407in}}{\pgfqpoint{5.335840in}{5.447820in}}{\pgfqpoint{5.339959in}{5.443702in}}%
\pgfpathcurveto{\pgfqpoint{5.344077in}{5.439584in}}{\pgfqpoint{5.349663in}{5.437270in}}{\pgfqpoint{5.355487in}{5.437270in}}%
\pgfpathlineto{\pgfqpoint{5.355487in}{5.437270in}}%
\pgfpathclose%
\pgfusepath{stroke,fill}%
\end{pgfscope}%
\begin{pgfscope}%
\pgfpathrectangle{\pgfqpoint{1.542338in}{0.880000in}}{\pgfqpoint{5.115323in}{6.160000in}}%
\pgfusepath{clip}%
\pgfsetbuttcap%
\pgfsetroundjoin%
\definecolor{currentfill}{rgb}{0.200000,0.800000,0.200000}%
\pgfsetfillcolor{currentfill}%
\pgfsetlinewidth{1.003750pt}%
\definecolor{currentstroke}{rgb}{0.200000,0.800000,0.200000}%
\pgfsetstrokecolor{currentstroke}%
\pgfsetdash{}{0pt}%
\pgfpathmoveto{\pgfqpoint{5.398551in}{5.550787in}}%
\pgfpathcurveto{\pgfqpoint{5.404375in}{5.550787in}}{\pgfqpoint{5.409961in}{5.553100in}}{\pgfqpoint{5.414079in}{5.557219in}}%
\pgfpathcurveto{\pgfqpoint{5.418197in}{5.561337in}}{\pgfqpoint{5.420511in}{5.566923in}}{\pgfqpoint{5.420511in}{5.572747in}}%
\pgfpathcurveto{\pgfqpoint{5.420511in}{5.578571in}}{\pgfqpoint{5.418197in}{5.584157in}}{\pgfqpoint{5.414079in}{5.588275in}}%
\pgfpathcurveto{\pgfqpoint{5.409961in}{5.592393in}}{\pgfqpoint{5.404375in}{5.594707in}}{\pgfqpoint{5.398551in}{5.594707in}}%
\pgfpathcurveto{\pgfqpoint{5.392727in}{5.594707in}}{\pgfqpoint{5.387141in}{5.592393in}}{\pgfqpoint{5.383023in}{5.588275in}}%
\pgfpathcurveto{\pgfqpoint{5.378905in}{5.584157in}}{\pgfqpoint{5.376591in}{5.578571in}}{\pgfqpoint{5.376591in}{5.572747in}}%
\pgfpathcurveto{\pgfqpoint{5.376591in}{5.566923in}}{\pgfqpoint{5.378905in}{5.561337in}}{\pgfqpoint{5.383023in}{5.557219in}}%
\pgfpathcurveto{\pgfqpoint{5.387141in}{5.553100in}}{\pgfqpoint{5.392727in}{5.550787in}}{\pgfqpoint{5.398551in}{5.550787in}}%
\pgfpathlineto{\pgfqpoint{5.398551in}{5.550787in}}%
\pgfpathclose%
\pgfusepath{stroke,fill}%
\end{pgfscope}%
\begin{pgfscope}%
\pgfpathrectangle{\pgfqpoint{1.542338in}{0.880000in}}{\pgfqpoint{5.115323in}{6.160000in}}%
\pgfusepath{clip}%
\pgfsetbuttcap%
\pgfsetroundjoin%
\definecolor{currentfill}{rgb}{0.200000,0.800000,0.200000}%
\pgfsetfillcolor{currentfill}%
\pgfsetlinewidth{1.003750pt}%
\definecolor{currentstroke}{rgb}{0.200000,0.800000,0.200000}%
\pgfsetstrokecolor{currentstroke}%
\pgfsetdash{}{0pt}%
\pgfpathmoveto{\pgfqpoint{5.421678in}{5.670190in}}%
\pgfpathcurveto{\pgfqpoint{5.427502in}{5.670190in}}{\pgfqpoint{5.433088in}{5.672504in}}{\pgfqpoint{5.437206in}{5.676622in}}%
\pgfpathcurveto{\pgfqpoint{5.441325in}{5.680740in}}{\pgfqpoint{5.443638in}{5.686326in}}{\pgfqpoint{5.443638in}{5.692150in}}%
\pgfpathcurveto{\pgfqpoint{5.443638in}{5.697974in}}{\pgfqpoint{5.441325in}{5.703561in}}{\pgfqpoint{5.437206in}{5.707679in}}%
\pgfpathcurveto{\pgfqpoint{5.433088in}{5.711797in}}{\pgfqpoint{5.427502in}{5.714111in}}{\pgfqpoint{5.421678in}{5.714111in}}%
\pgfpathcurveto{\pgfqpoint{5.415854in}{5.714111in}}{\pgfqpoint{5.410268in}{5.711797in}}{\pgfqpoint{5.406150in}{5.707679in}}%
\pgfpathcurveto{\pgfqpoint{5.402032in}{5.703561in}}{\pgfqpoint{5.399718in}{5.697974in}}{\pgfqpoint{5.399718in}{5.692150in}}%
\pgfpathcurveto{\pgfqpoint{5.399718in}{5.686326in}}{\pgfqpoint{5.402032in}{5.680740in}}{\pgfqpoint{5.406150in}{5.676622in}}%
\pgfpathcurveto{\pgfqpoint{5.410268in}{5.672504in}}{\pgfqpoint{5.415854in}{5.670190in}}{\pgfqpoint{5.421678in}{5.670190in}}%
\pgfpathlineto{\pgfqpoint{5.421678in}{5.670190in}}%
\pgfpathclose%
\pgfusepath{stroke,fill}%
\end{pgfscope}%
\begin{pgfscope}%
\pgfpathrectangle{\pgfqpoint{1.542338in}{0.880000in}}{\pgfqpoint{5.115323in}{6.160000in}}%
\pgfusepath{clip}%
\pgfsetbuttcap%
\pgfsetroundjoin%
\definecolor{currentfill}{rgb}{0.200000,0.800000,0.200000}%
\pgfsetfillcolor{currentfill}%
\pgfsetlinewidth{1.003750pt}%
\definecolor{currentstroke}{rgb}{0.200000,0.800000,0.200000}%
\pgfsetstrokecolor{currentstroke}%
\pgfsetdash{}{0pt}%
\pgfpathmoveto{\pgfqpoint{5.429172in}{5.791555in}}%
\pgfpathcurveto{\pgfqpoint{5.434996in}{5.791555in}}{\pgfqpoint{5.440582in}{5.793869in}}{\pgfqpoint{5.444700in}{5.797987in}}%
\pgfpathcurveto{\pgfqpoint{5.448818in}{5.802105in}}{\pgfqpoint{5.451132in}{5.807691in}}{\pgfqpoint{5.451132in}{5.813515in}}%
\pgfpathcurveto{\pgfqpoint{5.451132in}{5.819339in}}{\pgfqpoint{5.448818in}{5.824925in}}{\pgfqpoint{5.444700in}{5.829043in}}%
\pgfpathcurveto{\pgfqpoint{5.440582in}{5.833162in}}{\pgfqpoint{5.434996in}{5.835475in}}{\pgfqpoint{5.429172in}{5.835475in}}%
\pgfpathcurveto{\pgfqpoint{5.423348in}{5.835475in}}{\pgfqpoint{5.417762in}{5.833162in}}{\pgfqpoint{5.413643in}{5.829043in}}%
\pgfpathcurveto{\pgfqpoint{5.409525in}{5.824925in}}{\pgfqpoint{5.407211in}{5.819339in}}{\pgfqpoint{5.407211in}{5.813515in}}%
\pgfpathcurveto{\pgfqpoint{5.407211in}{5.807691in}}{\pgfqpoint{5.409525in}{5.802105in}}{\pgfqpoint{5.413643in}{5.797987in}}%
\pgfpathcurveto{\pgfqpoint{5.417762in}{5.793869in}}{\pgfqpoint{5.423348in}{5.791555in}}{\pgfqpoint{5.429172in}{5.791555in}}%
\pgfpathlineto{\pgfqpoint{5.429172in}{5.791555in}}%
\pgfpathclose%
\pgfusepath{stroke,fill}%
\end{pgfscope}%
\begin{pgfscope}%
\pgfpathrectangle{\pgfqpoint{1.542338in}{0.880000in}}{\pgfqpoint{5.115323in}{6.160000in}}%
\pgfusepath{clip}%
\pgfsetbuttcap%
\pgfsetroundjoin%
\definecolor{currentfill}{rgb}{0.500000,0.000000,0.500000}%
\pgfsetfillcolor{currentfill}%
\pgfsetlinewidth{1.003750pt}%
\definecolor{currentstroke}{rgb}{0.500000,0.000000,0.500000}%
\pgfsetstrokecolor{currentstroke}%
\pgfsetdash{}{0pt}%
\pgfpathmoveto{\pgfqpoint{3.019964in}{5.535553in}}%
\pgfpathcurveto{\pgfqpoint{3.025788in}{5.535553in}}{\pgfqpoint{3.031374in}{5.537867in}}{\pgfqpoint{3.035492in}{5.541985in}}%
\pgfpathcurveto{\pgfqpoint{3.039610in}{5.546103in}}{\pgfqpoint{3.041924in}{5.551690in}}{\pgfqpoint{3.041924in}{5.557514in}}%
\pgfpathcurveto{\pgfqpoint{3.041924in}{5.563337in}}{\pgfqpoint{3.039610in}{5.568924in}}{\pgfqpoint{3.035492in}{5.573042in}}%
\pgfpathcurveto{\pgfqpoint{3.031374in}{5.577160in}}{\pgfqpoint{3.025788in}{5.579474in}}{\pgfqpoint{3.019964in}{5.579474in}}%
\pgfpathcurveto{\pgfqpoint{3.014140in}{5.579474in}}{\pgfqpoint{3.008554in}{5.577160in}}{\pgfqpoint{3.004436in}{5.573042in}}%
\pgfpathcurveto{\pgfqpoint{3.000317in}{5.568924in}}{\pgfqpoint{2.998004in}{5.563337in}}{\pgfqpoint{2.998004in}{5.557514in}}%
\pgfpathcurveto{\pgfqpoint{2.998004in}{5.551690in}}{\pgfqpoint{3.000317in}{5.546103in}}{\pgfqpoint{3.004436in}{5.541985in}}%
\pgfpathcurveto{\pgfqpoint{3.008554in}{5.537867in}}{\pgfqpoint{3.014140in}{5.535553in}}{\pgfqpoint{3.019964in}{5.535553in}}%
\pgfpathlineto{\pgfqpoint{3.019964in}{5.535553in}}%
\pgfpathclose%
\pgfusepath{stroke,fill}%
\end{pgfscope}%
\begin{pgfscope}%
\pgfpathrectangle{\pgfqpoint{1.542338in}{0.880000in}}{\pgfqpoint{5.115323in}{6.160000in}}%
\pgfusepath{clip}%
\pgfsetbuttcap%
\pgfsetroundjoin%
\definecolor{currentfill}{rgb}{0.500000,0.000000,0.500000}%
\pgfsetfillcolor{currentfill}%
\pgfsetlinewidth{1.003750pt}%
\definecolor{currentstroke}{rgb}{0.500000,0.000000,0.500000}%
\pgfsetstrokecolor{currentstroke}%
\pgfsetdash{}{0pt}%
\pgfpathmoveto{\pgfqpoint{2.020723in}{6.031441in}}%
\pgfpathcurveto{\pgfqpoint{2.026547in}{6.031441in}}{\pgfqpoint{2.032133in}{6.033755in}}{\pgfqpoint{2.036251in}{6.037873in}}%
\pgfpathcurveto{\pgfqpoint{2.040369in}{6.041992in}}{\pgfqpoint{2.042683in}{6.047578in}}{\pgfqpoint{2.042683in}{6.053402in}}%
\pgfpathcurveto{\pgfqpoint{2.042683in}{6.059226in}}{\pgfqpoint{2.040369in}{6.064812in}}{\pgfqpoint{2.036251in}{6.068930in}}%
\pgfpathcurveto{\pgfqpoint{2.032133in}{6.073048in}}{\pgfqpoint{2.026547in}{6.075362in}}{\pgfqpoint{2.020723in}{6.075362in}}%
\pgfpathcurveto{\pgfqpoint{2.014899in}{6.075362in}}{\pgfqpoint{2.009313in}{6.073048in}}{\pgfqpoint{2.005194in}{6.068930in}}%
\pgfpathcurveto{\pgfqpoint{2.001076in}{6.064812in}}{\pgfqpoint{1.998762in}{6.059226in}}{\pgfqpoint{1.998762in}{6.053402in}}%
\pgfpathcurveto{\pgfqpoint{1.998762in}{6.047578in}}{\pgfqpoint{2.001076in}{6.041992in}}{\pgfqpoint{2.005194in}{6.037873in}}%
\pgfpathcurveto{\pgfqpoint{2.009313in}{6.033755in}}{\pgfqpoint{2.014899in}{6.031441in}}{\pgfqpoint{2.020723in}{6.031441in}}%
\pgfpathlineto{\pgfqpoint{2.020723in}{6.031441in}}%
\pgfpathclose%
\pgfusepath{stroke,fill}%
\end{pgfscope}%
\begin{pgfscope}%
\pgfpathrectangle{\pgfqpoint{1.542338in}{0.880000in}}{\pgfqpoint{5.115323in}{6.160000in}}%
\pgfusepath{clip}%
\pgfsetbuttcap%
\pgfsetroundjoin%
\definecolor{currentfill}{rgb}{0.200000,0.200000,0.800000}%
\pgfsetfillcolor{currentfill}%
\pgfsetlinewidth{1.003750pt}%
\definecolor{currentstroke}{rgb}{0.200000,0.200000,0.800000}%
\pgfsetstrokecolor{currentstroke}%
\pgfsetdash{}{0pt}%
\pgfpathmoveto{\pgfqpoint{2.802067in}{3.661123in}}%
\pgfpathcurveto{\pgfqpoint{2.807891in}{3.661123in}}{\pgfqpoint{2.813477in}{3.663437in}}{\pgfqpoint{2.817595in}{3.667555in}}%
\pgfpathcurveto{\pgfqpoint{2.821713in}{3.671673in}}{\pgfqpoint{2.824027in}{3.677259in}}{\pgfqpoint{2.824027in}{3.683083in}}%
\pgfpathcurveto{\pgfqpoint{2.824027in}{3.688907in}}{\pgfqpoint{2.821713in}{3.694493in}}{\pgfqpoint{2.817595in}{3.698612in}}%
\pgfpathcurveto{\pgfqpoint{2.813477in}{3.702730in}}{\pgfqpoint{2.807891in}{3.705044in}}{\pgfqpoint{2.802067in}{3.705044in}}%
\pgfpathcurveto{\pgfqpoint{2.796243in}{3.705044in}}{\pgfqpoint{2.790657in}{3.702730in}}{\pgfqpoint{2.786539in}{3.698612in}}%
\pgfpathcurveto{\pgfqpoint{2.782420in}{3.694493in}}{\pgfqpoint{2.780106in}{3.688907in}}{\pgfqpoint{2.780106in}{3.683083in}}%
\pgfpathcurveto{\pgfqpoint{2.780106in}{3.677259in}}{\pgfqpoint{2.782420in}{3.671673in}}{\pgfqpoint{2.786539in}{3.667555in}}%
\pgfpathcurveto{\pgfqpoint{2.790657in}{3.663437in}}{\pgfqpoint{2.796243in}{3.661123in}}{\pgfqpoint{2.802067in}{3.661123in}}%
\pgfpathlineto{\pgfqpoint{2.802067in}{3.661123in}}%
\pgfpathclose%
\pgfusepath{stroke,fill}%
\end{pgfscope}%
\begin{pgfscope}%
\pgfpathrectangle{\pgfqpoint{1.542338in}{0.880000in}}{\pgfqpoint{5.115323in}{6.160000in}}%
\pgfusepath{clip}%
\pgfsetbuttcap%
\pgfsetroundjoin%
\definecolor{currentfill}{rgb}{0.500000,0.000000,0.500000}%
\pgfsetfillcolor{currentfill}%
\pgfsetlinewidth{1.003750pt}%
\definecolor{currentstroke}{rgb}{0.500000,0.000000,0.500000}%
\pgfsetstrokecolor{currentstroke}%
\pgfsetdash{}{0pt}%
\pgfpathmoveto{\pgfqpoint{2.785322in}{6.328746in}}%
\pgfpathcurveto{\pgfqpoint{2.791146in}{6.328746in}}{\pgfqpoint{2.796733in}{6.331060in}}{\pgfqpoint{2.800851in}{6.335178in}}%
\pgfpathcurveto{\pgfqpoint{2.804969in}{6.339296in}}{\pgfqpoint{2.807283in}{6.344883in}}{\pgfqpoint{2.807283in}{6.350707in}}%
\pgfpathcurveto{\pgfqpoint{2.807283in}{6.356531in}}{\pgfqpoint{2.804969in}{6.362117in}}{\pgfqpoint{2.800851in}{6.366235in}}%
\pgfpathcurveto{\pgfqpoint{2.796733in}{6.370353in}}{\pgfqpoint{2.791146in}{6.372667in}}{\pgfqpoint{2.785322in}{6.372667in}}%
\pgfpathcurveto{\pgfqpoint{2.779499in}{6.372667in}}{\pgfqpoint{2.773912in}{6.370353in}}{\pgfqpoint{2.769794in}{6.366235in}}%
\pgfpathcurveto{\pgfqpoint{2.765676in}{6.362117in}}{\pgfqpoint{2.763362in}{6.356531in}}{\pgfqpoint{2.763362in}{6.350707in}}%
\pgfpathcurveto{\pgfqpoint{2.763362in}{6.344883in}}{\pgfqpoint{2.765676in}{6.339296in}}{\pgfqpoint{2.769794in}{6.335178in}}%
\pgfpathcurveto{\pgfqpoint{2.773912in}{6.331060in}}{\pgfqpoint{2.779499in}{6.328746in}}{\pgfqpoint{2.785322in}{6.328746in}}%
\pgfpathlineto{\pgfqpoint{2.785322in}{6.328746in}}%
\pgfpathclose%
\pgfusepath{stroke,fill}%
\end{pgfscope}%
\begin{pgfscope}%
\pgfpathrectangle{\pgfqpoint{1.542338in}{0.880000in}}{\pgfqpoint{5.115323in}{6.160000in}}%
\pgfusepath{clip}%
\pgfsetbuttcap%
\pgfsetroundjoin%
\definecolor{currentfill}{rgb}{0.500000,0.000000,0.500000}%
\pgfsetfillcolor{currentfill}%
\pgfsetlinewidth{1.003750pt}%
\definecolor{currentstroke}{rgb}{0.500000,0.000000,0.500000}%
\pgfsetstrokecolor{currentstroke}%
\pgfsetdash{}{0pt}%
\pgfpathmoveto{\pgfqpoint{2.133033in}{2.624077in}}%
\pgfpathcurveto{\pgfqpoint{2.138857in}{2.624077in}}{\pgfqpoint{2.144443in}{2.626391in}}{\pgfqpoint{2.148562in}{2.630509in}}%
\pgfpathcurveto{\pgfqpoint{2.152680in}{2.634628in}}{\pgfqpoint{2.154994in}{2.640214in}}{\pgfqpoint{2.154994in}{2.646038in}}%
\pgfpathcurveto{\pgfqpoint{2.154994in}{2.651862in}}{\pgfqpoint{2.152680in}{2.657448in}}{\pgfqpoint{2.148562in}{2.661566in}}%
\pgfpathcurveto{\pgfqpoint{2.144443in}{2.665684in}}{\pgfqpoint{2.138857in}{2.667998in}}{\pgfqpoint{2.133033in}{2.667998in}}%
\pgfpathcurveto{\pgfqpoint{2.127209in}{2.667998in}}{\pgfqpoint{2.121623in}{2.665684in}}{\pgfqpoint{2.117505in}{2.661566in}}%
\pgfpathcurveto{\pgfqpoint{2.113387in}{2.657448in}}{\pgfqpoint{2.111073in}{2.651862in}}{\pgfqpoint{2.111073in}{2.646038in}}%
\pgfpathcurveto{\pgfqpoint{2.111073in}{2.640214in}}{\pgfqpoint{2.113387in}{2.634628in}}{\pgfqpoint{2.117505in}{2.630509in}}%
\pgfpathcurveto{\pgfqpoint{2.121623in}{2.626391in}}{\pgfqpoint{2.127209in}{2.624077in}}{\pgfqpoint{2.133033in}{2.624077in}}%
\pgfpathlineto{\pgfqpoint{2.133033in}{2.624077in}}%
\pgfpathclose%
\pgfusepath{stroke,fill}%
\end{pgfscope}%
\begin{pgfscope}%
\pgfpathrectangle{\pgfqpoint{1.542338in}{0.880000in}}{\pgfqpoint{5.115323in}{6.160000in}}%
\pgfusepath{clip}%
\pgfsetbuttcap%
\pgfsetroundjoin%
\definecolor{currentfill}{rgb}{0.200000,0.800000,0.200000}%
\pgfsetfillcolor{currentfill}%
\pgfsetlinewidth{1.003750pt}%
\definecolor{currentstroke}{rgb}{0.200000,0.800000,0.200000}%
\pgfsetstrokecolor{currentstroke}%
\pgfsetdash{}{0pt}%
\pgfpathmoveto{\pgfqpoint{5.316136in}{5.026479in}}%
\pgfpathcurveto{\pgfqpoint{5.321960in}{5.026479in}}{\pgfqpoint{5.327546in}{5.028793in}}{\pgfqpoint{5.331665in}{5.032911in}}%
\pgfpathcurveto{\pgfqpoint{5.335783in}{5.037029in}}{\pgfqpoint{5.338097in}{5.042615in}}{\pgfqpoint{5.338097in}{5.048439in}}%
\pgfpathcurveto{\pgfqpoint{5.338097in}{5.054263in}}{\pgfqpoint{5.335783in}{5.059849in}}{\pgfqpoint{5.331665in}{5.063967in}}%
\pgfpathcurveto{\pgfqpoint{5.327546in}{5.068085in}}{\pgfqpoint{5.321960in}{5.070399in}}{\pgfqpoint{5.316136in}{5.070399in}}%
\pgfpathcurveto{\pgfqpoint{5.310312in}{5.070399in}}{\pgfqpoint{5.304726in}{5.068085in}}{\pgfqpoint{5.300608in}{5.063967in}}%
\pgfpathcurveto{\pgfqpoint{5.296490in}{5.059849in}}{\pgfqpoint{5.294176in}{5.054263in}}{\pgfqpoint{5.294176in}{5.048439in}}%
\pgfpathcurveto{\pgfqpoint{5.294176in}{5.042615in}}{\pgfqpoint{5.296490in}{5.037029in}}{\pgfqpoint{5.300608in}{5.032911in}}%
\pgfpathcurveto{\pgfqpoint{5.304726in}{5.028793in}}{\pgfqpoint{5.310312in}{5.026479in}}{\pgfqpoint{5.316136in}{5.026479in}}%
\pgfpathlineto{\pgfqpoint{5.316136in}{5.026479in}}%
\pgfpathclose%
\pgfusepath{stroke,fill}%
\end{pgfscope}%
\begin{pgfscope}%
\pgfpathrectangle{\pgfqpoint{1.542338in}{0.880000in}}{\pgfqpoint{5.115323in}{6.160000in}}%
\pgfusepath{clip}%
\pgfsetbuttcap%
\pgfsetroundjoin%
\definecolor{currentfill}{rgb}{0.500000,0.000000,0.500000}%
\pgfsetfillcolor{currentfill}%
\pgfsetlinewidth{1.003750pt}%
\definecolor{currentstroke}{rgb}{0.500000,0.000000,0.500000}%
\pgfsetstrokecolor{currentstroke}%
\pgfsetdash{}{0pt}%
\pgfpathmoveto{\pgfqpoint{3.649199in}{4.157653in}}%
\pgfpathcurveto{\pgfqpoint{3.655023in}{4.157653in}}{\pgfqpoint{3.660609in}{4.159967in}}{\pgfqpoint{3.664727in}{4.164085in}}%
\pgfpathcurveto{\pgfqpoint{3.668846in}{4.168203in}}{\pgfqpoint{3.671159in}{4.173789in}}{\pgfqpoint{3.671159in}{4.179613in}}%
\pgfpathcurveto{\pgfqpoint{3.671159in}{4.185437in}}{\pgfqpoint{3.668846in}{4.191024in}}{\pgfqpoint{3.664727in}{4.195142in}}%
\pgfpathcurveto{\pgfqpoint{3.660609in}{4.199260in}}{\pgfqpoint{3.655023in}{4.201574in}}{\pgfqpoint{3.649199in}{4.201574in}}%
\pgfpathcurveto{\pgfqpoint{3.643375in}{4.201574in}}{\pgfqpoint{3.637789in}{4.199260in}}{\pgfqpoint{3.633671in}{4.195142in}}%
\pgfpathcurveto{\pgfqpoint{3.629553in}{4.191024in}}{\pgfqpoint{3.627239in}{4.185437in}}{\pgfqpoint{3.627239in}{4.179613in}}%
\pgfpathcurveto{\pgfqpoint{3.627239in}{4.173789in}}{\pgfqpoint{3.629553in}{4.168203in}}{\pgfqpoint{3.633671in}{4.164085in}}%
\pgfpathcurveto{\pgfqpoint{3.637789in}{4.159967in}}{\pgfqpoint{3.643375in}{4.157653in}}{\pgfqpoint{3.649199in}{4.157653in}}%
\pgfpathlineto{\pgfqpoint{3.649199in}{4.157653in}}%
\pgfpathclose%
\pgfusepath{stroke,fill}%
\end{pgfscope}%
\begin{pgfscope}%
\pgfpathrectangle{\pgfqpoint{1.542338in}{0.880000in}}{\pgfqpoint{5.115323in}{6.160000in}}%
\pgfusepath{clip}%
\pgfsetbuttcap%
\pgfsetroundjoin%
\definecolor{currentfill}{rgb}{0.500000,0.000000,0.500000}%
\pgfsetfillcolor{currentfill}%
\pgfsetlinewidth{1.003750pt}%
\definecolor{currentstroke}{rgb}{0.500000,0.000000,0.500000}%
\pgfsetstrokecolor{currentstroke}%
\pgfsetdash{}{0pt}%
\pgfpathmoveto{\pgfqpoint{6.079092in}{5.520421in}}%
\pgfpathcurveto{\pgfqpoint{6.084916in}{5.520421in}}{\pgfqpoint{6.090502in}{5.522735in}}{\pgfqpoint{6.094620in}{5.526853in}}%
\pgfpathcurveto{\pgfqpoint{6.098739in}{5.530971in}}{\pgfqpoint{6.101052in}{5.536558in}}{\pgfqpoint{6.101052in}{5.542381in}}%
\pgfpathcurveto{\pgfqpoint{6.101052in}{5.548205in}}{\pgfqpoint{6.098739in}{5.553792in}}{\pgfqpoint{6.094620in}{5.557910in}}%
\pgfpathcurveto{\pgfqpoint{6.090502in}{5.562028in}}{\pgfqpoint{6.084916in}{5.564342in}}{\pgfqpoint{6.079092in}{5.564342in}}%
\pgfpathcurveto{\pgfqpoint{6.073268in}{5.564342in}}{\pgfqpoint{6.067682in}{5.562028in}}{\pgfqpoint{6.063564in}{5.557910in}}%
\pgfpathcurveto{\pgfqpoint{6.059446in}{5.553792in}}{\pgfqpoint{6.057132in}{5.548205in}}{\pgfqpoint{6.057132in}{5.542381in}}%
\pgfpathcurveto{\pgfqpoint{6.057132in}{5.536558in}}{\pgfqpoint{6.059446in}{5.530971in}}{\pgfqpoint{6.063564in}{5.526853in}}%
\pgfpathcurveto{\pgfqpoint{6.067682in}{5.522735in}}{\pgfqpoint{6.073268in}{5.520421in}}{\pgfqpoint{6.079092in}{5.520421in}}%
\pgfpathlineto{\pgfqpoint{6.079092in}{5.520421in}}%
\pgfpathclose%
\pgfusepath{stroke,fill}%
\end{pgfscope}%
\begin{pgfscope}%
\pgfpathrectangle{\pgfqpoint{1.542338in}{0.880000in}}{\pgfqpoint{5.115323in}{6.160000in}}%
\pgfusepath{clip}%
\pgfsetbuttcap%
\pgfsetroundjoin%
\definecolor{currentfill}{rgb}{0.200000,0.200000,0.800000}%
\pgfsetfillcolor{currentfill}%
\pgfsetlinewidth{1.003750pt}%
\definecolor{currentstroke}{rgb}{0.200000,0.200000,0.800000}%
\pgfsetstrokecolor{currentstroke}%
\pgfsetdash{}{0pt}%
\pgfpathmoveto{\pgfqpoint{3.033532in}{3.881917in}}%
\pgfpathcurveto{\pgfqpoint{3.039356in}{3.881917in}}{\pgfqpoint{3.044942in}{3.884231in}}{\pgfqpoint{3.049060in}{3.888350in}}%
\pgfpathcurveto{\pgfqpoint{3.053178in}{3.892468in}}{\pgfqpoint{3.055492in}{3.898054in}}{\pgfqpoint{3.055492in}{3.903878in}}%
\pgfpathcurveto{\pgfqpoint{3.055492in}{3.909702in}}{\pgfqpoint{3.053178in}{3.915288in}}{\pgfqpoint{3.049060in}{3.919406in}}%
\pgfpathcurveto{\pgfqpoint{3.044942in}{3.923524in}}{\pgfqpoint{3.039356in}{3.925838in}}{\pgfqpoint{3.033532in}{3.925838in}}%
\pgfpathcurveto{\pgfqpoint{3.027708in}{3.925838in}}{\pgfqpoint{3.022122in}{3.923524in}}{\pgfqpoint{3.018003in}{3.919406in}}%
\pgfpathcurveto{\pgfqpoint{3.013885in}{3.915288in}}{\pgfqpoint{3.011571in}{3.909702in}}{\pgfqpoint{3.011571in}{3.903878in}}%
\pgfpathcurveto{\pgfqpoint{3.011571in}{3.898054in}}{\pgfqpoint{3.013885in}{3.892468in}}{\pgfqpoint{3.018003in}{3.888350in}}%
\pgfpathcurveto{\pgfqpoint{3.022122in}{3.884231in}}{\pgfqpoint{3.027708in}{3.881917in}}{\pgfqpoint{3.033532in}{3.881917in}}%
\pgfpathlineto{\pgfqpoint{3.033532in}{3.881917in}}%
\pgfpathclose%
\pgfusepath{stroke,fill}%
\end{pgfscope}%
\begin{pgfscope}%
\pgfpathrectangle{\pgfqpoint{1.542338in}{0.880000in}}{\pgfqpoint{5.115323in}{6.160000in}}%
\pgfusepath{clip}%
\pgfsetbuttcap%
\pgfsetroundjoin%
\definecolor{currentfill}{rgb}{0.500000,0.000000,0.500000}%
\pgfsetfillcolor{currentfill}%
\pgfsetlinewidth{1.003750pt}%
\definecolor{currentstroke}{rgb}{0.500000,0.000000,0.500000}%
\pgfsetstrokecolor{currentstroke}%
\pgfsetdash{}{0pt}%
\pgfpathmoveto{\pgfqpoint{4.002100in}{5.361634in}}%
\pgfpathcurveto{\pgfqpoint{4.007924in}{5.361634in}}{\pgfqpoint{4.013510in}{5.363948in}}{\pgfqpoint{4.017628in}{5.368066in}}%
\pgfpathcurveto{\pgfqpoint{4.021747in}{5.372184in}}{\pgfqpoint{4.024060in}{5.377770in}}{\pgfqpoint{4.024060in}{5.383594in}}%
\pgfpathcurveto{\pgfqpoint{4.024060in}{5.389418in}}{\pgfqpoint{4.021747in}{5.395004in}}{\pgfqpoint{4.017628in}{5.399123in}}%
\pgfpathcurveto{\pgfqpoint{4.013510in}{5.403241in}}{\pgfqpoint{4.007924in}{5.405555in}}{\pgfqpoint{4.002100in}{5.405555in}}%
\pgfpathcurveto{\pgfqpoint{3.996276in}{5.405555in}}{\pgfqpoint{3.990690in}{5.403241in}}{\pgfqpoint{3.986572in}{5.399123in}}%
\pgfpathcurveto{\pgfqpoint{3.982454in}{5.395004in}}{\pgfqpoint{3.980140in}{5.389418in}}{\pgfqpoint{3.980140in}{5.383594in}}%
\pgfpathcurveto{\pgfqpoint{3.980140in}{5.377770in}}{\pgfqpoint{3.982454in}{5.372184in}}{\pgfqpoint{3.986572in}{5.368066in}}%
\pgfpathcurveto{\pgfqpoint{3.990690in}{5.363948in}}{\pgfqpoint{3.996276in}{5.361634in}}{\pgfqpoint{4.002100in}{5.361634in}}%
\pgfpathlineto{\pgfqpoint{4.002100in}{5.361634in}}%
\pgfpathclose%
\pgfusepath{stroke,fill}%
\end{pgfscope}%
\begin{pgfscope}%
\pgfpathrectangle{\pgfqpoint{1.542338in}{0.880000in}}{\pgfqpoint{5.115323in}{6.160000in}}%
\pgfusepath{clip}%
\pgfsetbuttcap%
\pgfsetroundjoin%
\definecolor{currentfill}{rgb}{0.500000,0.000000,0.500000}%
\pgfsetfillcolor{currentfill}%
\pgfsetlinewidth{1.003750pt}%
\definecolor{currentstroke}{rgb}{0.500000,0.000000,0.500000}%
\pgfsetstrokecolor{currentstroke}%
\pgfsetdash{}{0pt}%
\pgfpathmoveto{\pgfqpoint{4.897081in}{6.066066in}}%
\pgfpathcurveto{\pgfqpoint{4.902905in}{6.066066in}}{\pgfqpoint{4.908491in}{6.068380in}}{\pgfqpoint{4.912609in}{6.072498in}}%
\pgfpathcurveto{\pgfqpoint{4.916727in}{6.076616in}}{\pgfqpoint{4.919041in}{6.082202in}}{\pgfqpoint{4.919041in}{6.088026in}}%
\pgfpathcurveto{\pgfqpoint{4.919041in}{6.093850in}}{\pgfqpoint{4.916727in}{6.099436in}}{\pgfqpoint{4.912609in}{6.103555in}}%
\pgfpathcurveto{\pgfqpoint{4.908491in}{6.107673in}}{\pgfqpoint{4.902905in}{6.109987in}}{\pgfqpoint{4.897081in}{6.109987in}}%
\pgfpathcurveto{\pgfqpoint{4.891257in}{6.109987in}}{\pgfqpoint{4.885671in}{6.107673in}}{\pgfqpoint{4.881553in}{6.103555in}}%
\pgfpathcurveto{\pgfqpoint{4.877435in}{6.099436in}}{\pgfqpoint{4.875121in}{6.093850in}}{\pgfqpoint{4.875121in}{6.088026in}}%
\pgfpathcurveto{\pgfqpoint{4.875121in}{6.082202in}}{\pgfqpoint{4.877435in}{6.076616in}}{\pgfqpoint{4.881553in}{6.072498in}}%
\pgfpathcurveto{\pgfqpoint{4.885671in}{6.068380in}}{\pgfqpoint{4.891257in}{6.066066in}}{\pgfqpoint{4.897081in}{6.066066in}}%
\pgfpathlineto{\pgfqpoint{4.897081in}{6.066066in}}%
\pgfpathclose%
\pgfusepath{stroke,fill}%
\end{pgfscope}%
\begin{pgfscope}%
\pgfpathrectangle{\pgfqpoint{1.542338in}{0.880000in}}{\pgfqpoint{5.115323in}{6.160000in}}%
\pgfusepath{clip}%
\pgfsetbuttcap%
\pgfsetroundjoin%
\definecolor{currentfill}{rgb}{0.500000,0.000000,0.500000}%
\pgfsetfillcolor{currentfill}%
\pgfsetlinewidth{1.003750pt}%
\definecolor{currentstroke}{rgb}{0.500000,0.000000,0.500000}%
\pgfsetstrokecolor{currentstroke}%
\pgfsetdash{}{0pt}%
\pgfpathmoveto{\pgfqpoint{6.045751in}{5.112848in}}%
\pgfpathcurveto{\pgfqpoint{6.051575in}{5.112848in}}{\pgfqpoint{6.057161in}{5.115162in}}{\pgfqpoint{6.061279in}{5.119280in}}%
\pgfpathcurveto{\pgfqpoint{6.065397in}{5.123399in}}{\pgfqpoint{6.067711in}{5.128985in}}{\pgfqpoint{6.067711in}{5.134809in}}%
\pgfpathcurveto{\pgfqpoint{6.067711in}{5.140633in}}{\pgfqpoint{6.065397in}{5.146219in}}{\pgfqpoint{6.061279in}{5.150337in}}%
\pgfpathcurveto{\pgfqpoint{6.057161in}{5.154455in}}{\pgfqpoint{6.051575in}{5.156769in}}{\pgfqpoint{6.045751in}{5.156769in}}%
\pgfpathcurveto{\pgfqpoint{6.039927in}{5.156769in}}{\pgfqpoint{6.034341in}{5.154455in}}{\pgfqpoint{6.030222in}{5.150337in}}%
\pgfpathcurveto{\pgfqpoint{6.026104in}{5.146219in}}{\pgfqpoint{6.023790in}{5.140633in}}{\pgfqpoint{6.023790in}{5.134809in}}%
\pgfpathcurveto{\pgfqpoint{6.023790in}{5.128985in}}{\pgfqpoint{6.026104in}{5.123399in}}{\pgfqpoint{6.030222in}{5.119280in}}%
\pgfpathcurveto{\pgfqpoint{6.034341in}{5.115162in}}{\pgfqpoint{6.039927in}{5.112848in}}{\pgfqpoint{6.045751in}{5.112848in}}%
\pgfpathlineto{\pgfqpoint{6.045751in}{5.112848in}}%
\pgfpathclose%
\pgfusepath{stroke,fill}%
\end{pgfscope}%
\begin{pgfscope}%
\pgfpathrectangle{\pgfqpoint{1.542338in}{0.880000in}}{\pgfqpoint{5.115323in}{6.160000in}}%
\pgfusepath{clip}%
\pgfsetbuttcap%
\pgfsetroundjoin%
\definecolor{currentfill}{rgb}{0.500000,0.000000,0.500000}%
\pgfsetfillcolor{currentfill}%
\pgfsetlinewidth{1.003750pt}%
\definecolor{currentstroke}{rgb}{0.500000,0.000000,0.500000}%
\pgfsetstrokecolor{currentstroke}%
\pgfsetdash{}{0pt}%
\pgfpathmoveto{\pgfqpoint{4.346001in}{5.323846in}}%
\pgfpathcurveto{\pgfqpoint{4.351825in}{5.323846in}}{\pgfqpoint{4.357411in}{5.326159in}}{\pgfqpoint{4.361529in}{5.330278in}}%
\pgfpathcurveto{\pgfqpoint{4.365648in}{5.334396in}}{\pgfqpoint{4.367961in}{5.339982in}}{\pgfqpoint{4.367961in}{5.345806in}}%
\pgfpathcurveto{\pgfqpoint{4.367961in}{5.351630in}}{\pgfqpoint{4.365648in}{5.357216in}}{\pgfqpoint{4.361529in}{5.361334in}}%
\pgfpathcurveto{\pgfqpoint{4.357411in}{5.365452in}}{\pgfqpoint{4.351825in}{5.367766in}}{\pgfqpoint{4.346001in}{5.367766in}}%
\pgfpathcurveto{\pgfqpoint{4.340177in}{5.367766in}}{\pgfqpoint{4.334591in}{5.365452in}}{\pgfqpoint{4.330473in}{5.361334in}}%
\pgfpathcurveto{\pgfqpoint{4.326355in}{5.357216in}}{\pgfqpoint{4.324041in}{5.351630in}}{\pgfqpoint{4.324041in}{5.345806in}}%
\pgfpathcurveto{\pgfqpoint{4.324041in}{5.339982in}}{\pgfqpoint{4.326355in}{5.334396in}}{\pgfqpoint{4.330473in}{5.330278in}}%
\pgfpathcurveto{\pgfqpoint{4.334591in}{5.326159in}}{\pgfqpoint{4.340177in}{5.323846in}}{\pgfqpoint{4.346001in}{5.323846in}}%
\pgfpathlineto{\pgfqpoint{4.346001in}{5.323846in}}%
\pgfpathclose%
\pgfusepath{stroke,fill}%
\end{pgfscope}%
\begin{pgfscope}%
\pgfpathrectangle{\pgfqpoint{1.542338in}{0.880000in}}{\pgfqpoint{5.115323in}{6.160000in}}%
\pgfusepath{clip}%
\pgfsetbuttcap%
\pgfsetroundjoin%
\definecolor{currentfill}{rgb}{0.500000,0.000000,0.500000}%
\pgfsetfillcolor{currentfill}%
\pgfsetlinewidth{1.003750pt}%
\definecolor{currentstroke}{rgb}{0.500000,0.000000,0.500000}%
\pgfsetstrokecolor{currentstroke}%
\pgfsetdash{}{0pt}%
\pgfpathmoveto{\pgfqpoint{3.618126in}{1.908781in}}%
\pgfpathcurveto{\pgfqpoint{3.623950in}{1.908781in}}{\pgfqpoint{3.629536in}{1.911095in}}{\pgfqpoint{3.633654in}{1.915213in}}%
\pgfpathcurveto{\pgfqpoint{3.637773in}{1.919331in}}{\pgfqpoint{3.640086in}{1.924917in}}{\pgfqpoint{3.640086in}{1.930741in}}%
\pgfpathcurveto{\pgfqpoint{3.640086in}{1.936565in}}{\pgfqpoint{3.637773in}{1.942151in}}{\pgfqpoint{3.633654in}{1.946269in}}%
\pgfpathcurveto{\pgfqpoint{3.629536in}{1.950388in}}{\pgfqpoint{3.623950in}{1.952701in}}{\pgfqpoint{3.618126in}{1.952701in}}%
\pgfpathcurveto{\pgfqpoint{3.612302in}{1.952701in}}{\pgfqpoint{3.606716in}{1.950388in}}{\pgfqpoint{3.602598in}{1.946269in}}%
\pgfpathcurveto{\pgfqpoint{3.598480in}{1.942151in}}{\pgfqpoint{3.596166in}{1.936565in}}{\pgfqpoint{3.596166in}{1.930741in}}%
\pgfpathcurveto{\pgfqpoint{3.596166in}{1.924917in}}{\pgfqpoint{3.598480in}{1.919331in}}{\pgfqpoint{3.602598in}{1.915213in}}%
\pgfpathcurveto{\pgfqpoint{3.606716in}{1.911095in}}{\pgfqpoint{3.612302in}{1.908781in}}{\pgfqpoint{3.618126in}{1.908781in}}%
\pgfpathlineto{\pgfqpoint{3.618126in}{1.908781in}}%
\pgfpathclose%
\pgfusepath{stroke,fill}%
\end{pgfscope}%
\begin{pgfscope}%
\pgfpathrectangle{\pgfqpoint{1.542338in}{0.880000in}}{\pgfqpoint{5.115323in}{6.160000in}}%
\pgfusepath{clip}%
\pgfsetbuttcap%
\pgfsetroundjoin%
\definecolor{currentfill}{rgb}{0.500000,0.000000,0.500000}%
\pgfsetfillcolor{currentfill}%
\pgfsetlinewidth{1.003750pt}%
\definecolor{currentstroke}{rgb}{0.500000,0.000000,0.500000}%
\pgfsetstrokecolor{currentstroke}%
\pgfsetdash{}{0pt}%
\pgfpathmoveto{\pgfqpoint{5.995411in}{2.572517in}}%
\pgfpathcurveto{\pgfqpoint{6.001235in}{2.572517in}}{\pgfqpoint{6.006821in}{2.574831in}}{\pgfqpoint{6.010939in}{2.578949in}}%
\pgfpathcurveto{\pgfqpoint{6.015057in}{2.583067in}}{\pgfqpoint{6.017371in}{2.588654in}}{\pgfqpoint{6.017371in}{2.594478in}}%
\pgfpathcurveto{\pgfqpoint{6.017371in}{2.600302in}}{\pgfqpoint{6.015057in}{2.605888in}}{\pgfqpoint{6.010939in}{2.610006in}}%
\pgfpathcurveto{\pgfqpoint{6.006821in}{2.614124in}}{\pgfqpoint{6.001235in}{2.616438in}}{\pgfqpoint{5.995411in}{2.616438in}}%
\pgfpathcurveto{\pgfqpoint{5.989587in}{2.616438in}}{\pgfqpoint{5.984001in}{2.614124in}}{\pgfqpoint{5.979882in}{2.610006in}}%
\pgfpathcurveto{\pgfqpoint{5.975764in}{2.605888in}}{\pgfqpoint{5.973450in}{2.600302in}}{\pgfqpoint{5.973450in}{2.594478in}}%
\pgfpathcurveto{\pgfqpoint{5.973450in}{2.588654in}}{\pgfqpoint{5.975764in}{2.583067in}}{\pgfqpoint{5.979882in}{2.578949in}}%
\pgfpathcurveto{\pgfqpoint{5.984001in}{2.574831in}}{\pgfqpoint{5.989587in}{2.572517in}}{\pgfqpoint{5.995411in}{2.572517in}}%
\pgfpathlineto{\pgfqpoint{5.995411in}{2.572517in}}%
\pgfpathclose%
\pgfusepath{stroke,fill}%
\end{pgfscope}%
\begin{pgfscope}%
\pgfpathrectangle{\pgfqpoint{1.542338in}{0.880000in}}{\pgfqpoint{5.115323in}{6.160000in}}%
\pgfusepath{clip}%
\pgfsetbuttcap%
\pgfsetroundjoin%
\definecolor{currentfill}{rgb}{0.500000,0.000000,0.500000}%
\pgfsetfillcolor{currentfill}%
\pgfsetlinewidth{1.003750pt}%
\definecolor{currentstroke}{rgb}{0.500000,0.000000,0.500000}%
\pgfsetstrokecolor{currentstroke}%
\pgfsetdash{}{0pt}%
\pgfpathmoveto{\pgfqpoint{3.313297in}{6.278677in}}%
\pgfpathcurveto{\pgfqpoint{3.319121in}{6.278677in}}{\pgfqpoint{3.324707in}{6.280991in}}{\pgfqpoint{3.328825in}{6.285109in}}%
\pgfpathcurveto{\pgfqpoint{3.332943in}{6.289227in}}{\pgfqpoint{3.335257in}{6.294813in}}{\pgfqpoint{3.335257in}{6.300637in}}%
\pgfpathcurveto{\pgfqpoint{3.335257in}{6.306461in}}{\pgfqpoint{3.332943in}{6.312048in}}{\pgfqpoint{3.328825in}{6.316166in}}%
\pgfpathcurveto{\pgfqpoint{3.324707in}{6.320284in}}{\pgfqpoint{3.319121in}{6.322598in}}{\pgfqpoint{3.313297in}{6.322598in}}%
\pgfpathcurveto{\pgfqpoint{3.307473in}{6.322598in}}{\pgfqpoint{3.301887in}{6.320284in}}{\pgfqpoint{3.297769in}{6.316166in}}%
\pgfpathcurveto{\pgfqpoint{3.293651in}{6.312048in}}{\pgfqpoint{3.291337in}{6.306461in}}{\pgfqpoint{3.291337in}{6.300637in}}%
\pgfpathcurveto{\pgfqpoint{3.291337in}{6.294813in}}{\pgfqpoint{3.293651in}{6.289227in}}{\pgfqpoint{3.297769in}{6.285109in}}%
\pgfpathcurveto{\pgfqpoint{3.301887in}{6.280991in}}{\pgfqpoint{3.307473in}{6.278677in}}{\pgfqpoint{3.313297in}{6.278677in}}%
\pgfpathlineto{\pgfqpoint{3.313297in}{6.278677in}}%
\pgfpathclose%
\pgfusepath{stroke,fill}%
\end{pgfscope}%
\begin{pgfscope}%
\pgfpathrectangle{\pgfqpoint{1.542338in}{0.880000in}}{\pgfqpoint{5.115323in}{6.160000in}}%
\pgfusepath{clip}%
\pgfsetbuttcap%
\pgfsetroundjoin%
\definecolor{currentfill}{rgb}{0.500000,0.000000,0.500000}%
\pgfsetfillcolor{currentfill}%
\pgfsetlinewidth{1.003750pt}%
\definecolor{currentstroke}{rgb}{0.500000,0.000000,0.500000}%
\pgfsetstrokecolor{currentstroke}%
\pgfsetdash{}{0pt}%
\pgfpathmoveto{\pgfqpoint{2.201358in}{4.924458in}}%
\pgfpathcurveto{\pgfqpoint{2.207182in}{4.924458in}}{\pgfqpoint{2.212768in}{4.926771in}}{\pgfqpoint{2.216886in}{4.930890in}}%
\pgfpathcurveto{\pgfqpoint{2.221004in}{4.935008in}}{\pgfqpoint{2.223318in}{4.940594in}}{\pgfqpoint{2.223318in}{4.946418in}}%
\pgfpathcurveto{\pgfqpoint{2.223318in}{4.952242in}}{\pgfqpoint{2.221004in}{4.957828in}}{\pgfqpoint{2.216886in}{4.961946in}}%
\pgfpathcurveto{\pgfqpoint{2.212768in}{4.966064in}}{\pgfqpoint{2.207182in}{4.968378in}}{\pgfqpoint{2.201358in}{4.968378in}}%
\pgfpathcurveto{\pgfqpoint{2.195534in}{4.968378in}}{\pgfqpoint{2.189948in}{4.966064in}}{\pgfqpoint{2.185830in}{4.961946in}}%
\pgfpathcurveto{\pgfqpoint{2.181712in}{4.957828in}}{\pgfqpoint{2.179398in}{4.952242in}}{\pgfqpoint{2.179398in}{4.946418in}}%
\pgfpathcurveto{\pgfqpoint{2.179398in}{4.940594in}}{\pgfqpoint{2.181712in}{4.935008in}}{\pgfqpoint{2.185830in}{4.930890in}}%
\pgfpathcurveto{\pgfqpoint{2.189948in}{4.926771in}}{\pgfqpoint{2.195534in}{4.924458in}}{\pgfqpoint{2.201358in}{4.924458in}}%
\pgfpathlineto{\pgfqpoint{2.201358in}{4.924458in}}%
\pgfpathclose%
\pgfusepath{stroke,fill}%
\end{pgfscope}%
\begin{pgfscope}%
\pgfpathrectangle{\pgfqpoint{1.542338in}{0.880000in}}{\pgfqpoint{5.115323in}{6.160000in}}%
\pgfusepath{clip}%
\pgfsetbuttcap%
\pgfsetroundjoin%
\definecolor{currentfill}{rgb}{0.500000,0.000000,0.500000}%
\pgfsetfillcolor{currentfill}%
\pgfsetlinewidth{1.003750pt}%
\definecolor{currentstroke}{rgb}{0.500000,0.000000,0.500000}%
\pgfsetstrokecolor{currentstroke}%
\pgfsetdash{}{0pt}%
\pgfpathmoveto{\pgfqpoint{3.962646in}{4.289838in}}%
\pgfpathcurveto{\pgfqpoint{3.968470in}{4.289838in}}{\pgfqpoint{3.974056in}{4.292152in}}{\pgfqpoint{3.978175in}{4.296270in}}%
\pgfpathcurveto{\pgfqpoint{3.982293in}{4.300388in}}{\pgfqpoint{3.984607in}{4.305974in}}{\pgfqpoint{3.984607in}{4.311798in}}%
\pgfpathcurveto{\pgfqpoint{3.984607in}{4.317622in}}{\pgfqpoint{3.982293in}{4.323208in}}{\pgfqpoint{3.978175in}{4.327327in}}%
\pgfpathcurveto{\pgfqpoint{3.974056in}{4.331445in}}{\pgfqpoint{3.968470in}{4.333759in}}{\pgfqpoint{3.962646in}{4.333759in}}%
\pgfpathcurveto{\pgfqpoint{3.956822in}{4.333759in}}{\pgfqpoint{3.951236in}{4.331445in}}{\pgfqpoint{3.947118in}{4.327327in}}%
\pgfpathcurveto{\pgfqpoint{3.943000in}{4.323208in}}{\pgfqpoint{3.940686in}{4.317622in}}{\pgfqpoint{3.940686in}{4.311798in}}%
\pgfpathcurveto{\pgfqpoint{3.940686in}{4.305974in}}{\pgfqpoint{3.943000in}{4.300388in}}{\pgfqpoint{3.947118in}{4.296270in}}%
\pgfpathcurveto{\pgfqpoint{3.951236in}{4.292152in}}{\pgfqpoint{3.956822in}{4.289838in}}{\pgfqpoint{3.962646in}{4.289838in}}%
\pgfpathlineto{\pgfqpoint{3.962646in}{4.289838in}}%
\pgfpathclose%
\pgfusepath{stroke,fill}%
\end{pgfscope}%
\begin{pgfscope}%
\pgfpathrectangle{\pgfqpoint{1.542338in}{0.880000in}}{\pgfqpoint{5.115323in}{6.160000in}}%
\pgfusepath{clip}%
\pgfsetbuttcap%
\pgfsetroundjoin%
\definecolor{currentfill}{rgb}{0.800000,0.200000,0.200000}%
\pgfsetfillcolor{currentfill}%
\pgfsetlinewidth{1.003750pt}%
\definecolor{currentstroke}{rgb}{0.800000,0.200000,0.200000}%
\pgfsetstrokecolor{currentstroke}%
\pgfsetdash{}{0pt}%
\pgfpathmoveto{\pgfqpoint{5.600770in}{3.353008in}}%
\pgfpathcurveto{\pgfqpoint{5.606594in}{3.353008in}}{\pgfqpoint{5.612180in}{3.355322in}}{\pgfqpoint{5.616298in}{3.359440in}}%
\pgfpathcurveto{\pgfqpoint{5.620417in}{3.363558in}}{\pgfqpoint{5.622730in}{3.369144in}}{\pgfqpoint{5.622730in}{3.374968in}}%
\pgfpathcurveto{\pgfqpoint{5.622730in}{3.380792in}}{\pgfqpoint{5.620417in}{3.386378in}}{\pgfqpoint{5.616298in}{3.390496in}}%
\pgfpathcurveto{\pgfqpoint{5.612180in}{3.394614in}}{\pgfqpoint{5.606594in}{3.396928in}}{\pgfqpoint{5.600770in}{3.396928in}}%
\pgfpathcurveto{\pgfqpoint{5.594946in}{3.396928in}}{\pgfqpoint{5.589360in}{3.394614in}}{\pgfqpoint{5.585242in}{3.390496in}}%
\pgfpathcurveto{\pgfqpoint{5.581124in}{3.386378in}}{\pgfqpoint{5.578810in}{3.380792in}}{\pgfqpoint{5.578810in}{3.374968in}}%
\pgfpathcurveto{\pgfqpoint{5.578810in}{3.369144in}}{\pgfqpoint{5.581124in}{3.363558in}}{\pgfqpoint{5.585242in}{3.359440in}}%
\pgfpathcurveto{\pgfqpoint{5.589360in}{3.355322in}}{\pgfqpoint{5.594946in}{3.353008in}}{\pgfqpoint{5.600770in}{3.353008in}}%
\pgfpathlineto{\pgfqpoint{5.600770in}{3.353008in}}%
\pgfpathclose%
\pgfusepath{stroke,fill}%
\end{pgfscope}%
\begin{pgfscope}%
\pgfpathrectangle{\pgfqpoint{1.542338in}{0.880000in}}{\pgfqpoint{5.115323in}{6.160000in}}%
\pgfusepath{clip}%
\pgfsetbuttcap%
\pgfsetroundjoin%
\definecolor{currentfill}{rgb}{0.200000,0.200000,0.800000}%
\pgfsetfillcolor{currentfill}%
\pgfsetlinewidth{1.003750pt}%
\definecolor{currentstroke}{rgb}{0.200000,0.200000,0.800000}%
\pgfsetstrokecolor{currentstroke}%
\pgfsetdash{}{0pt}%
\pgfpathmoveto{\pgfqpoint{1.967031in}{1.799012in}}%
\pgfpathcurveto{\pgfqpoint{1.972855in}{1.799012in}}{\pgfqpoint{1.978442in}{1.801326in}}{\pgfqpoint{1.982560in}{1.805444in}}%
\pgfpathcurveto{\pgfqpoint{1.986678in}{1.809562in}}{\pgfqpoint{1.988992in}{1.815149in}}{\pgfqpoint{1.988992in}{1.820972in}}%
\pgfpathcurveto{\pgfqpoint{1.988992in}{1.826796in}}{\pgfqpoint{1.986678in}{1.832383in}}{\pgfqpoint{1.982560in}{1.836501in}}%
\pgfpathcurveto{\pgfqpoint{1.978442in}{1.840619in}}{\pgfqpoint{1.972855in}{1.842933in}}{\pgfqpoint{1.967031in}{1.842933in}}%
\pgfpathcurveto{\pgfqpoint{1.961208in}{1.842933in}}{\pgfqpoint{1.955621in}{1.840619in}}{\pgfqpoint{1.951503in}{1.836501in}}%
\pgfpathcurveto{\pgfqpoint{1.947385in}{1.832383in}}{\pgfqpoint{1.945071in}{1.826796in}}{\pgfqpoint{1.945071in}{1.820972in}}%
\pgfpathcurveto{\pgfqpoint{1.945071in}{1.815149in}}{\pgfqpoint{1.947385in}{1.809562in}}{\pgfqpoint{1.951503in}{1.805444in}}%
\pgfpathcurveto{\pgfqpoint{1.955621in}{1.801326in}}{\pgfqpoint{1.961208in}{1.799012in}}{\pgfqpoint{1.967031in}{1.799012in}}%
\pgfpathlineto{\pgfqpoint{1.967031in}{1.799012in}}%
\pgfpathclose%
\pgfusepath{stroke,fill}%
\end{pgfscope}%
\begin{pgfscope}%
\pgfpathrectangle{\pgfqpoint{1.542338in}{0.880000in}}{\pgfqpoint{5.115323in}{6.160000in}}%
\pgfusepath{clip}%
\pgfsetbuttcap%
\pgfsetroundjoin%
\definecolor{currentfill}{rgb}{0.500000,0.000000,0.500000}%
\pgfsetfillcolor{currentfill}%
\pgfsetlinewidth{1.003750pt}%
\definecolor{currentstroke}{rgb}{0.500000,0.000000,0.500000}%
\pgfsetstrokecolor{currentstroke}%
\pgfsetdash{}{0pt}%
\pgfpathmoveto{\pgfqpoint{4.641114in}{3.363411in}}%
\pgfpathcurveto{\pgfqpoint{4.646937in}{3.363411in}}{\pgfqpoint{4.652524in}{3.365725in}}{\pgfqpoint{4.656642in}{3.369843in}}%
\pgfpathcurveto{\pgfqpoint{4.660760in}{3.373961in}}{\pgfqpoint{4.663074in}{3.379547in}}{\pgfqpoint{4.663074in}{3.385371in}}%
\pgfpathcurveto{\pgfqpoint{4.663074in}{3.391195in}}{\pgfqpoint{4.660760in}{3.396781in}}{\pgfqpoint{4.656642in}{3.400899in}}%
\pgfpathcurveto{\pgfqpoint{4.652524in}{3.405017in}}{\pgfqpoint{4.646937in}{3.407331in}}{\pgfqpoint{4.641114in}{3.407331in}}%
\pgfpathcurveto{\pgfqpoint{4.635290in}{3.407331in}}{\pgfqpoint{4.629703in}{3.405017in}}{\pgfqpoint{4.625585in}{3.400899in}}%
\pgfpathcurveto{\pgfqpoint{4.621467in}{3.396781in}}{\pgfqpoint{4.619153in}{3.391195in}}{\pgfqpoint{4.619153in}{3.385371in}}%
\pgfpathcurveto{\pgfqpoint{4.619153in}{3.379547in}}{\pgfqpoint{4.621467in}{3.373961in}}{\pgfqpoint{4.625585in}{3.369843in}}%
\pgfpathcurveto{\pgfqpoint{4.629703in}{3.365725in}}{\pgfqpoint{4.635290in}{3.363411in}}{\pgfqpoint{4.641114in}{3.363411in}}%
\pgfpathlineto{\pgfqpoint{4.641114in}{3.363411in}}%
\pgfpathclose%
\pgfusepath{stroke,fill}%
\end{pgfscope}%
\begin{pgfscope}%
\pgfpathrectangle{\pgfqpoint{1.542338in}{0.880000in}}{\pgfqpoint{5.115323in}{6.160000in}}%
\pgfusepath{clip}%
\pgfsetbuttcap%
\pgfsetroundjoin%
\definecolor{currentfill}{rgb}{0.500000,0.000000,0.500000}%
\pgfsetfillcolor{currentfill}%
\pgfsetlinewidth{1.003750pt}%
\definecolor{currentstroke}{rgb}{0.500000,0.000000,0.500000}%
\pgfsetstrokecolor{currentstroke}%
\pgfsetdash{}{0pt}%
\pgfpathmoveto{\pgfqpoint{4.682455in}{3.988584in}}%
\pgfpathcurveto{\pgfqpoint{4.688278in}{3.988584in}}{\pgfqpoint{4.693865in}{3.990897in}}{\pgfqpoint{4.697983in}{3.995016in}}%
\pgfpathcurveto{\pgfqpoint{4.702101in}{3.999134in}}{\pgfqpoint{4.704415in}{4.004720in}}{\pgfqpoint{4.704415in}{4.010544in}}%
\pgfpathcurveto{\pgfqpoint{4.704415in}{4.016368in}}{\pgfqpoint{4.702101in}{4.021954in}}{\pgfqpoint{4.697983in}{4.026072in}}%
\pgfpathcurveto{\pgfqpoint{4.693865in}{4.030190in}}{\pgfqpoint{4.688278in}{4.032504in}}{\pgfqpoint{4.682455in}{4.032504in}}%
\pgfpathcurveto{\pgfqpoint{4.676631in}{4.032504in}}{\pgfqpoint{4.671044in}{4.030190in}}{\pgfqpoint{4.666926in}{4.026072in}}%
\pgfpathcurveto{\pgfqpoint{4.662808in}{4.021954in}}{\pgfqpoint{4.660494in}{4.016368in}}{\pgfqpoint{4.660494in}{4.010544in}}%
\pgfpathcurveto{\pgfqpoint{4.660494in}{4.004720in}}{\pgfqpoint{4.662808in}{3.999134in}}{\pgfqpoint{4.666926in}{3.995016in}}%
\pgfpathcurveto{\pgfqpoint{4.671044in}{3.990897in}}{\pgfqpoint{4.676631in}{3.988584in}}{\pgfqpoint{4.682455in}{3.988584in}}%
\pgfpathlineto{\pgfqpoint{4.682455in}{3.988584in}}%
\pgfpathclose%
\pgfusepath{stroke,fill}%
\end{pgfscope}%
\begin{pgfscope}%
\pgfpathrectangle{\pgfqpoint{1.542338in}{0.880000in}}{\pgfqpoint{5.115323in}{6.160000in}}%
\pgfusepath{clip}%
\pgfsetbuttcap%
\pgfsetroundjoin%
\definecolor{currentfill}{rgb}{0.800000,0.200000,0.200000}%
\pgfsetfillcolor{currentfill}%
\pgfsetlinewidth{1.003750pt}%
\definecolor{currentstroke}{rgb}{0.800000,0.200000,0.200000}%
\pgfsetstrokecolor{currentstroke}%
\pgfsetdash{}{0pt}%
\pgfpathmoveto{\pgfqpoint{6.291802in}{2.603839in}}%
\pgfpathcurveto{\pgfqpoint{6.297626in}{2.603839in}}{\pgfqpoint{6.303212in}{2.606152in}}{\pgfqpoint{6.307331in}{2.610271in}}%
\pgfpathcurveto{\pgfqpoint{6.311449in}{2.614389in}}{\pgfqpoint{6.313763in}{2.619975in}}{\pgfqpoint{6.313763in}{2.625799in}}%
\pgfpathcurveto{\pgfqpoint{6.313763in}{2.631623in}}{\pgfqpoint{6.311449in}{2.637209in}}{\pgfqpoint{6.307331in}{2.641327in}}%
\pgfpathcurveto{\pgfqpoint{6.303212in}{2.645445in}}{\pgfqpoint{6.297626in}{2.647759in}}{\pgfqpoint{6.291802in}{2.647759in}}%
\pgfpathcurveto{\pgfqpoint{6.285978in}{2.647759in}}{\pgfqpoint{6.280392in}{2.645445in}}{\pgfqpoint{6.276274in}{2.641327in}}%
\pgfpathcurveto{\pgfqpoint{6.272156in}{2.637209in}}{\pgfqpoint{6.269842in}{2.631623in}}{\pgfqpoint{6.269842in}{2.625799in}}%
\pgfpathcurveto{\pgfqpoint{6.269842in}{2.619975in}}{\pgfqpoint{6.272156in}{2.614389in}}{\pgfqpoint{6.276274in}{2.610271in}}%
\pgfpathcurveto{\pgfqpoint{6.280392in}{2.606152in}}{\pgfqpoint{6.285978in}{2.603839in}}{\pgfqpoint{6.291802in}{2.603839in}}%
\pgfpathlineto{\pgfqpoint{6.291802in}{2.603839in}}%
\pgfpathclose%
\pgfusepath{stroke,fill}%
\end{pgfscope}%
\begin{pgfscope}%
\pgfpathrectangle{\pgfqpoint{1.542338in}{0.880000in}}{\pgfqpoint{5.115323in}{6.160000in}}%
\pgfusepath{clip}%
\pgfsetbuttcap%
\pgfsetroundjoin%
\definecolor{currentfill}{rgb}{0.500000,0.000000,0.500000}%
\pgfsetfillcolor{currentfill}%
\pgfsetlinewidth{1.003750pt}%
\definecolor{currentstroke}{rgb}{0.500000,0.000000,0.500000}%
\pgfsetstrokecolor{currentstroke}%
\pgfsetdash{}{0pt}%
\pgfpathmoveto{\pgfqpoint{3.198734in}{2.419227in}}%
\pgfpathcurveto{\pgfqpoint{3.204558in}{2.419227in}}{\pgfqpoint{3.210144in}{2.421541in}}{\pgfqpoint{3.214263in}{2.425659in}}%
\pgfpathcurveto{\pgfqpoint{3.218381in}{2.429777in}}{\pgfqpoint{3.220695in}{2.435363in}}{\pgfqpoint{3.220695in}{2.441187in}}%
\pgfpathcurveto{\pgfqpoint{3.220695in}{2.447011in}}{\pgfqpoint{3.218381in}{2.452597in}}{\pgfqpoint{3.214263in}{2.456716in}}%
\pgfpathcurveto{\pgfqpoint{3.210144in}{2.460834in}}{\pgfqpoint{3.204558in}{2.463148in}}{\pgfqpoint{3.198734in}{2.463148in}}%
\pgfpathcurveto{\pgfqpoint{3.192910in}{2.463148in}}{\pgfqpoint{3.187324in}{2.460834in}}{\pgfqpoint{3.183206in}{2.456716in}}%
\pgfpathcurveto{\pgfqpoint{3.179088in}{2.452597in}}{\pgfqpoint{3.176774in}{2.447011in}}{\pgfqpoint{3.176774in}{2.441187in}}%
\pgfpathcurveto{\pgfqpoint{3.176774in}{2.435363in}}{\pgfqpoint{3.179088in}{2.429777in}}{\pgfqpoint{3.183206in}{2.425659in}}%
\pgfpathcurveto{\pgfqpoint{3.187324in}{2.421541in}}{\pgfqpoint{3.192910in}{2.419227in}}{\pgfqpoint{3.198734in}{2.419227in}}%
\pgfpathlineto{\pgfqpoint{3.198734in}{2.419227in}}%
\pgfpathclose%
\pgfusepath{stroke,fill}%
\end{pgfscope}%
\begin{pgfscope}%
\pgfpathrectangle{\pgfqpoint{1.542338in}{0.880000in}}{\pgfqpoint{5.115323in}{6.160000in}}%
\pgfusepath{clip}%
\pgfsetbuttcap%
\pgfsetroundjoin%
\definecolor{currentfill}{rgb}{0.500000,0.000000,0.500000}%
\pgfsetfillcolor{currentfill}%
\pgfsetlinewidth{1.003750pt}%
\definecolor{currentstroke}{rgb}{0.500000,0.000000,0.500000}%
\pgfsetstrokecolor{currentstroke}%
\pgfsetdash{}{0pt}%
\pgfpathmoveto{\pgfqpoint{3.161987in}{4.034365in}}%
\pgfpathcurveto{\pgfqpoint{3.167811in}{4.034365in}}{\pgfqpoint{3.173397in}{4.036679in}}{\pgfqpoint{3.177515in}{4.040797in}}%
\pgfpathcurveto{\pgfqpoint{3.181633in}{4.044916in}}{\pgfqpoint{3.183947in}{4.050502in}}{\pgfqpoint{3.183947in}{4.056326in}}%
\pgfpathcurveto{\pgfqpoint{3.183947in}{4.062150in}}{\pgfqpoint{3.181633in}{4.067736in}}{\pgfqpoint{3.177515in}{4.071854in}}%
\pgfpathcurveto{\pgfqpoint{3.173397in}{4.075972in}}{\pgfqpoint{3.167811in}{4.078286in}}{\pgfqpoint{3.161987in}{4.078286in}}%
\pgfpathcurveto{\pgfqpoint{3.156163in}{4.078286in}}{\pgfqpoint{3.150577in}{4.075972in}}{\pgfqpoint{3.146459in}{4.071854in}}%
\pgfpathcurveto{\pgfqpoint{3.142341in}{4.067736in}}{\pgfqpoint{3.140027in}{4.062150in}}{\pgfqpoint{3.140027in}{4.056326in}}%
\pgfpathcurveto{\pgfqpoint{3.140027in}{4.050502in}}{\pgfqpoint{3.142341in}{4.044916in}}{\pgfqpoint{3.146459in}{4.040797in}}%
\pgfpathcurveto{\pgfqpoint{3.150577in}{4.036679in}}{\pgfqpoint{3.156163in}{4.034365in}}{\pgfqpoint{3.161987in}{4.034365in}}%
\pgfpathlineto{\pgfqpoint{3.161987in}{4.034365in}}%
\pgfpathclose%
\pgfusepath{stroke,fill}%
\end{pgfscope}%
\begin{pgfscope}%
\pgfpathrectangle{\pgfqpoint{1.542338in}{0.880000in}}{\pgfqpoint{5.115323in}{6.160000in}}%
\pgfusepath{clip}%
\pgfsetbuttcap%
\pgfsetroundjoin%
\definecolor{currentfill}{rgb}{0.500000,0.000000,0.500000}%
\pgfsetfillcolor{currentfill}%
\pgfsetlinewidth{1.003750pt}%
\definecolor{currentstroke}{rgb}{0.500000,0.000000,0.500000}%
\pgfsetstrokecolor{currentstroke}%
\pgfsetdash{}{0pt}%
\pgfpathmoveto{\pgfqpoint{3.940687in}{4.407646in}}%
\pgfpathcurveto{\pgfqpoint{3.946511in}{4.407646in}}{\pgfqpoint{3.952097in}{4.409960in}}{\pgfqpoint{3.956215in}{4.414078in}}%
\pgfpathcurveto{\pgfqpoint{3.960333in}{4.418196in}}{\pgfqpoint{3.962647in}{4.423782in}}{\pgfqpoint{3.962647in}{4.429606in}}%
\pgfpathcurveto{\pgfqpoint{3.962647in}{4.435430in}}{\pgfqpoint{3.960333in}{4.441016in}}{\pgfqpoint{3.956215in}{4.445134in}}%
\pgfpathcurveto{\pgfqpoint{3.952097in}{4.449252in}}{\pgfqpoint{3.946511in}{4.451566in}}{\pgfqpoint{3.940687in}{4.451566in}}%
\pgfpathcurveto{\pgfqpoint{3.934863in}{4.451566in}}{\pgfqpoint{3.929277in}{4.449252in}}{\pgfqpoint{3.925159in}{4.445134in}}%
\pgfpathcurveto{\pgfqpoint{3.921041in}{4.441016in}}{\pgfqpoint{3.918727in}{4.435430in}}{\pgfqpoint{3.918727in}{4.429606in}}%
\pgfpathcurveto{\pgfqpoint{3.918727in}{4.423782in}}{\pgfqpoint{3.921041in}{4.418196in}}{\pgfqpoint{3.925159in}{4.414078in}}%
\pgfpathcurveto{\pgfqpoint{3.929277in}{4.409960in}}{\pgfqpoint{3.934863in}{4.407646in}}{\pgfqpoint{3.940687in}{4.407646in}}%
\pgfpathlineto{\pgfqpoint{3.940687in}{4.407646in}}%
\pgfpathclose%
\pgfusepath{stroke,fill}%
\end{pgfscope}%
\begin{pgfscope}%
\pgfpathrectangle{\pgfqpoint{1.542338in}{0.880000in}}{\pgfqpoint{5.115323in}{6.160000in}}%
\pgfusepath{clip}%
\pgfsetbuttcap%
\pgfsetroundjoin%
\definecolor{currentfill}{rgb}{0.500000,0.000000,0.500000}%
\pgfsetfillcolor{currentfill}%
\pgfsetlinewidth{1.003750pt}%
\definecolor{currentstroke}{rgb}{0.500000,0.000000,0.500000}%
\pgfsetstrokecolor{currentstroke}%
\pgfsetdash{}{0pt}%
\pgfpathmoveto{\pgfqpoint{3.538112in}{2.740931in}}%
\pgfpathcurveto{\pgfqpoint{3.543936in}{2.740931in}}{\pgfqpoint{3.549523in}{2.743245in}}{\pgfqpoint{3.553641in}{2.747363in}}%
\pgfpathcurveto{\pgfqpoint{3.557759in}{2.751481in}}{\pgfqpoint{3.560073in}{2.757067in}}{\pgfqpoint{3.560073in}{2.762891in}}%
\pgfpathcurveto{\pgfqpoint{3.560073in}{2.768715in}}{\pgfqpoint{3.557759in}{2.774301in}}{\pgfqpoint{3.553641in}{2.778419in}}%
\pgfpathcurveto{\pgfqpoint{3.549523in}{2.782538in}}{\pgfqpoint{3.543936in}{2.784851in}}{\pgfqpoint{3.538112in}{2.784851in}}%
\pgfpathcurveto{\pgfqpoint{3.532289in}{2.784851in}}{\pgfqpoint{3.526702in}{2.782538in}}{\pgfqpoint{3.522584in}{2.778419in}}%
\pgfpathcurveto{\pgfqpoint{3.518466in}{2.774301in}}{\pgfqpoint{3.516152in}{2.768715in}}{\pgfqpoint{3.516152in}{2.762891in}}%
\pgfpathcurveto{\pgfqpoint{3.516152in}{2.757067in}}{\pgfqpoint{3.518466in}{2.751481in}}{\pgfqpoint{3.522584in}{2.747363in}}%
\pgfpathcurveto{\pgfqpoint{3.526702in}{2.743245in}}{\pgfqpoint{3.532289in}{2.740931in}}{\pgfqpoint{3.538112in}{2.740931in}}%
\pgfpathlineto{\pgfqpoint{3.538112in}{2.740931in}}%
\pgfpathclose%
\pgfusepath{stroke,fill}%
\end{pgfscope}%
\begin{pgfscope}%
\pgfpathrectangle{\pgfqpoint{1.542338in}{0.880000in}}{\pgfqpoint{5.115323in}{6.160000in}}%
\pgfusepath{clip}%
\pgfsetbuttcap%
\pgfsetroundjoin%
\definecolor{currentfill}{rgb}{0.500000,0.000000,0.500000}%
\pgfsetfillcolor{currentfill}%
\pgfsetlinewidth{1.003750pt}%
\definecolor{currentstroke}{rgb}{0.500000,0.000000,0.500000}%
\pgfsetstrokecolor{currentstroke}%
\pgfsetdash{}{0pt}%
\pgfpathmoveto{\pgfqpoint{5.365940in}{4.414801in}}%
\pgfpathcurveto{\pgfqpoint{5.371764in}{4.414801in}}{\pgfqpoint{5.377350in}{4.417114in}}{\pgfqpoint{5.381468in}{4.421233in}}%
\pgfpathcurveto{\pgfqpoint{5.385586in}{4.425351in}}{\pgfqpoint{5.387900in}{4.430937in}}{\pgfqpoint{5.387900in}{4.436761in}}%
\pgfpathcurveto{\pgfqpoint{5.387900in}{4.442585in}}{\pgfqpoint{5.385586in}{4.448171in}}{\pgfqpoint{5.381468in}{4.452289in}}%
\pgfpathcurveto{\pgfqpoint{5.377350in}{4.456407in}}{\pgfqpoint{5.371764in}{4.458721in}}{\pgfqpoint{5.365940in}{4.458721in}}%
\pgfpathcurveto{\pgfqpoint{5.360116in}{4.458721in}}{\pgfqpoint{5.354530in}{4.456407in}}{\pgfqpoint{5.350412in}{4.452289in}}%
\pgfpathcurveto{\pgfqpoint{5.346293in}{4.448171in}}{\pgfqpoint{5.343980in}{4.442585in}}{\pgfqpoint{5.343980in}{4.436761in}}%
\pgfpathcurveto{\pgfqpoint{5.343980in}{4.430937in}}{\pgfqpoint{5.346293in}{4.425351in}}{\pgfqpoint{5.350412in}{4.421233in}}%
\pgfpathcurveto{\pgfqpoint{5.354530in}{4.417114in}}{\pgfqpoint{5.360116in}{4.414801in}}{\pgfqpoint{5.365940in}{4.414801in}}%
\pgfpathlineto{\pgfqpoint{5.365940in}{4.414801in}}%
\pgfpathclose%
\pgfusepath{stroke,fill}%
\end{pgfscope}%
\begin{pgfscope}%
\pgfpathrectangle{\pgfqpoint{1.542338in}{0.880000in}}{\pgfqpoint{5.115323in}{6.160000in}}%
\pgfusepath{clip}%
\pgfsetbuttcap%
\pgfsetroundjoin%
\definecolor{currentfill}{rgb}{0.500000,0.000000,0.500000}%
\pgfsetfillcolor{currentfill}%
\pgfsetlinewidth{1.003750pt}%
\definecolor{currentstroke}{rgb}{0.500000,0.000000,0.500000}%
\pgfsetstrokecolor{currentstroke}%
\pgfsetdash{}{0pt}%
\pgfpathmoveto{\pgfqpoint{4.910254in}{5.922373in}}%
\pgfpathcurveto{\pgfqpoint{4.916078in}{5.922373in}}{\pgfqpoint{4.921665in}{5.924686in}}{\pgfqpoint{4.925783in}{5.928805in}}%
\pgfpathcurveto{\pgfqpoint{4.929901in}{5.932923in}}{\pgfqpoint{4.932215in}{5.938509in}}{\pgfqpoint{4.932215in}{5.944333in}}%
\pgfpathcurveto{\pgfqpoint{4.932215in}{5.950157in}}{\pgfqpoint{4.929901in}{5.955743in}}{\pgfqpoint{4.925783in}{5.959861in}}%
\pgfpathcurveto{\pgfqpoint{4.921665in}{5.963979in}}{\pgfqpoint{4.916078in}{5.966293in}}{\pgfqpoint{4.910254in}{5.966293in}}%
\pgfpathcurveto{\pgfqpoint{4.904431in}{5.966293in}}{\pgfqpoint{4.898844in}{5.963979in}}{\pgfqpoint{4.894726in}{5.959861in}}%
\pgfpathcurveto{\pgfqpoint{4.890608in}{5.955743in}}{\pgfqpoint{4.888294in}{5.950157in}}{\pgfqpoint{4.888294in}{5.944333in}}%
\pgfpathcurveto{\pgfqpoint{4.888294in}{5.938509in}}{\pgfqpoint{4.890608in}{5.932923in}}{\pgfqpoint{4.894726in}{5.928805in}}%
\pgfpathcurveto{\pgfqpoint{4.898844in}{5.924686in}}{\pgfqpoint{4.904431in}{5.922373in}}{\pgfqpoint{4.910254in}{5.922373in}}%
\pgfpathlineto{\pgfqpoint{4.910254in}{5.922373in}}%
\pgfpathclose%
\pgfusepath{stroke,fill}%
\end{pgfscope}%
\begin{pgfscope}%
\pgfpathrectangle{\pgfqpoint{1.542338in}{0.880000in}}{\pgfqpoint{5.115323in}{6.160000in}}%
\pgfusepath{clip}%
\pgfsetbuttcap%
\pgfsetroundjoin%
\definecolor{currentfill}{rgb}{0.500000,0.000000,0.500000}%
\pgfsetfillcolor{currentfill}%
\pgfsetlinewidth{1.003750pt}%
\definecolor{currentstroke}{rgb}{0.500000,0.000000,0.500000}%
\pgfsetstrokecolor{currentstroke}%
\pgfsetdash{}{0pt}%
\pgfpathmoveto{\pgfqpoint{5.474691in}{3.864588in}}%
\pgfpathcurveto{\pgfqpoint{5.480515in}{3.864588in}}{\pgfqpoint{5.486101in}{3.866902in}}{\pgfqpoint{5.490219in}{3.871020in}}%
\pgfpathcurveto{\pgfqpoint{5.494337in}{3.875139in}}{\pgfqpoint{5.496651in}{3.880725in}}{\pgfqpoint{5.496651in}{3.886549in}}%
\pgfpathcurveto{\pgfqpoint{5.496651in}{3.892373in}}{\pgfqpoint{5.494337in}{3.897959in}}{\pgfqpoint{5.490219in}{3.902077in}}%
\pgfpathcurveto{\pgfqpoint{5.486101in}{3.906195in}}{\pgfqpoint{5.480515in}{3.908509in}}{\pgfqpoint{5.474691in}{3.908509in}}%
\pgfpathcurveto{\pgfqpoint{5.468867in}{3.908509in}}{\pgfqpoint{5.463281in}{3.906195in}}{\pgfqpoint{5.459163in}{3.902077in}}%
\pgfpathcurveto{\pgfqpoint{5.455044in}{3.897959in}}{\pgfqpoint{5.452731in}{3.892373in}}{\pgfqpoint{5.452731in}{3.886549in}}%
\pgfpathcurveto{\pgfqpoint{5.452731in}{3.880725in}}{\pgfqpoint{5.455044in}{3.875139in}}{\pgfqpoint{5.459163in}{3.871020in}}%
\pgfpathcurveto{\pgfqpoint{5.463281in}{3.866902in}}{\pgfqpoint{5.468867in}{3.864588in}}{\pgfqpoint{5.474691in}{3.864588in}}%
\pgfpathlineto{\pgfqpoint{5.474691in}{3.864588in}}%
\pgfpathclose%
\pgfusepath{stroke,fill}%
\end{pgfscope}%
\begin{pgfscope}%
\pgfpathrectangle{\pgfqpoint{1.542338in}{0.880000in}}{\pgfqpoint{5.115323in}{6.160000in}}%
\pgfusepath{clip}%
\pgfsetbuttcap%
\pgfsetmiterjoin%
\pgfsetlinewidth{1.003750pt}%
\definecolor{currentstroke}{rgb}{0.800000,0.200000,0.200000}%
\pgfsetstrokecolor{currentstroke}%
\pgfsetdash{}{0pt}%
\pgfpathmoveto{\pgfqpoint{5.448380in}{1.188476in}}%
\pgfpathcurveto{\pgfqpoint{5.718627in}{1.188476in}}{\pgfqpoint{5.977841in}{1.295846in}}{\pgfqpoint{6.168935in}{1.486939in}}%
\pgfpathcurveto{\pgfqpoint{6.360028in}{1.678033in}}{\pgfqpoint{6.467398in}{1.937247in}}{\pgfqpoint{6.467398in}{2.207494in}}%
\pgfpathcurveto{\pgfqpoint{6.467398in}{2.477741in}}{\pgfqpoint{6.360028in}{2.736956in}}{\pgfqpoint{6.168935in}{2.928049in}}%
\pgfpathcurveto{\pgfqpoint{5.977841in}{3.119142in}}{\pgfqpoint{5.718627in}{3.226513in}}{\pgfqpoint{5.448380in}{3.226513in}}%
\pgfpathcurveto{\pgfqpoint{5.178133in}{3.226513in}}{\pgfqpoint{4.918918in}{3.119142in}}{\pgfqpoint{4.727825in}{2.928049in}}%
\pgfpathcurveto{\pgfqpoint{4.536732in}{2.736956in}}{\pgfqpoint{4.429361in}{2.477741in}}{\pgfqpoint{4.429361in}{2.207494in}}%
\pgfpathcurveto{\pgfqpoint{4.429361in}{1.937247in}}{\pgfqpoint{4.536732in}{1.678033in}}{\pgfqpoint{4.727825in}{1.486939in}}%
\pgfpathcurveto{\pgfqpoint{4.918918in}{1.295846in}}{\pgfqpoint{5.178133in}{1.188476in}}{\pgfqpoint{5.448380in}{1.188476in}}%
\pgfpathlineto{\pgfqpoint{5.448380in}{1.188476in}}%
\pgfpathclose%
\pgfusepath{stroke}%
\end{pgfscope}%
\begin{pgfscope}%
\pgfpathrectangle{\pgfqpoint{1.542338in}{0.880000in}}{\pgfqpoint{5.115323in}{6.160000in}}%
\pgfusepath{clip}%
\pgfsetbuttcap%
\pgfsetroundjoin%
\definecolor{currentfill}{rgb}{0.000000,0.000000,0.000000}%
\pgfsetfillcolor{currentfill}%
\pgfsetlinewidth{1.003750pt}%
\definecolor{currentstroke}{rgb}{0.000000,0.000000,0.000000}%
\pgfsetstrokecolor{currentstroke}%
\pgfsetdash{}{0pt}%
\pgfsys@defobject{currentmarker}{\pgfqpoint{-0.021960in}{-0.021960in}}{\pgfqpoint{0.021960in}{0.021960in}}{%
\pgfpathmoveto{\pgfqpoint{0.000000in}{-0.021960in}}%
\pgfpathcurveto{\pgfqpoint{0.005824in}{-0.021960in}}{\pgfqpoint{0.011410in}{-0.019646in}}{\pgfqpoint{0.015528in}{-0.015528in}}%
\pgfpathcurveto{\pgfqpoint{0.019646in}{-0.011410in}}{\pgfqpoint{0.021960in}{-0.005824in}}{\pgfqpoint{0.021960in}{0.000000in}}%
\pgfpathcurveto{\pgfqpoint{0.021960in}{0.005824in}}{\pgfqpoint{0.019646in}{0.011410in}}{\pgfqpoint{0.015528in}{0.015528in}}%
\pgfpathcurveto{\pgfqpoint{0.011410in}{0.019646in}}{\pgfqpoint{0.005824in}{0.021960in}}{\pgfqpoint{0.000000in}{0.021960in}}%
\pgfpathcurveto{\pgfqpoint{-0.005824in}{0.021960in}}{\pgfqpoint{-0.011410in}{0.019646in}}{\pgfqpoint{-0.015528in}{0.015528in}}%
\pgfpathcurveto{\pgfqpoint{-0.019646in}{0.011410in}}{\pgfqpoint{-0.021960in}{0.005824in}}{\pgfqpoint{-0.021960in}{0.000000in}}%
\pgfpathcurveto{\pgfqpoint{-0.021960in}{-0.005824in}}{\pgfqpoint{-0.019646in}{-0.011410in}}{\pgfqpoint{-0.015528in}{-0.015528in}}%
\pgfpathcurveto{\pgfqpoint{-0.011410in}{-0.019646in}}{\pgfqpoint{-0.005824in}{-0.021960in}}{\pgfqpoint{0.000000in}{-0.021960in}}%
\pgfpathlineto{\pgfqpoint{0.000000in}{-0.021960in}}%
\pgfpathclose%
\pgfusepath{stroke,fill}%
}%
\begin{pgfscope}%
\pgfsys@transformshift{5.448380in}{2.207494in}%
\pgfsys@useobject{currentmarker}{}%
\end{pgfscope}%
\end{pgfscope}%
\begin{pgfscope}%
\pgfpathrectangle{\pgfqpoint{1.542338in}{0.880000in}}{\pgfqpoint{5.115323in}{6.160000in}}%
\pgfusepath{clip}%
\pgfsetbuttcap%
\pgfsetmiterjoin%
\pgfsetlinewidth{1.003750pt}%
\definecolor{currentstroke}{rgb}{0.200000,0.800000,0.200000}%
\pgfsetstrokecolor{currentstroke}%
\pgfsetdash{}{0pt}%
\pgfpathmoveto{\pgfqpoint{4.516112in}{4.845237in}}%
\pgfpathcurveto{\pgfqpoint{4.768988in}{4.845237in}}{\pgfqpoint{5.011541in}{4.945706in}}{\pgfqpoint{5.190352in}{5.124516in}}%
\pgfpathcurveto{\pgfqpoint{5.369162in}{5.303327in}}{\pgfqpoint{5.469631in}{5.545880in}}{\pgfqpoint{5.469631in}{5.798756in}}%
\pgfpathcurveto{\pgfqpoint{5.469631in}{6.051633in}}{\pgfqpoint{5.369162in}{6.294186in}}{\pgfqpoint{5.190352in}{6.472996in}}%
\pgfpathcurveto{\pgfqpoint{5.011541in}{6.651807in}}{\pgfqpoint{4.768988in}{6.752276in}}{\pgfqpoint{4.516112in}{6.752276in}}%
\pgfpathcurveto{\pgfqpoint{4.263236in}{6.752276in}}{\pgfqpoint{4.020682in}{6.651807in}}{\pgfqpoint{3.841872in}{6.472996in}}%
\pgfpathcurveto{\pgfqpoint{3.663061in}{6.294186in}}{\pgfqpoint{3.562593in}{6.051633in}}{\pgfqpoint{3.562593in}{5.798756in}}%
\pgfpathcurveto{\pgfqpoint{3.562593in}{5.545880in}}{\pgfqpoint{3.663061in}{5.303327in}}{\pgfqpoint{3.841872in}{5.124516in}}%
\pgfpathcurveto{\pgfqpoint{4.020682in}{4.945706in}}{\pgfqpoint{4.263236in}{4.845237in}}{\pgfqpoint{4.516112in}{4.845237in}}%
\pgfpathlineto{\pgfqpoint{4.516112in}{4.845237in}}%
\pgfpathclose%
\pgfusepath{stroke}%
\end{pgfscope}%
\begin{pgfscope}%
\pgfpathrectangle{\pgfqpoint{1.542338in}{0.880000in}}{\pgfqpoint{5.115323in}{6.160000in}}%
\pgfusepath{clip}%
\pgfsetbuttcap%
\pgfsetroundjoin%
\definecolor{currentfill}{rgb}{0.000000,0.000000,0.000000}%
\pgfsetfillcolor{currentfill}%
\pgfsetlinewidth{1.003750pt}%
\definecolor{currentstroke}{rgb}{0.000000,0.000000,0.000000}%
\pgfsetstrokecolor{currentstroke}%
\pgfsetdash{}{0pt}%
\pgfsys@defobject{currentmarker}{\pgfqpoint{-0.021960in}{-0.021960in}}{\pgfqpoint{0.021960in}{0.021960in}}{%
\pgfpathmoveto{\pgfqpoint{0.000000in}{-0.021960in}}%
\pgfpathcurveto{\pgfqpoint{0.005824in}{-0.021960in}}{\pgfqpoint{0.011410in}{-0.019646in}}{\pgfqpoint{0.015528in}{-0.015528in}}%
\pgfpathcurveto{\pgfqpoint{0.019646in}{-0.011410in}}{\pgfqpoint{0.021960in}{-0.005824in}}{\pgfqpoint{0.021960in}{0.000000in}}%
\pgfpathcurveto{\pgfqpoint{0.021960in}{0.005824in}}{\pgfqpoint{0.019646in}{0.011410in}}{\pgfqpoint{0.015528in}{0.015528in}}%
\pgfpathcurveto{\pgfqpoint{0.011410in}{0.019646in}}{\pgfqpoint{0.005824in}{0.021960in}}{\pgfqpoint{0.000000in}{0.021960in}}%
\pgfpathcurveto{\pgfqpoint{-0.005824in}{0.021960in}}{\pgfqpoint{-0.011410in}{0.019646in}}{\pgfqpoint{-0.015528in}{0.015528in}}%
\pgfpathcurveto{\pgfqpoint{-0.019646in}{0.011410in}}{\pgfqpoint{-0.021960in}{0.005824in}}{\pgfqpoint{-0.021960in}{0.000000in}}%
\pgfpathcurveto{\pgfqpoint{-0.021960in}{-0.005824in}}{\pgfqpoint{-0.019646in}{-0.011410in}}{\pgfqpoint{-0.015528in}{-0.015528in}}%
\pgfpathcurveto{\pgfqpoint{-0.011410in}{-0.019646in}}{\pgfqpoint{-0.005824in}{-0.021960in}}{\pgfqpoint{0.000000in}{-0.021960in}}%
\pgfpathlineto{\pgfqpoint{0.000000in}{-0.021960in}}%
\pgfpathclose%
\pgfusepath{stroke,fill}%
}%
\begin{pgfscope}%
\pgfsys@transformshift{4.516112in}{5.798756in}%
\pgfsys@useobject{currentmarker}{}%
\end{pgfscope}%
\end{pgfscope}%
\begin{pgfscope}%
\pgfpathrectangle{\pgfqpoint{1.542338in}{0.880000in}}{\pgfqpoint{5.115323in}{6.160000in}}%
\pgfusepath{clip}%
\pgfsetbuttcap%
\pgfsetmiterjoin%
\pgfsetlinewidth{1.003750pt}%
\definecolor{currentstroke}{rgb}{0.200000,0.200000,0.800000}%
\pgfsetstrokecolor{currentstroke}%
\pgfsetdash{}{0pt}%
\pgfpathmoveto{\pgfqpoint{2.975988in}{1.387924in}}%
\pgfpathcurveto{\pgfqpoint{3.293874in}{1.387924in}}{\pgfqpoint{3.598782in}{1.514221in}}{\pgfqpoint{3.823561in}{1.739000in}}%
\pgfpathcurveto{\pgfqpoint{4.048340in}{1.963779in}}{\pgfqpoint{4.174637in}{2.268688in}}{\pgfqpoint{4.174637in}{2.586574in}}%
\pgfpathcurveto{\pgfqpoint{4.174637in}{2.904459in}}{\pgfqpoint{4.048340in}{3.209368in}}{\pgfqpoint{3.823561in}{3.434147in}}%
\pgfpathcurveto{\pgfqpoint{3.598782in}{3.658926in}}{\pgfqpoint{3.293874in}{3.785223in}}{\pgfqpoint{2.975988in}{3.785223in}}%
\pgfpathcurveto{\pgfqpoint{2.658103in}{3.785223in}}{\pgfqpoint{2.353194in}{3.658926in}}{\pgfqpoint{2.128415in}{3.434147in}}%
\pgfpathcurveto{\pgfqpoint{1.903636in}{3.209368in}}{\pgfqpoint{1.777339in}{2.904459in}}{\pgfqpoint{1.777339in}{2.586574in}}%
\pgfpathcurveto{\pgfqpoint{1.777339in}{2.268688in}}{\pgfqpoint{1.903636in}{1.963779in}}{\pgfqpoint{2.128415in}{1.739000in}}%
\pgfpathcurveto{\pgfqpoint{2.353194in}{1.514221in}}{\pgfqpoint{2.658103in}{1.387924in}}{\pgfqpoint{2.975988in}{1.387924in}}%
\pgfpathlineto{\pgfqpoint{2.975988in}{1.387924in}}%
\pgfpathclose%
\pgfusepath{stroke}%
\end{pgfscope}%
\begin{pgfscope}%
\pgfpathrectangle{\pgfqpoint{1.542338in}{0.880000in}}{\pgfqpoint{5.115323in}{6.160000in}}%
\pgfusepath{clip}%
\pgfsetbuttcap%
\pgfsetroundjoin%
\definecolor{currentfill}{rgb}{0.000000,0.000000,0.000000}%
\pgfsetfillcolor{currentfill}%
\pgfsetlinewidth{1.003750pt}%
\definecolor{currentstroke}{rgb}{0.000000,0.000000,0.000000}%
\pgfsetstrokecolor{currentstroke}%
\pgfsetdash{}{0pt}%
\pgfsys@defobject{currentmarker}{\pgfqpoint{-0.021960in}{-0.021960in}}{\pgfqpoint{0.021960in}{0.021960in}}{%
\pgfpathmoveto{\pgfqpoint{0.000000in}{-0.021960in}}%
\pgfpathcurveto{\pgfqpoint{0.005824in}{-0.021960in}}{\pgfqpoint{0.011410in}{-0.019646in}}{\pgfqpoint{0.015528in}{-0.015528in}}%
\pgfpathcurveto{\pgfqpoint{0.019646in}{-0.011410in}}{\pgfqpoint{0.021960in}{-0.005824in}}{\pgfqpoint{0.021960in}{0.000000in}}%
\pgfpathcurveto{\pgfqpoint{0.021960in}{0.005824in}}{\pgfqpoint{0.019646in}{0.011410in}}{\pgfqpoint{0.015528in}{0.015528in}}%
\pgfpathcurveto{\pgfqpoint{0.011410in}{0.019646in}}{\pgfqpoint{0.005824in}{0.021960in}}{\pgfqpoint{0.000000in}{0.021960in}}%
\pgfpathcurveto{\pgfqpoint{-0.005824in}{0.021960in}}{\pgfqpoint{-0.011410in}{0.019646in}}{\pgfqpoint{-0.015528in}{0.015528in}}%
\pgfpathcurveto{\pgfqpoint{-0.019646in}{0.011410in}}{\pgfqpoint{-0.021960in}{0.005824in}}{\pgfqpoint{-0.021960in}{0.000000in}}%
\pgfpathcurveto{\pgfqpoint{-0.021960in}{-0.005824in}}{\pgfqpoint{-0.019646in}{-0.011410in}}{\pgfqpoint{-0.015528in}{-0.015528in}}%
\pgfpathcurveto{\pgfqpoint{-0.011410in}{-0.019646in}}{\pgfqpoint{-0.005824in}{-0.021960in}}{\pgfqpoint{0.000000in}{-0.021960in}}%
\pgfpathlineto{\pgfqpoint{0.000000in}{-0.021960in}}%
\pgfpathclose%
\pgfusepath{stroke,fill}%
}%
\begin{pgfscope}%
\pgfsys@transformshift{2.975988in}{2.586574in}%
\pgfsys@useobject{currentmarker}{}%
\end{pgfscope}%
\end{pgfscope}%
\begin{pgfscope}%
\pgfsetbuttcap%
\pgfsetroundjoin%
\definecolor{currentfill}{rgb}{0.000000,0.000000,0.000000}%
\pgfsetfillcolor{currentfill}%
\pgfsetlinewidth{0.803000pt}%
\definecolor{currentstroke}{rgb}{0.000000,0.000000,0.000000}%
\pgfsetstrokecolor{currentstroke}%
\pgfsetdash{}{0pt}%
\pgfsys@defobject{currentmarker}{\pgfqpoint{0.000000in}{-0.048611in}}{\pgfqpoint{0.000000in}{0.000000in}}{%
\pgfpathmoveto{\pgfqpoint{0.000000in}{0.000000in}}%
\pgfpathlineto{\pgfqpoint{0.000000in}{-0.048611in}}%
\pgfusepath{stroke,fill}%
}%
\begin{pgfscope}%
\pgfsys@transformshift{1.818751in}{0.880000in}%
\pgfsys@useobject{currentmarker}{}%
\end{pgfscope}%
\end{pgfscope}%
\begin{pgfscope}%
\definecolor{textcolor}{rgb}{0.000000,0.000000,0.000000}%
\pgfsetstrokecolor{textcolor}%
\pgfsetfillcolor{textcolor}%
\pgftext[x=1.818751in,y=0.782778in,,top]{\color{textcolor}{\sffamily\fontsize{10.000000}{12.000000}\selectfont\catcode`\^=\active\def^{\ifmmode\sp\else\^{}\fi}\catcode`\%=\active\def%{\%}\ensuremath{-}600}}%
\end{pgfscope}%
\begin{pgfscope}%
\pgfsetbuttcap%
\pgfsetroundjoin%
\definecolor{currentfill}{rgb}{0.000000,0.000000,0.000000}%
\pgfsetfillcolor{currentfill}%
\pgfsetlinewidth{0.803000pt}%
\definecolor{currentstroke}{rgb}{0.000000,0.000000,0.000000}%
\pgfsetstrokecolor{currentstroke}%
\pgfsetdash{}{0pt}%
\pgfsys@defobject{currentmarker}{\pgfqpoint{0.000000in}{-0.048611in}}{\pgfqpoint{0.000000in}{0.000000in}}{%
\pgfpathmoveto{\pgfqpoint{0.000000in}{0.000000in}}%
\pgfpathlineto{\pgfqpoint{0.000000in}{-0.048611in}}%
\pgfusepath{stroke,fill}%
}%
\begin{pgfscope}%
\pgfsys@transformshift{2.621632in}{0.880000in}%
\pgfsys@useobject{currentmarker}{}%
\end{pgfscope}%
\end{pgfscope}%
\begin{pgfscope}%
\definecolor{textcolor}{rgb}{0.000000,0.000000,0.000000}%
\pgfsetstrokecolor{textcolor}%
\pgfsetfillcolor{textcolor}%
\pgftext[x=2.621632in,y=0.782778in,,top]{\color{textcolor}{\sffamily\fontsize{10.000000}{12.000000}\selectfont\catcode`\^=\active\def^{\ifmmode\sp\else\^{}\fi}\catcode`\%=\active\def%{\%}\ensuremath{-}400}}%
\end{pgfscope}%
\begin{pgfscope}%
\pgfsetbuttcap%
\pgfsetroundjoin%
\definecolor{currentfill}{rgb}{0.000000,0.000000,0.000000}%
\pgfsetfillcolor{currentfill}%
\pgfsetlinewidth{0.803000pt}%
\definecolor{currentstroke}{rgb}{0.000000,0.000000,0.000000}%
\pgfsetstrokecolor{currentstroke}%
\pgfsetdash{}{0pt}%
\pgfsys@defobject{currentmarker}{\pgfqpoint{0.000000in}{-0.048611in}}{\pgfqpoint{0.000000in}{0.000000in}}{%
\pgfpathmoveto{\pgfqpoint{0.000000in}{0.000000in}}%
\pgfpathlineto{\pgfqpoint{0.000000in}{-0.048611in}}%
\pgfusepath{stroke,fill}%
}%
\begin{pgfscope}%
\pgfsys@transformshift{3.424513in}{0.880000in}%
\pgfsys@useobject{currentmarker}{}%
\end{pgfscope}%
\end{pgfscope}%
\begin{pgfscope}%
\definecolor{textcolor}{rgb}{0.000000,0.000000,0.000000}%
\pgfsetstrokecolor{textcolor}%
\pgfsetfillcolor{textcolor}%
\pgftext[x=3.424513in,y=0.782778in,,top]{\color{textcolor}{\sffamily\fontsize{10.000000}{12.000000}\selectfont\catcode`\^=\active\def^{\ifmmode\sp\else\^{}\fi}\catcode`\%=\active\def%{\%}\ensuremath{-}200}}%
\end{pgfscope}%
\begin{pgfscope}%
\pgfsetbuttcap%
\pgfsetroundjoin%
\definecolor{currentfill}{rgb}{0.000000,0.000000,0.000000}%
\pgfsetfillcolor{currentfill}%
\pgfsetlinewidth{0.803000pt}%
\definecolor{currentstroke}{rgb}{0.000000,0.000000,0.000000}%
\pgfsetstrokecolor{currentstroke}%
\pgfsetdash{}{0pt}%
\pgfsys@defobject{currentmarker}{\pgfqpoint{0.000000in}{-0.048611in}}{\pgfqpoint{0.000000in}{0.000000in}}{%
\pgfpathmoveto{\pgfqpoint{0.000000in}{0.000000in}}%
\pgfpathlineto{\pgfqpoint{0.000000in}{-0.048611in}}%
\pgfusepath{stroke,fill}%
}%
\begin{pgfscope}%
\pgfsys@transformshift{4.227394in}{0.880000in}%
\pgfsys@useobject{currentmarker}{}%
\end{pgfscope}%
\end{pgfscope}%
\begin{pgfscope}%
\definecolor{textcolor}{rgb}{0.000000,0.000000,0.000000}%
\pgfsetstrokecolor{textcolor}%
\pgfsetfillcolor{textcolor}%
\pgftext[x=4.227394in,y=0.782778in,,top]{\color{textcolor}{\sffamily\fontsize{10.000000}{12.000000}\selectfont\catcode`\^=\active\def^{\ifmmode\sp\else\^{}\fi}\catcode`\%=\active\def%{\%}0}}%
\end{pgfscope}%
\begin{pgfscope}%
\pgfsetbuttcap%
\pgfsetroundjoin%
\definecolor{currentfill}{rgb}{0.000000,0.000000,0.000000}%
\pgfsetfillcolor{currentfill}%
\pgfsetlinewidth{0.803000pt}%
\definecolor{currentstroke}{rgb}{0.000000,0.000000,0.000000}%
\pgfsetstrokecolor{currentstroke}%
\pgfsetdash{}{0pt}%
\pgfsys@defobject{currentmarker}{\pgfqpoint{0.000000in}{-0.048611in}}{\pgfqpoint{0.000000in}{0.000000in}}{%
\pgfpathmoveto{\pgfqpoint{0.000000in}{0.000000in}}%
\pgfpathlineto{\pgfqpoint{0.000000in}{-0.048611in}}%
\pgfusepath{stroke,fill}%
}%
\begin{pgfscope}%
\pgfsys@transformshift{5.030275in}{0.880000in}%
\pgfsys@useobject{currentmarker}{}%
\end{pgfscope}%
\end{pgfscope}%
\begin{pgfscope}%
\definecolor{textcolor}{rgb}{0.000000,0.000000,0.000000}%
\pgfsetstrokecolor{textcolor}%
\pgfsetfillcolor{textcolor}%
\pgftext[x=5.030275in,y=0.782778in,,top]{\color{textcolor}{\sffamily\fontsize{10.000000}{12.000000}\selectfont\catcode`\^=\active\def^{\ifmmode\sp\else\^{}\fi}\catcode`\%=\active\def%{\%}200}}%
\end{pgfscope}%
\begin{pgfscope}%
\pgfsetbuttcap%
\pgfsetroundjoin%
\definecolor{currentfill}{rgb}{0.000000,0.000000,0.000000}%
\pgfsetfillcolor{currentfill}%
\pgfsetlinewidth{0.803000pt}%
\definecolor{currentstroke}{rgb}{0.000000,0.000000,0.000000}%
\pgfsetstrokecolor{currentstroke}%
\pgfsetdash{}{0pt}%
\pgfsys@defobject{currentmarker}{\pgfqpoint{0.000000in}{-0.048611in}}{\pgfqpoint{0.000000in}{0.000000in}}{%
\pgfpathmoveto{\pgfqpoint{0.000000in}{0.000000in}}%
\pgfpathlineto{\pgfqpoint{0.000000in}{-0.048611in}}%
\pgfusepath{stroke,fill}%
}%
\begin{pgfscope}%
\pgfsys@transformshift{5.833156in}{0.880000in}%
\pgfsys@useobject{currentmarker}{}%
\end{pgfscope}%
\end{pgfscope}%
\begin{pgfscope}%
\definecolor{textcolor}{rgb}{0.000000,0.000000,0.000000}%
\pgfsetstrokecolor{textcolor}%
\pgfsetfillcolor{textcolor}%
\pgftext[x=5.833156in,y=0.782778in,,top]{\color{textcolor}{\sffamily\fontsize{10.000000}{12.000000}\selectfont\catcode`\^=\active\def^{\ifmmode\sp\else\^{}\fi}\catcode`\%=\active\def%{\%}400}}%
\end{pgfscope}%
\begin{pgfscope}%
\pgfsetbuttcap%
\pgfsetroundjoin%
\definecolor{currentfill}{rgb}{0.000000,0.000000,0.000000}%
\pgfsetfillcolor{currentfill}%
\pgfsetlinewidth{0.803000pt}%
\definecolor{currentstroke}{rgb}{0.000000,0.000000,0.000000}%
\pgfsetstrokecolor{currentstroke}%
\pgfsetdash{}{0pt}%
\pgfsys@defobject{currentmarker}{\pgfqpoint{0.000000in}{-0.048611in}}{\pgfqpoint{0.000000in}{0.000000in}}{%
\pgfpathmoveto{\pgfqpoint{0.000000in}{0.000000in}}%
\pgfpathlineto{\pgfqpoint{0.000000in}{-0.048611in}}%
\pgfusepath{stroke,fill}%
}%
\begin{pgfscope}%
\pgfsys@transformshift{6.636037in}{0.880000in}%
\pgfsys@useobject{currentmarker}{}%
\end{pgfscope}%
\end{pgfscope}%
\begin{pgfscope}%
\definecolor{textcolor}{rgb}{0.000000,0.000000,0.000000}%
\pgfsetstrokecolor{textcolor}%
\pgfsetfillcolor{textcolor}%
\pgftext[x=6.636037in,y=0.782778in,,top]{\color{textcolor}{\sffamily\fontsize{10.000000}{12.000000}\selectfont\catcode`\^=\active\def^{\ifmmode\sp\else\^{}\fi}\catcode`\%=\active\def%{\%}600}}%
\end{pgfscope}%
\begin{pgfscope}%
\pgfsetbuttcap%
\pgfsetroundjoin%
\definecolor{currentfill}{rgb}{0.000000,0.000000,0.000000}%
\pgfsetfillcolor{currentfill}%
\pgfsetlinewidth{0.803000pt}%
\definecolor{currentstroke}{rgb}{0.000000,0.000000,0.000000}%
\pgfsetstrokecolor{currentstroke}%
\pgfsetdash{}{0pt}%
\pgfsys@defobject{currentmarker}{\pgfqpoint{-0.048611in}{0.000000in}}{\pgfqpoint{-0.000000in}{0.000000in}}{%
\pgfpathmoveto{\pgfqpoint{-0.000000in}{0.000000in}}%
\pgfpathlineto{\pgfqpoint{-0.048611in}{0.000000in}}%
\pgfusepath{stroke,fill}%
}%
\begin{pgfscope}%
\pgfsys@transformshift{1.542338in}{1.607138in}%
\pgfsys@useobject{currentmarker}{}%
\end{pgfscope}%
\end{pgfscope}%
\begin{pgfscope}%
\definecolor{textcolor}{rgb}{0.000000,0.000000,0.000000}%
\pgfsetstrokecolor{textcolor}%
\pgfsetfillcolor{textcolor}%
\pgftext[x=1.071995in, y=1.554376in, left, base]{\color{textcolor}{\sffamily\fontsize{10.000000}{12.000000}\selectfont\catcode`\^=\active\def^{\ifmmode\sp\else\^{}\fi}\catcode`\%=\active\def%{\%}\ensuremath{-}600}}%
\end{pgfscope}%
\begin{pgfscope}%
\pgfsetbuttcap%
\pgfsetroundjoin%
\definecolor{currentfill}{rgb}{0.000000,0.000000,0.000000}%
\pgfsetfillcolor{currentfill}%
\pgfsetlinewidth{0.803000pt}%
\definecolor{currentstroke}{rgb}{0.000000,0.000000,0.000000}%
\pgfsetstrokecolor{currentstroke}%
\pgfsetdash{}{0pt}%
\pgfsys@defobject{currentmarker}{\pgfqpoint{-0.048611in}{0.000000in}}{\pgfqpoint{-0.000000in}{0.000000in}}{%
\pgfpathmoveto{\pgfqpoint{-0.000000in}{0.000000in}}%
\pgfpathlineto{\pgfqpoint{-0.048611in}{0.000000in}}%
\pgfusepath{stroke,fill}%
}%
\begin{pgfscope}%
\pgfsys@transformshift{1.542338in}{2.410019in}%
\pgfsys@useobject{currentmarker}{}%
\end{pgfscope}%
\end{pgfscope}%
\begin{pgfscope}%
\definecolor{textcolor}{rgb}{0.000000,0.000000,0.000000}%
\pgfsetstrokecolor{textcolor}%
\pgfsetfillcolor{textcolor}%
\pgftext[x=1.071995in, y=2.357257in, left, base]{\color{textcolor}{\sffamily\fontsize{10.000000}{12.000000}\selectfont\catcode`\^=\active\def^{\ifmmode\sp\else\^{}\fi}\catcode`\%=\active\def%{\%}\ensuremath{-}400}}%
\end{pgfscope}%
\begin{pgfscope}%
\pgfsetbuttcap%
\pgfsetroundjoin%
\definecolor{currentfill}{rgb}{0.000000,0.000000,0.000000}%
\pgfsetfillcolor{currentfill}%
\pgfsetlinewidth{0.803000pt}%
\definecolor{currentstroke}{rgb}{0.000000,0.000000,0.000000}%
\pgfsetstrokecolor{currentstroke}%
\pgfsetdash{}{0pt}%
\pgfsys@defobject{currentmarker}{\pgfqpoint{-0.048611in}{0.000000in}}{\pgfqpoint{-0.000000in}{0.000000in}}{%
\pgfpathmoveto{\pgfqpoint{-0.000000in}{0.000000in}}%
\pgfpathlineto{\pgfqpoint{-0.048611in}{0.000000in}}%
\pgfusepath{stroke,fill}%
}%
\begin{pgfscope}%
\pgfsys@transformshift{1.542338in}{3.212900in}%
\pgfsys@useobject{currentmarker}{}%
\end{pgfscope}%
\end{pgfscope}%
\begin{pgfscope}%
\definecolor{textcolor}{rgb}{0.000000,0.000000,0.000000}%
\pgfsetstrokecolor{textcolor}%
\pgfsetfillcolor{textcolor}%
\pgftext[x=1.071995in, y=3.160138in, left, base]{\color{textcolor}{\sffamily\fontsize{10.000000}{12.000000}\selectfont\catcode`\^=\active\def^{\ifmmode\sp\else\^{}\fi}\catcode`\%=\active\def%{\%}\ensuremath{-}200}}%
\end{pgfscope}%
\begin{pgfscope}%
\pgfsetbuttcap%
\pgfsetroundjoin%
\definecolor{currentfill}{rgb}{0.000000,0.000000,0.000000}%
\pgfsetfillcolor{currentfill}%
\pgfsetlinewidth{0.803000pt}%
\definecolor{currentstroke}{rgb}{0.000000,0.000000,0.000000}%
\pgfsetstrokecolor{currentstroke}%
\pgfsetdash{}{0pt}%
\pgfsys@defobject{currentmarker}{\pgfqpoint{-0.048611in}{0.000000in}}{\pgfqpoint{-0.000000in}{0.000000in}}{%
\pgfpathmoveto{\pgfqpoint{-0.000000in}{0.000000in}}%
\pgfpathlineto{\pgfqpoint{-0.048611in}{0.000000in}}%
\pgfusepath{stroke,fill}%
}%
\begin{pgfscope}%
\pgfsys@transformshift{1.542338in}{4.015781in}%
\pgfsys@useobject{currentmarker}{}%
\end{pgfscope}%
\end{pgfscope}%
\begin{pgfscope}%
\definecolor{textcolor}{rgb}{0.000000,0.000000,0.000000}%
\pgfsetstrokecolor{textcolor}%
\pgfsetfillcolor{textcolor}%
\pgftext[x=1.356751in, y=3.963019in, left, base]{\color{textcolor}{\sffamily\fontsize{10.000000}{12.000000}\selectfont\catcode`\^=\active\def^{\ifmmode\sp\else\^{}\fi}\catcode`\%=\active\def%{\%}0}}%
\end{pgfscope}%
\begin{pgfscope}%
\pgfsetbuttcap%
\pgfsetroundjoin%
\definecolor{currentfill}{rgb}{0.000000,0.000000,0.000000}%
\pgfsetfillcolor{currentfill}%
\pgfsetlinewidth{0.803000pt}%
\definecolor{currentstroke}{rgb}{0.000000,0.000000,0.000000}%
\pgfsetstrokecolor{currentstroke}%
\pgfsetdash{}{0pt}%
\pgfsys@defobject{currentmarker}{\pgfqpoint{-0.048611in}{0.000000in}}{\pgfqpoint{-0.000000in}{0.000000in}}{%
\pgfpathmoveto{\pgfqpoint{-0.000000in}{0.000000in}}%
\pgfpathlineto{\pgfqpoint{-0.048611in}{0.000000in}}%
\pgfusepath{stroke,fill}%
}%
\begin{pgfscope}%
\pgfsys@transformshift{1.542338in}{4.818662in}%
\pgfsys@useobject{currentmarker}{}%
\end{pgfscope}%
\end{pgfscope}%
\begin{pgfscope}%
\definecolor{textcolor}{rgb}{0.000000,0.000000,0.000000}%
\pgfsetstrokecolor{textcolor}%
\pgfsetfillcolor{textcolor}%
\pgftext[x=1.180020in, y=4.765900in, left, base]{\color{textcolor}{\sffamily\fontsize{10.000000}{12.000000}\selectfont\catcode`\^=\active\def^{\ifmmode\sp\else\^{}\fi}\catcode`\%=\active\def%{\%}200}}%
\end{pgfscope}%
\begin{pgfscope}%
\pgfsetbuttcap%
\pgfsetroundjoin%
\definecolor{currentfill}{rgb}{0.000000,0.000000,0.000000}%
\pgfsetfillcolor{currentfill}%
\pgfsetlinewidth{0.803000pt}%
\definecolor{currentstroke}{rgb}{0.000000,0.000000,0.000000}%
\pgfsetstrokecolor{currentstroke}%
\pgfsetdash{}{0pt}%
\pgfsys@defobject{currentmarker}{\pgfqpoint{-0.048611in}{0.000000in}}{\pgfqpoint{-0.000000in}{0.000000in}}{%
\pgfpathmoveto{\pgfqpoint{-0.000000in}{0.000000in}}%
\pgfpathlineto{\pgfqpoint{-0.048611in}{0.000000in}}%
\pgfusepath{stroke,fill}%
}%
\begin{pgfscope}%
\pgfsys@transformshift{1.542338in}{5.621543in}%
\pgfsys@useobject{currentmarker}{}%
\end{pgfscope}%
\end{pgfscope}%
\begin{pgfscope}%
\definecolor{textcolor}{rgb}{0.000000,0.000000,0.000000}%
\pgfsetstrokecolor{textcolor}%
\pgfsetfillcolor{textcolor}%
\pgftext[x=1.180020in, y=5.568781in, left, base]{\color{textcolor}{\sffamily\fontsize{10.000000}{12.000000}\selectfont\catcode`\^=\active\def^{\ifmmode\sp\else\^{}\fi}\catcode`\%=\active\def%{\%}400}}%
\end{pgfscope}%
\begin{pgfscope}%
\pgfsetbuttcap%
\pgfsetroundjoin%
\definecolor{currentfill}{rgb}{0.000000,0.000000,0.000000}%
\pgfsetfillcolor{currentfill}%
\pgfsetlinewidth{0.803000pt}%
\definecolor{currentstroke}{rgb}{0.000000,0.000000,0.000000}%
\pgfsetstrokecolor{currentstroke}%
\pgfsetdash{}{0pt}%
\pgfsys@defobject{currentmarker}{\pgfqpoint{-0.048611in}{0.000000in}}{\pgfqpoint{-0.000000in}{0.000000in}}{%
\pgfpathmoveto{\pgfqpoint{-0.000000in}{0.000000in}}%
\pgfpathlineto{\pgfqpoint{-0.048611in}{0.000000in}}%
\pgfusepath{stroke,fill}%
}%
\begin{pgfscope}%
\pgfsys@transformshift{1.542338in}{6.424424in}%
\pgfsys@useobject{currentmarker}{}%
\end{pgfscope}%
\end{pgfscope}%
\begin{pgfscope}%
\definecolor{textcolor}{rgb}{0.000000,0.000000,0.000000}%
\pgfsetstrokecolor{textcolor}%
\pgfsetfillcolor{textcolor}%
\pgftext[x=1.180020in, y=6.371662in, left, base]{\color{textcolor}{\sffamily\fontsize{10.000000}{12.000000}\selectfont\catcode`\^=\active\def^{\ifmmode\sp\else\^{}\fi}\catcode`\%=\active\def%{\%}600}}%
\end{pgfscope}%
\begin{pgfscope}%
\pgfsetrectcap%
\pgfsetmiterjoin%
\pgfsetlinewidth{0.803000pt}%
\definecolor{currentstroke}{rgb}{0.000000,0.000000,0.000000}%
\pgfsetstrokecolor{currentstroke}%
\pgfsetdash{}{0pt}%
\pgfpathmoveto{\pgfqpoint{1.542338in}{0.880000in}}%
\pgfpathlineto{\pgfqpoint{1.542338in}{7.040000in}}%
\pgfusepath{stroke}%
\end{pgfscope}%
\begin{pgfscope}%
\pgfsetrectcap%
\pgfsetmiterjoin%
\pgfsetlinewidth{0.803000pt}%
\definecolor{currentstroke}{rgb}{0.000000,0.000000,0.000000}%
\pgfsetstrokecolor{currentstroke}%
\pgfsetdash{}{0pt}%
\pgfpathmoveto{\pgfqpoint{6.657662in}{0.880000in}}%
\pgfpathlineto{\pgfqpoint{6.657662in}{7.040000in}}%
\pgfusepath{stroke}%
\end{pgfscope}%
\begin{pgfscope}%
\pgfsetrectcap%
\pgfsetmiterjoin%
\pgfsetlinewidth{0.803000pt}%
\definecolor{currentstroke}{rgb}{0.000000,0.000000,0.000000}%
\pgfsetstrokecolor{currentstroke}%
\pgfsetdash{}{0pt}%
\pgfpathmoveto{\pgfqpoint{1.542338in}{0.880000in}}%
\pgfpathlineto{\pgfqpoint{6.657662in}{0.880000in}}%
\pgfusepath{stroke}%
\end{pgfscope}%
\begin{pgfscope}%
\pgfsetrectcap%
\pgfsetmiterjoin%
\pgfsetlinewidth{0.803000pt}%
\definecolor{currentstroke}{rgb}{0.000000,0.000000,0.000000}%
\pgfsetstrokecolor{currentstroke}%
\pgfsetdash{}{0pt}%
\pgfpathmoveto{\pgfqpoint{1.542338in}{7.040000in}}%
\pgfpathlineto{\pgfqpoint{6.657662in}{7.040000in}}%
\pgfusepath{stroke}%
\end{pgfscope}%
\end{pgfpicture}%
\makeatother%
\endgroup%
}
    \label{fig:noisy_bg}
    \caption{Example of a dataset with different rings and background noise, and their correct classification (purple means noise).}
\end{figure}
The algorithm takes an additional noise threshold, which is the maximum distance a point can have to a cluster center to be considered noise.
We can express the equation as:
\begin{equation}
    \text{is\_noise}(X_j) = \begin{cases}
        1 & \text{if } \min_{i} d_{ij} > \text{noise\_distance\_threshold} \\
        0 & \text{otherwise}
    \end{cases}
\end{equation}
When computing the centers and radii, points that are considered noise are ignored, that is, all their weights are set to zero.
It is important to note that it is only for the computation of the centers and radii. The stored membership degrees are not modified.
This is done with the use of a mask, where 1 means not noise, and 0 means noise, and then multiplying the mask by the membership degrees and the distances.


\subsection{Overall Steps}
The overall steps of the algorithm are as follows:
\begin{enumerate}
    \item Initialize the parameters $U$, $V$, and $R$.
    \item While iter < max\_iter:
    \begin{enumerate}
        \item Update the membership degrees $U$.
        \item Update the cluster radii $R$ and centers $V$ using the equations \eqref{eq:d_dr} and \eqref{eq:r_i}.
        \item If convergence criterion is met, compute the noise mask, and continue to the next step.
        \item If convergence criterion is met, and the noise mask is not the same as the old one, go to the start of the loop.
        \item Else, break the loop.
    \end{enumerate}
    \item Return the membership degrees $U$, the cluster centers $V$, and the cluster radii $R$.
\end{enumerate}

\subsection{Obtaining the results}
After the algorithm has converged or we have reached the maximum number of iterations, we can obtain the results.
Recall that the membership degree can be seen as 'how much a point belongs to a cluster', and each vector can be seen as a probability distribution.
Having this in mind, there are multiple ways we could obtain the results:
\begin{enumerate}
    \item We can assign each point to the cluster with the highest membership degree.
    \item We can sample from a multinomial distribution with the membership degrees as the probabilities.
    \item We can simply use the membership degrees directly.
\end{enumerate}
We chose the first option, that is, assigning each point to the cluster with the highest membership degree.
As for the radius and the center, obtaining them is a direct result of the algorithm, and we can use them directly.

\subsection{Implementation}
\subsubsection{Data Structures}
We implement the algorithm in Python, using the NumPy \cite{harris2020array} library. The state of the algorithm is stored in three matrices:
\begin{itemize}
    \item $U$: A matrix of size $k \times n$, where $n$ is the number of data samples, and $k$ is the number of clusters. It stores the membership degrees.
    \item $V$: A matrix of size $k \times d$, where $d$ is the number of dimensions of the data samples. It stores the cluster centers.
    \item $R$: A vector of size $k$. It stores the cluster radii.
\end{itemize}
\subsubsection{Pseudocode}
All of the above steps can be formulated as Tensor/NDArray operations, with no need for explicit loops,
by defining the different operations as element-wise operations, tensor multiplications and exploiting
expansion and broadcasting rules as needed. Code is shown at \cite{github}.
This allows to exploit the parallelism of modern computing frameworks, like NumPy, Eigen, or PyTorch.
The algorithm can be implemented with the following pseudocode:
\begin{verbatim}
fn fkr(X, k, q, epsilon):
    U, V, R = initialize(X, k)
    while True:
        U = update_membership(X, V, q)
        R = update_radii(X, U, q)
        V = update_centers(X, U, R, q)
        if converged(U, U_old, epsilon):
            break
    return U, V, R
\end{verbatim}
\begin{verbatim}
fn update_membership(X, V, q):
    D = distance_matrix(X, V)
    U = D ** (-1 / (q - 1))
    U = U / U.sum(axis=1)
    return u
\end{verbatim}

\section{Experiments}
We conducted different experiments to test the performance of the algorithm. Given a dataset, that is, a set of points, we could define, informally,
two types of points, those that belong to a ring, and those that don't belong to any. We call the second one 'background noise' or 'noise' from now on.
In order to generate the dataset, we generate N rings with n noise. The noise of the rings can be seen as 'imperfection' that would occur in a real dataset.
Moreover, we generate $N_2$ noise points randomly accross the space.
\subsection{Evaluation Metrics}
We use the following metrics to evaluate the performance of the algorithm:
\begin{itemize}
    \item Squared distance error (with hard labels)
    \begin{equation}
        \text{SDE} = \sum_{i=1}^{n} distance(X_i, ring_i)^2
    \end{equation}
    Where $distance(X_i, ring_i)$ is the distance between the point $X_i$ and the circunference of the ring $ring_i$.
    $ring_i$ denotes the classified ring of the point $X_i$.
    Recall that we could classify some points as noise. In the case of a noise point, distance returns 0, that is,
    we do not take noise points into account when computing the SDE.
\end{itemize}
On the other hand, we are also interested in getting the lowest runtime possible. We measure the runtime of the algorithm, as well as the total number of iterations it takes to converge.

\subsection{Results}
We conducted different experiments to test the performance of the algorithm.
\subsubsection{General test with excentric rings}
We performed a general test with excentric rings, with different levels of noise and different numbers of rings.
The hyperparameters were set as follows:

\begin{itemize}
    \item q: 1.1
    \item convergence\_eps: $10^{-5}$
    \item max\_iters: 10000
    \item noise\_distance\_threshold: 100
    \item max\_noise\_checks: 20
    \item apply\_noise\_removal: True
    \item init\_method: "fuzzycmeans"
\end{itemize}

Each circle had 100 samples.
The data was generated in the following way:
\begin{itemize}
    \item For each ring, we randomly select a center in a rect centered at $(0, 0)$ with sides of length 1200.
    \item Each circle had a radius between 100 and 400.
    \item For each ring, we randomly select 100 samples in the circunference. For each sample, we add a noise with the following equation:
    \begin{equation}
        X_{\text{noise}} = X_{\text{ring}} + \text{randn}(0, 1) \cdot \text{noise\_level}
    \end{equation}
    Where $X_{\text{ring}}$ is the point in the circunference, and $\text{noise\_level}$ is the noise level, and $\text{randn}(0, 1)$ is a random number from a normal distribution with mean 0 and variance 1.
    \item To add the background noise, we select N points in the rect, sampled from an uniform distribution.
\end{itemize}

It is noteworthy to say that the algorithm is sensible to the different hyperparametrs.
As we can see in the results, both the runtime and performance degrades with higher noise and higher number of rings.
% tik picture
\begin{figure}[H]
    \centering
    \resizebox{0.9\linewidth}{!}{%% Creator: Matplotlib, PGF backend
%%
%% To include the figure in your LaTeX document, write
%%   \input{<filename>.pgf}
%%
%% Make sure the required packages are loaded in your preamble
%%   \usepackage{pgf}
%%
%% Also ensure that all the required font packages are loaded; for instance,
%% the lmodern package is sometimes necessary when using math font.
%%   \usepackage{lmodern}
%%
%% Figures using additional raster images can only be included by \input if
%% they are in the same directory as the main LaTeX file. For loading figures
%% from other directories you can use the `import` package
%%   \usepackage{import}
%%
%% and then include the figures with
%%   \import{<path to file>}{<filename>.pgf}
%%
%% Matplotlib used the following preamble
%%   \def\mathdefault#1{#1}
%%   \everymath=\expandafter{\the\everymath\displaystyle}
%%   
%%   \usepackage{fontspec}
%%   \setmainfont{DejaVuSerif.ttf}[Path=\detokenize{C:/Users/dagom/anaconda3/envs/pytorch/lib/site-packages/matplotlib/mpl-data/fonts/ttf/}]
%%   \setsansfont{DejaVuSans.ttf}[Path=\detokenize{C:/Users/dagom/anaconda3/envs/pytorch/lib/site-packages/matplotlib/mpl-data/fonts/ttf/}]
%%   \setmonofont{DejaVuSansMono.ttf}[Path=\detokenize{C:/Users/dagom/anaconda3/envs/pytorch/lib/site-packages/matplotlib/mpl-data/fonts/ttf/}]
%%   \makeatletter\@ifpackageloaded{underscore}{}{\usepackage[strings]{underscore}}\makeatother
%%
\begingroup%
\makeatletter%
\begin{pgfpicture}%
\pgfpathrectangle{\pgfpointorigin}{\pgfqpoint{6.400000in}{4.800000in}}%
\pgfusepath{use as bounding box, clip}%
\begin{pgfscope}%
\pgfsetbuttcap%
\pgfsetmiterjoin%
\definecolor{currentfill}{rgb}{1.000000,1.000000,1.000000}%
\pgfsetfillcolor{currentfill}%
\pgfsetlinewidth{0.000000pt}%
\definecolor{currentstroke}{rgb}{1.000000,1.000000,1.000000}%
\pgfsetstrokecolor{currentstroke}%
\pgfsetdash{}{0pt}%
\pgfpathmoveto{\pgfqpoint{0.000000in}{0.000000in}}%
\pgfpathlineto{\pgfqpoint{6.400000in}{0.000000in}}%
\pgfpathlineto{\pgfqpoint{6.400000in}{4.800000in}}%
\pgfpathlineto{\pgfqpoint{0.000000in}{4.800000in}}%
\pgfpathlineto{\pgfqpoint{0.000000in}{0.000000in}}%
\pgfpathclose%
\pgfusepath{fill}%
\end{pgfscope}%
\begin{pgfscope}%
\pgfsetbuttcap%
\pgfsetmiterjoin%
\definecolor{currentfill}{rgb}{1.000000,1.000000,1.000000}%
\pgfsetfillcolor{currentfill}%
\pgfsetlinewidth{0.000000pt}%
\definecolor{currentstroke}{rgb}{0.000000,0.000000,0.000000}%
\pgfsetstrokecolor{currentstroke}%
\pgfsetstrokeopacity{0.000000}%
\pgfsetdash{}{0pt}%
\pgfpathmoveto{\pgfqpoint{0.997489in}{0.528000in}}%
\pgfpathlineto{\pgfqpoint{5.562511in}{0.528000in}}%
\pgfpathlineto{\pgfqpoint{5.562511in}{4.224000in}}%
\pgfpathlineto{\pgfqpoint{0.997489in}{4.224000in}}%
\pgfpathlineto{\pgfqpoint{0.997489in}{0.528000in}}%
\pgfpathclose%
\pgfusepath{fill}%
\end{pgfscope}%
\begin{pgfscope}%
\pgfpathrectangle{\pgfqpoint{0.997489in}{0.528000in}}{\pgfqpoint{4.565023in}{3.696000in}}%
\pgfusepath{clip}%
\pgfsetbuttcap%
\pgfsetroundjoin%
\definecolor{currentfill}{rgb}{0.800000,0.800000,0.200000}%
\pgfsetfillcolor{currentfill}%
\pgfsetlinewidth{1.003750pt}%
\definecolor{currentstroke}{rgb}{0.800000,0.800000,0.200000}%
\pgfsetstrokecolor{currentstroke}%
\pgfsetdash{}{0pt}%
\pgfpathmoveto{\pgfqpoint{4.255958in}{2.297347in}}%
\pgfpathcurveto{\pgfqpoint{4.261782in}{2.297347in}}{\pgfqpoint{4.267368in}{2.299661in}}{\pgfqpoint{4.271486in}{2.303779in}}%
\pgfpathcurveto{\pgfqpoint{4.275604in}{2.307898in}}{\pgfqpoint{4.277918in}{2.313484in}}{\pgfqpoint{4.277918in}{2.319308in}}%
\pgfpathcurveto{\pgfqpoint{4.277918in}{2.325132in}}{\pgfqpoint{4.275604in}{2.330718in}}{\pgfqpoint{4.271486in}{2.334836in}}%
\pgfpathcurveto{\pgfqpoint{4.267368in}{2.338954in}}{\pgfqpoint{4.261782in}{2.341268in}}{\pgfqpoint{4.255958in}{2.341268in}}%
\pgfpathcurveto{\pgfqpoint{4.250134in}{2.341268in}}{\pgfqpoint{4.244548in}{2.338954in}}{\pgfqpoint{4.240430in}{2.334836in}}%
\pgfpathcurveto{\pgfqpoint{4.236311in}{2.330718in}}{\pgfqpoint{4.233998in}{2.325132in}}{\pgfqpoint{4.233998in}{2.319308in}}%
\pgfpathcurveto{\pgfqpoint{4.233998in}{2.313484in}}{\pgfqpoint{4.236311in}{2.307898in}}{\pgfqpoint{4.240430in}{2.303779in}}%
\pgfpathcurveto{\pgfqpoint{4.244548in}{2.299661in}}{\pgfqpoint{4.250134in}{2.297347in}}{\pgfqpoint{4.255958in}{2.297347in}}%
\pgfpathlineto{\pgfqpoint{4.255958in}{2.297347in}}%
\pgfpathclose%
\pgfusepath{stroke,fill}%
\end{pgfscope}%
\begin{pgfscope}%
\pgfpathrectangle{\pgfqpoint{0.997489in}{0.528000in}}{\pgfqpoint{4.565023in}{3.696000in}}%
\pgfusepath{clip}%
\pgfsetbuttcap%
\pgfsetroundjoin%
\definecolor{currentfill}{rgb}{0.800000,0.200000,0.200000}%
\pgfsetfillcolor{currentfill}%
\pgfsetlinewidth{1.003750pt}%
\definecolor{currentstroke}{rgb}{0.800000,0.200000,0.200000}%
\pgfsetstrokecolor{currentstroke}%
\pgfsetdash{}{0pt}%
\pgfpathmoveto{\pgfqpoint{4.189490in}{2.359150in}}%
\pgfpathcurveto{\pgfqpoint{4.195314in}{2.359150in}}{\pgfqpoint{4.200900in}{2.361464in}}{\pgfqpoint{4.205019in}{2.365582in}}%
\pgfpathcurveto{\pgfqpoint{4.209137in}{2.369700in}}{\pgfqpoint{4.211451in}{2.375287in}}{\pgfqpoint{4.211451in}{2.381110in}}%
\pgfpathcurveto{\pgfqpoint{4.211451in}{2.386934in}}{\pgfqpoint{4.209137in}{2.392521in}}{\pgfqpoint{4.205019in}{2.396639in}}%
\pgfpathcurveto{\pgfqpoint{4.200900in}{2.400757in}}{\pgfqpoint{4.195314in}{2.403071in}}{\pgfqpoint{4.189490in}{2.403071in}}%
\pgfpathcurveto{\pgfqpoint{4.183666in}{2.403071in}}{\pgfqpoint{4.178080in}{2.400757in}}{\pgfqpoint{4.173962in}{2.396639in}}%
\pgfpathcurveto{\pgfqpoint{4.169844in}{2.392521in}}{\pgfqpoint{4.167530in}{2.386934in}}{\pgfqpoint{4.167530in}{2.381110in}}%
\pgfpathcurveto{\pgfqpoint{4.167530in}{2.375287in}}{\pgfqpoint{4.169844in}{2.369700in}}{\pgfqpoint{4.173962in}{2.365582in}}%
\pgfpathcurveto{\pgfqpoint{4.178080in}{2.361464in}}{\pgfqpoint{4.183666in}{2.359150in}}{\pgfqpoint{4.189490in}{2.359150in}}%
\pgfpathlineto{\pgfqpoint{4.189490in}{2.359150in}}%
\pgfpathclose%
\pgfusepath{stroke,fill}%
\end{pgfscope}%
\begin{pgfscope}%
\pgfpathrectangle{\pgfqpoint{0.997489in}{0.528000in}}{\pgfqpoint{4.565023in}{3.696000in}}%
\pgfusepath{clip}%
\pgfsetbuttcap%
\pgfsetroundjoin%
\definecolor{currentfill}{rgb}{0.800000,0.200000,0.200000}%
\pgfsetfillcolor{currentfill}%
\pgfsetlinewidth{1.003750pt}%
\definecolor{currentstroke}{rgb}{0.800000,0.200000,0.200000}%
\pgfsetstrokecolor{currentstroke}%
\pgfsetdash{}{0pt}%
\pgfpathmoveto{\pgfqpoint{4.291317in}{2.434449in}}%
\pgfpathcurveto{\pgfqpoint{4.297141in}{2.434449in}}{\pgfqpoint{4.302727in}{2.436763in}}{\pgfqpoint{4.306846in}{2.440881in}}%
\pgfpathcurveto{\pgfqpoint{4.310964in}{2.444999in}}{\pgfqpoint{4.313278in}{2.450586in}}{\pgfqpoint{4.313278in}{2.456410in}}%
\pgfpathcurveto{\pgfqpoint{4.313278in}{2.462233in}}{\pgfqpoint{4.310964in}{2.467820in}}{\pgfqpoint{4.306846in}{2.471938in}}%
\pgfpathcurveto{\pgfqpoint{4.302727in}{2.476056in}}{\pgfqpoint{4.297141in}{2.478370in}}{\pgfqpoint{4.291317in}{2.478370in}}%
\pgfpathcurveto{\pgfqpoint{4.285493in}{2.478370in}}{\pgfqpoint{4.279907in}{2.476056in}}{\pgfqpoint{4.275789in}{2.471938in}}%
\pgfpathcurveto{\pgfqpoint{4.271671in}{2.467820in}}{\pgfqpoint{4.269357in}{2.462233in}}{\pgfqpoint{4.269357in}{2.456410in}}%
\pgfpathcurveto{\pgfqpoint{4.269357in}{2.450586in}}{\pgfqpoint{4.271671in}{2.444999in}}{\pgfqpoint{4.275789in}{2.440881in}}%
\pgfpathcurveto{\pgfqpoint{4.279907in}{2.436763in}}{\pgfqpoint{4.285493in}{2.434449in}}{\pgfqpoint{4.291317in}{2.434449in}}%
\pgfpathlineto{\pgfqpoint{4.291317in}{2.434449in}}%
\pgfpathclose%
\pgfusepath{stroke,fill}%
\end{pgfscope}%
\begin{pgfscope}%
\pgfpathrectangle{\pgfqpoint{0.997489in}{0.528000in}}{\pgfqpoint{4.565023in}{3.696000in}}%
\pgfusepath{clip}%
\pgfsetbuttcap%
\pgfsetroundjoin%
\definecolor{currentfill}{rgb}{0.800000,0.200000,0.200000}%
\pgfsetfillcolor{currentfill}%
\pgfsetlinewidth{1.003750pt}%
\definecolor{currentstroke}{rgb}{0.800000,0.200000,0.200000}%
\pgfsetstrokecolor{currentstroke}%
\pgfsetdash{}{0pt}%
\pgfpathmoveto{\pgfqpoint{4.223145in}{2.491263in}}%
\pgfpathcurveto{\pgfqpoint{4.228968in}{2.491263in}}{\pgfqpoint{4.234555in}{2.493577in}}{\pgfqpoint{4.238673in}{2.497695in}}%
\pgfpathcurveto{\pgfqpoint{4.242791in}{2.501814in}}{\pgfqpoint{4.245105in}{2.507400in}}{\pgfqpoint{4.245105in}{2.513224in}}%
\pgfpathcurveto{\pgfqpoint{4.245105in}{2.519048in}}{\pgfqpoint{4.242791in}{2.524634in}}{\pgfqpoint{4.238673in}{2.528752in}}%
\pgfpathcurveto{\pgfqpoint{4.234555in}{2.532870in}}{\pgfqpoint{4.228968in}{2.535184in}}{\pgfqpoint{4.223145in}{2.535184in}}%
\pgfpathcurveto{\pgfqpoint{4.217321in}{2.535184in}}{\pgfqpoint{4.211734in}{2.532870in}}{\pgfqpoint{4.207616in}{2.528752in}}%
\pgfpathcurveto{\pgfqpoint{4.203498in}{2.524634in}}{\pgfqpoint{4.201184in}{2.519048in}}{\pgfqpoint{4.201184in}{2.513224in}}%
\pgfpathcurveto{\pgfqpoint{4.201184in}{2.507400in}}{\pgfqpoint{4.203498in}{2.501814in}}{\pgfqpoint{4.207616in}{2.497695in}}%
\pgfpathcurveto{\pgfqpoint{4.211734in}{2.493577in}}{\pgfqpoint{4.217321in}{2.491263in}}{\pgfqpoint{4.223145in}{2.491263in}}%
\pgfpathlineto{\pgfqpoint{4.223145in}{2.491263in}}%
\pgfpathclose%
\pgfusepath{stroke,fill}%
\end{pgfscope}%
\begin{pgfscope}%
\pgfpathrectangle{\pgfqpoint{0.997489in}{0.528000in}}{\pgfqpoint{4.565023in}{3.696000in}}%
\pgfusepath{clip}%
\pgfsetbuttcap%
\pgfsetroundjoin%
\definecolor{currentfill}{rgb}{0.800000,0.200000,0.200000}%
\pgfsetfillcolor{currentfill}%
\pgfsetlinewidth{1.003750pt}%
\definecolor{currentstroke}{rgb}{0.800000,0.200000,0.200000}%
\pgfsetstrokecolor{currentstroke}%
\pgfsetdash{}{0pt}%
\pgfpathmoveto{\pgfqpoint{4.168580in}{2.544245in}}%
\pgfpathcurveto{\pgfqpoint{4.174404in}{2.544245in}}{\pgfqpoint{4.179990in}{2.546559in}}{\pgfqpoint{4.184109in}{2.550677in}}%
\pgfpathcurveto{\pgfqpoint{4.188227in}{2.554795in}}{\pgfqpoint{4.190541in}{2.560381in}}{\pgfqpoint{4.190541in}{2.566205in}}%
\pgfpathcurveto{\pgfqpoint{4.190541in}{2.572029in}}{\pgfqpoint{4.188227in}{2.577615in}}{\pgfqpoint{4.184109in}{2.581734in}}%
\pgfpathcurveto{\pgfqpoint{4.179990in}{2.585852in}}{\pgfqpoint{4.174404in}{2.588166in}}{\pgfqpoint{4.168580in}{2.588166in}}%
\pgfpathcurveto{\pgfqpoint{4.162756in}{2.588166in}}{\pgfqpoint{4.157170in}{2.585852in}}{\pgfqpoint{4.153052in}{2.581734in}}%
\pgfpathcurveto{\pgfqpoint{4.148934in}{2.577615in}}{\pgfqpoint{4.146620in}{2.572029in}}{\pgfqpoint{4.146620in}{2.566205in}}%
\pgfpathcurveto{\pgfqpoint{4.146620in}{2.560381in}}{\pgfqpoint{4.148934in}{2.554795in}}{\pgfqpoint{4.153052in}{2.550677in}}%
\pgfpathcurveto{\pgfqpoint{4.157170in}{2.546559in}}{\pgfqpoint{4.162756in}{2.544245in}}{\pgfqpoint{4.168580in}{2.544245in}}%
\pgfpathlineto{\pgfqpoint{4.168580in}{2.544245in}}%
\pgfpathclose%
\pgfusepath{stroke,fill}%
\end{pgfscope}%
\begin{pgfscope}%
\pgfpathrectangle{\pgfqpoint{0.997489in}{0.528000in}}{\pgfqpoint{4.565023in}{3.696000in}}%
\pgfusepath{clip}%
\pgfsetbuttcap%
\pgfsetroundjoin%
\definecolor{currentfill}{rgb}{0.800000,0.200000,0.200000}%
\pgfsetfillcolor{currentfill}%
\pgfsetlinewidth{1.003750pt}%
\definecolor{currentstroke}{rgb}{0.800000,0.200000,0.200000}%
\pgfsetstrokecolor{currentstroke}%
\pgfsetdash{}{0pt}%
\pgfpathmoveto{\pgfqpoint{4.208591in}{2.623013in}}%
\pgfpathcurveto{\pgfqpoint{4.214415in}{2.623013in}}{\pgfqpoint{4.220001in}{2.625327in}}{\pgfqpoint{4.224120in}{2.629445in}}%
\pgfpathcurveto{\pgfqpoint{4.228238in}{2.633563in}}{\pgfqpoint{4.230552in}{2.639149in}}{\pgfqpoint{4.230552in}{2.644973in}}%
\pgfpathcurveto{\pgfqpoint{4.230552in}{2.650797in}}{\pgfqpoint{4.228238in}{2.656384in}}{\pgfqpoint{4.224120in}{2.660502in}}%
\pgfpathcurveto{\pgfqpoint{4.220001in}{2.664620in}}{\pgfqpoint{4.214415in}{2.666934in}}{\pgfqpoint{4.208591in}{2.666934in}}%
\pgfpathcurveto{\pgfqpoint{4.202767in}{2.666934in}}{\pgfqpoint{4.197181in}{2.664620in}}{\pgfqpoint{4.193063in}{2.660502in}}%
\pgfpathcurveto{\pgfqpoint{4.188945in}{2.656384in}}{\pgfqpoint{4.186631in}{2.650797in}}{\pgfqpoint{4.186631in}{2.644973in}}%
\pgfpathcurveto{\pgfqpoint{4.186631in}{2.639149in}}{\pgfqpoint{4.188945in}{2.633563in}}{\pgfqpoint{4.193063in}{2.629445in}}%
\pgfpathcurveto{\pgfqpoint{4.197181in}{2.625327in}}{\pgfqpoint{4.202767in}{2.623013in}}{\pgfqpoint{4.208591in}{2.623013in}}%
\pgfpathlineto{\pgfqpoint{4.208591in}{2.623013in}}%
\pgfpathclose%
\pgfusepath{stroke,fill}%
\end{pgfscope}%
\begin{pgfscope}%
\pgfpathrectangle{\pgfqpoint{0.997489in}{0.528000in}}{\pgfqpoint{4.565023in}{3.696000in}}%
\pgfusepath{clip}%
\pgfsetbuttcap%
\pgfsetroundjoin%
\definecolor{currentfill}{rgb}{0.800000,0.200000,0.200000}%
\pgfsetfillcolor{currentfill}%
\pgfsetlinewidth{1.003750pt}%
\definecolor{currentstroke}{rgb}{0.800000,0.200000,0.200000}%
\pgfsetstrokecolor{currentstroke}%
\pgfsetdash{}{0pt}%
\pgfpathmoveto{\pgfqpoint{4.211263in}{2.695385in}}%
\pgfpathcurveto{\pgfqpoint{4.217087in}{2.695385in}}{\pgfqpoint{4.222673in}{2.697699in}}{\pgfqpoint{4.226792in}{2.701817in}}%
\pgfpathcurveto{\pgfqpoint{4.230910in}{2.705935in}}{\pgfqpoint{4.233224in}{2.711522in}}{\pgfqpoint{4.233224in}{2.717346in}}%
\pgfpathcurveto{\pgfqpoint{4.233224in}{2.723169in}}{\pgfqpoint{4.230910in}{2.728756in}}{\pgfqpoint{4.226792in}{2.732874in}}%
\pgfpathcurveto{\pgfqpoint{4.222673in}{2.736992in}}{\pgfqpoint{4.217087in}{2.739306in}}{\pgfqpoint{4.211263in}{2.739306in}}%
\pgfpathcurveto{\pgfqpoint{4.205439in}{2.739306in}}{\pgfqpoint{4.199853in}{2.736992in}}{\pgfqpoint{4.195735in}{2.732874in}}%
\pgfpathcurveto{\pgfqpoint{4.191617in}{2.728756in}}{\pgfqpoint{4.189303in}{2.723169in}}{\pgfqpoint{4.189303in}{2.717346in}}%
\pgfpathcurveto{\pgfqpoint{4.189303in}{2.711522in}}{\pgfqpoint{4.191617in}{2.705935in}}{\pgfqpoint{4.195735in}{2.701817in}}%
\pgfpathcurveto{\pgfqpoint{4.199853in}{2.697699in}}{\pgfqpoint{4.205439in}{2.695385in}}{\pgfqpoint{4.211263in}{2.695385in}}%
\pgfpathlineto{\pgfqpoint{4.211263in}{2.695385in}}%
\pgfpathclose%
\pgfusepath{stroke,fill}%
\end{pgfscope}%
\begin{pgfscope}%
\pgfpathrectangle{\pgfqpoint{0.997489in}{0.528000in}}{\pgfqpoint{4.565023in}{3.696000in}}%
\pgfusepath{clip}%
\pgfsetbuttcap%
\pgfsetroundjoin%
\definecolor{currentfill}{rgb}{0.800000,0.200000,0.200000}%
\pgfsetfillcolor{currentfill}%
\pgfsetlinewidth{1.003750pt}%
\definecolor{currentstroke}{rgb}{0.800000,0.200000,0.200000}%
\pgfsetstrokecolor{currentstroke}%
\pgfsetdash{}{0pt}%
\pgfpathmoveto{\pgfqpoint{4.154197in}{2.743449in}}%
\pgfpathcurveto{\pgfqpoint{4.160021in}{2.743449in}}{\pgfqpoint{4.165607in}{2.745763in}}{\pgfqpoint{4.169725in}{2.749881in}}%
\pgfpathcurveto{\pgfqpoint{4.173844in}{2.753999in}}{\pgfqpoint{4.176157in}{2.759585in}}{\pgfqpoint{4.176157in}{2.765409in}}%
\pgfpathcurveto{\pgfqpoint{4.176157in}{2.771233in}}{\pgfqpoint{4.173844in}{2.776819in}}{\pgfqpoint{4.169725in}{2.780938in}}%
\pgfpathcurveto{\pgfqpoint{4.165607in}{2.785056in}}{\pgfqpoint{4.160021in}{2.787370in}}{\pgfqpoint{4.154197in}{2.787370in}}%
\pgfpathcurveto{\pgfqpoint{4.148373in}{2.787370in}}{\pgfqpoint{4.142787in}{2.785056in}}{\pgfqpoint{4.138669in}{2.780938in}}%
\pgfpathcurveto{\pgfqpoint{4.134551in}{2.776819in}}{\pgfqpoint{4.132237in}{2.771233in}}{\pgfqpoint{4.132237in}{2.765409in}}%
\pgfpathcurveto{\pgfqpoint{4.132237in}{2.759585in}}{\pgfqpoint{4.134551in}{2.753999in}}{\pgfqpoint{4.138669in}{2.749881in}}%
\pgfpathcurveto{\pgfqpoint{4.142787in}{2.745763in}}{\pgfqpoint{4.148373in}{2.743449in}}{\pgfqpoint{4.154197in}{2.743449in}}%
\pgfpathlineto{\pgfqpoint{4.154197in}{2.743449in}}%
\pgfpathclose%
\pgfusepath{stroke,fill}%
\end{pgfscope}%
\begin{pgfscope}%
\pgfpathrectangle{\pgfqpoint{0.997489in}{0.528000in}}{\pgfqpoint{4.565023in}{3.696000in}}%
\pgfusepath{clip}%
\pgfsetbuttcap%
\pgfsetroundjoin%
\definecolor{currentfill}{rgb}{0.800000,0.200000,0.200000}%
\pgfsetfillcolor{currentfill}%
\pgfsetlinewidth{1.003750pt}%
\definecolor{currentstroke}{rgb}{0.800000,0.200000,0.200000}%
\pgfsetstrokecolor{currentstroke}%
\pgfsetdash{}{0pt}%
\pgfpathmoveto{\pgfqpoint{4.090339in}{2.783253in}}%
\pgfpathcurveto{\pgfqpoint{4.096163in}{2.783253in}}{\pgfqpoint{4.101749in}{2.785567in}}{\pgfqpoint{4.105867in}{2.789685in}}%
\pgfpathcurveto{\pgfqpoint{4.109986in}{2.793803in}}{\pgfqpoint{4.112299in}{2.799389in}}{\pgfqpoint{4.112299in}{2.805213in}}%
\pgfpathcurveto{\pgfqpoint{4.112299in}{2.811037in}}{\pgfqpoint{4.109986in}{2.816623in}}{\pgfqpoint{4.105867in}{2.820742in}}%
\pgfpathcurveto{\pgfqpoint{4.101749in}{2.824860in}}{\pgfqpoint{4.096163in}{2.827174in}}{\pgfqpoint{4.090339in}{2.827174in}}%
\pgfpathcurveto{\pgfqpoint{4.084515in}{2.827174in}}{\pgfqpoint{4.078929in}{2.824860in}}{\pgfqpoint{4.074811in}{2.820742in}}%
\pgfpathcurveto{\pgfqpoint{4.070693in}{2.816623in}}{\pgfqpoint{4.068379in}{2.811037in}}{\pgfqpoint{4.068379in}{2.805213in}}%
\pgfpathcurveto{\pgfqpoint{4.068379in}{2.799389in}}{\pgfqpoint{4.070693in}{2.793803in}}{\pgfqpoint{4.074811in}{2.789685in}}%
\pgfpathcurveto{\pgfqpoint{4.078929in}{2.785567in}}{\pgfqpoint{4.084515in}{2.783253in}}{\pgfqpoint{4.090339in}{2.783253in}}%
\pgfpathlineto{\pgfqpoint{4.090339in}{2.783253in}}%
\pgfpathclose%
\pgfusepath{stroke,fill}%
\end{pgfscope}%
\begin{pgfscope}%
\pgfpathrectangle{\pgfqpoint{0.997489in}{0.528000in}}{\pgfqpoint{4.565023in}{3.696000in}}%
\pgfusepath{clip}%
\pgfsetbuttcap%
\pgfsetroundjoin%
\definecolor{currentfill}{rgb}{0.800000,0.200000,0.200000}%
\pgfsetfillcolor{currentfill}%
\pgfsetlinewidth{1.003750pt}%
\definecolor{currentstroke}{rgb}{0.800000,0.200000,0.200000}%
\pgfsetstrokecolor{currentstroke}%
\pgfsetdash{}{0pt}%
\pgfpathmoveto{\pgfqpoint{4.041792in}{2.827401in}}%
\pgfpathcurveto{\pgfqpoint{4.047616in}{2.827401in}}{\pgfqpoint{4.053202in}{2.829715in}}{\pgfqpoint{4.057320in}{2.833833in}}%
\pgfpathcurveto{\pgfqpoint{4.061438in}{2.837951in}}{\pgfqpoint{4.063752in}{2.843537in}}{\pgfqpoint{4.063752in}{2.849361in}}%
\pgfpathcurveto{\pgfqpoint{4.063752in}{2.855185in}}{\pgfqpoint{4.061438in}{2.860771in}}{\pgfqpoint{4.057320in}{2.864889in}}%
\pgfpathcurveto{\pgfqpoint{4.053202in}{2.869007in}}{\pgfqpoint{4.047616in}{2.871321in}}{\pgfqpoint{4.041792in}{2.871321in}}%
\pgfpathcurveto{\pgfqpoint{4.035968in}{2.871321in}}{\pgfqpoint{4.030382in}{2.869007in}}{\pgfqpoint{4.026264in}{2.864889in}}%
\pgfpathcurveto{\pgfqpoint{4.022146in}{2.860771in}}{\pgfqpoint{4.019832in}{2.855185in}}{\pgfqpoint{4.019832in}{2.849361in}}%
\pgfpathcurveto{\pgfqpoint{4.019832in}{2.843537in}}{\pgfqpoint{4.022146in}{2.837951in}}{\pgfqpoint{4.026264in}{2.833833in}}%
\pgfpathcurveto{\pgfqpoint{4.030382in}{2.829715in}}{\pgfqpoint{4.035968in}{2.827401in}}{\pgfqpoint{4.041792in}{2.827401in}}%
\pgfpathlineto{\pgfqpoint{4.041792in}{2.827401in}}%
\pgfpathclose%
\pgfusepath{stroke,fill}%
\end{pgfscope}%
\begin{pgfscope}%
\pgfpathrectangle{\pgfqpoint{0.997489in}{0.528000in}}{\pgfqpoint{4.565023in}{3.696000in}}%
\pgfusepath{clip}%
\pgfsetbuttcap%
\pgfsetroundjoin%
\definecolor{currentfill}{rgb}{0.800000,0.200000,0.200000}%
\pgfsetfillcolor{currentfill}%
\pgfsetlinewidth{1.003750pt}%
\definecolor{currentstroke}{rgb}{0.800000,0.200000,0.200000}%
\pgfsetstrokecolor{currentstroke}%
\pgfsetdash{}{0pt}%
\pgfpathmoveto{\pgfqpoint{4.005700in}{2.878046in}}%
\pgfpathcurveto{\pgfqpoint{4.011524in}{2.878046in}}{\pgfqpoint{4.017111in}{2.880360in}}{\pgfqpoint{4.021229in}{2.884478in}}%
\pgfpathcurveto{\pgfqpoint{4.025347in}{2.888596in}}{\pgfqpoint{4.027661in}{2.894182in}}{\pgfqpoint{4.027661in}{2.900006in}}%
\pgfpathcurveto{\pgfqpoint{4.027661in}{2.905830in}}{\pgfqpoint{4.025347in}{2.911416in}}{\pgfqpoint{4.021229in}{2.915534in}}%
\pgfpathcurveto{\pgfqpoint{4.017111in}{2.919652in}}{\pgfqpoint{4.011524in}{2.921966in}}{\pgfqpoint{4.005700in}{2.921966in}}%
\pgfpathcurveto{\pgfqpoint{3.999876in}{2.921966in}}{\pgfqpoint{3.994290in}{2.919652in}}{\pgfqpoint{3.990172in}{2.915534in}}%
\pgfpathcurveto{\pgfqpoint{3.986054in}{2.911416in}}{\pgfqpoint{3.983740in}{2.905830in}}{\pgfqpoint{3.983740in}{2.900006in}}%
\pgfpathcurveto{\pgfqpoint{3.983740in}{2.894182in}}{\pgfqpoint{3.986054in}{2.888596in}}{\pgfqpoint{3.990172in}{2.884478in}}%
\pgfpathcurveto{\pgfqpoint{3.994290in}{2.880360in}}{\pgfqpoint{3.999876in}{2.878046in}}{\pgfqpoint{4.005700in}{2.878046in}}%
\pgfpathlineto{\pgfqpoint{4.005700in}{2.878046in}}%
\pgfpathclose%
\pgfusepath{stroke,fill}%
\end{pgfscope}%
\begin{pgfscope}%
\pgfpathrectangle{\pgfqpoint{0.997489in}{0.528000in}}{\pgfqpoint{4.565023in}{3.696000in}}%
\pgfusepath{clip}%
\pgfsetbuttcap%
\pgfsetroundjoin%
\definecolor{currentfill}{rgb}{0.800000,0.200000,0.200000}%
\pgfsetfillcolor{currentfill}%
\pgfsetlinewidth{1.003750pt}%
\definecolor{currentstroke}{rgb}{0.800000,0.200000,0.200000}%
\pgfsetstrokecolor{currentstroke}%
\pgfsetdash{}{0pt}%
\pgfpathmoveto{\pgfqpoint{4.031040in}{2.980398in}}%
\pgfpathcurveto{\pgfqpoint{4.036864in}{2.980398in}}{\pgfqpoint{4.042451in}{2.982711in}}{\pgfqpoint{4.046569in}{2.986830in}}%
\pgfpathcurveto{\pgfqpoint{4.050687in}{2.990948in}}{\pgfqpoint{4.053001in}{2.996534in}}{\pgfqpoint{4.053001in}{3.002358in}}%
\pgfpathcurveto{\pgfqpoint{4.053001in}{3.008182in}}{\pgfqpoint{4.050687in}{3.013768in}}{\pgfqpoint{4.046569in}{3.017886in}}%
\pgfpathcurveto{\pgfqpoint{4.042451in}{3.022004in}}{\pgfqpoint{4.036864in}{3.024318in}}{\pgfqpoint{4.031040in}{3.024318in}}%
\pgfpathcurveto{\pgfqpoint{4.025217in}{3.024318in}}{\pgfqpoint{4.019630in}{3.022004in}}{\pgfqpoint{4.015512in}{3.017886in}}%
\pgfpathcurveto{\pgfqpoint{4.011394in}{3.013768in}}{\pgfqpoint{4.009080in}{3.008182in}}{\pgfqpoint{4.009080in}{3.002358in}}%
\pgfpathcurveto{\pgfqpoint{4.009080in}{2.996534in}}{\pgfqpoint{4.011394in}{2.990948in}}{\pgfqpoint{4.015512in}{2.986830in}}%
\pgfpathcurveto{\pgfqpoint{4.019630in}{2.982711in}}{\pgfqpoint{4.025217in}{2.980398in}}{\pgfqpoint{4.031040in}{2.980398in}}%
\pgfpathlineto{\pgfqpoint{4.031040in}{2.980398in}}%
\pgfpathclose%
\pgfusepath{stroke,fill}%
\end{pgfscope}%
\begin{pgfscope}%
\pgfpathrectangle{\pgfqpoint{0.997489in}{0.528000in}}{\pgfqpoint{4.565023in}{3.696000in}}%
\pgfusepath{clip}%
\pgfsetbuttcap%
\pgfsetroundjoin%
\definecolor{currentfill}{rgb}{0.800000,0.200000,0.200000}%
\pgfsetfillcolor{currentfill}%
\pgfsetlinewidth{1.003750pt}%
\definecolor{currentstroke}{rgb}{0.800000,0.200000,0.200000}%
\pgfsetstrokecolor{currentstroke}%
\pgfsetdash{}{0pt}%
\pgfpathmoveto{\pgfqpoint{3.991635in}{3.035948in}}%
\pgfpathcurveto{\pgfqpoint{3.997459in}{3.035948in}}{\pgfqpoint{4.003045in}{3.038262in}}{\pgfqpoint{4.007163in}{3.042380in}}%
\pgfpathcurveto{\pgfqpoint{4.011281in}{3.046498in}}{\pgfqpoint{4.013595in}{3.052084in}}{\pgfqpoint{4.013595in}{3.057908in}}%
\pgfpathcurveto{\pgfqpoint{4.013595in}{3.063732in}}{\pgfqpoint{4.011281in}{3.069318in}}{\pgfqpoint{4.007163in}{3.073436in}}%
\pgfpathcurveto{\pgfqpoint{4.003045in}{3.077554in}}{\pgfqpoint{3.997459in}{3.079868in}}{\pgfqpoint{3.991635in}{3.079868in}}%
\pgfpathcurveto{\pgfqpoint{3.985811in}{3.079868in}}{\pgfqpoint{3.980225in}{3.077554in}}{\pgfqpoint{3.976107in}{3.073436in}}%
\pgfpathcurveto{\pgfqpoint{3.971988in}{3.069318in}}{\pgfqpoint{3.969674in}{3.063732in}}{\pgfqpoint{3.969674in}{3.057908in}}%
\pgfpathcurveto{\pgfqpoint{3.969674in}{3.052084in}}{\pgfqpoint{3.971988in}{3.046498in}}{\pgfqpoint{3.976107in}{3.042380in}}%
\pgfpathcurveto{\pgfqpoint{3.980225in}{3.038262in}}{\pgfqpoint{3.985811in}{3.035948in}}{\pgfqpoint{3.991635in}{3.035948in}}%
\pgfpathlineto{\pgfqpoint{3.991635in}{3.035948in}}%
\pgfpathclose%
\pgfusepath{stroke,fill}%
\end{pgfscope}%
\begin{pgfscope}%
\pgfpathrectangle{\pgfqpoint{0.997489in}{0.528000in}}{\pgfqpoint{4.565023in}{3.696000in}}%
\pgfusepath{clip}%
\pgfsetbuttcap%
\pgfsetroundjoin%
\definecolor{currentfill}{rgb}{0.800000,0.200000,0.200000}%
\pgfsetfillcolor{currentfill}%
\pgfsetlinewidth{1.003750pt}%
\definecolor{currentstroke}{rgb}{0.800000,0.200000,0.200000}%
\pgfsetstrokecolor{currentstroke}%
\pgfsetdash{}{0pt}%
\pgfpathmoveto{\pgfqpoint{3.901286in}{3.038179in}}%
\pgfpathcurveto{\pgfqpoint{3.907110in}{3.038179in}}{\pgfqpoint{3.912696in}{3.040493in}}{\pgfqpoint{3.916814in}{3.044611in}}%
\pgfpathcurveto{\pgfqpoint{3.920932in}{3.048729in}}{\pgfqpoint{3.923246in}{3.054315in}}{\pgfqpoint{3.923246in}{3.060139in}}%
\pgfpathcurveto{\pgfqpoint{3.923246in}{3.065963in}}{\pgfqpoint{3.920932in}{3.071549in}}{\pgfqpoint{3.916814in}{3.075667in}}%
\pgfpathcurveto{\pgfqpoint{3.912696in}{3.079785in}}{\pgfqpoint{3.907110in}{3.082099in}}{\pgfqpoint{3.901286in}{3.082099in}}%
\pgfpathcurveto{\pgfqpoint{3.895462in}{3.082099in}}{\pgfqpoint{3.889876in}{3.079785in}}{\pgfqpoint{3.885757in}{3.075667in}}%
\pgfpathcurveto{\pgfqpoint{3.881639in}{3.071549in}}{\pgfqpoint{3.879325in}{3.065963in}}{\pgfqpoint{3.879325in}{3.060139in}}%
\pgfpathcurveto{\pgfqpoint{3.879325in}{3.054315in}}{\pgfqpoint{3.881639in}{3.048729in}}{\pgfqpoint{3.885757in}{3.044611in}}%
\pgfpathcurveto{\pgfqpoint{3.889876in}{3.040493in}}{\pgfqpoint{3.895462in}{3.038179in}}{\pgfqpoint{3.901286in}{3.038179in}}%
\pgfpathlineto{\pgfqpoint{3.901286in}{3.038179in}}%
\pgfpathclose%
\pgfusepath{stroke,fill}%
\end{pgfscope}%
\begin{pgfscope}%
\pgfpathrectangle{\pgfqpoint{0.997489in}{0.528000in}}{\pgfqpoint{4.565023in}{3.696000in}}%
\pgfusepath{clip}%
\pgfsetbuttcap%
\pgfsetroundjoin%
\definecolor{currentfill}{rgb}{0.800000,0.200000,0.200000}%
\pgfsetfillcolor{currentfill}%
\pgfsetlinewidth{1.003750pt}%
\definecolor{currentstroke}{rgb}{0.800000,0.200000,0.200000}%
\pgfsetstrokecolor{currentstroke}%
\pgfsetdash{}{0pt}%
\pgfpathmoveto{\pgfqpoint{3.937307in}{3.183957in}}%
\pgfpathcurveto{\pgfqpoint{3.943131in}{3.183957in}}{\pgfqpoint{3.948718in}{3.186271in}}{\pgfqpoint{3.952836in}{3.190389in}}%
\pgfpathcurveto{\pgfqpoint{3.956954in}{3.194507in}}{\pgfqpoint{3.959268in}{3.200093in}}{\pgfqpoint{3.959268in}{3.205917in}}%
\pgfpathcurveto{\pgfqpoint{3.959268in}{3.211741in}}{\pgfqpoint{3.956954in}{3.217327in}}{\pgfqpoint{3.952836in}{3.221445in}}%
\pgfpathcurveto{\pgfqpoint{3.948718in}{3.225564in}}{\pgfqpoint{3.943131in}{3.227877in}}{\pgfqpoint{3.937307in}{3.227877in}}%
\pgfpathcurveto{\pgfqpoint{3.931484in}{3.227877in}}{\pgfqpoint{3.925897in}{3.225564in}}{\pgfqpoint{3.921779in}{3.221445in}}%
\pgfpathcurveto{\pgfqpoint{3.917661in}{3.217327in}}{\pgfqpoint{3.915347in}{3.211741in}}{\pgfqpoint{3.915347in}{3.205917in}}%
\pgfpathcurveto{\pgfqpoint{3.915347in}{3.200093in}}{\pgfqpoint{3.917661in}{3.194507in}}{\pgfqpoint{3.921779in}{3.190389in}}%
\pgfpathcurveto{\pgfqpoint{3.925897in}{3.186271in}}{\pgfqpoint{3.931484in}{3.183957in}}{\pgfqpoint{3.937307in}{3.183957in}}%
\pgfpathlineto{\pgfqpoint{3.937307in}{3.183957in}}%
\pgfpathclose%
\pgfusepath{stroke,fill}%
\end{pgfscope}%
\begin{pgfscope}%
\pgfpathrectangle{\pgfqpoint{0.997489in}{0.528000in}}{\pgfqpoint{4.565023in}{3.696000in}}%
\pgfusepath{clip}%
\pgfsetbuttcap%
\pgfsetroundjoin%
\definecolor{currentfill}{rgb}{0.800000,0.200000,0.200000}%
\pgfsetfillcolor{currentfill}%
\pgfsetlinewidth{1.003750pt}%
\definecolor{currentstroke}{rgb}{0.800000,0.200000,0.200000}%
\pgfsetstrokecolor{currentstroke}%
\pgfsetdash{}{0pt}%
\pgfpathmoveto{\pgfqpoint{3.802395in}{3.119401in}}%
\pgfpathcurveto{\pgfqpoint{3.808219in}{3.119401in}}{\pgfqpoint{3.813805in}{3.121715in}}{\pgfqpoint{3.817923in}{3.125833in}}%
\pgfpathcurveto{\pgfqpoint{3.822041in}{3.129951in}}{\pgfqpoint{3.824355in}{3.135538in}}{\pgfqpoint{3.824355in}{3.141361in}}%
\pgfpathcurveto{\pgfqpoint{3.824355in}{3.147185in}}{\pgfqpoint{3.822041in}{3.152772in}}{\pgfqpoint{3.817923in}{3.156890in}}%
\pgfpathcurveto{\pgfqpoint{3.813805in}{3.161008in}}{\pgfqpoint{3.808219in}{3.163322in}}{\pgfqpoint{3.802395in}{3.163322in}}%
\pgfpathcurveto{\pgfqpoint{3.796571in}{3.163322in}}{\pgfqpoint{3.790985in}{3.161008in}}{\pgfqpoint{3.786867in}{3.156890in}}%
\pgfpathcurveto{\pgfqpoint{3.782749in}{3.152772in}}{\pgfqpoint{3.780435in}{3.147185in}}{\pgfqpoint{3.780435in}{3.141361in}}%
\pgfpathcurveto{\pgfqpoint{3.780435in}{3.135538in}}{\pgfqpoint{3.782749in}{3.129951in}}{\pgfqpoint{3.786867in}{3.125833in}}%
\pgfpathcurveto{\pgfqpoint{3.790985in}{3.121715in}}{\pgfqpoint{3.796571in}{3.119401in}}{\pgfqpoint{3.802395in}{3.119401in}}%
\pgfpathlineto{\pgfqpoint{3.802395in}{3.119401in}}%
\pgfpathclose%
\pgfusepath{stroke,fill}%
\end{pgfscope}%
\begin{pgfscope}%
\pgfpathrectangle{\pgfqpoint{0.997489in}{0.528000in}}{\pgfqpoint{4.565023in}{3.696000in}}%
\pgfusepath{clip}%
\pgfsetbuttcap%
\pgfsetroundjoin%
\definecolor{currentfill}{rgb}{0.800000,0.200000,0.200000}%
\pgfsetfillcolor{currentfill}%
\pgfsetlinewidth{1.003750pt}%
\definecolor{currentstroke}{rgb}{0.800000,0.200000,0.200000}%
\pgfsetstrokecolor{currentstroke}%
\pgfsetdash{}{0pt}%
\pgfpathmoveto{\pgfqpoint{3.753535in}{3.162057in}}%
\pgfpathcurveto{\pgfqpoint{3.759359in}{3.162057in}}{\pgfqpoint{3.764946in}{3.164370in}}{\pgfqpoint{3.769064in}{3.168489in}}%
\pgfpathcurveto{\pgfqpoint{3.773182in}{3.172607in}}{\pgfqpoint{3.775496in}{3.178193in}}{\pgfqpoint{3.775496in}{3.184017in}}%
\pgfpathcurveto{\pgfqpoint{3.775496in}{3.189841in}}{\pgfqpoint{3.773182in}{3.195427in}}{\pgfqpoint{3.769064in}{3.199545in}}%
\pgfpathcurveto{\pgfqpoint{3.764946in}{3.203663in}}{\pgfqpoint{3.759359in}{3.205977in}}{\pgfqpoint{3.753535in}{3.205977in}}%
\pgfpathcurveto{\pgfqpoint{3.747711in}{3.205977in}}{\pgfqpoint{3.742125in}{3.203663in}}{\pgfqpoint{3.738007in}{3.199545in}}%
\pgfpathcurveto{\pgfqpoint{3.733889in}{3.195427in}}{\pgfqpoint{3.731575in}{3.189841in}}{\pgfqpoint{3.731575in}{3.184017in}}%
\pgfpathcurveto{\pgfqpoint{3.731575in}{3.178193in}}{\pgfqpoint{3.733889in}{3.172607in}}{\pgfqpoint{3.738007in}{3.168489in}}%
\pgfpathcurveto{\pgfqpoint{3.742125in}{3.164370in}}{\pgfqpoint{3.747711in}{3.162057in}}{\pgfqpoint{3.753535in}{3.162057in}}%
\pgfpathlineto{\pgfqpoint{3.753535in}{3.162057in}}%
\pgfpathclose%
\pgfusepath{stroke,fill}%
\end{pgfscope}%
\begin{pgfscope}%
\pgfpathrectangle{\pgfqpoint{0.997489in}{0.528000in}}{\pgfqpoint{4.565023in}{3.696000in}}%
\pgfusepath{clip}%
\pgfsetbuttcap%
\pgfsetroundjoin%
\definecolor{currentfill}{rgb}{0.800000,0.200000,0.200000}%
\pgfsetfillcolor{currentfill}%
\pgfsetlinewidth{1.003750pt}%
\definecolor{currentstroke}{rgb}{0.800000,0.200000,0.200000}%
\pgfsetstrokecolor{currentstroke}%
\pgfsetdash{}{0pt}%
\pgfpathmoveto{\pgfqpoint{3.689776in}{3.179718in}}%
\pgfpathcurveto{\pgfqpoint{3.695599in}{3.179718in}}{\pgfqpoint{3.701186in}{3.182032in}}{\pgfqpoint{3.705304in}{3.186150in}}%
\pgfpathcurveto{\pgfqpoint{3.709422in}{3.190268in}}{\pgfqpoint{3.711736in}{3.195854in}}{\pgfqpoint{3.711736in}{3.201678in}}%
\pgfpathcurveto{\pgfqpoint{3.711736in}{3.207502in}}{\pgfqpoint{3.709422in}{3.213088in}}{\pgfqpoint{3.705304in}{3.217206in}}%
\pgfpathcurveto{\pgfqpoint{3.701186in}{3.221325in}}{\pgfqpoint{3.695599in}{3.223638in}}{\pgfqpoint{3.689776in}{3.223638in}}%
\pgfpathcurveto{\pgfqpoint{3.683952in}{3.223638in}}{\pgfqpoint{3.678365in}{3.221325in}}{\pgfqpoint{3.674247in}{3.217206in}}%
\pgfpathcurveto{\pgfqpoint{3.670129in}{3.213088in}}{\pgfqpoint{3.667815in}{3.207502in}}{\pgfqpoint{3.667815in}{3.201678in}}%
\pgfpathcurveto{\pgfqpoint{3.667815in}{3.195854in}}{\pgfqpoint{3.670129in}{3.190268in}}{\pgfqpoint{3.674247in}{3.186150in}}%
\pgfpathcurveto{\pgfqpoint{3.678365in}{3.182032in}}{\pgfqpoint{3.683952in}{3.179718in}}{\pgfqpoint{3.689776in}{3.179718in}}%
\pgfpathlineto{\pgfqpoint{3.689776in}{3.179718in}}%
\pgfpathclose%
\pgfusepath{stroke,fill}%
\end{pgfscope}%
\begin{pgfscope}%
\pgfpathrectangle{\pgfqpoint{0.997489in}{0.528000in}}{\pgfqpoint{4.565023in}{3.696000in}}%
\pgfusepath{clip}%
\pgfsetbuttcap%
\pgfsetroundjoin%
\definecolor{currentfill}{rgb}{0.800000,0.800000,0.200000}%
\pgfsetfillcolor{currentfill}%
\pgfsetlinewidth{1.003750pt}%
\definecolor{currentstroke}{rgb}{0.800000,0.800000,0.200000}%
\pgfsetstrokecolor{currentstroke}%
\pgfsetdash{}{0pt}%
\pgfpathmoveto{\pgfqpoint{3.638843in}{3.221025in}}%
\pgfpathcurveto{\pgfqpoint{3.644667in}{3.221025in}}{\pgfqpoint{3.650253in}{3.223339in}}{\pgfqpoint{3.654371in}{3.227457in}}%
\pgfpathcurveto{\pgfqpoint{3.658489in}{3.231575in}}{\pgfqpoint{3.660803in}{3.237162in}}{\pgfqpoint{3.660803in}{3.242986in}}%
\pgfpathcurveto{\pgfqpoint{3.660803in}{3.248809in}}{\pgfqpoint{3.658489in}{3.254396in}}{\pgfqpoint{3.654371in}{3.258514in}}%
\pgfpathcurveto{\pgfqpoint{3.650253in}{3.262632in}}{\pgfqpoint{3.644667in}{3.264946in}}{\pgfqpoint{3.638843in}{3.264946in}}%
\pgfpathcurveto{\pgfqpoint{3.633019in}{3.264946in}}{\pgfqpoint{3.627432in}{3.262632in}}{\pgfqpoint{3.623314in}{3.258514in}}%
\pgfpathcurveto{\pgfqpoint{3.619196in}{3.254396in}}{\pgfqpoint{3.616882in}{3.248809in}}{\pgfqpoint{3.616882in}{3.242986in}}%
\pgfpathcurveto{\pgfqpoint{3.616882in}{3.237162in}}{\pgfqpoint{3.619196in}{3.231575in}}{\pgfqpoint{3.623314in}{3.227457in}}%
\pgfpathcurveto{\pgfqpoint{3.627432in}{3.223339in}}{\pgfqpoint{3.633019in}{3.221025in}}{\pgfqpoint{3.638843in}{3.221025in}}%
\pgfpathlineto{\pgfqpoint{3.638843in}{3.221025in}}%
\pgfpathclose%
\pgfusepath{stroke,fill}%
\end{pgfscope}%
\begin{pgfscope}%
\pgfpathrectangle{\pgfqpoint{0.997489in}{0.528000in}}{\pgfqpoint{4.565023in}{3.696000in}}%
\pgfusepath{clip}%
\pgfsetbuttcap%
\pgfsetroundjoin%
\definecolor{currentfill}{rgb}{0.800000,0.800000,0.200000}%
\pgfsetfillcolor{currentfill}%
\pgfsetlinewidth{1.003750pt}%
\definecolor{currentstroke}{rgb}{0.800000,0.800000,0.200000}%
\pgfsetstrokecolor{currentstroke}%
\pgfsetdash{}{0pt}%
\pgfpathmoveto{\pgfqpoint{3.580863in}{3.249723in}}%
\pgfpathcurveto{\pgfqpoint{3.586687in}{3.249723in}}{\pgfqpoint{3.592273in}{3.252037in}}{\pgfqpoint{3.596391in}{3.256155in}}%
\pgfpathcurveto{\pgfqpoint{3.600509in}{3.260273in}}{\pgfqpoint{3.602823in}{3.265859in}}{\pgfqpoint{3.602823in}{3.271683in}}%
\pgfpathcurveto{\pgfqpoint{3.602823in}{3.277507in}}{\pgfqpoint{3.600509in}{3.283093in}}{\pgfqpoint{3.596391in}{3.287211in}}%
\pgfpathcurveto{\pgfqpoint{3.592273in}{3.291329in}}{\pgfqpoint{3.586687in}{3.293643in}}{\pgfqpoint{3.580863in}{3.293643in}}%
\pgfpathcurveto{\pgfqpoint{3.575039in}{3.293643in}}{\pgfqpoint{3.569453in}{3.291329in}}{\pgfqpoint{3.565335in}{3.287211in}}%
\pgfpathcurveto{\pgfqpoint{3.561217in}{3.283093in}}{\pgfqpoint{3.558903in}{3.277507in}}{\pgfqpoint{3.558903in}{3.271683in}}%
\pgfpathcurveto{\pgfqpoint{3.558903in}{3.265859in}}{\pgfqpoint{3.561217in}{3.260273in}}{\pgfqpoint{3.565335in}{3.256155in}}%
\pgfpathcurveto{\pgfqpoint{3.569453in}{3.252037in}}{\pgfqpoint{3.575039in}{3.249723in}}{\pgfqpoint{3.580863in}{3.249723in}}%
\pgfpathlineto{\pgfqpoint{3.580863in}{3.249723in}}%
\pgfpathclose%
\pgfusepath{stroke,fill}%
\end{pgfscope}%
\begin{pgfscope}%
\pgfpathrectangle{\pgfqpoint{0.997489in}{0.528000in}}{\pgfqpoint{4.565023in}{3.696000in}}%
\pgfusepath{clip}%
\pgfsetbuttcap%
\pgfsetroundjoin%
\definecolor{currentfill}{rgb}{0.800000,0.800000,0.200000}%
\pgfsetfillcolor{currentfill}%
\pgfsetlinewidth{1.003750pt}%
\definecolor{currentstroke}{rgb}{0.800000,0.800000,0.200000}%
\pgfsetstrokecolor{currentstroke}%
\pgfsetdash{}{0pt}%
\pgfpathmoveto{\pgfqpoint{3.529239in}{3.301472in}}%
\pgfpathcurveto{\pgfqpoint{3.535063in}{3.301472in}}{\pgfqpoint{3.540649in}{3.303786in}}{\pgfqpoint{3.544767in}{3.307904in}}%
\pgfpathcurveto{\pgfqpoint{3.548885in}{3.312022in}}{\pgfqpoint{3.551199in}{3.317608in}}{\pgfqpoint{3.551199in}{3.323432in}}%
\pgfpathcurveto{\pgfqpoint{3.551199in}{3.329256in}}{\pgfqpoint{3.548885in}{3.334843in}}{\pgfqpoint{3.544767in}{3.338961in}}%
\pgfpathcurveto{\pgfqpoint{3.540649in}{3.343079in}}{\pgfqpoint{3.535063in}{3.345393in}}{\pgfqpoint{3.529239in}{3.345393in}}%
\pgfpathcurveto{\pgfqpoint{3.523415in}{3.345393in}}{\pgfqpoint{3.517829in}{3.343079in}}{\pgfqpoint{3.513711in}{3.338961in}}%
\pgfpathcurveto{\pgfqpoint{3.509592in}{3.334843in}}{\pgfqpoint{3.507279in}{3.329256in}}{\pgfqpoint{3.507279in}{3.323432in}}%
\pgfpathcurveto{\pgfqpoint{3.507279in}{3.317608in}}{\pgfqpoint{3.509592in}{3.312022in}}{\pgfqpoint{3.513711in}{3.307904in}}%
\pgfpathcurveto{\pgfqpoint{3.517829in}{3.303786in}}{\pgfqpoint{3.523415in}{3.301472in}}{\pgfqpoint{3.529239in}{3.301472in}}%
\pgfpathlineto{\pgfqpoint{3.529239in}{3.301472in}}%
\pgfpathclose%
\pgfusepath{stroke,fill}%
\end{pgfscope}%
\begin{pgfscope}%
\pgfpathrectangle{\pgfqpoint{0.997489in}{0.528000in}}{\pgfqpoint{4.565023in}{3.696000in}}%
\pgfusepath{clip}%
\pgfsetbuttcap%
\pgfsetroundjoin%
\definecolor{currentfill}{rgb}{0.800000,0.800000,0.200000}%
\pgfsetfillcolor{currentfill}%
\pgfsetlinewidth{1.003750pt}%
\definecolor{currentstroke}{rgb}{0.800000,0.800000,0.200000}%
\pgfsetstrokecolor{currentstroke}%
\pgfsetdash{}{0pt}%
\pgfpathmoveto{\pgfqpoint{3.455470in}{3.280282in}}%
\pgfpathcurveto{\pgfqpoint{3.461294in}{3.280282in}}{\pgfqpoint{3.466881in}{3.282596in}}{\pgfqpoint{3.470999in}{3.286714in}}%
\pgfpathcurveto{\pgfqpoint{3.475117in}{3.290832in}}{\pgfqpoint{3.477431in}{3.296418in}}{\pgfqpoint{3.477431in}{3.302242in}}%
\pgfpathcurveto{\pgfqpoint{3.477431in}{3.308066in}}{\pgfqpoint{3.475117in}{3.313653in}}{\pgfqpoint{3.470999in}{3.317771in}}%
\pgfpathcurveto{\pgfqpoint{3.466881in}{3.321889in}}{\pgfqpoint{3.461294in}{3.324203in}}{\pgfqpoint{3.455470in}{3.324203in}}%
\pgfpathcurveto{\pgfqpoint{3.449647in}{3.324203in}}{\pgfqpoint{3.444060in}{3.321889in}}{\pgfqpoint{3.439942in}{3.317771in}}%
\pgfpathcurveto{\pgfqpoint{3.435824in}{3.313653in}}{\pgfqpoint{3.433510in}{3.308066in}}{\pgfqpoint{3.433510in}{3.302242in}}%
\pgfpathcurveto{\pgfqpoint{3.433510in}{3.296418in}}{\pgfqpoint{3.435824in}{3.290832in}}{\pgfqpoint{3.439942in}{3.286714in}}%
\pgfpathcurveto{\pgfqpoint{3.444060in}{3.282596in}}{\pgfqpoint{3.449647in}{3.280282in}}{\pgfqpoint{3.455470in}{3.280282in}}%
\pgfpathlineto{\pgfqpoint{3.455470in}{3.280282in}}%
\pgfpathclose%
\pgfusepath{stroke,fill}%
\end{pgfscope}%
\begin{pgfscope}%
\pgfpathrectangle{\pgfqpoint{0.997489in}{0.528000in}}{\pgfqpoint{4.565023in}{3.696000in}}%
\pgfusepath{clip}%
\pgfsetbuttcap%
\pgfsetroundjoin%
\definecolor{currentfill}{rgb}{0.800000,0.200000,0.200000}%
\pgfsetfillcolor{currentfill}%
\pgfsetlinewidth{1.003750pt}%
\definecolor{currentstroke}{rgb}{0.800000,0.200000,0.200000}%
\pgfsetstrokecolor{currentstroke}%
\pgfsetdash{}{0pt}%
\pgfpathmoveto{\pgfqpoint{3.399845in}{3.334241in}}%
\pgfpathcurveto{\pgfqpoint{3.405669in}{3.334241in}}{\pgfqpoint{3.411256in}{3.336555in}}{\pgfqpoint{3.415374in}{3.340673in}}%
\pgfpathcurveto{\pgfqpoint{3.419492in}{3.344791in}}{\pgfqpoint{3.421806in}{3.350378in}}{\pgfqpoint{3.421806in}{3.356202in}}%
\pgfpathcurveto{\pgfqpoint{3.421806in}{3.362025in}}{\pgfqpoint{3.419492in}{3.367612in}}{\pgfqpoint{3.415374in}{3.371730in}}%
\pgfpathcurveto{\pgfqpoint{3.411256in}{3.375848in}}{\pgfqpoint{3.405669in}{3.378162in}}{\pgfqpoint{3.399845in}{3.378162in}}%
\pgfpathcurveto{\pgfqpoint{3.394021in}{3.378162in}}{\pgfqpoint{3.388435in}{3.375848in}}{\pgfqpoint{3.384317in}{3.371730in}}%
\pgfpathcurveto{\pgfqpoint{3.380199in}{3.367612in}}{\pgfqpoint{3.377885in}{3.362025in}}{\pgfqpoint{3.377885in}{3.356202in}}%
\pgfpathcurveto{\pgfqpoint{3.377885in}{3.350378in}}{\pgfqpoint{3.380199in}{3.344791in}}{\pgfqpoint{3.384317in}{3.340673in}}%
\pgfpathcurveto{\pgfqpoint{3.388435in}{3.336555in}}{\pgfqpoint{3.394021in}{3.334241in}}{\pgfqpoint{3.399845in}{3.334241in}}%
\pgfpathlineto{\pgfqpoint{3.399845in}{3.334241in}}%
\pgfpathclose%
\pgfusepath{stroke,fill}%
\end{pgfscope}%
\begin{pgfscope}%
\pgfpathrectangle{\pgfqpoint{0.997489in}{0.528000in}}{\pgfqpoint{4.565023in}{3.696000in}}%
\pgfusepath{clip}%
\pgfsetbuttcap%
\pgfsetroundjoin%
\definecolor{currentfill}{rgb}{0.800000,0.200000,0.200000}%
\pgfsetfillcolor{currentfill}%
\pgfsetlinewidth{1.003750pt}%
\definecolor{currentstroke}{rgb}{0.800000,0.200000,0.200000}%
\pgfsetstrokecolor{currentstroke}%
\pgfsetdash{}{0pt}%
\pgfpathmoveto{\pgfqpoint{3.321077in}{3.230445in}}%
\pgfpathcurveto{\pgfqpoint{3.326901in}{3.230445in}}{\pgfqpoint{3.332487in}{3.232759in}}{\pgfqpoint{3.336605in}{3.236877in}}%
\pgfpathcurveto{\pgfqpoint{3.340723in}{3.240995in}}{\pgfqpoint{3.343037in}{3.246581in}}{\pgfqpoint{3.343037in}{3.252405in}}%
\pgfpathcurveto{\pgfqpoint{3.343037in}{3.258229in}}{\pgfqpoint{3.340723in}{3.263815in}}{\pgfqpoint{3.336605in}{3.267933in}}%
\pgfpathcurveto{\pgfqpoint{3.332487in}{3.272051in}}{\pgfqpoint{3.326901in}{3.274365in}}{\pgfqpoint{3.321077in}{3.274365in}}%
\pgfpathcurveto{\pgfqpoint{3.315253in}{3.274365in}}{\pgfqpoint{3.309667in}{3.272051in}}{\pgfqpoint{3.305549in}{3.267933in}}%
\pgfpathcurveto{\pgfqpoint{3.301431in}{3.263815in}}{\pgfqpoint{3.299117in}{3.258229in}}{\pgfqpoint{3.299117in}{3.252405in}}%
\pgfpathcurveto{\pgfqpoint{3.299117in}{3.246581in}}{\pgfqpoint{3.301431in}{3.240995in}}{\pgfqpoint{3.305549in}{3.236877in}}%
\pgfpathcurveto{\pgfqpoint{3.309667in}{3.232759in}}{\pgfqpoint{3.315253in}{3.230445in}}{\pgfqpoint{3.321077in}{3.230445in}}%
\pgfpathlineto{\pgfqpoint{3.321077in}{3.230445in}}%
\pgfpathclose%
\pgfusepath{stroke,fill}%
\end{pgfscope}%
\begin{pgfscope}%
\pgfpathrectangle{\pgfqpoint{0.997489in}{0.528000in}}{\pgfqpoint{4.565023in}{3.696000in}}%
\pgfusepath{clip}%
\pgfsetbuttcap%
\pgfsetroundjoin%
\definecolor{currentfill}{rgb}{0.800000,0.200000,0.200000}%
\pgfsetfillcolor{currentfill}%
\pgfsetlinewidth{1.003750pt}%
\definecolor{currentstroke}{rgb}{0.800000,0.200000,0.200000}%
\pgfsetstrokecolor{currentstroke}%
\pgfsetdash{}{0pt}%
\pgfpathmoveto{\pgfqpoint{3.264082in}{3.285448in}}%
\pgfpathcurveto{\pgfqpoint{3.269906in}{3.285448in}}{\pgfqpoint{3.275492in}{3.287761in}}{\pgfqpoint{3.279610in}{3.291880in}}%
\pgfpathcurveto{\pgfqpoint{3.283728in}{3.295998in}}{\pgfqpoint{3.286042in}{3.301584in}}{\pgfqpoint{3.286042in}{3.307408in}}%
\pgfpathcurveto{\pgfqpoint{3.286042in}{3.313232in}}{\pgfqpoint{3.283728in}{3.318818in}}{\pgfqpoint{3.279610in}{3.322936in}}%
\pgfpathcurveto{\pgfqpoint{3.275492in}{3.327054in}}{\pgfqpoint{3.269906in}{3.329368in}}{\pgfqpoint{3.264082in}{3.329368in}}%
\pgfpathcurveto{\pgfqpoint{3.258258in}{3.329368in}}{\pgfqpoint{3.252672in}{3.327054in}}{\pgfqpoint{3.248554in}{3.322936in}}%
\pgfpathcurveto{\pgfqpoint{3.244436in}{3.318818in}}{\pgfqpoint{3.242122in}{3.313232in}}{\pgfqpoint{3.242122in}{3.307408in}}%
\pgfpathcurveto{\pgfqpoint{3.242122in}{3.301584in}}{\pgfqpoint{3.244436in}{3.295998in}}{\pgfqpoint{3.248554in}{3.291880in}}%
\pgfpathcurveto{\pgfqpoint{3.252672in}{3.287761in}}{\pgfqpoint{3.258258in}{3.285448in}}{\pgfqpoint{3.264082in}{3.285448in}}%
\pgfpathlineto{\pgfqpoint{3.264082in}{3.285448in}}%
\pgfpathclose%
\pgfusepath{stroke,fill}%
\end{pgfscope}%
\begin{pgfscope}%
\pgfpathrectangle{\pgfqpoint{0.997489in}{0.528000in}}{\pgfqpoint{4.565023in}{3.696000in}}%
\pgfusepath{clip}%
\pgfsetbuttcap%
\pgfsetroundjoin%
\definecolor{currentfill}{rgb}{0.800000,0.200000,0.200000}%
\pgfsetfillcolor{currentfill}%
\pgfsetlinewidth{1.003750pt}%
\definecolor{currentstroke}{rgb}{0.800000,0.200000,0.200000}%
\pgfsetstrokecolor{currentstroke}%
\pgfsetdash{}{0pt}%
\pgfpathmoveto{\pgfqpoint{3.201069in}{3.302156in}}%
\pgfpathcurveto{\pgfqpoint{3.206893in}{3.302156in}}{\pgfqpoint{3.212479in}{3.304470in}}{\pgfqpoint{3.216597in}{3.308588in}}%
\pgfpathcurveto{\pgfqpoint{3.220715in}{3.312706in}}{\pgfqpoint{3.223029in}{3.318292in}}{\pgfqpoint{3.223029in}{3.324116in}}%
\pgfpathcurveto{\pgfqpoint{3.223029in}{3.329940in}}{\pgfqpoint{3.220715in}{3.335526in}}{\pgfqpoint{3.216597in}{3.339645in}}%
\pgfpathcurveto{\pgfqpoint{3.212479in}{3.343763in}}{\pgfqpoint{3.206893in}{3.346077in}}{\pgfqpoint{3.201069in}{3.346077in}}%
\pgfpathcurveto{\pgfqpoint{3.195245in}{3.346077in}}{\pgfqpoint{3.189659in}{3.343763in}}{\pgfqpoint{3.185541in}{3.339645in}}%
\pgfpathcurveto{\pgfqpoint{3.181422in}{3.335526in}}{\pgfqpoint{3.179109in}{3.329940in}}{\pgfqpoint{3.179109in}{3.324116in}}%
\pgfpathcurveto{\pgfqpoint{3.179109in}{3.318292in}}{\pgfqpoint{3.181422in}{3.312706in}}{\pgfqpoint{3.185541in}{3.308588in}}%
\pgfpathcurveto{\pgfqpoint{3.189659in}{3.304470in}}{\pgfqpoint{3.195245in}{3.302156in}}{\pgfqpoint{3.201069in}{3.302156in}}%
\pgfpathlineto{\pgfqpoint{3.201069in}{3.302156in}}%
\pgfpathclose%
\pgfusepath{stroke,fill}%
\end{pgfscope}%
\begin{pgfscope}%
\pgfpathrectangle{\pgfqpoint{0.997489in}{0.528000in}}{\pgfqpoint{4.565023in}{3.696000in}}%
\pgfusepath{clip}%
\pgfsetbuttcap%
\pgfsetroundjoin%
\definecolor{currentfill}{rgb}{0.800000,0.200000,0.200000}%
\pgfsetfillcolor{currentfill}%
\pgfsetlinewidth{1.003750pt}%
\definecolor{currentstroke}{rgb}{0.800000,0.200000,0.200000}%
\pgfsetstrokecolor{currentstroke}%
\pgfsetdash{}{0pt}%
\pgfpathmoveto{\pgfqpoint{3.137104in}{3.302494in}}%
\pgfpathcurveto{\pgfqpoint{3.142928in}{3.302494in}}{\pgfqpoint{3.148514in}{3.304808in}}{\pgfqpoint{3.152632in}{3.308926in}}%
\pgfpathcurveto{\pgfqpoint{3.156750in}{3.313044in}}{\pgfqpoint{3.159064in}{3.318630in}}{\pgfqpoint{3.159064in}{3.324454in}}%
\pgfpathcurveto{\pgfqpoint{3.159064in}{3.330278in}}{\pgfqpoint{3.156750in}{3.335864in}}{\pgfqpoint{3.152632in}{3.339982in}}%
\pgfpathcurveto{\pgfqpoint{3.148514in}{3.344100in}}{\pgfqpoint{3.142928in}{3.346414in}}{\pgfqpoint{3.137104in}{3.346414in}}%
\pgfpathcurveto{\pgfqpoint{3.131280in}{3.346414in}}{\pgfqpoint{3.125694in}{3.344100in}}{\pgfqpoint{3.121576in}{3.339982in}}%
\pgfpathcurveto{\pgfqpoint{3.117458in}{3.335864in}}{\pgfqpoint{3.115144in}{3.330278in}}{\pgfqpoint{3.115144in}{3.324454in}}%
\pgfpathcurveto{\pgfqpoint{3.115144in}{3.318630in}}{\pgfqpoint{3.117458in}{3.313044in}}{\pgfqpoint{3.121576in}{3.308926in}}%
\pgfpathcurveto{\pgfqpoint{3.125694in}{3.304808in}}{\pgfqpoint{3.131280in}{3.302494in}}{\pgfqpoint{3.137104in}{3.302494in}}%
\pgfpathlineto{\pgfqpoint{3.137104in}{3.302494in}}%
\pgfpathclose%
\pgfusepath{stroke,fill}%
\end{pgfscope}%
\begin{pgfscope}%
\pgfpathrectangle{\pgfqpoint{0.997489in}{0.528000in}}{\pgfqpoint{4.565023in}{3.696000in}}%
\pgfusepath{clip}%
\pgfsetbuttcap%
\pgfsetroundjoin%
\definecolor{currentfill}{rgb}{0.800000,0.200000,0.200000}%
\pgfsetfillcolor{currentfill}%
\pgfsetlinewidth{1.003750pt}%
\definecolor{currentstroke}{rgb}{0.800000,0.200000,0.200000}%
\pgfsetstrokecolor{currentstroke}%
\pgfsetdash{}{0pt}%
\pgfpathmoveto{\pgfqpoint{3.085138in}{3.214561in}}%
\pgfpathcurveto{\pgfqpoint{3.090962in}{3.214561in}}{\pgfqpoint{3.096548in}{3.216875in}}{\pgfqpoint{3.100666in}{3.220993in}}%
\pgfpathcurveto{\pgfqpoint{3.104784in}{3.225111in}}{\pgfqpoint{3.107098in}{3.230697in}}{\pgfqpoint{3.107098in}{3.236521in}}%
\pgfpathcurveto{\pgfqpoint{3.107098in}{3.242345in}}{\pgfqpoint{3.104784in}{3.247931in}}{\pgfqpoint{3.100666in}{3.252049in}}%
\pgfpathcurveto{\pgfqpoint{3.096548in}{3.256167in}}{\pgfqpoint{3.090962in}{3.258481in}}{\pgfqpoint{3.085138in}{3.258481in}}%
\pgfpathcurveto{\pgfqpoint{3.079314in}{3.258481in}}{\pgfqpoint{3.073728in}{3.256167in}}{\pgfqpoint{3.069609in}{3.252049in}}%
\pgfpathcurveto{\pgfqpoint{3.065491in}{3.247931in}}{\pgfqpoint{3.063177in}{3.242345in}}{\pgfqpoint{3.063177in}{3.236521in}}%
\pgfpathcurveto{\pgfqpoint{3.063177in}{3.230697in}}{\pgfqpoint{3.065491in}{3.225111in}}{\pgfqpoint{3.069609in}{3.220993in}}%
\pgfpathcurveto{\pgfqpoint{3.073728in}{3.216875in}}{\pgfqpoint{3.079314in}{3.214561in}}{\pgfqpoint{3.085138in}{3.214561in}}%
\pgfpathlineto{\pgfqpoint{3.085138in}{3.214561in}}%
\pgfpathclose%
\pgfusepath{stroke,fill}%
\end{pgfscope}%
\begin{pgfscope}%
\pgfpathrectangle{\pgfqpoint{0.997489in}{0.528000in}}{\pgfqpoint{4.565023in}{3.696000in}}%
\pgfusepath{clip}%
\pgfsetbuttcap%
\pgfsetroundjoin%
\definecolor{currentfill}{rgb}{0.200000,0.800000,0.200000}%
\pgfsetfillcolor{currentfill}%
\pgfsetlinewidth{1.003750pt}%
\definecolor{currentstroke}{rgb}{0.200000,0.800000,0.200000}%
\pgfsetstrokecolor{currentstroke}%
\pgfsetdash{}{0pt}%
\pgfpathmoveto{\pgfqpoint{2.998317in}{3.342530in}}%
\pgfpathcurveto{\pgfqpoint{3.004141in}{3.342530in}}{\pgfqpoint{3.009727in}{3.344844in}}{\pgfqpoint{3.013845in}{3.348962in}}%
\pgfpathcurveto{\pgfqpoint{3.017963in}{3.353080in}}{\pgfqpoint{3.020277in}{3.358666in}}{\pgfqpoint{3.020277in}{3.364490in}}%
\pgfpathcurveto{\pgfqpoint{3.020277in}{3.370314in}}{\pgfqpoint{3.017963in}{3.375900in}}{\pgfqpoint{3.013845in}{3.380018in}}%
\pgfpathcurveto{\pgfqpoint{3.009727in}{3.384136in}}{\pgfqpoint{3.004141in}{3.386450in}}{\pgfqpoint{2.998317in}{3.386450in}}%
\pgfpathcurveto{\pgfqpoint{2.992493in}{3.386450in}}{\pgfqpoint{2.986907in}{3.384136in}}{\pgfqpoint{2.982789in}{3.380018in}}%
\pgfpathcurveto{\pgfqpoint{2.978671in}{3.375900in}}{\pgfqpoint{2.976357in}{3.370314in}}{\pgfqpoint{2.976357in}{3.364490in}}%
\pgfpathcurveto{\pgfqpoint{2.976357in}{3.358666in}}{\pgfqpoint{2.978671in}{3.353080in}}{\pgfqpoint{2.982789in}{3.348962in}}%
\pgfpathcurveto{\pgfqpoint{2.986907in}{3.344844in}}{\pgfqpoint{2.992493in}{3.342530in}}{\pgfqpoint{2.998317in}{3.342530in}}%
\pgfpathlineto{\pgfqpoint{2.998317in}{3.342530in}}%
\pgfpathclose%
\pgfusepath{stroke,fill}%
\end{pgfscope}%
\begin{pgfscope}%
\pgfpathrectangle{\pgfqpoint{0.997489in}{0.528000in}}{\pgfqpoint{4.565023in}{3.696000in}}%
\pgfusepath{clip}%
\pgfsetbuttcap%
\pgfsetroundjoin%
\definecolor{currentfill}{rgb}{0.200000,0.800000,0.200000}%
\pgfsetfillcolor{currentfill}%
\pgfsetlinewidth{1.003750pt}%
\definecolor{currentstroke}{rgb}{0.200000,0.800000,0.200000}%
\pgfsetstrokecolor{currentstroke}%
\pgfsetdash{}{0pt}%
\pgfpathmoveto{\pgfqpoint{2.965060in}{3.208667in}}%
\pgfpathcurveto{\pgfqpoint{2.970884in}{3.208667in}}{\pgfqpoint{2.976470in}{3.210981in}}{\pgfqpoint{2.980588in}{3.215099in}}%
\pgfpathcurveto{\pgfqpoint{2.984707in}{3.219217in}}{\pgfqpoint{2.987020in}{3.224803in}}{\pgfqpoint{2.987020in}{3.230627in}}%
\pgfpathcurveto{\pgfqpoint{2.987020in}{3.236451in}}{\pgfqpoint{2.984707in}{3.242037in}}{\pgfqpoint{2.980588in}{3.246156in}}%
\pgfpathcurveto{\pgfqpoint{2.976470in}{3.250274in}}{\pgfqpoint{2.970884in}{3.252588in}}{\pgfqpoint{2.965060in}{3.252588in}}%
\pgfpathcurveto{\pgfqpoint{2.959236in}{3.252588in}}{\pgfqpoint{2.953650in}{3.250274in}}{\pgfqpoint{2.949532in}{3.246156in}}%
\pgfpathcurveto{\pgfqpoint{2.945414in}{3.242037in}}{\pgfqpoint{2.943100in}{3.236451in}}{\pgfqpoint{2.943100in}{3.230627in}}%
\pgfpathcurveto{\pgfqpoint{2.943100in}{3.224803in}}{\pgfqpoint{2.945414in}{3.219217in}}{\pgfqpoint{2.949532in}{3.215099in}}%
\pgfpathcurveto{\pgfqpoint{2.953650in}{3.210981in}}{\pgfqpoint{2.959236in}{3.208667in}}{\pgfqpoint{2.965060in}{3.208667in}}%
\pgfpathlineto{\pgfqpoint{2.965060in}{3.208667in}}%
\pgfpathclose%
\pgfusepath{stroke,fill}%
\end{pgfscope}%
\begin{pgfscope}%
\pgfpathrectangle{\pgfqpoint{0.997489in}{0.528000in}}{\pgfqpoint{4.565023in}{3.696000in}}%
\pgfusepath{clip}%
\pgfsetbuttcap%
\pgfsetroundjoin%
\definecolor{currentfill}{rgb}{0.800000,0.200000,0.200000}%
\pgfsetfillcolor{currentfill}%
\pgfsetlinewidth{1.003750pt}%
\definecolor{currentstroke}{rgb}{0.800000,0.200000,0.200000}%
\pgfsetstrokecolor{currentstroke}%
\pgfsetdash{}{0pt}%
\pgfpathmoveto{\pgfqpoint{2.893600in}{3.231788in}}%
\pgfpathcurveto{\pgfqpoint{2.899424in}{3.231788in}}{\pgfqpoint{2.905010in}{3.234102in}}{\pgfqpoint{2.909128in}{3.238220in}}%
\pgfpathcurveto{\pgfqpoint{2.913246in}{3.242339in}}{\pgfqpoint{2.915560in}{3.247925in}}{\pgfqpoint{2.915560in}{3.253749in}}%
\pgfpathcurveto{\pgfqpoint{2.915560in}{3.259573in}}{\pgfqpoint{2.913246in}{3.265159in}}{\pgfqpoint{2.909128in}{3.269277in}}%
\pgfpathcurveto{\pgfqpoint{2.905010in}{3.273395in}}{\pgfqpoint{2.899424in}{3.275709in}}{\pgfqpoint{2.893600in}{3.275709in}}%
\pgfpathcurveto{\pgfqpoint{2.887776in}{3.275709in}}{\pgfqpoint{2.882190in}{3.273395in}}{\pgfqpoint{2.878072in}{3.269277in}}%
\pgfpathcurveto{\pgfqpoint{2.873954in}{3.265159in}}{\pgfqpoint{2.871640in}{3.259573in}}{\pgfqpoint{2.871640in}{3.253749in}}%
\pgfpathcurveto{\pgfqpoint{2.871640in}{3.247925in}}{\pgfqpoint{2.873954in}{3.242339in}}{\pgfqpoint{2.878072in}{3.238220in}}%
\pgfpathcurveto{\pgfqpoint{2.882190in}{3.234102in}}{\pgfqpoint{2.887776in}{3.231788in}}{\pgfqpoint{2.893600in}{3.231788in}}%
\pgfpathlineto{\pgfqpoint{2.893600in}{3.231788in}}%
\pgfpathclose%
\pgfusepath{stroke,fill}%
\end{pgfscope}%
\begin{pgfscope}%
\pgfpathrectangle{\pgfqpoint{0.997489in}{0.528000in}}{\pgfqpoint{4.565023in}{3.696000in}}%
\pgfusepath{clip}%
\pgfsetbuttcap%
\pgfsetroundjoin%
\definecolor{currentfill}{rgb}{0.800000,0.200000,0.200000}%
\pgfsetfillcolor{currentfill}%
\pgfsetlinewidth{1.003750pt}%
\definecolor{currentstroke}{rgb}{0.800000,0.200000,0.200000}%
\pgfsetstrokecolor{currentstroke}%
\pgfsetdash{}{0pt}%
\pgfpathmoveto{\pgfqpoint{2.822919in}{3.238203in}}%
\pgfpathcurveto{\pgfqpoint{2.828742in}{3.238203in}}{\pgfqpoint{2.834329in}{3.240516in}}{\pgfqpoint{2.838447in}{3.244635in}}%
\pgfpathcurveto{\pgfqpoint{2.842565in}{3.248753in}}{\pgfqpoint{2.844879in}{3.254339in}}{\pgfqpoint{2.844879in}{3.260163in}}%
\pgfpathcurveto{\pgfqpoint{2.844879in}{3.265987in}}{\pgfqpoint{2.842565in}{3.271573in}}{\pgfqpoint{2.838447in}{3.275691in}}%
\pgfpathcurveto{\pgfqpoint{2.834329in}{3.279809in}}{\pgfqpoint{2.828742in}{3.282123in}}{\pgfqpoint{2.822919in}{3.282123in}}%
\pgfpathcurveto{\pgfqpoint{2.817095in}{3.282123in}}{\pgfqpoint{2.811508in}{3.279809in}}{\pgfqpoint{2.807390in}{3.275691in}}%
\pgfpathcurveto{\pgfqpoint{2.803272in}{3.271573in}}{\pgfqpoint{2.800958in}{3.265987in}}{\pgfqpoint{2.800958in}{3.260163in}}%
\pgfpathcurveto{\pgfqpoint{2.800958in}{3.254339in}}{\pgfqpoint{2.803272in}{3.248753in}}{\pgfqpoint{2.807390in}{3.244635in}}%
\pgfpathcurveto{\pgfqpoint{2.811508in}{3.240516in}}{\pgfqpoint{2.817095in}{3.238203in}}{\pgfqpoint{2.822919in}{3.238203in}}%
\pgfpathlineto{\pgfqpoint{2.822919in}{3.238203in}}%
\pgfpathclose%
\pgfusepath{stroke,fill}%
\end{pgfscope}%
\begin{pgfscope}%
\pgfpathrectangle{\pgfqpoint{0.997489in}{0.528000in}}{\pgfqpoint{4.565023in}{3.696000in}}%
\pgfusepath{clip}%
\pgfsetbuttcap%
\pgfsetroundjoin%
\definecolor{currentfill}{rgb}{0.200000,0.800000,0.200000}%
\pgfsetfillcolor{currentfill}%
\pgfsetlinewidth{1.003750pt}%
\definecolor{currentstroke}{rgb}{0.200000,0.800000,0.200000}%
\pgfsetstrokecolor{currentstroke}%
\pgfsetdash{}{0pt}%
\pgfpathmoveto{\pgfqpoint{2.716474in}{3.307285in}}%
\pgfpathcurveto{\pgfqpoint{2.722298in}{3.307285in}}{\pgfqpoint{2.727885in}{3.309599in}}{\pgfqpoint{2.732003in}{3.313717in}}%
\pgfpathcurveto{\pgfqpoint{2.736121in}{3.317835in}}{\pgfqpoint{2.738435in}{3.323421in}}{\pgfqpoint{2.738435in}{3.329245in}}%
\pgfpathcurveto{\pgfqpoint{2.738435in}{3.335069in}}{\pgfqpoint{2.736121in}{3.340655in}}{\pgfqpoint{2.732003in}{3.344774in}}%
\pgfpathcurveto{\pgfqpoint{2.727885in}{3.348892in}}{\pgfqpoint{2.722298in}{3.351206in}}{\pgfqpoint{2.716474in}{3.351206in}}%
\pgfpathcurveto{\pgfqpoint{2.710650in}{3.351206in}}{\pgfqpoint{2.705064in}{3.348892in}}{\pgfqpoint{2.700946in}{3.344774in}}%
\pgfpathcurveto{\pgfqpoint{2.696828in}{3.340655in}}{\pgfqpoint{2.694514in}{3.335069in}}{\pgfqpoint{2.694514in}{3.329245in}}%
\pgfpathcurveto{\pgfqpoint{2.694514in}{3.323421in}}{\pgfqpoint{2.696828in}{3.317835in}}{\pgfqpoint{2.700946in}{3.313717in}}%
\pgfpathcurveto{\pgfqpoint{2.705064in}{3.309599in}}{\pgfqpoint{2.710650in}{3.307285in}}{\pgfqpoint{2.716474in}{3.307285in}}%
\pgfpathlineto{\pgfqpoint{2.716474in}{3.307285in}}%
\pgfpathclose%
\pgfusepath{stroke,fill}%
\end{pgfscope}%
\begin{pgfscope}%
\pgfpathrectangle{\pgfqpoint{0.997489in}{0.528000in}}{\pgfqpoint{4.565023in}{3.696000in}}%
\pgfusepath{clip}%
\pgfsetbuttcap%
\pgfsetroundjoin%
\definecolor{currentfill}{rgb}{0.800000,0.200000,0.200000}%
\pgfsetfillcolor{currentfill}%
\pgfsetlinewidth{1.003750pt}%
\definecolor{currentstroke}{rgb}{0.800000,0.200000,0.200000}%
\pgfsetstrokecolor{currentstroke}%
\pgfsetdash{}{0pt}%
\pgfpathmoveto{\pgfqpoint{2.707629in}{3.179627in}}%
\pgfpathcurveto{\pgfqpoint{2.713453in}{3.179627in}}{\pgfqpoint{2.719039in}{3.181941in}}{\pgfqpoint{2.723157in}{3.186059in}}%
\pgfpathcurveto{\pgfqpoint{2.727275in}{3.190177in}}{\pgfqpoint{2.729589in}{3.195763in}}{\pgfqpoint{2.729589in}{3.201587in}}%
\pgfpathcurveto{\pgfqpoint{2.729589in}{3.207411in}}{\pgfqpoint{2.727275in}{3.212997in}}{\pgfqpoint{2.723157in}{3.217115in}}%
\pgfpathcurveto{\pgfqpoint{2.719039in}{3.221234in}}{\pgfqpoint{2.713453in}{3.223547in}}{\pgfqpoint{2.707629in}{3.223547in}}%
\pgfpathcurveto{\pgfqpoint{2.701805in}{3.223547in}}{\pgfqpoint{2.696219in}{3.221234in}}{\pgfqpoint{2.692101in}{3.217115in}}%
\pgfpathcurveto{\pgfqpoint{2.687982in}{3.212997in}}{\pgfqpoint{2.685669in}{3.207411in}}{\pgfqpoint{2.685669in}{3.201587in}}%
\pgfpathcurveto{\pgfqpoint{2.685669in}{3.195763in}}{\pgfqpoint{2.687982in}{3.190177in}}{\pgfqpoint{2.692101in}{3.186059in}}%
\pgfpathcurveto{\pgfqpoint{2.696219in}{3.181941in}}{\pgfqpoint{2.701805in}{3.179627in}}{\pgfqpoint{2.707629in}{3.179627in}}%
\pgfpathlineto{\pgfqpoint{2.707629in}{3.179627in}}%
\pgfpathclose%
\pgfusepath{stroke,fill}%
\end{pgfscope}%
\begin{pgfscope}%
\pgfpathrectangle{\pgfqpoint{0.997489in}{0.528000in}}{\pgfqpoint{4.565023in}{3.696000in}}%
\pgfusepath{clip}%
\pgfsetbuttcap%
\pgfsetroundjoin%
\definecolor{currentfill}{rgb}{0.800000,0.200000,0.200000}%
\pgfsetfillcolor{currentfill}%
\pgfsetlinewidth{1.003750pt}%
\definecolor{currentstroke}{rgb}{0.800000,0.200000,0.200000}%
\pgfsetstrokecolor{currentstroke}%
\pgfsetdash{}{0pt}%
\pgfpathmoveto{\pgfqpoint{2.638327in}{3.167142in}}%
\pgfpathcurveto{\pgfqpoint{2.644151in}{3.167142in}}{\pgfqpoint{2.649737in}{3.169456in}}{\pgfqpoint{2.653855in}{3.173574in}}%
\pgfpathcurveto{\pgfqpoint{2.657973in}{3.177692in}}{\pgfqpoint{2.660287in}{3.183278in}}{\pgfqpoint{2.660287in}{3.189102in}}%
\pgfpathcurveto{\pgfqpoint{2.660287in}{3.194926in}}{\pgfqpoint{2.657973in}{3.200512in}}{\pgfqpoint{2.653855in}{3.204630in}}%
\pgfpathcurveto{\pgfqpoint{2.649737in}{3.208749in}}{\pgfqpoint{2.644151in}{3.211062in}}{\pgfqpoint{2.638327in}{3.211062in}}%
\pgfpathcurveto{\pgfqpoint{2.632503in}{3.211062in}}{\pgfqpoint{2.626917in}{3.208749in}}{\pgfqpoint{2.622799in}{3.204630in}}%
\pgfpathcurveto{\pgfqpoint{2.618681in}{3.200512in}}{\pgfqpoint{2.616367in}{3.194926in}}{\pgfqpoint{2.616367in}{3.189102in}}%
\pgfpathcurveto{\pgfqpoint{2.616367in}{3.183278in}}{\pgfqpoint{2.618681in}{3.177692in}}{\pgfqpoint{2.622799in}{3.173574in}}%
\pgfpathcurveto{\pgfqpoint{2.626917in}{3.169456in}}{\pgfqpoint{2.632503in}{3.167142in}}{\pgfqpoint{2.638327in}{3.167142in}}%
\pgfpathlineto{\pgfqpoint{2.638327in}{3.167142in}}%
\pgfpathclose%
\pgfusepath{stroke,fill}%
\end{pgfscope}%
\begin{pgfscope}%
\pgfpathrectangle{\pgfqpoint{0.997489in}{0.528000in}}{\pgfqpoint{4.565023in}{3.696000in}}%
\pgfusepath{clip}%
\pgfsetbuttcap%
\pgfsetroundjoin%
\definecolor{currentfill}{rgb}{0.800000,0.200000,0.200000}%
\pgfsetfillcolor{currentfill}%
\pgfsetlinewidth{1.003750pt}%
\definecolor{currentstroke}{rgb}{0.800000,0.200000,0.200000}%
\pgfsetstrokecolor{currentstroke}%
\pgfsetdash{}{0pt}%
\pgfpathmoveto{\pgfqpoint{2.644880in}{3.049121in}}%
\pgfpathcurveto{\pgfqpoint{2.650704in}{3.049121in}}{\pgfqpoint{2.656290in}{3.051435in}}{\pgfqpoint{2.660408in}{3.055553in}}%
\pgfpathcurveto{\pgfqpoint{2.664527in}{3.059671in}}{\pgfqpoint{2.666840in}{3.065258in}}{\pgfqpoint{2.666840in}{3.071082in}}%
\pgfpathcurveto{\pgfqpoint{2.666840in}{3.076906in}}{\pgfqpoint{2.664527in}{3.082492in}}{\pgfqpoint{2.660408in}{3.086610in}}%
\pgfpathcurveto{\pgfqpoint{2.656290in}{3.090728in}}{\pgfqpoint{2.650704in}{3.093042in}}{\pgfqpoint{2.644880in}{3.093042in}}%
\pgfpathcurveto{\pgfqpoint{2.639056in}{3.093042in}}{\pgfqpoint{2.633470in}{3.090728in}}{\pgfqpoint{2.629352in}{3.086610in}}%
\pgfpathcurveto{\pgfqpoint{2.625234in}{3.082492in}}{\pgfqpoint{2.622920in}{3.076906in}}{\pgfqpoint{2.622920in}{3.071082in}}%
\pgfpathcurveto{\pgfqpoint{2.622920in}{3.065258in}}{\pgfqpoint{2.625234in}{3.059671in}}{\pgfqpoint{2.629352in}{3.055553in}}%
\pgfpathcurveto{\pgfqpoint{2.633470in}{3.051435in}}{\pgfqpoint{2.639056in}{3.049121in}}{\pgfqpoint{2.644880in}{3.049121in}}%
\pgfpathlineto{\pgfqpoint{2.644880in}{3.049121in}}%
\pgfpathclose%
\pgfusepath{stroke,fill}%
\end{pgfscope}%
\begin{pgfscope}%
\pgfpathrectangle{\pgfqpoint{0.997489in}{0.528000in}}{\pgfqpoint{4.565023in}{3.696000in}}%
\pgfusepath{clip}%
\pgfsetbuttcap%
\pgfsetroundjoin%
\definecolor{currentfill}{rgb}{0.800000,0.200000,0.200000}%
\pgfsetfillcolor{currentfill}%
\pgfsetlinewidth{1.003750pt}%
\definecolor{currentstroke}{rgb}{0.800000,0.200000,0.200000}%
\pgfsetstrokecolor{currentstroke}%
\pgfsetdash{}{0pt}%
\pgfpathmoveto{\pgfqpoint{2.576360in}{3.036700in}}%
\pgfpathcurveto{\pgfqpoint{2.582184in}{3.036700in}}{\pgfqpoint{2.587770in}{3.039014in}}{\pgfqpoint{2.591889in}{3.043132in}}%
\pgfpathcurveto{\pgfqpoint{2.596007in}{3.047250in}}{\pgfqpoint{2.598321in}{3.052836in}}{\pgfqpoint{2.598321in}{3.058660in}}%
\pgfpathcurveto{\pgfqpoint{2.598321in}{3.064484in}}{\pgfqpoint{2.596007in}{3.070071in}}{\pgfqpoint{2.591889in}{3.074189in}}%
\pgfpathcurveto{\pgfqpoint{2.587770in}{3.078307in}}{\pgfqpoint{2.582184in}{3.080621in}}{\pgfqpoint{2.576360in}{3.080621in}}%
\pgfpathcurveto{\pgfqpoint{2.570536in}{3.080621in}}{\pgfqpoint{2.564950in}{3.078307in}}{\pgfqpoint{2.560832in}{3.074189in}}%
\pgfpathcurveto{\pgfqpoint{2.556714in}{3.070071in}}{\pgfqpoint{2.554400in}{3.064484in}}{\pgfqpoint{2.554400in}{3.058660in}}%
\pgfpathcurveto{\pgfqpoint{2.554400in}{3.052836in}}{\pgfqpoint{2.556714in}{3.047250in}}{\pgfqpoint{2.560832in}{3.043132in}}%
\pgfpathcurveto{\pgfqpoint{2.564950in}{3.039014in}}{\pgfqpoint{2.570536in}{3.036700in}}{\pgfqpoint{2.576360in}{3.036700in}}%
\pgfpathlineto{\pgfqpoint{2.576360in}{3.036700in}}%
\pgfpathclose%
\pgfusepath{stroke,fill}%
\end{pgfscope}%
\begin{pgfscope}%
\pgfpathrectangle{\pgfqpoint{0.997489in}{0.528000in}}{\pgfqpoint{4.565023in}{3.696000in}}%
\pgfusepath{clip}%
\pgfsetbuttcap%
\pgfsetroundjoin%
\definecolor{currentfill}{rgb}{0.800000,0.200000,0.200000}%
\pgfsetfillcolor{currentfill}%
\pgfsetlinewidth{1.003750pt}%
\definecolor{currentstroke}{rgb}{0.800000,0.200000,0.200000}%
\pgfsetstrokecolor{currentstroke}%
\pgfsetdash{}{0pt}%
\pgfpathmoveto{\pgfqpoint{2.525703in}{2.999715in}}%
\pgfpathcurveto{\pgfqpoint{2.531527in}{2.999715in}}{\pgfqpoint{2.537113in}{3.002028in}}{\pgfqpoint{2.541231in}{3.006147in}}%
\pgfpathcurveto{\pgfqpoint{2.545349in}{3.010265in}}{\pgfqpoint{2.547663in}{3.015851in}}{\pgfqpoint{2.547663in}{3.021675in}}%
\pgfpathcurveto{\pgfqpoint{2.547663in}{3.027499in}}{\pgfqpoint{2.545349in}{3.033085in}}{\pgfqpoint{2.541231in}{3.037203in}}%
\pgfpathcurveto{\pgfqpoint{2.537113in}{3.041321in}}{\pgfqpoint{2.531527in}{3.043635in}}{\pgfqpoint{2.525703in}{3.043635in}}%
\pgfpathcurveto{\pgfqpoint{2.519879in}{3.043635in}}{\pgfqpoint{2.514293in}{3.041321in}}{\pgfqpoint{2.510174in}{3.037203in}}%
\pgfpathcurveto{\pgfqpoint{2.506056in}{3.033085in}}{\pgfqpoint{2.503742in}{3.027499in}}{\pgfqpoint{2.503742in}{3.021675in}}%
\pgfpathcurveto{\pgfqpoint{2.503742in}{3.015851in}}{\pgfqpoint{2.506056in}{3.010265in}}{\pgfqpoint{2.510174in}{3.006147in}}%
\pgfpathcurveto{\pgfqpoint{2.514293in}{3.002028in}}{\pgfqpoint{2.519879in}{2.999715in}}{\pgfqpoint{2.525703in}{2.999715in}}%
\pgfpathlineto{\pgfqpoint{2.525703in}{2.999715in}}%
\pgfpathclose%
\pgfusepath{stroke,fill}%
\end{pgfscope}%
\begin{pgfscope}%
\pgfpathrectangle{\pgfqpoint{0.997489in}{0.528000in}}{\pgfqpoint{4.565023in}{3.696000in}}%
\pgfusepath{clip}%
\pgfsetbuttcap%
\pgfsetroundjoin%
\definecolor{currentfill}{rgb}{0.800000,0.200000,0.200000}%
\pgfsetfillcolor{currentfill}%
\pgfsetlinewidth{1.003750pt}%
\definecolor{currentstroke}{rgb}{0.800000,0.200000,0.200000}%
\pgfsetstrokecolor{currentstroke}%
\pgfsetdash{}{0pt}%
\pgfpathmoveto{\pgfqpoint{2.476296in}{2.960048in}}%
\pgfpathcurveto{\pgfqpoint{2.482120in}{2.960048in}}{\pgfqpoint{2.487706in}{2.962362in}}{\pgfqpoint{2.491825in}{2.966480in}}%
\pgfpathcurveto{\pgfqpoint{2.495943in}{2.970598in}}{\pgfqpoint{2.498257in}{2.976184in}}{\pgfqpoint{2.498257in}{2.982008in}}%
\pgfpathcurveto{\pgfqpoint{2.498257in}{2.987832in}}{\pgfqpoint{2.495943in}{2.993418in}}{\pgfqpoint{2.491825in}{2.997536in}}%
\pgfpathcurveto{\pgfqpoint{2.487706in}{3.001654in}}{\pgfqpoint{2.482120in}{3.003968in}}{\pgfqpoint{2.476296in}{3.003968in}}%
\pgfpathcurveto{\pgfqpoint{2.470472in}{3.003968in}}{\pgfqpoint{2.464886in}{3.001654in}}{\pgfqpoint{2.460768in}{2.997536in}}%
\pgfpathcurveto{\pgfqpoint{2.456650in}{2.993418in}}{\pgfqpoint{2.454336in}{2.987832in}}{\pgfqpoint{2.454336in}{2.982008in}}%
\pgfpathcurveto{\pgfqpoint{2.454336in}{2.976184in}}{\pgfqpoint{2.456650in}{2.970598in}}{\pgfqpoint{2.460768in}{2.966480in}}%
\pgfpathcurveto{\pgfqpoint{2.464886in}{2.962362in}}{\pgfqpoint{2.470472in}{2.960048in}}{\pgfqpoint{2.476296in}{2.960048in}}%
\pgfpathlineto{\pgfqpoint{2.476296in}{2.960048in}}%
\pgfpathclose%
\pgfusepath{stroke,fill}%
\end{pgfscope}%
\begin{pgfscope}%
\pgfpathrectangle{\pgfqpoint{0.997489in}{0.528000in}}{\pgfqpoint{4.565023in}{3.696000in}}%
\pgfusepath{clip}%
\pgfsetbuttcap%
\pgfsetroundjoin%
\definecolor{currentfill}{rgb}{0.800000,0.200000,0.200000}%
\pgfsetfillcolor{currentfill}%
\pgfsetlinewidth{1.003750pt}%
\definecolor{currentstroke}{rgb}{0.800000,0.200000,0.200000}%
\pgfsetstrokecolor{currentstroke}%
\pgfsetdash{}{0pt}%
\pgfpathmoveto{\pgfqpoint{2.372694in}{2.961327in}}%
\pgfpathcurveto{\pgfqpoint{2.378518in}{2.961327in}}{\pgfqpoint{2.384104in}{2.963641in}}{\pgfqpoint{2.388223in}{2.967759in}}%
\pgfpathcurveto{\pgfqpoint{2.392341in}{2.971877in}}{\pgfqpoint{2.394655in}{2.977463in}}{\pgfqpoint{2.394655in}{2.983287in}}%
\pgfpathcurveto{\pgfqpoint{2.394655in}{2.989111in}}{\pgfqpoint{2.392341in}{2.994697in}}{\pgfqpoint{2.388223in}{2.998816in}}%
\pgfpathcurveto{\pgfqpoint{2.384104in}{3.002934in}}{\pgfqpoint{2.378518in}{3.005248in}}{\pgfqpoint{2.372694in}{3.005248in}}%
\pgfpathcurveto{\pgfqpoint{2.366870in}{3.005248in}}{\pgfqpoint{2.361284in}{3.002934in}}{\pgfqpoint{2.357166in}{2.998816in}}%
\pgfpathcurveto{\pgfqpoint{2.353048in}{2.994697in}}{\pgfqpoint{2.350734in}{2.989111in}}{\pgfqpoint{2.350734in}{2.983287in}}%
\pgfpathcurveto{\pgfqpoint{2.350734in}{2.977463in}}{\pgfqpoint{2.353048in}{2.971877in}}{\pgfqpoint{2.357166in}{2.967759in}}%
\pgfpathcurveto{\pgfqpoint{2.361284in}{2.963641in}}{\pgfqpoint{2.366870in}{2.961327in}}{\pgfqpoint{2.372694in}{2.961327in}}%
\pgfpathlineto{\pgfqpoint{2.372694in}{2.961327in}}%
\pgfpathclose%
\pgfusepath{stroke,fill}%
\end{pgfscope}%
\begin{pgfscope}%
\pgfpathrectangle{\pgfqpoint{0.997489in}{0.528000in}}{\pgfqpoint{4.565023in}{3.696000in}}%
\pgfusepath{clip}%
\pgfsetbuttcap%
\pgfsetroundjoin%
\definecolor{currentfill}{rgb}{0.800000,0.200000,0.200000}%
\pgfsetfillcolor{currentfill}%
\pgfsetlinewidth{1.003750pt}%
\definecolor{currentstroke}{rgb}{0.800000,0.200000,0.200000}%
\pgfsetstrokecolor{currentstroke}%
\pgfsetdash{}{0pt}%
\pgfpathmoveto{\pgfqpoint{2.399001in}{2.860508in}}%
\pgfpathcurveto{\pgfqpoint{2.404825in}{2.860508in}}{\pgfqpoint{2.410411in}{2.862821in}}{\pgfqpoint{2.414529in}{2.866940in}}%
\pgfpathcurveto{\pgfqpoint{2.418647in}{2.871058in}}{\pgfqpoint{2.420961in}{2.876644in}}{\pgfqpoint{2.420961in}{2.882468in}}%
\pgfpathcurveto{\pgfqpoint{2.420961in}{2.888292in}}{\pgfqpoint{2.418647in}{2.893878in}}{\pgfqpoint{2.414529in}{2.897996in}}%
\pgfpathcurveto{\pgfqpoint{2.410411in}{2.902114in}}{\pgfqpoint{2.404825in}{2.904428in}}{\pgfqpoint{2.399001in}{2.904428in}}%
\pgfpathcurveto{\pgfqpoint{2.393177in}{2.904428in}}{\pgfqpoint{2.387591in}{2.902114in}}{\pgfqpoint{2.383473in}{2.897996in}}%
\pgfpathcurveto{\pgfqpoint{2.379355in}{2.893878in}}{\pgfqpoint{2.377041in}{2.888292in}}{\pgfqpoint{2.377041in}{2.882468in}}%
\pgfpathcurveto{\pgfqpoint{2.377041in}{2.876644in}}{\pgfqpoint{2.379355in}{2.871058in}}{\pgfqpoint{2.383473in}{2.866940in}}%
\pgfpathcurveto{\pgfqpoint{2.387591in}{2.862821in}}{\pgfqpoint{2.393177in}{2.860508in}}{\pgfqpoint{2.399001in}{2.860508in}}%
\pgfpathlineto{\pgfqpoint{2.399001in}{2.860508in}}%
\pgfpathclose%
\pgfusepath{stroke,fill}%
\end{pgfscope}%
\begin{pgfscope}%
\pgfpathrectangle{\pgfqpoint{0.997489in}{0.528000in}}{\pgfqpoint{4.565023in}{3.696000in}}%
\pgfusepath{clip}%
\pgfsetbuttcap%
\pgfsetroundjoin%
\definecolor{currentfill}{rgb}{0.800000,0.200000,0.200000}%
\pgfsetfillcolor{currentfill}%
\pgfsetlinewidth{1.003750pt}%
\definecolor{currentstroke}{rgb}{0.800000,0.200000,0.200000}%
\pgfsetstrokecolor{currentstroke}%
\pgfsetdash{}{0pt}%
\pgfpathmoveto{\pgfqpoint{2.313562in}{2.838246in}}%
\pgfpathcurveto{\pgfqpoint{2.319386in}{2.838246in}}{\pgfqpoint{2.324972in}{2.840560in}}{\pgfqpoint{2.329090in}{2.844678in}}%
\pgfpathcurveto{\pgfqpoint{2.333208in}{2.848797in}}{\pgfqpoint{2.335522in}{2.854383in}}{\pgfqpoint{2.335522in}{2.860207in}}%
\pgfpathcurveto{\pgfqpoint{2.335522in}{2.866031in}}{\pgfqpoint{2.333208in}{2.871617in}}{\pgfqpoint{2.329090in}{2.875735in}}%
\pgfpathcurveto{\pgfqpoint{2.324972in}{2.879853in}}{\pgfqpoint{2.319386in}{2.882167in}}{\pgfqpoint{2.313562in}{2.882167in}}%
\pgfpathcurveto{\pgfqpoint{2.307738in}{2.882167in}}{\pgfqpoint{2.302152in}{2.879853in}}{\pgfqpoint{2.298034in}{2.875735in}}%
\pgfpathcurveto{\pgfqpoint{2.293916in}{2.871617in}}{\pgfqpoint{2.291602in}{2.866031in}}{\pgfqpoint{2.291602in}{2.860207in}}%
\pgfpathcurveto{\pgfqpoint{2.291602in}{2.854383in}}{\pgfqpoint{2.293916in}{2.848797in}}{\pgfqpoint{2.298034in}{2.844678in}}%
\pgfpathcurveto{\pgfqpoint{2.302152in}{2.840560in}}{\pgfqpoint{2.307738in}{2.838246in}}{\pgfqpoint{2.313562in}{2.838246in}}%
\pgfpathlineto{\pgfqpoint{2.313562in}{2.838246in}}%
\pgfpathclose%
\pgfusepath{stroke,fill}%
\end{pgfscope}%
\begin{pgfscope}%
\pgfpathrectangle{\pgfqpoint{0.997489in}{0.528000in}}{\pgfqpoint{4.565023in}{3.696000in}}%
\pgfusepath{clip}%
\pgfsetbuttcap%
\pgfsetroundjoin%
\definecolor{currentfill}{rgb}{0.800000,0.200000,0.200000}%
\pgfsetfillcolor{currentfill}%
\pgfsetlinewidth{1.003750pt}%
\definecolor{currentstroke}{rgb}{0.800000,0.200000,0.200000}%
\pgfsetstrokecolor{currentstroke}%
\pgfsetdash{}{0pt}%
\pgfpathmoveto{\pgfqpoint{2.344500in}{2.747159in}}%
\pgfpathcurveto{\pgfqpoint{2.350324in}{2.747159in}}{\pgfqpoint{2.355910in}{2.749473in}}{\pgfqpoint{2.360029in}{2.753591in}}%
\pgfpathcurveto{\pgfqpoint{2.364147in}{2.757709in}}{\pgfqpoint{2.366461in}{2.763295in}}{\pgfqpoint{2.366461in}{2.769119in}}%
\pgfpathcurveto{\pgfqpoint{2.366461in}{2.774943in}}{\pgfqpoint{2.364147in}{2.780529in}}{\pgfqpoint{2.360029in}{2.784648in}}%
\pgfpathcurveto{\pgfqpoint{2.355910in}{2.788766in}}{\pgfqpoint{2.350324in}{2.791080in}}{\pgfqpoint{2.344500in}{2.791080in}}%
\pgfpathcurveto{\pgfqpoint{2.338676in}{2.791080in}}{\pgfqpoint{2.333090in}{2.788766in}}{\pgfqpoint{2.328972in}{2.784648in}}%
\pgfpathcurveto{\pgfqpoint{2.324854in}{2.780529in}}{\pgfqpoint{2.322540in}{2.774943in}}{\pgfqpoint{2.322540in}{2.769119in}}%
\pgfpathcurveto{\pgfqpoint{2.322540in}{2.763295in}}{\pgfqpoint{2.324854in}{2.757709in}}{\pgfqpoint{2.328972in}{2.753591in}}%
\pgfpathcurveto{\pgfqpoint{2.333090in}{2.749473in}}{\pgfqpoint{2.338676in}{2.747159in}}{\pgfqpoint{2.344500in}{2.747159in}}%
\pgfpathlineto{\pgfqpoint{2.344500in}{2.747159in}}%
\pgfpathclose%
\pgfusepath{stroke,fill}%
\end{pgfscope}%
\begin{pgfscope}%
\pgfpathrectangle{\pgfqpoint{0.997489in}{0.528000in}}{\pgfqpoint{4.565023in}{3.696000in}}%
\pgfusepath{clip}%
\pgfsetbuttcap%
\pgfsetroundjoin%
\definecolor{currentfill}{rgb}{0.800000,0.200000,0.200000}%
\pgfsetfillcolor{currentfill}%
\pgfsetlinewidth{1.003750pt}%
\definecolor{currentstroke}{rgb}{0.800000,0.200000,0.200000}%
\pgfsetstrokecolor{currentstroke}%
\pgfsetdash{}{0pt}%
\pgfpathmoveto{\pgfqpoint{2.294450in}{2.701103in}}%
\pgfpathcurveto{\pgfqpoint{2.300274in}{2.701103in}}{\pgfqpoint{2.305860in}{2.703417in}}{\pgfqpoint{2.309978in}{2.707535in}}%
\pgfpathcurveto{\pgfqpoint{2.314097in}{2.711653in}}{\pgfqpoint{2.316410in}{2.717239in}}{\pgfqpoint{2.316410in}{2.723063in}}%
\pgfpathcurveto{\pgfqpoint{2.316410in}{2.728887in}}{\pgfqpoint{2.314097in}{2.734473in}}{\pgfqpoint{2.309978in}{2.738591in}}%
\pgfpathcurveto{\pgfqpoint{2.305860in}{2.742709in}}{\pgfqpoint{2.300274in}{2.745023in}}{\pgfqpoint{2.294450in}{2.745023in}}%
\pgfpathcurveto{\pgfqpoint{2.288626in}{2.745023in}}{\pgfqpoint{2.283040in}{2.742709in}}{\pgfqpoint{2.278922in}{2.738591in}}%
\pgfpathcurveto{\pgfqpoint{2.274804in}{2.734473in}}{\pgfqpoint{2.272490in}{2.728887in}}{\pgfqpoint{2.272490in}{2.723063in}}%
\pgfpathcurveto{\pgfqpoint{2.272490in}{2.717239in}}{\pgfqpoint{2.274804in}{2.711653in}}{\pgfqpoint{2.278922in}{2.707535in}}%
\pgfpathcurveto{\pgfqpoint{2.283040in}{2.703417in}}{\pgfqpoint{2.288626in}{2.701103in}}{\pgfqpoint{2.294450in}{2.701103in}}%
\pgfpathlineto{\pgfqpoint{2.294450in}{2.701103in}}%
\pgfpathclose%
\pgfusepath{stroke,fill}%
\end{pgfscope}%
\begin{pgfscope}%
\pgfpathrectangle{\pgfqpoint{0.997489in}{0.528000in}}{\pgfqpoint{4.565023in}{3.696000in}}%
\pgfusepath{clip}%
\pgfsetbuttcap%
\pgfsetroundjoin%
\definecolor{currentfill}{rgb}{0.800000,0.200000,0.200000}%
\pgfsetfillcolor{currentfill}%
\pgfsetlinewidth{1.003750pt}%
\definecolor{currentstroke}{rgb}{0.800000,0.200000,0.200000}%
\pgfsetstrokecolor{currentstroke}%
\pgfsetdash{}{0pt}%
\pgfpathmoveto{\pgfqpoint{2.373750in}{2.604270in}}%
\pgfpathcurveto{\pgfqpoint{2.379574in}{2.604270in}}{\pgfqpoint{2.385161in}{2.606584in}}{\pgfqpoint{2.389279in}{2.610702in}}%
\pgfpathcurveto{\pgfqpoint{2.393397in}{2.614820in}}{\pgfqpoint{2.395711in}{2.620406in}}{\pgfqpoint{2.395711in}{2.626230in}}%
\pgfpathcurveto{\pgfqpoint{2.395711in}{2.632054in}}{\pgfqpoint{2.393397in}{2.637640in}}{\pgfqpoint{2.389279in}{2.641758in}}%
\pgfpathcurveto{\pgfqpoint{2.385161in}{2.645877in}}{\pgfqpoint{2.379574in}{2.648190in}}{\pgfqpoint{2.373750in}{2.648190in}}%
\pgfpathcurveto{\pgfqpoint{2.367927in}{2.648190in}}{\pgfqpoint{2.362340in}{2.645877in}}{\pgfqpoint{2.358222in}{2.641758in}}%
\pgfpathcurveto{\pgfqpoint{2.354104in}{2.637640in}}{\pgfqpoint{2.351790in}{2.632054in}}{\pgfqpoint{2.351790in}{2.626230in}}%
\pgfpathcurveto{\pgfqpoint{2.351790in}{2.620406in}}{\pgfqpoint{2.354104in}{2.614820in}}{\pgfqpoint{2.358222in}{2.610702in}}%
\pgfpathcurveto{\pgfqpoint{2.362340in}{2.606584in}}{\pgfqpoint{2.367927in}{2.604270in}}{\pgfqpoint{2.373750in}{2.604270in}}%
\pgfpathlineto{\pgfqpoint{2.373750in}{2.604270in}}%
\pgfpathclose%
\pgfusepath{stroke,fill}%
\end{pgfscope}%
\begin{pgfscope}%
\pgfpathrectangle{\pgfqpoint{0.997489in}{0.528000in}}{\pgfqpoint{4.565023in}{3.696000in}}%
\pgfusepath{clip}%
\pgfsetbuttcap%
\pgfsetroundjoin%
\definecolor{currentfill}{rgb}{0.800000,0.200000,0.200000}%
\pgfsetfillcolor{currentfill}%
\pgfsetlinewidth{1.003750pt}%
\definecolor{currentstroke}{rgb}{0.800000,0.200000,0.200000}%
\pgfsetstrokecolor{currentstroke}%
\pgfsetdash{}{0pt}%
\pgfpathmoveto{\pgfqpoint{2.240402in}{2.584106in}}%
\pgfpathcurveto{\pgfqpoint{2.246226in}{2.584106in}}{\pgfqpoint{2.251812in}{2.586420in}}{\pgfqpoint{2.255930in}{2.590538in}}%
\pgfpathcurveto{\pgfqpoint{2.260048in}{2.594657in}}{\pgfqpoint{2.262362in}{2.600243in}}{\pgfqpoint{2.262362in}{2.606067in}}%
\pgfpathcurveto{\pgfqpoint{2.262362in}{2.611891in}}{\pgfqpoint{2.260048in}{2.617477in}}{\pgfqpoint{2.255930in}{2.621595in}}%
\pgfpathcurveto{\pgfqpoint{2.251812in}{2.625713in}}{\pgfqpoint{2.246226in}{2.628027in}}{\pgfqpoint{2.240402in}{2.628027in}}%
\pgfpathcurveto{\pgfqpoint{2.234578in}{2.628027in}}{\pgfqpoint{2.228992in}{2.625713in}}{\pgfqpoint{2.224874in}{2.621595in}}%
\pgfpathcurveto{\pgfqpoint{2.220756in}{2.617477in}}{\pgfqpoint{2.218442in}{2.611891in}}{\pgfqpoint{2.218442in}{2.606067in}}%
\pgfpathcurveto{\pgfqpoint{2.218442in}{2.600243in}}{\pgfqpoint{2.220756in}{2.594657in}}{\pgfqpoint{2.224874in}{2.590538in}}%
\pgfpathcurveto{\pgfqpoint{2.228992in}{2.586420in}}{\pgfqpoint{2.234578in}{2.584106in}}{\pgfqpoint{2.240402in}{2.584106in}}%
\pgfpathlineto{\pgfqpoint{2.240402in}{2.584106in}}%
\pgfpathclose%
\pgfusepath{stroke,fill}%
\end{pgfscope}%
\begin{pgfscope}%
\pgfpathrectangle{\pgfqpoint{0.997489in}{0.528000in}}{\pgfqpoint{4.565023in}{3.696000in}}%
\pgfusepath{clip}%
\pgfsetbuttcap%
\pgfsetroundjoin%
\definecolor{currentfill}{rgb}{0.800000,0.200000,0.200000}%
\pgfsetfillcolor{currentfill}%
\pgfsetlinewidth{1.003750pt}%
\definecolor{currentstroke}{rgb}{0.800000,0.200000,0.200000}%
\pgfsetstrokecolor{currentstroke}%
\pgfsetdash{}{0pt}%
\pgfpathmoveto{\pgfqpoint{2.221299in}{2.522239in}}%
\pgfpathcurveto{\pgfqpoint{2.227123in}{2.522239in}}{\pgfqpoint{2.232709in}{2.524553in}}{\pgfqpoint{2.236827in}{2.528671in}}%
\pgfpathcurveto{\pgfqpoint{2.240945in}{2.532790in}}{\pgfqpoint{2.243259in}{2.538376in}}{\pgfqpoint{2.243259in}{2.544200in}}%
\pgfpathcurveto{\pgfqpoint{2.243259in}{2.550024in}}{\pgfqpoint{2.240945in}{2.555610in}}{\pgfqpoint{2.236827in}{2.559728in}}%
\pgfpathcurveto{\pgfqpoint{2.232709in}{2.563846in}}{\pgfqpoint{2.227123in}{2.566160in}}{\pgfqpoint{2.221299in}{2.566160in}}%
\pgfpathcurveto{\pgfqpoint{2.215475in}{2.566160in}}{\pgfqpoint{2.209889in}{2.563846in}}{\pgfqpoint{2.205771in}{2.559728in}}%
\pgfpathcurveto{\pgfqpoint{2.201653in}{2.555610in}}{\pgfqpoint{2.199339in}{2.550024in}}{\pgfqpoint{2.199339in}{2.544200in}}%
\pgfpathcurveto{\pgfqpoint{2.199339in}{2.538376in}}{\pgfqpoint{2.201653in}{2.532790in}}{\pgfqpoint{2.205771in}{2.528671in}}%
\pgfpathcurveto{\pgfqpoint{2.209889in}{2.524553in}}{\pgfqpoint{2.215475in}{2.522239in}}{\pgfqpoint{2.221299in}{2.522239in}}%
\pgfpathlineto{\pgfqpoint{2.221299in}{2.522239in}}%
\pgfpathclose%
\pgfusepath{stroke,fill}%
\end{pgfscope}%
\begin{pgfscope}%
\pgfpathrectangle{\pgfqpoint{0.997489in}{0.528000in}}{\pgfqpoint{4.565023in}{3.696000in}}%
\pgfusepath{clip}%
\pgfsetbuttcap%
\pgfsetroundjoin%
\definecolor{currentfill}{rgb}{0.800000,0.200000,0.200000}%
\pgfsetfillcolor{currentfill}%
\pgfsetlinewidth{1.003750pt}%
\definecolor{currentstroke}{rgb}{0.800000,0.200000,0.200000}%
\pgfsetstrokecolor{currentstroke}%
\pgfsetdash{}{0pt}%
\pgfpathmoveto{\pgfqpoint{2.218249in}{2.457161in}}%
\pgfpathcurveto{\pgfqpoint{2.224073in}{2.457161in}}{\pgfqpoint{2.229659in}{2.459475in}}{\pgfqpoint{2.233777in}{2.463593in}}%
\pgfpathcurveto{\pgfqpoint{2.237895in}{2.467711in}}{\pgfqpoint{2.240209in}{2.473297in}}{\pgfqpoint{2.240209in}{2.479121in}}%
\pgfpathcurveto{\pgfqpoint{2.240209in}{2.484945in}}{\pgfqpoint{2.237895in}{2.490531in}}{\pgfqpoint{2.233777in}{2.494650in}}%
\pgfpathcurveto{\pgfqpoint{2.229659in}{2.498768in}}{\pgfqpoint{2.224073in}{2.501082in}}{\pgfqpoint{2.218249in}{2.501082in}}%
\pgfpathcurveto{\pgfqpoint{2.212425in}{2.501082in}}{\pgfqpoint{2.206839in}{2.498768in}}{\pgfqpoint{2.202721in}{2.494650in}}%
\pgfpathcurveto{\pgfqpoint{2.198603in}{2.490531in}}{\pgfqpoint{2.196289in}{2.484945in}}{\pgfqpoint{2.196289in}{2.479121in}}%
\pgfpathcurveto{\pgfqpoint{2.196289in}{2.473297in}}{\pgfqpoint{2.198603in}{2.467711in}}{\pgfqpoint{2.202721in}{2.463593in}}%
\pgfpathcurveto{\pgfqpoint{2.206839in}{2.459475in}}{\pgfqpoint{2.212425in}{2.457161in}}{\pgfqpoint{2.218249in}{2.457161in}}%
\pgfpathlineto{\pgfqpoint{2.218249in}{2.457161in}}%
\pgfpathclose%
\pgfusepath{stroke,fill}%
\end{pgfscope}%
\begin{pgfscope}%
\pgfpathrectangle{\pgfqpoint{0.997489in}{0.528000in}}{\pgfqpoint{4.565023in}{3.696000in}}%
\pgfusepath{clip}%
\pgfsetbuttcap%
\pgfsetroundjoin%
\definecolor{currentfill}{rgb}{0.800000,0.200000,0.200000}%
\pgfsetfillcolor{currentfill}%
\pgfsetlinewidth{1.003750pt}%
\definecolor{currentstroke}{rgb}{0.800000,0.200000,0.200000}%
\pgfsetstrokecolor{currentstroke}%
\pgfsetdash{}{0pt}%
\pgfpathmoveto{\pgfqpoint{2.098896in}{2.404115in}}%
\pgfpathcurveto{\pgfqpoint{2.104720in}{2.404115in}}{\pgfqpoint{2.110306in}{2.406429in}}{\pgfqpoint{2.114425in}{2.410547in}}%
\pgfpathcurveto{\pgfqpoint{2.118543in}{2.414665in}}{\pgfqpoint{2.120857in}{2.420251in}}{\pgfqpoint{2.120857in}{2.426075in}}%
\pgfpathcurveto{\pgfqpoint{2.120857in}{2.431899in}}{\pgfqpoint{2.118543in}{2.437485in}}{\pgfqpoint{2.114425in}{2.441603in}}%
\pgfpathcurveto{\pgfqpoint{2.110306in}{2.445721in}}{\pgfqpoint{2.104720in}{2.448035in}}{\pgfqpoint{2.098896in}{2.448035in}}%
\pgfpathcurveto{\pgfqpoint{2.093072in}{2.448035in}}{\pgfqpoint{2.087486in}{2.445721in}}{\pgfqpoint{2.083368in}{2.441603in}}%
\pgfpathcurveto{\pgfqpoint{2.079250in}{2.437485in}}{\pgfqpoint{2.076936in}{2.431899in}}{\pgfqpoint{2.076936in}{2.426075in}}%
\pgfpathcurveto{\pgfqpoint{2.076936in}{2.420251in}}{\pgfqpoint{2.079250in}{2.414665in}}{\pgfqpoint{2.083368in}{2.410547in}}%
\pgfpathcurveto{\pgfqpoint{2.087486in}{2.406429in}}{\pgfqpoint{2.093072in}{2.404115in}}{\pgfqpoint{2.098896in}{2.404115in}}%
\pgfpathlineto{\pgfqpoint{2.098896in}{2.404115in}}%
\pgfpathclose%
\pgfusepath{stroke,fill}%
\end{pgfscope}%
\begin{pgfscope}%
\pgfpathrectangle{\pgfqpoint{0.997489in}{0.528000in}}{\pgfqpoint{4.565023in}{3.696000in}}%
\pgfusepath{clip}%
\pgfsetbuttcap%
\pgfsetroundjoin%
\definecolor{currentfill}{rgb}{0.800000,0.200000,0.200000}%
\pgfsetfillcolor{currentfill}%
\pgfsetlinewidth{1.003750pt}%
\definecolor{currentstroke}{rgb}{0.800000,0.200000,0.200000}%
\pgfsetstrokecolor{currentstroke}%
\pgfsetdash{}{0pt}%
\pgfpathmoveto{\pgfqpoint{2.199391in}{2.329651in}}%
\pgfpathcurveto{\pgfqpoint{2.205215in}{2.329651in}}{\pgfqpoint{2.210801in}{2.331965in}}{\pgfqpoint{2.214919in}{2.336083in}}%
\pgfpathcurveto{\pgfqpoint{2.219037in}{2.340201in}}{\pgfqpoint{2.221351in}{2.345787in}}{\pgfqpoint{2.221351in}{2.351611in}}%
\pgfpathcurveto{\pgfqpoint{2.221351in}{2.357435in}}{\pgfqpoint{2.219037in}{2.363021in}}{\pgfqpoint{2.214919in}{2.367139in}}%
\pgfpathcurveto{\pgfqpoint{2.210801in}{2.371257in}}{\pgfqpoint{2.205215in}{2.373571in}}{\pgfqpoint{2.199391in}{2.373571in}}%
\pgfpathcurveto{\pgfqpoint{2.193567in}{2.373571in}}{\pgfqpoint{2.187981in}{2.371257in}}{\pgfqpoint{2.183863in}{2.367139in}}%
\pgfpathcurveto{\pgfqpoint{2.179744in}{2.363021in}}{\pgfqpoint{2.177431in}{2.357435in}}{\pgfqpoint{2.177431in}{2.351611in}}%
\pgfpathcurveto{\pgfqpoint{2.177431in}{2.345787in}}{\pgfqpoint{2.179744in}{2.340201in}}{\pgfqpoint{2.183863in}{2.336083in}}%
\pgfpathcurveto{\pgfqpoint{2.187981in}{2.331965in}}{\pgfqpoint{2.193567in}{2.329651in}}{\pgfqpoint{2.199391in}{2.329651in}}%
\pgfpathlineto{\pgfqpoint{2.199391in}{2.329651in}}%
\pgfpathclose%
\pgfusepath{stroke,fill}%
\end{pgfscope}%
\begin{pgfscope}%
\pgfpathrectangle{\pgfqpoint{0.997489in}{0.528000in}}{\pgfqpoint{4.565023in}{3.696000in}}%
\pgfusepath{clip}%
\pgfsetbuttcap%
\pgfsetroundjoin%
\definecolor{currentfill}{rgb}{0.800000,0.200000,0.200000}%
\pgfsetfillcolor{currentfill}%
\pgfsetlinewidth{1.003750pt}%
\definecolor{currentstroke}{rgb}{0.800000,0.200000,0.200000}%
\pgfsetstrokecolor{currentstroke}%
\pgfsetdash{}{0pt}%
\pgfpathmoveto{\pgfqpoint{2.336813in}{2.269406in}}%
\pgfpathcurveto{\pgfqpoint{2.342637in}{2.269406in}}{\pgfqpoint{2.348224in}{2.271720in}}{\pgfqpoint{2.352342in}{2.275839in}}%
\pgfpathcurveto{\pgfqpoint{2.356460in}{2.279957in}}{\pgfqpoint{2.358774in}{2.285543in}}{\pgfqpoint{2.358774in}{2.291367in}}%
\pgfpathcurveto{\pgfqpoint{2.358774in}{2.297191in}}{\pgfqpoint{2.356460in}{2.302777in}}{\pgfqpoint{2.352342in}{2.306895in}}%
\pgfpathcurveto{\pgfqpoint{2.348224in}{2.311013in}}{\pgfqpoint{2.342637in}{2.313327in}}{\pgfqpoint{2.336813in}{2.313327in}}%
\pgfpathcurveto{\pgfqpoint{2.330989in}{2.313327in}}{\pgfqpoint{2.325403in}{2.311013in}}{\pgfqpoint{2.321285in}{2.306895in}}%
\pgfpathcurveto{\pgfqpoint{2.317167in}{2.302777in}}{\pgfqpoint{2.314853in}{2.297191in}}{\pgfqpoint{2.314853in}{2.291367in}}%
\pgfpathcurveto{\pgfqpoint{2.314853in}{2.285543in}}{\pgfqpoint{2.317167in}{2.279957in}}{\pgfqpoint{2.321285in}{2.275839in}}%
\pgfpathcurveto{\pgfqpoint{2.325403in}{2.271720in}}{\pgfqpoint{2.330989in}{2.269406in}}{\pgfqpoint{2.336813in}{2.269406in}}%
\pgfpathlineto{\pgfqpoint{2.336813in}{2.269406in}}%
\pgfpathclose%
\pgfusepath{stroke,fill}%
\end{pgfscope}%
\begin{pgfscope}%
\pgfpathrectangle{\pgfqpoint{0.997489in}{0.528000in}}{\pgfqpoint{4.565023in}{3.696000in}}%
\pgfusepath{clip}%
\pgfsetbuttcap%
\pgfsetroundjoin%
\definecolor{currentfill}{rgb}{0.800000,0.200000,0.200000}%
\pgfsetfillcolor{currentfill}%
\pgfsetlinewidth{1.003750pt}%
\definecolor{currentstroke}{rgb}{0.800000,0.200000,0.200000}%
\pgfsetstrokecolor{currentstroke}%
\pgfsetdash{}{0pt}%
\pgfpathmoveto{\pgfqpoint{2.262944in}{2.206245in}}%
\pgfpathcurveto{\pgfqpoint{2.268768in}{2.206245in}}{\pgfqpoint{2.274354in}{2.208559in}}{\pgfqpoint{2.278472in}{2.212677in}}%
\pgfpathcurveto{\pgfqpoint{2.282590in}{2.216795in}}{\pgfqpoint{2.284904in}{2.222381in}}{\pgfqpoint{2.284904in}{2.228205in}}%
\pgfpathcurveto{\pgfqpoint{2.284904in}{2.234029in}}{\pgfqpoint{2.282590in}{2.239615in}}{\pgfqpoint{2.278472in}{2.243733in}}%
\pgfpathcurveto{\pgfqpoint{2.274354in}{2.247852in}}{\pgfqpoint{2.268768in}{2.250165in}}{\pgfqpoint{2.262944in}{2.250165in}}%
\pgfpathcurveto{\pgfqpoint{2.257120in}{2.250165in}}{\pgfqpoint{2.251533in}{2.247852in}}{\pgfqpoint{2.247415in}{2.243733in}}%
\pgfpathcurveto{\pgfqpoint{2.243297in}{2.239615in}}{\pgfqpoint{2.240983in}{2.234029in}}{\pgfqpoint{2.240983in}{2.228205in}}%
\pgfpathcurveto{\pgfqpoint{2.240983in}{2.222381in}}{\pgfqpoint{2.243297in}{2.216795in}}{\pgfqpoint{2.247415in}{2.212677in}}%
\pgfpathcurveto{\pgfqpoint{2.251533in}{2.208559in}}{\pgfqpoint{2.257120in}{2.206245in}}{\pgfqpoint{2.262944in}{2.206245in}}%
\pgfpathlineto{\pgfqpoint{2.262944in}{2.206245in}}%
\pgfpathclose%
\pgfusepath{stroke,fill}%
\end{pgfscope}%
\begin{pgfscope}%
\pgfpathrectangle{\pgfqpoint{0.997489in}{0.528000in}}{\pgfqpoint{4.565023in}{3.696000in}}%
\pgfusepath{clip}%
\pgfsetbuttcap%
\pgfsetroundjoin%
\definecolor{currentfill}{rgb}{0.800000,0.200000,0.200000}%
\pgfsetfillcolor{currentfill}%
\pgfsetlinewidth{1.003750pt}%
\definecolor{currentstroke}{rgb}{0.800000,0.200000,0.200000}%
\pgfsetstrokecolor{currentstroke}%
\pgfsetdash{}{0pt}%
\pgfpathmoveto{\pgfqpoint{2.210537in}{2.136300in}}%
\pgfpathcurveto{\pgfqpoint{2.216360in}{2.136300in}}{\pgfqpoint{2.221947in}{2.138614in}}{\pgfqpoint{2.226065in}{2.142732in}}%
\pgfpathcurveto{\pgfqpoint{2.230183in}{2.146850in}}{\pgfqpoint{2.232497in}{2.152436in}}{\pgfqpoint{2.232497in}{2.158260in}}%
\pgfpathcurveto{\pgfqpoint{2.232497in}{2.164084in}}{\pgfqpoint{2.230183in}{2.169670in}}{\pgfqpoint{2.226065in}{2.173788in}}%
\pgfpathcurveto{\pgfqpoint{2.221947in}{2.177906in}}{\pgfqpoint{2.216360in}{2.180220in}}{\pgfqpoint{2.210537in}{2.180220in}}%
\pgfpathcurveto{\pgfqpoint{2.204713in}{2.180220in}}{\pgfqpoint{2.199126in}{2.177906in}}{\pgfqpoint{2.195008in}{2.173788in}}%
\pgfpathcurveto{\pgfqpoint{2.190890in}{2.169670in}}{\pgfqpoint{2.188576in}{2.164084in}}{\pgfqpoint{2.188576in}{2.158260in}}%
\pgfpathcurveto{\pgfqpoint{2.188576in}{2.152436in}}{\pgfqpoint{2.190890in}{2.146850in}}{\pgfqpoint{2.195008in}{2.142732in}}%
\pgfpathcurveto{\pgfqpoint{2.199126in}{2.138614in}}{\pgfqpoint{2.204713in}{2.136300in}}{\pgfqpoint{2.210537in}{2.136300in}}%
\pgfpathlineto{\pgfqpoint{2.210537in}{2.136300in}}%
\pgfpathclose%
\pgfusepath{stroke,fill}%
\end{pgfscope}%
\begin{pgfscope}%
\pgfpathrectangle{\pgfqpoint{0.997489in}{0.528000in}}{\pgfqpoint{4.565023in}{3.696000in}}%
\pgfusepath{clip}%
\pgfsetbuttcap%
\pgfsetroundjoin%
\definecolor{currentfill}{rgb}{0.200000,0.800000,0.200000}%
\pgfsetfillcolor{currentfill}%
\pgfsetlinewidth{1.003750pt}%
\definecolor{currentstroke}{rgb}{0.200000,0.800000,0.200000}%
\pgfsetstrokecolor{currentstroke}%
\pgfsetdash{}{0pt}%
\pgfpathmoveto{\pgfqpoint{2.224618in}{2.073205in}}%
\pgfpathcurveto{\pgfqpoint{2.230442in}{2.073205in}}{\pgfqpoint{2.236028in}{2.075519in}}{\pgfqpoint{2.240146in}{2.079637in}}%
\pgfpathcurveto{\pgfqpoint{2.244264in}{2.083755in}}{\pgfqpoint{2.246578in}{2.089341in}}{\pgfqpoint{2.246578in}{2.095165in}}%
\pgfpathcurveto{\pgfqpoint{2.246578in}{2.100989in}}{\pgfqpoint{2.244264in}{2.106576in}}{\pgfqpoint{2.240146in}{2.110694in}}%
\pgfpathcurveto{\pgfqpoint{2.236028in}{2.114812in}}{\pgfqpoint{2.230442in}{2.117126in}}{\pgfqpoint{2.224618in}{2.117126in}}%
\pgfpathcurveto{\pgfqpoint{2.218794in}{2.117126in}}{\pgfqpoint{2.213208in}{2.114812in}}{\pgfqpoint{2.209090in}{2.110694in}}%
\pgfpathcurveto{\pgfqpoint{2.204972in}{2.106576in}}{\pgfqpoint{2.202658in}{2.100989in}}{\pgfqpoint{2.202658in}{2.095165in}}%
\pgfpathcurveto{\pgfqpoint{2.202658in}{2.089341in}}{\pgfqpoint{2.204972in}{2.083755in}}{\pgfqpoint{2.209090in}{2.079637in}}%
\pgfpathcurveto{\pgfqpoint{2.213208in}{2.075519in}}{\pgfqpoint{2.218794in}{2.073205in}}{\pgfqpoint{2.224618in}{2.073205in}}%
\pgfpathlineto{\pgfqpoint{2.224618in}{2.073205in}}%
\pgfpathclose%
\pgfusepath{stroke,fill}%
\end{pgfscope}%
\begin{pgfscope}%
\pgfpathrectangle{\pgfqpoint{0.997489in}{0.528000in}}{\pgfqpoint{4.565023in}{3.696000in}}%
\pgfusepath{clip}%
\pgfsetbuttcap%
\pgfsetroundjoin%
\definecolor{currentfill}{rgb}{0.200000,0.800000,0.200000}%
\pgfsetfillcolor{currentfill}%
\pgfsetlinewidth{1.003750pt}%
\definecolor{currentstroke}{rgb}{0.200000,0.800000,0.200000}%
\pgfsetstrokecolor{currentstroke}%
\pgfsetdash{}{0pt}%
\pgfpathmoveto{\pgfqpoint{2.277240in}{2.021405in}}%
\pgfpathcurveto{\pgfqpoint{2.283064in}{2.021405in}}{\pgfqpoint{2.288650in}{2.023719in}}{\pgfqpoint{2.292768in}{2.027837in}}%
\pgfpathcurveto{\pgfqpoint{2.296886in}{2.031955in}}{\pgfqpoint{2.299200in}{2.037542in}}{\pgfqpoint{2.299200in}{2.043365in}}%
\pgfpathcurveto{\pgfqpoint{2.299200in}{2.049189in}}{\pgfqpoint{2.296886in}{2.054776in}}{\pgfqpoint{2.292768in}{2.058894in}}%
\pgfpathcurveto{\pgfqpoint{2.288650in}{2.063012in}}{\pgfqpoint{2.283064in}{2.065326in}}{\pgfqpoint{2.277240in}{2.065326in}}%
\pgfpathcurveto{\pgfqpoint{2.271416in}{2.065326in}}{\pgfqpoint{2.265830in}{2.063012in}}{\pgfqpoint{2.261712in}{2.058894in}}%
\pgfpathcurveto{\pgfqpoint{2.257594in}{2.054776in}}{\pgfqpoint{2.255280in}{2.049189in}}{\pgfqpoint{2.255280in}{2.043365in}}%
\pgfpathcurveto{\pgfqpoint{2.255280in}{2.037542in}}{\pgfqpoint{2.257594in}{2.031955in}}{\pgfqpoint{2.261712in}{2.027837in}}%
\pgfpathcurveto{\pgfqpoint{2.265830in}{2.023719in}}{\pgfqpoint{2.271416in}{2.021405in}}{\pgfqpoint{2.277240in}{2.021405in}}%
\pgfpathlineto{\pgfqpoint{2.277240in}{2.021405in}}%
\pgfpathclose%
\pgfusepath{stroke,fill}%
\end{pgfscope}%
\begin{pgfscope}%
\pgfpathrectangle{\pgfqpoint{0.997489in}{0.528000in}}{\pgfqpoint{4.565023in}{3.696000in}}%
\pgfusepath{clip}%
\pgfsetbuttcap%
\pgfsetroundjoin%
\definecolor{currentfill}{rgb}{0.200000,0.800000,0.200000}%
\pgfsetfillcolor{currentfill}%
\pgfsetlinewidth{1.003750pt}%
\definecolor{currentstroke}{rgb}{0.200000,0.800000,0.200000}%
\pgfsetstrokecolor{currentstroke}%
\pgfsetdash{}{0pt}%
\pgfpathmoveto{\pgfqpoint{2.306763in}{1.966044in}}%
\pgfpathcurveto{\pgfqpoint{2.312587in}{1.966044in}}{\pgfqpoint{2.318173in}{1.968357in}}{\pgfqpoint{2.322291in}{1.972476in}}%
\pgfpathcurveto{\pgfqpoint{2.326409in}{1.976594in}}{\pgfqpoint{2.328723in}{1.982180in}}{\pgfqpoint{2.328723in}{1.988004in}}%
\pgfpathcurveto{\pgfqpoint{2.328723in}{1.993828in}}{\pgfqpoint{2.326409in}{1.999414in}}{\pgfqpoint{2.322291in}{2.003532in}}%
\pgfpathcurveto{\pgfqpoint{2.318173in}{2.007650in}}{\pgfqpoint{2.312587in}{2.009964in}}{\pgfqpoint{2.306763in}{2.009964in}}%
\pgfpathcurveto{\pgfqpoint{2.300939in}{2.009964in}}{\pgfqpoint{2.295353in}{2.007650in}}{\pgfqpoint{2.291235in}{2.003532in}}%
\pgfpathcurveto{\pgfqpoint{2.287117in}{1.999414in}}{\pgfqpoint{2.284803in}{1.993828in}}{\pgfqpoint{2.284803in}{1.988004in}}%
\pgfpathcurveto{\pgfqpoint{2.284803in}{1.982180in}}{\pgfqpoint{2.287117in}{1.976594in}}{\pgfqpoint{2.291235in}{1.972476in}}%
\pgfpathcurveto{\pgfqpoint{2.295353in}{1.968357in}}{\pgfqpoint{2.300939in}{1.966044in}}{\pgfqpoint{2.306763in}{1.966044in}}%
\pgfpathlineto{\pgfqpoint{2.306763in}{1.966044in}}%
\pgfpathclose%
\pgfusepath{stroke,fill}%
\end{pgfscope}%
\begin{pgfscope}%
\pgfpathrectangle{\pgfqpoint{0.997489in}{0.528000in}}{\pgfqpoint{4.565023in}{3.696000in}}%
\pgfusepath{clip}%
\pgfsetbuttcap%
\pgfsetroundjoin%
\definecolor{currentfill}{rgb}{0.800000,0.200000,0.200000}%
\pgfsetfillcolor{currentfill}%
\pgfsetlinewidth{1.003750pt}%
\definecolor{currentstroke}{rgb}{0.800000,0.200000,0.200000}%
\pgfsetstrokecolor{currentstroke}%
\pgfsetdash{}{0pt}%
\pgfpathmoveto{\pgfqpoint{2.215517in}{1.859047in}}%
\pgfpathcurveto{\pgfqpoint{2.221341in}{1.859047in}}{\pgfqpoint{2.226927in}{1.861361in}}{\pgfqpoint{2.231045in}{1.865479in}}%
\pgfpathcurveto{\pgfqpoint{2.235163in}{1.869598in}}{\pgfqpoint{2.237477in}{1.875184in}}{\pgfqpoint{2.237477in}{1.881008in}}%
\pgfpathcurveto{\pgfqpoint{2.237477in}{1.886832in}}{\pgfqpoint{2.235163in}{1.892418in}}{\pgfqpoint{2.231045in}{1.896536in}}%
\pgfpathcurveto{\pgfqpoint{2.226927in}{1.900654in}}{\pgfqpoint{2.221341in}{1.902968in}}{\pgfqpoint{2.215517in}{1.902968in}}%
\pgfpathcurveto{\pgfqpoint{2.209693in}{1.902968in}}{\pgfqpoint{2.204107in}{1.900654in}}{\pgfqpoint{2.199989in}{1.896536in}}%
\pgfpathcurveto{\pgfqpoint{2.195871in}{1.892418in}}{\pgfqpoint{2.193557in}{1.886832in}}{\pgfqpoint{2.193557in}{1.881008in}}%
\pgfpathcurveto{\pgfqpoint{2.193557in}{1.875184in}}{\pgfqpoint{2.195871in}{1.869598in}}{\pgfqpoint{2.199989in}{1.865479in}}%
\pgfpathcurveto{\pgfqpoint{2.204107in}{1.861361in}}{\pgfqpoint{2.209693in}{1.859047in}}{\pgfqpoint{2.215517in}{1.859047in}}%
\pgfpathlineto{\pgfqpoint{2.215517in}{1.859047in}}%
\pgfpathclose%
\pgfusepath{stroke,fill}%
\end{pgfscope}%
\begin{pgfscope}%
\pgfpathrectangle{\pgfqpoint{0.997489in}{0.528000in}}{\pgfqpoint{4.565023in}{3.696000in}}%
\pgfusepath{clip}%
\pgfsetbuttcap%
\pgfsetroundjoin%
\definecolor{currentfill}{rgb}{0.800000,0.200000,0.200000}%
\pgfsetfillcolor{currentfill}%
\pgfsetlinewidth{1.003750pt}%
\definecolor{currentstroke}{rgb}{0.800000,0.200000,0.200000}%
\pgfsetstrokecolor{currentstroke}%
\pgfsetdash{}{0pt}%
\pgfpathmoveto{\pgfqpoint{2.263540in}{1.805798in}}%
\pgfpathcurveto{\pgfqpoint{2.269364in}{1.805798in}}{\pgfqpoint{2.274950in}{1.808112in}}{\pgfqpoint{2.279068in}{1.812230in}}%
\pgfpathcurveto{\pgfqpoint{2.283186in}{1.816348in}}{\pgfqpoint{2.285500in}{1.821934in}}{\pgfqpoint{2.285500in}{1.827758in}}%
\pgfpathcurveto{\pgfqpoint{2.285500in}{1.833582in}}{\pgfqpoint{2.283186in}{1.839168in}}{\pgfqpoint{2.279068in}{1.843286in}}%
\pgfpathcurveto{\pgfqpoint{2.274950in}{1.847404in}}{\pgfqpoint{2.269364in}{1.849718in}}{\pgfqpoint{2.263540in}{1.849718in}}%
\pgfpathcurveto{\pgfqpoint{2.257716in}{1.849718in}}{\pgfqpoint{2.252130in}{1.847404in}}{\pgfqpoint{2.248012in}{1.843286in}}%
\pgfpathcurveto{\pgfqpoint{2.243893in}{1.839168in}}{\pgfqpoint{2.241580in}{1.833582in}}{\pgfqpoint{2.241580in}{1.827758in}}%
\pgfpathcurveto{\pgfqpoint{2.241580in}{1.821934in}}{\pgfqpoint{2.243893in}{1.816348in}}{\pgfqpoint{2.248012in}{1.812230in}}%
\pgfpathcurveto{\pgfqpoint{2.252130in}{1.808112in}}{\pgfqpoint{2.257716in}{1.805798in}}{\pgfqpoint{2.263540in}{1.805798in}}%
\pgfpathlineto{\pgfqpoint{2.263540in}{1.805798in}}%
\pgfpathclose%
\pgfusepath{stroke,fill}%
\end{pgfscope}%
\begin{pgfscope}%
\pgfpathrectangle{\pgfqpoint{0.997489in}{0.528000in}}{\pgfqpoint{4.565023in}{3.696000in}}%
\pgfusepath{clip}%
\pgfsetbuttcap%
\pgfsetroundjoin%
\definecolor{currentfill}{rgb}{0.800000,0.200000,0.200000}%
\pgfsetfillcolor{currentfill}%
\pgfsetlinewidth{1.003750pt}%
\definecolor{currentstroke}{rgb}{0.800000,0.200000,0.200000}%
\pgfsetstrokecolor{currentstroke}%
\pgfsetdash{}{0pt}%
\pgfpathmoveto{\pgfqpoint{2.290624in}{1.742716in}}%
\pgfpathcurveto{\pgfqpoint{2.296448in}{1.742716in}}{\pgfqpoint{2.302034in}{1.745029in}}{\pgfqpoint{2.306153in}{1.749148in}}%
\pgfpathcurveto{\pgfqpoint{2.310271in}{1.753266in}}{\pgfqpoint{2.312585in}{1.758852in}}{\pgfqpoint{2.312585in}{1.764676in}}%
\pgfpathcurveto{\pgfqpoint{2.312585in}{1.770500in}}{\pgfqpoint{2.310271in}{1.776086in}}{\pgfqpoint{2.306153in}{1.780204in}}%
\pgfpathcurveto{\pgfqpoint{2.302034in}{1.784322in}}{\pgfqpoint{2.296448in}{1.786636in}}{\pgfqpoint{2.290624in}{1.786636in}}%
\pgfpathcurveto{\pgfqpoint{2.284800in}{1.786636in}}{\pgfqpoint{2.279214in}{1.784322in}}{\pgfqpoint{2.275096in}{1.780204in}}%
\pgfpathcurveto{\pgfqpoint{2.270978in}{1.776086in}}{\pgfqpoint{2.268664in}{1.770500in}}{\pgfqpoint{2.268664in}{1.764676in}}%
\pgfpathcurveto{\pgfqpoint{2.268664in}{1.758852in}}{\pgfqpoint{2.270978in}{1.753266in}}{\pgfqpoint{2.275096in}{1.749148in}}%
\pgfpathcurveto{\pgfqpoint{2.279214in}{1.745029in}}{\pgfqpoint{2.284800in}{1.742716in}}{\pgfqpoint{2.290624in}{1.742716in}}%
\pgfpathlineto{\pgfqpoint{2.290624in}{1.742716in}}%
\pgfpathclose%
\pgfusepath{stroke,fill}%
\end{pgfscope}%
\begin{pgfscope}%
\pgfpathrectangle{\pgfqpoint{0.997489in}{0.528000in}}{\pgfqpoint{4.565023in}{3.696000in}}%
\pgfusepath{clip}%
\pgfsetbuttcap%
\pgfsetroundjoin%
\definecolor{currentfill}{rgb}{0.800000,0.200000,0.200000}%
\pgfsetfillcolor{currentfill}%
\pgfsetlinewidth{1.003750pt}%
\definecolor{currentstroke}{rgb}{0.800000,0.200000,0.200000}%
\pgfsetstrokecolor{currentstroke}%
\pgfsetdash{}{0pt}%
\pgfpathmoveto{\pgfqpoint{2.402393in}{1.736523in}}%
\pgfpathcurveto{\pgfqpoint{2.408217in}{1.736523in}}{\pgfqpoint{2.413804in}{1.738837in}}{\pgfqpoint{2.417922in}{1.742955in}}%
\pgfpathcurveto{\pgfqpoint{2.422040in}{1.747073in}}{\pgfqpoint{2.424354in}{1.752659in}}{\pgfqpoint{2.424354in}{1.758483in}}%
\pgfpathcurveto{\pgfqpoint{2.424354in}{1.764307in}}{\pgfqpoint{2.422040in}{1.769893in}}{\pgfqpoint{2.417922in}{1.774011in}}%
\pgfpathcurveto{\pgfqpoint{2.413804in}{1.778130in}}{\pgfqpoint{2.408217in}{1.780443in}}{\pgfqpoint{2.402393in}{1.780443in}}%
\pgfpathcurveto{\pgfqpoint{2.396570in}{1.780443in}}{\pgfqpoint{2.390983in}{1.778130in}}{\pgfqpoint{2.386865in}{1.774011in}}%
\pgfpathcurveto{\pgfqpoint{2.382747in}{1.769893in}}{\pgfqpoint{2.380433in}{1.764307in}}{\pgfqpoint{2.380433in}{1.758483in}}%
\pgfpathcurveto{\pgfqpoint{2.380433in}{1.752659in}}{\pgfqpoint{2.382747in}{1.747073in}}{\pgfqpoint{2.386865in}{1.742955in}}%
\pgfpathcurveto{\pgfqpoint{2.390983in}{1.738837in}}{\pgfqpoint{2.396570in}{1.736523in}}{\pgfqpoint{2.402393in}{1.736523in}}%
\pgfpathlineto{\pgfqpoint{2.402393in}{1.736523in}}%
\pgfpathclose%
\pgfusepath{stroke,fill}%
\end{pgfscope}%
\begin{pgfscope}%
\pgfpathrectangle{\pgfqpoint{0.997489in}{0.528000in}}{\pgfqpoint{4.565023in}{3.696000in}}%
\pgfusepath{clip}%
\pgfsetbuttcap%
\pgfsetroundjoin%
\definecolor{currentfill}{rgb}{0.800000,0.200000,0.200000}%
\pgfsetfillcolor{currentfill}%
\pgfsetlinewidth{1.003750pt}%
\definecolor{currentstroke}{rgb}{0.800000,0.200000,0.200000}%
\pgfsetstrokecolor{currentstroke}%
\pgfsetdash{}{0pt}%
\pgfpathmoveto{\pgfqpoint{2.385574in}{1.643497in}}%
\pgfpathcurveto{\pgfqpoint{2.391398in}{1.643497in}}{\pgfqpoint{2.396984in}{1.645811in}}{\pgfqpoint{2.401102in}{1.649929in}}%
\pgfpathcurveto{\pgfqpoint{2.405221in}{1.654047in}}{\pgfqpoint{2.407534in}{1.659633in}}{\pgfqpoint{2.407534in}{1.665457in}}%
\pgfpathcurveto{\pgfqpoint{2.407534in}{1.671281in}}{\pgfqpoint{2.405221in}{1.676867in}}{\pgfqpoint{2.401102in}{1.680985in}}%
\pgfpathcurveto{\pgfqpoint{2.396984in}{1.685103in}}{\pgfqpoint{2.391398in}{1.687417in}}{\pgfqpoint{2.385574in}{1.687417in}}%
\pgfpathcurveto{\pgfqpoint{2.379750in}{1.687417in}}{\pgfqpoint{2.374164in}{1.685103in}}{\pgfqpoint{2.370046in}{1.680985in}}%
\pgfpathcurveto{\pgfqpoint{2.365928in}{1.676867in}}{\pgfqpoint{2.363614in}{1.671281in}}{\pgfqpoint{2.363614in}{1.665457in}}%
\pgfpathcurveto{\pgfqpoint{2.363614in}{1.659633in}}{\pgfqpoint{2.365928in}{1.654047in}}{\pgfqpoint{2.370046in}{1.649929in}}%
\pgfpathcurveto{\pgfqpoint{2.374164in}{1.645811in}}{\pgfqpoint{2.379750in}{1.643497in}}{\pgfqpoint{2.385574in}{1.643497in}}%
\pgfpathlineto{\pgfqpoint{2.385574in}{1.643497in}}%
\pgfpathclose%
\pgfusepath{stroke,fill}%
\end{pgfscope}%
\begin{pgfscope}%
\pgfpathrectangle{\pgfqpoint{0.997489in}{0.528000in}}{\pgfqpoint{4.565023in}{3.696000in}}%
\pgfusepath{clip}%
\pgfsetbuttcap%
\pgfsetroundjoin%
\definecolor{currentfill}{rgb}{0.800000,0.200000,0.200000}%
\pgfsetfillcolor{currentfill}%
\pgfsetlinewidth{1.003750pt}%
\definecolor{currentstroke}{rgb}{0.800000,0.200000,0.200000}%
\pgfsetstrokecolor{currentstroke}%
\pgfsetdash{}{0pt}%
\pgfpathmoveto{\pgfqpoint{2.554716in}{1.704807in}}%
\pgfpathcurveto{\pgfqpoint{2.560540in}{1.704807in}}{\pgfqpoint{2.566126in}{1.707121in}}{\pgfqpoint{2.570244in}{1.711239in}}%
\pgfpathcurveto{\pgfqpoint{2.574363in}{1.715358in}}{\pgfqpoint{2.576676in}{1.720944in}}{\pgfqpoint{2.576676in}{1.726768in}}%
\pgfpathcurveto{\pgfqpoint{2.576676in}{1.732592in}}{\pgfqpoint{2.574363in}{1.738178in}}{\pgfqpoint{2.570244in}{1.742296in}}%
\pgfpathcurveto{\pgfqpoint{2.566126in}{1.746414in}}{\pgfqpoint{2.560540in}{1.748728in}}{\pgfqpoint{2.554716in}{1.748728in}}%
\pgfpathcurveto{\pgfqpoint{2.548892in}{1.748728in}}{\pgfqpoint{2.543306in}{1.746414in}}{\pgfqpoint{2.539188in}{1.742296in}}%
\pgfpathcurveto{\pgfqpoint{2.535070in}{1.738178in}}{\pgfqpoint{2.532756in}{1.732592in}}{\pgfqpoint{2.532756in}{1.726768in}}%
\pgfpathcurveto{\pgfqpoint{2.532756in}{1.720944in}}{\pgfqpoint{2.535070in}{1.715358in}}{\pgfqpoint{2.539188in}{1.711239in}}%
\pgfpathcurveto{\pgfqpoint{2.543306in}{1.707121in}}{\pgfqpoint{2.548892in}{1.704807in}}{\pgfqpoint{2.554716in}{1.704807in}}%
\pgfpathlineto{\pgfqpoint{2.554716in}{1.704807in}}%
\pgfpathclose%
\pgfusepath{stroke,fill}%
\end{pgfscope}%
\begin{pgfscope}%
\pgfpathrectangle{\pgfqpoint{0.997489in}{0.528000in}}{\pgfqpoint{4.565023in}{3.696000in}}%
\pgfusepath{clip}%
\pgfsetbuttcap%
\pgfsetroundjoin%
\definecolor{currentfill}{rgb}{0.800000,0.200000,0.200000}%
\pgfsetfillcolor{currentfill}%
\pgfsetlinewidth{1.003750pt}%
\definecolor{currentstroke}{rgb}{0.800000,0.200000,0.200000}%
\pgfsetstrokecolor{currentstroke}%
\pgfsetdash{}{0pt}%
\pgfpathmoveto{\pgfqpoint{2.515532in}{1.584647in}}%
\pgfpathcurveto{\pgfqpoint{2.521356in}{1.584647in}}{\pgfqpoint{2.526942in}{1.586961in}}{\pgfqpoint{2.531060in}{1.591079in}}%
\pgfpathcurveto{\pgfqpoint{2.535178in}{1.595197in}}{\pgfqpoint{2.537492in}{1.600783in}}{\pgfqpoint{2.537492in}{1.606607in}}%
\pgfpathcurveto{\pgfqpoint{2.537492in}{1.612431in}}{\pgfqpoint{2.535178in}{1.618017in}}{\pgfqpoint{2.531060in}{1.622135in}}%
\pgfpathcurveto{\pgfqpoint{2.526942in}{1.626253in}}{\pgfqpoint{2.521356in}{1.628567in}}{\pgfqpoint{2.515532in}{1.628567in}}%
\pgfpathcurveto{\pgfqpoint{2.509708in}{1.628567in}}{\pgfqpoint{2.504122in}{1.626253in}}{\pgfqpoint{2.500003in}{1.622135in}}%
\pgfpathcurveto{\pgfqpoint{2.495885in}{1.618017in}}{\pgfqpoint{2.493571in}{1.612431in}}{\pgfqpoint{2.493571in}{1.606607in}}%
\pgfpathcurveto{\pgfqpoint{2.493571in}{1.600783in}}{\pgfqpoint{2.495885in}{1.595197in}}{\pgfqpoint{2.500003in}{1.591079in}}%
\pgfpathcurveto{\pgfqpoint{2.504122in}{1.586961in}}{\pgfqpoint{2.509708in}{1.584647in}}{\pgfqpoint{2.515532in}{1.584647in}}%
\pgfpathlineto{\pgfqpoint{2.515532in}{1.584647in}}%
\pgfpathclose%
\pgfusepath{stroke,fill}%
\end{pgfscope}%
\begin{pgfscope}%
\pgfpathrectangle{\pgfqpoint{0.997489in}{0.528000in}}{\pgfqpoint{4.565023in}{3.696000in}}%
\pgfusepath{clip}%
\pgfsetbuttcap%
\pgfsetroundjoin%
\definecolor{currentfill}{rgb}{0.800000,0.200000,0.200000}%
\pgfsetfillcolor{currentfill}%
\pgfsetlinewidth{1.003750pt}%
\definecolor{currentstroke}{rgb}{0.800000,0.200000,0.200000}%
\pgfsetstrokecolor{currentstroke}%
\pgfsetdash{}{0pt}%
\pgfpathmoveto{\pgfqpoint{2.566750in}{1.546904in}}%
\pgfpathcurveto{\pgfqpoint{2.572574in}{1.546904in}}{\pgfqpoint{2.578161in}{1.549218in}}{\pgfqpoint{2.582279in}{1.553336in}}%
\pgfpathcurveto{\pgfqpoint{2.586397in}{1.557455in}}{\pgfqpoint{2.588711in}{1.563041in}}{\pgfqpoint{2.588711in}{1.568865in}}%
\pgfpathcurveto{\pgfqpoint{2.588711in}{1.574689in}}{\pgfqpoint{2.586397in}{1.580275in}}{\pgfqpoint{2.582279in}{1.584393in}}%
\pgfpathcurveto{\pgfqpoint{2.578161in}{1.588511in}}{\pgfqpoint{2.572574in}{1.590825in}}{\pgfqpoint{2.566750in}{1.590825in}}%
\pgfpathcurveto{\pgfqpoint{2.560926in}{1.590825in}}{\pgfqpoint{2.555340in}{1.588511in}}{\pgfqpoint{2.551222in}{1.584393in}}%
\pgfpathcurveto{\pgfqpoint{2.547104in}{1.580275in}}{\pgfqpoint{2.544790in}{1.574689in}}{\pgfqpoint{2.544790in}{1.568865in}}%
\pgfpathcurveto{\pgfqpoint{2.544790in}{1.563041in}}{\pgfqpoint{2.547104in}{1.557455in}}{\pgfqpoint{2.551222in}{1.553336in}}%
\pgfpathcurveto{\pgfqpoint{2.555340in}{1.549218in}}{\pgfqpoint{2.560926in}{1.546904in}}{\pgfqpoint{2.566750in}{1.546904in}}%
\pgfpathlineto{\pgfqpoint{2.566750in}{1.546904in}}%
\pgfpathclose%
\pgfusepath{stroke,fill}%
\end{pgfscope}%
\begin{pgfscope}%
\pgfpathrectangle{\pgfqpoint{0.997489in}{0.528000in}}{\pgfqpoint{4.565023in}{3.696000in}}%
\pgfusepath{clip}%
\pgfsetbuttcap%
\pgfsetroundjoin%
\definecolor{currentfill}{rgb}{0.800000,0.200000,0.200000}%
\pgfsetfillcolor{currentfill}%
\pgfsetlinewidth{1.003750pt}%
\definecolor{currentstroke}{rgb}{0.800000,0.200000,0.200000}%
\pgfsetstrokecolor{currentstroke}%
\pgfsetdash{}{0pt}%
\pgfpathmoveto{\pgfqpoint{2.606252in}{1.494817in}}%
\pgfpathcurveto{\pgfqpoint{2.612076in}{1.494817in}}{\pgfqpoint{2.617662in}{1.497130in}}{\pgfqpoint{2.621780in}{1.501249in}}%
\pgfpathcurveto{\pgfqpoint{2.625898in}{1.505367in}}{\pgfqpoint{2.628212in}{1.510953in}}{\pgfqpoint{2.628212in}{1.516777in}}%
\pgfpathcurveto{\pgfqpoint{2.628212in}{1.522601in}}{\pgfqpoint{2.625898in}{1.528187in}}{\pgfqpoint{2.621780in}{1.532305in}}%
\pgfpathcurveto{\pgfqpoint{2.617662in}{1.536423in}}{\pgfqpoint{2.612076in}{1.538737in}}{\pgfqpoint{2.606252in}{1.538737in}}%
\pgfpathcurveto{\pgfqpoint{2.600428in}{1.538737in}}{\pgfqpoint{2.594842in}{1.536423in}}{\pgfqpoint{2.590724in}{1.532305in}}%
\pgfpathcurveto{\pgfqpoint{2.586605in}{1.528187in}}{\pgfqpoint{2.584292in}{1.522601in}}{\pgfqpoint{2.584292in}{1.516777in}}%
\pgfpathcurveto{\pgfqpoint{2.584292in}{1.510953in}}{\pgfqpoint{2.586605in}{1.505367in}}{\pgfqpoint{2.590724in}{1.501249in}}%
\pgfpathcurveto{\pgfqpoint{2.594842in}{1.497130in}}{\pgfqpoint{2.600428in}{1.494817in}}{\pgfqpoint{2.606252in}{1.494817in}}%
\pgfpathlineto{\pgfqpoint{2.606252in}{1.494817in}}%
\pgfpathclose%
\pgfusepath{stroke,fill}%
\end{pgfscope}%
\begin{pgfscope}%
\pgfpathrectangle{\pgfqpoint{0.997489in}{0.528000in}}{\pgfqpoint{4.565023in}{3.696000in}}%
\pgfusepath{clip}%
\pgfsetbuttcap%
\pgfsetroundjoin%
\definecolor{currentfill}{rgb}{0.800000,0.200000,0.200000}%
\pgfsetfillcolor{currentfill}%
\pgfsetlinewidth{1.003750pt}%
\definecolor{currentstroke}{rgb}{0.800000,0.200000,0.200000}%
\pgfsetstrokecolor{currentstroke}%
\pgfsetdash{}{0pt}%
\pgfpathmoveto{\pgfqpoint{2.647872in}{1.441900in}}%
\pgfpathcurveto{\pgfqpoint{2.653696in}{1.441900in}}{\pgfqpoint{2.659282in}{1.444213in}}{\pgfqpoint{2.663400in}{1.448332in}}%
\pgfpathcurveto{\pgfqpoint{2.667518in}{1.452450in}}{\pgfqpoint{2.669832in}{1.458036in}}{\pgfqpoint{2.669832in}{1.463860in}}%
\pgfpathcurveto{\pgfqpoint{2.669832in}{1.469684in}}{\pgfqpoint{2.667518in}{1.475270in}}{\pgfqpoint{2.663400in}{1.479388in}}%
\pgfpathcurveto{\pgfqpoint{2.659282in}{1.483506in}}{\pgfqpoint{2.653696in}{1.485820in}}{\pgfqpoint{2.647872in}{1.485820in}}%
\pgfpathcurveto{\pgfqpoint{2.642048in}{1.485820in}}{\pgfqpoint{2.636462in}{1.483506in}}{\pgfqpoint{2.632344in}{1.479388in}}%
\pgfpathcurveto{\pgfqpoint{2.628226in}{1.475270in}}{\pgfqpoint{2.625912in}{1.469684in}}{\pgfqpoint{2.625912in}{1.463860in}}%
\pgfpathcurveto{\pgfqpoint{2.625912in}{1.458036in}}{\pgfqpoint{2.628226in}{1.452450in}}{\pgfqpoint{2.632344in}{1.448332in}}%
\pgfpathcurveto{\pgfqpoint{2.636462in}{1.444213in}}{\pgfqpoint{2.642048in}{1.441900in}}{\pgfqpoint{2.647872in}{1.441900in}}%
\pgfpathlineto{\pgfqpoint{2.647872in}{1.441900in}}%
\pgfpathclose%
\pgfusepath{stroke,fill}%
\end{pgfscope}%
\begin{pgfscope}%
\pgfpathrectangle{\pgfqpoint{0.997489in}{0.528000in}}{\pgfqpoint{4.565023in}{3.696000in}}%
\pgfusepath{clip}%
\pgfsetbuttcap%
\pgfsetroundjoin%
\definecolor{currentfill}{rgb}{0.800000,0.200000,0.200000}%
\pgfsetfillcolor{currentfill}%
\pgfsetlinewidth{1.003750pt}%
\definecolor{currentstroke}{rgb}{0.800000,0.200000,0.200000}%
\pgfsetstrokecolor{currentstroke}%
\pgfsetdash{}{0pt}%
\pgfpathmoveto{\pgfqpoint{2.710687in}{1.420366in}}%
\pgfpathcurveto{\pgfqpoint{2.716511in}{1.420366in}}{\pgfqpoint{2.722098in}{1.422680in}}{\pgfqpoint{2.726216in}{1.426798in}}%
\pgfpathcurveto{\pgfqpoint{2.730334in}{1.430916in}}{\pgfqpoint{2.732648in}{1.436502in}}{\pgfqpoint{2.732648in}{1.442326in}}%
\pgfpathcurveto{\pgfqpoint{2.732648in}{1.448150in}}{\pgfqpoint{2.730334in}{1.453736in}}{\pgfqpoint{2.726216in}{1.457854in}}%
\pgfpathcurveto{\pgfqpoint{2.722098in}{1.461972in}}{\pgfqpoint{2.716511in}{1.464286in}}{\pgfqpoint{2.710687in}{1.464286in}}%
\pgfpathcurveto{\pgfqpoint{2.704864in}{1.464286in}}{\pgfqpoint{2.699277in}{1.461972in}}{\pgfqpoint{2.695159in}{1.457854in}}%
\pgfpathcurveto{\pgfqpoint{2.691041in}{1.453736in}}{\pgfqpoint{2.688727in}{1.448150in}}{\pgfqpoint{2.688727in}{1.442326in}}%
\pgfpathcurveto{\pgfqpoint{2.688727in}{1.436502in}}{\pgfqpoint{2.691041in}{1.430916in}}{\pgfqpoint{2.695159in}{1.426798in}}%
\pgfpathcurveto{\pgfqpoint{2.699277in}{1.422680in}}{\pgfqpoint{2.704864in}{1.420366in}}{\pgfqpoint{2.710687in}{1.420366in}}%
\pgfpathlineto{\pgfqpoint{2.710687in}{1.420366in}}%
\pgfpathclose%
\pgfusepath{stroke,fill}%
\end{pgfscope}%
\begin{pgfscope}%
\pgfpathrectangle{\pgfqpoint{0.997489in}{0.528000in}}{\pgfqpoint{4.565023in}{3.696000in}}%
\pgfusepath{clip}%
\pgfsetbuttcap%
\pgfsetroundjoin%
\definecolor{currentfill}{rgb}{0.800000,0.200000,0.200000}%
\pgfsetfillcolor{currentfill}%
\pgfsetlinewidth{1.003750pt}%
\definecolor{currentstroke}{rgb}{0.800000,0.200000,0.200000}%
\pgfsetstrokecolor{currentstroke}%
\pgfsetdash{}{0pt}%
\pgfpathmoveto{\pgfqpoint{2.789302in}{1.434355in}}%
\pgfpathcurveto{\pgfqpoint{2.795126in}{1.434355in}}{\pgfqpoint{2.800712in}{1.436669in}}{\pgfqpoint{2.804831in}{1.440787in}}%
\pgfpathcurveto{\pgfqpoint{2.808949in}{1.444905in}}{\pgfqpoint{2.811263in}{1.450491in}}{\pgfqpoint{2.811263in}{1.456315in}}%
\pgfpathcurveto{\pgfqpoint{2.811263in}{1.462139in}}{\pgfqpoint{2.808949in}{1.467725in}}{\pgfqpoint{2.804831in}{1.471843in}}%
\pgfpathcurveto{\pgfqpoint{2.800712in}{1.475962in}}{\pgfqpoint{2.795126in}{1.478276in}}{\pgfqpoint{2.789302in}{1.478276in}}%
\pgfpathcurveto{\pgfqpoint{2.783478in}{1.478276in}}{\pgfqpoint{2.777892in}{1.475962in}}{\pgfqpoint{2.773774in}{1.471843in}}%
\pgfpathcurveto{\pgfqpoint{2.769656in}{1.467725in}}{\pgfqpoint{2.767342in}{1.462139in}}{\pgfqpoint{2.767342in}{1.456315in}}%
\pgfpathcurveto{\pgfqpoint{2.767342in}{1.450491in}}{\pgfqpoint{2.769656in}{1.444905in}}{\pgfqpoint{2.773774in}{1.440787in}}%
\pgfpathcurveto{\pgfqpoint{2.777892in}{1.436669in}}{\pgfqpoint{2.783478in}{1.434355in}}{\pgfqpoint{2.789302in}{1.434355in}}%
\pgfpathlineto{\pgfqpoint{2.789302in}{1.434355in}}%
\pgfpathclose%
\pgfusepath{stroke,fill}%
\end{pgfscope}%
\begin{pgfscope}%
\pgfpathrectangle{\pgfqpoint{0.997489in}{0.528000in}}{\pgfqpoint{4.565023in}{3.696000in}}%
\pgfusepath{clip}%
\pgfsetbuttcap%
\pgfsetroundjoin%
\definecolor{currentfill}{rgb}{0.800000,0.200000,0.200000}%
\pgfsetfillcolor{currentfill}%
\pgfsetlinewidth{1.003750pt}%
\definecolor{currentstroke}{rgb}{0.800000,0.200000,0.200000}%
\pgfsetstrokecolor{currentstroke}%
\pgfsetdash{}{0pt}%
\pgfpathmoveto{\pgfqpoint{2.837238in}{1.390679in}}%
\pgfpathcurveto{\pgfqpoint{2.843062in}{1.390679in}}{\pgfqpoint{2.848648in}{1.392993in}}{\pgfqpoint{2.852766in}{1.397111in}}%
\pgfpathcurveto{\pgfqpoint{2.856885in}{1.401229in}}{\pgfqpoint{2.859198in}{1.406815in}}{\pgfqpoint{2.859198in}{1.412639in}}%
\pgfpathcurveto{\pgfqpoint{2.859198in}{1.418463in}}{\pgfqpoint{2.856885in}{1.424049in}}{\pgfqpoint{2.852766in}{1.428167in}}%
\pgfpathcurveto{\pgfqpoint{2.848648in}{1.432286in}}{\pgfqpoint{2.843062in}{1.434599in}}{\pgfqpoint{2.837238in}{1.434599in}}%
\pgfpathcurveto{\pgfqpoint{2.831414in}{1.434599in}}{\pgfqpoint{2.825828in}{1.432286in}}{\pgfqpoint{2.821710in}{1.428167in}}%
\pgfpathcurveto{\pgfqpoint{2.817592in}{1.424049in}}{\pgfqpoint{2.815278in}{1.418463in}}{\pgfqpoint{2.815278in}{1.412639in}}%
\pgfpathcurveto{\pgfqpoint{2.815278in}{1.406815in}}{\pgfqpoint{2.817592in}{1.401229in}}{\pgfqpoint{2.821710in}{1.397111in}}%
\pgfpathcurveto{\pgfqpoint{2.825828in}{1.392993in}}{\pgfqpoint{2.831414in}{1.390679in}}{\pgfqpoint{2.837238in}{1.390679in}}%
\pgfpathlineto{\pgfqpoint{2.837238in}{1.390679in}}%
\pgfpathclose%
\pgfusepath{stroke,fill}%
\end{pgfscope}%
\begin{pgfscope}%
\pgfpathrectangle{\pgfqpoint{0.997489in}{0.528000in}}{\pgfqpoint{4.565023in}{3.696000in}}%
\pgfusepath{clip}%
\pgfsetbuttcap%
\pgfsetroundjoin%
\definecolor{currentfill}{rgb}{0.800000,0.200000,0.200000}%
\pgfsetfillcolor{currentfill}%
\pgfsetlinewidth{1.003750pt}%
\definecolor{currentstroke}{rgb}{0.800000,0.200000,0.200000}%
\pgfsetstrokecolor{currentstroke}%
\pgfsetdash{}{0pt}%
\pgfpathmoveto{\pgfqpoint{2.874140in}{1.306681in}}%
\pgfpathcurveto{\pgfqpoint{2.879964in}{1.306681in}}{\pgfqpoint{2.885550in}{1.308995in}}{\pgfqpoint{2.889668in}{1.313113in}}%
\pgfpathcurveto{\pgfqpoint{2.893786in}{1.317231in}}{\pgfqpoint{2.896100in}{1.322817in}}{\pgfqpoint{2.896100in}{1.328641in}}%
\pgfpathcurveto{\pgfqpoint{2.896100in}{1.334465in}}{\pgfqpoint{2.893786in}{1.340051in}}{\pgfqpoint{2.889668in}{1.344169in}}%
\pgfpathcurveto{\pgfqpoint{2.885550in}{1.348288in}}{\pgfqpoint{2.879964in}{1.350601in}}{\pgfqpoint{2.874140in}{1.350601in}}%
\pgfpathcurveto{\pgfqpoint{2.868316in}{1.350601in}}{\pgfqpoint{2.862730in}{1.348288in}}{\pgfqpoint{2.858612in}{1.344169in}}%
\pgfpathcurveto{\pgfqpoint{2.854494in}{1.340051in}}{\pgfqpoint{2.852180in}{1.334465in}}{\pgfqpoint{2.852180in}{1.328641in}}%
\pgfpathcurveto{\pgfqpoint{2.852180in}{1.322817in}}{\pgfqpoint{2.854494in}{1.317231in}}{\pgfqpoint{2.858612in}{1.313113in}}%
\pgfpathcurveto{\pgfqpoint{2.862730in}{1.308995in}}{\pgfqpoint{2.868316in}{1.306681in}}{\pgfqpoint{2.874140in}{1.306681in}}%
\pgfpathlineto{\pgfqpoint{2.874140in}{1.306681in}}%
\pgfpathclose%
\pgfusepath{stroke,fill}%
\end{pgfscope}%
\begin{pgfscope}%
\pgfpathrectangle{\pgfqpoint{0.997489in}{0.528000in}}{\pgfqpoint{4.565023in}{3.696000in}}%
\pgfusepath{clip}%
\pgfsetbuttcap%
\pgfsetroundjoin%
\definecolor{currentfill}{rgb}{0.800000,0.200000,0.200000}%
\pgfsetfillcolor{currentfill}%
\pgfsetlinewidth{1.003750pt}%
\definecolor{currentstroke}{rgb}{0.800000,0.200000,0.200000}%
\pgfsetstrokecolor{currentstroke}%
\pgfsetdash{}{0pt}%
\pgfpathmoveto{\pgfqpoint{2.939919in}{1.295091in}}%
\pgfpathcurveto{\pgfqpoint{2.945743in}{1.295091in}}{\pgfqpoint{2.951329in}{1.297404in}}{\pgfqpoint{2.955447in}{1.301523in}}%
\pgfpathcurveto{\pgfqpoint{2.959565in}{1.305641in}}{\pgfqpoint{2.961879in}{1.311227in}}{\pgfqpoint{2.961879in}{1.317051in}}%
\pgfpathcurveto{\pgfqpoint{2.961879in}{1.322875in}}{\pgfqpoint{2.959565in}{1.328461in}}{\pgfqpoint{2.955447in}{1.332579in}}%
\pgfpathcurveto{\pgfqpoint{2.951329in}{1.336697in}}{\pgfqpoint{2.945743in}{1.339011in}}{\pgfqpoint{2.939919in}{1.339011in}}%
\pgfpathcurveto{\pgfqpoint{2.934095in}{1.339011in}}{\pgfqpoint{2.928509in}{1.336697in}}{\pgfqpoint{2.924390in}{1.332579in}}%
\pgfpathcurveto{\pgfqpoint{2.920272in}{1.328461in}}{\pgfqpoint{2.917958in}{1.322875in}}{\pgfqpoint{2.917958in}{1.317051in}}%
\pgfpathcurveto{\pgfqpoint{2.917958in}{1.311227in}}{\pgfqpoint{2.920272in}{1.305641in}}{\pgfqpoint{2.924390in}{1.301523in}}%
\pgfpathcurveto{\pgfqpoint{2.928509in}{1.297404in}}{\pgfqpoint{2.934095in}{1.295091in}}{\pgfqpoint{2.939919in}{1.295091in}}%
\pgfpathlineto{\pgfqpoint{2.939919in}{1.295091in}}%
\pgfpathclose%
\pgfusepath{stroke,fill}%
\end{pgfscope}%
\begin{pgfscope}%
\pgfpathrectangle{\pgfqpoint{0.997489in}{0.528000in}}{\pgfqpoint{4.565023in}{3.696000in}}%
\pgfusepath{clip}%
\pgfsetbuttcap%
\pgfsetroundjoin%
\definecolor{currentfill}{rgb}{0.800000,0.200000,0.200000}%
\pgfsetfillcolor{currentfill}%
\pgfsetlinewidth{1.003750pt}%
\definecolor{currentstroke}{rgb}{0.800000,0.200000,0.200000}%
\pgfsetstrokecolor{currentstroke}%
\pgfsetdash{}{0pt}%
\pgfpathmoveto{\pgfqpoint{2.990922in}{1.216824in}}%
\pgfpathcurveto{\pgfqpoint{2.996746in}{1.216824in}}{\pgfqpoint{3.002332in}{1.219138in}}{\pgfqpoint{3.006450in}{1.223256in}}%
\pgfpathcurveto{\pgfqpoint{3.010569in}{1.227374in}}{\pgfqpoint{3.012882in}{1.232961in}}{\pgfqpoint{3.012882in}{1.238785in}}%
\pgfpathcurveto{\pgfqpoint{3.012882in}{1.244609in}}{\pgfqpoint{3.010569in}{1.250195in}}{\pgfqpoint{3.006450in}{1.254313in}}%
\pgfpathcurveto{\pgfqpoint{3.002332in}{1.258431in}}{\pgfqpoint{2.996746in}{1.260745in}}{\pgfqpoint{2.990922in}{1.260745in}}%
\pgfpathcurveto{\pgfqpoint{2.985098in}{1.260745in}}{\pgfqpoint{2.979512in}{1.258431in}}{\pgfqpoint{2.975394in}{1.254313in}}%
\pgfpathcurveto{\pgfqpoint{2.971276in}{1.250195in}}{\pgfqpoint{2.968962in}{1.244609in}}{\pgfqpoint{2.968962in}{1.238785in}}%
\pgfpathcurveto{\pgfqpoint{2.968962in}{1.232961in}}{\pgfqpoint{2.971276in}{1.227374in}}{\pgfqpoint{2.975394in}{1.223256in}}%
\pgfpathcurveto{\pgfqpoint{2.979512in}{1.219138in}}{\pgfqpoint{2.985098in}{1.216824in}}{\pgfqpoint{2.990922in}{1.216824in}}%
\pgfpathlineto{\pgfqpoint{2.990922in}{1.216824in}}%
\pgfpathclose%
\pgfusepath{stroke,fill}%
\end{pgfscope}%
\begin{pgfscope}%
\pgfpathrectangle{\pgfqpoint{0.997489in}{0.528000in}}{\pgfqpoint{4.565023in}{3.696000in}}%
\pgfusepath{clip}%
\pgfsetbuttcap%
\pgfsetroundjoin%
\definecolor{currentfill}{rgb}{0.800000,0.200000,0.200000}%
\pgfsetfillcolor{currentfill}%
\pgfsetlinewidth{1.003750pt}%
\definecolor{currentstroke}{rgb}{0.800000,0.200000,0.200000}%
\pgfsetstrokecolor{currentstroke}%
\pgfsetdash{}{0pt}%
\pgfpathmoveto{\pgfqpoint{3.069337in}{1.270236in}}%
\pgfpathcurveto{\pgfqpoint{3.075161in}{1.270236in}}{\pgfqpoint{3.080747in}{1.272550in}}{\pgfqpoint{3.084865in}{1.276668in}}%
\pgfpathcurveto{\pgfqpoint{3.088983in}{1.280787in}}{\pgfqpoint{3.091297in}{1.286373in}}{\pgfqpoint{3.091297in}{1.292197in}}%
\pgfpathcurveto{\pgfqpoint{3.091297in}{1.298021in}}{\pgfqpoint{3.088983in}{1.303607in}}{\pgfqpoint{3.084865in}{1.307725in}}%
\pgfpathcurveto{\pgfqpoint{3.080747in}{1.311843in}}{\pgfqpoint{3.075161in}{1.314157in}}{\pgfqpoint{3.069337in}{1.314157in}}%
\pgfpathcurveto{\pgfqpoint{3.063513in}{1.314157in}}{\pgfqpoint{3.057927in}{1.311843in}}{\pgfqpoint{3.053809in}{1.307725in}}%
\pgfpathcurveto{\pgfqpoint{3.049690in}{1.303607in}}{\pgfqpoint{3.047377in}{1.298021in}}{\pgfqpoint{3.047377in}{1.292197in}}%
\pgfpathcurveto{\pgfqpoint{3.047377in}{1.286373in}}{\pgfqpoint{3.049690in}{1.280787in}}{\pgfqpoint{3.053809in}{1.276668in}}%
\pgfpathcurveto{\pgfqpoint{3.057927in}{1.272550in}}{\pgfqpoint{3.063513in}{1.270236in}}{\pgfqpoint{3.069337in}{1.270236in}}%
\pgfpathlineto{\pgfqpoint{3.069337in}{1.270236in}}%
\pgfpathclose%
\pgfusepath{stroke,fill}%
\end{pgfscope}%
\begin{pgfscope}%
\pgfpathrectangle{\pgfqpoint{0.997489in}{0.528000in}}{\pgfqpoint{4.565023in}{3.696000in}}%
\pgfusepath{clip}%
\pgfsetbuttcap%
\pgfsetroundjoin%
\definecolor{currentfill}{rgb}{0.800000,0.200000,0.200000}%
\pgfsetfillcolor{currentfill}%
\pgfsetlinewidth{1.003750pt}%
\definecolor{currentstroke}{rgb}{0.800000,0.200000,0.200000}%
\pgfsetstrokecolor{currentstroke}%
\pgfsetdash{}{0pt}%
\pgfpathmoveto{\pgfqpoint{3.129957in}{1.202301in}}%
\pgfpathcurveto{\pgfqpoint{3.135781in}{1.202301in}}{\pgfqpoint{3.141367in}{1.204615in}}{\pgfqpoint{3.145485in}{1.208733in}}%
\pgfpathcurveto{\pgfqpoint{3.149603in}{1.212852in}}{\pgfqpoint{3.151917in}{1.218438in}}{\pgfqpoint{3.151917in}{1.224262in}}%
\pgfpathcurveto{\pgfqpoint{3.151917in}{1.230086in}}{\pgfqpoint{3.149603in}{1.235672in}}{\pgfqpoint{3.145485in}{1.239790in}}%
\pgfpathcurveto{\pgfqpoint{3.141367in}{1.243908in}}{\pgfqpoint{3.135781in}{1.246222in}}{\pgfqpoint{3.129957in}{1.246222in}}%
\pgfpathcurveto{\pgfqpoint{3.124133in}{1.246222in}}{\pgfqpoint{3.118547in}{1.243908in}}{\pgfqpoint{3.114429in}{1.239790in}}%
\pgfpathcurveto{\pgfqpoint{3.110311in}{1.235672in}}{\pgfqpoint{3.107997in}{1.230086in}}{\pgfqpoint{3.107997in}{1.224262in}}%
\pgfpathcurveto{\pgfqpoint{3.107997in}{1.218438in}}{\pgfqpoint{3.110311in}{1.212852in}}{\pgfqpoint{3.114429in}{1.208733in}}%
\pgfpathcurveto{\pgfqpoint{3.118547in}{1.204615in}}{\pgfqpoint{3.124133in}{1.202301in}}{\pgfqpoint{3.129957in}{1.202301in}}%
\pgfpathlineto{\pgfqpoint{3.129957in}{1.202301in}}%
\pgfpathclose%
\pgfusepath{stroke,fill}%
\end{pgfscope}%
\begin{pgfscope}%
\pgfpathrectangle{\pgfqpoint{0.997489in}{0.528000in}}{\pgfqpoint{4.565023in}{3.696000in}}%
\pgfusepath{clip}%
\pgfsetbuttcap%
\pgfsetroundjoin%
\definecolor{currentfill}{rgb}{0.800000,0.200000,0.200000}%
\pgfsetfillcolor{currentfill}%
\pgfsetlinewidth{1.003750pt}%
\definecolor{currentstroke}{rgb}{0.800000,0.200000,0.200000}%
\pgfsetstrokecolor{currentstroke}%
\pgfsetdash{}{0pt}%
\pgfpathmoveto{\pgfqpoint{3.201994in}{1.350860in}}%
\pgfpathcurveto{\pgfqpoint{3.207818in}{1.350860in}}{\pgfqpoint{3.213404in}{1.353174in}}{\pgfqpoint{3.217522in}{1.357292in}}%
\pgfpathcurveto{\pgfqpoint{3.221641in}{1.361410in}}{\pgfqpoint{3.223955in}{1.366996in}}{\pgfqpoint{3.223955in}{1.372820in}}%
\pgfpathcurveto{\pgfqpoint{3.223955in}{1.378644in}}{\pgfqpoint{3.221641in}{1.384230in}}{\pgfqpoint{3.217522in}{1.388349in}}%
\pgfpathcurveto{\pgfqpoint{3.213404in}{1.392467in}}{\pgfqpoint{3.207818in}{1.394781in}}{\pgfqpoint{3.201994in}{1.394781in}}%
\pgfpathcurveto{\pgfqpoint{3.196170in}{1.394781in}}{\pgfqpoint{3.190584in}{1.392467in}}{\pgfqpoint{3.186466in}{1.388349in}}%
\pgfpathcurveto{\pgfqpoint{3.182348in}{1.384230in}}{\pgfqpoint{3.180034in}{1.378644in}}{\pgfqpoint{3.180034in}{1.372820in}}%
\pgfpathcurveto{\pgfqpoint{3.180034in}{1.366996in}}{\pgfqpoint{3.182348in}{1.361410in}}{\pgfqpoint{3.186466in}{1.357292in}}%
\pgfpathcurveto{\pgfqpoint{3.190584in}{1.353174in}}{\pgfqpoint{3.196170in}{1.350860in}}{\pgfqpoint{3.201994in}{1.350860in}}%
\pgfpathlineto{\pgfqpoint{3.201994in}{1.350860in}}%
\pgfpathclose%
\pgfusepath{stroke,fill}%
\end{pgfscope}%
\begin{pgfscope}%
\pgfpathrectangle{\pgfqpoint{0.997489in}{0.528000in}}{\pgfqpoint{4.565023in}{3.696000in}}%
\pgfusepath{clip}%
\pgfsetbuttcap%
\pgfsetroundjoin%
\definecolor{currentfill}{rgb}{0.800000,0.200000,0.200000}%
\pgfsetfillcolor{currentfill}%
\pgfsetlinewidth{1.003750pt}%
\definecolor{currentstroke}{rgb}{0.800000,0.200000,0.200000}%
\pgfsetstrokecolor{currentstroke}%
\pgfsetdash{}{0pt}%
\pgfpathmoveto{\pgfqpoint{3.267073in}{1.246466in}}%
\pgfpathcurveto{\pgfqpoint{3.272897in}{1.246466in}}{\pgfqpoint{3.278483in}{1.248780in}}{\pgfqpoint{3.282601in}{1.252898in}}%
\pgfpathcurveto{\pgfqpoint{3.286719in}{1.257016in}}{\pgfqpoint{3.289033in}{1.262602in}}{\pgfqpoint{3.289033in}{1.268426in}}%
\pgfpathcurveto{\pgfqpoint{3.289033in}{1.274250in}}{\pgfqpoint{3.286719in}{1.279836in}}{\pgfqpoint{3.282601in}{1.283954in}}%
\pgfpathcurveto{\pgfqpoint{3.278483in}{1.288072in}}{\pgfqpoint{3.272897in}{1.290386in}}{\pgfqpoint{3.267073in}{1.290386in}}%
\pgfpathcurveto{\pgfqpoint{3.261249in}{1.290386in}}{\pgfqpoint{3.255663in}{1.288072in}}{\pgfqpoint{3.251544in}{1.283954in}}%
\pgfpathcurveto{\pgfqpoint{3.247426in}{1.279836in}}{\pgfqpoint{3.245112in}{1.274250in}}{\pgfqpoint{3.245112in}{1.268426in}}%
\pgfpathcurveto{\pgfqpoint{3.245112in}{1.262602in}}{\pgfqpoint{3.247426in}{1.257016in}}{\pgfqpoint{3.251544in}{1.252898in}}%
\pgfpathcurveto{\pgfqpoint{3.255663in}{1.248780in}}{\pgfqpoint{3.261249in}{1.246466in}}{\pgfqpoint{3.267073in}{1.246466in}}%
\pgfpathlineto{\pgfqpoint{3.267073in}{1.246466in}}%
\pgfpathclose%
\pgfusepath{stroke,fill}%
\end{pgfscope}%
\begin{pgfscope}%
\pgfpathrectangle{\pgfqpoint{0.997489in}{0.528000in}}{\pgfqpoint{4.565023in}{3.696000in}}%
\pgfusepath{clip}%
\pgfsetbuttcap%
\pgfsetroundjoin%
\definecolor{currentfill}{rgb}{0.800000,0.200000,0.200000}%
\pgfsetfillcolor{currentfill}%
\pgfsetlinewidth{1.003750pt}%
\definecolor{currentstroke}{rgb}{0.800000,0.200000,0.200000}%
\pgfsetstrokecolor{currentstroke}%
\pgfsetdash{}{0pt}%
\pgfpathmoveto{\pgfqpoint{3.334580in}{1.243177in}}%
\pgfpathcurveto{\pgfqpoint{3.340404in}{1.243177in}}{\pgfqpoint{3.345990in}{1.245491in}}{\pgfqpoint{3.350108in}{1.249609in}}%
\pgfpathcurveto{\pgfqpoint{3.354226in}{1.253727in}}{\pgfqpoint{3.356540in}{1.259313in}}{\pgfqpoint{3.356540in}{1.265137in}}%
\pgfpathcurveto{\pgfqpoint{3.356540in}{1.270961in}}{\pgfqpoint{3.354226in}{1.276547in}}{\pgfqpoint{3.350108in}{1.280665in}}%
\pgfpathcurveto{\pgfqpoint{3.345990in}{1.284784in}}{\pgfqpoint{3.340404in}{1.287097in}}{\pgfqpoint{3.334580in}{1.287097in}}%
\pgfpathcurveto{\pgfqpoint{3.328756in}{1.287097in}}{\pgfqpoint{3.323170in}{1.284784in}}{\pgfqpoint{3.319052in}{1.280665in}}%
\pgfpathcurveto{\pgfqpoint{3.314933in}{1.276547in}}{\pgfqpoint{3.312620in}{1.270961in}}{\pgfqpoint{3.312620in}{1.265137in}}%
\pgfpathcurveto{\pgfqpoint{3.312620in}{1.259313in}}{\pgfqpoint{3.314933in}{1.253727in}}{\pgfqpoint{3.319052in}{1.249609in}}%
\pgfpathcurveto{\pgfqpoint{3.323170in}{1.245491in}}{\pgfqpoint{3.328756in}{1.243177in}}{\pgfqpoint{3.334580in}{1.243177in}}%
\pgfpathlineto{\pgfqpoint{3.334580in}{1.243177in}}%
\pgfpathclose%
\pgfusepath{stroke,fill}%
\end{pgfscope}%
\begin{pgfscope}%
\pgfpathrectangle{\pgfqpoint{0.997489in}{0.528000in}}{\pgfqpoint{4.565023in}{3.696000in}}%
\pgfusepath{clip}%
\pgfsetbuttcap%
\pgfsetroundjoin%
\definecolor{currentfill}{rgb}{0.800000,0.200000,0.200000}%
\pgfsetfillcolor{currentfill}%
\pgfsetlinewidth{1.003750pt}%
\definecolor{currentstroke}{rgb}{0.800000,0.200000,0.200000}%
\pgfsetstrokecolor{currentstroke}%
\pgfsetdash{}{0pt}%
\pgfpathmoveto{\pgfqpoint{3.398753in}{1.266650in}}%
\pgfpathcurveto{\pgfqpoint{3.404577in}{1.266650in}}{\pgfqpoint{3.410163in}{1.268964in}}{\pgfqpoint{3.414281in}{1.273082in}}%
\pgfpathcurveto{\pgfqpoint{3.418399in}{1.277200in}}{\pgfqpoint{3.420713in}{1.282786in}}{\pgfqpoint{3.420713in}{1.288610in}}%
\pgfpathcurveto{\pgfqpoint{3.420713in}{1.294434in}}{\pgfqpoint{3.418399in}{1.300020in}}{\pgfqpoint{3.414281in}{1.304138in}}%
\pgfpathcurveto{\pgfqpoint{3.410163in}{1.308257in}}{\pgfqpoint{3.404577in}{1.310570in}}{\pgfqpoint{3.398753in}{1.310570in}}%
\pgfpathcurveto{\pgfqpoint{3.392929in}{1.310570in}}{\pgfqpoint{3.387343in}{1.308257in}}{\pgfqpoint{3.383225in}{1.304138in}}%
\pgfpathcurveto{\pgfqpoint{3.379106in}{1.300020in}}{\pgfqpoint{3.376793in}{1.294434in}}{\pgfqpoint{3.376793in}{1.288610in}}%
\pgfpathcurveto{\pgfqpoint{3.376793in}{1.282786in}}{\pgfqpoint{3.379106in}{1.277200in}}{\pgfqpoint{3.383225in}{1.273082in}}%
\pgfpathcurveto{\pgfqpoint{3.387343in}{1.268964in}}{\pgfqpoint{3.392929in}{1.266650in}}{\pgfqpoint{3.398753in}{1.266650in}}%
\pgfpathlineto{\pgfqpoint{3.398753in}{1.266650in}}%
\pgfpathclose%
\pgfusepath{stroke,fill}%
\end{pgfscope}%
\begin{pgfscope}%
\pgfpathrectangle{\pgfqpoint{0.997489in}{0.528000in}}{\pgfqpoint{4.565023in}{3.696000in}}%
\pgfusepath{clip}%
\pgfsetbuttcap%
\pgfsetroundjoin%
\definecolor{currentfill}{rgb}{0.800000,0.200000,0.200000}%
\pgfsetfillcolor{currentfill}%
\pgfsetlinewidth{1.003750pt}%
\definecolor{currentstroke}{rgb}{0.800000,0.200000,0.200000}%
\pgfsetstrokecolor{currentstroke}%
\pgfsetdash{}{0pt}%
\pgfpathmoveto{\pgfqpoint{3.443197in}{1.365005in}}%
\pgfpathcurveto{\pgfqpoint{3.449021in}{1.365005in}}{\pgfqpoint{3.454607in}{1.367319in}}{\pgfqpoint{3.458725in}{1.371437in}}%
\pgfpathcurveto{\pgfqpoint{3.462843in}{1.375555in}}{\pgfqpoint{3.465157in}{1.381141in}}{\pgfqpoint{3.465157in}{1.386965in}}%
\pgfpathcurveto{\pgfqpoint{3.465157in}{1.392789in}}{\pgfqpoint{3.462843in}{1.398376in}}{\pgfqpoint{3.458725in}{1.402494in}}%
\pgfpathcurveto{\pgfqpoint{3.454607in}{1.406612in}}{\pgfqpoint{3.449021in}{1.408926in}}{\pgfqpoint{3.443197in}{1.408926in}}%
\pgfpathcurveto{\pgfqpoint{3.437373in}{1.408926in}}{\pgfqpoint{3.431787in}{1.406612in}}{\pgfqpoint{3.427669in}{1.402494in}}%
\pgfpathcurveto{\pgfqpoint{3.423550in}{1.398376in}}{\pgfqpoint{3.421237in}{1.392789in}}{\pgfqpoint{3.421237in}{1.386965in}}%
\pgfpathcurveto{\pgfqpoint{3.421237in}{1.381141in}}{\pgfqpoint{3.423550in}{1.375555in}}{\pgfqpoint{3.427669in}{1.371437in}}%
\pgfpathcurveto{\pgfqpoint{3.431787in}{1.367319in}}{\pgfqpoint{3.437373in}{1.365005in}}{\pgfqpoint{3.443197in}{1.365005in}}%
\pgfpathlineto{\pgfqpoint{3.443197in}{1.365005in}}%
\pgfpathclose%
\pgfusepath{stroke,fill}%
\end{pgfscope}%
\begin{pgfscope}%
\pgfpathrectangle{\pgfqpoint{0.997489in}{0.528000in}}{\pgfqpoint{4.565023in}{3.696000in}}%
\pgfusepath{clip}%
\pgfsetbuttcap%
\pgfsetroundjoin%
\definecolor{currentfill}{rgb}{0.800000,0.200000,0.200000}%
\pgfsetfillcolor{currentfill}%
\pgfsetlinewidth{1.003750pt}%
\definecolor{currentstroke}{rgb}{0.800000,0.200000,0.200000}%
\pgfsetstrokecolor{currentstroke}%
\pgfsetdash{}{0pt}%
\pgfpathmoveto{\pgfqpoint{3.525197in}{1.306221in}}%
\pgfpathcurveto{\pgfqpoint{3.531021in}{1.306221in}}{\pgfqpoint{3.536607in}{1.308535in}}{\pgfqpoint{3.540725in}{1.312653in}}%
\pgfpathcurveto{\pgfqpoint{3.544843in}{1.316771in}}{\pgfqpoint{3.547157in}{1.322358in}}{\pgfqpoint{3.547157in}{1.328181in}}%
\pgfpathcurveto{\pgfqpoint{3.547157in}{1.334005in}}{\pgfqpoint{3.544843in}{1.339592in}}{\pgfqpoint{3.540725in}{1.343710in}}%
\pgfpathcurveto{\pgfqpoint{3.536607in}{1.347828in}}{\pgfqpoint{3.531021in}{1.350142in}}{\pgfqpoint{3.525197in}{1.350142in}}%
\pgfpathcurveto{\pgfqpoint{3.519373in}{1.350142in}}{\pgfqpoint{3.513787in}{1.347828in}}{\pgfqpoint{3.509669in}{1.343710in}}%
\pgfpathcurveto{\pgfqpoint{3.505551in}{1.339592in}}{\pgfqpoint{3.503237in}{1.334005in}}{\pgfqpoint{3.503237in}{1.328181in}}%
\pgfpathcurveto{\pgfqpoint{3.503237in}{1.322358in}}{\pgfqpoint{3.505551in}{1.316771in}}{\pgfqpoint{3.509669in}{1.312653in}}%
\pgfpathcurveto{\pgfqpoint{3.513787in}{1.308535in}}{\pgfqpoint{3.519373in}{1.306221in}}{\pgfqpoint{3.525197in}{1.306221in}}%
\pgfpathlineto{\pgfqpoint{3.525197in}{1.306221in}}%
\pgfpathclose%
\pgfusepath{stroke,fill}%
\end{pgfscope}%
\begin{pgfscope}%
\pgfpathrectangle{\pgfqpoint{0.997489in}{0.528000in}}{\pgfqpoint{4.565023in}{3.696000in}}%
\pgfusepath{clip}%
\pgfsetbuttcap%
\pgfsetroundjoin%
\definecolor{currentfill}{rgb}{0.800000,0.200000,0.200000}%
\pgfsetfillcolor{currentfill}%
\pgfsetlinewidth{1.003750pt}%
\definecolor{currentstroke}{rgb}{0.800000,0.200000,0.200000}%
\pgfsetstrokecolor{currentstroke}%
\pgfsetdash{}{0pt}%
\pgfpathmoveto{\pgfqpoint{3.574908in}{1.360558in}}%
\pgfpathcurveto{\pgfqpoint{3.580732in}{1.360558in}}{\pgfqpoint{3.586319in}{1.362872in}}{\pgfqpoint{3.590437in}{1.366990in}}%
\pgfpathcurveto{\pgfqpoint{3.594555in}{1.371108in}}{\pgfqpoint{3.596869in}{1.376695in}}{\pgfqpoint{3.596869in}{1.382518in}}%
\pgfpathcurveto{\pgfqpoint{3.596869in}{1.388342in}}{\pgfqpoint{3.594555in}{1.393929in}}{\pgfqpoint{3.590437in}{1.398047in}}%
\pgfpathcurveto{\pgfqpoint{3.586319in}{1.402165in}}{\pgfqpoint{3.580732in}{1.404479in}}{\pgfqpoint{3.574908in}{1.404479in}}%
\pgfpathcurveto{\pgfqpoint{3.569085in}{1.404479in}}{\pgfqpoint{3.563498in}{1.402165in}}{\pgfqpoint{3.559380in}{1.398047in}}%
\pgfpathcurveto{\pgfqpoint{3.555262in}{1.393929in}}{\pgfqpoint{3.552948in}{1.388342in}}{\pgfqpoint{3.552948in}{1.382518in}}%
\pgfpathcurveto{\pgfqpoint{3.552948in}{1.376695in}}{\pgfqpoint{3.555262in}{1.371108in}}{\pgfqpoint{3.559380in}{1.366990in}}%
\pgfpathcurveto{\pgfqpoint{3.563498in}{1.362872in}}{\pgfqpoint{3.569085in}{1.360558in}}{\pgfqpoint{3.574908in}{1.360558in}}%
\pgfpathlineto{\pgfqpoint{3.574908in}{1.360558in}}%
\pgfpathclose%
\pgfusepath{stroke,fill}%
\end{pgfscope}%
\begin{pgfscope}%
\pgfpathrectangle{\pgfqpoint{0.997489in}{0.528000in}}{\pgfqpoint{4.565023in}{3.696000in}}%
\pgfusepath{clip}%
\pgfsetbuttcap%
\pgfsetroundjoin%
\definecolor{currentfill}{rgb}{0.800000,0.200000,0.200000}%
\pgfsetfillcolor{currentfill}%
\pgfsetlinewidth{1.003750pt}%
\definecolor{currentstroke}{rgb}{0.800000,0.200000,0.200000}%
\pgfsetstrokecolor{currentstroke}%
\pgfsetdash{}{0pt}%
\pgfpathmoveto{\pgfqpoint{3.653797in}{1.340924in}}%
\pgfpathcurveto{\pgfqpoint{3.659621in}{1.340924in}}{\pgfqpoint{3.665207in}{1.343238in}}{\pgfqpoint{3.669325in}{1.347356in}}%
\pgfpathcurveto{\pgfqpoint{3.673443in}{1.351474in}}{\pgfqpoint{3.675757in}{1.357060in}}{\pgfqpoint{3.675757in}{1.362884in}}%
\pgfpathcurveto{\pgfqpoint{3.675757in}{1.368708in}}{\pgfqpoint{3.673443in}{1.374294in}}{\pgfqpoint{3.669325in}{1.378413in}}%
\pgfpathcurveto{\pgfqpoint{3.665207in}{1.382531in}}{\pgfqpoint{3.659621in}{1.384845in}}{\pgfqpoint{3.653797in}{1.384845in}}%
\pgfpathcurveto{\pgfqpoint{3.647973in}{1.384845in}}{\pgfqpoint{3.642387in}{1.382531in}}{\pgfqpoint{3.638269in}{1.378413in}}%
\pgfpathcurveto{\pgfqpoint{3.634151in}{1.374294in}}{\pgfqpoint{3.631837in}{1.368708in}}{\pgfqpoint{3.631837in}{1.362884in}}%
\pgfpathcurveto{\pgfqpoint{3.631837in}{1.357060in}}{\pgfqpoint{3.634151in}{1.351474in}}{\pgfqpoint{3.638269in}{1.347356in}}%
\pgfpathcurveto{\pgfqpoint{3.642387in}{1.343238in}}{\pgfqpoint{3.647973in}{1.340924in}}{\pgfqpoint{3.653797in}{1.340924in}}%
\pgfpathlineto{\pgfqpoint{3.653797in}{1.340924in}}%
\pgfpathclose%
\pgfusepath{stroke,fill}%
\end{pgfscope}%
\begin{pgfscope}%
\pgfpathrectangle{\pgfqpoint{0.997489in}{0.528000in}}{\pgfqpoint{4.565023in}{3.696000in}}%
\pgfusepath{clip}%
\pgfsetbuttcap%
\pgfsetroundjoin%
\definecolor{currentfill}{rgb}{0.800000,0.200000,0.200000}%
\pgfsetfillcolor{currentfill}%
\pgfsetlinewidth{1.003750pt}%
\definecolor{currentstroke}{rgb}{0.800000,0.200000,0.200000}%
\pgfsetstrokecolor{currentstroke}%
\pgfsetdash{}{0pt}%
\pgfpathmoveto{\pgfqpoint{3.697319in}{1.400897in}}%
\pgfpathcurveto{\pgfqpoint{3.703143in}{1.400897in}}{\pgfqpoint{3.708729in}{1.403211in}}{\pgfqpoint{3.712848in}{1.407329in}}%
\pgfpathcurveto{\pgfqpoint{3.716966in}{1.411447in}}{\pgfqpoint{3.719280in}{1.417034in}}{\pgfqpoint{3.719280in}{1.422857in}}%
\pgfpathcurveto{\pgfqpoint{3.719280in}{1.428681in}}{\pgfqpoint{3.716966in}{1.434268in}}{\pgfqpoint{3.712848in}{1.438386in}}%
\pgfpathcurveto{\pgfqpoint{3.708729in}{1.442504in}}{\pgfqpoint{3.703143in}{1.444818in}}{\pgfqpoint{3.697319in}{1.444818in}}%
\pgfpathcurveto{\pgfqpoint{3.691495in}{1.444818in}}{\pgfqpoint{3.685909in}{1.442504in}}{\pgfqpoint{3.681791in}{1.438386in}}%
\pgfpathcurveto{\pgfqpoint{3.677673in}{1.434268in}}{\pgfqpoint{3.675359in}{1.428681in}}{\pgfqpoint{3.675359in}{1.422857in}}%
\pgfpathcurveto{\pgfqpoint{3.675359in}{1.417034in}}{\pgfqpoint{3.677673in}{1.411447in}}{\pgfqpoint{3.681791in}{1.407329in}}%
\pgfpathcurveto{\pgfqpoint{3.685909in}{1.403211in}}{\pgfqpoint{3.691495in}{1.400897in}}{\pgfqpoint{3.697319in}{1.400897in}}%
\pgfpathlineto{\pgfqpoint{3.697319in}{1.400897in}}%
\pgfpathclose%
\pgfusepath{stroke,fill}%
\end{pgfscope}%
\begin{pgfscope}%
\pgfpathrectangle{\pgfqpoint{0.997489in}{0.528000in}}{\pgfqpoint{4.565023in}{3.696000in}}%
\pgfusepath{clip}%
\pgfsetbuttcap%
\pgfsetroundjoin%
\definecolor{currentfill}{rgb}{0.800000,0.200000,0.200000}%
\pgfsetfillcolor{currentfill}%
\pgfsetlinewidth{1.003750pt}%
\definecolor{currentstroke}{rgb}{0.800000,0.200000,0.200000}%
\pgfsetstrokecolor{currentstroke}%
\pgfsetdash{}{0pt}%
\pgfpathmoveto{\pgfqpoint{3.740058in}{1.454360in}}%
\pgfpathcurveto{\pgfqpoint{3.745882in}{1.454360in}}{\pgfqpoint{3.751468in}{1.456673in}}{\pgfqpoint{3.755586in}{1.460792in}}%
\pgfpathcurveto{\pgfqpoint{3.759705in}{1.464910in}}{\pgfqpoint{3.762018in}{1.470496in}}{\pgfqpoint{3.762018in}{1.476320in}}%
\pgfpathcurveto{\pgfqpoint{3.762018in}{1.482144in}}{\pgfqpoint{3.759705in}{1.487730in}}{\pgfqpoint{3.755586in}{1.491848in}}%
\pgfpathcurveto{\pgfqpoint{3.751468in}{1.495966in}}{\pgfqpoint{3.745882in}{1.498280in}}{\pgfqpoint{3.740058in}{1.498280in}}%
\pgfpathcurveto{\pgfqpoint{3.734234in}{1.498280in}}{\pgfqpoint{3.728648in}{1.495966in}}{\pgfqpoint{3.724530in}{1.491848in}}%
\pgfpathcurveto{\pgfqpoint{3.720412in}{1.487730in}}{\pgfqpoint{3.718098in}{1.482144in}}{\pgfqpoint{3.718098in}{1.476320in}}%
\pgfpathcurveto{\pgfqpoint{3.718098in}{1.470496in}}{\pgfqpoint{3.720412in}{1.464910in}}{\pgfqpoint{3.724530in}{1.460792in}}%
\pgfpathcurveto{\pgfqpoint{3.728648in}{1.456673in}}{\pgfqpoint{3.734234in}{1.454360in}}{\pgfqpoint{3.740058in}{1.454360in}}%
\pgfpathlineto{\pgfqpoint{3.740058in}{1.454360in}}%
\pgfpathclose%
\pgfusepath{stroke,fill}%
\end{pgfscope}%
\begin{pgfscope}%
\pgfpathrectangle{\pgfqpoint{0.997489in}{0.528000in}}{\pgfqpoint{4.565023in}{3.696000in}}%
\pgfusepath{clip}%
\pgfsetbuttcap%
\pgfsetroundjoin%
\definecolor{currentfill}{rgb}{0.800000,0.200000,0.200000}%
\pgfsetfillcolor{currentfill}%
\pgfsetlinewidth{1.003750pt}%
\definecolor{currentstroke}{rgb}{0.800000,0.200000,0.200000}%
\pgfsetstrokecolor{currentstroke}%
\pgfsetdash{}{0pt}%
\pgfpathmoveto{\pgfqpoint{3.878025in}{1.369086in}}%
\pgfpathcurveto{\pgfqpoint{3.883849in}{1.369086in}}{\pgfqpoint{3.889436in}{1.371399in}}{\pgfqpoint{3.893554in}{1.375518in}}%
\pgfpathcurveto{\pgfqpoint{3.897672in}{1.379636in}}{\pgfqpoint{3.899986in}{1.385222in}}{\pgfqpoint{3.899986in}{1.391046in}}%
\pgfpathcurveto{\pgfqpoint{3.899986in}{1.396870in}}{\pgfqpoint{3.897672in}{1.402456in}}{\pgfqpoint{3.893554in}{1.406574in}}%
\pgfpathcurveto{\pgfqpoint{3.889436in}{1.410692in}}{\pgfqpoint{3.883849in}{1.413006in}}{\pgfqpoint{3.878025in}{1.413006in}}%
\pgfpathcurveto{\pgfqpoint{3.872202in}{1.413006in}}{\pgfqpoint{3.866615in}{1.410692in}}{\pgfqpoint{3.862497in}{1.406574in}}%
\pgfpathcurveto{\pgfqpoint{3.858379in}{1.402456in}}{\pgfqpoint{3.856065in}{1.396870in}}{\pgfqpoint{3.856065in}{1.391046in}}%
\pgfpathcurveto{\pgfqpoint{3.856065in}{1.385222in}}{\pgfqpoint{3.858379in}{1.379636in}}{\pgfqpoint{3.862497in}{1.375518in}}%
\pgfpathcurveto{\pgfqpoint{3.866615in}{1.371399in}}{\pgfqpoint{3.872202in}{1.369086in}}{\pgfqpoint{3.878025in}{1.369086in}}%
\pgfpathlineto{\pgfqpoint{3.878025in}{1.369086in}}%
\pgfpathclose%
\pgfusepath{stroke,fill}%
\end{pgfscope}%
\begin{pgfscope}%
\pgfpathrectangle{\pgfqpoint{0.997489in}{0.528000in}}{\pgfqpoint{4.565023in}{3.696000in}}%
\pgfusepath{clip}%
\pgfsetbuttcap%
\pgfsetroundjoin%
\definecolor{currentfill}{rgb}{0.800000,0.200000,0.200000}%
\pgfsetfillcolor{currentfill}%
\pgfsetlinewidth{1.003750pt}%
\definecolor{currentstroke}{rgb}{0.800000,0.200000,0.200000}%
\pgfsetstrokecolor{currentstroke}%
\pgfsetdash{}{0pt}%
\pgfpathmoveto{\pgfqpoint{3.836877in}{1.534358in}}%
\pgfpathcurveto{\pgfqpoint{3.842701in}{1.534358in}}{\pgfqpoint{3.848287in}{1.536672in}}{\pgfqpoint{3.852405in}{1.540790in}}%
\pgfpathcurveto{\pgfqpoint{3.856523in}{1.544908in}}{\pgfqpoint{3.858837in}{1.550494in}}{\pgfqpoint{3.858837in}{1.556318in}}%
\pgfpathcurveto{\pgfqpoint{3.858837in}{1.562142in}}{\pgfqpoint{3.856523in}{1.567728in}}{\pgfqpoint{3.852405in}{1.571846in}}%
\pgfpathcurveto{\pgfqpoint{3.848287in}{1.575964in}}{\pgfqpoint{3.842701in}{1.578278in}}{\pgfqpoint{3.836877in}{1.578278in}}%
\pgfpathcurveto{\pgfqpoint{3.831053in}{1.578278in}}{\pgfqpoint{3.825467in}{1.575964in}}{\pgfqpoint{3.821349in}{1.571846in}}%
\pgfpathcurveto{\pgfqpoint{3.817231in}{1.567728in}}{\pgfqpoint{3.814917in}{1.562142in}}{\pgfqpoint{3.814917in}{1.556318in}}%
\pgfpathcurveto{\pgfqpoint{3.814917in}{1.550494in}}{\pgfqpoint{3.817231in}{1.544908in}}{\pgfqpoint{3.821349in}{1.540790in}}%
\pgfpathcurveto{\pgfqpoint{3.825467in}{1.536672in}}{\pgfqpoint{3.831053in}{1.534358in}}{\pgfqpoint{3.836877in}{1.534358in}}%
\pgfpathlineto{\pgfqpoint{3.836877in}{1.534358in}}%
\pgfpathclose%
\pgfusepath{stroke,fill}%
\end{pgfscope}%
\begin{pgfscope}%
\pgfpathrectangle{\pgfqpoint{0.997489in}{0.528000in}}{\pgfqpoint{4.565023in}{3.696000in}}%
\pgfusepath{clip}%
\pgfsetbuttcap%
\pgfsetroundjoin%
\definecolor{currentfill}{rgb}{0.800000,0.200000,0.200000}%
\pgfsetfillcolor{currentfill}%
\pgfsetlinewidth{1.003750pt}%
\definecolor{currentstroke}{rgb}{0.800000,0.200000,0.200000}%
\pgfsetstrokecolor{currentstroke}%
\pgfsetdash{}{0pt}%
\pgfpathmoveto{\pgfqpoint{3.919926in}{1.536335in}}%
\pgfpathcurveto{\pgfqpoint{3.925750in}{1.536335in}}{\pgfqpoint{3.931336in}{1.538649in}}{\pgfqpoint{3.935454in}{1.542767in}}%
\pgfpathcurveto{\pgfqpoint{3.939572in}{1.546885in}}{\pgfqpoint{3.941886in}{1.552472in}}{\pgfqpoint{3.941886in}{1.558296in}}%
\pgfpathcurveto{\pgfqpoint{3.941886in}{1.564119in}}{\pgfqpoint{3.939572in}{1.569706in}}{\pgfqpoint{3.935454in}{1.573824in}}%
\pgfpathcurveto{\pgfqpoint{3.931336in}{1.577942in}}{\pgfqpoint{3.925750in}{1.580256in}}{\pgfqpoint{3.919926in}{1.580256in}}%
\pgfpathcurveto{\pgfqpoint{3.914102in}{1.580256in}}{\pgfqpoint{3.908516in}{1.577942in}}{\pgfqpoint{3.904398in}{1.573824in}}%
\pgfpathcurveto{\pgfqpoint{3.900280in}{1.569706in}}{\pgfqpoint{3.897966in}{1.564119in}}{\pgfqpoint{3.897966in}{1.558296in}}%
\pgfpathcurveto{\pgfqpoint{3.897966in}{1.552472in}}{\pgfqpoint{3.900280in}{1.546885in}}{\pgfqpoint{3.904398in}{1.542767in}}%
\pgfpathcurveto{\pgfqpoint{3.908516in}{1.538649in}}{\pgfqpoint{3.914102in}{1.536335in}}{\pgfqpoint{3.919926in}{1.536335in}}%
\pgfpathlineto{\pgfqpoint{3.919926in}{1.536335in}}%
\pgfpathclose%
\pgfusepath{stroke,fill}%
\end{pgfscope}%
\begin{pgfscope}%
\pgfpathrectangle{\pgfqpoint{0.997489in}{0.528000in}}{\pgfqpoint{4.565023in}{3.696000in}}%
\pgfusepath{clip}%
\pgfsetbuttcap%
\pgfsetroundjoin%
\definecolor{currentfill}{rgb}{0.800000,0.200000,0.200000}%
\pgfsetfillcolor{currentfill}%
\pgfsetlinewidth{1.003750pt}%
\definecolor{currentstroke}{rgb}{0.800000,0.200000,0.200000}%
\pgfsetstrokecolor{currentstroke}%
\pgfsetdash{}{0pt}%
\pgfpathmoveto{\pgfqpoint{3.976979in}{1.572721in}}%
\pgfpathcurveto{\pgfqpoint{3.982803in}{1.572721in}}{\pgfqpoint{3.988389in}{1.575035in}}{\pgfqpoint{3.992508in}{1.579153in}}%
\pgfpathcurveto{\pgfqpoint{3.996626in}{1.583271in}}{\pgfqpoint{3.998940in}{1.588858in}}{\pgfqpoint{3.998940in}{1.594681in}}%
\pgfpathcurveto{\pgfqpoint{3.998940in}{1.600505in}}{\pgfqpoint{3.996626in}{1.606092in}}{\pgfqpoint{3.992508in}{1.610210in}}%
\pgfpathcurveto{\pgfqpoint{3.988389in}{1.614328in}}{\pgfqpoint{3.982803in}{1.616642in}}{\pgfqpoint{3.976979in}{1.616642in}}%
\pgfpathcurveto{\pgfqpoint{3.971155in}{1.616642in}}{\pgfqpoint{3.965569in}{1.614328in}}{\pgfqpoint{3.961451in}{1.610210in}}%
\pgfpathcurveto{\pgfqpoint{3.957333in}{1.606092in}}{\pgfqpoint{3.955019in}{1.600505in}}{\pgfqpoint{3.955019in}{1.594681in}}%
\pgfpathcurveto{\pgfqpoint{3.955019in}{1.588858in}}{\pgfqpoint{3.957333in}{1.583271in}}{\pgfqpoint{3.961451in}{1.579153in}}%
\pgfpathcurveto{\pgfqpoint{3.965569in}{1.575035in}}{\pgfqpoint{3.971155in}{1.572721in}}{\pgfqpoint{3.976979in}{1.572721in}}%
\pgfpathlineto{\pgfqpoint{3.976979in}{1.572721in}}%
\pgfpathclose%
\pgfusepath{stroke,fill}%
\end{pgfscope}%
\begin{pgfscope}%
\pgfpathrectangle{\pgfqpoint{0.997489in}{0.528000in}}{\pgfqpoint{4.565023in}{3.696000in}}%
\pgfusepath{clip}%
\pgfsetbuttcap%
\pgfsetroundjoin%
\definecolor{currentfill}{rgb}{0.800000,0.200000,0.200000}%
\pgfsetfillcolor{currentfill}%
\pgfsetlinewidth{1.003750pt}%
\definecolor{currentstroke}{rgb}{0.800000,0.200000,0.200000}%
\pgfsetstrokecolor{currentstroke}%
\pgfsetdash{}{0pt}%
\pgfpathmoveto{\pgfqpoint{3.987909in}{1.650489in}}%
\pgfpathcurveto{\pgfqpoint{3.993732in}{1.650489in}}{\pgfqpoint{3.999319in}{1.652803in}}{\pgfqpoint{4.003437in}{1.656921in}}%
\pgfpathcurveto{\pgfqpoint{4.007555in}{1.661040in}}{\pgfqpoint{4.009869in}{1.666626in}}{\pgfqpoint{4.009869in}{1.672450in}}%
\pgfpathcurveto{\pgfqpoint{4.009869in}{1.678274in}}{\pgfqpoint{4.007555in}{1.683860in}}{\pgfqpoint{4.003437in}{1.687978in}}%
\pgfpathcurveto{\pgfqpoint{3.999319in}{1.692096in}}{\pgfqpoint{3.993732in}{1.694410in}}{\pgfqpoint{3.987909in}{1.694410in}}%
\pgfpathcurveto{\pgfqpoint{3.982085in}{1.694410in}}{\pgfqpoint{3.976498in}{1.692096in}}{\pgfqpoint{3.972380in}{1.687978in}}%
\pgfpathcurveto{\pgfqpoint{3.968262in}{1.683860in}}{\pgfqpoint{3.965948in}{1.678274in}}{\pgfqpoint{3.965948in}{1.672450in}}%
\pgfpathcurveto{\pgfqpoint{3.965948in}{1.666626in}}{\pgfqpoint{3.968262in}{1.661040in}}{\pgfqpoint{3.972380in}{1.656921in}}%
\pgfpathcurveto{\pgfqpoint{3.976498in}{1.652803in}}{\pgfqpoint{3.982085in}{1.650489in}}{\pgfqpoint{3.987909in}{1.650489in}}%
\pgfpathlineto{\pgfqpoint{3.987909in}{1.650489in}}%
\pgfpathclose%
\pgfusepath{stroke,fill}%
\end{pgfscope}%
\begin{pgfscope}%
\pgfpathrectangle{\pgfqpoint{0.997489in}{0.528000in}}{\pgfqpoint{4.565023in}{3.696000in}}%
\pgfusepath{clip}%
\pgfsetbuttcap%
\pgfsetroundjoin%
\definecolor{currentfill}{rgb}{0.800000,0.200000,0.200000}%
\pgfsetfillcolor{currentfill}%
\pgfsetlinewidth{1.003750pt}%
\definecolor{currentstroke}{rgb}{0.800000,0.200000,0.200000}%
\pgfsetstrokecolor{currentstroke}%
\pgfsetdash{}{0pt}%
\pgfpathmoveto{\pgfqpoint{4.087069in}{1.656739in}}%
\pgfpathcurveto{\pgfqpoint{4.092893in}{1.656739in}}{\pgfqpoint{4.098479in}{1.659053in}}{\pgfqpoint{4.102597in}{1.663171in}}%
\pgfpathcurveto{\pgfqpoint{4.106715in}{1.667289in}}{\pgfqpoint{4.109029in}{1.672875in}}{\pgfqpoint{4.109029in}{1.678699in}}%
\pgfpathcurveto{\pgfqpoint{4.109029in}{1.684523in}}{\pgfqpoint{4.106715in}{1.690109in}}{\pgfqpoint{4.102597in}{1.694227in}}%
\pgfpathcurveto{\pgfqpoint{4.098479in}{1.698345in}}{\pgfqpoint{4.092893in}{1.700659in}}{\pgfqpoint{4.087069in}{1.700659in}}%
\pgfpathcurveto{\pgfqpoint{4.081245in}{1.700659in}}{\pgfqpoint{4.075659in}{1.698345in}}{\pgfqpoint{4.071541in}{1.694227in}}%
\pgfpathcurveto{\pgfqpoint{4.067423in}{1.690109in}}{\pgfqpoint{4.065109in}{1.684523in}}{\pgfqpoint{4.065109in}{1.678699in}}%
\pgfpathcurveto{\pgfqpoint{4.065109in}{1.672875in}}{\pgfqpoint{4.067423in}{1.667289in}}{\pgfqpoint{4.071541in}{1.663171in}}%
\pgfpathcurveto{\pgfqpoint{4.075659in}{1.659053in}}{\pgfqpoint{4.081245in}{1.656739in}}{\pgfqpoint{4.087069in}{1.656739in}}%
\pgfpathlineto{\pgfqpoint{4.087069in}{1.656739in}}%
\pgfpathclose%
\pgfusepath{stroke,fill}%
\end{pgfscope}%
\begin{pgfscope}%
\pgfpathrectangle{\pgfqpoint{0.997489in}{0.528000in}}{\pgfqpoint{4.565023in}{3.696000in}}%
\pgfusepath{clip}%
\pgfsetbuttcap%
\pgfsetroundjoin%
\definecolor{currentfill}{rgb}{0.800000,0.200000,0.200000}%
\pgfsetfillcolor{currentfill}%
\pgfsetlinewidth{1.003750pt}%
\definecolor{currentstroke}{rgb}{0.800000,0.200000,0.200000}%
\pgfsetstrokecolor{currentstroke}%
\pgfsetdash{}{0pt}%
\pgfpathmoveto{\pgfqpoint{4.113936in}{1.720930in}}%
\pgfpathcurveto{\pgfqpoint{4.119760in}{1.720930in}}{\pgfqpoint{4.125346in}{1.723244in}}{\pgfqpoint{4.129464in}{1.727362in}}%
\pgfpathcurveto{\pgfqpoint{4.133582in}{1.731480in}}{\pgfqpoint{4.135896in}{1.737067in}}{\pgfqpoint{4.135896in}{1.742891in}}%
\pgfpathcurveto{\pgfqpoint{4.135896in}{1.748715in}}{\pgfqpoint{4.133582in}{1.754301in}}{\pgfqpoint{4.129464in}{1.758419in}}%
\pgfpathcurveto{\pgfqpoint{4.125346in}{1.762537in}}{\pgfqpoint{4.119760in}{1.764851in}}{\pgfqpoint{4.113936in}{1.764851in}}%
\pgfpathcurveto{\pgfqpoint{4.108112in}{1.764851in}}{\pgfqpoint{4.102526in}{1.762537in}}{\pgfqpoint{4.098407in}{1.758419in}}%
\pgfpathcurveto{\pgfqpoint{4.094289in}{1.754301in}}{\pgfqpoint{4.091975in}{1.748715in}}{\pgfqpoint{4.091975in}{1.742891in}}%
\pgfpathcurveto{\pgfqpoint{4.091975in}{1.737067in}}{\pgfqpoint{4.094289in}{1.731480in}}{\pgfqpoint{4.098407in}{1.727362in}}%
\pgfpathcurveto{\pgfqpoint{4.102526in}{1.723244in}}{\pgfqpoint{4.108112in}{1.720930in}}{\pgfqpoint{4.113936in}{1.720930in}}%
\pgfpathlineto{\pgfqpoint{4.113936in}{1.720930in}}%
\pgfpathclose%
\pgfusepath{stroke,fill}%
\end{pgfscope}%
\begin{pgfscope}%
\pgfpathrectangle{\pgfqpoint{0.997489in}{0.528000in}}{\pgfqpoint{4.565023in}{3.696000in}}%
\pgfusepath{clip}%
\pgfsetbuttcap%
\pgfsetroundjoin%
\definecolor{currentfill}{rgb}{0.800000,0.200000,0.200000}%
\pgfsetfillcolor{currentfill}%
\pgfsetlinewidth{1.003750pt}%
\definecolor{currentstroke}{rgb}{0.800000,0.200000,0.200000}%
\pgfsetstrokecolor{currentstroke}%
\pgfsetdash{}{0pt}%
\pgfpathmoveto{\pgfqpoint{4.033394in}{1.843125in}}%
\pgfpathcurveto{\pgfqpoint{4.039218in}{1.843125in}}{\pgfqpoint{4.044805in}{1.845439in}}{\pgfqpoint{4.048923in}{1.849557in}}%
\pgfpathcurveto{\pgfqpoint{4.053041in}{1.853675in}}{\pgfqpoint{4.055355in}{1.859261in}}{\pgfqpoint{4.055355in}{1.865085in}}%
\pgfpathcurveto{\pgfqpoint{4.055355in}{1.870909in}}{\pgfqpoint{4.053041in}{1.876495in}}{\pgfqpoint{4.048923in}{1.880614in}}%
\pgfpathcurveto{\pgfqpoint{4.044805in}{1.884732in}}{\pgfqpoint{4.039218in}{1.887046in}}{\pgfqpoint{4.033394in}{1.887046in}}%
\pgfpathcurveto{\pgfqpoint{4.027571in}{1.887046in}}{\pgfqpoint{4.021984in}{1.884732in}}{\pgfqpoint{4.017866in}{1.880614in}}%
\pgfpathcurveto{\pgfqpoint{4.013748in}{1.876495in}}{\pgfqpoint{4.011434in}{1.870909in}}{\pgfqpoint{4.011434in}{1.865085in}}%
\pgfpathcurveto{\pgfqpoint{4.011434in}{1.859261in}}{\pgfqpoint{4.013748in}{1.853675in}}{\pgfqpoint{4.017866in}{1.849557in}}%
\pgfpathcurveto{\pgfqpoint{4.021984in}{1.845439in}}{\pgfqpoint{4.027571in}{1.843125in}}{\pgfqpoint{4.033394in}{1.843125in}}%
\pgfpathlineto{\pgfqpoint{4.033394in}{1.843125in}}%
\pgfpathclose%
\pgfusepath{stroke,fill}%
\end{pgfscope}%
\begin{pgfscope}%
\pgfpathrectangle{\pgfqpoint{0.997489in}{0.528000in}}{\pgfqpoint{4.565023in}{3.696000in}}%
\pgfusepath{clip}%
\pgfsetbuttcap%
\pgfsetroundjoin%
\definecolor{currentfill}{rgb}{0.800000,0.200000,0.200000}%
\pgfsetfillcolor{currentfill}%
\pgfsetlinewidth{1.003750pt}%
\definecolor{currentstroke}{rgb}{0.800000,0.200000,0.200000}%
\pgfsetstrokecolor{currentstroke}%
\pgfsetdash{}{0pt}%
\pgfpathmoveto{\pgfqpoint{4.168277in}{1.844544in}}%
\pgfpathcurveto{\pgfqpoint{4.174101in}{1.844544in}}{\pgfqpoint{4.179687in}{1.846858in}}{\pgfqpoint{4.183805in}{1.850976in}}%
\pgfpathcurveto{\pgfqpoint{4.187923in}{1.855094in}}{\pgfqpoint{4.190237in}{1.860680in}}{\pgfqpoint{4.190237in}{1.866504in}}%
\pgfpathcurveto{\pgfqpoint{4.190237in}{1.872328in}}{\pgfqpoint{4.187923in}{1.877914in}}{\pgfqpoint{4.183805in}{1.882032in}}%
\pgfpathcurveto{\pgfqpoint{4.179687in}{1.886151in}}{\pgfqpoint{4.174101in}{1.888465in}}{\pgfqpoint{4.168277in}{1.888465in}}%
\pgfpathcurveto{\pgfqpoint{4.162453in}{1.888465in}}{\pgfqpoint{4.156867in}{1.886151in}}{\pgfqpoint{4.152749in}{1.882032in}}%
\pgfpathcurveto{\pgfqpoint{4.148630in}{1.877914in}}{\pgfqpoint{4.146316in}{1.872328in}}{\pgfqpoint{4.146316in}{1.866504in}}%
\pgfpathcurveto{\pgfqpoint{4.146316in}{1.860680in}}{\pgfqpoint{4.148630in}{1.855094in}}{\pgfqpoint{4.152749in}{1.850976in}}%
\pgfpathcurveto{\pgfqpoint{4.156867in}{1.846858in}}{\pgfqpoint{4.162453in}{1.844544in}}{\pgfqpoint{4.168277in}{1.844544in}}%
\pgfpathlineto{\pgfqpoint{4.168277in}{1.844544in}}%
\pgfpathclose%
\pgfusepath{stroke,fill}%
\end{pgfscope}%
\begin{pgfscope}%
\pgfpathrectangle{\pgfqpoint{0.997489in}{0.528000in}}{\pgfqpoint{4.565023in}{3.696000in}}%
\pgfusepath{clip}%
\pgfsetbuttcap%
\pgfsetroundjoin%
\definecolor{currentfill}{rgb}{0.800000,0.200000,0.200000}%
\pgfsetfillcolor{currentfill}%
\pgfsetlinewidth{1.003750pt}%
\definecolor{currentstroke}{rgb}{0.800000,0.200000,0.200000}%
\pgfsetstrokecolor{currentstroke}%
\pgfsetdash{}{0pt}%
\pgfpathmoveto{\pgfqpoint{4.250511in}{1.883597in}}%
\pgfpathcurveto{\pgfqpoint{4.256335in}{1.883597in}}{\pgfqpoint{4.261921in}{1.885911in}}{\pgfqpoint{4.266039in}{1.890029in}}%
\pgfpathcurveto{\pgfqpoint{4.270157in}{1.894148in}}{\pgfqpoint{4.272471in}{1.899734in}}{\pgfqpoint{4.272471in}{1.905558in}}%
\pgfpathcurveto{\pgfqpoint{4.272471in}{1.911382in}}{\pgfqpoint{4.270157in}{1.916968in}}{\pgfqpoint{4.266039in}{1.921086in}}%
\pgfpathcurveto{\pgfqpoint{4.261921in}{1.925204in}}{\pgfqpoint{4.256335in}{1.927518in}}{\pgfqpoint{4.250511in}{1.927518in}}%
\pgfpathcurveto{\pgfqpoint{4.244687in}{1.927518in}}{\pgfqpoint{4.239101in}{1.925204in}}{\pgfqpoint{4.234983in}{1.921086in}}%
\pgfpathcurveto{\pgfqpoint{4.230864in}{1.916968in}}{\pgfqpoint{4.228551in}{1.911382in}}{\pgfqpoint{4.228551in}{1.905558in}}%
\pgfpathcurveto{\pgfqpoint{4.228551in}{1.899734in}}{\pgfqpoint{4.230864in}{1.894148in}}{\pgfqpoint{4.234983in}{1.890029in}}%
\pgfpathcurveto{\pgfqpoint{4.239101in}{1.885911in}}{\pgfqpoint{4.244687in}{1.883597in}}{\pgfqpoint{4.250511in}{1.883597in}}%
\pgfpathlineto{\pgfqpoint{4.250511in}{1.883597in}}%
\pgfpathclose%
\pgfusepath{stroke,fill}%
\end{pgfscope}%
\begin{pgfscope}%
\pgfpathrectangle{\pgfqpoint{0.997489in}{0.528000in}}{\pgfqpoint{4.565023in}{3.696000in}}%
\pgfusepath{clip}%
\pgfsetbuttcap%
\pgfsetroundjoin%
\definecolor{currentfill}{rgb}{0.800000,0.200000,0.200000}%
\pgfsetfillcolor{currentfill}%
\pgfsetlinewidth{1.003750pt}%
\definecolor{currentstroke}{rgb}{0.800000,0.200000,0.200000}%
\pgfsetstrokecolor{currentstroke}%
\pgfsetdash{}{0pt}%
\pgfpathmoveto{\pgfqpoint{4.150060in}{1.990905in}}%
\pgfpathcurveto{\pgfqpoint{4.155884in}{1.990905in}}{\pgfqpoint{4.161470in}{1.993219in}}{\pgfqpoint{4.165588in}{1.997337in}}%
\pgfpathcurveto{\pgfqpoint{4.169706in}{2.001456in}}{\pgfqpoint{4.172020in}{2.007042in}}{\pgfqpoint{4.172020in}{2.012866in}}%
\pgfpathcurveto{\pgfqpoint{4.172020in}{2.018690in}}{\pgfqpoint{4.169706in}{2.024276in}}{\pgfqpoint{4.165588in}{2.028394in}}%
\pgfpathcurveto{\pgfqpoint{4.161470in}{2.032512in}}{\pgfqpoint{4.155884in}{2.034826in}}{\pgfqpoint{4.150060in}{2.034826in}}%
\pgfpathcurveto{\pgfqpoint{4.144236in}{2.034826in}}{\pgfqpoint{4.138650in}{2.032512in}}{\pgfqpoint{4.134531in}{2.028394in}}%
\pgfpathcurveto{\pgfqpoint{4.130413in}{2.024276in}}{\pgfqpoint{4.128099in}{2.018690in}}{\pgfqpoint{4.128099in}{2.012866in}}%
\pgfpathcurveto{\pgfqpoint{4.128099in}{2.007042in}}{\pgfqpoint{4.130413in}{2.001456in}}{\pgfqpoint{4.134531in}{1.997337in}}%
\pgfpathcurveto{\pgfqpoint{4.138650in}{1.993219in}}{\pgfqpoint{4.144236in}{1.990905in}}{\pgfqpoint{4.150060in}{1.990905in}}%
\pgfpathlineto{\pgfqpoint{4.150060in}{1.990905in}}%
\pgfpathclose%
\pgfusepath{stroke,fill}%
\end{pgfscope}%
\begin{pgfscope}%
\pgfpathrectangle{\pgfqpoint{0.997489in}{0.528000in}}{\pgfqpoint{4.565023in}{3.696000in}}%
\pgfusepath{clip}%
\pgfsetbuttcap%
\pgfsetroundjoin%
\definecolor{currentfill}{rgb}{0.800000,0.200000,0.200000}%
\pgfsetfillcolor{currentfill}%
\pgfsetlinewidth{1.003750pt}%
\definecolor{currentstroke}{rgb}{0.800000,0.200000,0.200000}%
\pgfsetstrokecolor{currentstroke}%
\pgfsetdash{}{0pt}%
\pgfpathmoveto{\pgfqpoint{4.136797in}{2.058696in}}%
\pgfpathcurveto{\pgfqpoint{4.142621in}{2.058696in}}{\pgfqpoint{4.148208in}{2.061010in}}{\pgfqpoint{4.152326in}{2.065128in}}%
\pgfpathcurveto{\pgfqpoint{4.156444in}{2.069247in}}{\pgfqpoint{4.158758in}{2.074833in}}{\pgfqpoint{4.158758in}{2.080657in}}%
\pgfpathcurveto{\pgfqpoint{4.158758in}{2.086481in}}{\pgfqpoint{4.156444in}{2.092067in}}{\pgfqpoint{4.152326in}{2.096185in}}%
\pgfpathcurveto{\pgfqpoint{4.148208in}{2.100303in}}{\pgfqpoint{4.142621in}{2.102617in}}{\pgfqpoint{4.136797in}{2.102617in}}%
\pgfpathcurveto{\pgfqpoint{4.130974in}{2.102617in}}{\pgfqpoint{4.125387in}{2.100303in}}{\pgfqpoint{4.121269in}{2.096185in}}%
\pgfpathcurveto{\pgfqpoint{4.117151in}{2.092067in}}{\pgfqpoint{4.114837in}{2.086481in}}{\pgfqpoint{4.114837in}{2.080657in}}%
\pgfpathcurveto{\pgfqpoint{4.114837in}{2.074833in}}{\pgfqpoint{4.117151in}{2.069247in}}{\pgfqpoint{4.121269in}{2.065128in}}%
\pgfpathcurveto{\pgfqpoint{4.125387in}{2.061010in}}{\pgfqpoint{4.130974in}{2.058696in}}{\pgfqpoint{4.136797in}{2.058696in}}%
\pgfpathlineto{\pgfqpoint{4.136797in}{2.058696in}}%
\pgfpathclose%
\pgfusepath{stroke,fill}%
\end{pgfscope}%
\begin{pgfscope}%
\pgfpathrectangle{\pgfqpoint{0.997489in}{0.528000in}}{\pgfqpoint{4.565023in}{3.696000in}}%
\pgfusepath{clip}%
\pgfsetbuttcap%
\pgfsetroundjoin%
\definecolor{currentfill}{rgb}{0.800000,0.200000,0.200000}%
\pgfsetfillcolor{currentfill}%
\pgfsetlinewidth{1.003750pt}%
\definecolor{currentstroke}{rgb}{0.800000,0.200000,0.200000}%
\pgfsetstrokecolor{currentstroke}%
\pgfsetdash{}{0pt}%
\pgfpathmoveto{\pgfqpoint{4.204711in}{2.106984in}}%
\pgfpathcurveto{\pgfqpoint{4.210535in}{2.106984in}}{\pgfqpoint{4.216121in}{2.109298in}}{\pgfqpoint{4.220239in}{2.113416in}}%
\pgfpathcurveto{\pgfqpoint{4.224357in}{2.117534in}}{\pgfqpoint{4.226671in}{2.123121in}}{\pgfqpoint{4.226671in}{2.128945in}}%
\pgfpathcurveto{\pgfqpoint{4.226671in}{2.134769in}}{\pgfqpoint{4.224357in}{2.140355in}}{\pgfqpoint{4.220239in}{2.144473in}}%
\pgfpathcurveto{\pgfqpoint{4.216121in}{2.148591in}}{\pgfqpoint{4.210535in}{2.150905in}}{\pgfqpoint{4.204711in}{2.150905in}}%
\pgfpathcurveto{\pgfqpoint{4.198887in}{2.150905in}}{\pgfqpoint{4.193301in}{2.148591in}}{\pgfqpoint{4.189183in}{2.144473in}}%
\pgfpathcurveto{\pgfqpoint{4.185065in}{2.140355in}}{\pgfqpoint{4.182751in}{2.134769in}}{\pgfqpoint{4.182751in}{2.128945in}}%
\pgfpathcurveto{\pgfqpoint{4.182751in}{2.123121in}}{\pgfqpoint{4.185065in}{2.117534in}}{\pgfqpoint{4.189183in}{2.113416in}}%
\pgfpathcurveto{\pgfqpoint{4.193301in}{2.109298in}}{\pgfqpoint{4.198887in}{2.106984in}}{\pgfqpoint{4.204711in}{2.106984in}}%
\pgfpathlineto{\pgfqpoint{4.204711in}{2.106984in}}%
\pgfpathclose%
\pgfusepath{stroke,fill}%
\end{pgfscope}%
\begin{pgfscope}%
\pgfpathrectangle{\pgfqpoint{0.997489in}{0.528000in}}{\pgfqpoint{4.565023in}{3.696000in}}%
\pgfusepath{clip}%
\pgfsetbuttcap%
\pgfsetroundjoin%
\definecolor{currentfill}{rgb}{0.800000,0.200000,0.200000}%
\pgfsetfillcolor{currentfill}%
\pgfsetlinewidth{1.003750pt}%
\definecolor{currentstroke}{rgb}{0.800000,0.200000,0.200000}%
\pgfsetstrokecolor{currentstroke}%
\pgfsetdash{}{0pt}%
\pgfpathmoveto{\pgfqpoint{4.228654in}{2.168243in}}%
\pgfpathcurveto{\pgfqpoint{4.234478in}{2.168243in}}{\pgfqpoint{4.240064in}{2.170557in}}{\pgfqpoint{4.244182in}{2.174675in}}%
\pgfpathcurveto{\pgfqpoint{4.248301in}{2.178793in}}{\pgfqpoint{4.250614in}{2.184379in}}{\pgfqpoint{4.250614in}{2.190203in}}%
\pgfpathcurveto{\pgfqpoint{4.250614in}{2.196027in}}{\pgfqpoint{4.248301in}{2.201613in}}{\pgfqpoint{4.244182in}{2.205731in}}%
\pgfpathcurveto{\pgfqpoint{4.240064in}{2.209849in}}{\pgfqpoint{4.234478in}{2.212163in}}{\pgfqpoint{4.228654in}{2.212163in}}%
\pgfpathcurveto{\pgfqpoint{4.222830in}{2.212163in}}{\pgfqpoint{4.217244in}{2.209849in}}{\pgfqpoint{4.213126in}{2.205731in}}%
\pgfpathcurveto{\pgfqpoint{4.209008in}{2.201613in}}{\pgfqpoint{4.206694in}{2.196027in}}{\pgfqpoint{4.206694in}{2.190203in}}%
\pgfpathcurveto{\pgfqpoint{4.206694in}{2.184379in}}{\pgfqpoint{4.209008in}{2.178793in}}{\pgfqpoint{4.213126in}{2.174675in}}%
\pgfpathcurveto{\pgfqpoint{4.217244in}{2.170557in}}{\pgfqpoint{4.222830in}{2.168243in}}{\pgfqpoint{4.228654in}{2.168243in}}%
\pgfpathlineto{\pgfqpoint{4.228654in}{2.168243in}}%
\pgfpathclose%
\pgfusepath{stroke,fill}%
\end{pgfscope}%
\begin{pgfscope}%
\pgfpathrectangle{\pgfqpoint{0.997489in}{0.528000in}}{\pgfqpoint{4.565023in}{3.696000in}}%
\pgfusepath{clip}%
\pgfsetbuttcap%
\pgfsetroundjoin%
\definecolor{currentfill}{rgb}{0.800000,0.200000,0.200000}%
\pgfsetfillcolor{currentfill}%
\pgfsetlinewidth{1.003750pt}%
\definecolor{currentstroke}{rgb}{0.800000,0.200000,0.200000}%
\pgfsetstrokecolor{currentstroke}%
\pgfsetdash{}{0pt}%
\pgfpathmoveto{\pgfqpoint{4.238090in}{2.232456in}}%
\pgfpathcurveto{\pgfqpoint{4.243914in}{2.232456in}}{\pgfqpoint{4.249500in}{2.234770in}}{\pgfqpoint{4.253619in}{2.238888in}}%
\pgfpathcurveto{\pgfqpoint{4.257737in}{2.243006in}}{\pgfqpoint{4.260051in}{2.248592in}}{\pgfqpoint{4.260051in}{2.254416in}}%
\pgfpathcurveto{\pgfqpoint{4.260051in}{2.260240in}}{\pgfqpoint{4.257737in}{2.265826in}}{\pgfqpoint{4.253619in}{2.269945in}}%
\pgfpathcurveto{\pgfqpoint{4.249500in}{2.274063in}}{\pgfqpoint{4.243914in}{2.276377in}}{\pgfqpoint{4.238090in}{2.276377in}}%
\pgfpathcurveto{\pgfqpoint{4.232266in}{2.276377in}}{\pgfqpoint{4.226680in}{2.274063in}}{\pgfqpoint{4.222562in}{2.269945in}}%
\pgfpathcurveto{\pgfqpoint{4.218444in}{2.265826in}}{\pgfqpoint{4.216130in}{2.260240in}}{\pgfqpoint{4.216130in}{2.254416in}}%
\pgfpathcurveto{\pgfqpoint{4.216130in}{2.248592in}}{\pgfqpoint{4.218444in}{2.243006in}}{\pgfqpoint{4.222562in}{2.238888in}}%
\pgfpathcurveto{\pgfqpoint{4.226680in}{2.234770in}}{\pgfqpoint{4.232266in}{2.232456in}}{\pgfqpoint{4.238090in}{2.232456in}}%
\pgfpathlineto{\pgfqpoint{4.238090in}{2.232456in}}%
\pgfpathclose%
\pgfusepath{stroke,fill}%
\end{pgfscope}%
\begin{pgfscope}%
\pgfpathrectangle{\pgfqpoint{0.997489in}{0.528000in}}{\pgfqpoint{4.565023in}{3.696000in}}%
\pgfusepath{clip}%
\pgfsetbuttcap%
\pgfsetroundjoin%
\definecolor{currentfill}{rgb}{0.800000,0.800000,0.200000}%
\pgfsetfillcolor{currentfill}%
\pgfsetlinewidth{1.003750pt}%
\definecolor{currentstroke}{rgb}{0.800000,0.800000,0.200000}%
\pgfsetstrokecolor{currentstroke}%
\pgfsetdash{}{0pt}%
\pgfpathmoveto{\pgfqpoint{4.292707in}{2.297347in}}%
\pgfpathcurveto{\pgfqpoint{4.298531in}{2.297347in}}{\pgfqpoint{4.304117in}{2.299661in}}{\pgfqpoint{4.308235in}{2.303779in}}%
\pgfpathcurveto{\pgfqpoint{4.312353in}{2.307898in}}{\pgfqpoint{4.314667in}{2.313484in}}{\pgfqpoint{4.314667in}{2.319308in}}%
\pgfpathcurveto{\pgfqpoint{4.314667in}{2.325132in}}{\pgfqpoint{4.312353in}{2.330718in}}{\pgfqpoint{4.308235in}{2.334836in}}%
\pgfpathcurveto{\pgfqpoint{4.304117in}{2.338954in}}{\pgfqpoint{4.298531in}{2.341268in}}{\pgfqpoint{4.292707in}{2.341268in}}%
\pgfpathcurveto{\pgfqpoint{4.286883in}{2.341268in}}{\pgfqpoint{4.281297in}{2.338954in}}{\pgfqpoint{4.277179in}{2.334836in}}%
\pgfpathcurveto{\pgfqpoint{4.273061in}{2.330718in}}{\pgfqpoint{4.270747in}{2.325132in}}{\pgfqpoint{4.270747in}{2.319308in}}%
\pgfpathcurveto{\pgfqpoint{4.270747in}{2.313484in}}{\pgfqpoint{4.273061in}{2.307898in}}{\pgfqpoint{4.277179in}{2.303779in}}%
\pgfpathcurveto{\pgfqpoint{4.281297in}{2.299661in}}{\pgfqpoint{4.286883in}{2.297347in}}{\pgfqpoint{4.292707in}{2.297347in}}%
\pgfpathlineto{\pgfqpoint{4.292707in}{2.297347in}}%
\pgfpathclose%
\pgfusepath{stroke,fill}%
\end{pgfscope}%
\begin{pgfscope}%
\pgfpathrectangle{\pgfqpoint{0.997489in}{0.528000in}}{\pgfqpoint{4.565023in}{3.696000in}}%
\pgfusepath{clip}%
\pgfsetbuttcap%
\pgfsetroundjoin%
\definecolor{currentfill}{rgb}{0.800000,0.800000,0.200000}%
\pgfsetfillcolor{currentfill}%
\pgfsetlinewidth{1.003750pt}%
\definecolor{currentstroke}{rgb}{0.800000,0.800000,0.200000}%
\pgfsetstrokecolor{currentstroke}%
\pgfsetdash{}{0pt}%
\pgfpathmoveto{\pgfqpoint{5.283390in}{3.139883in}}%
\pgfpathcurveto{\pgfqpoint{5.289214in}{3.139883in}}{\pgfqpoint{5.294800in}{3.142197in}}{\pgfqpoint{5.298918in}{3.146315in}}%
\pgfpathcurveto{\pgfqpoint{5.303036in}{3.150433in}}{\pgfqpoint{5.305350in}{3.156019in}}{\pgfqpoint{5.305350in}{3.161843in}}%
\pgfpathcurveto{\pgfqpoint{5.305350in}{3.167667in}}{\pgfqpoint{5.303036in}{3.173253in}}{\pgfqpoint{5.298918in}{3.177371in}}%
\pgfpathcurveto{\pgfqpoint{5.294800in}{3.181489in}}{\pgfqpoint{5.289214in}{3.183803in}}{\pgfqpoint{5.283390in}{3.183803in}}%
\pgfpathcurveto{\pgfqpoint{5.277566in}{3.183803in}}{\pgfqpoint{5.271980in}{3.181489in}}{\pgfqpoint{5.267861in}{3.177371in}}%
\pgfpathcurveto{\pgfqpoint{5.263743in}{3.173253in}}{\pgfqpoint{5.261429in}{3.167667in}}{\pgfqpoint{5.261429in}{3.161843in}}%
\pgfpathcurveto{\pgfqpoint{5.261429in}{3.156019in}}{\pgfqpoint{5.263743in}{3.150433in}}{\pgfqpoint{5.267861in}{3.146315in}}%
\pgfpathcurveto{\pgfqpoint{5.271980in}{3.142197in}}{\pgfqpoint{5.277566in}{3.139883in}}{\pgfqpoint{5.283390in}{3.139883in}}%
\pgfpathlineto{\pgfqpoint{5.283390in}{3.139883in}}%
\pgfpathclose%
\pgfusepath{stroke,fill}%
\end{pgfscope}%
\begin{pgfscope}%
\pgfpathrectangle{\pgfqpoint{0.997489in}{0.528000in}}{\pgfqpoint{4.565023in}{3.696000in}}%
\pgfusepath{clip}%
\pgfsetbuttcap%
\pgfsetroundjoin%
\definecolor{currentfill}{rgb}{0.800000,0.800000,0.200000}%
\pgfsetfillcolor{currentfill}%
\pgfsetlinewidth{1.003750pt}%
\definecolor{currentstroke}{rgb}{0.800000,0.800000,0.200000}%
\pgfsetstrokecolor{currentstroke}%
\pgfsetdash{}{0pt}%
\pgfpathmoveto{\pgfqpoint{5.205572in}{3.189136in}}%
\pgfpathcurveto{\pgfqpoint{5.211396in}{3.189136in}}{\pgfqpoint{5.216982in}{3.191450in}}{\pgfqpoint{5.221100in}{3.195568in}}%
\pgfpathcurveto{\pgfqpoint{5.225218in}{3.199686in}}{\pgfqpoint{5.227532in}{3.205272in}}{\pgfqpoint{5.227532in}{3.211096in}}%
\pgfpathcurveto{\pgfqpoint{5.227532in}{3.216920in}}{\pgfqpoint{5.225218in}{3.222506in}}{\pgfqpoint{5.221100in}{3.226625in}}%
\pgfpathcurveto{\pgfqpoint{5.216982in}{3.230743in}}{\pgfqpoint{5.211396in}{3.233057in}}{\pgfqpoint{5.205572in}{3.233057in}}%
\pgfpathcurveto{\pgfqpoint{5.199748in}{3.233057in}}{\pgfqpoint{5.194162in}{3.230743in}}{\pgfqpoint{5.190044in}{3.226625in}}%
\pgfpathcurveto{\pgfqpoint{5.185925in}{3.222506in}}{\pgfqpoint{5.183612in}{3.216920in}}{\pgfqpoint{5.183612in}{3.211096in}}%
\pgfpathcurveto{\pgfqpoint{5.183612in}{3.205272in}}{\pgfqpoint{5.185925in}{3.199686in}}{\pgfqpoint{5.190044in}{3.195568in}}%
\pgfpathcurveto{\pgfqpoint{5.194162in}{3.191450in}}{\pgfqpoint{5.199748in}{3.189136in}}{\pgfqpoint{5.205572in}{3.189136in}}%
\pgfpathlineto{\pgfqpoint{5.205572in}{3.189136in}}%
\pgfpathclose%
\pgfusepath{stroke,fill}%
\end{pgfscope}%
\begin{pgfscope}%
\pgfpathrectangle{\pgfqpoint{0.997489in}{0.528000in}}{\pgfqpoint{4.565023in}{3.696000in}}%
\pgfusepath{clip}%
\pgfsetbuttcap%
\pgfsetroundjoin%
\definecolor{currentfill}{rgb}{0.800000,0.800000,0.200000}%
\pgfsetfillcolor{currentfill}%
\pgfsetlinewidth{1.003750pt}%
\definecolor{currentstroke}{rgb}{0.800000,0.800000,0.200000}%
\pgfsetstrokecolor{currentstroke}%
\pgfsetdash{}{0pt}%
\pgfpathmoveto{\pgfqpoint{5.206207in}{3.238870in}}%
\pgfpathcurveto{\pgfqpoint{5.212031in}{3.238870in}}{\pgfqpoint{5.217617in}{3.241184in}}{\pgfqpoint{5.221735in}{3.245302in}}%
\pgfpathcurveto{\pgfqpoint{5.225853in}{3.249420in}}{\pgfqpoint{5.228167in}{3.255006in}}{\pgfqpoint{5.228167in}{3.260830in}}%
\pgfpathcurveto{\pgfqpoint{5.228167in}{3.266654in}}{\pgfqpoint{5.225853in}{3.272240in}}{\pgfqpoint{5.221735in}{3.276358in}}%
\pgfpathcurveto{\pgfqpoint{5.217617in}{3.280477in}}{\pgfqpoint{5.212031in}{3.282790in}}{\pgfqpoint{5.206207in}{3.282790in}}%
\pgfpathcurveto{\pgfqpoint{5.200383in}{3.282790in}}{\pgfqpoint{5.194797in}{3.280477in}}{\pgfqpoint{5.190678in}{3.276358in}}%
\pgfpathcurveto{\pgfqpoint{5.186560in}{3.272240in}}{\pgfqpoint{5.184246in}{3.266654in}}{\pgfqpoint{5.184246in}{3.260830in}}%
\pgfpathcurveto{\pgfqpoint{5.184246in}{3.255006in}}{\pgfqpoint{5.186560in}{3.249420in}}{\pgfqpoint{5.190678in}{3.245302in}}%
\pgfpathcurveto{\pgfqpoint{5.194797in}{3.241184in}}{\pgfqpoint{5.200383in}{3.238870in}}{\pgfqpoint{5.206207in}{3.238870in}}%
\pgfpathlineto{\pgfqpoint{5.206207in}{3.238870in}}%
\pgfpathclose%
\pgfusepath{stroke,fill}%
\end{pgfscope}%
\begin{pgfscope}%
\pgfpathrectangle{\pgfqpoint{0.997489in}{0.528000in}}{\pgfqpoint{4.565023in}{3.696000in}}%
\pgfusepath{clip}%
\pgfsetbuttcap%
\pgfsetroundjoin%
\definecolor{currentfill}{rgb}{0.800000,0.800000,0.200000}%
\pgfsetfillcolor{currentfill}%
\pgfsetlinewidth{1.003750pt}%
\definecolor{currentstroke}{rgb}{0.800000,0.800000,0.200000}%
\pgfsetstrokecolor{currentstroke}%
\pgfsetdash{}{0pt}%
\pgfpathmoveto{\pgfqpoint{5.176088in}{3.283571in}}%
\pgfpathcurveto{\pgfqpoint{5.181912in}{3.283571in}}{\pgfqpoint{5.187498in}{3.285885in}}{\pgfqpoint{5.191616in}{3.290003in}}%
\pgfpathcurveto{\pgfqpoint{5.195735in}{3.294121in}}{\pgfqpoint{5.198049in}{3.299708in}}{\pgfqpoint{5.198049in}{3.305531in}}%
\pgfpathcurveto{\pgfqpoint{5.198049in}{3.311355in}}{\pgfqpoint{5.195735in}{3.316942in}}{\pgfqpoint{5.191616in}{3.321060in}}%
\pgfpathcurveto{\pgfqpoint{5.187498in}{3.325178in}}{\pgfqpoint{5.181912in}{3.327492in}}{\pgfqpoint{5.176088in}{3.327492in}}%
\pgfpathcurveto{\pgfqpoint{5.170264in}{3.327492in}}{\pgfqpoint{5.164678in}{3.325178in}}{\pgfqpoint{5.160560in}{3.321060in}}%
\pgfpathcurveto{\pgfqpoint{5.156442in}{3.316942in}}{\pgfqpoint{5.154128in}{3.311355in}}{\pgfqpoint{5.154128in}{3.305531in}}%
\pgfpathcurveto{\pgfqpoint{5.154128in}{3.299708in}}{\pgfqpoint{5.156442in}{3.294121in}}{\pgfqpoint{5.160560in}{3.290003in}}%
\pgfpathcurveto{\pgfqpoint{5.164678in}{3.285885in}}{\pgfqpoint{5.170264in}{3.283571in}}{\pgfqpoint{5.176088in}{3.283571in}}%
\pgfpathlineto{\pgfqpoint{5.176088in}{3.283571in}}%
\pgfpathclose%
\pgfusepath{stroke,fill}%
\end{pgfscope}%
\begin{pgfscope}%
\pgfpathrectangle{\pgfqpoint{0.997489in}{0.528000in}}{\pgfqpoint{4.565023in}{3.696000in}}%
\pgfusepath{clip}%
\pgfsetbuttcap%
\pgfsetroundjoin%
\definecolor{currentfill}{rgb}{0.800000,0.800000,0.200000}%
\pgfsetfillcolor{currentfill}%
\pgfsetlinewidth{1.003750pt}%
\definecolor{currentstroke}{rgb}{0.800000,0.800000,0.200000}%
\pgfsetstrokecolor{currentstroke}%
\pgfsetdash{}{0pt}%
\pgfpathmoveto{\pgfqpoint{5.316155in}{3.369663in}}%
\pgfpathcurveto{\pgfqpoint{5.321979in}{3.369663in}}{\pgfqpoint{5.327565in}{3.371976in}}{\pgfqpoint{5.331683in}{3.376095in}}%
\pgfpathcurveto{\pgfqpoint{5.335802in}{3.380213in}}{\pgfqpoint{5.338115in}{3.385799in}}{\pgfqpoint{5.338115in}{3.391623in}}%
\pgfpathcurveto{\pgfqpoint{5.338115in}{3.397447in}}{\pgfqpoint{5.335802in}{3.403033in}}{\pgfqpoint{5.331683in}{3.407151in}}%
\pgfpathcurveto{\pgfqpoint{5.327565in}{3.411269in}}{\pgfqpoint{5.321979in}{3.413583in}}{\pgfqpoint{5.316155in}{3.413583in}}%
\pgfpathcurveto{\pgfqpoint{5.310331in}{3.413583in}}{\pgfqpoint{5.304745in}{3.411269in}}{\pgfqpoint{5.300627in}{3.407151in}}%
\pgfpathcurveto{\pgfqpoint{5.296509in}{3.403033in}}{\pgfqpoint{5.294195in}{3.397447in}}{\pgfqpoint{5.294195in}{3.391623in}}%
\pgfpathcurveto{\pgfqpoint{5.294195in}{3.385799in}}{\pgfqpoint{5.296509in}{3.380213in}}{\pgfqpoint{5.300627in}{3.376095in}}%
\pgfpathcurveto{\pgfqpoint{5.304745in}{3.371976in}}{\pgfqpoint{5.310331in}{3.369663in}}{\pgfqpoint{5.316155in}{3.369663in}}%
\pgfpathlineto{\pgfqpoint{5.316155in}{3.369663in}}%
\pgfpathclose%
\pgfusepath{stroke,fill}%
\end{pgfscope}%
\begin{pgfscope}%
\pgfpathrectangle{\pgfqpoint{0.997489in}{0.528000in}}{\pgfqpoint{4.565023in}{3.696000in}}%
\pgfusepath{clip}%
\pgfsetbuttcap%
\pgfsetroundjoin%
\definecolor{currentfill}{rgb}{0.800000,0.800000,0.200000}%
\pgfsetfillcolor{currentfill}%
\pgfsetlinewidth{1.003750pt}%
\definecolor{currentstroke}{rgb}{0.800000,0.800000,0.200000}%
\pgfsetstrokecolor{currentstroke}%
\pgfsetdash{}{0pt}%
\pgfpathmoveto{\pgfqpoint{5.149359in}{3.375959in}}%
\pgfpathcurveto{\pgfqpoint{5.155183in}{3.375959in}}{\pgfqpoint{5.160770in}{3.378273in}}{\pgfqpoint{5.164888in}{3.382391in}}%
\pgfpathcurveto{\pgfqpoint{5.169006in}{3.386509in}}{\pgfqpoint{5.171320in}{3.392095in}}{\pgfqpoint{5.171320in}{3.397919in}}%
\pgfpathcurveto{\pgfqpoint{5.171320in}{3.403743in}}{\pgfqpoint{5.169006in}{3.409329in}}{\pgfqpoint{5.164888in}{3.413447in}}%
\pgfpathcurveto{\pgfqpoint{5.160770in}{3.417565in}}{\pgfqpoint{5.155183in}{3.419879in}}{\pgfqpoint{5.149359in}{3.419879in}}%
\pgfpathcurveto{\pgfqpoint{5.143535in}{3.419879in}}{\pgfqpoint{5.137949in}{3.417565in}}{\pgfqpoint{5.133831in}{3.413447in}}%
\pgfpathcurveto{\pgfqpoint{5.129713in}{3.409329in}}{\pgfqpoint{5.127399in}{3.403743in}}{\pgfqpoint{5.127399in}{3.397919in}}%
\pgfpathcurveto{\pgfqpoint{5.127399in}{3.392095in}}{\pgfqpoint{5.129713in}{3.386509in}}{\pgfqpoint{5.133831in}{3.382391in}}%
\pgfpathcurveto{\pgfqpoint{5.137949in}{3.378273in}}{\pgfqpoint{5.143535in}{3.375959in}}{\pgfqpoint{5.149359in}{3.375959in}}%
\pgfpathlineto{\pgfqpoint{5.149359in}{3.375959in}}%
\pgfpathclose%
\pgfusepath{stroke,fill}%
\end{pgfscope}%
\begin{pgfscope}%
\pgfpathrectangle{\pgfqpoint{0.997489in}{0.528000in}}{\pgfqpoint{4.565023in}{3.696000in}}%
\pgfusepath{clip}%
\pgfsetbuttcap%
\pgfsetroundjoin%
\definecolor{currentfill}{rgb}{0.800000,0.800000,0.200000}%
\pgfsetfillcolor{currentfill}%
\pgfsetlinewidth{1.003750pt}%
\definecolor{currentstroke}{rgb}{0.800000,0.800000,0.200000}%
\pgfsetstrokecolor{currentstroke}%
\pgfsetdash{}{0pt}%
\pgfpathmoveto{\pgfqpoint{5.289540in}{3.483766in}}%
\pgfpathcurveto{\pgfqpoint{5.295364in}{3.483766in}}{\pgfqpoint{5.300950in}{3.486080in}}{\pgfqpoint{5.305068in}{3.490198in}}%
\pgfpathcurveto{\pgfqpoint{5.309186in}{3.494316in}}{\pgfqpoint{5.311500in}{3.499902in}}{\pgfqpoint{5.311500in}{3.505726in}}%
\pgfpathcurveto{\pgfqpoint{5.311500in}{3.511550in}}{\pgfqpoint{5.309186in}{3.517136in}}{\pgfqpoint{5.305068in}{3.521254in}}%
\pgfpathcurveto{\pgfqpoint{5.300950in}{3.525372in}}{\pgfqpoint{5.295364in}{3.527686in}}{\pgfqpoint{5.289540in}{3.527686in}}%
\pgfpathcurveto{\pgfqpoint{5.283716in}{3.527686in}}{\pgfqpoint{5.278130in}{3.525372in}}{\pgfqpoint{5.274012in}{3.521254in}}%
\pgfpathcurveto{\pgfqpoint{5.269894in}{3.517136in}}{\pgfqpoint{5.267580in}{3.511550in}}{\pgfqpoint{5.267580in}{3.505726in}}%
\pgfpathcurveto{\pgfqpoint{5.267580in}{3.499902in}}{\pgfqpoint{5.269894in}{3.494316in}}{\pgfqpoint{5.274012in}{3.490198in}}%
\pgfpathcurveto{\pgfqpoint{5.278130in}{3.486080in}}{\pgfqpoint{5.283716in}{3.483766in}}{\pgfqpoint{5.289540in}{3.483766in}}%
\pgfpathlineto{\pgfqpoint{5.289540in}{3.483766in}}%
\pgfpathclose%
\pgfusepath{stroke,fill}%
\end{pgfscope}%
\begin{pgfscope}%
\pgfpathrectangle{\pgfqpoint{0.997489in}{0.528000in}}{\pgfqpoint{4.565023in}{3.696000in}}%
\pgfusepath{clip}%
\pgfsetbuttcap%
\pgfsetroundjoin%
\definecolor{currentfill}{rgb}{0.800000,0.800000,0.200000}%
\pgfsetfillcolor{currentfill}%
\pgfsetlinewidth{1.003750pt}%
\definecolor{currentstroke}{rgb}{0.800000,0.800000,0.200000}%
\pgfsetstrokecolor{currentstroke}%
\pgfsetdash{}{0pt}%
\pgfpathmoveto{\pgfqpoint{5.144365in}{3.479655in}}%
\pgfpathcurveto{\pgfqpoint{5.150189in}{3.479655in}}{\pgfqpoint{5.155775in}{3.481969in}}{\pgfqpoint{5.159893in}{3.486087in}}%
\pgfpathcurveto{\pgfqpoint{5.164011in}{3.490205in}}{\pgfqpoint{5.166325in}{3.495791in}}{\pgfqpoint{5.166325in}{3.501615in}}%
\pgfpathcurveto{\pgfqpoint{5.166325in}{3.507439in}}{\pgfqpoint{5.164011in}{3.513025in}}{\pgfqpoint{5.159893in}{3.517143in}}%
\pgfpathcurveto{\pgfqpoint{5.155775in}{3.521261in}}{\pgfqpoint{5.150189in}{3.523575in}}{\pgfqpoint{5.144365in}{3.523575in}}%
\pgfpathcurveto{\pgfqpoint{5.138541in}{3.523575in}}{\pgfqpoint{5.132955in}{3.521261in}}{\pgfqpoint{5.128837in}{3.517143in}}%
\pgfpathcurveto{\pgfqpoint{5.124719in}{3.513025in}}{\pgfqpoint{5.122405in}{3.507439in}}{\pgfqpoint{5.122405in}{3.501615in}}%
\pgfpathcurveto{\pgfqpoint{5.122405in}{3.495791in}}{\pgfqpoint{5.124719in}{3.490205in}}{\pgfqpoint{5.128837in}{3.486087in}}%
\pgfpathcurveto{\pgfqpoint{5.132955in}{3.481969in}}{\pgfqpoint{5.138541in}{3.479655in}}{\pgfqpoint{5.144365in}{3.479655in}}%
\pgfpathlineto{\pgfqpoint{5.144365in}{3.479655in}}%
\pgfpathclose%
\pgfusepath{stroke,fill}%
\end{pgfscope}%
\begin{pgfscope}%
\pgfpathrectangle{\pgfqpoint{0.997489in}{0.528000in}}{\pgfqpoint{4.565023in}{3.696000in}}%
\pgfusepath{clip}%
\pgfsetbuttcap%
\pgfsetroundjoin%
\definecolor{currentfill}{rgb}{0.800000,0.800000,0.200000}%
\pgfsetfillcolor{currentfill}%
\pgfsetlinewidth{1.003750pt}%
\definecolor{currentstroke}{rgb}{0.800000,0.800000,0.200000}%
\pgfsetstrokecolor{currentstroke}%
\pgfsetdash{}{0pt}%
\pgfpathmoveto{\pgfqpoint{5.110476in}{3.518177in}}%
\pgfpathcurveto{\pgfqpoint{5.116300in}{3.518177in}}{\pgfqpoint{5.121887in}{3.520491in}}{\pgfqpoint{5.126005in}{3.524609in}}%
\pgfpathcurveto{\pgfqpoint{5.130123in}{3.528727in}}{\pgfqpoint{5.132437in}{3.534314in}}{\pgfqpoint{5.132437in}{3.540137in}}%
\pgfpathcurveto{\pgfqpoint{5.132437in}{3.545961in}}{\pgfqpoint{5.130123in}{3.551548in}}{\pgfqpoint{5.126005in}{3.555666in}}%
\pgfpathcurveto{\pgfqpoint{5.121887in}{3.559784in}}{\pgfqpoint{5.116300in}{3.562098in}}{\pgfqpoint{5.110476in}{3.562098in}}%
\pgfpathcurveto{\pgfqpoint{5.104652in}{3.562098in}}{\pgfqpoint{5.099066in}{3.559784in}}{\pgfqpoint{5.094948in}{3.555666in}}%
\pgfpathcurveto{\pgfqpoint{5.090830in}{3.551548in}}{\pgfqpoint{5.088516in}{3.545961in}}{\pgfqpoint{5.088516in}{3.540137in}}%
\pgfpathcurveto{\pgfqpoint{5.088516in}{3.534314in}}{\pgfqpoint{5.090830in}{3.528727in}}{\pgfqpoint{5.094948in}{3.524609in}}%
\pgfpathcurveto{\pgfqpoint{5.099066in}{3.520491in}}{\pgfqpoint{5.104652in}{3.518177in}}{\pgfqpoint{5.110476in}{3.518177in}}%
\pgfpathlineto{\pgfqpoint{5.110476in}{3.518177in}}%
\pgfpathclose%
\pgfusepath{stroke,fill}%
\end{pgfscope}%
\begin{pgfscope}%
\pgfpathrectangle{\pgfqpoint{0.997489in}{0.528000in}}{\pgfqpoint{4.565023in}{3.696000in}}%
\pgfusepath{clip}%
\pgfsetbuttcap%
\pgfsetroundjoin%
\definecolor{currentfill}{rgb}{0.800000,0.800000,0.200000}%
\pgfsetfillcolor{currentfill}%
\pgfsetlinewidth{1.003750pt}%
\definecolor{currentstroke}{rgb}{0.800000,0.800000,0.200000}%
\pgfsetstrokecolor{currentstroke}%
\pgfsetdash{}{0pt}%
\pgfpathmoveto{\pgfqpoint{5.136151in}{3.593338in}}%
\pgfpathcurveto{\pgfqpoint{5.141975in}{3.593338in}}{\pgfqpoint{5.147561in}{3.595652in}}{\pgfqpoint{5.151680in}{3.599770in}}%
\pgfpathcurveto{\pgfqpoint{5.155798in}{3.603888in}}{\pgfqpoint{5.158112in}{3.609474in}}{\pgfqpoint{5.158112in}{3.615298in}}%
\pgfpathcurveto{\pgfqpoint{5.158112in}{3.621122in}}{\pgfqpoint{5.155798in}{3.626708in}}{\pgfqpoint{5.151680in}{3.630826in}}%
\pgfpathcurveto{\pgfqpoint{5.147561in}{3.634944in}}{\pgfqpoint{5.141975in}{3.637258in}}{\pgfqpoint{5.136151in}{3.637258in}}%
\pgfpathcurveto{\pgfqpoint{5.130327in}{3.637258in}}{\pgfqpoint{5.124741in}{3.634944in}}{\pgfqpoint{5.120623in}{3.630826in}}%
\pgfpathcurveto{\pgfqpoint{5.116505in}{3.626708in}}{\pgfqpoint{5.114191in}{3.621122in}}{\pgfqpoint{5.114191in}{3.615298in}}%
\pgfpathcurveto{\pgfqpoint{5.114191in}{3.609474in}}{\pgfqpoint{5.116505in}{3.603888in}}{\pgfqpoint{5.120623in}{3.599770in}}%
\pgfpathcurveto{\pgfqpoint{5.124741in}{3.595652in}}{\pgfqpoint{5.130327in}{3.593338in}}{\pgfqpoint{5.136151in}{3.593338in}}%
\pgfpathlineto{\pgfqpoint{5.136151in}{3.593338in}}%
\pgfpathclose%
\pgfusepath{stroke,fill}%
\end{pgfscope}%
\begin{pgfscope}%
\pgfpathrectangle{\pgfqpoint{0.997489in}{0.528000in}}{\pgfqpoint{4.565023in}{3.696000in}}%
\pgfusepath{clip}%
\pgfsetbuttcap%
\pgfsetroundjoin%
\definecolor{currentfill}{rgb}{0.800000,0.800000,0.200000}%
\pgfsetfillcolor{currentfill}%
\pgfsetlinewidth{1.003750pt}%
\definecolor{currentstroke}{rgb}{0.800000,0.800000,0.200000}%
\pgfsetstrokecolor{currentstroke}%
\pgfsetdash{}{0pt}%
\pgfpathmoveto{\pgfqpoint{5.079770in}{3.617885in}}%
\pgfpathcurveto{\pgfqpoint{5.085594in}{3.617885in}}{\pgfqpoint{5.091180in}{3.620199in}}{\pgfqpoint{5.095298in}{3.624317in}}%
\pgfpathcurveto{\pgfqpoint{5.099416in}{3.628435in}}{\pgfqpoint{5.101730in}{3.634021in}}{\pgfqpoint{5.101730in}{3.639845in}}%
\pgfpathcurveto{\pgfqpoint{5.101730in}{3.645669in}}{\pgfqpoint{5.099416in}{3.651255in}}{\pgfqpoint{5.095298in}{3.655373in}}%
\pgfpathcurveto{\pgfqpoint{5.091180in}{3.659491in}}{\pgfqpoint{5.085594in}{3.661805in}}{\pgfqpoint{5.079770in}{3.661805in}}%
\pgfpathcurveto{\pgfqpoint{5.073946in}{3.661805in}}{\pgfqpoint{5.068360in}{3.659491in}}{\pgfqpoint{5.064241in}{3.655373in}}%
\pgfpathcurveto{\pgfqpoint{5.060123in}{3.651255in}}{\pgfqpoint{5.057809in}{3.645669in}}{\pgfqpoint{5.057809in}{3.639845in}}%
\pgfpathcurveto{\pgfqpoint{5.057809in}{3.634021in}}{\pgfqpoint{5.060123in}{3.628435in}}{\pgfqpoint{5.064241in}{3.624317in}}%
\pgfpathcurveto{\pgfqpoint{5.068360in}{3.620199in}}{\pgfqpoint{5.073946in}{3.617885in}}{\pgfqpoint{5.079770in}{3.617885in}}%
\pgfpathlineto{\pgfqpoint{5.079770in}{3.617885in}}%
\pgfpathclose%
\pgfusepath{stroke,fill}%
\end{pgfscope}%
\begin{pgfscope}%
\pgfpathrectangle{\pgfqpoint{0.997489in}{0.528000in}}{\pgfqpoint{4.565023in}{3.696000in}}%
\pgfusepath{clip}%
\pgfsetbuttcap%
\pgfsetroundjoin%
\definecolor{currentfill}{rgb}{0.800000,0.800000,0.200000}%
\pgfsetfillcolor{currentfill}%
\pgfsetlinewidth{1.003750pt}%
\definecolor{currentstroke}{rgb}{0.800000,0.800000,0.200000}%
\pgfsetstrokecolor{currentstroke}%
\pgfsetdash{}{0pt}%
\pgfpathmoveto{\pgfqpoint{5.079122in}{3.684090in}}%
\pgfpathcurveto{\pgfqpoint{5.084946in}{3.684090in}}{\pgfqpoint{5.090532in}{3.686403in}}{\pgfqpoint{5.094650in}{3.690522in}}%
\pgfpathcurveto{\pgfqpoint{5.098768in}{3.694640in}}{\pgfqpoint{5.101082in}{3.700226in}}{\pgfqpoint{5.101082in}{3.706050in}}%
\pgfpathcurveto{\pgfqpoint{5.101082in}{3.711874in}}{\pgfqpoint{5.098768in}{3.717460in}}{\pgfqpoint{5.094650in}{3.721578in}}%
\pgfpathcurveto{\pgfqpoint{5.090532in}{3.725696in}}{\pgfqpoint{5.084946in}{3.728010in}}{\pgfqpoint{5.079122in}{3.728010in}}%
\pgfpathcurveto{\pgfqpoint{5.073298in}{3.728010in}}{\pgfqpoint{5.067712in}{3.725696in}}{\pgfqpoint{5.063594in}{3.721578in}}%
\pgfpathcurveto{\pgfqpoint{5.059476in}{3.717460in}}{\pgfqpoint{5.057162in}{3.711874in}}{\pgfqpoint{5.057162in}{3.706050in}}%
\pgfpathcurveto{\pgfqpoint{5.057162in}{3.700226in}}{\pgfqpoint{5.059476in}{3.694640in}}{\pgfqpoint{5.063594in}{3.690522in}}%
\pgfpathcurveto{\pgfqpoint{5.067712in}{3.686403in}}{\pgfqpoint{5.073298in}{3.684090in}}{\pgfqpoint{5.079122in}{3.684090in}}%
\pgfpathlineto{\pgfqpoint{5.079122in}{3.684090in}}%
\pgfpathclose%
\pgfusepath{stroke,fill}%
\end{pgfscope}%
\begin{pgfscope}%
\pgfpathrectangle{\pgfqpoint{0.997489in}{0.528000in}}{\pgfqpoint{4.565023in}{3.696000in}}%
\pgfusepath{clip}%
\pgfsetbuttcap%
\pgfsetroundjoin%
\definecolor{currentfill}{rgb}{0.800000,0.800000,0.200000}%
\pgfsetfillcolor{currentfill}%
\pgfsetlinewidth{1.003750pt}%
\definecolor{currentstroke}{rgb}{0.800000,0.800000,0.200000}%
\pgfsetstrokecolor{currentstroke}%
\pgfsetdash{}{0pt}%
\pgfpathmoveto{\pgfqpoint{5.006810in}{3.689335in}}%
\pgfpathcurveto{\pgfqpoint{5.012634in}{3.689335in}}{\pgfqpoint{5.018220in}{3.691648in}}{\pgfqpoint{5.022338in}{3.695767in}}%
\pgfpathcurveto{\pgfqpoint{5.026456in}{3.699885in}}{\pgfqpoint{5.028770in}{3.705471in}}{\pgfqpoint{5.028770in}{3.711295in}}%
\pgfpathcurveto{\pgfqpoint{5.028770in}{3.717119in}}{\pgfqpoint{5.026456in}{3.722705in}}{\pgfqpoint{5.022338in}{3.726823in}}%
\pgfpathcurveto{\pgfqpoint{5.018220in}{3.730941in}}{\pgfqpoint{5.012634in}{3.733255in}}{\pgfqpoint{5.006810in}{3.733255in}}%
\pgfpathcurveto{\pgfqpoint{5.000986in}{3.733255in}}{\pgfqpoint{4.995400in}{3.730941in}}{\pgfqpoint{4.991282in}{3.726823in}}%
\pgfpathcurveto{\pgfqpoint{4.987164in}{3.722705in}}{\pgfqpoint{4.984850in}{3.717119in}}{\pgfqpoint{4.984850in}{3.711295in}}%
\pgfpathcurveto{\pgfqpoint{4.984850in}{3.705471in}}{\pgfqpoint{4.987164in}{3.699885in}}{\pgfqpoint{4.991282in}{3.695767in}}%
\pgfpathcurveto{\pgfqpoint{4.995400in}{3.691648in}}{\pgfqpoint{5.000986in}{3.689335in}}{\pgfqpoint{5.006810in}{3.689335in}}%
\pgfpathlineto{\pgfqpoint{5.006810in}{3.689335in}}%
\pgfpathclose%
\pgfusepath{stroke,fill}%
\end{pgfscope}%
\begin{pgfscope}%
\pgfpathrectangle{\pgfqpoint{0.997489in}{0.528000in}}{\pgfqpoint{4.565023in}{3.696000in}}%
\pgfusepath{clip}%
\pgfsetbuttcap%
\pgfsetroundjoin%
\definecolor{currentfill}{rgb}{0.800000,0.800000,0.200000}%
\pgfsetfillcolor{currentfill}%
\pgfsetlinewidth{1.003750pt}%
\definecolor{currentstroke}{rgb}{0.800000,0.800000,0.200000}%
\pgfsetstrokecolor{currentstroke}%
\pgfsetdash{}{0pt}%
\pgfpathmoveto{\pgfqpoint{4.990590in}{3.746201in}}%
\pgfpathcurveto{\pgfqpoint{4.996414in}{3.746201in}}{\pgfqpoint{5.002000in}{3.748515in}}{\pgfqpoint{5.006119in}{3.752633in}}%
\pgfpathcurveto{\pgfqpoint{5.010237in}{3.756751in}}{\pgfqpoint{5.012551in}{3.762337in}}{\pgfqpoint{5.012551in}{3.768161in}}%
\pgfpathcurveto{\pgfqpoint{5.012551in}{3.773985in}}{\pgfqpoint{5.010237in}{3.779571in}}{\pgfqpoint{5.006119in}{3.783689in}}%
\pgfpathcurveto{\pgfqpoint{5.002000in}{3.787808in}}{\pgfqpoint{4.996414in}{3.790121in}}{\pgfqpoint{4.990590in}{3.790121in}}%
\pgfpathcurveto{\pgfqpoint{4.984766in}{3.790121in}}{\pgfqpoint{4.979180in}{3.787808in}}{\pgfqpoint{4.975062in}{3.783689in}}%
\pgfpathcurveto{\pgfqpoint{4.970944in}{3.779571in}}{\pgfqpoint{4.968630in}{3.773985in}}{\pgfqpoint{4.968630in}{3.768161in}}%
\pgfpathcurveto{\pgfqpoint{4.968630in}{3.762337in}}{\pgfqpoint{4.970944in}{3.756751in}}{\pgfqpoint{4.975062in}{3.752633in}}%
\pgfpathcurveto{\pgfqpoint{4.979180in}{3.748515in}}{\pgfqpoint{4.984766in}{3.746201in}}{\pgfqpoint{4.990590in}{3.746201in}}%
\pgfpathlineto{\pgfqpoint{4.990590in}{3.746201in}}%
\pgfpathclose%
\pgfusepath{stroke,fill}%
\end{pgfscope}%
\begin{pgfscope}%
\pgfpathrectangle{\pgfqpoint{0.997489in}{0.528000in}}{\pgfqpoint{4.565023in}{3.696000in}}%
\pgfusepath{clip}%
\pgfsetbuttcap%
\pgfsetroundjoin%
\definecolor{currentfill}{rgb}{0.800000,0.800000,0.200000}%
\pgfsetfillcolor{currentfill}%
\pgfsetlinewidth{1.003750pt}%
\definecolor{currentstroke}{rgb}{0.800000,0.800000,0.200000}%
\pgfsetstrokecolor{currentstroke}%
\pgfsetdash{}{0pt}%
\pgfpathmoveto{\pgfqpoint{4.936492in}{3.762632in}}%
\pgfpathcurveto{\pgfqpoint{4.942316in}{3.762632in}}{\pgfqpoint{4.947902in}{3.764946in}}{\pgfqpoint{4.952020in}{3.769064in}}%
\pgfpathcurveto{\pgfqpoint{4.956139in}{3.773182in}}{\pgfqpoint{4.958452in}{3.778769in}}{\pgfqpoint{4.958452in}{3.784593in}}%
\pgfpathcurveto{\pgfqpoint{4.958452in}{3.790416in}}{\pgfqpoint{4.956139in}{3.796003in}}{\pgfqpoint{4.952020in}{3.800121in}}%
\pgfpathcurveto{\pgfqpoint{4.947902in}{3.804239in}}{\pgfqpoint{4.942316in}{3.806553in}}{\pgfqpoint{4.936492in}{3.806553in}}%
\pgfpathcurveto{\pgfqpoint{4.930668in}{3.806553in}}{\pgfqpoint{4.925082in}{3.804239in}}{\pgfqpoint{4.920964in}{3.800121in}}%
\pgfpathcurveto{\pgfqpoint{4.916846in}{3.796003in}}{\pgfqpoint{4.914532in}{3.790416in}}{\pgfqpoint{4.914532in}{3.784593in}}%
\pgfpathcurveto{\pgfqpoint{4.914532in}{3.778769in}}{\pgfqpoint{4.916846in}{3.773182in}}{\pgfqpoint{4.920964in}{3.769064in}}%
\pgfpathcurveto{\pgfqpoint{4.925082in}{3.764946in}}{\pgfqpoint{4.930668in}{3.762632in}}{\pgfqpoint{4.936492in}{3.762632in}}%
\pgfpathlineto{\pgfqpoint{4.936492in}{3.762632in}}%
\pgfpathclose%
\pgfusepath{stroke,fill}%
\end{pgfscope}%
\begin{pgfscope}%
\pgfpathrectangle{\pgfqpoint{0.997489in}{0.528000in}}{\pgfqpoint{4.565023in}{3.696000in}}%
\pgfusepath{clip}%
\pgfsetbuttcap%
\pgfsetroundjoin%
\definecolor{currentfill}{rgb}{0.800000,0.800000,0.200000}%
\pgfsetfillcolor{currentfill}%
\pgfsetlinewidth{1.003750pt}%
\definecolor{currentstroke}{rgb}{0.800000,0.800000,0.200000}%
\pgfsetstrokecolor{currentstroke}%
\pgfsetdash{}{0pt}%
\pgfpathmoveto{\pgfqpoint{4.944586in}{3.861729in}}%
\pgfpathcurveto{\pgfqpoint{4.950410in}{3.861729in}}{\pgfqpoint{4.955996in}{3.864043in}}{\pgfqpoint{4.960114in}{3.868161in}}%
\pgfpathcurveto{\pgfqpoint{4.964233in}{3.872280in}}{\pgfqpoint{4.966546in}{3.877866in}}{\pgfqpoint{4.966546in}{3.883690in}}%
\pgfpathcurveto{\pgfqpoint{4.966546in}{3.889514in}}{\pgfqpoint{4.964233in}{3.895100in}}{\pgfqpoint{4.960114in}{3.899218in}}%
\pgfpathcurveto{\pgfqpoint{4.955996in}{3.903336in}}{\pgfqpoint{4.950410in}{3.905650in}}{\pgfqpoint{4.944586in}{3.905650in}}%
\pgfpathcurveto{\pgfqpoint{4.938762in}{3.905650in}}{\pgfqpoint{4.933176in}{3.903336in}}{\pgfqpoint{4.929058in}{3.899218in}}%
\pgfpathcurveto{\pgfqpoint{4.924940in}{3.895100in}}{\pgfqpoint{4.922626in}{3.889514in}}{\pgfqpoint{4.922626in}{3.883690in}}%
\pgfpathcurveto{\pgfqpoint{4.922626in}{3.877866in}}{\pgfqpoint{4.924940in}{3.872280in}}{\pgfqpoint{4.929058in}{3.868161in}}%
\pgfpathcurveto{\pgfqpoint{4.933176in}{3.864043in}}{\pgfqpoint{4.938762in}{3.861729in}}{\pgfqpoint{4.944586in}{3.861729in}}%
\pgfpathlineto{\pgfqpoint{4.944586in}{3.861729in}}%
\pgfpathclose%
\pgfusepath{stroke,fill}%
\end{pgfscope}%
\begin{pgfscope}%
\pgfpathrectangle{\pgfqpoint{0.997489in}{0.528000in}}{\pgfqpoint{4.565023in}{3.696000in}}%
\pgfusepath{clip}%
\pgfsetbuttcap%
\pgfsetroundjoin%
\definecolor{currentfill}{rgb}{0.800000,0.800000,0.200000}%
\pgfsetfillcolor{currentfill}%
\pgfsetlinewidth{1.003750pt}%
\definecolor{currentstroke}{rgb}{0.800000,0.800000,0.200000}%
\pgfsetstrokecolor{currentstroke}%
\pgfsetdash{}{0pt}%
\pgfpathmoveto{\pgfqpoint{4.888259in}{3.877551in}}%
\pgfpathcurveto{\pgfqpoint{4.894083in}{3.877551in}}{\pgfqpoint{4.899669in}{3.879865in}}{\pgfqpoint{4.903787in}{3.883983in}}%
\pgfpathcurveto{\pgfqpoint{4.907906in}{3.888101in}}{\pgfqpoint{4.910219in}{3.893687in}}{\pgfqpoint{4.910219in}{3.899511in}}%
\pgfpathcurveto{\pgfqpoint{4.910219in}{3.905335in}}{\pgfqpoint{4.907906in}{3.910921in}}{\pgfqpoint{4.903787in}{3.915039in}}%
\pgfpathcurveto{\pgfqpoint{4.899669in}{3.919158in}}{\pgfqpoint{4.894083in}{3.921471in}}{\pgfqpoint{4.888259in}{3.921471in}}%
\pgfpathcurveto{\pgfqpoint{4.882435in}{3.921471in}}{\pgfqpoint{4.876849in}{3.919158in}}{\pgfqpoint{4.872731in}{3.915039in}}%
\pgfpathcurveto{\pgfqpoint{4.868613in}{3.910921in}}{\pgfqpoint{4.866299in}{3.905335in}}{\pgfqpoint{4.866299in}{3.899511in}}%
\pgfpathcurveto{\pgfqpoint{4.866299in}{3.893687in}}{\pgfqpoint{4.868613in}{3.888101in}}{\pgfqpoint{4.872731in}{3.883983in}}%
\pgfpathcurveto{\pgfqpoint{4.876849in}{3.879865in}}{\pgfqpoint{4.882435in}{3.877551in}}{\pgfqpoint{4.888259in}{3.877551in}}%
\pgfpathlineto{\pgfqpoint{4.888259in}{3.877551in}}%
\pgfpathclose%
\pgfusepath{stroke,fill}%
\end{pgfscope}%
\begin{pgfscope}%
\pgfpathrectangle{\pgfqpoint{0.997489in}{0.528000in}}{\pgfqpoint{4.565023in}{3.696000in}}%
\pgfusepath{clip}%
\pgfsetbuttcap%
\pgfsetroundjoin%
\definecolor{currentfill}{rgb}{0.800000,0.800000,0.200000}%
\pgfsetfillcolor{currentfill}%
\pgfsetlinewidth{1.003750pt}%
\definecolor{currentstroke}{rgb}{0.800000,0.800000,0.200000}%
\pgfsetstrokecolor{currentstroke}%
\pgfsetdash{}{0pt}%
\pgfpathmoveto{\pgfqpoint{4.834079in}{3.893013in}}%
\pgfpathcurveto{\pgfqpoint{4.839903in}{3.893013in}}{\pgfqpoint{4.845489in}{3.895327in}}{\pgfqpoint{4.849607in}{3.899445in}}%
\pgfpathcurveto{\pgfqpoint{4.853725in}{3.903563in}}{\pgfqpoint{4.856039in}{3.909149in}}{\pgfqpoint{4.856039in}{3.914973in}}%
\pgfpathcurveto{\pgfqpoint{4.856039in}{3.920797in}}{\pgfqpoint{4.853725in}{3.926383in}}{\pgfqpoint{4.849607in}{3.930502in}}%
\pgfpathcurveto{\pgfqpoint{4.845489in}{3.934620in}}{\pgfqpoint{4.839903in}{3.936934in}}{\pgfqpoint{4.834079in}{3.936934in}}%
\pgfpathcurveto{\pgfqpoint{4.828255in}{3.936934in}}{\pgfqpoint{4.822669in}{3.934620in}}{\pgfqpoint{4.818550in}{3.930502in}}%
\pgfpathcurveto{\pgfqpoint{4.814432in}{3.926383in}}{\pgfqpoint{4.812118in}{3.920797in}}{\pgfqpoint{4.812118in}{3.914973in}}%
\pgfpathcurveto{\pgfqpoint{4.812118in}{3.909149in}}{\pgfqpoint{4.814432in}{3.903563in}}{\pgfqpoint{4.818550in}{3.899445in}}%
\pgfpathcurveto{\pgfqpoint{4.822669in}{3.895327in}}{\pgfqpoint{4.828255in}{3.893013in}}{\pgfqpoint{4.834079in}{3.893013in}}%
\pgfpathlineto{\pgfqpoint{4.834079in}{3.893013in}}%
\pgfpathclose%
\pgfusepath{stroke,fill}%
\end{pgfscope}%
\begin{pgfscope}%
\pgfpathrectangle{\pgfqpoint{0.997489in}{0.528000in}}{\pgfqpoint{4.565023in}{3.696000in}}%
\pgfusepath{clip}%
\pgfsetbuttcap%
\pgfsetroundjoin%
\definecolor{currentfill}{rgb}{0.800000,0.800000,0.200000}%
\pgfsetfillcolor{currentfill}%
\pgfsetlinewidth{1.003750pt}%
\definecolor{currentstroke}{rgb}{0.800000,0.800000,0.200000}%
\pgfsetstrokecolor{currentstroke}%
\pgfsetdash{}{0pt}%
\pgfpathmoveto{\pgfqpoint{4.802820in}{3.955016in}}%
\pgfpathcurveto{\pgfqpoint{4.808644in}{3.955016in}}{\pgfqpoint{4.814231in}{3.957330in}}{\pgfqpoint{4.818349in}{3.961448in}}%
\pgfpathcurveto{\pgfqpoint{4.822467in}{3.965567in}}{\pgfqpoint{4.824781in}{3.971153in}}{\pgfqpoint{4.824781in}{3.976977in}}%
\pgfpathcurveto{\pgfqpoint{4.824781in}{3.982801in}}{\pgfqpoint{4.822467in}{3.988387in}}{\pgfqpoint{4.818349in}{3.992505in}}%
\pgfpathcurveto{\pgfqpoint{4.814231in}{3.996623in}}{\pgfqpoint{4.808644in}{3.998937in}}{\pgfqpoint{4.802820in}{3.998937in}}%
\pgfpathcurveto{\pgfqpoint{4.796997in}{3.998937in}}{\pgfqpoint{4.791410in}{3.996623in}}{\pgfqpoint{4.787292in}{3.992505in}}%
\pgfpathcurveto{\pgfqpoint{4.783174in}{3.988387in}}{\pgfqpoint{4.780860in}{3.982801in}}{\pgfqpoint{4.780860in}{3.976977in}}%
\pgfpathcurveto{\pgfqpoint{4.780860in}{3.971153in}}{\pgfqpoint{4.783174in}{3.965567in}}{\pgfqpoint{4.787292in}{3.961448in}}%
\pgfpathcurveto{\pgfqpoint{4.791410in}{3.957330in}}{\pgfqpoint{4.796997in}{3.955016in}}{\pgfqpoint{4.802820in}{3.955016in}}%
\pgfpathlineto{\pgfqpoint{4.802820in}{3.955016in}}%
\pgfpathclose%
\pgfusepath{stroke,fill}%
\end{pgfscope}%
\begin{pgfscope}%
\pgfpathrectangle{\pgfqpoint{0.997489in}{0.528000in}}{\pgfqpoint{4.565023in}{3.696000in}}%
\pgfusepath{clip}%
\pgfsetbuttcap%
\pgfsetroundjoin%
\definecolor{currentfill}{rgb}{0.800000,0.800000,0.200000}%
\pgfsetfillcolor{currentfill}%
\pgfsetlinewidth{1.003750pt}%
\definecolor{currentstroke}{rgb}{0.800000,0.800000,0.200000}%
\pgfsetstrokecolor{currentstroke}%
\pgfsetdash{}{0pt}%
\pgfpathmoveto{\pgfqpoint{4.735777in}{3.938783in}}%
\pgfpathcurveto{\pgfqpoint{4.741601in}{3.938783in}}{\pgfqpoint{4.747187in}{3.941097in}}{\pgfqpoint{4.751305in}{3.945215in}}%
\pgfpathcurveto{\pgfqpoint{4.755424in}{3.949334in}}{\pgfqpoint{4.757737in}{3.954920in}}{\pgfqpoint{4.757737in}{3.960744in}}%
\pgfpathcurveto{\pgfqpoint{4.757737in}{3.966568in}}{\pgfqpoint{4.755424in}{3.972154in}}{\pgfqpoint{4.751305in}{3.976272in}}%
\pgfpathcurveto{\pgfqpoint{4.747187in}{3.980390in}}{\pgfqpoint{4.741601in}{3.982704in}}{\pgfqpoint{4.735777in}{3.982704in}}%
\pgfpathcurveto{\pgfqpoint{4.729953in}{3.982704in}}{\pgfqpoint{4.724367in}{3.980390in}}{\pgfqpoint{4.720249in}{3.976272in}}%
\pgfpathcurveto{\pgfqpoint{4.716131in}{3.972154in}}{\pgfqpoint{4.713817in}{3.966568in}}{\pgfqpoint{4.713817in}{3.960744in}}%
\pgfpathcurveto{\pgfqpoint{4.713817in}{3.954920in}}{\pgfqpoint{4.716131in}{3.949334in}}{\pgfqpoint{4.720249in}{3.945215in}}%
\pgfpathcurveto{\pgfqpoint{4.724367in}{3.941097in}}{\pgfqpoint{4.729953in}{3.938783in}}{\pgfqpoint{4.735777in}{3.938783in}}%
\pgfpathlineto{\pgfqpoint{4.735777in}{3.938783in}}%
\pgfpathclose%
\pgfusepath{stroke,fill}%
\end{pgfscope}%
\begin{pgfscope}%
\pgfpathrectangle{\pgfqpoint{0.997489in}{0.528000in}}{\pgfqpoint{4.565023in}{3.696000in}}%
\pgfusepath{clip}%
\pgfsetbuttcap%
\pgfsetroundjoin%
\definecolor{currentfill}{rgb}{0.800000,0.800000,0.200000}%
\pgfsetfillcolor{currentfill}%
\pgfsetlinewidth{1.003750pt}%
\definecolor{currentstroke}{rgb}{0.800000,0.800000,0.200000}%
\pgfsetstrokecolor{currentstroke}%
\pgfsetdash{}{0pt}%
\pgfpathmoveto{\pgfqpoint{4.683197in}{3.952364in}}%
\pgfpathcurveto{\pgfqpoint{4.689021in}{3.952364in}}{\pgfqpoint{4.694607in}{3.954678in}}{\pgfqpoint{4.698725in}{3.958796in}}%
\pgfpathcurveto{\pgfqpoint{4.702843in}{3.962914in}}{\pgfqpoint{4.705157in}{3.968500in}}{\pgfqpoint{4.705157in}{3.974324in}}%
\pgfpathcurveto{\pgfqpoint{4.705157in}{3.980148in}}{\pgfqpoint{4.702843in}{3.985734in}}{\pgfqpoint{4.698725in}{3.989852in}}%
\pgfpathcurveto{\pgfqpoint{4.694607in}{3.993971in}}{\pgfqpoint{4.689021in}{3.996284in}}{\pgfqpoint{4.683197in}{3.996284in}}%
\pgfpathcurveto{\pgfqpoint{4.677373in}{3.996284in}}{\pgfqpoint{4.671787in}{3.993971in}}{\pgfqpoint{4.667669in}{3.989852in}}%
\pgfpathcurveto{\pgfqpoint{4.663551in}{3.985734in}}{\pgfqpoint{4.661237in}{3.980148in}}{\pgfqpoint{4.661237in}{3.974324in}}%
\pgfpathcurveto{\pgfqpoint{4.661237in}{3.968500in}}{\pgfqpoint{4.663551in}{3.962914in}}{\pgfqpoint{4.667669in}{3.958796in}}%
\pgfpathcurveto{\pgfqpoint{4.671787in}{3.954678in}}{\pgfqpoint{4.677373in}{3.952364in}}{\pgfqpoint{4.683197in}{3.952364in}}%
\pgfpathlineto{\pgfqpoint{4.683197in}{3.952364in}}%
\pgfpathclose%
\pgfusepath{stroke,fill}%
\end{pgfscope}%
\begin{pgfscope}%
\pgfpathrectangle{\pgfqpoint{0.997489in}{0.528000in}}{\pgfqpoint{4.565023in}{3.696000in}}%
\pgfusepath{clip}%
\pgfsetbuttcap%
\pgfsetroundjoin%
\definecolor{currentfill}{rgb}{0.800000,0.800000,0.200000}%
\pgfsetfillcolor{currentfill}%
\pgfsetlinewidth{1.003750pt}%
\definecolor{currentstroke}{rgb}{0.800000,0.800000,0.200000}%
\pgfsetstrokecolor{currentstroke}%
\pgfsetdash{}{0pt}%
\pgfpathmoveto{\pgfqpoint{4.627513in}{3.951729in}}%
\pgfpathcurveto{\pgfqpoint{4.633337in}{3.951729in}}{\pgfqpoint{4.638923in}{3.954043in}}{\pgfqpoint{4.643041in}{3.958161in}}%
\pgfpathcurveto{\pgfqpoint{4.647160in}{3.962279in}}{\pgfqpoint{4.649473in}{3.967865in}}{\pgfqpoint{4.649473in}{3.973689in}}%
\pgfpathcurveto{\pgfqpoint{4.649473in}{3.979513in}}{\pgfqpoint{4.647160in}{3.985099in}}{\pgfqpoint{4.643041in}{3.989218in}}%
\pgfpathcurveto{\pgfqpoint{4.638923in}{3.993336in}}{\pgfqpoint{4.633337in}{3.995650in}}{\pgfqpoint{4.627513in}{3.995650in}}%
\pgfpathcurveto{\pgfqpoint{4.621689in}{3.995650in}}{\pgfqpoint{4.616103in}{3.993336in}}{\pgfqpoint{4.611985in}{3.989218in}}%
\pgfpathcurveto{\pgfqpoint{4.607867in}{3.985099in}}{\pgfqpoint{4.605553in}{3.979513in}}{\pgfqpoint{4.605553in}{3.973689in}}%
\pgfpathcurveto{\pgfqpoint{4.605553in}{3.967865in}}{\pgfqpoint{4.607867in}{3.962279in}}{\pgfqpoint{4.611985in}{3.958161in}}%
\pgfpathcurveto{\pgfqpoint{4.616103in}{3.954043in}}{\pgfqpoint{4.621689in}{3.951729in}}{\pgfqpoint{4.627513in}{3.951729in}}%
\pgfpathlineto{\pgfqpoint{4.627513in}{3.951729in}}%
\pgfpathclose%
\pgfusepath{stroke,fill}%
\end{pgfscope}%
\begin{pgfscope}%
\pgfpathrectangle{\pgfqpoint{0.997489in}{0.528000in}}{\pgfqpoint{4.565023in}{3.696000in}}%
\pgfusepath{clip}%
\pgfsetbuttcap%
\pgfsetroundjoin%
\definecolor{currentfill}{rgb}{0.800000,0.800000,0.200000}%
\pgfsetfillcolor{currentfill}%
\pgfsetlinewidth{1.003750pt}%
\definecolor{currentstroke}{rgb}{0.800000,0.800000,0.200000}%
\pgfsetstrokecolor{currentstroke}%
\pgfsetdash{}{0pt}%
\pgfpathmoveto{\pgfqpoint{4.570269in}{3.932201in}}%
\pgfpathcurveto{\pgfqpoint{4.576092in}{3.932201in}}{\pgfqpoint{4.581679in}{3.934515in}}{\pgfqpoint{4.585797in}{3.938633in}}%
\pgfpathcurveto{\pgfqpoint{4.589915in}{3.942752in}}{\pgfqpoint{4.592229in}{3.948338in}}{\pgfqpoint{4.592229in}{3.954162in}}%
\pgfpathcurveto{\pgfqpoint{4.592229in}{3.959986in}}{\pgfqpoint{4.589915in}{3.965572in}}{\pgfqpoint{4.585797in}{3.969690in}}%
\pgfpathcurveto{\pgfqpoint{4.581679in}{3.973808in}}{\pgfqpoint{4.576092in}{3.976122in}}{\pgfqpoint{4.570269in}{3.976122in}}%
\pgfpathcurveto{\pgfqpoint{4.564445in}{3.976122in}}{\pgfqpoint{4.558858in}{3.973808in}}{\pgfqpoint{4.554740in}{3.969690in}}%
\pgfpathcurveto{\pgfqpoint{4.550622in}{3.965572in}}{\pgfqpoint{4.548308in}{3.959986in}}{\pgfqpoint{4.548308in}{3.954162in}}%
\pgfpathcurveto{\pgfqpoint{4.548308in}{3.948338in}}{\pgfqpoint{4.550622in}{3.942752in}}{\pgfqpoint{4.554740in}{3.938633in}}%
\pgfpathcurveto{\pgfqpoint{4.558858in}{3.934515in}}{\pgfqpoint{4.564445in}{3.932201in}}{\pgfqpoint{4.570269in}{3.932201in}}%
\pgfpathlineto{\pgfqpoint{4.570269in}{3.932201in}}%
\pgfpathclose%
\pgfusepath{stroke,fill}%
\end{pgfscope}%
\begin{pgfscope}%
\pgfpathrectangle{\pgfqpoint{0.997489in}{0.528000in}}{\pgfqpoint{4.565023in}{3.696000in}}%
\pgfusepath{clip}%
\pgfsetbuttcap%
\pgfsetroundjoin%
\definecolor{currentfill}{rgb}{0.800000,0.800000,0.200000}%
\pgfsetfillcolor{currentfill}%
\pgfsetlinewidth{1.003750pt}%
\definecolor{currentstroke}{rgb}{0.800000,0.800000,0.200000}%
\pgfsetstrokecolor{currentstroke}%
\pgfsetdash{}{0pt}%
\pgfpathmoveto{\pgfqpoint{4.519748in}{3.939580in}}%
\pgfpathcurveto{\pgfqpoint{4.525572in}{3.939580in}}{\pgfqpoint{4.531158in}{3.941894in}}{\pgfqpoint{4.535276in}{3.946012in}}%
\pgfpathcurveto{\pgfqpoint{4.539394in}{3.950131in}}{\pgfqpoint{4.541708in}{3.955717in}}{\pgfqpoint{4.541708in}{3.961541in}}%
\pgfpathcurveto{\pgfqpoint{4.541708in}{3.967365in}}{\pgfqpoint{4.539394in}{3.972951in}}{\pgfqpoint{4.535276in}{3.977069in}}%
\pgfpathcurveto{\pgfqpoint{4.531158in}{3.981187in}}{\pgfqpoint{4.525572in}{3.983501in}}{\pgfqpoint{4.519748in}{3.983501in}}%
\pgfpathcurveto{\pgfqpoint{4.513924in}{3.983501in}}{\pgfqpoint{4.508338in}{3.981187in}}{\pgfqpoint{4.504220in}{3.977069in}}%
\pgfpathcurveto{\pgfqpoint{4.500102in}{3.972951in}}{\pgfqpoint{4.497788in}{3.967365in}}{\pgfqpoint{4.497788in}{3.961541in}}%
\pgfpathcurveto{\pgfqpoint{4.497788in}{3.955717in}}{\pgfqpoint{4.500102in}{3.950131in}}{\pgfqpoint{4.504220in}{3.946012in}}%
\pgfpathcurveto{\pgfqpoint{4.508338in}{3.941894in}}{\pgfqpoint{4.513924in}{3.939580in}}{\pgfqpoint{4.519748in}{3.939580in}}%
\pgfpathlineto{\pgfqpoint{4.519748in}{3.939580in}}%
\pgfpathclose%
\pgfusepath{stroke,fill}%
\end{pgfscope}%
\begin{pgfscope}%
\pgfpathrectangle{\pgfqpoint{0.997489in}{0.528000in}}{\pgfqpoint{4.565023in}{3.696000in}}%
\pgfusepath{clip}%
\pgfsetbuttcap%
\pgfsetroundjoin%
\definecolor{currentfill}{rgb}{0.800000,0.800000,0.200000}%
\pgfsetfillcolor{currentfill}%
\pgfsetlinewidth{1.003750pt}%
\definecolor{currentstroke}{rgb}{0.800000,0.800000,0.200000}%
\pgfsetstrokecolor{currentstroke}%
\pgfsetdash{}{0pt}%
\pgfpathmoveto{\pgfqpoint{4.470969in}{3.988135in}}%
\pgfpathcurveto{\pgfqpoint{4.476792in}{3.988135in}}{\pgfqpoint{4.482379in}{3.990448in}}{\pgfqpoint{4.486497in}{3.994567in}}%
\pgfpathcurveto{\pgfqpoint{4.490615in}{3.998685in}}{\pgfqpoint{4.492929in}{4.004271in}}{\pgfqpoint{4.492929in}{4.010095in}}%
\pgfpathcurveto{\pgfqpoint{4.492929in}{4.015919in}}{\pgfqpoint{4.490615in}{4.021505in}}{\pgfqpoint{4.486497in}{4.025623in}}%
\pgfpathcurveto{\pgfqpoint{4.482379in}{4.029741in}}{\pgfqpoint{4.476792in}{4.032055in}}{\pgfqpoint{4.470969in}{4.032055in}}%
\pgfpathcurveto{\pgfqpoint{4.465145in}{4.032055in}}{\pgfqpoint{4.459558in}{4.029741in}}{\pgfqpoint{4.455440in}{4.025623in}}%
\pgfpathcurveto{\pgfqpoint{4.451322in}{4.021505in}}{\pgfqpoint{4.449008in}{4.015919in}}{\pgfqpoint{4.449008in}{4.010095in}}%
\pgfpathcurveto{\pgfqpoint{4.449008in}{4.004271in}}{\pgfqpoint{4.451322in}{3.998685in}}{\pgfqpoint{4.455440in}{3.994567in}}%
\pgfpathcurveto{\pgfqpoint{4.459558in}{3.990448in}}{\pgfqpoint{4.465145in}{3.988135in}}{\pgfqpoint{4.470969in}{3.988135in}}%
\pgfpathlineto{\pgfqpoint{4.470969in}{3.988135in}}%
\pgfpathclose%
\pgfusepath{stroke,fill}%
\end{pgfscope}%
\begin{pgfscope}%
\pgfpathrectangle{\pgfqpoint{0.997489in}{0.528000in}}{\pgfqpoint{4.565023in}{3.696000in}}%
\pgfusepath{clip}%
\pgfsetbuttcap%
\pgfsetroundjoin%
\definecolor{currentfill}{rgb}{0.800000,0.800000,0.200000}%
\pgfsetfillcolor{currentfill}%
\pgfsetlinewidth{1.003750pt}%
\definecolor{currentstroke}{rgb}{0.800000,0.800000,0.200000}%
\pgfsetstrokecolor{currentstroke}%
\pgfsetdash{}{0pt}%
\pgfpathmoveto{\pgfqpoint{4.416373in}{4.034040in}}%
\pgfpathcurveto{\pgfqpoint{4.422197in}{4.034040in}}{\pgfqpoint{4.427783in}{4.036354in}}{\pgfqpoint{4.431901in}{4.040472in}}%
\pgfpathcurveto{\pgfqpoint{4.436019in}{4.044590in}}{\pgfqpoint{4.438333in}{4.050176in}}{\pgfqpoint{4.438333in}{4.056000in}}%
\pgfpathcurveto{\pgfqpoint{4.438333in}{4.061824in}}{\pgfqpoint{4.436019in}{4.067410in}}{\pgfqpoint{4.431901in}{4.071528in}}%
\pgfpathcurveto{\pgfqpoint{4.427783in}{4.075646in}}{\pgfqpoint{4.422197in}{4.077960in}}{\pgfqpoint{4.416373in}{4.077960in}}%
\pgfpathcurveto{\pgfqpoint{4.410549in}{4.077960in}}{\pgfqpoint{4.404963in}{4.075646in}}{\pgfqpoint{4.400845in}{4.071528in}}%
\pgfpathcurveto{\pgfqpoint{4.396727in}{4.067410in}}{\pgfqpoint{4.394413in}{4.061824in}}{\pgfqpoint{4.394413in}{4.056000in}}%
\pgfpathcurveto{\pgfqpoint{4.394413in}{4.050176in}}{\pgfqpoint{4.396727in}{4.044590in}}{\pgfqpoint{4.400845in}{4.040472in}}%
\pgfpathcurveto{\pgfqpoint{4.404963in}{4.036354in}}{\pgfqpoint{4.410549in}{4.034040in}}{\pgfqpoint{4.416373in}{4.034040in}}%
\pgfpathlineto{\pgfqpoint{4.416373in}{4.034040in}}%
\pgfpathclose%
\pgfusepath{stroke,fill}%
\end{pgfscope}%
\begin{pgfscope}%
\pgfpathrectangle{\pgfqpoint{0.997489in}{0.528000in}}{\pgfqpoint{4.565023in}{3.696000in}}%
\pgfusepath{clip}%
\pgfsetbuttcap%
\pgfsetroundjoin%
\definecolor{currentfill}{rgb}{0.800000,0.800000,0.200000}%
\pgfsetfillcolor{currentfill}%
\pgfsetlinewidth{1.003750pt}%
\definecolor{currentstroke}{rgb}{0.800000,0.800000,0.200000}%
\pgfsetstrokecolor{currentstroke}%
\pgfsetdash{}{0pt}%
\pgfpathmoveto{\pgfqpoint{4.361526in}{4.008256in}}%
\pgfpathcurveto{\pgfqpoint{4.367350in}{4.008256in}}{\pgfqpoint{4.372936in}{4.010570in}}{\pgfqpoint{4.377054in}{4.014688in}}%
\pgfpathcurveto{\pgfqpoint{4.381172in}{4.018806in}}{\pgfqpoint{4.383486in}{4.024392in}}{\pgfqpoint{4.383486in}{4.030216in}}%
\pgfpathcurveto{\pgfqpoint{4.383486in}{4.036040in}}{\pgfqpoint{4.381172in}{4.041626in}}{\pgfqpoint{4.377054in}{4.045744in}}%
\pgfpathcurveto{\pgfqpoint{4.372936in}{4.049862in}}{\pgfqpoint{4.367350in}{4.052176in}}{\pgfqpoint{4.361526in}{4.052176in}}%
\pgfpathcurveto{\pgfqpoint{4.355702in}{4.052176in}}{\pgfqpoint{4.350116in}{4.049862in}}{\pgfqpoint{4.345997in}{4.045744in}}%
\pgfpathcurveto{\pgfqpoint{4.341879in}{4.041626in}}{\pgfqpoint{4.339565in}{4.036040in}}{\pgfqpoint{4.339565in}{4.030216in}}%
\pgfpathcurveto{\pgfqpoint{4.339565in}{4.024392in}}{\pgfqpoint{4.341879in}{4.018806in}}{\pgfqpoint{4.345997in}{4.014688in}}%
\pgfpathcurveto{\pgfqpoint{4.350116in}{4.010570in}}{\pgfqpoint{4.355702in}{4.008256in}}{\pgfqpoint{4.361526in}{4.008256in}}%
\pgfpathlineto{\pgfqpoint{4.361526in}{4.008256in}}%
\pgfpathclose%
\pgfusepath{stroke,fill}%
\end{pgfscope}%
\begin{pgfscope}%
\pgfpathrectangle{\pgfqpoint{0.997489in}{0.528000in}}{\pgfqpoint{4.565023in}{3.696000in}}%
\pgfusepath{clip}%
\pgfsetbuttcap%
\pgfsetroundjoin%
\definecolor{currentfill}{rgb}{0.800000,0.800000,0.200000}%
\pgfsetfillcolor{currentfill}%
\pgfsetlinewidth{1.003750pt}%
\definecolor{currentstroke}{rgb}{0.800000,0.800000,0.200000}%
\pgfsetstrokecolor{currentstroke}%
\pgfsetdash{}{0pt}%
\pgfpathmoveto{\pgfqpoint{4.306631in}{4.001835in}}%
\pgfpathcurveto{\pgfqpoint{4.312455in}{4.001835in}}{\pgfqpoint{4.318041in}{4.004149in}}{\pgfqpoint{4.322160in}{4.008267in}}%
\pgfpathcurveto{\pgfqpoint{4.326278in}{4.012385in}}{\pgfqpoint{4.328592in}{4.017971in}}{\pgfqpoint{4.328592in}{4.023795in}}%
\pgfpathcurveto{\pgfqpoint{4.328592in}{4.029619in}}{\pgfqpoint{4.326278in}{4.035205in}}{\pgfqpoint{4.322160in}{4.039323in}}%
\pgfpathcurveto{\pgfqpoint{4.318041in}{4.043441in}}{\pgfqpoint{4.312455in}{4.045755in}}{\pgfqpoint{4.306631in}{4.045755in}}%
\pgfpathcurveto{\pgfqpoint{4.300807in}{4.045755in}}{\pgfqpoint{4.295221in}{4.043441in}}{\pgfqpoint{4.291103in}{4.039323in}}%
\pgfpathcurveto{\pgfqpoint{4.286985in}{4.035205in}}{\pgfqpoint{4.284671in}{4.029619in}}{\pgfqpoint{4.284671in}{4.023795in}}%
\pgfpathcurveto{\pgfqpoint{4.284671in}{4.017971in}}{\pgfqpoint{4.286985in}{4.012385in}}{\pgfqpoint{4.291103in}{4.008267in}}%
\pgfpathcurveto{\pgfqpoint{4.295221in}{4.004149in}}{\pgfqpoint{4.300807in}{4.001835in}}{\pgfqpoint{4.306631in}{4.001835in}}%
\pgfpathlineto{\pgfqpoint{4.306631in}{4.001835in}}%
\pgfpathclose%
\pgfusepath{stroke,fill}%
\end{pgfscope}%
\begin{pgfscope}%
\pgfpathrectangle{\pgfqpoint{0.997489in}{0.528000in}}{\pgfqpoint{4.565023in}{3.696000in}}%
\pgfusepath{clip}%
\pgfsetbuttcap%
\pgfsetroundjoin%
\definecolor{currentfill}{rgb}{0.800000,0.800000,0.200000}%
\pgfsetfillcolor{currentfill}%
\pgfsetlinewidth{1.003750pt}%
\definecolor{currentstroke}{rgb}{0.800000,0.800000,0.200000}%
\pgfsetstrokecolor{currentstroke}%
\pgfsetdash{}{0pt}%
\pgfpathmoveto{\pgfqpoint{4.269918in}{3.907623in}}%
\pgfpathcurveto{\pgfqpoint{4.275742in}{3.907623in}}{\pgfqpoint{4.281328in}{3.909937in}}{\pgfqpoint{4.285446in}{3.914055in}}%
\pgfpathcurveto{\pgfqpoint{4.289564in}{3.918173in}}{\pgfqpoint{4.291878in}{3.923759in}}{\pgfqpoint{4.291878in}{3.929583in}}%
\pgfpathcurveto{\pgfqpoint{4.291878in}{3.935407in}}{\pgfqpoint{4.289564in}{3.940993in}}{\pgfqpoint{4.285446in}{3.945112in}}%
\pgfpathcurveto{\pgfqpoint{4.281328in}{3.949230in}}{\pgfqpoint{4.275742in}{3.951544in}}{\pgfqpoint{4.269918in}{3.951544in}}%
\pgfpathcurveto{\pgfqpoint{4.264094in}{3.951544in}}{\pgfqpoint{4.258508in}{3.949230in}}{\pgfqpoint{4.254389in}{3.945112in}}%
\pgfpathcurveto{\pgfqpoint{4.250271in}{3.940993in}}{\pgfqpoint{4.247957in}{3.935407in}}{\pgfqpoint{4.247957in}{3.929583in}}%
\pgfpathcurveto{\pgfqpoint{4.247957in}{3.923759in}}{\pgfqpoint{4.250271in}{3.918173in}}{\pgfqpoint{4.254389in}{3.914055in}}%
\pgfpathcurveto{\pgfqpoint{4.258508in}{3.909937in}}{\pgfqpoint{4.264094in}{3.907623in}}{\pgfqpoint{4.269918in}{3.907623in}}%
\pgfpathlineto{\pgfqpoint{4.269918in}{3.907623in}}%
\pgfpathclose%
\pgfusepath{stroke,fill}%
\end{pgfscope}%
\begin{pgfscope}%
\pgfpathrectangle{\pgfqpoint{0.997489in}{0.528000in}}{\pgfqpoint{4.565023in}{3.696000in}}%
\pgfusepath{clip}%
\pgfsetbuttcap%
\pgfsetroundjoin%
\definecolor{currentfill}{rgb}{0.800000,0.800000,0.200000}%
\pgfsetfillcolor{currentfill}%
\pgfsetlinewidth{1.003750pt}%
\definecolor{currentstroke}{rgb}{0.800000,0.800000,0.200000}%
\pgfsetstrokecolor{currentstroke}%
\pgfsetdash{}{0pt}%
\pgfpathmoveto{\pgfqpoint{4.199088in}{3.977128in}}%
\pgfpathcurveto{\pgfqpoint{4.204912in}{3.977128in}}{\pgfqpoint{4.210498in}{3.979442in}}{\pgfqpoint{4.214616in}{3.983560in}}%
\pgfpathcurveto{\pgfqpoint{4.218734in}{3.987678in}}{\pgfqpoint{4.221048in}{3.993264in}}{\pgfqpoint{4.221048in}{3.999088in}}%
\pgfpathcurveto{\pgfqpoint{4.221048in}{4.004912in}}{\pgfqpoint{4.218734in}{4.010498in}}{\pgfqpoint{4.214616in}{4.014616in}}%
\pgfpathcurveto{\pgfqpoint{4.210498in}{4.018734in}}{\pgfqpoint{4.204912in}{4.021048in}}{\pgfqpoint{4.199088in}{4.021048in}}%
\pgfpathcurveto{\pgfqpoint{4.193264in}{4.021048in}}{\pgfqpoint{4.187678in}{4.018734in}}{\pgfqpoint{4.183560in}{4.014616in}}%
\pgfpathcurveto{\pgfqpoint{4.179442in}{4.010498in}}{\pgfqpoint{4.177128in}{4.004912in}}{\pgfqpoint{4.177128in}{3.999088in}}%
\pgfpathcurveto{\pgfqpoint{4.177128in}{3.993264in}}{\pgfqpoint{4.179442in}{3.987678in}}{\pgfqpoint{4.183560in}{3.983560in}}%
\pgfpathcurveto{\pgfqpoint{4.187678in}{3.979442in}}{\pgfqpoint{4.193264in}{3.977128in}}{\pgfqpoint{4.199088in}{3.977128in}}%
\pgfpathlineto{\pgfqpoint{4.199088in}{3.977128in}}%
\pgfpathclose%
\pgfusepath{stroke,fill}%
\end{pgfscope}%
\begin{pgfscope}%
\pgfpathrectangle{\pgfqpoint{0.997489in}{0.528000in}}{\pgfqpoint{4.565023in}{3.696000in}}%
\pgfusepath{clip}%
\pgfsetbuttcap%
\pgfsetroundjoin%
\definecolor{currentfill}{rgb}{0.800000,0.800000,0.200000}%
\pgfsetfillcolor{currentfill}%
\pgfsetlinewidth{1.003750pt}%
\definecolor{currentstroke}{rgb}{0.800000,0.800000,0.200000}%
\pgfsetstrokecolor{currentstroke}%
\pgfsetdash{}{0pt}%
\pgfpathmoveto{\pgfqpoint{4.162056in}{3.915677in}}%
\pgfpathcurveto{\pgfqpoint{4.167880in}{3.915677in}}{\pgfqpoint{4.173466in}{3.917991in}}{\pgfqpoint{4.177585in}{3.922109in}}%
\pgfpathcurveto{\pgfqpoint{4.181703in}{3.926227in}}{\pgfqpoint{4.184017in}{3.931813in}}{\pgfqpoint{4.184017in}{3.937637in}}%
\pgfpathcurveto{\pgfqpoint{4.184017in}{3.943461in}}{\pgfqpoint{4.181703in}{3.949047in}}{\pgfqpoint{4.177585in}{3.953166in}}%
\pgfpathcurveto{\pgfqpoint{4.173466in}{3.957284in}}{\pgfqpoint{4.167880in}{3.959598in}}{\pgfqpoint{4.162056in}{3.959598in}}%
\pgfpathcurveto{\pgfqpoint{4.156232in}{3.959598in}}{\pgfqpoint{4.150646in}{3.957284in}}{\pgfqpoint{4.146528in}{3.953166in}}%
\pgfpathcurveto{\pgfqpoint{4.142410in}{3.949047in}}{\pgfqpoint{4.140096in}{3.943461in}}{\pgfqpoint{4.140096in}{3.937637in}}%
\pgfpathcurveto{\pgfqpoint{4.140096in}{3.931813in}}{\pgfqpoint{4.142410in}{3.926227in}}{\pgfqpoint{4.146528in}{3.922109in}}%
\pgfpathcurveto{\pgfqpoint{4.150646in}{3.917991in}}{\pgfqpoint{4.156232in}{3.915677in}}{\pgfqpoint{4.162056in}{3.915677in}}%
\pgfpathlineto{\pgfqpoint{4.162056in}{3.915677in}}%
\pgfpathclose%
\pgfusepath{stroke,fill}%
\end{pgfscope}%
\begin{pgfscope}%
\pgfpathrectangle{\pgfqpoint{0.997489in}{0.528000in}}{\pgfqpoint{4.565023in}{3.696000in}}%
\pgfusepath{clip}%
\pgfsetbuttcap%
\pgfsetroundjoin%
\definecolor{currentfill}{rgb}{0.800000,0.800000,0.200000}%
\pgfsetfillcolor{currentfill}%
\pgfsetlinewidth{1.003750pt}%
\definecolor{currentstroke}{rgb}{0.800000,0.800000,0.200000}%
\pgfsetstrokecolor{currentstroke}%
\pgfsetdash{}{0pt}%
\pgfpathmoveto{\pgfqpoint{4.101900in}{3.924524in}}%
\pgfpathcurveto{\pgfqpoint{4.107724in}{3.924524in}}{\pgfqpoint{4.113310in}{3.926838in}}{\pgfqpoint{4.117428in}{3.930956in}}%
\pgfpathcurveto{\pgfqpoint{4.121546in}{3.935074in}}{\pgfqpoint{4.123860in}{3.940661in}}{\pgfqpoint{4.123860in}{3.946485in}}%
\pgfpathcurveto{\pgfqpoint{4.123860in}{3.952308in}}{\pgfqpoint{4.121546in}{3.957895in}}{\pgfqpoint{4.117428in}{3.962013in}}%
\pgfpathcurveto{\pgfqpoint{4.113310in}{3.966131in}}{\pgfqpoint{4.107724in}{3.968445in}}{\pgfqpoint{4.101900in}{3.968445in}}%
\pgfpathcurveto{\pgfqpoint{4.096076in}{3.968445in}}{\pgfqpoint{4.090490in}{3.966131in}}{\pgfqpoint{4.086371in}{3.962013in}}%
\pgfpathcurveto{\pgfqpoint{4.082253in}{3.957895in}}{\pgfqpoint{4.079939in}{3.952308in}}{\pgfqpoint{4.079939in}{3.946485in}}%
\pgfpathcurveto{\pgfqpoint{4.079939in}{3.940661in}}{\pgfqpoint{4.082253in}{3.935074in}}{\pgfqpoint{4.086371in}{3.930956in}}%
\pgfpathcurveto{\pgfqpoint{4.090490in}{3.926838in}}{\pgfqpoint{4.096076in}{3.924524in}}{\pgfqpoint{4.101900in}{3.924524in}}%
\pgfpathlineto{\pgfqpoint{4.101900in}{3.924524in}}%
\pgfpathclose%
\pgfusepath{stroke,fill}%
\end{pgfscope}%
\begin{pgfscope}%
\pgfpathrectangle{\pgfqpoint{0.997489in}{0.528000in}}{\pgfqpoint{4.565023in}{3.696000in}}%
\pgfusepath{clip}%
\pgfsetbuttcap%
\pgfsetroundjoin%
\definecolor{currentfill}{rgb}{0.800000,0.800000,0.200000}%
\pgfsetfillcolor{currentfill}%
\pgfsetlinewidth{1.003750pt}%
\definecolor{currentstroke}{rgb}{0.800000,0.800000,0.200000}%
\pgfsetstrokecolor{currentstroke}%
\pgfsetdash{}{0pt}%
\pgfpathmoveto{\pgfqpoint{4.044151in}{3.919544in}}%
\pgfpathcurveto{\pgfqpoint{4.049975in}{3.919544in}}{\pgfqpoint{4.055561in}{3.921858in}}{\pgfqpoint{4.059679in}{3.925976in}}%
\pgfpathcurveto{\pgfqpoint{4.063797in}{3.930094in}}{\pgfqpoint{4.066111in}{3.935680in}}{\pgfqpoint{4.066111in}{3.941504in}}%
\pgfpathcurveto{\pgfqpoint{4.066111in}{3.947328in}}{\pgfqpoint{4.063797in}{3.952914in}}{\pgfqpoint{4.059679in}{3.957032in}}%
\pgfpathcurveto{\pgfqpoint{4.055561in}{3.961150in}}{\pgfqpoint{4.049975in}{3.963464in}}{\pgfqpoint{4.044151in}{3.963464in}}%
\pgfpathcurveto{\pgfqpoint{4.038327in}{3.963464in}}{\pgfqpoint{4.032741in}{3.961150in}}{\pgfqpoint{4.028622in}{3.957032in}}%
\pgfpathcurveto{\pgfqpoint{4.024504in}{3.952914in}}{\pgfqpoint{4.022190in}{3.947328in}}{\pgfqpoint{4.022190in}{3.941504in}}%
\pgfpathcurveto{\pgfqpoint{4.022190in}{3.935680in}}{\pgfqpoint{4.024504in}{3.930094in}}{\pgfqpoint{4.028622in}{3.925976in}}%
\pgfpathcurveto{\pgfqpoint{4.032741in}{3.921858in}}{\pgfqpoint{4.038327in}{3.919544in}}{\pgfqpoint{4.044151in}{3.919544in}}%
\pgfpathlineto{\pgfqpoint{4.044151in}{3.919544in}}%
\pgfpathclose%
\pgfusepath{stroke,fill}%
\end{pgfscope}%
\begin{pgfscope}%
\pgfpathrectangle{\pgfqpoint{0.997489in}{0.528000in}}{\pgfqpoint{4.565023in}{3.696000in}}%
\pgfusepath{clip}%
\pgfsetbuttcap%
\pgfsetroundjoin%
\definecolor{currentfill}{rgb}{0.800000,0.800000,0.200000}%
\pgfsetfillcolor{currentfill}%
\pgfsetlinewidth{1.003750pt}%
\definecolor{currentstroke}{rgb}{0.800000,0.800000,0.200000}%
\pgfsetstrokecolor{currentstroke}%
\pgfsetdash{}{0pt}%
\pgfpathmoveto{\pgfqpoint{4.039673in}{3.816922in}}%
\pgfpathcurveto{\pgfqpoint{4.045497in}{3.816922in}}{\pgfqpoint{4.051083in}{3.819236in}}{\pgfqpoint{4.055201in}{3.823354in}}%
\pgfpathcurveto{\pgfqpoint{4.059319in}{3.827472in}}{\pgfqpoint{4.061633in}{3.833058in}}{\pgfqpoint{4.061633in}{3.838882in}}%
\pgfpathcurveto{\pgfqpoint{4.061633in}{3.844706in}}{\pgfqpoint{4.059319in}{3.850292in}}{\pgfqpoint{4.055201in}{3.854410in}}%
\pgfpathcurveto{\pgfqpoint{4.051083in}{3.858528in}}{\pgfqpoint{4.045497in}{3.860842in}}{\pgfqpoint{4.039673in}{3.860842in}}%
\pgfpathcurveto{\pgfqpoint{4.033849in}{3.860842in}}{\pgfqpoint{4.028263in}{3.858528in}}{\pgfqpoint{4.024144in}{3.854410in}}%
\pgfpathcurveto{\pgfqpoint{4.020026in}{3.850292in}}{\pgfqpoint{4.017712in}{3.844706in}}{\pgfqpoint{4.017712in}{3.838882in}}%
\pgfpathcurveto{\pgfqpoint{4.017712in}{3.833058in}}{\pgfqpoint{4.020026in}{3.827472in}}{\pgfqpoint{4.024144in}{3.823354in}}%
\pgfpathcurveto{\pgfqpoint{4.028263in}{3.819236in}}{\pgfqpoint{4.033849in}{3.816922in}}{\pgfqpoint{4.039673in}{3.816922in}}%
\pgfpathlineto{\pgfqpoint{4.039673in}{3.816922in}}%
\pgfpathclose%
\pgfusepath{stroke,fill}%
\end{pgfscope}%
\begin{pgfscope}%
\pgfpathrectangle{\pgfqpoint{0.997489in}{0.528000in}}{\pgfqpoint{4.565023in}{3.696000in}}%
\pgfusepath{clip}%
\pgfsetbuttcap%
\pgfsetroundjoin%
\definecolor{currentfill}{rgb}{0.800000,0.800000,0.200000}%
\pgfsetfillcolor{currentfill}%
\pgfsetlinewidth{1.003750pt}%
\definecolor{currentstroke}{rgb}{0.800000,0.800000,0.200000}%
\pgfsetstrokecolor{currentstroke}%
\pgfsetdash{}{0pt}%
\pgfpathmoveto{\pgfqpoint{3.959870in}{3.847355in}}%
\pgfpathcurveto{\pgfqpoint{3.965694in}{3.847355in}}{\pgfqpoint{3.971280in}{3.849669in}}{\pgfqpoint{3.975398in}{3.853787in}}%
\pgfpathcurveto{\pgfqpoint{3.979517in}{3.857906in}}{\pgfqpoint{3.981830in}{3.863492in}}{\pgfqpoint{3.981830in}{3.869316in}}%
\pgfpathcurveto{\pgfqpoint{3.981830in}{3.875140in}}{\pgfqpoint{3.979517in}{3.880726in}}{\pgfqpoint{3.975398in}{3.884844in}}%
\pgfpathcurveto{\pgfqpoint{3.971280in}{3.888962in}}{\pgfqpoint{3.965694in}{3.891276in}}{\pgfqpoint{3.959870in}{3.891276in}}%
\pgfpathcurveto{\pgfqpoint{3.954046in}{3.891276in}}{\pgfqpoint{3.948460in}{3.888962in}}{\pgfqpoint{3.944342in}{3.884844in}}%
\pgfpathcurveto{\pgfqpoint{3.940224in}{3.880726in}}{\pgfqpoint{3.937910in}{3.875140in}}{\pgfqpoint{3.937910in}{3.869316in}}%
\pgfpathcurveto{\pgfqpoint{3.937910in}{3.863492in}}{\pgfqpoint{3.940224in}{3.857906in}}{\pgfqpoint{3.944342in}{3.853787in}}%
\pgfpathcurveto{\pgfqpoint{3.948460in}{3.849669in}}{\pgfqpoint{3.954046in}{3.847355in}}{\pgfqpoint{3.959870in}{3.847355in}}%
\pgfpathlineto{\pgfqpoint{3.959870in}{3.847355in}}%
\pgfpathclose%
\pgfusepath{stroke,fill}%
\end{pgfscope}%
\begin{pgfscope}%
\pgfpathrectangle{\pgfqpoint{0.997489in}{0.528000in}}{\pgfqpoint{4.565023in}{3.696000in}}%
\pgfusepath{clip}%
\pgfsetbuttcap%
\pgfsetroundjoin%
\definecolor{currentfill}{rgb}{0.800000,0.800000,0.200000}%
\pgfsetfillcolor{currentfill}%
\pgfsetlinewidth{1.003750pt}%
\definecolor{currentstroke}{rgb}{0.800000,0.800000,0.200000}%
\pgfsetstrokecolor{currentstroke}%
\pgfsetdash{}{0pt}%
\pgfpathmoveto{\pgfqpoint{3.913476in}{3.819325in}}%
\pgfpathcurveto{\pgfqpoint{3.919300in}{3.819325in}}{\pgfqpoint{3.924886in}{3.821639in}}{\pgfqpoint{3.929004in}{3.825757in}}%
\pgfpathcurveto{\pgfqpoint{3.933122in}{3.829875in}}{\pgfqpoint{3.935436in}{3.835461in}}{\pgfqpoint{3.935436in}{3.841285in}}%
\pgfpathcurveto{\pgfqpoint{3.935436in}{3.847109in}}{\pgfqpoint{3.933122in}{3.852695in}}{\pgfqpoint{3.929004in}{3.856813in}}%
\pgfpathcurveto{\pgfqpoint{3.924886in}{3.860932in}}{\pgfqpoint{3.919300in}{3.863245in}}{\pgfqpoint{3.913476in}{3.863245in}}%
\pgfpathcurveto{\pgfqpoint{3.907652in}{3.863245in}}{\pgfqpoint{3.902066in}{3.860932in}}{\pgfqpoint{3.897948in}{3.856813in}}%
\pgfpathcurveto{\pgfqpoint{3.893830in}{3.852695in}}{\pgfqpoint{3.891516in}{3.847109in}}{\pgfqpoint{3.891516in}{3.841285in}}%
\pgfpathcurveto{\pgfqpoint{3.891516in}{3.835461in}}{\pgfqpoint{3.893830in}{3.829875in}}{\pgfqpoint{3.897948in}{3.825757in}}%
\pgfpathcurveto{\pgfqpoint{3.902066in}{3.821639in}}{\pgfqpoint{3.907652in}{3.819325in}}{\pgfqpoint{3.913476in}{3.819325in}}%
\pgfpathlineto{\pgfqpoint{3.913476in}{3.819325in}}%
\pgfpathclose%
\pgfusepath{stroke,fill}%
\end{pgfscope}%
\begin{pgfscope}%
\pgfpathrectangle{\pgfqpoint{0.997489in}{0.528000in}}{\pgfqpoint{4.565023in}{3.696000in}}%
\pgfusepath{clip}%
\pgfsetbuttcap%
\pgfsetroundjoin%
\definecolor{currentfill}{rgb}{0.800000,0.800000,0.200000}%
\pgfsetfillcolor{currentfill}%
\pgfsetlinewidth{1.003750pt}%
\definecolor{currentstroke}{rgb}{0.800000,0.800000,0.200000}%
\pgfsetstrokecolor{currentstroke}%
\pgfsetdash{}{0pt}%
\pgfpathmoveto{\pgfqpoint{3.845592in}{3.814973in}}%
\pgfpathcurveto{\pgfqpoint{3.851416in}{3.814973in}}{\pgfqpoint{3.857002in}{3.817287in}}{\pgfqpoint{3.861121in}{3.821405in}}%
\pgfpathcurveto{\pgfqpoint{3.865239in}{3.825523in}}{\pgfqpoint{3.867553in}{3.831109in}}{\pgfqpoint{3.867553in}{3.836933in}}%
\pgfpathcurveto{\pgfqpoint{3.867553in}{3.842757in}}{\pgfqpoint{3.865239in}{3.848343in}}{\pgfqpoint{3.861121in}{3.852461in}}%
\pgfpathcurveto{\pgfqpoint{3.857002in}{3.856580in}}{\pgfqpoint{3.851416in}{3.858893in}}{\pgfqpoint{3.845592in}{3.858893in}}%
\pgfpathcurveto{\pgfqpoint{3.839768in}{3.858893in}}{\pgfqpoint{3.834182in}{3.856580in}}{\pgfqpoint{3.830064in}{3.852461in}}%
\pgfpathcurveto{\pgfqpoint{3.825946in}{3.848343in}}{\pgfqpoint{3.823632in}{3.842757in}}{\pgfqpoint{3.823632in}{3.836933in}}%
\pgfpathcurveto{\pgfqpoint{3.823632in}{3.831109in}}{\pgfqpoint{3.825946in}{3.825523in}}{\pgfqpoint{3.830064in}{3.821405in}}%
\pgfpathcurveto{\pgfqpoint{3.834182in}{3.817287in}}{\pgfqpoint{3.839768in}{3.814973in}}{\pgfqpoint{3.845592in}{3.814973in}}%
\pgfpathlineto{\pgfqpoint{3.845592in}{3.814973in}}%
\pgfpathclose%
\pgfusepath{stroke,fill}%
\end{pgfscope}%
\begin{pgfscope}%
\pgfpathrectangle{\pgfqpoint{0.997489in}{0.528000in}}{\pgfqpoint{4.565023in}{3.696000in}}%
\pgfusepath{clip}%
\pgfsetbuttcap%
\pgfsetroundjoin%
\definecolor{currentfill}{rgb}{0.800000,0.800000,0.200000}%
\pgfsetfillcolor{currentfill}%
\pgfsetlinewidth{1.003750pt}%
\definecolor{currentstroke}{rgb}{0.800000,0.800000,0.200000}%
\pgfsetstrokecolor{currentstroke}%
\pgfsetdash{}{0pt}%
\pgfpathmoveto{\pgfqpoint{3.858959in}{3.720627in}}%
\pgfpathcurveto{\pgfqpoint{3.864783in}{3.720627in}}{\pgfqpoint{3.870369in}{3.722941in}}{\pgfqpoint{3.874488in}{3.727059in}}%
\pgfpathcurveto{\pgfqpoint{3.878606in}{3.731177in}}{\pgfqpoint{3.880920in}{3.736763in}}{\pgfqpoint{3.880920in}{3.742587in}}%
\pgfpathcurveto{\pgfqpoint{3.880920in}{3.748411in}}{\pgfqpoint{3.878606in}{3.753997in}}{\pgfqpoint{3.874488in}{3.758115in}}%
\pgfpathcurveto{\pgfqpoint{3.870369in}{3.762233in}}{\pgfqpoint{3.864783in}{3.764547in}}{\pgfqpoint{3.858959in}{3.764547in}}%
\pgfpathcurveto{\pgfqpoint{3.853135in}{3.764547in}}{\pgfqpoint{3.847549in}{3.762233in}}{\pgfqpoint{3.843431in}{3.758115in}}%
\pgfpathcurveto{\pgfqpoint{3.839313in}{3.753997in}}{\pgfqpoint{3.836999in}{3.748411in}}{\pgfqpoint{3.836999in}{3.742587in}}%
\pgfpathcurveto{\pgfqpoint{3.836999in}{3.736763in}}{\pgfqpoint{3.839313in}{3.731177in}}{\pgfqpoint{3.843431in}{3.727059in}}%
\pgfpathcurveto{\pgfqpoint{3.847549in}{3.722941in}}{\pgfqpoint{3.853135in}{3.720627in}}{\pgfqpoint{3.858959in}{3.720627in}}%
\pgfpathlineto{\pgfqpoint{3.858959in}{3.720627in}}%
\pgfpathclose%
\pgfusepath{stroke,fill}%
\end{pgfscope}%
\begin{pgfscope}%
\pgfpathrectangle{\pgfqpoint{0.997489in}{0.528000in}}{\pgfqpoint{4.565023in}{3.696000in}}%
\pgfusepath{clip}%
\pgfsetbuttcap%
\pgfsetroundjoin%
\definecolor{currentfill}{rgb}{0.800000,0.800000,0.200000}%
\pgfsetfillcolor{currentfill}%
\pgfsetlinewidth{1.003750pt}%
\definecolor{currentstroke}{rgb}{0.800000,0.800000,0.200000}%
\pgfsetstrokecolor{currentstroke}%
\pgfsetdash{}{0pt}%
\pgfpathmoveto{\pgfqpoint{3.805706in}{3.698925in}}%
\pgfpathcurveto{\pgfqpoint{3.811530in}{3.698925in}}{\pgfqpoint{3.817116in}{3.701238in}}{\pgfqpoint{3.821234in}{3.705357in}}%
\pgfpathcurveto{\pgfqpoint{3.825353in}{3.709475in}}{\pgfqpoint{3.827666in}{3.715061in}}{\pgfqpoint{3.827666in}{3.720885in}}%
\pgfpathcurveto{\pgfqpoint{3.827666in}{3.726709in}}{\pgfqpoint{3.825353in}{3.732295in}}{\pgfqpoint{3.821234in}{3.736413in}}%
\pgfpathcurveto{\pgfqpoint{3.817116in}{3.740531in}}{\pgfqpoint{3.811530in}{3.742845in}}{\pgfqpoint{3.805706in}{3.742845in}}%
\pgfpathcurveto{\pgfqpoint{3.799882in}{3.742845in}}{\pgfqpoint{3.794296in}{3.740531in}}{\pgfqpoint{3.790178in}{3.736413in}}%
\pgfpathcurveto{\pgfqpoint{3.786060in}{3.732295in}}{\pgfqpoint{3.783746in}{3.726709in}}{\pgfqpoint{3.783746in}{3.720885in}}%
\pgfpathcurveto{\pgfqpoint{3.783746in}{3.715061in}}{\pgfqpoint{3.786060in}{3.709475in}}{\pgfqpoint{3.790178in}{3.705357in}}%
\pgfpathcurveto{\pgfqpoint{3.794296in}{3.701238in}}{\pgfqpoint{3.799882in}{3.698925in}}{\pgfqpoint{3.805706in}{3.698925in}}%
\pgfpathlineto{\pgfqpoint{3.805706in}{3.698925in}}%
\pgfpathclose%
\pgfusepath{stroke,fill}%
\end{pgfscope}%
\begin{pgfscope}%
\pgfpathrectangle{\pgfqpoint{0.997489in}{0.528000in}}{\pgfqpoint{4.565023in}{3.696000in}}%
\pgfusepath{clip}%
\pgfsetbuttcap%
\pgfsetroundjoin%
\definecolor{currentfill}{rgb}{0.800000,0.800000,0.200000}%
\pgfsetfillcolor{currentfill}%
\pgfsetlinewidth{1.003750pt}%
\definecolor{currentstroke}{rgb}{0.800000,0.800000,0.200000}%
\pgfsetstrokecolor{currentstroke}%
\pgfsetdash{}{0pt}%
\pgfpathmoveto{\pgfqpoint{3.762105in}{3.665562in}}%
\pgfpathcurveto{\pgfqpoint{3.767929in}{3.665562in}}{\pgfqpoint{3.773515in}{3.667876in}}{\pgfqpoint{3.777633in}{3.671994in}}%
\pgfpathcurveto{\pgfqpoint{3.781752in}{3.676113in}}{\pgfqpoint{3.784065in}{3.681699in}}{\pgfqpoint{3.784065in}{3.687523in}}%
\pgfpathcurveto{\pgfqpoint{3.784065in}{3.693347in}}{\pgfqpoint{3.781752in}{3.698933in}}{\pgfqpoint{3.777633in}{3.703051in}}%
\pgfpathcurveto{\pgfqpoint{3.773515in}{3.707169in}}{\pgfqpoint{3.767929in}{3.709483in}}{\pgfqpoint{3.762105in}{3.709483in}}%
\pgfpathcurveto{\pgfqpoint{3.756281in}{3.709483in}}{\pgfqpoint{3.750695in}{3.707169in}}{\pgfqpoint{3.746577in}{3.703051in}}%
\pgfpathcurveto{\pgfqpoint{3.742459in}{3.698933in}}{\pgfqpoint{3.740145in}{3.693347in}}{\pgfqpoint{3.740145in}{3.687523in}}%
\pgfpathcurveto{\pgfqpoint{3.740145in}{3.681699in}}{\pgfqpoint{3.742459in}{3.676113in}}{\pgfqpoint{3.746577in}{3.671994in}}%
\pgfpathcurveto{\pgfqpoint{3.750695in}{3.667876in}}{\pgfqpoint{3.756281in}{3.665562in}}{\pgfqpoint{3.762105in}{3.665562in}}%
\pgfpathlineto{\pgfqpoint{3.762105in}{3.665562in}}%
\pgfpathclose%
\pgfusepath{stroke,fill}%
\end{pgfscope}%
\begin{pgfscope}%
\pgfpathrectangle{\pgfqpoint{0.997489in}{0.528000in}}{\pgfqpoint{4.565023in}{3.696000in}}%
\pgfusepath{clip}%
\pgfsetbuttcap%
\pgfsetroundjoin%
\definecolor{currentfill}{rgb}{0.800000,0.800000,0.200000}%
\pgfsetfillcolor{currentfill}%
\pgfsetlinewidth{1.003750pt}%
\definecolor{currentstroke}{rgb}{0.800000,0.800000,0.200000}%
\pgfsetstrokecolor{currentstroke}%
\pgfsetdash{}{0pt}%
\pgfpathmoveto{\pgfqpoint{3.759379in}{3.601958in}}%
\pgfpathcurveto{\pgfqpoint{3.765202in}{3.601958in}}{\pgfqpoint{3.770789in}{3.604272in}}{\pgfqpoint{3.774907in}{3.608390in}}%
\pgfpathcurveto{\pgfqpoint{3.779025in}{3.612508in}}{\pgfqpoint{3.781339in}{3.618095in}}{\pgfqpoint{3.781339in}{3.623919in}}%
\pgfpathcurveto{\pgfqpoint{3.781339in}{3.629742in}}{\pgfqpoint{3.779025in}{3.635329in}}{\pgfqpoint{3.774907in}{3.639447in}}%
\pgfpathcurveto{\pgfqpoint{3.770789in}{3.643565in}}{\pgfqpoint{3.765202in}{3.645879in}}{\pgfqpoint{3.759379in}{3.645879in}}%
\pgfpathcurveto{\pgfqpoint{3.753555in}{3.645879in}}{\pgfqpoint{3.747968in}{3.643565in}}{\pgfqpoint{3.743850in}{3.639447in}}%
\pgfpathcurveto{\pgfqpoint{3.739732in}{3.635329in}}{\pgfqpoint{3.737418in}{3.629742in}}{\pgfqpoint{3.737418in}{3.623919in}}%
\pgfpathcurveto{\pgfqpoint{3.737418in}{3.618095in}}{\pgfqpoint{3.739732in}{3.612508in}}{\pgfqpoint{3.743850in}{3.608390in}}%
\pgfpathcurveto{\pgfqpoint{3.747968in}{3.604272in}}{\pgfqpoint{3.753555in}{3.601958in}}{\pgfqpoint{3.759379in}{3.601958in}}%
\pgfpathlineto{\pgfqpoint{3.759379in}{3.601958in}}%
\pgfpathclose%
\pgfusepath{stroke,fill}%
\end{pgfscope}%
\begin{pgfscope}%
\pgfpathrectangle{\pgfqpoint{0.997489in}{0.528000in}}{\pgfqpoint{4.565023in}{3.696000in}}%
\pgfusepath{clip}%
\pgfsetbuttcap%
\pgfsetroundjoin%
\definecolor{currentfill}{rgb}{0.800000,0.800000,0.200000}%
\pgfsetfillcolor{currentfill}%
\pgfsetlinewidth{1.003750pt}%
\definecolor{currentstroke}{rgb}{0.800000,0.800000,0.200000}%
\pgfsetstrokecolor{currentstroke}%
\pgfsetdash{}{0pt}%
\pgfpathmoveto{\pgfqpoint{3.732234in}{3.557974in}}%
\pgfpathcurveto{\pgfqpoint{3.738058in}{3.557974in}}{\pgfqpoint{3.743644in}{3.560287in}}{\pgfqpoint{3.747762in}{3.564406in}}%
\pgfpathcurveto{\pgfqpoint{3.751880in}{3.568524in}}{\pgfqpoint{3.754194in}{3.574110in}}{\pgfqpoint{3.754194in}{3.579934in}}%
\pgfpathcurveto{\pgfqpoint{3.754194in}{3.585758in}}{\pgfqpoint{3.751880in}{3.591344in}}{\pgfqpoint{3.747762in}{3.595462in}}%
\pgfpathcurveto{\pgfqpoint{3.743644in}{3.599580in}}{\pgfqpoint{3.738058in}{3.601894in}}{\pgfqpoint{3.732234in}{3.601894in}}%
\pgfpathcurveto{\pgfqpoint{3.726410in}{3.601894in}}{\pgfqpoint{3.720824in}{3.599580in}}{\pgfqpoint{3.716706in}{3.595462in}}%
\pgfpathcurveto{\pgfqpoint{3.712588in}{3.591344in}}{\pgfqpoint{3.710274in}{3.585758in}}{\pgfqpoint{3.710274in}{3.579934in}}%
\pgfpathcurveto{\pgfqpoint{3.710274in}{3.574110in}}{\pgfqpoint{3.712588in}{3.568524in}}{\pgfqpoint{3.716706in}{3.564406in}}%
\pgfpathcurveto{\pgfqpoint{3.720824in}{3.560287in}}{\pgfqpoint{3.726410in}{3.557974in}}{\pgfqpoint{3.732234in}{3.557974in}}%
\pgfpathlineto{\pgfqpoint{3.732234in}{3.557974in}}%
\pgfpathclose%
\pgfusepath{stroke,fill}%
\end{pgfscope}%
\begin{pgfscope}%
\pgfpathrectangle{\pgfqpoint{0.997489in}{0.528000in}}{\pgfqpoint{4.565023in}{3.696000in}}%
\pgfusepath{clip}%
\pgfsetbuttcap%
\pgfsetroundjoin%
\definecolor{currentfill}{rgb}{0.800000,0.800000,0.200000}%
\pgfsetfillcolor{currentfill}%
\pgfsetlinewidth{1.003750pt}%
\definecolor{currentstroke}{rgb}{0.800000,0.800000,0.200000}%
\pgfsetstrokecolor{currentstroke}%
\pgfsetdash{}{0pt}%
\pgfpathmoveto{\pgfqpoint{3.708898in}{3.511926in}}%
\pgfpathcurveto{\pgfqpoint{3.714722in}{3.511926in}}{\pgfqpoint{3.720308in}{3.514240in}}{\pgfqpoint{3.724426in}{3.518358in}}%
\pgfpathcurveto{\pgfqpoint{3.728544in}{3.522476in}}{\pgfqpoint{3.730858in}{3.528062in}}{\pgfqpoint{3.730858in}{3.533886in}}%
\pgfpathcurveto{\pgfqpoint{3.730858in}{3.539710in}}{\pgfqpoint{3.728544in}{3.545296in}}{\pgfqpoint{3.724426in}{3.549415in}}%
\pgfpathcurveto{\pgfqpoint{3.720308in}{3.553533in}}{\pgfqpoint{3.714722in}{3.555847in}}{\pgfqpoint{3.708898in}{3.555847in}}%
\pgfpathcurveto{\pgfqpoint{3.703074in}{3.555847in}}{\pgfqpoint{3.697488in}{3.553533in}}{\pgfqpoint{3.693370in}{3.549415in}}%
\pgfpathcurveto{\pgfqpoint{3.689251in}{3.545296in}}{\pgfqpoint{3.686938in}{3.539710in}}{\pgfqpoint{3.686938in}{3.533886in}}%
\pgfpathcurveto{\pgfqpoint{3.686938in}{3.528062in}}{\pgfqpoint{3.689251in}{3.522476in}}{\pgfqpoint{3.693370in}{3.518358in}}%
\pgfpathcurveto{\pgfqpoint{3.697488in}{3.514240in}}{\pgfqpoint{3.703074in}{3.511926in}}{\pgfqpoint{3.708898in}{3.511926in}}%
\pgfpathlineto{\pgfqpoint{3.708898in}{3.511926in}}%
\pgfpathclose%
\pgfusepath{stroke,fill}%
\end{pgfscope}%
\begin{pgfscope}%
\pgfpathrectangle{\pgfqpoint{0.997489in}{0.528000in}}{\pgfqpoint{4.565023in}{3.696000in}}%
\pgfusepath{clip}%
\pgfsetbuttcap%
\pgfsetroundjoin%
\definecolor{currentfill}{rgb}{0.800000,0.800000,0.200000}%
\pgfsetfillcolor{currentfill}%
\pgfsetlinewidth{1.003750pt}%
\definecolor{currentstroke}{rgb}{0.800000,0.800000,0.200000}%
\pgfsetstrokecolor{currentstroke}%
\pgfsetdash{}{0pt}%
\pgfpathmoveto{\pgfqpoint{3.665617in}{3.474657in}}%
\pgfpathcurveto{\pgfqpoint{3.671441in}{3.474657in}}{\pgfqpoint{3.677027in}{3.476971in}}{\pgfqpoint{3.681146in}{3.481089in}}%
\pgfpathcurveto{\pgfqpoint{3.685264in}{3.485207in}}{\pgfqpoint{3.687578in}{3.490793in}}{\pgfqpoint{3.687578in}{3.496617in}}%
\pgfpathcurveto{\pgfqpoint{3.687578in}{3.502441in}}{\pgfqpoint{3.685264in}{3.508027in}}{\pgfqpoint{3.681146in}{3.512145in}}%
\pgfpathcurveto{\pgfqpoint{3.677027in}{3.516263in}}{\pgfqpoint{3.671441in}{3.518577in}}{\pgfqpoint{3.665617in}{3.518577in}}%
\pgfpathcurveto{\pgfqpoint{3.659793in}{3.518577in}}{\pgfqpoint{3.654207in}{3.516263in}}{\pgfqpoint{3.650089in}{3.512145in}}%
\pgfpathcurveto{\pgfqpoint{3.645971in}{3.508027in}}{\pgfqpoint{3.643657in}{3.502441in}}{\pgfqpoint{3.643657in}{3.496617in}}%
\pgfpathcurveto{\pgfqpoint{3.643657in}{3.490793in}}{\pgfqpoint{3.645971in}{3.485207in}}{\pgfqpoint{3.650089in}{3.481089in}}%
\pgfpathcurveto{\pgfqpoint{3.654207in}{3.476971in}}{\pgfqpoint{3.659793in}{3.474657in}}{\pgfqpoint{3.665617in}{3.474657in}}%
\pgfpathlineto{\pgfqpoint{3.665617in}{3.474657in}}%
\pgfpathclose%
\pgfusepath{stroke,fill}%
\end{pgfscope}%
\begin{pgfscope}%
\pgfpathrectangle{\pgfqpoint{0.997489in}{0.528000in}}{\pgfqpoint{4.565023in}{3.696000in}}%
\pgfusepath{clip}%
\pgfsetbuttcap%
\pgfsetroundjoin%
\definecolor{currentfill}{rgb}{0.800000,0.800000,0.200000}%
\pgfsetfillcolor{currentfill}%
\pgfsetlinewidth{1.003750pt}%
\definecolor{currentstroke}{rgb}{0.800000,0.800000,0.200000}%
\pgfsetstrokecolor{currentstroke}%
\pgfsetdash{}{0pt}%
\pgfpathmoveto{\pgfqpoint{3.554769in}{3.458645in}}%
\pgfpathcurveto{\pgfqpoint{3.560593in}{3.458645in}}{\pgfqpoint{3.566179in}{3.460959in}}{\pgfqpoint{3.570297in}{3.465077in}}%
\pgfpathcurveto{\pgfqpoint{3.574415in}{3.469195in}}{\pgfqpoint{3.576729in}{3.474781in}}{\pgfqpoint{3.576729in}{3.480605in}}%
\pgfpathcurveto{\pgfqpoint{3.576729in}{3.486429in}}{\pgfqpoint{3.574415in}{3.492015in}}{\pgfqpoint{3.570297in}{3.496133in}}%
\pgfpathcurveto{\pgfqpoint{3.566179in}{3.500252in}}{\pgfqpoint{3.560593in}{3.502565in}}{\pgfqpoint{3.554769in}{3.502565in}}%
\pgfpathcurveto{\pgfqpoint{3.548945in}{3.502565in}}{\pgfqpoint{3.543359in}{3.500252in}}{\pgfqpoint{3.539241in}{3.496133in}}%
\pgfpathcurveto{\pgfqpoint{3.535123in}{3.492015in}}{\pgfqpoint{3.532809in}{3.486429in}}{\pgfqpoint{3.532809in}{3.480605in}}%
\pgfpathcurveto{\pgfqpoint{3.532809in}{3.474781in}}{\pgfqpoint{3.535123in}{3.469195in}}{\pgfqpoint{3.539241in}{3.465077in}}%
\pgfpathcurveto{\pgfqpoint{3.543359in}{3.460959in}}{\pgfqpoint{3.548945in}{3.458645in}}{\pgfqpoint{3.554769in}{3.458645in}}%
\pgfpathlineto{\pgfqpoint{3.554769in}{3.458645in}}%
\pgfpathclose%
\pgfusepath{stroke,fill}%
\end{pgfscope}%
\begin{pgfscope}%
\pgfpathrectangle{\pgfqpoint{0.997489in}{0.528000in}}{\pgfqpoint{4.565023in}{3.696000in}}%
\pgfusepath{clip}%
\pgfsetbuttcap%
\pgfsetroundjoin%
\definecolor{currentfill}{rgb}{0.800000,0.800000,0.200000}%
\pgfsetfillcolor{currentfill}%
\pgfsetlinewidth{1.003750pt}%
\definecolor{currentstroke}{rgb}{0.800000,0.800000,0.200000}%
\pgfsetstrokecolor{currentstroke}%
\pgfsetdash{}{0pt}%
\pgfpathmoveto{\pgfqpoint{3.627622in}{3.375647in}}%
\pgfpathcurveto{\pgfqpoint{3.633446in}{3.375647in}}{\pgfqpoint{3.639032in}{3.377961in}}{\pgfqpoint{3.643150in}{3.382079in}}%
\pgfpathcurveto{\pgfqpoint{3.647268in}{3.386197in}}{\pgfqpoint{3.649582in}{3.391783in}}{\pgfqpoint{3.649582in}{3.397607in}}%
\pgfpathcurveto{\pgfqpoint{3.649582in}{3.403431in}}{\pgfqpoint{3.647268in}{3.409017in}}{\pgfqpoint{3.643150in}{3.413135in}}%
\pgfpathcurveto{\pgfqpoint{3.639032in}{3.417254in}}{\pgfqpoint{3.633446in}{3.419568in}}{\pgfqpoint{3.627622in}{3.419568in}}%
\pgfpathcurveto{\pgfqpoint{3.621798in}{3.419568in}}{\pgfqpoint{3.616212in}{3.417254in}}{\pgfqpoint{3.612094in}{3.413135in}}%
\pgfpathcurveto{\pgfqpoint{3.607975in}{3.409017in}}{\pgfqpoint{3.605662in}{3.403431in}}{\pgfqpoint{3.605662in}{3.397607in}}%
\pgfpathcurveto{\pgfqpoint{3.605662in}{3.391783in}}{\pgfqpoint{3.607975in}{3.386197in}}{\pgfqpoint{3.612094in}{3.382079in}}%
\pgfpathcurveto{\pgfqpoint{3.616212in}{3.377961in}}{\pgfqpoint{3.621798in}{3.375647in}}{\pgfqpoint{3.627622in}{3.375647in}}%
\pgfpathlineto{\pgfqpoint{3.627622in}{3.375647in}}%
\pgfpathclose%
\pgfusepath{stroke,fill}%
\end{pgfscope}%
\begin{pgfscope}%
\pgfpathrectangle{\pgfqpoint{0.997489in}{0.528000in}}{\pgfqpoint{4.565023in}{3.696000in}}%
\pgfusepath{clip}%
\pgfsetbuttcap%
\pgfsetroundjoin%
\definecolor{currentfill}{rgb}{0.800000,0.800000,0.200000}%
\pgfsetfillcolor{currentfill}%
\pgfsetlinewidth{1.003750pt}%
\definecolor{currentstroke}{rgb}{0.800000,0.800000,0.200000}%
\pgfsetstrokecolor{currentstroke}%
\pgfsetdash{}{0pt}%
\pgfpathmoveto{\pgfqpoint{3.585032in}{3.330854in}}%
\pgfpathcurveto{\pgfqpoint{3.590856in}{3.330854in}}{\pgfqpoint{3.596442in}{3.333168in}}{\pgfqpoint{3.600560in}{3.337286in}}%
\pgfpathcurveto{\pgfqpoint{3.604678in}{3.341404in}}{\pgfqpoint{3.606992in}{3.346990in}}{\pgfqpoint{3.606992in}{3.352814in}}%
\pgfpathcurveto{\pgfqpoint{3.606992in}{3.358638in}}{\pgfqpoint{3.604678in}{3.364224in}}{\pgfqpoint{3.600560in}{3.368342in}}%
\pgfpathcurveto{\pgfqpoint{3.596442in}{3.372461in}}{\pgfqpoint{3.590856in}{3.374774in}}{\pgfqpoint{3.585032in}{3.374774in}}%
\pgfpathcurveto{\pgfqpoint{3.579208in}{3.374774in}}{\pgfqpoint{3.573622in}{3.372461in}}{\pgfqpoint{3.569504in}{3.368342in}}%
\pgfpathcurveto{\pgfqpoint{3.565386in}{3.364224in}}{\pgfqpoint{3.563072in}{3.358638in}}{\pgfqpoint{3.563072in}{3.352814in}}%
\pgfpathcurveto{\pgfqpoint{3.563072in}{3.346990in}}{\pgfqpoint{3.565386in}{3.341404in}}{\pgfqpoint{3.569504in}{3.337286in}}%
\pgfpathcurveto{\pgfqpoint{3.573622in}{3.333168in}}{\pgfqpoint{3.579208in}{3.330854in}}{\pgfqpoint{3.585032in}{3.330854in}}%
\pgfpathlineto{\pgfqpoint{3.585032in}{3.330854in}}%
\pgfpathclose%
\pgfusepath{stroke,fill}%
\end{pgfscope}%
\begin{pgfscope}%
\pgfpathrectangle{\pgfqpoint{0.997489in}{0.528000in}}{\pgfqpoint{4.565023in}{3.696000in}}%
\pgfusepath{clip}%
\pgfsetbuttcap%
\pgfsetroundjoin%
\definecolor{currentfill}{rgb}{0.800000,0.800000,0.200000}%
\pgfsetfillcolor{currentfill}%
\pgfsetlinewidth{1.003750pt}%
\definecolor{currentstroke}{rgb}{0.800000,0.800000,0.200000}%
\pgfsetstrokecolor{currentstroke}%
\pgfsetdash{}{0pt}%
\pgfpathmoveto{\pgfqpoint{3.597917in}{3.273115in}}%
\pgfpathcurveto{\pgfqpoint{3.603741in}{3.273115in}}{\pgfqpoint{3.609327in}{3.275429in}}{\pgfqpoint{3.613445in}{3.279547in}}%
\pgfpathcurveto{\pgfqpoint{3.617564in}{3.283665in}}{\pgfqpoint{3.619877in}{3.289251in}}{\pgfqpoint{3.619877in}{3.295075in}}%
\pgfpathcurveto{\pgfqpoint{3.619877in}{3.300899in}}{\pgfqpoint{3.617564in}{3.306486in}}{\pgfqpoint{3.613445in}{3.310604in}}%
\pgfpathcurveto{\pgfqpoint{3.609327in}{3.314722in}}{\pgfqpoint{3.603741in}{3.317036in}}{\pgfqpoint{3.597917in}{3.317036in}}%
\pgfpathcurveto{\pgfqpoint{3.592093in}{3.317036in}}{\pgfqpoint{3.586507in}{3.314722in}}{\pgfqpoint{3.582389in}{3.310604in}}%
\pgfpathcurveto{\pgfqpoint{3.578271in}{3.306486in}}{\pgfqpoint{3.575957in}{3.300899in}}{\pgfqpoint{3.575957in}{3.295075in}}%
\pgfpathcurveto{\pgfqpoint{3.575957in}{3.289251in}}{\pgfqpoint{3.578271in}{3.283665in}}{\pgfqpoint{3.582389in}{3.279547in}}%
\pgfpathcurveto{\pgfqpoint{3.586507in}{3.275429in}}{\pgfqpoint{3.592093in}{3.273115in}}{\pgfqpoint{3.597917in}{3.273115in}}%
\pgfpathlineto{\pgfqpoint{3.597917in}{3.273115in}}%
\pgfpathclose%
\pgfusepath{stroke,fill}%
\end{pgfscope}%
\begin{pgfscope}%
\pgfpathrectangle{\pgfqpoint{0.997489in}{0.528000in}}{\pgfqpoint{4.565023in}{3.696000in}}%
\pgfusepath{clip}%
\pgfsetbuttcap%
\pgfsetroundjoin%
\definecolor{currentfill}{rgb}{0.800000,0.800000,0.200000}%
\pgfsetfillcolor{currentfill}%
\pgfsetlinewidth{1.003750pt}%
\definecolor{currentstroke}{rgb}{0.800000,0.800000,0.200000}%
\pgfsetstrokecolor{currentstroke}%
\pgfsetdash{}{0pt}%
\pgfpathmoveto{\pgfqpoint{3.608611in}{3.218369in}}%
\pgfpathcurveto{\pgfqpoint{3.614435in}{3.218369in}}{\pgfqpoint{3.620021in}{3.220683in}}{\pgfqpoint{3.624139in}{3.224801in}}%
\pgfpathcurveto{\pgfqpoint{3.628257in}{3.228920in}}{\pgfqpoint{3.630571in}{3.234506in}}{\pgfqpoint{3.630571in}{3.240330in}}%
\pgfpathcurveto{\pgfqpoint{3.630571in}{3.246154in}}{\pgfqpoint{3.628257in}{3.251740in}}{\pgfqpoint{3.624139in}{3.255858in}}%
\pgfpathcurveto{\pgfqpoint{3.620021in}{3.259976in}}{\pgfqpoint{3.614435in}{3.262290in}}{\pgfqpoint{3.608611in}{3.262290in}}%
\pgfpathcurveto{\pgfqpoint{3.602787in}{3.262290in}}{\pgfqpoint{3.597201in}{3.259976in}}{\pgfqpoint{3.593082in}{3.255858in}}%
\pgfpathcurveto{\pgfqpoint{3.588964in}{3.251740in}}{\pgfqpoint{3.586650in}{3.246154in}}{\pgfqpoint{3.586650in}{3.240330in}}%
\pgfpathcurveto{\pgfqpoint{3.586650in}{3.234506in}}{\pgfqpoint{3.588964in}{3.228920in}}{\pgfqpoint{3.593082in}{3.224801in}}%
\pgfpathcurveto{\pgfqpoint{3.597201in}{3.220683in}}{\pgfqpoint{3.602787in}{3.218369in}}{\pgfqpoint{3.608611in}{3.218369in}}%
\pgfpathlineto{\pgfqpoint{3.608611in}{3.218369in}}%
\pgfpathclose%
\pgfusepath{stroke,fill}%
\end{pgfscope}%
\begin{pgfscope}%
\pgfpathrectangle{\pgfqpoint{0.997489in}{0.528000in}}{\pgfqpoint{4.565023in}{3.696000in}}%
\pgfusepath{clip}%
\pgfsetbuttcap%
\pgfsetroundjoin%
\definecolor{currentfill}{rgb}{0.800000,0.200000,0.200000}%
\pgfsetfillcolor{currentfill}%
\pgfsetlinewidth{1.003750pt}%
\definecolor{currentstroke}{rgb}{0.800000,0.200000,0.200000}%
\pgfsetstrokecolor{currentstroke}%
\pgfsetdash{}{0pt}%
\pgfpathmoveto{\pgfqpoint{3.590635in}{3.166545in}}%
\pgfpathcurveto{\pgfqpoint{3.596459in}{3.166545in}}{\pgfqpoint{3.602045in}{3.168859in}}{\pgfqpoint{3.606163in}{3.172977in}}%
\pgfpathcurveto{\pgfqpoint{3.610281in}{3.177095in}}{\pgfqpoint{3.612595in}{3.182682in}}{\pgfqpoint{3.612595in}{3.188505in}}%
\pgfpathcurveto{\pgfqpoint{3.612595in}{3.194329in}}{\pgfqpoint{3.610281in}{3.199916in}}{\pgfqpoint{3.606163in}{3.204034in}}%
\pgfpathcurveto{\pgfqpoint{3.602045in}{3.208152in}}{\pgfqpoint{3.596459in}{3.210466in}}{\pgfqpoint{3.590635in}{3.210466in}}%
\pgfpathcurveto{\pgfqpoint{3.584811in}{3.210466in}}{\pgfqpoint{3.579225in}{3.208152in}}{\pgfqpoint{3.575107in}{3.204034in}}%
\pgfpathcurveto{\pgfqpoint{3.570989in}{3.199916in}}{\pgfqpoint{3.568675in}{3.194329in}}{\pgfqpoint{3.568675in}{3.188505in}}%
\pgfpathcurveto{\pgfqpoint{3.568675in}{3.182682in}}{\pgfqpoint{3.570989in}{3.177095in}}{\pgfqpoint{3.575107in}{3.172977in}}%
\pgfpathcurveto{\pgfqpoint{3.579225in}{3.168859in}}{\pgfqpoint{3.584811in}{3.166545in}}{\pgfqpoint{3.590635in}{3.166545in}}%
\pgfpathlineto{\pgfqpoint{3.590635in}{3.166545in}}%
\pgfpathclose%
\pgfusepath{stroke,fill}%
\end{pgfscope}%
\begin{pgfscope}%
\pgfpathrectangle{\pgfqpoint{0.997489in}{0.528000in}}{\pgfqpoint{4.565023in}{3.696000in}}%
\pgfusepath{clip}%
\pgfsetbuttcap%
\pgfsetroundjoin%
\definecolor{currentfill}{rgb}{0.800000,0.800000,0.200000}%
\pgfsetfillcolor{currentfill}%
\pgfsetlinewidth{1.003750pt}%
\definecolor{currentstroke}{rgb}{0.800000,0.800000,0.200000}%
\pgfsetstrokecolor{currentstroke}%
\pgfsetdash{}{0pt}%
\pgfpathmoveto{\pgfqpoint{3.549596in}{3.111917in}}%
\pgfpathcurveto{\pgfqpoint{3.555420in}{3.111917in}}{\pgfqpoint{3.561007in}{3.114231in}}{\pgfqpoint{3.565125in}{3.118349in}}%
\pgfpathcurveto{\pgfqpoint{3.569243in}{3.122468in}}{\pgfqpoint{3.571557in}{3.128054in}}{\pgfqpoint{3.571557in}{3.133878in}}%
\pgfpathcurveto{\pgfqpoint{3.571557in}{3.139702in}}{\pgfqpoint{3.569243in}{3.145288in}}{\pgfqpoint{3.565125in}{3.149406in}}%
\pgfpathcurveto{\pgfqpoint{3.561007in}{3.153524in}}{\pgfqpoint{3.555420in}{3.155838in}}{\pgfqpoint{3.549596in}{3.155838in}}%
\pgfpathcurveto{\pgfqpoint{3.543773in}{3.155838in}}{\pgfqpoint{3.538186in}{3.153524in}}{\pgfqpoint{3.534068in}{3.149406in}}%
\pgfpathcurveto{\pgfqpoint{3.529950in}{3.145288in}}{\pgfqpoint{3.527636in}{3.139702in}}{\pgfqpoint{3.527636in}{3.133878in}}%
\pgfpathcurveto{\pgfqpoint{3.527636in}{3.128054in}}{\pgfqpoint{3.529950in}{3.122468in}}{\pgfqpoint{3.534068in}{3.118349in}}%
\pgfpathcurveto{\pgfqpoint{3.538186in}{3.114231in}}{\pgfqpoint{3.543773in}{3.111917in}}{\pgfqpoint{3.549596in}{3.111917in}}%
\pgfpathlineto{\pgfqpoint{3.549596in}{3.111917in}}%
\pgfpathclose%
\pgfusepath{stroke,fill}%
\end{pgfscope}%
\begin{pgfscope}%
\pgfpathrectangle{\pgfqpoint{0.997489in}{0.528000in}}{\pgfqpoint{4.565023in}{3.696000in}}%
\pgfusepath{clip}%
\pgfsetbuttcap%
\pgfsetroundjoin%
\definecolor{currentfill}{rgb}{0.800000,0.800000,0.200000}%
\pgfsetfillcolor{currentfill}%
\pgfsetlinewidth{1.003750pt}%
\definecolor{currentstroke}{rgb}{0.800000,0.800000,0.200000}%
\pgfsetstrokecolor{currentstroke}%
\pgfsetdash{}{0pt}%
\pgfpathmoveto{\pgfqpoint{3.595525in}{3.060146in}}%
\pgfpathcurveto{\pgfqpoint{3.601349in}{3.060146in}}{\pgfqpoint{3.606935in}{3.062460in}}{\pgfqpoint{3.611053in}{3.066578in}}%
\pgfpathcurveto{\pgfqpoint{3.615171in}{3.070696in}}{\pgfqpoint{3.617485in}{3.076283in}}{\pgfqpoint{3.617485in}{3.082107in}}%
\pgfpathcurveto{\pgfqpoint{3.617485in}{3.087931in}}{\pgfqpoint{3.615171in}{3.093517in}}{\pgfqpoint{3.611053in}{3.097635in}}%
\pgfpathcurveto{\pgfqpoint{3.606935in}{3.101753in}}{\pgfqpoint{3.601349in}{3.104067in}}{\pgfqpoint{3.595525in}{3.104067in}}%
\pgfpathcurveto{\pgfqpoint{3.589701in}{3.104067in}}{\pgfqpoint{3.584115in}{3.101753in}}{\pgfqpoint{3.579996in}{3.097635in}}%
\pgfpathcurveto{\pgfqpoint{3.575878in}{3.093517in}}{\pgfqpoint{3.573564in}{3.087931in}}{\pgfqpoint{3.573564in}{3.082107in}}%
\pgfpathcurveto{\pgfqpoint{3.573564in}{3.076283in}}{\pgfqpoint{3.575878in}{3.070696in}}{\pgfqpoint{3.579996in}{3.066578in}}%
\pgfpathcurveto{\pgfqpoint{3.584115in}{3.062460in}}{\pgfqpoint{3.589701in}{3.060146in}}{\pgfqpoint{3.595525in}{3.060146in}}%
\pgfpathlineto{\pgfqpoint{3.595525in}{3.060146in}}%
\pgfpathclose%
\pgfusepath{stroke,fill}%
\end{pgfscope}%
\begin{pgfscope}%
\pgfpathrectangle{\pgfqpoint{0.997489in}{0.528000in}}{\pgfqpoint{4.565023in}{3.696000in}}%
\pgfusepath{clip}%
\pgfsetbuttcap%
\pgfsetroundjoin%
\definecolor{currentfill}{rgb}{0.800000,0.800000,0.200000}%
\pgfsetfillcolor{currentfill}%
\pgfsetlinewidth{1.003750pt}%
\definecolor{currentstroke}{rgb}{0.800000,0.800000,0.200000}%
\pgfsetstrokecolor{currentstroke}%
\pgfsetdash{}{0pt}%
\pgfpathmoveto{\pgfqpoint{3.607754in}{3.008224in}}%
\pgfpathcurveto{\pgfqpoint{3.613578in}{3.008224in}}{\pgfqpoint{3.619164in}{3.010538in}}{\pgfqpoint{3.623282in}{3.014656in}}%
\pgfpathcurveto{\pgfqpoint{3.627400in}{3.018774in}}{\pgfqpoint{3.629714in}{3.024360in}}{\pgfqpoint{3.629714in}{3.030184in}}%
\pgfpathcurveto{\pgfqpoint{3.629714in}{3.036008in}}{\pgfqpoint{3.627400in}{3.041595in}}{\pgfqpoint{3.623282in}{3.045713in}}%
\pgfpathcurveto{\pgfqpoint{3.619164in}{3.049831in}}{\pgfqpoint{3.613578in}{3.052145in}}{\pgfqpoint{3.607754in}{3.052145in}}%
\pgfpathcurveto{\pgfqpoint{3.601930in}{3.052145in}}{\pgfqpoint{3.596344in}{3.049831in}}{\pgfqpoint{3.592226in}{3.045713in}}%
\pgfpathcurveto{\pgfqpoint{3.588107in}{3.041595in}}{\pgfqpoint{3.585794in}{3.036008in}}{\pgfqpoint{3.585794in}{3.030184in}}%
\pgfpathcurveto{\pgfqpoint{3.585794in}{3.024360in}}{\pgfqpoint{3.588107in}{3.018774in}}{\pgfqpoint{3.592226in}{3.014656in}}%
\pgfpathcurveto{\pgfqpoint{3.596344in}{3.010538in}}{\pgfqpoint{3.601930in}{3.008224in}}{\pgfqpoint{3.607754in}{3.008224in}}%
\pgfpathlineto{\pgfqpoint{3.607754in}{3.008224in}}%
\pgfpathclose%
\pgfusepath{stroke,fill}%
\end{pgfscope}%
\begin{pgfscope}%
\pgfpathrectangle{\pgfqpoint{0.997489in}{0.528000in}}{\pgfqpoint{4.565023in}{3.696000in}}%
\pgfusepath{clip}%
\pgfsetbuttcap%
\pgfsetroundjoin%
\definecolor{currentfill}{rgb}{0.800000,0.800000,0.200000}%
\pgfsetfillcolor{currentfill}%
\pgfsetlinewidth{1.003750pt}%
\definecolor{currentstroke}{rgb}{0.800000,0.800000,0.200000}%
\pgfsetstrokecolor{currentstroke}%
\pgfsetdash{}{0pt}%
\pgfpathmoveto{\pgfqpoint{3.621487in}{2.957145in}}%
\pgfpathcurveto{\pgfqpoint{3.627311in}{2.957145in}}{\pgfqpoint{3.632897in}{2.959459in}}{\pgfqpoint{3.637016in}{2.963577in}}%
\pgfpathcurveto{\pgfqpoint{3.641134in}{2.967695in}}{\pgfqpoint{3.643448in}{2.973282in}}{\pgfqpoint{3.643448in}{2.979105in}}%
\pgfpathcurveto{\pgfqpoint{3.643448in}{2.984929in}}{\pgfqpoint{3.641134in}{2.990516in}}{\pgfqpoint{3.637016in}{2.994634in}}%
\pgfpathcurveto{\pgfqpoint{3.632897in}{2.998752in}}{\pgfqpoint{3.627311in}{3.001066in}}{\pgfqpoint{3.621487in}{3.001066in}}%
\pgfpathcurveto{\pgfqpoint{3.615663in}{3.001066in}}{\pgfqpoint{3.610077in}{2.998752in}}{\pgfqpoint{3.605959in}{2.994634in}}%
\pgfpathcurveto{\pgfqpoint{3.601841in}{2.990516in}}{\pgfqpoint{3.599527in}{2.984929in}}{\pgfqpoint{3.599527in}{2.979105in}}%
\pgfpathcurveto{\pgfqpoint{3.599527in}{2.973282in}}{\pgfqpoint{3.601841in}{2.967695in}}{\pgfqpoint{3.605959in}{2.963577in}}%
\pgfpathcurveto{\pgfqpoint{3.610077in}{2.959459in}}{\pgfqpoint{3.615663in}{2.957145in}}{\pgfqpoint{3.621487in}{2.957145in}}%
\pgfpathlineto{\pgfqpoint{3.621487in}{2.957145in}}%
\pgfpathclose%
\pgfusepath{stroke,fill}%
\end{pgfscope}%
\begin{pgfscope}%
\pgfpathrectangle{\pgfqpoint{0.997489in}{0.528000in}}{\pgfqpoint{4.565023in}{3.696000in}}%
\pgfusepath{clip}%
\pgfsetbuttcap%
\pgfsetroundjoin%
\definecolor{currentfill}{rgb}{0.200000,0.200000,0.800000}%
\pgfsetfillcolor{currentfill}%
\pgfsetlinewidth{1.003750pt}%
\definecolor{currentstroke}{rgb}{0.200000,0.200000,0.800000}%
\pgfsetstrokecolor{currentstroke}%
\pgfsetdash{}{0pt}%
\pgfpathmoveto{\pgfqpoint{3.672373in}{2.917258in}}%
\pgfpathcurveto{\pgfqpoint{3.678197in}{2.917258in}}{\pgfqpoint{3.683783in}{2.919572in}}{\pgfqpoint{3.687901in}{2.923690in}}%
\pgfpathcurveto{\pgfqpoint{3.692019in}{2.927809in}}{\pgfqpoint{3.694333in}{2.933395in}}{\pgfqpoint{3.694333in}{2.939219in}}%
\pgfpathcurveto{\pgfqpoint{3.694333in}{2.945043in}}{\pgfqpoint{3.692019in}{2.950629in}}{\pgfqpoint{3.687901in}{2.954747in}}%
\pgfpathcurveto{\pgfqpoint{3.683783in}{2.958865in}}{\pgfqpoint{3.678197in}{2.961179in}}{\pgfqpoint{3.672373in}{2.961179in}}%
\pgfpathcurveto{\pgfqpoint{3.666549in}{2.961179in}}{\pgfqpoint{3.660963in}{2.958865in}}{\pgfqpoint{3.656845in}{2.954747in}}%
\pgfpathcurveto{\pgfqpoint{3.652726in}{2.950629in}}{\pgfqpoint{3.650413in}{2.945043in}}{\pgfqpoint{3.650413in}{2.939219in}}%
\pgfpathcurveto{\pgfqpoint{3.650413in}{2.933395in}}{\pgfqpoint{3.652726in}{2.927809in}}{\pgfqpoint{3.656845in}{2.923690in}}%
\pgfpathcurveto{\pgfqpoint{3.660963in}{2.919572in}}{\pgfqpoint{3.666549in}{2.917258in}}{\pgfqpoint{3.672373in}{2.917258in}}%
\pgfpathlineto{\pgfqpoint{3.672373in}{2.917258in}}%
\pgfpathclose%
\pgfusepath{stroke,fill}%
\end{pgfscope}%
\begin{pgfscope}%
\pgfpathrectangle{\pgfqpoint{0.997489in}{0.528000in}}{\pgfqpoint{4.565023in}{3.696000in}}%
\pgfusepath{clip}%
\pgfsetbuttcap%
\pgfsetroundjoin%
\definecolor{currentfill}{rgb}{0.800000,0.800000,0.200000}%
\pgfsetfillcolor{currentfill}%
\pgfsetlinewidth{1.003750pt}%
\definecolor{currentstroke}{rgb}{0.800000,0.800000,0.200000}%
\pgfsetstrokecolor{currentstroke}%
\pgfsetdash{}{0pt}%
\pgfpathmoveto{\pgfqpoint{3.589144in}{2.833632in}}%
\pgfpathcurveto{\pgfqpoint{3.594968in}{2.833632in}}{\pgfqpoint{3.600554in}{2.835946in}}{\pgfqpoint{3.604672in}{2.840064in}}%
\pgfpathcurveto{\pgfqpoint{3.608791in}{2.844182in}}{\pgfqpoint{3.611104in}{2.849768in}}{\pgfqpoint{3.611104in}{2.855592in}}%
\pgfpathcurveto{\pgfqpoint{3.611104in}{2.861416in}}{\pgfqpoint{3.608791in}{2.867002in}}{\pgfqpoint{3.604672in}{2.871120in}}%
\pgfpathcurveto{\pgfqpoint{3.600554in}{2.875239in}}{\pgfqpoint{3.594968in}{2.877552in}}{\pgfqpoint{3.589144in}{2.877552in}}%
\pgfpathcurveto{\pgfqpoint{3.583320in}{2.877552in}}{\pgfqpoint{3.577734in}{2.875239in}}{\pgfqpoint{3.573616in}{2.871120in}}%
\pgfpathcurveto{\pgfqpoint{3.569498in}{2.867002in}}{\pgfqpoint{3.567184in}{2.861416in}}{\pgfqpoint{3.567184in}{2.855592in}}%
\pgfpathcurveto{\pgfqpoint{3.567184in}{2.849768in}}{\pgfqpoint{3.569498in}{2.844182in}}{\pgfqpoint{3.573616in}{2.840064in}}%
\pgfpathcurveto{\pgfqpoint{3.577734in}{2.835946in}}{\pgfqpoint{3.583320in}{2.833632in}}{\pgfqpoint{3.589144in}{2.833632in}}%
\pgfpathlineto{\pgfqpoint{3.589144in}{2.833632in}}%
\pgfpathclose%
\pgfusepath{stroke,fill}%
\end{pgfscope}%
\begin{pgfscope}%
\pgfpathrectangle{\pgfqpoint{0.997489in}{0.528000in}}{\pgfqpoint{4.565023in}{3.696000in}}%
\pgfusepath{clip}%
\pgfsetbuttcap%
\pgfsetroundjoin%
\definecolor{currentfill}{rgb}{0.200000,0.200000,0.800000}%
\pgfsetfillcolor{currentfill}%
\pgfsetlinewidth{1.003750pt}%
\definecolor{currentstroke}{rgb}{0.200000,0.200000,0.800000}%
\pgfsetstrokecolor{currentstroke}%
\pgfsetdash{}{0pt}%
\pgfpathmoveto{\pgfqpoint{3.680748in}{2.811731in}}%
\pgfpathcurveto{\pgfqpoint{3.686572in}{2.811731in}}{\pgfqpoint{3.692158in}{2.814044in}}{\pgfqpoint{3.696277in}{2.818163in}}%
\pgfpathcurveto{\pgfqpoint{3.700395in}{2.822281in}}{\pgfqpoint{3.702709in}{2.827867in}}{\pgfqpoint{3.702709in}{2.833691in}}%
\pgfpathcurveto{\pgfqpoint{3.702709in}{2.839515in}}{\pgfqpoint{3.700395in}{2.845101in}}{\pgfqpoint{3.696277in}{2.849219in}}%
\pgfpathcurveto{\pgfqpoint{3.692158in}{2.853337in}}{\pgfqpoint{3.686572in}{2.855651in}}{\pgfqpoint{3.680748in}{2.855651in}}%
\pgfpathcurveto{\pgfqpoint{3.674924in}{2.855651in}}{\pgfqpoint{3.669338in}{2.853337in}}{\pgfqpoint{3.665220in}{2.849219in}}%
\pgfpathcurveto{\pgfqpoint{3.661102in}{2.845101in}}{\pgfqpoint{3.658788in}{2.839515in}}{\pgfqpoint{3.658788in}{2.833691in}}%
\pgfpathcurveto{\pgfqpoint{3.658788in}{2.827867in}}{\pgfqpoint{3.661102in}{2.822281in}}{\pgfqpoint{3.665220in}{2.818163in}}%
\pgfpathcurveto{\pgfqpoint{3.669338in}{2.814044in}}{\pgfqpoint{3.674924in}{2.811731in}}{\pgfqpoint{3.680748in}{2.811731in}}%
\pgfpathlineto{\pgfqpoint{3.680748in}{2.811731in}}%
\pgfpathclose%
\pgfusepath{stroke,fill}%
\end{pgfscope}%
\begin{pgfscope}%
\pgfpathrectangle{\pgfqpoint{0.997489in}{0.528000in}}{\pgfqpoint{4.565023in}{3.696000in}}%
\pgfusepath{clip}%
\pgfsetbuttcap%
\pgfsetroundjoin%
\definecolor{currentfill}{rgb}{0.800000,0.800000,0.200000}%
\pgfsetfillcolor{currentfill}%
\pgfsetlinewidth{1.003750pt}%
\definecolor{currentstroke}{rgb}{0.800000,0.800000,0.200000}%
\pgfsetstrokecolor{currentstroke}%
\pgfsetdash{}{0pt}%
\pgfpathmoveto{\pgfqpoint{3.614938in}{2.719400in}}%
\pgfpathcurveto{\pgfqpoint{3.620762in}{2.719400in}}{\pgfqpoint{3.626348in}{2.721714in}}{\pgfqpoint{3.630467in}{2.725832in}}%
\pgfpathcurveto{\pgfqpoint{3.634585in}{2.729950in}}{\pgfqpoint{3.636899in}{2.735536in}}{\pgfqpoint{3.636899in}{2.741360in}}%
\pgfpathcurveto{\pgfqpoint{3.636899in}{2.747184in}}{\pgfqpoint{3.634585in}{2.752770in}}{\pgfqpoint{3.630467in}{2.756888in}}%
\pgfpathcurveto{\pgfqpoint{3.626348in}{2.761006in}}{\pgfqpoint{3.620762in}{2.763320in}}{\pgfqpoint{3.614938in}{2.763320in}}%
\pgfpathcurveto{\pgfqpoint{3.609114in}{2.763320in}}{\pgfqpoint{3.603528in}{2.761006in}}{\pgfqpoint{3.599410in}{2.756888in}}%
\pgfpathcurveto{\pgfqpoint{3.595292in}{2.752770in}}{\pgfqpoint{3.592978in}{2.747184in}}{\pgfqpoint{3.592978in}{2.741360in}}%
\pgfpathcurveto{\pgfqpoint{3.592978in}{2.735536in}}{\pgfqpoint{3.595292in}{2.729950in}}{\pgfqpoint{3.599410in}{2.725832in}}%
\pgfpathcurveto{\pgfqpoint{3.603528in}{2.721714in}}{\pgfqpoint{3.609114in}{2.719400in}}{\pgfqpoint{3.614938in}{2.719400in}}%
\pgfpathlineto{\pgfqpoint{3.614938in}{2.719400in}}%
\pgfpathclose%
\pgfusepath{stroke,fill}%
\end{pgfscope}%
\begin{pgfscope}%
\pgfpathrectangle{\pgfqpoint{0.997489in}{0.528000in}}{\pgfqpoint{4.565023in}{3.696000in}}%
\pgfusepath{clip}%
\pgfsetbuttcap%
\pgfsetroundjoin%
\definecolor{currentfill}{rgb}{0.200000,0.200000,0.800000}%
\pgfsetfillcolor{currentfill}%
\pgfsetlinewidth{1.003750pt}%
\definecolor{currentstroke}{rgb}{0.200000,0.200000,0.800000}%
\pgfsetstrokecolor{currentstroke}%
\pgfsetdash{}{0pt}%
\pgfpathmoveto{\pgfqpoint{3.697618in}{2.701067in}}%
\pgfpathcurveto{\pgfqpoint{3.703442in}{2.701067in}}{\pgfqpoint{3.709028in}{2.703381in}}{\pgfqpoint{3.713146in}{2.707499in}}%
\pgfpathcurveto{\pgfqpoint{3.717264in}{2.711617in}}{\pgfqpoint{3.719578in}{2.717203in}}{\pgfqpoint{3.719578in}{2.723027in}}%
\pgfpathcurveto{\pgfqpoint{3.719578in}{2.728851in}}{\pgfqpoint{3.717264in}{2.734437in}}{\pgfqpoint{3.713146in}{2.738556in}}%
\pgfpathcurveto{\pgfqpoint{3.709028in}{2.742674in}}{\pgfqpoint{3.703442in}{2.744988in}}{\pgfqpoint{3.697618in}{2.744988in}}%
\pgfpathcurveto{\pgfqpoint{3.691794in}{2.744988in}}{\pgfqpoint{3.686208in}{2.742674in}}{\pgfqpoint{3.682090in}{2.738556in}}%
\pgfpathcurveto{\pgfqpoint{3.677972in}{2.734437in}}{\pgfqpoint{3.675658in}{2.728851in}}{\pgfqpoint{3.675658in}{2.723027in}}%
\pgfpathcurveto{\pgfqpoint{3.675658in}{2.717203in}}{\pgfqpoint{3.677972in}{2.711617in}}{\pgfqpoint{3.682090in}{2.707499in}}%
\pgfpathcurveto{\pgfqpoint{3.686208in}{2.703381in}}{\pgfqpoint{3.691794in}{2.701067in}}{\pgfqpoint{3.697618in}{2.701067in}}%
\pgfpathlineto{\pgfqpoint{3.697618in}{2.701067in}}%
\pgfpathclose%
\pgfusepath{stroke,fill}%
\end{pgfscope}%
\begin{pgfscope}%
\pgfpathrectangle{\pgfqpoint{0.997489in}{0.528000in}}{\pgfqpoint{4.565023in}{3.696000in}}%
\pgfusepath{clip}%
\pgfsetbuttcap%
\pgfsetroundjoin%
\definecolor{currentfill}{rgb}{0.800000,0.800000,0.200000}%
\pgfsetfillcolor{currentfill}%
\pgfsetlinewidth{1.003750pt}%
\definecolor{currentstroke}{rgb}{0.800000,0.800000,0.200000}%
\pgfsetstrokecolor{currentstroke}%
\pgfsetdash{}{0pt}%
\pgfpathmoveto{\pgfqpoint{3.677437in}{2.621395in}}%
\pgfpathcurveto{\pgfqpoint{3.683261in}{2.621395in}}{\pgfqpoint{3.688847in}{2.623709in}}{\pgfqpoint{3.692966in}{2.627827in}}%
\pgfpathcurveto{\pgfqpoint{3.697084in}{2.631945in}}{\pgfqpoint{3.699398in}{2.637531in}}{\pgfqpoint{3.699398in}{2.643355in}}%
\pgfpathcurveto{\pgfqpoint{3.699398in}{2.649179in}}{\pgfqpoint{3.697084in}{2.654765in}}{\pgfqpoint{3.692966in}{2.658883in}}%
\pgfpathcurveto{\pgfqpoint{3.688847in}{2.663001in}}{\pgfqpoint{3.683261in}{2.665315in}}{\pgfqpoint{3.677437in}{2.665315in}}%
\pgfpathcurveto{\pgfqpoint{3.671613in}{2.665315in}}{\pgfqpoint{3.666027in}{2.663001in}}{\pgfqpoint{3.661909in}{2.658883in}}%
\pgfpathcurveto{\pgfqpoint{3.657791in}{2.654765in}}{\pgfqpoint{3.655477in}{2.649179in}}{\pgfqpoint{3.655477in}{2.643355in}}%
\pgfpathcurveto{\pgfqpoint{3.655477in}{2.637531in}}{\pgfqpoint{3.657791in}{2.631945in}}{\pgfqpoint{3.661909in}{2.627827in}}%
\pgfpathcurveto{\pgfqpoint{3.666027in}{2.623709in}}{\pgfqpoint{3.671613in}{2.621395in}}{\pgfqpoint{3.677437in}{2.621395in}}%
\pgfpathlineto{\pgfqpoint{3.677437in}{2.621395in}}%
\pgfpathclose%
\pgfusepath{stroke,fill}%
\end{pgfscope}%
\begin{pgfscope}%
\pgfpathrectangle{\pgfqpoint{0.997489in}{0.528000in}}{\pgfqpoint{4.565023in}{3.696000in}}%
\pgfusepath{clip}%
\pgfsetbuttcap%
\pgfsetroundjoin%
\definecolor{currentfill}{rgb}{0.800000,0.800000,0.200000}%
\pgfsetfillcolor{currentfill}%
\pgfsetlinewidth{1.003750pt}%
\definecolor{currentstroke}{rgb}{0.800000,0.800000,0.200000}%
\pgfsetstrokecolor{currentstroke}%
\pgfsetdash{}{0pt}%
\pgfpathmoveto{\pgfqpoint{3.777150in}{2.626034in}}%
\pgfpathcurveto{\pgfqpoint{3.782974in}{2.626034in}}{\pgfqpoint{3.788560in}{2.628348in}}{\pgfqpoint{3.792678in}{2.632466in}}%
\pgfpathcurveto{\pgfqpoint{3.796796in}{2.636584in}}{\pgfqpoint{3.799110in}{2.642171in}}{\pgfqpoint{3.799110in}{2.647995in}}%
\pgfpathcurveto{\pgfqpoint{3.799110in}{2.653818in}}{\pgfqpoint{3.796796in}{2.659405in}}{\pgfqpoint{3.792678in}{2.663523in}}%
\pgfpathcurveto{\pgfqpoint{3.788560in}{2.667641in}}{\pgfqpoint{3.782974in}{2.669955in}}{\pgfqpoint{3.777150in}{2.669955in}}%
\pgfpathcurveto{\pgfqpoint{3.771326in}{2.669955in}}{\pgfqpoint{3.765740in}{2.667641in}}{\pgfqpoint{3.761622in}{2.663523in}}%
\pgfpathcurveto{\pgfqpoint{3.757504in}{2.659405in}}{\pgfqpoint{3.755190in}{2.653818in}}{\pgfqpoint{3.755190in}{2.647995in}}%
\pgfpathcurveto{\pgfqpoint{3.755190in}{2.642171in}}{\pgfqpoint{3.757504in}{2.636584in}}{\pgfqpoint{3.761622in}{2.632466in}}%
\pgfpathcurveto{\pgfqpoint{3.765740in}{2.628348in}}{\pgfqpoint{3.771326in}{2.626034in}}{\pgfqpoint{3.777150in}{2.626034in}}%
\pgfpathlineto{\pgfqpoint{3.777150in}{2.626034in}}%
\pgfpathclose%
\pgfusepath{stroke,fill}%
\end{pgfscope}%
\begin{pgfscope}%
\pgfpathrectangle{\pgfqpoint{0.997489in}{0.528000in}}{\pgfqpoint{4.565023in}{3.696000in}}%
\pgfusepath{clip}%
\pgfsetbuttcap%
\pgfsetroundjoin%
\definecolor{currentfill}{rgb}{0.800000,0.800000,0.200000}%
\pgfsetfillcolor{currentfill}%
\pgfsetlinewidth{1.003750pt}%
\definecolor{currentstroke}{rgb}{0.800000,0.800000,0.200000}%
\pgfsetstrokecolor{currentstroke}%
\pgfsetdash{}{0pt}%
\pgfpathmoveto{\pgfqpoint{3.831618in}{2.604023in}}%
\pgfpathcurveto{\pgfqpoint{3.837442in}{2.604023in}}{\pgfqpoint{3.843028in}{2.606337in}}{\pgfqpoint{3.847146in}{2.610455in}}%
\pgfpathcurveto{\pgfqpoint{3.851264in}{2.614573in}}{\pgfqpoint{3.853578in}{2.620160in}}{\pgfqpoint{3.853578in}{2.625984in}}%
\pgfpathcurveto{\pgfqpoint{3.853578in}{2.631808in}}{\pgfqpoint{3.851264in}{2.637394in}}{\pgfqpoint{3.847146in}{2.641512in}}%
\pgfpathcurveto{\pgfqpoint{3.843028in}{2.645630in}}{\pgfqpoint{3.837442in}{2.647944in}}{\pgfqpoint{3.831618in}{2.647944in}}%
\pgfpathcurveto{\pgfqpoint{3.825794in}{2.647944in}}{\pgfqpoint{3.820208in}{2.645630in}}{\pgfqpoint{3.816090in}{2.641512in}}%
\pgfpathcurveto{\pgfqpoint{3.811972in}{2.637394in}}{\pgfqpoint{3.809658in}{2.631808in}}{\pgfqpoint{3.809658in}{2.625984in}}%
\pgfpathcurveto{\pgfqpoint{3.809658in}{2.620160in}}{\pgfqpoint{3.811972in}{2.614573in}}{\pgfqpoint{3.816090in}{2.610455in}}%
\pgfpathcurveto{\pgfqpoint{3.820208in}{2.606337in}}{\pgfqpoint{3.825794in}{2.604023in}}{\pgfqpoint{3.831618in}{2.604023in}}%
\pgfpathlineto{\pgfqpoint{3.831618in}{2.604023in}}%
\pgfpathclose%
\pgfusepath{stroke,fill}%
\end{pgfscope}%
\begin{pgfscope}%
\pgfpathrectangle{\pgfqpoint{0.997489in}{0.528000in}}{\pgfqpoint{4.565023in}{3.696000in}}%
\pgfusepath{clip}%
\pgfsetbuttcap%
\pgfsetroundjoin%
\definecolor{currentfill}{rgb}{0.800000,0.800000,0.200000}%
\pgfsetfillcolor{currentfill}%
\pgfsetlinewidth{1.003750pt}%
\definecolor{currentstroke}{rgb}{0.800000,0.800000,0.200000}%
\pgfsetstrokecolor{currentstroke}%
\pgfsetdash{}{0pt}%
\pgfpathmoveto{\pgfqpoint{3.802335in}{2.501608in}}%
\pgfpathcurveto{\pgfqpoint{3.808159in}{2.501608in}}{\pgfqpoint{3.813745in}{2.503922in}}{\pgfqpoint{3.817863in}{2.508040in}}%
\pgfpathcurveto{\pgfqpoint{3.821981in}{2.512158in}}{\pgfqpoint{3.824295in}{2.517745in}}{\pgfqpoint{3.824295in}{2.523568in}}%
\pgfpathcurveto{\pgfqpoint{3.824295in}{2.529392in}}{\pgfqpoint{3.821981in}{2.534979in}}{\pgfqpoint{3.817863in}{2.539097in}}%
\pgfpathcurveto{\pgfqpoint{3.813745in}{2.543215in}}{\pgfqpoint{3.808159in}{2.545529in}}{\pgfqpoint{3.802335in}{2.545529in}}%
\pgfpathcurveto{\pgfqpoint{3.796511in}{2.545529in}}{\pgfqpoint{3.790925in}{2.543215in}}{\pgfqpoint{3.786806in}{2.539097in}}%
\pgfpathcurveto{\pgfqpoint{3.782688in}{2.534979in}}{\pgfqpoint{3.780374in}{2.529392in}}{\pgfqpoint{3.780374in}{2.523568in}}%
\pgfpathcurveto{\pgfqpoint{3.780374in}{2.517745in}}{\pgfqpoint{3.782688in}{2.512158in}}{\pgfqpoint{3.786806in}{2.508040in}}%
\pgfpathcurveto{\pgfqpoint{3.790925in}{2.503922in}}{\pgfqpoint{3.796511in}{2.501608in}}{\pgfqpoint{3.802335in}{2.501608in}}%
\pgfpathlineto{\pgfqpoint{3.802335in}{2.501608in}}%
\pgfpathclose%
\pgfusepath{stroke,fill}%
\end{pgfscope}%
\begin{pgfscope}%
\pgfpathrectangle{\pgfqpoint{0.997489in}{0.528000in}}{\pgfqpoint{4.565023in}{3.696000in}}%
\pgfusepath{clip}%
\pgfsetbuttcap%
\pgfsetroundjoin%
\definecolor{currentfill}{rgb}{0.800000,0.800000,0.200000}%
\pgfsetfillcolor{currentfill}%
\pgfsetlinewidth{1.003750pt}%
\definecolor{currentstroke}{rgb}{0.800000,0.800000,0.200000}%
\pgfsetstrokecolor{currentstroke}%
\pgfsetdash{}{0pt}%
\pgfpathmoveto{\pgfqpoint{3.898439in}{2.525780in}}%
\pgfpathcurveto{\pgfqpoint{3.904263in}{2.525780in}}{\pgfqpoint{3.909849in}{2.528094in}}{\pgfqpoint{3.913967in}{2.532212in}}%
\pgfpathcurveto{\pgfqpoint{3.918085in}{2.536331in}}{\pgfqpoint{3.920399in}{2.541917in}}{\pgfqpoint{3.920399in}{2.547741in}}%
\pgfpathcurveto{\pgfqpoint{3.920399in}{2.553565in}}{\pgfqpoint{3.918085in}{2.559151in}}{\pgfqpoint{3.913967in}{2.563269in}}%
\pgfpathcurveto{\pgfqpoint{3.909849in}{2.567387in}}{\pgfqpoint{3.904263in}{2.569701in}}{\pgfqpoint{3.898439in}{2.569701in}}%
\pgfpathcurveto{\pgfqpoint{3.892615in}{2.569701in}}{\pgfqpoint{3.887029in}{2.567387in}}{\pgfqpoint{3.882911in}{2.563269in}}%
\pgfpathcurveto{\pgfqpoint{3.878792in}{2.559151in}}{\pgfqpoint{3.876479in}{2.553565in}}{\pgfqpoint{3.876479in}{2.547741in}}%
\pgfpathcurveto{\pgfqpoint{3.876479in}{2.541917in}}{\pgfqpoint{3.878792in}{2.536331in}}{\pgfqpoint{3.882911in}{2.532212in}}%
\pgfpathcurveto{\pgfqpoint{3.887029in}{2.528094in}}{\pgfqpoint{3.892615in}{2.525780in}}{\pgfqpoint{3.898439in}{2.525780in}}%
\pgfpathlineto{\pgfqpoint{3.898439in}{2.525780in}}%
\pgfpathclose%
\pgfusepath{stroke,fill}%
\end{pgfscope}%
\begin{pgfscope}%
\pgfpathrectangle{\pgfqpoint{0.997489in}{0.528000in}}{\pgfqpoint{4.565023in}{3.696000in}}%
\pgfusepath{clip}%
\pgfsetbuttcap%
\pgfsetroundjoin%
\definecolor{currentfill}{rgb}{0.800000,0.800000,0.200000}%
\pgfsetfillcolor{currentfill}%
\pgfsetlinewidth{1.003750pt}%
\definecolor{currentstroke}{rgb}{0.800000,0.800000,0.200000}%
\pgfsetstrokecolor{currentstroke}%
\pgfsetdash{}{0pt}%
\pgfpathmoveto{\pgfqpoint{3.880597in}{2.417238in}}%
\pgfpathcurveto{\pgfqpoint{3.886421in}{2.417238in}}{\pgfqpoint{3.892007in}{2.419551in}}{\pgfqpoint{3.896125in}{2.423670in}}%
\pgfpathcurveto{\pgfqpoint{3.900243in}{2.427788in}}{\pgfqpoint{3.902557in}{2.433374in}}{\pgfqpoint{3.902557in}{2.439198in}}%
\pgfpathcurveto{\pgfqpoint{3.902557in}{2.445022in}}{\pgfqpoint{3.900243in}{2.450608in}}{\pgfqpoint{3.896125in}{2.454726in}}%
\pgfpathcurveto{\pgfqpoint{3.892007in}{2.458844in}}{\pgfqpoint{3.886421in}{2.461158in}}{\pgfqpoint{3.880597in}{2.461158in}}%
\pgfpathcurveto{\pgfqpoint{3.874773in}{2.461158in}}{\pgfqpoint{3.869187in}{2.458844in}}{\pgfqpoint{3.865068in}{2.454726in}}%
\pgfpathcurveto{\pgfqpoint{3.860950in}{2.450608in}}{\pgfqpoint{3.858636in}{2.445022in}}{\pgfqpoint{3.858636in}{2.439198in}}%
\pgfpathcurveto{\pgfqpoint{3.858636in}{2.433374in}}{\pgfqpoint{3.860950in}{2.427788in}}{\pgfqpoint{3.865068in}{2.423670in}}%
\pgfpathcurveto{\pgfqpoint{3.869187in}{2.419551in}}{\pgfqpoint{3.874773in}{2.417238in}}{\pgfqpoint{3.880597in}{2.417238in}}%
\pgfpathlineto{\pgfqpoint{3.880597in}{2.417238in}}%
\pgfpathclose%
\pgfusepath{stroke,fill}%
\end{pgfscope}%
\begin{pgfscope}%
\pgfpathrectangle{\pgfqpoint{0.997489in}{0.528000in}}{\pgfqpoint{4.565023in}{3.696000in}}%
\pgfusepath{clip}%
\pgfsetbuttcap%
\pgfsetroundjoin%
\definecolor{currentfill}{rgb}{0.800000,0.800000,0.200000}%
\pgfsetfillcolor{currentfill}%
\pgfsetlinewidth{1.003750pt}%
\definecolor{currentstroke}{rgb}{0.800000,0.800000,0.200000}%
\pgfsetstrokecolor{currentstroke}%
\pgfsetdash{}{0pt}%
\pgfpathmoveto{\pgfqpoint{3.947795in}{2.414261in}}%
\pgfpathcurveto{\pgfqpoint{3.953619in}{2.414261in}}{\pgfqpoint{3.959205in}{2.416574in}}{\pgfqpoint{3.963323in}{2.420693in}}%
\pgfpathcurveto{\pgfqpoint{3.967441in}{2.424811in}}{\pgfqpoint{3.969755in}{2.430397in}}{\pgfqpoint{3.969755in}{2.436221in}}%
\pgfpathcurveto{\pgfqpoint{3.969755in}{2.442045in}}{\pgfqpoint{3.967441in}{2.447631in}}{\pgfqpoint{3.963323in}{2.451749in}}%
\pgfpathcurveto{\pgfqpoint{3.959205in}{2.455867in}}{\pgfqpoint{3.953619in}{2.458181in}}{\pgfqpoint{3.947795in}{2.458181in}}%
\pgfpathcurveto{\pgfqpoint{3.941971in}{2.458181in}}{\pgfqpoint{3.936385in}{2.455867in}}{\pgfqpoint{3.932267in}{2.451749in}}%
\pgfpathcurveto{\pgfqpoint{3.928149in}{2.447631in}}{\pgfqpoint{3.925835in}{2.442045in}}{\pgfqpoint{3.925835in}{2.436221in}}%
\pgfpathcurveto{\pgfqpoint{3.925835in}{2.430397in}}{\pgfqpoint{3.928149in}{2.424811in}}{\pgfqpoint{3.932267in}{2.420693in}}%
\pgfpathcurveto{\pgfqpoint{3.936385in}{2.416574in}}{\pgfqpoint{3.941971in}{2.414261in}}{\pgfqpoint{3.947795in}{2.414261in}}%
\pgfpathlineto{\pgfqpoint{3.947795in}{2.414261in}}%
\pgfpathclose%
\pgfusepath{stroke,fill}%
\end{pgfscope}%
\begin{pgfscope}%
\pgfpathrectangle{\pgfqpoint{0.997489in}{0.528000in}}{\pgfqpoint{4.565023in}{3.696000in}}%
\pgfusepath{clip}%
\pgfsetbuttcap%
\pgfsetroundjoin%
\definecolor{currentfill}{rgb}{0.800000,0.800000,0.200000}%
\pgfsetfillcolor{currentfill}%
\pgfsetlinewidth{1.003750pt}%
\definecolor{currentstroke}{rgb}{0.800000,0.800000,0.200000}%
\pgfsetstrokecolor{currentstroke}%
\pgfsetdash{}{0pt}%
\pgfpathmoveto{\pgfqpoint{3.971874in}{2.345412in}}%
\pgfpathcurveto{\pgfqpoint{3.977698in}{2.345412in}}{\pgfqpoint{3.983284in}{2.347726in}}{\pgfqpoint{3.987402in}{2.351844in}}%
\pgfpathcurveto{\pgfqpoint{3.991520in}{2.355963in}}{\pgfqpoint{3.993834in}{2.361549in}}{\pgfqpoint{3.993834in}{2.367373in}}%
\pgfpathcurveto{\pgfqpoint{3.993834in}{2.373197in}}{\pgfqpoint{3.991520in}{2.378783in}}{\pgfqpoint{3.987402in}{2.382901in}}%
\pgfpathcurveto{\pgfqpoint{3.983284in}{2.387019in}}{\pgfqpoint{3.977698in}{2.389333in}}{\pgfqpoint{3.971874in}{2.389333in}}%
\pgfpathcurveto{\pgfqpoint{3.966050in}{2.389333in}}{\pgfqpoint{3.960464in}{2.387019in}}{\pgfqpoint{3.956346in}{2.382901in}}%
\pgfpathcurveto{\pgfqpoint{3.952227in}{2.378783in}}{\pgfqpoint{3.949914in}{2.373197in}}{\pgfqpoint{3.949914in}{2.367373in}}%
\pgfpathcurveto{\pgfqpoint{3.949914in}{2.361549in}}{\pgfqpoint{3.952227in}{2.355963in}}{\pgfqpoint{3.956346in}{2.351844in}}%
\pgfpathcurveto{\pgfqpoint{3.960464in}{2.347726in}}{\pgfqpoint{3.966050in}{2.345412in}}{\pgfqpoint{3.971874in}{2.345412in}}%
\pgfpathlineto{\pgfqpoint{3.971874in}{2.345412in}}%
\pgfpathclose%
\pgfusepath{stroke,fill}%
\end{pgfscope}%
\begin{pgfscope}%
\pgfpathrectangle{\pgfqpoint{0.997489in}{0.528000in}}{\pgfqpoint{4.565023in}{3.696000in}}%
\pgfusepath{clip}%
\pgfsetbuttcap%
\pgfsetroundjoin%
\definecolor{currentfill}{rgb}{0.800000,0.800000,0.200000}%
\pgfsetfillcolor{currentfill}%
\pgfsetlinewidth{1.003750pt}%
\definecolor{currentstroke}{rgb}{0.800000,0.800000,0.200000}%
\pgfsetstrokecolor{currentstroke}%
\pgfsetdash{}{0pt}%
\pgfpathmoveto{\pgfqpoint{4.079270in}{2.431082in}}%
\pgfpathcurveto{\pgfqpoint{4.085094in}{2.431082in}}{\pgfqpoint{4.090681in}{2.433396in}}{\pgfqpoint{4.094799in}{2.437514in}}%
\pgfpathcurveto{\pgfqpoint{4.098917in}{2.441633in}}{\pgfqpoint{4.101231in}{2.447219in}}{\pgfqpoint{4.101231in}{2.453043in}}%
\pgfpathcurveto{\pgfqpoint{4.101231in}{2.458867in}}{\pgfqpoint{4.098917in}{2.464453in}}{\pgfqpoint{4.094799in}{2.468571in}}%
\pgfpathcurveto{\pgfqpoint{4.090681in}{2.472689in}}{\pgfqpoint{4.085094in}{2.475003in}}{\pgfqpoint{4.079270in}{2.475003in}}%
\pgfpathcurveto{\pgfqpoint{4.073446in}{2.475003in}}{\pgfqpoint{4.067860in}{2.472689in}}{\pgfqpoint{4.063742in}{2.468571in}}%
\pgfpathcurveto{\pgfqpoint{4.059624in}{2.464453in}}{\pgfqpoint{4.057310in}{2.458867in}}{\pgfqpoint{4.057310in}{2.453043in}}%
\pgfpathcurveto{\pgfqpoint{4.057310in}{2.447219in}}{\pgfqpoint{4.059624in}{2.441633in}}{\pgfqpoint{4.063742in}{2.437514in}}%
\pgfpathcurveto{\pgfqpoint{4.067860in}{2.433396in}}{\pgfqpoint{4.073446in}{2.431082in}}{\pgfqpoint{4.079270in}{2.431082in}}%
\pgfpathlineto{\pgfqpoint{4.079270in}{2.431082in}}%
\pgfpathclose%
\pgfusepath{stroke,fill}%
\end{pgfscope}%
\begin{pgfscope}%
\pgfpathrectangle{\pgfqpoint{0.997489in}{0.528000in}}{\pgfqpoint{4.565023in}{3.696000in}}%
\pgfusepath{clip}%
\pgfsetbuttcap%
\pgfsetroundjoin%
\definecolor{currentfill}{rgb}{0.800000,0.800000,0.200000}%
\pgfsetfillcolor{currentfill}%
\pgfsetlinewidth{1.003750pt}%
\definecolor{currentstroke}{rgb}{0.800000,0.800000,0.200000}%
\pgfsetstrokecolor{currentstroke}%
\pgfsetdash{}{0pt}%
\pgfpathmoveto{\pgfqpoint{4.086292in}{2.317979in}}%
\pgfpathcurveto{\pgfqpoint{4.092116in}{2.317979in}}{\pgfqpoint{4.097702in}{2.320293in}}{\pgfqpoint{4.101820in}{2.324411in}}%
\pgfpathcurveto{\pgfqpoint{4.105938in}{2.328529in}}{\pgfqpoint{4.108252in}{2.334116in}}{\pgfqpoint{4.108252in}{2.339940in}}%
\pgfpathcurveto{\pgfqpoint{4.108252in}{2.345763in}}{\pgfqpoint{4.105938in}{2.351350in}}{\pgfqpoint{4.101820in}{2.355468in}}%
\pgfpathcurveto{\pgfqpoint{4.097702in}{2.359586in}}{\pgfqpoint{4.092116in}{2.361900in}}{\pgfqpoint{4.086292in}{2.361900in}}%
\pgfpathcurveto{\pgfqpoint{4.080468in}{2.361900in}}{\pgfqpoint{4.074882in}{2.359586in}}{\pgfqpoint{4.070764in}{2.355468in}}%
\pgfpathcurveto{\pgfqpoint{4.066646in}{2.351350in}}{\pgfqpoint{4.064332in}{2.345763in}}{\pgfqpoint{4.064332in}{2.339940in}}%
\pgfpathcurveto{\pgfqpoint{4.064332in}{2.334116in}}{\pgfqpoint{4.066646in}{2.328529in}}{\pgfqpoint{4.070764in}{2.324411in}}%
\pgfpathcurveto{\pgfqpoint{4.074882in}{2.320293in}}{\pgfqpoint{4.080468in}{2.317979in}}{\pgfqpoint{4.086292in}{2.317979in}}%
\pgfpathlineto{\pgfqpoint{4.086292in}{2.317979in}}%
\pgfpathclose%
\pgfusepath{stroke,fill}%
\end{pgfscope}%
\begin{pgfscope}%
\pgfpathrectangle{\pgfqpoint{0.997489in}{0.528000in}}{\pgfqpoint{4.565023in}{3.696000in}}%
\pgfusepath{clip}%
\pgfsetbuttcap%
\pgfsetroundjoin%
\definecolor{currentfill}{rgb}{0.800000,0.200000,0.200000}%
\pgfsetfillcolor{currentfill}%
\pgfsetlinewidth{1.003750pt}%
\definecolor{currentstroke}{rgb}{0.800000,0.200000,0.200000}%
\pgfsetstrokecolor{currentstroke}%
\pgfsetdash{}{0pt}%
\pgfpathmoveto{\pgfqpoint{4.167870in}{2.380886in}}%
\pgfpathcurveto{\pgfqpoint{4.173694in}{2.380886in}}{\pgfqpoint{4.179280in}{2.383200in}}{\pgfqpoint{4.183398in}{2.387318in}}%
\pgfpathcurveto{\pgfqpoint{4.187516in}{2.391436in}}{\pgfqpoint{4.189830in}{2.397022in}}{\pgfqpoint{4.189830in}{2.402846in}}%
\pgfpathcurveto{\pgfqpoint{4.189830in}{2.408670in}}{\pgfqpoint{4.187516in}{2.414256in}}{\pgfqpoint{4.183398in}{2.418374in}}%
\pgfpathcurveto{\pgfqpoint{4.179280in}{2.422493in}}{\pgfqpoint{4.173694in}{2.424806in}}{\pgfqpoint{4.167870in}{2.424806in}}%
\pgfpathcurveto{\pgfqpoint{4.162046in}{2.424806in}}{\pgfqpoint{4.156460in}{2.422493in}}{\pgfqpoint{4.152342in}{2.418374in}}%
\pgfpathcurveto{\pgfqpoint{4.148224in}{2.414256in}}{\pgfqpoint{4.145910in}{2.408670in}}{\pgfqpoint{4.145910in}{2.402846in}}%
\pgfpathcurveto{\pgfqpoint{4.145910in}{2.397022in}}{\pgfqpoint{4.148224in}{2.391436in}}{\pgfqpoint{4.152342in}{2.387318in}}%
\pgfpathcurveto{\pgfqpoint{4.156460in}{2.383200in}}{\pgfqpoint{4.162046in}{2.380886in}}{\pgfqpoint{4.167870in}{2.380886in}}%
\pgfpathlineto{\pgfqpoint{4.167870in}{2.380886in}}%
\pgfpathclose%
\pgfusepath{stroke,fill}%
\end{pgfscope}%
\begin{pgfscope}%
\pgfpathrectangle{\pgfqpoint{0.997489in}{0.528000in}}{\pgfqpoint{4.565023in}{3.696000in}}%
\pgfusepath{clip}%
\pgfsetbuttcap%
\pgfsetroundjoin%
\definecolor{currentfill}{rgb}{0.800000,0.200000,0.200000}%
\pgfsetfillcolor{currentfill}%
\pgfsetlinewidth{1.003750pt}%
\definecolor{currentstroke}{rgb}{0.800000,0.200000,0.200000}%
\pgfsetstrokecolor{currentstroke}%
\pgfsetdash{}{0pt}%
\pgfpathmoveto{\pgfqpoint{4.239338in}{2.448223in}}%
\pgfpathcurveto{\pgfqpoint{4.245162in}{2.448223in}}{\pgfqpoint{4.250748in}{2.450537in}}{\pgfqpoint{4.254866in}{2.454655in}}%
\pgfpathcurveto{\pgfqpoint{4.258984in}{2.458773in}}{\pgfqpoint{4.261298in}{2.464360in}}{\pgfqpoint{4.261298in}{2.470184in}}%
\pgfpathcurveto{\pgfqpoint{4.261298in}{2.476008in}}{\pgfqpoint{4.258984in}{2.481594in}}{\pgfqpoint{4.254866in}{2.485712in}}%
\pgfpathcurveto{\pgfqpoint{4.250748in}{2.489830in}}{\pgfqpoint{4.245162in}{2.492144in}}{\pgfqpoint{4.239338in}{2.492144in}}%
\pgfpathcurveto{\pgfqpoint{4.233514in}{2.492144in}}{\pgfqpoint{4.227928in}{2.489830in}}{\pgfqpoint{4.223810in}{2.485712in}}%
\pgfpathcurveto{\pgfqpoint{4.219692in}{2.481594in}}{\pgfqpoint{4.217378in}{2.476008in}}{\pgfqpoint{4.217378in}{2.470184in}}%
\pgfpathcurveto{\pgfqpoint{4.217378in}{2.464360in}}{\pgfqpoint{4.219692in}{2.458773in}}{\pgfqpoint{4.223810in}{2.454655in}}%
\pgfpathcurveto{\pgfqpoint{4.227928in}{2.450537in}}{\pgfqpoint{4.233514in}{2.448223in}}{\pgfqpoint{4.239338in}{2.448223in}}%
\pgfpathlineto{\pgfqpoint{4.239338in}{2.448223in}}%
\pgfpathclose%
\pgfusepath{stroke,fill}%
\end{pgfscope}%
\begin{pgfscope}%
\pgfpathrectangle{\pgfqpoint{0.997489in}{0.528000in}}{\pgfqpoint{4.565023in}{3.696000in}}%
\pgfusepath{clip}%
\pgfsetbuttcap%
\pgfsetroundjoin%
\definecolor{currentfill}{rgb}{0.800000,0.800000,0.200000}%
\pgfsetfillcolor{currentfill}%
\pgfsetlinewidth{1.003750pt}%
\definecolor{currentstroke}{rgb}{0.800000,0.800000,0.200000}%
\pgfsetstrokecolor{currentstroke}%
\pgfsetdash{}{0pt}%
\pgfpathmoveto{\pgfqpoint{4.248702in}{2.270748in}}%
\pgfpathcurveto{\pgfqpoint{4.254526in}{2.270748in}}{\pgfqpoint{4.260112in}{2.273061in}}{\pgfqpoint{4.264230in}{2.277180in}}%
\pgfpathcurveto{\pgfqpoint{4.268348in}{2.281298in}}{\pgfqpoint{4.270662in}{2.286884in}}{\pgfqpoint{4.270662in}{2.292708in}}%
\pgfpathcurveto{\pgfqpoint{4.270662in}{2.298532in}}{\pgfqpoint{4.268348in}{2.304118in}}{\pgfqpoint{4.264230in}{2.308236in}}%
\pgfpathcurveto{\pgfqpoint{4.260112in}{2.312354in}}{\pgfqpoint{4.254526in}{2.314668in}}{\pgfqpoint{4.248702in}{2.314668in}}%
\pgfpathcurveto{\pgfqpoint{4.242878in}{2.314668in}}{\pgfqpoint{4.237292in}{2.312354in}}{\pgfqpoint{4.233173in}{2.308236in}}%
\pgfpathcurveto{\pgfqpoint{4.229055in}{2.304118in}}{\pgfqpoint{4.226741in}{2.298532in}}{\pgfqpoint{4.226741in}{2.292708in}}%
\pgfpathcurveto{\pgfqpoint{4.226741in}{2.286884in}}{\pgfqpoint{4.229055in}{2.281298in}}{\pgfqpoint{4.233173in}{2.277180in}}%
\pgfpathcurveto{\pgfqpoint{4.237292in}{2.273061in}}{\pgfqpoint{4.242878in}{2.270748in}}{\pgfqpoint{4.248702in}{2.270748in}}%
\pgfpathlineto{\pgfqpoint{4.248702in}{2.270748in}}%
\pgfpathclose%
\pgfusepath{stroke,fill}%
\end{pgfscope}%
\begin{pgfscope}%
\pgfpathrectangle{\pgfqpoint{0.997489in}{0.528000in}}{\pgfqpoint{4.565023in}{3.696000in}}%
\pgfusepath{clip}%
\pgfsetbuttcap%
\pgfsetroundjoin%
\definecolor{currentfill}{rgb}{0.800000,0.800000,0.200000}%
\pgfsetfillcolor{currentfill}%
\pgfsetlinewidth{1.003750pt}%
\definecolor{currentstroke}{rgb}{0.800000,0.800000,0.200000}%
\pgfsetstrokecolor{currentstroke}%
\pgfsetdash{}{0pt}%
\pgfpathmoveto{\pgfqpoint{4.310836in}{2.307175in}}%
\pgfpathcurveto{\pgfqpoint{4.316660in}{2.307175in}}{\pgfqpoint{4.322246in}{2.309489in}}{\pgfqpoint{4.326364in}{2.313607in}}%
\pgfpathcurveto{\pgfqpoint{4.330483in}{2.317726in}}{\pgfqpoint{4.332796in}{2.323312in}}{\pgfqpoint{4.332796in}{2.329136in}}%
\pgfpathcurveto{\pgfqpoint{4.332796in}{2.334960in}}{\pgfqpoint{4.330483in}{2.340546in}}{\pgfqpoint{4.326364in}{2.344664in}}%
\pgfpathcurveto{\pgfqpoint{4.322246in}{2.348782in}}{\pgfqpoint{4.316660in}{2.351096in}}{\pgfqpoint{4.310836in}{2.351096in}}%
\pgfpathcurveto{\pgfqpoint{4.305012in}{2.351096in}}{\pgfqpoint{4.299426in}{2.348782in}}{\pgfqpoint{4.295308in}{2.344664in}}%
\pgfpathcurveto{\pgfqpoint{4.291190in}{2.340546in}}{\pgfqpoint{4.288876in}{2.334960in}}{\pgfqpoint{4.288876in}{2.329136in}}%
\pgfpathcurveto{\pgfqpoint{4.288876in}{2.323312in}}{\pgfqpoint{4.291190in}{2.317726in}}{\pgfqpoint{4.295308in}{2.313607in}}%
\pgfpathcurveto{\pgfqpoint{4.299426in}{2.309489in}}{\pgfqpoint{4.305012in}{2.307175in}}{\pgfqpoint{4.310836in}{2.307175in}}%
\pgfpathlineto{\pgfqpoint{4.310836in}{2.307175in}}%
\pgfpathclose%
\pgfusepath{stroke,fill}%
\end{pgfscope}%
\begin{pgfscope}%
\pgfpathrectangle{\pgfqpoint{0.997489in}{0.528000in}}{\pgfqpoint{4.565023in}{3.696000in}}%
\pgfusepath{clip}%
\pgfsetbuttcap%
\pgfsetroundjoin%
\definecolor{currentfill}{rgb}{0.800000,0.800000,0.200000}%
\pgfsetfillcolor{currentfill}%
\pgfsetlinewidth{1.003750pt}%
\definecolor{currentstroke}{rgb}{0.800000,0.800000,0.200000}%
\pgfsetstrokecolor{currentstroke}%
\pgfsetdash{}{0pt}%
\pgfpathmoveto{\pgfqpoint{4.359918in}{2.251288in}}%
\pgfpathcurveto{\pgfqpoint{4.365742in}{2.251288in}}{\pgfqpoint{4.371328in}{2.253602in}}{\pgfqpoint{4.375446in}{2.257720in}}%
\pgfpathcurveto{\pgfqpoint{4.379564in}{2.261838in}}{\pgfqpoint{4.381878in}{2.267424in}}{\pgfqpoint{4.381878in}{2.273248in}}%
\pgfpathcurveto{\pgfqpoint{4.381878in}{2.279072in}}{\pgfqpoint{4.379564in}{2.284658in}}{\pgfqpoint{4.375446in}{2.288776in}}%
\pgfpathcurveto{\pgfqpoint{4.371328in}{2.292895in}}{\pgfqpoint{4.365742in}{2.295208in}}{\pgfqpoint{4.359918in}{2.295208in}}%
\pgfpathcurveto{\pgfqpoint{4.354094in}{2.295208in}}{\pgfqpoint{4.348508in}{2.292895in}}{\pgfqpoint{4.344390in}{2.288776in}}%
\pgfpathcurveto{\pgfqpoint{4.340272in}{2.284658in}}{\pgfqpoint{4.337958in}{2.279072in}}{\pgfqpoint{4.337958in}{2.273248in}}%
\pgfpathcurveto{\pgfqpoint{4.337958in}{2.267424in}}{\pgfqpoint{4.340272in}{2.261838in}}{\pgfqpoint{4.344390in}{2.257720in}}%
\pgfpathcurveto{\pgfqpoint{4.348508in}{2.253602in}}{\pgfqpoint{4.354094in}{2.251288in}}{\pgfqpoint{4.359918in}{2.251288in}}%
\pgfpathlineto{\pgfqpoint{4.359918in}{2.251288in}}%
\pgfpathclose%
\pgfusepath{stroke,fill}%
\end{pgfscope}%
\begin{pgfscope}%
\pgfpathrectangle{\pgfqpoint{0.997489in}{0.528000in}}{\pgfqpoint{4.565023in}{3.696000in}}%
\pgfusepath{clip}%
\pgfsetbuttcap%
\pgfsetroundjoin%
\definecolor{currentfill}{rgb}{0.800000,0.800000,0.200000}%
\pgfsetfillcolor{currentfill}%
\pgfsetlinewidth{1.003750pt}%
\definecolor{currentstroke}{rgb}{0.800000,0.800000,0.200000}%
\pgfsetstrokecolor{currentstroke}%
\pgfsetdash{}{0pt}%
\pgfpathmoveto{\pgfqpoint{4.417133in}{2.293627in}}%
\pgfpathcurveto{\pgfqpoint{4.422957in}{2.293627in}}{\pgfqpoint{4.428543in}{2.295941in}}{\pgfqpoint{4.432661in}{2.300059in}}%
\pgfpathcurveto{\pgfqpoint{4.436779in}{2.304177in}}{\pgfqpoint{4.439093in}{2.309763in}}{\pgfqpoint{4.439093in}{2.315587in}}%
\pgfpathcurveto{\pgfqpoint{4.439093in}{2.321411in}}{\pgfqpoint{4.436779in}{2.326997in}}{\pgfqpoint{4.432661in}{2.331115in}}%
\pgfpathcurveto{\pgfqpoint{4.428543in}{2.335233in}}{\pgfqpoint{4.422957in}{2.337547in}}{\pgfqpoint{4.417133in}{2.337547in}}%
\pgfpathcurveto{\pgfqpoint{4.411309in}{2.337547in}}{\pgfqpoint{4.405723in}{2.335233in}}{\pgfqpoint{4.401605in}{2.331115in}}%
\pgfpathcurveto{\pgfqpoint{4.397487in}{2.326997in}}{\pgfqpoint{4.395173in}{2.321411in}}{\pgfqpoint{4.395173in}{2.315587in}}%
\pgfpathcurveto{\pgfqpoint{4.395173in}{2.309763in}}{\pgfqpoint{4.397487in}{2.304177in}}{\pgfqpoint{4.401605in}{2.300059in}}%
\pgfpathcurveto{\pgfqpoint{4.405723in}{2.295941in}}{\pgfqpoint{4.411309in}{2.293627in}}{\pgfqpoint{4.417133in}{2.293627in}}%
\pgfpathlineto{\pgfqpoint{4.417133in}{2.293627in}}%
\pgfpathclose%
\pgfusepath{stroke,fill}%
\end{pgfscope}%
\begin{pgfscope}%
\pgfpathrectangle{\pgfqpoint{0.997489in}{0.528000in}}{\pgfqpoint{4.565023in}{3.696000in}}%
\pgfusepath{clip}%
\pgfsetbuttcap%
\pgfsetroundjoin%
\definecolor{currentfill}{rgb}{0.800000,0.800000,0.200000}%
\pgfsetfillcolor{currentfill}%
\pgfsetlinewidth{1.003750pt}%
\definecolor{currentstroke}{rgb}{0.800000,0.800000,0.200000}%
\pgfsetstrokecolor{currentstroke}%
\pgfsetdash{}{0pt}%
\pgfpathmoveto{\pgfqpoint{4.470659in}{2.298125in}}%
\pgfpathcurveto{\pgfqpoint{4.476483in}{2.298125in}}{\pgfqpoint{4.482069in}{2.300438in}}{\pgfqpoint{4.486187in}{2.304557in}}%
\pgfpathcurveto{\pgfqpoint{4.490306in}{2.308675in}}{\pgfqpoint{4.492619in}{2.314261in}}{\pgfqpoint{4.492619in}{2.320085in}}%
\pgfpathcurveto{\pgfqpoint{4.492619in}{2.325909in}}{\pgfqpoint{4.490306in}{2.331495in}}{\pgfqpoint{4.486187in}{2.335613in}}%
\pgfpathcurveto{\pgfqpoint{4.482069in}{2.339731in}}{\pgfqpoint{4.476483in}{2.342045in}}{\pgfqpoint{4.470659in}{2.342045in}}%
\pgfpathcurveto{\pgfqpoint{4.464835in}{2.342045in}}{\pgfqpoint{4.459249in}{2.339731in}}{\pgfqpoint{4.455131in}{2.335613in}}%
\pgfpathcurveto{\pgfqpoint{4.451013in}{2.331495in}}{\pgfqpoint{4.448699in}{2.325909in}}{\pgfqpoint{4.448699in}{2.320085in}}%
\pgfpathcurveto{\pgfqpoint{4.448699in}{2.314261in}}{\pgfqpoint{4.451013in}{2.308675in}}{\pgfqpoint{4.455131in}{2.304557in}}%
\pgfpathcurveto{\pgfqpoint{4.459249in}{2.300438in}}{\pgfqpoint{4.464835in}{2.298125in}}{\pgfqpoint{4.470659in}{2.298125in}}%
\pgfpathlineto{\pgfqpoint{4.470659in}{2.298125in}}%
\pgfpathclose%
\pgfusepath{stroke,fill}%
\end{pgfscope}%
\begin{pgfscope}%
\pgfpathrectangle{\pgfqpoint{0.997489in}{0.528000in}}{\pgfqpoint{4.565023in}{3.696000in}}%
\pgfusepath{clip}%
\pgfsetbuttcap%
\pgfsetroundjoin%
\definecolor{currentfill}{rgb}{0.800000,0.800000,0.200000}%
\pgfsetfillcolor{currentfill}%
\pgfsetlinewidth{1.003750pt}%
\definecolor{currentstroke}{rgb}{0.800000,0.800000,0.200000}%
\pgfsetstrokecolor{currentstroke}%
\pgfsetdash{}{0pt}%
\pgfpathmoveto{\pgfqpoint{4.532432in}{2.226451in}}%
\pgfpathcurveto{\pgfqpoint{4.538256in}{2.226451in}}{\pgfqpoint{4.543842in}{2.228764in}}{\pgfqpoint{4.547960in}{2.232883in}}%
\pgfpathcurveto{\pgfqpoint{4.552079in}{2.237001in}}{\pgfqpoint{4.554392in}{2.242587in}}{\pgfqpoint{4.554392in}{2.248411in}}%
\pgfpathcurveto{\pgfqpoint{4.554392in}{2.254235in}}{\pgfqpoint{4.552079in}{2.259821in}}{\pgfqpoint{4.547960in}{2.263939in}}%
\pgfpathcurveto{\pgfqpoint{4.543842in}{2.268057in}}{\pgfqpoint{4.538256in}{2.270371in}}{\pgfqpoint{4.532432in}{2.270371in}}%
\pgfpathcurveto{\pgfqpoint{4.526608in}{2.270371in}}{\pgfqpoint{4.521022in}{2.268057in}}{\pgfqpoint{4.516904in}{2.263939in}}%
\pgfpathcurveto{\pgfqpoint{4.512786in}{2.259821in}}{\pgfqpoint{4.510472in}{2.254235in}}{\pgfqpoint{4.510472in}{2.248411in}}%
\pgfpathcurveto{\pgfqpoint{4.510472in}{2.242587in}}{\pgfqpoint{4.512786in}{2.237001in}}{\pgfqpoint{4.516904in}{2.232883in}}%
\pgfpathcurveto{\pgfqpoint{4.521022in}{2.228764in}}{\pgfqpoint{4.526608in}{2.226451in}}{\pgfqpoint{4.532432in}{2.226451in}}%
\pgfpathlineto{\pgfqpoint{4.532432in}{2.226451in}}%
\pgfpathclose%
\pgfusepath{stroke,fill}%
\end{pgfscope}%
\begin{pgfscope}%
\pgfpathrectangle{\pgfqpoint{0.997489in}{0.528000in}}{\pgfqpoint{4.565023in}{3.696000in}}%
\pgfusepath{clip}%
\pgfsetbuttcap%
\pgfsetroundjoin%
\definecolor{currentfill}{rgb}{0.800000,0.800000,0.200000}%
\pgfsetfillcolor{currentfill}%
\pgfsetlinewidth{1.003750pt}%
\definecolor{currentstroke}{rgb}{0.800000,0.800000,0.200000}%
\pgfsetstrokecolor{currentstroke}%
\pgfsetdash{}{0pt}%
\pgfpathmoveto{\pgfqpoint{4.577767in}{2.305039in}}%
\pgfpathcurveto{\pgfqpoint{4.583591in}{2.305039in}}{\pgfqpoint{4.589177in}{2.307353in}}{\pgfqpoint{4.593295in}{2.311471in}}%
\pgfpathcurveto{\pgfqpoint{4.597413in}{2.315589in}}{\pgfqpoint{4.599727in}{2.321176in}}{\pgfqpoint{4.599727in}{2.327000in}}%
\pgfpathcurveto{\pgfqpoint{4.599727in}{2.332824in}}{\pgfqpoint{4.597413in}{2.338410in}}{\pgfqpoint{4.593295in}{2.342528in}}%
\pgfpathcurveto{\pgfqpoint{4.589177in}{2.346646in}}{\pgfqpoint{4.583591in}{2.348960in}}{\pgfqpoint{4.577767in}{2.348960in}}%
\pgfpathcurveto{\pgfqpoint{4.571943in}{2.348960in}}{\pgfqpoint{4.566357in}{2.346646in}}{\pgfqpoint{4.562239in}{2.342528in}}%
\pgfpathcurveto{\pgfqpoint{4.558120in}{2.338410in}}{\pgfqpoint{4.555806in}{2.332824in}}{\pgfqpoint{4.555806in}{2.327000in}}%
\pgfpathcurveto{\pgfqpoint{4.555806in}{2.321176in}}{\pgfqpoint{4.558120in}{2.315589in}}{\pgfqpoint{4.562239in}{2.311471in}}%
\pgfpathcurveto{\pgfqpoint{4.566357in}{2.307353in}}{\pgfqpoint{4.571943in}{2.305039in}}{\pgfqpoint{4.577767in}{2.305039in}}%
\pgfpathlineto{\pgfqpoint{4.577767in}{2.305039in}}%
\pgfpathclose%
\pgfusepath{stroke,fill}%
\end{pgfscope}%
\begin{pgfscope}%
\pgfpathrectangle{\pgfqpoint{0.997489in}{0.528000in}}{\pgfqpoint{4.565023in}{3.696000in}}%
\pgfusepath{clip}%
\pgfsetbuttcap%
\pgfsetroundjoin%
\definecolor{currentfill}{rgb}{0.800000,0.800000,0.200000}%
\pgfsetfillcolor{currentfill}%
\pgfsetlinewidth{1.003750pt}%
\definecolor{currentstroke}{rgb}{0.800000,0.800000,0.200000}%
\pgfsetstrokecolor{currentstroke}%
\pgfsetdash{}{0pt}%
\pgfpathmoveto{\pgfqpoint{4.623176in}{2.345914in}}%
\pgfpathcurveto{\pgfqpoint{4.629000in}{2.345914in}}{\pgfqpoint{4.634586in}{2.348228in}}{\pgfqpoint{4.638704in}{2.352346in}}%
\pgfpathcurveto{\pgfqpoint{4.642822in}{2.356464in}}{\pgfqpoint{4.645136in}{2.362051in}}{\pgfqpoint{4.645136in}{2.367875in}}%
\pgfpathcurveto{\pgfqpoint{4.645136in}{2.373698in}}{\pgfqpoint{4.642822in}{2.379285in}}{\pgfqpoint{4.638704in}{2.383403in}}%
\pgfpathcurveto{\pgfqpoint{4.634586in}{2.387521in}}{\pgfqpoint{4.629000in}{2.389835in}}{\pgfqpoint{4.623176in}{2.389835in}}%
\pgfpathcurveto{\pgfqpoint{4.617352in}{2.389835in}}{\pgfqpoint{4.611766in}{2.387521in}}{\pgfqpoint{4.607648in}{2.383403in}}%
\pgfpathcurveto{\pgfqpoint{4.603530in}{2.379285in}}{\pgfqpoint{4.601216in}{2.373698in}}{\pgfqpoint{4.601216in}{2.367875in}}%
\pgfpathcurveto{\pgfqpoint{4.601216in}{2.362051in}}{\pgfqpoint{4.603530in}{2.356464in}}{\pgfqpoint{4.607648in}{2.352346in}}%
\pgfpathcurveto{\pgfqpoint{4.611766in}{2.348228in}}{\pgfqpoint{4.617352in}{2.345914in}}{\pgfqpoint{4.623176in}{2.345914in}}%
\pgfpathlineto{\pgfqpoint{4.623176in}{2.345914in}}%
\pgfpathclose%
\pgfusepath{stroke,fill}%
\end{pgfscope}%
\begin{pgfscope}%
\pgfpathrectangle{\pgfqpoint{0.997489in}{0.528000in}}{\pgfqpoint{4.565023in}{3.696000in}}%
\pgfusepath{clip}%
\pgfsetbuttcap%
\pgfsetroundjoin%
\definecolor{currentfill}{rgb}{0.800000,0.800000,0.200000}%
\pgfsetfillcolor{currentfill}%
\pgfsetlinewidth{1.003750pt}%
\definecolor{currentstroke}{rgb}{0.800000,0.800000,0.200000}%
\pgfsetstrokecolor{currentstroke}%
\pgfsetdash{}{0pt}%
\pgfpathmoveto{\pgfqpoint{4.679092in}{2.340602in}}%
\pgfpathcurveto{\pgfqpoint{4.684916in}{2.340602in}}{\pgfqpoint{4.690502in}{2.342916in}}{\pgfqpoint{4.694620in}{2.347034in}}%
\pgfpathcurveto{\pgfqpoint{4.698739in}{2.351152in}}{\pgfqpoint{4.701053in}{2.356738in}}{\pgfqpoint{4.701053in}{2.362562in}}%
\pgfpathcurveto{\pgfqpoint{4.701053in}{2.368386in}}{\pgfqpoint{4.698739in}{2.373972in}}{\pgfqpoint{4.694620in}{2.378091in}}%
\pgfpathcurveto{\pgfqpoint{4.690502in}{2.382209in}}{\pgfqpoint{4.684916in}{2.384523in}}{\pgfqpoint{4.679092in}{2.384523in}}%
\pgfpathcurveto{\pgfqpoint{4.673268in}{2.384523in}}{\pgfqpoint{4.667682in}{2.382209in}}{\pgfqpoint{4.663564in}{2.378091in}}%
\pgfpathcurveto{\pgfqpoint{4.659446in}{2.373972in}}{\pgfqpoint{4.657132in}{2.368386in}}{\pgfqpoint{4.657132in}{2.362562in}}%
\pgfpathcurveto{\pgfqpoint{4.657132in}{2.356738in}}{\pgfqpoint{4.659446in}{2.351152in}}{\pgfqpoint{4.663564in}{2.347034in}}%
\pgfpathcurveto{\pgfqpoint{4.667682in}{2.342916in}}{\pgfqpoint{4.673268in}{2.340602in}}{\pgfqpoint{4.679092in}{2.340602in}}%
\pgfpathlineto{\pgfqpoint{4.679092in}{2.340602in}}%
\pgfpathclose%
\pgfusepath{stroke,fill}%
\end{pgfscope}%
\begin{pgfscope}%
\pgfpathrectangle{\pgfqpoint{0.997489in}{0.528000in}}{\pgfqpoint{4.565023in}{3.696000in}}%
\pgfusepath{clip}%
\pgfsetbuttcap%
\pgfsetroundjoin%
\definecolor{currentfill}{rgb}{0.800000,0.800000,0.200000}%
\pgfsetfillcolor{currentfill}%
\pgfsetlinewidth{1.003750pt}%
\definecolor{currentstroke}{rgb}{0.800000,0.800000,0.200000}%
\pgfsetstrokecolor{currentstroke}%
\pgfsetdash{}{0pt}%
\pgfpathmoveto{\pgfqpoint{4.733828in}{2.346083in}}%
\pgfpathcurveto{\pgfqpoint{4.739652in}{2.346083in}}{\pgfqpoint{4.745238in}{2.348397in}}{\pgfqpoint{4.749356in}{2.352515in}}%
\pgfpathcurveto{\pgfqpoint{4.753475in}{2.356633in}}{\pgfqpoint{4.755789in}{2.362219in}}{\pgfqpoint{4.755789in}{2.368043in}}%
\pgfpathcurveto{\pgfqpoint{4.755789in}{2.373867in}}{\pgfqpoint{4.753475in}{2.379453in}}{\pgfqpoint{4.749356in}{2.383572in}}%
\pgfpathcurveto{\pgfqpoint{4.745238in}{2.387690in}}{\pgfqpoint{4.739652in}{2.390004in}}{\pgfqpoint{4.733828in}{2.390004in}}%
\pgfpathcurveto{\pgfqpoint{4.728004in}{2.390004in}}{\pgfqpoint{4.722418in}{2.387690in}}{\pgfqpoint{4.718300in}{2.383572in}}%
\pgfpathcurveto{\pgfqpoint{4.714182in}{2.379453in}}{\pgfqpoint{4.711868in}{2.373867in}}{\pgfqpoint{4.711868in}{2.368043in}}%
\pgfpathcurveto{\pgfqpoint{4.711868in}{2.362219in}}{\pgfqpoint{4.714182in}{2.356633in}}{\pgfqpoint{4.718300in}{2.352515in}}%
\pgfpathcurveto{\pgfqpoint{4.722418in}{2.348397in}}{\pgfqpoint{4.728004in}{2.346083in}}{\pgfqpoint{4.733828in}{2.346083in}}%
\pgfpathlineto{\pgfqpoint{4.733828in}{2.346083in}}%
\pgfpathclose%
\pgfusepath{stroke,fill}%
\end{pgfscope}%
\begin{pgfscope}%
\pgfpathrectangle{\pgfqpoint{0.997489in}{0.528000in}}{\pgfqpoint{4.565023in}{3.696000in}}%
\pgfusepath{clip}%
\pgfsetbuttcap%
\pgfsetroundjoin%
\definecolor{currentfill}{rgb}{0.800000,0.800000,0.200000}%
\pgfsetfillcolor{currentfill}%
\pgfsetlinewidth{1.003750pt}%
\definecolor{currentstroke}{rgb}{0.800000,0.800000,0.200000}%
\pgfsetstrokecolor{currentstroke}%
\pgfsetdash{}{0pt}%
\pgfpathmoveto{\pgfqpoint{4.719032in}{2.508221in}}%
\pgfpathcurveto{\pgfqpoint{4.724855in}{2.508221in}}{\pgfqpoint{4.730442in}{2.510535in}}{\pgfqpoint{4.734560in}{2.514653in}}%
\pgfpathcurveto{\pgfqpoint{4.738678in}{2.518771in}}{\pgfqpoint{4.740992in}{2.524357in}}{\pgfqpoint{4.740992in}{2.530181in}}%
\pgfpathcurveto{\pgfqpoint{4.740992in}{2.536005in}}{\pgfqpoint{4.738678in}{2.541592in}}{\pgfqpoint{4.734560in}{2.545710in}}%
\pgfpathcurveto{\pgfqpoint{4.730442in}{2.549828in}}{\pgfqpoint{4.724855in}{2.552142in}}{\pgfqpoint{4.719032in}{2.552142in}}%
\pgfpathcurveto{\pgfqpoint{4.713208in}{2.552142in}}{\pgfqpoint{4.707621in}{2.549828in}}{\pgfqpoint{4.703503in}{2.545710in}}%
\pgfpathcurveto{\pgfqpoint{4.699385in}{2.541592in}}{\pgfqpoint{4.697071in}{2.536005in}}{\pgfqpoint{4.697071in}{2.530181in}}%
\pgfpathcurveto{\pgfqpoint{4.697071in}{2.524357in}}{\pgfqpoint{4.699385in}{2.518771in}}{\pgfqpoint{4.703503in}{2.514653in}}%
\pgfpathcurveto{\pgfqpoint{4.707621in}{2.510535in}}{\pgfqpoint{4.713208in}{2.508221in}}{\pgfqpoint{4.719032in}{2.508221in}}%
\pgfpathlineto{\pgfqpoint{4.719032in}{2.508221in}}%
\pgfpathclose%
\pgfusepath{stroke,fill}%
\end{pgfscope}%
\begin{pgfscope}%
\pgfpathrectangle{\pgfqpoint{0.997489in}{0.528000in}}{\pgfqpoint{4.565023in}{3.696000in}}%
\pgfusepath{clip}%
\pgfsetbuttcap%
\pgfsetroundjoin%
\definecolor{currentfill}{rgb}{0.800000,0.800000,0.200000}%
\pgfsetfillcolor{currentfill}%
\pgfsetlinewidth{1.003750pt}%
\definecolor{currentstroke}{rgb}{0.800000,0.800000,0.200000}%
\pgfsetstrokecolor{currentstroke}%
\pgfsetdash{}{0pt}%
\pgfpathmoveto{\pgfqpoint{4.840026in}{2.375652in}}%
\pgfpathcurveto{\pgfqpoint{4.845850in}{2.375652in}}{\pgfqpoint{4.851436in}{2.377966in}}{\pgfqpoint{4.855554in}{2.382084in}}%
\pgfpathcurveto{\pgfqpoint{4.859672in}{2.386202in}}{\pgfqpoint{4.861986in}{2.391788in}}{\pgfqpoint{4.861986in}{2.397612in}}%
\pgfpathcurveto{\pgfqpoint{4.861986in}{2.403436in}}{\pgfqpoint{4.859672in}{2.409022in}}{\pgfqpoint{4.855554in}{2.413140in}}%
\pgfpathcurveto{\pgfqpoint{4.851436in}{2.417259in}}{\pgfqpoint{4.845850in}{2.419573in}}{\pgfqpoint{4.840026in}{2.419573in}}%
\pgfpathcurveto{\pgfqpoint{4.834202in}{2.419573in}}{\pgfqpoint{4.828616in}{2.417259in}}{\pgfqpoint{4.824498in}{2.413140in}}%
\pgfpathcurveto{\pgfqpoint{4.820380in}{2.409022in}}{\pgfqpoint{4.818066in}{2.403436in}}{\pgfqpoint{4.818066in}{2.397612in}}%
\pgfpathcurveto{\pgfqpoint{4.818066in}{2.391788in}}{\pgfqpoint{4.820380in}{2.386202in}}{\pgfqpoint{4.824498in}{2.382084in}}%
\pgfpathcurveto{\pgfqpoint{4.828616in}{2.377966in}}{\pgfqpoint{4.834202in}{2.375652in}}{\pgfqpoint{4.840026in}{2.375652in}}%
\pgfpathlineto{\pgfqpoint{4.840026in}{2.375652in}}%
\pgfpathclose%
\pgfusepath{stroke,fill}%
\end{pgfscope}%
\begin{pgfscope}%
\pgfpathrectangle{\pgfqpoint{0.997489in}{0.528000in}}{\pgfqpoint{4.565023in}{3.696000in}}%
\pgfusepath{clip}%
\pgfsetbuttcap%
\pgfsetroundjoin%
\definecolor{currentfill}{rgb}{0.800000,0.800000,0.200000}%
\pgfsetfillcolor{currentfill}%
\pgfsetlinewidth{1.003750pt}%
\definecolor{currentstroke}{rgb}{0.800000,0.800000,0.200000}%
\pgfsetstrokecolor{currentstroke}%
\pgfsetdash{}{0pt}%
\pgfpathmoveto{\pgfqpoint{4.911345in}{2.365007in}}%
\pgfpathcurveto{\pgfqpoint{4.917169in}{2.365007in}}{\pgfqpoint{4.922755in}{2.367321in}}{\pgfqpoint{4.926874in}{2.371439in}}%
\pgfpathcurveto{\pgfqpoint{4.930992in}{2.375557in}}{\pgfqpoint{4.933306in}{2.381143in}}{\pgfqpoint{4.933306in}{2.386967in}}%
\pgfpathcurveto{\pgfqpoint{4.933306in}{2.392791in}}{\pgfqpoint{4.930992in}{2.398377in}}{\pgfqpoint{4.926874in}{2.402495in}}%
\pgfpathcurveto{\pgfqpoint{4.922755in}{2.406613in}}{\pgfqpoint{4.917169in}{2.408927in}}{\pgfqpoint{4.911345in}{2.408927in}}%
\pgfpathcurveto{\pgfqpoint{4.905521in}{2.408927in}}{\pgfqpoint{4.899935in}{2.406613in}}{\pgfqpoint{4.895817in}{2.402495in}}%
\pgfpathcurveto{\pgfqpoint{4.891699in}{2.398377in}}{\pgfqpoint{4.889385in}{2.392791in}}{\pgfqpoint{4.889385in}{2.386967in}}%
\pgfpathcurveto{\pgfqpoint{4.889385in}{2.381143in}}{\pgfqpoint{4.891699in}{2.375557in}}{\pgfqpoint{4.895817in}{2.371439in}}%
\pgfpathcurveto{\pgfqpoint{4.899935in}{2.367321in}}{\pgfqpoint{4.905521in}{2.365007in}}{\pgfqpoint{4.911345in}{2.365007in}}%
\pgfpathlineto{\pgfqpoint{4.911345in}{2.365007in}}%
\pgfpathclose%
\pgfusepath{stroke,fill}%
\end{pgfscope}%
\begin{pgfscope}%
\pgfpathrectangle{\pgfqpoint{0.997489in}{0.528000in}}{\pgfqpoint{4.565023in}{3.696000in}}%
\pgfusepath{clip}%
\pgfsetbuttcap%
\pgfsetroundjoin%
\definecolor{currentfill}{rgb}{0.800000,0.800000,0.200000}%
\pgfsetfillcolor{currentfill}%
\pgfsetlinewidth{1.003750pt}%
\definecolor{currentstroke}{rgb}{0.800000,0.800000,0.200000}%
\pgfsetstrokecolor{currentstroke}%
\pgfsetdash{}{0pt}%
\pgfpathmoveto{\pgfqpoint{4.917268in}{2.456399in}}%
\pgfpathcurveto{\pgfqpoint{4.923092in}{2.456399in}}{\pgfqpoint{4.928678in}{2.458713in}}{\pgfqpoint{4.932796in}{2.462831in}}%
\pgfpathcurveto{\pgfqpoint{4.936914in}{2.466949in}}{\pgfqpoint{4.939228in}{2.472535in}}{\pgfqpoint{4.939228in}{2.478359in}}%
\pgfpathcurveto{\pgfqpoint{4.939228in}{2.484183in}}{\pgfqpoint{4.936914in}{2.489769in}}{\pgfqpoint{4.932796in}{2.493888in}}%
\pgfpathcurveto{\pgfqpoint{4.928678in}{2.498006in}}{\pgfqpoint{4.923092in}{2.500320in}}{\pgfqpoint{4.917268in}{2.500320in}}%
\pgfpathcurveto{\pgfqpoint{4.911444in}{2.500320in}}{\pgfqpoint{4.905858in}{2.498006in}}{\pgfqpoint{4.901740in}{2.493888in}}%
\pgfpathcurveto{\pgfqpoint{4.897621in}{2.489769in}}{\pgfqpoint{4.895308in}{2.484183in}}{\pgfqpoint{4.895308in}{2.478359in}}%
\pgfpathcurveto{\pgfqpoint{4.895308in}{2.472535in}}{\pgfqpoint{4.897621in}{2.466949in}}{\pgfqpoint{4.901740in}{2.462831in}}%
\pgfpathcurveto{\pgfqpoint{4.905858in}{2.458713in}}{\pgfqpoint{4.911444in}{2.456399in}}{\pgfqpoint{4.917268in}{2.456399in}}%
\pgfpathlineto{\pgfqpoint{4.917268in}{2.456399in}}%
\pgfpathclose%
\pgfusepath{stroke,fill}%
\end{pgfscope}%
\begin{pgfscope}%
\pgfpathrectangle{\pgfqpoint{0.997489in}{0.528000in}}{\pgfqpoint{4.565023in}{3.696000in}}%
\pgfusepath{clip}%
\pgfsetbuttcap%
\pgfsetroundjoin%
\definecolor{currentfill}{rgb}{0.800000,0.800000,0.200000}%
\pgfsetfillcolor{currentfill}%
\pgfsetlinewidth{1.003750pt}%
\definecolor{currentstroke}{rgb}{0.800000,0.800000,0.200000}%
\pgfsetstrokecolor{currentstroke}%
\pgfsetdash{}{0pt}%
\pgfpathmoveto{\pgfqpoint{4.947508in}{2.503574in}}%
\pgfpathcurveto{\pgfqpoint{4.953332in}{2.503574in}}{\pgfqpoint{4.958918in}{2.505888in}}{\pgfqpoint{4.963036in}{2.510006in}}%
\pgfpathcurveto{\pgfqpoint{4.967154in}{2.514124in}}{\pgfqpoint{4.969468in}{2.519710in}}{\pgfqpoint{4.969468in}{2.525534in}}%
\pgfpathcurveto{\pgfqpoint{4.969468in}{2.531358in}}{\pgfqpoint{4.967154in}{2.536944in}}{\pgfqpoint{4.963036in}{2.541062in}}%
\pgfpathcurveto{\pgfqpoint{4.958918in}{2.545181in}}{\pgfqpoint{4.953332in}{2.547494in}}{\pgfqpoint{4.947508in}{2.547494in}}%
\pgfpathcurveto{\pgfqpoint{4.941684in}{2.547494in}}{\pgfqpoint{4.936098in}{2.545181in}}{\pgfqpoint{4.931980in}{2.541062in}}%
\pgfpathcurveto{\pgfqpoint{4.927861in}{2.536944in}}{\pgfqpoint{4.925548in}{2.531358in}}{\pgfqpoint{4.925548in}{2.525534in}}%
\pgfpathcurveto{\pgfqpoint{4.925548in}{2.519710in}}{\pgfqpoint{4.927861in}{2.514124in}}{\pgfqpoint{4.931980in}{2.510006in}}%
\pgfpathcurveto{\pgfqpoint{4.936098in}{2.505888in}}{\pgfqpoint{4.941684in}{2.503574in}}{\pgfqpoint{4.947508in}{2.503574in}}%
\pgfpathlineto{\pgfqpoint{4.947508in}{2.503574in}}%
\pgfpathclose%
\pgfusepath{stroke,fill}%
\end{pgfscope}%
\begin{pgfscope}%
\pgfpathrectangle{\pgfqpoint{0.997489in}{0.528000in}}{\pgfqpoint{4.565023in}{3.696000in}}%
\pgfusepath{clip}%
\pgfsetbuttcap%
\pgfsetroundjoin%
\definecolor{currentfill}{rgb}{0.800000,0.800000,0.200000}%
\pgfsetfillcolor{currentfill}%
\pgfsetlinewidth{1.003750pt}%
\definecolor{currentstroke}{rgb}{0.800000,0.800000,0.200000}%
\pgfsetstrokecolor{currentstroke}%
\pgfsetdash{}{0pt}%
\pgfpathmoveto{\pgfqpoint{5.015209in}{2.506911in}}%
\pgfpathcurveto{\pgfqpoint{5.021033in}{2.506911in}}{\pgfqpoint{5.026619in}{2.509224in}}{\pgfqpoint{5.030738in}{2.513343in}}%
\pgfpathcurveto{\pgfqpoint{5.034856in}{2.517461in}}{\pgfqpoint{5.037170in}{2.523047in}}{\pgfqpoint{5.037170in}{2.528871in}}%
\pgfpathcurveto{\pgfqpoint{5.037170in}{2.534695in}}{\pgfqpoint{5.034856in}{2.540281in}}{\pgfqpoint{5.030738in}{2.544399in}}%
\pgfpathcurveto{\pgfqpoint{5.026619in}{2.548517in}}{\pgfqpoint{5.021033in}{2.550831in}}{\pgfqpoint{5.015209in}{2.550831in}}%
\pgfpathcurveto{\pgfqpoint{5.009385in}{2.550831in}}{\pgfqpoint{5.003799in}{2.548517in}}{\pgfqpoint{4.999681in}{2.544399in}}%
\pgfpathcurveto{\pgfqpoint{4.995563in}{2.540281in}}{\pgfqpoint{4.993249in}{2.534695in}}{\pgfqpoint{4.993249in}{2.528871in}}%
\pgfpathcurveto{\pgfqpoint{4.993249in}{2.523047in}}{\pgfqpoint{4.995563in}{2.517461in}}{\pgfqpoint{4.999681in}{2.513343in}}%
\pgfpathcurveto{\pgfqpoint{5.003799in}{2.509224in}}{\pgfqpoint{5.009385in}{2.506911in}}{\pgfqpoint{5.015209in}{2.506911in}}%
\pgfpathlineto{\pgfqpoint{5.015209in}{2.506911in}}%
\pgfpathclose%
\pgfusepath{stroke,fill}%
\end{pgfscope}%
\begin{pgfscope}%
\pgfpathrectangle{\pgfqpoint{0.997489in}{0.528000in}}{\pgfqpoint{4.565023in}{3.696000in}}%
\pgfusepath{clip}%
\pgfsetbuttcap%
\pgfsetroundjoin%
\definecolor{currentfill}{rgb}{0.800000,0.800000,0.200000}%
\pgfsetfillcolor{currentfill}%
\pgfsetlinewidth{1.003750pt}%
\definecolor{currentstroke}{rgb}{0.800000,0.800000,0.200000}%
\pgfsetstrokecolor{currentstroke}%
\pgfsetdash{}{0pt}%
\pgfpathmoveto{\pgfqpoint{5.104948in}{2.496856in}}%
\pgfpathcurveto{\pgfqpoint{5.110772in}{2.496856in}}{\pgfqpoint{5.116358in}{2.499170in}}{\pgfqpoint{5.120476in}{2.503288in}}%
\pgfpathcurveto{\pgfqpoint{5.124594in}{2.507407in}}{\pgfqpoint{5.126908in}{2.512993in}}{\pgfqpoint{5.126908in}{2.518817in}}%
\pgfpathcurveto{\pgfqpoint{5.126908in}{2.524641in}}{\pgfqpoint{5.124594in}{2.530227in}}{\pgfqpoint{5.120476in}{2.534345in}}%
\pgfpathcurveto{\pgfqpoint{5.116358in}{2.538463in}}{\pgfqpoint{5.110772in}{2.540777in}}{\pgfqpoint{5.104948in}{2.540777in}}%
\pgfpathcurveto{\pgfqpoint{5.099124in}{2.540777in}}{\pgfqpoint{5.093538in}{2.538463in}}{\pgfqpoint{5.089420in}{2.534345in}}%
\pgfpathcurveto{\pgfqpoint{5.085302in}{2.530227in}}{\pgfqpoint{5.082988in}{2.524641in}}{\pgfqpoint{5.082988in}{2.518817in}}%
\pgfpathcurveto{\pgfqpoint{5.082988in}{2.512993in}}{\pgfqpoint{5.085302in}{2.507407in}}{\pgfqpoint{5.089420in}{2.503288in}}%
\pgfpathcurveto{\pgfqpoint{5.093538in}{2.499170in}}{\pgfqpoint{5.099124in}{2.496856in}}{\pgfqpoint{5.104948in}{2.496856in}}%
\pgfpathlineto{\pgfqpoint{5.104948in}{2.496856in}}%
\pgfpathclose%
\pgfusepath{stroke,fill}%
\end{pgfscope}%
\begin{pgfscope}%
\pgfpathrectangle{\pgfqpoint{0.997489in}{0.528000in}}{\pgfqpoint{4.565023in}{3.696000in}}%
\pgfusepath{clip}%
\pgfsetbuttcap%
\pgfsetroundjoin%
\definecolor{currentfill}{rgb}{0.800000,0.800000,0.200000}%
\pgfsetfillcolor{currentfill}%
\pgfsetlinewidth{1.003750pt}%
\definecolor{currentstroke}{rgb}{0.800000,0.800000,0.200000}%
\pgfsetstrokecolor{currentstroke}%
\pgfsetdash{}{0pt}%
\pgfpathmoveto{\pgfqpoint{5.093863in}{2.583307in}}%
\pgfpathcurveto{\pgfqpoint{5.099687in}{2.583307in}}{\pgfqpoint{5.105273in}{2.585621in}}{\pgfqpoint{5.109391in}{2.589739in}}%
\pgfpathcurveto{\pgfqpoint{5.113509in}{2.593857in}}{\pgfqpoint{5.115823in}{2.599443in}}{\pgfqpoint{5.115823in}{2.605267in}}%
\pgfpathcurveto{\pgfqpoint{5.115823in}{2.611091in}}{\pgfqpoint{5.113509in}{2.616677in}}{\pgfqpoint{5.109391in}{2.620795in}}%
\pgfpathcurveto{\pgfqpoint{5.105273in}{2.624913in}}{\pgfqpoint{5.099687in}{2.627227in}}{\pgfqpoint{5.093863in}{2.627227in}}%
\pgfpathcurveto{\pgfqpoint{5.088039in}{2.627227in}}{\pgfqpoint{5.082453in}{2.624913in}}{\pgfqpoint{5.078334in}{2.620795in}}%
\pgfpathcurveto{\pgfqpoint{5.074216in}{2.616677in}}{\pgfqpoint{5.071902in}{2.611091in}}{\pgfqpoint{5.071902in}{2.605267in}}%
\pgfpathcurveto{\pgfqpoint{5.071902in}{2.599443in}}{\pgfqpoint{5.074216in}{2.593857in}}{\pgfqpoint{5.078334in}{2.589739in}}%
\pgfpathcurveto{\pgfqpoint{5.082453in}{2.585621in}}{\pgfqpoint{5.088039in}{2.583307in}}{\pgfqpoint{5.093863in}{2.583307in}}%
\pgfpathlineto{\pgfqpoint{5.093863in}{2.583307in}}%
\pgfpathclose%
\pgfusepath{stroke,fill}%
\end{pgfscope}%
\begin{pgfscope}%
\pgfpathrectangle{\pgfqpoint{0.997489in}{0.528000in}}{\pgfqpoint{4.565023in}{3.696000in}}%
\pgfusepath{clip}%
\pgfsetbuttcap%
\pgfsetroundjoin%
\definecolor{currentfill}{rgb}{0.800000,0.800000,0.200000}%
\pgfsetfillcolor{currentfill}%
\pgfsetlinewidth{1.003750pt}%
\definecolor{currentstroke}{rgb}{0.800000,0.800000,0.200000}%
\pgfsetstrokecolor{currentstroke}%
\pgfsetdash{}{0pt}%
\pgfpathmoveto{\pgfqpoint{5.117210in}{2.634314in}}%
\pgfpathcurveto{\pgfqpoint{5.123034in}{2.634314in}}{\pgfqpoint{5.128620in}{2.636628in}}{\pgfqpoint{5.132738in}{2.640746in}}%
\pgfpathcurveto{\pgfqpoint{5.136856in}{2.644864in}}{\pgfqpoint{5.139170in}{2.650450in}}{\pgfqpoint{5.139170in}{2.656274in}}%
\pgfpathcurveto{\pgfqpoint{5.139170in}{2.662098in}}{\pgfqpoint{5.136856in}{2.667684in}}{\pgfqpoint{5.132738in}{2.671802in}}%
\pgfpathcurveto{\pgfqpoint{5.128620in}{2.675921in}}{\pgfqpoint{5.123034in}{2.678234in}}{\pgfqpoint{5.117210in}{2.678234in}}%
\pgfpathcurveto{\pgfqpoint{5.111386in}{2.678234in}}{\pgfqpoint{5.105800in}{2.675921in}}{\pgfqpoint{5.101682in}{2.671802in}}%
\pgfpathcurveto{\pgfqpoint{5.097564in}{2.667684in}}{\pgfqpoint{5.095250in}{2.662098in}}{\pgfqpoint{5.095250in}{2.656274in}}%
\pgfpathcurveto{\pgfqpoint{5.095250in}{2.650450in}}{\pgfqpoint{5.097564in}{2.644864in}}{\pgfqpoint{5.101682in}{2.640746in}}%
\pgfpathcurveto{\pgfqpoint{5.105800in}{2.636628in}}{\pgfqpoint{5.111386in}{2.634314in}}{\pgfqpoint{5.117210in}{2.634314in}}%
\pgfpathlineto{\pgfqpoint{5.117210in}{2.634314in}}%
\pgfpathclose%
\pgfusepath{stroke,fill}%
\end{pgfscope}%
\begin{pgfscope}%
\pgfpathrectangle{\pgfqpoint{0.997489in}{0.528000in}}{\pgfqpoint{4.565023in}{3.696000in}}%
\pgfusepath{clip}%
\pgfsetbuttcap%
\pgfsetroundjoin%
\definecolor{currentfill}{rgb}{0.800000,0.800000,0.200000}%
\pgfsetfillcolor{currentfill}%
\pgfsetlinewidth{1.003750pt}%
\definecolor{currentstroke}{rgb}{0.800000,0.800000,0.200000}%
\pgfsetstrokecolor{currentstroke}%
\pgfsetdash{}{0pt}%
\pgfpathmoveto{\pgfqpoint{5.121560in}{2.695805in}}%
\pgfpathcurveto{\pgfqpoint{5.127384in}{2.695805in}}{\pgfqpoint{5.132971in}{2.698118in}}{\pgfqpoint{5.137089in}{2.702237in}}%
\pgfpathcurveto{\pgfqpoint{5.141207in}{2.706355in}}{\pgfqpoint{5.143521in}{2.711941in}}{\pgfqpoint{5.143521in}{2.717765in}}%
\pgfpathcurveto{\pgfqpoint{5.143521in}{2.723589in}}{\pgfqpoint{5.141207in}{2.729175in}}{\pgfqpoint{5.137089in}{2.733293in}}%
\pgfpathcurveto{\pgfqpoint{5.132971in}{2.737411in}}{\pgfqpoint{5.127384in}{2.739725in}}{\pgfqpoint{5.121560in}{2.739725in}}%
\pgfpathcurveto{\pgfqpoint{5.115737in}{2.739725in}}{\pgfqpoint{5.110150in}{2.737411in}}{\pgfqpoint{5.106032in}{2.733293in}}%
\pgfpathcurveto{\pgfqpoint{5.101914in}{2.729175in}}{\pgfqpoint{5.099600in}{2.723589in}}{\pgfqpoint{5.099600in}{2.717765in}}%
\pgfpathcurveto{\pgfqpoint{5.099600in}{2.711941in}}{\pgfqpoint{5.101914in}{2.706355in}}{\pgfqpoint{5.106032in}{2.702237in}}%
\pgfpathcurveto{\pgfqpoint{5.110150in}{2.698118in}}{\pgfqpoint{5.115737in}{2.695805in}}{\pgfqpoint{5.121560in}{2.695805in}}%
\pgfpathlineto{\pgfqpoint{5.121560in}{2.695805in}}%
\pgfpathclose%
\pgfusepath{stroke,fill}%
\end{pgfscope}%
\begin{pgfscope}%
\pgfpathrectangle{\pgfqpoint{0.997489in}{0.528000in}}{\pgfqpoint{4.565023in}{3.696000in}}%
\pgfusepath{clip}%
\pgfsetbuttcap%
\pgfsetroundjoin%
\definecolor{currentfill}{rgb}{0.800000,0.800000,0.200000}%
\pgfsetfillcolor{currentfill}%
\pgfsetlinewidth{1.003750pt}%
\definecolor{currentstroke}{rgb}{0.800000,0.800000,0.200000}%
\pgfsetstrokecolor{currentstroke}%
\pgfsetdash{}{0pt}%
\pgfpathmoveto{\pgfqpoint{5.117638in}{2.757604in}}%
\pgfpathcurveto{\pgfqpoint{5.123462in}{2.757604in}}{\pgfqpoint{5.129048in}{2.759918in}}{\pgfqpoint{5.133166in}{2.764036in}}%
\pgfpathcurveto{\pgfqpoint{5.137284in}{2.768154in}}{\pgfqpoint{5.139598in}{2.773740in}}{\pgfqpoint{5.139598in}{2.779564in}}%
\pgfpathcurveto{\pgfqpoint{5.139598in}{2.785388in}}{\pgfqpoint{5.137284in}{2.790974in}}{\pgfqpoint{5.133166in}{2.795092in}}%
\pgfpathcurveto{\pgfqpoint{5.129048in}{2.799210in}}{\pgfqpoint{5.123462in}{2.801524in}}{\pgfqpoint{5.117638in}{2.801524in}}%
\pgfpathcurveto{\pgfqpoint{5.111814in}{2.801524in}}{\pgfqpoint{5.106227in}{2.799210in}}{\pgfqpoint{5.102109in}{2.795092in}}%
\pgfpathcurveto{\pgfqpoint{5.097991in}{2.790974in}}{\pgfqpoint{5.095677in}{2.785388in}}{\pgfqpoint{5.095677in}{2.779564in}}%
\pgfpathcurveto{\pgfqpoint{5.095677in}{2.773740in}}{\pgfqpoint{5.097991in}{2.768154in}}{\pgfqpoint{5.102109in}{2.764036in}}%
\pgfpathcurveto{\pgfqpoint{5.106227in}{2.759918in}}{\pgfqpoint{5.111814in}{2.757604in}}{\pgfqpoint{5.117638in}{2.757604in}}%
\pgfpathlineto{\pgfqpoint{5.117638in}{2.757604in}}%
\pgfpathclose%
\pgfusepath{stroke,fill}%
\end{pgfscope}%
\begin{pgfscope}%
\pgfpathrectangle{\pgfqpoint{0.997489in}{0.528000in}}{\pgfqpoint{4.565023in}{3.696000in}}%
\pgfusepath{clip}%
\pgfsetbuttcap%
\pgfsetroundjoin%
\definecolor{currentfill}{rgb}{0.800000,0.800000,0.200000}%
\pgfsetfillcolor{currentfill}%
\pgfsetlinewidth{1.003750pt}%
\definecolor{currentstroke}{rgb}{0.800000,0.800000,0.200000}%
\pgfsetstrokecolor{currentstroke}%
\pgfsetdash{}{0pt}%
\pgfpathmoveto{\pgfqpoint{5.191559in}{2.777646in}}%
\pgfpathcurveto{\pgfqpoint{5.197383in}{2.777646in}}{\pgfqpoint{5.202969in}{2.779960in}}{\pgfqpoint{5.207087in}{2.784078in}}%
\pgfpathcurveto{\pgfqpoint{5.211206in}{2.788196in}}{\pgfqpoint{5.213519in}{2.793782in}}{\pgfqpoint{5.213519in}{2.799606in}}%
\pgfpathcurveto{\pgfqpoint{5.213519in}{2.805430in}}{\pgfqpoint{5.211206in}{2.811016in}}{\pgfqpoint{5.207087in}{2.815135in}}%
\pgfpathcurveto{\pgfqpoint{5.202969in}{2.819253in}}{\pgfqpoint{5.197383in}{2.821567in}}{\pgfqpoint{5.191559in}{2.821567in}}%
\pgfpathcurveto{\pgfqpoint{5.185735in}{2.821567in}}{\pgfqpoint{5.180149in}{2.819253in}}{\pgfqpoint{5.176031in}{2.815135in}}%
\pgfpathcurveto{\pgfqpoint{5.171913in}{2.811016in}}{\pgfqpoint{5.169599in}{2.805430in}}{\pgfqpoint{5.169599in}{2.799606in}}%
\pgfpathcurveto{\pgfqpoint{5.169599in}{2.793782in}}{\pgfqpoint{5.171913in}{2.788196in}}{\pgfqpoint{5.176031in}{2.784078in}}%
\pgfpathcurveto{\pgfqpoint{5.180149in}{2.779960in}}{\pgfqpoint{5.185735in}{2.777646in}}{\pgfqpoint{5.191559in}{2.777646in}}%
\pgfpathlineto{\pgfqpoint{5.191559in}{2.777646in}}%
\pgfpathclose%
\pgfusepath{stroke,fill}%
\end{pgfscope}%
\begin{pgfscope}%
\pgfpathrectangle{\pgfqpoint{0.997489in}{0.528000in}}{\pgfqpoint{4.565023in}{3.696000in}}%
\pgfusepath{clip}%
\pgfsetbuttcap%
\pgfsetroundjoin%
\definecolor{currentfill}{rgb}{0.800000,0.800000,0.200000}%
\pgfsetfillcolor{currentfill}%
\pgfsetlinewidth{1.003750pt}%
\definecolor{currentstroke}{rgb}{0.800000,0.800000,0.200000}%
\pgfsetstrokecolor{currentstroke}%
\pgfsetdash{}{0pt}%
\pgfpathmoveto{\pgfqpoint{5.213874in}{2.826291in}}%
\pgfpathcurveto{\pgfqpoint{5.219698in}{2.826291in}}{\pgfqpoint{5.225284in}{2.828605in}}{\pgfqpoint{5.229403in}{2.832723in}}%
\pgfpathcurveto{\pgfqpoint{5.233521in}{2.836842in}}{\pgfqpoint{5.235835in}{2.842428in}}{\pgfqpoint{5.235835in}{2.848252in}}%
\pgfpathcurveto{\pgfqpoint{5.235835in}{2.854076in}}{\pgfqpoint{5.233521in}{2.859662in}}{\pgfqpoint{5.229403in}{2.863780in}}%
\pgfpathcurveto{\pgfqpoint{5.225284in}{2.867898in}}{\pgfqpoint{5.219698in}{2.870212in}}{\pgfqpoint{5.213874in}{2.870212in}}%
\pgfpathcurveto{\pgfqpoint{5.208050in}{2.870212in}}{\pgfqpoint{5.202464in}{2.867898in}}{\pgfqpoint{5.198346in}{2.863780in}}%
\pgfpathcurveto{\pgfqpoint{5.194228in}{2.859662in}}{\pgfqpoint{5.191914in}{2.854076in}}{\pgfqpoint{5.191914in}{2.848252in}}%
\pgfpathcurveto{\pgfqpoint{5.191914in}{2.842428in}}{\pgfqpoint{5.194228in}{2.836842in}}{\pgfqpoint{5.198346in}{2.832723in}}%
\pgfpathcurveto{\pgfqpoint{5.202464in}{2.828605in}}{\pgfqpoint{5.208050in}{2.826291in}}{\pgfqpoint{5.213874in}{2.826291in}}%
\pgfpathlineto{\pgfqpoint{5.213874in}{2.826291in}}%
\pgfpathclose%
\pgfusepath{stroke,fill}%
\end{pgfscope}%
\begin{pgfscope}%
\pgfpathrectangle{\pgfqpoint{0.997489in}{0.528000in}}{\pgfqpoint{4.565023in}{3.696000in}}%
\pgfusepath{clip}%
\pgfsetbuttcap%
\pgfsetroundjoin%
\definecolor{currentfill}{rgb}{0.800000,0.800000,0.200000}%
\pgfsetfillcolor{currentfill}%
\pgfsetlinewidth{1.003750pt}%
\definecolor{currentstroke}{rgb}{0.800000,0.800000,0.200000}%
\pgfsetstrokecolor{currentstroke}%
\pgfsetdash{}{0pt}%
\pgfpathmoveto{\pgfqpoint{5.135164in}{2.908469in}}%
\pgfpathcurveto{\pgfqpoint{5.140988in}{2.908469in}}{\pgfqpoint{5.146574in}{2.910783in}}{\pgfqpoint{5.150692in}{2.914901in}}%
\pgfpathcurveto{\pgfqpoint{5.154810in}{2.919019in}}{\pgfqpoint{5.157124in}{2.924605in}}{\pgfqpoint{5.157124in}{2.930429in}}%
\pgfpathcurveto{\pgfqpoint{5.157124in}{2.936253in}}{\pgfqpoint{5.154810in}{2.941839in}}{\pgfqpoint{5.150692in}{2.945957in}}%
\pgfpathcurveto{\pgfqpoint{5.146574in}{2.950075in}}{\pgfqpoint{5.140988in}{2.952389in}}{\pgfqpoint{5.135164in}{2.952389in}}%
\pgfpathcurveto{\pgfqpoint{5.129340in}{2.952389in}}{\pgfqpoint{5.123754in}{2.950075in}}{\pgfqpoint{5.119636in}{2.945957in}}%
\pgfpathcurveto{\pgfqpoint{5.115518in}{2.941839in}}{\pgfqpoint{5.113204in}{2.936253in}}{\pgfqpoint{5.113204in}{2.930429in}}%
\pgfpathcurveto{\pgfqpoint{5.113204in}{2.924605in}}{\pgfqpoint{5.115518in}{2.919019in}}{\pgfqpoint{5.119636in}{2.914901in}}%
\pgfpathcurveto{\pgfqpoint{5.123754in}{2.910783in}}{\pgfqpoint{5.129340in}{2.908469in}}{\pgfqpoint{5.135164in}{2.908469in}}%
\pgfpathlineto{\pgfqpoint{5.135164in}{2.908469in}}%
\pgfpathclose%
\pgfusepath{stroke,fill}%
\end{pgfscope}%
\begin{pgfscope}%
\pgfpathrectangle{\pgfqpoint{0.997489in}{0.528000in}}{\pgfqpoint{4.565023in}{3.696000in}}%
\pgfusepath{clip}%
\pgfsetbuttcap%
\pgfsetroundjoin%
\definecolor{currentfill}{rgb}{0.800000,0.800000,0.200000}%
\pgfsetfillcolor{currentfill}%
\pgfsetlinewidth{1.003750pt}%
\definecolor{currentstroke}{rgb}{0.800000,0.800000,0.200000}%
\pgfsetstrokecolor{currentstroke}%
\pgfsetdash{}{0pt}%
\pgfpathmoveto{\pgfqpoint{5.208789in}{2.937961in}}%
\pgfpathcurveto{\pgfqpoint{5.214613in}{2.937961in}}{\pgfqpoint{5.220199in}{2.940274in}}{\pgfqpoint{5.224317in}{2.944393in}}%
\pgfpathcurveto{\pgfqpoint{5.228435in}{2.948511in}}{\pgfqpoint{5.230749in}{2.954097in}}{\pgfqpoint{5.230749in}{2.959921in}}%
\pgfpathcurveto{\pgfqpoint{5.230749in}{2.965745in}}{\pgfqpoint{5.228435in}{2.971331in}}{\pgfqpoint{5.224317in}{2.975449in}}%
\pgfpathcurveto{\pgfqpoint{5.220199in}{2.979567in}}{\pgfqpoint{5.214613in}{2.981881in}}{\pgfqpoint{5.208789in}{2.981881in}}%
\pgfpathcurveto{\pgfqpoint{5.202965in}{2.981881in}}{\pgfqpoint{5.197379in}{2.979567in}}{\pgfqpoint{5.193260in}{2.975449in}}%
\pgfpathcurveto{\pgfqpoint{5.189142in}{2.971331in}}{\pgfqpoint{5.186828in}{2.965745in}}{\pgfqpoint{5.186828in}{2.959921in}}%
\pgfpathcurveto{\pgfqpoint{5.186828in}{2.954097in}}{\pgfqpoint{5.189142in}{2.948511in}}{\pgfqpoint{5.193260in}{2.944393in}}%
\pgfpathcurveto{\pgfqpoint{5.197379in}{2.940274in}}{\pgfqpoint{5.202965in}{2.937961in}}{\pgfqpoint{5.208789in}{2.937961in}}%
\pgfpathlineto{\pgfqpoint{5.208789in}{2.937961in}}%
\pgfpathclose%
\pgfusepath{stroke,fill}%
\end{pgfscope}%
\begin{pgfscope}%
\pgfpathrectangle{\pgfqpoint{0.997489in}{0.528000in}}{\pgfqpoint{4.565023in}{3.696000in}}%
\pgfusepath{clip}%
\pgfsetbuttcap%
\pgfsetroundjoin%
\definecolor{currentfill}{rgb}{0.800000,0.800000,0.200000}%
\pgfsetfillcolor{currentfill}%
\pgfsetlinewidth{1.003750pt}%
\definecolor{currentstroke}{rgb}{0.800000,0.800000,0.200000}%
\pgfsetstrokecolor{currentstroke}%
\pgfsetdash{}{0pt}%
\pgfpathmoveto{\pgfqpoint{5.315832in}{2.969261in}}%
\pgfpathcurveto{\pgfqpoint{5.321656in}{2.969261in}}{\pgfqpoint{5.327242in}{2.971575in}}{\pgfqpoint{5.331360in}{2.975693in}}%
\pgfpathcurveto{\pgfqpoint{5.335478in}{2.979811in}}{\pgfqpoint{5.337792in}{2.985397in}}{\pgfqpoint{5.337792in}{2.991221in}}%
\pgfpathcurveto{\pgfqpoint{5.337792in}{2.997045in}}{\pgfqpoint{5.335478in}{3.002631in}}{\pgfqpoint{5.331360in}{3.006749in}}%
\pgfpathcurveto{\pgfqpoint{5.327242in}{3.010867in}}{\pgfqpoint{5.321656in}{3.013181in}}{\pgfqpoint{5.315832in}{3.013181in}}%
\pgfpathcurveto{\pgfqpoint{5.310008in}{3.013181in}}{\pgfqpoint{5.304422in}{3.010867in}}{\pgfqpoint{5.300304in}{3.006749in}}%
\pgfpathcurveto{\pgfqpoint{5.296185in}{3.002631in}}{\pgfqpoint{5.293872in}{2.997045in}}{\pgfqpoint{5.293872in}{2.991221in}}%
\pgfpathcurveto{\pgfqpoint{5.293872in}{2.985397in}}{\pgfqpoint{5.296185in}{2.979811in}}{\pgfqpoint{5.300304in}{2.975693in}}%
\pgfpathcurveto{\pgfqpoint{5.304422in}{2.971575in}}{\pgfqpoint{5.310008in}{2.969261in}}{\pgfqpoint{5.315832in}{2.969261in}}%
\pgfpathlineto{\pgfqpoint{5.315832in}{2.969261in}}%
\pgfpathclose%
\pgfusepath{stroke,fill}%
\end{pgfscope}%
\begin{pgfscope}%
\pgfpathrectangle{\pgfqpoint{0.997489in}{0.528000in}}{\pgfqpoint{4.565023in}{3.696000in}}%
\pgfusepath{clip}%
\pgfsetbuttcap%
\pgfsetroundjoin%
\definecolor{currentfill}{rgb}{0.800000,0.800000,0.200000}%
\pgfsetfillcolor{currentfill}%
\pgfsetlinewidth{1.003750pt}%
\definecolor{currentstroke}{rgb}{0.800000,0.800000,0.200000}%
\pgfsetstrokecolor{currentstroke}%
\pgfsetdash{}{0pt}%
\pgfpathmoveto{\pgfqpoint{5.252146in}{3.035033in}}%
\pgfpathcurveto{\pgfqpoint{5.257970in}{3.035033in}}{\pgfqpoint{5.263557in}{3.037347in}}{\pgfqpoint{5.267675in}{3.041465in}}%
\pgfpathcurveto{\pgfqpoint{5.271793in}{3.045583in}}{\pgfqpoint{5.274107in}{3.051169in}}{\pgfqpoint{5.274107in}{3.056993in}}%
\pgfpathcurveto{\pgfqpoint{5.274107in}{3.062817in}}{\pgfqpoint{5.271793in}{3.068403in}}{\pgfqpoint{5.267675in}{3.072521in}}%
\pgfpathcurveto{\pgfqpoint{5.263557in}{3.076639in}}{\pgfqpoint{5.257970in}{3.078953in}}{\pgfqpoint{5.252146in}{3.078953in}}%
\pgfpathcurveto{\pgfqpoint{5.246323in}{3.078953in}}{\pgfqpoint{5.240736in}{3.076639in}}{\pgfqpoint{5.236618in}{3.072521in}}%
\pgfpathcurveto{\pgfqpoint{5.232500in}{3.068403in}}{\pgfqpoint{5.230186in}{3.062817in}}{\pgfqpoint{5.230186in}{3.056993in}}%
\pgfpathcurveto{\pgfqpoint{5.230186in}{3.051169in}}{\pgfqpoint{5.232500in}{3.045583in}}{\pgfqpoint{5.236618in}{3.041465in}}%
\pgfpathcurveto{\pgfqpoint{5.240736in}{3.037347in}}{\pgfqpoint{5.246323in}{3.035033in}}{\pgfqpoint{5.252146in}{3.035033in}}%
\pgfpathlineto{\pgfqpoint{5.252146in}{3.035033in}}%
\pgfpathclose%
\pgfusepath{stroke,fill}%
\end{pgfscope}%
\begin{pgfscope}%
\pgfpathrectangle{\pgfqpoint{0.997489in}{0.528000in}}{\pgfqpoint{4.565023in}{3.696000in}}%
\pgfusepath{clip}%
\pgfsetbuttcap%
\pgfsetroundjoin%
\definecolor{currentfill}{rgb}{0.800000,0.800000,0.200000}%
\pgfsetfillcolor{currentfill}%
\pgfsetlinewidth{1.003750pt}%
\definecolor{currentstroke}{rgb}{0.800000,0.800000,0.200000}%
\pgfsetstrokecolor{currentstroke}%
\pgfsetdash{}{0pt}%
\pgfpathmoveto{\pgfqpoint{5.143656in}{3.094564in}}%
\pgfpathcurveto{\pgfqpoint{5.149480in}{3.094564in}}{\pgfqpoint{5.155066in}{3.096878in}}{\pgfqpoint{5.159184in}{3.100996in}}%
\pgfpathcurveto{\pgfqpoint{5.163302in}{3.105114in}}{\pgfqpoint{5.165616in}{3.110701in}}{\pgfqpoint{5.165616in}{3.116524in}}%
\pgfpathcurveto{\pgfqpoint{5.165616in}{3.122348in}}{\pgfqpoint{5.163302in}{3.127935in}}{\pgfqpoint{5.159184in}{3.132053in}}%
\pgfpathcurveto{\pgfqpoint{5.155066in}{3.136171in}}{\pgfqpoint{5.149480in}{3.138485in}}{\pgfqpoint{5.143656in}{3.138485in}}%
\pgfpathcurveto{\pgfqpoint{5.137832in}{3.138485in}}{\pgfqpoint{5.132246in}{3.136171in}}{\pgfqpoint{5.128127in}{3.132053in}}%
\pgfpathcurveto{\pgfqpoint{5.124009in}{3.127935in}}{\pgfqpoint{5.121695in}{3.122348in}}{\pgfqpoint{5.121695in}{3.116524in}}%
\pgfpathcurveto{\pgfqpoint{5.121695in}{3.110701in}}{\pgfqpoint{5.124009in}{3.105114in}}{\pgfqpoint{5.128127in}{3.100996in}}%
\pgfpathcurveto{\pgfqpoint{5.132246in}{3.096878in}}{\pgfqpoint{5.137832in}{3.094564in}}{\pgfqpoint{5.143656in}{3.094564in}}%
\pgfpathlineto{\pgfqpoint{5.143656in}{3.094564in}}%
\pgfpathclose%
\pgfusepath{stroke,fill}%
\end{pgfscope}%
\begin{pgfscope}%
\pgfpathrectangle{\pgfqpoint{0.997489in}{0.528000in}}{\pgfqpoint{4.565023in}{3.696000in}}%
\pgfusepath{clip}%
\pgfsetbuttcap%
\pgfsetroundjoin%
\definecolor{currentfill}{rgb}{0.800000,0.800000,0.200000}%
\pgfsetfillcolor{currentfill}%
\pgfsetlinewidth{1.003750pt}%
\definecolor{currentstroke}{rgb}{0.800000,0.800000,0.200000}%
\pgfsetstrokecolor{currentstroke}%
\pgfsetdash{}{0pt}%
\pgfpathmoveto{\pgfqpoint{5.355010in}{3.139883in}}%
\pgfpathcurveto{\pgfqpoint{5.360834in}{3.139883in}}{\pgfqpoint{5.366420in}{3.142197in}}{\pgfqpoint{5.370539in}{3.146315in}}%
\pgfpathcurveto{\pgfqpoint{5.374657in}{3.150433in}}{\pgfqpoint{5.376971in}{3.156019in}}{\pgfqpoint{5.376971in}{3.161843in}}%
\pgfpathcurveto{\pgfqpoint{5.376971in}{3.167667in}}{\pgfqpoint{5.374657in}{3.173253in}}{\pgfqpoint{5.370539in}{3.177371in}}%
\pgfpathcurveto{\pgfqpoint{5.366420in}{3.181489in}}{\pgfqpoint{5.360834in}{3.183803in}}{\pgfqpoint{5.355010in}{3.183803in}}%
\pgfpathcurveto{\pgfqpoint{5.349186in}{3.183803in}}{\pgfqpoint{5.343600in}{3.181489in}}{\pgfqpoint{5.339482in}{3.177371in}}%
\pgfpathcurveto{\pgfqpoint{5.335364in}{3.173253in}}{\pgfqpoint{5.333050in}{3.167667in}}{\pgfqpoint{5.333050in}{3.161843in}}%
\pgfpathcurveto{\pgfqpoint{5.333050in}{3.156019in}}{\pgfqpoint{5.335364in}{3.150433in}}{\pgfqpoint{5.339482in}{3.146315in}}%
\pgfpathcurveto{\pgfqpoint{5.343600in}{3.142197in}}{\pgfqpoint{5.349186in}{3.139883in}}{\pgfqpoint{5.355010in}{3.139883in}}%
\pgfpathlineto{\pgfqpoint{5.355010in}{3.139883in}}%
\pgfpathclose%
\pgfusepath{stroke,fill}%
\end{pgfscope}%
\begin{pgfscope}%
\pgfpathrectangle{\pgfqpoint{0.997489in}{0.528000in}}{\pgfqpoint{4.565023in}{3.696000in}}%
\pgfusepath{clip}%
\pgfsetbuttcap%
\pgfsetroundjoin%
\definecolor{currentfill}{rgb}{0.200000,0.200000,0.800000}%
\pgfsetfillcolor{currentfill}%
\pgfsetlinewidth{1.003750pt}%
\definecolor{currentstroke}{rgb}{0.200000,0.200000,0.800000}%
\pgfsetstrokecolor{currentstroke}%
\pgfsetdash{}{0pt}%
\pgfpathmoveto{\pgfqpoint{3.728347in}{2.549302in}}%
\pgfpathcurveto{\pgfqpoint{3.734170in}{2.549302in}}{\pgfqpoint{3.739757in}{2.551615in}}{\pgfqpoint{3.743875in}{2.555734in}}%
\pgfpathcurveto{\pgfqpoint{3.747993in}{2.559852in}}{\pgfqpoint{3.750307in}{2.565438in}}{\pgfqpoint{3.750307in}{2.571262in}}%
\pgfpathcurveto{\pgfqpoint{3.750307in}{2.577086in}}{\pgfqpoint{3.747993in}{2.582672in}}{\pgfqpoint{3.743875in}{2.586790in}}%
\pgfpathcurveto{\pgfqpoint{3.739757in}{2.590908in}}{\pgfqpoint{3.734170in}{2.593222in}}{\pgfqpoint{3.728347in}{2.593222in}}%
\pgfpathcurveto{\pgfqpoint{3.722523in}{2.593222in}}{\pgfqpoint{3.716936in}{2.590908in}}{\pgfqpoint{3.712818in}{2.586790in}}%
\pgfpathcurveto{\pgfqpoint{3.708700in}{2.582672in}}{\pgfqpoint{3.706386in}{2.577086in}}{\pgfqpoint{3.706386in}{2.571262in}}%
\pgfpathcurveto{\pgfqpoint{3.706386in}{2.565438in}}{\pgfqpoint{3.708700in}{2.559852in}}{\pgfqpoint{3.712818in}{2.555734in}}%
\pgfpathcurveto{\pgfqpoint{3.716936in}{2.551615in}}{\pgfqpoint{3.722523in}{2.549302in}}{\pgfqpoint{3.728347in}{2.549302in}}%
\pgfpathlineto{\pgfqpoint{3.728347in}{2.549302in}}%
\pgfpathclose%
\pgfusepath{stroke,fill}%
\end{pgfscope}%
\begin{pgfscope}%
\pgfpathrectangle{\pgfqpoint{0.997489in}{0.528000in}}{\pgfqpoint{4.565023in}{3.696000in}}%
\pgfusepath{clip}%
\pgfsetbuttcap%
\pgfsetroundjoin%
\definecolor{currentfill}{rgb}{0.800000,0.800000,0.200000}%
\pgfsetfillcolor{currentfill}%
\pgfsetlinewidth{1.003750pt}%
\definecolor{currentstroke}{rgb}{0.800000,0.800000,0.200000}%
\pgfsetstrokecolor{currentstroke}%
\pgfsetdash{}{0pt}%
\pgfpathmoveto{\pgfqpoint{3.801942in}{2.586549in}}%
\pgfpathcurveto{\pgfqpoint{3.807766in}{2.586549in}}{\pgfqpoint{3.813353in}{2.588863in}}{\pgfqpoint{3.817471in}{2.592981in}}%
\pgfpathcurveto{\pgfqpoint{3.821589in}{2.597099in}}{\pgfqpoint{3.823903in}{2.602686in}}{\pgfqpoint{3.823903in}{2.608509in}}%
\pgfpathcurveto{\pgfqpoint{3.823903in}{2.614333in}}{\pgfqpoint{3.821589in}{2.619920in}}{\pgfqpoint{3.817471in}{2.624038in}}%
\pgfpathcurveto{\pgfqpoint{3.813353in}{2.628156in}}{\pgfqpoint{3.807766in}{2.630470in}}{\pgfqpoint{3.801942in}{2.630470in}}%
\pgfpathcurveto{\pgfqpoint{3.796119in}{2.630470in}}{\pgfqpoint{3.790532in}{2.628156in}}{\pgfqpoint{3.786414in}{2.624038in}}%
\pgfpathcurveto{\pgfqpoint{3.782296in}{2.619920in}}{\pgfqpoint{3.779982in}{2.614333in}}{\pgfqpoint{3.779982in}{2.608509in}}%
\pgfpathcurveto{\pgfqpoint{3.779982in}{2.602686in}}{\pgfqpoint{3.782296in}{2.597099in}}{\pgfqpoint{3.786414in}{2.592981in}}%
\pgfpathcurveto{\pgfqpoint{3.790532in}{2.588863in}}{\pgfqpoint{3.796119in}{2.586549in}}{\pgfqpoint{3.801942in}{2.586549in}}%
\pgfpathlineto{\pgfqpoint{3.801942in}{2.586549in}}%
\pgfpathclose%
\pgfusepath{stroke,fill}%
\end{pgfscope}%
\begin{pgfscope}%
\pgfpathrectangle{\pgfqpoint{0.997489in}{0.528000in}}{\pgfqpoint{4.565023in}{3.696000in}}%
\pgfusepath{clip}%
\pgfsetbuttcap%
\pgfsetroundjoin%
\definecolor{currentfill}{rgb}{0.800000,0.800000,0.200000}%
\pgfsetfillcolor{currentfill}%
\pgfsetlinewidth{1.003750pt}%
\definecolor{currentstroke}{rgb}{0.800000,0.800000,0.200000}%
\pgfsetstrokecolor{currentstroke}%
\pgfsetdash{}{0pt}%
\pgfpathmoveto{\pgfqpoint{3.641428in}{2.603614in}}%
\pgfpathcurveto{\pgfqpoint{3.647252in}{2.603614in}}{\pgfqpoint{3.652838in}{2.605928in}}{\pgfqpoint{3.656956in}{2.610046in}}%
\pgfpathcurveto{\pgfqpoint{3.661074in}{2.614164in}}{\pgfqpoint{3.663388in}{2.619751in}}{\pgfqpoint{3.663388in}{2.625575in}}%
\pgfpathcurveto{\pgfqpoint{3.663388in}{2.631398in}}{\pgfqpoint{3.661074in}{2.636985in}}{\pgfqpoint{3.656956in}{2.641103in}}%
\pgfpathcurveto{\pgfqpoint{3.652838in}{2.645221in}}{\pgfqpoint{3.647252in}{2.647535in}}{\pgfqpoint{3.641428in}{2.647535in}}%
\pgfpathcurveto{\pgfqpoint{3.635604in}{2.647535in}}{\pgfqpoint{3.630018in}{2.645221in}}{\pgfqpoint{3.625899in}{2.641103in}}%
\pgfpathcurveto{\pgfqpoint{3.621781in}{2.636985in}}{\pgfqpoint{3.619467in}{2.631398in}}{\pgfqpoint{3.619467in}{2.625575in}}%
\pgfpathcurveto{\pgfqpoint{3.619467in}{2.619751in}}{\pgfqpoint{3.621781in}{2.614164in}}{\pgfqpoint{3.625899in}{2.610046in}}%
\pgfpathcurveto{\pgfqpoint{3.630018in}{2.605928in}}{\pgfqpoint{3.635604in}{2.603614in}}{\pgfqpoint{3.641428in}{2.603614in}}%
\pgfpathlineto{\pgfqpoint{3.641428in}{2.603614in}}%
\pgfpathclose%
\pgfusepath{stroke,fill}%
\end{pgfscope}%
\begin{pgfscope}%
\pgfpathrectangle{\pgfqpoint{0.997489in}{0.528000in}}{\pgfqpoint{4.565023in}{3.696000in}}%
\pgfusepath{clip}%
\pgfsetbuttcap%
\pgfsetroundjoin%
\definecolor{currentfill}{rgb}{0.200000,0.200000,0.800000}%
\pgfsetfillcolor{currentfill}%
\pgfsetlinewidth{1.003750pt}%
\definecolor{currentstroke}{rgb}{0.200000,0.200000,0.800000}%
\pgfsetstrokecolor{currentstroke}%
\pgfsetdash{}{0pt}%
\pgfpathmoveto{\pgfqpoint{3.757169in}{2.653634in}}%
\pgfpathcurveto{\pgfqpoint{3.762993in}{2.653634in}}{\pgfqpoint{3.768579in}{2.655947in}}{\pgfqpoint{3.772697in}{2.660066in}}%
\pgfpathcurveto{\pgfqpoint{3.776815in}{2.664184in}}{\pgfqpoint{3.779129in}{2.669770in}}{\pgfqpoint{3.779129in}{2.675594in}}%
\pgfpathcurveto{\pgfqpoint{3.779129in}{2.681418in}}{\pgfqpoint{3.776815in}{2.687004in}}{\pgfqpoint{3.772697in}{2.691122in}}%
\pgfpathcurveto{\pgfqpoint{3.768579in}{2.695240in}}{\pgfqpoint{3.762993in}{2.697554in}}{\pgfqpoint{3.757169in}{2.697554in}}%
\pgfpathcurveto{\pgfqpoint{3.751345in}{2.697554in}}{\pgfqpoint{3.745759in}{2.695240in}}{\pgfqpoint{3.741641in}{2.691122in}}%
\pgfpathcurveto{\pgfqpoint{3.737523in}{2.687004in}}{\pgfqpoint{3.735209in}{2.681418in}}{\pgfqpoint{3.735209in}{2.675594in}}%
\pgfpathcurveto{\pgfqpoint{3.735209in}{2.669770in}}{\pgfqpoint{3.737523in}{2.664184in}}{\pgfqpoint{3.741641in}{2.660066in}}%
\pgfpathcurveto{\pgfqpoint{3.745759in}{2.655947in}}{\pgfqpoint{3.751345in}{2.653634in}}{\pgfqpoint{3.757169in}{2.653634in}}%
\pgfpathlineto{\pgfqpoint{3.757169in}{2.653634in}}%
\pgfpathclose%
\pgfusepath{stroke,fill}%
\end{pgfscope}%
\begin{pgfscope}%
\pgfpathrectangle{\pgfqpoint{0.997489in}{0.528000in}}{\pgfqpoint{4.565023in}{3.696000in}}%
\pgfusepath{clip}%
\pgfsetbuttcap%
\pgfsetroundjoin%
\definecolor{currentfill}{rgb}{0.200000,0.200000,0.800000}%
\pgfsetfillcolor{currentfill}%
\pgfsetlinewidth{1.003750pt}%
\definecolor{currentstroke}{rgb}{0.200000,0.200000,0.800000}%
\pgfsetstrokecolor{currentstroke}%
\pgfsetdash{}{0pt}%
\pgfpathmoveto{\pgfqpoint{3.697761in}{2.674342in}}%
\pgfpathcurveto{\pgfqpoint{3.703585in}{2.674342in}}{\pgfqpoint{3.709171in}{2.676656in}}{\pgfqpoint{3.713290in}{2.680774in}}%
\pgfpathcurveto{\pgfqpoint{3.717408in}{2.684892in}}{\pgfqpoint{3.719722in}{2.690478in}}{\pgfqpoint{3.719722in}{2.696302in}}%
\pgfpathcurveto{\pgfqpoint{3.719722in}{2.702126in}}{\pgfqpoint{3.717408in}{2.707712in}}{\pgfqpoint{3.713290in}{2.711831in}}%
\pgfpathcurveto{\pgfqpoint{3.709171in}{2.715949in}}{\pgfqpoint{3.703585in}{2.718263in}}{\pgfqpoint{3.697761in}{2.718263in}}%
\pgfpathcurveto{\pgfqpoint{3.691937in}{2.718263in}}{\pgfqpoint{3.686351in}{2.715949in}}{\pgfqpoint{3.682233in}{2.711831in}}%
\pgfpathcurveto{\pgfqpoint{3.678115in}{2.707712in}}{\pgfqpoint{3.675801in}{2.702126in}}{\pgfqpoint{3.675801in}{2.696302in}}%
\pgfpathcurveto{\pgfqpoint{3.675801in}{2.690478in}}{\pgfqpoint{3.678115in}{2.684892in}}{\pgfqpoint{3.682233in}{2.680774in}}%
\pgfpathcurveto{\pgfqpoint{3.686351in}{2.676656in}}{\pgfqpoint{3.691937in}{2.674342in}}{\pgfqpoint{3.697761in}{2.674342in}}%
\pgfpathlineto{\pgfqpoint{3.697761in}{2.674342in}}%
\pgfpathclose%
\pgfusepath{stroke,fill}%
\end{pgfscope}%
\begin{pgfscope}%
\pgfpathrectangle{\pgfqpoint{0.997489in}{0.528000in}}{\pgfqpoint{4.565023in}{3.696000in}}%
\pgfusepath{clip}%
\pgfsetbuttcap%
\pgfsetroundjoin%
\definecolor{currentfill}{rgb}{0.200000,0.200000,0.800000}%
\pgfsetfillcolor{currentfill}%
\pgfsetlinewidth{1.003750pt}%
\definecolor{currentstroke}{rgb}{0.200000,0.200000,0.800000}%
\pgfsetstrokecolor{currentstroke}%
\pgfsetdash{}{0pt}%
\pgfpathmoveto{\pgfqpoint{3.704872in}{2.709914in}}%
\pgfpathcurveto{\pgfqpoint{3.710696in}{2.709914in}}{\pgfqpoint{3.716282in}{2.712228in}}{\pgfqpoint{3.720400in}{2.716346in}}%
\pgfpathcurveto{\pgfqpoint{3.724518in}{2.720464in}}{\pgfqpoint{3.726832in}{2.726050in}}{\pgfqpoint{3.726832in}{2.731874in}}%
\pgfpathcurveto{\pgfqpoint{3.726832in}{2.737698in}}{\pgfqpoint{3.724518in}{2.743285in}}{\pgfqpoint{3.720400in}{2.747403in}}%
\pgfpathcurveto{\pgfqpoint{3.716282in}{2.751521in}}{\pgfqpoint{3.710696in}{2.753835in}}{\pgfqpoint{3.704872in}{2.753835in}}%
\pgfpathcurveto{\pgfqpoint{3.699048in}{2.753835in}}{\pgfqpoint{3.693462in}{2.751521in}}{\pgfqpoint{3.689344in}{2.747403in}}%
\pgfpathcurveto{\pgfqpoint{3.685226in}{2.743285in}}{\pgfqpoint{3.682912in}{2.737698in}}{\pgfqpoint{3.682912in}{2.731874in}}%
\pgfpathcurveto{\pgfqpoint{3.682912in}{2.726050in}}{\pgfqpoint{3.685226in}{2.720464in}}{\pgfqpoint{3.689344in}{2.716346in}}%
\pgfpathcurveto{\pgfqpoint{3.693462in}{2.712228in}}{\pgfqpoint{3.699048in}{2.709914in}}{\pgfqpoint{3.704872in}{2.709914in}}%
\pgfpathlineto{\pgfqpoint{3.704872in}{2.709914in}}%
\pgfpathclose%
\pgfusepath{stroke,fill}%
\end{pgfscope}%
\begin{pgfscope}%
\pgfpathrectangle{\pgfqpoint{0.997489in}{0.528000in}}{\pgfqpoint{4.565023in}{3.696000in}}%
\pgfusepath{clip}%
\pgfsetbuttcap%
\pgfsetroundjoin%
\definecolor{currentfill}{rgb}{0.200000,0.200000,0.800000}%
\pgfsetfillcolor{currentfill}%
\pgfsetlinewidth{1.003750pt}%
\definecolor{currentstroke}{rgb}{0.200000,0.200000,0.800000}%
\pgfsetstrokecolor{currentstroke}%
\pgfsetdash{}{0pt}%
\pgfpathmoveto{\pgfqpoint{3.744872in}{2.761093in}}%
\pgfpathcurveto{\pgfqpoint{3.750696in}{2.761093in}}{\pgfqpoint{3.756282in}{2.763407in}}{\pgfqpoint{3.760401in}{2.767525in}}%
\pgfpathcurveto{\pgfqpoint{3.764519in}{2.771643in}}{\pgfqpoint{3.766833in}{2.777229in}}{\pgfqpoint{3.766833in}{2.783053in}}%
\pgfpathcurveto{\pgfqpoint{3.766833in}{2.788877in}}{\pgfqpoint{3.764519in}{2.794463in}}{\pgfqpoint{3.760401in}{2.798581in}}%
\pgfpathcurveto{\pgfqpoint{3.756282in}{2.802700in}}{\pgfqpoint{3.750696in}{2.805013in}}{\pgfqpoint{3.744872in}{2.805013in}}%
\pgfpathcurveto{\pgfqpoint{3.739048in}{2.805013in}}{\pgfqpoint{3.733462in}{2.802700in}}{\pgfqpoint{3.729344in}{2.798581in}}%
\pgfpathcurveto{\pgfqpoint{3.725226in}{2.794463in}}{\pgfqpoint{3.722912in}{2.788877in}}{\pgfqpoint{3.722912in}{2.783053in}}%
\pgfpathcurveto{\pgfqpoint{3.722912in}{2.777229in}}{\pgfqpoint{3.725226in}{2.771643in}}{\pgfqpoint{3.729344in}{2.767525in}}%
\pgfpathcurveto{\pgfqpoint{3.733462in}{2.763407in}}{\pgfqpoint{3.739048in}{2.761093in}}{\pgfqpoint{3.744872in}{2.761093in}}%
\pgfpathlineto{\pgfqpoint{3.744872in}{2.761093in}}%
\pgfpathclose%
\pgfusepath{stroke,fill}%
\end{pgfscope}%
\begin{pgfscope}%
\pgfpathrectangle{\pgfqpoint{0.997489in}{0.528000in}}{\pgfqpoint{4.565023in}{3.696000in}}%
\pgfusepath{clip}%
\pgfsetbuttcap%
\pgfsetroundjoin%
\definecolor{currentfill}{rgb}{0.200000,0.200000,0.800000}%
\pgfsetfillcolor{currentfill}%
\pgfsetlinewidth{1.003750pt}%
\definecolor{currentstroke}{rgb}{0.200000,0.200000,0.800000}%
\pgfsetstrokecolor{currentstroke}%
\pgfsetdash{}{0pt}%
\pgfpathmoveto{\pgfqpoint{3.666865in}{2.763989in}}%
\pgfpathcurveto{\pgfqpoint{3.672689in}{2.763989in}}{\pgfqpoint{3.678275in}{2.766303in}}{\pgfqpoint{3.682393in}{2.770421in}}%
\pgfpathcurveto{\pgfqpoint{3.686511in}{2.774539in}}{\pgfqpoint{3.688825in}{2.780125in}}{\pgfqpoint{3.688825in}{2.785949in}}%
\pgfpathcurveto{\pgfqpoint{3.688825in}{2.791773in}}{\pgfqpoint{3.686511in}{2.797359in}}{\pgfqpoint{3.682393in}{2.801477in}}%
\pgfpathcurveto{\pgfqpoint{3.678275in}{2.805596in}}{\pgfqpoint{3.672689in}{2.807909in}}{\pgfqpoint{3.666865in}{2.807909in}}%
\pgfpathcurveto{\pgfqpoint{3.661041in}{2.807909in}}{\pgfqpoint{3.655455in}{2.805596in}}{\pgfqpoint{3.651337in}{2.801477in}}%
\pgfpathcurveto{\pgfqpoint{3.647219in}{2.797359in}}{\pgfqpoint{3.644905in}{2.791773in}}{\pgfqpoint{3.644905in}{2.785949in}}%
\pgfpathcurveto{\pgfqpoint{3.644905in}{2.780125in}}{\pgfqpoint{3.647219in}{2.774539in}}{\pgfqpoint{3.651337in}{2.770421in}}%
\pgfpathcurveto{\pgfqpoint{3.655455in}{2.766303in}}{\pgfqpoint{3.661041in}{2.763989in}}{\pgfqpoint{3.666865in}{2.763989in}}%
\pgfpathlineto{\pgfqpoint{3.666865in}{2.763989in}}%
\pgfpathclose%
\pgfusepath{stroke,fill}%
\end{pgfscope}%
\begin{pgfscope}%
\pgfpathrectangle{\pgfqpoint{0.997489in}{0.528000in}}{\pgfqpoint{4.565023in}{3.696000in}}%
\pgfusepath{clip}%
\pgfsetbuttcap%
\pgfsetroundjoin%
\definecolor{currentfill}{rgb}{0.200000,0.200000,0.800000}%
\pgfsetfillcolor{currentfill}%
\pgfsetlinewidth{1.003750pt}%
\definecolor{currentstroke}{rgb}{0.200000,0.200000,0.800000}%
\pgfsetstrokecolor{currentstroke}%
\pgfsetdash{}{0pt}%
\pgfpathmoveto{\pgfqpoint{3.656819in}{2.794654in}}%
\pgfpathcurveto{\pgfqpoint{3.662643in}{2.794654in}}{\pgfqpoint{3.668229in}{2.796968in}}{\pgfqpoint{3.672347in}{2.801086in}}%
\pgfpathcurveto{\pgfqpoint{3.676466in}{2.805204in}}{\pgfqpoint{3.678779in}{2.810790in}}{\pgfqpoint{3.678779in}{2.816614in}}%
\pgfpathcurveto{\pgfqpoint{3.678779in}{2.822438in}}{\pgfqpoint{3.676466in}{2.828024in}}{\pgfqpoint{3.672347in}{2.832143in}}%
\pgfpathcurveto{\pgfqpoint{3.668229in}{2.836261in}}{\pgfqpoint{3.662643in}{2.838575in}}{\pgfqpoint{3.656819in}{2.838575in}}%
\pgfpathcurveto{\pgfqpoint{3.650995in}{2.838575in}}{\pgfqpoint{3.645409in}{2.836261in}}{\pgfqpoint{3.641291in}{2.832143in}}%
\pgfpathcurveto{\pgfqpoint{3.637173in}{2.828024in}}{\pgfqpoint{3.634859in}{2.822438in}}{\pgfqpoint{3.634859in}{2.816614in}}%
\pgfpathcurveto{\pgfqpoint{3.634859in}{2.810790in}}{\pgfqpoint{3.637173in}{2.805204in}}{\pgfqpoint{3.641291in}{2.801086in}}%
\pgfpathcurveto{\pgfqpoint{3.645409in}{2.796968in}}{\pgfqpoint{3.650995in}{2.794654in}}{\pgfqpoint{3.656819in}{2.794654in}}%
\pgfpathlineto{\pgfqpoint{3.656819in}{2.794654in}}%
\pgfpathclose%
\pgfusepath{stroke,fill}%
\end{pgfscope}%
\begin{pgfscope}%
\pgfpathrectangle{\pgfqpoint{0.997489in}{0.528000in}}{\pgfqpoint{4.565023in}{3.696000in}}%
\pgfusepath{clip}%
\pgfsetbuttcap%
\pgfsetroundjoin%
\definecolor{currentfill}{rgb}{0.200000,0.200000,0.800000}%
\pgfsetfillcolor{currentfill}%
\pgfsetlinewidth{1.003750pt}%
\definecolor{currentstroke}{rgb}{0.200000,0.200000,0.800000}%
\pgfsetstrokecolor{currentstroke}%
\pgfsetdash{}{0pt}%
\pgfpathmoveto{\pgfqpoint{3.647667in}{2.826818in}}%
\pgfpathcurveto{\pgfqpoint{3.653491in}{2.826818in}}{\pgfqpoint{3.659077in}{2.829132in}}{\pgfqpoint{3.663195in}{2.833250in}}%
\pgfpathcurveto{\pgfqpoint{3.667313in}{2.837368in}}{\pgfqpoint{3.669627in}{2.842954in}}{\pgfqpoint{3.669627in}{2.848778in}}%
\pgfpathcurveto{\pgfqpoint{3.669627in}{2.854602in}}{\pgfqpoint{3.667313in}{2.860188in}}{\pgfqpoint{3.663195in}{2.864306in}}%
\pgfpathcurveto{\pgfqpoint{3.659077in}{2.868424in}}{\pgfqpoint{3.653491in}{2.870738in}}{\pgfqpoint{3.647667in}{2.870738in}}%
\pgfpathcurveto{\pgfqpoint{3.641843in}{2.870738in}}{\pgfqpoint{3.636257in}{2.868424in}}{\pgfqpoint{3.632138in}{2.864306in}}%
\pgfpathcurveto{\pgfqpoint{3.628020in}{2.860188in}}{\pgfqpoint{3.625706in}{2.854602in}}{\pgfqpoint{3.625706in}{2.848778in}}%
\pgfpathcurveto{\pgfqpoint{3.625706in}{2.842954in}}{\pgfqpoint{3.628020in}{2.837368in}}{\pgfqpoint{3.632138in}{2.833250in}}%
\pgfpathcurveto{\pgfqpoint{3.636257in}{2.829132in}}{\pgfqpoint{3.641843in}{2.826818in}}{\pgfqpoint{3.647667in}{2.826818in}}%
\pgfpathlineto{\pgfqpoint{3.647667in}{2.826818in}}%
\pgfpathclose%
\pgfusepath{stroke,fill}%
\end{pgfscope}%
\begin{pgfscope}%
\pgfpathrectangle{\pgfqpoint{0.997489in}{0.528000in}}{\pgfqpoint{4.565023in}{3.696000in}}%
\pgfusepath{clip}%
\pgfsetbuttcap%
\pgfsetroundjoin%
\definecolor{currentfill}{rgb}{0.200000,0.200000,0.800000}%
\pgfsetfillcolor{currentfill}%
\pgfsetlinewidth{1.003750pt}%
\definecolor{currentstroke}{rgb}{0.200000,0.200000,0.800000}%
\pgfsetstrokecolor{currentstroke}%
\pgfsetdash{}{0pt}%
\pgfpathmoveto{\pgfqpoint{3.631961in}{2.855683in}}%
\pgfpathcurveto{\pgfqpoint{3.637785in}{2.855683in}}{\pgfqpoint{3.643371in}{2.857997in}}{\pgfqpoint{3.647489in}{2.862115in}}%
\pgfpathcurveto{\pgfqpoint{3.651607in}{2.866233in}}{\pgfqpoint{3.653921in}{2.871819in}}{\pgfqpoint{3.653921in}{2.877643in}}%
\pgfpathcurveto{\pgfqpoint{3.653921in}{2.883467in}}{\pgfqpoint{3.651607in}{2.889053in}}{\pgfqpoint{3.647489in}{2.893171in}}%
\pgfpathcurveto{\pgfqpoint{3.643371in}{2.897289in}}{\pgfqpoint{3.637785in}{2.899603in}}{\pgfqpoint{3.631961in}{2.899603in}}%
\pgfpathcurveto{\pgfqpoint{3.626137in}{2.899603in}}{\pgfqpoint{3.620551in}{2.897289in}}{\pgfqpoint{3.616433in}{2.893171in}}%
\pgfpathcurveto{\pgfqpoint{3.612315in}{2.889053in}}{\pgfqpoint{3.610001in}{2.883467in}}{\pgfqpoint{3.610001in}{2.877643in}}%
\pgfpathcurveto{\pgfqpoint{3.610001in}{2.871819in}}{\pgfqpoint{3.612315in}{2.866233in}}{\pgfqpoint{3.616433in}{2.862115in}}%
\pgfpathcurveto{\pgfqpoint{3.620551in}{2.857997in}}{\pgfqpoint{3.626137in}{2.855683in}}{\pgfqpoint{3.631961in}{2.855683in}}%
\pgfpathlineto{\pgfqpoint{3.631961in}{2.855683in}}%
\pgfpathclose%
\pgfusepath{stroke,fill}%
\end{pgfscope}%
\begin{pgfscope}%
\pgfpathrectangle{\pgfqpoint{0.997489in}{0.528000in}}{\pgfqpoint{4.565023in}{3.696000in}}%
\pgfusepath{clip}%
\pgfsetbuttcap%
\pgfsetroundjoin%
\definecolor{currentfill}{rgb}{0.800000,0.800000,0.200000}%
\pgfsetfillcolor{currentfill}%
\pgfsetlinewidth{1.003750pt}%
\definecolor{currentstroke}{rgb}{0.800000,0.800000,0.200000}%
\pgfsetstrokecolor{currentstroke}%
\pgfsetdash{}{0pt}%
\pgfpathmoveto{\pgfqpoint{3.560559in}{2.838552in}}%
\pgfpathcurveto{\pgfqpoint{3.566382in}{2.838552in}}{\pgfqpoint{3.571969in}{2.840866in}}{\pgfqpoint{3.576087in}{2.844984in}}%
\pgfpathcurveto{\pgfqpoint{3.580205in}{2.849102in}}{\pgfqpoint{3.582519in}{2.854689in}}{\pgfqpoint{3.582519in}{2.860512in}}%
\pgfpathcurveto{\pgfqpoint{3.582519in}{2.866336in}}{\pgfqpoint{3.580205in}{2.871923in}}{\pgfqpoint{3.576087in}{2.876041in}}%
\pgfpathcurveto{\pgfqpoint{3.571969in}{2.880159in}}{\pgfqpoint{3.566382in}{2.882473in}}{\pgfqpoint{3.560559in}{2.882473in}}%
\pgfpathcurveto{\pgfqpoint{3.554735in}{2.882473in}}{\pgfqpoint{3.549148in}{2.880159in}}{\pgfqpoint{3.545030in}{2.876041in}}%
\pgfpathcurveto{\pgfqpoint{3.540912in}{2.871923in}}{\pgfqpoint{3.538598in}{2.866336in}}{\pgfqpoint{3.538598in}{2.860512in}}%
\pgfpathcurveto{\pgfqpoint{3.538598in}{2.854689in}}{\pgfqpoint{3.540912in}{2.849102in}}{\pgfqpoint{3.545030in}{2.844984in}}%
\pgfpathcurveto{\pgfqpoint{3.549148in}{2.840866in}}{\pgfqpoint{3.554735in}{2.838552in}}{\pgfqpoint{3.560559in}{2.838552in}}%
\pgfpathlineto{\pgfqpoint{3.560559in}{2.838552in}}%
\pgfpathclose%
\pgfusepath{stroke,fill}%
\end{pgfscope}%
\begin{pgfscope}%
\pgfpathrectangle{\pgfqpoint{0.997489in}{0.528000in}}{\pgfqpoint{4.565023in}{3.696000in}}%
\pgfusepath{clip}%
\pgfsetbuttcap%
\pgfsetroundjoin%
\definecolor{currentfill}{rgb}{0.800000,0.800000,0.200000}%
\pgfsetfillcolor{currentfill}%
\pgfsetlinewidth{1.003750pt}%
\definecolor{currentstroke}{rgb}{0.800000,0.800000,0.200000}%
\pgfsetstrokecolor{currentstroke}%
\pgfsetdash{}{0pt}%
\pgfpathmoveto{\pgfqpoint{3.588351in}{2.904487in}}%
\pgfpathcurveto{\pgfqpoint{3.594175in}{2.904487in}}{\pgfqpoint{3.599761in}{2.906801in}}{\pgfqpoint{3.603880in}{2.910919in}}%
\pgfpathcurveto{\pgfqpoint{3.607998in}{2.915038in}}{\pgfqpoint{3.610312in}{2.920624in}}{\pgfqpoint{3.610312in}{2.926448in}}%
\pgfpathcurveto{\pgfqpoint{3.610312in}{2.932272in}}{\pgfqpoint{3.607998in}{2.937858in}}{\pgfqpoint{3.603880in}{2.941976in}}%
\pgfpathcurveto{\pgfqpoint{3.599761in}{2.946094in}}{\pgfqpoint{3.594175in}{2.948408in}}{\pgfqpoint{3.588351in}{2.948408in}}%
\pgfpathcurveto{\pgfqpoint{3.582527in}{2.948408in}}{\pgfqpoint{3.576941in}{2.946094in}}{\pgfqpoint{3.572823in}{2.941976in}}%
\pgfpathcurveto{\pgfqpoint{3.568705in}{2.937858in}}{\pgfqpoint{3.566391in}{2.932272in}}{\pgfqpoint{3.566391in}{2.926448in}}%
\pgfpathcurveto{\pgfqpoint{3.566391in}{2.920624in}}{\pgfqpoint{3.568705in}{2.915038in}}{\pgfqpoint{3.572823in}{2.910919in}}%
\pgfpathcurveto{\pgfqpoint{3.576941in}{2.906801in}}{\pgfqpoint{3.582527in}{2.904487in}}{\pgfqpoint{3.588351in}{2.904487in}}%
\pgfpathlineto{\pgfqpoint{3.588351in}{2.904487in}}%
\pgfpathclose%
\pgfusepath{stroke,fill}%
\end{pgfscope}%
\begin{pgfscope}%
\pgfpathrectangle{\pgfqpoint{0.997489in}{0.528000in}}{\pgfqpoint{4.565023in}{3.696000in}}%
\pgfusepath{clip}%
\pgfsetbuttcap%
\pgfsetroundjoin%
\definecolor{currentfill}{rgb}{0.800000,0.800000,0.200000}%
\pgfsetfillcolor{currentfill}%
\pgfsetlinewidth{1.003750pt}%
\definecolor{currentstroke}{rgb}{0.800000,0.800000,0.200000}%
\pgfsetstrokecolor{currentstroke}%
\pgfsetdash{}{0pt}%
\pgfpathmoveto{\pgfqpoint{3.574629in}{2.937744in}}%
\pgfpathcurveto{\pgfqpoint{3.580453in}{2.937744in}}{\pgfqpoint{3.586040in}{2.940057in}}{\pgfqpoint{3.590158in}{2.944176in}}%
\pgfpathcurveto{\pgfqpoint{3.594276in}{2.948294in}}{\pgfqpoint{3.596590in}{2.953880in}}{\pgfqpoint{3.596590in}{2.959704in}}%
\pgfpathcurveto{\pgfqpoint{3.596590in}{2.965528in}}{\pgfqpoint{3.594276in}{2.971114in}}{\pgfqpoint{3.590158in}{2.975232in}}%
\pgfpathcurveto{\pgfqpoint{3.586040in}{2.979350in}}{\pgfqpoint{3.580453in}{2.981664in}}{\pgfqpoint{3.574629in}{2.981664in}}%
\pgfpathcurveto{\pgfqpoint{3.568806in}{2.981664in}}{\pgfqpoint{3.563219in}{2.979350in}}{\pgfqpoint{3.559101in}{2.975232in}}%
\pgfpathcurveto{\pgfqpoint{3.554983in}{2.971114in}}{\pgfqpoint{3.552669in}{2.965528in}}{\pgfqpoint{3.552669in}{2.959704in}}%
\pgfpathcurveto{\pgfqpoint{3.552669in}{2.953880in}}{\pgfqpoint{3.554983in}{2.948294in}}{\pgfqpoint{3.559101in}{2.944176in}}%
\pgfpathcurveto{\pgfqpoint{3.563219in}{2.940057in}}{\pgfqpoint{3.568806in}{2.937744in}}{\pgfqpoint{3.574629in}{2.937744in}}%
\pgfpathlineto{\pgfqpoint{3.574629in}{2.937744in}}%
\pgfpathclose%
\pgfusepath{stroke,fill}%
\end{pgfscope}%
\begin{pgfscope}%
\pgfpathrectangle{\pgfqpoint{0.997489in}{0.528000in}}{\pgfqpoint{4.565023in}{3.696000in}}%
\pgfusepath{clip}%
\pgfsetbuttcap%
\pgfsetroundjoin%
\definecolor{currentfill}{rgb}{0.200000,0.200000,0.800000}%
\pgfsetfillcolor{currentfill}%
\pgfsetlinewidth{1.003750pt}%
\definecolor{currentstroke}{rgb}{0.200000,0.200000,0.800000}%
\pgfsetstrokecolor{currentstroke}%
\pgfsetdash{}{0pt}%
\pgfpathmoveto{\pgfqpoint{3.482462in}{2.877483in}}%
\pgfpathcurveto{\pgfqpoint{3.488286in}{2.877483in}}{\pgfqpoint{3.493872in}{2.879797in}}{\pgfqpoint{3.497991in}{2.883915in}}%
\pgfpathcurveto{\pgfqpoint{3.502109in}{2.888033in}}{\pgfqpoint{3.504423in}{2.893619in}}{\pgfqpoint{3.504423in}{2.899443in}}%
\pgfpathcurveto{\pgfqpoint{3.504423in}{2.905267in}}{\pgfqpoint{3.502109in}{2.910853in}}{\pgfqpoint{3.497991in}{2.914971in}}%
\pgfpathcurveto{\pgfqpoint{3.493872in}{2.919090in}}{\pgfqpoint{3.488286in}{2.921404in}}{\pgfqpoint{3.482462in}{2.921404in}}%
\pgfpathcurveto{\pgfqpoint{3.476638in}{2.921404in}}{\pgfqpoint{3.471052in}{2.919090in}}{\pgfqpoint{3.466934in}{2.914971in}}%
\pgfpathcurveto{\pgfqpoint{3.462816in}{2.910853in}}{\pgfqpoint{3.460502in}{2.905267in}}{\pgfqpoint{3.460502in}{2.899443in}}%
\pgfpathcurveto{\pgfqpoint{3.460502in}{2.893619in}}{\pgfqpoint{3.462816in}{2.888033in}}{\pgfqpoint{3.466934in}{2.883915in}}%
\pgfpathcurveto{\pgfqpoint{3.471052in}{2.879797in}}{\pgfqpoint{3.476638in}{2.877483in}}{\pgfqpoint{3.482462in}{2.877483in}}%
\pgfpathlineto{\pgfqpoint{3.482462in}{2.877483in}}%
\pgfpathclose%
\pgfusepath{stroke,fill}%
\end{pgfscope}%
\begin{pgfscope}%
\pgfpathrectangle{\pgfqpoint{0.997489in}{0.528000in}}{\pgfqpoint{4.565023in}{3.696000in}}%
\pgfusepath{clip}%
\pgfsetbuttcap%
\pgfsetroundjoin%
\definecolor{currentfill}{rgb}{0.800000,0.800000,0.200000}%
\pgfsetfillcolor{currentfill}%
\pgfsetlinewidth{1.003750pt}%
\definecolor{currentstroke}{rgb}{0.800000,0.800000,0.200000}%
\pgfsetstrokecolor{currentstroke}%
\pgfsetdash{}{0pt}%
\pgfpathmoveto{\pgfqpoint{3.535483in}{2.998174in}}%
\pgfpathcurveto{\pgfqpoint{3.541307in}{2.998174in}}{\pgfqpoint{3.546893in}{3.000487in}}{\pgfqpoint{3.551012in}{3.004606in}}%
\pgfpathcurveto{\pgfqpoint{3.555130in}{3.008724in}}{\pgfqpoint{3.557444in}{3.014310in}}{\pgfqpoint{3.557444in}{3.020134in}}%
\pgfpathcurveto{\pgfqpoint{3.557444in}{3.025958in}}{\pgfqpoint{3.555130in}{3.031544in}}{\pgfqpoint{3.551012in}{3.035662in}}%
\pgfpathcurveto{\pgfqpoint{3.546893in}{3.039780in}}{\pgfqpoint{3.541307in}{3.042094in}}{\pgfqpoint{3.535483in}{3.042094in}}%
\pgfpathcurveto{\pgfqpoint{3.529659in}{3.042094in}}{\pgfqpoint{3.524073in}{3.039780in}}{\pgfqpoint{3.519955in}{3.035662in}}%
\pgfpathcurveto{\pgfqpoint{3.515837in}{3.031544in}}{\pgfqpoint{3.513523in}{3.025958in}}{\pgfqpoint{3.513523in}{3.020134in}}%
\pgfpathcurveto{\pgfqpoint{3.513523in}{3.014310in}}{\pgfqpoint{3.515837in}{3.008724in}}{\pgfqpoint{3.519955in}{3.004606in}}%
\pgfpathcurveto{\pgfqpoint{3.524073in}{3.000487in}}{\pgfqpoint{3.529659in}{2.998174in}}{\pgfqpoint{3.535483in}{2.998174in}}%
\pgfpathlineto{\pgfqpoint{3.535483in}{2.998174in}}%
\pgfpathclose%
\pgfusepath{stroke,fill}%
\end{pgfscope}%
\begin{pgfscope}%
\pgfpathrectangle{\pgfqpoint{0.997489in}{0.528000in}}{\pgfqpoint{4.565023in}{3.696000in}}%
\pgfusepath{clip}%
\pgfsetbuttcap%
\pgfsetroundjoin%
\definecolor{currentfill}{rgb}{0.800000,0.800000,0.200000}%
\pgfsetfillcolor{currentfill}%
\pgfsetlinewidth{1.003750pt}%
\definecolor{currentstroke}{rgb}{0.800000,0.800000,0.200000}%
\pgfsetstrokecolor{currentstroke}%
\pgfsetdash{}{0pt}%
\pgfpathmoveto{\pgfqpoint{3.507756in}{3.019775in}}%
\pgfpathcurveto{\pgfqpoint{3.513580in}{3.019775in}}{\pgfqpoint{3.519166in}{3.022089in}}{\pgfqpoint{3.523284in}{3.026207in}}%
\pgfpathcurveto{\pgfqpoint{3.527402in}{3.030325in}}{\pgfqpoint{3.529716in}{3.035912in}}{\pgfqpoint{3.529716in}{3.041735in}}%
\pgfpathcurveto{\pgfqpoint{3.529716in}{3.047559in}}{\pgfqpoint{3.527402in}{3.053146in}}{\pgfqpoint{3.523284in}{3.057264in}}%
\pgfpathcurveto{\pgfqpoint{3.519166in}{3.061382in}}{\pgfqpoint{3.513580in}{3.063696in}}{\pgfqpoint{3.507756in}{3.063696in}}%
\pgfpathcurveto{\pgfqpoint{3.501932in}{3.063696in}}{\pgfqpoint{3.496346in}{3.061382in}}{\pgfqpoint{3.492227in}{3.057264in}}%
\pgfpathcurveto{\pgfqpoint{3.488109in}{3.053146in}}{\pgfqpoint{3.485795in}{3.047559in}}{\pgfqpoint{3.485795in}{3.041735in}}%
\pgfpathcurveto{\pgfqpoint{3.485795in}{3.035912in}}{\pgfqpoint{3.488109in}{3.030325in}}{\pgfqpoint{3.492227in}{3.026207in}}%
\pgfpathcurveto{\pgfqpoint{3.496346in}{3.022089in}}{\pgfqpoint{3.501932in}{3.019775in}}{\pgfqpoint{3.507756in}{3.019775in}}%
\pgfpathlineto{\pgfqpoint{3.507756in}{3.019775in}}%
\pgfpathclose%
\pgfusepath{stroke,fill}%
\end{pgfscope}%
\begin{pgfscope}%
\pgfpathrectangle{\pgfqpoint{0.997489in}{0.528000in}}{\pgfqpoint{4.565023in}{3.696000in}}%
\pgfusepath{clip}%
\pgfsetbuttcap%
\pgfsetroundjoin%
\definecolor{currentfill}{rgb}{0.200000,0.200000,0.800000}%
\pgfsetfillcolor{currentfill}%
\pgfsetlinewidth{1.003750pt}%
\definecolor{currentstroke}{rgb}{0.200000,0.200000,0.800000}%
\pgfsetstrokecolor{currentstroke}%
\pgfsetdash{}{0pt}%
\pgfpathmoveto{\pgfqpoint{3.461148in}{3.007142in}}%
\pgfpathcurveto{\pgfqpoint{3.466972in}{3.007142in}}{\pgfqpoint{3.472558in}{3.009456in}}{\pgfqpoint{3.476676in}{3.013574in}}%
\pgfpathcurveto{\pgfqpoint{3.480794in}{3.017692in}}{\pgfqpoint{3.483108in}{3.023278in}}{\pgfqpoint{3.483108in}{3.029102in}}%
\pgfpathcurveto{\pgfqpoint{3.483108in}{3.034926in}}{\pgfqpoint{3.480794in}{3.040512in}}{\pgfqpoint{3.476676in}{3.044630in}}%
\pgfpathcurveto{\pgfqpoint{3.472558in}{3.048748in}}{\pgfqpoint{3.466972in}{3.051062in}}{\pgfqpoint{3.461148in}{3.051062in}}%
\pgfpathcurveto{\pgfqpoint{3.455324in}{3.051062in}}{\pgfqpoint{3.449738in}{3.048748in}}{\pgfqpoint{3.445619in}{3.044630in}}%
\pgfpathcurveto{\pgfqpoint{3.441501in}{3.040512in}}{\pgfqpoint{3.439187in}{3.034926in}}{\pgfqpoint{3.439187in}{3.029102in}}%
\pgfpathcurveto{\pgfqpoint{3.439187in}{3.023278in}}{\pgfqpoint{3.441501in}{3.017692in}}{\pgfqpoint{3.445619in}{3.013574in}}%
\pgfpathcurveto{\pgfqpoint{3.449738in}{3.009456in}}{\pgfqpoint{3.455324in}{3.007142in}}{\pgfqpoint{3.461148in}{3.007142in}}%
\pgfpathlineto{\pgfqpoint{3.461148in}{3.007142in}}%
\pgfpathclose%
\pgfusepath{stroke,fill}%
\end{pgfscope}%
\begin{pgfscope}%
\pgfpathrectangle{\pgfqpoint{0.997489in}{0.528000in}}{\pgfqpoint{4.565023in}{3.696000in}}%
\pgfusepath{clip}%
\pgfsetbuttcap%
\pgfsetroundjoin%
\definecolor{currentfill}{rgb}{0.200000,0.200000,0.800000}%
\pgfsetfillcolor{currentfill}%
\pgfsetlinewidth{1.003750pt}%
\definecolor{currentstroke}{rgb}{0.200000,0.200000,0.800000}%
\pgfsetstrokecolor{currentstroke}%
\pgfsetdash{}{0pt}%
\pgfpathmoveto{\pgfqpoint{3.397278in}{2.946589in}}%
\pgfpathcurveto{\pgfqpoint{3.403102in}{2.946589in}}{\pgfqpoint{3.408688in}{2.948903in}}{\pgfqpoint{3.412806in}{2.953021in}}%
\pgfpathcurveto{\pgfqpoint{3.416925in}{2.957139in}}{\pgfqpoint{3.419238in}{2.962726in}}{\pgfqpoint{3.419238in}{2.968549in}}%
\pgfpathcurveto{\pgfqpoint{3.419238in}{2.974373in}}{\pgfqpoint{3.416925in}{2.979960in}}{\pgfqpoint{3.412806in}{2.984078in}}%
\pgfpathcurveto{\pgfqpoint{3.408688in}{2.988196in}}{\pgfqpoint{3.403102in}{2.990510in}}{\pgfqpoint{3.397278in}{2.990510in}}%
\pgfpathcurveto{\pgfqpoint{3.391454in}{2.990510in}}{\pgfqpoint{3.385868in}{2.988196in}}{\pgfqpoint{3.381750in}{2.984078in}}%
\pgfpathcurveto{\pgfqpoint{3.377632in}{2.979960in}}{\pgfqpoint{3.375318in}{2.974373in}}{\pgfqpoint{3.375318in}{2.968549in}}%
\pgfpathcurveto{\pgfqpoint{3.375318in}{2.962726in}}{\pgfqpoint{3.377632in}{2.957139in}}{\pgfqpoint{3.381750in}{2.953021in}}%
\pgfpathcurveto{\pgfqpoint{3.385868in}{2.948903in}}{\pgfqpoint{3.391454in}{2.946589in}}{\pgfqpoint{3.397278in}{2.946589in}}%
\pgfpathlineto{\pgfqpoint{3.397278in}{2.946589in}}%
\pgfpathclose%
\pgfusepath{stroke,fill}%
\end{pgfscope}%
\begin{pgfscope}%
\pgfpathrectangle{\pgfqpoint{0.997489in}{0.528000in}}{\pgfqpoint{4.565023in}{3.696000in}}%
\pgfusepath{clip}%
\pgfsetbuttcap%
\pgfsetroundjoin%
\definecolor{currentfill}{rgb}{0.200000,0.200000,0.800000}%
\pgfsetfillcolor{currentfill}%
\pgfsetlinewidth{1.003750pt}%
\definecolor{currentstroke}{rgb}{0.200000,0.200000,0.800000}%
\pgfsetstrokecolor{currentstroke}%
\pgfsetdash{}{0pt}%
\pgfpathmoveto{\pgfqpoint{3.384562in}{2.990924in}}%
\pgfpathcurveto{\pgfqpoint{3.390386in}{2.990924in}}{\pgfqpoint{3.395973in}{2.993238in}}{\pgfqpoint{3.400091in}{2.997356in}}%
\pgfpathcurveto{\pgfqpoint{3.404209in}{3.001474in}}{\pgfqpoint{3.406523in}{3.007060in}}{\pgfqpoint{3.406523in}{3.012884in}}%
\pgfpathcurveto{\pgfqpoint{3.406523in}{3.018708in}}{\pgfqpoint{3.404209in}{3.024294in}}{\pgfqpoint{3.400091in}{3.028412in}}%
\pgfpathcurveto{\pgfqpoint{3.395973in}{3.032531in}}{\pgfqpoint{3.390386in}{3.034844in}}{\pgfqpoint{3.384562in}{3.034844in}}%
\pgfpathcurveto{\pgfqpoint{3.378739in}{3.034844in}}{\pgfqpoint{3.373152in}{3.032531in}}{\pgfqpoint{3.369034in}{3.028412in}}%
\pgfpathcurveto{\pgfqpoint{3.364916in}{3.024294in}}{\pgfqpoint{3.362602in}{3.018708in}}{\pgfqpoint{3.362602in}{3.012884in}}%
\pgfpathcurveto{\pgfqpoint{3.362602in}{3.007060in}}{\pgfqpoint{3.364916in}{3.001474in}}{\pgfqpoint{3.369034in}{2.997356in}}%
\pgfpathcurveto{\pgfqpoint{3.373152in}{2.993238in}}{\pgfqpoint{3.378739in}{2.990924in}}{\pgfqpoint{3.384562in}{2.990924in}}%
\pgfpathlineto{\pgfqpoint{3.384562in}{2.990924in}}%
\pgfpathclose%
\pgfusepath{stroke,fill}%
\end{pgfscope}%
\begin{pgfscope}%
\pgfpathrectangle{\pgfqpoint{0.997489in}{0.528000in}}{\pgfqpoint{4.565023in}{3.696000in}}%
\pgfusepath{clip}%
\pgfsetbuttcap%
\pgfsetroundjoin%
\definecolor{currentfill}{rgb}{0.200000,0.200000,0.800000}%
\pgfsetfillcolor{currentfill}%
\pgfsetlinewidth{1.003750pt}%
\definecolor{currentstroke}{rgb}{0.200000,0.200000,0.800000}%
\pgfsetstrokecolor{currentstroke}%
\pgfsetdash{}{0pt}%
\pgfpathmoveto{\pgfqpoint{3.330452in}{2.917886in}}%
\pgfpathcurveto{\pgfqpoint{3.336276in}{2.917886in}}{\pgfqpoint{3.341862in}{2.920200in}}{\pgfqpoint{3.345980in}{2.924318in}}%
\pgfpathcurveto{\pgfqpoint{3.350098in}{2.928436in}}{\pgfqpoint{3.352412in}{2.934022in}}{\pgfqpoint{3.352412in}{2.939846in}}%
\pgfpathcurveto{\pgfqpoint{3.352412in}{2.945670in}}{\pgfqpoint{3.350098in}{2.951256in}}{\pgfqpoint{3.345980in}{2.955375in}}%
\pgfpathcurveto{\pgfqpoint{3.341862in}{2.959493in}}{\pgfqpoint{3.336276in}{2.961807in}}{\pgfqpoint{3.330452in}{2.961807in}}%
\pgfpathcurveto{\pgfqpoint{3.324628in}{2.961807in}}{\pgfqpoint{3.319042in}{2.959493in}}{\pgfqpoint{3.314924in}{2.955375in}}%
\pgfpathcurveto{\pgfqpoint{3.310805in}{2.951256in}}{\pgfqpoint{3.308492in}{2.945670in}}{\pgfqpoint{3.308492in}{2.939846in}}%
\pgfpathcurveto{\pgfqpoint{3.308492in}{2.934022in}}{\pgfqpoint{3.310805in}{2.928436in}}{\pgfqpoint{3.314924in}{2.924318in}}%
\pgfpathcurveto{\pgfqpoint{3.319042in}{2.920200in}}{\pgfqpoint{3.324628in}{2.917886in}}{\pgfqpoint{3.330452in}{2.917886in}}%
\pgfpathlineto{\pgfqpoint{3.330452in}{2.917886in}}%
\pgfpathclose%
\pgfusepath{stroke,fill}%
\end{pgfscope}%
\begin{pgfscope}%
\pgfpathrectangle{\pgfqpoint{0.997489in}{0.528000in}}{\pgfqpoint{4.565023in}{3.696000in}}%
\pgfusepath{clip}%
\pgfsetbuttcap%
\pgfsetroundjoin%
\definecolor{currentfill}{rgb}{0.200000,0.200000,0.800000}%
\pgfsetfillcolor{currentfill}%
\pgfsetlinewidth{1.003750pt}%
\definecolor{currentstroke}{rgb}{0.200000,0.200000,0.800000}%
\pgfsetstrokecolor{currentstroke}%
\pgfsetdash{}{0pt}%
\pgfpathmoveto{\pgfqpoint{3.330850in}{3.023369in}}%
\pgfpathcurveto{\pgfqpoint{3.336674in}{3.023369in}}{\pgfqpoint{3.342260in}{3.025682in}}{\pgfqpoint{3.346379in}{3.029801in}}%
\pgfpathcurveto{\pgfqpoint{3.350497in}{3.033919in}}{\pgfqpoint{3.352811in}{3.039505in}}{\pgfqpoint{3.352811in}{3.045329in}}%
\pgfpathcurveto{\pgfqpoint{3.352811in}{3.051153in}}{\pgfqpoint{3.350497in}{3.056739in}}{\pgfqpoint{3.346379in}{3.060857in}}%
\pgfpathcurveto{\pgfqpoint{3.342260in}{3.064975in}}{\pgfqpoint{3.336674in}{3.067289in}}{\pgfqpoint{3.330850in}{3.067289in}}%
\pgfpathcurveto{\pgfqpoint{3.325026in}{3.067289in}}{\pgfqpoint{3.319440in}{3.064975in}}{\pgfqpoint{3.315322in}{3.060857in}}%
\pgfpathcurveto{\pgfqpoint{3.311204in}{3.056739in}}{\pgfqpoint{3.308890in}{3.051153in}}{\pgfqpoint{3.308890in}{3.045329in}}%
\pgfpathcurveto{\pgfqpoint{3.308890in}{3.039505in}}{\pgfqpoint{3.311204in}{3.033919in}}{\pgfqpoint{3.315322in}{3.029801in}}%
\pgfpathcurveto{\pgfqpoint{3.319440in}{3.025682in}}{\pgfqpoint{3.325026in}{3.023369in}}{\pgfqpoint{3.330850in}{3.023369in}}%
\pgfpathlineto{\pgfqpoint{3.330850in}{3.023369in}}%
\pgfpathclose%
\pgfusepath{stroke,fill}%
\end{pgfscope}%
\begin{pgfscope}%
\pgfpathrectangle{\pgfqpoint{0.997489in}{0.528000in}}{\pgfqpoint{4.565023in}{3.696000in}}%
\pgfusepath{clip}%
\pgfsetbuttcap%
\pgfsetroundjoin%
\definecolor{currentfill}{rgb}{0.200000,0.200000,0.800000}%
\pgfsetfillcolor{currentfill}%
\pgfsetlinewidth{1.003750pt}%
\definecolor{currentstroke}{rgb}{0.200000,0.200000,0.800000}%
\pgfsetstrokecolor{currentstroke}%
\pgfsetdash{}{0pt}%
\pgfpathmoveto{\pgfqpoint{3.295558in}{3.001388in}}%
\pgfpathcurveto{\pgfqpoint{3.301382in}{3.001388in}}{\pgfqpoint{3.306968in}{3.003702in}}{\pgfqpoint{3.311086in}{3.007820in}}%
\pgfpathcurveto{\pgfqpoint{3.315204in}{3.011938in}}{\pgfqpoint{3.317518in}{3.017524in}}{\pgfqpoint{3.317518in}{3.023348in}}%
\pgfpathcurveto{\pgfqpoint{3.317518in}{3.029172in}}{\pgfqpoint{3.315204in}{3.034758in}}{\pgfqpoint{3.311086in}{3.038876in}}%
\pgfpathcurveto{\pgfqpoint{3.306968in}{3.042994in}}{\pgfqpoint{3.301382in}{3.045308in}}{\pgfqpoint{3.295558in}{3.045308in}}%
\pgfpathcurveto{\pgfqpoint{3.289734in}{3.045308in}}{\pgfqpoint{3.284148in}{3.042994in}}{\pgfqpoint{3.280030in}{3.038876in}}%
\pgfpathcurveto{\pgfqpoint{3.275912in}{3.034758in}}{\pgfqpoint{3.273598in}{3.029172in}}{\pgfqpoint{3.273598in}{3.023348in}}%
\pgfpathcurveto{\pgfqpoint{3.273598in}{3.017524in}}{\pgfqpoint{3.275912in}{3.011938in}}{\pgfqpoint{3.280030in}{3.007820in}}%
\pgfpathcurveto{\pgfqpoint{3.284148in}{3.003702in}}{\pgfqpoint{3.289734in}{3.001388in}}{\pgfqpoint{3.295558in}{3.001388in}}%
\pgfpathlineto{\pgfqpoint{3.295558in}{3.001388in}}%
\pgfpathclose%
\pgfusepath{stroke,fill}%
\end{pgfscope}%
\begin{pgfscope}%
\pgfpathrectangle{\pgfqpoint{0.997489in}{0.528000in}}{\pgfqpoint{4.565023in}{3.696000in}}%
\pgfusepath{clip}%
\pgfsetbuttcap%
\pgfsetroundjoin%
\definecolor{currentfill}{rgb}{0.200000,0.200000,0.800000}%
\pgfsetfillcolor{currentfill}%
\pgfsetlinewidth{1.003750pt}%
\definecolor{currentstroke}{rgb}{0.200000,0.200000,0.800000}%
\pgfsetstrokecolor{currentstroke}%
\pgfsetdash{}{0pt}%
\pgfpathmoveto{\pgfqpoint{3.271573in}{3.049013in}}%
\pgfpathcurveto{\pgfqpoint{3.277397in}{3.049013in}}{\pgfqpoint{3.282984in}{3.051326in}}{\pgfqpoint{3.287102in}{3.055445in}}%
\pgfpathcurveto{\pgfqpoint{3.291220in}{3.059563in}}{\pgfqpoint{3.293534in}{3.065149in}}{\pgfqpoint{3.293534in}{3.070973in}}%
\pgfpathcurveto{\pgfqpoint{3.293534in}{3.076797in}}{\pgfqpoint{3.291220in}{3.082383in}}{\pgfqpoint{3.287102in}{3.086501in}}%
\pgfpathcurveto{\pgfqpoint{3.282984in}{3.090619in}}{\pgfqpoint{3.277397in}{3.092933in}}{\pgfqpoint{3.271573in}{3.092933in}}%
\pgfpathcurveto{\pgfqpoint{3.265749in}{3.092933in}}{\pgfqpoint{3.260163in}{3.090619in}}{\pgfqpoint{3.256045in}{3.086501in}}%
\pgfpathcurveto{\pgfqpoint{3.251927in}{3.082383in}}{\pgfqpoint{3.249613in}{3.076797in}}{\pgfqpoint{3.249613in}{3.070973in}}%
\pgfpathcurveto{\pgfqpoint{3.249613in}{3.065149in}}{\pgfqpoint{3.251927in}{3.059563in}}{\pgfqpoint{3.256045in}{3.055445in}}%
\pgfpathcurveto{\pgfqpoint{3.260163in}{3.051326in}}{\pgfqpoint{3.265749in}{3.049013in}}{\pgfqpoint{3.271573in}{3.049013in}}%
\pgfpathlineto{\pgfqpoint{3.271573in}{3.049013in}}%
\pgfpathclose%
\pgfusepath{stroke,fill}%
\end{pgfscope}%
\begin{pgfscope}%
\pgfpathrectangle{\pgfqpoint{0.997489in}{0.528000in}}{\pgfqpoint{4.565023in}{3.696000in}}%
\pgfusepath{clip}%
\pgfsetbuttcap%
\pgfsetroundjoin%
\definecolor{currentfill}{rgb}{0.200000,0.200000,0.800000}%
\pgfsetfillcolor{currentfill}%
\pgfsetlinewidth{1.003750pt}%
\definecolor{currentstroke}{rgb}{0.200000,0.200000,0.800000}%
\pgfsetstrokecolor{currentstroke}%
\pgfsetdash{}{0pt}%
\pgfpathmoveto{\pgfqpoint{3.242401in}{3.106813in}}%
\pgfpathcurveto{\pgfqpoint{3.248224in}{3.106813in}}{\pgfqpoint{3.253811in}{3.109127in}}{\pgfqpoint{3.257929in}{3.113245in}}%
\pgfpathcurveto{\pgfqpoint{3.262047in}{3.117363in}}{\pgfqpoint{3.264361in}{3.122949in}}{\pgfqpoint{3.264361in}{3.128773in}}%
\pgfpathcurveto{\pgfqpoint{3.264361in}{3.134597in}}{\pgfqpoint{3.262047in}{3.140184in}}{\pgfqpoint{3.257929in}{3.144302in}}%
\pgfpathcurveto{\pgfqpoint{3.253811in}{3.148420in}}{\pgfqpoint{3.248224in}{3.150734in}}{\pgfqpoint{3.242401in}{3.150734in}}%
\pgfpathcurveto{\pgfqpoint{3.236577in}{3.150734in}}{\pgfqpoint{3.230990in}{3.148420in}}{\pgfqpoint{3.226872in}{3.144302in}}%
\pgfpathcurveto{\pgfqpoint{3.222754in}{3.140184in}}{\pgfqpoint{3.220440in}{3.134597in}}{\pgfqpoint{3.220440in}{3.128773in}}%
\pgfpathcurveto{\pgfqpoint{3.220440in}{3.122949in}}{\pgfqpoint{3.222754in}{3.117363in}}{\pgfqpoint{3.226872in}{3.113245in}}%
\pgfpathcurveto{\pgfqpoint{3.230990in}{3.109127in}}{\pgfqpoint{3.236577in}{3.106813in}}{\pgfqpoint{3.242401in}{3.106813in}}%
\pgfpathlineto{\pgfqpoint{3.242401in}{3.106813in}}%
\pgfpathclose%
\pgfusepath{stroke,fill}%
\end{pgfscope}%
\begin{pgfscope}%
\pgfpathrectangle{\pgfqpoint{0.997489in}{0.528000in}}{\pgfqpoint{4.565023in}{3.696000in}}%
\pgfusepath{clip}%
\pgfsetbuttcap%
\pgfsetroundjoin%
\definecolor{currentfill}{rgb}{0.200000,0.200000,0.800000}%
\pgfsetfillcolor{currentfill}%
\pgfsetlinewidth{1.003750pt}%
\definecolor{currentstroke}{rgb}{0.200000,0.200000,0.800000}%
\pgfsetstrokecolor{currentstroke}%
\pgfsetdash{}{0pt}%
\pgfpathmoveto{\pgfqpoint{3.208222in}{3.029575in}}%
\pgfpathcurveto{\pgfqpoint{3.214046in}{3.029575in}}{\pgfqpoint{3.219632in}{3.031889in}}{\pgfqpoint{3.223750in}{3.036007in}}%
\pgfpathcurveto{\pgfqpoint{3.227868in}{3.040125in}}{\pgfqpoint{3.230182in}{3.045711in}}{\pgfqpoint{3.230182in}{3.051535in}}%
\pgfpathcurveto{\pgfqpoint{3.230182in}{3.057359in}}{\pgfqpoint{3.227868in}{3.062945in}}{\pgfqpoint{3.223750in}{3.067064in}}%
\pgfpathcurveto{\pgfqpoint{3.219632in}{3.071182in}}{\pgfqpoint{3.214046in}{3.073496in}}{\pgfqpoint{3.208222in}{3.073496in}}%
\pgfpathcurveto{\pgfqpoint{3.202398in}{3.073496in}}{\pgfqpoint{3.196812in}{3.071182in}}{\pgfqpoint{3.192694in}{3.067064in}}%
\pgfpathcurveto{\pgfqpoint{3.188576in}{3.062945in}}{\pgfqpoint{3.186262in}{3.057359in}}{\pgfqpoint{3.186262in}{3.051535in}}%
\pgfpathcurveto{\pgfqpoint{3.186262in}{3.045711in}}{\pgfqpoint{3.188576in}{3.040125in}}{\pgfqpoint{3.192694in}{3.036007in}}%
\pgfpathcurveto{\pgfqpoint{3.196812in}{3.031889in}}{\pgfqpoint{3.202398in}{3.029575in}}{\pgfqpoint{3.208222in}{3.029575in}}%
\pgfpathlineto{\pgfqpoint{3.208222in}{3.029575in}}%
\pgfpathclose%
\pgfusepath{stroke,fill}%
\end{pgfscope}%
\begin{pgfscope}%
\pgfpathrectangle{\pgfqpoint{0.997489in}{0.528000in}}{\pgfqpoint{4.565023in}{3.696000in}}%
\pgfusepath{clip}%
\pgfsetbuttcap%
\pgfsetroundjoin%
\definecolor{currentfill}{rgb}{0.200000,0.200000,0.800000}%
\pgfsetfillcolor{currentfill}%
\pgfsetlinewidth{1.003750pt}%
\definecolor{currentstroke}{rgb}{0.200000,0.200000,0.800000}%
\pgfsetstrokecolor{currentstroke}%
\pgfsetdash{}{0pt}%
\pgfpathmoveto{\pgfqpoint{3.174031in}{3.075239in}}%
\pgfpathcurveto{\pgfqpoint{3.179855in}{3.075239in}}{\pgfqpoint{3.185441in}{3.077553in}}{\pgfqpoint{3.189559in}{3.081671in}}%
\pgfpathcurveto{\pgfqpoint{3.193677in}{3.085789in}}{\pgfqpoint{3.195991in}{3.091375in}}{\pgfqpoint{3.195991in}{3.097199in}}%
\pgfpathcurveto{\pgfqpoint{3.195991in}{3.103023in}}{\pgfqpoint{3.193677in}{3.108609in}}{\pgfqpoint{3.189559in}{3.112728in}}%
\pgfpathcurveto{\pgfqpoint{3.185441in}{3.116846in}}{\pgfqpoint{3.179855in}{3.119160in}}{\pgfqpoint{3.174031in}{3.119160in}}%
\pgfpathcurveto{\pgfqpoint{3.168207in}{3.119160in}}{\pgfqpoint{3.162621in}{3.116846in}}{\pgfqpoint{3.158503in}{3.112728in}}%
\pgfpathcurveto{\pgfqpoint{3.154385in}{3.108609in}}{\pgfqpoint{3.152071in}{3.103023in}}{\pgfqpoint{3.152071in}{3.097199in}}%
\pgfpathcurveto{\pgfqpoint{3.152071in}{3.091375in}}{\pgfqpoint{3.154385in}{3.085789in}}{\pgfqpoint{3.158503in}{3.081671in}}%
\pgfpathcurveto{\pgfqpoint{3.162621in}{3.077553in}}{\pgfqpoint{3.168207in}{3.075239in}}{\pgfqpoint{3.174031in}{3.075239in}}%
\pgfpathlineto{\pgfqpoint{3.174031in}{3.075239in}}%
\pgfpathclose%
\pgfusepath{stroke,fill}%
\end{pgfscope}%
\begin{pgfscope}%
\pgfpathrectangle{\pgfqpoint{0.997489in}{0.528000in}}{\pgfqpoint{4.565023in}{3.696000in}}%
\pgfusepath{clip}%
\pgfsetbuttcap%
\pgfsetroundjoin%
\definecolor{currentfill}{rgb}{0.200000,0.200000,0.800000}%
\pgfsetfillcolor{currentfill}%
\pgfsetlinewidth{1.003750pt}%
\definecolor{currentstroke}{rgb}{0.200000,0.200000,0.800000}%
\pgfsetstrokecolor{currentstroke}%
\pgfsetdash{}{0pt}%
\pgfpathmoveto{\pgfqpoint{3.144430in}{3.045988in}}%
\pgfpathcurveto{\pgfqpoint{3.150254in}{3.045988in}}{\pgfqpoint{3.155840in}{3.048301in}}{\pgfqpoint{3.159959in}{3.052420in}}%
\pgfpathcurveto{\pgfqpoint{3.164077in}{3.056538in}}{\pgfqpoint{3.166391in}{3.062124in}}{\pgfqpoint{3.166391in}{3.067948in}}%
\pgfpathcurveto{\pgfqpoint{3.166391in}{3.073772in}}{\pgfqpoint{3.164077in}{3.079358in}}{\pgfqpoint{3.159959in}{3.083476in}}%
\pgfpathcurveto{\pgfqpoint{3.155840in}{3.087594in}}{\pgfqpoint{3.150254in}{3.089908in}}{\pgfqpoint{3.144430in}{3.089908in}}%
\pgfpathcurveto{\pgfqpoint{3.138606in}{3.089908in}}{\pgfqpoint{3.133020in}{3.087594in}}{\pgfqpoint{3.128902in}{3.083476in}}%
\pgfpathcurveto{\pgfqpoint{3.124784in}{3.079358in}}{\pgfqpoint{3.122470in}{3.073772in}}{\pgfqpoint{3.122470in}{3.067948in}}%
\pgfpathcurveto{\pgfqpoint{3.122470in}{3.062124in}}{\pgfqpoint{3.124784in}{3.056538in}}{\pgfqpoint{3.128902in}{3.052420in}}%
\pgfpathcurveto{\pgfqpoint{3.133020in}{3.048301in}}{\pgfqpoint{3.138606in}{3.045988in}}{\pgfqpoint{3.144430in}{3.045988in}}%
\pgfpathlineto{\pgfqpoint{3.144430in}{3.045988in}}%
\pgfpathclose%
\pgfusepath{stroke,fill}%
\end{pgfscope}%
\begin{pgfscope}%
\pgfpathrectangle{\pgfqpoint{0.997489in}{0.528000in}}{\pgfqpoint{4.565023in}{3.696000in}}%
\pgfusepath{clip}%
\pgfsetbuttcap%
\pgfsetroundjoin%
\definecolor{currentfill}{rgb}{0.200000,0.200000,0.800000}%
\pgfsetfillcolor{currentfill}%
\pgfsetlinewidth{1.003750pt}%
\definecolor{currentstroke}{rgb}{0.200000,0.200000,0.800000}%
\pgfsetstrokecolor{currentstroke}%
\pgfsetdash{}{0pt}%
\pgfpathmoveto{\pgfqpoint{3.121723in}{2.999115in}}%
\pgfpathcurveto{\pgfqpoint{3.127547in}{2.999115in}}{\pgfqpoint{3.133133in}{3.001429in}}{\pgfqpoint{3.137251in}{3.005547in}}%
\pgfpathcurveto{\pgfqpoint{3.141369in}{3.009665in}}{\pgfqpoint{3.143683in}{3.015252in}}{\pgfqpoint{3.143683in}{3.021076in}}%
\pgfpathcurveto{\pgfqpoint{3.143683in}{3.026900in}}{\pgfqpoint{3.141369in}{3.032486in}}{\pgfqpoint{3.137251in}{3.036604in}}%
\pgfpathcurveto{\pgfqpoint{3.133133in}{3.040722in}}{\pgfqpoint{3.127547in}{3.043036in}}{\pgfqpoint{3.121723in}{3.043036in}}%
\pgfpathcurveto{\pgfqpoint{3.115899in}{3.043036in}}{\pgfqpoint{3.110313in}{3.040722in}}{\pgfqpoint{3.106195in}{3.036604in}}%
\pgfpathcurveto{\pgfqpoint{3.102077in}{3.032486in}}{\pgfqpoint{3.099763in}{3.026900in}}{\pgfqpoint{3.099763in}{3.021076in}}%
\pgfpathcurveto{\pgfqpoint{3.099763in}{3.015252in}}{\pgfqpoint{3.102077in}{3.009665in}}{\pgfqpoint{3.106195in}{3.005547in}}%
\pgfpathcurveto{\pgfqpoint{3.110313in}{3.001429in}}{\pgfqpoint{3.115899in}{2.999115in}}{\pgfqpoint{3.121723in}{2.999115in}}%
\pgfpathlineto{\pgfqpoint{3.121723in}{2.999115in}}%
\pgfpathclose%
\pgfusepath{stroke,fill}%
\end{pgfscope}%
\begin{pgfscope}%
\pgfpathrectangle{\pgfqpoint{0.997489in}{0.528000in}}{\pgfqpoint{4.565023in}{3.696000in}}%
\pgfusepath{clip}%
\pgfsetbuttcap%
\pgfsetroundjoin%
\definecolor{currentfill}{rgb}{0.200000,0.800000,0.200000}%
\pgfsetfillcolor{currentfill}%
\pgfsetlinewidth{1.003750pt}%
\definecolor{currentstroke}{rgb}{0.200000,0.800000,0.200000}%
\pgfsetstrokecolor{currentstroke}%
\pgfsetdash{}{0pt}%
\pgfpathmoveto{\pgfqpoint{3.061491in}{3.107596in}}%
\pgfpathcurveto{\pgfqpoint{3.067315in}{3.107596in}}{\pgfqpoint{3.072901in}{3.109909in}}{\pgfqpoint{3.077019in}{3.114028in}}%
\pgfpathcurveto{\pgfqpoint{3.081138in}{3.118146in}}{\pgfqpoint{3.083451in}{3.123732in}}{\pgfqpoint{3.083451in}{3.129556in}}%
\pgfpathcurveto{\pgfqpoint{3.083451in}{3.135380in}}{\pgfqpoint{3.081138in}{3.140966in}}{\pgfqpoint{3.077019in}{3.145084in}}%
\pgfpathcurveto{\pgfqpoint{3.072901in}{3.149202in}}{\pgfqpoint{3.067315in}{3.151516in}}{\pgfqpoint{3.061491in}{3.151516in}}%
\pgfpathcurveto{\pgfqpoint{3.055667in}{3.151516in}}{\pgfqpoint{3.050081in}{3.149202in}}{\pgfqpoint{3.045963in}{3.145084in}}%
\pgfpathcurveto{\pgfqpoint{3.041845in}{3.140966in}}{\pgfqpoint{3.039531in}{3.135380in}}{\pgfqpoint{3.039531in}{3.129556in}}%
\pgfpathcurveto{\pgfqpoint{3.039531in}{3.123732in}}{\pgfqpoint{3.041845in}{3.118146in}}{\pgfqpoint{3.045963in}{3.114028in}}%
\pgfpathcurveto{\pgfqpoint{3.050081in}{3.109909in}}{\pgfqpoint{3.055667in}{3.107596in}}{\pgfqpoint{3.061491in}{3.107596in}}%
\pgfpathlineto{\pgfqpoint{3.061491in}{3.107596in}}%
\pgfpathclose%
\pgfusepath{stroke,fill}%
\end{pgfscope}%
\begin{pgfscope}%
\pgfpathrectangle{\pgfqpoint{0.997489in}{0.528000in}}{\pgfqpoint{4.565023in}{3.696000in}}%
\pgfusepath{clip}%
\pgfsetbuttcap%
\pgfsetroundjoin%
\definecolor{currentfill}{rgb}{0.200000,0.800000,0.200000}%
\pgfsetfillcolor{currentfill}%
\pgfsetlinewidth{1.003750pt}%
\definecolor{currentstroke}{rgb}{0.200000,0.800000,0.200000}%
\pgfsetstrokecolor{currentstroke}%
\pgfsetdash{}{0pt}%
\pgfpathmoveto{\pgfqpoint{3.036237in}{3.068239in}}%
\pgfpathcurveto{\pgfqpoint{3.042061in}{3.068239in}}{\pgfqpoint{3.047647in}{3.070553in}}{\pgfqpoint{3.051765in}{3.074671in}}%
\pgfpathcurveto{\pgfqpoint{3.055883in}{3.078789in}}{\pgfqpoint{3.058197in}{3.084375in}}{\pgfqpoint{3.058197in}{3.090199in}}%
\pgfpathcurveto{\pgfqpoint{3.058197in}{3.096023in}}{\pgfqpoint{3.055883in}{3.101609in}}{\pgfqpoint{3.051765in}{3.105727in}}%
\pgfpathcurveto{\pgfqpoint{3.047647in}{3.109845in}}{\pgfqpoint{3.042061in}{3.112159in}}{\pgfqpoint{3.036237in}{3.112159in}}%
\pgfpathcurveto{\pgfqpoint{3.030413in}{3.112159in}}{\pgfqpoint{3.024827in}{3.109845in}}{\pgfqpoint{3.020709in}{3.105727in}}%
\pgfpathcurveto{\pgfqpoint{3.016591in}{3.101609in}}{\pgfqpoint{3.014277in}{3.096023in}}{\pgfqpoint{3.014277in}{3.090199in}}%
\pgfpathcurveto{\pgfqpoint{3.014277in}{3.084375in}}{\pgfqpoint{3.016591in}{3.078789in}}{\pgfqpoint{3.020709in}{3.074671in}}%
\pgfpathcurveto{\pgfqpoint{3.024827in}{3.070553in}}{\pgfqpoint{3.030413in}{3.068239in}}{\pgfqpoint{3.036237in}{3.068239in}}%
\pgfpathlineto{\pgfqpoint{3.036237in}{3.068239in}}%
\pgfpathclose%
\pgfusepath{stroke,fill}%
\end{pgfscope}%
\begin{pgfscope}%
\pgfpathrectangle{\pgfqpoint{0.997489in}{0.528000in}}{\pgfqpoint{4.565023in}{3.696000in}}%
\pgfusepath{clip}%
\pgfsetbuttcap%
\pgfsetroundjoin%
\definecolor{currentfill}{rgb}{0.200000,0.800000,0.200000}%
\pgfsetfillcolor{currentfill}%
\pgfsetlinewidth{1.003750pt}%
\definecolor{currentstroke}{rgb}{0.200000,0.800000,0.200000}%
\pgfsetstrokecolor{currentstroke}%
\pgfsetdash{}{0pt}%
\pgfpathmoveto{\pgfqpoint{3.042387in}{2.963409in}}%
\pgfpathcurveto{\pgfqpoint{3.048211in}{2.963409in}}{\pgfqpoint{3.053797in}{2.965722in}}{\pgfqpoint{3.057915in}{2.969841in}}%
\pgfpathcurveto{\pgfqpoint{3.062033in}{2.973959in}}{\pgfqpoint{3.064347in}{2.979545in}}{\pgfqpoint{3.064347in}{2.985369in}}%
\pgfpathcurveto{\pgfqpoint{3.064347in}{2.991193in}}{\pgfqpoint{3.062033in}{2.996779in}}{\pgfqpoint{3.057915in}{3.000897in}}%
\pgfpathcurveto{\pgfqpoint{3.053797in}{3.005015in}}{\pgfqpoint{3.048211in}{3.007329in}}{\pgfqpoint{3.042387in}{3.007329in}}%
\pgfpathcurveto{\pgfqpoint{3.036563in}{3.007329in}}{\pgfqpoint{3.030976in}{3.005015in}}{\pgfqpoint{3.026858in}{3.000897in}}%
\pgfpathcurveto{\pgfqpoint{3.022740in}{2.996779in}}{\pgfqpoint{3.020426in}{2.991193in}}{\pgfqpoint{3.020426in}{2.985369in}}%
\pgfpathcurveto{\pgfqpoint{3.020426in}{2.979545in}}{\pgfqpoint{3.022740in}{2.973959in}}{\pgfqpoint{3.026858in}{2.969841in}}%
\pgfpathcurveto{\pgfqpoint{3.030976in}{2.965722in}}{\pgfqpoint{3.036563in}{2.963409in}}{\pgfqpoint{3.042387in}{2.963409in}}%
\pgfpathlineto{\pgfqpoint{3.042387in}{2.963409in}}%
\pgfpathclose%
\pgfusepath{stroke,fill}%
\end{pgfscope}%
\begin{pgfscope}%
\pgfpathrectangle{\pgfqpoint{0.997489in}{0.528000in}}{\pgfqpoint{4.565023in}{3.696000in}}%
\pgfusepath{clip}%
\pgfsetbuttcap%
\pgfsetroundjoin%
\definecolor{currentfill}{rgb}{0.200000,0.800000,0.200000}%
\pgfsetfillcolor{currentfill}%
\pgfsetlinewidth{1.003750pt}%
\definecolor{currentstroke}{rgb}{0.200000,0.800000,0.200000}%
\pgfsetstrokecolor{currentstroke}%
\pgfsetdash{}{0pt}%
\pgfpathmoveto{\pgfqpoint{2.980696in}{3.023757in}}%
\pgfpathcurveto{\pgfqpoint{2.986520in}{3.023757in}}{\pgfqpoint{2.992106in}{3.026071in}}{\pgfqpoint{2.996225in}{3.030189in}}%
\pgfpathcurveto{\pgfqpoint{3.000343in}{3.034307in}}{\pgfqpoint{3.002657in}{3.039893in}}{\pgfqpoint{3.002657in}{3.045717in}}%
\pgfpathcurveto{\pgfqpoint{3.002657in}{3.051541in}}{\pgfqpoint{3.000343in}{3.057127in}}{\pgfqpoint{2.996225in}{3.061246in}}%
\pgfpathcurveto{\pgfqpoint{2.992106in}{3.065364in}}{\pgfqpoint{2.986520in}{3.067678in}}{\pgfqpoint{2.980696in}{3.067678in}}%
\pgfpathcurveto{\pgfqpoint{2.974872in}{3.067678in}}{\pgfqpoint{2.969286in}{3.065364in}}{\pgfqpoint{2.965168in}{3.061246in}}%
\pgfpathcurveto{\pgfqpoint{2.961050in}{3.057127in}}{\pgfqpoint{2.958736in}{3.051541in}}{\pgfqpoint{2.958736in}{3.045717in}}%
\pgfpathcurveto{\pgfqpoint{2.958736in}{3.039893in}}{\pgfqpoint{2.961050in}{3.034307in}}{\pgfqpoint{2.965168in}{3.030189in}}%
\pgfpathcurveto{\pgfqpoint{2.969286in}{3.026071in}}{\pgfqpoint{2.974872in}{3.023757in}}{\pgfqpoint{2.980696in}{3.023757in}}%
\pgfpathlineto{\pgfqpoint{2.980696in}{3.023757in}}%
\pgfpathclose%
\pgfusepath{stroke,fill}%
\end{pgfscope}%
\begin{pgfscope}%
\pgfpathrectangle{\pgfqpoint{0.997489in}{0.528000in}}{\pgfqpoint{4.565023in}{3.696000in}}%
\pgfusepath{clip}%
\pgfsetbuttcap%
\pgfsetroundjoin%
\definecolor{currentfill}{rgb}{0.200000,0.200000,0.800000}%
\pgfsetfillcolor{currentfill}%
\pgfsetlinewidth{1.003750pt}%
\definecolor{currentstroke}{rgb}{0.200000,0.200000,0.800000}%
\pgfsetstrokecolor{currentstroke}%
\pgfsetdash{}{0pt}%
\pgfpathmoveto{\pgfqpoint{2.968050in}{2.978492in}}%
\pgfpathcurveto{\pgfqpoint{2.973874in}{2.978492in}}{\pgfqpoint{2.979460in}{2.980806in}}{\pgfqpoint{2.983578in}{2.984924in}}%
\pgfpathcurveto{\pgfqpoint{2.987696in}{2.989042in}}{\pgfqpoint{2.990010in}{2.994629in}}{\pgfqpoint{2.990010in}{3.000452in}}%
\pgfpathcurveto{\pgfqpoint{2.990010in}{3.006276in}}{\pgfqpoint{2.987696in}{3.011863in}}{\pgfqpoint{2.983578in}{3.015981in}}%
\pgfpathcurveto{\pgfqpoint{2.979460in}{3.020099in}}{\pgfqpoint{2.973874in}{3.022413in}}{\pgfqpoint{2.968050in}{3.022413in}}%
\pgfpathcurveto{\pgfqpoint{2.962226in}{3.022413in}}{\pgfqpoint{2.956639in}{3.020099in}}{\pgfqpoint{2.952521in}{3.015981in}}%
\pgfpathcurveto{\pgfqpoint{2.948403in}{3.011863in}}{\pgfqpoint{2.946089in}{3.006276in}}{\pgfqpoint{2.946089in}{3.000452in}}%
\pgfpathcurveto{\pgfqpoint{2.946089in}{2.994629in}}{\pgfqpoint{2.948403in}{2.989042in}}{\pgfqpoint{2.952521in}{2.984924in}}%
\pgfpathcurveto{\pgfqpoint{2.956639in}{2.980806in}}{\pgfqpoint{2.962226in}{2.978492in}}{\pgfqpoint{2.968050in}{2.978492in}}%
\pgfpathlineto{\pgfqpoint{2.968050in}{2.978492in}}%
\pgfpathclose%
\pgfusepath{stroke,fill}%
\end{pgfscope}%
\begin{pgfscope}%
\pgfpathrectangle{\pgfqpoint{0.997489in}{0.528000in}}{\pgfqpoint{4.565023in}{3.696000in}}%
\pgfusepath{clip}%
\pgfsetbuttcap%
\pgfsetroundjoin%
\definecolor{currentfill}{rgb}{0.200000,0.200000,0.800000}%
\pgfsetfillcolor{currentfill}%
\pgfsetlinewidth{1.003750pt}%
\definecolor{currentstroke}{rgb}{0.200000,0.200000,0.800000}%
\pgfsetstrokecolor{currentstroke}%
\pgfsetdash{}{0pt}%
\pgfpathmoveto{\pgfqpoint{2.920689in}{2.992933in}}%
\pgfpathcurveto{\pgfqpoint{2.926513in}{2.992933in}}{\pgfqpoint{2.932099in}{2.995247in}}{\pgfqpoint{2.936217in}{2.999365in}}%
\pgfpathcurveto{\pgfqpoint{2.940335in}{3.003483in}}{\pgfqpoint{2.942649in}{3.009069in}}{\pgfqpoint{2.942649in}{3.014893in}}%
\pgfpathcurveto{\pgfqpoint{2.942649in}{3.020717in}}{\pgfqpoint{2.940335in}{3.026303in}}{\pgfqpoint{2.936217in}{3.030421in}}%
\pgfpathcurveto{\pgfqpoint{2.932099in}{3.034539in}}{\pgfqpoint{2.926513in}{3.036853in}}{\pgfqpoint{2.920689in}{3.036853in}}%
\pgfpathcurveto{\pgfqpoint{2.914865in}{3.036853in}}{\pgfqpoint{2.909279in}{3.034539in}}{\pgfqpoint{2.905161in}{3.030421in}}%
\pgfpathcurveto{\pgfqpoint{2.901043in}{3.026303in}}{\pgfqpoint{2.898729in}{3.020717in}}{\pgfqpoint{2.898729in}{3.014893in}}%
\pgfpathcurveto{\pgfqpoint{2.898729in}{3.009069in}}{\pgfqpoint{2.901043in}{3.003483in}}{\pgfqpoint{2.905161in}{2.999365in}}%
\pgfpathcurveto{\pgfqpoint{2.909279in}{2.995247in}}{\pgfqpoint{2.914865in}{2.992933in}}{\pgfqpoint{2.920689in}{2.992933in}}%
\pgfpathlineto{\pgfqpoint{2.920689in}{2.992933in}}%
\pgfpathclose%
\pgfusepath{stroke,fill}%
\end{pgfscope}%
\begin{pgfscope}%
\pgfpathrectangle{\pgfqpoint{0.997489in}{0.528000in}}{\pgfqpoint{4.565023in}{3.696000in}}%
\pgfusepath{clip}%
\pgfsetbuttcap%
\pgfsetroundjoin%
\definecolor{currentfill}{rgb}{0.200000,0.200000,0.800000}%
\pgfsetfillcolor{currentfill}%
\pgfsetlinewidth{1.003750pt}%
\definecolor{currentstroke}{rgb}{0.200000,0.200000,0.800000}%
\pgfsetstrokecolor{currentstroke}%
\pgfsetdash{}{0pt}%
\pgfpathmoveto{\pgfqpoint{2.951232in}{2.896997in}}%
\pgfpathcurveto{\pgfqpoint{2.957056in}{2.896997in}}{\pgfqpoint{2.962642in}{2.899311in}}{\pgfqpoint{2.966760in}{2.903429in}}%
\pgfpathcurveto{\pgfqpoint{2.970878in}{2.907547in}}{\pgfqpoint{2.973192in}{2.913133in}}{\pgfqpoint{2.973192in}{2.918957in}}%
\pgfpathcurveto{\pgfqpoint{2.973192in}{2.924781in}}{\pgfqpoint{2.970878in}{2.930367in}}{\pgfqpoint{2.966760in}{2.934485in}}%
\pgfpathcurveto{\pgfqpoint{2.962642in}{2.938603in}}{\pgfqpoint{2.957056in}{2.940917in}}{\pgfqpoint{2.951232in}{2.940917in}}%
\pgfpathcurveto{\pgfqpoint{2.945408in}{2.940917in}}{\pgfqpoint{2.939822in}{2.938603in}}{\pgfqpoint{2.935703in}{2.934485in}}%
\pgfpathcurveto{\pgfqpoint{2.931585in}{2.930367in}}{\pgfqpoint{2.929271in}{2.924781in}}{\pgfqpoint{2.929271in}{2.918957in}}%
\pgfpathcurveto{\pgfqpoint{2.929271in}{2.913133in}}{\pgfqpoint{2.931585in}{2.907547in}}{\pgfqpoint{2.935703in}{2.903429in}}%
\pgfpathcurveto{\pgfqpoint{2.939822in}{2.899311in}}{\pgfqpoint{2.945408in}{2.896997in}}{\pgfqpoint{2.951232in}{2.896997in}}%
\pgfpathlineto{\pgfqpoint{2.951232in}{2.896997in}}%
\pgfpathclose%
\pgfusepath{stroke,fill}%
\end{pgfscope}%
\begin{pgfscope}%
\pgfpathrectangle{\pgfqpoint{0.997489in}{0.528000in}}{\pgfqpoint{4.565023in}{3.696000in}}%
\pgfusepath{clip}%
\pgfsetbuttcap%
\pgfsetroundjoin%
\definecolor{currentfill}{rgb}{0.200000,0.200000,0.800000}%
\pgfsetfillcolor{currentfill}%
\pgfsetlinewidth{1.003750pt}%
\definecolor{currentstroke}{rgb}{0.200000,0.200000,0.800000}%
\pgfsetstrokecolor{currentstroke}%
\pgfsetdash{}{0pt}%
\pgfpathmoveto{\pgfqpoint{2.896074in}{2.918335in}}%
\pgfpathcurveto{\pgfqpoint{2.901898in}{2.918335in}}{\pgfqpoint{2.907484in}{2.920649in}}{\pgfqpoint{2.911602in}{2.924767in}}%
\pgfpathcurveto{\pgfqpoint{2.915720in}{2.928885in}}{\pgfqpoint{2.918034in}{2.934471in}}{\pgfqpoint{2.918034in}{2.940295in}}%
\pgfpathcurveto{\pgfqpoint{2.918034in}{2.946119in}}{\pgfqpoint{2.915720in}{2.951705in}}{\pgfqpoint{2.911602in}{2.955824in}}%
\pgfpathcurveto{\pgfqpoint{2.907484in}{2.959942in}}{\pgfqpoint{2.901898in}{2.962256in}}{\pgfqpoint{2.896074in}{2.962256in}}%
\pgfpathcurveto{\pgfqpoint{2.890250in}{2.962256in}}{\pgfqpoint{2.884663in}{2.959942in}}{\pgfqpoint{2.880545in}{2.955824in}}%
\pgfpathcurveto{\pgfqpoint{2.876427in}{2.951705in}}{\pgfqpoint{2.874113in}{2.946119in}}{\pgfqpoint{2.874113in}{2.940295in}}%
\pgfpathcurveto{\pgfqpoint{2.874113in}{2.934471in}}{\pgfqpoint{2.876427in}{2.928885in}}{\pgfqpoint{2.880545in}{2.924767in}}%
\pgfpathcurveto{\pgfqpoint{2.884663in}{2.920649in}}{\pgfqpoint{2.890250in}{2.918335in}}{\pgfqpoint{2.896074in}{2.918335in}}%
\pgfpathlineto{\pgfqpoint{2.896074in}{2.918335in}}%
\pgfpathclose%
\pgfusepath{stroke,fill}%
\end{pgfscope}%
\begin{pgfscope}%
\pgfpathrectangle{\pgfqpoint{0.997489in}{0.528000in}}{\pgfqpoint{4.565023in}{3.696000in}}%
\pgfusepath{clip}%
\pgfsetbuttcap%
\pgfsetroundjoin%
\definecolor{currentfill}{rgb}{0.200000,0.200000,0.800000}%
\pgfsetfillcolor{currentfill}%
\pgfsetlinewidth{1.003750pt}%
\definecolor{currentstroke}{rgb}{0.200000,0.200000,0.800000}%
\pgfsetstrokecolor{currentstroke}%
\pgfsetdash{}{0pt}%
\pgfpathmoveto{\pgfqpoint{2.847830in}{2.923200in}}%
\pgfpathcurveto{\pgfqpoint{2.853654in}{2.923200in}}{\pgfqpoint{2.859240in}{2.925514in}}{\pgfqpoint{2.863358in}{2.929632in}}%
\pgfpathcurveto{\pgfqpoint{2.867477in}{2.933750in}}{\pgfqpoint{2.869790in}{2.939337in}}{\pgfqpoint{2.869790in}{2.945161in}}%
\pgfpathcurveto{\pgfqpoint{2.869790in}{2.950984in}}{\pgfqpoint{2.867477in}{2.956571in}}{\pgfqpoint{2.863358in}{2.960689in}}%
\pgfpathcurveto{\pgfqpoint{2.859240in}{2.964807in}}{\pgfqpoint{2.853654in}{2.967121in}}{\pgfqpoint{2.847830in}{2.967121in}}%
\pgfpathcurveto{\pgfqpoint{2.842006in}{2.967121in}}{\pgfqpoint{2.836420in}{2.964807in}}{\pgfqpoint{2.832302in}{2.960689in}}%
\pgfpathcurveto{\pgfqpoint{2.828184in}{2.956571in}}{\pgfqpoint{2.825870in}{2.950984in}}{\pgfqpoint{2.825870in}{2.945161in}}%
\pgfpathcurveto{\pgfqpoint{2.825870in}{2.939337in}}{\pgfqpoint{2.828184in}{2.933750in}}{\pgfqpoint{2.832302in}{2.929632in}}%
\pgfpathcurveto{\pgfqpoint{2.836420in}{2.925514in}}{\pgfqpoint{2.842006in}{2.923200in}}{\pgfqpoint{2.847830in}{2.923200in}}%
\pgfpathlineto{\pgfqpoint{2.847830in}{2.923200in}}%
\pgfpathclose%
\pgfusepath{stroke,fill}%
\end{pgfscope}%
\begin{pgfscope}%
\pgfpathrectangle{\pgfqpoint{0.997489in}{0.528000in}}{\pgfqpoint{4.565023in}{3.696000in}}%
\pgfusepath{clip}%
\pgfsetbuttcap%
\pgfsetroundjoin%
\definecolor{currentfill}{rgb}{0.200000,0.200000,0.800000}%
\pgfsetfillcolor{currentfill}%
\pgfsetlinewidth{1.003750pt}%
\definecolor{currentstroke}{rgb}{0.200000,0.200000,0.800000}%
\pgfsetstrokecolor{currentstroke}%
\pgfsetdash{}{0pt}%
\pgfpathmoveto{\pgfqpoint{2.880440in}{2.849378in}}%
\pgfpathcurveto{\pgfqpoint{2.886264in}{2.849378in}}{\pgfqpoint{2.891850in}{2.851692in}}{\pgfqpoint{2.895968in}{2.855810in}}%
\pgfpathcurveto{\pgfqpoint{2.900086in}{2.859928in}}{\pgfqpoint{2.902400in}{2.865514in}}{\pgfqpoint{2.902400in}{2.871338in}}%
\pgfpathcurveto{\pgfqpoint{2.902400in}{2.877162in}}{\pgfqpoint{2.900086in}{2.882748in}}{\pgfqpoint{2.895968in}{2.886867in}}%
\pgfpathcurveto{\pgfqpoint{2.891850in}{2.890985in}}{\pgfqpoint{2.886264in}{2.893299in}}{\pgfqpoint{2.880440in}{2.893299in}}%
\pgfpathcurveto{\pgfqpoint{2.874616in}{2.893299in}}{\pgfqpoint{2.869030in}{2.890985in}}{\pgfqpoint{2.864912in}{2.886867in}}%
\pgfpathcurveto{\pgfqpoint{2.860793in}{2.882748in}}{\pgfqpoint{2.858480in}{2.877162in}}{\pgfqpoint{2.858480in}{2.871338in}}%
\pgfpathcurveto{\pgfqpoint{2.858480in}{2.865514in}}{\pgfqpoint{2.860793in}{2.859928in}}{\pgfqpoint{2.864912in}{2.855810in}}%
\pgfpathcurveto{\pgfqpoint{2.869030in}{2.851692in}}{\pgfqpoint{2.874616in}{2.849378in}}{\pgfqpoint{2.880440in}{2.849378in}}%
\pgfpathlineto{\pgfqpoint{2.880440in}{2.849378in}}%
\pgfpathclose%
\pgfusepath{stroke,fill}%
\end{pgfscope}%
\begin{pgfscope}%
\pgfpathrectangle{\pgfqpoint{0.997489in}{0.528000in}}{\pgfqpoint{4.565023in}{3.696000in}}%
\pgfusepath{clip}%
\pgfsetbuttcap%
\pgfsetroundjoin%
\definecolor{currentfill}{rgb}{0.200000,0.200000,0.800000}%
\pgfsetfillcolor{currentfill}%
\pgfsetlinewidth{1.003750pt}%
\definecolor{currentstroke}{rgb}{0.200000,0.200000,0.800000}%
\pgfsetstrokecolor{currentstroke}%
\pgfsetdash{}{0pt}%
\pgfpathmoveto{\pgfqpoint{2.807312in}{2.870574in}}%
\pgfpathcurveto{\pgfqpoint{2.813136in}{2.870574in}}{\pgfqpoint{2.818722in}{2.872888in}}{\pgfqpoint{2.822840in}{2.877006in}}%
\pgfpathcurveto{\pgfqpoint{2.826958in}{2.881124in}}{\pgfqpoint{2.829272in}{2.886710in}}{\pgfqpoint{2.829272in}{2.892534in}}%
\pgfpathcurveto{\pgfqpoint{2.829272in}{2.898358in}}{\pgfqpoint{2.826958in}{2.903944in}}{\pgfqpoint{2.822840in}{2.908062in}}%
\pgfpathcurveto{\pgfqpoint{2.818722in}{2.912181in}}{\pgfqpoint{2.813136in}{2.914494in}}{\pgfqpoint{2.807312in}{2.914494in}}%
\pgfpathcurveto{\pgfqpoint{2.801488in}{2.914494in}}{\pgfqpoint{2.795902in}{2.912181in}}{\pgfqpoint{2.791784in}{2.908062in}}%
\pgfpathcurveto{\pgfqpoint{2.787665in}{2.903944in}}{\pgfqpoint{2.785352in}{2.898358in}}{\pgfqpoint{2.785352in}{2.892534in}}%
\pgfpathcurveto{\pgfqpoint{2.785352in}{2.886710in}}{\pgfqpoint{2.787665in}{2.881124in}}{\pgfqpoint{2.791784in}{2.877006in}}%
\pgfpathcurveto{\pgfqpoint{2.795902in}{2.872888in}}{\pgfqpoint{2.801488in}{2.870574in}}{\pgfqpoint{2.807312in}{2.870574in}}%
\pgfpathlineto{\pgfqpoint{2.807312in}{2.870574in}}%
\pgfpathclose%
\pgfusepath{stroke,fill}%
\end{pgfscope}%
\begin{pgfscope}%
\pgfpathrectangle{\pgfqpoint{0.997489in}{0.528000in}}{\pgfqpoint{4.565023in}{3.696000in}}%
\pgfusepath{clip}%
\pgfsetbuttcap%
\pgfsetroundjoin%
\definecolor{currentfill}{rgb}{0.200000,0.200000,0.800000}%
\pgfsetfillcolor{currentfill}%
\pgfsetlinewidth{1.003750pt}%
\definecolor{currentstroke}{rgb}{0.200000,0.200000,0.800000}%
\pgfsetstrokecolor{currentstroke}%
\pgfsetdash{}{0pt}%
\pgfpathmoveto{\pgfqpoint{2.813067in}{2.826593in}}%
\pgfpathcurveto{\pgfqpoint{2.818891in}{2.826593in}}{\pgfqpoint{2.824477in}{2.828906in}}{\pgfqpoint{2.828595in}{2.833025in}}%
\pgfpathcurveto{\pgfqpoint{2.832713in}{2.837143in}}{\pgfqpoint{2.835027in}{2.842729in}}{\pgfqpoint{2.835027in}{2.848553in}}%
\pgfpathcurveto{\pgfqpoint{2.835027in}{2.854377in}}{\pgfqpoint{2.832713in}{2.859963in}}{\pgfqpoint{2.828595in}{2.864081in}}%
\pgfpathcurveto{\pgfqpoint{2.824477in}{2.868199in}}{\pgfqpoint{2.818891in}{2.870513in}}{\pgfqpoint{2.813067in}{2.870513in}}%
\pgfpathcurveto{\pgfqpoint{2.807243in}{2.870513in}}{\pgfqpoint{2.801657in}{2.868199in}}{\pgfqpoint{2.797539in}{2.864081in}}%
\pgfpathcurveto{\pgfqpoint{2.793420in}{2.859963in}}{\pgfqpoint{2.791107in}{2.854377in}}{\pgfqpoint{2.791107in}{2.848553in}}%
\pgfpathcurveto{\pgfqpoint{2.791107in}{2.842729in}}{\pgfqpoint{2.793420in}{2.837143in}}{\pgfqpoint{2.797539in}{2.833025in}}%
\pgfpathcurveto{\pgfqpoint{2.801657in}{2.828906in}}{\pgfqpoint{2.807243in}{2.826593in}}{\pgfqpoint{2.813067in}{2.826593in}}%
\pgfpathlineto{\pgfqpoint{2.813067in}{2.826593in}}%
\pgfpathclose%
\pgfusepath{stroke,fill}%
\end{pgfscope}%
\begin{pgfscope}%
\pgfpathrectangle{\pgfqpoint{0.997489in}{0.528000in}}{\pgfqpoint{4.565023in}{3.696000in}}%
\pgfusepath{clip}%
\pgfsetbuttcap%
\pgfsetroundjoin%
\definecolor{currentfill}{rgb}{0.200000,0.200000,0.800000}%
\pgfsetfillcolor{currentfill}%
\pgfsetlinewidth{1.003750pt}%
\definecolor{currentstroke}{rgb}{0.200000,0.200000,0.800000}%
\pgfsetstrokecolor{currentstroke}%
\pgfsetdash{}{0pt}%
\pgfpathmoveto{\pgfqpoint{2.779475in}{2.810556in}}%
\pgfpathcurveto{\pgfqpoint{2.785299in}{2.810556in}}{\pgfqpoint{2.790885in}{2.812870in}}{\pgfqpoint{2.795003in}{2.816988in}}%
\pgfpathcurveto{\pgfqpoint{2.799121in}{2.821107in}}{\pgfqpoint{2.801435in}{2.826693in}}{\pgfqpoint{2.801435in}{2.832517in}}%
\pgfpathcurveto{\pgfqpoint{2.801435in}{2.838341in}}{\pgfqpoint{2.799121in}{2.843927in}}{\pgfqpoint{2.795003in}{2.848045in}}%
\pgfpathcurveto{\pgfqpoint{2.790885in}{2.852163in}}{\pgfqpoint{2.785299in}{2.854477in}}{\pgfqpoint{2.779475in}{2.854477in}}%
\pgfpathcurveto{\pgfqpoint{2.773651in}{2.854477in}}{\pgfqpoint{2.768065in}{2.852163in}}{\pgfqpoint{2.763947in}{2.848045in}}%
\pgfpathcurveto{\pgfqpoint{2.759829in}{2.843927in}}{\pgfqpoint{2.757515in}{2.838341in}}{\pgfqpoint{2.757515in}{2.832517in}}%
\pgfpathcurveto{\pgfqpoint{2.757515in}{2.826693in}}{\pgfqpoint{2.759829in}{2.821107in}}{\pgfqpoint{2.763947in}{2.816988in}}%
\pgfpathcurveto{\pgfqpoint{2.768065in}{2.812870in}}{\pgfqpoint{2.773651in}{2.810556in}}{\pgfqpoint{2.779475in}{2.810556in}}%
\pgfpathlineto{\pgfqpoint{2.779475in}{2.810556in}}%
\pgfpathclose%
\pgfusepath{stroke,fill}%
\end{pgfscope}%
\begin{pgfscope}%
\pgfpathrectangle{\pgfqpoint{0.997489in}{0.528000in}}{\pgfqpoint{4.565023in}{3.696000in}}%
\pgfusepath{clip}%
\pgfsetbuttcap%
\pgfsetroundjoin%
\definecolor{currentfill}{rgb}{0.200000,0.200000,0.800000}%
\pgfsetfillcolor{currentfill}%
\pgfsetlinewidth{1.003750pt}%
\definecolor{currentstroke}{rgb}{0.200000,0.200000,0.800000}%
\pgfsetstrokecolor{currentstroke}%
\pgfsetdash{}{0pt}%
\pgfpathmoveto{\pgfqpoint{2.718915in}{2.805486in}}%
\pgfpathcurveto{\pgfqpoint{2.724739in}{2.805486in}}{\pgfqpoint{2.730325in}{2.807800in}}{\pgfqpoint{2.734443in}{2.811918in}}%
\pgfpathcurveto{\pgfqpoint{2.738561in}{2.816036in}}{\pgfqpoint{2.740875in}{2.821622in}}{\pgfqpoint{2.740875in}{2.827446in}}%
\pgfpathcurveto{\pgfqpoint{2.740875in}{2.833270in}}{\pgfqpoint{2.738561in}{2.838856in}}{\pgfqpoint{2.734443in}{2.842974in}}%
\pgfpathcurveto{\pgfqpoint{2.730325in}{2.847092in}}{\pgfqpoint{2.724739in}{2.849406in}}{\pgfqpoint{2.718915in}{2.849406in}}%
\pgfpathcurveto{\pgfqpoint{2.713091in}{2.849406in}}{\pgfqpoint{2.707505in}{2.847092in}}{\pgfqpoint{2.703387in}{2.842974in}}%
\pgfpathcurveto{\pgfqpoint{2.699269in}{2.838856in}}{\pgfqpoint{2.696955in}{2.833270in}}{\pgfqpoint{2.696955in}{2.827446in}}%
\pgfpathcurveto{\pgfqpoint{2.696955in}{2.821622in}}{\pgfqpoint{2.699269in}{2.816036in}}{\pgfqpoint{2.703387in}{2.811918in}}%
\pgfpathcurveto{\pgfqpoint{2.707505in}{2.807800in}}{\pgfqpoint{2.713091in}{2.805486in}}{\pgfqpoint{2.718915in}{2.805486in}}%
\pgfpathlineto{\pgfqpoint{2.718915in}{2.805486in}}%
\pgfpathclose%
\pgfusepath{stroke,fill}%
\end{pgfscope}%
\begin{pgfscope}%
\pgfpathrectangle{\pgfqpoint{0.997489in}{0.528000in}}{\pgfqpoint{4.565023in}{3.696000in}}%
\pgfusepath{clip}%
\pgfsetbuttcap%
\pgfsetroundjoin%
\definecolor{currentfill}{rgb}{0.200000,0.200000,0.800000}%
\pgfsetfillcolor{currentfill}%
\pgfsetlinewidth{1.003750pt}%
\definecolor{currentstroke}{rgb}{0.200000,0.200000,0.800000}%
\pgfsetstrokecolor{currentstroke}%
\pgfsetdash{}{0pt}%
\pgfpathmoveto{\pgfqpoint{2.623703in}{2.808449in}}%
\pgfpathcurveto{\pgfqpoint{2.629527in}{2.808449in}}{\pgfqpoint{2.635113in}{2.810762in}}{\pgfqpoint{2.639232in}{2.814881in}}%
\pgfpathcurveto{\pgfqpoint{2.643350in}{2.818999in}}{\pgfqpoint{2.645664in}{2.824585in}}{\pgfqpoint{2.645664in}{2.830409in}}%
\pgfpathcurveto{\pgfqpoint{2.645664in}{2.836233in}}{\pgfqpoint{2.643350in}{2.841819in}}{\pgfqpoint{2.639232in}{2.845937in}}%
\pgfpathcurveto{\pgfqpoint{2.635113in}{2.850055in}}{\pgfqpoint{2.629527in}{2.852369in}}{\pgfqpoint{2.623703in}{2.852369in}}%
\pgfpathcurveto{\pgfqpoint{2.617879in}{2.852369in}}{\pgfqpoint{2.612293in}{2.850055in}}{\pgfqpoint{2.608175in}{2.845937in}}%
\pgfpathcurveto{\pgfqpoint{2.604057in}{2.841819in}}{\pgfqpoint{2.601743in}{2.836233in}}{\pgfqpoint{2.601743in}{2.830409in}}%
\pgfpathcurveto{\pgfqpoint{2.601743in}{2.824585in}}{\pgfqpoint{2.604057in}{2.818999in}}{\pgfqpoint{2.608175in}{2.814881in}}%
\pgfpathcurveto{\pgfqpoint{2.612293in}{2.810762in}}{\pgfqpoint{2.617879in}{2.808449in}}{\pgfqpoint{2.623703in}{2.808449in}}%
\pgfpathlineto{\pgfqpoint{2.623703in}{2.808449in}}%
\pgfpathclose%
\pgfusepath{stroke,fill}%
\end{pgfscope}%
\begin{pgfscope}%
\pgfpathrectangle{\pgfqpoint{0.997489in}{0.528000in}}{\pgfqpoint{4.565023in}{3.696000in}}%
\pgfusepath{clip}%
\pgfsetbuttcap%
\pgfsetroundjoin%
\definecolor{currentfill}{rgb}{0.200000,0.200000,0.800000}%
\pgfsetfillcolor{currentfill}%
\pgfsetlinewidth{1.003750pt}%
\definecolor{currentstroke}{rgb}{0.200000,0.200000,0.800000}%
\pgfsetstrokecolor{currentstroke}%
\pgfsetdash{}{0pt}%
\pgfpathmoveto{\pgfqpoint{2.736262in}{2.723855in}}%
\pgfpathcurveto{\pgfqpoint{2.742086in}{2.723855in}}{\pgfqpoint{2.747672in}{2.726169in}}{\pgfqpoint{2.751790in}{2.730287in}}%
\pgfpathcurveto{\pgfqpoint{2.755908in}{2.734405in}}{\pgfqpoint{2.758222in}{2.739991in}}{\pgfqpoint{2.758222in}{2.745815in}}%
\pgfpathcurveto{\pgfqpoint{2.758222in}{2.751639in}}{\pgfqpoint{2.755908in}{2.757225in}}{\pgfqpoint{2.751790in}{2.761343in}}%
\pgfpathcurveto{\pgfqpoint{2.747672in}{2.765461in}}{\pgfqpoint{2.742086in}{2.767775in}}{\pgfqpoint{2.736262in}{2.767775in}}%
\pgfpathcurveto{\pgfqpoint{2.730438in}{2.767775in}}{\pgfqpoint{2.724852in}{2.765461in}}{\pgfqpoint{2.720734in}{2.761343in}}%
\pgfpathcurveto{\pgfqpoint{2.716615in}{2.757225in}}{\pgfqpoint{2.714301in}{2.751639in}}{\pgfqpoint{2.714301in}{2.745815in}}%
\pgfpathcurveto{\pgfqpoint{2.714301in}{2.739991in}}{\pgfqpoint{2.716615in}{2.734405in}}{\pgfqpoint{2.720734in}{2.730287in}}%
\pgfpathcurveto{\pgfqpoint{2.724852in}{2.726169in}}{\pgfqpoint{2.730438in}{2.723855in}}{\pgfqpoint{2.736262in}{2.723855in}}%
\pgfpathlineto{\pgfqpoint{2.736262in}{2.723855in}}%
\pgfpathclose%
\pgfusepath{stroke,fill}%
\end{pgfscope}%
\begin{pgfscope}%
\pgfpathrectangle{\pgfqpoint{0.997489in}{0.528000in}}{\pgfqpoint{4.565023in}{3.696000in}}%
\pgfusepath{clip}%
\pgfsetbuttcap%
\pgfsetroundjoin%
\definecolor{currentfill}{rgb}{0.200000,0.200000,0.800000}%
\pgfsetfillcolor{currentfill}%
\pgfsetlinewidth{1.003750pt}%
\definecolor{currentstroke}{rgb}{0.200000,0.200000,0.800000}%
\pgfsetstrokecolor{currentstroke}%
\pgfsetdash{}{0pt}%
\pgfpathmoveto{\pgfqpoint{2.739370in}{2.689207in}}%
\pgfpathcurveto{\pgfqpoint{2.745194in}{2.689207in}}{\pgfqpoint{2.750780in}{2.691520in}}{\pgfqpoint{2.754898in}{2.695639in}}%
\pgfpathcurveto{\pgfqpoint{2.759016in}{2.699757in}}{\pgfqpoint{2.761330in}{2.705343in}}{\pgfqpoint{2.761330in}{2.711167in}}%
\pgfpathcurveto{\pgfqpoint{2.761330in}{2.716991in}}{\pgfqpoint{2.759016in}{2.722577in}}{\pgfqpoint{2.754898in}{2.726695in}}%
\pgfpathcurveto{\pgfqpoint{2.750780in}{2.730813in}}{\pgfqpoint{2.745194in}{2.733127in}}{\pgfqpoint{2.739370in}{2.733127in}}%
\pgfpathcurveto{\pgfqpoint{2.733546in}{2.733127in}}{\pgfqpoint{2.727960in}{2.730813in}}{\pgfqpoint{2.723842in}{2.726695in}}%
\pgfpathcurveto{\pgfqpoint{2.719724in}{2.722577in}}{\pgfqpoint{2.717410in}{2.716991in}}{\pgfqpoint{2.717410in}{2.711167in}}%
\pgfpathcurveto{\pgfqpoint{2.717410in}{2.705343in}}{\pgfqpoint{2.719724in}{2.699757in}}{\pgfqpoint{2.723842in}{2.695639in}}%
\pgfpathcurveto{\pgfqpoint{2.727960in}{2.691520in}}{\pgfqpoint{2.733546in}{2.689207in}}{\pgfqpoint{2.739370in}{2.689207in}}%
\pgfpathlineto{\pgfqpoint{2.739370in}{2.689207in}}%
\pgfpathclose%
\pgfusepath{stroke,fill}%
\end{pgfscope}%
\begin{pgfscope}%
\pgfpathrectangle{\pgfqpoint{0.997489in}{0.528000in}}{\pgfqpoint{4.565023in}{3.696000in}}%
\pgfusepath{clip}%
\pgfsetbuttcap%
\pgfsetroundjoin%
\definecolor{currentfill}{rgb}{0.200000,0.200000,0.800000}%
\pgfsetfillcolor{currentfill}%
\pgfsetlinewidth{1.003750pt}%
\definecolor{currentstroke}{rgb}{0.200000,0.200000,0.800000}%
\pgfsetstrokecolor{currentstroke}%
\pgfsetdash{}{0pt}%
\pgfpathmoveto{\pgfqpoint{2.624490in}{2.682864in}}%
\pgfpathcurveto{\pgfqpoint{2.630314in}{2.682864in}}{\pgfqpoint{2.635901in}{2.685178in}}{\pgfqpoint{2.640019in}{2.689296in}}%
\pgfpathcurveto{\pgfqpoint{2.644137in}{2.693415in}}{\pgfqpoint{2.646451in}{2.699001in}}{\pgfqpoint{2.646451in}{2.704825in}}%
\pgfpathcurveto{\pgfqpoint{2.646451in}{2.710649in}}{\pgfqpoint{2.644137in}{2.716235in}}{\pgfqpoint{2.640019in}{2.720353in}}%
\pgfpathcurveto{\pgfqpoint{2.635901in}{2.724471in}}{\pgfqpoint{2.630314in}{2.726785in}}{\pgfqpoint{2.624490in}{2.726785in}}%
\pgfpathcurveto{\pgfqpoint{2.618666in}{2.726785in}}{\pgfqpoint{2.613080in}{2.724471in}}{\pgfqpoint{2.608962in}{2.720353in}}%
\pgfpathcurveto{\pgfqpoint{2.604844in}{2.716235in}}{\pgfqpoint{2.602530in}{2.710649in}}{\pgfqpoint{2.602530in}{2.704825in}}%
\pgfpathcurveto{\pgfqpoint{2.602530in}{2.699001in}}{\pgfqpoint{2.604844in}{2.693415in}}{\pgfqpoint{2.608962in}{2.689296in}}%
\pgfpathcurveto{\pgfqpoint{2.613080in}{2.685178in}}{\pgfqpoint{2.618666in}{2.682864in}}{\pgfqpoint{2.624490in}{2.682864in}}%
\pgfpathlineto{\pgfqpoint{2.624490in}{2.682864in}}%
\pgfpathclose%
\pgfusepath{stroke,fill}%
\end{pgfscope}%
\begin{pgfscope}%
\pgfpathrectangle{\pgfqpoint{0.997489in}{0.528000in}}{\pgfqpoint{4.565023in}{3.696000in}}%
\pgfusepath{clip}%
\pgfsetbuttcap%
\pgfsetroundjoin%
\definecolor{currentfill}{rgb}{0.200000,0.200000,0.800000}%
\pgfsetfillcolor{currentfill}%
\pgfsetlinewidth{1.003750pt}%
\definecolor{currentstroke}{rgb}{0.200000,0.200000,0.800000}%
\pgfsetstrokecolor{currentstroke}%
\pgfsetdash{}{0pt}%
\pgfpathmoveto{\pgfqpoint{2.657963in}{2.638569in}}%
\pgfpathcurveto{\pgfqpoint{2.663787in}{2.638569in}}{\pgfqpoint{2.669373in}{2.640883in}}{\pgfqpoint{2.673491in}{2.645001in}}%
\pgfpathcurveto{\pgfqpoint{2.677609in}{2.649119in}}{\pgfqpoint{2.679923in}{2.654705in}}{\pgfqpoint{2.679923in}{2.660529in}}%
\pgfpathcurveto{\pgfqpoint{2.679923in}{2.666353in}}{\pgfqpoint{2.677609in}{2.671939in}}{\pgfqpoint{2.673491in}{2.676057in}}%
\pgfpathcurveto{\pgfqpoint{2.669373in}{2.680175in}}{\pgfqpoint{2.663787in}{2.682489in}}{\pgfqpoint{2.657963in}{2.682489in}}%
\pgfpathcurveto{\pgfqpoint{2.652139in}{2.682489in}}{\pgfqpoint{2.646552in}{2.680175in}}{\pgfqpoint{2.642434in}{2.676057in}}%
\pgfpathcurveto{\pgfqpoint{2.638316in}{2.671939in}}{\pgfqpoint{2.636002in}{2.666353in}}{\pgfqpoint{2.636002in}{2.660529in}}%
\pgfpathcurveto{\pgfqpoint{2.636002in}{2.654705in}}{\pgfqpoint{2.638316in}{2.649119in}}{\pgfqpoint{2.642434in}{2.645001in}}%
\pgfpathcurveto{\pgfqpoint{2.646552in}{2.640883in}}{\pgfqpoint{2.652139in}{2.638569in}}{\pgfqpoint{2.657963in}{2.638569in}}%
\pgfpathlineto{\pgfqpoint{2.657963in}{2.638569in}}%
\pgfpathclose%
\pgfusepath{stroke,fill}%
\end{pgfscope}%
\begin{pgfscope}%
\pgfpathrectangle{\pgfqpoint{0.997489in}{0.528000in}}{\pgfqpoint{4.565023in}{3.696000in}}%
\pgfusepath{clip}%
\pgfsetbuttcap%
\pgfsetroundjoin%
\definecolor{currentfill}{rgb}{0.200000,0.200000,0.800000}%
\pgfsetfillcolor{currentfill}%
\pgfsetlinewidth{1.003750pt}%
\definecolor{currentstroke}{rgb}{0.200000,0.200000,0.800000}%
\pgfsetstrokecolor{currentstroke}%
\pgfsetdash{}{0pt}%
\pgfpathmoveto{\pgfqpoint{2.620315in}{2.606168in}}%
\pgfpathcurveto{\pgfqpoint{2.626139in}{2.606168in}}{\pgfqpoint{2.631725in}{2.608481in}}{\pgfqpoint{2.635844in}{2.612600in}}%
\pgfpathcurveto{\pgfqpoint{2.639962in}{2.616718in}}{\pgfqpoint{2.642276in}{2.622304in}}{\pgfqpoint{2.642276in}{2.628128in}}%
\pgfpathcurveto{\pgfqpoint{2.642276in}{2.633952in}}{\pgfqpoint{2.639962in}{2.639538in}}{\pgfqpoint{2.635844in}{2.643656in}}%
\pgfpathcurveto{\pgfqpoint{2.631725in}{2.647774in}}{\pgfqpoint{2.626139in}{2.650088in}}{\pgfqpoint{2.620315in}{2.650088in}}%
\pgfpathcurveto{\pgfqpoint{2.614491in}{2.650088in}}{\pgfqpoint{2.608905in}{2.647774in}}{\pgfqpoint{2.604787in}{2.643656in}}%
\pgfpathcurveto{\pgfqpoint{2.600669in}{2.639538in}}{\pgfqpoint{2.598355in}{2.633952in}}{\pgfqpoint{2.598355in}{2.628128in}}%
\pgfpathcurveto{\pgfqpoint{2.598355in}{2.622304in}}{\pgfqpoint{2.600669in}{2.616718in}}{\pgfqpoint{2.604787in}{2.612600in}}%
\pgfpathcurveto{\pgfqpoint{2.608905in}{2.608481in}}{\pgfqpoint{2.614491in}{2.606168in}}{\pgfqpoint{2.620315in}{2.606168in}}%
\pgfpathlineto{\pgfqpoint{2.620315in}{2.606168in}}%
\pgfpathclose%
\pgfusepath{stroke,fill}%
\end{pgfscope}%
\begin{pgfscope}%
\pgfpathrectangle{\pgfqpoint{0.997489in}{0.528000in}}{\pgfqpoint{4.565023in}{3.696000in}}%
\pgfusepath{clip}%
\pgfsetbuttcap%
\pgfsetroundjoin%
\definecolor{currentfill}{rgb}{0.200000,0.200000,0.800000}%
\pgfsetfillcolor{currentfill}%
\pgfsetlinewidth{1.003750pt}%
\definecolor{currentstroke}{rgb}{0.200000,0.200000,0.800000}%
\pgfsetstrokecolor{currentstroke}%
\pgfsetdash{}{0pt}%
\pgfpathmoveto{\pgfqpoint{2.697452in}{2.565757in}}%
\pgfpathcurveto{\pgfqpoint{2.703276in}{2.565757in}}{\pgfqpoint{2.708862in}{2.568071in}}{\pgfqpoint{2.712980in}{2.572189in}}%
\pgfpathcurveto{\pgfqpoint{2.717098in}{2.576307in}}{\pgfqpoint{2.719412in}{2.581894in}}{\pgfqpoint{2.719412in}{2.587718in}}%
\pgfpathcurveto{\pgfqpoint{2.719412in}{2.593541in}}{\pgfqpoint{2.717098in}{2.599128in}}{\pgfqpoint{2.712980in}{2.603246in}}%
\pgfpathcurveto{\pgfqpoint{2.708862in}{2.607364in}}{\pgfqpoint{2.703276in}{2.609678in}}{\pgfqpoint{2.697452in}{2.609678in}}%
\pgfpathcurveto{\pgfqpoint{2.691628in}{2.609678in}}{\pgfqpoint{2.686042in}{2.607364in}}{\pgfqpoint{2.681923in}{2.603246in}}%
\pgfpathcurveto{\pgfqpoint{2.677805in}{2.599128in}}{\pgfqpoint{2.675491in}{2.593541in}}{\pgfqpoint{2.675491in}{2.587718in}}%
\pgfpathcurveto{\pgfqpoint{2.675491in}{2.581894in}}{\pgfqpoint{2.677805in}{2.576307in}}{\pgfqpoint{2.681923in}{2.572189in}}%
\pgfpathcurveto{\pgfqpoint{2.686042in}{2.568071in}}{\pgfqpoint{2.691628in}{2.565757in}}{\pgfqpoint{2.697452in}{2.565757in}}%
\pgfpathlineto{\pgfqpoint{2.697452in}{2.565757in}}%
\pgfpathclose%
\pgfusepath{stroke,fill}%
\end{pgfscope}%
\begin{pgfscope}%
\pgfpathrectangle{\pgfqpoint{0.997489in}{0.528000in}}{\pgfqpoint{4.565023in}{3.696000in}}%
\pgfusepath{clip}%
\pgfsetbuttcap%
\pgfsetroundjoin%
\definecolor{currentfill}{rgb}{0.200000,0.200000,0.800000}%
\pgfsetfillcolor{currentfill}%
\pgfsetlinewidth{1.003750pt}%
\definecolor{currentstroke}{rgb}{0.200000,0.200000,0.800000}%
\pgfsetstrokecolor{currentstroke}%
\pgfsetdash{}{0pt}%
\pgfpathmoveto{\pgfqpoint{2.667552in}{2.531897in}}%
\pgfpathcurveto{\pgfqpoint{2.673376in}{2.531897in}}{\pgfqpoint{2.678962in}{2.534210in}}{\pgfqpoint{2.683080in}{2.538329in}}%
\pgfpathcurveto{\pgfqpoint{2.687198in}{2.542447in}}{\pgfqpoint{2.689512in}{2.548033in}}{\pgfqpoint{2.689512in}{2.553857in}}%
\pgfpathcurveto{\pgfqpoint{2.689512in}{2.559681in}}{\pgfqpoint{2.687198in}{2.565267in}}{\pgfqpoint{2.683080in}{2.569385in}}%
\pgfpathcurveto{\pgfqpoint{2.678962in}{2.573503in}}{\pgfqpoint{2.673376in}{2.575817in}}{\pgfqpoint{2.667552in}{2.575817in}}%
\pgfpathcurveto{\pgfqpoint{2.661728in}{2.575817in}}{\pgfqpoint{2.656142in}{2.573503in}}{\pgfqpoint{2.652023in}{2.569385in}}%
\pgfpathcurveto{\pgfqpoint{2.647905in}{2.565267in}}{\pgfqpoint{2.645591in}{2.559681in}}{\pgfqpoint{2.645591in}{2.553857in}}%
\pgfpathcurveto{\pgfqpoint{2.645591in}{2.548033in}}{\pgfqpoint{2.647905in}{2.542447in}}{\pgfqpoint{2.652023in}{2.538329in}}%
\pgfpathcurveto{\pgfqpoint{2.656142in}{2.534210in}}{\pgfqpoint{2.661728in}{2.531897in}}{\pgfqpoint{2.667552in}{2.531897in}}%
\pgfpathlineto{\pgfqpoint{2.667552in}{2.531897in}}%
\pgfpathclose%
\pgfusepath{stroke,fill}%
\end{pgfscope}%
\begin{pgfscope}%
\pgfpathrectangle{\pgfqpoint{0.997489in}{0.528000in}}{\pgfqpoint{4.565023in}{3.696000in}}%
\pgfusepath{clip}%
\pgfsetbuttcap%
\pgfsetroundjoin%
\definecolor{currentfill}{rgb}{0.200000,0.200000,0.800000}%
\pgfsetfillcolor{currentfill}%
\pgfsetlinewidth{1.003750pt}%
\definecolor{currentstroke}{rgb}{0.200000,0.200000,0.800000}%
\pgfsetstrokecolor{currentstroke}%
\pgfsetdash{}{0pt}%
\pgfpathmoveto{\pgfqpoint{2.709320in}{2.500934in}}%
\pgfpathcurveto{\pgfqpoint{2.715144in}{2.500934in}}{\pgfqpoint{2.720731in}{2.503248in}}{\pgfqpoint{2.724849in}{2.507366in}}%
\pgfpathcurveto{\pgfqpoint{2.728967in}{2.511485in}}{\pgfqpoint{2.731281in}{2.517071in}}{\pgfqpoint{2.731281in}{2.522895in}}%
\pgfpathcurveto{\pgfqpoint{2.731281in}{2.528719in}}{\pgfqpoint{2.728967in}{2.534305in}}{\pgfqpoint{2.724849in}{2.538423in}}%
\pgfpathcurveto{\pgfqpoint{2.720731in}{2.542541in}}{\pgfqpoint{2.715144in}{2.544855in}}{\pgfqpoint{2.709320in}{2.544855in}}%
\pgfpathcurveto{\pgfqpoint{2.703497in}{2.544855in}}{\pgfqpoint{2.697910in}{2.542541in}}{\pgfqpoint{2.693792in}{2.538423in}}%
\pgfpathcurveto{\pgfqpoint{2.689674in}{2.534305in}}{\pgfqpoint{2.687360in}{2.528719in}}{\pgfqpoint{2.687360in}{2.522895in}}%
\pgfpathcurveto{\pgfqpoint{2.687360in}{2.517071in}}{\pgfqpoint{2.689674in}{2.511485in}}{\pgfqpoint{2.693792in}{2.507366in}}%
\pgfpathcurveto{\pgfqpoint{2.697910in}{2.503248in}}{\pgfqpoint{2.703497in}{2.500934in}}{\pgfqpoint{2.709320in}{2.500934in}}%
\pgfpathlineto{\pgfqpoint{2.709320in}{2.500934in}}%
\pgfpathclose%
\pgfusepath{stroke,fill}%
\end{pgfscope}%
\begin{pgfscope}%
\pgfpathrectangle{\pgfqpoint{0.997489in}{0.528000in}}{\pgfqpoint{4.565023in}{3.696000in}}%
\pgfusepath{clip}%
\pgfsetbuttcap%
\pgfsetroundjoin%
\definecolor{currentfill}{rgb}{0.200000,0.200000,0.800000}%
\pgfsetfillcolor{currentfill}%
\pgfsetlinewidth{1.003750pt}%
\definecolor{currentstroke}{rgb}{0.200000,0.200000,0.800000}%
\pgfsetstrokecolor{currentstroke}%
\pgfsetdash{}{0pt}%
\pgfpathmoveto{\pgfqpoint{2.712992in}{2.468840in}}%
\pgfpathcurveto{\pgfqpoint{2.718816in}{2.468840in}}{\pgfqpoint{2.724402in}{2.471154in}}{\pgfqpoint{2.728520in}{2.475272in}}%
\pgfpathcurveto{\pgfqpoint{2.732639in}{2.479390in}}{\pgfqpoint{2.734952in}{2.484976in}}{\pgfqpoint{2.734952in}{2.490800in}}%
\pgfpathcurveto{\pgfqpoint{2.734952in}{2.496624in}}{\pgfqpoint{2.732639in}{2.502210in}}{\pgfqpoint{2.728520in}{2.506328in}}%
\pgfpathcurveto{\pgfqpoint{2.724402in}{2.510446in}}{\pgfqpoint{2.718816in}{2.512760in}}{\pgfqpoint{2.712992in}{2.512760in}}%
\pgfpathcurveto{\pgfqpoint{2.707168in}{2.512760in}}{\pgfqpoint{2.701582in}{2.510446in}}{\pgfqpoint{2.697464in}{2.506328in}}%
\pgfpathcurveto{\pgfqpoint{2.693346in}{2.502210in}}{\pgfqpoint{2.691032in}{2.496624in}}{\pgfqpoint{2.691032in}{2.490800in}}%
\pgfpathcurveto{\pgfqpoint{2.691032in}{2.484976in}}{\pgfqpoint{2.693346in}{2.479390in}}{\pgfqpoint{2.697464in}{2.475272in}}%
\pgfpathcurveto{\pgfqpoint{2.701582in}{2.471154in}}{\pgfqpoint{2.707168in}{2.468840in}}{\pgfqpoint{2.712992in}{2.468840in}}%
\pgfpathlineto{\pgfqpoint{2.712992in}{2.468840in}}%
\pgfpathclose%
\pgfusepath{stroke,fill}%
\end{pgfscope}%
\begin{pgfscope}%
\pgfpathrectangle{\pgfqpoint{0.997489in}{0.528000in}}{\pgfqpoint{4.565023in}{3.696000in}}%
\pgfusepath{clip}%
\pgfsetbuttcap%
\pgfsetroundjoin%
\definecolor{currentfill}{rgb}{0.200000,0.200000,0.800000}%
\pgfsetfillcolor{currentfill}%
\pgfsetlinewidth{1.003750pt}%
\definecolor{currentstroke}{rgb}{0.200000,0.200000,0.800000}%
\pgfsetstrokecolor{currentstroke}%
\pgfsetdash{}{0pt}%
\pgfpathmoveto{\pgfqpoint{2.716381in}{2.436493in}}%
\pgfpathcurveto{\pgfqpoint{2.722205in}{2.436493in}}{\pgfqpoint{2.727791in}{2.438807in}}{\pgfqpoint{2.731909in}{2.442925in}}%
\pgfpathcurveto{\pgfqpoint{2.736028in}{2.447043in}}{\pgfqpoint{2.738341in}{2.452629in}}{\pgfqpoint{2.738341in}{2.458453in}}%
\pgfpathcurveto{\pgfqpoint{2.738341in}{2.464277in}}{\pgfqpoint{2.736028in}{2.469863in}}{\pgfqpoint{2.731909in}{2.473982in}}%
\pgfpathcurveto{\pgfqpoint{2.727791in}{2.478100in}}{\pgfqpoint{2.722205in}{2.480414in}}{\pgfqpoint{2.716381in}{2.480414in}}%
\pgfpathcurveto{\pgfqpoint{2.710557in}{2.480414in}}{\pgfqpoint{2.704971in}{2.478100in}}{\pgfqpoint{2.700853in}{2.473982in}}%
\pgfpathcurveto{\pgfqpoint{2.696735in}{2.469863in}}{\pgfqpoint{2.694421in}{2.464277in}}{\pgfqpoint{2.694421in}{2.458453in}}%
\pgfpathcurveto{\pgfqpoint{2.694421in}{2.452629in}}{\pgfqpoint{2.696735in}{2.447043in}}{\pgfqpoint{2.700853in}{2.442925in}}%
\pgfpathcurveto{\pgfqpoint{2.704971in}{2.438807in}}{\pgfqpoint{2.710557in}{2.436493in}}{\pgfqpoint{2.716381in}{2.436493in}}%
\pgfpathlineto{\pgfqpoint{2.716381in}{2.436493in}}%
\pgfpathclose%
\pgfusepath{stroke,fill}%
\end{pgfscope}%
\begin{pgfscope}%
\pgfpathrectangle{\pgfqpoint{0.997489in}{0.528000in}}{\pgfqpoint{4.565023in}{3.696000in}}%
\pgfusepath{clip}%
\pgfsetbuttcap%
\pgfsetroundjoin%
\definecolor{currentfill}{rgb}{0.200000,0.200000,0.800000}%
\pgfsetfillcolor{currentfill}%
\pgfsetlinewidth{1.003750pt}%
\definecolor{currentstroke}{rgb}{0.200000,0.200000,0.800000}%
\pgfsetstrokecolor{currentstroke}%
\pgfsetdash{}{0pt}%
\pgfpathmoveto{\pgfqpoint{2.709355in}{2.400583in}}%
\pgfpathcurveto{\pgfqpoint{2.715179in}{2.400583in}}{\pgfqpoint{2.720765in}{2.402897in}}{\pgfqpoint{2.724883in}{2.407015in}}%
\pgfpathcurveto{\pgfqpoint{2.729001in}{2.411133in}}{\pgfqpoint{2.731315in}{2.416719in}}{\pgfqpoint{2.731315in}{2.422543in}}%
\pgfpathcurveto{\pgfqpoint{2.731315in}{2.428367in}}{\pgfqpoint{2.729001in}{2.433954in}}{\pgfqpoint{2.724883in}{2.438072in}}%
\pgfpathcurveto{\pgfqpoint{2.720765in}{2.442190in}}{\pgfqpoint{2.715179in}{2.444504in}}{\pgfqpoint{2.709355in}{2.444504in}}%
\pgfpathcurveto{\pgfqpoint{2.703531in}{2.444504in}}{\pgfqpoint{2.697945in}{2.442190in}}{\pgfqpoint{2.693826in}{2.438072in}}%
\pgfpathcurveto{\pgfqpoint{2.689708in}{2.433954in}}{\pgfqpoint{2.687394in}{2.428367in}}{\pgfqpoint{2.687394in}{2.422543in}}%
\pgfpathcurveto{\pgfqpoint{2.687394in}{2.416719in}}{\pgfqpoint{2.689708in}{2.411133in}}{\pgfqpoint{2.693826in}{2.407015in}}%
\pgfpathcurveto{\pgfqpoint{2.697945in}{2.402897in}}{\pgfqpoint{2.703531in}{2.400583in}}{\pgfqpoint{2.709355in}{2.400583in}}%
\pgfpathlineto{\pgfqpoint{2.709355in}{2.400583in}}%
\pgfpathclose%
\pgfusepath{stroke,fill}%
\end{pgfscope}%
\begin{pgfscope}%
\pgfpathrectangle{\pgfqpoint{0.997489in}{0.528000in}}{\pgfqpoint{4.565023in}{3.696000in}}%
\pgfusepath{clip}%
\pgfsetbuttcap%
\pgfsetroundjoin%
\definecolor{currentfill}{rgb}{0.200000,0.200000,0.800000}%
\pgfsetfillcolor{currentfill}%
\pgfsetlinewidth{1.003750pt}%
\definecolor{currentstroke}{rgb}{0.200000,0.200000,0.800000}%
\pgfsetstrokecolor{currentstroke}%
\pgfsetdash{}{0pt}%
\pgfpathmoveto{\pgfqpoint{2.736434in}{2.374811in}}%
\pgfpathcurveto{\pgfqpoint{2.742258in}{2.374811in}}{\pgfqpoint{2.747844in}{2.377125in}}{\pgfqpoint{2.751963in}{2.381243in}}%
\pgfpathcurveto{\pgfqpoint{2.756081in}{2.385361in}}{\pgfqpoint{2.758395in}{2.390947in}}{\pgfqpoint{2.758395in}{2.396771in}}%
\pgfpathcurveto{\pgfqpoint{2.758395in}{2.402595in}}{\pgfqpoint{2.756081in}{2.408181in}}{\pgfqpoint{2.751963in}{2.412300in}}%
\pgfpathcurveto{\pgfqpoint{2.747844in}{2.416418in}}{\pgfqpoint{2.742258in}{2.418732in}}{\pgfqpoint{2.736434in}{2.418732in}}%
\pgfpathcurveto{\pgfqpoint{2.730610in}{2.418732in}}{\pgfqpoint{2.725024in}{2.416418in}}{\pgfqpoint{2.720906in}{2.412300in}}%
\pgfpathcurveto{\pgfqpoint{2.716788in}{2.408181in}}{\pgfqpoint{2.714474in}{2.402595in}}{\pgfqpoint{2.714474in}{2.396771in}}%
\pgfpathcurveto{\pgfqpoint{2.714474in}{2.390947in}}{\pgfqpoint{2.716788in}{2.385361in}}{\pgfqpoint{2.720906in}{2.381243in}}%
\pgfpathcurveto{\pgfqpoint{2.725024in}{2.377125in}}{\pgfqpoint{2.730610in}{2.374811in}}{\pgfqpoint{2.736434in}{2.374811in}}%
\pgfpathlineto{\pgfqpoint{2.736434in}{2.374811in}}%
\pgfpathclose%
\pgfusepath{stroke,fill}%
\end{pgfscope}%
\begin{pgfscope}%
\pgfpathrectangle{\pgfqpoint{0.997489in}{0.528000in}}{\pgfqpoint{4.565023in}{3.696000in}}%
\pgfusepath{clip}%
\pgfsetbuttcap%
\pgfsetroundjoin%
\definecolor{currentfill}{rgb}{0.200000,0.200000,0.800000}%
\pgfsetfillcolor{currentfill}%
\pgfsetlinewidth{1.003750pt}%
\definecolor{currentstroke}{rgb}{0.200000,0.200000,0.800000}%
\pgfsetstrokecolor{currentstroke}%
\pgfsetdash{}{0pt}%
\pgfpathmoveto{\pgfqpoint{2.686173in}{2.317494in}}%
\pgfpathcurveto{\pgfqpoint{2.691997in}{2.317494in}}{\pgfqpoint{2.697583in}{2.319808in}}{\pgfqpoint{2.701701in}{2.323926in}}%
\pgfpathcurveto{\pgfqpoint{2.705819in}{2.328044in}}{\pgfqpoint{2.708133in}{2.333630in}}{\pgfqpoint{2.708133in}{2.339454in}}%
\pgfpathcurveto{\pgfqpoint{2.708133in}{2.345278in}}{\pgfqpoint{2.705819in}{2.350864in}}{\pgfqpoint{2.701701in}{2.354982in}}%
\pgfpathcurveto{\pgfqpoint{2.697583in}{2.359101in}}{\pgfqpoint{2.691997in}{2.361414in}}{\pgfqpoint{2.686173in}{2.361414in}}%
\pgfpathcurveto{\pgfqpoint{2.680349in}{2.361414in}}{\pgfqpoint{2.674763in}{2.359101in}}{\pgfqpoint{2.670645in}{2.354982in}}%
\pgfpathcurveto{\pgfqpoint{2.666526in}{2.350864in}}{\pgfqpoint{2.664213in}{2.345278in}}{\pgfqpoint{2.664213in}{2.339454in}}%
\pgfpathcurveto{\pgfqpoint{2.664213in}{2.333630in}}{\pgfqpoint{2.666526in}{2.328044in}}{\pgfqpoint{2.670645in}{2.323926in}}%
\pgfpathcurveto{\pgfqpoint{2.674763in}{2.319808in}}{\pgfqpoint{2.680349in}{2.317494in}}{\pgfqpoint{2.686173in}{2.317494in}}%
\pgfpathlineto{\pgfqpoint{2.686173in}{2.317494in}}%
\pgfpathclose%
\pgfusepath{stroke,fill}%
\end{pgfscope}%
\begin{pgfscope}%
\pgfpathrectangle{\pgfqpoint{0.997489in}{0.528000in}}{\pgfqpoint{4.565023in}{3.696000in}}%
\pgfusepath{clip}%
\pgfsetbuttcap%
\pgfsetroundjoin%
\definecolor{currentfill}{rgb}{0.200000,0.200000,0.800000}%
\pgfsetfillcolor{currentfill}%
\pgfsetlinewidth{1.003750pt}%
\definecolor{currentstroke}{rgb}{0.200000,0.200000,0.800000}%
\pgfsetstrokecolor{currentstroke}%
\pgfsetdash{}{0pt}%
\pgfpathmoveto{\pgfqpoint{2.739334in}{2.303644in}}%
\pgfpathcurveto{\pgfqpoint{2.745158in}{2.303644in}}{\pgfqpoint{2.750744in}{2.305958in}}{\pgfqpoint{2.754862in}{2.310076in}}%
\pgfpathcurveto{\pgfqpoint{2.758980in}{2.314194in}}{\pgfqpoint{2.761294in}{2.319780in}}{\pgfqpoint{2.761294in}{2.325604in}}%
\pgfpathcurveto{\pgfqpoint{2.761294in}{2.331428in}}{\pgfqpoint{2.758980in}{2.337014in}}{\pgfqpoint{2.754862in}{2.341133in}}%
\pgfpathcurveto{\pgfqpoint{2.750744in}{2.345251in}}{\pgfqpoint{2.745158in}{2.347565in}}{\pgfqpoint{2.739334in}{2.347565in}}%
\pgfpathcurveto{\pgfqpoint{2.733510in}{2.347565in}}{\pgfqpoint{2.727924in}{2.345251in}}{\pgfqpoint{2.723806in}{2.341133in}}%
\pgfpathcurveto{\pgfqpoint{2.719688in}{2.337014in}}{\pgfqpoint{2.717374in}{2.331428in}}{\pgfqpoint{2.717374in}{2.325604in}}%
\pgfpathcurveto{\pgfqpoint{2.717374in}{2.319780in}}{\pgfqpoint{2.719688in}{2.314194in}}{\pgfqpoint{2.723806in}{2.310076in}}%
\pgfpathcurveto{\pgfqpoint{2.727924in}{2.305958in}}{\pgfqpoint{2.733510in}{2.303644in}}{\pgfqpoint{2.739334in}{2.303644in}}%
\pgfpathlineto{\pgfqpoint{2.739334in}{2.303644in}}%
\pgfpathclose%
\pgfusepath{stroke,fill}%
\end{pgfscope}%
\begin{pgfscope}%
\pgfpathrectangle{\pgfqpoint{0.997489in}{0.528000in}}{\pgfqpoint{4.565023in}{3.696000in}}%
\pgfusepath{clip}%
\pgfsetbuttcap%
\pgfsetroundjoin%
\definecolor{currentfill}{rgb}{0.200000,0.800000,0.200000}%
\pgfsetfillcolor{currentfill}%
\pgfsetlinewidth{1.003750pt}%
\definecolor{currentstroke}{rgb}{0.200000,0.800000,0.200000}%
\pgfsetstrokecolor{currentstroke}%
\pgfsetdash{}{0pt}%
\pgfpathmoveto{\pgfqpoint{2.772481in}{2.283859in}}%
\pgfpathcurveto{\pgfqpoint{2.778305in}{2.283859in}}{\pgfqpoint{2.783891in}{2.286173in}}{\pgfqpoint{2.788009in}{2.290291in}}%
\pgfpathcurveto{\pgfqpoint{2.792128in}{2.294410in}}{\pgfqpoint{2.794441in}{2.299996in}}{\pgfqpoint{2.794441in}{2.305820in}}%
\pgfpathcurveto{\pgfqpoint{2.794441in}{2.311644in}}{\pgfqpoint{2.792128in}{2.317230in}}{\pgfqpoint{2.788009in}{2.321348in}}%
\pgfpathcurveto{\pgfqpoint{2.783891in}{2.325466in}}{\pgfqpoint{2.778305in}{2.327780in}}{\pgfqpoint{2.772481in}{2.327780in}}%
\pgfpathcurveto{\pgfqpoint{2.766657in}{2.327780in}}{\pgfqpoint{2.761071in}{2.325466in}}{\pgfqpoint{2.756953in}{2.321348in}}%
\pgfpathcurveto{\pgfqpoint{2.752835in}{2.317230in}}{\pgfqpoint{2.750521in}{2.311644in}}{\pgfqpoint{2.750521in}{2.305820in}}%
\pgfpathcurveto{\pgfqpoint{2.750521in}{2.299996in}}{\pgfqpoint{2.752835in}{2.294410in}}{\pgfqpoint{2.756953in}{2.290291in}}%
\pgfpathcurveto{\pgfqpoint{2.761071in}{2.286173in}}{\pgfqpoint{2.766657in}{2.283859in}}{\pgfqpoint{2.772481in}{2.283859in}}%
\pgfpathlineto{\pgfqpoint{2.772481in}{2.283859in}}%
\pgfpathclose%
\pgfusepath{stroke,fill}%
\end{pgfscope}%
\begin{pgfscope}%
\pgfpathrectangle{\pgfqpoint{0.997489in}{0.528000in}}{\pgfqpoint{4.565023in}{3.696000in}}%
\pgfusepath{clip}%
\pgfsetbuttcap%
\pgfsetroundjoin%
\definecolor{currentfill}{rgb}{0.200000,0.200000,0.800000}%
\pgfsetfillcolor{currentfill}%
\pgfsetlinewidth{1.003750pt}%
\definecolor{currentstroke}{rgb}{0.200000,0.200000,0.800000}%
\pgfsetstrokecolor{currentstroke}%
\pgfsetdash{}{0pt}%
\pgfpathmoveto{\pgfqpoint{2.791920in}{2.257452in}}%
\pgfpathcurveto{\pgfqpoint{2.797744in}{2.257452in}}{\pgfqpoint{2.803330in}{2.259766in}}{\pgfqpoint{2.807448in}{2.263884in}}%
\pgfpathcurveto{\pgfqpoint{2.811566in}{2.268002in}}{\pgfqpoint{2.813880in}{2.273588in}}{\pgfqpoint{2.813880in}{2.279412in}}%
\pgfpathcurveto{\pgfqpoint{2.813880in}{2.285236in}}{\pgfqpoint{2.811566in}{2.290822in}}{\pgfqpoint{2.807448in}{2.294940in}}%
\pgfpathcurveto{\pgfqpoint{2.803330in}{2.299058in}}{\pgfqpoint{2.797744in}{2.301372in}}{\pgfqpoint{2.791920in}{2.301372in}}%
\pgfpathcurveto{\pgfqpoint{2.786096in}{2.301372in}}{\pgfqpoint{2.780510in}{2.299058in}}{\pgfqpoint{2.776392in}{2.294940in}}%
\pgfpathcurveto{\pgfqpoint{2.772273in}{2.290822in}}{\pgfqpoint{2.769960in}{2.285236in}}{\pgfqpoint{2.769960in}{2.279412in}}%
\pgfpathcurveto{\pgfqpoint{2.769960in}{2.273588in}}{\pgfqpoint{2.772273in}{2.268002in}}{\pgfqpoint{2.776392in}{2.263884in}}%
\pgfpathcurveto{\pgfqpoint{2.780510in}{2.259766in}}{\pgfqpoint{2.786096in}{2.257452in}}{\pgfqpoint{2.791920in}{2.257452in}}%
\pgfpathlineto{\pgfqpoint{2.791920in}{2.257452in}}%
\pgfpathclose%
\pgfusepath{stroke,fill}%
\end{pgfscope}%
\begin{pgfscope}%
\pgfpathrectangle{\pgfqpoint{0.997489in}{0.528000in}}{\pgfqpoint{4.565023in}{3.696000in}}%
\pgfusepath{clip}%
\pgfsetbuttcap%
\pgfsetroundjoin%
\definecolor{currentfill}{rgb}{0.200000,0.200000,0.800000}%
\pgfsetfillcolor{currentfill}%
\pgfsetlinewidth{1.003750pt}%
\definecolor{currentstroke}{rgb}{0.200000,0.200000,0.800000}%
\pgfsetstrokecolor{currentstroke}%
\pgfsetdash{}{0pt}%
\pgfpathmoveto{\pgfqpoint{2.835137in}{2.249911in}}%
\pgfpathcurveto{\pgfqpoint{2.840960in}{2.249911in}}{\pgfqpoint{2.846547in}{2.252225in}}{\pgfqpoint{2.850665in}{2.256343in}}%
\pgfpathcurveto{\pgfqpoint{2.854783in}{2.260461in}}{\pgfqpoint{2.857097in}{2.266047in}}{\pgfqpoint{2.857097in}{2.271871in}}%
\pgfpathcurveto{\pgfqpoint{2.857097in}{2.277695in}}{\pgfqpoint{2.854783in}{2.283281in}}{\pgfqpoint{2.850665in}{2.287399in}}%
\pgfpathcurveto{\pgfqpoint{2.846547in}{2.291517in}}{\pgfqpoint{2.840960in}{2.293831in}}{\pgfqpoint{2.835137in}{2.293831in}}%
\pgfpathcurveto{\pgfqpoint{2.829313in}{2.293831in}}{\pgfqpoint{2.823726in}{2.291517in}}{\pgfqpoint{2.819608in}{2.287399in}}%
\pgfpathcurveto{\pgfqpoint{2.815490in}{2.283281in}}{\pgfqpoint{2.813176in}{2.277695in}}{\pgfqpoint{2.813176in}{2.271871in}}%
\pgfpathcurveto{\pgfqpoint{2.813176in}{2.266047in}}{\pgfqpoint{2.815490in}{2.260461in}}{\pgfqpoint{2.819608in}{2.256343in}}%
\pgfpathcurveto{\pgfqpoint{2.823726in}{2.252225in}}{\pgfqpoint{2.829313in}{2.249911in}}{\pgfqpoint{2.835137in}{2.249911in}}%
\pgfpathlineto{\pgfqpoint{2.835137in}{2.249911in}}%
\pgfpathclose%
\pgfusepath{stroke,fill}%
\end{pgfscope}%
\begin{pgfscope}%
\pgfpathrectangle{\pgfqpoint{0.997489in}{0.528000in}}{\pgfqpoint{4.565023in}{3.696000in}}%
\pgfusepath{clip}%
\pgfsetbuttcap%
\pgfsetroundjoin%
\definecolor{currentfill}{rgb}{0.200000,0.200000,0.800000}%
\pgfsetfillcolor{currentfill}%
\pgfsetlinewidth{1.003750pt}%
\definecolor{currentstroke}{rgb}{0.200000,0.200000,0.800000}%
\pgfsetstrokecolor{currentstroke}%
\pgfsetdash{}{0pt}%
\pgfpathmoveto{\pgfqpoint{2.880324in}{2.249121in}}%
\pgfpathcurveto{\pgfqpoint{2.886148in}{2.249121in}}{\pgfqpoint{2.891734in}{2.251435in}}{\pgfqpoint{2.895852in}{2.255553in}}%
\pgfpathcurveto{\pgfqpoint{2.899970in}{2.259671in}}{\pgfqpoint{2.902284in}{2.265257in}}{\pgfqpoint{2.902284in}{2.271081in}}%
\pgfpathcurveto{\pgfqpoint{2.902284in}{2.276905in}}{\pgfqpoint{2.899970in}{2.282491in}}{\pgfqpoint{2.895852in}{2.286609in}}%
\pgfpathcurveto{\pgfqpoint{2.891734in}{2.290728in}}{\pgfqpoint{2.886148in}{2.293041in}}{\pgfqpoint{2.880324in}{2.293041in}}%
\pgfpathcurveto{\pgfqpoint{2.874500in}{2.293041in}}{\pgfqpoint{2.868913in}{2.290728in}}{\pgfqpoint{2.864795in}{2.286609in}}%
\pgfpathcurveto{\pgfqpoint{2.860677in}{2.282491in}}{\pgfqpoint{2.858363in}{2.276905in}}{\pgfqpoint{2.858363in}{2.271081in}}%
\pgfpathcurveto{\pgfqpoint{2.858363in}{2.265257in}}{\pgfqpoint{2.860677in}{2.259671in}}{\pgfqpoint{2.864795in}{2.255553in}}%
\pgfpathcurveto{\pgfqpoint{2.868913in}{2.251435in}}{\pgfqpoint{2.874500in}{2.249121in}}{\pgfqpoint{2.880324in}{2.249121in}}%
\pgfpathlineto{\pgfqpoint{2.880324in}{2.249121in}}%
\pgfpathclose%
\pgfusepath{stroke,fill}%
\end{pgfscope}%
\begin{pgfscope}%
\pgfpathrectangle{\pgfqpoint{0.997489in}{0.528000in}}{\pgfqpoint{4.565023in}{3.696000in}}%
\pgfusepath{clip}%
\pgfsetbuttcap%
\pgfsetroundjoin%
\definecolor{currentfill}{rgb}{0.200000,0.200000,0.800000}%
\pgfsetfillcolor{currentfill}%
\pgfsetlinewidth{1.003750pt}%
\definecolor{currentstroke}{rgb}{0.200000,0.200000,0.800000}%
\pgfsetstrokecolor{currentstroke}%
\pgfsetdash{}{0pt}%
\pgfpathmoveto{\pgfqpoint{2.824342in}{2.151539in}}%
\pgfpathcurveto{\pgfqpoint{2.830166in}{2.151539in}}{\pgfqpoint{2.835752in}{2.153853in}}{\pgfqpoint{2.839870in}{2.157971in}}%
\pgfpathcurveto{\pgfqpoint{2.843988in}{2.162089in}}{\pgfqpoint{2.846302in}{2.167675in}}{\pgfqpoint{2.846302in}{2.173499in}}%
\pgfpathcurveto{\pgfqpoint{2.846302in}{2.179323in}}{\pgfqpoint{2.843988in}{2.184909in}}{\pgfqpoint{2.839870in}{2.189027in}}%
\pgfpathcurveto{\pgfqpoint{2.835752in}{2.193146in}}{\pgfqpoint{2.830166in}{2.195459in}}{\pgfqpoint{2.824342in}{2.195459in}}%
\pgfpathcurveto{\pgfqpoint{2.818518in}{2.195459in}}{\pgfqpoint{2.812932in}{2.193146in}}{\pgfqpoint{2.808814in}{2.189027in}}%
\pgfpathcurveto{\pgfqpoint{2.804696in}{2.184909in}}{\pgfqpoint{2.802382in}{2.179323in}}{\pgfqpoint{2.802382in}{2.173499in}}%
\pgfpathcurveto{\pgfqpoint{2.802382in}{2.167675in}}{\pgfqpoint{2.804696in}{2.162089in}}{\pgfqpoint{2.808814in}{2.157971in}}%
\pgfpathcurveto{\pgfqpoint{2.812932in}{2.153853in}}{\pgfqpoint{2.818518in}{2.151539in}}{\pgfqpoint{2.824342in}{2.151539in}}%
\pgfpathlineto{\pgfqpoint{2.824342in}{2.151539in}}%
\pgfpathclose%
\pgfusepath{stroke,fill}%
\end{pgfscope}%
\begin{pgfscope}%
\pgfpathrectangle{\pgfqpoint{0.997489in}{0.528000in}}{\pgfqpoint{4.565023in}{3.696000in}}%
\pgfusepath{clip}%
\pgfsetbuttcap%
\pgfsetroundjoin%
\definecolor{currentfill}{rgb}{0.200000,0.200000,0.800000}%
\pgfsetfillcolor{currentfill}%
\pgfsetlinewidth{1.003750pt}%
\definecolor{currentstroke}{rgb}{0.200000,0.200000,0.800000}%
\pgfsetstrokecolor{currentstroke}%
\pgfsetdash{}{0pt}%
\pgfpathmoveto{\pgfqpoint{2.850524in}{2.127701in}}%
\pgfpathcurveto{\pgfqpoint{2.856348in}{2.127701in}}{\pgfqpoint{2.861934in}{2.130015in}}{\pgfqpoint{2.866052in}{2.134133in}}%
\pgfpathcurveto{\pgfqpoint{2.870171in}{2.138251in}}{\pgfqpoint{2.872484in}{2.143837in}}{\pgfqpoint{2.872484in}{2.149661in}}%
\pgfpathcurveto{\pgfqpoint{2.872484in}{2.155485in}}{\pgfqpoint{2.870171in}{2.161071in}}{\pgfqpoint{2.866052in}{2.165190in}}%
\pgfpathcurveto{\pgfqpoint{2.861934in}{2.169308in}}{\pgfqpoint{2.856348in}{2.171622in}}{\pgfqpoint{2.850524in}{2.171622in}}%
\pgfpathcurveto{\pgfqpoint{2.844700in}{2.171622in}}{\pgfqpoint{2.839114in}{2.169308in}}{\pgfqpoint{2.834996in}{2.165190in}}%
\pgfpathcurveto{\pgfqpoint{2.830878in}{2.161071in}}{\pgfqpoint{2.828564in}{2.155485in}}{\pgfqpoint{2.828564in}{2.149661in}}%
\pgfpathcurveto{\pgfqpoint{2.828564in}{2.143837in}}{\pgfqpoint{2.830878in}{2.138251in}}{\pgfqpoint{2.834996in}{2.134133in}}%
\pgfpathcurveto{\pgfqpoint{2.839114in}{2.130015in}}{\pgfqpoint{2.844700in}{2.127701in}}{\pgfqpoint{2.850524in}{2.127701in}}%
\pgfpathlineto{\pgfqpoint{2.850524in}{2.127701in}}%
\pgfpathclose%
\pgfusepath{stroke,fill}%
\end{pgfscope}%
\begin{pgfscope}%
\pgfpathrectangle{\pgfqpoint{0.997489in}{0.528000in}}{\pgfqpoint{4.565023in}{3.696000in}}%
\pgfusepath{clip}%
\pgfsetbuttcap%
\pgfsetroundjoin%
\definecolor{currentfill}{rgb}{0.200000,0.200000,0.800000}%
\pgfsetfillcolor{currentfill}%
\pgfsetlinewidth{1.003750pt}%
\definecolor{currentstroke}{rgb}{0.200000,0.200000,0.800000}%
\pgfsetstrokecolor{currentstroke}%
\pgfsetdash{}{0pt}%
\pgfpathmoveto{\pgfqpoint{2.943895in}{2.191967in}}%
\pgfpathcurveto{\pgfqpoint{2.949719in}{2.191967in}}{\pgfqpoint{2.955306in}{2.194281in}}{\pgfqpoint{2.959424in}{2.198399in}}%
\pgfpathcurveto{\pgfqpoint{2.963542in}{2.202517in}}{\pgfqpoint{2.965856in}{2.208103in}}{\pgfqpoint{2.965856in}{2.213927in}}%
\pgfpathcurveto{\pgfqpoint{2.965856in}{2.219751in}}{\pgfqpoint{2.963542in}{2.225337in}}{\pgfqpoint{2.959424in}{2.229455in}}%
\pgfpathcurveto{\pgfqpoint{2.955306in}{2.233573in}}{\pgfqpoint{2.949719in}{2.235887in}}{\pgfqpoint{2.943895in}{2.235887in}}%
\pgfpathcurveto{\pgfqpoint{2.938072in}{2.235887in}}{\pgfqpoint{2.932485in}{2.233573in}}{\pgfqpoint{2.928367in}{2.229455in}}%
\pgfpathcurveto{\pgfqpoint{2.924249in}{2.225337in}}{\pgfqpoint{2.921935in}{2.219751in}}{\pgfqpoint{2.921935in}{2.213927in}}%
\pgfpathcurveto{\pgfqpoint{2.921935in}{2.208103in}}{\pgfqpoint{2.924249in}{2.202517in}}{\pgfqpoint{2.928367in}{2.198399in}}%
\pgfpathcurveto{\pgfqpoint{2.932485in}{2.194281in}}{\pgfqpoint{2.938072in}{2.191967in}}{\pgfqpoint{2.943895in}{2.191967in}}%
\pgfpathlineto{\pgfqpoint{2.943895in}{2.191967in}}%
\pgfpathclose%
\pgfusepath{stroke,fill}%
\end{pgfscope}%
\begin{pgfscope}%
\pgfpathrectangle{\pgfqpoint{0.997489in}{0.528000in}}{\pgfqpoint{4.565023in}{3.696000in}}%
\pgfusepath{clip}%
\pgfsetbuttcap%
\pgfsetroundjoin%
\definecolor{currentfill}{rgb}{0.200000,0.200000,0.800000}%
\pgfsetfillcolor{currentfill}%
\pgfsetlinewidth{1.003750pt}%
\definecolor{currentstroke}{rgb}{0.200000,0.200000,0.800000}%
\pgfsetstrokecolor{currentstroke}%
\pgfsetdash{}{0pt}%
\pgfpathmoveto{\pgfqpoint{2.966486in}{2.174505in}}%
\pgfpathcurveto{\pgfqpoint{2.972310in}{2.174505in}}{\pgfqpoint{2.977896in}{2.176819in}}{\pgfqpoint{2.982014in}{2.180937in}}%
\pgfpathcurveto{\pgfqpoint{2.986132in}{2.185055in}}{\pgfqpoint{2.988446in}{2.190641in}}{\pgfqpoint{2.988446in}{2.196465in}}%
\pgfpathcurveto{\pgfqpoint{2.988446in}{2.202289in}}{\pgfqpoint{2.986132in}{2.207875in}}{\pgfqpoint{2.982014in}{2.211993in}}%
\pgfpathcurveto{\pgfqpoint{2.977896in}{2.216111in}}{\pgfqpoint{2.972310in}{2.218425in}}{\pgfqpoint{2.966486in}{2.218425in}}%
\pgfpathcurveto{\pgfqpoint{2.960662in}{2.218425in}}{\pgfqpoint{2.955075in}{2.216111in}}{\pgfqpoint{2.950957in}{2.211993in}}%
\pgfpathcurveto{\pgfqpoint{2.946839in}{2.207875in}}{\pgfqpoint{2.944525in}{2.202289in}}{\pgfqpoint{2.944525in}{2.196465in}}%
\pgfpathcurveto{\pgfqpoint{2.944525in}{2.190641in}}{\pgfqpoint{2.946839in}{2.185055in}}{\pgfqpoint{2.950957in}{2.180937in}}%
\pgfpathcurveto{\pgfqpoint{2.955075in}{2.176819in}}{\pgfqpoint{2.960662in}{2.174505in}}{\pgfqpoint{2.966486in}{2.174505in}}%
\pgfpathlineto{\pgfqpoint{2.966486in}{2.174505in}}%
\pgfpathclose%
\pgfusepath{stroke,fill}%
\end{pgfscope}%
\begin{pgfscope}%
\pgfpathrectangle{\pgfqpoint{0.997489in}{0.528000in}}{\pgfqpoint{4.565023in}{3.696000in}}%
\pgfusepath{clip}%
\pgfsetbuttcap%
\pgfsetroundjoin%
\definecolor{currentfill}{rgb}{0.200000,0.200000,0.800000}%
\pgfsetfillcolor{currentfill}%
\pgfsetlinewidth{1.003750pt}%
\definecolor{currentstroke}{rgb}{0.200000,0.200000,0.800000}%
\pgfsetstrokecolor{currentstroke}%
\pgfsetdash{}{0pt}%
\pgfpathmoveto{\pgfqpoint{2.944571in}{2.079444in}}%
\pgfpathcurveto{\pgfqpoint{2.950395in}{2.079444in}}{\pgfqpoint{2.955981in}{2.081758in}}{\pgfqpoint{2.960099in}{2.085876in}}%
\pgfpathcurveto{\pgfqpoint{2.964217in}{2.089994in}}{\pgfqpoint{2.966531in}{2.095580in}}{\pgfqpoint{2.966531in}{2.101404in}}%
\pgfpathcurveto{\pgfqpoint{2.966531in}{2.107228in}}{\pgfqpoint{2.964217in}{2.112815in}}{\pgfqpoint{2.960099in}{2.116933in}}%
\pgfpathcurveto{\pgfqpoint{2.955981in}{2.121051in}}{\pgfqpoint{2.950395in}{2.123365in}}{\pgfqpoint{2.944571in}{2.123365in}}%
\pgfpathcurveto{\pgfqpoint{2.938747in}{2.123365in}}{\pgfqpoint{2.933161in}{2.121051in}}{\pgfqpoint{2.929042in}{2.116933in}}%
\pgfpathcurveto{\pgfqpoint{2.924924in}{2.112815in}}{\pgfqpoint{2.922610in}{2.107228in}}{\pgfqpoint{2.922610in}{2.101404in}}%
\pgfpathcurveto{\pgfqpoint{2.922610in}{2.095580in}}{\pgfqpoint{2.924924in}{2.089994in}}{\pgfqpoint{2.929042in}{2.085876in}}%
\pgfpathcurveto{\pgfqpoint{2.933161in}{2.081758in}}{\pgfqpoint{2.938747in}{2.079444in}}{\pgfqpoint{2.944571in}{2.079444in}}%
\pgfpathlineto{\pgfqpoint{2.944571in}{2.079444in}}%
\pgfpathclose%
\pgfusepath{stroke,fill}%
\end{pgfscope}%
\begin{pgfscope}%
\pgfpathrectangle{\pgfqpoint{0.997489in}{0.528000in}}{\pgfqpoint{4.565023in}{3.696000in}}%
\pgfusepath{clip}%
\pgfsetbuttcap%
\pgfsetroundjoin%
\definecolor{currentfill}{rgb}{0.200000,0.200000,0.800000}%
\pgfsetfillcolor{currentfill}%
\pgfsetlinewidth{1.003750pt}%
\definecolor{currentstroke}{rgb}{0.200000,0.200000,0.800000}%
\pgfsetstrokecolor{currentstroke}%
\pgfsetdash{}{0pt}%
\pgfpathmoveto{\pgfqpoint{2.998358in}{2.110483in}}%
\pgfpathcurveto{\pgfqpoint{3.004182in}{2.110483in}}{\pgfqpoint{3.009769in}{2.112797in}}{\pgfqpoint{3.013887in}{2.116915in}}%
\pgfpathcurveto{\pgfqpoint{3.018005in}{2.121033in}}{\pgfqpoint{3.020319in}{2.126619in}}{\pgfqpoint{3.020319in}{2.132443in}}%
\pgfpathcurveto{\pgfqpoint{3.020319in}{2.138267in}}{\pgfqpoint{3.018005in}{2.143853in}}{\pgfqpoint{3.013887in}{2.147971in}}%
\pgfpathcurveto{\pgfqpoint{3.009769in}{2.152089in}}{\pgfqpoint{3.004182in}{2.154403in}}{\pgfqpoint{2.998358in}{2.154403in}}%
\pgfpathcurveto{\pgfqpoint{2.992534in}{2.154403in}}{\pgfqpoint{2.986948in}{2.152089in}}{\pgfqpoint{2.982830in}{2.147971in}}%
\pgfpathcurveto{\pgfqpoint{2.978712in}{2.143853in}}{\pgfqpoint{2.976398in}{2.138267in}}{\pgfqpoint{2.976398in}{2.132443in}}%
\pgfpathcurveto{\pgfqpoint{2.976398in}{2.126619in}}{\pgfqpoint{2.978712in}{2.121033in}}{\pgfqpoint{2.982830in}{2.116915in}}%
\pgfpathcurveto{\pgfqpoint{2.986948in}{2.112797in}}{\pgfqpoint{2.992534in}{2.110483in}}{\pgfqpoint{2.998358in}{2.110483in}}%
\pgfpathlineto{\pgfqpoint{2.998358in}{2.110483in}}%
\pgfpathclose%
\pgfusepath{stroke,fill}%
\end{pgfscope}%
\begin{pgfscope}%
\pgfpathrectangle{\pgfqpoint{0.997489in}{0.528000in}}{\pgfqpoint{4.565023in}{3.696000in}}%
\pgfusepath{clip}%
\pgfsetbuttcap%
\pgfsetroundjoin%
\definecolor{currentfill}{rgb}{0.200000,0.200000,0.800000}%
\pgfsetfillcolor{currentfill}%
\pgfsetlinewidth{1.003750pt}%
\definecolor{currentstroke}{rgb}{0.200000,0.200000,0.800000}%
\pgfsetstrokecolor{currentstroke}%
\pgfsetdash{}{0pt}%
\pgfpathmoveto{\pgfqpoint{3.040016in}{2.129536in}}%
\pgfpathcurveto{\pgfqpoint{3.045840in}{2.129536in}}{\pgfqpoint{3.051427in}{2.131850in}}{\pgfqpoint{3.055545in}{2.135968in}}%
\pgfpathcurveto{\pgfqpoint{3.059663in}{2.140086in}}{\pgfqpoint{3.061977in}{2.145672in}}{\pgfqpoint{3.061977in}{2.151496in}}%
\pgfpathcurveto{\pgfqpoint{3.061977in}{2.157320in}}{\pgfqpoint{3.059663in}{2.162907in}}{\pgfqpoint{3.055545in}{2.167025in}}%
\pgfpathcurveto{\pgfqpoint{3.051427in}{2.171143in}}{\pgfqpoint{3.045840in}{2.173457in}}{\pgfqpoint{3.040016in}{2.173457in}}%
\pgfpathcurveto{\pgfqpoint{3.034193in}{2.173457in}}{\pgfqpoint{3.028606in}{2.171143in}}{\pgfqpoint{3.024488in}{2.167025in}}%
\pgfpathcurveto{\pgfqpoint{3.020370in}{2.162907in}}{\pgfqpoint{3.018056in}{2.157320in}}{\pgfqpoint{3.018056in}{2.151496in}}%
\pgfpathcurveto{\pgfqpoint{3.018056in}{2.145672in}}{\pgfqpoint{3.020370in}{2.140086in}}{\pgfqpoint{3.024488in}{2.135968in}}%
\pgfpathcurveto{\pgfqpoint{3.028606in}{2.131850in}}{\pgfqpoint{3.034193in}{2.129536in}}{\pgfqpoint{3.040016in}{2.129536in}}%
\pgfpathlineto{\pgfqpoint{3.040016in}{2.129536in}}%
\pgfpathclose%
\pgfusepath{stroke,fill}%
\end{pgfscope}%
\begin{pgfscope}%
\pgfpathrectangle{\pgfqpoint{0.997489in}{0.528000in}}{\pgfqpoint{4.565023in}{3.696000in}}%
\pgfusepath{clip}%
\pgfsetbuttcap%
\pgfsetroundjoin%
\definecolor{currentfill}{rgb}{0.200000,0.200000,0.800000}%
\pgfsetfillcolor{currentfill}%
\pgfsetlinewidth{1.003750pt}%
\definecolor{currentstroke}{rgb}{0.200000,0.200000,0.800000}%
\pgfsetstrokecolor{currentstroke}%
\pgfsetdash{}{0pt}%
\pgfpathmoveto{\pgfqpoint{3.077393in}{2.149278in}}%
\pgfpathcurveto{\pgfqpoint{3.083217in}{2.149278in}}{\pgfqpoint{3.088803in}{2.151592in}}{\pgfqpoint{3.092922in}{2.155710in}}%
\pgfpathcurveto{\pgfqpoint{3.097040in}{2.159828in}}{\pgfqpoint{3.099354in}{2.165414in}}{\pgfqpoint{3.099354in}{2.171238in}}%
\pgfpathcurveto{\pgfqpoint{3.099354in}{2.177062in}}{\pgfqpoint{3.097040in}{2.182648in}}{\pgfqpoint{3.092922in}{2.186766in}}%
\pgfpathcurveto{\pgfqpoint{3.088803in}{2.190884in}}{\pgfqpoint{3.083217in}{2.193198in}}{\pgfqpoint{3.077393in}{2.193198in}}%
\pgfpathcurveto{\pgfqpoint{3.071569in}{2.193198in}}{\pgfqpoint{3.065983in}{2.190884in}}{\pgfqpoint{3.061865in}{2.186766in}}%
\pgfpathcurveto{\pgfqpoint{3.057747in}{2.182648in}}{\pgfqpoint{3.055433in}{2.177062in}}{\pgfqpoint{3.055433in}{2.171238in}}%
\pgfpathcurveto{\pgfqpoint{3.055433in}{2.165414in}}{\pgfqpoint{3.057747in}{2.159828in}}{\pgfqpoint{3.061865in}{2.155710in}}%
\pgfpathcurveto{\pgfqpoint{3.065983in}{2.151592in}}{\pgfqpoint{3.071569in}{2.149278in}}{\pgfqpoint{3.077393in}{2.149278in}}%
\pgfpathlineto{\pgfqpoint{3.077393in}{2.149278in}}%
\pgfpathclose%
\pgfusepath{stroke,fill}%
\end{pgfscope}%
\begin{pgfscope}%
\pgfpathrectangle{\pgfqpoint{0.997489in}{0.528000in}}{\pgfqpoint{4.565023in}{3.696000in}}%
\pgfusepath{clip}%
\pgfsetbuttcap%
\pgfsetroundjoin%
\definecolor{currentfill}{rgb}{0.200000,0.200000,0.800000}%
\pgfsetfillcolor{currentfill}%
\pgfsetlinewidth{1.003750pt}%
\definecolor{currentstroke}{rgb}{0.200000,0.200000,0.800000}%
\pgfsetstrokecolor{currentstroke}%
\pgfsetdash{}{0pt}%
\pgfpathmoveto{\pgfqpoint{3.087294in}{2.084337in}}%
\pgfpathcurveto{\pgfqpoint{3.093118in}{2.084337in}}{\pgfqpoint{3.098704in}{2.086651in}}{\pgfqpoint{3.102822in}{2.090769in}}%
\pgfpathcurveto{\pgfqpoint{3.106940in}{2.094887in}}{\pgfqpoint{3.109254in}{2.100473in}}{\pgfqpoint{3.109254in}{2.106297in}}%
\pgfpathcurveto{\pgfqpoint{3.109254in}{2.112121in}}{\pgfqpoint{3.106940in}{2.117707in}}{\pgfqpoint{3.102822in}{2.121825in}}%
\pgfpathcurveto{\pgfqpoint{3.098704in}{2.125943in}}{\pgfqpoint{3.093118in}{2.128257in}}{\pgfqpoint{3.087294in}{2.128257in}}%
\pgfpathcurveto{\pgfqpoint{3.081470in}{2.128257in}}{\pgfqpoint{3.075884in}{2.125943in}}{\pgfqpoint{3.071766in}{2.121825in}}%
\pgfpathcurveto{\pgfqpoint{3.067648in}{2.117707in}}{\pgfqpoint{3.065334in}{2.112121in}}{\pgfqpoint{3.065334in}{2.106297in}}%
\pgfpathcurveto{\pgfqpoint{3.065334in}{2.100473in}}{\pgfqpoint{3.067648in}{2.094887in}}{\pgfqpoint{3.071766in}{2.090769in}}%
\pgfpathcurveto{\pgfqpoint{3.075884in}{2.086651in}}{\pgfqpoint{3.081470in}{2.084337in}}{\pgfqpoint{3.087294in}{2.084337in}}%
\pgfpathlineto{\pgfqpoint{3.087294in}{2.084337in}}%
\pgfpathclose%
\pgfusepath{stroke,fill}%
\end{pgfscope}%
\begin{pgfscope}%
\pgfpathrectangle{\pgfqpoint{0.997489in}{0.528000in}}{\pgfqpoint{4.565023in}{3.696000in}}%
\pgfusepath{clip}%
\pgfsetbuttcap%
\pgfsetroundjoin%
\definecolor{currentfill}{rgb}{0.200000,0.200000,0.800000}%
\pgfsetfillcolor{currentfill}%
\pgfsetlinewidth{1.003750pt}%
\definecolor{currentstroke}{rgb}{0.200000,0.200000,0.800000}%
\pgfsetstrokecolor{currentstroke}%
\pgfsetdash{}{0pt}%
\pgfpathmoveto{\pgfqpoint{3.110760in}{2.047095in}}%
\pgfpathcurveto{\pgfqpoint{3.116584in}{2.047095in}}{\pgfqpoint{3.122170in}{2.049409in}}{\pgfqpoint{3.126288in}{2.053527in}}%
\pgfpathcurveto{\pgfqpoint{3.130407in}{2.057645in}}{\pgfqpoint{3.132720in}{2.063231in}}{\pgfqpoint{3.132720in}{2.069055in}}%
\pgfpathcurveto{\pgfqpoint{3.132720in}{2.074879in}}{\pgfqpoint{3.130407in}{2.080465in}}{\pgfqpoint{3.126288in}{2.084583in}}%
\pgfpathcurveto{\pgfqpoint{3.122170in}{2.088702in}}{\pgfqpoint{3.116584in}{2.091015in}}{\pgfqpoint{3.110760in}{2.091015in}}%
\pgfpathcurveto{\pgfqpoint{3.104936in}{2.091015in}}{\pgfqpoint{3.099350in}{2.088702in}}{\pgfqpoint{3.095232in}{2.084583in}}%
\pgfpathcurveto{\pgfqpoint{3.091114in}{2.080465in}}{\pgfqpoint{3.088800in}{2.074879in}}{\pgfqpoint{3.088800in}{2.069055in}}%
\pgfpathcurveto{\pgfqpoint{3.088800in}{2.063231in}}{\pgfqpoint{3.091114in}{2.057645in}}{\pgfqpoint{3.095232in}{2.053527in}}%
\pgfpathcurveto{\pgfqpoint{3.099350in}{2.049409in}}{\pgfqpoint{3.104936in}{2.047095in}}{\pgfqpoint{3.110760in}{2.047095in}}%
\pgfpathlineto{\pgfqpoint{3.110760in}{2.047095in}}%
\pgfpathclose%
\pgfusepath{stroke,fill}%
\end{pgfscope}%
\begin{pgfscope}%
\pgfpathrectangle{\pgfqpoint{0.997489in}{0.528000in}}{\pgfqpoint{4.565023in}{3.696000in}}%
\pgfusepath{clip}%
\pgfsetbuttcap%
\pgfsetroundjoin%
\definecolor{currentfill}{rgb}{0.200000,0.200000,0.800000}%
\pgfsetfillcolor{currentfill}%
\pgfsetlinewidth{1.003750pt}%
\definecolor{currentstroke}{rgb}{0.200000,0.200000,0.800000}%
\pgfsetstrokecolor{currentstroke}%
\pgfsetdash{}{0pt}%
\pgfpathmoveto{\pgfqpoint{3.145866in}{2.062604in}}%
\pgfpathcurveto{\pgfqpoint{3.151690in}{2.062604in}}{\pgfqpoint{3.157277in}{2.064918in}}{\pgfqpoint{3.161395in}{2.069036in}}%
\pgfpathcurveto{\pgfqpoint{3.165513in}{2.073154in}}{\pgfqpoint{3.167827in}{2.078740in}}{\pgfqpoint{3.167827in}{2.084564in}}%
\pgfpathcurveto{\pgfqpoint{3.167827in}{2.090388in}}{\pgfqpoint{3.165513in}{2.095974in}}{\pgfqpoint{3.161395in}{2.100092in}}%
\pgfpathcurveto{\pgfqpoint{3.157277in}{2.104211in}}{\pgfqpoint{3.151690in}{2.106524in}}{\pgfqpoint{3.145866in}{2.106524in}}%
\pgfpathcurveto{\pgfqpoint{3.140042in}{2.106524in}}{\pgfqpoint{3.134456in}{2.104211in}}{\pgfqpoint{3.130338in}{2.100092in}}%
\pgfpathcurveto{\pgfqpoint{3.126220in}{2.095974in}}{\pgfqpoint{3.123906in}{2.090388in}}{\pgfqpoint{3.123906in}{2.084564in}}%
\pgfpathcurveto{\pgfqpoint{3.123906in}{2.078740in}}{\pgfqpoint{3.126220in}{2.073154in}}{\pgfqpoint{3.130338in}{2.069036in}}%
\pgfpathcurveto{\pgfqpoint{3.134456in}{2.064918in}}{\pgfqpoint{3.140042in}{2.062604in}}{\pgfqpoint{3.145866in}{2.062604in}}%
\pgfpathlineto{\pgfqpoint{3.145866in}{2.062604in}}%
\pgfpathclose%
\pgfusepath{stroke,fill}%
\end{pgfscope}%
\begin{pgfscope}%
\pgfpathrectangle{\pgfqpoint{0.997489in}{0.528000in}}{\pgfqpoint{4.565023in}{3.696000in}}%
\pgfusepath{clip}%
\pgfsetbuttcap%
\pgfsetroundjoin%
\definecolor{currentfill}{rgb}{0.200000,0.200000,0.800000}%
\pgfsetfillcolor{currentfill}%
\pgfsetlinewidth{1.003750pt}%
\definecolor{currentstroke}{rgb}{0.200000,0.200000,0.800000}%
\pgfsetstrokecolor{currentstroke}%
\pgfsetdash{}{0pt}%
\pgfpathmoveto{\pgfqpoint{3.181155in}{2.112981in}}%
\pgfpathcurveto{\pgfqpoint{3.186979in}{2.112981in}}{\pgfqpoint{3.192566in}{2.115294in}}{\pgfqpoint{3.196684in}{2.119413in}}%
\pgfpathcurveto{\pgfqpoint{3.200802in}{2.123531in}}{\pgfqpoint{3.203116in}{2.129117in}}{\pgfqpoint{3.203116in}{2.134941in}}%
\pgfpathcurveto{\pgfqpoint{3.203116in}{2.140765in}}{\pgfqpoint{3.200802in}{2.146351in}}{\pgfqpoint{3.196684in}{2.150469in}}%
\pgfpathcurveto{\pgfqpoint{3.192566in}{2.154587in}}{\pgfqpoint{3.186979in}{2.156901in}}{\pgfqpoint{3.181155in}{2.156901in}}%
\pgfpathcurveto{\pgfqpoint{3.175332in}{2.156901in}}{\pgfqpoint{3.169745in}{2.154587in}}{\pgfqpoint{3.165627in}{2.150469in}}%
\pgfpathcurveto{\pgfqpoint{3.161509in}{2.146351in}}{\pgfqpoint{3.159195in}{2.140765in}}{\pgfqpoint{3.159195in}{2.134941in}}%
\pgfpathcurveto{\pgfqpoint{3.159195in}{2.129117in}}{\pgfqpoint{3.161509in}{2.123531in}}{\pgfqpoint{3.165627in}{2.119413in}}%
\pgfpathcurveto{\pgfqpoint{3.169745in}{2.115294in}}{\pgfqpoint{3.175332in}{2.112981in}}{\pgfqpoint{3.181155in}{2.112981in}}%
\pgfpathlineto{\pgfqpoint{3.181155in}{2.112981in}}%
\pgfpathclose%
\pgfusepath{stroke,fill}%
\end{pgfscope}%
\begin{pgfscope}%
\pgfpathrectangle{\pgfqpoint{0.997489in}{0.528000in}}{\pgfqpoint{4.565023in}{3.696000in}}%
\pgfusepath{clip}%
\pgfsetbuttcap%
\pgfsetroundjoin%
\definecolor{currentfill}{rgb}{0.200000,0.200000,0.800000}%
\pgfsetfillcolor{currentfill}%
\pgfsetlinewidth{1.003750pt}%
\definecolor{currentstroke}{rgb}{0.200000,0.200000,0.800000}%
\pgfsetstrokecolor{currentstroke}%
\pgfsetdash{}{0pt}%
\pgfpathmoveto{\pgfqpoint{3.208293in}{2.073478in}}%
\pgfpathcurveto{\pgfqpoint{3.214117in}{2.073478in}}{\pgfqpoint{3.219703in}{2.075791in}}{\pgfqpoint{3.223821in}{2.079910in}}%
\pgfpathcurveto{\pgfqpoint{3.227939in}{2.084028in}}{\pgfqpoint{3.230253in}{2.089614in}}{\pgfqpoint{3.230253in}{2.095438in}}%
\pgfpathcurveto{\pgfqpoint{3.230253in}{2.101262in}}{\pgfqpoint{3.227939in}{2.106848in}}{\pgfqpoint{3.223821in}{2.110966in}}%
\pgfpathcurveto{\pgfqpoint{3.219703in}{2.115084in}}{\pgfqpoint{3.214117in}{2.117398in}}{\pgfqpoint{3.208293in}{2.117398in}}%
\pgfpathcurveto{\pgfqpoint{3.202469in}{2.117398in}}{\pgfqpoint{3.196882in}{2.115084in}}{\pgfqpoint{3.192764in}{2.110966in}}%
\pgfpathcurveto{\pgfqpoint{3.188646in}{2.106848in}}{\pgfqpoint{3.186332in}{2.101262in}}{\pgfqpoint{3.186332in}{2.095438in}}%
\pgfpathcurveto{\pgfqpoint{3.186332in}{2.089614in}}{\pgfqpoint{3.188646in}{2.084028in}}{\pgfqpoint{3.192764in}{2.079910in}}%
\pgfpathcurveto{\pgfqpoint{3.196882in}{2.075791in}}{\pgfqpoint{3.202469in}{2.073478in}}{\pgfqpoint{3.208293in}{2.073478in}}%
\pgfpathlineto{\pgfqpoint{3.208293in}{2.073478in}}%
\pgfpathclose%
\pgfusepath{stroke,fill}%
\end{pgfscope}%
\begin{pgfscope}%
\pgfpathrectangle{\pgfqpoint{0.997489in}{0.528000in}}{\pgfqpoint{4.565023in}{3.696000in}}%
\pgfusepath{clip}%
\pgfsetbuttcap%
\pgfsetroundjoin%
\definecolor{currentfill}{rgb}{0.200000,0.200000,0.800000}%
\pgfsetfillcolor{currentfill}%
\pgfsetlinewidth{1.003750pt}%
\definecolor{currentstroke}{rgb}{0.200000,0.200000,0.800000}%
\pgfsetstrokecolor{currentstroke}%
\pgfsetdash{}{0pt}%
\pgfpathmoveto{\pgfqpoint{3.241301in}{2.014862in}}%
\pgfpathcurveto{\pgfqpoint{3.247125in}{2.014862in}}{\pgfqpoint{3.252712in}{2.017176in}}{\pgfqpoint{3.256830in}{2.021294in}}%
\pgfpathcurveto{\pgfqpoint{3.260948in}{2.025412in}}{\pgfqpoint{3.263262in}{2.030998in}}{\pgfqpoint{3.263262in}{2.036822in}}%
\pgfpathcurveto{\pgfqpoint{3.263262in}{2.042646in}}{\pgfqpoint{3.260948in}{2.048232in}}{\pgfqpoint{3.256830in}{2.052350in}}%
\pgfpathcurveto{\pgfqpoint{3.252712in}{2.056468in}}{\pgfqpoint{3.247125in}{2.058782in}}{\pgfqpoint{3.241301in}{2.058782in}}%
\pgfpathcurveto{\pgfqpoint{3.235478in}{2.058782in}}{\pgfqpoint{3.229891in}{2.056468in}}{\pgfqpoint{3.225773in}{2.052350in}}%
\pgfpathcurveto{\pgfqpoint{3.221655in}{2.048232in}}{\pgfqpoint{3.219341in}{2.042646in}}{\pgfqpoint{3.219341in}{2.036822in}}%
\pgfpathcurveto{\pgfqpoint{3.219341in}{2.030998in}}{\pgfqpoint{3.221655in}{2.025412in}}{\pgfqpoint{3.225773in}{2.021294in}}%
\pgfpathcurveto{\pgfqpoint{3.229891in}{2.017176in}}{\pgfqpoint{3.235478in}{2.014862in}}{\pgfqpoint{3.241301in}{2.014862in}}%
\pgfpathlineto{\pgfqpoint{3.241301in}{2.014862in}}%
\pgfpathclose%
\pgfusepath{stroke,fill}%
\end{pgfscope}%
\begin{pgfscope}%
\pgfpathrectangle{\pgfqpoint{0.997489in}{0.528000in}}{\pgfqpoint{4.565023in}{3.696000in}}%
\pgfusepath{clip}%
\pgfsetbuttcap%
\pgfsetroundjoin%
\definecolor{currentfill}{rgb}{0.200000,0.200000,0.800000}%
\pgfsetfillcolor{currentfill}%
\pgfsetlinewidth{1.003750pt}%
\definecolor{currentstroke}{rgb}{0.200000,0.200000,0.800000}%
\pgfsetstrokecolor{currentstroke}%
\pgfsetdash{}{0pt}%
\pgfpathmoveto{\pgfqpoint{3.281042in}{1.964692in}}%
\pgfpathcurveto{\pgfqpoint{3.286866in}{1.964692in}}{\pgfqpoint{3.292452in}{1.967006in}}{\pgfqpoint{3.296570in}{1.971124in}}%
\pgfpathcurveto{\pgfqpoint{3.300688in}{1.975242in}}{\pgfqpoint{3.303002in}{1.980829in}}{\pgfqpoint{3.303002in}{1.986652in}}%
\pgfpathcurveto{\pgfqpoint{3.303002in}{1.992476in}}{\pgfqpoint{3.300688in}{1.998063in}}{\pgfqpoint{3.296570in}{2.002181in}}%
\pgfpathcurveto{\pgfqpoint{3.292452in}{2.006299in}}{\pgfqpoint{3.286866in}{2.008613in}}{\pgfqpoint{3.281042in}{2.008613in}}%
\pgfpathcurveto{\pgfqpoint{3.275218in}{2.008613in}}{\pgfqpoint{3.269632in}{2.006299in}}{\pgfqpoint{3.265513in}{2.002181in}}%
\pgfpathcurveto{\pgfqpoint{3.261395in}{1.998063in}}{\pgfqpoint{3.259081in}{1.992476in}}{\pgfqpoint{3.259081in}{1.986652in}}%
\pgfpathcurveto{\pgfqpoint{3.259081in}{1.980829in}}{\pgfqpoint{3.261395in}{1.975242in}}{\pgfqpoint{3.265513in}{1.971124in}}%
\pgfpathcurveto{\pgfqpoint{3.269632in}{1.967006in}}{\pgfqpoint{3.275218in}{1.964692in}}{\pgfqpoint{3.281042in}{1.964692in}}%
\pgfpathlineto{\pgfqpoint{3.281042in}{1.964692in}}%
\pgfpathclose%
\pgfusepath{stroke,fill}%
\end{pgfscope}%
\begin{pgfscope}%
\pgfpathrectangle{\pgfqpoint{0.997489in}{0.528000in}}{\pgfqpoint{4.565023in}{3.696000in}}%
\pgfusepath{clip}%
\pgfsetbuttcap%
\pgfsetroundjoin%
\definecolor{currentfill}{rgb}{0.200000,0.200000,0.800000}%
\pgfsetfillcolor{currentfill}%
\pgfsetlinewidth{1.003750pt}%
\definecolor{currentstroke}{rgb}{0.200000,0.200000,0.800000}%
\pgfsetstrokecolor{currentstroke}%
\pgfsetdash{}{0pt}%
\pgfpathmoveto{\pgfqpoint{3.304184in}{2.048296in}}%
\pgfpathcurveto{\pgfqpoint{3.310008in}{2.048296in}}{\pgfqpoint{3.315594in}{2.050610in}}{\pgfqpoint{3.319712in}{2.054728in}}%
\pgfpathcurveto{\pgfqpoint{3.323830in}{2.058846in}}{\pgfqpoint{3.326144in}{2.064433in}}{\pgfqpoint{3.326144in}{2.070257in}}%
\pgfpathcurveto{\pgfqpoint{3.326144in}{2.076081in}}{\pgfqpoint{3.323830in}{2.081667in}}{\pgfqpoint{3.319712in}{2.085785in}}%
\pgfpathcurveto{\pgfqpoint{3.315594in}{2.089903in}}{\pgfqpoint{3.310008in}{2.092217in}}{\pgfqpoint{3.304184in}{2.092217in}}%
\pgfpathcurveto{\pgfqpoint{3.298360in}{2.092217in}}{\pgfqpoint{3.292774in}{2.089903in}}{\pgfqpoint{3.288655in}{2.085785in}}%
\pgfpathcurveto{\pgfqpoint{3.284537in}{2.081667in}}{\pgfqpoint{3.282223in}{2.076081in}}{\pgfqpoint{3.282223in}{2.070257in}}%
\pgfpathcurveto{\pgfqpoint{3.282223in}{2.064433in}}{\pgfqpoint{3.284537in}{2.058846in}}{\pgfqpoint{3.288655in}{2.054728in}}%
\pgfpathcurveto{\pgfqpoint{3.292774in}{2.050610in}}{\pgfqpoint{3.298360in}{2.048296in}}{\pgfqpoint{3.304184in}{2.048296in}}%
\pgfpathlineto{\pgfqpoint{3.304184in}{2.048296in}}%
\pgfpathclose%
\pgfusepath{stroke,fill}%
\end{pgfscope}%
\begin{pgfscope}%
\pgfpathrectangle{\pgfqpoint{0.997489in}{0.528000in}}{\pgfqpoint{4.565023in}{3.696000in}}%
\pgfusepath{clip}%
\pgfsetbuttcap%
\pgfsetroundjoin%
\definecolor{currentfill}{rgb}{0.200000,0.200000,0.800000}%
\pgfsetfillcolor{currentfill}%
\pgfsetlinewidth{1.003750pt}%
\definecolor{currentstroke}{rgb}{0.200000,0.200000,0.800000}%
\pgfsetstrokecolor{currentstroke}%
\pgfsetdash{}{0pt}%
\pgfpathmoveto{\pgfqpoint{3.347812in}{2.005316in}}%
\pgfpathcurveto{\pgfqpoint{3.353636in}{2.005316in}}{\pgfqpoint{3.359223in}{2.007630in}}{\pgfqpoint{3.363341in}{2.011748in}}%
\pgfpathcurveto{\pgfqpoint{3.367459in}{2.015866in}}{\pgfqpoint{3.369773in}{2.021452in}}{\pgfqpoint{3.369773in}{2.027276in}}%
\pgfpathcurveto{\pgfqpoint{3.369773in}{2.033100in}}{\pgfqpoint{3.367459in}{2.038686in}}{\pgfqpoint{3.363341in}{2.042804in}}%
\pgfpathcurveto{\pgfqpoint{3.359223in}{2.046922in}}{\pgfqpoint{3.353636in}{2.049236in}}{\pgfqpoint{3.347812in}{2.049236in}}%
\pgfpathcurveto{\pgfqpoint{3.341989in}{2.049236in}}{\pgfqpoint{3.336402in}{2.046922in}}{\pgfqpoint{3.332284in}{2.042804in}}%
\pgfpathcurveto{\pgfqpoint{3.328166in}{2.038686in}}{\pgfqpoint{3.325852in}{2.033100in}}{\pgfqpoint{3.325852in}{2.027276in}}%
\pgfpathcurveto{\pgfqpoint{3.325852in}{2.021452in}}{\pgfqpoint{3.328166in}{2.015866in}}{\pgfqpoint{3.332284in}{2.011748in}}%
\pgfpathcurveto{\pgfqpoint{3.336402in}{2.007630in}}{\pgfqpoint{3.341989in}{2.005316in}}{\pgfqpoint{3.347812in}{2.005316in}}%
\pgfpathlineto{\pgfqpoint{3.347812in}{2.005316in}}%
\pgfpathclose%
\pgfusepath{stroke,fill}%
\end{pgfscope}%
\begin{pgfscope}%
\pgfpathrectangle{\pgfqpoint{0.997489in}{0.528000in}}{\pgfqpoint{4.565023in}{3.696000in}}%
\pgfusepath{clip}%
\pgfsetbuttcap%
\pgfsetroundjoin%
\definecolor{currentfill}{rgb}{0.200000,0.200000,0.800000}%
\pgfsetfillcolor{currentfill}%
\pgfsetlinewidth{1.003750pt}%
\definecolor{currentstroke}{rgb}{0.200000,0.200000,0.800000}%
\pgfsetstrokecolor{currentstroke}%
\pgfsetdash{}{0pt}%
\pgfpathmoveto{\pgfqpoint{3.377258in}{2.030188in}}%
\pgfpathcurveto{\pgfqpoint{3.383082in}{2.030188in}}{\pgfqpoint{3.388668in}{2.032501in}}{\pgfqpoint{3.392786in}{2.036620in}}%
\pgfpathcurveto{\pgfqpoint{3.396904in}{2.040738in}}{\pgfqpoint{3.399218in}{2.046324in}}{\pgfqpoint{3.399218in}{2.052148in}}%
\pgfpathcurveto{\pgfqpoint{3.399218in}{2.057972in}}{\pgfqpoint{3.396904in}{2.063558in}}{\pgfqpoint{3.392786in}{2.067676in}}%
\pgfpathcurveto{\pgfqpoint{3.388668in}{2.071794in}}{\pgfqpoint{3.383082in}{2.074108in}}{\pgfqpoint{3.377258in}{2.074108in}}%
\pgfpathcurveto{\pgfqpoint{3.371434in}{2.074108in}}{\pgfqpoint{3.365848in}{2.071794in}}{\pgfqpoint{3.361730in}{2.067676in}}%
\pgfpathcurveto{\pgfqpoint{3.357612in}{2.063558in}}{\pgfqpoint{3.355298in}{2.057972in}}{\pgfqpoint{3.355298in}{2.052148in}}%
\pgfpathcurveto{\pgfqpoint{3.355298in}{2.046324in}}{\pgfqpoint{3.357612in}{2.040738in}}{\pgfqpoint{3.361730in}{2.036620in}}%
\pgfpathcurveto{\pgfqpoint{3.365848in}{2.032501in}}{\pgfqpoint{3.371434in}{2.030188in}}{\pgfqpoint{3.377258in}{2.030188in}}%
\pgfpathlineto{\pgfqpoint{3.377258in}{2.030188in}}%
\pgfpathclose%
\pgfusepath{stroke,fill}%
\end{pgfscope}%
\begin{pgfscope}%
\pgfpathrectangle{\pgfqpoint{0.997489in}{0.528000in}}{\pgfqpoint{4.565023in}{3.696000in}}%
\pgfusepath{clip}%
\pgfsetbuttcap%
\pgfsetroundjoin%
\definecolor{currentfill}{rgb}{0.200000,0.200000,0.800000}%
\pgfsetfillcolor{currentfill}%
\pgfsetlinewidth{1.003750pt}%
\definecolor{currentstroke}{rgb}{0.200000,0.200000,0.800000}%
\pgfsetstrokecolor{currentstroke}%
\pgfsetdash{}{0pt}%
\pgfpathmoveto{\pgfqpoint{3.385818in}{2.104392in}}%
\pgfpathcurveto{\pgfqpoint{3.391642in}{2.104392in}}{\pgfqpoint{3.397228in}{2.106706in}}{\pgfqpoint{3.401346in}{2.110824in}}%
\pgfpathcurveto{\pgfqpoint{3.405464in}{2.114943in}}{\pgfqpoint{3.407778in}{2.120529in}}{\pgfqpoint{3.407778in}{2.126353in}}%
\pgfpathcurveto{\pgfqpoint{3.407778in}{2.132177in}}{\pgfqpoint{3.405464in}{2.137763in}}{\pgfqpoint{3.401346in}{2.141881in}}%
\pgfpathcurveto{\pgfqpoint{3.397228in}{2.145999in}}{\pgfqpoint{3.391642in}{2.148313in}}{\pgfqpoint{3.385818in}{2.148313in}}%
\pgfpathcurveto{\pgfqpoint{3.379994in}{2.148313in}}{\pgfqpoint{3.374408in}{2.145999in}}{\pgfqpoint{3.370290in}{2.141881in}}%
\pgfpathcurveto{\pgfqpoint{3.366172in}{2.137763in}}{\pgfqpoint{3.363858in}{2.132177in}}{\pgfqpoint{3.363858in}{2.126353in}}%
\pgfpathcurveto{\pgfqpoint{3.363858in}{2.120529in}}{\pgfqpoint{3.366172in}{2.114943in}}{\pgfqpoint{3.370290in}{2.110824in}}%
\pgfpathcurveto{\pgfqpoint{3.374408in}{2.106706in}}{\pgfqpoint{3.379994in}{2.104392in}}{\pgfqpoint{3.385818in}{2.104392in}}%
\pgfpathlineto{\pgfqpoint{3.385818in}{2.104392in}}%
\pgfpathclose%
\pgfusepath{stroke,fill}%
\end{pgfscope}%
\begin{pgfscope}%
\pgfpathrectangle{\pgfqpoint{0.997489in}{0.528000in}}{\pgfqpoint{4.565023in}{3.696000in}}%
\pgfusepath{clip}%
\pgfsetbuttcap%
\pgfsetroundjoin%
\definecolor{currentfill}{rgb}{0.200000,0.200000,0.800000}%
\pgfsetfillcolor{currentfill}%
\pgfsetlinewidth{1.003750pt}%
\definecolor{currentstroke}{rgb}{0.200000,0.200000,0.800000}%
\pgfsetstrokecolor{currentstroke}%
\pgfsetdash{}{0pt}%
\pgfpathmoveto{\pgfqpoint{3.446332in}{2.044602in}}%
\pgfpathcurveto{\pgfqpoint{3.452156in}{2.044602in}}{\pgfqpoint{3.457742in}{2.046916in}}{\pgfqpoint{3.461860in}{2.051034in}}%
\pgfpathcurveto{\pgfqpoint{3.465978in}{2.055152in}}{\pgfqpoint{3.468292in}{2.060738in}}{\pgfqpoint{3.468292in}{2.066562in}}%
\pgfpathcurveto{\pgfqpoint{3.468292in}{2.072386in}}{\pgfqpoint{3.465978in}{2.077972in}}{\pgfqpoint{3.461860in}{2.082090in}}%
\pgfpathcurveto{\pgfqpoint{3.457742in}{2.086208in}}{\pgfqpoint{3.452156in}{2.088522in}}{\pgfqpoint{3.446332in}{2.088522in}}%
\pgfpathcurveto{\pgfqpoint{3.440508in}{2.088522in}}{\pgfqpoint{3.434921in}{2.086208in}}{\pgfqpoint{3.430803in}{2.082090in}}%
\pgfpathcurveto{\pgfqpoint{3.426685in}{2.077972in}}{\pgfqpoint{3.424371in}{2.072386in}}{\pgfqpoint{3.424371in}{2.066562in}}%
\pgfpathcurveto{\pgfqpoint{3.424371in}{2.060738in}}{\pgfqpoint{3.426685in}{2.055152in}}{\pgfqpoint{3.430803in}{2.051034in}}%
\pgfpathcurveto{\pgfqpoint{3.434921in}{2.046916in}}{\pgfqpoint{3.440508in}{2.044602in}}{\pgfqpoint{3.446332in}{2.044602in}}%
\pgfpathlineto{\pgfqpoint{3.446332in}{2.044602in}}%
\pgfpathclose%
\pgfusepath{stroke,fill}%
\end{pgfscope}%
\begin{pgfscope}%
\pgfpathrectangle{\pgfqpoint{0.997489in}{0.528000in}}{\pgfqpoint{4.565023in}{3.696000in}}%
\pgfusepath{clip}%
\pgfsetbuttcap%
\pgfsetroundjoin%
\definecolor{currentfill}{rgb}{0.200000,0.200000,0.800000}%
\pgfsetfillcolor{currentfill}%
\pgfsetlinewidth{1.003750pt}%
\definecolor{currentstroke}{rgb}{0.200000,0.200000,0.800000}%
\pgfsetstrokecolor{currentstroke}%
\pgfsetdash{}{0pt}%
\pgfpathmoveto{\pgfqpoint{3.482439in}{2.051722in}}%
\pgfpathcurveto{\pgfqpoint{3.488263in}{2.051722in}}{\pgfqpoint{3.493850in}{2.054036in}}{\pgfqpoint{3.497968in}{2.058154in}}%
\pgfpathcurveto{\pgfqpoint{3.502086in}{2.062272in}}{\pgfqpoint{3.504400in}{2.067858in}}{\pgfqpoint{3.504400in}{2.073682in}}%
\pgfpathcurveto{\pgfqpoint{3.504400in}{2.079506in}}{\pgfqpoint{3.502086in}{2.085093in}}{\pgfqpoint{3.497968in}{2.089211in}}%
\pgfpathcurveto{\pgfqpoint{3.493850in}{2.093329in}}{\pgfqpoint{3.488263in}{2.095643in}}{\pgfqpoint{3.482439in}{2.095643in}}%
\pgfpathcurveto{\pgfqpoint{3.476615in}{2.095643in}}{\pgfqpoint{3.471029in}{2.093329in}}{\pgfqpoint{3.466911in}{2.089211in}}%
\pgfpathcurveto{\pgfqpoint{3.462793in}{2.085093in}}{\pgfqpoint{3.460479in}{2.079506in}}{\pgfqpoint{3.460479in}{2.073682in}}%
\pgfpathcurveto{\pgfqpoint{3.460479in}{2.067858in}}{\pgfqpoint{3.462793in}{2.062272in}}{\pgfqpoint{3.466911in}{2.058154in}}%
\pgfpathcurveto{\pgfqpoint{3.471029in}{2.054036in}}{\pgfqpoint{3.476615in}{2.051722in}}{\pgfqpoint{3.482439in}{2.051722in}}%
\pgfpathlineto{\pgfqpoint{3.482439in}{2.051722in}}%
\pgfpathclose%
\pgfusepath{stroke,fill}%
\end{pgfscope}%
\begin{pgfscope}%
\pgfpathrectangle{\pgfqpoint{0.997489in}{0.528000in}}{\pgfqpoint{4.565023in}{3.696000in}}%
\pgfusepath{clip}%
\pgfsetbuttcap%
\pgfsetroundjoin%
\definecolor{currentfill}{rgb}{0.200000,0.200000,0.800000}%
\pgfsetfillcolor{currentfill}%
\pgfsetlinewidth{1.003750pt}%
\definecolor{currentstroke}{rgb}{0.200000,0.200000,0.800000}%
\pgfsetstrokecolor{currentstroke}%
\pgfsetdash{}{0pt}%
\pgfpathmoveto{\pgfqpoint{3.469583in}{2.140351in}}%
\pgfpathcurveto{\pgfqpoint{3.475407in}{2.140351in}}{\pgfqpoint{3.480993in}{2.142665in}}{\pgfqpoint{3.485111in}{2.146783in}}%
\pgfpathcurveto{\pgfqpoint{3.489229in}{2.150901in}}{\pgfqpoint{3.491543in}{2.156487in}}{\pgfqpoint{3.491543in}{2.162311in}}%
\pgfpathcurveto{\pgfqpoint{3.491543in}{2.168135in}}{\pgfqpoint{3.489229in}{2.173721in}}{\pgfqpoint{3.485111in}{2.177840in}}%
\pgfpathcurveto{\pgfqpoint{3.480993in}{2.181958in}}{\pgfqpoint{3.475407in}{2.184272in}}{\pgfqpoint{3.469583in}{2.184272in}}%
\pgfpathcurveto{\pgfqpoint{3.463759in}{2.184272in}}{\pgfqpoint{3.458173in}{2.181958in}}{\pgfqpoint{3.454054in}{2.177840in}}%
\pgfpathcurveto{\pgfqpoint{3.449936in}{2.173721in}}{\pgfqpoint{3.447622in}{2.168135in}}{\pgfqpoint{3.447622in}{2.162311in}}%
\pgfpathcurveto{\pgfqpoint{3.447622in}{2.156487in}}{\pgfqpoint{3.449936in}{2.150901in}}{\pgfqpoint{3.454054in}{2.146783in}}%
\pgfpathcurveto{\pgfqpoint{3.458173in}{2.142665in}}{\pgfqpoint{3.463759in}{2.140351in}}{\pgfqpoint{3.469583in}{2.140351in}}%
\pgfpathlineto{\pgfqpoint{3.469583in}{2.140351in}}%
\pgfpathclose%
\pgfusepath{stroke,fill}%
\end{pgfscope}%
\begin{pgfscope}%
\pgfpathrectangle{\pgfqpoint{0.997489in}{0.528000in}}{\pgfqpoint{4.565023in}{3.696000in}}%
\pgfusepath{clip}%
\pgfsetbuttcap%
\pgfsetroundjoin%
\definecolor{currentfill}{rgb}{0.200000,0.200000,0.800000}%
\pgfsetfillcolor{currentfill}%
\pgfsetlinewidth{1.003750pt}%
\definecolor{currentstroke}{rgb}{0.200000,0.200000,0.800000}%
\pgfsetstrokecolor{currentstroke}%
\pgfsetdash{}{0pt}%
\pgfpathmoveto{\pgfqpoint{3.510072in}{2.136115in}}%
\pgfpathcurveto{\pgfqpoint{3.515896in}{2.136115in}}{\pgfqpoint{3.521482in}{2.138429in}}{\pgfqpoint{3.525600in}{2.142547in}}%
\pgfpathcurveto{\pgfqpoint{3.529718in}{2.146665in}}{\pgfqpoint{3.532032in}{2.152251in}}{\pgfqpoint{3.532032in}{2.158075in}}%
\pgfpathcurveto{\pgfqpoint{3.532032in}{2.163899in}}{\pgfqpoint{3.529718in}{2.169485in}}{\pgfqpoint{3.525600in}{2.173603in}}%
\pgfpathcurveto{\pgfqpoint{3.521482in}{2.177721in}}{\pgfqpoint{3.515896in}{2.180035in}}{\pgfqpoint{3.510072in}{2.180035in}}%
\pgfpathcurveto{\pgfqpoint{3.504248in}{2.180035in}}{\pgfqpoint{3.498662in}{2.177721in}}{\pgfqpoint{3.494544in}{2.173603in}}%
\pgfpathcurveto{\pgfqpoint{3.490425in}{2.169485in}}{\pgfqpoint{3.488112in}{2.163899in}}{\pgfqpoint{3.488112in}{2.158075in}}%
\pgfpathcurveto{\pgfqpoint{3.488112in}{2.152251in}}{\pgfqpoint{3.490425in}{2.146665in}}{\pgfqpoint{3.494544in}{2.142547in}}%
\pgfpathcurveto{\pgfqpoint{3.498662in}{2.138429in}}{\pgfqpoint{3.504248in}{2.136115in}}{\pgfqpoint{3.510072in}{2.136115in}}%
\pgfpathlineto{\pgfqpoint{3.510072in}{2.136115in}}%
\pgfpathclose%
\pgfusepath{stroke,fill}%
\end{pgfscope}%
\begin{pgfscope}%
\pgfpathrectangle{\pgfqpoint{0.997489in}{0.528000in}}{\pgfqpoint{4.565023in}{3.696000in}}%
\pgfusepath{clip}%
\pgfsetbuttcap%
\pgfsetroundjoin%
\definecolor{currentfill}{rgb}{0.200000,0.200000,0.800000}%
\pgfsetfillcolor{currentfill}%
\pgfsetlinewidth{1.003750pt}%
\definecolor{currentstroke}{rgb}{0.200000,0.200000,0.800000}%
\pgfsetstrokecolor{currentstroke}%
\pgfsetdash{}{0pt}%
\pgfpathmoveto{\pgfqpoint{3.516667in}{2.179018in}}%
\pgfpathcurveto{\pgfqpoint{3.522491in}{2.179018in}}{\pgfqpoint{3.528077in}{2.181331in}}{\pgfqpoint{3.532195in}{2.185450in}}%
\pgfpathcurveto{\pgfqpoint{3.536313in}{2.189568in}}{\pgfqpoint{3.538627in}{2.195154in}}{\pgfqpoint{3.538627in}{2.200978in}}%
\pgfpathcurveto{\pgfqpoint{3.538627in}{2.206802in}}{\pgfqpoint{3.536313in}{2.212388in}}{\pgfqpoint{3.532195in}{2.216506in}}%
\pgfpathcurveto{\pgfqpoint{3.528077in}{2.220624in}}{\pgfqpoint{3.522491in}{2.222938in}}{\pgfqpoint{3.516667in}{2.222938in}}%
\pgfpathcurveto{\pgfqpoint{3.510843in}{2.222938in}}{\pgfqpoint{3.505257in}{2.220624in}}{\pgfqpoint{3.501139in}{2.216506in}}%
\pgfpathcurveto{\pgfqpoint{3.497021in}{2.212388in}}{\pgfqpoint{3.494707in}{2.206802in}}{\pgfqpoint{3.494707in}{2.200978in}}%
\pgfpathcurveto{\pgfqpoint{3.494707in}{2.195154in}}{\pgfqpoint{3.497021in}{2.189568in}}{\pgfqpoint{3.501139in}{2.185450in}}%
\pgfpathcurveto{\pgfqpoint{3.505257in}{2.181331in}}{\pgfqpoint{3.510843in}{2.179018in}}{\pgfqpoint{3.516667in}{2.179018in}}%
\pgfpathlineto{\pgfqpoint{3.516667in}{2.179018in}}%
\pgfpathclose%
\pgfusepath{stroke,fill}%
\end{pgfscope}%
\begin{pgfscope}%
\pgfpathrectangle{\pgfqpoint{0.997489in}{0.528000in}}{\pgfqpoint{4.565023in}{3.696000in}}%
\pgfusepath{clip}%
\pgfsetbuttcap%
\pgfsetroundjoin%
\definecolor{currentfill}{rgb}{0.200000,0.200000,0.800000}%
\pgfsetfillcolor{currentfill}%
\pgfsetlinewidth{1.003750pt}%
\definecolor{currentstroke}{rgb}{0.200000,0.200000,0.800000}%
\pgfsetstrokecolor{currentstroke}%
\pgfsetdash{}{0pt}%
\pgfpathmoveto{\pgfqpoint{3.515259in}{2.225137in}}%
\pgfpathcurveto{\pgfqpoint{3.521083in}{2.225137in}}{\pgfqpoint{3.526669in}{2.227451in}}{\pgfqpoint{3.530787in}{2.231569in}}%
\pgfpathcurveto{\pgfqpoint{3.534905in}{2.235687in}}{\pgfqpoint{3.537219in}{2.241274in}}{\pgfqpoint{3.537219in}{2.247097in}}%
\pgfpathcurveto{\pgfqpoint{3.537219in}{2.252921in}}{\pgfqpoint{3.534905in}{2.258508in}}{\pgfqpoint{3.530787in}{2.262626in}}%
\pgfpathcurveto{\pgfqpoint{3.526669in}{2.266744in}}{\pgfqpoint{3.521083in}{2.269058in}}{\pgfqpoint{3.515259in}{2.269058in}}%
\pgfpathcurveto{\pgfqpoint{3.509435in}{2.269058in}}{\pgfqpoint{3.503849in}{2.266744in}}{\pgfqpoint{3.499731in}{2.262626in}}%
\pgfpathcurveto{\pgfqpoint{3.495613in}{2.258508in}}{\pgfqpoint{3.493299in}{2.252921in}}{\pgfqpoint{3.493299in}{2.247097in}}%
\pgfpathcurveto{\pgfqpoint{3.493299in}{2.241274in}}{\pgfqpoint{3.495613in}{2.235687in}}{\pgfqpoint{3.499731in}{2.231569in}}%
\pgfpathcurveto{\pgfqpoint{3.503849in}{2.227451in}}{\pgfqpoint{3.509435in}{2.225137in}}{\pgfqpoint{3.515259in}{2.225137in}}%
\pgfpathlineto{\pgfqpoint{3.515259in}{2.225137in}}%
\pgfpathclose%
\pgfusepath{stroke,fill}%
\end{pgfscope}%
\begin{pgfscope}%
\pgfpathrectangle{\pgfqpoint{0.997489in}{0.528000in}}{\pgfqpoint{4.565023in}{3.696000in}}%
\pgfusepath{clip}%
\pgfsetbuttcap%
\pgfsetroundjoin%
\definecolor{currentfill}{rgb}{0.200000,0.200000,0.800000}%
\pgfsetfillcolor{currentfill}%
\pgfsetlinewidth{1.003750pt}%
\definecolor{currentstroke}{rgb}{0.200000,0.200000,0.800000}%
\pgfsetstrokecolor{currentstroke}%
\pgfsetdash{}{0pt}%
\pgfpathmoveto{\pgfqpoint{3.620601in}{2.163366in}}%
\pgfpathcurveto{\pgfqpoint{3.626425in}{2.163366in}}{\pgfqpoint{3.632011in}{2.165679in}}{\pgfqpoint{3.636129in}{2.169798in}}%
\pgfpathcurveto{\pgfqpoint{3.640247in}{2.173916in}}{\pgfqpoint{3.642561in}{2.179502in}}{\pgfqpoint{3.642561in}{2.185326in}}%
\pgfpathcurveto{\pgfqpoint{3.642561in}{2.191150in}}{\pgfqpoint{3.640247in}{2.196736in}}{\pgfqpoint{3.636129in}{2.200854in}}%
\pgfpathcurveto{\pgfqpoint{3.632011in}{2.204972in}}{\pgfqpoint{3.626425in}{2.207286in}}{\pgfqpoint{3.620601in}{2.207286in}}%
\pgfpathcurveto{\pgfqpoint{3.614777in}{2.207286in}}{\pgfqpoint{3.609191in}{2.204972in}}{\pgfqpoint{3.605073in}{2.200854in}}%
\pgfpathcurveto{\pgfqpoint{3.600954in}{2.196736in}}{\pgfqpoint{3.598641in}{2.191150in}}{\pgfqpoint{3.598641in}{2.185326in}}%
\pgfpathcurveto{\pgfqpoint{3.598641in}{2.179502in}}{\pgfqpoint{3.600954in}{2.173916in}}{\pgfqpoint{3.605073in}{2.169798in}}%
\pgfpathcurveto{\pgfqpoint{3.609191in}{2.165679in}}{\pgfqpoint{3.614777in}{2.163366in}}{\pgfqpoint{3.620601in}{2.163366in}}%
\pgfpathlineto{\pgfqpoint{3.620601in}{2.163366in}}%
\pgfpathclose%
\pgfusepath{stroke,fill}%
\end{pgfscope}%
\begin{pgfscope}%
\pgfpathrectangle{\pgfqpoint{0.997489in}{0.528000in}}{\pgfqpoint{4.565023in}{3.696000in}}%
\pgfusepath{clip}%
\pgfsetbuttcap%
\pgfsetroundjoin%
\definecolor{currentfill}{rgb}{0.200000,0.200000,0.800000}%
\pgfsetfillcolor{currentfill}%
\pgfsetlinewidth{1.003750pt}%
\definecolor{currentstroke}{rgb}{0.200000,0.200000,0.800000}%
\pgfsetstrokecolor{currentstroke}%
\pgfsetdash{}{0pt}%
\pgfpathmoveto{\pgfqpoint{3.667239in}{2.170535in}}%
\pgfpathcurveto{\pgfqpoint{3.673063in}{2.170535in}}{\pgfqpoint{3.678649in}{2.172849in}}{\pgfqpoint{3.682767in}{2.176967in}}%
\pgfpathcurveto{\pgfqpoint{3.686885in}{2.181086in}}{\pgfqpoint{3.689199in}{2.186672in}}{\pgfqpoint{3.689199in}{2.192496in}}%
\pgfpathcurveto{\pgfqpoint{3.689199in}{2.198320in}}{\pgfqpoint{3.686885in}{2.203906in}}{\pgfqpoint{3.682767in}{2.208024in}}%
\pgfpathcurveto{\pgfqpoint{3.678649in}{2.212142in}}{\pgfqpoint{3.673063in}{2.214456in}}{\pgfqpoint{3.667239in}{2.214456in}}%
\pgfpathcurveto{\pgfqpoint{3.661415in}{2.214456in}}{\pgfqpoint{3.655829in}{2.212142in}}{\pgfqpoint{3.651711in}{2.208024in}}%
\pgfpathcurveto{\pgfqpoint{3.647592in}{2.203906in}}{\pgfqpoint{3.645279in}{2.198320in}}{\pgfqpoint{3.645279in}{2.192496in}}%
\pgfpathcurveto{\pgfqpoint{3.645279in}{2.186672in}}{\pgfqpoint{3.647592in}{2.181086in}}{\pgfqpoint{3.651711in}{2.176967in}}%
\pgfpathcurveto{\pgfqpoint{3.655829in}{2.172849in}}{\pgfqpoint{3.661415in}{2.170535in}}{\pgfqpoint{3.667239in}{2.170535in}}%
\pgfpathlineto{\pgfqpoint{3.667239in}{2.170535in}}%
\pgfpathclose%
\pgfusepath{stroke,fill}%
\end{pgfscope}%
\begin{pgfscope}%
\pgfpathrectangle{\pgfqpoint{0.997489in}{0.528000in}}{\pgfqpoint{4.565023in}{3.696000in}}%
\pgfusepath{clip}%
\pgfsetbuttcap%
\pgfsetroundjoin%
\definecolor{currentfill}{rgb}{0.200000,0.200000,0.800000}%
\pgfsetfillcolor{currentfill}%
\pgfsetlinewidth{1.003750pt}%
\definecolor{currentstroke}{rgb}{0.200000,0.200000,0.800000}%
\pgfsetstrokecolor{currentstroke}%
\pgfsetdash{}{0pt}%
\pgfpathmoveto{\pgfqpoint{3.621010in}{2.250983in}}%
\pgfpathcurveto{\pgfqpoint{3.626834in}{2.250983in}}{\pgfqpoint{3.632420in}{2.253297in}}{\pgfqpoint{3.636538in}{2.257415in}}%
\pgfpathcurveto{\pgfqpoint{3.640657in}{2.261533in}}{\pgfqpoint{3.642970in}{2.267120in}}{\pgfqpoint{3.642970in}{2.272943in}}%
\pgfpathcurveto{\pgfqpoint{3.642970in}{2.278767in}}{\pgfqpoint{3.640657in}{2.284354in}}{\pgfqpoint{3.636538in}{2.288472in}}%
\pgfpathcurveto{\pgfqpoint{3.632420in}{2.292590in}}{\pgfqpoint{3.626834in}{2.294904in}}{\pgfqpoint{3.621010in}{2.294904in}}%
\pgfpathcurveto{\pgfqpoint{3.615186in}{2.294904in}}{\pgfqpoint{3.609600in}{2.292590in}}{\pgfqpoint{3.605482in}{2.288472in}}%
\pgfpathcurveto{\pgfqpoint{3.601364in}{2.284354in}}{\pgfqpoint{3.599050in}{2.278767in}}{\pgfqpoint{3.599050in}{2.272943in}}%
\pgfpathcurveto{\pgfqpoint{3.599050in}{2.267120in}}{\pgfqpoint{3.601364in}{2.261533in}}{\pgfqpoint{3.605482in}{2.257415in}}%
\pgfpathcurveto{\pgfqpoint{3.609600in}{2.253297in}}{\pgfqpoint{3.615186in}{2.250983in}}{\pgfqpoint{3.621010in}{2.250983in}}%
\pgfpathlineto{\pgfqpoint{3.621010in}{2.250983in}}%
\pgfpathclose%
\pgfusepath{stroke,fill}%
\end{pgfscope}%
\begin{pgfscope}%
\pgfpathrectangle{\pgfqpoint{0.997489in}{0.528000in}}{\pgfqpoint{4.565023in}{3.696000in}}%
\pgfusepath{clip}%
\pgfsetbuttcap%
\pgfsetroundjoin%
\definecolor{currentfill}{rgb}{0.200000,0.200000,0.800000}%
\pgfsetfillcolor{currentfill}%
\pgfsetlinewidth{1.003750pt}%
\definecolor{currentstroke}{rgb}{0.200000,0.200000,0.800000}%
\pgfsetstrokecolor{currentstroke}%
\pgfsetdash{}{0pt}%
\pgfpathmoveto{\pgfqpoint{3.618526in}{2.290513in}}%
\pgfpathcurveto{\pgfqpoint{3.624350in}{2.290513in}}{\pgfqpoint{3.629936in}{2.292826in}}{\pgfqpoint{3.634055in}{2.296945in}}%
\pgfpathcurveto{\pgfqpoint{3.638173in}{2.301063in}}{\pgfqpoint{3.640487in}{2.306649in}}{\pgfqpoint{3.640487in}{2.312473in}}%
\pgfpathcurveto{\pgfqpoint{3.640487in}{2.318297in}}{\pgfqpoint{3.638173in}{2.323883in}}{\pgfqpoint{3.634055in}{2.328001in}}%
\pgfpathcurveto{\pgfqpoint{3.629936in}{2.332119in}}{\pgfqpoint{3.624350in}{2.334433in}}{\pgfqpoint{3.618526in}{2.334433in}}%
\pgfpathcurveto{\pgfqpoint{3.612702in}{2.334433in}}{\pgfqpoint{3.607116in}{2.332119in}}{\pgfqpoint{3.602998in}{2.328001in}}%
\pgfpathcurveto{\pgfqpoint{3.598880in}{2.323883in}}{\pgfqpoint{3.596566in}{2.318297in}}{\pgfqpoint{3.596566in}{2.312473in}}%
\pgfpathcurveto{\pgfqpoint{3.596566in}{2.306649in}}{\pgfqpoint{3.598880in}{2.301063in}}{\pgfqpoint{3.602998in}{2.296945in}}%
\pgfpathcurveto{\pgfqpoint{3.607116in}{2.292826in}}{\pgfqpoint{3.612702in}{2.290513in}}{\pgfqpoint{3.618526in}{2.290513in}}%
\pgfpathlineto{\pgfqpoint{3.618526in}{2.290513in}}%
\pgfpathclose%
\pgfusepath{stroke,fill}%
\end{pgfscope}%
\begin{pgfscope}%
\pgfpathrectangle{\pgfqpoint{0.997489in}{0.528000in}}{\pgfqpoint{4.565023in}{3.696000in}}%
\pgfusepath{clip}%
\pgfsetbuttcap%
\pgfsetroundjoin%
\definecolor{currentfill}{rgb}{0.200000,0.200000,0.800000}%
\pgfsetfillcolor{currentfill}%
\pgfsetlinewidth{1.003750pt}%
\definecolor{currentstroke}{rgb}{0.200000,0.200000,0.800000}%
\pgfsetstrokecolor{currentstroke}%
\pgfsetdash{}{0pt}%
\pgfpathmoveto{\pgfqpoint{3.706583in}{2.276261in}}%
\pgfpathcurveto{\pgfqpoint{3.712407in}{2.276261in}}{\pgfqpoint{3.717993in}{2.278575in}}{\pgfqpoint{3.722111in}{2.282693in}}%
\pgfpathcurveto{\pgfqpoint{3.726229in}{2.286811in}}{\pgfqpoint{3.728543in}{2.292397in}}{\pgfqpoint{3.728543in}{2.298221in}}%
\pgfpathcurveto{\pgfqpoint{3.728543in}{2.304045in}}{\pgfqpoint{3.726229in}{2.309631in}}{\pgfqpoint{3.722111in}{2.313750in}}%
\pgfpathcurveto{\pgfqpoint{3.717993in}{2.317868in}}{\pgfqpoint{3.712407in}{2.320182in}}{\pgfqpoint{3.706583in}{2.320182in}}%
\pgfpathcurveto{\pgfqpoint{3.700759in}{2.320182in}}{\pgfqpoint{3.695173in}{2.317868in}}{\pgfqpoint{3.691055in}{2.313750in}}%
\pgfpathcurveto{\pgfqpoint{3.686937in}{2.309631in}}{\pgfqpoint{3.684623in}{2.304045in}}{\pgfqpoint{3.684623in}{2.298221in}}%
\pgfpathcurveto{\pgfqpoint{3.684623in}{2.292397in}}{\pgfqpoint{3.686937in}{2.286811in}}{\pgfqpoint{3.691055in}{2.282693in}}%
\pgfpathcurveto{\pgfqpoint{3.695173in}{2.278575in}}{\pgfqpoint{3.700759in}{2.276261in}}{\pgfqpoint{3.706583in}{2.276261in}}%
\pgfpathlineto{\pgfqpoint{3.706583in}{2.276261in}}%
\pgfpathclose%
\pgfusepath{stroke,fill}%
\end{pgfscope}%
\begin{pgfscope}%
\pgfpathrectangle{\pgfqpoint{0.997489in}{0.528000in}}{\pgfqpoint{4.565023in}{3.696000in}}%
\pgfusepath{clip}%
\pgfsetbuttcap%
\pgfsetroundjoin%
\definecolor{currentfill}{rgb}{0.200000,0.200000,0.800000}%
\pgfsetfillcolor{currentfill}%
\pgfsetlinewidth{1.003750pt}%
\definecolor{currentstroke}{rgb}{0.200000,0.200000,0.800000}%
\pgfsetstrokecolor{currentstroke}%
\pgfsetdash{}{0pt}%
\pgfpathmoveto{\pgfqpoint{3.727964in}{2.305531in}}%
\pgfpathcurveto{\pgfqpoint{3.733788in}{2.305531in}}{\pgfqpoint{3.739374in}{2.307845in}}{\pgfqpoint{3.743492in}{2.311963in}}%
\pgfpathcurveto{\pgfqpoint{3.747611in}{2.316081in}}{\pgfqpoint{3.749924in}{2.321667in}}{\pgfqpoint{3.749924in}{2.327491in}}%
\pgfpathcurveto{\pgfqpoint{3.749924in}{2.333315in}}{\pgfqpoint{3.747611in}{2.338901in}}{\pgfqpoint{3.743492in}{2.343019in}}%
\pgfpathcurveto{\pgfqpoint{3.739374in}{2.347137in}}{\pgfqpoint{3.733788in}{2.349451in}}{\pgfqpoint{3.727964in}{2.349451in}}%
\pgfpathcurveto{\pgfqpoint{3.722140in}{2.349451in}}{\pgfqpoint{3.716554in}{2.347137in}}{\pgfqpoint{3.712436in}{2.343019in}}%
\pgfpathcurveto{\pgfqpoint{3.708318in}{2.338901in}}{\pgfqpoint{3.706004in}{2.333315in}}{\pgfqpoint{3.706004in}{2.327491in}}%
\pgfpathcurveto{\pgfqpoint{3.706004in}{2.321667in}}{\pgfqpoint{3.708318in}{2.316081in}}{\pgfqpoint{3.712436in}{2.311963in}}%
\pgfpathcurveto{\pgfqpoint{3.716554in}{2.307845in}}{\pgfqpoint{3.722140in}{2.305531in}}{\pgfqpoint{3.727964in}{2.305531in}}%
\pgfpathlineto{\pgfqpoint{3.727964in}{2.305531in}}%
\pgfpathclose%
\pgfusepath{stroke,fill}%
\end{pgfscope}%
\begin{pgfscope}%
\pgfpathrectangle{\pgfqpoint{0.997489in}{0.528000in}}{\pgfqpoint{4.565023in}{3.696000in}}%
\pgfusepath{clip}%
\pgfsetbuttcap%
\pgfsetroundjoin%
\definecolor{currentfill}{rgb}{0.200000,0.200000,0.800000}%
\pgfsetfillcolor{currentfill}%
\pgfsetlinewidth{1.003750pt}%
\definecolor{currentstroke}{rgb}{0.200000,0.200000,0.800000}%
\pgfsetstrokecolor{currentstroke}%
\pgfsetdash{}{0pt}%
\pgfpathmoveto{\pgfqpoint{3.644294in}{2.377776in}}%
\pgfpathcurveto{\pgfqpoint{3.650118in}{2.377776in}}{\pgfqpoint{3.655704in}{2.380089in}}{\pgfqpoint{3.659822in}{2.384208in}}%
\pgfpathcurveto{\pgfqpoint{3.663940in}{2.388326in}}{\pgfqpoint{3.666254in}{2.393912in}}{\pgfqpoint{3.666254in}{2.399736in}}%
\pgfpathcurveto{\pgfqpoint{3.666254in}{2.405560in}}{\pgfqpoint{3.663940in}{2.411146in}}{\pgfqpoint{3.659822in}{2.415264in}}%
\pgfpathcurveto{\pgfqpoint{3.655704in}{2.419382in}}{\pgfqpoint{3.650118in}{2.421696in}}{\pgfqpoint{3.644294in}{2.421696in}}%
\pgfpathcurveto{\pgfqpoint{3.638470in}{2.421696in}}{\pgfqpoint{3.632884in}{2.419382in}}{\pgfqpoint{3.628766in}{2.415264in}}%
\pgfpathcurveto{\pgfqpoint{3.624648in}{2.411146in}}{\pgfqpoint{3.622334in}{2.405560in}}{\pgfqpoint{3.622334in}{2.399736in}}%
\pgfpathcurveto{\pgfqpoint{3.622334in}{2.393912in}}{\pgfqpoint{3.624648in}{2.388326in}}{\pgfqpoint{3.628766in}{2.384208in}}%
\pgfpathcurveto{\pgfqpoint{3.632884in}{2.380089in}}{\pgfqpoint{3.638470in}{2.377776in}}{\pgfqpoint{3.644294in}{2.377776in}}%
\pgfpathlineto{\pgfqpoint{3.644294in}{2.377776in}}%
\pgfpathclose%
\pgfusepath{stroke,fill}%
\end{pgfscope}%
\begin{pgfscope}%
\pgfpathrectangle{\pgfqpoint{0.997489in}{0.528000in}}{\pgfqpoint{4.565023in}{3.696000in}}%
\pgfusepath{clip}%
\pgfsetbuttcap%
\pgfsetroundjoin%
\definecolor{currentfill}{rgb}{0.200000,0.200000,0.800000}%
\pgfsetfillcolor{currentfill}%
\pgfsetlinewidth{1.003750pt}%
\definecolor{currentstroke}{rgb}{0.200000,0.200000,0.800000}%
\pgfsetstrokecolor{currentstroke}%
\pgfsetdash{}{0pt}%
\pgfpathmoveto{\pgfqpoint{3.703779in}{2.389048in}}%
\pgfpathcurveto{\pgfqpoint{3.709603in}{2.389048in}}{\pgfqpoint{3.715189in}{2.391362in}}{\pgfqpoint{3.719307in}{2.395480in}}%
\pgfpathcurveto{\pgfqpoint{3.723425in}{2.399598in}}{\pgfqpoint{3.725739in}{2.405184in}}{\pgfqpoint{3.725739in}{2.411008in}}%
\pgfpathcurveto{\pgfqpoint{3.725739in}{2.416832in}}{\pgfqpoint{3.723425in}{2.422418in}}{\pgfqpoint{3.719307in}{2.426536in}}%
\pgfpathcurveto{\pgfqpoint{3.715189in}{2.430655in}}{\pgfqpoint{3.709603in}{2.432968in}}{\pgfqpoint{3.703779in}{2.432968in}}%
\pgfpathcurveto{\pgfqpoint{3.697955in}{2.432968in}}{\pgfqpoint{3.692369in}{2.430655in}}{\pgfqpoint{3.688251in}{2.426536in}}%
\pgfpathcurveto{\pgfqpoint{3.684132in}{2.422418in}}{\pgfqpoint{3.681819in}{2.416832in}}{\pgfqpoint{3.681819in}{2.411008in}}%
\pgfpathcurveto{\pgfqpoint{3.681819in}{2.405184in}}{\pgfqpoint{3.684132in}{2.399598in}}{\pgfqpoint{3.688251in}{2.395480in}}%
\pgfpathcurveto{\pgfqpoint{3.692369in}{2.391362in}}{\pgfqpoint{3.697955in}{2.389048in}}{\pgfqpoint{3.703779in}{2.389048in}}%
\pgfpathlineto{\pgfqpoint{3.703779in}{2.389048in}}%
\pgfpathclose%
\pgfusepath{stroke,fill}%
\end{pgfscope}%
\begin{pgfscope}%
\pgfpathrectangle{\pgfqpoint{0.997489in}{0.528000in}}{\pgfqpoint{4.565023in}{3.696000in}}%
\pgfusepath{clip}%
\pgfsetbuttcap%
\pgfsetroundjoin%
\definecolor{currentfill}{rgb}{0.800000,0.800000,0.200000}%
\pgfsetfillcolor{currentfill}%
\pgfsetlinewidth{1.003750pt}%
\definecolor{currentstroke}{rgb}{0.800000,0.800000,0.200000}%
\pgfsetstrokecolor{currentstroke}%
\pgfsetdash{}{0pt}%
\pgfpathmoveto{\pgfqpoint{3.805076in}{2.396417in}}%
\pgfpathcurveto{\pgfqpoint{3.810900in}{2.396417in}}{\pgfqpoint{3.816486in}{2.398730in}}{\pgfqpoint{3.820604in}{2.402849in}}%
\pgfpathcurveto{\pgfqpoint{3.824723in}{2.406967in}}{\pgfqpoint{3.827036in}{2.412553in}}{\pgfqpoint{3.827036in}{2.418377in}}%
\pgfpathcurveto{\pgfqpoint{3.827036in}{2.424201in}}{\pgfqpoint{3.824723in}{2.429787in}}{\pgfqpoint{3.820604in}{2.433905in}}%
\pgfpathcurveto{\pgfqpoint{3.816486in}{2.438023in}}{\pgfqpoint{3.810900in}{2.440337in}}{\pgfqpoint{3.805076in}{2.440337in}}%
\pgfpathcurveto{\pgfqpoint{3.799252in}{2.440337in}}{\pgfqpoint{3.793666in}{2.438023in}}{\pgfqpoint{3.789548in}{2.433905in}}%
\pgfpathcurveto{\pgfqpoint{3.785430in}{2.429787in}}{\pgfqpoint{3.783116in}{2.424201in}}{\pgfqpoint{3.783116in}{2.418377in}}%
\pgfpathcurveto{\pgfqpoint{3.783116in}{2.412553in}}{\pgfqpoint{3.785430in}{2.406967in}}{\pgfqpoint{3.789548in}{2.402849in}}%
\pgfpathcurveto{\pgfqpoint{3.793666in}{2.398730in}}{\pgfqpoint{3.799252in}{2.396417in}}{\pgfqpoint{3.805076in}{2.396417in}}%
\pgfpathlineto{\pgfqpoint{3.805076in}{2.396417in}}%
\pgfpathclose%
\pgfusepath{stroke,fill}%
\end{pgfscope}%
\begin{pgfscope}%
\pgfpathrectangle{\pgfqpoint{0.997489in}{0.528000in}}{\pgfqpoint{4.565023in}{3.696000in}}%
\pgfusepath{clip}%
\pgfsetbuttcap%
\pgfsetroundjoin%
\definecolor{currentfill}{rgb}{0.800000,0.800000,0.200000}%
\pgfsetfillcolor{currentfill}%
\pgfsetlinewidth{1.003750pt}%
\definecolor{currentstroke}{rgb}{0.800000,0.800000,0.200000}%
\pgfsetstrokecolor{currentstroke}%
\pgfsetdash{}{0pt}%
\pgfpathmoveto{\pgfqpoint{3.805554in}{2.435644in}}%
\pgfpathcurveto{\pgfqpoint{3.811378in}{2.435644in}}{\pgfqpoint{3.816964in}{2.437958in}}{\pgfqpoint{3.821082in}{2.442076in}}%
\pgfpathcurveto{\pgfqpoint{3.825200in}{2.446194in}}{\pgfqpoint{3.827514in}{2.451780in}}{\pgfqpoint{3.827514in}{2.457604in}}%
\pgfpathcurveto{\pgfqpoint{3.827514in}{2.463428in}}{\pgfqpoint{3.825200in}{2.469014in}}{\pgfqpoint{3.821082in}{2.473133in}}%
\pgfpathcurveto{\pgfqpoint{3.816964in}{2.477251in}}{\pgfqpoint{3.811378in}{2.479565in}}{\pgfqpoint{3.805554in}{2.479565in}}%
\pgfpathcurveto{\pgfqpoint{3.799730in}{2.479565in}}{\pgfqpoint{3.794144in}{2.477251in}}{\pgfqpoint{3.790026in}{2.473133in}}%
\pgfpathcurveto{\pgfqpoint{3.785908in}{2.469014in}}{\pgfqpoint{3.783594in}{2.463428in}}{\pgfqpoint{3.783594in}{2.457604in}}%
\pgfpathcurveto{\pgfqpoint{3.783594in}{2.451780in}}{\pgfqpoint{3.785908in}{2.446194in}}{\pgfqpoint{3.790026in}{2.442076in}}%
\pgfpathcurveto{\pgfqpoint{3.794144in}{2.437958in}}{\pgfqpoint{3.799730in}{2.435644in}}{\pgfqpoint{3.805554in}{2.435644in}}%
\pgfpathlineto{\pgfqpoint{3.805554in}{2.435644in}}%
\pgfpathclose%
\pgfusepath{stroke,fill}%
\end{pgfscope}%
\begin{pgfscope}%
\pgfpathrectangle{\pgfqpoint{0.997489in}{0.528000in}}{\pgfqpoint{4.565023in}{3.696000in}}%
\pgfusepath{clip}%
\pgfsetbuttcap%
\pgfsetroundjoin%
\definecolor{currentfill}{rgb}{0.200000,0.200000,0.800000}%
\pgfsetfillcolor{currentfill}%
\pgfsetlinewidth{1.003750pt}%
\definecolor{currentstroke}{rgb}{0.200000,0.200000,0.800000}%
\pgfsetstrokecolor{currentstroke}%
\pgfsetdash{}{0pt}%
\pgfpathmoveto{\pgfqpoint{3.670665in}{2.491258in}}%
\pgfpathcurveto{\pgfqpoint{3.676488in}{2.491258in}}{\pgfqpoint{3.682075in}{2.493571in}}{\pgfqpoint{3.686193in}{2.497690in}}%
\pgfpathcurveto{\pgfqpoint{3.690311in}{2.501808in}}{\pgfqpoint{3.692625in}{2.507394in}}{\pgfqpoint{3.692625in}{2.513218in}}%
\pgfpathcurveto{\pgfqpoint{3.692625in}{2.519042in}}{\pgfqpoint{3.690311in}{2.524628in}}{\pgfqpoint{3.686193in}{2.528746in}}%
\pgfpathcurveto{\pgfqpoint{3.682075in}{2.532864in}}{\pgfqpoint{3.676488in}{2.535178in}}{\pgfqpoint{3.670665in}{2.535178in}}%
\pgfpathcurveto{\pgfqpoint{3.664841in}{2.535178in}}{\pgfqpoint{3.659254in}{2.532864in}}{\pgfqpoint{3.655136in}{2.528746in}}%
\pgfpathcurveto{\pgfqpoint{3.651018in}{2.524628in}}{\pgfqpoint{3.648704in}{2.519042in}}{\pgfqpoint{3.648704in}{2.513218in}}%
\pgfpathcurveto{\pgfqpoint{3.648704in}{2.507394in}}{\pgfqpoint{3.651018in}{2.501808in}}{\pgfqpoint{3.655136in}{2.497690in}}%
\pgfpathcurveto{\pgfqpoint{3.659254in}{2.493571in}}{\pgfqpoint{3.664841in}{2.491258in}}{\pgfqpoint{3.670665in}{2.491258in}}%
\pgfpathlineto{\pgfqpoint{3.670665in}{2.491258in}}%
\pgfpathclose%
\pgfusepath{stroke,fill}%
\end{pgfscope}%
\begin{pgfscope}%
\pgfpathrectangle{\pgfqpoint{0.997489in}{0.528000in}}{\pgfqpoint{4.565023in}{3.696000in}}%
\pgfusepath{clip}%
\pgfsetbuttcap%
\pgfsetroundjoin%
\definecolor{currentfill}{rgb}{0.200000,0.200000,0.800000}%
\pgfsetfillcolor{currentfill}%
\pgfsetlinewidth{1.003750pt}%
\definecolor{currentstroke}{rgb}{0.200000,0.200000,0.800000}%
\pgfsetstrokecolor{currentstroke}%
\pgfsetdash{}{0pt}%
\pgfpathmoveto{\pgfqpoint{3.649297in}{2.521755in}}%
\pgfpathcurveto{\pgfqpoint{3.655121in}{2.521755in}}{\pgfqpoint{3.660708in}{2.524069in}}{\pgfqpoint{3.664826in}{2.528187in}}%
\pgfpathcurveto{\pgfqpoint{3.668944in}{2.532305in}}{\pgfqpoint{3.671258in}{2.537891in}}{\pgfqpoint{3.671258in}{2.543715in}}%
\pgfpathcurveto{\pgfqpoint{3.671258in}{2.549539in}}{\pgfqpoint{3.668944in}{2.555125in}}{\pgfqpoint{3.664826in}{2.559243in}}%
\pgfpathcurveto{\pgfqpoint{3.660708in}{2.563361in}}{\pgfqpoint{3.655121in}{2.565675in}}{\pgfqpoint{3.649297in}{2.565675in}}%
\pgfpathcurveto{\pgfqpoint{3.643473in}{2.565675in}}{\pgfqpoint{3.637887in}{2.563361in}}{\pgfqpoint{3.633769in}{2.559243in}}%
\pgfpathcurveto{\pgfqpoint{3.629651in}{2.555125in}}{\pgfqpoint{3.627337in}{2.549539in}}{\pgfqpoint{3.627337in}{2.543715in}}%
\pgfpathcurveto{\pgfqpoint{3.627337in}{2.537891in}}{\pgfqpoint{3.629651in}{2.532305in}}{\pgfqpoint{3.633769in}{2.528187in}}%
\pgfpathcurveto{\pgfqpoint{3.637887in}{2.524069in}}{\pgfqpoint{3.643473in}{2.521755in}}{\pgfqpoint{3.649297in}{2.521755in}}%
\pgfpathlineto{\pgfqpoint{3.649297in}{2.521755in}}%
\pgfpathclose%
\pgfusepath{stroke,fill}%
\end{pgfscope}%
\begin{pgfscope}%
\pgfpathrectangle{\pgfqpoint{0.997489in}{0.528000in}}{\pgfqpoint{4.565023in}{3.696000in}}%
\pgfusepath{clip}%
\pgfsetbuttcap%
\pgfsetroundjoin%
\definecolor{currentfill}{rgb}{0.800000,0.800000,0.200000}%
\pgfsetfillcolor{currentfill}%
\pgfsetlinewidth{1.003750pt}%
\definecolor{currentstroke}{rgb}{0.800000,0.800000,0.200000}%
\pgfsetstrokecolor{currentstroke}%
\pgfsetdash{}{0pt}%
\pgfpathmoveto{\pgfqpoint{3.740330in}{2.549302in}}%
\pgfpathcurveto{\pgfqpoint{3.746154in}{2.549302in}}{\pgfqpoint{3.751740in}{2.551615in}}{\pgfqpoint{3.755859in}{2.555734in}}%
\pgfpathcurveto{\pgfqpoint{3.759977in}{2.559852in}}{\pgfqpoint{3.762291in}{2.565438in}}{\pgfqpoint{3.762291in}{2.571262in}}%
\pgfpathcurveto{\pgfqpoint{3.762291in}{2.577086in}}{\pgfqpoint{3.759977in}{2.582672in}}{\pgfqpoint{3.755859in}{2.586790in}}%
\pgfpathcurveto{\pgfqpoint{3.751740in}{2.590908in}}{\pgfqpoint{3.746154in}{2.593222in}}{\pgfqpoint{3.740330in}{2.593222in}}%
\pgfpathcurveto{\pgfqpoint{3.734506in}{2.593222in}}{\pgfqpoint{3.728920in}{2.590908in}}{\pgfqpoint{3.724802in}{2.586790in}}%
\pgfpathcurveto{\pgfqpoint{3.720684in}{2.582672in}}{\pgfqpoint{3.718370in}{2.577086in}}{\pgfqpoint{3.718370in}{2.571262in}}%
\pgfpathcurveto{\pgfqpoint{3.718370in}{2.565438in}}{\pgfqpoint{3.720684in}{2.559852in}}{\pgfqpoint{3.724802in}{2.555734in}}%
\pgfpathcurveto{\pgfqpoint{3.728920in}{2.551615in}}{\pgfqpoint{3.734506in}{2.549302in}}{\pgfqpoint{3.740330in}{2.549302in}}%
\pgfpathlineto{\pgfqpoint{3.740330in}{2.549302in}}%
\pgfpathclose%
\pgfusepath{stroke,fill}%
\end{pgfscope}%
\begin{pgfscope}%
\pgfpathrectangle{\pgfqpoint{0.997489in}{0.528000in}}{\pgfqpoint{4.565023in}{3.696000in}}%
\pgfusepath{clip}%
\pgfsetbuttcap%
\pgfsetroundjoin%
\definecolor{currentfill}{rgb}{0.200000,0.800000,0.200000}%
\pgfsetfillcolor{currentfill}%
\pgfsetlinewidth{1.003750pt}%
\definecolor{currentstroke}{rgb}{0.200000,0.800000,0.200000}%
\pgfsetstrokecolor{currentstroke}%
\pgfsetdash{}{0pt}%
\pgfpathmoveto{\pgfqpoint{3.038518in}{2.947196in}}%
\pgfpathcurveto{\pgfqpoint{3.044341in}{2.947196in}}{\pgfqpoint{3.049928in}{2.949510in}}{\pgfqpoint{3.054046in}{2.953628in}}%
\pgfpathcurveto{\pgfqpoint{3.058164in}{2.957746in}}{\pgfqpoint{3.060478in}{2.963332in}}{\pgfqpoint{3.060478in}{2.969156in}}%
\pgfpathcurveto{\pgfqpoint{3.060478in}{2.974980in}}{\pgfqpoint{3.058164in}{2.980566in}}{\pgfqpoint{3.054046in}{2.984684in}}%
\pgfpathcurveto{\pgfqpoint{3.049928in}{2.988802in}}{\pgfqpoint{3.044341in}{2.991116in}}{\pgfqpoint{3.038518in}{2.991116in}}%
\pgfpathcurveto{\pgfqpoint{3.032694in}{2.991116in}}{\pgfqpoint{3.027107in}{2.988802in}}{\pgfqpoint{3.022989in}{2.984684in}}%
\pgfpathcurveto{\pgfqpoint{3.018871in}{2.980566in}}{\pgfqpoint{3.016557in}{2.974980in}}{\pgfqpoint{3.016557in}{2.969156in}}%
\pgfpathcurveto{\pgfqpoint{3.016557in}{2.963332in}}{\pgfqpoint{3.018871in}{2.957746in}}{\pgfqpoint{3.022989in}{2.953628in}}%
\pgfpathcurveto{\pgfqpoint{3.027107in}{2.949510in}}{\pgfqpoint{3.032694in}{2.947196in}}{\pgfqpoint{3.038518in}{2.947196in}}%
\pgfpathlineto{\pgfqpoint{3.038518in}{2.947196in}}%
\pgfpathclose%
\pgfusepath{stroke,fill}%
\end{pgfscope}%
\begin{pgfscope}%
\pgfpathrectangle{\pgfqpoint{0.997489in}{0.528000in}}{\pgfqpoint{4.565023in}{3.696000in}}%
\pgfusepath{clip}%
\pgfsetbuttcap%
\pgfsetroundjoin%
\definecolor{currentfill}{rgb}{0.200000,0.200000,0.800000}%
\pgfsetfillcolor{currentfill}%
\pgfsetlinewidth{1.003750pt}%
\definecolor{currentstroke}{rgb}{0.200000,0.200000,0.800000}%
\pgfsetstrokecolor{currentstroke}%
\pgfsetdash{}{0pt}%
\pgfpathmoveto{\pgfqpoint{3.048173in}{3.004923in}}%
\pgfpathcurveto{\pgfqpoint{3.053997in}{3.004923in}}{\pgfqpoint{3.059583in}{3.007237in}}{\pgfqpoint{3.063701in}{3.011355in}}%
\pgfpathcurveto{\pgfqpoint{3.067819in}{3.015473in}}{\pgfqpoint{3.070133in}{3.021059in}}{\pgfqpoint{3.070133in}{3.026883in}}%
\pgfpathcurveto{\pgfqpoint{3.070133in}{3.032707in}}{\pgfqpoint{3.067819in}{3.038293in}}{\pgfqpoint{3.063701in}{3.042411in}}%
\pgfpathcurveto{\pgfqpoint{3.059583in}{3.046529in}}{\pgfqpoint{3.053997in}{3.048843in}}{\pgfqpoint{3.048173in}{3.048843in}}%
\pgfpathcurveto{\pgfqpoint{3.042349in}{3.048843in}}{\pgfqpoint{3.036762in}{3.046529in}}{\pgfqpoint{3.032644in}{3.042411in}}%
\pgfpathcurveto{\pgfqpoint{3.028526in}{3.038293in}}{\pgfqpoint{3.026212in}{3.032707in}}{\pgfqpoint{3.026212in}{3.026883in}}%
\pgfpathcurveto{\pgfqpoint{3.026212in}{3.021059in}}{\pgfqpoint{3.028526in}{3.015473in}}{\pgfqpoint{3.032644in}{3.011355in}}%
\pgfpathcurveto{\pgfqpoint{3.036762in}{3.007237in}}{\pgfqpoint{3.042349in}{3.004923in}}{\pgfqpoint{3.048173in}{3.004923in}}%
\pgfpathlineto{\pgfqpoint{3.048173in}{3.004923in}}%
\pgfpathclose%
\pgfusepath{stroke,fill}%
\end{pgfscope}%
\begin{pgfscope}%
\pgfpathrectangle{\pgfqpoint{0.997489in}{0.528000in}}{\pgfqpoint{4.565023in}{3.696000in}}%
\pgfusepath{clip}%
\pgfsetbuttcap%
\pgfsetroundjoin%
\definecolor{currentfill}{rgb}{0.200000,0.800000,0.200000}%
\pgfsetfillcolor{currentfill}%
\pgfsetlinewidth{1.003750pt}%
\definecolor{currentstroke}{rgb}{0.200000,0.800000,0.200000}%
\pgfsetstrokecolor{currentstroke}%
\pgfsetdash{}{0pt}%
\pgfpathmoveto{\pgfqpoint{2.948183in}{3.050357in}}%
\pgfpathcurveto{\pgfqpoint{2.954007in}{3.050357in}}{\pgfqpoint{2.959593in}{3.052671in}}{\pgfqpoint{2.963711in}{3.056789in}}%
\pgfpathcurveto{\pgfqpoint{2.967829in}{3.060908in}}{\pgfqpoint{2.970143in}{3.066494in}}{\pgfqpoint{2.970143in}{3.072318in}}%
\pgfpathcurveto{\pgfqpoint{2.970143in}{3.078142in}}{\pgfqpoint{2.967829in}{3.083728in}}{\pgfqpoint{2.963711in}{3.087846in}}%
\pgfpathcurveto{\pgfqpoint{2.959593in}{3.091964in}}{\pgfqpoint{2.954007in}{3.094278in}}{\pgfqpoint{2.948183in}{3.094278in}}%
\pgfpathcurveto{\pgfqpoint{2.942359in}{3.094278in}}{\pgfqpoint{2.936773in}{3.091964in}}{\pgfqpoint{2.932655in}{3.087846in}}%
\pgfpathcurveto{\pgfqpoint{2.928536in}{3.083728in}}{\pgfqpoint{2.926223in}{3.078142in}}{\pgfqpoint{2.926223in}{3.072318in}}%
\pgfpathcurveto{\pgfqpoint{2.926223in}{3.066494in}}{\pgfqpoint{2.928536in}{3.060908in}}{\pgfqpoint{2.932655in}{3.056789in}}%
\pgfpathcurveto{\pgfqpoint{2.936773in}{3.052671in}}{\pgfqpoint{2.942359in}{3.050357in}}{\pgfqpoint{2.948183in}{3.050357in}}%
\pgfpathlineto{\pgfqpoint{2.948183in}{3.050357in}}%
\pgfpathclose%
\pgfusepath{stroke,fill}%
\end{pgfscope}%
\begin{pgfscope}%
\pgfpathrectangle{\pgfqpoint{0.997489in}{0.528000in}}{\pgfqpoint{4.565023in}{3.696000in}}%
\pgfusepath{clip}%
\pgfsetbuttcap%
\pgfsetroundjoin%
\definecolor{currentfill}{rgb}{0.200000,0.800000,0.200000}%
\pgfsetfillcolor{currentfill}%
\pgfsetlinewidth{1.003750pt}%
\definecolor{currentstroke}{rgb}{0.200000,0.800000,0.200000}%
\pgfsetstrokecolor{currentstroke}%
\pgfsetdash{}{0pt}%
\pgfpathmoveto{\pgfqpoint{2.941254in}{3.101658in}}%
\pgfpathcurveto{\pgfqpoint{2.947078in}{3.101658in}}{\pgfqpoint{2.952664in}{3.103972in}}{\pgfqpoint{2.956782in}{3.108090in}}%
\pgfpathcurveto{\pgfqpoint{2.960901in}{3.112208in}}{\pgfqpoint{2.963214in}{3.117794in}}{\pgfqpoint{2.963214in}{3.123618in}}%
\pgfpathcurveto{\pgfqpoint{2.963214in}{3.129442in}}{\pgfqpoint{2.960901in}{3.135028in}}{\pgfqpoint{2.956782in}{3.139146in}}%
\pgfpathcurveto{\pgfqpoint{2.952664in}{3.143265in}}{\pgfqpoint{2.947078in}{3.145579in}}{\pgfqpoint{2.941254in}{3.145579in}}%
\pgfpathcurveto{\pgfqpoint{2.935430in}{3.145579in}}{\pgfqpoint{2.929844in}{3.143265in}}{\pgfqpoint{2.925726in}{3.139146in}}%
\pgfpathcurveto{\pgfqpoint{2.921608in}{3.135028in}}{\pgfqpoint{2.919294in}{3.129442in}}{\pgfqpoint{2.919294in}{3.123618in}}%
\pgfpathcurveto{\pgfqpoint{2.919294in}{3.117794in}}{\pgfqpoint{2.921608in}{3.112208in}}{\pgfqpoint{2.925726in}{3.108090in}}%
\pgfpathcurveto{\pgfqpoint{2.929844in}{3.103972in}}{\pgfqpoint{2.935430in}{3.101658in}}{\pgfqpoint{2.941254in}{3.101658in}}%
\pgfpathlineto{\pgfqpoint{2.941254in}{3.101658in}}%
\pgfpathclose%
\pgfusepath{stroke,fill}%
\end{pgfscope}%
\begin{pgfscope}%
\pgfpathrectangle{\pgfqpoint{0.997489in}{0.528000in}}{\pgfqpoint{4.565023in}{3.696000in}}%
\pgfusepath{clip}%
\pgfsetbuttcap%
\pgfsetroundjoin%
\definecolor{currentfill}{rgb}{0.200000,0.800000,0.200000}%
\pgfsetfillcolor{currentfill}%
\pgfsetlinewidth{1.003750pt}%
\definecolor{currentstroke}{rgb}{0.200000,0.800000,0.200000}%
\pgfsetstrokecolor{currentstroke}%
\pgfsetdash{}{0pt}%
\pgfpathmoveto{\pgfqpoint{2.977993in}{3.164669in}}%
\pgfpathcurveto{\pgfqpoint{2.983817in}{3.164669in}}{\pgfqpoint{2.989403in}{3.166983in}}{\pgfqpoint{2.993521in}{3.171101in}}%
\pgfpathcurveto{\pgfqpoint{2.997639in}{3.175220in}}{\pgfqpoint{2.999953in}{3.180806in}}{\pgfqpoint{2.999953in}{3.186630in}}%
\pgfpathcurveto{\pgfqpoint{2.999953in}{3.192454in}}{\pgfqpoint{2.997639in}{3.198040in}}{\pgfqpoint{2.993521in}{3.202158in}}%
\pgfpathcurveto{\pgfqpoint{2.989403in}{3.206276in}}{\pgfqpoint{2.983817in}{3.208590in}}{\pgfqpoint{2.977993in}{3.208590in}}%
\pgfpathcurveto{\pgfqpoint{2.972169in}{3.208590in}}{\pgfqpoint{2.966583in}{3.206276in}}{\pgfqpoint{2.962465in}{3.202158in}}%
\pgfpathcurveto{\pgfqpoint{2.958346in}{3.198040in}}{\pgfqpoint{2.956033in}{3.192454in}}{\pgfqpoint{2.956033in}{3.186630in}}%
\pgfpathcurveto{\pgfqpoint{2.956033in}{3.180806in}}{\pgfqpoint{2.958346in}{3.175220in}}{\pgfqpoint{2.962465in}{3.171101in}}%
\pgfpathcurveto{\pgfqpoint{2.966583in}{3.166983in}}{\pgfqpoint{2.972169in}{3.164669in}}{\pgfqpoint{2.977993in}{3.164669in}}%
\pgfpathlineto{\pgfqpoint{2.977993in}{3.164669in}}%
\pgfpathclose%
\pgfusepath{stroke,fill}%
\end{pgfscope}%
\begin{pgfscope}%
\pgfpathrectangle{\pgfqpoint{0.997489in}{0.528000in}}{\pgfqpoint{4.565023in}{3.696000in}}%
\pgfusepath{clip}%
\pgfsetbuttcap%
\pgfsetroundjoin%
\definecolor{currentfill}{rgb}{0.800000,0.200000,0.200000}%
\pgfsetfillcolor{currentfill}%
\pgfsetlinewidth{1.003750pt}%
\definecolor{currentstroke}{rgb}{0.800000,0.200000,0.200000}%
\pgfsetstrokecolor{currentstroke}%
\pgfsetdash{}{0pt}%
\pgfpathmoveto{\pgfqpoint{3.030842in}{3.239833in}}%
\pgfpathcurveto{\pgfqpoint{3.036666in}{3.239833in}}{\pgfqpoint{3.042252in}{3.242147in}}{\pgfqpoint{3.046371in}{3.246265in}}%
\pgfpathcurveto{\pgfqpoint{3.050489in}{3.250383in}}{\pgfqpoint{3.052803in}{3.255969in}}{\pgfqpoint{3.052803in}{3.261793in}}%
\pgfpathcurveto{\pgfqpoint{3.052803in}{3.267617in}}{\pgfqpoint{3.050489in}{3.273203in}}{\pgfqpoint{3.046371in}{3.277322in}}%
\pgfpathcurveto{\pgfqpoint{3.042252in}{3.281440in}}{\pgfqpoint{3.036666in}{3.283754in}}{\pgfqpoint{3.030842in}{3.283754in}}%
\pgfpathcurveto{\pgfqpoint{3.025018in}{3.283754in}}{\pgfqpoint{3.019432in}{3.281440in}}{\pgfqpoint{3.015314in}{3.277322in}}%
\pgfpathcurveto{\pgfqpoint{3.011196in}{3.273203in}}{\pgfqpoint{3.008882in}{3.267617in}}{\pgfqpoint{3.008882in}{3.261793in}}%
\pgfpathcurveto{\pgfqpoint{3.008882in}{3.255969in}}{\pgfqpoint{3.011196in}{3.250383in}}{\pgfqpoint{3.015314in}{3.246265in}}%
\pgfpathcurveto{\pgfqpoint{3.019432in}{3.242147in}}{\pgfqpoint{3.025018in}{3.239833in}}{\pgfqpoint{3.030842in}{3.239833in}}%
\pgfpathlineto{\pgfqpoint{3.030842in}{3.239833in}}%
\pgfpathclose%
\pgfusepath{stroke,fill}%
\end{pgfscope}%
\begin{pgfscope}%
\pgfpathrectangle{\pgfqpoint{0.997489in}{0.528000in}}{\pgfqpoint{4.565023in}{3.696000in}}%
\pgfusepath{clip}%
\pgfsetbuttcap%
\pgfsetroundjoin%
\definecolor{currentfill}{rgb}{0.200000,0.800000,0.200000}%
\pgfsetfillcolor{currentfill}%
\pgfsetlinewidth{1.003750pt}%
\definecolor{currentstroke}{rgb}{0.200000,0.800000,0.200000}%
\pgfsetstrokecolor{currentstroke}%
\pgfsetdash{}{0pt}%
\pgfpathmoveto{\pgfqpoint{2.909789in}{3.255441in}}%
\pgfpathcurveto{\pgfqpoint{2.915613in}{3.255441in}}{\pgfqpoint{2.921199in}{3.257755in}}{\pgfqpoint{2.925317in}{3.261873in}}%
\pgfpathcurveto{\pgfqpoint{2.929435in}{3.265992in}}{\pgfqpoint{2.931749in}{3.271578in}}{\pgfqpoint{2.931749in}{3.277402in}}%
\pgfpathcurveto{\pgfqpoint{2.931749in}{3.283226in}}{\pgfqpoint{2.929435in}{3.288812in}}{\pgfqpoint{2.925317in}{3.292930in}}%
\pgfpathcurveto{\pgfqpoint{2.921199in}{3.297048in}}{\pgfqpoint{2.915613in}{3.299362in}}{\pgfqpoint{2.909789in}{3.299362in}}%
\pgfpathcurveto{\pgfqpoint{2.903965in}{3.299362in}}{\pgfqpoint{2.898379in}{3.297048in}}{\pgfqpoint{2.894260in}{3.292930in}}%
\pgfpathcurveto{\pgfqpoint{2.890142in}{3.288812in}}{\pgfqpoint{2.887828in}{3.283226in}}{\pgfqpoint{2.887828in}{3.277402in}}%
\pgfpathcurveto{\pgfqpoint{2.887828in}{3.271578in}}{\pgfqpoint{2.890142in}{3.265992in}}{\pgfqpoint{2.894260in}{3.261873in}}%
\pgfpathcurveto{\pgfqpoint{2.898379in}{3.257755in}}{\pgfqpoint{2.903965in}{3.255441in}}{\pgfqpoint{2.909789in}{3.255441in}}%
\pgfpathlineto{\pgfqpoint{2.909789in}{3.255441in}}%
\pgfpathclose%
\pgfusepath{stroke,fill}%
\end{pgfscope}%
\begin{pgfscope}%
\pgfpathrectangle{\pgfqpoint{0.997489in}{0.528000in}}{\pgfqpoint{4.565023in}{3.696000in}}%
\pgfusepath{clip}%
\pgfsetbuttcap%
\pgfsetroundjoin%
\definecolor{currentfill}{rgb}{0.200000,0.800000,0.200000}%
\pgfsetfillcolor{currentfill}%
\pgfsetlinewidth{1.003750pt}%
\definecolor{currentstroke}{rgb}{0.200000,0.800000,0.200000}%
\pgfsetstrokecolor{currentstroke}%
\pgfsetdash{}{0pt}%
\pgfpathmoveto{\pgfqpoint{2.897542in}{3.307869in}}%
\pgfpathcurveto{\pgfqpoint{2.903366in}{3.307869in}}{\pgfqpoint{2.908952in}{3.310183in}}{\pgfqpoint{2.913070in}{3.314301in}}%
\pgfpathcurveto{\pgfqpoint{2.917189in}{3.318419in}}{\pgfqpoint{2.919502in}{3.324006in}}{\pgfqpoint{2.919502in}{3.329829in}}%
\pgfpathcurveto{\pgfqpoint{2.919502in}{3.335653in}}{\pgfqpoint{2.917189in}{3.341240in}}{\pgfqpoint{2.913070in}{3.345358in}}%
\pgfpathcurveto{\pgfqpoint{2.908952in}{3.349476in}}{\pgfqpoint{2.903366in}{3.351790in}}{\pgfqpoint{2.897542in}{3.351790in}}%
\pgfpathcurveto{\pgfqpoint{2.891718in}{3.351790in}}{\pgfqpoint{2.886132in}{3.349476in}}{\pgfqpoint{2.882014in}{3.345358in}}%
\pgfpathcurveto{\pgfqpoint{2.877896in}{3.341240in}}{\pgfqpoint{2.875582in}{3.335653in}}{\pgfqpoint{2.875582in}{3.329829in}}%
\pgfpathcurveto{\pgfqpoint{2.875582in}{3.324006in}}{\pgfqpoint{2.877896in}{3.318419in}}{\pgfqpoint{2.882014in}{3.314301in}}%
\pgfpathcurveto{\pgfqpoint{2.886132in}{3.310183in}}{\pgfqpoint{2.891718in}{3.307869in}}{\pgfqpoint{2.897542in}{3.307869in}}%
\pgfpathlineto{\pgfqpoint{2.897542in}{3.307869in}}%
\pgfpathclose%
\pgfusepath{stroke,fill}%
\end{pgfscope}%
\begin{pgfscope}%
\pgfpathrectangle{\pgfqpoint{0.997489in}{0.528000in}}{\pgfqpoint{4.565023in}{3.696000in}}%
\pgfusepath{clip}%
\pgfsetbuttcap%
\pgfsetroundjoin%
\definecolor{currentfill}{rgb}{0.200000,0.800000,0.200000}%
\pgfsetfillcolor{currentfill}%
\pgfsetlinewidth{1.003750pt}%
\definecolor{currentstroke}{rgb}{0.200000,0.800000,0.200000}%
\pgfsetstrokecolor{currentstroke}%
\pgfsetdash{}{0pt}%
\pgfpathmoveto{\pgfqpoint{2.905889in}{3.373420in}}%
\pgfpathcurveto{\pgfqpoint{2.911713in}{3.373420in}}{\pgfqpoint{2.917299in}{3.375734in}}{\pgfqpoint{2.921417in}{3.379852in}}%
\pgfpathcurveto{\pgfqpoint{2.925535in}{3.383971in}}{\pgfqpoint{2.927849in}{3.389557in}}{\pgfqpoint{2.927849in}{3.395381in}}%
\pgfpathcurveto{\pgfqpoint{2.927849in}{3.401205in}}{\pgfqpoint{2.925535in}{3.406791in}}{\pgfqpoint{2.921417in}{3.410909in}}%
\pgfpathcurveto{\pgfqpoint{2.917299in}{3.415027in}}{\pgfqpoint{2.911713in}{3.417341in}}{\pgfqpoint{2.905889in}{3.417341in}}%
\pgfpathcurveto{\pgfqpoint{2.900065in}{3.417341in}}{\pgfqpoint{2.894479in}{3.415027in}}{\pgfqpoint{2.890360in}{3.410909in}}%
\pgfpathcurveto{\pgfqpoint{2.886242in}{3.406791in}}{\pgfqpoint{2.883928in}{3.401205in}}{\pgfqpoint{2.883928in}{3.395381in}}%
\pgfpathcurveto{\pgfqpoint{2.883928in}{3.389557in}}{\pgfqpoint{2.886242in}{3.383971in}}{\pgfqpoint{2.890360in}{3.379852in}}%
\pgfpathcurveto{\pgfqpoint{2.894479in}{3.375734in}}{\pgfqpoint{2.900065in}{3.373420in}}{\pgfqpoint{2.905889in}{3.373420in}}%
\pgfpathlineto{\pgfqpoint{2.905889in}{3.373420in}}%
\pgfpathclose%
\pgfusepath{stroke,fill}%
\end{pgfscope}%
\begin{pgfscope}%
\pgfpathrectangle{\pgfqpoint{0.997489in}{0.528000in}}{\pgfqpoint{4.565023in}{3.696000in}}%
\pgfusepath{clip}%
\pgfsetbuttcap%
\pgfsetroundjoin%
\definecolor{currentfill}{rgb}{0.200000,0.800000,0.200000}%
\pgfsetfillcolor{currentfill}%
\pgfsetlinewidth{1.003750pt}%
\definecolor{currentstroke}{rgb}{0.200000,0.800000,0.200000}%
\pgfsetstrokecolor{currentstroke}%
\pgfsetdash{}{0pt}%
\pgfpathmoveto{\pgfqpoint{2.891926in}{3.430540in}}%
\pgfpathcurveto{\pgfqpoint{2.897750in}{3.430540in}}{\pgfqpoint{2.903336in}{3.432854in}}{\pgfqpoint{2.907454in}{3.436972in}}%
\pgfpathcurveto{\pgfqpoint{2.911573in}{3.441090in}}{\pgfqpoint{2.913886in}{3.446676in}}{\pgfqpoint{2.913886in}{3.452500in}}%
\pgfpathcurveto{\pgfqpoint{2.913886in}{3.458324in}}{\pgfqpoint{2.911573in}{3.463910in}}{\pgfqpoint{2.907454in}{3.468028in}}%
\pgfpathcurveto{\pgfqpoint{2.903336in}{3.472146in}}{\pgfqpoint{2.897750in}{3.474460in}}{\pgfqpoint{2.891926in}{3.474460in}}%
\pgfpathcurveto{\pgfqpoint{2.886102in}{3.474460in}}{\pgfqpoint{2.880516in}{3.472146in}}{\pgfqpoint{2.876398in}{3.468028in}}%
\pgfpathcurveto{\pgfqpoint{2.872280in}{3.463910in}}{\pgfqpoint{2.869966in}{3.458324in}}{\pgfqpoint{2.869966in}{3.452500in}}%
\pgfpathcurveto{\pgfqpoint{2.869966in}{3.446676in}}{\pgfqpoint{2.872280in}{3.441090in}}{\pgfqpoint{2.876398in}{3.436972in}}%
\pgfpathcurveto{\pgfqpoint{2.880516in}{3.432854in}}{\pgfqpoint{2.886102in}{3.430540in}}{\pgfqpoint{2.891926in}{3.430540in}}%
\pgfpathlineto{\pgfqpoint{2.891926in}{3.430540in}}%
\pgfpathclose%
\pgfusepath{stroke,fill}%
\end{pgfscope}%
\begin{pgfscope}%
\pgfpathrectangle{\pgfqpoint{0.997489in}{0.528000in}}{\pgfqpoint{4.565023in}{3.696000in}}%
\pgfusepath{clip}%
\pgfsetbuttcap%
\pgfsetroundjoin%
\definecolor{currentfill}{rgb}{0.200000,0.800000,0.200000}%
\pgfsetfillcolor{currentfill}%
\pgfsetlinewidth{1.003750pt}%
\definecolor{currentstroke}{rgb}{0.200000,0.800000,0.200000}%
\pgfsetstrokecolor{currentstroke}%
\pgfsetdash{}{0pt}%
\pgfpathmoveto{\pgfqpoint{2.854348in}{3.473285in}}%
\pgfpathcurveto{\pgfqpoint{2.860172in}{3.473285in}}{\pgfqpoint{2.865758in}{3.475599in}}{\pgfqpoint{2.869876in}{3.479717in}}%
\pgfpathcurveto{\pgfqpoint{2.873994in}{3.483835in}}{\pgfqpoint{2.876308in}{3.489422in}}{\pgfqpoint{2.876308in}{3.495245in}}%
\pgfpathcurveto{\pgfqpoint{2.876308in}{3.501069in}}{\pgfqpoint{2.873994in}{3.506656in}}{\pgfqpoint{2.869876in}{3.510774in}}%
\pgfpathcurveto{\pgfqpoint{2.865758in}{3.514892in}}{\pgfqpoint{2.860172in}{3.517206in}}{\pgfqpoint{2.854348in}{3.517206in}}%
\pgfpathcurveto{\pgfqpoint{2.848524in}{3.517206in}}{\pgfqpoint{2.842938in}{3.514892in}}{\pgfqpoint{2.838819in}{3.510774in}}%
\pgfpathcurveto{\pgfqpoint{2.834701in}{3.506656in}}{\pgfqpoint{2.832387in}{3.501069in}}{\pgfqpoint{2.832387in}{3.495245in}}%
\pgfpathcurveto{\pgfqpoint{2.832387in}{3.489422in}}{\pgfqpoint{2.834701in}{3.483835in}}{\pgfqpoint{2.838819in}{3.479717in}}%
\pgfpathcurveto{\pgfqpoint{2.842938in}{3.475599in}}{\pgfqpoint{2.848524in}{3.473285in}}{\pgfqpoint{2.854348in}{3.473285in}}%
\pgfpathlineto{\pgfqpoint{2.854348in}{3.473285in}}%
\pgfpathclose%
\pgfusepath{stroke,fill}%
\end{pgfscope}%
\begin{pgfscope}%
\pgfpathrectangle{\pgfqpoint{0.997489in}{0.528000in}}{\pgfqpoint{4.565023in}{3.696000in}}%
\pgfusepath{clip}%
\pgfsetbuttcap%
\pgfsetroundjoin%
\definecolor{currentfill}{rgb}{0.200000,0.800000,0.200000}%
\pgfsetfillcolor{currentfill}%
\pgfsetlinewidth{1.003750pt}%
\definecolor{currentstroke}{rgb}{0.200000,0.800000,0.200000}%
\pgfsetstrokecolor{currentstroke}%
\pgfsetdash{}{0pt}%
\pgfpathmoveto{\pgfqpoint{2.799935in}{3.501091in}}%
\pgfpathcurveto{\pgfqpoint{2.805759in}{3.501091in}}{\pgfqpoint{2.811345in}{3.503405in}}{\pgfqpoint{2.815464in}{3.507523in}}%
\pgfpathcurveto{\pgfqpoint{2.819582in}{3.511642in}}{\pgfqpoint{2.821896in}{3.517228in}}{\pgfqpoint{2.821896in}{3.523052in}}%
\pgfpathcurveto{\pgfqpoint{2.821896in}{3.528876in}}{\pgfqpoint{2.819582in}{3.534462in}}{\pgfqpoint{2.815464in}{3.538580in}}%
\pgfpathcurveto{\pgfqpoint{2.811345in}{3.542698in}}{\pgfqpoint{2.805759in}{3.545012in}}{\pgfqpoint{2.799935in}{3.545012in}}%
\pgfpathcurveto{\pgfqpoint{2.794111in}{3.545012in}}{\pgfqpoint{2.788525in}{3.542698in}}{\pgfqpoint{2.784407in}{3.538580in}}%
\pgfpathcurveto{\pgfqpoint{2.780289in}{3.534462in}}{\pgfqpoint{2.777975in}{3.528876in}}{\pgfqpoint{2.777975in}{3.523052in}}%
\pgfpathcurveto{\pgfqpoint{2.777975in}{3.517228in}}{\pgfqpoint{2.780289in}{3.511642in}}{\pgfqpoint{2.784407in}{3.507523in}}%
\pgfpathcurveto{\pgfqpoint{2.788525in}{3.503405in}}{\pgfqpoint{2.794111in}{3.501091in}}{\pgfqpoint{2.799935in}{3.501091in}}%
\pgfpathlineto{\pgfqpoint{2.799935in}{3.501091in}}%
\pgfpathclose%
\pgfusepath{stroke,fill}%
\end{pgfscope}%
\begin{pgfscope}%
\pgfpathrectangle{\pgfqpoint{0.997489in}{0.528000in}}{\pgfqpoint{4.565023in}{3.696000in}}%
\pgfusepath{clip}%
\pgfsetbuttcap%
\pgfsetroundjoin%
\definecolor{currentfill}{rgb}{0.200000,0.800000,0.200000}%
\pgfsetfillcolor{currentfill}%
\pgfsetlinewidth{1.003750pt}%
\definecolor{currentstroke}{rgb}{0.200000,0.800000,0.200000}%
\pgfsetstrokecolor{currentstroke}%
\pgfsetdash{}{0pt}%
\pgfpathmoveto{\pgfqpoint{2.750826in}{3.529781in}}%
\pgfpathcurveto{\pgfqpoint{2.756650in}{3.529781in}}{\pgfqpoint{2.762236in}{3.532095in}}{\pgfqpoint{2.766354in}{3.536213in}}%
\pgfpathcurveto{\pgfqpoint{2.770472in}{3.540331in}}{\pgfqpoint{2.772786in}{3.545917in}}{\pgfqpoint{2.772786in}{3.551741in}}%
\pgfpathcurveto{\pgfqpoint{2.772786in}{3.557565in}}{\pgfqpoint{2.770472in}{3.563151in}}{\pgfqpoint{2.766354in}{3.567269in}}%
\pgfpathcurveto{\pgfqpoint{2.762236in}{3.571388in}}{\pgfqpoint{2.756650in}{3.573701in}}{\pgfqpoint{2.750826in}{3.573701in}}%
\pgfpathcurveto{\pgfqpoint{2.745002in}{3.573701in}}{\pgfqpoint{2.739416in}{3.571388in}}{\pgfqpoint{2.735298in}{3.567269in}}%
\pgfpathcurveto{\pgfqpoint{2.731180in}{3.563151in}}{\pgfqpoint{2.728866in}{3.557565in}}{\pgfqpoint{2.728866in}{3.551741in}}%
\pgfpathcurveto{\pgfqpoint{2.728866in}{3.545917in}}{\pgfqpoint{2.731180in}{3.540331in}}{\pgfqpoint{2.735298in}{3.536213in}}%
\pgfpathcurveto{\pgfqpoint{2.739416in}{3.532095in}}{\pgfqpoint{2.745002in}{3.529781in}}{\pgfqpoint{2.750826in}{3.529781in}}%
\pgfpathlineto{\pgfqpoint{2.750826in}{3.529781in}}%
\pgfpathclose%
\pgfusepath{stroke,fill}%
\end{pgfscope}%
\begin{pgfscope}%
\pgfpathrectangle{\pgfqpoint{0.997489in}{0.528000in}}{\pgfqpoint{4.565023in}{3.696000in}}%
\pgfusepath{clip}%
\pgfsetbuttcap%
\pgfsetroundjoin%
\definecolor{currentfill}{rgb}{0.200000,0.800000,0.200000}%
\pgfsetfillcolor{currentfill}%
\pgfsetlinewidth{1.003750pt}%
\definecolor{currentstroke}{rgb}{0.200000,0.800000,0.200000}%
\pgfsetstrokecolor{currentstroke}%
\pgfsetdash{}{0pt}%
\pgfpathmoveto{\pgfqpoint{2.735021in}{3.591585in}}%
\pgfpathcurveto{\pgfqpoint{2.740845in}{3.591585in}}{\pgfqpoint{2.746431in}{3.593898in}}{\pgfqpoint{2.750550in}{3.598017in}}%
\pgfpathcurveto{\pgfqpoint{2.754668in}{3.602135in}}{\pgfqpoint{2.756982in}{3.607721in}}{\pgfqpoint{2.756982in}{3.613545in}}%
\pgfpathcurveto{\pgfqpoint{2.756982in}{3.619369in}}{\pgfqpoint{2.754668in}{3.624955in}}{\pgfqpoint{2.750550in}{3.629073in}}%
\pgfpathcurveto{\pgfqpoint{2.746431in}{3.633191in}}{\pgfqpoint{2.740845in}{3.635505in}}{\pgfqpoint{2.735021in}{3.635505in}}%
\pgfpathcurveto{\pgfqpoint{2.729197in}{3.635505in}}{\pgfqpoint{2.723611in}{3.633191in}}{\pgfqpoint{2.719493in}{3.629073in}}%
\pgfpathcurveto{\pgfqpoint{2.715375in}{3.624955in}}{\pgfqpoint{2.713061in}{3.619369in}}{\pgfqpoint{2.713061in}{3.613545in}}%
\pgfpathcurveto{\pgfqpoint{2.713061in}{3.607721in}}{\pgfqpoint{2.715375in}{3.602135in}}{\pgfqpoint{2.719493in}{3.598017in}}%
\pgfpathcurveto{\pgfqpoint{2.723611in}{3.593898in}}{\pgfqpoint{2.729197in}{3.591585in}}{\pgfqpoint{2.735021in}{3.591585in}}%
\pgfpathlineto{\pgfqpoint{2.735021in}{3.591585in}}%
\pgfpathclose%
\pgfusepath{stroke,fill}%
\end{pgfscope}%
\begin{pgfscope}%
\pgfpathrectangle{\pgfqpoint{0.997489in}{0.528000in}}{\pgfqpoint{4.565023in}{3.696000in}}%
\pgfusepath{clip}%
\pgfsetbuttcap%
\pgfsetroundjoin%
\definecolor{currentfill}{rgb}{0.200000,0.800000,0.200000}%
\pgfsetfillcolor{currentfill}%
\pgfsetlinewidth{1.003750pt}%
\definecolor{currentstroke}{rgb}{0.200000,0.800000,0.200000}%
\pgfsetstrokecolor{currentstroke}%
\pgfsetdash{}{0pt}%
\pgfpathmoveto{\pgfqpoint{2.577706in}{3.486179in}}%
\pgfpathcurveto{\pgfqpoint{2.583530in}{3.486179in}}{\pgfqpoint{2.589116in}{3.488493in}}{\pgfqpoint{2.593234in}{3.492611in}}%
\pgfpathcurveto{\pgfqpoint{2.597352in}{3.496729in}}{\pgfqpoint{2.599666in}{3.502315in}}{\pgfqpoint{2.599666in}{3.508139in}}%
\pgfpathcurveto{\pgfqpoint{2.599666in}{3.513963in}}{\pgfqpoint{2.597352in}{3.519549in}}{\pgfqpoint{2.593234in}{3.523667in}}%
\pgfpathcurveto{\pgfqpoint{2.589116in}{3.527786in}}{\pgfqpoint{2.583530in}{3.530099in}}{\pgfqpoint{2.577706in}{3.530099in}}%
\pgfpathcurveto{\pgfqpoint{2.571882in}{3.530099in}}{\pgfqpoint{2.566296in}{3.527786in}}{\pgfqpoint{2.562178in}{3.523667in}}%
\pgfpathcurveto{\pgfqpoint{2.558060in}{3.519549in}}{\pgfqpoint{2.555746in}{3.513963in}}{\pgfqpoint{2.555746in}{3.508139in}}%
\pgfpathcurveto{\pgfqpoint{2.555746in}{3.502315in}}{\pgfqpoint{2.558060in}{3.496729in}}{\pgfqpoint{2.562178in}{3.492611in}}%
\pgfpathcurveto{\pgfqpoint{2.566296in}{3.488493in}}{\pgfqpoint{2.571882in}{3.486179in}}{\pgfqpoint{2.577706in}{3.486179in}}%
\pgfpathlineto{\pgfqpoint{2.577706in}{3.486179in}}%
\pgfpathclose%
\pgfusepath{stroke,fill}%
\end{pgfscope}%
\begin{pgfscope}%
\pgfpathrectangle{\pgfqpoint{0.997489in}{0.528000in}}{\pgfqpoint{4.565023in}{3.696000in}}%
\pgfusepath{clip}%
\pgfsetbuttcap%
\pgfsetroundjoin%
\definecolor{currentfill}{rgb}{0.200000,0.800000,0.200000}%
\pgfsetfillcolor{currentfill}%
\pgfsetlinewidth{1.003750pt}%
\definecolor{currentstroke}{rgb}{0.200000,0.800000,0.200000}%
\pgfsetstrokecolor{currentstroke}%
\pgfsetdash{}{0pt}%
\pgfpathmoveto{\pgfqpoint{2.694801in}{3.726546in}}%
\pgfpathcurveto{\pgfqpoint{2.700625in}{3.726546in}}{\pgfqpoint{2.706211in}{3.728860in}}{\pgfqpoint{2.710329in}{3.732978in}}%
\pgfpathcurveto{\pgfqpoint{2.714448in}{3.737096in}}{\pgfqpoint{2.716761in}{3.742682in}}{\pgfqpoint{2.716761in}{3.748506in}}%
\pgfpathcurveto{\pgfqpoint{2.716761in}{3.754330in}}{\pgfqpoint{2.714448in}{3.759917in}}{\pgfqpoint{2.710329in}{3.764035in}}%
\pgfpathcurveto{\pgfqpoint{2.706211in}{3.768153in}}{\pgfqpoint{2.700625in}{3.770467in}}{\pgfqpoint{2.694801in}{3.770467in}}%
\pgfpathcurveto{\pgfqpoint{2.688977in}{3.770467in}}{\pgfqpoint{2.683391in}{3.768153in}}{\pgfqpoint{2.679273in}{3.764035in}}%
\pgfpathcurveto{\pgfqpoint{2.675155in}{3.759917in}}{\pgfqpoint{2.672841in}{3.754330in}}{\pgfqpoint{2.672841in}{3.748506in}}%
\pgfpathcurveto{\pgfqpoint{2.672841in}{3.742682in}}{\pgfqpoint{2.675155in}{3.737096in}}{\pgfqpoint{2.679273in}{3.732978in}}%
\pgfpathcurveto{\pgfqpoint{2.683391in}{3.728860in}}{\pgfqpoint{2.688977in}{3.726546in}}{\pgfqpoint{2.694801in}{3.726546in}}%
\pgfpathlineto{\pgfqpoint{2.694801in}{3.726546in}}%
\pgfpathclose%
\pgfusepath{stroke,fill}%
\end{pgfscope}%
\begin{pgfscope}%
\pgfpathrectangle{\pgfqpoint{0.997489in}{0.528000in}}{\pgfqpoint{4.565023in}{3.696000in}}%
\pgfusepath{clip}%
\pgfsetbuttcap%
\pgfsetroundjoin%
\definecolor{currentfill}{rgb}{0.200000,0.800000,0.200000}%
\pgfsetfillcolor{currentfill}%
\pgfsetlinewidth{1.003750pt}%
\definecolor{currentstroke}{rgb}{0.200000,0.800000,0.200000}%
\pgfsetstrokecolor{currentstroke}%
\pgfsetdash{}{0pt}%
\pgfpathmoveto{\pgfqpoint{2.588193in}{3.669822in}}%
\pgfpathcurveto{\pgfqpoint{2.594017in}{3.669822in}}{\pgfqpoint{2.599603in}{3.672136in}}{\pgfqpoint{2.603722in}{3.676254in}}%
\pgfpathcurveto{\pgfqpoint{2.607840in}{3.680373in}}{\pgfqpoint{2.610154in}{3.685959in}}{\pgfqpoint{2.610154in}{3.691783in}}%
\pgfpathcurveto{\pgfqpoint{2.610154in}{3.697607in}}{\pgfqpoint{2.607840in}{3.703193in}}{\pgfqpoint{2.603722in}{3.707311in}}%
\pgfpathcurveto{\pgfqpoint{2.599603in}{3.711429in}}{\pgfqpoint{2.594017in}{3.713743in}}{\pgfqpoint{2.588193in}{3.713743in}}%
\pgfpathcurveto{\pgfqpoint{2.582369in}{3.713743in}}{\pgfqpoint{2.576783in}{3.711429in}}{\pgfqpoint{2.572665in}{3.707311in}}%
\pgfpathcurveto{\pgfqpoint{2.568547in}{3.703193in}}{\pgfqpoint{2.566233in}{3.697607in}}{\pgfqpoint{2.566233in}{3.691783in}}%
\pgfpathcurveto{\pgfqpoint{2.566233in}{3.685959in}}{\pgfqpoint{2.568547in}{3.680373in}}{\pgfqpoint{2.572665in}{3.676254in}}%
\pgfpathcurveto{\pgfqpoint{2.576783in}{3.672136in}}{\pgfqpoint{2.582369in}{3.669822in}}{\pgfqpoint{2.588193in}{3.669822in}}%
\pgfpathlineto{\pgfqpoint{2.588193in}{3.669822in}}%
\pgfpathclose%
\pgfusepath{stroke,fill}%
\end{pgfscope}%
\begin{pgfscope}%
\pgfpathrectangle{\pgfqpoint{0.997489in}{0.528000in}}{\pgfqpoint{4.565023in}{3.696000in}}%
\pgfusepath{clip}%
\pgfsetbuttcap%
\pgfsetroundjoin%
\definecolor{currentfill}{rgb}{0.200000,0.800000,0.200000}%
\pgfsetfillcolor{currentfill}%
\pgfsetlinewidth{1.003750pt}%
\definecolor{currentstroke}{rgb}{0.200000,0.800000,0.200000}%
\pgfsetstrokecolor{currentstroke}%
\pgfsetdash{}{0pt}%
\pgfpathmoveto{\pgfqpoint{2.550454in}{3.713593in}}%
\pgfpathcurveto{\pgfqpoint{2.556278in}{3.713593in}}{\pgfqpoint{2.561864in}{3.715907in}}{\pgfqpoint{2.565982in}{3.720025in}}%
\pgfpathcurveto{\pgfqpoint{2.570100in}{3.724143in}}{\pgfqpoint{2.572414in}{3.729729in}}{\pgfqpoint{2.572414in}{3.735553in}}%
\pgfpathcurveto{\pgfqpoint{2.572414in}{3.741377in}}{\pgfqpoint{2.570100in}{3.746963in}}{\pgfqpoint{2.565982in}{3.751082in}}%
\pgfpathcurveto{\pgfqpoint{2.561864in}{3.755200in}}{\pgfqpoint{2.556278in}{3.757514in}}{\pgfqpoint{2.550454in}{3.757514in}}%
\pgfpathcurveto{\pgfqpoint{2.544630in}{3.757514in}}{\pgfqpoint{2.539044in}{3.755200in}}{\pgfqpoint{2.534926in}{3.751082in}}%
\pgfpathcurveto{\pgfqpoint{2.530807in}{3.746963in}}{\pgfqpoint{2.528494in}{3.741377in}}{\pgfqpoint{2.528494in}{3.735553in}}%
\pgfpathcurveto{\pgfqpoint{2.528494in}{3.729729in}}{\pgfqpoint{2.530807in}{3.724143in}}{\pgfqpoint{2.534926in}{3.720025in}}%
\pgfpathcurveto{\pgfqpoint{2.539044in}{3.715907in}}{\pgfqpoint{2.544630in}{3.713593in}}{\pgfqpoint{2.550454in}{3.713593in}}%
\pgfpathlineto{\pgfqpoint{2.550454in}{3.713593in}}%
\pgfpathclose%
\pgfusepath{stroke,fill}%
\end{pgfscope}%
\begin{pgfscope}%
\pgfpathrectangle{\pgfqpoint{0.997489in}{0.528000in}}{\pgfqpoint{4.565023in}{3.696000in}}%
\pgfusepath{clip}%
\pgfsetbuttcap%
\pgfsetroundjoin%
\definecolor{currentfill}{rgb}{0.200000,0.800000,0.200000}%
\pgfsetfillcolor{currentfill}%
\pgfsetlinewidth{1.003750pt}%
\definecolor{currentstroke}{rgb}{0.200000,0.800000,0.200000}%
\pgfsetstrokecolor{currentstroke}%
\pgfsetdash{}{0pt}%
\pgfpathmoveto{\pgfqpoint{2.542225in}{3.828323in}}%
\pgfpathcurveto{\pgfqpoint{2.548049in}{3.828323in}}{\pgfqpoint{2.553636in}{3.830637in}}{\pgfqpoint{2.557754in}{3.834755in}}%
\pgfpathcurveto{\pgfqpoint{2.561872in}{3.838873in}}{\pgfqpoint{2.564186in}{3.844459in}}{\pgfqpoint{2.564186in}{3.850283in}}%
\pgfpathcurveto{\pgfqpoint{2.564186in}{3.856107in}}{\pgfqpoint{2.561872in}{3.861693in}}{\pgfqpoint{2.557754in}{3.865811in}}%
\pgfpathcurveto{\pgfqpoint{2.553636in}{3.869929in}}{\pgfqpoint{2.548049in}{3.872243in}}{\pgfqpoint{2.542225in}{3.872243in}}%
\pgfpathcurveto{\pgfqpoint{2.536402in}{3.872243in}}{\pgfqpoint{2.530815in}{3.869929in}}{\pgfqpoint{2.526697in}{3.865811in}}%
\pgfpathcurveto{\pgfqpoint{2.522579in}{3.861693in}}{\pgfqpoint{2.520265in}{3.856107in}}{\pgfqpoint{2.520265in}{3.850283in}}%
\pgfpathcurveto{\pgfqpoint{2.520265in}{3.844459in}}{\pgfqpoint{2.522579in}{3.838873in}}{\pgfqpoint{2.526697in}{3.834755in}}%
\pgfpathcurveto{\pgfqpoint{2.530815in}{3.830637in}}{\pgfqpoint{2.536402in}{3.828323in}}{\pgfqpoint{2.542225in}{3.828323in}}%
\pgfpathlineto{\pgfqpoint{2.542225in}{3.828323in}}%
\pgfpathclose%
\pgfusepath{stroke,fill}%
\end{pgfscope}%
\begin{pgfscope}%
\pgfpathrectangle{\pgfqpoint{0.997489in}{0.528000in}}{\pgfqpoint{4.565023in}{3.696000in}}%
\pgfusepath{clip}%
\pgfsetbuttcap%
\pgfsetroundjoin%
\definecolor{currentfill}{rgb}{0.200000,0.800000,0.200000}%
\pgfsetfillcolor{currentfill}%
\pgfsetlinewidth{1.003750pt}%
\definecolor{currentstroke}{rgb}{0.200000,0.800000,0.200000}%
\pgfsetstrokecolor{currentstroke}%
\pgfsetdash{}{0pt}%
\pgfpathmoveto{\pgfqpoint{2.433075in}{3.714767in}}%
\pgfpathcurveto{\pgfqpoint{2.438899in}{3.714767in}}{\pgfqpoint{2.444485in}{3.717081in}}{\pgfqpoint{2.448603in}{3.721199in}}%
\pgfpathcurveto{\pgfqpoint{2.452721in}{3.725317in}}{\pgfqpoint{2.455035in}{3.730904in}}{\pgfqpoint{2.455035in}{3.736728in}}%
\pgfpathcurveto{\pgfqpoint{2.455035in}{3.742551in}}{\pgfqpoint{2.452721in}{3.748138in}}{\pgfqpoint{2.448603in}{3.752256in}}%
\pgfpathcurveto{\pgfqpoint{2.444485in}{3.756374in}}{\pgfqpoint{2.438899in}{3.758688in}}{\pgfqpoint{2.433075in}{3.758688in}}%
\pgfpathcurveto{\pgfqpoint{2.427251in}{3.758688in}}{\pgfqpoint{2.421665in}{3.756374in}}{\pgfqpoint{2.417547in}{3.752256in}}%
\pgfpathcurveto{\pgfqpoint{2.413429in}{3.748138in}}{\pgfqpoint{2.411115in}{3.742551in}}{\pgfqpoint{2.411115in}{3.736728in}}%
\pgfpathcurveto{\pgfqpoint{2.411115in}{3.730904in}}{\pgfqpoint{2.413429in}{3.725317in}}{\pgfqpoint{2.417547in}{3.721199in}}%
\pgfpathcurveto{\pgfqpoint{2.421665in}{3.717081in}}{\pgfqpoint{2.427251in}{3.714767in}}{\pgfqpoint{2.433075in}{3.714767in}}%
\pgfpathlineto{\pgfqpoint{2.433075in}{3.714767in}}%
\pgfpathclose%
\pgfusepath{stroke,fill}%
\end{pgfscope}%
\begin{pgfscope}%
\pgfpathrectangle{\pgfqpoint{0.997489in}{0.528000in}}{\pgfqpoint{4.565023in}{3.696000in}}%
\pgfusepath{clip}%
\pgfsetbuttcap%
\pgfsetroundjoin%
\definecolor{currentfill}{rgb}{0.200000,0.800000,0.200000}%
\pgfsetfillcolor{currentfill}%
\pgfsetlinewidth{1.003750pt}%
\definecolor{currentstroke}{rgb}{0.200000,0.800000,0.200000}%
\pgfsetstrokecolor{currentstroke}%
\pgfsetdash{}{0pt}%
\pgfpathmoveto{\pgfqpoint{2.431020in}{3.883673in}}%
\pgfpathcurveto{\pgfqpoint{2.436844in}{3.883673in}}{\pgfqpoint{2.442430in}{3.885987in}}{\pgfqpoint{2.446548in}{3.890105in}}%
\pgfpathcurveto{\pgfqpoint{2.450666in}{3.894224in}}{\pgfqpoint{2.452980in}{3.899810in}}{\pgfqpoint{2.452980in}{3.905634in}}%
\pgfpathcurveto{\pgfqpoint{2.452980in}{3.911458in}}{\pgfqpoint{2.450666in}{3.917044in}}{\pgfqpoint{2.446548in}{3.921162in}}%
\pgfpathcurveto{\pgfqpoint{2.442430in}{3.925280in}}{\pgfqpoint{2.436844in}{3.927594in}}{\pgfqpoint{2.431020in}{3.927594in}}%
\pgfpathcurveto{\pgfqpoint{2.425196in}{3.927594in}}{\pgfqpoint{2.419610in}{3.925280in}}{\pgfqpoint{2.415491in}{3.921162in}}%
\pgfpathcurveto{\pgfqpoint{2.411373in}{3.917044in}}{\pgfqpoint{2.409059in}{3.911458in}}{\pgfqpoint{2.409059in}{3.905634in}}%
\pgfpathcurveto{\pgfqpoint{2.409059in}{3.899810in}}{\pgfqpoint{2.411373in}{3.894224in}}{\pgfqpoint{2.415491in}{3.890105in}}%
\pgfpathcurveto{\pgfqpoint{2.419610in}{3.885987in}}{\pgfqpoint{2.425196in}{3.883673in}}{\pgfqpoint{2.431020in}{3.883673in}}%
\pgfpathlineto{\pgfqpoint{2.431020in}{3.883673in}}%
\pgfpathclose%
\pgfusepath{stroke,fill}%
\end{pgfscope}%
\begin{pgfscope}%
\pgfpathrectangle{\pgfqpoint{0.997489in}{0.528000in}}{\pgfqpoint{4.565023in}{3.696000in}}%
\pgfusepath{clip}%
\pgfsetbuttcap%
\pgfsetroundjoin%
\definecolor{currentfill}{rgb}{0.200000,0.800000,0.200000}%
\pgfsetfillcolor{currentfill}%
\pgfsetlinewidth{1.003750pt}%
\definecolor{currentstroke}{rgb}{0.200000,0.800000,0.200000}%
\pgfsetstrokecolor{currentstroke}%
\pgfsetdash{}{0pt}%
\pgfpathmoveto{\pgfqpoint{2.351613in}{3.820183in}}%
\pgfpathcurveto{\pgfqpoint{2.357437in}{3.820183in}}{\pgfqpoint{2.363023in}{3.822497in}}{\pgfqpoint{2.367141in}{3.826615in}}%
\pgfpathcurveto{\pgfqpoint{2.371259in}{3.830733in}}{\pgfqpoint{2.373573in}{3.836319in}}{\pgfqpoint{2.373573in}{3.842143in}}%
\pgfpathcurveto{\pgfqpoint{2.373573in}{3.847967in}}{\pgfqpoint{2.371259in}{3.853553in}}{\pgfqpoint{2.367141in}{3.857671in}}%
\pgfpathcurveto{\pgfqpoint{2.363023in}{3.861789in}}{\pgfqpoint{2.357437in}{3.864103in}}{\pgfqpoint{2.351613in}{3.864103in}}%
\pgfpathcurveto{\pgfqpoint{2.345789in}{3.864103in}}{\pgfqpoint{2.340203in}{3.861789in}}{\pgfqpoint{2.336084in}{3.857671in}}%
\pgfpathcurveto{\pgfqpoint{2.331966in}{3.853553in}}{\pgfqpoint{2.329652in}{3.847967in}}{\pgfqpoint{2.329652in}{3.842143in}}%
\pgfpathcurveto{\pgfqpoint{2.329652in}{3.836319in}}{\pgfqpoint{2.331966in}{3.830733in}}{\pgfqpoint{2.336084in}{3.826615in}}%
\pgfpathcurveto{\pgfqpoint{2.340203in}{3.822497in}}{\pgfqpoint{2.345789in}{3.820183in}}{\pgfqpoint{2.351613in}{3.820183in}}%
\pgfpathlineto{\pgfqpoint{2.351613in}{3.820183in}}%
\pgfpathclose%
\pgfusepath{stroke,fill}%
\end{pgfscope}%
\begin{pgfscope}%
\pgfpathrectangle{\pgfqpoint{0.997489in}{0.528000in}}{\pgfqpoint{4.565023in}{3.696000in}}%
\pgfusepath{clip}%
\pgfsetbuttcap%
\pgfsetroundjoin%
\definecolor{currentfill}{rgb}{0.200000,0.800000,0.200000}%
\pgfsetfillcolor{currentfill}%
\pgfsetlinewidth{1.003750pt}%
\definecolor{currentstroke}{rgb}{0.200000,0.800000,0.200000}%
\pgfsetstrokecolor{currentstroke}%
\pgfsetdash{}{0pt}%
\pgfpathmoveto{\pgfqpoint{2.293184in}{3.816921in}}%
\pgfpathcurveto{\pgfqpoint{2.299008in}{3.816921in}}{\pgfqpoint{2.304594in}{3.819235in}}{\pgfqpoint{2.308713in}{3.823353in}}%
\pgfpathcurveto{\pgfqpoint{2.312831in}{3.827471in}}{\pgfqpoint{2.315145in}{3.833058in}}{\pgfqpoint{2.315145in}{3.838881in}}%
\pgfpathcurveto{\pgfqpoint{2.315145in}{3.844705in}}{\pgfqpoint{2.312831in}{3.850292in}}{\pgfqpoint{2.308713in}{3.854410in}}%
\pgfpathcurveto{\pgfqpoint{2.304594in}{3.858528in}}{\pgfqpoint{2.299008in}{3.860842in}}{\pgfqpoint{2.293184in}{3.860842in}}%
\pgfpathcurveto{\pgfqpoint{2.287360in}{3.860842in}}{\pgfqpoint{2.281774in}{3.858528in}}{\pgfqpoint{2.277656in}{3.854410in}}%
\pgfpathcurveto{\pgfqpoint{2.273538in}{3.850292in}}{\pgfqpoint{2.271224in}{3.844705in}}{\pgfqpoint{2.271224in}{3.838881in}}%
\pgfpathcurveto{\pgfqpoint{2.271224in}{3.833058in}}{\pgfqpoint{2.273538in}{3.827471in}}{\pgfqpoint{2.277656in}{3.823353in}}%
\pgfpathcurveto{\pgfqpoint{2.281774in}{3.819235in}}{\pgfqpoint{2.287360in}{3.816921in}}{\pgfqpoint{2.293184in}{3.816921in}}%
\pgfpathlineto{\pgfqpoint{2.293184in}{3.816921in}}%
\pgfpathclose%
\pgfusepath{stroke,fill}%
\end{pgfscope}%
\begin{pgfscope}%
\pgfpathrectangle{\pgfqpoint{0.997489in}{0.528000in}}{\pgfqpoint{4.565023in}{3.696000in}}%
\pgfusepath{clip}%
\pgfsetbuttcap%
\pgfsetroundjoin%
\definecolor{currentfill}{rgb}{0.200000,0.800000,0.200000}%
\pgfsetfillcolor{currentfill}%
\pgfsetlinewidth{1.003750pt}%
\definecolor{currentstroke}{rgb}{0.200000,0.800000,0.200000}%
\pgfsetstrokecolor{currentstroke}%
\pgfsetdash{}{0pt}%
\pgfpathmoveto{\pgfqpoint{2.238846in}{3.835045in}}%
\pgfpathcurveto{\pgfqpoint{2.244670in}{3.835045in}}{\pgfqpoint{2.250256in}{3.837359in}}{\pgfqpoint{2.254374in}{3.841477in}}%
\pgfpathcurveto{\pgfqpoint{2.258492in}{3.845595in}}{\pgfqpoint{2.260806in}{3.851182in}}{\pgfqpoint{2.260806in}{3.857006in}}%
\pgfpathcurveto{\pgfqpoint{2.260806in}{3.862830in}}{\pgfqpoint{2.258492in}{3.868416in}}{\pgfqpoint{2.254374in}{3.872534in}}%
\pgfpathcurveto{\pgfqpoint{2.250256in}{3.876652in}}{\pgfqpoint{2.244670in}{3.878966in}}{\pgfqpoint{2.238846in}{3.878966in}}%
\pgfpathcurveto{\pgfqpoint{2.233022in}{3.878966in}}{\pgfqpoint{2.227436in}{3.876652in}}{\pgfqpoint{2.223318in}{3.872534in}}%
\pgfpathcurveto{\pgfqpoint{2.219200in}{3.868416in}}{\pgfqpoint{2.216886in}{3.862830in}}{\pgfqpoint{2.216886in}{3.857006in}}%
\pgfpathcurveto{\pgfqpoint{2.216886in}{3.851182in}}{\pgfqpoint{2.219200in}{3.845595in}}{\pgfqpoint{2.223318in}{3.841477in}}%
\pgfpathcurveto{\pgfqpoint{2.227436in}{3.837359in}}{\pgfqpoint{2.233022in}{3.835045in}}{\pgfqpoint{2.238846in}{3.835045in}}%
\pgfpathlineto{\pgfqpoint{2.238846in}{3.835045in}}%
\pgfpathclose%
\pgfusepath{stroke,fill}%
\end{pgfscope}%
\begin{pgfscope}%
\pgfpathrectangle{\pgfqpoint{0.997489in}{0.528000in}}{\pgfqpoint{4.565023in}{3.696000in}}%
\pgfusepath{clip}%
\pgfsetbuttcap%
\pgfsetroundjoin%
\definecolor{currentfill}{rgb}{0.200000,0.800000,0.200000}%
\pgfsetfillcolor{currentfill}%
\pgfsetlinewidth{1.003750pt}%
\definecolor{currentstroke}{rgb}{0.200000,0.800000,0.200000}%
\pgfsetstrokecolor{currentstroke}%
\pgfsetdash{}{0pt}%
\pgfpathmoveto{\pgfqpoint{2.178679in}{3.762769in}}%
\pgfpathcurveto{\pgfqpoint{2.184503in}{3.762769in}}{\pgfqpoint{2.190089in}{3.765083in}}{\pgfqpoint{2.194207in}{3.769201in}}%
\pgfpathcurveto{\pgfqpoint{2.198325in}{3.773319in}}{\pgfqpoint{2.200639in}{3.778905in}}{\pgfqpoint{2.200639in}{3.784729in}}%
\pgfpathcurveto{\pgfqpoint{2.200639in}{3.790553in}}{\pgfqpoint{2.198325in}{3.796139in}}{\pgfqpoint{2.194207in}{3.800257in}}%
\pgfpathcurveto{\pgfqpoint{2.190089in}{3.804375in}}{\pgfqpoint{2.184503in}{3.806689in}}{\pgfqpoint{2.178679in}{3.806689in}}%
\pgfpathcurveto{\pgfqpoint{2.172855in}{3.806689in}}{\pgfqpoint{2.167269in}{3.804375in}}{\pgfqpoint{2.163151in}{3.800257in}}%
\pgfpathcurveto{\pgfqpoint{2.159032in}{3.796139in}}{\pgfqpoint{2.156719in}{3.790553in}}{\pgfqpoint{2.156719in}{3.784729in}}%
\pgfpathcurveto{\pgfqpoint{2.156719in}{3.778905in}}{\pgfqpoint{2.159032in}{3.773319in}}{\pgfqpoint{2.163151in}{3.769201in}}%
\pgfpathcurveto{\pgfqpoint{2.167269in}{3.765083in}}{\pgfqpoint{2.172855in}{3.762769in}}{\pgfqpoint{2.178679in}{3.762769in}}%
\pgfpathlineto{\pgfqpoint{2.178679in}{3.762769in}}%
\pgfpathclose%
\pgfusepath{stroke,fill}%
\end{pgfscope}%
\begin{pgfscope}%
\pgfpathrectangle{\pgfqpoint{0.997489in}{0.528000in}}{\pgfqpoint{4.565023in}{3.696000in}}%
\pgfusepath{clip}%
\pgfsetbuttcap%
\pgfsetroundjoin%
\definecolor{currentfill}{rgb}{0.200000,0.800000,0.200000}%
\pgfsetfillcolor{currentfill}%
\pgfsetlinewidth{1.003750pt}%
\definecolor{currentstroke}{rgb}{0.200000,0.800000,0.200000}%
\pgfsetstrokecolor{currentstroke}%
\pgfsetdash{}{0pt}%
\pgfpathmoveto{\pgfqpoint{2.127537in}{3.721816in}}%
\pgfpathcurveto{\pgfqpoint{2.133361in}{3.721816in}}{\pgfqpoint{2.138947in}{3.724130in}}{\pgfqpoint{2.143065in}{3.728248in}}%
\pgfpathcurveto{\pgfqpoint{2.147183in}{3.732366in}}{\pgfqpoint{2.149497in}{3.737952in}}{\pgfqpoint{2.149497in}{3.743776in}}%
\pgfpathcurveto{\pgfqpoint{2.149497in}{3.749600in}}{\pgfqpoint{2.147183in}{3.755186in}}{\pgfqpoint{2.143065in}{3.759305in}}%
\pgfpathcurveto{\pgfqpoint{2.138947in}{3.763423in}}{\pgfqpoint{2.133361in}{3.765737in}}{\pgfqpoint{2.127537in}{3.765737in}}%
\pgfpathcurveto{\pgfqpoint{2.121713in}{3.765737in}}{\pgfqpoint{2.116127in}{3.763423in}}{\pgfqpoint{2.112008in}{3.759305in}}%
\pgfpathcurveto{\pgfqpoint{2.107890in}{3.755186in}}{\pgfqpoint{2.105576in}{3.749600in}}{\pgfqpoint{2.105576in}{3.743776in}}%
\pgfpathcurveto{\pgfqpoint{2.105576in}{3.737952in}}{\pgfqpoint{2.107890in}{3.732366in}}{\pgfqpoint{2.112008in}{3.728248in}}%
\pgfpathcurveto{\pgfqpoint{2.116127in}{3.724130in}}{\pgfqpoint{2.121713in}{3.721816in}}{\pgfqpoint{2.127537in}{3.721816in}}%
\pgfpathlineto{\pgfqpoint{2.127537in}{3.721816in}}%
\pgfpathclose%
\pgfusepath{stroke,fill}%
\end{pgfscope}%
\begin{pgfscope}%
\pgfpathrectangle{\pgfqpoint{0.997489in}{0.528000in}}{\pgfqpoint{4.565023in}{3.696000in}}%
\pgfusepath{clip}%
\pgfsetbuttcap%
\pgfsetroundjoin%
\definecolor{currentfill}{rgb}{0.200000,0.800000,0.200000}%
\pgfsetfillcolor{currentfill}%
\pgfsetlinewidth{1.003750pt}%
\definecolor{currentstroke}{rgb}{0.200000,0.800000,0.200000}%
\pgfsetstrokecolor{currentstroke}%
\pgfsetdash{}{0pt}%
\pgfpathmoveto{\pgfqpoint{2.076511in}{3.743641in}}%
\pgfpathcurveto{\pgfqpoint{2.082335in}{3.743641in}}{\pgfqpoint{2.087921in}{3.745955in}}{\pgfqpoint{2.092039in}{3.750073in}}%
\pgfpathcurveto{\pgfqpoint{2.096157in}{3.754191in}}{\pgfqpoint{2.098471in}{3.759778in}}{\pgfqpoint{2.098471in}{3.765602in}}%
\pgfpathcurveto{\pgfqpoint{2.098471in}{3.771426in}}{\pgfqpoint{2.096157in}{3.777012in}}{\pgfqpoint{2.092039in}{3.781130in}}%
\pgfpathcurveto{\pgfqpoint{2.087921in}{3.785248in}}{\pgfqpoint{2.082335in}{3.787562in}}{\pgfqpoint{2.076511in}{3.787562in}}%
\pgfpathcurveto{\pgfqpoint{2.070687in}{3.787562in}}{\pgfqpoint{2.065101in}{3.785248in}}{\pgfqpoint{2.060983in}{3.781130in}}%
\pgfpathcurveto{\pgfqpoint{2.056865in}{3.777012in}}{\pgfqpoint{2.054551in}{3.771426in}}{\pgfqpoint{2.054551in}{3.765602in}}%
\pgfpathcurveto{\pgfqpoint{2.054551in}{3.759778in}}{\pgfqpoint{2.056865in}{3.754191in}}{\pgfqpoint{2.060983in}{3.750073in}}%
\pgfpathcurveto{\pgfqpoint{2.065101in}{3.745955in}}{\pgfqpoint{2.070687in}{3.743641in}}{\pgfqpoint{2.076511in}{3.743641in}}%
\pgfpathlineto{\pgfqpoint{2.076511in}{3.743641in}}%
\pgfpathclose%
\pgfusepath{stroke,fill}%
\end{pgfscope}%
\begin{pgfscope}%
\pgfpathrectangle{\pgfqpoint{0.997489in}{0.528000in}}{\pgfqpoint{4.565023in}{3.696000in}}%
\pgfusepath{clip}%
\pgfsetbuttcap%
\pgfsetroundjoin%
\definecolor{currentfill}{rgb}{0.200000,0.800000,0.200000}%
\pgfsetfillcolor{currentfill}%
\pgfsetlinewidth{1.003750pt}%
\definecolor{currentstroke}{rgb}{0.200000,0.800000,0.200000}%
\pgfsetstrokecolor{currentstroke}%
\pgfsetdash{}{0pt}%
\pgfpathmoveto{\pgfqpoint{2.009425in}{3.854171in}}%
\pgfpathcurveto{\pgfqpoint{2.015249in}{3.854171in}}{\pgfqpoint{2.020835in}{3.856485in}}{\pgfqpoint{2.024953in}{3.860603in}}%
\pgfpathcurveto{\pgfqpoint{2.029071in}{3.864722in}}{\pgfqpoint{2.031385in}{3.870308in}}{\pgfqpoint{2.031385in}{3.876132in}}%
\pgfpathcurveto{\pgfqpoint{2.031385in}{3.881956in}}{\pgfqpoint{2.029071in}{3.887542in}}{\pgfqpoint{2.024953in}{3.891660in}}%
\pgfpathcurveto{\pgfqpoint{2.020835in}{3.895778in}}{\pgfqpoint{2.015249in}{3.898092in}}{\pgfqpoint{2.009425in}{3.898092in}}%
\pgfpathcurveto{\pgfqpoint{2.003601in}{3.898092in}}{\pgfqpoint{1.998015in}{3.895778in}}{\pgfqpoint{1.993897in}{3.891660in}}%
\pgfpathcurveto{\pgfqpoint{1.989779in}{3.887542in}}{\pgfqpoint{1.987465in}{3.881956in}}{\pgfqpoint{1.987465in}{3.876132in}}%
\pgfpathcurveto{\pgfqpoint{1.987465in}{3.870308in}}{\pgfqpoint{1.989779in}{3.864722in}}{\pgfqpoint{1.993897in}{3.860603in}}%
\pgfpathcurveto{\pgfqpoint{1.998015in}{3.856485in}}{\pgfqpoint{2.003601in}{3.854171in}}{\pgfqpoint{2.009425in}{3.854171in}}%
\pgfpathlineto{\pgfqpoint{2.009425in}{3.854171in}}%
\pgfpathclose%
\pgfusepath{stroke,fill}%
\end{pgfscope}%
\begin{pgfscope}%
\pgfpathrectangle{\pgfqpoint{0.997489in}{0.528000in}}{\pgfqpoint{4.565023in}{3.696000in}}%
\pgfusepath{clip}%
\pgfsetbuttcap%
\pgfsetroundjoin%
\definecolor{currentfill}{rgb}{0.200000,0.800000,0.200000}%
\pgfsetfillcolor{currentfill}%
\pgfsetlinewidth{1.003750pt}%
\definecolor{currentstroke}{rgb}{0.200000,0.800000,0.200000}%
\pgfsetstrokecolor{currentstroke}%
\pgfsetdash{}{0pt}%
\pgfpathmoveto{\pgfqpoint{1.973601in}{3.741621in}}%
\pgfpathcurveto{\pgfqpoint{1.979425in}{3.741621in}}{\pgfqpoint{1.985011in}{3.743935in}}{\pgfqpoint{1.989129in}{3.748053in}}%
\pgfpathcurveto{\pgfqpoint{1.993248in}{3.752171in}}{\pgfqpoint{1.995561in}{3.757757in}}{\pgfqpoint{1.995561in}{3.763581in}}%
\pgfpathcurveto{\pgfqpoint{1.995561in}{3.769405in}}{\pgfqpoint{1.993248in}{3.774991in}}{\pgfqpoint{1.989129in}{3.779109in}}%
\pgfpathcurveto{\pgfqpoint{1.985011in}{3.783227in}}{\pgfqpoint{1.979425in}{3.785541in}}{\pgfqpoint{1.973601in}{3.785541in}}%
\pgfpathcurveto{\pgfqpoint{1.967777in}{3.785541in}}{\pgfqpoint{1.962191in}{3.783227in}}{\pgfqpoint{1.958073in}{3.779109in}}%
\pgfpathcurveto{\pgfqpoint{1.953955in}{3.774991in}}{\pgfqpoint{1.951641in}{3.769405in}}{\pgfqpoint{1.951641in}{3.763581in}}%
\pgfpathcurveto{\pgfqpoint{1.951641in}{3.757757in}}{\pgfqpoint{1.953955in}{3.752171in}}{\pgfqpoint{1.958073in}{3.748053in}}%
\pgfpathcurveto{\pgfqpoint{1.962191in}{3.743935in}}{\pgfqpoint{1.967777in}{3.741621in}}{\pgfqpoint{1.973601in}{3.741621in}}%
\pgfpathlineto{\pgfqpoint{1.973601in}{3.741621in}}%
\pgfpathclose%
\pgfusepath{stroke,fill}%
\end{pgfscope}%
\begin{pgfscope}%
\pgfpathrectangle{\pgfqpoint{0.997489in}{0.528000in}}{\pgfqpoint{4.565023in}{3.696000in}}%
\pgfusepath{clip}%
\pgfsetbuttcap%
\pgfsetroundjoin%
\definecolor{currentfill}{rgb}{0.200000,0.800000,0.200000}%
\pgfsetfillcolor{currentfill}%
\pgfsetlinewidth{1.003750pt}%
\definecolor{currentstroke}{rgb}{0.200000,0.800000,0.200000}%
\pgfsetstrokecolor{currentstroke}%
\pgfsetdash{}{0pt}%
\pgfpathmoveto{\pgfqpoint{1.933457in}{3.693645in}}%
\pgfpathcurveto{\pgfqpoint{1.939281in}{3.693645in}}{\pgfqpoint{1.944867in}{3.695959in}}{\pgfqpoint{1.948985in}{3.700077in}}%
\pgfpathcurveto{\pgfqpoint{1.953104in}{3.704195in}}{\pgfqpoint{1.955417in}{3.709781in}}{\pgfqpoint{1.955417in}{3.715605in}}%
\pgfpathcurveto{\pgfqpoint{1.955417in}{3.721429in}}{\pgfqpoint{1.953104in}{3.727015in}}{\pgfqpoint{1.948985in}{3.731134in}}%
\pgfpathcurveto{\pgfqpoint{1.944867in}{3.735252in}}{\pgfqpoint{1.939281in}{3.737566in}}{\pgfqpoint{1.933457in}{3.737566in}}%
\pgfpathcurveto{\pgfqpoint{1.927633in}{3.737566in}}{\pgfqpoint{1.922047in}{3.735252in}}{\pgfqpoint{1.917929in}{3.731134in}}%
\pgfpathcurveto{\pgfqpoint{1.913811in}{3.727015in}}{\pgfqpoint{1.911497in}{3.721429in}}{\pgfqpoint{1.911497in}{3.715605in}}%
\pgfpathcurveto{\pgfqpoint{1.911497in}{3.709781in}}{\pgfqpoint{1.913811in}{3.704195in}}{\pgfqpoint{1.917929in}{3.700077in}}%
\pgfpathcurveto{\pgfqpoint{1.922047in}{3.695959in}}{\pgfqpoint{1.927633in}{3.693645in}}{\pgfqpoint{1.933457in}{3.693645in}}%
\pgfpathlineto{\pgfqpoint{1.933457in}{3.693645in}}%
\pgfpathclose%
\pgfusepath{stroke,fill}%
\end{pgfscope}%
\begin{pgfscope}%
\pgfpathrectangle{\pgfqpoint{0.997489in}{0.528000in}}{\pgfqpoint{4.565023in}{3.696000in}}%
\pgfusepath{clip}%
\pgfsetbuttcap%
\pgfsetroundjoin%
\definecolor{currentfill}{rgb}{0.200000,0.800000,0.200000}%
\pgfsetfillcolor{currentfill}%
\pgfsetlinewidth{1.003750pt}%
\definecolor{currentstroke}{rgb}{0.200000,0.800000,0.200000}%
\pgfsetstrokecolor{currentstroke}%
\pgfsetdash{}{0pt}%
\pgfpathmoveto{\pgfqpoint{1.859880in}{3.756054in}}%
\pgfpathcurveto{\pgfqpoint{1.865704in}{3.756054in}}{\pgfqpoint{1.871290in}{3.758368in}}{\pgfqpoint{1.875408in}{3.762486in}}%
\pgfpathcurveto{\pgfqpoint{1.879526in}{3.766604in}}{\pgfqpoint{1.881840in}{3.772190in}}{\pgfqpoint{1.881840in}{3.778014in}}%
\pgfpathcurveto{\pgfqpoint{1.881840in}{3.783838in}}{\pgfqpoint{1.879526in}{3.789424in}}{\pgfqpoint{1.875408in}{3.793542in}}%
\pgfpathcurveto{\pgfqpoint{1.871290in}{3.797660in}}{\pgfqpoint{1.865704in}{3.799974in}}{\pgfqpoint{1.859880in}{3.799974in}}%
\pgfpathcurveto{\pgfqpoint{1.854056in}{3.799974in}}{\pgfqpoint{1.848470in}{3.797660in}}{\pgfqpoint{1.844352in}{3.793542in}}%
\pgfpathcurveto{\pgfqpoint{1.840233in}{3.789424in}}{\pgfqpoint{1.837920in}{3.783838in}}{\pgfqpoint{1.837920in}{3.778014in}}%
\pgfpathcurveto{\pgfqpoint{1.837920in}{3.772190in}}{\pgfqpoint{1.840233in}{3.766604in}}{\pgfqpoint{1.844352in}{3.762486in}}%
\pgfpathcurveto{\pgfqpoint{1.848470in}{3.758368in}}{\pgfqpoint{1.854056in}{3.756054in}}{\pgfqpoint{1.859880in}{3.756054in}}%
\pgfpathlineto{\pgfqpoint{1.859880in}{3.756054in}}%
\pgfpathclose%
\pgfusepath{stroke,fill}%
\end{pgfscope}%
\begin{pgfscope}%
\pgfpathrectangle{\pgfqpoint{0.997489in}{0.528000in}}{\pgfqpoint{4.565023in}{3.696000in}}%
\pgfusepath{clip}%
\pgfsetbuttcap%
\pgfsetroundjoin%
\definecolor{currentfill}{rgb}{0.200000,0.800000,0.200000}%
\pgfsetfillcolor{currentfill}%
\pgfsetlinewidth{1.003750pt}%
\definecolor{currentstroke}{rgb}{0.200000,0.800000,0.200000}%
\pgfsetstrokecolor{currentstroke}%
\pgfsetdash{}{0pt}%
\pgfpathmoveto{\pgfqpoint{1.796069in}{3.767882in}}%
\pgfpathcurveto{\pgfqpoint{1.801893in}{3.767882in}}{\pgfqpoint{1.807479in}{3.770196in}}{\pgfqpoint{1.811597in}{3.774314in}}%
\pgfpathcurveto{\pgfqpoint{1.815715in}{3.778432in}}{\pgfqpoint{1.818029in}{3.784019in}}{\pgfqpoint{1.818029in}{3.789843in}}%
\pgfpathcurveto{\pgfqpoint{1.818029in}{3.795667in}}{\pgfqpoint{1.815715in}{3.801253in}}{\pgfqpoint{1.811597in}{3.805371in}}%
\pgfpathcurveto{\pgfqpoint{1.807479in}{3.809489in}}{\pgfqpoint{1.801893in}{3.811803in}}{\pgfqpoint{1.796069in}{3.811803in}}%
\pgfpathcurveto{\pgfqpoint{1.790245in}{3.811803in}}{\pgfqpoint{1.784659in}{3.809489in}}{\pgfqpoint{1.780540in}{3.805371in}}%
\pgfpathcurveto{\pgfqpoint{1.776422in}{3.801253in}}{\pgfqpoint{1.774108in}{3.795667in}}{\pgfqpoint{1.774108in}{3.789843in}}%
\pgfpathcurveto{\pgfqpoint{1.774108in}{3.784019in}}{\pgfqpoint{1.776422in}{3.778432in}}{\pgfqpoint{1.780540in}{3.774314in}}%
\pgfpathcurveto{\pgfqpoint{1.784659in}{3.770196in}}{\pgfqpoint{1.790245in}{3.767882in}}{\pgfqpoint{1.796069in}{3.767882in}}%
\pgfpathlineto{\pgfqpoint{1.796069in}{3.767882in}}%
\pgfpathclose%
\pgfusepath{stroke,fill}%
\end{pgfscope}%
\begin{pgfscope}%
\pgfpathrectangle{\pgfqpoint{0.997489in}{0.528000in}}{\pgfqpoint{4.565023in}{3.696000in}}%
\pgfusepath{clip}%
\pgfsetbuttcap%
\pgfsetroundjoin%
\definecolor{currentfill}{rgb}{0.200000,0.800000,0.200000}%
\pgfsetfillcolor{currentfill}%
\pgfsetlinewidth{1.003750pt}%
\definecolor{currentstroke}{rgb}{0.200000,0.800000,0.200000}%
\pgfsetstrokecolor{currentstroke}%
\pgfsetdash{}{0pt}%
\pgfpathmoveto{\pgfqpoint{1.757026in}{3.719576in}}%
\pgfpathcurveto{\pgfqpoint{1.762850in}{3.719576in}}{\pgfqpoint{1.768436in}{3.721890in}}{\pgfqpoint{1.772555in}{3.726008in}}%
\pgfpathcurveto{\pgfqpoint{1.776673in}{3.730126in}}{\pgfqpoint{1.778987in}{3.735712in}}{\pgfqpoint{1.778987in}{3.741536in}}%
\pgfpathcurveto{\pgfqpoint{1.778987in}{3.747360in}}{\pgfqpoint{1.776673in}{3.752946in}}{\pgfqpoint{1.772555in}{3.757065in}}%
\pgfpathcurveto{\pgfqpoint{1.768436in}{3.761183in}}{\pgfqpoint{1.762850in}{3.763497in}}{\pgfqpoint{1.757026in}{3.763497in}}%
\pgfpathcurveto{\pgfqpoint{1.751202in}{3.763497in}}{\pgfqpoint{1.745616in}{3.761183in}}{\pgfqpoint{1.741498in}{3.757065in}}%
\pgfpathcurveto{\pgfqpoint{1.737380in}{3.752946in}}{\pgfqpoint{1.735066in}{3.747360in}}{\pgfqpoint{1.735066in}{3.741536in}}%
\pgfpathcurveto{\pgfqpoint{1.735066in}{3.735712in}}{\pgfqpoint{1.737380in}{3.730126in}}{\pgfqpoint{1.741498in}{3.726008in}}%
\pgfpathcurveto{\pgfqpoint{1.745616in}{3.721890in}}{\pgfqpoint{1.751202in}{3.719576in}}{\pgfqpoint{1.757026in}{3.719576in}}%
\pgfpathlineto{\pgfqpoint{1.757026in}{3.719576in}}%
\pgfpathclose%
\pgfusepath{stroke,fill}%
\end{pgfscope}%
\begin{pgfscope}%
\pgfpathrectangle{\pgfqpoint{0.997489in}{0.528000in}}{\pgfqpoint{4.565023in}{3.696000in}}%
\pgfusepath{clip}%
\pgfsetbuttcap%
\pgfsetroundjoin%
\definecolor{currentfill}{rgb}{0.200000,0.800000,0.200000}%
\pgfsetfillcolor{currentfill}%
\pgfsetlinewidth{1.003750pt}%
\definecolor{currentstroke}{rgb}{0.200000,0.800000,0.200000}%
\pgfsetstrokecolor{currentstroke}%
\pgfsetdash{}{0pt}%
\pgfpathmoveto{\pgfqpoint{1.694122in}{3.719182in}}%
\pgfpathcurveto{\pgfqpoint{1.699946in}{3.719182in}}{\pgfqpoint{1.705532in}{3.721496in}}{\pgfqpoint{1.709650in}{3.725614in}}%
\pgfpathcurveto{\pgfqpoint{1.713768in}{3.729732in}}{\pgfqpoint{1.716082in}{3.735318in}}{\pgfqpoint{1.716082in}{3.741142in}}%
\pgfpathcurveto{\pgfqpoint{1.716082in}{3.746966in}}{\pgfqpoint{1.713768in}{3.752553in}}{\pgfqpoint{1.709650in}{3.756671in}}%
\pgfpathcurveto{\pgfqpoint{1.705532in}{3.760789in}}{\pgfqpoint{1.699946in}{3.763103in}}{\pgfqpoint{1.694122in}{3.763103in}}%
\pgfpathcurveto{\pgfqpoint{1.688298in}{3.763103in}}{\pgfqpoint{1.682712in}{3.760789in}}{\pgfqpoint{1.678594in}{3.756671in}}%
\pgfpathcurveto{\pgfqpoint{1.674475in}{3.752553in}}{\pgfqpoint{1.672162in}{3.746966in}}{\pgfqpoint{1.672162in}{3.741142in}}%
\pgfpathcurveto{\pgfqpoint{1.672162in}{3.735318in}}{\pgfqpoint{1.674475in}{3.729732in}}{\pgfqpoint{1.678594in}{3.725614in}}%
\pgfpathcurveto{\pgfqpoint{1.682712in}{3.721496in}}{\pgfqpoint{1.688298in}{3.719182in}}{\pgfqpoint{1.694122in}{3.719182in}}%
\pgfpathlineto{\pgfqpoint{1.694122in}{3.719182in}}%
\pgfpathclose%
\pgfusepath{stroke,fill}%
\end{pgfscope}%
\begin{pgfscope}%
\pgfpathrectangle{\pgfqpoint{0.997489in}{0.528000in}}{\pgfqpoint{4.565023in}{3.696000in}}%
\pgfusepath{clip}%
\pgfsetbuttcap%
\pgfsetroundjoin%
\definecolor{currentfill}{rgb}{0.200000,0.800000,0.200000}%
\pgfsetfillcolor{currentfill}%
\pgfsetlinewidth{1.003750pt}%
\definecolor{currentstroke}{rgb}{0.200000,0.800000,0.200000}%
\pgfsetstrokecolor{currentstroke}%
\pgfsetdash{}{0pt}%
\pgfpathmoveto{\pgfqpoint{1.630333in}{3.712993in}}%
\pgfpathcurveto{\pgfqpoint{1.636157in}{3.712993in}}{\pgfqpoint{1.641743in}{3.715307in}}{\pgfqpoint{1.645861in}{3.719425in}}%
\pgfpathcurveto{\pgfqpoint{1.649979in}{3.723544in}}{\pgfqpoint{1.652293in}{3.729130in}}{\pgfqpoint{1.652293in}{3.734954in}}%
\pgfpathcurveto{\pgfqpoint{1.652293in}{3.740778in}}{\pgfqpoint{1.649979in}{3.746364in}}{\pgfqpoint{1.645861in}{3.750482in}}%
\pgfpathcurveto{\pgfqpoint{1.641743in}{3.754600in}}{\pgfqpoint{1.636157in}{3.756914in}}{\pgfqpoint{1.630333in}{3.756914in}}%
\pgfpathcurveto{\pgfqpoint{1.624509in}{3.756914in}}{\pgfqpoint{1.618923in}{3.754600in}}{\pgfqpoint{1.614804in}{3.750482in}}%
\pgfpathcurveto{\pgfqpoint{1.610686in}{3.746364in}}{\pgfqpoint{1.608372in}{3.740778in}}{\pgfqpoint{1.608372in}{3.734954in}}%
\pgfpathcurveto{\pgfqpoint{1.608372in}{3.729130in}}{\pgfqpoint{1.610686in}{3.723544in}}{\pgfqpoint{1.614804in}{3.719425in}}%
\pgfpathcurveto{\pgfqpoint{1.618923in}{3.715307in}}{\pgfqpoint{1.624509in}{3.712993in}}{\pgfqpoint{1.630333in}{3.712993in}}%
\pgfpathlineto{\pgfqpoint{1.630333in}{3.712993in}}%
\pgfpathclose%
\pgfusepath{stroke,fill}%
\end{pgfscope}%
\begin{pgfscope}%
\pgfpathrectangle{\pgfqpoint{0.997489in}{0.528000in}}{\pgfqpoint{4.565023in}{3.696000in}}%
\pgfusepath{clip}%
\pgfsetbuttcap%
\pgfsetroundjoin%
\definecolor{currentfill}{rgb}{0.200000,0.800000,0.200000}%
\pgfsetfillcolor{currentfill}%
\pgfsetlinewidth{1.003750pt}%
\definecolor{currentstroke}{rgb}{0.200000,0.800000,0.200000}%
\pgfsetstrokecolor{currentstroke}%
\pgfsetdash{}{0pt}%
\pgfpathmoveto{\pgfqpoint{1.601080in}{3.655103in}}%
\pgfpathcurveto{\pgfqpoint{1.606904in}{3.655103in}}{\pgfqpoint{1.612490in}{3.657417in}}{\pgfqpoint{1.616608in}{3.661535in}}%
\pgfpathcurveto{\pgfqpoint{1.620726in}{3.665653in}}{\pgfqpoint{1.623040in}{3.671239in}}{\pgfqpoint{1.623040in}{3.677063in}}%
\pgfpathcurveto{\pgfqpoint{1.623040in}{3.682887in}}{\pgfqpoint{1.620726in}{3.688473in}}{\pgfqpoint{1.616608in}{3.692591in}}%
\pgfpathcurveto{\pgfqpoint{1.612490in}{3.696709in}}{\pgfqpoint{1.606904in}{3.699023in}}{\pgfqpoint{1.601080in}{3.699023in}}%
\pgfpathcurveto{\pgfqpoint{1.595256in}{3.699023in}}{\pgfqpoint{1.589670in}{3.696709in}}{\pgfqpoint{1.585552in}{3.692591in}}%
\pgfpathcurveto{\pgfqpoint{1.581434in}{3.688473in}}{\pgfqpoint{1.579120in}{3.682887in}}{\pgfqpoint{1.579120in}{3.677063in}}%
\pgfpathcurveto{\pgfqpoint{1.579120in}{3.671239in}}{\pgfqpoint{1.581434in}{3.665653in}}{\pgfqpoint{1.585552in}{3.661535in}}%
\pgfpathcurveto{\pgfqpoint{1.589670in}{3.657417in}}{\pgfqpoint{1.595256in}{3.655103in}}{\pgfqpoint{1.601080in}{3.655103in}}%
\pgfpathlineto{\pgfqpoint{1.601080in}{3.655103in}}%
\pgfpathclose%
\pgfusepath{stroke,fill}%
\end{pgfscope}%
\begin{pgfscope}%
\pgfpathrectangle{\pgfqpoint{0.997489in}{0.528000in}}{\pgfqpoint{4.565023in}{3.696000in}}%
\pgfusepath{clip}%
\pgfsetbuttcap%
\pgfsetroundjoin%
\definecolor{currentfill}{rgb}{0.200000,0.800000,0.200000}%
\pgfsetfillcolor{currentfill}%
\pgfsetlinewidth{1.003750pt}%
\definecolor{currentstroke}{rgb}{0.200000,0.800000,0.200000}%
\pgfsetstrokecolor{currentstroke}%
\pgfsetdash{}{0pt}%
\pgfpathmoveto{\pgfqpoint{1.581843in}{3.591145in}}%
\pgfpathcurveto{\pgfqpoint{1.587667in}{3.591145in}}{\pgfqpoint{1.593253in}{3.593459in}}{\pgfqpoint{1.597372in}{3.597577in}}%
\pgfpathcurveto{\pgfqpoint{1.601490in}{3.601695in}}{\pgfqpoint{1.603804in}{3.607281in}}{\pgfqpoint{1.603804in}{3.613105in}}%
\pgfpathcurveto{\pgfqpoint{1.603804in}{3.618929in}}{\pgfqpoint{1.601490in}{3.624515in}}{\pgfqpoint{1.597372in}{3.628633in}}%
\pgfpathcurveto{\pgfqpoint{1.593253in}{3.632751in}}{\pgfqpoint{1.587667in}{3.635065in}}{\pgfqpoint{1.581843in}{3.635065in}}%
\pgfpathcurveto{\pgfqpoint{1.576019in}{3.635065in}}{\pgfqpoint{1.570433in}{3.632751in}}{\pgfqpoint{1.566315in}{3.628633in}}%
\pgfpathcurveto{\pgfqpoint{1.562197in}{3.624515in}}{\pgfqpoint{1.559883in}{3.618929in}}{\pgfqpoint{1.559883in}{3.613105in}}%
\pgfpathcurveto{\pgfqpoint{1.559883in}{3.607281in}}{\pgfqpoint{1.562197in}{3.601695in}}{\pgfqpoint{1.566315in}{3.597577in}}%
\pgfpathcurveto{\pgfqpoint{1.570433in}{3.593459in}}{\pgfqpoint{1.576019in}{3.591145in}}{\pgfqpoint{1.581843in}{3.591145in}}%
\pgfpathlineto{\pgfqpoint{1.581843in}{3.591145in}}%
\pgfpathclose%
\pgfusepath{stroke,fill}%
\end{pgfscope}%
\begin{pgfscope}%
\pgfpathrectangle{\pgfqpoint{0.997489in}{0.528000in}}{\pgfqpoint{4.565023in}{3.696000in}}%
\pgfusepath{clip}%
\pgfsetbuttcap%
\pgfsetroundjoin%
\definecolor{currentfill}{rgb}{0.200000,0.800000,0.200000}%
\pgfsetfillcolor{currentfill}%
\pgfsetlinewidth{1.003750pt}%
\definecolor{currentstroke}{rgb}{0.200000,0.800000,0.200000}%
\pgfsetstrokecolor{currentstroke}%
\pgfsetdash{}{0pt}%
\pgfpathmoveto{\pgfqpoint{1.556290in}{3.540067in}}%
\pgfpathcurveto{\pgfqpoint{1.562114in}{3.540067in}}{\pgfqpoint{1.567700in}{3.542381in}}{\pgfqpoint{1.571818in}{3.546499in}}%
\pgfpathcurveto{\pgfqpoint{1.575936in}{3.550617in}}{\pgfqpoint{1.578250in}{3.556204in}}{\pgfqpoint{1.578250in}{3.562028in}}%
\pgfpathcurveto{\pgfqpoint{1.578250in}{3.567852in}}{\pgfqpoint{1.575936in}{3.573438in}}{\pgfqpoint{1.571818in}{3.577556in}}%
\pgfpathcurveto{\pgfqpoint{1.567700in}{3.581674in}}{\pgfqpoint{1.562114in}{3.583988in}}{\pgfqpoint{1.556290in}{3.583988in}}%
\pgfpathcurveto{\pgfqpoint{1.550466in}{3.583988in}}{\pgfqpoint{1.544880in}{3.581674in}}{\pgfqpoint{1.540762in}{3.577556in}}%
\pgfpathcurveto{\pgfqpoint{1.536643in}{3.573438in}}{\pgfqpoint{1.534330in}{3.567852in}}{\pgfqpoint{1.534330in}{3.562028in}}%
\pgfpathcurveto{\pgfqpoint{1.534330in}{3.556204in}}{\pgfqpoint{1.536643in}{3.550617in}}{\pgfqpoint{1.540762in}{3.546499in}}%
\pgfpathcurveto{\pgfqpoint{1.544880in}{3.542381in}}{\pgfqpoint{1.550466in}{3.540067in}}{\pgfqpoint{1.556290in}{3.540067in}}%
\pgfpathlineto{\pgfqpoint{1.556290in}{3.540067in}}%
\pgfpathclose%
\pgfusepath{stroke,fill}%
\end{pgfscope}%
\begin{pgfscope}%
\pgfpathrectangle{\pgfqpoint{0.997489in}{0.528000in}}{\pgfqpoint{4.565023in}{3.696000in}}%
\pgfusepath{clip}%
\pgfsetbuttcap%
\pgfsetroundjoin%
\definecolor{currentfill}{rgb}{0.200000,0.800000,0.200000}%
\pgfsetfillcolor{currentfill}%
\pgfsetlinewidth{1.003750pt}%
\definecolor{currentstroke}{rgb}{0.200000,0.800000,0.200000}%
\pgfsetstrokecolor{currentstroke}%
\pgfsetdash{}{0pt}%
\pgfpathmoveto{\pgfqpoint{1.469150in}{3.547235in}}%
\pgfpathcurveto{\pgfqpoint{1.474974in}{3.547235in}}{\pgfqpoint{1.480560in}{3.549549in}}{\pgfqpoint{1.484678in}{3.553667in}}%
\pgfpathcurveto{\pgfqpoint{1.488796in}{3.557785in}}{\pgfqpoint{1.491110in}{3.563371in}}{\pgfqpoint{1.491110in}{3.569195in}}%
\pgfpathcurveto{\pgfqpoint{1.491110in}{3.575019in}}{\pgfqpoint{1.488796in}{3.580605in}}{\pgfqpoint{1.484678in}{3.584723in}}%
\pgfpathcurveto{\pgfqpoint{1.480560in}{3.588841in}}{\pgfqpoint{1.474974in}{3.591155in}}{\pgfqpoint{1.469150in}{3.591155in}}%
\pgfpathcurveto{\pgfqpoint{1.463326in}{3.591155in}}{\pgfqpoint{1.457740in}{3.588841in}}{\pgfqpoint{1.453621in}{3.584723in}}%
\pgfpathcurveto{\pgfqpoint{1.449503in}{3.580605in}}{\pgfqpoint{1.447189in}{3.575019in}}{\pgfqpoint{1.447189in}{3.569195in}}%
\pgfpathcurveto{\pgfqpoint{1.447189in}{3.563371in}}{\pgfqpoint{1.449503in}{3.557785in}}{\pgfqpoint{1.453621in}{3.553667in}}%
\pgfpathcurveto{\pgfqpoint{1.457740in}{3.549549in}}{\pgfqpoint{1.463326in}{3.547235in}}{\pgfqpoint{1.469150in}{3.547235in}}%
\pgfpathlineto{\pgfqpoint{1.469150in}{3.547235in}}%
\pgfpathclose%
\pgfusepath{stroke,fill}%
\end{pgfscope}%
\begin{pgfscope}%
\pgfpathrectangle{\pgfqpoint{0.997489in}{0.528000in}}{\pgfqpoint{4.565023in}{3.696000in}}%
\pgfusepath{clip}%
\pgfsetbuttcap%
\pgfsetroundjoin%
\definecolor{currentfill}{rgb}{0.200000,0.800000,0.200000}%
\pgfsetfillcolor{currentfill}%
\pgfsetlinewidth{1.003750pt}%
\definecolor{currentstroke}{rgb}{0.200000,0.800000,0.200000}%
\pgfsetstrokecolor{currentstroke}%
\pgfsetdash{}{0pt}%
\pgfpathmoveto{\pgfqpoint{1.452596in}{3.487642in}}%
\pgfpathcurveto{\pgfqpoint{1.458420in}{3.487642in}}{\pgfqpoint{1.464006in}{3.489955in}}{\pgfqpoint{1.468124in}{3.494074in}}%
\pgfpathcurveto{\pgfqpoint{1.472242in}{3.498192in}}{\pgfqpoint{1.474556in}{3.503778in}}{\pgfqpoint{1.474556in}{3.509602in}}%
\pgfpathcurveto{\pgfqpoint{1.474556in}{3.515426in}}{\pgfqpoint{1.472242in}{3.521012in}}{\pgfqpoint{1.468124in}{3.525130in}}%
\pgfpathcurveto{\pgfqpoint{1.464006in}{3.529248in}}{\pgfqpoint{1.458420in}{3.531562in}}{\pgfqpoint{1.452596in}{3.531562in}}%
\pgfpathcurveto{\pgfqpoint{1.446772in}{3.531562in}}{\pgfqpoint{1.441186in}{3.529248in}}{\pgfqpoint{1.437067in}{3.525130in}}%
\pgfpathcurveto{\pgfqpoint{1.432949in}{3.521012in}}{\pgfqpoint{1.430635in}{3.515426in}}{\pgfqpoint{1.430635in}{3.509602in}}%
\pgfpathcurveto{\pgfqpoint{1.430635in}{3.503778in}}{\pgfqpoint{1.432949in}{3.498192in}}{\pgfqpoint{1.437067in}{3.494074in}}%
\pgfpathcurveto{\pgfqpoint{1.441186in}{3.489955in}}{\pgfqpoint{1.446772in}{3.487642in}}{\pgfqpoint{1.452596in}{3.487642in}}%
\pgfpathlineto{\pgfqpoint{1.452596in}{3.487642in}}%
\pgfpathclose%
\pgfusepath{stroke,fill}%
\end{pgfscope}%
\begin{pgfscope}%
\pgfpathrectangle{\pgfqpoint{0.997489in}{0.528000in}}{\pgfqpoint{4.565023in}{3.696000in}}%
\pgfusepath{clip}%
\pgfsetbuttcap%
\pgfsetroundjoin%
\definecolor{currentfill}{rgb}{0.200000,0.800000,0.200000}%
\pgfsetfillcolor{currentfill}%
\pgfsetlinewidth{1.003750pt}%
\definecolor{currentstroke}{rgb}{0.200000,0.800000,0.200000}%
\pgfsetstrokecolor{currentstroke}%
\pgfsetdash{}{0pt}%
\pgfpathmoveto{\pgfqpoint{1.468120in}{3.409633in}}%
\pgfpathcurveto{\pgfqpoint{1.473944in}{3.409633in}}{\pgfqpoint{1.479530in}{3.411947in}}{\pgfqpoint{1.483648in}{3.416065in}}%
\pgfpathcurveto{\pgfqpoint{1.487766in}{3.420183in}}{\pgfqpoint{1.490080in}{3.425770in}}{\pgfqpoint{1.490080in}{3.431593in}}%
\pgfpathcurveto{\pgfqpoint{1.490080in}{3.437417in}}{\pgfqpoint{1.487766in}{3.443004in}}{\pgfqpoint{1.483648in}{3.447122in}}%
\pgfpathcurveto{\pgfqpoint{1.479530in}{3.451240in}}{\pgfqpoint{1.473944in}{3.453554in}}{\pgfqpoint{1.468120in}{3.453554in}}%
\pgfpathcurveto{\pgfqpoint{1.462296in}{3.453554in}}{\pgfqpoint{1.456710in}{3.451240in}}{\pgfqpoint{1.452592in}{3.447122in}}%
\pgfpathcurveto{\pgfqpoint{1.448474in}{3.443004in}}{\pgfqpoint{1.446160in}{3.437417in}}{\pgfqpoint{1.446160in}{3.431593in}}%
\pgfpathcurveto{\pgfqpoint{1.446160in}{3.425770in}}{\pgfqpoint{1.448474in}{3.420183in}}{\pgfqpoint{1.452592in}{3.416065in}}%
\pgfpathcurveto{\pgfqpoint{1.456710in}{3.411947in}}{\pgfqpoint{1.462296in}{3.409633in}}{\pgfqpoint{1.468120in}{3.409633in}}%
\pgfpathlineto{\pgfqpoint{1.468120in}{3.409633in}}%
\pgfpathclose%
\pgfusepath{stroke,fill}%
\end{pgfscope}%
\begin{pgfscope}%
\pgfpathrectangle{\pgfqpoint{0.997489in}{0.528000in}}{\pgfqpoint{4.565023in}{3.696000in}}%
\pgfusepath{clip}%
\pgfsetbuttcap%
\pgfsetroundjoin%
\definecolor{currentfill}{rgb}{0.200000,0.800000,0.200000}%
\pgfsetfillcolor{currentfill}%
\pgfsetlinewidth{1.003750pt}%
\definecolor{currentstroke}{rgb}{0.200000,0.800000,0.200000}%
\pgfsetstrokecolor{currentstroke}%
\pgfsetdash{}{0pt}%
\pgfpathmoveto{\pgfqpoint{1.344122in}{3.423588in}}%
\pgfpathcurveto{\pgfqpoint{1.349946in}{3.423588in}}{\pgfqpoint{1.355532in}{3.425902in}}{\pgfqpoint{1.359650in}{3.430020in}}%
\pgfpathcurveto{\pgfqpoint{1.363768in}{3.434138in}}{\pgfqpoint{1.366082in}{3.439724in}}{\pgfqpoint{1.366082in}{3.445548in}}%
\pgfpathcurveto{\pgfqpoint{1.366082in}{3.451372in}}{\pgfqpoint{1.363768in}{3.456958in}}{\pgfqpoint{1.359650in}{3.461076in}}%
\pgfpathcurveto{\pgfqpoint{1.355532in}{3.465194in}}{\pgfqpoint{1.349946in}{3.467508in}}{\pgfqpoint{1.344122in}{3.467508in}}%
\pgfpathcurveto{\pgfqpoint{1.338298in}{3.467508in}}{\pgfqpoint{1.332712in}{3.465194in}}{\pgfqpoint{1.328594in}{3.461076in}}%
\pgfpathcurveto{\pgfqpoint{1.324476in}{3.456958in}}{\pgfqpoint{1.322162in}{3.451372in}}{\pgfqpoint{1.322162in}{3.445548in}}%
\pgfpathcurveto{\pgfqpoint{1.322162in}{3.439724in}}{\pgfqpoint{1.324476in}{3.434138in}}{\pgfqpoint{1.328594in}{3.430020in}}%
\pgfpathcurveto{\pgfqpoint{1.332712in}{3.425902in}}{\pgfqpoint{1.338298in}{3.423588in}}{\pgfqpoint{1.344122in}{3.423588in}}%
\pgfpathlineto{\pgfqpoint{1.344122in}{3.423588in}}%
\pgfpathclose%
\pgfusepath{stroke,fill}%
\end{pgfscope}%
\begin{pgfscope}%
\pgfpathrectangle{\pgfqpoint{0.997489in}{0.528000in}}{\pgfqpoint{4.565023in}{3.696000in}}%
\pgfusepath{clip}%
\pgfsetbuttcap%
\pgfsetroundjoin%
\definecolor{currentfill}{rgb}{0.200000,0.800000,0.200000}%
\pgfsetfillcolor{currentfill}%
\pgfsetlinewidth{1.003750pt}%
\definecolor{currentstroke}{rgb}{0.200000,0.800000,0.200000}%
\pgfsetstrokecolor{currentstroke}%
\pgfsetdash{}{0pt}%
\pgfpathmoveto{\pgfqpoint{1.363340in}{3.347504in}}%
\pgfpathcurveto{\pgfqpoint{1.369164in}{3.347504in}}{\pgfqpoint{1.374750in}{3.349817in}}{\pgfqpoint{1.378868in}{3.353936in}}%
\pgfpathcurveto{\pgfqpoint{1.382986in}{3.358054in}}{\pgfqpoint{1.385300in}{3.363640in}}{\pgfqpoint{1.385300in}{3.369464in}}%
\pgfpathcurveto{\pgfqpoint{1.385300in}{3.375288in}}{\pgfqpoint{1.382986in}{3.380874in}}{\pgfqpoint{1.378868in}{3.384992in}}%
\pgfpathcurveto{\pgfqpoint{1.374750in}{3.389110in}}{\pgfqpoint{1.369164in}{3.391424in}}{\pgfqpoint{1.363340in}{3.391424in}}%
\pgfpathcurveto{\pgfqpoint{1.357516in}{3.391424in}}{\pgfqpoint{1.351930in}{3.389110in}}{\pgfqpoint{1.347811in}{3.384992in}}%
\pgfpathcurveto{\pgfqpoint{1.343693in}{3.380874in}}{\pgfqpoint{1.341379in}{3.375288in}}{\pgfqpoint{1.341379in}{3.369464in}}%
\pgfpathcurveto{\pgfqpoint{1.341379in}{3.363640in}}{\pgfqpoint{1.343693in}{3.358054in}}{\pgfqpoint{1.347811in}{3.353936in}}%
\pgfpathcurveto{\pgfqpoint{1.351930in}{3.349817in}}{\pgfqpoint{1.357516in}{3.347504in}}{\pgfqpoint{1.363340in}{3.347504in}}%
\pgfpathlineto{\pgfqpoint{1.363340in}{3.347504in}}%
\pgfpathclose%
\pgfusepath{stroke,fill}%
\end{pgfscope}%
\begin{pgfscope}%
\pgfpathrectangle{\pgfqpoint{0.997489in}{0.528000in}}{\pgfqpoint{4.565023in}{3.696000in}}%
\pgfusepath{clip}%
\pgfsetbuttcap%
\pgfsetroundjoin%
\definecolor{currentfill}{rgb}{0.200000,0.800000,0.200000}%
\pgfsetfillcolor{currentfill}%
\pgfsetlinewidth{1.003750pt}%
\definecolor{currentstroke}{rgb}{0.200000,0.800000,0.200000}%
\pgfsetstrokecolor{currentstroke}%
\pgfsetdash{}{0pt}%
\pgfpathmoveto{\pgfqpoint{1.320992in}{3.305556in}}%
\pgfpathcurveto{\pgfqpoint{1.326816in}{3.305556in}}{\pgfqpoint{1.332402in}{3.307870in}}{\pgfqpoint{1.336520in}{3.311988in}}%
\pgfpathcurveto{\pgfqpoint{1.340638in}{3.316106in}}{\pgfqpoint{1.342952in}{3.321692in}}{\pgfqpoint{1.342952in}{3.327516in}}%
\pgfpathcurveto{\pgfqpoint{1.342952in}{3.333340in}}{\pgfqpoint{1.340638in}{3.338926in}}{\pgfqpoint{1.336520in}{3.343044in}}%
\pgfpathcurveto{\pgfqpoint{1.332402in}{3.347162in}}{\pgfqpoint{1.326816in}{3.349476in}}{\pgfqpoint{1.320992in}{3.349476in}}%
\pgfpathcurveto{\pgfqpoint{1.315168in}{3.349476in}}{\pgfqpoint{1.309582in}{3.347162in}}{\pgfqpoint{1.305464in}{3.343044in}}%
\pgfpathcurveto{\pgfqpoint{1.301346in}{3.338926in}}{\pgfqpoint{1.299032in}{3.333340in}}{\pgfqpoint{1.299032in}{3.327516in}}%
\pgfpathcurveto{\pgfqpoint{1.299032in}{3.321692in}}{\pgfqpoint{1.301346in}{3.316106in}}{\pgfqpoint{1.305464in}{3.311988in}}%
\pgfpathcurveto{\pgfqpoint{1.309582in}{3.307870in}}{\pgfqpoint{1.315168in}{3.305556in}}{\pgfqpoint{1.320992in}{3.305556in}}%
\pgfpathlineto{\pgfqpoint{1.320992in}{3.305556in}}%
\pgfpathclose%
\pgfusepath{stroke,fill}%
\end{pgfscope}%
\begin{pgfscope}%
\pgfpathrectangle{\pgfqpoint{0.997489in}{0.528000in}}{\pgfqpoint{4.565023in}{3.696000in}}%
\pgfusepath{clip}%
\pgfsetbuttcap%
\pgfsetroundjoin%
\definecolor{currentfill}{rgb}{0.200000,0.800000,0.200000}%
\pgfsetfillcolor{currentfill}%
\pgfsetlinewidth{1.003750pt}%
\definecolor{currentstroke}{rgb}{0.200000,0.800000,0.200000}%
\pgfsetstrokecolor{currentstroke}%
\pgfsetdash{}{0pt}%
\pgfpathmoveto{\pgfqpoint{1.350718in}{3.234409in}}%
\pgfpathcurveto{\pgfqpoint{1.356542in}{3.234409in}}{\pgfqpoint{1.362128in}{3.236722in}}{\pgfqpoint{1.366246in}{3.240841in}}%
\pgfpathcurveto{\pgfqpoint{1.370364in}{3.244959in}}{\pgfqpoint{1.372678in}{3.250545in}}{\pgfqpoint{1.372678in}{3.256369in}}%
\pgfpathcurveto{\pgfqpoint{1.372678in}{3.262193in}}{\pgfqpoint{1.370364in}{3.267779in}}{\pgfqpoint{1.366246in}{3.271897in}}%
\pgfpathcurveto{\pgfqpoint{1.362128in}{3.276015in}}{\pgfqpoint{1.356542in}{3.278329in}}{\pgfqpoint{1.350718in}{3.278329in}}%
\pgfpathcurveto{\pgfqpoint{1.344894in}{3.278329in}}{\pgfqpoint{1.339308in}{3.276015in}}{\pgfqpoint{1.335190in}{3.271897in}}%
\pgfpathcurveto{\pgfqpoint{1.331071in}{3.267779in}}{\pgfqpoint{1.328758in}{3.262193in}}{\pgfqpoint{1.328758in}{3.256369in}}%
\pgfpathcurveto{\pgfqpoint{1.328758in}{3.250545in}}{\pgfqpoint{1.331071in}{3.244959in}}{\pgfqpoint{1.335190in}{3.240841in}}%
\pgfpathcurveto{\pgfqpoint{1.339308in}{3.236722in}}{\pgfqpoint{1.344894in}{3.234409in}}{\pgfqpoint{1.350718in}{3.234409in}}%
\pgfpathlineto{\pgfqpoint{1.350718in}{3.234409in}}%
\pgfpathclose%
\pgfusepath{stroke,fill}%
\end{pgfscope}%
\begin{pgfscope}%
\pgfpathrectangle{\pgfqpoint{0.997489in}{0.528000in}}{\pgfqpoint{4.565023in}{3.696000in}}%
\pgfusepath{clip}%
\pgfsetbuttcap%
\pgfsetroundjoin%
\definecolor{currentfill}{rgb}{0.200000,0.800000,0.200000}%
\pgfsetfillcolor{currentfill}%
\pgfsetlinewidth{1.003750pt}%
\definecolor{currentstroke}{rgb}{0.200000,0.800000,0.200000}%
\pgfsetstrokecolor{currentstroke}%
\pgfsetdash{}{0pt}%
\pgfpathmoveto{\pgfqpoint{1.258493in}{3.205979in}}%
\pgfpathcurveto{\pgfqpoint{1.264317in}{3.205979in}}{\pgfqpoint{1.269903in}{3.208293in}}{\pgfqpoint{1.274021in}{3.212411in}}%
\pgfpathcurveto{\pgfqpoint{1.278140in}{3.216529in}}{\pgfqpoint{1.280453in}{3.222116in}}{\pgfqpoint{1.280453in}{3.227939in}}%
\pgfpathcurveto{\pgfqpoint{1.280453in}{3.233763in}}{\pgfqpoint{1.278140in}{3.239350in}}{\pgfqpoint{1.274021in}{3.243468in}}%
\pgfpathcurveto{\pgfqpoint{1.269903in}{3.247586in}}{\pgfqpoint{1.264317in}{3.249900in}}{\pgfqpoint{1.258493in}{3.249900in}}%
\pgfpathcurveto{\pgfqpoint{1.252669in}{3.249900in}}{\pgfqpoint{1.247083in}{3.247586in}}{\pgfqpoint{1.242965in}{3.243468in}}%
\pgfpathcurveto{\pgfqpoint{1.238847in}{3.239350in}}{\pgfqpoint{1.236533in}{3.233763in}}{\pgfqpoint{1.236533in}{3.227939in}}%
\pgfpathcurveto{\pgfqpoint{1.236533in}{3.222116in}}{\pgfqpoint{1.238847in}{3.216529in}}{\pgfqpoint{1.242965in}{3.212411in}}%
\pgfpathcurveto{\pgfqpoint{1.247083in}{3.208293in}}{\pgfqpoint{1.252669in}{3.205979in}}{\pgfqpoint{1.258493in}{3.205979in}}%
\pgfpathlineto{\pgfqpoint{1.258493in}{3.205979in}}%
\pgfpathclose%
\pgfusepath{stroke,fill}%
\end{pgfscope}%
\begin{pgfscope}%
\pgfpathrectangle{\pgfqpoint{0.997489in}{0.528000in}}{\pgfqpoint{4.565023in}{3.696000in}}%
\pgfusepath{clip}%
\pgfsetbuttcap%
\pgfsetroundjoin%
\definecolor{currentfill}{rgb}{0.200000,0.800000,0.200000}%
\pgfsetfillcolor{currentfill}%
\pgfsetlinewidth{1.003750pt}%
\definecolor{currentstroke}{rgb}{0.200000,0.800000,0.200000}%
\pgfsetstrokecolor{currentstroke}%
\pgfsetdash{}{0pt}%
\pgfpathmoveto{\pgfqpoint{1.380936in}{3.118599in}}%
\pgfpathcurveto{\pgfqpoint{1.386760in}{3.118599in}}{\pgfqpoint{1.392346in}{3.120913in}}{\pgfqpoint{1.396464in}{3.125031in}}%
\pgfpathcurveto{\pgfqpoint{1.400582in}{3.129149in}}{\pgfqpoint{1.402896in}{3.134736in}}{\pgfqpoint{1.402896in}{3.140560in}}%
\pgfpathcurveto{\pgfqpoint{1.402896in}{3.146383in}}{\pgfqpoint{1.400582in}{3.151970in}}{\pgfqpoint{1.396464in}{3.156088in}}%
\pgfpathcurveto{\pgfqpoint{1.392346in}{3.160206in}}{\pgfqpoint{1.386760in}{3.162520in}}{\pgfqpoint{1.380936in}{3.162520in}}%
\pgfpathcurveto{\pgfqpoint{1.375112in}{3.162520in}}{\pgfqpoint{1.369526in}{3.160206in}}{\pgfqpoint{1.365407in}{3.156088in}}%
\pgfpathcurveto{\pgfqpoint{1.361289in}{3.151970in}}{\pgfqpoint{1.358975in}{3.146383in}}{\pgfqpoint{1.358975in}{3.140560in}}%
\pgfpathcurveto{\pgfqpoint{1.358975in}{3.134736in}}{\pgfqpoint{1.361289in}{3.129149in}}{\pgfqpoint{1.365407in}{3.125031in}}%
\pgfpathcurveto{\pgfqpoint{1.369526in}{3.120913in}}{\pgfqpoint{1.375112in}{3.118599in}}{\pgfqpoint{1.380936in}{3.118599in}}%
\pgfpathlineto{\pgfqpoint{1.380936in}{3.118599in}}%
\pgfpathclose%
\pgfusepath{stroke,fill}%
\end{pgfscope}%
\begin{pgfscope}%
\pgfpathrectangle{\pgfqpoint{0.997489in}{0.528000in}}{\pgfqpoint{4.565023in}{3.696000in}}%
\pgfusepath{clip}%
\pgfsetbuttcap%
\pgfsetroundjoin%
\definecolor{currentfill}{rgb}{0.200000,0.800000,0.200000}%
\pgfsetfillcolor{currentfill}%
\pgfsetlinewidth{1.003750pt}%
\definecolor{currentstroke}{rgb}{0.200000,0.800000,0.200000}%
\pgfsetstrokecolor{currentstroke}%
\pgfsetdash{}{0pt}%
\pgfpathmoveto{\pgfqpoint{1.204990in}{3.096781in}}%
\pgfpathcurveto{\pgfqpoint{1.210814in}{3.096781in}}{\pgfqpoint{1.216400in}{3.099094in}}{\pgfqpoint{1.220518in}{3.103213in}}%
\pgfpathcurveto{\pgfqpoint{1.224636in}{3.107331in}}{\pgfqpoint{1.226950in}{3.112917in}}{\pgfqpoint{1.226950in}{3.118741in}}%
\pgfpathcurveto{\pgfqpoint{1.226950in}{3.124565in}}{\pgfqpoint{1.224636in}{3.130151in}}{\pgfqpoint{1.220518in}{3.134269in}}%
\pgfpathcurveto{\pgfqpoint{1.216400in}{3.138387in}}{\pgfqpoint{1.210814in}{3.140701in}}{\pgfqpoint{1.204990in}{3.140701in}}%
\pgfpathcurveto{\pgfqpoint{1.199166in}{3.140701in}}{\pgfqpoint{1.193580in}{3.138387in}}{\pgfqpoint{1.189461in}{3.134269in}}%
\pgfpathcurveto{\pgfqpoint{1.185343in}{3.130151in}}{\pgfqpoint{1.183029in}{3.124565in}}{\pgfqpoint{1.183029in}{3.118741in}}%
\pgfpathcurveto{\pgfqpoint{1.183029in}{3.112917in}}{\pgfqpoint{1.185343in}{3.107331in}}{\pgfqpoint{1.189461in}{3.103213in}}%
\pgfpathcurveto{\pgfqpoint{1.193580in}{3.099094in}}{\pgfqpoint{1.199166in}{3.096781in}}{\pgfqpoint{1.204990in}{3.096781in}}%
\pgfpathlineto{\pgfqpoint{1.204990in}{3.096781in}}%
\pgfpathclose%
\pgfusepath{stroke,fill}%
\end{pgfscope}%
\begin{pgfscope}%
\pgfpathrectangle{\pgfqpoint{0.997489in}{0.528000in}}{\pgfqpoint{4.565023in}{3.696000in}}%
\pgfusepath{clip}%
\pgfsetbuttcap%
\pgfsetroundjoin%
\definecolor{currentfill}{rgb}{0.200000,0.800000,0.200000}%
\pgfsetfillcolor{currentfill}%
\pgfsetlinewidth{1.003750pt}%
\definecolor{currentstroke}{rgb}{0.200000,0.800000,0.200000}%
\pgfsetstrokecolor{currentstroke}%
\pgfsetdash{}{0pt}%
\pgfpathmoveto{\pgfqpoint{1.383550in}{3.019412in}}%
\pgfpathcurveto{\pgfqpoint{1.389374in}{3.019412in}}{\pgfqpoint{1.394960in}{3.021726in}}{\pgfqpoint{1.399078in}{3.025844in}}%
\pgfpathcurveto{\pgfqpoint{1.403196in}{3.029962in}}{\pgfqpoint{1.405510in}{3.035548in}}{\pgfqpoint{1.405510in}{3.041372in}}%
\pgfpathcurveto{\pgfqpoint{1.405510in}{3.047196in}}{\pgfqpoint{1.403196in}{3.052782in}}{\pgfqpoint{1.399078in}{3.056900in}}%
\pgfpathcurveto{\pgfqpoint{1.394960in}{3.061018in}}{\pgfqpoint{1.389374in}{3.063332in}}{\pgfqpoint{1.383550in}{3.063332in}}%
\pgfpathcurveto{\pgfqpoint{1.377726in}{3.063332in}}{\pgfqpoint{1.372140in}{3.061018in}}{\pgfqpoint{1.368021in}{3.056900in}}%
\pgfpathcurveto{\pgfqpoint{1.363903in}{3.052782in}}{\pgfqpoint{1.361589in}{3.047196in}}{\pgfqpoint{1.361589in}{3.041372in}}%
\pgfpathcurveto{\pgfqpoint{1.361589in}{3.035548in}}{\pgfqpoint{1.363903in}{3.029962in}}{\pgfqpoint{1.368021in}{3.025844in}}%
\pgfpathcurveto{\pgfqpoint{1.372140in}{3.021726in}}{\pgfqpoint{1.377726in}{3.019412in}}{\pgfqpoint{1.383550in}{3.019412in}}%
\pgfpathlineto{\pgfqpoint{1.383550in}{3.019412in}}%
\pgfpathclose%
\pgfusepath{stroke,fill}%
\end{pgfscope}%
\begin{pgfscope}%
\pgfpathrectangle{\pgfqpoint{0.997489in}{0.528000in}}{\pgfqpoint{4.565023in}{3.696000in}}%
\pgfusepath{clip}%
\pgfsetbuttcap%
\pgfsetroundjoin%
\definecolor{currentfill}{rgb}{0.200000,0.800000,0.200000}%
\pgfsetfillcolor{currentfill}%
\pgfsetlinewidth{1.003750pt}%
\definecolor{currentstroke}{rgb}{0.200000,0.800000,0.200000}%
\pgfsetstrokecolor{currentstroke}%
\pgfsetdash{}{0pt}%
\pgfpathmoveto{\pgfqpoint{1.351417in}{2.972223in}}%
\pgfpathcurveto{\pgfqpoint{1.357241in}{2.972223in}}{\pgfqpoint{1.362827in}{2.974537in}}{\pgfqpoint{1.366945in}{2.978655in}}%
\pgfpathcurveto{\pgfqpoint{1.371063in}{2.982773in}}{\pgfqpoint{1.373377in}{2.988359in}}{\pgfqpoint{1.373377in}{2.994183in}}%
\pgfpathcurveto{\pgfqpoint{1.373377in}{3.000007in}}{\pgfqpoint{1.371063in}{3.005593in}}{\pgfqpoint{1.366945in}{3.009712in}}%
\pgfpathcurveto{\pgfqpoint{1.362827in}{3.013830in}}{\pgfqpoint{1.357241in}{3.016144in}}{\pgfqpoint{1.351417in}{3.016144in}}%
\pgfpathcurveto{\pgfqpoint{1.345593in}{3.016144in}}{\pgfqpoint{1.340006in}{3.013830in}}{\pgfqpoint{1.335888in}{3.009712in}}%
\pgfpathcurveto{\pgfqpoint{1.331770in}{3.005593in}}{\pgfqpoint{1.329456in}{3.000007in}}{\pgfqpoint{1.329456in}{2.994183in}}%
\pgfpathcurveto{\pgfqpoint{1.329456in}{2.988359in}}{\pgfqpoint{1.331770in}{2.982773in}}{\pgfqpoint{1.335888in}{2.978655in}}%
\pgfpathcurveto{\pgfqpoint{1.340006in}{2.974537in}}{\pgfqpoint{1.345593in}{2.972223in}}{\pgfqpoint{1.351417in}{2.972223in}}%
\pgfpathlineto{\pgfqpoint{1.351417in}{2.972223in}}%
\pgfpathclose%
\pgfusepath{stroke,fill}%
\end{pgfscope}%
\begin{pgfscope}%
\pgfpathrectangle{\pgfqpoint{0.997489in}{0.528000in}}{\pgfqpoint{4.565023in}{3.696000in}}%
\pgfusepath{clip}%
\pgfsetbuttcap%
\pgfsetroundjoin%
\definecolor{currentfill}{rgb}{0.200000,0.800000,0.200000}%
\pgfsetfillcolor{currentfill}%
\pgfsetlinewidth{1.003750pt}%
\definecolor{currentstroke}{rgb}{0.200000,0.800000,0.200000}%
\pgfsetstrokecolor{currentstroke}%
\pgfsetdash{}{0pt}%
\pgfpathmoveto{\pgfqpoint{1.267200in}{2.919495in}}%
\pgfpathcurveto{\pgfqpoint{1.273024in}{2.919495in}}{\pgfqpoint{1.278610in}{2.921809in}}{\pgfqpoint{1.282728in}{2.925927in}}%
\pgfpathcurveto{\pgfqpoint{1.286847in}{2.930045in}}{\pgfqpoint{1.289160in}{2.935632in}}{\pgfqpoint{1.289160in}{2.941455in}}%
\pgfpathcurveto{\pgfqpoint{1.289160in}{2.947279in}}{\pgfqpoint{1.286847in}{2.952866in}}{\pgfqpoint{1.282728in}{2.956984in}}%
\pgfpathcurveto{\pgfqpoint{1.278610in}{2.961102in}}{\pgfqpoint{1.273024in}{2.963416in}}{\pgfqpoint{1.267200in}{2.963416in}}%
\pgfpathcurveto{\pgfqpoint{1.261376in}{2.963416in}}{\pgfqpoint{1.255790in}{2.961102in}}{\pgfqpoint{1.251672in}{2.956984in}}%
\pgfpathcurveto{\pgfqpoint{1.247554in}{2.952866in}}{\pgfqpoint{1.245240in}{2.947279in}}{\pgfqpoint{1.245240in}{2.941455in}}%
\pgfpathcurveto{\pgfqpoint{1.245240in}{2.935632in}}{\pgfqpoint{1.247554in}{2.930045in}}{\pgfqpoint{1.251672in}{2.925927in}}%
\pgfpathcurveto{\pgfqpoint{1.255790in}{2.921809in}}{\pgfqpoint{1.261376in}{2.919495in}}{\pgfqpoint{1.267200in}{2.919495in}}%
\pgfpathlineto{\pgfqpoint{1.267200in}{2.919495in}}%
\pgfpathclose%
\pgfusepath{stroke,fill}%
\end{pgfscope}%
\begin{pgfscope}%
\pgfpathrectangle{\pgfqpoint{0.997489in}{0.528000in}}{\pgfqpoint{4.565023in}{3.696000in}}%
\pgfusepath{clip}%
\pgfsetbuttcap%
\pgfsetroundjoin%
\definecolor{currentfill}{rgb}{0.200000,0.800000,0.200000}%
\pgfsetfillcolor{currentfill}%
\pgfsetlinewidth{1.003750pt}%
\definecolor{currentstroke}{rgb}{0.200000,0.800000,0.200000}%
\pgfsetstrokecolor{currentstroke}%
\pgfsetdash{}{0pt}%
\pgfpathmoveto{\pgfqpoint{1.311023in}{2.868055in}}%
\pgfpathcurveto{\pgfqpoint{1.316847in}{2.868055in}}{\pgfqpoint{1.322433in}{2.870368in}}{\pgfqpoint{1.326551in}{2.874487in}}%
\pgfpathcurveto{\pgfqpoint{1.330669in}{2.878605in}}{\pgfqpoint{1.332983in}{2.884191in}}{\pgfqpoint{1.332983in}{2.890015in}}%
\pgfpathcurveto{\pgfqpoint{1.332983in}{2.895839in}}{\pgfqpoint{1.330669in}{2.901425in}}{\pgfqpoint{1.326551in}{2.905543in}}%
\pgfpathcurveto{\pgfqpoint{1.322433in}{2.909661in}}{\pgfqpoint{1.316847in}{2.911975in}}{\pgfqpoint{1.311023in}{2.911975in}}%
\pgfpathcurveto{\pgfqpoint{1.305199in}{2.911975in}}{\pgfqpoint{1.299613in}{2.909661in}}{\pgfqpoint{1.295495in}{2.905543in}}%
\pgfpathcurveto{\pgfqpoint{1.291377in}{2.901425in}}{\pgfqpoint{1.289063in}{2.895839in}}{\pgfqpoint{1.289063in}{2.890015in}}%
\pgfpathcurveto{\pgfqpoint{1.289063in}{2.884191in}}{\pgfqpoint{1.291377in}{2.878605in}}{\pgfqpoint{1.295495in}{2.874487in}}%
\pgfpathcurveto{\pgfqpoint{1.299613in}{2.870368in}}{\pgfqpoint{1.305199in}{2.868055in}}{\pgfqpoint{1.311023in}{2.868055in}}%
\pgfpathlineto{\pgfqpoint{1.311023in}{2.868055in}}%
\pgfpathclose%
\pgfusepath{stroke,fill}%
\end{pgfscope}%
\begin{pgfscope}%
\pgfpathrectangle{\pgfqpoint{0.997489in}{0.528000in}}{\pgfqpoint{4.565023in}{3.696000in}}%
\pgfusepath{clip}%
\pgfsetbuttcap%
\pgfsetroundjoin%
\definecolor{currentfill}{rgb}{0.200000,0.800000,0.200000}%
\pgfsetfillcolor{currentfill}%
\pgfsetlinewidth{1.003750pt}%
\definecolor{currentstroke}{rgb}{0.200000,0.800000,0.200000}%
\pgfsetstrokecolor{currentstroke}%
\pgfsetdash{}{0pt}%
\pgfpathmoveto{\pgfqpoint{1.307369in}{2.813993in}}%
\pgfpathcurveto{\pgfqpoint{1.313193in}{2.813993in}}{\pgfqpoint{1.318779in}{2.816307in}}{\pgfqpoint{1.322897in}{2.820425in}}%
\pgfpathcurveto{\pgfqpoint{1.327015in}{2.824543in}}{\pgfqpoint{1.329329in}{2.830129in}}{\pgfqpoint{1.329329in}{2.835953in}}%
\pgfpathcurveto{\pgfqpoint{1.329329in}{2.841777in}}{\pgfqpoint{1.327015in}{2.847363in}}{\pgfqpoint{1.322897in}{2.851481in}}%
\pgfpathcurveto{\pgfqpoint{1.318779in}{2.855600in}}{\pgfqpoint{1.313193in}{2.857913in}}{\pgfqpoint{1.307369in}{2.857913in}}%
\pgfpathcurveto{\pgfqpoint{1.301545in}{2.857913in}}{\pgfqpoint{1.295959in}{2.855600in}}{\pgfqpoint{1.291841in}{2.851481in}}%
\pgfpathcurveto{\pgfqpoint{1.287722in}{2.847363in}}{\pgfqpoint{1.285409in}{2.841777in}}{\pgfqpoint{1.285409in}{2.835953in}}%
\pgfpathcurveto{\pgfqpoint{1.285409in}{2.830129in}}{\pgfqpoint{1.287722in}{2.824543in}}{\pgfqpoint{1.291841in}{2.820425in}}%
\pgfpathcurveto{\pgfqpoint{1.295959in}{2.816307in}}{\pgfqpoint{1.301545in}{2.813993in}}{\pgfqpoint{1.307369in}{2.813993in}}%
\pgfpathlineto{\pgfqpoint{1.307369in}{2.813993in}}%
\pgfpathclose%
\pgfusepath{stroke,fill}%
\end{pgfscope}%
\begin{pgfscope}%
\pgfpathrectangle{\pgfqpoint{0.997489in}{0.528000in}}{\pgfqpoint{4.565023in}{3.696000in}}%
\pgfusepath{clip}%
\pgfsetbuttcap%
\pgfsetroundjoin%
\definecolor{currentfill}{rgb}{0.200000,0.800000,0.200000}%
\pgfsetfillcolor{currentfill}%
\pgfsetlinewidth{1.003750pt}%
\definecolor{currentstroke}{rgb}{0.200000,0.800000,0.200000}%
\pgfsetstrokecolor{currentstroke}%
\pgfsetdash{}{0pt}%
\pgfpathmoveto{\pgfqpoint{1.327726in}{2.763775in}}%
\pgfpathcurveto{\pgfqpoint{1.333550in}{2.763775in}}{\pgfqpoint{1.339137in}{2.766088in}}{\pgfqpoint{1.343255in}{2.770207in}}%
\pgfpathcurveto{\pgfqpoint{1.347373in}{2.774325in}}{\pgfqpoint{1.349687in}{2.779911in}}{\pgfqpoint{1.349687in}{2.785735in}}%
\pgfpathcurveto{\pgfqpoint{1.349687in}{2.791559in}}{\pgfqpoint{1.347373in}{2.797145in}}{\pgfqpoint{1.343255in}{2.801263in}}%
\pgfpathcurveto{\pgfqpoint{1.339137in}{2.805381in}}{\pgfqpoint{1.333550in}{2.807695in}}{\pgfqpoint{1.327726in}{2.807695in}}%
\pgfpathcurveto{\pgfqpoint{1.321902in}{2.807695in}}{\pgfqpoint{1.316316in}{2.805381in}}{\pgfqpoint{1.312198in}{2.801263in}}%
\pgfpathcurveto{\pgfqpoint{1.308080in}{2.797145in}}{\pgfqpoint{1.305766in}{2.791559in}}{\pgfqpoint{1.305766in}{2.785735in}}%
\pgfpathcurveto{\pgfqpoint{1.305766in}{2.779911in}}{\pgfqpoint{1.308080in}{2.774325in}}{\pgfqpoint{1.312198in}{2.770207in}}%
\pgfpathcurveto{\pgfqpoint{1.316316in}{2.766088in}}{\pgfqpoint{1.321902in}{2.763775in}}{\pgfqpoint{1.327726in}{2.763775in}}%
\pgfpathlineto{\pgfqpoint{1.327726in}{2.763775in}}%
\pgfpathclose%
\pgfusepath{stroke,fill}%
\end{pgfscope}%
\begin{pgfscope}%
\pgfpathrectangle{\pgfqpoint{0.997489in}{0.528000in}}{\pgfqpoint{4.565023in}{3.696000in}}%
\pgfusepath{clip}%
\pgfsetbuttcap%
\pgfsetroundjoin%
\definecolor{currentfill}{rgb}{0.200000,0.800000,0.200000}%
\pgfsetfillcolor{currentfill}%
\pgfsetlinewidth{1.003750pt}%
\definecolor{currentstroke}{rgb}{0.200000,0.800000,0.200000}%
\pgfsetstrokecolor{currentstroke}%
\pgfsetdash{}{0pt}%
\pgfpathmoveto{\pgfqpoint{1.353505in}{2.716310in}}%
\pgfpathcurveto{\pgfqpoint{1.359329in}{2.716310in}}{\pgfqpoint{1.364915in}{2.718624in}}{\pgfqpoint{1.369033in}{2.722742in}}%
\pgfpathcurveto{\pgfqpoint{1.373151in}{2.726861in}}{\pgfqpoint{1.375465in}{2.732447in}}{\pgfqpoint{1.375465in}{2.738271in}}%
\pgfpathcurveto{\pgfqpoint{1.375465in}{2.744095in}}{\pgfqpoint{1.373151in}{2.749681in}}{\pgfqpoint{1.369033in}{2.753799in}}%
\pgfpathcurveto{\pgfqpoint{1.364915in}{2.757917in}}{\pgfqpoint{1.359329in}{2.760231in}}{\pgfqpoint{1.353505in}{2.760231in}}%
\pgfpathcurveto{\pgfqpoint{1.347681in}{2.760231in}}{\pgfqpoint{1.342095in}{2.757917in}}{\pgfqpoint{1.337976in}{2.753799in}}%
\pgfpathcurveto{\pgfqpoint{1.333858in}{2.749681in}}{\pgfqpoint{1.331544in}{2.744095in}}{\pgfqpoint{1.331544in}{2.738271in}}%
\pgfpathcurveto{\pgfqpoint{1.331544in}{2.732447in}}{\pgfqpoint{1.333858in}{2.726861in}}{\pgfqpoint{1.337976in}{2.722742in}}%
\pgfpathcurveto{\pgfqpoint{1.342095in}{2.718624in}}{\pgfqpoint{1.347681in}{2.716310in}}{\pgfqpoint{1.353505in}{2.716310in}}%
\pgfpathlineto{\pgfqpoint{1.353505in}{2.716310in}}%
\pgfpathclose%
\pgfusepath{stroke,fill}%
\end{pgfscope}%
\begin{pgfscope}%
\pgfpathrectangle{\pgfqpoint{0.997489in}{0.528000in}}{\pgfqpoint{4.565023in}{3.696000in}}%
\pgfusepath{clip}%
\pgfsetbuttcap%
\pgfsetroundjoin%
\definecolor{currentfill}{rgb}{0.200000,0.800000,0.200000}%
\pgfsetfillcolor{currentfill}%
\pgfsetlinewidth{1.003750pt}%
\definecolor{currentstroke}{rgb}{0.200000,0.800000,0.200000}%
\pgfsetstrokecolor{currentstroke}%
\pgfsetdash{}{0pt}%
\pgfpathmoveto{\pgfqpoint{1.316246in}{2.647436in}}%
\pgfpathcurveto{\pgfqpoint{1.322069in}{2.647436in}}{\pgfqpoint{1.327656in}{2.649750in}}{\pgfqpoint{1.331774in}{2.653868in}}%
\pgfpathcurveto{\pgfqpoint{1.335892in}{2.657986in}}{\pgfqpoint{1.338206in}{2.663573in}}{\pgfqpoint{1.338206in}{2.669396in}}%
\pgfpathcurveto{\pgfqpoint{1.338206in}{2.675220in}}{\pgfqpoint{1.335892in}{2.680807in}}{\pgfqpoint{1.331774in}{2.684925in}}%
\pgfpathcurveto{\pgfqpoint{1.327656in}{2.689043in}}{\pgfqpoint{1.322069in}{2.691357in}}{\pgfqpoint{1.316246in}{2.691357in}}%
\pgfpathcurveto{\pgfqpoint{1.310422in}{2.691357in}}{\pgfqpoint{1.304835in}{2.689043in}}{\pgfqpoint{1.300717in}{2.684925in}}%
\pgfpathcurveto{\pgfqpoint{1.296599in}{2.680807in}}{\pgfqpoint{1.294285in}{2.675220in}}{\pgfqpoint{1.294285in}{2.669396in}}%
\pgfpathcurveto{\pgfqpoint{1.294285in}{2.663573in}}{\pgfqpoint{1.296599in}{2.657986in}}{\pgfqpoint{1.300717in}{2.653868in}}%
\pgfpathcurveto{\pgfqpoint{1.304835in}{2.649750in}}{\pgfqpoint{1.310422in}{2.647436in}}{\pgfqpoint{1.316246in}{2.647436in}}%
\pgfpathlineto{\pgfqpoint{1.316246in}{2.647436in}}%
\pgfpathclose%
\pgfusepath{stroke,fill}%
\end{pgfscope}%
\begin{pgfscope}%
\pgfpathrectangle{\pgfqpoint{0.997489in}{0.528000in}}{\pgfqpoint{4.565023in}{3.696000in}}%
\pgfusepath{clip}%
\pgfsetbuttcap%
\pgfsetroundjoin%
\definecolor{currentfill}{rgb}{0.200000,0.800000,0.200000}%
\pgfsetfillcolor{currentfill}%
\pgfsetlinewidth{1.003750pt}%
\definecolor{currentstroke}{rgb}{0.200000,0.800000,0.200000}%
\pgfsetstrokecolor{currentstroke}%
\pgfsetdash{}{0pt}%
\pgfpathmoveto{\pgfqpoint{1.354243in}{2.603388in}}%
\pgfpathcurveto{\pgfqpoint{1.360066in}{2.603388in}}{\pgfqpoint{1.365653in}{2.605702in}}{\pgfqpoint{1.369771in}{2.609820in}}%
\pgfpathcurveto{\pgfqpoint{1.373889in}{2.613938in}}{\pgfqpoint{1.376203in}{2.619524in}}{\pgfqpoint{1.376203in}{2.625348in}}%
\pgfpathcurveto{\pgfqpoint{1.376203in}{2.631172in}}{\pgfqpoint{1.373889in}{2.636758in}}{\pgfqpoint{1.369771in}{2.640876in}}%
\pgfpathcurveto{\pgfqpoint{1.365653in}{2.644995in}}{\pgfqpoint{1.360066in}{2.647308in}}{\pgfqpoint{1.354243in}{2.647308in}}%
\pgfpathcurveto{\pgfqpoint{1.348419in}{2.647308in}}{\pgfqpoint{1.342832in}{2.644995in}}{\pgfqpoint{1.338714in}{2.640876in}}%
\pgfpathcurveto{\pgfqpoint{1.334596in}{2.636758in}}{\pgfqpoint{1.332282in}{2.631172in}}{\pgfqpoint{1.332282in}{2.625348in}}%
\pgfpathcurveto{\pgfqpoint{1.332282in}{2.619524in}}{\pgfqpoint{1.334596in}{2.613938in}}{\pgfqpoint{1.338714in}{2.609820in}}%
\pgfpathcurveto{\pgfqpoint{1.342832in}{2.605702in}}{\pgfqpoint{1.348419in}{2.603388in}}{\pgfqpoint{1.354243in}{2.603388in}}%
\pgfpathlineto{\pgfqpoint{1.354243in}{2.603388in}}%
\pgfpathclose%
\pgfusepath{stroke,fill}%
\end{pgfscope}%
\begin{pgfscope}%
\pgfpathrectangle{\pgfqpoint{0.997489in}{0.528000in}}{\pgfqpoint{4.565023in}{3.696000in}}%
\pgfusepath{clip}%
\pgfsetbuttcap%
\pgfsetroundjoin%
\definecolor{currentfill}{rgb}{0.200000,0.800000,0.200000}%
\pgfsetfillcolor{currentfill}%
\pgfsetlinewidth{1.003750pt}%
\definecolor{currentstroke}{rgb}{0.200000,0.800000,0.200000}%
\pgfsetstrokecolor{currentstroke}%
\pgfsetdash{}{0pt}%
\pgfpathmoveto{\pgfqpoint{1.334703in}{2.532125in}}%
\pgfpathcurveto{\pgfqpoint{1.340527in}{2.532125in}}{\pgfqpoint{1.346113in}{2.534439in}}{\pgfqpoint{1.350231in}{2.538557in}}%
\pgfpathcurveto{\pgfqpoint{1.354349in}{2.542675in}}{\pgfqpoint{1.356663in}{2.548261in}}{\pgfqpoint{1.356663in}{2.554085in}}%
\pgfpathcurveto{\pgfqpoint{1.356663in}{2.559909in}}{\pgfqpoint{1.354349in}{2.565495in}}{\pgfqpoint{1.350231in}{2.569613in}}%
\pgfpathcurveto{\pgfqpoint{1.346113in}{2.573732in}}{\pgfqpoint{1.340527in}{2.576045in}}{\pgfqpoint{1.334703in}{2.576045in}}%
\pgfpathcurveto{\pgfqpoint{1.328879in}{2.576045in}}{\pgfqpoint{1.323293in}{2.573732in}}{\pgfqpoint{1.319175in}{2.569613in}}%
\pgfpathcurveto{\pgfqpoint{1.315057in}{2.565495in}}{\pgfqpoint{1.312743in}{2.559909in}}{\pgfqpoint{1.312743in}{2.554085in}}%
\pgfpathcurveto{\pgfqpoint{1.312743in}{2.548261in}}{\pgfqpoint{1.315057in}{2.542675in}}{\pgfqpoint{1.319175in}{2.538557in}}%
\pgfpathcurveto{\pgfqpoint{1.323293in}{2.534439in}}{\pgfqpoint{1.328879in}{2.532125in}}{\pgfqpoint{1.334703in}{2.532125in}}%
\pgfpathlineto{\pgfqpoint{1.334703in}{2.532125in}}%
\pgfpathclose%
\pgfusepath{stroke,fill}%
\end{pgfscope}%
\begin{pgfscope}%
\pgfpathrectangle{\pgfqpoint{0.997489in}{0.528000in}}{\pgfqpoint{4.565023in}{3.696000in}}%
\pgfusepath{clip}%
\pgfsetbuttcap%
\pgfsetroundjoin%
\definecolor{currentfill}{rgb}{0.200000,0.800000,0.200000}%
\pgfsetfillcolor{currentfill}%
\pgfsetlinewidth{1.003750pt}%
\definecolor{currentstroke}{rgb}{0.200000,0.800000,0.200000}%
\pgfsetstrokecolor{currentstroke}%
\pgfsetdash{}{0pt}%
\pgfpathmoveto{\pgfqpoint{1.398371in}{2.503283in}}%
\pgfpathcurveto{\pgfqpoint{1.404195in}{2.503283in}}{\pgfqpoint{1.409781in}{2.505597in}}{\pgfqpoint{1.413900in}{2.509715in}}%
\pgfpathcurveto{\pgfqpoint{1.418018in}{2.513833in}}{\pgfqpoint{1.420332in}{2.519419in}}{\pgfqpoint{1.420332in}{2.525243in}}%
\pgfpathcurveto{\pgfqpoint{1.420332in}{2.531067in}}{\pgfqpoint{1.418018in}{2.536653in}}{\pgfqpoint{1.413900in}{2.540772in}}%
\pgfpathcurveto{\pgfqpoint{1.409781in}{2.544890in}}{\pgfqpoint{1.404195in}{2.547204in}}{\pgfqpoint{1.398371in}{2.547204in}}%
\pgfpathcurveto{\pgfqpoint{1.392547in}{2.547204in}}{\pgfqpoint{1.386961in}{2.544890in}}{\pgfqpoint{1.382843in}{2.540772in}}%
\pgfpathcurveto{\pgfqpoint{1.378725in}{2.536653in}}{\pgfqpoint{1.376411in}{2.531067in}}{\pgfqpoint{1.376411in}{2.525243in}}%
\pgfpathcurveto{\pgfqpoint{1.376411in}{2.519419in}}{\pgfqpoint{1.378725in}{2.513833in}}{\pgfqpoint{1.382843in}{2.509715in}}%
\pgfpathcurveto{\pgfqpoint{1.386961in}{2.505597in}}{\pgfqpoint{1.392547in}{2.503283in}}{\pgfqpoint{1.398371in}{2.503283in}}%
\pgfpathlineto{\pgfqpoint{1.398371in}{2.503283in}}%
\pgfpathclose%
\pgfusepath{stroke,fill}%
\end{pgfscope}%
\begin{pgfscope}%
\pgfpathrectangle{\pgfqpoint{0.997489in}{0.528000in}}{\pgfqpoint{4.565023in}{3.696000in}}%
\pgfusepath{clip}%
\pgfsetbuttcap%
\pgfsetroundjoin%
\definecolor{currentfill}{rgb}{0.200000,0.800000,0.200000}%
\pgfsetfillcolor{currentfill}%
\pgfsetlinewidth{1.003750pt}%
\definecolor{currentstroke}{rgb}{0.200000,0.800000,0.200000}%
\pgfsetstrokecolor{currentstroke}%
\pgfsetdash{}{0pt}%
\pgfpathmoveto{\pgfqpoint{1.449200in}{2.471733in}}%
\pgfpathcurveto{\pgfqpoint{1.455023in}{2.471733in}}{\pgfqpoint{1.460610in}{2.474047in}}{\pgfqpoint{1.464728in}{2.478165in}}%
\pgfpathcurveto{\pgfqpoint{1.468846in}{2.482283in}}{\pgfqpoint{1.471160in}{2.487869in}}{\pgfqpoint{1.471160in}{2.493693in}}%
\pgfpathcurveto{\pgfqpoint{1.471160in}{2.499517in}}{\pgfqpoint{1.468846in}{2.505103in}}{\pgfqpoint{1.464728in}{2.509221in}}%
\pgfpathcurveto{\pgfqpoint{1.460610in}{2.513339in}}{\pgfqpoint{1.455023in}{2.515653in}}{\pgfqpoint{1.449200in}{2.515653in}}%
\pgfpathcurveto{\pgfqpoint{1.443376in}{2.515653in}}{\pgfqpoint{1.437789in}{2.513339in}}{\pgfqpoint{1.433671in}{2.509221in}}%
\pgfpathcurveto{\pgfqpoint{1.429553in}{2.505103in}}{\pgfqpoint{1.427239in}{2.499517in}}{\pgfqpoint{1.427239in}{2.493693in}}%
\pgfpathcurveto{\pgfqpoint{1.427239in}{2.487869in}}{\pgfqpoint{1.429553in}{2.482283in}}{\pgfqpoint{1.433671in}{2.478165in}}%
\pgfpathcurveto{\pgfqpoint{1.437789in}{2.474047in}}{\pgfqpoint{1.443376in}{2.471733in}}{\pgfqpoint{1.449200in}{2.471733in}}%
\pgfpathlineto{\pgfqpoint{1.449200in}{2.471733in}}%
\pgfpathclose%
\pgfusepath{stroke,fill}%
\end{pgfscope}%
\begin{pgfscope}%
\pgfpathrectangle{\pgfqpoint{0.997489in}{0.528000in}}{\pgfqpoint{4.565023in}{3.696000in}}%
\pgfusepath{clip}%
\pgfsetbuttcap%
\pgfsetroundjoin%
\definecolor{currentfill}{rgb}{0.200000,0.800000,0.200000}%
\pgfsetfillcolor{currentfill}%
\pgfsetlinewidth{1.003750pt}%
\definecolor{currentstroke}{rgb}{0.200000,0.800000,0.200000}%
\pgfsetstrokecolor{currentstroke}%
\pgfsetdash{}{0pt}%
\pgfpathmoveto{\pgfqpoint{1.432294in}{2.390785in}}%
\pgfpathcurveto{\pgfqpoint{1.438118in}{2.390785in}}{\pgfqpoint{1.443704in}{2.393099in}}{\pgfqpoint{1.447822in}{2.397217in}}%
\pgfpathcurveto{\pgfqpoint{1.451941in}{2.401335in}}{\pgfqpoint{1.454254in}{2.406921in}}{\pgfqpoint{1.454254in}{2.412745in}}%
\pgfpathcurveto{\pgfqpoint{1.454254in}{2.418569in}}{\pgfqpoint{1.451941in}{2.424155in}}{\pgfqpoint{1.447822in}{2.428273in}}%
\pgfpathcurveto{\pgfqpoint{1.443704in}{2.432391in}}{\pgfqpoint{1.438118in}{2.434705in}}{\pgfqpoint{1.432294in}{2.434705in}}%
\pgfpathcurveto{\pgfqpoint{1.426470in}{2.434705in}}{\pgfqpoint{1.420884in}{2.432391in}}{\pgfqpoint{1.416766in}{2.428273in}}%
\pgfpathcurveto{\pgfqpoint{1.412648in}{2.424155in}}{\pgfqpoint{1.410334in}{2.418569in}}{\pgfqpoint{1.410334in}{2.412745in}}%
\pgfpathcurveto{\pgfqpoint{1.410334in}{2.406921in}}{\pgfqpoint{1.412648in}{2.401335in}}{\pgfqpoint{1.416766in}{2.397217in}}%
\pgfpathcurveto{\pgfqpoint{1.420884in}{2.393099in}}{\pgfqpoint{1.426470in}{2.390785in}}{\pgfqpoint{1.432294in}{2.390785in}}%
\pgfpathlineto{\pgfqpoint{1.432294in}{2.390785in}}%
\pgfpathclose%
\pgfusepath{stroke,fill}%
\end{pgfscope}%
\begin{pgfscope}%
\pgfpathrectangle{\pgfqpoint{0.997489in}{0.528000in}}{\pgfqpoint{4.565023in}{3.696000in}}%
\pgfusepath{clip}%
\pgfsetbuttcap%
\pgfsetroundjoin%
\definecolor{currentfill}{rgb}{0.200000,0.800000,0.200000}%
\pgfsetfillcolor{currentfill}%
\pgfsetlinewidth{1.003750pt}%
\definecolor{currentstroke}{rgb}{0.200000,0.800000,0.200000}%
\pgfsetstrokecolor{currentstroke}%
\pgfsetdash{}{0pt}%
\pgfpathmoveto{\pgfqpoint{1.569526in}{2.436961in}}%
\pgfpathcurveto{\pgfqpoint{1.575350in}{2.436961in}}{\pgfqpoint{1.580936in}{2.439275in}}{\pgfqpoint{1.585054in}{2.443393in}}%
\pgfpathcurveto{\pgfqpoint{1.589173in}{2.447511in}}{\pgfqpoint{1.591486in}{2.453097in}}{\pgfqpoint{1.591486in}{2.458921in}}%
\pgfpathcurveto{\pgfqpoint{1.591486in}{2.464745in}}{\pgfqpoint{1.589173in}{2.470332in}}{\pgfqpoint{1.585054in}{2.474450in}}%
\pgfpathcurveto{\pgfqpoint{1.580936in}{2.478568in}}{\pgfqpoint{1.575350in}{2.480882in}}{\pgfqpoint{1.569526in}{2.480882in}}%
\pgfpathcurveto{\pgfqpoint{1.563702in}{2.480882in}}{\pgfqpoint{1.558116in}{2.478568in}}{\pgfqpoint{1.553998in}{2.474450in}}%
\pgfpathcurveto{\pgfqpoint{1.549880in}{2.470332in}}{\pgfqpoint{1.547566in}{2.464745in}}{\pgfqpoint{1.547566in}{2.458921in}}%
\pgfpathcurveto{\pgfqpoint{1.547566in}{2.453097in}}{\pgfqpoint{1.549880in}{2.447511in}}{\pgfqpoint{1.553998in}{2.443393in}}%
\pgfpathcurveto{\pgfqpoint{1.558116in}{2.439275in}}{\pgfqpoint{1.563702in}{2.436961in}}{\pgfqpoint{1.569526in}{2.436961in}}%
\pgfpathlineto{\pgfqpoint{1.569526in}{2.436961in}}%
\pgfpathclose%
\pgfusepath{stroke,fill}%
\end{pgfscope}%
\begin{pgfscope}%
\pgfpathrectangle{\pgfqpoint{0.997489in}{0.528000in}}{\pgfqpoint{4.565023in}{3.696000in}}%
\pgfusepath{clip}%
\pgfsetbuttcap%
\pgfsetroundjoin%
\definecolor{currentfill}{rgb}{0.200000,0.800000,0.200000}%
\pgfsetfillcolor{currentfill}%
\pgfsetlinewidth{1.003750pt}%
\definecolor{currentstroke}{rgb}{0.200000,0.800000,0.200000}%
\pgfsetstrokecolor{currentstroke}%
\pgfsetdash{}{0pt}%
\pgfpathmoveto{\pgfqpoint{1.630239in}{2.429456in}}%
\pgfpathcurveto{\pgfqpoint{1.636063in}{2.429456in}}{\pgfqpoint{1.641649in}{2.431770in}}{\pgfqpoint{1.645767in}{2.435888in}}%
\pgfpathcurveto{\pgfqpoint{1.649885in}{2.440006in}}{\pgfqpoint{1.652199in}{2.445593in}}{\pgfqpoint{1.652199in}{2.451417in}}%
\pgfpathcurveto{\pgfqpoint{1.652199in}{2.457240in}}{\pgfqpoint{1.649885in}{2.462827in}}{\pgfqpoint{1.645767in}{2.466945in}}%
\pgfpathcurveto{\pgfqpoint{1.641649in}{2.471063in}}{\pgfqpoint{1.636063in}{2.473377in}}{\pgfqpoint{1.630239in}{2.473377in}}%
\pgfpathcurveto{\pgfqpoint{1.624415in}{2.473377in}}{\pgfqpoint{1.618829in}{2.471063in}}{\pgfqpoint{1.614711in}{2.466945in}}%
\pgfpathcurveto{\pgfqpoint{1.610593in}{2.462827in}}{\pgfqpoint{1.608279in}{2.457240in}}{\pgfqpoint{1.608279in}{2.451417in}}%
\pgfpathcurveto{\pgfqpoint{1.608279in}{2.445593in}}{\pgfqpoint{1.610593in}{2.440006in}}{\pgfqpoint{1.614711in}{2.435888in}}%
\pgfpathcurveto{\pgfqpoint{1.618829in}{2.431770in}}{\pgfqpoint{1.624415in}{2.429456in}}{\pgfqpoint{1.630239in}{2.429456in}}%
\pgfpathlineto{\pgfqpoint{1.630239in}{2.429456in}}%
\pgfpathclose%
\pgfusepath{stroke,fill}%
\end{pgfscope}%
\begin{pgfscope}%
\pgfpathrectangle{\pgfqpoint{0.997489in}{0.528000in}}{\pgfqpoint{4.565023in}{3.696000in}}%
\pgfusepath{clip}%
\pgfsetbuttcap%
\pgfsetroundjoin%
\definecolor{currentfill}{rgb}{0.200000,0.800000,0.200000}%
\pgfsetfillcolor{currentfill}%
\pgfsetlinewidth{1.003750pt}%
\definecolor{currentstroke}{rgb}{0.200000,0.800000,0.200000}%
\pgfsetstrokecolor{currentstroke}%
\pgfsetdash{}{0pt}%
\pgfpathmoveto{\pgfqpoint{1.630425in}{2.359313in}}%
\pgfpathcurveto{\pgfqpoint{1.636249in}{2.359313in}}{\pgfqpoint{1.641835in}{2.361626in}}{\pgfqpoint{1.645953in}{2.365745in}}%
\pgfpathcurveto{\pgfqpoint{1.650071in}{2.369863in}}{\pgfqpoint{1.652385in}{2.375449in}}{\pgfqpoint{1.652385in}{2.381273in}}%
\pgfpathcurveto{\pgfqpoint{1.652385in}{2.387097in}}{\pgfqpoint{1.650071in}{2.392683in}}{\pgfqpoint{1.645953in}{2.396801in}}%
\pgfpathcurveto{\pgfqpoint{1.641835in}{2.400919in}}{\pgfqpoint{1.636249in}{2.403233in}}{\pgfqpoint{1.630425in}{2.403233in}}%
\pgfpathcurveto{\pgfqpoint{1.624601in}{2.403233in}}{\pgfqpoint{1.619014in}{2.400919in}}{\pgfqpoint{1.614896in}{2.396801in}}%
\pgfpathcurveto{\pgfqpoint{1.610778in}{2.392683in}}{\pgfqpoint{1.608464in}{2.387097in}}{\pgfqpoint{1.608464in}{2.381273in}}%
\pgfpathcurveto{\pgfqpoint{1.608464in}{2.375449in}}{\pgfqpoint{1.610778in}{2.369863in}}{\pgfqpoint{1.614896in}{2.365745in}}%
\pgfpathcurveto{\pgfqpoint{1.619014in}{2.361626in}}{\pgfqpoint{1.624601in}{2.359313in}}{\pgfqpoint{1.630425in}{2.359313in}}%
\pgfpathlineto{\pgfqpoint{1.630425in}{2.359313in}}%
\pgfpathclose%
\pgfusepath{stroke,fill}%
\end{pgfscope}%
\begin{pgfscope}%
\pgfpathrectangle{\pgfqpoint{0.997489in}{0.528000in}}{\pgfqpoint{4.565023in}{3.696000in}}%
\pgfusepath{clip}%
\pgfsetbuttcap%
\pgfsetroundjoin%
\definecolor{currentfill}{rgb}{0.200000,0.800000,0.200000}%
\pgfsetfillcolor{currentfill}%
\pgfsetlinewidth{1.003750pt}%
\definecolor{currentstroke}{rgb}{0.200000,0.800000,0.200000}%
\pgfsetstrokecolor{currentstroke}%
\pgfsetdash{}{0pt}%
\pgfpathmoveto{\pgfqpoint{1.629746in}{2.276955in}}%
\pgfpathcurveto{\pgfqpoint{1.635570in}{2.276955in}}{\pgfqpoint{1.641156in}{2.279269in}}{\pgfqpoint{1.645274in}{2.283387in}}%
\pgfpathcurveto{\pgfqpoint{1.649392in}{2.287505in}}{\pgfqpoint{1.651706in}{2.293091in}}{\pgfqpoint{1.651706in}{2.298915in}}%
\pgfpathcurveto{\pgfqpoint{1.651706in}{2.304739in}}{\pgfqpoint{1.649392in}{2.310325in}}{\pgfqpoint{1.645274in}{2.314443in}}%
\pgfpathcurveto{\pgfqpoint{1.641156in}{2.318562in}}{\pgfqpoint{1.635570in}{2.320875in}}{\pgfqpoint{1.629746in}{2.320875in}}%
\pgfpathcurveto{\pgfqpoint{1.623922in}{2.320875in}}{\pgfqpoint{1.618335in}{2.318562in}}{\pgfqpoint{1.614217in}{2.314443in}}%
\pgfpathcurveto{\pgfqpoint{1.610099in}{2.310325in}}{\pgfqpoint{1.607785in}{2.304739in}}{\pgfqpoint{1.607785in}{2.298915in}}%
\pgfpathcurveto{\pgfqpoint{1.607785in}{2.293091in}}{\pgfqpoint{1.610099in}{2.287505in}}{\pgfqpoint{1.614217in}{2.283387in}}%
\pgfpathcurveto{\pgfqpoint{1.618335in}{2.279269in}}{\pgfqpoint{1.623922in}{2.276955in}}{\pgfqpoint{1.629746in}{2.276955in}}%
\pgfpathlineto{\pgfqpoint{1.629746in}{2.276955in}}%
\pgfpathclose%
\pgfusepath{stroke,fill}%
\end{pgfscope}%
\begin{pgfscope}%
\pgfpathrectangle{\pgfqpoint{0.997489in}{0.528000in}}{\pgfqpoint{4.565023in}{3.696000in}}%
\pgfusepath{clip}%
\pgfsetbuttcap%
\pgfsetroundjoin%
\definecolor{currentfill}{rgb}{0.200000,0.800000,0.200000}%
\pgfsetfillcolor{currentfill}%
\pgfsetlinewidth{1.003750pt}%
\definecolor{currentstroke}{rgb}{0.200000,0.800000,0.200000}%
\pgfsetstrokecolor{currentstroke}%
\pgfsetdash{}{0pt}%
\pgfpathmoveto{\pgfqpoint{1.652216in}{2.214291in}}%
\pgfpathcurveto{\pgfqpoint{1.658040in}{2.214291in}}{\pgfqpoint{1.663627in}{2.216604in}}{\pgfqpoint{1.667745in}{2.220723in}}%
\pgfpathcurveto{\pgfqpoint{1.671863in}{2.224841in}}{\pgfqpoint{1.674177in}{2.230427in}}{\pgfqpoint{1.674177in}{2.236251in}}%
\pgfpathcurveto{\pgfqpoint{1.674177in}{2.242075in}}{\pgfqpoint{1.671863in}{2.247661in}}{\pgfqpoint{1.667745in}{2.251779in}}%
\pgfpathcurveto{\pgfqpoint{1.663627in}{2.255897in}}{\pgfqpoint{1.658040in}{2.258211in}}{\pgfqpoint{1.652216in}{2.258211in}}%
\pgfpathcurveto{\pgfqpoint{1.646393in}{2.258211in}}{\pgfqpoint{1.640806in}{2.255897in}}{\pgfqpoint{1.636688in}{2.251779in}}%
\pgfpathcurveto{\pgfqpoint{1.632570in}{2.247661in}}{\pgfqpoint{1.630256in}{2.242075in}}{\pgfqpoint{1.630256in}{2.236251in}}%
\pgfpathcurveto{\pgfqpoint{1.630256in}{2.230427in}}{\pgfqpoint{1.632570in}{2.224841in}}{\pgfqpoint{1.636688in}{2.220723in}}%
\pgfpathcurveto{\pgfqpoint{1.640806in}{2.216604in}}{\pgfqpoint{1.646393in}{2.214291in}}{\pgfqpoint{1.652216in}{2.214291in}}%
\pgfpathlineto{\pgfqpoint{1.652216in}{2.214291in}}%
\pgfpathclose%
\pgfusepath{stroke,fill}%
\end{pgfscope}%
\begin{pgfscope}%
\pgfpathrectangle{\pgfqpoint{0.997489in}{0.528000in}}{\pgfqpoint{4.565023in}{3.696000in}}%
\pgfusepath{clip}%
\pgfsetbuttcap%
\pgfsetroundjoin%
\definecolor{currentfill}{rgb}{0.200000,0.800000,0.200000}%
\pgfsetfillcolor{currentfill}%
\pgfsetlinewidth{1.003750pt}%
\definecolor{currentstroke}{rgb}{0.200000,0.800000,0.200000}%
\pgfsetstrokecolor{currentstroke}%
\pgfsetdash{}{0pt}%
\pgfpathmoveto{\pgfqpoint{1.778915in}{2.322076in}}%
\pgfpathcurveto{\pgfqpoint{1.784739in}{2.322076in}}{\pgfqpoint{1.790325in}{2.324390in}}{\pgfqpoint{1.794443in}{2.328508in}}%
\pgfpathcurveto{\pgfqpoint{1.798562in}{2.332626in}}{\pgfqpoint{1.800875in}{2.338212in}}{\pgfqpoint{1.800875in}{2.344036in}}%
\pgfpathcurveto{\pgfqpoint{1.800875in}{2.349860in}}{\pgfqpoint{1.798562in}{2.355446in}}{\pgfqpoint{1.794443in}{2.359564in}}%
\pgfpathcurveto{\pgfqpoint{1.790325in}{2.363683in}}{\pgfqpoint{1.784739in}{2.365996in}}{\pgfqpoint{1.778915in}{2.365996in}}%
\pgfpathcurveto{\pgfqpoint{1.773091in}{2.365996in}}{\pgfqpoint{1.767505in}{2.363683in}}{\pgfqpoint{1.763387in}{2.359564in}}%
\pgfpathcurveto{\pgfqpoint{1.759269in}{2.355446in}}{\pgfqpoint{1.756955in}{2.349860in}}{\pgfqpoint{1.756955in}{2.344036in}}%
\pgfpathcurveto{\pgfqpoint{1.756955in}{2.338212in}}{\pgfqpoint{1.759269in}{2.332626in}}{\pgfqpoint{1.763387in}{2.328508in}}%
\pgfpathcurveto{\pgfqpoint{1.767505in}{2.324390in}}{\pgfqpoint{1.773091in}{2.322076in}}{\pgfqpoint{1.778915in}{2.322076in}}%
\pgfpathlineto{\pgfqpoint{1.778915in}{2.322076in}}%
\pgfpathclose%
\pgfusepath{stroke,fill}%
\end{pgfscope}%
\begin{pgfscope}%
\pgfpathrectangle{\pgfqpoint{0.997489in}{0.528000in}}{\pgfqpoint{4.565023in}{3.696000in}}%
\pgfusepath{clip}%
\pgfsetbuttcap%
\pgfsetroundjoin%
\definecolor{currentfill}{rgb}{0.200000,0.800000,0.200000}%
\pgfsetfillcolor{currentfill}%
\pgfsetlinewidth{1.003750pt}%
\definecolor{currentstroke}{rgb}{0.200000,0.800000,0.200000}%
\pgfsetstrokecolor{currentstroke}%
\pgfsetdash{}{0pt}%
\pgfpathmoveto{\pgfqpoint{1.758181in}{2.177146in}}%
\pgfpathcurveto{\pgfqpoint{1.764005in}{2.177146in}}{\pgfqpoint{1.769591in}{2.179460in}}{\pgfqpoint{1.773709in}{2.183578in}}%
\pgfpathcurveto{\pgfqpoint{1.777828in}{2.187696in}}{\pgfqpoint{1.780141in}{2.193282in}}{\pgfqpoint{1.780141in}{2.199106in}}%
\pgfpathcurveto{\pgfqpoint{1.780141in}{2.204930in}}{\pgfqpoint{1.777828in}{2.210516in}}{\pgfqpoint{1.773709in}{2.214634in}}%
\pgfpathcurveto{\pgfqpoint{1.769591in}{2.218752in}}{\pgfqpoint{1.764005in}{2.221066in}}{\pgfqpoint{1.758181in}{2.221066in}}%
\pgfpathcurveto{\pgfqpoint{1.752357in}{2.221066in}}{\pgfqpoint{1.746771in}{2.218752in}}{\pgfqpoint{1.742653in}{2.214634in}}%
\pgfpathcurveto{\pgfqpoint{1.738535in}{2.210516in}}{\pgfqpoint{1.736221in}{2.204930in}}{\pgfqpoint{1.736221in}{2.199106in}}%
\pgfpathcurveto{\pgfqpoint{1.736221in}{2.193282in}}{\pgfqpoint{1.738535in}{2.187696in}}{\pgfqpoint{1.742653in}{2.183578in}}%
\pgfpathcurveto{\pgfqpoint{1.746771in}{2.179460in}}{\pgfqpoint{1.752357in}{2.177146in}}{\pgfqpoint{1.758181in}{2.177146in}}%
\pgfpathlineto{\pgfqpoint{1.758181in}{2.177146in}}%
\pgfpathclose%
\pgfusepath{stroke,fill}%
\end{pgfscope}%
\begin{pgfscope}%
\pgfpathrectangle{\pgfqpoint{0.997489in}{0.528000in}}{\pgfqpoint{4.565023in}{3.696000in}}%
\pgfusepath{clip}%
\pgfsetbuttcap%
\pgfsetroundjoin%
\definecolor{currentfill}{rgb}{0.200000,0.800000,0.200000}%
\pgfsetfillcolor{currentfill}%
\pgfsetlinewidth{1.003750pt}%
\definecolor{currentstroke}{rgb}{0.200000,0.800000,0.200000}%
\pgfsetstrokecolor{currentstroke}%
\pgfsetdash{}{0pt}%
\pgfpathmoveto{\pgfqpoint{1.814645in}{2.170858in}}%
\pgfpathcurveto{\pgfqpoint{1.820469in}{2.170858in}}{\pgfqpoint{1.826055in}{2.173172in}}{\pgfqpoint{1.830173in}{2.177290in}}%
\pgfpathcurveto{\pgfqpoint{1.834291in}{2.181409in}}{\pgfqpoint{1.836605in}{2.186995in}}{\pgfqpoint{1.836605in}{2.192819in}}%
\pgfpathcurveto{\pgfqpoint{1.836605in}{2.198643in}}{\pgfqpoint{1.834291in}{2.204229in}}{\pgfqpoint{1.830173in}{2.208347in}}%
\pgfpathcurveto{\pgfqpoint{1.826055in}{2.212465in}}{\pgfqpoint{1.820469in}{2.214779in}}{\pgfqpoint{1.814645in}{2.214779in}}%
\pgfpathcurveto{\pgfqpoint{1.808821in}{2.214779in}}{\pgfqpoint{1.803235in}{2.212465in}}{\pgfqpoint{1.799117in}{2.208347in}}%
\pgfpathcurveto{\pgfqpoint{1.794999in}{2.204229in}}{\pgfqpoint{1.792685in}{2.198643in}}{\pgfqpoint{1.792685in}{2.192819in}}%
\pgfpathcurveto{\pgfqpoint{1.792685in}{2.186995in}}{\pgfqpoint{1.794999in}{2.181409in}}{\pgfqpoint{1.799117in}{2.177290in}}%
\pgfpathcurveto{\pgfqpoint{1.803235in}{2.173172in}}{\pgfqpoint{1.808821in}{2.170858in}}{\pgfqpoint{1.814645in}{2.170858in}}%
\pgfpathlineto{\pgfqpoint{1.814645in}{2.170858in}}%
\pgfpathclose%
\pgfusepath{stroke,fill}%
\end{pgfscope}%
\begin{pgfscope}%
\pgfpathrectangle{\pgfqpoint{0.997489in}{0.528000in}}{\pgfqpoint{4.565023in}{3.696000in}}%
\pgfusepath{clip}%
\pgfsetbuttcap%
\pgfsetroundjoin%
\definecolor{currentfill}{rgb}{0.200000,0.800000,0.200000}%
\pgfsetfillcolor{currentfill}%
\pgfsetlinewidth{1.003750pt}%
\definecolor{currentstroke}{rgb}{0.200000,0.800000,0.200000}%
\pgfsetstrokecolor{currentstroke}%
\pgfsetdash{}{0pt}%
\pgfpathmoveto{\pgfqpoint{1.856220in}{2.127763in}}%
\pgfpathcurveto{\pgfqpoint{1.862044in}{2.127763in}}{\pgfqpoint{1.867630in}{2.130077in}}{\pgfqpoint{1.871748in}{2.134195in}}%
\pgfpathcurveto{\pgfqpoint{1.875866in}{2.138313in}}{\pgfqpoint{1.878180in}{2.143899in}}{\pgfqpoint{1.878180in}{2.149723in}}%
\pgfpathcurveto{\pgfqpoint{1.878180in}{2.155547in}}{\pgfqpoint{1.875866in}{2.161133in}}{\pgfqpoint{1.871748in}{2.165251in}}%
\pgfpathcurveto{\pgfqpoint{1.867630in}{2.169369in}}{\pgfqpoint{1.862044in}{2.171683in}}{\pgfqpoint{1.856220in}{2.171683in}}%
\pgfpathcurveto{\pgfqpoint{1.850396in}{2.171683in}}{\pgfqpoint{1.844810in}{2.169369in}}{\pgfqpoint{1.840692in}{2.165251in}}%
\pgfpathcurveto{\pgfqpoint{1.836573in}{2.161133in}}{\pgfqpoint{1.834260in}{2.155547in}}{\pgfqpoint{1.834260in}{2.149723in}}%
\pgfpathcurveto{\pgfqpoint{1.834260in}{2.143899in}}{\pgfqpoint{1.836573in}{2.138313in}}{\pgfqpoint{1.840692in}{2.134195in}}%
\pgfpathcurveto{\pgfqpoint{1.844810in}{2.130077in}}{\pgfqpoint{1.850396in}{2.127763in}}{\pgfqpoint{1.856220in}{2.127763in}}%
\pgfpathlineto{\pgfqpoint{1.856220in}{2.127763in}}%
\pgfpathclose%
\pgfusepath{stroke,fill}%
\end{pgfscope}%
\begin{pgfscope}%
\pgfpathrectangle{\pgfqpoint{0.997489in}{0.528000in}}{\pgfqpoint{4.565023in}{3.696000in}}%
\pgfusepath{clip}%
\pgfsetbuttcap%
\pgfsetroundjoin%
\definecolor{currentfill}{rgb}{0.200000,0.800000,0.200000}%
\pgfsetfillcolor{currentfill}%
\pgfsetlinewidth{1.003750pt}%
\definecolor{currentstroke}{rgb}{0.200000,0.800000,0.200000}%
\pgfsetstrokecolor{currentstroke}%
\pgfsetdash{}{0pt}%
\pgfpathmoveto{\pgfqpoint{1.936952in}{2.213388in}}%
\pgfpathcurveto{\pgfqpoint{1.942776in}{2.213388in}}{\pgfqpoint{1.948362in}{2.215702in}}{\pgfqpoint{1.952480in}{2.219820in}}%
\pgfpathcurveto{\pgfqpoint{1.956599in}{2.223938in}}{\pgfqpoint{1.958912in}{2.229524in}}{\pgfqpoint{1.958912in}{2.235348in}}%
\pgfpathcurveto{\pgfqpoint{1.958912in}{2.241172in}}{\pgfqpoint{1.956599in}{2.246758in}}{\pgfqpoint{1.952480in}{2.250877in}}%
\pgfpathcurveto{\pgfqpoint{1.948362in}{2.254995in}}{\pgfqpoint{1.942776in}{2.257309in}}{\pgfqpoint{1.936952in}{2.257309in}}%
\pgfpathcurveto{\pgfqpoint{1.931128in}{2.257309in}}{\pgfqpoint{1.925542in}{2.254995in}}{\pgfqpoint{1.921424in}{2.250877in}}%
\pgfpathcurveto{\pgfqpoint{1.917306in}{2.246758in}}{\pgfqpoint{1.914992in}{2.241172in}}{\pgfqpoint{1.914992in}{2.235348in}}%
\pgfpathcurveto{\pgfqpoint{1.914992in}{2.229524in}}{\pgfqpoint{1.917306in}{2.223938in}}{\pgfqpoint{1.921424in}{2.219820in}}%
\pgfpathcurveto{\pgfqpoint{1.925542in}{2.215702in}}{\pgfqpoint{1.931128in}{2.213388in}}{\pgfqpoint{1.936952in}{2.213388in}}%
\pgfpathlineto{\pgfqpoint{1.936952in}{2.213388in}}%
\pgfpathclose%
\pgfusepath{stroke,fill}%
\end{pgfscope}%
\begin{pgfscope}%
\pgfpathrectangle{\pgfqpoint{0.997489in}{0.528000in}}{\pgfqpoint{4.565023in}{3.696000in}}%
\pgfusepath{clip}%
\pgfsetbuttcap%
\pgfsetroundjoin%
\definecolor{currentfill}{rgb}{0.200000,0.800000,0.200000}%
\pgfsetfillcolor{currentfill}%
\pgfsetlinewidth{1.003750pt}%
\definecolor{currentstroke}{rgb}{0.200000,0.800000,0.200000}%
\pgfsetstrokecolor{currentstroke}%
\pgfsetdash{}{0pt}%
\pgfpathmoveto{\pgfqpoint{1.935973in}{1.972939in}}%
\pgfpathcurveto{\pgfqpoint{1.941797in}{1.972939in}}{\pgfqpoint{1.947383in}{1.975253in}}{\pgfqpoint{1.951501in}{1.979371in}}%
\pgfpathcurveto{\pgfqpoint{1.955619in}{1.983489in}}{\pgfqpoint{1.957933in}{1.989075in}}{\pgfqpoint{1.957933in}{1.994899in}}%
\pgfpathcurveto{\pgfqpoint{1.957933in}{2.000723in}}{\pgfqpoint{1.955619in}{2.006309in}}{\pgfqpoint{1.951501in}{2.010428in}}%
\pgfpathcurveto{\pgfqpoint{1.947383in}{2.014546in}}{\pgfqpoint{1.941797in}{2.016860in}}{\pgfqpoint{1.935973in}{2.016860in}}%
\pgfpathcurveto{\pgfqpoint{1.930149in}{2.016860in}}{\pgfqpoint{1.924563in}{2.014546in}}{\pgfqpoint{1.920445in}{2.010428in}}%
\pgfpathcurveto{\pgfqpoint{1.916326in}{2.006309in}}{\pgfqpoint{1.914013in}{2.000723in}}{\pgfqpoint{1.914013in}{1.994899in}}%
\pgfpathcurveto{\pgfqpoint{1.914013in}{1.989075in}}{\pgfqpoint{1.916326in}{1.983489in}}{\pgfqpoint{1.920445in}{1.979371in}}%
\pgfpathcurveto{\pgfqpoint{1.924563in}{1.975253in}}{\pgfqpoint{1.930149in}{1.972939in}}{\pgfqpoint{1.935973in}{1.972939in}}%
\pgfpathlineto{\pgfqpoint{1.935973in}{1.972939in}}%
\pgfpathclose%
\pgfusepath{stroke,fill}%
\end{pgfscope}%
\begin{pgfscope}%
\pgfpathrectangle{\pgfqpoint{0.997489in}{0.528000in}}{\pgfqpoint{4.565023in}{3.696000in}}%
\pgfusepath{clip}%
\pgfsetbuttcap%
\pgfsetroundjoin%
\definecolor{currentfill}{rgb}{0.200000,0.800000,0.200000}%
\pgfsetfillcolor{currentfill}%
\pgfsetlinewidth{1.003750pt}%
\definecolor{currentstroke}{rgb}{0.200000,0.800000,0.200000}%
\pgfsetstrokecolor{currentstroke}%
\pgfsetdash{}{0pt}%
\pgfpathmoveto{\pgfqpoint{2.017193in}{2.094247in}}%
\pgfpathcurveto{\pgfqpoint{2.023017in}{2.094247in}}{\pgfqpoint{2.028603in}{2.096561in}}{\pgfqpoint{2.032721in}{2.100679in}}%
\pgfpathcurveto{\pgfqpoint{2.036839in}{2.104798in}}{\pgfqpoint{2.039153in}{2.110384in}}{\pgfqpoint{2.039153in}{2.116208in}}%
\pgfpathcurveto{\pgfqpoint{2.039153in}{2.122032in}}{\pgfqpoint{2.036839in}{2.127618in}}{\pgfqpoint{2.032721in}{2.131736in}}%
\pgfpathcurveto{\pgfqpoint{2.028603in}{2.135854in}}{\pgfqpoint{2.023017in}{2.138168in}}{\pgfqpoint{2.017193in}{2.138168in}}%
\pgfpathcurveto{\pgfqpoint{2.011369in}{2.138168in}}{\pgfqpoint{2.005783in}{2.135854in}}{\pgfqpoint{2.001665in}{2.131736in}}%
\pgfpathcurveto{\pgfqpoint{1.997546in}{2.127618in}}{\pgfqpoint{1.995233in}{2.122032in}}{\pgfqpoint{1.995233in}{2.116208in}}%
\pgfpathcurveto{\pgfqpoint{1.995233in}{2.110384in}}{\pgfqpoint{1.997546in}{2.104798in}}{\pgfqpoint{2.001665in}{2.100679in}}%
\pgfpathcurveto{\pgfqpoint{2.005783in}{2.096561in}}{\pgfqpoint{2.011369in}{2.094247in}}{\pgfqpoint{2.017193in}{2.094247in}}%
\pgfpathlineto{\pgfqpoint{2.017193in}{2.094247in}}%
\pgfpathclose%
\pgfusepath{stroke,fill}%
\end{pgfscope}%
\begin{pgfscope}%
\pgfpathrectangle{\pgfqpoint{0.997489in}{0.528000in}}{\pgfqpoint{4.565023in}{3.696000in}}%
\pgfusepath{clip}%
\pgfsetbuttcap%
\pgfsetroundjoin%
\definecolor{currentfill}{rgb}{0.800000,0.200000,0.200000}%
\pgfsetfillcolor{currentfill}%
\pgfsetlinewidth{1.003750pt}%
\definecolor{currentstroke}{rgb}{0.800000,0.200000,0.200000}%
\pgfsetstrokecolor{currentstroke}%
\pgfsetdash{}{0pt}%
\pgfpathmoveto{\pgfqpoint{2.076661in}{2.152644in}}%
\pgfpathcurveto{\pgfqpoint{2.082485in}{2.152644in}}{\pgfqpoint{2.088072in}{2.154958in}}{\pgfqpoint{2.092190in}{2.159076in}}%
\pgfpathcurveto{\pgfqpoint{2.096308in}{2.163194in}}{\pgfqpoint{2.098622in}{2.168780in}}{\pgfqpoint{2.098622in}{2.174604in}}%
\pgfpathcurveto{\pgfqpoint{2.098622in}{2.180428in}}{\pgfqpoint{2.096308in}{2.186014in}}{\pgfqpoint{2.092190in}{2.190133in}}%
\pgfpathcurveto{\pgfqpoint{2.088072in}{2.194251in}}{\pgfqpoint{2.082485in}{2.196565in}}{\pgfqpoint{2.076661in}{2.196565in}}%
\pgfpathcurveto{\pgfqpoint{2.070838in}{2.196565in}}{\pgfqpoint{2.065251in}{2.194251in}}{\pgfqpoint{2.061133in}{2.190133in}}%
\pgfpathcurveto{\pgfqpoint{2.057015in}{2.186014in}}{\pgfqpoint{2.054701in}{2.180428in}}{\pgfqpoint{2.054701in}{2.174604in}}%
\pgfpathcurveto{\pgfqpoint{2.054701in}{2.168780in}}{\pgfqpoint{2.057015in}{2.163194in}}{\pgfqpoint{2.061133in}{2.159076in}}%
\pgfpathcurveto{\pgfqpoint{2.065251in}{2.154958in}}{\pgfqpoint{2.070838in}{2.152644in}}{\pgfqpoint{2.076661in}{2.152644in}}%
\pgfpathlineto{\pgfqpoint{2.076661in}{2.152644in}}%
\pgfpathclose%
\pgfusepath{stroke,fill}%
\end{pgfscope}%
\begin{pgfscope}%
\pgfpathrectangle{\pgfqpoint{0.997489in}{0.528000in}}{\pgfqpoint{4.565023in}{3.696000in}}%
\pgfusepath{clip}%
\pgfsetbuttcap%
\pgfsetroundjoin%
\definecolor{currentfill}{rgb}{0.800000,0.200000,0.200000}%
\pgfsetfillcolor{currentfill}%
\pgfsetlinewidth{1.003750pt}%
\definecolor{currentstroke}{rgb}{0.800000,0.200000,0.200000}%
\pgfsetstrokecolor{currentstroke}%
\pgfsetdash{}{0pt}%
\pgfpathmoveto{\pgfqpoint{2.126077in}{2.080570in}}%
\pgfpathcurveto{\pgfqpoint{2.131901in}{2.080570in}}{\pgfqpoint{2.137487in}{2.082884in}}{\pgfqpoint{2.141605in}{2.087002in}}%
\pgfpathcurveto{\pgfqpoint{2.145723in}{2.091120in}}{\pgfqpoint{2.148037in}{2.096706in}}{\pgfqpoint{2.148037in}{2.102530in}}%
\pgfpathcurveto{\pgfqpoint{2.148037in}{2.108354in}}{\pgfqpoint{2.145723in}{2.113940in}}{\pgfqpoint{2.141605in}{2.118058in}}%
\pgfpathcurveto{\pgfqpoint{2.137487in}{2.122177in}}{\pgfqpoint{2.131901in}{2.124490in}}{\pgfqpoint{2.126077in}{2.124490in}}%
\pgfpathcurveto{\pgfqpoint{2.120253in}{2.124490in}}{\pgfqpoint{2.114667in}{2.122177in}}{\pgfqpoint{2.110548in}{2.118058in}}%
\pgfpathcurveto{\pgfqpoint{2.106430in}{2.113940in}}{\pgfqpoint{2.104116in}{2.108354in}}{\pgfqpoint{2.104116in}{2.102530in}}%
\pgfpathcurveto{\pgfqpoint{2.104116in}{2.096706in}}{\pgfqpoint{2.106430in}{2.091120in}}{\pgfqpoint{2.110548in}{2.087002in}}%
\pgfpathcurveto{\pgfqpoint{2.114667in}{2.082884in}}{\pgfqpoint{2.120253in}{2.080570in}}{\pgfqpoint{2.126077in}{2.080570in}}%
\pgfpathlineto{\pgfqpoint{2.126077in}{2.080570in}}%
\pgfpathclose%
\pgfusepath{stroke,fill}%
\end{pgfscope}%
\begin{pgfscope}%
\pgfpathrectangle{\pgfqpoint{0.997489in}{0.528000in}}{\pgfqpoint{4.565023in}{3.696000in}}%
\pgfusepath{clip}%
\pgfsetbuttcap%
\pgfsetroundjoin%
\definecolor{currentfill}{rgb}{0.200000,0.800000,0.200000}%
\pgfsetfillcolor{currentfill}%
\pgfsetlinewidth{1.003750pt}%
\definecolor{currentstroke}{rgb}{0.200000,0.800000,0.200000}%
\pgfsetstrokecolor{currentstroke}%
\pgfsetdash{}{0pt}%
\pgfpathmoveto{\pgfqpoint{2.184431in}{2.010874in}}%
\pgfpathcurveto{\pgfqpoint{2.190255in}{2.010874in}}{\pgfqpoint{2.195841in}{2.013188in}}{\pgfqpoint{2.199959in}{2.017306in}}%
\pgfpathcurveto{\pgfqpoint{2.204077in}{2.021425in}}{\pgfqpoint{2.206391in}{2.027011in}}{\pgfqpoint{2.206391in}{2.032835in}}%
\pgfpathcurveto{\pgfqpoint{2.206391in}{2.038659in}}{\pgfqpoint{2.204077in}{2.044245in}}{\pgfqpoint{2.199959in}{2.048363in}}%
\pgfpathcurveto{\pgfqpoint{2.195841in}{2.052481in}}{\pgfqpoint{2.190255in}{2.054795in}}{\pgfqpoint{2.184431in}{2.054795in}}%
\pgfpathcurveto{\pgfqpoint{2.178607in}{2.054795in}}{\pgfqpoint{2.173021in}{2.052481in}}{\pgfqpoint{2.168903in}{2.048363in}}%
\pgfpathcurveto{\pgfqpoint{2.164784in}{2.044245in}}{\pgfqpoint{2.162471in}{2.038659in}}{\pgfqpoint{2.162471in}{2.032835in}}%
\pgfpathcurveto{\pgfqpoint{2.162471in}{2.027011in}}{\pgfqpoint{2.164784in}{2.021425in}}{\pgfqpoint{2.168903in}{2.017306in}}%
\pgfpathcurveto{\pgfqpoint{2.173021in}{2.013188in}}{\pgfqpoint{2.178607in}{2.010874in}}{\pgfqpoint{2.184431in}{2.010874in}}%
\pgfpathlineto{\pgfqpoint{2.184431in}{2.010874in}}%
\pgfpathclose%
\pgfusepath{stroke,fill}%
\end{pgfscope}%
\begin{pgfscope}%
\pgfpathrectangle{\pgfqpoint{0.997489in}{0.528000in}}{\pgfqpoint{4.565023in}{3.696000in}}%
\pgfusepath{clip}%
\pgfsetbuttcap%
\pgfsetroundjoin%
\definecolor{currentfill}{rgb}{0.800000,0.200000,0.200000}%
\pgfsetfillcolor{currentfill}%
\pgfsetlinewidth{1.003750pt}%
\definecolor{currentstroke}{rgb}{0.800000,0.200000,0.200000}%
\pgfsetstrokecolor{currentstroke}%
\pgfsetdash{}{0pt}%
\pgfpathmoveto{\pgfqpoint{2.227774in}{2.158621in}}%
\pgfpathcurveto{\pgfqpoint{2.233598in}{2.158621in}}{\pgfqpoint{2.239185in}{2.160934in}}{\pgfqpoint{2.243303in}{2.165053in}}%
\pgfpathcurveto{\pgfqpoint{2.247421in}{2.169171in}}{\pgfqpoint{2.249735in}{2.174757in}}{\pgfqpoint{2.249735in}{2.180581in}}%
\pgfpathcurveto{\pgfqpoint{2.249735in}{2.186405in}}{\pgfqpoint{2.247421in}{2.191991in}}{\pgfqpoint{2.243303in}{2.196109in}}%
\pgfpathcurveto{\pgfqpoint{2.239185in}{2.200227in}}{\pgfqpoint{2.233598in}{2.202541in}}{\pgfqpoint{2.227774in}{2.202541in}}%
\pgfpathcurveto{\pgfqpoint{2.221951in}{2.202541in}}{\pgfqpoint{2.216364in}{2.200227in}}{\pgfqpoint{2.212246in}{2.196109in}}%
\pgfpathcurveto{\pgfqpoint{2.208128in}{2.191991in}}{\pgfqpoint{2.205814in}{2.186405in}}{\pgfqpoint{2.205814in}{2.180581in}}%
\pgfpathcurveto{\pgfqpoint{2.205814in}{2.174757in}}{\pgfqpoint{2.208128in}{2.169171in}}{\pgfqpoint{2.212246in}{2.165053in}}%
\pgfpathcurveto{\pgfqpoint{2.216364in}{2.160934in}}{\pgfqpoint{2.221951in}{2.158621in}}{\pgfqpoint{2.227774in}{2.158621in}}%
\pgfpathlineto{\pgfqpoint{2.227774in}{2.158621in}}%
\pgfpathclose%
\pgfusepath{stroke,fill}%
\end{pgfscope}%
\begin{pgfscope}%
\pgfpathrectangle{\pgfqpoint{0.997489in}{0.528000in}}{\pgfqpoint{4.565023in}{3.696000in}}%
\pgfusepath{clip}%
\pgfsetbuttcap%
\pgfsetroundjoin%
\definecolor{currentfill}{rgb}{0.200000,0.800000,0.200000}%
\pgfsetfillcolor{currentfill}%
\pgfsetlinewidth{1.003750pt}%
\definecolor{currentstroke}{rgb}{0.200000,0.800000,0.200000}%
\pgfsetstrokecolor{currentstroke}%
\pgfsetdash{}{0pt}%
\pgfpathmoveto{\pgfqpoint{2.275742in}{2.176393in}}%
\pgfpathcurveto{\pgfqpoint{2.281566in}{2.176393in}}{\pgfqpoint{2.287152in}{2.178707in}}{\pgfqpoint{2.291270in}{2.182825in}}%
\pgfpathcurveto{\pgfqpoint{2.295388in}{2.186943in}}{\pgfqpoint{2.297702in}{2.192529in}}{\pgfqpoint{2.297702in}{2.198353in}}%
\pgfpathcurveto{\pgfqpoint{2.297702in}{2.204177in}}{\pgfqpoint{2.295388in}{2.209763in}}{\pgfqpoint{2.291270in}{2.213881in}}%
\pgfpathcurveto{\pgfqpoint{2.287152in}{2.217999in}}{\pgfqpoint{2.281566in}{2.220313in}}{\pgfqpoint{2.275742in}{2.220313in}}%
\pgfpathcurveto{\pgfqpoint{2.269918in}{2.220313in}}{\pgfqpoint{2.264332in}{2.217999in}}{\pgfqpoint{2.260213in}{2.213881in}}%
\pgfpathcurveto{\pgfqpoint{2.256095in}{2.209763in}}{\pgfqpoint{2.253781in}{2.204177in}}{\pgfqpoint{2.253781in}{2.198353in}}%
\pgfpathcurveto{\pgfqpoint{2.253781in}{2.192529in}}{\pgfqpoint{2.256095in}{2.186943in}}{\pgfqpoint{2.260213in}{2.182825in}}%
\pgfpathcurveto{\pgfqpoint{2.264332in}{2.178707in}}{\pgfqpoint{2.269918in}{2.176393in}}{\pgfqpoint{2.275742in}{2.176393in}}%
\pgfpathlineto{\pgfqpoint{2.275742in}{2.176393in}}%
\pgfpathclose%
\pgfusepath{stroke,fill}%
\end{pgfscope}%
\begin{pgfscope}%
\pgfpathrectangle{\pgfqpoint{0.997489in}{0.528000in}}{\pgfqpoint{4.565023in}{3.696000in}}%
\pgfusepath{clip}%
\pgfsetbuttcap%
\pgfsetroundjoin%
\definecolor{currentfill}{rgb}{0.200000,0.800000,0.200000}%
\pgfsetfillcolor{currentfill}%
\pgfsetlinewidth{1.003750pt}%
\definecolor{currentstroke}{rgb}{0.200000,0.800000,0.200000}%
\pgfsetstrokecolor{currentstroke}%
\pgfsetdash{}{0pt}%
\pgfpathmoveto{\pgfqpoint{2.341150in}{2.117337in}}%
\pgfpathcurveto{\pgfqpoint{2.346974in}{2.117337in}}{\pgfqpoint{2.352560in}{2.119651in}}{\pgfqpoint{2.356678in}{2.123769in}}%
\pgfpathcurveto{\pgfqpoint{2.360796in}{2.127887in}}{\pgfqpoint{2.363110in}{2.133473in}}{\pgfqpoint{2.363110in}{2.139297in}}%
\pgfpathcurveto{\pgfqpoint{2.363110in}{2.145121in}}{\pgfqpoint{2.360796in}{2.150707in}}{\pgfqpoint{2.356678in}{2.154825in}}%
\pgfpathcurveto{\pgfqpoint{2.352560in}{2.158943in}}{\pgfqpoint{2.346974in}{2.161257in}}{\pgfqpoint{2.341150in}{2.161257in}}%
\pgfpathcurveto{\pgfqpoint{2.335326in}{2.161257in}}{\pgfqpoint{2.329740in}{2.158943in}}{\pgfqpoint{2.325622in}{2.154825in}}%
\pgfpathcurveto{\pgfqpoint{2.321504in}{2.150707in}}{\pgfqpoint{2.319190in}{2.145121in}}{\pgfqpoint{2.319190in}{2.139297in}}%
\pgfpathcurveto{\pgfqpoint{2.319190in}{2.133473in}}{\pgfqpoint{2.321504in}{2.127887in}}{\pgfqpoint{2.325622in}{2.123769in}}%
\pgfpathcurveto{\pgfqpoint{2.329740in}{2.119651in}}{\pgfqpoint{2.335326in}{2.117337in}}{\pgfqpoint{2.341150in}{2.117337in}}%
\pgfpathlineto{\pgfqpoint{2.341150in}{2.117337in}}%
\pgfpathclose%
\pgfusepath{stroke,fill}%
\end{pgfscope}%
\begin{pgfscope}%
\pgfpathrectangle{\pgfqpoint{0.997489in}{0.528000in}}{\pgfqpoint{4.565023in}{3.696000in}}%
\pgfusepath{clip}%
\pgfsetbuttcap%
\pgfsetroundjoin%
\definecolor{currentfill}{rgb}{0.200000,0.800000,0.200000}%
\pgfsetfillcolor{currentfill}%
\pgfsetlinewidth{1.003750pt}%
\definecolor{currentstroke}{rgb}{0.200000,0.800000,0.200000}%
\pgfsetstrokecolor{currentstroke}%
\pgfsetdash{}{0pt}%
\pgfpathmoveto{\pgfqpoint{2.394689in}{2.127560in}}%
\pgfpathcurveto{\pgfqpoint{2.400513in}{2.127560in}}{\pgfqpoint{2.406099in}{2.129874in}}{\pgfqpoint{2.410217in}{2.133992in}}%
\pgfpathcurveto{\pgfqpoint{2.414335in}{2.138110in}}{\pgfqpoint{2.416649in}{2.143696in}}{\pgfqpoint{2.416649in}{2.149520in}}%
\pgfpathcurveto{\pgfqpoint{2.416649in}{2.155344in}}{\pgfqpoint{2.414335in}{2.160930in}}{\pgfqpoint{2.410217in}{2.165048in}}%
\pgfpathcurveto{\pgfqpoint{2.406099in}{2.169166in}}{\pgfqpoint{2.400513in}{2.171480in}}{\pgfqpoint{2.394689in}{2.171480in}}%
\pgfpathcurveto{\pgfqpoint{2.388865in}{2.171480in}}{\pgfqpoint{2.383278in}{2.169166in}}{\pgfqpoint{2.379160in}{2.165048in}}%
\pgfpathcurveto{\pgfqpoint{2.375042in}{2.160930in}}{\pgfqpoint{2.372728in}{2.155344in}}{\pgfqpoint{2.372728in}{2.149520in}}%
\pgfpathcurveto{\pgfqpoint{2.372728in}{2.143696in}}{\pgfqpoint{2.375042in}{2.138110in}}{\pgfqpoint{2.379160in}{2.133992in}}%
\pgfpathcurveto{\pgfqpoint{2.383278in}{2.129874in}}{\pgfqpoint{2.388865in}{2.127560in}}{\pgfqpoint{2.394689in}{2.127560in}}%
\pgfpathlineto{\pgfqpoint{2.394689in}{2.127560in}}%
\pgfpathclose%
\pgfusepath{stroke,fill}%
\end{pgfscope}%
\begin{pgfscope}%
\pgfpathrectangle{\pgfqpoint{0.997489in}{0.528000in}}{\pgfqpoint{4.565023in}{3.696000in}}%
\pgfusepath{clip}%
\pgfsetbuttcap%
\pgfsetroundjoin%
\definecolor{currentfill}{rgb}{0.200000,0.800000,0.200000}%
\pgfsetfillcolor{currentfill}%
\pgfsetlinewidth{1.003750pt}%
\definecolor{currentstroke}{rgb}{0.200000,0.800000,0.200000}%
\pgfsetstrokecolor{currentstroke}%
\pgfsetdash{}{0pt}%
\pgfpathmoveto{\pgfqpoint{2.459864in}{2.109505in}}%
\pgfpathcurveto{\pgfqpoint{2.465688in}{2.109505in}}{\pgfqpoint{2.471274in}{2.111819in}}{\pgfqpoint{2.475392in}{2.115937in}}%
\pgfpathcurveto{\pgfqpoint{2.479510in}{2.120055in}}{\pgfqpoint{2.481824in}{2.125641in}}{\pgfqpoint{2.481824in}{2.131465in}}%
\pgfpathcurveto{\pgfqpoint{2.481824in}{2.137289in}}{\pgfqpoint{2.479510in}{2.142875in}}{\pgfqpoint{2.475392in}{2.146993in}}%
\pgfpathcurveto{\pgfqpoint{2.471274in}{2.151112in}}{\pgfqpoint{2.465688in}{2.153425in}}{\pgfqpoint{2.459864in}{2.153425in}}%
\pgfpathcurveto{\pgfqpoint{2.454040in}{2.153425in}}{\pgfqpoint{2.448454in}{2.151112in}}{\pgfqpoint{2.444335in}{2.146993in}}%
\pgfpathcurveto{\pgfqpoint{2.440217in}{2.142875in}}{\pgfqpoint{2.437903in}{2.137289in}}{\pgfqpoint{2.437903in}{2.131465in}}%
\pgfpathcurveto{\pgfqpoint{2.437903in}{2.125641in}}{\pgfqpoint{2.440217in}{2.120055in}}{\pgfqpoint{2.444335in}{2.115937in}}%
\pgfpathcurveto{\pgfqpoint{2.448454in}{2.111819in}}{\pgfqpoint{2.454040in}{2.109505in}}{\pgfqpoint{2.459864in}{2.109505in}}%
\pgfpathlineto{\pgfqpoint{2.459864in}{2.109505in}}%
\pgfpathclose%
\pgfusepath{stroke,fill}%
\end{pgfscope}%
\begin{pgfscope}%
\pgfpathrectangle{\pgfqpoint{0.997489in}{0.528000in}}{\pgfqpoint{4.565023in}{3.696000in}}%
\pgfusepath{clip}%
\pgfsetbuttcap%
\pgfsetroundjoin%
\definecolor{currentfill}{rgb}{0.200000,0.800000,0.200000}%
\pgfsetfillcolor{currentfill}%
\pgfsetlinewidth{1.003750pt}%
\definecolor{currentstroke}{rgb}{0.200000,0.800000,0.200000}%
\pgfsetstrokecolor{currentstroke}%
\pgfsetdash{}{0pt}%
\pgfpathmoveto{\pgfqpoint{2.528811in}{2.095442in}}%
\pgfpathcurveto{\pgfqpoint{2.534635in}{2.095442in}}{\pgfqpoint{2.540221in}{2.097756in}}{\pgfqpoint{2.544339in}{2.101874in}}%
\pgfpathcurveto{\pgfqpoint{2.548458in}{2.105992in}}{\pgfqpoint{2.550771in}{2.111579in}}{\pgfqpoint{2.550771in}{2.117402in}}%
\pgfpathcurveto{\pgfqpoint{2.550771in}{2.123226in}}{\pgfqpoint{2.548458in}{2.128813in}}{\pgfqpoint{2.544339in}{2.132931in}}%
\pgfpathcurveto{\pgfqpoint{2.540221in}{2.137049in}}{\pgfqpoint{2.534635in}{2.139363in}}{\pgfqpoint{2.528811in}{2.139363in}}%
\pgfpathcurveto{\pgfqpoint{2.522987in}{2.139363in}}{\pgfqpoint{2.517401in}{2.137049in}}{\pgfqpoint{2.513283in}{2.132931in}}%
\pgfpathcurveto{\pgfqpoint{2.509165in}{2.128813in}}{\pgfqpoint{2.506851in}{2.123226in}}{\pgfqpoint{2.506851in}{2.117402in}}%
\pgfpathcurveto{\pgfqpoint{2.506851in}{2.111579in}}{\pgfqpoint{2.509165in}{2.105992in}}{\pgfqpoint{2.513283in}{2.101874in}}%
\pgfpathcurveto{\pgfqpoint{2.517401in}{2.097756in}}{\pgfqpoint{2.522987in}{2.095442in}}{\pgfqpoint{2.528811in}{2.095442in}}%
\pgfpathlineto{\pgfqpoint{2.528811in}{2.095442in}}%
\pgfpathclose%
\pgfusepath{stroke,fill}%
\end{pgfscope}%
\begin{pgfscope}%
\pgfpathrectangle{\pgfqpoint{0.997489in}{0.528000in}}{\pgfqpoint{4.565023in}{3.696000in}}%
\pgfusepath{clip}%
\pgfsetbuttcap%
\pgfsetroundjoin%
\definecolor{currentfill}{rgb}{0.200000,0.800000,0.200000}%
\pgfsetfillcolor{currentfill}%
\pgfsetlinewidth{1.003750pt}%
\definecolor{currentstroke}{rgb}{0.200000,0.800000,0.200000}%
\pgfsetstrokecolor{currentstroke}%
\pgfsetdash{}{0pt}%
\pgfpathmoveto{\pgfqpoint{2.565059in}{2.153539in}}%
\pgfpathcurveto{\pgfqpoint{2.570883in}{2.153539in}}{\pgfqpoint{2.576469in}{2.155853in}}{\pgfqpoint{2.580588in}{2.159971in}}%
\pgfpathcurveto{\pgfqpoint{2.584706in}{2.164089in}}{\pgfqpoint{2.587020in}{2.169675in}}{\pgfqpoint{2.587020in}{2.175499in}}%
\pgfpathcurveto{\pgfqpoint{2.587020in}{2.181323in}}{\pgfqpoint{2.584706in}{2.186909in}}{\pgfqpoint{2.580588in}{2.191027in}}%
\pgfpathcurveto{\pgfqpoint{2.576469in}{2.195145in}}{\pgfqpoint{2.570883in}{2.197459in}}{\pgfqpoint{2.565059in}{2.197459in}}%
\pgfpathcurveto{\pgfqpoint{2.559235in}{2.197459in}}{\pgfqpoint{2.553649in}{2.195145in}}{\pgfqpoint{2.549531in}{2.191027in}}%
\pgfpathcurveto{\pgfqpoint{2.545413in}{2.186909in}}{\pgfqpoint{2.543099in}{2.181323in}}{\pgfqpoint{2.543099in}{2.175499in}}%
\pgfpathcurveto{\pgfqpoint{2.543099in}{2.169675in}}{\pgfqpoint{2.545413in}{2.164089in}}{\pgfqpoint{2.549531in}{2.159971in}}%
\pgfpathcurveto{\pgfqpoint{2.553649in}{2.155853in}}{\pgfqpoint{2.559235in}{2.153539in}}{\pgfqpoint{2.565059in}{2.153539in}}%
\pgfpathlineto{\pgfqpoint{2.565059in}{2.153539in}}%
\pgfpathclose%
\pgfusepath{stroke,fill}%
\end{pgfscope}%
\begin{pgfscope}%
\pgfpathrectangle{\pgfqpoint{0.997489in}{0.528000in}}{\pgfqpoint{4.565023in}{3.696000in}}%
\pgfusepath{clip}%
\pgfsetbuttcap%
\pgfsetroundjoin%
\definecolor{currentfill}{rgb}{0.200000,0.800000,0.200000}%
\pgfsetfillcolor{currentfill}%
\pgfsetlinewidth{1.003750pt}%
\definecolor{currentstroke}{rgb}{0.200000,0.800000,0.200000}%
\pgfsetstrokecolor{currentstroke}%
\pgfsetdash{}{0pt}%
\pgfpathmoveto{\pgfqpoint{2.555407in}{2.277411in}}%
\pgfpathcurveto{\pgfqpoint{2.561231in}{2.277411in}}{\pgfqpoint{2.566817in}{2.279725in}}{\pgfqpoint{2.570935in}{2.283843in}}%
\pgfpathcurveto{\pgfqpoint{2.575053in}{2.287962in}}{\pgfqpoint{2.577367in}{2.293548in}}{\pgfqpoint{2.577367in}{2.299372in}}%
\pgfpathcurveto{\pgfqpoint{2.577367in}{2.305196in}}{\pgfqpoint{2.575053in}{2.310782in}}{\pgfqpoint{2.570935in}{2.314900in}}%
\pgfpathcurveto{\pgfqpoint{2.566817in}{2.319018in}}{\pgfqpoint{2.561231in}{2.321332in}}{\pgfqpoint{2.555407in}{2.321332in}}%
\pgfpathcurveto{\pgfqpoint{2.549583in}{2.321332in}}{\pgfqpoint{2.543996in}{2.319018in}}{\pgfqpoint{2.539878in}{2.314900in}}%
\pgfpathcurveto{\pgfqpoint{2.535760in}{2.310782in}}{\pgfqpoint{2.533446in}{2.305196in}}{\pgfqpoint{2.533446in}{2.299372in}}%
\pgfpathcurveto{\pgfqpoint{2.533446in}{2.293548in}}{\pgfqpoint{2.535760in}{2.287962in}}{\pgfqpoint{2.539878in}{2.283843in}}%
\pgfpathcurveto{\pgfqpoint{2.543996in}{2.279725in}}{\pgfqpoint{2.549583in}{2.277411in}}{\pgfqpoint{2.555407in}{2.277411in}}%
\pgfpathlineto{\pgfqpoint{2.555407in}{2.277411in}}%
\pgfpathclose%
\pgfusepath{stroke,fill}%
\end{pgfscope}%
\begin{pgfscope}%
\pgfpathrectangle{\pgfqpoint{0.997489in}{0.528000in}}{\pgfqpoint{4.565023in}{3.696000in}}%
\pgfusepath{clip}%
\pgfsetbuttcap%
\pgfsetroundjoin%
\definecolor{currentfill}{rgb}{0.200000,0.800000,0.200000}%
\pgfsetfillcolor{currentfill}%
\pgfsetlinewidth{1.003750pt}%
\definecolor{currentstroke}{rgb}{0.200000,0.800000,0.200000}%
\pgfsetstrokecolor{currentstroke}%
\pgfsetdash{}{0pt}%
\pgfpathmoveto{\pgfqpoint{2.606554in}{2.291772in}}%
\pgfpathcurveto{\pgfqpoint{2.612378in}{2.291772in}}{\pgfqpoint{2.617964in}{2.294086in}}{\pgfqpoint{2.622082in}{2.298204in}}%
\pgfpathcurveto{\pgfqpoint{2.626200in}{2.302322in}}{\pgfqpoint{2.628514in}{2.307908in}}{\pgfqpoint{2.628514in}{2.313732in}}%
\pgfpathcurveto{\pgfqpoint{2.628514in}{2.319556in}}{\pgfqpoint{2.626200in}{2.325142in}}{\pgfqpoint{2.622082in}{2.329260in}}%
\pgfpathcurveto{\pgfqpoint{2.617964in}{2.333378in}}{\pgfqpoint{2.612378in}{2.335692in}}{\pgfqpoint{2.606554in}{2.335692in}}%
\pgfpathcurveto{\pgfqpoint{2.600730in}{2.335692in}}{\pgfqpoint{2.595144in}{2.333378in}}{\pgfqpoint{2.591025in}{2.329260in}}%
\pgfpathcurveto{\pgfqpoint{2.586907in}{2.325142in}}{\pgfqpoint{2.584593in}{2.319556in}}{\pgfqpoint{2.584593in}{2.313732in}}%
\pgfpathcurveto{\pgfqpoint{2.584593in}{2.307908in}}{\pgfqpoint{2.586907in}{2.302322in}}{\pgfqpoint{2.591025in}{2.298204in}}%
\pgfpathcurveto{\pgfqpoint{2.595144in}{2.294086in}}{\pgfqpoint{2.600730in}{2.291772in}}{\pgfqpoint{2.606554in}{2.291772in}}%
\pgfpathlineto{\pgfqpoint{2.606554in}{2.291772in}}%
\pgfpathclose%
\pgfusepath{stroke,fill}%
\end{pgfscope}%
\begin{pgfscope}%
\pgfpathrectangle{\pgfqpoint{0.997489in}{0.528000in}}{\pgfqpoint{4.565023in}{3.696000in}}%
\pgfusepath{clip}%
\pgfsetbuttcap%
\pgfsetroundjoin%
\definecolor{currentfill}{rgb}{0.200000,0.200000,0.800000}%
\pgfsetfillcolor{currentfill}%
\pgfsetlinewidth{1.003750pt}%
\definecolor{currentstroke}{rgb}{0.200000,0.200000,0.800000}%
\pgfsetstrokecolor{currentstroke}%
\pgfsetdash{}{0pt}%
\pgfpathmoveto{\pgfqpoint{2.656082in}{2.311740in}}%
\pgfpathcurveto{\pgfqpoint{2.661906in}{2.311740in}}{\pgfqpoint{2.667492in}{2.314054in}}{\pgfqpoint{2.671610in}{2.318172in}}%
\pgfpathcurveto{\pgfqpoint{2.675728in}{2.322290in}}{\pgfqpoint{2.678042in}{2.327876in}}{\pgfqpoint{2.678042in}{2.333700in}}%
\pgfpathcurveto{\pgfqpoint{2.678042in}{2.339524in}}{\pgfqpoint{2.675728in}{2.345110in}}{\pgfqpoint{2.671610in}{2.349228in}}%
\pgfpathcurveto{\pgfqpoint{2.667492in}{2.353346in}}{\pgfqpoint{2.661906in}{2.355660in}}{\pgfqpoint{2.656082in}{2.355660in}}%
\pgfpathcurveto{\pgfqpoint{2.650258in}{2.355660in}}{\pgfqpoint{2.644672in}{2.353346in}}{\pgfqpoint{2.640554in}{2.349228in}}%
\pgfpathcurveto{\pgfqpoint{2.636436in}{2.345110in}}{\pgfqpoint{2.634122in}{2.339524in}}{\pgfqpoint{2.634122in}{2.333700in}}%
\pgfpathcurveto{\pgfqpoint{2.634122in}{2.327876in}}{\pgfqpoint{2.636436in}{2.322290in}}{\pgfqpoint{2.640554in}{2.318172in}}%
\pgfpathcurveto{\pgfqpoint{2.644672in}{2.314054in}}{\pgfqpoint{2.650258in}{2.311740in}}{\pgfqpoint{2.656082in}{2.311740in}}%
\pgfpathlineto{\pgfqpoint{2.656082in}{2.311740in}}%
\pgfpathclose%
\pgfusepath{stroke,fill}%
\end{pgfscope}%
\begin{pgfscope}%
\pgfpathrectangle{\pgfqpoint{0.997489in}{0.528000in}}{\pgfqpoint{4.565023in}{3.696000in}}%
\pgfusepath{clip}%
\pgfsetbuttcap%
\pgfsetroundjoin%
\definecolor{currentfill}{rgb}{0.200000,0.200000,0.800000}%
\pgfsetfillcolor{currentfill}%
\pgfsetlinewidth{1.003750pt}%
\definecolor{currentstroke}{rgb}{0.200000,0.200000,0.800000}%
\pgfsetstrokecolor{currentstroke}%
\pgfsetdash{}{0pt}%
\pgfpathmoveto{\pgfqpoint{2.750674in}{2.285861in}}%
\pgfpathcurveto{\pgfqpoint{2.756498in}{2.285861in}}{\pgfqpoint{2.762084in}{2.288175in}}{\pgfqpoint{2.766202in}{2.292293in}}%
\pgfpathcurveto{\pgfqpoint{2.770320in}{2.296411in}}{\pgfqpoint{2.772634in}{2.301997in}}{\pgfqpoint{2.772634in}{2.307821in}}%
\pgfpathcurveto{\pgfqpoint{2.772634in}{2.313645in}}{\pgfqpoint{2.770320in}{2.319231in}}{\pgfqpoint{2.766202in}{2.323349in}}%
\pgfpathcurveto{\pgfqpoint{2.762084in}{2.327467in}}{\pgfqpoint{2.756498in}{2.329781in}}{\pgfqpoint{2.750674in}{2.329781in}}%
\pgfpathcurveto{\pgfqpoint{2.744850in}{2.329781in}}{\pgfqpoint{2.739264in}{2.327467in}}{\pgfqpoint{2.735146in}{2.323349in}}%
\pgfpathcurveto{\pgfqpoint{2.731028in}{2.319231in}}{\pgfqpoint{2.728714in}{2.313645in}}{\pgfqpoint{2.728714in}{2.307821in}}%
\pgfpathcurveto{\pgfqpoint{2.728714in}{2.301997in}}{\pgfqpoint{2.731028in}{2.296411in}}{\pgfqpoint{2.735146in}{2.292293in}}%
\pgfpathcurveto{\pgfqpoint{2.739264in}{2.288175in}}{\pgfqpoint{2.744850in}{2.285861in}}{\pgfqpoint{2.750674in}{2.285861in}}%
\pgfpathlineto{\pgfqpoint{2.750674in}{2.285861in}}%
\pgfpathclose%
\pgfusepath{stroke,fill}%
\end{pgfscope}%
\begin{pgfscope}%
\pgfpathrectangle{\pgfqpoint{0.997489in}{0.528000in}}{\pgfqpoint{4.565023in}{3.696000in}}%
\pgfusepath{clip}%
\pgfsetbuttcap%
\pgfsetroundjoin%
\definecolor{currentfill}{rgb}{0.200000,0.800000,0.200000}%
\pgfsetfillcolor{currentfill}%
\pgfsetlinewidth{1.003750pt}%
\definecolor{currentstroke}{rgb}{0.200000,0.800000,0.200000}%
\pgfsetstrokecolor{currentstroke}%
\pgfsetdash{}{0pt}%
\pgfpathmoveto{\pgfqpoint{2.784986in}{2.332039in}}%
\pgfpathcurveto{\pgfqpoint{2.790810in}{2.332039in}}{\pgfqpoint{2.796396in}{2.334353in}}{\pgfqpoint{2.800514in}{2.338471in}}%
\pgfpathcurveto{\pgfqpoint{2.804633in}{2.342589in}}{\pgfqpoint{2.806946in}{2.348175in}}{\pgfqpoint{2.806946in}{2.353999in}}%
\pgfpathcurveto{\pgfqpoint{2.806946in}{2.359823in}}{\pgfqpoint{2.804633in}{2.365409in}}{\pgfqpoint{2.800514in}{2.369528in}}%
\pgfpathcurveto{\pgfqpoint{2.796396in}{2.373646in}}{\pgfqpoint{2.790810in}{2.375960in}}{\pgfqpoint{2.784986in}{2.375960in}}%
\pgfpathcurveto{\pgfqpoint{2.779162in}{2.375960in}}{\pgfqpoint{2.773576in}{2.373646in}}{\pgfqpoint{2.769458in}{2.369528in}}%
\pgfpathcurveto{\pgfqpoint{2.765340in}{2.365409in}}{\pgfqpoint{2.763026in}{2.359823in}}{\pgfqpoint{2.763026in}{2.353999in}}%
\pgfpathcurveto{\pgfqpoint{2.763026in}{2.348175in}}{\pgfqpoint{2.765340in}{2.342589in}}{\pgfqpoint{2.769458in}{2.338471in}}%
\pgfpathcurveto{\pgfqpoint{2.773576in}{2.334353in}}{\pgfqpoint{2.779162in}{2.332039in}}{\pgfqpoint{2.784986in}{2.332039in}}%
\pgfpathlineto{\pgfqpoint{2.784986in}{2.332039in}}%
\pgfpathclose%
\pgfusepath{stroke,fill}%
\end{pgfscope}%
\begin{pgfscope}%
\pgfpathrectangle{\pgfqpoint{0.997489in}{0.528000in}}{\pgfqpoint{4.565023in}{3.696000in}}%
\pgfusepath{clip}%
\pgfsetbuttcap%
\pgfsetroundjoin%
\definecolor{currentfill}{rgb}{0.200000,0.800000,0.200000}%
\pgfsetfillcolor{currentfill}%
\pgfsetlinewidth{1.003750pt}%
\definecolor{currentstroke}{rgb}{0.200000,0.800000,0.200000}%
\pgfsetstrokecolor{currentstroke}%
\pgfsetdash{}{0pt}%
\pgfpathmoveto{\pgfqpoint{2.848658in}{2.352417in}}%
\pgfpathcurveto{\pgfqpoint{2.854482in}{2.352417in}}{\pgfqpoint{2.860068in}{2.354731in}}{\pgfqpoint{2.864186in}{2.358849in}}%
\pgfpathcurveto{\pgfqpoint{2.868305in}{2.362967in}}{\pgfqpoint{2.870618in}{2.368553in}}{\pgfqpoint{2.870618in}{2.374377in}}%
\pgfpathcurveto{\pgfqpoint{2.870618in}{2.380201in}}{\pgfqpoint{2.868305in}{2.385787in}}{\pgfqpoint{2.864186in}{2.389906in}}%
\pgfpathcurveto{\pgfqpoint{2.860068in}{2.394024in}}{\pgfqpoint{2.854482in}{2.396338in}}{\pgfqpoint{2.848658in}{2.396338in}}%
\pgfpathcurveto{\pgfqpoint{2.842834in}{2.396338in}}{\pgfqpoint{2.837248in}{2.394024in}}{\pgfqpoint{2.833130in}{2.389906in}}%
\pgfpathcurveto{\pgfqpoint{2.829012in}{2.385787in}}{\pgfqpoint{2.826698in}{2.380201in}}{\pgfqpoint{2.826698in}{2.374377in}}%
\pgfpathcurveto{\pgfqpoint{2.826698in}{2.368553in}}{\pgfqpoint{2.829012in}{2.362967in}}{\pgfqpoint{2.833130in}{2.358849in}}%
\pgfpathcurveto{\pgfqpoint{2.837248in}{2.354731in}}{\pgfqpoint{2.842834in}{2.352417in}}{\pgfqpoint{2.848658in}{2.352417in}}%
\pgfpathlineto{\pgfqpoint{2.848658in}{2.352417in}}%
\pgfpathclose%
\pgfusepath{stroke,fill}%
\end{pgfscope}%
\begin{pgfscope}%
\pgfpathrectangle{\pgfqpoint{0.997489in}{0.528000in}}{\pgfqpoint{4.565023in}{3.696000in}}%
\pgfusepath{clip}%
\pgfsetbuttcap%
\pgfsetroundjoin%
\definecolor{currentfill}{rgb}{0.200000,0.800000,0.200000}%
\pgfsetfillcolor{currentfill}%
\pgfsetlinewidth{1.003750pt}%
\definecolor{currentstroke}{rgb}{0.200000,0.800000,0.200000}%
\pgfsetstrokecolor{currentstroke}%
\pgfsetdash{}{0pt}%
\pgfpathmoveto{\pgfqpoint{2.855645in}{2.420151in}}%
\pgfpathcurveto{\pgfqpoint{2.861469in}{2.420151in}}{\pgfqpoint{2.867055in}{2.422465in}}{\pgfqpoint{2.871174in}{2.426583in}}%
\pgfpathcurveto{\pgfqpoint{2.875292in}{2.430701in}}{\pgfqpoint{2.877606in}{2.436287in}}{\pgfqpoint{2.877606in}{2.442111in}}%
\pgfpathcurveto{\pgfqpoint{2.877606in}{2.447935in}}{\pgfqpoint{2.875292in}{2.453521in}}{\pgfqpoint{2.871174in}{2.457639in}}%
\pgfpathcurveto{\pgfqpoint{2.867055in}{2.461758in}}{\pgfqpoint{2.861469in}{2.464071in}}{\pgfqpoint{2.855645in}{2.464071in}}%
\pgfpathcurveto{\pgfqpoint{2.849821in}{2.464071in}}{\pgfqpoint{2.844235in}{2.461758in}}{\pgfqpoint{2.840117in}{2.457639in}}%
\pgfpathcurveto{\pgfqpoint{2.835999in}{2.453521in}}{\pgfqpoint{2.833685in}{2.447935in}}{\pgfqpoint{2.833685in}{2.442111in}}%
\pgfpathcurveto{\pgfqpoint{2.833685in}{2.436287in}}{\pgfqpoint{2.835999in}{2.430701in}}{\pgfqpoint{2.840117in}{2.426583in}}%
\pgfpathcurveto{\pgfqpoint{2.844235in}{2.422465in}}{\pgfqpoint{2.849821in}{2.420151in}}{\pgfqpoint{2.855645in}{2.420151in}}%
\pgfpathlineto{\pgfqpoint{2.855645in}{2.420151in}}%
\pgfpathclose%
\pgfusepath{stroke,fill}%
\end{pgfscope}%
\begin{pgfscope}%
\pgfpathrectangle{\pgfqpoint{0.997489in}{0.528000in}}{\pgfqpoint{4.565023in}{3.696000in}}%
\pgfusepath{clip}%
\pgfsetbuttcap%
\pgfsetroundjoin%
\definecolor{currentfill}{rgb}{0.200000,0.800000,0.200000}%
\pgfsetfillcolor{currentfill}%
\pgfsetlinewidth{1.003750pt}%
\definecolor{currentstroke}{rgb}{0.200000,0.800000,0.200000}%
\pgfsetstrokecolor{currentstroke}%
\pgfsetdash{}{0pt}%
\pgfpathmoveto{\pgfqpoint{2.836025in}{2.499777in}}%
\pgfpathcurveto{\pgfqpoint{2.841849in}{2.499777in}}{\pgfqpoint{2.847436in}{2.502091in}}{\pgfqpoint{2.851554in}{2.506209in}}%
\pgfpathcurveto{\pgfqpoint{2.855672in}{2.510327in}}{\pgfqpoint{2.857986in}{2.515913in}}{\pgfqpoint{2.857986in}{2.521737in}}%
\pgfpathcurveto{\pgfqpoint{2.857986in}{2.527561in}}{\pgfqpoint{2.855672in}{2.533147in}}{\pgfqpoint{2.851554in}{2.537266in}}%
\pgfpathcurveto{\pgfqpoint{2.847436in}{2.541384in}}{\pgfqpoint{2.841849in}{2.543698in}}{\pgfqpoint{2.836025in}{2.543698in}}%
\pgfpathcurveto{\pgfqpoint{2.830202in}{2.543698in}}{\pgfqpoint{2.824615in}{2.541384in}}{\pgfqpoint{2.820497in}{2.537266in}}%
\pgfpathcurveto{\pgfqpoint{2.816379in}{2.533147in}}{\pgfqpoint{2.814065in}{2.527561in}}{\pgfqpoint{2.814065in}{2.521737in}}%
\pgfpathcurveto{\pgfqpoint{2.814065in}{2.515913in}}{\pgfqpoint{2.816379in}{2.510327in}}{\pgfqpoint{2.820497in}{2.506209in}}%
\pgfpathcurveto{\pgfqpoint{2.824615in}{2.502091in}}{\pgfqpoint{2.830202in}{2.499777in}}{\pgfqpoint{2.836025in}{2.499777in}}%
\pgfpathlineto{\pgfqpoint{2.836025in}{2.499777in}}%
\pgfpathclose%
\pgfusepath{stroke,fill}%
\end{pgfscope}%
\begin{pgfscope}%
\pgfpathrectangle{\pgfqpoint{0.997489in}{0.528000in}}{\pgfqpoint{4.565023in}{3.696000in}}%
\pgfusepath{clip}%
\pgfsetbuttcap%
\pgfsetroundjoin%
\definecolor{currentfill}{rgb}{0.200000,0.800000,0.200000}%
\pgfsetfillcolor{currentfill}%
\pgfsetlinewidth{1.003750pt}%
\definecolor{currentstroke}{rgb}{0.200000,0.800000,0.200000}%
\pgfsetstrokecolor{currentstroke}%
\pgfsetdash{}{0pt}%
\pgfpathmoveto{\pgfqpoint{2.882473in}{2.533999in}}%
\pgfpathcurveto{\pgfqpoint{2.888297in}{2.533999in}}{\pgfqpoint{2.893883in}{2.536313in}}{\pgfqpoint{2.898002in}{2.540431in}}%
\pgfpathcurveto{\pgfqpoint{2.902120in}{2.544549in}}{\pgfqpoint{2.904434in}{2.550136in}}{\pgfqpoint{2.904434in}{2.555959in}}%
\pgfpathcurveto{\pgfqpoint{2.904434in}{2.561783in}}{\pgfqpoint{2.902120in}{2.567370in}}{\pgfqpoint{2.898002in}{2.571488in}}%
\pgfpathcurveto{\pgfqpoint{2.893883in}{2.575606in}}{\pgfqpoint{2.888297in}{2.577920in}}{\pgfqpoint{2.882473in}{2.577920in}}%
\pgfpathcurveto{\pgfqpoint{2.876649in}{2.577920in}}{\pgfqpoint{2.871063in}{2.575606in}}{\pgfqpoint{2.866945in}{2.571488in}}%
\pgfpathcurveto{\pgfqpoint{2.862827in}{2.567370in}}{\pgfqpoint{2.860513in}{2.561783in}}{\pgfqpoint{2.860513in}{2.555959in}}%
\pgfpathcurveto{\pgfqpoint{2.860513in}{2.550136in}}{\pgfqpoint{2.862827in}{2.544549in}}{\pgfqpoint{2.866945in}{2.540431in}}%
\pgfpathcurveto{\pgfqpoint{2.871063in}{2.536313in}}{\pgfqpoint{2.876649in}{2.533999in}}{\pgfqpoint{2.882473in}{2.533999in}}%
\pgfpathlineto{\pgfqpoint{2.882473in}{2.533999in}}%
\pgfpathclose%
\pgfusepath{stroke,fill}%
\end{pgfscope}%
\begin{pgfscope}%
\pgfpathrectangle{\pgfqpoint{0.997489in}{0.528000in}}{\pgfqpoint{4.565023in}{3.696000in}}%
\pgfusepath{clip}%
\pgfsetbuttcap%
\pgfsetroundjoin%
\definecolor{currentfill}{rgb}{0.200000,0.800000,0.200000}%
\pgfsetfillcolor{currentfill}%
\pgfsetlinewidth{1.003750pt}%
\definecolor{currentstroke}{rgb}{0.200000,0.800000,0.200000}%
\pgfsetstrokecolor{currentstroke}%
\pgfsetdash{}{0pt}%
\pgfpathmoveto{\pgfqpoint{2.848103in}{2.610056in}}%
\pgfpathcurveto{\pgfqpoint{2.853927in}{2.610056in}}{\pgfqpoint{2.859513in}{2.612369in}}{\pgfqpoint{2.863631in}{2.616488in}}%
\pgfpathcurveto{\pgfqpoint{2.867749in}{2.620606in}}{\pgfqpoint{2.870063in}{2.626192in}}{\pgfqpoint{2.870063in}{2.632016in}}%
\pgfpathcurveto{\pgfqpoint{2.870063in}{2.637840in}}{\pgfqpoint{2.867749in}{2.643426in}}{\pgfqpoint{2.863631in}{2.647544in}}%
\pgfpathcurveto{\pgfqpoint{2.859513in}{2.651662in}}{\pgfqpoint{2.853927in}{2.653976in}}{\pgfqpoint{2.848103in}{2.653976in}}%
\pgfpathcurveto{\pgfqpoint{2.842279in}{2.653976in}}{\pgfqpoint{2.836693in}{2.651662in}}{\pgfqpoint{2.832575in}{2.647544in}}%
\pgfpathcurveto{\pgfqpoint{2.828457in}{2.643426in}}{\pgfqpoint{2.826143in}{2.637840in}}{\pgfqpoint{2.826143in}{2.632016in}}%
\pgfpathcurveto{\pgfqpoint{2.826143in}{2.626192in}}{\pgfqpoint{2.828457in}{2.620606in}}{\pgfqpoint{2.832575in}{2.616488in}}%
\pgfpathcurveto{\pgfqpoint{2.836693in}{2.612369in}}{\pgfqpoint{2.842279in}{2.610056in}}{\pgfqpoint{2.848103in}{2.610056in}}%
\pgfpathlineto{\pgfqpoint{2.848103in}{2.610056in}}%
\pgfpathclose%
\pgfusepath{stroke,fill}%
\end{pgfscope}%
\begin{pgfscope}%
\pgfpathrectangle{\pgfqpoint{0.997489in}{0.528000in}}{\pgfqpoint{4.565023in}{3.696000in}}%
\pgfusepath{clip}%
\pgfsetbuttcap%
\pgfsetroundjoin%
\definecolor{currentfill}{rgb}{0.200000,0.800000,0.200000}%
\pgfsetfillcolor{currentfill}%
\pgfsetlinewidth{1.003750pt}%
\definecolor{currentstroke}{rgb}{0.200000,0.800000,0.200000}%
\pgfsetstrokecolor{currentstroke}%
\pgfsetdash{}{0pt}%
\pgfpathmoveto{\pgfqpoint{2.968433in}{2.615473in}}%
\pgfpathcurveto{\pgfqpoint{2.974257in}{2.615473in}}{\pgfqpoint{2.979843in}{2.617787in}}{\pgfqpoint{2.983961in}{2.621905in}}%
\pgfpathcurveto{\pgfqpoint{2.988079in}{2.626023in}}{\pgfqpoint{2.990393in}{2.631609in}}{\pgfqpoint{2.990393in}{2.637433in}}%
\pgfpathcurveto{\pgfqpoint{2.990393in}{2.643257in}}{\pgfqpoint{2.988079in}{2.648843in}}{\pgfqpoint{2.983961in}{2.652961in}}%
\pgfpathcurveto{\pgfqpoint{2.979843in}{2.657079in}}{\pgfqpoint{2.974257in}{2.659393in}}{\pgfqpoint{2.968433in}{2.659393in}}%
\pgfpathcurveto{\pgfqpoint{2.962609in}{2.659393in}}{\pgfqpoint{2.957023in}{2.657079in}}{\pgfqpoint{2.952904in}{2.652961in}}%
\pgfpathcurveto{\pgfqpoint{2.948786in}{2.648843in}}{\pgfqpoint{2.946472in}{2.643257in}}{\pgfqpoint{2.946472in}{2.637433in}}%
\pgfpathcurveto{\pgfqpoint{2.946472in}{2.631609in}}{\pgfqpoint{2.948786in}{2.626023in}}{\pgfqpoint{2.952904in}{2.621905in}}%
\pgfpathcurveto{\pgfqpoint{2.957023in}{2.617787in}}{\pgfqpoint{2.962609in}{2.615473in}}{\pgfqpoint{2.968433in}{2.615473in}}%
\pgfpathlineto{\pgfqpoint{2.968433in}{2.615473in}}%
\pgfpathclose%
\pgfusepath{stroke,fill}%
\end{pgfscope}%
\begin{pgfscope}%
\pgfpathrectangle{\pgfqpoint{0.997489in}{0.528000in}}{\pgfqpoint{4.565023in}{3.696000in}}%
\pgfusepath{clip}%
\pgfsetbuttcap%
\pgfsetroundjoin%
\definecolor{currentfill}{rgb}{0.200000,0.800000,0.200000}%
\pgfsetfillcolor{currentfill}%
\pgfsetlinewidth{1.003750pt}%
\definecolor{currentstroke}{rgb}{0.200000,0.800000,0.200000}%
\pgfsetstrokecolor{currentstroke}%
\pgfsetdash{}{0pt}%
\pgfpathmoveto{\pgfqpoint{2.926149in}{2.688943in}}%
\pgfpathcurveto{\pgfqpoint{2.931973in}{2.688943in}}{\pgfqpoint{2.937559in}{2.691257in}}{\pgfqpoint{2.941677in}{2.695375in}}%
\pgfpathcurveto{\pgfqpoint{2.945796in}{2.699493in}}{\pgfqpoint{2.948109in}{2.705079in}}{\pgfqpoint{2.948109in}{2.710903in}}%
\pgfpathcurveto{\pgfqpoint{2.948109in}{2.716727in}}{\pgfqpoint{2.945796in}{2.722314in}}{\pgfqpoint{2.941677in}{2.726432in}}%
\pgfpathcurveto{\pgfqpoint{2.937559in}{2.730550in}}{\pgfqpoint{2.931973in}{2.732864in}}{\pgfqpoint{2.926149in}{2.732864in}}%
\pgfpathcurveto{\pgfqpoint{2.920325in}{2.732864in}}{\pgfqpoint{2.914739in}{2.730550in}}{\pgfqpoint{2.910621in}{2.726432in}}%
\pgfpathcurveto{\pgfqpoint{2.906503in}{2.722314in}}{\pgfqpoint{2.904189in}{2.716727in}}{\pgfqpoint{2.904189in}{2.710903in}}%
\pgfpathcurveto{\pgfqpoint{2.904189in}{2.705079in}}{\pgfqpoint{2.906503in}{2.699493in}}{\pgfqpoint{2.910621in}{2.695375in}}%
\pgfpathcurveto{\pgfqpoint{2.914739in}{2.691257in}}{\pgfqpoint{2.920325in}{2.688943in}}{\pgfqpoint{2.926149in}{2.688943in}}%
\pgfpathlineto{\pgfqpoint{2.926149in}{2.688943in}}%
\pgfpathclose%
\pgfusepath{stroke,fill}%
\end{pgfscope}%
\begin{pgfscope}%
\pgfpathrectangle{\pgfqpoint{0.997489in}{0.528000in}}{\pgfqpoint{4.565023in}{3.696000in}}%
\pgfusepath{clip}%
\pgfsetbuttcap%
\pgfsetroundjoin%
\definecolor{currentfill}{rgb}{0.200000,0.800000,0.200000}%
\pgfsetfillcolor{currentfill}%
\pgfsetlinewidth{1.003750pt}%
\definecolor{currentstroke}{rgb}{0.200000,0.800000,0.200000}%
\pgfsetstrokecolor{currentstroke}%
\pgfsetdash{}{0pt}%
\pgfpathmoveto{\pgfqpoint{3.033617in}{2.715290in}}%
\pgfpathcurveto{\pgfqpoint{3.039441in}{2.715290in}}{\pgfqpoint{3.045028in}{2.717603in}}{\pgfqpoint{3.049146in}{2.721722in}}%
\pgfpathcurveto{\pgfqpoint{3.053264in}{2.725840in}}{\pgfqpoint{3.055578in}{2.731426in}}{\pgfqpoint{3.055578in}{2.737250in}}%
\pgfpathcurveto{\pgfqpoint{3.055578in}{2.743074in}}{\pgfqpoint{3.053264in}{2.748660in}}{\pgfqpoint{3.049146in}{2.752778in}}%
\pgfpathcurveto{\pgfqpoint{3.045028in}{2.756896in}}{\pgfqpoint{3.039441in}{2.759210in}}{\pgfqpoint{3.033617in}{2.759210in}}%
\pgfpathcurveto{\pgfqpoint{3.027793in}{2.759210in}}{\pgfqpoint{3.022207in}{2.756896in}}{\pgfqpoint{3.018089in}{2.752778in}}%
\pgfpathcurveto{\pgfqpoint{3.013971in}{2.748660in}}{\pgfqpoint{3.011657in}{2.743074in}}{\pgfqpoint{3.011657in}{2.737250in}}%
\pgfpathcurveto{\pgfqpoint{3.011657in}{2.731426in}}{\pgfqpoint{3.013971in}{2.725840in}}{\pgfqpoint{3.018089in}{2.721722in}}%
\pgfpathcurveto{\pgfqpoint{3.022207in}{2.717603in}}{\pgfqpoint{3.027793in}{2.715290in}}{\pgfqpoint{3.033617in}{2.715290in}}%
\pgfpathlineto{\pgfqpoint{3.033617in}{2.715290in}}%
\pgfpathclose%
\pgfusepath{stroke,fill}%
\end{pgfscope}%
\begin{pgfscope}%
\pgfpathrectangle{\pgfqpoint{0.997489in}{0.528000in}}{\pgfqpoint{4.565023in}{3.696000in}}%
\pgfusepath{clip}%
\pgfsetbuttcap%
\pgfsetroundjoin%
\definecolor{currentfill}{rgb}{0.200000,0.800000,0.200000}%
\pgfsetfillcolor{currentfill}%
\pgfsetlinewidth{1.003750pt}%
\definecolor{currentstroke}{rgb}{0.200000,0.800000,0.200000}%
\pgfsetstrokecolor{currentstroke}%
\pgfsetdash{}{0pt}%
\pgfpathmoveto{\pgfqpoint{3.014823in}{2.778554in}}%
\pgfpathcurveto{\pgfqpoint{3.020647in}{2.778554in}}{\pgfqpoint{3.026233in}{2.780868in}}{\pgfqpoint{3.030351in}{2.784986in}}%
\pgfpathcurveto{\pgfqpoint{3.034469in}{2.789105in}}{\pgfqpoint{3.036783in}{2.794691in}}{\pgfqpoint{3.036783in}{2.800515in}}%
\pgfpathcurveto{\pgfqpoint{3.036783in}{2.806339in}}{\pgfqpoint{3.034469in}{2.811925in}}{\pgfqpoint{3.030351in}{2.816043in}}%
\pgfpathcurveto{\pgfqpoint{3.026233in}{2.820161in}}{\pgfqpoint{3.020647in}{2.822475in}}{\pgfqpoint{3.014823in}{2.822475in}}%
\pgfpathcurveto{\pgfqpoint{3.008999in}{2.822475in}}{\pgfqpoint{3.003413in}{2.820161in}}{\pgfqpoint{2.999295in}{2.816043in}}%
\pgfpathcurveto{\pgfqpoint{2.995177in}{2.811925in}}{\pgfqpoint{2.992863in}{2.806339in}}{\pgfqpoint{2.992863in}{2.800515in}}%
\pgfpathcurveto{\pgfqpoint{2.992863in}{2.794691in}}{\pgfqpoint{2.995177in}{2.789105in}}{\pgfqpoint{2.999295in}{2.784986in}}%
\pgfpathcurveto{\pgfqpoint{3.003413in}{2.780868in}}{\pgfqpoint{3.008999in}{2.778554in}}{\pgfqpoint{3.014823in}{2.778554in}}%
\pgfpathlineto{\pgfqpoint{3.014823in}{2.778554in}}%
\pgfpathclose%
\pgfusepath{stroke,fill}%
\end{pgfscope}%
\begin{pgfscope}%
\pgfpathrectangle{\pgfqpoint{0.997489in}{0.528000in}}{\pgfqpoint{4.565023in}{3.696000in}}%
\pgfusepath{clip}%
\pgfsetbuttcap%
\pgfsetroundjoin%
\definecolor{currentfill}{rgb}{0.200000,0.800000,0.200000}%
\pgfsetfillcolor{currentfill}%
\pgfsetlinewidth{1.003750pt}%
\definecolor{currentstroke}{rgb}{0.200000,0.800000,0.200000}%
\pgfsetstrokecolor{currentstroke}%
\pgfsetdash{}{0pt}%
\pgfpathmoveto{\pgfqpoint{3.092415in}{2.825627in}}%
\pgfpathcurveto{\pgfqpoint{3.098239in}{2.825627in}}{\pgfqpoint{3.103825in}{2.827941in}}{\pgfqpoint{3.107944in}{2.832059in}}%
\pgfpathcurveto{\pgfqpoint{3.112062in}{2.836178in}}{\pgfqpoint{3.114376in}{2.841764in}}{\pgfqpoint{3.114376in}{2.847588in}}%
\pgfpathcurveto{\pgfqpoint{3.114376in}{2.853412in}}{\pgfqpoint{3.112062in}{2.858998in}}{\pgfqpoint{3.107944in}{2.863116in}}%
\pgfpathcurveto{\pgfqpoint{3.103825in}{2.867234in}}{\pgfqpoint{3.098239in}{2.869548in}}{\pgfqpoint{3.092415in}{2.869548in}}%
\pgfpathcurveto{\pgfqpoint{3.086591in}{2.869548in}}{\pgfqpoint{3.081005in}{2.867234in}}{\pgfqpoint{3.076887in}{2.863116in}}%
\pgfpathcurveto{\pgfqpoint{3.072769in}{2.858998in}}{\pgfqpoint{3.070455in}{2.853412in}}{\pgfqpoint{3.070455in}{2.847588in}}%
\pgfpathcurveto{\pgfqpoint{3.070455in}{2.841764in}}{\pgfqpoint{3.072769in}{2.836178in}}{\pgfqpoint{3.076887in}{2.832059in}}%
\pgfpathcurveto{\pgfqpoint{3.081005in}{2.827941in}}{\pgfqpoint{3.086591in}{2.825627in}}{\pgfqpoint{3.092415in}{2.825627in}}%
\pgfpathlineto{\pgfqpoint{3.092415in}{2.825627in}}%
\pgfpathclose%
\pgfusepath{stroke,fill}%
\end{pgfscope}%
\begin{pgfscope}%
\pgfpathrectangle{\pgfqpoint{0.997489in}{0.528000in}}{\pgfqpoint{4.565023in}{3.696000in}}%
\pgfusepath{clip}%
\pgfsetbuttcap%
\pgfsetroundjoin%
\definecolor{currentfill}{rgb}{0.200000,0.800000,0.200000}%
\pgfsetfillcolor{currentfill}%
\pgfsetlinewidth{1.003750pt}%
\definecolor{currentstroke}{rgb}{0.200000,0.800000,0.200000}%
\pgfsetstrokecolor{currentstroke}%
\pgfsetdash{}{0pt}%
\pgfpathmoveto{\pgfqpoint{3.019812in}{2.891271in}}%
\pgfpathcurveto{\pgfqpoint{3.025635in}{2.891271in}}{\pgfqpoint{3.031222in}{2.893585in}}{\pgfqpoint{3.035340in}{2.897703in}}%
\pgfpathcurveto{\pgfqpoint{3.039458in}{2.901821in}}{\pgfqpoint{3.041772in}{2.907408in}}{\pgfqpoint{3.041772in}{2.913232in}}%
\pgfpathcurveto{\pgfqpoint{3.041772in}{2.919055in}}{\pgfqpoint{3.039458in}{2.924642in}}{\pgfqpoint{3.035340in}{2.928760in}}%
\pgfpathcurveto{\pgfqpoint{3.031222in}{2.932878in}}{\pgfqpoint{3.025635in}{2.935192in}}{\pgfqpoint{3.019812in}{2.935192in}}%
\pgfpathcurveto{\pgfqpoint{3.013988in}{2.935192in}}{\pgfqpoint{3.008401in}{2.932878in}}{\pgfqpoint{3.004283in}{2.928760in}}%
\pgfpathcurveto{\pgfqpoint{3.000165in}{2.924642in}}{\pgfqpoint{2.997851in}{2.919055in}}{\pgfqpoint{2.997851in}{2.913232in}}%
\pgfpathcurveto{\pgfqpoint{2.997851in}{2.907408in}}{\pgfqpoint{3.000165in}{2.901821in}}{\pgfqpoint{3.004283in}{2.897703in}}%
\pgfpathcurveto{\pgfqpoint{3.008401in}{2.893585in}}{\pgfqpoint{3.013988in}{2.891271in}}{\pgfqpoint{3.019812in}{2.891271in}}%
\pgfpathlineto{\pgfqpoint{3.019812in}{2.891271in}}%
\pgfpathclose%
\pgfusepath{stroke,fill}%
\end{pgfscope}%
\begin{pgfscope}%
\pgfpathrectangle{\pgfqpoint{0.997489in}{0.528000in}}{\pgfqpoint{4.565023in}{3.696000in}}%
\pgfusepath{clip}%
\pgfsetbuttcap%
\pgfsetroundjoin%
\definecolor{currentfill}{rgb}{0.200000,0.800000,0.200000}%
\pgfsetfillcolor{currentfill}%
\pgfsetlinewidth{1.003750pt}%
\definecolor{currentstroke}{rgb}{0.200000,0.800000,0.200000}%
\pgfsetstrokecolor{currentstroke}%
\pgfsetdash{}{0pt}%
\pgfpathmoveto{\pgfqpoint{3.020618in}{2.947196in}}%
\pgfpathcurveto{\pgfqpoint{3.026442in}{2.947196in}}{\pgfqpoint{3.032028in}{2.949510in}}{\pgfqpoint{3.036146in}{2.953628in}}%
\pgfpathcurveto{\pgfqpoint{3.040264in}{2.957746in}}{\pgfqpoint{3.042578in}{2.963332in}}{\pgfqpoint{3.042578in}{2.969156in}}%
\pgfpathcurveto{\pgfqpoint{3.042578in}{2.974980in}}{\pgfqpoint{3.040264in}{2.980566in}}{\pgfqpoint{3.036146in}{2.984684in}}%
\pgfpathcurveto{\pgfqpoint{3.032028in}{2.988802in}}{\pgfqpoint{3.026442in}{2.991116in}}{\pgfqpoint{3.020618in}{2.991116in}}%
\pgfpathcurveto{\pgfqpoint{3.014794in}{2.991116in}}{\pgfqpoint{3.009208in}{2.988802in}}{\pgfqpoint{3.005090in}{2.984684in}}%
\pgfpathcurveto{\pgfqpoint{3.000972in}{2.980566in}}{\pgfqpoint{2.998658in}{2.974980in}}{\pgfqpoint{2.998658in}{2.969156in}}%
\pgfpathcurveto{\pgfqpoint{2.998658in}{2.963332in}}{\pgfqpoint{3.000972in}{2.957746in}}{\pgfqpoint{3.005090in}{2.953628in}}%
\pgfpathcurveto{\pgfqpoint{3.009208in}{2.949510in}}{\pgfqpoint{3.014794in}{2.947196in}}{\pgfqpoint{3.020618in}{2.947196in}}%
\pgfpathlineto{\pgfqpoint{3.020618in}{2.947196in}}%
\pgfpathclose%
\pgfusepath{stroke,fill}%
\end{pgfscope}%
\begin{pgfscope}%
\pgfpathrectangle{\pgfqpoint{0.997489in}{0.528000in}}{\pgfqpoint{4.565023in}{3.696000in}}%
\pgfusepath{clip}%
\pgfsetbuttcap%
\pgfsetroundjoin%
\definecolor{currentfill}{rgb}{0.200000,0.800000,0.200000}%
\pgfsetfillcolor{currentfill}%
\pgfsetlinewidth{1.003750pt}%
\definecolor{currentstroke}{rgb}{0.200000,0.800000,0.200000}%
\pgfsetstrokecolor{currentstroke}%
\pgfsetdash{}{0pt}%
\pgfpathmoveto{\pgfqpoint{1.579454in}{2.910712in}}%
\pgfpathcurveto{\pgfqpoint{1.585278in}{2.910712in}}{\pgfqpoint{1.590864in}{2.913026in}}{\pgfqpoint{1.594982in}{2.917144in}}%
\pgfpathcurveto{\pgfqpoint{1.599100in}{2.921262in}}{\pgfqpoint{1.601414in}{2.926848in}}{\pgfqpoint{1.601414in}{2.932672in}}%
\pgfpathcurveto{\pgfqpoint{1.601414in}{2.938496in}}{\pgfqpoint{1.599100in}{2.944082in}}{\pgfqpoint{1.594982in}{2.948200in}}%
\pgfpathcurveto{\pgfqpoint{1.590864in}{2.952318in}}{\pgfqpoint{1.585278in}{2.954632in}}{\pgfqpoint{1.579454in}{2.954632in}}%
\pgfpathcurveto{\pgfqpoint{1.573630in}{2.954632in}}{\pgfqpoint{1.568044in}{2.952318in}}{\pgfqpoint{1.563925in}{2.948200in}}%
\pgfpathcurveto{\pgfqpoint{1.559807in}{2.944082in}}{\pgfqpoint{1.557493in}{2.938496in}}{\pgfqpoint{1.557493in}{2.932672in}}%
\pgfpathcurveto{\pgfqpoint{1.557493in}{2.926848in}}{\pgfqpoint{1.559807in}{2.921262in}}{\pgfqpoint{1.563925in}{2.917144in}}%
\pgfpathcurveto{\pgfqpoint{1.568044in}{2.913026in}}{\pgfqpoint{1.573630in}{2.910712in}}{\pgfqpoint{1.579454in}{2.910712in}}%
\pgfpathlineto{\pgfqpoint{1.579454in}{2.910712in}}%
\pgfpathclose%
\pgfusepath{stroke,fill}%
\end{pgfscope}%
\begin{pgfscope}%
\pgfpathrectangle{\pgfqpoint{0.997489in}{0.528000in}}{\pgfqpoint{4.565023in}{3.696000in}}%
\pgfusepath{clip}%
\pgfsetbuttcap%
\pgfsetroundjoin%
\definecolor{currentfill}{rgb}{0.800000,0.200000,0.200000}%
\pgfsetfillcolor{currentfill}%
\pgfsetlinewidth{1.003750pt}%
\definecolor{currentstroke}{rgb}{0.800000,0.200000,0.200000}%
\pgfsetstrokecolor{currentstroke}%
\pgfsetdash{}{0pt}%
\pgfpathmoveto{\pgfqpoint{2.283610in}{0.898975in}}%
\pgfpathcurveto{\pgfqpoint{2.289434in}{0.898975in}}{\pgfqpoint{2.295020in}{0.901289in}}{\pgfqpoint{2.299138in}{0.905407in}}%
\pgfpathcurveto{\pgfqpoint{2.303256in}{0.909525in}}{\pgfqpoint{2.305570in}{0.915111in}}{\pgfqpoint{2.305570in}{0.920935in}}%
\pgfpathcurveto{\pgfqpoint{2.305570in}{0.926759in}}{\pgfqpoint{2.303256in}{0.932345in}}{\pgfqpoint{2.299138in}{0.936464in}}%
\pgfpathcurveto{\pgfqpoint{2.295020in}{0.940582in}}{\pgfqpoint{2.289434in}{0.942896in}}{\pgfqpoint{2.283610in}{0.942896in}}%
\pgfpathcurveto{\pgfqpoint{2.277786in}{0.942896in}}{\pgfqpoint{2.272200in}{0.940582in}}{\pgfqpoint{2.268082in}{0.936464in}}%
\pgfpathcurveto{\pgfqpoint{2.263964in}{0.932345in}}{\pgfqpoint{2.261650in}{0.926759in}}{\pgfqpoint{2.261650in}{0.920935in}}%
\pgfpathcurveto{\pgfqpoint{2.261650in}{0.915111in}}{\pgfqpoint{2.263964in}{0.909525in}}{\pgfqpoint{2.268082in}{0.905407in}}%
\pgfpathcurveto{\pgfqpoint{2.272200in}{0.901289in}}{\pgfqpoint{2.277786in}{0.898975in}}{\pgfqpoint{2.283610in}{0.898975in}}%
\pgfpathlineto{\pgfqpoint{2.283610in}{0.898975in}}%
\pgfpathclose%
\pgfusepath{stroke,fill}%
\end{pgfscope}%
\begin{pgfscope}%
\pgfpathrectangle{\pgfqpoint{0.997489in}{0.528000in}}{\pgfqpoint{4.565023in}{3.696000in}}%
\pgfusepath{clip}%
\pgfsetbuttcap%
\pgfsetroundjoin%
\definecolor{currentfill}{rgb}{0.800000,0.800000,0.200000}%
\pgfsetfillcolor{currentfill}%
\pgfsetlinewidth{1.003750pt}%
\definecolor{currentstroke}{rgb}{0.800000,0.800000,0.200000}%
\pgfsetstrokecolor{currentstroke}%
\pgfsetdash{}{0pt}%
\pgfpathmoveto{\pgfqpoint{4.071068in}{2.250458in}}%
\pgfpathcurveto{\pgfqpoint{4.076892in}{2.250458in}}{\pgfqpoint{4.082478in}{2.252772in}}{\pgfqpoint{4.086596in}{2.256890in}}%
\pgfpathcurveto{\pgfqpoint{4.090714in}{2.261008in}}{\pgfqpoint{4.093028in}{2.266594in}}{\pgfqpoint{4.093028in}{2.272418in}}%
\pgfpathcurveto{\pgfqpoint{4.093028in}{2.278242in}}{\pgfqpoint{4.090714in}{2.283828in}}{\pgfqpoint{4.086596in}{2.287947in}}%
\pgfpathcurveto{\pgfqpoint{4.082478in}{2.292065in}}{\pgfqpoint{4.076892in}{2.294379in}}{\pgfqpoint{4.071068in}{2.294379in}}%
\pgfpathcurveto{\pgfqpoint{4.065244in}{2.294379in}}{\pgfqpoint{4.059658in}{2.292065in}}{\pgfqpoint{4.055539in}{2.287947in}}%
\pgfpathcurveto{\pgfqpoint{4.051421in}{2.283828in}}{\pgfqpoint{4.049107in}{2.278242in}}{\pgfqpoint{4.049107in}{2.272418in}}%
\pgfpathcurveto{\pgfqpoint{4.049107in}{2.266594in}}{\pgfqpoint{4.051421in}{2.261008in}}{\pgfqpoint{4.055539in}{2.256890in}}%
\pgfpathcurveto{\pgfqpoint{4.059658in}{2.252772in}}{\pgfqpoint{4.065244in}{2.250458in}}{\pgfqpoint{4.071068in}{2.250458in}}%
\pgfpathlineto{\pgfqpoint{4.071068in}{2.250458in}}%
\pgfpathclose%
\pgfusepath{stroke,fill}%
\end{pgfscope}%
\begin{pgfscope}%
\pgfpathrectangle{\pgfqpoint{0.997489in}{0.528000in}}{\pgfqpoint{4.565023in}{3.696000in}}%
\pgfusepath{clip}%
\pgfsetbuttcap%
\pgfsetroundjoin%
\definecolor{currentfill}{rgb}{0.800000,0.200000,0.200000}%
\pgfsetfillcolor{currentfill}%
\pgfsetlinewidth{1.003750pt}%
\definecolor{currentstroke}{rgb}{0.800000,0.200000,0.200000}%
\pgfsetstrokecolor{currentstroke}%
\pgfsetdash{}{0pt}%
\pgfpathmoveto{\pgfqpoint{2.621609in}{3.275738in}}%
\pgfpathcurveto{\pgfqpoint{2.627433in}{3.275738in}}{\pgfqpoint{2.633019in}{3.278052in}}{\pgfqpoint{2.637137in}{3.282170in}}%
\pgfpathcurveto{\pgfqpoint{2.641256in}{3.286288in}}{\pgfqpoint{2.643569in}{3.291874in}}{\pgfqpoint{2.643569in}{3.297698in}}%
\pgfpathcurveto{\pgfqpoint{2.643569in}{3.303522in}}{\pgfqpoint{2.641256in}{3.309108in}}{\pgfqpoint{2.637137in}{3.313226in}}%
\pgfpathcurveto{\pgfqpoint{2.633019in}{3.317345in}}{\pgfqpoint{2.627433in}{3.319658in}}{\pgfqpoint{2.621609in}{3.319658in}}%
\pgfpathcurveto{\pgfqpoint{2.615785in}{3.319658in}}{\pgfqpoint{2.610199in}{3.317345in}}{\pgfqpoint{2.606081in}{3.313226in}}%
\pgfpathcurveto{\pgfqpoint{2.601963in}{3.309108in}}{\pgfqpoint{2.599649in}{3.303522in}}{\pgfqpoint{2.599649in}{3.297698in}}%
\pgfpathcurveto{\pgfqpoint{2.599649in}{3.291874in}}{\pgfqpoint{2.601963in}{3.286288in}}{\pgfqpoint{2.606081in}{3.282170in}}%
\pgfpathcurveto{\pgfqpoint{2.610199in}{3.278052in}}{\pgfqpoint{2.615785in}{3.275738in}}{\pgfqpoint{2.621609in}{3.275738in}}%
\pgfpathlineto{\pgfqpoint{2.621609in}{3.275738in}}%
\pgfpathclose%
\pgfusepath{stroke,fill}%
\end{pgfscope}%
\begin{pgfscope}%
\pgfpathrectangle{\pgfqpoint{0.997489in}{0.528000in}}{\pgfqpoint{4.565023in}{3.696000in}}%
\pgfusepath{clip}%
\pgfsetbuttcap%
\pgfsetroundjoin%
\definecolor{currentfill}{rgb}{0.200000,0.800000,0.200000}%
\pgfsetfillcolor{currentfill}%
\pgfsetlinewidth{1.003750pt}%
\definecolor{currentstroke}{rgb}{0.200000,0.800000,0.200000}%
\pgfsetstrokecolor{currentstroke}%
\pgfsetdash{}{0pt}%
\pgfpathmoveto{\pgfqpoint{2.965835in}{2.880724in}}%
\pgfpathcurveto{\pgfqpoint{2.971659in}{2.880724in}}{\pgfqpoint{2.977245in}{2.883038in}}{\pgfqpoint{2.981364in}{2.887156in}}%
\pgfpathcurveto{\pgfqpoint{2.985482in}{2.891274in}}{\pgfqpoint{2.987796in}{2.896860in}}{\pgfqpoint{2.987796in}{2.902684in}}%
\pgfpathcurveto{\pgfqpoint{2.987796in}{2.908508in}}{\pgfqpoint{2.985482in}{2.914094in}}{\pgfqpoint{2.981364in}{2.918212in}}%
\pgfpathcurveto{\pgfqpoint{2.977245in}{2.922330in}}{\pgfqpoint{2.971659in}{2.924644in}}{\pgfqpoint{2.965835in}{2.924644in}}%
\pgfpathcurveto{\pgfqpoint{2.960011in}{2.924644in}}{\pgfqpoint{2.954425in}{2.922330in}}{\pgfqpoint{2.950307in}{2.918212in}}%
\pgfpathcurveto{\pgfqpoint{2.946189in}{2.914094in}}{\pgfqpoint{2.943875in}{2.908508in}}{\pgfqpoint{2.943875in}{2.902684in}}%
\pgfpathcurveto{\pgfqpoint{2.943875in}{2.896860in}}{\pgfqpoint{2.946189in}{2.891274in}}{\pgfqpoint{2.950307in}{2.887156in}}%
\pgfpathcurveto{\pgfqpoint{2.954425in}{2.883038in}}{\pgfqpoint{2.960011in}{2.880724in}}{\pgfqpoint{2.965835in}{2.880724in}}%
\pgfpathlineto{\pgfqpoint{2.965835in}{2.880724in}}%
\pgfpathclose%
\pgfusepath{stroke,fill}%
\end{pgfscope}%
\begin{pgfscope}%
\pgfpathrectangle{\pgfqpoint{0.997489in}{0.528000in}}{\pgfqpoint{4.565023in}{3.696000in}}%
\pgfusepath{clip}%
\pgfsetbuttcap%
\pgfsetroundjoin%
\definecolor{currentfill}{rgb}{0.200000,0.800000,0.200000}%
\pgfsetfillcolor{currentfill}%
\pgfsetlinewidth{1.003750pt}%
\definecolor{currentstroke}{rgb}{0.200000,0.800000,0.200000}%
\pgfsetstrokecolor{currentstroke}%
\pgfsetdash{}{0pt}%
\pgfpathmoveto{\pgfqpoint{1.936614in}{3.341452in}}%
\pgfpathcurveto{\pgfqpoint{1.942438in}{3.341452in}}{\pgfqpoint{1.948024in}{3.343766in}}{\pgfqpoint{1.952142in}{3.347884in}}%
\pgfpathcurveto{\pgfqpoint{1.956261in}{3.352002in}}{\pgfqpoint{1.958574in}{3.357588in}}{\pgfqpoint{1.958574in}{3.363412in}}%
\pgfpathcurveto{\pgfqpoint{1.958574in}{3.369236in}}{\pgfqpoint{1.956261in}{3.374822in}}{\pgfqpoint{1.952142in}{3.378940in}}%
\pgfpathcurveto{\pgfqpoint{1.948024in}{3.383059in}}{\pgfqpoint{1.942438in}{3.385372in}}{\pgfqpoint{1.936614in}{3.385372in}}%
\pgfpathcurveto{\pgfqpoint{1.930790in}{3.385372in}}{\pgfqpoint{1.925204in}{3.383059in}}{\pgfqpoint{1.921086in}{3.378940in}}%
\pgfpathcurveto{\pgfqpoint{1.916968in}{3.374822in}}{\pgfqpoint{1.914654in}{3.369236in}}{\pgfqpoint{1.914654in}{3.363412in}}%
\pgfpathcurveto{\pgfqpoint{1.914654in}{3.357588in}}{\pgfqpoint{1.916968in}{3.352002in}}{\pgfqpoint{1.921086in}{3.347884in}}%
\pgfpathcurveto{\pgfqpoint{1.925204in}{3.343766in}}{\pgfqpoint{1.930790in}{3.341452in}}{\pgfqpoint{1.936614in}{3.341452in}}%
\pgfpathlineto{\pgfqpoint{1.936614in}{3.341452in}}%
\pgfpathclose%
\pgfusepath{stroke,fill}%
\end{pgfscope}%
\begin{pgfscope}%
\pgfpathrectangle{\pgfqpoint{0.997489in}{0.528000in}}{\pgfqpoint{4.565023in}{3.696000in}}%
\pgfusepath{clip}%
\pgfsetbuttcap%
\pgfsetroundjoin%
\definecolor{currentfill}{rgb}{0.800000,0.200000,0.200000}%
\pgfsetfillcolor{currentfill}%
\pgfsetlinewidth{1.003750pt}%
\definecolor{currentstroke}{rgb}{0.800000,0.200000,0.200000}%
\pgfsetstrokecolor{currentstroke}%
\pgfsetdash{}{0pt}%
\pgfpathmoveto{\pgfqpoint{3.083238in}{1.306007in}}%
\pgfpathcurveto{\pgfqpoint{3.089062in}{1.306007in}}{\pgfqpoint{3.094648in}{1.308321in}}{\pgfqpoint{3.098766in}{1.312439in}}%
\pgfpathcurveto{\pgfqpoint{3.102884in}{1.316557in}}{\pgfqpoint{3.105198in}{1.322143in}}{\pgfqpoint{3.105198in}{1.327967in}}%
\pgfpathcurveto{\pgfqpoint{3.105198in}{1.333791in}}{\pgfqpoint{3.102884in}{1.339377in}}{\pgfqpoint{3.098766in}{1.343495in}}%
\pgfpathcurveto{\pgfqpoint{3.094648in}{1.347614in}}{\pgfqpoint{3.089062in}{1.349927in}}{\pgfqpoint{3.083238in}{1.349927in}}%
\pgfpathcurveto{\pgfqpoint{3.077414in}{1.349927in}}{\pgfqpoint{3.071828in}{1.347614in}}{\pgfqpoint{3.067710in}{1.343495in}}%
\pgfpathcurveto{\pgfqpoint{3.063592in}{1.339377in}}{\pgfqpoint{3.061278in}{1.333791in}}{\pgfqpoint{3.061278in}{1.327967in}}%
\pgfpathcurveto{\pgfqpoint{3.061278in}{1.322143in}}{\pgfqpoint{3.063592in}{1.316557in}}{\pgfqpoint{3.067710in}{1.312439in}}%
\pgfpathcurveto{\pgfqpoint{3.071828in}{1.308321in}}{\pgfqpoint{3.077414in}{1.306007in}}{\pgfqpoint{3.083238in}{1.306007in}}%
\pgfpathlineto{\pgfqpoint{3.083238in}{1.306007in}}%
\pgfpathclose%
\pgfusepath{stroke,fill}%
\end{pgfscope}%
\begin{pgfscope}%
\pgfpathrectangle{\pgfqpoint{0.997489in}{0.528000in}}{\pgfqpoint{4.565023in}{3.696000in}}%
\pgfusepath{clip}%
\pgfsetbuttcap%
\pgfsetroundjoin%
\definecolor{currentfill}{rgb}{0.800000,0.200000,0.200000}%
\pgfsetfillcolor{currentfill}%
\pgfsetlinewidth{1.003750pt}%
\definecolor{currentstroke}{rgb}{0.800000,0.200000,0.200000}%
\pgfsetstrokecolor{currentstroke}%
\pgfsetdash{}{0pt}%
\pgfpathmoveto{\pgfqpoint{2.188920in}{1.556838in}}%
\pgfpathcurveto{\pgfqpoint{2.194744in}{1.556838in}}{\pgfqpoint{2.200331in}{1.559152in}}{\pgfqpoint{2.204449in}{1.563270in}}%
\pgfpathcurveto{\pgfqpoint{2.208567in}{1.567389in}}{\pgfqpoint{2.210881in}{1.572975in}}{\pgfqpoint{2.210881in}{1.578799in}}%
\pgfpathcurveto{\pgfqpoint{2.210881in}{1.584623in}}{\pgfqpoint{2.208567in}{1.590209in}}{\pgfqpoint{2.204449in}{1.594327in}}%
\pgfpathcurveto{\pgfqpoint{2.200331in}{1.598445in}}{\pgfqpoint{2.194744in}{1.600759in}}{\pgfqpoint{2.188920in}{1.600759in}}%
\pgfpathcurveto{\pgfqpoint{2.183097in}{1.600759in}}{\pgfqpoint{2.177510in}{1.598445in}}{\pgfqpoint{2.173392in}{1.594327in}}%
\pgfpathcurveto{\pgfqpoint{2.169274in}{1.590209in}}{\pgfqpoint{2.166960in}{1.584623in}}{\pgfqpoint{2.166960in}{1.578799in}}%
\pgfpathcurveto{\pgfqpoint{2.166960in}{1.572975in}}{\pgfqpoint{2.169274in}{1.567389in}}{\pgfqpoint{2.173392in}{1.563270in}}%
\pgfpathcurveto{\pgfqpoint{2.177510in}{1.559152in}}{\pgfqpoint{2.183097in}{1.556838in}}{\pgfqpoint{2.188920in}{1.556838in}}%
\pgfpathlineto{\pgfqpoint{2.188920in}{1.556838in}}%
\pgfpathclose%
\pgfusepath{stroke,fill}%
\end{pgfscope}%
\begin{pgfscope}%
\pgfpathrectangle{\pgfqpoint{0.997489in}{0.528000in}}{\pgfqpoint{4.565023in}{3.696000in}}%
\pgfusepath{clip}%
\pgfsetbuttcap%
\pgfsetroundjoin%
\definecolor{currentfill}{rgb}{0.800000,0.200000,0.200000}%
\pgfsetfillcolor{currentfill}%
\pgfsetlinewidth{1.003750pt}%
\definecolor{currentstroke}{rgb}{0.800000,0.200000,0.200000}%
\pgfsetstrokecolor{currentstroke}%
\pgfsetdash{}{0pt}%
\pgfpathmoveto{\pgfqpoint{3.941779in}{2.036106in}}%
\pgfpathcurveto{\pgfqpoint{3.947603in}{2.036106in}}{\pgfqpoint{3.953189in}{2.038420in}}{\pgfqpoint{3.957307in}{2.042538in}}%
\pgfpathcurveto{\pgfqpoint{3.961425in}{2.046656in}}{\pgfqpoint{3.963739in}{2.052243in}}{\pgfqpoint{3.963739in}{2.058067in}}%
\pgfpathcurveto{\pgfqpoint{3.963739in}{2.063890in}}{\pgfqpoint{3.961425in}{2.069477in}}{\pgfqpoint{3.957307in}{2.073595in}}%
\pgfpathcurveto{\pgfqpoint{3.953189in}{2.077713in}}{\pgfqpoint{3.947603in}{2.080027in}}{\pgfqpoint{3.941779in}{2.080027in}}%
\pgfpathcurveto{\pgfqpoint{3.935955in}{2.080027in}}{\pgfqpoint{3.930369in}{2.077713in}}{\pgfqpoint{3.926251in}{2.073595in}}%
\pgfpathcurveto{\pgfqpoint{3.922133in}{2.069477in}}{\pgfqpoint{3.919819in}{2.063890in}}{\pgfqpoint{3.919819in}{2.058067in}}%
\pgfpathcurveto{\pgfqpoint{3.919819in}{2.052243in}}{\pgfqpoint{3.922133in}{2.046656in}}{\pgfqpoint{3.926251in}{2.042538in}}%
\pgfpathcurveto{\pgfqpoint{3.930369in}{2.038420in}}{\pgfqpoint{3.935955in}{2.036106in}}{\pgfqpoint{3.941779in}{2.036106in}}%
\pgfpathlineto{\pgfqpoint{3.941779in}{2.036106in}}%
\pgfpathclose%
\pgfusepath{stroke,fill}%
\end{pgfscope}%
\begin{pgfscope}%
\pgfpathrectangle{\pgfqpoint{0.997489in}{0.528000in}}{\pgfqpoint{4.565023in}{3.696000in}}%
\pgfusepath{clip}%
\pgfsetbuttcap%
\pgfsetroundjoin%
\definecolor{currentfill}{rgb}{0.800000,0.200000,0.200000}%
\pgfsetfillcolor{currentfill}%
\pgfsetlinewidth{1.003750pt}%
\definecolor{currentstroke}{rgb}{0.800000,0.200000,0.200000}%
\pgfsetstrokecolor{currentstroke}%
\pgfsetdash{}{0pt}%
\pgfpathmoveto{\pgfqpoint{1.989083in}{2.290607in}}%
\pgfpathcurveto{\pgfqpoint{1.994907in}{2.290607in}}{\pgfqpoint{2.000493in}{2.292921in}}{\pgfqpoint{2.004611in}{2.297039in}}%
\pgfpathcurveto{\pgfqpoint{2.008730in}{2.301157in}}{\pgfqpoint{2.011043in}{2.306743in}}{\pgfqpoint{2.011043in}{2.312567in}}%
\pgfpathcurveto{\pgfqpoint{2.011043in}{2.318391in}}{\pgfqpoint{2.008730in}{2.323977in}}{\pgfqpoint{2.004611in}{2.328096in}}%
\pgfpathcurveto{\pgfqpoint{2.000493in}{2.332214in}}{\pgfqpoint{1.994907in}{2.334528in}}{\pgfqpoint{1.989083in}{2.334528in}}%
\pgfpathcurveto{\pgfqpoint{1.983259in}{2.334528in}}{\pgfqpoint{1.977673in}{2.332214in}}{\pgfqpoint{1.973555in}{2.328096in}}%
\pgfpathcurveto{\pgfqpoint{1.969437in}{2.323977in}}{\pgfqpoint{1.967123in}{2.318391in}}{\pgfqpoint{1.967123in}{2.312567in}}%
\pgfpathcurveto{\pgfqpoint{1.967123in}{2.306743in}}{\pgfqpoint{1.969437in}{2.301157in}}{\pgfqpoint{1.973555in}{2.297039in}}%
\pgfpathcurveto{\pgfqpoint{1.977673in}{2.292921in}}{\pgfqpoint{1.983259in}{2.290607in}}{\pgfqpoint{1.989083in}{2.290607in}}%
\pgfpathlineto{\pgfqpoint{1.989083in}{2.290607in}}%
\pgfpathclose%
\pgfusepath{stroke,fill}%
\end{pgfscope}%
\begin{pgfscope}%
\pgfpathrectangle{\pgfqpoint{0.997489in}{0.528000in}}{\pgfqpoint{4.565023in}{3.696000in}}%
\pgfusepath{clip}%
\pgfsetbuttcap%
\pgfsetroundjoin%
\definecolor{currentfill}{rgb}{0.800000,0.200000,0.200000}%
\pgfsetfillcolor{currentfill}%
\pgfsetlinewidth{1.003750pt}%
\definecolor{currentstroke}{rgb}{0.800000,0.200000,0.200000}%
\pgfsetstrokecolor{currentstroke}%
\pgfsetdash{}{0pt}%
\pgfpathmoveto{\pgfqpoint{2.484594in}{1.448995in}}%
\pgfpathcurveto{\pgfqpoint{2.490418in}{1.448995in}}{\pgfqpoint{2.496004in}{1.451309in}}{\pgfqpoint{2.500122in}{1.455427in}}%
\pgfpathcurveto{\pgfqpoint{2.504240in}{1.459545in}}{\pgfqpoint{2.506554in}{1.465131in}}{\pgfqpoint{2.506554in}{1.470955in}}%
\pgfpathcurveto{\pgfqpoint{2.506554in}{1.476779in}}{\pgfqpoint{2.504240in}{1.482365in}}{\pgfqpoint{2.500122in}{1.486483in}}%
\pgfpathcurveto{\pgfqpoint{2.496004in}{1.490602in}}{\pgfqpoint{2.490418in}{1.492915in}}{\pgfqpoint{2.484594in}{1.492915in}}%
\pgfpathcurveto{\pgfqpoint{2.478770in}{1.492915in}}{\pgfqpoint{2.473184in}{1.490602in}}{\pgfqpoint{2.469066in}{1.486483in}}%
\pgfpathcurveto{\pgfqpoint{2.464948in}{1.482365in}}{\pgfqpoint{2.462634in}{1.476779in}}{\pgfqpoint{2.462634in}{1.470955in}}%
\pgfpathcurveto{\pgfqpoint{2.462634in}{1.465131in}}{\pgfqpoint{2.464948in}{1.459545in}}{\pgfqpoint{2.469066in}{1.455427in}}%
\pgfpathcurveto{\pgfqpoint{2.473184in}{1.451309in}}{\pgfqpoint{2.478770in}{1.448995in}}{\pgfqpoint{2.484594in}{1.448995in}}%
\pgfpathlineto{\pgfqpoint{2.484594in}{1.448995in}}%
\pgfpathclose%
\pgfusepath{stroke,fill}%
\end{pgfscope}%
\begin{pgfscope}%
\pgfpathrectangle{\pgfqpoint{0.997489in}{0.528000in}}{\pgfqpoint{4.565023in}{3.696000in}}%
\pgfusepath{clip}%
\pgfsetbuttcap%
\pgfsetroundjoin%
\definecolor{currentfill}{rgb}{0.200000,0.800000,0.200000}%
\pgfsetfillcolor{currentfill}%
\pgfsetlinewidth{1.003750pt}%
\definecolor{currentstroke}{rgb}{0.200000,0.800000,0.200000}%
\pgfsetstrokecolor{currentstroke}%
\pgfsetdash{}{0pt}%
\pgfpathmoveto{\pgfqpoint{1.662581in}{1.849707in}}%
\pgfpathcurveto{\pgfqpoint{1.668405in}{1.849707in}}{\pgfqpoint{1.673991in}{1.852021in}}{\pgfqpoint{1.678109in}{1.856139in}}%
\pgfpathcurveto{\pgfqpoint{1.682227in}{1.860257in}}{\pgfqpoint{1.684541in}{1.865843in}}{\pgfqpoint{1.684541in}{1.871667in}}%
\pgfpathcurveto{\pgfqpoint{1.684541in}{1.877491in}}{\pgfqpoint{1.682227in}{1.883077in}}{\pgfqpoint{1.678109in}{1.887195in}}%
\pgfpathcurveto{\pgfqpoint{1.673991in}{1.891313in}}{\pgfqpoint{1.668405in}{1.893627in}}{\pgfqpoint{1.662581in}{1.893627in}}%
\pgfpathcurveto{\pgfqpoint{1.656757in}{1.893627in}}{\pgfqpoint{1.651170in}{1.891313in}}{\pgfqpoint{1.647052in}{1.887195in}}%
\pgfpathcurveto{\pgfqpoint{1.642934in}{1.883077in}}{\pgfqpoint{1.640620in}{1.877491in}}{\pgfqpoint{1.640620in}{1.871667in}}%
\pgfpathcurveto{\pgfqpoint{1.640620in}{1.865843in}}{\pgfqpoint{1.642934in}{1.860257in}}{\pgfqpoint{1.647052in}{1.856139in}}%
\pgfpathcurveto{\pgfqpoint{1.651170in}{1.852021in}}{\pgfqpoint{1.656757in}{1.849707in}}{\pgfqpoint{1.662581in}{1.849707in}}%
\pgfpathlineto{\pgfqpoint{1.662581in}{1.849707in}}%
\pgfpathclose%
\pgfusepath{stroke,fill}%
\end{pgfscope}%
\begin{pgfscope}%
\pgfpathrectangle{\pgfqpoint{0.997489in}{0.528000in}}{\pgfqpoint{4.565023in}{3.696000in}}%
\pgfusepath{clip}%
\pgfsetbuttcap%
\pgfsetroundjoin%
\definecolor{currentfill}{rgb}{0.200000,0.800000,0.200000}%
\pgfsetfillcolor{currentfill}%
\pgfsetlinewidth{1.003750pt}%
\definecolor{currentstroke}{rgb}{0.200000,0.800000,0.200000}%
\pgfsetstrokecolor{currentstroke}%
\pgfsetdash{}{0pt}%
\pgfpathmoveto{\pgfqpoint{3.192814in}{2.799079in}}%
\pgfpathcurveto{\pgfqpoint{3.198638in}{2.799079in}}{\pgfqpoint{3.204224in}{2.801393in}}{\pgfqpoint{3.208342in}{2.805511in}}%
\pgfpathcurveto{\pgfqpoint{3.212460in}{2.809630in}}{\pgfqpoint{3.214774in}{2.815216in}}{\pgfqpoint{3.214774in}{2.821040in}}%
\pgfpathcurveto{\pgfqpoint{3.214774in}{2.826864in}}{\pgfqpoint{3.212460in}{2.832450in}}{\pgfqpoint{3.208342in}{2.836568in}}%
\pgfpathcurveto{\pgfqpoint{3.204224in}{2.840686in}}{\pgfqpoint{3.198638in}{2.843000in}}{\pgfqpoint{3.192814in}{2.843000in}}%
\pgfpathcurveto{\pgfqpoint{3.186990in}{2.843000in}}{\pgfqpoint{3.181404in}{2.840686in}}{\pgfqpoint{3.177286in}{2.836568in}}%
\pgfpathcurveto{\pgfqpoint{3.173167in}{2.832450in}}{\pgfqpoint{3.170854in}{2.826864in}}{\pgfqpoint{3.170854in}{2.821040in}}%
\pgfpathcurveto{\pgfqpoint{3.170854in}{2.815216in}}{\pgfqpoint{3.173167in}{2.809630in}}{\pgfqpoint{3.177286in}{2.805511in}}%
\pgfpathcurveto{\pgfqpoint{3.181404in}{2.801393in}}{\pgfqpoint{3.186990in}{2.799079in}}{\pgfqpoint{3.192814in}{2.799079in}}%
\pgfpathlineto{\pgfqpoint{3.192814in}{2.799079in}}%
\pgfpathclose%
\pgfusepath{stroke,fill}%
\end{pgfscope}%
\begin{pgfscope}%
\pgfpathrectangle{\pgfqpoint{0.997489in}{0.528000in}}{\pgfqpoint{4.565023in}{3.696000in}}%
\pgfusepath{clip}%
\pgfsetbuttcap%
\pgfsetroundjoin%
\definecolor{currentfill}{rgb}{0.800000,0.200000,0.200000}%
\pgfsetfillcolor{currentfill}%
\pgfsetlinewidth{1.003750pt}%
\definecolor{currentstroke}{rgb}{0.800000,0.200000,0.200000}%
\pgfsetstrokecolor{currentstroke}%
\pgfsetdash{}{0pt}%
\pgfpathmoveto{\pgfqpoint{4.291776in}{2.669742in}}%
\pgfpathcurveto{\pgfqpoint{4.297600in}{2.669742in}}{\pgfqpoint{4.303186in}{2.672056in}}{\pgfqpoint{4.307304in}{2.676174in}}%
\pgfpathcurveto{\pgfqpoint{4.311422in}{2.680292in}}{\pgfqpoint{4.313736in}{2.685878in}}{\pgfqpoint{4.313736in}{2.691702in}}%
\pgfpathcurveto{\pgfqpoint{4.313736in}{2.697526in}}{\pgfqpoint{4.311422in}{2.703112in}}{\pgfqpoint{4.307304in}{2.707230in}}%
\pgfpathcurveto{\pgfqpoint{4.303186in}{2.711348in}}{\pgfqpoint{4.297600in}{2.713662in}}{\pgfqpoint{4.291776in}{2.713662in}}%
\pgfpathcurveto{\pgfqpoint{4.285952in}{2.713662in}}{\pgfqpoint{4.280366in}{2.711348in}}{\pgfqpoint{4.276247in}{2.707230in}}%
\pgfpathcurveto{\pgfqpoint{4.272129in}{2.703112in}}{\pgfqpoint{4.269815in}{2.697526in}}{\pgfqpoint{4.269815in}{2.691702in}}%
\pgfpathcurveto{\pgfqpoint{4.269815in}{2.685878in}}{\pgfqpoint{4.272129in}{2.680292in}}{\pgfqpoint{4.276247in}{2.676174in}}%
\pgfpathcurveto{\pgfqpoint{4.280366in}{2.672056in}}{\pgfqpoint{4.285952in}{2.669742in}}{\pgfqpoint{4.291776in}{2.669742in}}%
\pgfpathlineto{\pgfqpoint{4.291776in}{2.669742in}}%
\pgfpathclose%
\pgfusepath{stroke,fill}%
\end{pgfscope}%
\begin{pgfscope}%
\pgfpathrectangle{\pgfqpoint{0.997489in}{0.528000in}}{\pgfqpoint{4.565023in}{3.696000in}}%
\pgfusepath{clip}%
\pgfsetbuttcap%
\pgfsetroundjoin%
\definecolor{currentfill}{rgb}{0.800000,0.200000,0.200000}%
\pgfsetfillcolor{currentfill}%
\pgfsetlinewidth{1.003750pt}%
\definecolor{currentstroke}{rgb}{0.800000,0.200000,0.200000}%
\pgfsetstrokecolor{currentstroke}%
\pgfsetdash{}{0pt}%
\pgfpathmoveto{\pgfqpoint{1.904986in}{1.009119in}}%
\pgfpathcurveto{\pgfqpoint{1.910810in}{1.009119in}}{\pgfqpoint{1.916396in}{1.011433in}}{\pgfqpoint{1.920514in}{1.015551in}}%
\pgfpathcurveto{\pgfqpoint{1.924632in}{1.019669in}}{\pgfqpoint{1.926946in}{1.025255in}}{\pgfqpoint{1.926946in}{1.031079in}}%
\pgfpathcurveto{\pgfqpoint{1.926946in}{1.036903in}}{\pgfqpoint{1.924632in}{1.042489in}}{\pgfqpoint{1.920514in}{1.046608in}}%
\pgfpathcurveto{\pgfqpoint{1.916396in}{1.050726in}}{\pgfqpoint{1.910810in}{1.053040in}}{\pgfqpoint{1.904986in}{1.053040in}}%
\pgfpathcurveto{\pgfqpoint{1.899162in}{1.053040in}}{\pgfqpoint{1.893576in}{1.050726in}}{\pgfqpoint{1.889458in}{1.046608in}}%
\pgfpathcurveto{\pgfqpoint{1.885340in}{1.042489in}}{\pgfqpoint{1.883026in}{1.036903in}}{\pgfqpoint{1.883026in}{1.031079in}}%
\pgfpathcurveto{\pgfqpoint{1.883026in}{1.025255in}}{\pgfqpoint{1.885340in}{1.019669in}}{\pgfqpoint{1.889458in}{1.015551in}}%
\pgfpathcurveto{\pgfqpoint{1.893576in}{1.011433in}}{\pgfqpoint{1.899162in}{1.009119in}}{\pgfqpoint{1.904986in}{1.009119in}}%
\pgfpathlineto{\pgfqpoint{1.904986in}{1.009119in}}%
\pgfpathclose%
\pgfusepath{stroke,fill}%
\end{pgfscope}%
\begin{pgfscope}%
\pgfpathrectangle{\pgfqpoint{0.997489in}{0.528000in}}{\pgfqpoint{4.565023in}{3.696000in}}%
\pgfusepath{clip}%
\pgfsetbuttcap%
\pgfsetroundjoin%
\definecolor{currentfill}{rgb}{0.200000,0.800000,0.200000}%
\pgfsetfillcolor{currentfill}%
\pgfsetlinewidth{1.003750pt}%
\definecolor{currentstroke}{rgb}{0.200000,0.800000,0.200000}%
\pgfsetstrokecolor{currentstroke}%
\pgfsetdash{}{0pt}%
\pgfpathmoveto{\pgfqpoint{1.428697in}{3.726269in}}%
\pgfpathcurveto{\pgfqpoint{1.434521in}{3.726269in}}{\pgfqpoint{1.440108in}{3.728583in}}{\pgfqpoint{1.444226in}{3.732701in}}%
\pgfpathcurveto{\pgfqpoint{1.448344in}{3.736819in}}{\pgfqpoint{1.450658in}{3.742405in}}{\pgfqpoint{1.450658in}{3.748229in}}%
\pgfpathcurveto{\pgfqpoint{1.450658in}{3.754053in}}{\pgfqpoint{1.448344in}{3.759640in}}{\pgfqpoint{1.444226in}{3.763758in}}%
\pgfpathcurveto{\pgfqpoint{1.440108in}{3.767876in}}{\pgfqpoint{1.434521in}{3.770190in}}{\pgfqpoint{1.428697in}{3.770190in}}%
\pgfpathcurveto{\pgfqpoint{1.422874in}{3.770190in}}{\pgfqpoint{1.417287in}{3.767876in}}{\pgfqpoint{1.413169in}{3.763758in}}%
\pgfpathcurveto{\pgfqpoint{1.409051in}{3.759640in}}{\pgfqpoint{1.406737in}{3.754053in}}{\pgfqpoint{1.406737in}{3.748229in}}%
\pgfpathcurveto{\pgfqpoint{1.406737in}{3.742405in}}{\pgfqpoint{1.409051in}{3.736819in}}{\pgfqpoint{1.413169in}{3.732701in}}%
\pgfpathcurveto{\pgfqpoint{1.417287in}{3.728583in}}{\pgfqpoint{1.422874in}{3.726269in}}{\pgfqpoint{1.428697in}{3.726269in}}%
\pgfpathlineto{\pgfqpoint{1.428697in}{3.726269in}}%
\pgfpathclose%
\pgfusepath{stroke,fill}%
\end{pgfscope}%
\begin{pgfscope}%
\pgfpathrectangle{\pgfqpoint{0.997489in}{0.528000in}}{\pgfqpoint{4.565023in}{3.696000in}}%
\pgfusepath{clip}%
\pgfsetbuttcap%
\pgfsetroundjoin%
\definecolor{currentfill}{rgb}{0.200000,0.200000,0.800000}%
\pgfsetfillcolor{currentfill}%
\pgfsetlinewidth{1.003750pt}%
\definecolor{currentstroke}{rgb}{0.200000,0.200000,0.800000}%
\pgfsetstrokecolor{currentstroke}%
\pgfsetdash{}{0pt}%
\pgfpathmoveto{\pgfqpoint{2.678960in}{2.413198in}}%
\pgfpathcurveto{\pgfqpoint{2.684784in}{2.413198in}}{\pgfqpoint{2.690370in}{2.415512in}}{\pgfqpoint{2.694488in}{2.419630in}}%
\pgfpathcurveto{\pgfqpoint{2.698607in}{2.423748in}}{\pgfqpoint{2.700920in}{2.429334in}}{\pgfqpoint{2.700920in}{2.435158in}}%
\pgfpathcurveto{\pgfqpoint{2.700920in}{2.440982in}}{\pgfqpoint{2.698607in}{2.446568in}}{\pgfqpoint{2.694488in}{2.450686in}}%
\pgfpathcurveto{\pgfqpoint{2.690370in}{2.454805in}}{\pgfqpoint{2.684784in}{2.457118in}}{\pgfqpoint{2.678960in}{2.457118in}}%
\pgfpathcurveto{\pgfqpoint{2.673136in}{2.457118in}}{\pgfqpoint{2.667550in}{2.454805in}}{\pgfqpoint{2.663432in}{2.450686in}}%
\pgfpathcurveto{\pgfqpoint{2.659314in}{2.446568in}}{\pgfqpoint{2.657000in}{2.440982in}}{\pgfqpoint{2.657000in}{2.435158in}}%
\pgfpathcurveto{\pgfqpoint{2.657000in}{2.429334in}}{\pgfqpoint{2.659314in}{2.423748in}}{\pgfqpoint{2.663432in}{2.419630in}}%
\pgfpathcurveto{\pgfqpoint{2.667550in}{2.415512in}}{\pgfqpoint{2.673136in}{2.413198in}}{\pgfqpoint{2.678960in}{2.413198in}}%
\pgfpathlineto{\pgfqpoint{2.678960in}{2.413198in}}%
\pgfpathclose%
\pgfusepath{stroke,fill}%
\end{pgfscope}%
\begin{pgfscope}%
\pgfpathrectangle{\pgfqpoint{0.997489in}{0.528000in}}{\pgfqpoint{4.565023in}{3.696000in}}%
\pgfusepath{clip}%
\pgfsetbuttcap%
\pgfsetroundjoin%
\definecolor{currentfill}{rgb}{0.200000,0.200000,0.800000}%
\pgfsetfillcolor{currentfill}%
\pgfsetlinewidth{1.003750pt}%
\definecolor{currentstroke}{rgb}{0.200000,0.200000,0.800000}%
\pgfsetstrokecolor{currentstroke}%
\pgfsetdash{}{0pt}%
\pgfpathmoveto{\pgfqpoint{3.690878in}{2.167375in}}%
\pgfpathcurveto{\pgfqpoint{3.696702in}{2.167375in}}{\pgfqpoint{3.702288in}{2.169689in}}{\pgfqpoint{3.706407in}{2.173807in}}%
\pgfpathcurveto{\pgfqpoint{3.710525in}{2.177925in}}{\pgfqpoint{3.712839in}{2.183511in}}{\pgfqpoint{3.712839in}{2.189335in}}%
\pgfpathcurveto{\pgfqpoint{3.712839in}{2.195159in}}{\pgfqpoint{3.710525in}{2.200745in}}{\pgfqpoint{3.706407in}{2.204863in}}%
\pgfpathcurveto{\pgfqpoint{3.702288in}{2.208981in}}{\pgfqpoint{3.696702in}{2.211295in}}{\pgfqpoint{3.690878in}{2.211295in}}%
\pgfpathcurveto{\pgfqpoint{3.685054in}{2.211295in}}{\pgfqpoint{3.679468in}{2.208981in}}{\pgfqpoint{3.675350in}{2.204863in}}%
\pgfpathcurveto{\pgfqpoint{3.671232in}{2.200745in}}{\pgfqpoint{3.668918in}{2.195159in}}{\pgfqpoint{3.668918in}{2.189335in}}%
\pgfpathcurveto{\pgfqpoint{3.668918in}{2.183511in}}{\pgfqpoint{3.671232in}{2.177925in}}{\pgfqpoint{3.675350in}{2.173807in}}%
\pgfpathcurveto{\pgfqpoint{3.679468in}{2.169689in}}{\pgfqpoint{3.685054in}{2.167375in}}{\pgfqpoint{3.690878in}{2.167375in}}%
\pgfpathlineto{\pgfqpoint{3.690878in}{2.167375in}}%
\pgfpathclose%
\pgfusepath{stroke,fill}%
\end{pgfscope}%
\begin{pgfscope}%
\pgfpathrectangle{\pgfqpoint{0.997489in}{0.528000in}}{\pgfqpoint{4.565023in}{3.696000in}}%
\pgfusepath{clip}%
\pgfsetbuttcap%
\pgfsetroundjoin%
\definecolor{currentfill}{rgb}{0.800000,0.800000,0.200000}%
\pgfsetfillcolor{currentfill}%
\pgfsetlinewidth{1.003750pt}%
\definecolor{currentstroke}{rgb}{0.800000,0.800000,0.200000}%
\pgfsetstrokecolor{currentstroke}%
\pgfsetdash{}{0pt}%
\pgfpathmoveto{\pgfqpoint{4.394623in}{3.668974in}}%
\pgfpathcurveto{\pgfqpoint{4.400447in}{3.668974in}}{\pgfqpoint{4.406033in}{3.671288in}}{\pgfqpoint{4.410152in}{3.675406in}}%
\pgfpathcurveto{\pgfqpoint{4.414270in}{3.679524in}}{\pgfqpoint{4.416584in}{3.685111in}}{\pgfqpoint{4.416584in}{3.690934in}}%
\pgfpathcurveto{\pgfqpoint{4.416584in}{3.696758in}}{\pgfqpoint{4.414270in}{3.702345in}}{\pgfqpoint{4.410152in}{3.706463in}}%
\pgfpathcurveto{\pgfqpoint{4.406033in}{3.710581in}}{\pgfqpoint{4.400447in}{3.712895in}}{\pgfqpoint{4.394623in}{3.712895in}}%
\pgfpathcurveto{\pgfqpoint{4.388799in}{3.712895in}}{\pgfqpoint{4.383213in}{3.710581in}}{\pgfqpoint{4.379095in}{3.706463in}}%
\pgfpathcurveto{\pgfqpoint{4.374977in}{3.702345in}}{\pgfqpoint{4.372663in}{3.696758in}}{\pgfqpoint{4.372663in}{3.690934in}}%
\pgfpathcurveto{\pgfqpoint{4.372663in}{3.685111in}}{\pgfqpoint{4.374977in}{3.679524in}}{\pgfqpoint{4.379095in}{3.675406in}}%
\pgfpathcurveto{\pgfqpoint{4.383213in}{3.671288in}}{\pgfqpoint{4.388799in}{3.668974in}}{\pgfqpoint{4.394623in}{3.668974in}}%
\pgfpathlineto{\pgfqpoint{4.394623in}{3.668974in}}%
\pgfpathclose%
\pgfusepath{stroke,fill}%
\end{pgfscope}%
\begin{pgfscope}%
\pgfpathrectangle{\pgfqpoint{0.997489in}{0.528000in}}{\pgfqpoint{4.565023in}{3.696000in}}%
\pgfusepath{clip}%
\pgfsetbuttcap%
\pgfsetroundjoin%
\definecolor{currentfill}{rgb}{0.800000,0.200000,0.200000}%
\pgfsetfillcolor{currentfill}%
\pgfsetlinewidth{1.003750pt}%
\definecolor{currentstroke}{rgb}{0.800000,0.200000,0.200000}%
\pgfsetstrokecolor{currentstroke}%
\pgfsetdash{}{0pt}%
\pgfpathmoveto{\pgfqpoint{1.526619in}{1.101808in}}%
\pgfpathcurveto{\pgfqpoint{1.532443in}{1.101808in}}{\pgfqpoint{1.538029in}{1.104122in}}{\pgfqpoint{1.542148in}{1.108240in}}%
\pgfpathcurveto{\pgfqpoint{1.546266in}{1.112358in}}{\pgfqpoint{1.548580in}{1.117944in}}{\pgfqpoint{1.548580in}{1.123768in}}%
\pgfpathcurveto{\pgfqpoint{1.548580in}{1.129592in}}{\pgfqpoint{1.546266in}{1.135178in}}{\pgfqpoint{1.542148in}{1.139297in}}%
\pgfpathcurveto{\pgfqpoint{1.538029in}{1.143415in}}{\pgfqpoint{1.532443in}{1.145729in}}{\pgfqpoint{1.526619in}{1.145729in}}%
\pgfpathcurveto{\pgfqpoint{1.520795in}{1.145729in}}{\pgfqpoint{1.515209in}{1.143415in}}{\pgfqpoint{1.511091in}{1.139297in}}%
\pgfpathcurveto{\pgfqpoint{1.506973in}{1.135178in}}{\pgfqpoint{1.504659in}{1.129592in}}{\pgfqpoint{1.504659in}{1.123768in}}%
\pgfpathcurveto{\pgfqpoint{1.504659in}{1.117944in}}{\pgfqpoint{1.506973in}{1.112358in}}{\pgfqpoint{1.511091in}{1.108240in}}%
\pgfpathcurveto{\pgfqpoint{1.515209in}{1.104122in}}{\pgfqpoint{1.520795in}{1.101808in}}{\pgfqpoint{1.526619in}{1.101808in}}%
\pgfpathlineto{\pgfqpoint{1.526619in}{1.101808in}}%
\pgfpathclose%
\pgfusepath{stroke,fill}%
\end{pgfscope}%
\begin{pgfscope}%
\pgfpathrectangle{\pgfqpoint{0.997489in}{0.528000in}}{\pgfqpoint{4.565023in}{3.696000in}}%
\pgfusepath{clip}%
\pgfsetbuttcap%
\pgfsetroundjoin%
\definecolor{currentfill}{rgb}{0.200000,0.200000,0.800000}%
\pgfsetfillcolor{currentfill}%
\pgfsetlinewidth{1.003750pt}%
\definecolor{currentstroke}{rgb}{0.200000,0.200000,0.800000}%
\pgfsetstrokecolor{currentstroke}%
\pgfsetdash{}{0pt}%
\pgfpathmoveto{\pgfqpoint{3.807442in}{2.353989in}}%
\pgfpathcurveto{\pgfqpoint{3.813266in}{2.353989in}}{\pgfqpoint{3.818852in}{2.356303in}}{\pgfqpoint{3.822970in}{2.360421in}}%
\pgfpathcurveto{\pgfqpoint{3.827088in}{2.364539in}}{\pgfqpoint{3.829402in}{2.370126in}}{\pgfqpoint{3.829402in}{2.375949in}}%
\pgfpathcurveto{\pgfqpoint{3.829402in}{2.381773in}}{\pgfqpoint{3.827088in}{2.387360in}}{\pgfqpoint{3.822970in}{2.391478in}}%
\pgfpathcurveto{\pgfqpoint{3.818852in}{2.395596in}}{\pgfqpoint{3.813266in}{2.397910in}}{\pgfqpoint{3.807442in}{2.397910in}}%
\pgfpathcurveto{\pgfqpoint{3.801618in}{2.397910in}}{\pgfqpoint{3.796032in}{2.395596in}}{\pgfqpoint{3.791914in}{2.391478in}}%
\pgfpathcurveto{\pgfqpoint{3.787796in}{2.387360in}}{\pgfqpoint{3.785482in}{2.381773in}}{\pgfqpoint{3.785482in}{2.375949in}}%
\pgfpathcurveto{\pgfqpoint{3.785482in}{2.370126in}}{\pgfqpoint{3.787796in}{2.364539in}}{\pgfqpoint{3.791914in}{2.360421in}}%
\pgfpathcurveto{\pgfqpoint{3.796032in}{2.356303in}}{\pgfqpoint{3.801618in}{2.353989in}}{\pgfqpoint{3.807442in}{2.353989in}}%
\pgfpathlineto{\pgfqpoint{3.807442in}{2.353989in}}%
\pgfpathclose%
\pgfusepath{stroke,fill}%
\end{pgfscope}%
\begin{pgfscope}%
\pgfpathrectangle{\pgfqpoint{0.997489in}{0.528000in}}{\pgfqpoint{4.565023in}{3.696000in}}%
\pgfusepath{clip}%
\pgfsetbuttcap%
\pgfsetroundjoin%
\definecolor{currentfill}{rgb}{0.800000,0.200000,0.200000}%
\pgfsetfillcolor{currentfill}%
\pgfsetlinewidth{1.003750pt}%
\definecolor{currentstroke}{rgb}{0.800000,0.200000,0.200000}%
\pgfsetstrokecolor{currentstroke}%
\pgfsetdash{}{0pt}%
\pgfpathmoveto{\pgfqpoint{2.511972in}{3.165519in}}%
\pgfpathcurveto{\pgfqpoint{2.517796in}{3.165519in}}{\pgfqpoint{2.523382in}{3.167833in}}{\pgfqpoint{2.527500in}{3.171951in}}%
\pgfpathcurveto{\pgfqpoint{2.531618in}{3.176069in}}{\pgfqpoint{2.533932in}{3.181655in}}{\pgfqpoint{2.533932in}{3.187479in}}%
\pgfpathcurveto{\pgfqpoint{2.533932in}{3.193303in}}{\pgfqpoint{2.531618in}{3.198889in}}{\pgfqpoint{2.527500in}{3.203007in}}%
\pgfpathcurveto{\pgfqpoint{2.523382in}{3.207125in}}{\pgfqpoint{2.517796in}{3.209439in}}{\pgfqpoint{2.511972in}{3.209439in}}%
\pgfpathcurveto{\pgfqpoint{2.506148in}{3.209439in}}{\pgfqpoint{2.500562in}{3.207125in}}{\pgfqpoint{2.496444in}{3.203007in}}%
\pgfpathcurveto{\pgfqpoint{2.492326in}{3.198889in}}{\pgfqpoint{2.490012in}{3.193303in}}{\pgfqpoint{2.490012in}{3.187479in}}%
\pgfpathcurveto{\pgfqpoint{2.490012in}{3.181655in}}{\pgfqpoint{2.492326in}{3.176069in}}{\pgfqpoint{2.496444in}{3.171951in}}%
\pgfpathcurveto{\pgfqpoint{2.500562in}{3.167833in}}{\pgfqpoint{2.506148in}{3.165519in}}{\pgfqpoint{2.511972in}{3.165519in}}%
\pgfpathlineto{\pgfqpoint{2.511972in}{3.165519in}}%
\pgfpathclose%
\pgfusepath{stroke,fill}%
\end{pgfscope}%
\begin{pgfscope}%
\pgfpathrectangle{\pgfqpoint{0.997489in}{0.528000in}}{\pgfqpoint{4.565023in}{3.696000in}}%
\pgfusepath{clip}%
\pgfsetbuttcap%
\pgfsetroundjoin%
\definecolor{currentfill}{rgb}{0.200000,0.800000,0.200000}%
\pgfsetfillcolor{currentfill}%
\pgfsetlinewidth{1.003750pt}%
\definecolor{currentstroke}{rgb}{0.200000,0.800000,0.200000}%
\pgfsetstrokecolor{currentstroke}%
\pgfsetdash{}{0pt}%
\pgfpathmoveto{\pgfqpoint{1.844683in}{1.904552in}}%
\pgfpathcurveto{\pgfqpoint{1.850507in}{1.904552in}}{\pgfqpoint{1.856093in}{1.906866in}}{\pgfqpoint{1.860211in}{1.910984in}}%
\pgfpathcurveto{\pgfqpoint{1.864329in}{1.915102in}}{\pgfqpoint{1.866643in}{1.920688in}}{\pgfqpoint{1.866643in}{1.926512in}}%
\pgfpathcurveto{\pgfqpoint{1.866643in}{1.932336in}}{\pgfqpoint{1.864329in}{1.937922in}}{\pgfqpoint{1.860211in}{1.942040in}}%
\pgfpathcurveto{\pgfqpoint{1.856093in}{1.946158in}}{\pgfqpoint{1.850507in}{1.948472in}}{\pgfqpoint{1.844683in}{1.948472in}}%
\pgfpathcurveto{\pgfqpoint{1.838859in}{1.948472in}}{\pgfqpoint{1.833273in}{1.946158in}}{\pgfqpoint{1.829155in}{1.942040in}}%
\pgfpathcurveto{\pgfqpoint{1.825036in}{1.937922in}}{\pgfqpoint{1.822723in}{1.932336in}}{\pgfqpoint{1.822723in}{1.926512in}}%
\pgfpathcurveto{\pgfqpoint{1.822723in}{1.920688in}}{\pgfqpoint{1.825036in}{1.915102in}}{\pgfqpoint{1.829155in}{1.910984in}}%
\pgfpathcurveto{\pgfqpoint{1.833273in}{1.906866in}}{\pgfqpoint{1.838859in}{1.904552in}}{\pgfqpoint{1.844683in}{1.904552in}}%
\pgfpathlineto{\pgfqpoint{1.844683in}{1.904552in}}%
\pgfpathclose%
\pgfusepath{stroke,fill}%
\end{pgfscope}%
\begin{pgfscope}%
\pgfpathrectangle{\pgfqpoint{0.997489in}{0.528000in}}{\pgfqpoint{4.565023in}{3.696000in}}%
\pgfusepath{clip}%
\pgfsetbuttcap%
\pgfsetroundjoin%
\definecolor{currentfill}{rgb}{0.800000,0.200000,0.200000}%
\pgfsetfillcolor{currentfill}%
\pgfsetlinewidth{1.003750pt}%
\definecolor{currentstroke}{rgb}{0.800000,0.200000,0.200000}%
\pgfsetstrokecolor{currentstroke}%
\pgfsetdash{}{0pt}%
\pgfpathmoveto{\pgfqpoint{1.964435in}{2.952013in}}%
\pgfpathcurveto{\pgfqpoint{1.970259in}{2.952013in}}{\pgfqpoint{1.975845in}{2.954327in}}{\pgfqpoint{1.979963in}{2.958445in}}%
\pgfpathcurveto{\pgfqpoint{1.984081in}{2.962563in}}{\pgfqpoint{1.986395in}{2.968150in}}{\pgfqpoint{1.986395in}{2.973973in}}%
\pgfpathcurveto{\pgfqpoint{1.986395in}{2.979797in}}{\pgfqpoint{1.984081in}{2.985384in}}{\pgfqpoint{1.979963in}{2.989502in}}%
\pgfpathcurveto{\pgfqpoint{1.975845in}{2.993620in}}{\pgfqpoint{1.970259in}{2.995934in}}{\pgfqpoint{1.964435in}{2.995934in}}%
\pgfpathcurveto{\pgfqpoint{1.958611in}{2.995934in}}{\pgfqpoint{1.953025in}{2.993620in}}{\pgfqpoint{1.948907in}{2.989502in}}%
\pgfpathcurveto{\pgfqpoint{1.944788in}{2.985384in}}{\pgfqpoint{1.942475in}{2.979797in}}{\pgfqpoint{1.942475in}{2.973973in}}%
\pgfpathcurveto{\pgfqpoint{1.942475in}{2.968150in}}{\pgfqpoint{1.944788in}{2.962563in}}{\pgfqpoint{1.948907in}{2.958445in}}%
\pgfpathcurveto{\pgfqpoint{1.953025in}{2.954327in}}{\pgfqpoint{1.958611in}{2.952013in}}{\pgfqpoint{1.964435in}{2.952013in}}%
\pgfpathlineto{\pgfqpoint{1.964435in}{2.952013in}}%
\pgfpathclose%
\pgfusepath{stroke,fill}%
\end{pgfscope}%
\begin{pgfscope}%
\pgfpathrectangle{\pgfqpoint{0.997489in}{0.528000in}}{\pgfqpoint{4.565023in}{3.696000in}}%
\pgfusepath{clip}%
\pgfsetbuttcap%
\pgfsetroundjoin%
\definecolor{currentfill}{rgb}{0.800000,0.800000,0.200000}%
\pgfsetfillcolor{currentfill}%
\pgfsetlinewidth{1.003750pt}%
\definecolor{currentstroke}{rgb}{0.800000,0.800000,0.200000}%
\pgfsetstrokecolor{currentstroke}%
\pgfsetdash{}{0pt}%
\pgfpathmoveto{\pgfqpoint{4.235621in}{3.457607in}}%
\pgfpathcurveto{\pgfqpoint{4.241444in}{3.457607in}}{\pgfqpoint{4.247031in}{3.459921in}}{\pgfqpoint{4.251149in}{3.464039in}}%
\pgfpathcurveto{\pgfqpoint{4.255267in}{3.468157in}}{\pgfqpoint{4.257581in}{3.473743in}}{\pgfqpoint{4.257581in}{3.479567in}}%
\pgfpathcurveto{\pgfqpoint{4.257581in}{3.485391in}}{\pgfqpoint{4.255267in}{3.490977in}}{\pgfqpoint{4.251149in}{3.495095in}}%
\pgfpathcurveto{\pgfqpoint{4.247031in}{3.499213in}}{\pgfqpoint{4.241444in}{3.501527in}}{\pgfqpoint{4.235621in}{3.501527in}}%
\pgfpathcurveto{\pgfqpoint{4.229797in}{3.501527in}}{\pgfqpoint{4.224210in}{3.499213in}}{\pgfqpoint{4.220092in}{3.495095in}}%
\pgfpathcurveto{\pgfqpoint{4.215974in}{3.490977in}}{\pgfqpoint{4.213660in}{3.485391in}}{\pgfqpoint{4.213660in}{3.479567in}}%
\pgfpathcurveto{\pgfqpoint{4.213660in}{3.473743in}}{\pgfqpoint{4.215974in}{3.468157in}}{\pgfqpoint{4.220092in}{3.464039in}}%
\pgfpathcurveto{\pgfqpoint{4.224210in}{3.459921in}}{\pgfqpoint{4.229797in}{3.457607in}}{\pgfqpoint{4.235621in}{3.457607in}}%
\pgfpathlineto{\pgfqpoint{4.235621in}{3.457607in}}%
\pgfpathclose%
\pgfusepath{stroke,fill}%
\end{pgfscope}%
\begin{pgfscope}%
\pgfpathrectangle{\pgfqpoint{0.997489in}{0.528000in}}{\pgfqpoint{4.565023in}{3.696000in}}%
\pgfusepath{clip}%
\pgfsetbuttcap%
\pgfsetroundjoin%
\definecolor{currentfill}{rgb}{0.800000,0.200000,0.200000}%
\pgfsetfillcolor{currentfill}%
\pgfsetlinewidth{1.003750pt}%
\definecolor{currentstroke}{rgb}{0.800000,0.200000,0.200000}%
\pgfsetstrokecolor{currentstroke}%
\pgfsetdash{}{0pt}%
\pgfpathmoveto{\pgfqpoint{3.880074in}{1.478457in}}%
\pgfpathcurveto{\pgfqpoint{3.885898in}{1.478457in}}{\pgfqpoint{3.891484in}{1.480771in}}{\pgfqpoint{3.895602in}{1.484889in}}%
\pgfpathcurveto{\pgfqpoint{3.899720in}{1.489007in}}{\pgfqpoint{3.902034in}{1.494594in}}{\pgfqpoint{3.902034in}{1.500418in}}%
\pgfpathcurveto{\pgfqpoint{3.902034in}{1.506242in}}{\pgfqpoint{3.899720in}{1.511828in}}{\pgfqpoint{3.895602in}{1.515946in}}%
\pgfpathcurveto{\pgfqpoint{3.891484in}{1.520064in}}{\pgfqpoint{3.885898in}{1.522378in}}{\pgfqpoint{3.880074in}{1.522378in}}%
\pgfpathcurveto{\pgfqpoint{3.874250in}{1.522378in}}{\pgfqpoint{3.868664in}{1.520064in}}{\pgfqpoint{3.864546in}{1.515946in}}%
\pgfpathcurveto{\pgfqpoint{3.860427in}{1.511828in}}{\pgfqpoint{3.858114in}{1.506242in}}{\pgfqpoint{3.858114in}{1.500418in}}%
\pgfpathcurveto{\pgfqpoint{3.858114in}{1.494594in}}{\pgfqpoint{3.860427in}{1.489007in}}{\pgfqpoint{3.864546in}{1.484889in}}%
\pgfpathcurveto{\pgfqpoint{3.868664in}{1.480771in}}{\pgfqpoint{3.874250in}{1.478457in}}{\pgfqpoint{3.880074in}{1.478457in}}%
\pgfpathlineto{\pgfqpoint{3.880074in}{1.478457in}}%
\pgfpathclose%
\pgfusepath{stroke,fill}%
\end{pgfscope}%
\begin{pgfscope}%
\pgfpathrectangle{\pgfqpoint{0.997489in}{0.528000in}}{\pgfqpoint{4.565023in}{3.696000in}}%
\pgfusepath{clip}%
\pgfsetbuttcap%
\pgfsetroundjoin%
\definecolor{currentfill}{rgb}{0.800000,0.200000,0.200000}%
\pgfsetfillcolor{currentfill}%
\pgfsetlinewidth{1.003750pt}%
\definecolor{currentstroke}{rgb}{0.800000,0.200000,0.200000}%
\pgfsetstrokecolor{currentstroke}%
\pgfsetdash{}{0pt}%
\pgfpathmoveto{\pgfqpoint{2.335186in}{1.415768in}}%
\pgfpathcurveto{\pgfqpoint{2.341010in}{1.415768in}}{\pgfqpoint{2.346596in}{1.418082in}}{\pgfqpoint{2.350715in}{1.422200in}}%
\pgfpathcurveto{\pgfqpoint{2.354833in}{1.426318in}}{\pgfqpoint{2.357147in}{1.431904in}}{\pgfqpoint{2.357147in}{1.437728in}}%
\pgfpathcurveto{\pgfqpoint{2.357147in}{1.443552in}}{\pgfqpoint{2.354833in}{1.449138in}}{\pgfqpoint{2.350715in}{1.453257in}}%
\pgfpathcurveto{\pgfqpoint{2.346596in}{1.457375in}}{\pgfqpoint{2.341010in}{1.459689in}}{\pgfqpoint{2.335186in}{1.459689in}}%
\pgfpathcurveto{\pgfqpoint{2.329362in}{1.459689in}}{\pgfqpoint{2.323776in}{1.457375in}}{\pgfqpoint{2.319658in}{1.453257in}}%
\pgfpathcurveto{\pgfqpoint{2.315540in}{1.449138in}}{\pgfqpoint{2.313226in}{1.443552in}}{\pgfqpoint{2.313226in}{1.437728in}}%
\pgfpathcurveto{\pgfqpoint{2.313226in}{1.431904in}}{\pgfqpoint{2.315540in}{1.426318in}}{\pgfqpoint{2.319658in}{1.422200in}}%
\pgfpathcurveto{\pgfqpoint{2.323776in}{1.418082in}}{\pgfqpoint{2.329362in}{1.415768in}}{\pgfqpoint{2.335186in}{1.415768in}}%
\pgfpathlineto{\pgfqpoint{2.335186in}{1.415768in}}%
\pgfpathclose%
\pgfusepath{stroke,fill}%
\end{pgfscope}%
\begin{pgfscope}%
\pgfpathrectangle{\pgfqpoint{0.997489in}{0.528000in}}{\pgfqpoint{4.565023in}{3.696000in}}%
\pgfusepath{clip}%
\pgfsetbuttcap%
\pgfsetroundjoin%
\definecolor{currentfill}{rgb}{0.800000,0.200000,0.200000}%
\pgfsetfillcolor{currentfill}%
\pgfsetlinewidth{1.003750pt}%
\definecolor{currentstroke}{rgb}{0.800000,0.200000,0.200000}%
\pgfsetstrokecolor{currentstroke}%
\pgfsetdash{}{0pt}%
\pgfpathmoveto{\pgfqpoint{4.286244in}{1.785904in}}%
\pgfpathcurveto{\pgfqpoint{4.292068in}{1.785904in}}{\pgfqpoint{4.297654in}{1.788217in}}{\pgfqpoint{4.301772in}{1.792336in}}%
\pgfpathcurveto{\pgfqpoint{4.305890in}{1.796454in}}{\pgfqpoint{4.308204in}{1.802040in}}{\pgfqpoint{4.308204in}{1.807864in}}%
\pgfpathcurveto{\pgfqpoint{4.308204in}{1.813688in}}{\pgfqpoint{4.305890in}{1.819274in}}{\pgfqpoint{4.301772in}{1.823392in}}%
\pgfpathcurveto{\pgfqpoint{4.297654in}{1.827510in}}{\pgfqpoint{4.292068in}{1.829824in}}{\pgfqpoint{4.286244in}{1.829824in}}%
\pgfpathcurveto{\pgfqpoint{4.280420in}{1.829824in}}{\pgfqpoint{4.274834in}{1.827510in}}{\pgfqpoint{4.270716in}{1.823392in}}%
\pgfpathcurveto{\pgfqpoint{4.266598in}{1.819274in}}{\pgfqpoint{4.264284in}{1.813688in}}{\pgfqpoint{4.264284in}{1.807864in}}%
\pgfpathcurveto{\pgfqpoint{4.264284in}{1.802040in}}{\pgfqpoint{4.266598in}{1.796454in}}{\pgfqpoint{4.270716in}{1.792336in}}%
\pgfpathcurveto{\pgfqpoint{4.274834in}{1.788217in}}{\pgfqpoint{4.280420in}{1.785904in}}{\pgfqpoint{4.286244in}{1.785904in}}%
\pgfpathlineto{\pgfqpoint{4.286244in}{1.785904in}}%
\pgfpathclose%
\pgfusepath{stroke,fill}%
\end{pgfscope}%
\begin{pgfscope}%
\pgfpathrectangle{\pgfqpoint{0.997489in}{0.528000in}}{\pgfqpoint{4.565023in}{3.696000in}}%
\pgfusepath{clip}%
\pgfsetbuttcap%
\pgfsetroundjoin%
\definecolor{currentfill}{rgb}{0.200000,0.800000,0.200000}%
\pgfsetfillcolor{currentfill}%
\pgfsetlinewidth{1.003750pt}%
\definecolor{currentstroke}{rgb}{0.200000,0.800000,0.200000}%
\pgfsetstrokecolor{currentstroke}%
\pgfsetdash{}{0pt}%
\pgfpathmoveto{\pgfqpoint{1.694182in}{1.965207in}}%
\pgfpathcurveto{\pgfqpoint{1.700006in}{1.965207in}}{\pgfqpoint{1.705592in}{1.967521in}}{\pgfqpoint{1.709711in}{1.971639in}}%
\pgfpathcurveto{\pgfqpoint{1.713829in}{1.975757in}}{\pgfqpoint{1.716143in}{1.981343in}}{\pgfqpoint{1.716143in}{1.987167in}}%
\pgfpathcurveto{\pgfqpoint{1.716143in}{1.992991in}}{\pgfqpoint{1.713829in}{1.998577in}}{\pgfqpoint{1.709711in}{2.002695in}}%
\pgfpathcurveto{\pgfqpoint{1.705592in}{2.006814in}}{\pgfqpoint{1.700006in}{2.009128in}}{\pgfqpoint{1.694182in}{2.009128in}}%
\pgfpathcurveto{\pgfqpoint{1.688358in}{2.009128in}}{\pgfqpoint{1.682772in}{2.006814in}}{\pgfqpoint{1.678654in}{2.002695in}}%
\pgfpathcurveto{\pgfqpoint{1.674536in}{1.998577in}}{\pgfqpoint{1.672222in}{1.992991in}}{\pgfqpoint{1.672222in}{1.987167in}}%
\pgfpathcurveto{\pgfqpoint{1.672222in}{1.981343in}}{\pgfqpoint{1.674536in}{1.975757in}}{\pgfqpoint{1.678654in}{1.971639in}}%
\pgfpathcurveto{\pgfqpoint{1.682772in}{1.967521in}}{\pgfqpoint{1.688358in}{1.965207in}}{\pgfqpoint{1.694182in}{1.965207in}}%
\pgfpathlineto{\pgfqpoint{1.694182in}{1.965207in}}%
\pgfpathclose%
\pgfusepath{stroke,fill}%
\end{pgfscope}%
\begin{pgfscope}%
\pgfpathrectangle{\pgfqpoint{0.997489in}{0.528000in}}{\pgfqpoint{4.565023in}{3.696000in}}%
\pgfusepath{clip}%
\pgfsetbuttcap%
\pgfsetroundjoin%
\definecolor{currentfill}{rgb}{0.800000,0.200000,0.200000}%
\pgfsetfillcolor{currentfill}%
\pgfsetlinewidth{1.003750pt}%
\definecolor{currentstroke}{rgb}{0.800000,0.200000,0.200000}%
\pgfsetstrokecolor{currentstroke}%
\pgfsetdash{}{0pt}%
\pgfpathmoveto{\pgfqpoint{2.368738in}{1.010893in}}%
\pgfpathcurveto{\pgfqpoint{2.374562in}{1.010893in}}{\pgfqpoint{2.380148in}{1.013206in}}{\pgfqpoint{2.384266in}{1.017325in}}%
\pgfpathcurveto{\pgfqpoint{2.388384in}{1.021443in}}{\pgfqpoint{2.390698in}{1.027029in}}{\pgfqpoint{2.390698in}{1.032853in}}%
\pgfpathcurveto{\pgfqpoint{2.390698in}{1.038677in}}{\pgfqpoint{2.388384in}{1.044263in}}{\pgfqpoint{2.384266in}{1.048381in}}%
\pgfpathcurveto{\pgfqpoint{2.380148in}{1.052499in}}{\pgfqpoint{2.374562in}{1.054813in}}{\pgfqpoint{2.368738in}{1.054813in}}%
\pgfpathcurveto{\pgfqpoint{2.362914in}{1.054813in}}{\pgfqpoint{2.357328in}{1.052499in}}{\pgfqpoint{2.353210in}{1.048381in}}%
\pgfpathcurveto{\pgfqpoint{2.349091in}{1.044263in}}{\pgfqpoint{2.346778in}{1.038677in}}{\pgfqpoint{2.346778in}{1.032853in}}%
\pgfpathcurveto{\pgfqpoint{2.346778in}{1.027029in}}{\pgfqpoint{2.349091in}{1.021443in}}{\pgfqpoint{2.353210in}{1.017325in}}%
\pgfpathcurveto{\pgfqpoint{2.357328in}{1.013206in}}{\pgfqpoint{2.362914in}{1.010893in}}{\pgfqpoint{2.368738in}{1.010893in}}%
\pgfpathlineto{\pgfqpoint{2.368738in}{1.010893in}}%
\pgfpathclose%
\pgfusepath{stroke,fill}%
\end{pgfscope}%
\begin{pgfscope}%
\pgfpathrectangle{\pgfqpoint{0.997489in}{0.528000in}}{\pgfqpoint{4.565023in}{3.696000in}}%
\pgfusepath{clip}%
\pgfsetbuttcap%
\pgfsetroundjoin%
\definecolor{currentfill}{rgb}{0.800000,0.200000,0.200000}%
\pgfsetfillcolor{currentfill}%
\pgfsetlinewidth{1.003750pt}%
\definecolor{currentstroke}{rgb}{0.800000,0.200000,0.200000}%
\pgfsetstrokecolor{currentstroke}%
\pgfsetdash{}{0pt}%
\pgfpathmoveto{\pgfqpoint{3.394417in}{1.384192in}}%
\pgfpathcurveto{\pgfqpoint{3.400241in}{1.384192in}}{\pgfqpoint{3.405827in}{1.386506in}}{\pgfqpoint{3.409945in}{1.390624in}}%
\pgfpathcurveto{\pgfqpoint{3.414064in}{1.394742in}}{\pgfqpoint{3.416377in}{1.400328in}}{\pgfqpoint{3.416377in}{1.406152in}}%
\pgfpathcurveto{\pgfqpoint{3.416377in}{1.411976in}}{\pgfqpoint{3.414064in}{1.417562in}}{\pgfqpoint{3.409945in}{1.421681in}}%
\pgfpathcurveto{\pgfqpoint{3.405827in}{1.425799in}}{\pgfqpoint{3.400241in}{1.428113in}}{\pgfqpoint{3.394417in}{1.428113in}}%
\pgfpathcurveto{\pgfqpoint{3.388593in}{1.428113in}}{\pgfqpoint{3.383007in}{1.425799in}}{\pgfqpoint{3.378889in}{1.421681in}}%
\pgfpathcurveto{\pgfqpoint{3.374771in}{1.417562in}}{\pgfqpoint{3.372457in}{1.411976in}}{\pgfqpoint{3.372457in}{1.406152in}}%
\pgfpathcurveto{\pgfqpoint{3.372457in}{1.400328in}}{\pgfqpoint{3.374771in}{1.394742in}}{\pgfqpoint{3.378889in}{1.390624in}}%
\pgfpathcurveto{\pgfqpoint{3.383007in}{1.386506in}}{\pgfqpoint{3.388593in}{1.384192in}}{\pgfqpoint{3.394417in}{1.384192in}}%
\pgfpathlineto{\pgfqpoint{3.394417in}{1.384192in}}%
\pgfpathclose%
\pgfusepath{stroke,fill}%
\end{pgfscope}%
\begin{pgfscope}%
\pgfpathrectangle{\pgfqpoint{0.997489in}{0.528000in}}{\pgfqpoint{4.565023in}{3.696000in}}%
\pgfusepath{clip}%
\pgfsetbuttcap%
\pgfsetroundjoin%
\definecolor{currentfill}{rgb}{0.800000,0.800000,0.200000}%
\pgfsetfillcolor{currentfill}%
\pgfsetlinewidth{1.003750pt}%
\definecolor{currentstroke}{rgb}{0.800000,0.800000,0.200000}%
\pgfsetstrokecolor{currentstroke}%
\pgfsetdash{}{0pt}%
\pgfpathmoveto{\pgfqpoint{4.470120in}{3.377092in}}%
\pgfpathcurveto{\pgfqpoint{4.475944in}{3.377092in}}{\pgfqpoint{4.481530in}{3.379406in}}{\pgfqpoint{4.485648in}{3.383524in}}%
\pgfpathcurveto{\pgfqpoint{4.489766in}{3.387642in}}{\pgfqpoint{4.492080in}{3.393228in}}{\pgfqpoint{4.492080in}{3.399052in}}%
\pgfpathcurveto{\pgfqpoint{4.492080in}{3.404876in}}{\pgfqpoint{4.489766in}{3.410462in}}{\pgfqpoint{4.485648in}{3.414580in}}%
\pgfpathcurveto{\pgfqpoint{4.481530in}{3.418698in}}{\pgfqpoint{4.475944in}{3.421012in}}{\pgfqpoint{4.470120in}{3.421012in}}%
\pgfpathcurveto{\pgfqpoint{4.464296in}{3.421012in}}{\pgfqpoint{4.458710in}{3.418698in}}{\pgfqpoint{4.454592in}{3.414580in}}%
\pgfpathcurveto{\pgfqpoint{4.450474in}{3.410462in}}{\pgfqpoint{4.448160in}{3.404876in}}{\pgfqpoint{4.448160in}{3.399052in}}%
\pgfpathcurveto{\pgfqpoint{4.448160in}{3.393228in}}{\pgfqpoint{4.450474in}{3.387642in}}{\pgfqpoint{4.454592in}{3.383524in}}%
\pgfpathcurveto{\pgfqpoint{4.458710in}{3.379406in}}{\pgfqpoint{4.464296in}{3.377092in}}{\pgfqpoint{4.470120in}{3.377092in}}%
\pgfpathlineto{\pgfqpoint{4.470120in}{3.377092in}}%
\pgfpathclose%
\pgfusepath{stroke,fill}%
\end{pgfscope}%
\begin{pgfscope}%
\pgfpathrectangle{\pgfqpoint{0.997489in}{0.528000in}}{\pgfqpoint{4.565023in}{3.696000in}}%
\pgfusepath{clip}%
\pgfsetbuttcap%
\pgfsetroundjoin%
\definecolor{currentfill}{rgb}{0.200000,0.800000,0.200000}%
\pgfsetfillcolor{currentfill}%
\pgfsetlinewidth{1.003750pt}%
\definecolor{currentstroke}{rgb}{0.200000,0.800000,0.200000}%
\pgfsetstrokecolor{currentstroke}%
\pgfsetdash{}{0pt}%
\pgfpathmoveto{\pgfqpoint{2.813008in}{2.425137in}}%
\pgfpathcurveto{\pgfqpoint{2.818832in}{2.425137in}}{\pgfqpoint{2.824418in}{2.427451in}}{\pgfqpoint{2.828536in}{2.431569in}}%
\pgfpathcurveto{\pgfqpoint{2.832654in}{2.435687in}}{\pgfqpoint{2.834968in}{2.441274in}}{\pgfqpoint{2.834968in}{2.447098in}}%
\pgfpathcurveto{\pgfqpoint{2.834968in}{2.452922in}}{\pgfqpoint{2.832654in}{2.458508in}}{\pgfqpoint{2.828536in}{2.462626in}}%
\pgfpathcurveto{\pgfqpoint{2.824418in}{2.466744in}}{\pgfqpoint{2.818832in}{2.469058in}}{\pgfqpoint{2.813008in}{2.469058in}}%
\pgfpathcurveto{\pgfqpoint{2.807184in}{2.469058in}}{\pgfqpoint{2.801598in}{2.466744in}}{\pgfqpoint{2.797480in}{2.462626in}}%
\pgfpathcurveto{\pgfqpoint{2.793362in}{2.458508in}}{\pgfqpoint{2.791048in}{2.452922in}}{\pgfqpoint{2.791048in}{2.447098in}}%
\pgfpathcurveto{\pgfqpoint{2.791048in}{2.441274in}}{\pgfqpoint{2.793362in}{2.435687in}}{\pgfqpoint{2.797480in}{2.431569in}}%
\pgfpathcurveto{\pgfqpoint{2.801598in}{2.427451in}}{\pgfqpoint{2.807184in}{2.425137in}}{\pgfqpoint{2.813008in}{2.425137in}}%
\pgfpathlineto{\pgfqpoint{2.813008in}{2.425137in}}%
\pgfpathclose%
\pgfusepath{stroke,fill}%
\end{pgfscope}%
\begin{pgfscope}%
\pgfpathrectangle{\pgfqpoint{0.997489in}{0.528000in}}{\pgfqpoint{4.565023in}{3.696000in}}%
\pgfusepath{clip}%
\pgfsetbuttcap%
\pgfsetroundjoin%
\definecolor{currentfill}{rgb}{0.800000,0.200000,0.200000}%
\pgfsetfillcolor{currentfill}%
\pgfsetlinewidth{1.003750pt}%
\definecolor{currentstroke}{rgb}{0.800000,0.200000,0.200000}%
\pgfsetstrokecolor{currentstroke}%
\pgfsetdash{}{0pt}%
\pgfpathmoveto{\pgfqpoint{3.884288in}{1.497603in}}%
\pgfpathcurveto{\pgfqpoint{3.890112in}{1.497603in}}{\pgfqpoint{3.895698in}{1.499917in}}{\pgfqpoint{3.899816in}{1.504035in}}%
\pgfpathcurveto{\pgfqpoint{3.903934in}{1.508154in}}{\pgfqpoint{3.906248in}{1.513740in}}{\pgfqpoint{3.906248in}{1.519564in}}%
\pgfpathcurveto{\pgfqpoint{3.906248in}{1.525388in}}{\pgfqpoint{3.903934in}{1.530974in}}{\pgfqpoint{3.899816in}{1.535092in}}%
\pgfpathcurveto{\pgfqpoint{3.895698in}{1.539210in}}{\pgfqpoint{3.890112in}{1.541524in}}{\pgfqpoint{3.884288in}{1.541524in}}%
\pgfpathcurveto{\pgfqpoint{3.878464in}{1.541524in}}{\pgfqpoint{3.872878in}{1.539210in}}{\pgfqpoint{3.868760in}{1.535092in}}%
\pgfpathcurveto{\pgfqpoint{3.864641in}{1.530974in}}{\pgfqpoint{3.862327in}{1.525388in}}{\pgfqpoint{3.862327in}{1.519564in}}%
\pgfpathcurveto{\pgfqpoint{3.862327in}{1.513740in}}{\pgfqpoint{3.864641in}{1.508154in}}{\pgfqpoint{3.868760in}{1.504035in}}%
\pgfpathcurveto{\pgfqpoint{3.872878in}{1.499917in}}{\pgfqpoint{3.878464in}{1.497603in}}{\pgfqpoint{3.884288in}{1.497603in}}%
\pgfpathlineto{\pgfqpoint{3.884288in}{1.497603in}}%
\pgfpathclose%
\pgfusepath{stroke,fill}%
\end{pgfscope}%
\begin{pgfscope}%
\pgfpathrectangle{\pgfqpoint{0.997489in}{0.528000in}}{\pgfqpoint{4.565023in}{3.696000in}}%
\pgfusepath{clip}%
\pgfsetbuttcap%
\pgfsetroundjoin%
\definecolor{currentfill}{rgb}{0.800000,0.800000,0.200000}%
\pgfsetfillcolor{currentfill}%
\pgfsetlinewidth{1.003750pt}%
\definecolor{currentstroke}{rgb}{0.800000,0.800000,0.200000}%
\pgfsetstrokecolor{currentstroke}%
\pgfsetdash{}{0pt}%
\pgfpathmoveto{\pgfqpoint{4.210729in}{3.595545in}}%
\pgfpathcurveto{\pgfqpoint{4.216553in}{3.595545in}}{\pgfqpoint{4.222139in}{3.597859in}}{\pgfqpoint{4.226257in}{3.601977in}}%
\pgfpathcurveto{\pgfqpoint{4.230375in}{3.606095in}}{\pgfqpoint{4.232689in}{3.611682in}}{\pgfqpoint{4.232689in}{3.617505in}}%
\pgfpathcurveto{\pgfqpoint{4.232689in}{3.623329in}}{\pgfqpoint{4.230375in}{3.628916in}}{\pgfqpoint{4.226257in}{3.633034in}}%
\pgfpathcurveto{\pgfqpoint{4.222139in}{3.637152in}}{\pgfqpoint{4.216553in}{3.639466in}}{\pgfqpoint{4.210729in}{3.639466in}}%
\pgfpathcurveto{\pgfqpoint{4.204905in}{3.639466in}}{\pgfqpoint{4.199319in}{3.637152in}}{\pgfqpoint{4.195201in}{3.633034in}}%
\pgfpathcurveto{\pgfqpoint{4.191083in}{3.628916in}}{\pgfqpoint{4.188769in}{3.623329in}}{\pgfqpoint{4.188769in}{3.617505in}}%
\pgfpathcurveto{\pgfqpoint{4.188769in}{3.611682in}}{\pgfqpoint{4.191083in}{3.606095in}}{\pgfqpoint{4.195201in}{3.601977in}}%
\pgfpathcurveto{\pgfqpoint{4.199319in}{3.597859in}}{\pgfqpoint{4.204905in}{3.595545in}}{\pgfqpoint{4.210729in}{3.595545in}}%
\pgfpathlineto{\pgfqpoint{4.210729in}{3.595545in}}%
\pgfpathclose%
\pgfusepath{stroke,fill}%
\end{pgfscope}%
\begin{pgfscope}%
\pgfpathrectangle{\pgfqpoint{0.997489in}{0.528000in}}{\pgfqpoint{4.565023in}{3.696000in}}%
\pgfusepath{clip}%
\pgfsetbuttcap%
\pgfsetroundjoin%
\definecolor{currentfill}{rgb}{0.200000,0.800000,0.200000}%
\pgfsetfillcolor{currentfill}%
\pgfsetlinewidth{1.003750pt}%
\definecolor{currentstroke}{rgb}{0.200000,0.800000,0.200000}%
\pgfsetstrokecolor{currentstroke}%
\pgfsetdash{}{0pt}%
\pgfpathmoveto{\pgfqpoint{1.397315in}{3.070004in}}%
\pgfpathcurveto{\pgfqpoint{1.403139in}{3.070004in}}{\pgfqpoint{1.408725in}{3.072318in}}{\pgfqpoint{1.412843in}{3.076436in}}%
\pgfpathcurveto{\pgfqpoint{1.416961in}{3.080554in}}{\pgfqpoint{1.419275in}{3.086140in}}{\pgfqpoint{1.419275in}{3.091964in}}%
\pgfpathcurveto{\pgfqpoint{1.419275in}{3.097788in}}{\pgfqpoint{1.416961in}{3.103374in}}{\pgfqpoint{1.412843in}{3.107493in}}%
\pgfpathcurveto{\pgfqpoint{1.408725in}{3.111611in}}{\pgfqpoint{1.403139in}{3.113925in}}{\pgfqpoint{1.397315in}{3.113925in}}%
\pgfpathcurveto{\pgfqpoint{1.391491in}{3.113925in}}{\pgfqpoint{1.385905in}{3.111611in}}{\pgfqpoint{1.381787in}{3.107493in}}%
\pgfpathcurveto{\pgfqpoint{1.377668in}{3.103374in}}{\pgfqpoint{1.375355in}{3.097788in}}{\pgfqpoint{1.375355in}{3.091964in}}%
\pgfpathcurveto{\pgfqpoint{1.375355in}{3.086140in}}{\pgfqpoint{1.377668in}{3.080554in}}{\pgfqpoint{1.381787in}{3.076436in}}%
\pgfpathcurveto{\pgfqpoint{1.385905in}{3.072318in}}{\pgfqpoint{1.391491in}{3.070004in}}{\pgfqpoint{1.397315in}{3.070004in}}%
\pgfpathlineto{\pgfqpoint{1.397315in}{3.070004in}}%
\pgfpathclose%
\pgfusepath{stroke,fill}%
\end{pgfscope}%
\begin{pgfscope}%
\pgfpathrectangle{\pgfqpoint{0.997489in}{0.528000in}}{\pgfqpoint{4.565023in}{3.696000in}}%
\pgfusepath{clip}%
\pgfsetbuttcap%
\pgfsetroundjoin%
\definecolor{currentfill}{rgb}{0.800000,0.200000,0.200000}%
\pgfsetfillcolor{currentfill}%
\pgfsetlinewidth{1.003750pt}%
\definecolor{currentstroke}{rgb}{0.800000,0.200000,0.200000}%
\pgfsetstrokecolor{currentstroke}%
\pgfsetdash{}{0pt}%
\pgfpathmoveto{\pgfqpoint{2.264440in}{1.791320in}}%
\pgfpathcurveto{\pgfqpoint{2.270264in}{1.791320in}}{\pgfqpoint{2.275850in}{1.793634in}}{\pgfqpoint{2.279968in}{1.797752in}}%
\pgfpathcurveto{\pgfqpoint{2.284086in}{1.801870in}}{\pgfqpoint{2.286400in}{1.807457in}}{\pgfqpoint{2.286400in}{1.813281in}}%
\pgfpathcurveto{\pgfqpoint{2.286400in}{1.819105in}}{\pgfqpoint{2.284086in}{1.824691in}}{\pgfqpoint{2.279968in}{1.828809in}}%
\pgfpathcurveto{\pgfqpoint{2.275850in}{1.832927in}}{\pgfqpoint{2.270264in}{1.835241in}}{\pgfqpoint{2.264440in}{1.835241in}}%
\pgfpathcurveto{\pgfqpoint{2.258616in}{1.835241in}}{\pgfqpoint{2.253030in}{1.832927in}}{\pgfqpoint{2.248912in}{1.828809in}}%
\pgfpathcurveto{\pgfqpoint{2.244794in}{1.824691in}}{\pgfqpoint{2.242480in}{1.819105in}}{\pgfqpoint{2.242480in}{1.813281in}}%
\pgfpathcurveto{\pgfqpoint{2.242480in}{1.807457in}}{\pgfqpoint{2.244794in}{1.801870in}}{\pgfqpoint{2.248912in}{1.797752in}}%
\pgfpathcurveto{\pgfqpoint{2.253030in}{1.793634in}}{\pgfqpoint{2.258616in}{1.791320in}}{\pgfqpoint{2.264440in}{1.791320in}}%
\pgfpathlineto{\pgfqpoint{2.264440in}{1.791320in}}%
\pgfpathclose%
\pgfusepath{stroke,fill}%
\end{pgfscope}%
\begin{pgfscope}%
\pgfpathrectangle{\pgfqpoint{0.997489in}{0.528000in}}{\pgfqpoint{4.565023in}{3.696000in}}%
\pgfusepath{clip}%
\pgfsetbuttcap%
\pgfsetroundjoin%
\definecolor{currentfill}{rgb}{0.200000,0.800000,0.200000}%
\pgfsetfillcolor{currentfill}%
\pgfsetlinewidth{1.003750pt}%
\definecolor{currentstroke}{rgb}{0.200000,0.800000,0.200000}%
\pgfsetstrokecolor{currentstroke}%
\pgfsetdash{}{0pt}%
\pgfpathmoveto{\pgfqpoint{1.678990in}{3.042473in}}%
\pgfpathcurveto{\pgfqpoint{1.684814in}{3.042473in}}{\pgfqpoint{1.690400in}{3.044787in}}{\pgfqpoint{1.694519in}{3.048905in}}%
\pgfpathcurveto{\pgfqpoint{1.698637in}{3.053023in}}{\pgfqpoint{1.700951in}{3.058609in}}{\pgfqpoint{1.700951in}{3.064433in}}%
\pgfpathcurveto{\pgfqpoint{1.700951in}{3.070257in}}{\pgfqpoint{1.698637in}{3.075843in}}{\pgfqpoint{1.694519in}{3.079961in}}%
\pgfpathcurveto{\pgfqpoint{1.690400in}{3.084079in}}{\pgfqpoint{1.684814in}{3.086393in}}{\pgfqpoint{1.678990in}{3.086393in}}%
\pgfpathcurveto{\pgfqpoint{1.673166in}{3.086393in}}{\pgfqpoint{1.667580in}{3.084079in}}{\pgfqpoint{1.663462in}{3.079961in}}%
\pgfpathcurveto{\pgfqpoint{1.659344in}{3.075843in}}{\pgfqpoint{1.657030in}{3.070257in}}{\pgfqpoint{1.657030in}{3.064433in}}%
\pgfpathcurveto{\pgfqpoint{1.657030in}{3.058609in}}{\pgfqpoint{1.659344in}{3.053023in}}{\pgfqpoint{1.663462in}{3.048905in}}%
\pgfpathcurveto{\pgfqpoint{1.667580in}{3.044787in}}{\pgfqpoint{1.673166in}{3.042473in}}{\pgfqpoint{1.678990in}{3.042473in}}%
\pgfpathlineto{\pgfqpoint{1.678990in}{3.042473in}}%
\pgfpathclose%
\pgfusepath{stroke,fill}%
\end{pgfscope}%
\begin{pgfscope}%
\pgfpathrectangle{\pgfqpoint{0.997489in}{0.528000in}}{\pgfqpoint{4.565023in}{3.696000in}}%
\pgfusepath{clip}%
\pgfsetbuttcap%
\pgfsetroundjoin%
\definecolor{currentfill}{rgb}{0.800000,0.800000,0.200000}%
\pgfsetfillcolor{currentfill}%
\pgfsetlinewidth{1.003750pt}%
\definecolor{currentstroke}{rgb}{0.800000,0.800000,0.200000}%
\pgfsetstrokecolor{currentstroke}%
\pgfsetdash{}{0pt}%
\pgfpathmoveto{\pgfqpoint{3.776702in}{2.516600in}}%
\pgfpathcurveto{\pgfqpoint{3.782526in}{2.516600in}}{\pgfqpoint{3.788113in}{2.518914in}}{\pgfqpoint{3.792231in}{2.523032in}}%
\pgfpathcurveto{\pgfqpoint{3.796349in}{2.527150in}}{\pgfqpoint{3.798663in}{2.532736in}}{\pgfqpoint{3.798663in}{2.538560in}}%
\pgfpathcurveto{\pgfqpoint{3.798663in}{2.544384in}}{\pgfqpoint{3.796349in}{2.549970in}}{\pgfqpoint{3.792231in}{2.554088in}}%
\pgfpathcurveto{\pgfqpoint{3.788113in}{2.558206in}}{\pgfqpoint{3.782526in}{2.560520in}}{\pgfqpoint{3.776702in}{2.560520in}}%
\pgfpathcurveto{\pgfqpoint{3.770879in}{2.560520in}}{\pgfqpoint{3.765292in}{2.558206in}}{\pgfqpoint{3.761174in}{2.554088in}}%
\pgfpathcurveto{\pgfqpoint{3.757056in}{2.549970in}}{\pgfqpoint{3.754742in}{2.544384in}}{\pgfqpoint{3.754742in}{2.538560in}}%
\pgfpathcurveto{\pgfqpoint{3.754742in}{2.532736in}}{\pgfqpoint{3.757056in}{2.527150in}}{\pgfqpoint{3.761174in}{2.523032in}}%
\pgfpathcurveto{\pgfqpoint{3.765292in}{2.518914in}}{\pgfqpoint{3.770879in}{2.516600in}}{\pgfqpoint{3.776702in}{2.516600in}}%
\pgfpathlineto{\pgfqpoint{3.776702in}{2.516600in}}%
\pgfpathclose%
\pgfusepath{stroke,fill}%
\end{pgfscope}%
\begin{pgfscope}%
\pgfpathrectangle{\pgfqpoint{0.997489in}{0.528000in}}{\pgfqpoint{4.565023in}{3.696000in}}%
\pgfusepath{clip}%
\pgfsetbuttcap%
\pgfsetroundjoin%
\definecolor{currentfill}{rgb}{0.200000,0.800000,0.200000}%
\pgfsetfillcolor{currentfill}%
\pgfsetlinewidth{1.003750pt}%
\definecolor{currentstroke}{rgb}{0.200000,0.800000,0.200000}%
\pgfsetstrokecolor{currentstroke}%
\pgfsetdash{}{0pt}%
\pgfpathmoveto{\pgfqpoint{3.127831in}{2.789069in}}%
\pgfpathcurveto{\pgfqpoint{3.133655in}{2.789069in}}{\pgfqpoint{3.139242in}{2.791383in}}{\pgfqpoint{3.143360in}{2.795501in}}%
\pgfpathcurveto{\pgfqpoint{3.147478in}{2.799620in}}{\pgfqpoint{3.149792in}{2.805206in}}{\pgfqpoint{3.149792in}{2.811030in}}%
\pgfpathcurveto{\pgfqpoint{3.149792in}{2.816854in}}{\pgfqpoint{3.147478in}{2.822440in}}{\pgfqpoint{3.143360in}{2.826558in}}%
\pgfpathcurveto{\pgfqpoint{3.139242in}{2.830676in}}{\pgfqpoint{3.133655in}{2.832990in}}{\pgfqpoint{3.127831in}{2.832990in}}%
\pgfpathcurveto{\pgfqpoint{3.122008in}{2.832990in}}{\pgfqpoint{3.116421in}{2.830676in}}{\pgfqpoint{3.112303in}{2.826558in}}%
\pgfpathcurveto{\pgfqpoint{3.108185in}{2.822440in}}{\pgfqpoint{3.105871in}{2.816854in}}{\pgfqpoint{3.105871in}{2.811030in}}%
\pgfpathcurveto{\pgfqpoint{3.105871in}{2.805206in}}{\pgfqpoint{3.108185in}{2.799620in}}{\pgfqpoint{3.112303in}{2.795501in}}%
\pgfpathcurveto{\pgfqpoint{3.116421in}{2.791383in}}{\pgfqpoint{3.122008in}{2.789069in}}{\pgfqpoint{3.127831in}{2.789069in}}%
\pgfpathlineto{\pgfqpoint{3.127831in}{2.789069in}}%
\pgfpathclose%
\pgfusepath{stroke,fill}%
\end{pgfscope}%
\begin{pgfscope}%
\pgfpathrectangle{\pgfqpoint{0.997489in}{0.528000in}}{\pgfqpoint{4.565023in}{3.696000in}}%
\pgfusepath{clip}%
\pgfsetbuttcap%
\pgfsetroundjoin%
\definecolor{currentfill}{rgb}{0.200000,0.800000,0.200000}%
\pgfsetfillcolor{currentfill}%
\pgfsetlinewidth{1.003750pt}%
\definecolor{currentstroke}{rgb}{0.200000,0.800000,0.200000}%
\pgfsetstrokecolor{currentstroke}%
\pgfsetdash{}{0pt}%
\pgfpathmoveto{\pgfqpoint{1.664089in}{1.664095in}}%
\pgfpathcurveto{\pgfqpoint{1.669913in}{1.664095in}}{\pgfqpoint{1.675499in}{1.666409in}}{\pgfqpoint{1.679618in}{1.670527in}}%
\pgfpathcurveto{\pgfqpoint{1.683736in}{1.674645in}}{\pgfqpoint{1.686050in}{1.680231in}}{\pgfqpoint{1.686050in}{1.686055in}}%
\pgfpathcurveto{\pgfqpoint{1.686050in}{1.691879in}}{\pgfqpoint{1.683736in}{1.697465in}}{\pgfqpoint{1.679618in}{1.701583in}}%
\pgfpathcurveto{\pgfqpoint{1.675499in}{1.705701in}}{\pgfqpoint{1.669913in}{1.708015in}}{\pgfqpoint{1.664089in}{1.708015in}}%
\pgfpathcurveto{\pgfqpoint{1.658265in}{1.708015in}}{\pgfqpoint{1.652679in}{1.705701in}}{\pgfqpoint{1.648561in}{1.701583in}}%
\pgfpathcurveto{\pgfqpoint{1.644443in}{1.697465in}}{\pgfqpoint{1.642129in}{1.691879in}}{\pgfqpoint{1.642129in}{1.686055in}}%
\pgfpathcurveto{\pgfqpoint{1.642129in}{1.680231in}}{\pgfqpoint{1.644443in}{1.674645in}}{\pgfqpoint{1.648561in}{1.670527in}}%
\pgfpathcurveto{\pgfqpoint{1.652679in}{1.666409in}}{\pgfqpoint{1.658265in}{1.664095in}}{\pgfqpoint{1.664089in}{1.664095in}}%
\pgfpathlineto{\pgfqpoint{1.664089in}{1.664095in}}%
\pgfpathclose%
\pgfusepath{stroke,fill}%
\end{pgfscope}%
\begin{pgfscope}%
\pgfpathrectangle{\pgfqpoint{0.997489in}{0.528000in}}{\pgfqpoint{4.565023in}{3.696000in}}%
\pgfusepath{clip}%
\pgfsetbuttcap%
\pgfsetroundjoin%
\definecolor{currentfill}{rgb}{0.800000,0.200000,0.200000}%
\pgfsetfillcolor{currentfill}%
\pgfsetlinewidth{1.003750pt}%
\definecolor{currentstroke}{rgb}{0.800000,0.200000,0.200000}%
\pgfsetstrokecolor{currentstroke}%
\pgfsetdash{}{0pt}%
\pgfpathmoveto{\pgfqpoint{2.673562in}{1.316534in}}%
\pgfpathcurveto{\pgfqpoint{2.679386in}{1.316534in}}{\pgfqpoint{2.684972in}{1.318848in}}{\pgfqpoint{2.689090in}{1.322966in}}%
\pgfpathcurveto{\pgfqpoint{2.693209in}{1.327084in}}{\pgfqpoint{2.695522in}{1.332670in}}{\pgfqpoint{2.695522in}{1.338494in}}%
\pgfpathcurveto{\pgfqpoint{2.695522in}{1.344318in}}{\pgfqpoint{2.693209in}{1.349904in}}{\pgfqpoint{2.689090in}{1.354022in}}%
\pgfpathcurveto{\pgfqpoint{2.684972in}{1.358141in}}{\pgfqpoint{2.679386in}{1.360454in}}{\pgfqpoint{2.673562in}{1.360454in}}%
\pgfpathcurveto{\pgfqpoint{2.667738in}{1.360454in}}{\pgfqpoint{2.662152in}{1.358141in}}{\pgfqpoint{2.658034in}{1.354022in}}%
\pgfpathcurveto{\pgfqpoint{2.653916in}{1.349904in}}{\pgfqpoint{2.651602in}{1.344318in}}{\pgfqpoint{2.651602in}{1.338494in}}%
\pgfpathcurveto{\pgfqpoint{2.651602in}{1.332670in}}{\pgfqpoint{2.653916in}{1.327084in}}{\pgfqpoint{2.658034in}{1.322966in}}%
\pgfpathcurveto{\pgfqpoint{2.662152in}{1.318848in}}{\pgfqpoint{2.667738in}{1.316534in}}{\pgfqpoint{2.673562in}{1.316534in}}%
\pgfpathlineto{\pgfqpoint{2.673562in}{1.316534in}}%
\pgfpathclose%
\pgfusepath{stroke,fill}%
\end{pgfscope}%
\begin{pgfscope}%
\pgfpathrectangle{\pgfqpoint{0.997489in}{0.528000in}}{\pgfqpoint{4.565023in}{3.696000in}}%
\pgfusepath{clip}%
\pgfsetbuttcap%
\pgfsetroundjoin%
\definecolor{currentfill}{rgb}{0.800000,0.200000,0.200000}%
\pgfsetfillcolor{currentfill}%
\pgfsetlinewidth{1.003750pt}%
\definecolor{currentstroke}{rgb}{0.800000,0.200000,0.200000}%
\pgfsetstrokecolor{currentstroke}%
\pgfsetdash{}{0pt}%
\pgfpathmoveto{\pgfqpoint{2.963041in}{1.453396in}}%
\pgfpathcurveto{\pgfqpoint{2.968865in}{1.453396in}}{\pgfqpoint{2.974451in}{1.455710in}}{\pgfqpoint{2.978570in}{1.459828in}}%
\pgfpathcurveto{\pgfqpoint{2.982688in}{1.463947in}}{\pgfqpoint{2.985002in}{1.469533in}}{\pgfqpoint{2.985002in}{1.475357in}}%
\pgfpathcurveto{\pgfqpoint{2.985002in}{1.481181in}}{\pgfqpoint{2.982688in}{1.486767in}}{\pgfqpoint{2.978570in}{1.490885in}}%
\pgfpathcurveto{\pgfqpoint{2.974451in}{1.495003in}}{\pgfqpoint{2.968865in}{1.497317in}}{\pgfqpoint{2.963041in}{1.497317in}}%
\pgfpathcurveto{\pgfqpoint{2.957217in}{1.497317in}}{\pgfqpoint{2.951631in}{1.495003in}}{\pgfqpoint{2.947513in}{1.490885in}}%
\pgfpathcurveto{\pgfqpoint{2.943395in}{1.486767in}}{\pgfqpoint{2.941081in}{1.481181in}}{\pgfqpoint{2.941081in}{1.475357in}}%
\pgfpathcurveto{\pgfqpoint{2.941081in}{1.469533in}}{\pgfqpoint{2.943395in}{1.463947in}}{\pgfqpoint{2.947513in}{1.459828in}}%
\pgfpathcurveto{\pgfqpoint{2.951631in}{1.455710in}}{\pgfqpoint{2.957217in}{1.453396in}}{\pgfqpoint{2.963041in}{1.453396in}}%
\pgfpathlineto{\pgfqpoint{2.963041in}{1.453396in}}%
\pgfpathclose%
\pgfusepath{stroke,fill}%
\end{pgfscope}%
\begin{pgfscope}%
\pgfpathrectangle{\pgfqpoint{0.997489in}{0.528000in}}{\pgfqpoint{4.565023in}{3.696000in}}%
\pgfusepath{clip}%
\pgfsetbuttcap%
\pgfsetroundjoin%
\definecolor{currentfill}{rgb}{0.800000,0.200000,0.200000}%
\pgfsetfillcolor{currentfill}%
\pgfsetlinewidth{1.003750pt}%
\definecolor{currentstroke}{rgb}{0.800000,0.200000,0.200000}%
\pgfsetstrokecolor{currentstroke}%
\pgfsetdash{}{0pt}%
\pgfpathmoveto{\pgfqpoint{3.533593in}{0.926109in}}%
\pgfpathcurveto{\pgfqpoint{3.539417in}{0.926109in}}{\pgfqpoint{3.545003in}{0.928423in}}{\pgfqpoint{3.549121in}{0.932541in}}%
\pgfpathcurveto{\pgfqpoint{3.553239in}{0.936659in}}{\pgfqpoint{3.555553in}{0.942245in}}{\pgfqpoint{3.555553in}{0.948069in}}%
\pgfpathcurveto{\pgfqpoint{3.555553in}{0.953893in}}{\pgfqpoint{3.553239in}{0.959479in}}{\pgfqpoint{3.549121in}{0.963597in}}%
\pgfpathcurveto{\pgfqpoint{3.545003in}{0.967716in}}{\pgfqpoint{3.539417in}{0.970029in}}{\pgfqpoint{3.533593in}{0.970029in}}%
\pgfpathcurveto{\pgfqpoint{3.527769in}{0.970029in}}{\pgfqpoint{3.522183in}{0.967716in}}{\pgfqpoint{3.518064in}{0.963597in}}%
\pgfpathcurveto{\pgfqpoint{3.513946in}{0.959479in}}{\pgfqpoint{3.511632in}{0.953893in}}{\pgfqpoint{3.511632in}{0.948069in}}%
\pgfpathcurveto{\pgfqpoint{3.511632in}{0.942245in}}{\pgfqpoint{3.513946in}{0.936659in}}{\pgfqpoint{3.518064in}{0.932541in}}%
\pgfpathcurveto{\pgfqpoint{3.522183in}{0.928423in}}{\pgfqpoint{3.527769in}{0.926109in}}{\pgfqpoint{3.533593in}{0.926109in}}%
\pgfpathlineto{\pgfqpoint{3.533593in}{0.926109in}}%
\pgfpathclose%
\pgfusepath{stroke,fill}%
\end{pgfscope}%
\begin{pgfscope}%
\pgfpathrectangle{\pgfqpoint{0.997489in}{0.528000in}}{\pgfqpoint{4.565023in}{3.696000in}}%
\pgfusepath{clip}%
\pgfsetbuttcap%
\pgfsetroundjoin%
\definecolor{currentfill}{rgb}{0.200000,0.200000,0.800000}%
\pgfsetfillcolor{currentfill}%
\pgfsetlinewidth{1.003750pt}%
\definecolor{currentstroke}{rgb}{0.200000,0.200000,0.800000}%
\pgfsetstrokecolor{currentstroke}%
\pgfsetdash{}{0pt}%
\pgfpathmoveto{\pgfqpoint{3.787332in}{2.128492in}}%
\pgfpathcurveto{\pgfqpoint{3.793156in}{2.128492in}}{\pgfqpoint{3.798742in}{2.130805in}}{\pgfqpoint{3.802860in}{2.134924in}}%
\pgfpathcurveto{\pgfqpoint{3.806979in}{2.139042in}}{\pgfqpoint{3.809292in}{2.144628in}}{\pgfqpoint{3.809292in}{2.150452in}}%
\pgfpathcurveto{\pgfqpoint{3.809292in}{2.156276in}}{\pgfqpoint{3.806979in}{2.161862in}}{\pgfqpoint{3.802860in}{2.165980in}}%
\pgfpathcurveto{\pgfqpoint{3.798742in}{2.170098in}}{\pgfqpoint{3.793156in}{2.172412in}}{\pgfqpoint{3.787332in}{2.172412in}}%
\pgfpathcurveto{\pgfqpoint{3.781508in}{2.172412in}}{\pgfqpoint{3.775922in}{2.170098in}}{\pgfqpoint{3.771804in}{2.165980in}}%
\pgfpathcurveto{\pgfqpoint{3.767686in}{2.161862in}}{\pgfqpoint{3.765372in}{2.156276in}}{\pgfqpoint{3.765372in}{2.150452in}}%
\pgfpathcurveto{\pgfqpoint{3.765372in}{2.144628in}}{\pgfqpoint{3.767686in}{2.139042in}}{\pgfqpoint{3.771804in}{2.134924in}}%
\pgfpathcurveto{\pgfqpoint{3.775922in}{2.130805in}}{\pgfqpoint{3.781508in}{2.128492in}}{\pgfqpoint{3.787332in}{2.128492in}}%
\pgfpathlineto{\pgfqpoint{3.787332in}{2.128492in}}%
\pgfpathclose%
\pgfusepath{stroke,fill}%
\end{pgfscope}%
\begin{pgfscope}%
\pgfpathrectangle{\pgfqpoint{0.997489in}{0.528000in}}{\pgfqpoint{4.565023in}{3.696000in}}%
\pgfusepath{clip}%
\pgfsetbuttcap%
\pgfsetroundjoin%
\definecolor{currentfill}{rgb}{0.200000,0.800000,0.200000}%
\pgfsetfillcolor{currentfill}%
\pgfsetlinewidth{1.003750pt}%
\definecolor{currentstroke}{rgb}{0.200000,0.800000,0.200000}%
\pgfsetstrokecolor{currentstroke}%
\pgfsetdash{}{0pt}%
\pgfpathmoveto{\pgfqpoint{1.640223in}{1.927307in}}%
\pgfpathcurveto{\pgfqpoint{1.646047in}{1.927307in}}{\pgfqpoint{1.651633in}{1.929620in}}{\pgfqpoint{1.655751in}{1.933739in}}%
\pgfpathcurveto{\pgfqpoint{1.659869in}{1.937857in}}{\pgfqpoint{1.662183in}{1.943443in}}{\pgfqpoint{1.662183in}{1.949267in}}%
\pgfpathcurveto{\pgfqpoint{1.662183in}{1.955091in}}{\pgfqpoint{1.659869in}{1.960677in}}{\pgfqpoint{1.655751in}{1.964795in}}%
\pgfpathcurveto{\pgfqpoint{1.651633in}{1.968913in}}{\pgfqpoint{1.646047in}{1.971227in}}{\pgfqpoint{1.640223in}{1.971227in}}%
\pgfpathcurveto{\pgfqpoint{1.634399in}{1.971227in}}{\pgfqpoint{1.628813in}{1.968913in}}{\pgfqpoint{1.624694in}{1.964795in}}%
\pgfpathcurveto{\pgfqpoint{1.620576in}{1.960677in}}{\pgfqpoint{1.618262in}{1.955091in}}{\pgfqpoint{1.618262in}{1.949267in}}%
\pgfpathcurveto{\pgfqpoint{1.618262in}{1.943443in}}{\pgfqpoint{1.620576in}{1.937857in}}{\pgfqpoint{1.624694in}{1.933739in}}%
\pgfpathcurveto{\pgfqpoint{1.628813in}{1.929620in}}{\pgfqpoint{1.634399in}{1.927307in}}{\pgfqpoint{1.640223in}{1.927307in}}%
\pgfpathlineto{\pgfqpoint{1.640223in}{1.927307in}}%
\pgfpathclose%
\pgfusepath{stroke,fill}%
\end{pgfscope}%
\begin{pgfscope}%
\pgfpathrectangle{\pgfqpoint{0.997489in}{0.528000in}}{\pgfqpoint{4.565023in}{3.696000in}}%
\pgfusepath{clip}%
\pgfsetbuttcap%
\pgfsetroundjoin%
\definecolor{currentfill}{rgb}{0.800000,0.200000,0.200000}%
\pgfsetfillcolor{currentfill}%
\pgfsetlinewidth{1.003750pt}%
\definecolor{currentstroke}{rgb}{0.800000,0.200000,0.200000}%
\pgfsetstrokecolor{currentstroke}%
\pgfsetdash{}{0pt}%
\pgfpathmoveto{\pgfqpoint{2.881319in}{0.674040in}}%
\pgfpathcurveto{\pgfqpoint{2.887143in}{0.674040in}}{\pgfqpoint{2.892729in}{0.676354in}}{\pgfqpoint{2.896847in}{0.680472in}}%
\pgfpathcurveto{\pgfqpoint{2.900966in}{0.684590in}}{\pgfqpoint{2.903279in}{0.690176in}}{\pgfqpoint{2.903279in}{0.696000in}}%
\pgfpathcurveto{\pgfqpoint{2.903279in}{0.701824in}}{\pgfqpoint{2.900966in}{0.707410in}}{\pgfqpoint{2.896847in}{0.711528in}}%
\pgfpathcurveto{\pgfqpoint{2.892729in}{0.715646in}}{\pgfqpoint{2.887143in}{0.717960in}}{\pgfqpoint{2.881319in}{0.717960in}}%
\pgfpathcurveto{\pgfqpoint{2.875495in}{0.717960in}}{\pgfqpoint{2.869909in}{0.715646in}}{\pgfqpoint{2.865791in}{0.711528in}}%
\pgfpathcurveto{\pgfqpoint{2.861673in}{0.707410in}}{\pgfqpoint{2.859359in}{0.701824in}}{\pgfqpoint{2.859359in}{0.696000in}}%
\pgfpathcurveto{\pgfqpoint{2.859359in}{0.690176in}}{\pgfqpoint{2.861673in}{0.684590in}}{\pgfqpoint{2.865791in}{0.680472in}}%
\pgfpathcurveto{\pgfqpoint{2.869909in}{0.676354in}}{\pgfqpoint{2.875495in}{0.674040in}}{\pgfqpoint{2.881319in}{0.674040in}}%
\pgfpathlineto{\pgfqpoint{2.881319in}{0.674040in}}%
\pgfpathclose%
\pgfusepath{stroke,fill}%
\end{pgfscope}%
\begin{pgfscope}%
\pgfpathrectangle{\pgfqpoint{0.997489in}{0.528000in}}{\pgfqpoint{4.565023in}{3.696000in}}%
\pgfusepath{clip}%
\pgfsetbuttcap%
\pgfsetroundjoin%
\definecolor{currentfill}{rgb}{0.200000,0.800000,0.200000}%
\pgfsetfillcolor{currentfill}%
\pgfsetlinewidth{1.003750pt}%
\definecolor{currentstroke}{rgb}{0.200000,0.800000,0.200000}%
\pgfsetstrokecolor{currentstroke}%
\pgfsetdash{}{0pt}%
\pgfpathmoveto{\pgfqpoint{1.867631in}{3.324746in}}%
\pgfpathcurveto{\pgfqpoint{1.873455in}{3.324746in}}{\pgfqpoint{1.879041in}{3.327059in}}{\pgfqpoint{1.883159in}{3.331178in}}%
\pgfpathcurveto{\pgfqpoint{1.887278in}{3.335296in}}{\pgfqpoint{1.889591in}{3.340882in}}{\pgfqpoint{1.889591in}{3.346706in}}%
\pgfpathcurveto{\pgfqpoint{1.889591in}{3.352530in}}{\pgfqpoint{1.887278in}{3.358116in}}{\pgfqpoint{1.883159in}{3.362234in}}%
\pgfpathcurveto{\pgfqpoint{1.879041in}{3.366352in}}{\pgfqpoint{1.873455in}{3.368666in}}{\pgfqpoint{1.867631in}{3.368666in}}%
\pgfpathcurveto{\pgfqpoint{1.861807in}{3.368666in}}{\pgfqpoint{1.856221in}{3.366352in}}{\pgfqpoint{1.852103in}{3.362234in}}%
\pgfpathcurveto{\pgfqpoint{1.847985in}{3.358116in}}{\pgfqpoint{1.845671in}{3.352530in}}{\pgfqpoint{1.845671in}{3.346706in}}%
\pgfpathcurveto{\pgfqpoint{1.845671in}{3.340882in}}{\pgfqpoint{1.847985in}{3.335296in}}{\pgfqpoint{1.852103in}{3.331178in}}%
\pgfpathcurveto{\pgfqpoint{1.856221in}{3.327059in}}{\pgfqpoint{1.861807in}{3.324746in}}{\pgfqpoint{1.867631in}{3.324746in}}%
\pgfpathlineto{\pgfqpoint{1.867631in}{3.324746in}}%
\pgfpathclose%
\pgfusepath{stroke,fill}%
\end{pgfscope}%
\begin{pgfscope}%
\pgfpathrectangle{\pgfqpoint{0.997489in}{0.528000in}}{\pgfqpoint{4.565023in}{3.696000in}}%
\pgfusepath{clip}%
\pgfsetbuttcap%
\pgfsetroundjoin%
\definecolor{currentfill}{rgb}{0.200000,0.200000,0.800000}%
\pgfsetfillcolor{currentfill}%
\pgfsetlinewidth{1.003750pt}%
\definecolor{currentstroke}{rgb}{0.200000,0.200000,0.800000}%
\pgfsetstrokecolor{currentstroke}%
\pgfsetdash{}{0pt}%
\pgfpathmoveto{\pgfqpoint{3.705905in}{2.783018in}}%
\pgfpathcurveto{\pgfqpoint{3.711729in}{2.783018in}}{\pgfqpoint{3.717315in}{2.785332in}}{\pgfqpoint{3.721434in}{2.789450in}}%
\pgfpathcurveto{\pgfqpoint{3.725552in}{2.793568in}}{\pgfqpoint{3.727866in}{2.799155in}}{\pgfqpoint{3.727866in}{2.804979in}}%
\pgfpathcurveto{\pgfqpoint{3.727866in}{2.810802in}}{\pgfqpoint{3.725552in}{2.816389in}}{\pgfqpoint{3.721434in}{2.820507in}}%
\pgfpathcurveto{\pgfqpoint{3.717315in}{2.824625in}}{\pgfqpoint{3.711729in}{2.826939in}}{\pgfqpoint{3.705905in}{2.826939in}}%
\pgfpathcurveto{\pgfqpoint{3.700081in}{2.826939in}}{\pgfqpoint{3.694495in}{2.824625in}}{\pgfqpoint{3.690377in}{2.820507in}}%
\pgfpathcurveto{\pgfqpoint{3.686259in}{2.816389in}}{\pgfqpoint{3.683945in}{2.810802in}}{\pgfqpoint{3.683945in}{2.804979in}}%
\pgfpathcurveto{\pgfqpoint{3.683945in}{2.799155in}}{\pgfqpoint{3.686259in}{2.793568in}}{\pgfqpoint{3.690377in}{2.789450in}}%
\pgfpathcurveto{\pgfqpoint{3.694495in}{2.785332in}}{\pgfqpoint{3.700081in}{2.783018in}}{\pgfqpoint{3.705905in}{2.783018in}}%
\pgfpathlineto{\pgfqpoint{3.705905in}{2.783018in}}%
\pgfpathclose%
\pgfusepath{stroke,fill}%
\end{pgfscope}%
\begin{pgfscope}%
\pgfpathrectangle{\pgfqpoint{0.997489in}{0.528000in}}{\pgfqpoint{4.565023in}{3.696000in}}%
\pgfusepath{clip}%
\pgfsetbuttcap%
\pgfsetroundjoin%
\definecolor{currentfill}{rgb}{0.200000,0.200000,0.800000}%
\pgfsetfillcolor{currentfill}%
\pgfsetlinewidth{1.003750pt}%
\definecolor{currentstroke}{rgb}{0.200000,0.200000,0.800000}%
\pgfsetstrokecolor{currentstroke}%
\pgfsetdash{}{0pt}%
\pgfpathmoveto{\pgfqpoint{3.409156in}{2.053063in}}%
\pgfpathcurveto{\pgfqpoint{3.414980in}{2.053063in}}{\pgfqpoint{3.420566in}{2.055377in}}{\pgfqpoint{3.424685in}{2.059495in}}%
\pgfpathcurveto{\pgfqpoint{3.428803in}{2.063613in}}{\pgfqpoint{3.431117in}{2.069200in}}{\pgfqpoint{3.431117in}{2.075023in}}%
\pgfpathcurveto{\pgfqpoint{3.431117in}{2.080847in}}{\pgfqpoint{3.428803in}{2.086434in}}{\pgfqpoint{3.424685in}{2.090552in}}%
\pgfpathcurveto{\pgfqpoint{3.420566in}{2.094670in}}{\pgfqpoint{3.414980in}{2.096984in}}{\pgfqpoint{3.409156in}{2.096984in}}%
\pgfpathcurveto{\pgfqpoint{3.403332in}{2.096984in}}{\pgfqpoint{3.397746in}{2.094670in}}{\pgfqpoint{3.393628in}{2.090552in}}%
\pgfpathcurveto{\pgfqpoint{3.389510in}{2.086434in}}{\pgfqpoint{3.387196in}{2.080847in}}{\pgfqpoint{3.387196in}{2.075023in}}%
\pgfpathcurveto{\pgfqpoint{3.387196in}{2.069200in}}{\pgfqpoint{3.389510in}{2.063613in}}{\pgfqpoint{3.393628in}{2.059495in}}%
\pgfpathcurveto{\pgfqpoint{3.397746in}{2.055377in}}{\pgfqpoint{3.403332in}{2.053063in}}{\pgfqpoint{3.409156in}{2.053063in}}%
\pgfpathlineto{\pgfqpoint{3.409156in}{2.053063in}}%
\pgfpathclose%
\pgfusepath{stroke,fill}%
\end{pgfscope}%
\begin{pgfscope}%
\pgfpathrectangle{\pgfqpoint{0.997489in}{0.528000in}}{\pgfqpoint{4.565023in}{3.696000in}}%
\pgfusepath{clip}%
\pgfsetbuttcap%
\pgfsetmiterjoin%
\pgfsetlinewidth{1.003750pt}%
\definecolor{currentstroke}{rgb}{0.800000,0.200000,0.200000}%
\pgfsetstrokecolor{currentstroke}%
\pgfsetdash{}{0pt}%
\pgfpathmoveto{\pgfqpoint{3.163229in}{1.115248in}}%
\pgfpathcurveto{\pgfqpoint{3.442342in}{1.115248in}}{\pgfqpoint{3.710061in}{1.226141in}}{\pgfqpoint{3.907424in}{1.423504in}}%
\pgfpathcurveto{\pgfqpoint{4.104787in}{1.620867in}}{\pgfqpoint{4.215680in}{1.888586in}}{\pgfqpoint{4.215680in}{2.167699in}}%
\pgfpathcurveto{\pgfqpoint{4.215680in}{2.446813in}}{\pgfqpoint{4.104787in}{2.714532in}}{\pgfqpoint{3.907424in}{2.911895in}}%
\pgfpathcurveto{\pgfqpoint{3.710061in}{3.109258in}}{\pgfqpoint{3.442342in}{3.220151in}}{\pgfqpoint{3.163229in}{3.220151in}}%
\pgfpathcurveto{\pgfqpoint{2.884115in}{3.220151in}}{\pgfqpoint{2.616396in}{3.109258in}}{\pgfqpoint{2.419033in}{2.911895in}}%
\pgfpathcurveto{\pgfqpoint{2.221670in}{2.714532in}}{\pgfqpoint{2.110777in}{2.446813in}}{\pgfqpoint{2.110777in}{2.167699in}}%
\pgfpathcurveto{\pgfqpoint{2.110777in}{1.888586in}}{\pgfqpoint{2.221670in}{1.620867in}}{\pgfqpoint{2.419033in}{1.423504in}}%
\pgfpathcurveto{\pgfqpoint{2.616396in}{1.226141in}}{\pgfqpoint{2.884115in}{1.115248in}}{\pgfqpoint{3.163229in}{1.115248in}}%
\pgfpathlineto{\pgfqpoint{3.163229in}{1.115248in}}%
\pgfpathclose%
\pgfusepath{stroke}%
\end{pgfscope}%
\begin{pgfscope}%
\pgfpathrectangle{\pgfqpoint{0.997489in}{0.528000in}}{\pgfqpoint{4.565023in}{3.696000in}}%
\pgfusepath{clip}%
\pgfsetbuttcap%
\pgfsetroundjoin%
\definecolor{currentfill}{rgb}{0.000000,0.000000,0.000000}%
\pgfsetfillcolor{currentfill}%
\pgfsetlinewidth{1.003750pt}%
\definecolor{currentstroke}{rgb}{0.000000,0.000000,0.000000}%
\pgfsetstrokecolor{currentstroke}%
\pgfsetdash{}{0pt}%
\pgfsys@defobject{currentmarker}{\pgfqpoint{-0.021960in}{-0.021960in}}{\pgfqpoint{0.021960in}{0.021960in}}{%
\pgfpathmoveto{\pgfqpoint{0.000000in}{-0.021960in}}%
\pgfpathcurveto{\pgfqpoint{0.005824in}{-0.021960in}}{\pgfqpoint{0.011410in}{-0.019646in}}{\pgfqpoint{0.015528in}{-0.015528in}}%
\pgfpathcurveto{\pgfqpoint{0.019646in}{-0.011410in}}{\pgfqpoint{0.021960in}{-0.005824in}}{\pgfqpoint{0.021960in}{0.000000in}}%
\pgfpathcurveto{\pgfqpoint{0.021960in}{0.005824in}}{\pgfqpoint{0.019646in}{0.011410in}}{\pgfqpoint{0.015528in}{0.015528in}}%
\pgfpathcurveto{\pgfqpoint{0.011410in}{0.019646in}}{\pgfqpoint{0.005824in}{0.021960in}}{\pgfqpoint{0.000000in}{0.021960in}}%
\pgfpathcurveto{\pgfqpoint{-0.005824in}{0.021960in}}{\pgfqpoint{-0.011410in}{0.019646in}}{\pgfqpoint{-0.015528in}{0.015528in}}%
\pgfpathcurveto{\pgfqpoint{-0.019646in}{0.011410in}}{\pgfqpoint{-0.021960in}{0.005824in}}{\pgfqpoint{-0.021960in}{0.000000in}}%
\pgfpathcurveto{\pgfqpoint{-0.021960in}{-0.005824in}}{\pgfqpoint{-0.019646in}{-0.011410in}}{\pgfqpoint{-0.015528in}{-0.015528in}}%
\pgfpathcurveto{\pgfqpoint{-0.011410in}{-0.019646in}}{\pgfqpoint{-0.005824in}{-0.021960in}}{\pgfqpoint{0.000000in}{-0.021960in}}%
\pgfpathlineto{\pgfqpoint{0.000000in}{-0.021960in}}%
\pgfpathclose%
\pgfusepath{stroke,fill}%
}%
\begin{pgfscope}%
\pgfsys@transformshift{3.163229in}{2.167699in}%
\pgfsys@useobject{currentmarker}{}%
\end{pgfscope}%
\end{pgfscope}%
\begin{pgfscope}%
\pgfpathrectangle{\pgfqpoint{0.997489in}{0.528000in}}{\pgfqpoint{4.565023in}{3.696000in}}%
\pgfusepath{clip}%
\pgfsetbuttcap%
\pgfsetmiterjoin%
\pgfsetlinewidth{1.003750pt}%
\definecolor{currentstroke}{rgb}{0.200000,0.800000,0.200000}%
\pgfsetstrokecolor{currentstroke}%
\pgfsetdash{}{0pt}%
\pgfpathmoveto{\pgfqpoint{2.168676in}{2.074679in}}%
\pgfpathcurveto{\pgfqpoint{2.398473in}{2.074679in}}{\pgfqpoint{2.618889in}{2.165979in}}{\pgfqpoint{2.781380in}{2.328470in}}%
\pgfpathcurveto{\pgfqpoint{2.943872in}{2.490961in}}{\pgfqpoint{3.035171in}{2.711377in}}{\pgfqpoint{3.035171in}{2.941174in}}%
\pgfpathcurveto{\pgfqpoint{3.035171in}{3.170972in}}{\pgfqpoint{2.943872in}{3.391388in}}{\pgfqpoint{2.781380in}{3.553879in}}%
\pgfpathcurveto{\pgfqpoint{2.618889in}{3.716370in}}{\pgfqpoint{2.398473in}{3.807669in}}{\pgfqpoint{2.168676in}{3.807669in}}%
\pgfpathcurveto{\pgfqpoint{1.938879in}{3.807669in}}{\pgfqpoint{1.718463in}{3.716370in}}{\pgfqpoint{1.555971in}{3.553879in}}%
\pgfpathcurveto{\pgfqpoint{1.393480in}{3.391388in}}{\pgfqpoint{1.302181in}{3.170972in}}{\pgfqpoint{1.302181in}{2.941174in}}%
\pgfpathcurveto{\pgfqpoint{1.302181in}{2.711377in}}{\pgfqpoint{1.393480in}{2.490961in}}{\pgfqpoint{1.555971in}{2.328470in}}%
\pgfpathcurveto{\pgfqpoint{1.718463in}{2.165979in}}{\pgfqpoint{1.938879in}{2.074679in}}{\pgfqpoint{2.168676in}{2.074679in}}%
\pgfpathlineto{\pgfqpoint{2.168676in}{2.074679in}}%
\pgfpathclose%
\pgfusepath{stroke}%
\end{pgfscope}%
\begin{pgfscope}%
\pgfpathrectangle{\pgfqpoint{0.997489in}{0.528000in}}{\pgfqpoint{4.565023in}{3.696000in}}%
\pgfusepath{clip}%
\pgfsetbuttcap%
\pgfsetroundjoin%
\definecolor{currentfill}{rgb}{0.000000,0.000000,0.000000}%
\pgfsetfillcolor{currentfill}%
\pgfsetlinewidth{1.003750pt}%
\definecolor{currentstroke}{rgb}{0.000000,0.000000,0.000000}%
\pgfsetstrokecolor{currentstroke}%
\pgfsetdash{}{0pt}%
\pgfsys@defobject{currentmarker}{\pgfqpoint{-0.021960in}{-0.021960in}}{\pgfqpoint{0.021960in}{0.021960in}}{%
\pgfpathmoveto{\pgfqpoint{0.000000in}{-0.021960in}}%
\pgfpathcurveto{\pgfqpoint{0.005824in}{-0.021960in}}{\pgfqpoint{0.011410in}{-0.019646in}}{\pgfqpoint{0.015528in}{-0.015528in}}%
\pgfpathcurveto{\pgfqpoint{0.019646in}{-0.011410in}}{\pgfqpoint{0.021960in}{-0.005824in}}{\pgfqpoint{0.021960in}{0.000000in}}%
\pgfpathcurveto{\pgfqpoint{0.021960in}{0.005824in}}{\pgfqpoint{0.019646in}{0.011410in}}{\pgfqpoint{0.015528in}{0.015528in}}%
\pgfpathcurveto{\pgfqpoint{0.011410in}{0.019646in}}{\pgfqpoint{0.005824in}{0.021960in}}{\pgfqpoint{0.000000in}{0.021960in}}%
\pgfpathcurveto{\pgfqpoint{-0.005824in}{0.021960in}}{\pgfqpoint{-0.011410in}{0.019646in}}{\pgfqpoint{-0.015528in}{0.015528in}}%
\pgfpathcurveto{\pgfqpoint{-0.019646in}{0.011410in}}{\pgfqpoint{-0.021960in}{0.005824in}}{\pgfqpoint{-0.021960in}{0.000000in}}%
\pgfpathcurveto{\pgfqpoint{-0.021960in}{-0.005824in}}{\pgfqpoint{-0.019646in}{-0.011410in}}{\pgfqpoint{-0.015528in}{-0.015528in}}%
\pgfpathcurveto{\pgfqpoint{-0.011410in}{-0.019646in}}{\pgfqpoint{-0.005824in}{-0.021960in}}{\pgfqpoint{0.000000in}{-0.021960in}}%
\pgfpathlineto{\pgfqpoint{0.000000in}{-0.021960in}}%
\pgfpathclose%
\pgfusepath{stroke,fill}%
}%
\begin{pgfscope}%
\pgfsys@transformshift{2.168676in}{2.941174in}%
\pgfsys@useobject{currentmarker}{}%
\end{pgfscope}%
\end{pgfscope}%
\begin{pgfscope}%
\pgfpathrectangle{\pgfqpoint{0.997489in}{0.528000in}}{\pgfqpoint{4.565023in}{3.696000in}}%
\pgfusepath{clip}%
\pgfsetbuttcap%
\pgfsetmiterjoin%
\pgfsetlinewidth{1.003750pt}%
\definecolor{currentstroke}{rgb}{0.200000,0.200000,0.800000}%
\pgfsetstrokecolor{currentstroke}%
\pgfsetdash{}{0pt}%
\pgfpathmoveto{\pgfqpoint{3.207991in}{2.013269in}}%
\pgfpathcurveto{\pgfqpoint{3.344109in}{2.013269in}}{\pgfqpoint{3.474671in}{2.067350in}}{\pgfqpoint{3.570921in}{2.163600in}}%
\pgfpathcurveto{\pgfqpoint{3.667172in}{2.259850in}}{\pgfqpoint{3.721252in}{2.390412in}}{\pgfqpoint{3.721252in}{2.526531in}}%
\pgfpathcurveto{\pgfqpoint{3.721252in}{2.662649in}}{\pgfqpoint{3.667172in}{2.793211in}}{\pgfqpoint{3.570921in}{2.889461in}}%
\pgfpathcurveto{\pgfqpoint{3.474671in}{2.985711in}}{\pgfqpoint{3.344109in}{3.039792in}}{\pgfqpoint{3.207991in}{3.039792in}}%
\pgfpathcurveto{\pgfqpoint{3.071872in}{3.039792in}}{\pgfqpoint{2.941310in}{2.985711in}}{\pgfqpoint{2.845060in}{2.889461in}}%
\pgfpathcurveto{\pgfqpoint{2.748810in}{2.793211in}}{\pgfqpoint{2.694729in}{2.662649in}}{\pgfqpoint{2.694729in}{2.526531in}}%
\pgfpathcurveto{\pgfqpoint{2.694729in}{2.390412in}}{\pgfqpoint{2.748810in}{2.259850in}}{\pgfqpoint{2.845060in}{2.163600in}}%
\pgfpathcurveto{\pgfqpoint{2.941310in}{2.067350in}}{\pgfqpoint{3.071872in}{2.013269in}}{\pgfqpoint{3.207991in}{2.013269in}}%
\pgfpathlineto{\pgfqpoint{3.207991in}{2.013269in}}%
\pgfpathclose%
\pgfusepath{stroke}%
\end{pgfscope}%
\begin{pgfscope}%
\pgfpathrectangle{\pgfqpoint{0.997489in}{0.528000in}}{\pgfqpoint{4.565023in}{3.696000in}}%
\pgfusepath{clip}%
\pgfsetbuttcap%
\pgfsetroundjoin%
\definecolor{currentfill}{rgb}{0.000000,0.000000,0.000000}%
\pgfsetfillcolor{currentfill}%
\pgfsetlinewidth{1.003750pt}%
\definecolor{currentstroke}{rgb}{0.000000,0.000000,0.000000}%
\pgfsetstrokecolor{currentstroke}%
\pgfsetdash{}{0pt}%
\pgfsys@defobject{currentmarker}{\pgfqpoint{-0.021960in}{-0.021960in}}{\pgfqpoint{0.021960in}{0.021960in}}{%
\pgfpathmoveto{\pgfqpoint{0.000000in}{-0.021960in}}%
\pgfpathcurveto{\pgfqpoint{0.005824in}{-0.021960in}}{\pgfqpoint{0.011410in}{-0.019646in}}{\pgfqpoint{0.015528in}{-0.015528in}}%
\pgfpathcurveto{\pgfqpoint{0.019646in}{-0.011410in}}{\pgfqpoint{0.021960in}{-0.005824in}}{\pgfqpoint{0.021960in}{0.000000in}}%
\pgfpathcurveto{\pgfqpoint{0.021960in}{0.005824in}}{\pgfqpoint{0.019646in}{0.011410in}}{\pgfqpoint{0.015528in}{0.015528in}}%
\pgfpathcurveto{\pgfqpoint{0.011410in}{0.019646in}}{\pgfqpoint{0.005824in}{0.021960in}}{\pgfqpoint{0.000000in}{0.021960in}}%
\pgfpathcurveto{\pgfqpoint{-0.005824in}{0.021960in}}{\pgfqpoint{-0.011410in}{0.019646in}}{\pgfqpoint{-0.015528in}{0.015528in}}%
\pgfpathcurveto{\pgfqpoint{-0.019646in}{0.011410in}}{\pgfqpoint{-0.021960in}{0.005824in}}{\pgfqpoint{-0.021960in}{0.000000in}}%
\pgfpathcurveto{\pgfqpoint{-0.021960in}{-0.005824in}}{\pgfqpoint{-0.019646in}{-0.011410in}}{\pgfqpoint{-0.015528in}{-0.015528in}}%
\pgfpathcurveto{\pgfqpoint{-0.011410in}{-0.019646in}}{\pgfqpoint{-0.005824in}{-0.021960in}}{\pgfqpoint{0.000000in}{-0.021960in}}%
\pgfpathlineto{\pgfqpoint{0.000000in}{-0.021960in}}%
\pgfpathclose%
\pgfusepath{stroke,fill}%
}%
\begin{pgfscope}%
\pgfsys@transformshift{3.207991in}{2.526531in}%
\pgfsys@useobject{currentmarker}{}%
\end{pgfscope}%
\end{pgfscope}%
\begin{pgfscope}%
\pgfpathrectangle{\pgfqpoint{0.997489in}{0.528000in}}{\pgfqpoint{4.565023in}{3.696000in}}%
\pgfusepath{clip}%
\pgfsetbuttcap%
\pgfsetmiterjoin%
\pgfsetlinewidth{1.003750pt}%
\definecolor{currentstroke}{rgb}{0.800000,0.800000,0.200000}%
\pgfsetstrokecolor{currentstroke}%
\pgfsetdash{}{0pt}%
\pgfpathmoveto{\pgfqpoint{4.361149in}{2.304247in}}%
\pgfpathcurveto{\pgfqpoint{4.582831in}{2.304247in}}{\pgfqpoint{4.795463in}{2.392322in}}{\pgfqpoint{4.952215in}{2.549074in}}%
\pgfpathcurveto{\pgfqpoint{5.108968in}{2.705827in}}{\pgfqpoint{5.197043in}{2.918459in}}{\pgfqpoint{5.197043in}{3.140141in}}%
\pgfpathcurveto{\pgfqpoint{5.197043in}{3.361822in}}{\pgfqpoint{5.108968in}{3.574454in}}{\pgfqpoint{4.952215in}{3.731207in}}%
\pgfpathcurveto{\pgfqpoint{4.795463in}{3.887960in}}{\pgfqpoint{4.582831in}{3.976035in}}{\pgfqpoint{4.361149in}{3.976035in}}%
\pgfpathcurveto{\pgfqpoint{4.139467in}{3.976035in}}{\pgfqpoint{3.926835in}{3.887960in}}{\pgfqpoint{3.770083in}{3.731207in}}%
\pgfpathcurveto{\pgfqpoint{3.613330in}{3.574454in}}{\pgfqpoint{3.525255in}{3.361822in}}{\pgfqpoint{3.525255in}{3.140141in}}%
\pgfpathcurveto{\pgfqpoint{3.525255in}{2.918459in}}{\pgfqpoint{3.613330in}{2.705827in}}{\pgfqpoint{3.770083in}{2.549074in}}%
\pgfpathcurveto{\pgfqpoint{3.926835in}{2.392322in}}{\pgfqpoint{4.139467in}{2.304247in}}{\pgfqpoint{4.361149in}{2.304247in}}%
\pgfpathlineto{\pgfqpoint{4.361149in}{2.304247in}}%
\pgfpathclose%
\pgfusepath{stroke}%
\end{pgfscope}%
\begin{pgfscope}%
\pgfpathrectangle{\pgfqpoint{0.997489in}{0.528000in}}{\pgfqpoint{4.565023in}{3.696000in}}%
\pgfusepath{clip}%
\pgfsetbuttcap%
\pgfsetroundjoin%
\definecolor{currentfill}{rgb}{0.000000,0.000000,0.000000}%
\pgfsetfillcolor{currentfill}%
\pgfsetlinewidth{1.003750pt}%
\definecolor{currentstroke}{rgb}{0.000000,0.000000,0.000000}%
\pgfsetstrokecolor{currentstroke}%
\pgfsetdash{}{0pt}%
\pgfsys@defobject{currentmarker}{\pgfqpoint{-0.021960in}{-0.021960in}}{\pgfqpoint{0.021960in}{0.021960in}}{%
\pgfpathmoveto{\pgfqpoint{0.000000in}{-0.021960in}}%
\pgfpathcurveto{\pgfqpoint{0.005824in}{-0.021960in}}{\pgfqpoint{0.011410in}{-0.019646in}}{\pgfqpoint{0.015528in}{-0.015528in}}%
\pgfpathcurveto{\pgfqpoint{0.019646in}{-0.011410in}}{\pgfqpoint{0.021960in}{-0.005824in}}{\pgfqpoint{0.021960in}{0.000000in}}%
\pgfpathcurveto{\pgfqpoint{0.021960in}{0.005824in}}{\pgfqpoint{0.019646in}{0.011410in}}{\pgfqpoint{0.015528in}{0.015528in}}%
\pgfpathcurveto{\pgfqpoint{0.011410in}{0.019646in}}{\pgfqpoint{0.005824in}{0.021960in}}{\pgfqpoint{0.000000in}{0.021960in}}%
\pgfpathcurveto{\pgfqpoint{-0.005824in}{0.021960in}}{\pgfqpoint{-0.011410in}{0.019646in}}{\pgfqpoint{-0.015528in}{0.015528in}}%
\pgfpathcurveto{\pgfqpoint{-0.019646in}{0.011410in}}{\pgfqpoint{-0.021960in}{0.005824in}}{\pgfqpoint{-0.021960in}{0.000000in}}%
\pgfpathcurveto{\pgfqpoint{-0.021960in}{-0.005824in}}{\pgfqpoint{-0.019646in}{-0.011410in}}{\pgfqpoint{-0.015528in}{-0.015528in}}%
\pgfpathcurveto{\pgfqpoint{-0.011410in}{-0.019646in}}{\pgfqpoint{-0.005824in}{-0.021960in}}{\pgfqpoint{0.000000in}{-0.021960in}}%
\pgfpathlineto{\pgfqpoint{0.000000in}{-0.021960in}}%
\pgfpathclose%
\pgfusepath{stroke,fill}%
}%
\begin{pgfscope}%
\pgfsys@transformshift{4.361149in}{3.140141in}%
\pgfsys@useobject{currentmarker}{}%
\end{pgfscope}%
\end{pgfscope}%
\begin{pgfscope}%
\pgfsetbuttcap%
\pgfsetroundjoin%
\definecolor{currentfill}{rgb}{0.000000,0.000000,0.000000}%
\pgfsetfillcolor{currentfill}%
\pgfsetlinewidth{0.803000pt}%
\definecolor{currentstroke}{rgb}{0.000000,0.000000,0.000000}%
\pgfsetstrokecolor{currentstroke}%
\pgfsetdash{}{0pt}%
\pgfsys@defobject{currentmarker}{\pgfqpoint{0.000000in}{-0.048611in}}{\pgfqpoint{0.000000in}{0.000000in}}{%
\pgfpathmoveto{\pgfqpoint{0.000000in}{0.000000in}}%
\pgfpathlineto{\pgfqpoint{0.000000in}{-0.048611in}}%
\pgfusepath{stroke,fill}%
}%
\begin{pgfscope}%
\pgfsys@transformshift{1.367423in}{0.528000in}%
\pgfsys@useobject{currentmarker}{}%
\end{pgfscope}%
\end{pgfscope}%
\begin{pgfscope}%
\definecolor{textcolor}{rgb}{0.000000,0.000000,0.000000}%
\pgfsetstrokecolor{textcolor}%
\pgfsetfillcolor{textcolor}%
\pgftext[x=1.367423in,y=0.430778in,,top]{\color{textcolor}{\sffamily\fontsize{10.000000}{12.000000}\selectfont\catcode`\^=\active\def^{\ifmmode\sp\else\^{}\fi}\catcode`\%=\active\def%{\%}\ensuremath{-}600}}%
\end{pgfscope}%
\begin{pgfscope}%
\pgfsetbuttcap%
\pgfsetroundjoin%
\definecolor{currentfill}{rgb}{0.000000,0.000000,0.000000}%
\pgfsetfillcolor{currentfill}%
\pgfsetlinewidth{0.803000pt}%
\definecolor{currentstroke}{rgb}{0.000000,0.000000,0.000000}%
\pgfsetstrokecolor{currentstroke}%
\pgfsetdash{}{0pt}%
\pgfsys@defobject{currentmarker}{\pgfqpoint{0.000000in}{-0.048611in}}{\pgfqpoint{0.000000in}{0.000000in}}{%
\pgfpathmoveto{\pgfqpoint{0.000000in}{0.000000in}}%
\pgfpathlineto{\pgfqpoint{0.000000in}{-0.048611in}}%
\pgfusepath{stroke,fill}%
}%
\begin{pgfscope}%
\pgfsys@transformshift{1.884840in}{0.528000in}%
\pgfsys@useobject{currentmarker}{}%
\end{pgfscope}%
\end{pgfscope}%
\begin{pgfscope}%
\definecolor{textcolor}{rgb}{0.000000,0.000000,0.000000}%
\pgfsetstrokecolor{textcolor}%
\pgfsetfillcolor{textcolor}%
\pgftext[x=1.884840in,y=0.430778in,,top]{\color{textcolor}{\sffamily\fontsize{10.000000}{12.000000}\selectfont\catcode`\^=\active\def^{\ifmmode\sp\else\^{}\fi}\catcode`\%=\active\def%{\%}\ensuremath{-}400}}%
\end{pgfscope}%
\begin{pgfscope}%
\pgfsetbuttcap%
\pgfsetroundjoin%
\definecolor{currentfill}{rgb}{0.000000,0.000000,0.000000}%
\pgfsetfillcolor{currentfill}%
\pgfsetlinewidth{0.803000pt}%
\definecolor{currentstroke}{rgb}{0.000000,0.000000,0.000000}%
\pgfsetstrokecolor{currentstroke}%
\pgfsetdash{}{0pt}%
\pgfsys@defobject{currentmarker}{\pgfqpoint{0.000000in}{-0.048611in}}{\pgfqpoint{0.000000in}{0.000000in}}{%
\pgfpathmoveto{\pgfqpoint{0.000000in}{0.000000in}}%
\pgfpathlineto{\pgfqpoint{0.000000in}{-0.048611in}}%
\pgfusepath{stroke,fill}%
}%
\begin{pgfscope}%
\pgfsys@transformshift{2.402257in}{0.528000in}%
\pgfsys@useobject{currentmarker}{}%
\end{pgfscope}%
\end{pgfscope}%
\begin{pgfscope}%
\definecolor{textcolor}{rgb}{0.000000,0.000000,0.000000}%
\pgfsetstrokecolor{textcolor}%
\pgfsetfillcolor{textcolor}%
\pgftext[x=2.402257in,y=0.430778in,,top]{\color{textcolor}{\sffamily\fontsize{10.000000}{12.000000}\selectfont\catcode`\^=\active\def^{\ifmmode\sp\else\^{}\fi}\catcode`\%=\active\def%{\%}\ensuremath{-}200}}%
\end{pgfscope}%
\begin{pgfscope}%
\pgfsetbuttcap%
\pgfsetroundjoin%
\definecolor{currentfill}{rgb}{0.000000,0.000000,0.000000}%
\pgfsetfillcolor{currentfill}%
\pgfsetlinewidth{0.803000pt}%
\definecolor{currentstroke}{rgb}{0.000000,0.000000,0.000000}%
\pgfsetstrokecolor{currentstroke}%
\pgfsetdash{}{0pt}%
\pgfsys@defobject{currentmarker}{\pgfqpoint{0.000000in}{-0.048611in}}{\pgfqpoint{0.000000in}{0.000000in}}{%
\pgfpathmoveto{\pgfqpoint{0.000000in}{0.000000in}}%
\pgfpathlineto{\pgfqpoint{0.000000in}{-0.048611in}}%
\pgfusepath{stroke,fill}%
}%
\begin{pgfscope}%
\pgfsys@transformshift{2.919673in}{0.528000in}%
\pgfsys@useobject{currentmarker}{}%
\end{pgfscope}%
\end{pgfscope}%
\begin{pgfscope}%
\definecolor{textcolor}{rgb}{0.000000,0.000000,0.000000}%
\pgfsetstrokecolor{textcolor}%
\pgfsetfillcolor{textcolor}%
\pgftext[x=2.919673in,y=0.430778in,,top]{\color{textcolor}{\sffamily\fontsize{10.000000}{12.000000}\selectfont\catcode`\^=\active\def^{\ifmmode\sp\else\^{}\fi}\catcode`\%=\active\def%{\%}0}}%
\end{pgfscope}%
\begin{pgfscope}%
\pgfsetbuttcap%
\pgfsetroundjoin%
\definecolor{currentfill}{rgb}{0.000000,0.000000,0.000000}%
\pgfsetfillcolor{currentfill}%
\pgfsetlinewidth{0.803000pt}%
\definecolor{currentstroke}{rgb}{0.000000,0.000000,0.000000}%
\pgfsetstrokecolor{currentstroke}%
\pgfsetdash{}{0pt}%
\pgfsys@defobject{currentmarker}{\pgfqpoint{0.000000in}{-0.048611in}}{\pgfqpoint{0.000000in}{0.000000in}}{%
\pgfpathmoveto{\pgfqpoint{0.000000in}{0.000000in}}%
\pgfpathlineto{\pgfqpoint{0.000000in}{-0.048611in}}%
\pgfusepath{stroke,fill}%
}%
\begin{pgfscope}%
\pgfsys@transformshift{3.437090in}{0.528000in}%
\pgfsys@useobject{currentmarker}{}%
\end{pgfscope}%
\end{pgfscope}%
\begin{pgfscope}%
\definecolor{textcolor}{rgb}{0.000000,0.000000,0.000000}%
\pgfsetstrokecolor{textcolor}%
\pgfsetfillcolor{textcolor}%
\pgftext[x=3.437090in,y=0.430778in,,top]{\color{textcolor}{\sffamily\fontsize{10.000000}{12.000000}\selectfont\catcode`\^=\active\def^{\ifmmode\sp\else\^{}\fi}\catcode`\%=\active\def%{\%}200}}%
\end{pgfscope}%
\begin{pgfscope}%
\pgfsetbuttcap%
\pgfsetroundjoin%
\definecolor{currentfill}{rgb}{0.000000,0.000000,0.000000}%
\pgfsetfillcolor{currentfill}%
\pgfsetlinewidth{0.803000pt}%
\definecolor{currentstroke}{rgb}{0.000000,0.000000,0.000000}%
\pgfsetstrokecolor{currentstroke}%
\pgfsetdash{}{0pt}%
\pgfsys@defobject{currentmarker}{\pgfqpoint{0.000000in}{-0.048611in}}{\pgfqpoint{0.000000in}{0.000000in}}{%
\pgfpathmoveto{\pgfqpoint{0.000000in}{0.000000in}}%
\pgfpathlineto{\pgfqpoint{0.000000in}{-0.048611in}}%
\pgfusepath{stroke,fill}%
}%
\begin{pgfscope}%
\pgfsys@transformshift{3.954507in}{0.528000in}%
\pgfsys@useobject{currentmarker}{}%
\end{pgfscope}%
\end{pgfscope}%
\begin{pgfscope}%
\definecolor{textcolor}{rgb}{0.000000,0.000000,0.000000}%
\pgfsetstrokecolor{textcolor}%
\pgfsetfillcolor{textcolor}%
\pgftext[x=3.954507in,y=0.430778in,,top]{\color{textcolor}{\sffamily\fontsize{10.000000}{12.000000}\selectfont\catcode`\^=\active\def^{\ifmmode\sp\else\^{}\fi}\catcode`\%=\active\def%{\%}400}}%
\end{pgfscope}%
\begin{pgfscope}%
\pgfsetbuttcap%
\pgfsetroundjoin%
\definecolor{currentfill}{rgb}{0.000000,0.000000,0.000000}%
\pgfsetfillcolor{currentfill}%
\pgfsetlinewidth{0.803000pt}%
\definecolor{currentstroke}{rgb}{0.000000,0.000000,0.000000}%
\pgfsetstrokecolor{currentstroke}%
\pgfsetdash{}{0pt}%
\pgfsys@defobject{currentmarker}{\pgfqpoint{0.000000in}{-0.048611in}}{\pgfqpoint{0.000000in}{0.000000in}}{%
\pgfpathmoveto{\pgfqpoint{0.000000in}{0.000000in}}%
\pgfpathlineto{\pgfqpoint{0.000000in}{-0.048611in}}%
\pgfusepath{stroke,fill}%
}%
\begin{pgfscope}%
\pgfsys@transformshift{4.471923in}{0.528000in}%
\pgfsys@useobject{currentmarker}{}%
\end{pgfscope}%
\end{pgfscope}%
\begin{pgfscope}%
\definecolor{textcolor}{rgb}{0.000000,0.000000,0.000000}%
\pgfsetstrokecolor{textcolor}%
\pgfsetfillcolor{textcolor}%
\pgftext[x=4.471923in,y=0.430778in,,top]{\color{textcolor}{\sffamily\fontsize{10.000000}{12.000000}\selectfont\catcode`\^=\active\def^{\ifmmode\sp\else\^{}\fi}\catcode`\%=\active\def%{\%}600}}%
\end{pgfscope}%
\begin{pgfscope}%
\pgfsetbuttcap%
\pgfsetroundjoin%
\definecolor{currentfill}{rgb}{0.000000,0.000000,0.000000}%
\pgfsetfillcolor{currentfill}%
\pgfsetlinewidth{0.803000pt}%
\definecolor{currentstroke}{rgb}{0.000000,0.000000,0.000000}%
\pgfsetstrokecolor{currentstroke}%
\pgfsetdash{}{0pt}%
\pgfsys@defobject{currentmarker}{\pgfqpoint{0.000000in}{-0.048611in}}{\pgfqpoint{0.000000in}{0.000000in}}{%
\pgfpathmoveto{\pgfqpoint{0.000000in}{0.000000in}}%
\pgfpathlineto{\pgfqpoint{0.000000in}{-0.048611in}}%
\pgfusepath{stroke,fill}%
}%
\begin{pgfscope}%
\pgfsys@transformshift{4.989340in}{0.528000in}%
\pgfsys@useobject{currentmarker}{}%
\end{pgfscope}%
\end{pgfscope}%
\begin{pgfscope}%
\definecolor{textcolor}{rgb}{0.000000,0.000000,0.000000}%
\pgfsetstrokecolor{textcolor}%
\pgfsetfillcolor{textcolor}%
\pgftext[x=4.989340in,y=0.430778in,,top]{\color{textcolor}{\sffamily\fontsize{10.000000}{12.000000}\selectfont\catcode`\^=\active\def^{\ifmmode\sp\else\^{}\fi}\catcode`\%=\active\def%{\%}800}}%
\end{pgfscope}%
\begin{pgfscope}%
\pgfsetbuttcap%
\pgfsetroundjoin%
\definecolor{currentfill}{rgb}{0.000000,0.000000,0.000000}%
\pgfsetfillcolor{currentfill}%
\pgfsetlinewidth{0.803000pt}%
\definecolor{currentstroke}{rgb}{0.000000,0.000000,0.000000}%
\pgfsetstrokecolor{currentstroke}%
\pgfsetdash{}{0pt}%
\pgfsys@defobject{currentmarker}{\pgfqpoint{0.000000in}{-0.048611in}}{\pgfqpoint{0.000000in}{0.000000in}}{%
\pgfpathmoveto{\pgfqpoint{0.000000in}{0.000000in}}%
\pgfpathlineto{\pgfqpoint{0.000000in}{-0.048611in}}%
\pgfusepath{stroke,fill}%
}%
\begin{pgfscope}%
\pgfsys@transformshift{5.506757in}{0.528000in}%
\pgfsys@useobject{currentmarker}{}%
\end{pgfscope}%
\end{pgfscope}%
\begin{pgfscope}%
\definecolor{textcolor}{rgb}{0.000000,0.000000,0.000000}%
\pgfsetstrokecolor{textcolor}%
\pgfsetfillcolor{textcolor}%
\pgftext[x=5.506757in,y=0.430778in,,top]{\color{textcolor}{\sffamily\fontsize{10.000000}{12.000000}\selectfont\catcode`\^=\active\def^{\ifmmode\sp\else\^{}\fi}\catcode`\%=\active\def%{\%}1000}}%
\end{pgfscope}%
\begin{pgfscope}%
\pgfsetbuttcap%
\pgfsetroundjoin%
\definecolor{currentfill}{rgb}{0.000000,0.000000,0.000000}%
\pgfsetfillcolor{currentfill}%
\pgfsetlinewidth{0.803000pt}%
\definecolor{currentstroke}{rgb}{0.000000,0.000000,0.000000}%
\pgfsetstrokecolor{currentstroke}%
\pgfsetdash{}{0pt}%
\pgfsys@defobject{currentmarker}{\pgfqpoint{-0.048611in}{0.000000in}}{\pgfqpoint{-0.000000in}{0.000000in}}{%
\pgfpathmoveto{\pgfqpoint{-0.000000in}{0.000000in}}%
\pgfpathlineto{\pgfqpoint{-0.048611in}{0.000000in}}%
\pgfusepath{stroke,fill}%
}%
\begin{pgfscope}%
\pgfsys@transformshift{0.997489in}{0.688116in}%
\pgfsys@useobject{currentmarker}{}%
\end{pgfscope}%
\end{pgfscope}%
\begin{pgfscope}%
\definecolor{textcolor}{rgb}{0.000000,0.000000,0.000000}%
\pgfsetstrokecolor{textcolor}%
\pgfsetfillcolor{textcolor}%
\pgftext[x=0.527145in, y=0.635354in, left, base]{\color{textcolor}{\sffamily\fontsize{10.000000}{12.000000}\selectfont\catcode`\^=\active\def^{\ifmmode\sp\else\^{}\fi}\catcode`\%=\active\def%{\%}\ensuremath{-}600}}%
\end{pgfscope}%
\begin{pgfscope}%
\pgfsetbuttcap%
\pgfsetroundjoin%
\definecolor{currentfill}{rgb}{0.000000,0.000000,0.000000}%
\pgfsetfillcolor{currentfill}%
\pgfsetlinewidth{0.803000pt}%
\definecolor{currentstroke}{rgb}{0.000000,0.000000,0.000000}%
\pgfsetstrokecolor{currentstroke}%
\pgfsetdash{}{0pt}%
\pgfsys@defobject{currentmarker}{\pgfqpoint{-0.048611in}{0.000000in}}{\pgfqpoint{-0.000000in}{0.000000in}}{%
\pgfpathmoveto{\pgfqpoint{-0.000000in}{0.000000in}}%
\pgfpathlineto{\pgfqpoint{-0.048611in}{0.000000in}}%
\pgfusepath{stroke,fill}%
}%
\begin{pgfscope}%
\pgfsys@transformshift{0.997489in}{1.205532in}%
\pgfsys@useobject{currentmarker}{}%
\end{pgfscope}%
\end{pgfscope}%
\begin{pgfscope}%
\definecolor{textcolor}{rgb}{0.000000,0.000000,0.000000}%
\pgfsetstrokecolor{textcolor}%
\pgfsetfillcolor{textcolor}%
\pgftext[x=0.527145in, y=1.152771in, left, base]{\color{textcolor}{\sffamily\fontsize{10.000000}{12.000000}\selectfont\catcode`\^=\active\def^{\ifmmode\sp\else\^{}\fi}\catcode`\%=\active\def%{\%}\ensuremath{-}400}}%
\end{pgfscope}%
\begin{pgfscope}%
\pgfsetbuttcap%
\pgfsetroundjoin%
\definecolor{currentfill}{rgb}{0.000000,0.000000,0.000000}%
\pgfsetfillcolor{currentfill}%
\pgfsetlinewidth{0.803000pt}%
\definecolor{currentstroke}{rgb}{0.000000,0.000000,0.000000}%
\pgfsetstrokecolor{currentstroke}%
\pgfsetdash{}{0pt}%
\pgfsys@defobject{currentmarker}{\pgfqpoint{-0.048611in}{0.000000in}}{\pgfqpoint{-0.000000in}{0.000000in}}{%
\pgfpathmoveto{\pgfqpoint{-0.000000in}{0.000000in}}%
\pgfpathlineto{\pgfqpoint{-0.048611in}{0.000000in}}%
\pgfusepath{stroke,fill}%
}%
\begin{pgfscope}%
\pgfsys@transformshift{0.997489in}{1.722949in}%
\pgfsys@useobject{currentmarker}{}%
\end{pgfscope}%
\end{pgfscope}%
\begin{pgfscope}%
\definecolor{textcolor}{rgb}{0.000000,0.000000,0.000000}%
\pgfsetstrokecolor{textcolor}%
\pgfsetfillcolor{textcolor}%
\pgftext[x=0.527145in, y=1.670187in, left, base]{\color{textcolor}{\sffamily\fontsize{10.000000}{12.000000}\selectfont\catcode`\^=\active\def^{\ifmmode\sp\else\^{}\fi}\catcode`\%=\active\def%{\%}\ensuremath{-}200}}%
\end{pgfscope}%
\begin{pgfscope}%
\pgfsetbuttcap%
\pgfsetroundjoin%
\definecolor{currentfill}{rgb}{0.000000,0.000000,0.000000}%
\pgfsetfillcolor{currentfill}%
\pgfsetlinewidth{0.803000pt}%
\definecolor{currentstroke}{rgb}{0.000000,0.000000,0.000000}%
\pgfsetstrokecolor{currentstroke}%
\pgfsetdash{}{0pt}%
\pgfsys@defobject{currentmarker}{\pgfqpoint{-0.048611in}{0.000000in}}{\pgfqpoint{-0.000000in}{0.000000in}}{%
\pgfpathmoveto{\pgfqpoint{-0.000000in}{0.000000in}}%
\pgfpathlineto{\pgfqpoint{-0.048611in}{0.000000in}}%
\pgfusepath{stroke,fill}%
}%
\begin{pgfscope}%
\pgfsys@transformshift{0.997489in}{2.240366in}%
\pgfsys@useobject{currentmarker}{}%
\end{pgfscope}%
\end{pgfscope}%
\begin{pgfscope}%
\definecolor{textcolor}{rgb}{0.000000,0.000000,0.000000}%
\pgfsetstrokecolor{textcolor}%
\pgfsetfillcolor{textcolor}%
\pgftext[x=0.811901in, y=2.187604in, left, base]{\color{textcolor}{\sffamily\fontsize{10.000000}{12.000000}\selectfont\catcode`\^=\active\def^{\ifmmode\sp\else\^{}\fi}\catcode`\%=\active\def%{\%}0}}%
\end{pgfscope}%
\begin{pgfscope}%
\pgfsetbuttcap%
\pgfsetroundjoin%
\definecolor{currentfill}{rgb}{0.000000,0.000000,0.000000}%
\pgfsetfillcolor{currentfill}%
\pgfsetlinewidth{0.803000pt}%
\definecolor{currentstroke}{rgb}{0.000000,0.000000,0.000000}%
\pgfsetstrokecolor{currentstroke}%
\pgfsetdash{}{0pt}%
\pgfsys@defobject{currentmarker}{\pgfqpoint{-0.048611in}{0.000000in}}{\pgfqpoint{-0.000000in}{0.000000in}}{%
\pgfpathmoveto{\pgfqpoint{-0.000000in}{0.000000in}}%
\pgfpathlineto{\pgfqpoint{-0.048611in}{0.000000in}}%
\pgfusepath{stroke,fill}%
}%
\begin{pgfscope}%
\pgfsys@transformshift{0.997489in}{2.757782in}%
\pgfsys@useobject{currentmarker}{}%
\end{pgfscope}%
\end{pgfscope}%
\begin{pgfscope}%
\definecolor{textcolor}{rgb}{0.000000,0.000000,0.000000}%
\pgfsetstrokecolor{textcolor}%
\pgfsetfillcolor{textcolor}%
\pgftext[x=0.635170in, y=2.705021in, left, base]{\color{textcolor}{\sffamily\fontsize{10.000000}{12.000000}\selectfont\catcode`\^=\active\def^{\ifmmode\sp\else\^{}\fi}\catcode`\%=\active\def%{\%}200}}%
\end{pgfscope}%
\begin{pgfscope}%
\pgfsetbuttcap%
\pgfsetroundjoin%
\definecolor{currentfill}{rgb}{0.000000,0.000000,0.000000}%
\pgfsetfillcolor{currentfill}%
\pgfsetlinewidth{0.803000pt}%
\definecolor{currentstroke}{rgb}{0.000000,0.000000,0.000000}%
\pgfsetstrokecolor{currentstroke}%
\pgfsetdash{}{0pt}%
\pgfsys@defobject{currentmarker}{\pgfqpoint{-0.048611in}{0.000000in}}{\pgfqpoint{-0.000000in}{0.000000in}}{%
\pgfpathmoveto{\pgfqpoint{-0.000000in}{0.000000in}}%
\pgfpathlineto{\pgfqpoint{-0.048611in}{0.000000in}}%
\pgfusepath{stroke,fill}%
}%
\begin{pgfscope}%
\pgfsys@transformshift{0.997489in}{3.275199in}%
\pgfsys@useobject{currentmarker}{}%
\end{pgfscope}%
\end{pgfscope}%
\begin{pgfscope}%
\definecolor{textcolor}{rgb}{0.000000,0.000000,0.000000}%
\pgfsetstrokecolor{textcolor}%
\pgfsetfillcolor{textcolor}%
\pgftext[x=0.635170in, y=3.222437in, left, base]{\color{textcolor}{\sffamily\fontsize{10.000000}{12.000000}\selectfont\catcode`\^=\active\def^{\ifmmode\sp\else\^{}\fi}\catcode`\%=\active\def%{\%}400}}%
\end{pgfscope}%
\begin{pgfscope}%
\pgfsetbuttcap%
\pgfsetroundjoin%
\definecolor{currentfill}{rgb}{0.000000,0.000000,0.000000}%
\pgfsetfillcolor{currentfill}%
\pgfsetlinewidth{0.803000pt}%
\definecolor{currentstroke}{rgb}{0.000000,0.000000,0.000000}%
\pgfsetstrokecolor{currentstroke}%
\pgfsetdash{}{0pt}%
\pgfsys@defobject{currentmarker}{\pgfqpoint{-0.048611in}{0.000000in}}{\pgfqpoint{-0.000000in}{0.000000in}}{%
\pgfpathmoveto{\pgfqpoint{-0.000000in}{0.000000in}}%
\pgfpathlineto{\pgfqpoint{-0.048611in}{0.000000in}}%
\pgfusepath{stroke,fill}%
}%
\begin{pgfscope}%
\pgfsys@transformshift{0.997489in}{3.792616in}%
\pgfsys@useobject{currentmarker}{}%
\end{pgfscope}%
\end{pgfscope}%
\begin{pgfscope}%
\definecolor{textcolor}{rgb}{0.000000,0.000000,0.000000}%
\pgfsetstrokecolor{textcolor}%
\pgfsetfillcolor{textcolor}%
\pgftext[x=0.635170in, y=3.739854in, left, base]{\color{textcolor}{\sffamily\fontsize{10.000000}{12.000000}\selectfont\catcode`\^=\active\def^{\ifmmode\sp\else\^{}\fi}\catcode`\%=\active\def%{\%}600}}%
\end{pgfscope}%
\begin{pgfscope}%
\pgfsetrectcap%
\pgfsetmiterjoin%
\pgfsetlinewidth{0.803000pt}%
\definecolor{currentstroke}{rgb}{0.000000,0.000000,0.000000}%
\pgfsetstrokecolor{currentstroke}%
\pgfsetdash{}{0pt}%
\pgfpathmoveto{\pgfqpoint{0.997489in}{0.528000in}}%
\pgfpathlineto{\pgfqpoint{0.997489in}{4.224000in}}%
\pgfusepath{stroke}%
\end{pgfscope}%
\begin{pgfscope}%
\pgfsetrectcap%
\pgfsetmiterjoin%
\pgfsetlinewidth{0.803000pt}%
\definecolor{currentstroke}{rgb}{0.000000,0.000000,0.000000}%
\pgfsetstrokecolor{currentstroke}%
\pgfsetdash{}{0pt}%
\pgfpathmoveto{\pgfqpoint{5.562511in}{0.528000in}}%
\pgfpathlineto{\pgfqpoint{5.562511in}{4.224000in}}%
\pgfusepath{stroke}%
\end{pgfscope}%
\begin{pgfscope}%
\pgfsetrectcap%
\pgfsetmiterjoin%
\pgfsetlinewidth{0.803000pt}%
\definecolor{currentstroke}{rgb}{0.000000,0.000000,0.000000}%
\pgfsetstrokecolor{currentstroke}%
\pgfsetdash{}{0pt}%
\pgfpathmoveto{\pgfqpoint{0.997489in}{0.528000in}}%
\pgfpathlineto{\pgfqpoint{5.562511in}{0.528000in}}%
\pgfusepath{stroke}%
\end{pgfscope}%
\begin{pgfscope}%
\pgfsetrectcap%
\pgfsetmiterjoin%
\pgfsetlinewidth{0.803000pt}%
\definecolor{currentstroke}{rgb}{0.000000,0.000000,0.000000}%
\pgfsetstrokecolor{currentstroke}%
\pgfsetdash{}{0pt}%
\pgfpathmoveto{\pgfqpoint{0.997489in}{4.224000in}}%
\pgfpathlineto{\pgfqpoint{5.562511in}{4.224000in}}%
\pgfusepath{stroke}%
\end{pgfscope}%
\end{pgfpicture}%
\makeatother%
\endgroup%
}
    \label{fig:excentric_rings}
    \caption{Example of a dataset with 4 rings, 20 ring noise, 50 background noise, and a good classification.}
\end{figure}


\begin{figure*}[!ht]
\centering
\begin{tabular}{rrrrrrr}
    \hline
       N rings &   Rings noise &   Background noise &   Avg Error &   Avg Time &       Iters &   N Samples \\
    \hline
             1 &             0 &                  0 &     1.08838 & 0.00091765 &     2       &           2 \\
             1 &            20 &                  0 &    18.1971  & 0.0007517  &     2       &           1 \\
             1 &            20 &                 30 &    18.5382  & 0.00187321 &     5.84615 &          13 \\
             2 &             0 &                  0 &     6.06266 & 1.33305    &  6002.6     &           5 \\
             2 &            20 &                  0 &    18.0959  & 0.30551    &  1436.57    &           7 \\
             2 &            20 &                  0 &    24.5638  & 2.04091    & 10000       &           1 \\
             2 &            20 &                 10 &    18.5584  & 0.0377052  &   154       &           3 \\
             2 &            20 &                 30 &    41.7539  & 1.7823     &  7502.75    &           4 \\
             2 &            20 &                 50 &    33.3655  & 1.60088    &  6681       &           3 \\
             3 &             0 &                  0 &    16.4274  & 1.98715    &  6005.2     &           5 \\
             3 &            20 &                  0 &    25.3432  & 2.13128    &  6675.67    &           3 \\
             3 &            20 &                 30 &    74.6996  & 2.48335    &  7506.75    &           4 \\
             3 &            20 &                 50 &    44.0436  & 2.61325    &  7534       &           4 \\
             4 &             0 &                  0 &     9.92773 & 4.57052    & 10000       &           4 \\
             4 &            20 &                  0 &    35.2127  & 2.40506    &  5038.75    &           4 \\
             4 &            20 &                 50 &    74.773   & 3.77698    &  7552.5     &           4 \\
             5 &             0 &                  0 &    30.2353  & 4.92909    &  7580       &           4 \\
             5 &            20 &                  0 &    27.7939  & 3.35194    &  5041.5     &           4 \\
             5 &            20 &                 50 &    72.9169  & 7.07606    & 10000       &           4 \\
    \hline
\end{tabular}
\caption{Results of the general test with excentric rings.}
\end{figure*}
\vspace{12pt}

\bibliographystyle{IEEEtran}
\bibliography{references.bib}

\vspace{12pt}
\color{red}
\end{document}
