\documentclass[conference]{IEEEtran}
\IEEEoverridecommandlockouts
% The preceding line is only needed to identify funding in the first footnote. If that is unneeded, please comment it out.
\usepackage{cite}
\usepackage{amsmath,amssymb,amsfonts}
\usepackage{algorithmic}
\usepackage{graphicx}
\usepackage{textcomp}
\usepackage{xcolor}
\usepackage{tikz}
\usepackage{pgfplots}
\usepackage{float}
\usepackage{stfloats} 


\usetikzlibrary{calc}
\def\BibTeX{{\rm B\kern-.05em{\sc i\kern-.025em b}\kern-.08em
    T\kern-.1667em\lower.7ex\hbox{E}\kern-.125emX}}
\begin{document}

\title{Clustering Rings in Noisy Data}

\author{\IEEEauthorblockN{1\textsuperscript{st} David González Martínez}

\IEEEauthorblockA{\textit{University of Seville} \\
Seville, Spain \\
davgonmar2@alum.us.es}
}
\maketitle

\begin{abstract}
In this paper, we apply the Fuzzy K-Rings algorithm, also known as the Fuzzy C-Shells algorithm to ring clustering in noisy data.
We further propose a modification to the algorithm to make it more robust to noise, and conduct different experiments to test the performance of the algorithm.
\end{abstract}

\begin{IEEEkeywords}
Fuzzy K-Rings, Fuzzy C-Shells, Clustering, Noisy Data, Ring Clustering, Hypersphere Clustering
\end{IEEEkeywords}

\section{Introduction}
We are presented with the following problem: we have a dataset that is composed different rings and noise. Our objective is to classify the points in different clusters,
each corresponding to a different ring. The Fuzzy K-Rings algorithm, also known as the Fuzzy C-Shells algorithm, is a clustering algorithm that has been used in the past
for similar tasks. The algorithm is inspired on the Fuzzy C-Means algorithm, and was introduced, altough in different variations, in \cite{308484} and \cite{DAVE1992713}.
From now on, we'll refer to the algorithm as Fuzzy K-Rings (FKR).
% here a plot from plots/noisy_rings.pgf
\begin{figure}[H]
    \centering
    \resizebox{0.9\linewidth}{!}{%% Creator: Matplotlib, PGF backend
%%
%% To include the figure in your LaTeX document, write
%%   \input{<filename>.pgf}
%%
%% Make sure the required packages are loaded in your preamble
%%   \usepackage{pgf}
%%
%% Also ensure that all the required font packages are loaded; for instance,
%% the lmodern package is sometimes necessary when using math font.
%%   \usepackage{lmodern}
%%
%% Figures using additional raster images can only be included by \input if
%% they are in the same directory as the main LaTeX file. For loading figures
%% from other directories you can use the `import` package
%%   \usepackage{import}
%%
%% and then include the figures with
%%   \import{<path to file>}{<filename>.pgf}
%%
%% Matplotlib used the following preamble
%%   \def\mathdefault#1{#1}
%%   \everymath=\expandafter{\the\everymath\displaystyle}
%%   
%%   \usepackage{fontspec}
%%   \setmainfont{DejaVuSerif.ttf}[Path=\detokenize{C:/Users/dagom/anaconda3/envs/pytorch/lib/site-packages/matplotlib/mpl-data/fonts/ttf/}]
%%   \setsansfont{DejaVuSans.ttf}[Path=\detokenize{C:/Users/dagom/anaconda3/envs/pytorch/lib/site-packages/matplotlib/mpl-data/fonts/ttf/}]
%%   \setmonofont{DejaVuSansMono.ttf}[Path=\detokenize{C:/Users/dagom/anaconda3/envs/pytorch/lib/site-packages/matplotlib/mpl-data/fonts/ttf/}]
%%   \makeatletter\@ifpackageloaded{underscore}{}{\usepackage[strings]{underscore}}\makeatother
%%
\begingroup%
\makeatletter%
\begin{pgfpicture}%
\pgfpathrectangle{\pgfpointorigin}{\pgfqpoint{8.000000in}{8.000000in}}%
\pgfusepath{use as bounding box, clip}%
\begin{pgfscope}%
\pgfsetbuttcap%
\pgfsetmiterjoin%
\definecolor{currentfill}{rgb}{1.000000,1.000000,1.000000}%
\pgfsetfillcolor{currentfill}%
\pgfsetlinewidth{0.000000pt}%
\definecolor{currentstroke}{rgb}{1.000000,1.000000,1.000000}%
\pgfsetstrokecolor{currentstroke}%
\pgfsetdash{}{0pt}%
\pgfpathmoveto{\pgfqpoint{0.000000in}{0.000000in}}%
\pgfpathlineto{\pgfqpoint{8.000000in}{0.000000in}}%
\pgfpathlineto{\pgfqpoint{8.000000in}{8.000000in}}%
\pgfpathlineto{\pgfqpoint{0.000000in}{8.000000in}}%
\pgfpathlineto{\pgfqpoint{0.000000in}{0.000000in}}%
\pgfpathclose%
\pgfusepath{fill}%
\end{pgfscope}%
\begin{pgfscope}%
\pgfsetbuttcap%
\pgfsetmiterjoin%
\definecolor{currentfill}{rgb}{1.000000,1.000000,1.000000}%
\pgfsetfillcolor{currentfill}%
\pgfsetlinewidth{0.000000pt}%
\definecolor{currentstroke}{rgb}{0.000000,0.000000,0.000000}%
\pgfsetstrokecolor{currentstroke}%
\pgfsetstrokeopacity{0.000000}%
\pgfsetdash{}{0pt}%
\pgfpathmoveto{\pgfqpoint{1.000000in}{0.979904in}}%
\pgfpathlineto{\pgfqpoint{7.200000in}{0.979904in}}%
\pgfpathlineto{\pgfqpoint{7.200000in}{6.940096in}}%
\pgfpathlineto{\pgfqpoint{1.000000in}{6.940096in}}%
\pgfpathlineto{\pgfqpoint{1.000000in}{0.979904in}}%
\pgfpathclose%
\pgfusepath{fill}%
\end{pgfscope}%
\begin{pgfscope}%
\pgfpathrectangle{\pgfqpoint{1.000000in}{0.979904in}}{\pgfqpoint{6.200000in}{5.960192in}}%
\pgfusepath{clip}%
\pgfsetbuttcap%
\pgfsetroundjoin%
\definecolor{currentfill}{rgb}{0.200000,0.200000,0.800000}%
\pgfsetfillcolor{currentfill}%
\pgfsetlinewidth{1.003750pt}%
\definecolor{currentstroke}{rgb}{0.200000,0.200000,0.800000}%
\pgfsetstrokecolor{currentstroke}%
\pgfsetdash{}{0pt}%
\pgfpathmoveto{\pgfqpoint{4.656257in}{5.676814in}}%
\pgfpathcurveto{\pgfqpoint{4.662081in}{5.676814in}}{\pgfqpoint{4.667667in}{5.679128in}}{\pgfqpoint{4.671785in}{5.683246in}}%
\pgfpathcurveto{\pgfqpoint{4.675903in}{5.687364in}}{\pgfqpoint{4.678217in}{5.692950in}}{\pgfqpoint{4.678217in}{5.698774in}}%
\pgfpathcurveto{\pgfqpoint{4.678217in}{5.704598in}}{\pgfqpoint{4.675903in}{5.710184in}}{\pgfqpoint{4.671785in}{5.714303in}}%
\pgfpathcurveto{\pgfqpoint{4.667667in}{5.718421in}}{\pgfqpoint{4.662081in}{5.720735in}}{\pgfqpoint{4.656257in}{5.720735in}}%
\pgfpathcurveto{\pgfqpoint{4.650433in}{5.720735in}}{\pgfqpoint{4.644847in}{5.718421in}}{\pgfqpoint{4.640729in}{5.714303in}}%
\pgfpathcurveto{\pgfqpoint{4.636610in}{5.710184in}}{\pgfqpoint{4.634297in}{5.704598in}}{\pgfqpoint{4.634297in}{5.698774in}}%
\pgfpathcurveto{\pgfqpoint{4.634297in}{5.692950in}}{\pgfqpoint{4.636610in}{5.687364in}}{\pgfqpoint{4.640729in}{5.683246in}}%
\pgfpathcurveto{\pgfqpoint{4.644847in}{5.679128in}}{\pgfqpoint{4.650433in}{5.676814in}}{\pgfqpoint{4.656257in}{5.676814in}}%
\pgfpathlineto{\pgfqpoint{4.656257in}{5.676814in}}%
\pgfpathclose%
\pgfusepath{stroke,fill}%
\end{pgfscope}%
\begin{pgfscope}%
\pgfpathrectangle{\pgfqpoint{1.000000in}{0.979904in}}{\pgfqpoint{6.200000in}{5.960192in}}%
\pgfusepath{clip}%
\pgfsetbuttcap%
\pgfsetroundjoin%
\definecolor{currentfill}{rgb}{0.200000,0.200000,0.800000}%
\pgfsetfillcolor{currentfill}%
\pgfsetlinewidth{1.003750pt}%
\definecolor{currentstroke}{rgb}{0.200000,0.200000,0.800000}%
\pgfsetstrokecolor{currentstroke}%
\pgfsetdash{}{0pt}%
\pgfpathmoveto{\pgfqpoint{4.693075in}{5.739556in}}%
\pgfpathcurveto{\pgfqpoint{4.698899in}{5.739556in}}{\pgfqpoint{4.704485in}{5.741870in}}{\pgfqpoint{4.708603in}{5.745988in}}%
\pgfpathcurveto{\pgfqpoint{4.712721in}{5.750107in}}{\pgfqpoint{4.715035in}{5.755693in}}{\pgfqpoint{4.715035in}{5.761517in}}%
\pgfpathcurveto{\pgfqpoint{4.715035in}{5.767341in}}{\pgfqpoint{4.712721in}{5.772927in}}{\pgfqpoint{4.708603in}{5.777045in}}%
\pgfpathcurveto{\pgfqpoint{4.704485in}{5.781163in}}{\pgfqpoint{4.698899in}{5.783477in}}{\pgfqpoint{4.693075in}{5.783477in}}%
\pgfpathcurveto{\pgfqpoint{4.687251in}{5.783477in}}{\pgfqpoint{4.681665in}{5.781163in}}{\pgfqpoint{4.677546in}{5.777045in}}%
\pgfpathcurveto{\pgfqpoint{4.673428in}{5.772927in}}{\pgfqpoint{4.671114in}{5.767341in}}{\pgfqpoint{4.671114in}{5.761517in}}%
\pgfpathcurveto{\pgfqpoint{4.671114in}{5.755693in}}{\pgfqpoint{4.673428in}{5.750107in}}{\pgfqpoint{4.677546in}{5.745988in}}%
\pgfpathcurveto{\pgfqpoint{4.681665in}{5.741870in}}{\pgfqpoint{4.687251in}{5.739556in}}{\pgfqpoint{4.693075in}{5.739556in}}%
\pgfpathlineto{\pgfqpoint{4.693075in}{5.739556in}}%
\pgfpathclose%
\pgfusepath{stroke,fill}%
\end{pgfscope}%
\begin{pgfscope}%
\pgfpathrectangle{\pgfqpoint{1.000000in}{0.979904in}}{\pgfqpoint{6.200000in}{5.960192in}}%
\pgfusepath{clip}%
\pgfsetbuttcap%
\pgfsetroundjoin%
\definecolor{currentfill}{rgb}{0.200000,0.200000,0.800000}%
\pgfsetfillcolor{currentfill}%
\pgfsetlinewidth{1.003750pt}%
\definecolor{currentstroke}{rgb}{0.200000,0.200000,0.800000}%
\pgfsetstrokecolor{currentstroke}%
\pgfsetdash{}{0pt}%
\pgfpathmoveto{\pgfqpoint{4.603353in}{5.791357in}}%
\pgfpathcurveto{\pgfqpoint{4.609177in}{5.791357in}}{\pgfqpoint{4.614763in}{5.793671in}}{\pgfqpoint{4.618881in}{5.797789in}}%
\pgfpathcurveto{\pgfqpoint{4.622999in}{5.801908in}}{\pgfqpoint{4.625313in}{5.807494in}}{\pgfqpoint{4.625313in}{5.813318in}}%
\pgfpathcurveto{\pgfqpoint{4.625313in}{5.819142in}}{\pgfqpoint{4.622999in}{5.824728in}}{\pgfqpoint{4.618881in}{5.828846in}}%
\pgfpathcurveto{\pgfqpoint{4.614763in}{5.832964in}}{\pgfqpoint{4.609177in}{5.835278in}}{\pgfqpoint{4.603353in}{5.835278in}}%
\pgfpathcurveto{\pgfqpoint{4.597529in}{5.835278in}}{\pgfqpoint{4.591943in}{5.832964in}}{\pgfqpoint{4.587825in}{5.828846in}}%
\pgfpathcurveto{\pgfqpoint{4.583707in}{5.824728in}}{\pgfqpoint{4.581393in}{5.819142in}}{\pgfqpoint{4.581393in}{5.813318in}}%
\pgfpathcurveto{\pgfqpoint{4.581393in}{5.807494in}}{\pgfqpoint{4.583707in}{5.801908in}}{\pgfqpoint{4.587825in}{5.797789in}}%
\pgfpathcurveto{\pgfqpoint{4.591943in}{5.793671in}}{\pgfqpoint{4.597529in}{5.791357in}}{\pgfqpoint{4.603353in}{5.791357in}}%
\pgfpathlineto{\pgfqpoint{4.603353in}{5.791357in}}%
\pgfpathclose%
\pgfusepath{stroke,fill}%
\end{pgfscope}%
\begin{pgfscope}%
\pgfpathrectangle{\pgfqpoint{1.000000in}{0.979904in}}{\pgfqpoint{6.200000in}{5.960192in}}%
\pgfusepath{clip}%
\pgfsetbuttcap%
\pgfsetroundjoin%
\definecolor{currentfill}{rgb}{0.200000,0.200000,0.800000}%
\pgfsetfillcolor{currentfill}%
\pgfsetlinewidth{1.003750pt}%
\definecolor{currentstroke}{rgb}{0.200000,0.200000,0.800000}%
\pgfsetstrokecolor{currentstroke}%
\pgfsetdash{}{0pt}%
\pgfpathmoveto{\pgfqpoint{4.597183in}{5.848612in}}%
\pgfpathcurveto{\pgfqpoint{4.603007in}{5.848612in}}{\pgfqpoint{4.608594in}{5.850926in}}{\pgfqpoint{4.612712in}{5.855044in}}%
\pgfpathcurveto{\pgfqpoint{4.616830in}{5.859162in}}{\pgfqpoint{4.619144in}{5.864748in}}{\pgfqpoint{4.619144in}{5.870572in}}%
\pgfpathcurveto{\pgfqpoint{4.619144in}{5.876396in}}{\pgfqpoint{4.616830in}{5.881982in}}{\pgfqpoint{4.612712in}{5.886100in}}%
\pgfpathcurveto{\pgfqpoint{4.608594in}{5.890218in}}{\pgfqpoint{4.603007in}{5.892532in}}{\pgfqpoint{4.597183in}{5.892532in}}%
\pgfpathcurveto{\pgfqpoint{4.591359in}{5.892532in}}{\pgfqpoint{4.585773in}{5.890218in}}{\pgfqpoint{4.581655in}{5.886100in}}%
\pgfpathcurveto{\pgfqpoint{4.577537in}{5.881982in}}{\pgfqpoint{4.575223in}{5.876396in}}{\pgfqpoint{4.575223in}{5.870572in}}%
\pgfpathcurveto{\pgfqpoint{4.575223in}{5.864748in}}{\pgfqpoint{4.577537in}{5.859162in}}{\pgfqpoint{4.581655in}{5.855044in}}%
\pgfpathcurveto{\pgfqpoint{4.585773in}{5.850926in}}{\pgfqpoint{4.591359in}{5.848612in}}{\pgfqpoint{4.597183in}{5.848612in}}%
\pgfpathlineto{\pgfqpoint{4.597183in}{5.848612in}}%
\pgfpathclose%
\pgfusepath{stroke,fill}%
\end{pgfscope}%
\begin{pgfscope}%
\pgfpathrectangle{\pgfqpoint{1.000000in}{0.979904in}}{\pgfqpoint{6.200000in}{5.960192in}}%
\pgfusepath{clip}%
\pgfsetbuttcap%
\pgfsetroundjoin%
\definecolor{currentfill}{rgb}{0.200000,0.200000,0.800000}%
\pgfsetfillcolor{currentfill}%
\pgfsetlinewidth{1.003750pt}%
\definecolor{currentstroke}{rgb}{0.200000,0.200000,0.800000}%
\pgfsetstrokecolor{currentstroke}%
\pgfsetdash{}{0pt}%
\pgfpathmoveto{\pgfqpoint{4.582120in}{5.904184in}}%
\pgfpathcurveto{\pgfqpoint{4.587943in}{5.904184in}}{\pgfqpoint{4.593530in}{5.906498in}}{\pgfqpoint{4.597648in}{5.910616in}}%
\pgfpathcurveto{\pgfqpoint{4.601766in}{5.914735in}}{\pgfqpoint{4.604080in}{5.920321in}}{\pgfqpoint{4.604080in}{5.926145in}}%
\pgfpathcurveto{\pgfqpoint{4.604080in}{5.931969in}}{\pgfqpoint{4.601766in}{5.937555in}}{\pgfqpoint{4.597648in}{5.941673in}}%
\pgfpathcurveto{\pgfqpoint{4.593530in}{5.945791in}}{\pgfqpoint{4.587943in}{5.948105in}}{\pgfqpoint{4.582120in}{5.948105in}}%
\pgfpathcurveto{\pgfqpoint{4.576296in}{5.948105in}}{\pgfqpoint{4.570709in}{5.945791in}}{\pgfqpoint{4.566591in}{5.941673in}}%
\pgfpathcurveto{\pgfqpoint{4.562473in}{5.937555in}}{\pgfqpoint{4.560159in}{5.931969in}}{\pgfqpoint{4.560159in}{5.926145in}}%
\pgfpathcurveto{\pgfqpoint{4.560159in}{5.920321in}}{\pgfqpoint{4.562473in}{5.914735in}}{\pgfqpoint{4.566591in}{5.910616in}}%
\pgfpathcurveto{\pgfqpoint{4.570709in}{5.906498in}}{\pgfqpoint{4.576296in}{5.904184in}}{\pgfqpoint{4.582120in}{5.904184in}}%
\pgfpathlineto{\pgfqpoint{4.582120in}{5.904184in}}%
\pgfpathclose%
\pgfusepath{stroke,fill}%
\end{pgfscope}%
\begin{pgfscope}%
\pgfpathrectangle{\pgfqpoint{1.000000in}{0.979904in}}{\pgfqpoint{6.200000in}{5.960192in}}%
\pgfusepath{clip}%
\pgfsetbuttcap%
\pgfsetroundjoin%
\definecolor{currentfill}{rgb}{0.200000,0.200000,0.800000}%
\pgfsetfillcolor{currentfill}%
\pgfsetlinewidth{1.003750pt}%
\definecolor{currentstroke}{rgb}{0.200000,0.200000,0.800000}%
\pgfsetstrokecolor{currentstroke}%
\pgfsetdash{}{0pt}%
\pgfpathmoveto{\pgfqpoint{4.548156in}{5.953466in}}%
\pgfpathcurveto{\pgfqpoint{4.553980in}{5.953466in}}{\pgfqpoint{4.559566in}{5.955780in}}{\pgfqpoint{4.563684in}{5.959898in}}%
\pgfpathcurveto{\pgfqpoint{4.567802in}{5.964016in}}{\pgfqpoint{4.570116in}{5.969603in}}{\pgfqpoint{4.570116in}{5.975427in}}%
\pgfpathcurveto{\pgfqpoint{4.570116in}{5.981251in}}{\pgfqpoint{4.567802in}{5.986837in}}{\pgfqpoint{4.563684in}{5.990955in}}%
\pgfpathcurveto{\pgfqpoint{4.559566in}{5.995073in}}{\pgfqpoint{4.553980in}{5.997387in}}{\pgfqpoint{4.548156in}{5.997387in}}%
\pgfpathcurveto{\pgfqpoint{4.542332in}{5.997387in}}{\pgfqpoint{4.536746in}{5.995073in}}{\pgfqpoint{4.532627in}{5.990955in}}%
\pgfpathcurveto{\pgfqpoint{4.528509in}{5.986837in}}{\pgfqpoint{4.526195in}{5.981251in}}{\pgfqpoint{4.526195in}{5.975427in}}%
\pgfpathcurveto{\pgfqpoint{4.526195in}{5.969603in}}{\pgfqpoint{4.528509in}{5.964016in}}{\pgfqpoint{4.532627in}{5.959898in}}%
\pgfpathcurveto{\pgfqpoint{4.536746in}{5.955780in}}{\pgfqpoint{4.542332in}{5.953466in}}{\pgfqpoint{4.548156in}{5.953466in}}%
\pgfpathlineto{\pgfqpoint{4.548156in}{5.953466in}}%
\pgfpathclose%
\pgfusepath{stroke,fill}%
\end{pgfscope}%
\begin{pgfscope}%
\pgfpathrectangle{\pgfqpoint{1.000000in}{0.979904in}}{\pgfqpoint{6.200000in}{5.960192in}}%
\pgfusepath{clip}%
\pgfsetbuttcap%
\pgfsetroundjoin%
\definecolor{currentfill}{rgb}{0.200000,0.200000,0.800000}%
\pgfsetfillcolor{currentfill}%
\pgfsetlinewidth{1.003750pt}%
\definecolor{currentstroke}{rgb}{0.200000,0.200000,0.800000}%
\pgfsetstrokecolor{currentstroke}%
\pgfsetdash{}{0pt}%
\pgfpathmoveto{\pgfqpoint{4.445746in}{5.973039in}}%
\pgfpathcurveto{\pgfqpoint{4.451570in}{5.973039in}}{\pgfqpoint{4.457156in}{5.975353in}}{\pgfqpoint{4.461274in}{5.979471in}}%
\pgfpathcurveto{\pgfqpoint{4.465392in}{5.983589in}}{\pgfqpoint{4.467706in}{5.989175in}}{\pgfqpoint{4.467706in}{5.994999in}}%
\pgfpathcurveto{\pgfqpoint{4.467706in}{6.000823in}}{\pgfqpoint{4.465392in}{6.006409in}}{\pgfqpoint{4.461274in}{6.010527in}}%
\pgfpathcurveto{\pgfqpoint{4.457156in}{6.014646in}}{\pgfqpoint{4.451570in}{6.016959in}}{\pgfqpoint{4.445746in}{6.016959in}}%
\pgfpathcurveto{\pgfqpoint{4.439922in}{6.016959in}}{\pgfqpoint{4.434336in}{6.014646in}}{\pgfqpoint{4.430218in}{6.010527in}}%
\pgfpathcurveto{\pgfqpoint{4.426100in}{6.006409in}}{\pgfqpoint{4.423786in}{6.000823in}}{\pgfqpoint{4.423786in}{5.994999in}}%
\pgfpathcurveto{\pgfqpoint{4.423786in}{5.989175in}}{\pgfqpoint{4.426100in}{5.983589in}}{\pgfqpoint{4.430218in}{5.979471in}}%
\pgfpathcurveto{\pgfqpoint{4.434336in}{5.975353in}}{\pgfqpoint{4.439922in}{5.973039in}}{\pgfqpoint{4.445746in}{5.973039in}}%
\pgfpathlineto{\pgfqpoint{4.445746in}{5.973039in}}%
\pgfpathclose%
\pgfusepath{stroke,fill}%
\end{pgfscope}%
\begin{pgfscope}%
\pgfpathrectangle{\pgfqpoint{1.000000in}{0.979904in}}{\pgfqpoint{6.200000in}{5.960192in}}%
\pgfusepath{clip}%
\pgfsetbuttcap%
\pgfsetroundjoin%
\definecolor{currentfill}{rgb}{0.200000,0.200000,0.800000}%
\pgfsetfillcolor{currentfill}%
\pgfsetlinewidth{1.003750pt}%
\definecolor{currentstroke}{rgb}{0.200000,0.200000,0.800000}%
\pgfsetstrokecolor{currentstroke}%
\pgfsetdash{}{0pt}%
\pgfpathmoveto{\pgfqpoint{4.596155in}{6.100619in}}%
\pgfpathcurveto{\pgfqpoint{4.601979in}{6.100619in}}{\pgfqpoint{4.607565in}{6.102933in}}{\pgfqpoint{4.611683in}{6.107051in}}%
\pgfpathcurveto{\pgfqpoint{4.615801in}{6.111169in}}{\pgfqpoint{4.618115in}{6.116755in}}{\pgfqpoint{4.618115in}{6.122579in}}%
\pgfpathcurveto{\pgfqpoint{4.618115in}{6.128403in}}{\pgfqpoint{4.615801in}{6.133989in}}{\pgfqpoint{4.611683in}{6.138108in}}%
\pgfpathcurveto{\pgfqpoint{4.607565in}{6.142226in}}{\pgfqpoint{4.601979in}{6.144540in}}{\pgfqpoint{4.596155in}{6.144540in}}%
\pgfpathcurveto{\pgfqpoint{4.590331in}{6.144540in}}{\pgfqpoint{4.584745in}{6.142226in}}{\pgfqpoint{4.580626in}{6.138108in}}%
\pgfpathcurveto{\pgfqpoint{4.576508in}{6.133989in}}{\pgfqpoint{4.574194in}{6.128403in}}{\pgfqpoint{4.574194in}{6.122579in}}%
\pgfpathcurveto{\pgfqpoint{4.574194in}{6.116755in}}{\pgfqpoint{4.576508in}{6.111169in}}{\pgfqpoint{4.580626in}{6.107051in}}%
\pgfpathcurveto{\pgfqpoint{4.584745in}{6.102933in}}{\pgfqpoint{4.590331in}{6.100619in}}{\pgfqpoint{4.596155in}{6.100619in}}%
\pgfpathlineto{\pgfqpoint{4.596155in}{6.100619in}}%
\pgfpathclose%
\pgfusepath{stroke,fill}%
\end{pgfscope}%
\begin{pgfscope}%
\pgfpathrectangle{\pgfqpoint{1.000000in}{0.979904in}}{\pgfqpoint{6.200000in}{5.960192in}}%
\pgfusepath{clip}%
\pgfsetbuttcap%
\pgfsetroundjoin%
\definecolor{currentfill}{rgb}{0.200000,0.800000,0.200000}%
\pgfsetfillcolor{currentfill}%
\pgfsetlinewidth{1.003750pt}%
\definecolor{currentstroke}{rgb}{0.200000,0.800000,0.200000}%
\pgfsetstrokecolor{currentstroke}%
\pgfsetdash{}{0pt}%
\pgfpathmoveto{\pgfqpoint{4.561274in}{6.152780in}}%
\pgfpathcurveto{\pgfqpoint{4.567098in}{6.152780in}}{\pgfqpoint{4.572684in}{6.155094in}}{\pgfqpoint{4.576802in}{6.159212in}}%
\pgfpathcurveto{\pgfqpoint{4.580920in}{6.163330in}}{\pgfqpoint{4.583234in}{6.168916in}}{\pgfqpoint{4.583234in}{6.174740in}}%
\pgfpathcurveto{\pgfqpoint{4.583234in}{6.180564in}}{\pgfqpoint{4.580920in}{6.186150in}}{\pgfqpoint{4.576802in}{6.190269in}}%
\pgfpathcurveto{\pgfqpoint{4.572684in}{6.194387in}}{\pgfqpoint{4.567098in}{6.196701in}}{\pgfqpoint{4.561274in}{6.196701in}}%
\pgfpathcurveto{\pgfqpoint{4.555450in}{6.196701in}}{\pgfqpoint{4.549864in}{6.194387in}}{\pgfqpoint{4.545745in}{6.190269in}}%
\pgfpathcurveto{\pgfqpoint{4.541627in}{6.186150in}}{\pgfqpoint{4.539313in}{6.180564in}}{\pgfqpoint{4.539313in}{6.174740in}}%
\pgfpathcurveto{\pgfqpoint{4.539313in}{6.168916in}}{\pgfqpoint{4.541627in}{6.163330in}}{\pgfqpoint{4.545745in}{6.159212in}}%
\pgfpathcurveto{\pgfqpoint{4.549864in}{6.155094in}}{\pgfqpoint{4.555450in}{6.152780in}}{\pgfqpoint{4.561274in}{6.152780in}}%
\pgfpathlineto{\pgfqpoint{4.561274in}{6.152780in}}%
\pgfpathclose%
\pgfusepath{stroke,fill}%
\end{pgfscope}%
\begin{pgfscope}%
\pgfpathrectangle{\pgfqpoint{1.000000in}{0.979904in}}{\pgfqpoint{6.200000in}{5.960192in}}%
\pgfusepath{clip}%
\pgfsetbuttcap%
\pgfsetroundjoin%
\definecolor{currentfill}{rgb}{0.200000,0.800000,0.200000}%
\pgfsetfillcolor{currentfill}%
\pgfsetlinewidth{1.003750pt}%
\definecolor{currentstroke}{rgb}{0.200000,0.800000,0.200000}%
\pgfsetstrokecolor{currentstroke}%
\pgfsetdash{}{0pt}%
\pgfpathmoveto{\pgfqpoint{4.424906in}{6.138947in}}%
\pgfpathcurveto{\pgfqpoint{4.430729in}{6.138947in}}{\pgfqpoint{4.436316in}{6.141261in}}{\pgfqpoint{4.440434in}{6.145379in}}%
\pgfpathcurveto{\pgfqpoint{4.444552in}{6.149497in}}{\pgfqpoint{4.446866in}{6.155084in}}{\pgfqpoint{4.446866in}{6.160908in}}%
\pgfpathcurveto{\pgfqpoint{4.446866in}{6.166732in}}{\pgfqpoint{4.444552in}{6.172318in}}{\pgfqpoint{4.440434in}{6.176436in}}%
\pgfpathcurveto{\pgfqpoint{4.436316in}{6.180554in}}{\pgfqpoint{4.430729in}{6.182868in}}{\pgfqpoint{4.424906in}{6.182868in}}%
\pgfpathcurveto{\pgfqpoint{4.419082in}{6.182868in}}{\pgfqpoint{4.413495in}{6.180554in}}{\pgfqpoint{4.409377in}{6.176436in}}%
\pgfpathcurveto{\pgfqpoint{4.405259in}{6.172318in}}{\pgfqpoint{4.402945in}{6.166732in}}{\pgfqpoint{4.402945in}{6.160908in}}%
\pgfpathcurveto{\pgfqpoint{4.402945in}{6.155084in}}{\pgfqpoint{4.405259in}{6.149497in}}{\pgfqpoint{4.409377in}{6.145379in}}%
\pgfpathcurveto{\pgfqpoint{4.413495in}{6.141261in}}{\pgfqpoint{4.419082in}{6.138947in}}{\pgfqpoint{4.424906in}{6.138947in}}%
\pgfpathlineto{\pgfqpoint{4.424906in}{6.138947in}}%
\pgfpathclose%
\pgfusepath{stroke,fill}%
\end{pgfscope}%
\begin{pgfscope}%
\pgfpathrectangle{\pgfqpoint{1.000000in}{0.979904in}}{\pgfqpoint{6.200000in}{5.960192in}}%
\pgfusepath{clip}%
\pgfsetbuttcap%
\pgfsetroundjoin%
\definecolor{currentfill}{rgb}{0.200000,0.200000,0.800000}%
\pgfsetfillcolor{currentfill}%
\pgfsetlinewidth{1.003750pt}%
\definecolor{currentstroke}{rgb}{0.200000,0.200000,0.800000}%
\pgfsetstrokecolor{currentstroke}%
\pgfsetdash{}{0pt}%
\pgfpathmoveto{\pgfqpoint{4.440967in}{6.218097in}}%
\pgfpathcurveto{\pgfqpoint{4.446791in}{6.218097in}}{\pgfqpoint{4.452377in}{6.220411in}}{\pgfqpoint{4.456495in}{6.224529in}}%
\pgfpathcurveto{\pgfqpoint{4.460613in}{6.228647in}}{\pgfqpoint{4.462927in}{6.234233in}}{\pgfqpoint{4.462927in}{6.240057in}}%
\pgfpathcurveto{\pgfqpoint{4.462927in}{6.245881in}}{\pgfqpoint{4.460613in}{6.251467in}}{\pgfqpoint{4.456495in}{6.255585in}}%
\pgfpathcurveto{\pgfqpoint{4.452377in}{6.259704in}}{\pgfqpoint{4.446791in}{6.262017in}}{\pgfqpoint{4.440967in}{6.262017in}}%
\pgfpathcurveto{\pgfqpoint{4.435143in}{6.262017in}}{\pgfqpoint{4.429557in}{6.259704in}}{\pgfqpoint{4.425438in}{6.255585in}}%
\pgfpathcurveto{\pgfqpoint{4.421320in}{6.251467in}}{\pgfqpoint{4.419006in}{6.245881in}}{\pgfqpoint{4.419006in}{6.240057in}}%
\pgfpathcurveto{\pgfqpoint{4.419006in}{6.234233in}}{\pgfqpoint{4.421320in}{6.228647in}}{\pgfqpoint{4.425438in}{6.224529in}}%
\pgfpathcurveto{\pgfqpoint{4.429557in}{6.220411in}}{\pgfqpoint{4.435143in}{6.218097in}}{\pgfqpoint{4.440967in}{6.218097in}}%
\pgfpathlineto{\pgfqpoint{4.440967in}{6.218097in}}%
\pgfpathclose%
\pgfusepath{stroke,fill}%
\end{pgfscope}%
\begin{pgfscope}%
\pgfpathrectangle{\pgfqpoint{1.000000in}{0.979904in}}{\pgfqpoint{6.200000in}{5.960192in}}%
\pgfusepath{clip}%
\pgfsetbuttcap%
\pgfsetroundjoin%
\definecolor{currentfill}{rgb}{0.200000,0.200000,0.800000}%
\pgfsetfillcolor{currentfill}%
\pgfsetlinewidth{1.003750pt}%
\definecolor{currentstroke}{rgb}{0.200000,0.200000,0.800000}%
\pgfsetstrokecolor{currentstroke}%
\pgfsetdash{}{0pt}%
\pgfpathmoveto{\pgfqpoint{4.392380in}{6.252913in}}%
\pgfpathcurveto{\pgfqpoint{4.398203in}{6.252913in}}{\pgfqpoint{4.403790in}{6.255226in}}{\pgfqpoint{4.407908in}{6.259345in}}%
\pgfpathcurveto{\pgfqpoint{4.412026in}{6.263463in}}{\pgfqpoint{4.414340in}{6.269049in}}{\pgfqpoint{4.414340in}{6.274873in}}%
\pgfpathcurveto{\pgfqpoint{4.414340in}{6.280697in}}{\pgfqpoint{4.412026in}{6.286283in}}{\pgfqpoint{4.407908in}{6.290401in}}%
\pgfpathcurveto{\pgfqpoint{4.403790in}{6.294519in}}{\pgfqpoint{4.398203in}{6.296833in}}{\pgfqpoint{4.392380in}{6.296833in}}%
\pgfpathcurveto{\pgfqpoint{4.386556in}{6.296833in}}{\pgfqpoint{4.380969in}{6.294519in}}{\pgfqpoint{4.376851in}{6.290401in}}%
\pgfpathcurveto{\pgfqpoint{4.372733in}{6.286283in}}{\pgfqpoint{4.370419in}{6.280697in}}{\pgfqpoint{4.370419in}{6.274873in}}%
\pgfpathcurveto{\pgfqpoint{4.370419in}{6.269049in}}{\pgfqpoint{4.372733in}{6.263463in}}{\pgfqpoint{4.376851in}{6.259345in}}%
\pgfpathcurveto{\pgfqpoint{4.380969in}{6.255226in}}{\pgfqpoint{4.386556in}{6.252913in}}{\pgfqpoint{4.392380in}{6.252913in}}%
\pgfpathlineto{\pgfqpoint{4.392380in}{6.252913in}}%
\pgfpathclose%
\pgfusepath{stroke,fill}%
\end{pgfscope}%
\begin{pgfscope}%
\pgfpathrectangle{\pgfqpoint{1.000000in}{0.979904in}}{\pgfqpoint{6.200000in}{5.960192in}}%
\pgfusepath{clip}%
\pgfsetbuttcap%
\pgfsetroundjoin%
\definecolor{currentfill}{rgb}{0.200000,0.200000,0.800000}%
\pgfsetfillcolor{currentfill}%
\pgfsetlinewidth{1.003750pt}%
\definecolor{currentstroke}{rgb}{0.200000,0.200000,0.800000}%
\pgfsetstrokecolor{currentstroke}%
\pgfsetdash{}{0pt}%
\pgfpathmoveto{\pgfqpoint{4.355959in}{6.296728in}}%
\pgfpathcurveto{\pgfqpoint{4.361783in}{6.296728in}}{\pgfqpoint{4.367369in}{6.299042in}}{\pgfqpoint{4.371487in}{6.303160in}}%
\pgfpathcurveto{\pgfqpoint{4.375605in}{6.307278in}}{\pgfqpoint{4.377919in}{6.312864in}}{\pgfqpoint{4.377919in}{6.318688in}}%
\pgfpathcurveto{\pgfqpoint{4.377919in}{6.324512in}}{\pgfqpoint{4.375605in}{6.330098in}}{\pgfqpoint{4.371487in}{6.334216in}}%
\pgfpathcurveto{\pgfqpoint{4.367369in}{6.338334in}}{\pgfqpoint{4.361783in}{6.340648in}}{\pgfqpoint{4.355959in}{6.340648in}}%
\pgfpathcurveto{\pgfqpoint{4.350135in}{6.340648in}}{\pgfqpoint{4.344549in}{6.338334in}}{\pgfqpoint{4.340431in}{6.334216in}}%
\pgfpathcurveto{\pgfqpoint{4.336312in}{6.330098in}}{\pgfqpoint{4.333999in}{6.324512in}}{\pgfqpoint{4.333999in}{6.318688in}}%
\pgfpathcurveto{\pgfqpoint{4.333999in}{6.312864in}}{\pgfqpoint{4.336312in}{6.307278in}}{\pgfqpoint{4.340431in}{6.303160in}}%
\pgfpathcurveto{\pgfqpoint{4.344549in}{6.299042in}}{\pgfqpoint{4.350135in}{6.296728in}}{\pgfqpoint{4.355959in}{6.296728in}}%
\pgfpathlineto{\pgfqpoint{4.355959in}{6.296728in}}%
\pgfpathclose%
\pgfusepath{stroke,fill}%
\end{pgfscope}%
\begin{pgfscope}%
\pgfpathrectangle{\pgfqpoint{1.000000in}{0.979904in}}{\pgfqpoint{6.200000in}{5.960192in}}%
\pgfusepath{clip}%
\pgfsetbuttcap%
\pgfsetroundjoin%
\definecolor{currentfill}{rgb}{0.200000,0.200000,0.800000}%
\pgfsetfillcolor{currentfill}%
\pgfsetlinewidth{1.003750pt}%
\definecolor{currentstroke}{rgb}{0.200000,0.200000,0.800000}%
\pgfsetstrokecolor{currentstroke}%
\pgfsetdash{}{0pt}%
\pgfpathmoveto{\pgfqpoint{4.263627in}{6.280736in}}%
\pgfpathcurveto{\pgfqpoint{4.269451in}{6.280736in}}{\pgfqpoint{4.275037in}{6.283049in}}{\pgfqpoint{4.279156in}{6.287168in}}%
\pgfpathcurveto{\pgfqpoint{4.283274in}{6.291286in}}{\pgfqpoint{4.285588in}{6.296872in}}{\pgfqpoint{4.285588in}{6.302696in}}%
\pgfpathcurveto{\pgfqpoint{4.285588in}{6.308520in}}{\pgfqpoint{4.283274in}{6.314106in}}{\pgfqpoint{4.279156in}{6.318224in}}%
\pgfpathcurveto{\pgfqpoint{4.275037in}{6.322342in}}{\pgfqpoint{4.269451in}{6.324656in}}{\pgfqpoint{4.263627in}{6.324656in}}%
\pgfpathcurveto{\pgfqpoint{4.257803in}{6.324656in}}{\pgfqpoint{4.252217in}{6.322342in}}{\pgfqpoint{4.248099in}{6.318224in}}%
\pgfpathcurveto{\pgfqpoint{4.243981in}{6.314106in}}{\pgfqpoint{4.241667in}{6.308520in}}{\pgfqpoint{4.241667in}{6.302696in}}%
\pgfpathcurveto{\pgfqpoint{4.241667in}{6.296872in}}{\pgfqpoint{4.243981in}{6.291286in}}{\pgfqpoint{4.248099in}{6.287168in}}%
\pgfpathcurveto{\pgfqpoint{4.252217in}{6.283049in}}{\pgfqpoint{4.257803in}{6.280736in}}{\pgfqpoint{4.263627in}{6.280736in}}%
\pgfpathlineto{\pgfqpoint{4.263627in}{6.280736in}}%
\pgfpathclose%
\pgfusepath{stroke,fill}%
\end{pgfscope}%
\begin{pgfscope}%
\pgfpathrectangle{\pgfqpoint{1.000000in}{0.979904in}}{\pgfqpoint{6.200000in}{5.960192in}}%
\pgfusepath{clip}%
\pgfsetbuttcap%
\pgfsetroundjoin%
\definecolor{currentfill}{rgb}{0.200000,0.200000,0.800000}%
\pgfsetfillcolor{currentfill}%
\pgfsetlinewidth{1.003750pt}%
\definecolor{currentstroke}{rgb}{0.200000,0.200000,0.800000}%
\pgfsetstrokecolor{currentstroke}%
\pgfsetdash{}{0pt}%
\pgfpathmoveto{\pgfqpoint{4.274723in}{6.377086in}}%
\pgfpathcurveto{\pgfqpoint{4.280547in}{6.377086in}}{\pgfqpoint{4.286133in}{6.379400in}}{\pgfqpoint{4.290251in}{6.383518in}}%
\pgfpathcurveto{\pgfqpoint{4.294370in}{6.387636in}}{\pgfqpoint{4.296683in}{6.393222in}}{\pgfqpoint{4.296683in}{6.399046in}}%
\pgfpathcurveto{\pgfqpoint{4.296683in}{6.404870in}}{\pgfqpoint{4.294370in}{6.410456in}}{\pgfqpoint{4.290251in}{6.414574in}}%
\pgfpathcurveto{\pgfqpoint{4.286133in}{6.418693in}}{\pgfqpoint{4.280547in}{6.421006in}}{\pgfqpoint{4.274723in}{6.421006in}}%
\pgfpathcurveto{\pgfqpoint{4.268899in}{6.421006in}}{\pgfqpoint{4.263313in}{6.418693in}}{\pgfqpoint{4.259195in}{6.414574in}}%
\pgfpathcurveto{\pgfqpoint{4.255077in}{6.410456in}}{\pgfqpoint{4.252763in}{6.404870in}}{\pgfqpoint{4.252763in}{6.399046in}}%
\pgfpathcurveto{\pgfqpoint{4.252763in}{6.393222in}}{\pgfqpoint{4.255077in}{6.387636in}}{\pgfqpoint{4.259195in}{6.383518in}}%
\pgfpathcurveto{\pgfqpoint{4.263313in}{6.379400in}}{\pgfqpoint{4.268899in}{6.377086in}}{\pgfqpoint{4.274723in}{6.377086in}}%
\pgfpathlineto{\pgfqpoint{4.274723in}{6.377086in}}%
\pgfpathclose%
\pgfusepath{stroke,fill}%
\end{pgfscope}%
\begin{pgfscope}%
\pgfpathrectangle{\pgfqpoint{1.000000in}{0.979904in}}{\pgfqpoint{6.200000in}{5.960192in}}%
\pgfusepath{clip}%
\pgfsetbuttcap%
\pgfsetroundjoin%
\definecolor{currentfill}{rgb}{0.200000,0.200000,0.800000}%
\pgfsetfillcolor{currentfill}%
\pgfsetlinewidth{1.003750pt}%
\definecolor{currentstroke}{rgb}{0.200000,0.200000,0.800000}%
\pgfsetstrokecolor{currentstroke}%
\pgfsetdash{}{0pt}%
\pgfpathmoveto{\pgfqpoint{4.266932in}{6.464796in}}%
\pgfpathcurveto{\pgfqpoint{4.272756in}{6.464796in}}{\pgfqpoint{4.278342in}{6.467110in}}{\pgfqpoint{4.282460in}{6.471228in}}%
\pgfpathcurveto{\pgfqpoint{4.286578in}{6.475347in}}{\pgfqpoint{4.288892in}{6.480933in}}{\pgfqpoint{4.288892in}{6.486757in}}%
\pgfpathcurveto{\pgfqpoint{4.288892in}{6.492581in}}{\pgfqpoint{4.286578in}{6.498167in}}{\pgfqpoint{4.282460in}{6.502285in}}%
\pgfpathcurveto{\pgfqpoint{4.278342in}{6.506403in}}{\pgfqpoint{4.272756in}{6.508717in}}{\pgfqpoint{4.266932in}{6.508717in}}%
\pgfpathcurveto{\pgfqpoint{4.261108in}{6.508717in}}{\pgfqpoint{4.255522in}{6.506403in}}{\pgfqpoint{4.251404in}{6.502285in}}%
\pgfpathcurveto{\pgfqpoint{4.247285in}{6.498167in}}{\pgfqpoint{4.244972in}{6.492581in}}{\pgfqpoint{4.244972in}{6.486757in}}%
\pgfpathcurveto{\pgfqpoint{4.244972in}{6.480933in}}{\pgfqpoint{4.247285in}{6.475347in}}{\pgfqpoint{4.251404in}{6.471228in}}%
\pgfpathcurveto{\pgfqpoint{4.255522in}{6.467110in}}{\pgfqpoint{4.261108in}{6.464796in}}{\pgfqpoint{4.266932in}{6.464796in}}%
\pgfpathlineto{\pgfqpoint{4.266932in}{6.464796in}}%
\pgfpathclose%
\pgfusepath{stroke,fill}%
\end{pgfscope}%
\begin{pgfscope}%
\pgfpathrectangle{\pgfqpoint{1.000000in}{0.979904in}}{\pgfqpoint{6.200000in}{5.960192in}}%
\pgfusepath{clip}%
\pgfsetbuttcap%
\pgfsetroundjoin%
\definecolor{currentfill}{rgb}{0.200000,0.200000,0.800000}%
\pgfsetfillcolor{currentfill}%
\pgfsetlinewidth{1.003750pt}%
\definecolor{currentstroke}{rgb}{0.200000,0.200000,0.800000}%
\pgfsetstrokecolor{currentstroke}%
\pgfsetdash{}{0pt}%
\pgfpathmoveto{\pgfqpoint{4.270701in}{6.587242in}}%
\pgfpathcurveto{\pgfqpoint{4.276525in}{6.587242in}}{\pgfqpoint{4.282111in}{6.589556in}}{\pgfqpoint{4.286230in}{6.593674in}}%
\pgfpathcurveto{\pgfqpoint{4.290348in}{6.597792in}}{\pgfqpoint{4.292662in}{6.603378in}}{\pgfqpoint{4.292662in}{6.609202in}}%
\pgfpathcurveto{\pgfqpoint{4.292662in}{6.615026in}}{\pgfqpoint{4.290348in}{6.620612in}}{\pgfqpoint{4.286230in}{6.624730in}}%
\pgfpathcurveto{\pgfqpoint{4.282111in}{6.628848in}}{\pgfqpoint{4.276525in}{6.631162in}}{\pgfqpoint{4.270701in}{6.631162in}}%
\pgfpathcurveto{\pgfqpoint{4.264877in}{6.631162in}}{\pgfqpoint{4.259291in}{6.628848in}}{\pgfqpoint{4.255173in}{6.624730in}}%
\pgfpathcurveto{\pgfqpoint{4.251055in}{6.620612in}}{\pgfqpoint{4.248741in}{6.615026in}}{\pgfqpoint{4.248741in}{6.609202in}}%
\pgfpathcurveto{\pgfqpoint{4.248741in}{6.603378in}}{\pgfqpoint{4.251055in}{6.597792in}}{\pgfqpoint{4.255173in}{6.593674in}}%
\pgfpathcurveto{\pgfqpoint{4.259291in}{6.589556in}}{\pgfqpoint{4.264877in}{6.587242in}}{\pgfqpoint{4.270701in}{6.587242in}}%
\pgfpathlineto{\pgfqpoint{4.270701in}{6.587242in}}%
\pgfpathclose%
\pgfusepath{stroke,fill}%
\end{pgfscope}%
\begin{pgfscope}%
\pgfpathrectangle{\pgfqpoint{1.000000in}{0.979904in}}{\pgfqpoint{6.200000in}{5.960192in}}%
\pgfusepath{clip}%
\pgfsetbuttcap%
\pgfsetroundjoin%
\definecolor{currentfill}{rgb}{0.200000,0.200000,0.800000}%
\pgfsetfillcolor{currentfill}%
\pgfsetlinewidth{1.003750pt}%
\definecolor{currentstroke}{rgb}{0.200000,0.200000,0.800000}%
\pgfsetstrokecolor{currentstroke}%
\pgfsetdash{}{0pt}%
\pgfpathmoveto{\pgfqpoint{4.157566in}{6.519974in}}%
\pgfpathcurveto{\pgfqpoint{4.163390in}{6.519974in}}{\pgfqpoint{4.168976in}{6.522288in}}{\pgfqpoint{4.173094in}{6.526406in}}%
\pgfpathcurveto{\pgfqpoint{4.177212in}{6.530524in}}{\pgfqpoint{4.179526in}{6.536110in}}{\pgfqpoint{4.179526in}{6.541934in}}%
\pgfpathcurveto{\pgfqpoint{4.179526in}{6.547758in}}{\pgfqpoint{4.177212in}{6.553344in}}{\pgfqpoint{4.173094in}{6.557462in}}%
\pgfpathcurveto{\pgfqpoint{4.168976in}{6.561581in}}{\pgfqpoint{4.163390in}{6.563894in}}{\pgfqpoint{4.157566in}{6.563894in}}%
\pgfpathcurveto{\pgfqpoint{4.151742in}{6.563894in}}{\pgfqpoint{4.146156in}{6.561581in}}{\pgfqpoint{4.142038in}{6.557462in}}%
\pgfpathcurveto{\pgfqpoint{4.137920in}{6.553344in}}{\pgfqpoint{4.135606in}{6.547758in}}{\pgfqpoint{4.135606in}{6.541934in}}%
\pgfpathcurveto{\pgfqpoint{4.135606in}{6.536110in}}{\pgfqpoint{4.137920in}{6.530524in}}{\pgfqpoint{4.142038in}{6.526406in}}%
\pgfpathcurveto{\pgfqpoint{4.146156in}{6.522288in}}{\pgfqpoint{4.151742in}{6.519974in}}{\pgfqpoint{4.157566in}{6.519974in}}%
\pgfpathlineto{\pgfqpoint{4.157566in}{6.519974in}}%
\pgfpathclose%
\pgfusepath{stroke,fill}%
\end{pgfscope}%
\begin{pgfscope}%
\pgfpathrectangle{\pgfqpoint{1.000000in}{0.979904in}}{\pgfqpoint{6.200000in}{5.960192in}}%
\pgfusepath{clip}%
\pgfsetbuttcap%
\pgfsetroundjoin%
\definecolor{currentfill}{rgb}{0.200000,0.200000,0.800000}%
\pgfsetfillcolor{currentfill}%
\pgfsetlinewidth{1.003750pt}%
\definecolor{currentstroke}{rgb}{0.200000,0.200000,0.800000}%
\pgfsetstrokecolor{currentstroke}%
\pgfsetdash{}{0pt}%
\pgfpathmoveto{\pgfqpoint{4.097185in}{6.533800in}}%
\pgfpathcurveto{\pgfqpoint{4.103009in}{6.533800in}}{\pgfqpoint{4.108595in}{6.536114in}}{\pgfqpoint{4.112713in}{6.540232in}}%
\pgfpathcurveto{\pgfqpoint{4.116831in}{6.544351in}}{\pgfqpoint{4.119145in}{6.549937in}}{\pgfqpoint{4.119145in}{6.555761in}}%
\pgfpathcurveto{\pgfqpoint{4.119145in}{6.561585in}}{\pgfqpoint{4.116831in}{6.567171in}}{\pgfqpoint{4.112713in}{6.571289in}}%
\pgfpathcurveto{\pgfqpoint{4.108595in}{6.575407in}}{\pgfqpoint{4.103009in}{6.577721in}}{\pgfqpoint{4.097185in}{6.577721in}}%
\pgfpathcurveto{\pgfqpoint{4.091361in}{6.577721in}}{\pgfqpoint{4.085775in}{6.575407in}}{\pgfqpoint{4.081656in}{6.571289in}}%
\pgfpathcurveto{\pgfqpoint{4.077538in}{6.567171in}}{\pgfqpoint{4.075224in}{6.561585in}}{\pgfqpoint{4.075224in}{6.555761in}}%
\pgfpathcurveto{\pgfqpoint{4.075224in}{6.549937in}}{\pgfqpoint{4.077538in}{6.544351in}}{\pgfqpoint{4.081656in}{6.540232in}}%
\pgfpathcurveto{\pgfqpoint{4.085775in}{6.536114in}}{\pgfqpoint{4.091361in}{6.533800in}}{\pgfqpoint{4.097185in}{6.533800in}}%
\pgfpathlineto{\pgfqpoint{4.097185in}{6.533800in}}%
\pgfpathclose%
\pgfusepath{stroke,fill}%
\end{pgfscope}%
\begin{pgfscope}%
\pgfpathrectangle{\pgfqpoint{1.000000in}{0.979904in}}{\pgfqpoint{6.200000in}{5.960192in}}%
\pgfusepath{clip}%
\pgfsetbuttcap%
\pgfsetroundjoin%
\definecolor{currentfill}{rgb}{0.200000,0.200000,0.800000}%
\pgfsetfillcolor{currentfill}%
\pgfsetlinewidth{1.003750pt}%
\definecolor{currentstroke}{rgb}{0.200000,0.200000,0.800000}%
\pgfsetstrokecolor{currentstroke}%
\pgfsetdash{}{0pt}%
\pgfpathmoveto{\pgfqpoint{4.029969in}{6.525293in}}%
\pgfpathcurveto{\pgfqpoint{4.035793in}{6.525293in}}{\pgfqpoint{4.041379in}{6.527607in}}{\pgfqpoint{4.045497in}{6.531725in}}%
\pgfpathcurveto{\pgfqpoint{4.049616in}{6.535844in}}{\pgfqpoint{4.051929in}{6.541430in}}{\pgfqpoint{4.051929in}{6.547254in}}%
\pgfpathcurveto{\pgfqpoint{4.051929in}{6.553078in}}{\pgfqpoint{4.049616in}{6.558664in}}{\pgfqpoint{4.045497in}{6.562782in}}%
\pgfpathcurveto{\pgfqpoint{4.041379in}{6.566900in}}{\pgfqpoint{4.035793in}{6.569214in}}{\pgfqpoint{4.029969in}{6.569214in}}%
\pgfpathcurveto{\pgfqpoint{4.024145in}{6.569214in}}{\pgfqpoint{4.018559in}{6.566900in}}{\pgfqpoint{4.014441in}{6.562782in}}%
\pgfpathcurveto{\pgfqpoint{4.010323in}{6.558664in}}{\pgfqpoint{4.008009in}{6.553078in}}{\pgfqpoint{4.008009in}{6.547254in}}%
\pgfpathcurveto{\pgfqpoint{4.008009in}{6.541430in}}{\pgfqpoint{4.010323in}{6.535844in}}{\pgfqpoint{4.014441in}{6.531725in}}%
\pgfpathcurveto{\pgfqpoint{4.018559in}{6.527607in}}{\pgfqpoint{4.024145in}{6.525293in}}{\pgfqpoint{4.029969in}{6.525293in}}%
\pgfpathlineto{\pgfqpoint{4.029969in}{6.525293in}}%
\pgfpathclose%
\pgfusepath{stroke,fill}%
\end{pgfscope}%
\begin{pgfscope}%
\pgfpathrectangle{\pgfqpoint{1.000000in}{0.979904in}}{\pgfqpoint{6.200000in}{5.960192in}}%
\pgfusepath{clip}%
\pgfsetbuttcap%
\pgfsetroundjoin%
\definecolor{currentfill}{rgb}{0.200000,0.200000,0.800000}%
\pgfsetfillcolor{currentfill}%
\pgfsetlinewidth{1.003750pt}%
\definecolor{currentstroke}{rgb}{0.200000,0.200000,0.800000}%
\pgfsetstrokecolor{currentstroke}%
\pgfsetdash{}{0pt}%
\pgfpathmoveto{\pgfqpoint{3.974166in}{6.539847in}}%
\pgfpathcurveto{\pgfqpoint{3.979990in}{6.539847in}}{\pgfqpoint{3.985577in}{6.542161in}}{\pgfqpoint{3.989695in}{6.546279in}}%
\pgfpathcurveto{\pgfqpoint{3.993813in}{6.550397in}}{\pgfqpoint{3.996127in}{6.555984in}}{\pgfqpoint{3.996127in}{6.561808in}}%
\pgfpathcurveto{\pgfqpoint{3.996127in}{6.567631in}}{\pgfqpoint{3.993813in}{6.573218in}}{\pgfqpoint{3.989695in}{6.577336in}}%
\pgfpathcurveto{\pgfqpoint{3.985577in}{6.581454in}}{\pgfqpoint{3.979990in}{6.583768in}}{\pgfqpoint{3.974166in}{6.583768in}}%
\pgfpathcurveto{\pgfqpoint{3.968343in}{6.583768in}}{\pgfqpoint{3.962756in}{6.581454in}}{\pgfqpoint{3.958638in}{6.577336in}}%
\pgfpathcurveto{\pgfqpoint{3.954520in}{6.573218in}}{\pgfqpoint{3.952206in}{6.567631in}}{\pgfqpoint{3.952206in}{6.561808in}}%
\pgfpathcurveto{\pgfqpoint{3.952206in}{6.555984in}}{\pgfqpoint{3.954520in}{6.550397in}}{\pgfqpoint{3.958638in}{6.546279in}}%
\pgfpathcurveto{\pgfqpoint{3.962756in}{6.542161in}}{\pgfqpoint{3.968343in}{6.539847in}}{\pgfqpoint{3.974166in}{6.539847in}}%
\pgfpathlineto{\pgfqpoint{3.974166in}{6.539847in}}%
\pgfpathclose%
\pgfusepath{stroke,fill}%
\end{pgfscope}%
\begin{pgfscope}%
\pgfpathrectangle{\pgfqpoint{1.000000in}{0.979904in}}{\pgfqpoint{6.200000in}{5.960192in}}%
\pgfusepath{clip}%
\pgfsetbuttcap%
\pgfsetroundjoin%
\definecolor{currentfill}{rgb}{0.200000,0.200000,0.800000}%
\pgfsetfillcolor{currentfill}%
\pgfsetlinewidth{1.003750pt}%
\definecolor{currentstroke}{rgb}{0.200000,0.200000,0.800000}%
\pgfsetstrokecolor{currentstroke}%
\pgfsetdash{}{0pt}%
\pgfpathmoveto{\pgfqpoint{3.938249in}{6.634931in}}%
\pgfpathcurveto{\pgfqpoint{3.944073in}{6.634931in}}{\pgfqpoint{3.949659in}{6.637244in}}{\pgfqpoint{3.953777in}{6.641363in}}%
\pgfpathcurveto{\pgfqpoint{3.957895in}{6.645481in}}{\pgfqpoint{3.960209in}{6.651067in}}{\pgfqpoint{3.960209in}{6.656891in}}%
\pgfpathcurveto{\pgfqpoint{3.960209in}{6.662715in}}{\pgfqpoint{3.957895in}{6.668301in}}{\pgfqpoint{3.953777in}{6.672419in}}%
\pgfpathcurveto{\pgfqpoint{3.949659in}{6.676537in}}{\pgfqpoint{3.944073in}{6.678851in}}{\pgfqpoint{3.938249in}{6.678851in}}%
\pgfpathcurveto{\pgfqpoint{3.932425in}{6.678851in}}{\pgfqpoint{3.926839in}{6.676537in}}{\pgfqpoint{3.922720in}{6.672419in}}%
\pgfpathcurveto{\pgfqpoint{3.918602in}{6.668301in}}{\pgfqpoint{3.916288in}{6.662715in}}{\pgfqpoint{3.916288in}{6.656891in}}%
\pgfpathcurveto{\pgfqpoint{3.916288in}{6.651067in}}{\pgfqpoint{3.918602in}{6.645481in}}{\pgfqpoint{3.922720in}{6.641363in}}%
\pgfpathcurveto{\pgfqpoint{3.926839in}{6.637244in}}{\pgfqpoint{3.932425in}{6.634931in}}{\pgfqpoint{3.938249in}{6.634931in}}%
\pgfpathlineto{\pgfqpoint{3.938249in}{6.634931in}}%
\pgfpathclose%
\pgfusepath{stroke,fill}%
\end{pgfscope}%
\begin{pgfscope}%
\pgfpathrectangle{\pgfqpoint{1.000000in}{0.979904in}}{\pgfqpoint{6.200000in}{5.960192in}}%
\pgfusepath{clip}%
\pgfsetbuttcap%
\pgfsetroundjoin%
\definecolor{currentfill}{rgb}{0.200000,0.200000,0.800000}%
\pgfsetfillcolor{currentfill}%
\pgfsetlinewidth{1.003750pt}%
\definecolor{currentstroke}{rgb}{0.200000,0.200000,0.800000}%
\pgfsetstrokecolor{currentstroke}%
\pgfsetdash{}{0pt}%
\pgfpathmoveto{\pgfqpoint{3.870966in}{6.613446in}}%
\pgfpathcurveto{\pgfqpoint{3.876790in}{6.613446in}}{\pgfqpoint{3.882376in}{6.615760in}}{\pgfqpoint{3.886494in}{6.619878in}}%
\pgfpathcurveto{\pgfqpoint{3.890612in}{6.623996in}}{\pgfqpoint{3.892926in}{6.629582in}}{\pgfqpoint{3.892926in}{6.635406in}}%
\pgfpathcurveto{\pgfqpoint{3.892926in}{6.641230in}}{\pgfqpoint{3.890612in}{6.646816in}}{\pgfqpoint{3.886494in}{6.650935in}}%
\pgfpathcurveto{\pgfqpoint{3.882376in}{6.655053in}}{\pgfqpoint{3.876790in}{6.657367in}}{\pgfqpoint{3.870966in}{6.657367in}}%
\pgfpathcurveto{\pgfqpoint{3.865142in}{6.657367in}}{\pgfqpoint{3.859556in}{6.655053in}}{\pgfqpoint{3.855437in}{6.650935in}}%
\pgfpathcurveto{\pgfqpoint{3.851319in}{6.646816in}}{\pgfqpoint{3.849005in}{6.641230in}}{\pgfqpoint{3.849005in}{6.635406in}}%
\pgfpathcurveto{\pgfqpoint{3.849005in}{6.629582in}}{\pgfqpoint{3.851319in}{6.623996in}}{\pgfqpoint{3.855437in}{6.619878in}}%
\pgfpathcurveto{\pgfqpoint{3.859556in}{6.615760in}}{\pgfqpoint{3.865142in}{6.613446in}}{\pgfqpoint{3.870966in}{6.613446in}}%
\pgfpathlineto{\pgfqpoint{3.870966in}{6.613446in}}%
\pgfpathclose%
\pgfusepath{stroke,fill}%
\end{pgfscope}%
\begin{pgfscope}%
\pgfpathrectangle{\pgfqpoint{1.000000in}{0.979904in}}{\pgfqpoint{6.200000in}{5.960192in}}%
\pgfusepath{clip}%
\pgfsetbuttcap%
\pgfsetroundjoin%
\definecolor{currentfill}{rgb}{0.200000,0.200000,0.800000}%
\pgfsetfillcolor{currentfill}%
\pgfsetlinewidth{1.003750pt}%
\definecolor{currentstroke}{rgb}{0.200000,0.200000,0.800000}%
\pgfsetstrokecolor{currentstroke}%
\pgfsetdash{}{0pt}%
\pgfpathmoveto{\pgfqpoint{3.805779in}{6.573176in}}%
\pgfpathcurveto{\pgfqpoint{3.811603in}{6.573176in}}{\pgfqpoint{3.817189in}{6.575490in}}{\pgfqpoint{3.821307in}{6.579608in}}%
\pgfpathcurveto{\pgfqpoint{3.825426in}{6.583726in}}{\pgfqpoint{3.827740in}{6.589312in}}{\pgfqpoint{3.827740in}{6.595136in}}%
\pgfpathcurveto{\pgfqpoint{3.827740in}{6.600960in}}{\pgfqpoint{3.825426in}{6.606546in}}{\pgfqpoint{3.821307in}{6.610664in}}%
\pgfpathcurveto{\pgfqpoint{3.817189in}{6.614783in}}{\pgfqpoint{3.811603in}{6.617096in}}{\pgfqpoint{3.805779in}{6.617096in}}%
\pgfpathcurveto{\pgfqpoint{3.799955in}{6.617096in}}{\pgfqpoint{3.794369in}{6.614783in}}{\pgfqpoint{3.790251in}{6.610664in}}%
\pgfpathcurveto{\pgfqpoint{3.786133in}{6.606546in}}{\pgfqpoint{3.783819in}{6.600960in}}{\pgfqpoint{3.783819in}{6.595136in}}%
\pgfpathcurveto{\pgfqpoint{3.783819in}{6.589312in}}{\pgfqpoint{3.786133in}{6.583726in}}{\pgfqpoint{3.790251in}{6.579608in}}%
\pgfpathcurveto{\pgfqpoint{3.794369in}{6.575490in}}{\pgfqpoint{3.799955in}{6.573176in}}{\pgfqpoint{3.805779in}{6.573176in}}%
\pgfpathlineto{\pgfqpoint{3.805779in}{6.573176in}}%
\pgfpathclose%
\pgfusepath{stroke,fill}%
\end{pgfscope}%
\begin{pgfscope}%
\pgfpathrectangle{\pgfqpoint{1.000000in}{0.979904in}}{\pgfqpoint{6.200000in}{5.960192in}}%
\pgfusepath{clip}%
\pgfsetbuttcap%
\pgfsetroundjoin%
\definecolor{currentfill}{rgb}{0.200000,0.200000,0.800000}%
\pgfsetfillcolor{currentfill}%
\pgfsetlinewidth{1.003750pt}%
\definecolor{currentstroke}{rgb}{0.200000,0.200000,0.800000}%
\pgfsetstrokecolor{currentstroke}%
\pgfsetdash{}{0pt}%
\pgfpathmoveto{\pgfqpoint{3.747753in}{6.557255in}}%
\pgfpathcurveto{\pgfqpoint{3.753577in}{6.557255in}}{\pgfqpoint{3.759163in}{6.559569in}}{\pgfqpoint{3.763281in}{6.563687in}}%
\pgfpathcurveto{\pgfqpoint{3.767399in}{6.567805in}}{\pgfqpoint{3.769713in}{6.573392in}}{\pgfqpoint{3.769713in}{6.579216in}}%
\pgfpathcurveto{\pgfqpoint{3.769713in}{6.585039in}}{\pgfqpoint{3.767399in}{6.590626in}}{\pgfqpoint{3.763281in}{6.594744in}}%
\pgfpathcurveto{\pgfqpoint{3.759163in}{6.598862in}}{\pgfqpoint{3.753577in}{6.601176in}}{\pgfqpoint{3.747753in}{6.601176in}}%
\pgfpathcurveto{\pgfqpoint{3.741929in}{6.601176in}}{\pgfqpoint{3.736343in}{6.598862in}}{\pgfqpoint{3.732224in}{6.594744in}}%
\pgfpathcurveto{\pgfqpoint{3.728106in}{6.590626in}}{\pgfqpoint{3.725792in}{6.585039in}}{\pgfqpoint{3.725792in}{6.579216in}}%
\pgfpathcurveto{\pgfqpoint{3.725792in}{6.573392in}}{\pgfqpoint{3.728106in}{6.567805in}}{\pgfqpoint{3.732224in}{6.563687in}}%
\pgfpathcurveto{\pgfqpoint{3.736343in}{6.559569in}}{\pgfqpoint{3.741929in}{6.557255in}}{\pgfqpoint{3.747753in}{6.557255in}}%
\pgfpathlineto{\pgfqpoint{3.747753in}{6.557255in}}%
\pgfpathclose%
\pgfusepath{stroke,fill}%
\end{pgfscope}%
\begin{pgfscope}%
\pgfpathrectangle{\pgfqpoint{1.000000in}{0.979904in}}{\pgfqpoint{6.200000in}{5.960192in}}%
\pgfusepath{clip}%
\pgfsetbuttcap%
\pgfsetroundjoin%
\definecolor{currentfill}{rgb}{0.200000,0.200000,0.800000}%
\pgfsetfillcolor{currentfill}%
\pgfsetlinewidth{1.003750pt}%
\definecolor{currentstroke}{rgb}{0.200000,0.200000,0.800000}%
\pgfsetstrokecolor{currentstroke}%
\pgfsetdash{}{0pt}%
\pgfpathmoveto{\pgfqpoint{3.691103in}{6.603755in}}%
\pgfpathcurveto{\pgfqpoint{3.696927in}{6.603755in}}{\pgfqpoint{3.702514in}{6.606069in}}{\pgfqpoint{3.706632in}{6.610187in}}%
\pgfpathcurveto{\pgfqpoint{3.710750in}{6.614305in}}{\pgfqpoint{3.713064in}{6.619891in}}{\pgfqpoint{3.713064in}{6.625715in}}%
\pgfpathcurveto{\pgfqpoint{3.713064in}{6.631539in}}{\pgfqpoint{3.710750in}{6.637125in}}{\pgfqpoint{3.706632in}{6.641243in}}%
\pgfpathcurveto{\pgfqpoint{3.702514in}{6.645362in}}{\pgfqpoint{3.696927in}{6.647675in}}{\pgfqpoint{3.691103in}{6.647675in}}%
\pgfpathcurveto{\pgfqpoint{3.685280in}{6.647675in}}{\pgfqpoint{3.679693in}{6.645362in}}{\pgfqpoint{3.675575in}{6.641243in}}%
\pgfpathcurveto{\pgfqpoint{3.671457in}{6.637125in}}{\pgfqpoint{3.669143in}{6.631539in}}{\pgfqpoint{3.669143in}{6.625715in}}%
\pgfpathcurveto{\pgfqpoint{3.669143in}{6.619891in}}{\pgfqpoint{3.671457in}{6.614305in}}{\pgfqpoint{3.675575in}{6.610187in}}%
\pgfpathcurveto{\pgfqpoint{3.679693in}{6.606069in}}{\pgfqpoint{3.685280in}{6.603755in}}{\pgfqpoint{3.691103in}{6.603755in}}%
\pgfpathlineto{\pgfqpoint{3.691103in}{6.603755in}}%
\pgfpathclose%
\pgfusepath{stroke,fill}%
\end{pgfscope}%
\begin{pgfscope}%
\pgfpathrectangle{\pgfqpoint{1.000000in}{0.979904in}}{\pgfqpoint{6.200000in}{5.960192in}}%
\pgfusepath{clip}%
\pgfsetbuttcap%
\pgfsetroundjoin%
\definecolor{currentfill}{rgb}{0.200000,0.200000,0.800000}%
\pgfsetfillcolor{currentfill}%
\pgfsetlinewidth{1.003750pt}%
\definecolor{currentstroke}{rgb}{0.200000,0.200000,0.800000}%
\pgfsetstrokecolor{currentstroke}%
\pgfsetdash{}{0pt}%
\pgfpathmoveto{\pgfqpoint{3.633907in}{6.581286in}}%
\pgfpathcurveto{\pgfqpoint{3.639731in}{6.581286in}}{\pgfqpoint{3.645317in}{6.583600in}}{\pgfqpoint{3.649435in}{6.587718in}}%
\pgfpathcurveto{\pgfqpoint{3.653553in}{6.591837in}}{\pgfqpoint{3.655867in}{6.597423in}}{\pgfqpoint{3.655867in}{6.603247in}}%
\pgfpathcurveto{\pgfqpoint{3.655867in}{6.609071in}}{\pgfqpoint{3.653553in}{6.614657in}}{\pgfqpoint{3.649435in}{6.618775in}}%
\pgfpathcurveto{\pgfqpoint{3.645317in}{6.622893in}}{\pgfqpoint{3.639731in}{6.625207in}}{\pgfqpoint{3.633907in}{6.625207in}}%
\pgfpathcurveto{\pgfqpoint{3.628083in}{6.625207in}}{\pgfqpoint{3.622496in}{6.622893in}}{\pgfqpoint{3.618378in}{6.618775in}}%
\pgfpathcurveto{\pgfqpoint{3.614260in}{6.614657in}}{\pgfqpoint{3.611946in}{6.609071in}}{\pgfqpoint{3.611946in}{6.603247in}}%
\pgfpathcurveto{\pgfqpoint{3.611946in}{6.597423in}}{\pgfqpoint{3.614260in}{6.591837in}}{\pgfqpoint{3.618378in}{6.587718in}}%
\pgfpathcurveto{\pgfqpoint{3.622496in}{6.583600in}}{\pgfqpoint{3.628083in}{6.581286in}}{\pgfqpoint{3.633907in}{6.581286in}}%
\pgfpathlineto{\pgfqpoint{3.633907in}{6.581286in}}%
\pgfpathclose%
\pgfusepath{stroke,fill}%
\end{pgfscope}%
\begin{pgfscope}%
\pgfpathrectangle{\pgfqpoint{1.000000in}{0.979904in}}{\pgfqpoint{6.200000in}{5.960192in}}%
\pgfusepath{clip}%
\pgfsetbuttcap%
\pgfsetroundjoin%
\definecolor{currentfill}{rgb}{0.200000,0.200000,0.800000}%
\pgfsetfillcolor{currentfill}%
\pgfsetlinewidth{1.003750pt}%
\definecolor{currentstroke}{rgb}{0.200000,0.200000,0.800000}%
\pgfsetstrokecolor{currentstroke}%
\pgfsetdash{}{0pt}%
\pgfpathmoveto{\pgfqpoint{3.572698in}{6.602640in}}%
\pgfpathcurveto{\pgfqpoint{3.578522in}{6.602640in}}{\pgfqpoint{3.584109in}{6.604954in}}{\pgfqpoint{3.588227in}{6.609072in}}%
\pgfpathcurveto{\pgfqpoint{3.592345in}{6.613190in}}{\pgfqpoint{3.594659in}{6.618776in}}{\pgfqpoint{3.594659in}{6.624600in}}%
\pgfpathcurveto{\pgfqpoint{3.594659in}{6.630424in}}{\pgfqpoint{3.592345in}{6.636010in}}{\pgfqpoint{3.588227in}{6.640129in}}%
\pgfpathcurveto{\pgfqpoint{3.584109in}{6.644247in}}{\pgfqpoint{3.578522in}{6.646561in}}{\pgfqpoint{3.572698in}{6.646561in}}%
\pgfpathcurveto{\pgfqpoint{3.566875in}{6.646561in}}{\pgfqpoint{3.561288in}{6.644247in}}{\pgfqpoint{3.557170in}{6.640129in}}%
\pgfpathcurveto{\pgfqpoint{3.553052in}{6.636010in}}{\pgfqpoint{3.550738in}{6.630424in}}{\pgfqpoint{3.550738in}{6.624600in}}%
\pgfpathcurveto{\pgfqpoint{3.550738in}{6.618776in}}{\pgfqpoint{3.553052in}{6.613190in}}{\pgfqpoint{3.557170in}{6.609072in}}%
\pgfpathcurveto{\pgfqpoint{3.561288in}{6.604954in}}{\pgfqpoint{3.566875in}{6.602640in}}{\pgfqpoint{3.572698in}{6.602640in}}%
\pgfpathlineto{\pgfqpoint{3.572698in}{6.602640in}}%
\pgfpathclose%
\pgfusepath{stroke,fill}%
\end{pgfscope}%
\begin{pgfscope}%
\pgfpathrectangle{\pgfqpoint{1.000000in}{0.979904in}}{\pgfqpoint{6.200000in}{5.960192in}}%
\pgfusepath{clip}%
\pgfsetbuttcap%
\pgfsetroundjoin%
\definecolor{currentfill}{rgb}{0.200000,0.200000,0.800000}%
\pgfsetfillcolor{currentfill}%
\pgfsetlinewidth{1.003750pt}%
\definecolor{currentstroke}{rgb}{0.200000,0.200000,0.800000}%
\pgfsetstrokecolor{currentstroke}%
\pgfsetdash{}{0pt}%
\pgfpathmoveto{\pgfqpoint{3.516680in}{6.580705in}}%
\pgfpathcurveto{\pgfqpoint{3.522504in}{6.580705in}}{\pgfqpoint{3.528090in}{6.583019in}}{\pgfqpoint{3.532208in}{6.587137in}}%
\pgfpathcurveto{\pgfqpoint{3.536327in}{6.591255in}}{\pgfqpoint{3.538640in}{6.596841in}}{\pgfqpoint{3.538640in}{6.602665in}}%
\pgfpathcurveto{\pgfqpoint{3.538640in}{6.608489in}}{\pgfqpoint{3.536327in}{6.614075in}}{\pgfqpoint{3.532208in}{6.618193in}}%
\pgfpathcurveto{\pgfqpoint{3.528090in}{6.622311in}}{\pgfqpoint{3.522504in}{6.624625in}}{\pgfqpoint{3.516680in}{6.624625in}}%
\pgfpathcurveto{\pgfqpoint{3.510856in}{6.624625in}}{\pgfqpoint{3.505270in}{6.622311in}}{\pgfqpoint{3.501152in}{6.618193in}}%
\pgfpathcurveto{\pgfqpoint{3.497034in}{6.614075in}}{\pgfqpoint{3.494720in}{6.608489in}}{\pgfqpoint{3.494720in}{6.602665in}}%
\pgfpathcurveto{\pgfqpoint{3.494720in}{6.596841in}}{\pgfqpoint{3.497034in}{6.591255in}}{\pgfqpoint{3.501152in}{6.587137in}}%
\pgfpathcurveto{\pgfqpoint{3.505270in}{6.583019in}}{\pgfqpoint{3.510856in}{6.580705in}}{\pgfqpoint{3.516680in}{6.580705in}}%
\pgfpathlineto{\pgfqpoint{3.516680in}{6.580705in}}%
\pgfpathclose%
\pgfusepath{stroke,fill}%
\end{pgfscope}%
\begin{pgfscope}%
\pgfpathrectangle{\pgfqpoint{1.000000in}{0.979904in}}{\pgfqpoint{6.200000in}{5.960192in}}%
\pgfusepath{clip}%
\pgfsetbuttcap%
\pgfsetroundjoin%
\definecolor{currentfill}{rgb}{0.200000,0.200000,0.800000}%
\pgfsetfillcolor{currentfill}%
\pgfsetlinewidth{1.003750pt}%
\definecolor{currentstroke}{rgb}{0.200000,0.200000,0.800000}%
\pgfsetstrokecolor{currentstroke}%
\pgfsetdash{}{0pt}%
\pgfpathmoveto{\pgfqpoint{3.450984in}{6.598534in}}%
\pgfpathcurveto{\pgfqpoint{3.456808in}{6.598534in}}{\pgfqpoint{3.462394in}{6.600848in}}{\pgfqpoint{3.466512in}{6.604966in}}%
\pgfpathcurveto{\pgfqpoint{3.470630in}{6.609084in}}{\pgfqpoint{3.472944in}{6.614670in}}{\pgfqpoint{3.472944in}{6.620494in}}%
\pgfpathcurveto{\pgfqpoint{3.472944in}{6.626318in}}{\pgfqpoint{3.470630in}{6.631904in}}{\pgfqpoint{3.466512in}{6.636022in}}%
\pgfpathcurveto{\pgfqpoint{3.462394in}{6.640140in}}{\pgfqpoint{3.456808in}{6.642454in}}{\pgfqpoint{3.450984in}{6.642454in}}%
\pgfpathcurveto{\pgfqpoint{3.445160in}{6.642454in}}{\pgfqpoint{3.439574in}{6.640140in}}{\pgfqpoint{3.435456in}{6.636022in}}%
\pgfpathcurveto{\pgfqpoint{3.431338in}{6.631904in}}{\pgfqpoint{3.429024in}{6.626318in}}{\pgfqpoint{3.429024in}{6.620494in}}%
\pgfpathcurveto{\pgfqpoint{3.429024in}{6.614670in}}{\pgfqpoint{3.431338in}{6.609084in}}{\pgfqpoint{3.435456in}{6.604966in}}%
\pgfpathcurveto{\pgfqpoint{3.439574in}{6.600848in}}{\pgfqpoint{3.445160in}{6.598534in}}{\pgfqpoint{3.450984in}{6.598534in}}%
\pgfpathlineto{\pgfqpoint{3.450984in}{6.598534in}}%
\pgfpathclose%
\pgfusepath{stroke,fill}%
\end{pgfscope}%
\begin{pgfscope}%
\pgfpathrectangle{\pgfqpoint{1.000000in}{0.979904in}}{\pgfqpoint{6.200000in}{5.960192in}}%
\pgfusepath{clip}%
\pgfsetbuttcap%
\pgfsetroundjoin%
\definecolor{currentfill}{rgb}{0.200000,0.200000,0.800000}%
\pgfsetfillcolor{currentfill}%
\pgfsetlinewidth{1.003750pt}%
\definecolor{currentstroke}{rgb}{0.200000,0.200000,0.800000}%
\pgfsetstrokecolor{currentstroke}%
\pgfsetdash{}{0pt}%
\pgfpathmoveto{\pgfqpoint{3.421748in}{6.497563in}}%
\pgfpathcurveto{\pgfqpoint{3.427572in}{6.497563in}}{\pgfqpoint{3.433158in}{6.499877in}}{\pgfqpoint{3.437276in}{6.503995in}}%
\pgfpathcurveto{\pgfqpoint{3.441395in}{6.508113in}}{\pgfqpoint{3.443708in}{6.513699in}}{\pgfqpoint{3.443708in}{6.519523in}}%
\pgfpathcurveto{\pgfqpoint{3.443708in}{6.525347in}}{\pgfqpoint{3.441395in}{6.530933in}}{\pgfqpoint{3.437276in}{6.535052in}}%
\pgfpathcurveto{\pgfqpoint{3.433158in}{6.539170in}}{\pgfqpoint{3.427572in}{6.541484in}}{\pgfqpoint{3.421748in}{6.541484in}}%
\pgfpathcurveto{\pgfqpoint{3.415924in}{6.541484in}}{\pgfqpoint{3.410338in}{6.539170in}}{\pgfqpoint{3.406220in}{6.535052in}}%
\pgfpathcurveto{\pgfqpoint{3.402102in}{6.530933in}}{\pgfqpoint{3.399788in}{6.525347in}}{\pgfqpoint{3.399788in}{6.519523in}}%
\pgfpathcurveto{\pgfqpoint{3.399788in}{6.513699in}}{\pgfqpoint{3.402102in}{6.508113in}}{\pgfqpoint{3.406220in}{6.503995in}}%
\pgfpathcurveto{\pgfqpoint{3.410338in}{6.499877in}}{\pgfqpoint{3.415924in}{6.497563in}}{\pgfqpoint{3.421748in}{6.497563in}}%
\pgfpathlineto{\pgfqpoint{3.421748in}{6.497563in}}%
\pgfpathclose%
\pgfusepath{stroke,fill}%
\end{pgfscope}%
\begin{pgfscope}%
\pgfpathrectangle{\pgfqpoint{1.000000in}{0.979904in}}{\pgfqpoint{6.200000in}{5.960192in}}%
\pgfusepath{clip}%
\pgfsetbuttcap%
\pgfsetroundjoin%
\definecolor{currentfill}{rgb}{0.200000,0.200000,0.800000}%
\pgfsetfillcolor{currentfill}%
\pgfsetlinewidth{1.003750pt}%
\definecolor{currentstroke}{rgb}{0.200000,0.200000,0.800000}%
\pgfsetstrokecolor{currentstroke}%
\pgfsetdash{}{0pt}%
\pgfpathmoveto{\pgfqpoint{3.321948in}{6.593244in}}%
\pgfpathcurveto{\pgfqpoint{3.327772in}{6.593244in}}{\pgfqpoint{3.333359in}{6.595558in}}{\pgfqpoint{3.337477in}{6.599676in}}%
\pgfpathcurveto{\pgfqpoint{3.341595in}{6.603794in}}{\pgfqpoint{3.343909in}{6.609381in}}{\pgfqpoint{3.343909in}{6.615204in}}%
\pgfpathcurveto{\pgfqpoint{3.343909in}{6.621028in}}{\pgfqpoint{3.341595in}{6.626615in}}{\pgfqpoint{3.337477in}{6.630733in}}%
\pgfpathcurveto{\pgfqpoint{3.333359in}{6.634851in}}{\pgfqpoint{3.327772in}{6.637165in}}{\pgfqpoint{3.321948in}{6.637165in}}%
\pgfpathcurveto{\pgfqpoint{3.316125in}{6.637165in}}{\pgfqpoint{3.310538in}{6.634851in}}{\pgfqpoint{3.306420in}{6.630733in}}%
\pgfpathcurveto{\pgfqpoint{3.302302in}{6.626615in}}{\pgfqpoint{3.299988in}{6.621028in}}{\pgfqpoint{3.299988in}{6.615204in}}%
\pgfpathcurveto{\pgfqpoint{3.299988in}{6.609381in}}{\pgfqpoint{3.302302in}{6.603794in}}{\pgfqpoint{3.306420in}{6.599676in}}%
\pgfpathcurveto{\pgfqpoint{3.310538in}{6.595558in}}{\pgfqpoint{3.316125in}{6.593244in}}{\pgfqpoint{3.321948in}{6.593244in}}%
\pgfpathlineto{\pgfqpoint{3.321948in}{6.593244in}}%
\pgfpathclose%
\pgfusepath{stroke,fill}%
\end{pgfscope}%
\begin{pgfscope}%
\pgfpathrectangle{\pgfqpoint{1.000000in}{0.979904in}}{\pgfqpoint{6.200000in}{5.960192in}}%
\pgfusepath{clip}%
\pgfsetbuttcap%
\pgfsetroundjoin%
\definecolor{currentfill}{rgb}{0.200000,0.200000,0.800000}%
\pgfsetfillcolor{currentfill}%
\pgfsetlinewidth{1.003750pt}%
\definecolor{currentstroke}{rgb}{0.200000,0.200000,0.800000}%
\pgfsetstrokecolor{currentstroke}%
\pgfsetdash{}{0pt}%
\pgfpathmoveto{\pgfqpoint{3.224867in}{6.647218in}}%
\pgfpathcurveto{\pgfqpoint{3.230691in}{6.647218in}}{\pgfqpoint{3.236277in}{6.649532in}}{\pgfqpoint{3.240395in}{6.653650in}}%
\pgfpathcurveto{\pgfqpoint{3.244513in}{6.657768in}}{\pgfqpoint{3.246827in}{6.663354in}}{\pgfqpoint{3.246827in}{6.669178in}}%
\pgfpathcurveto{\pgfqpoint{3.246827in}{6.675002in}}{\pgfqpoint{3.244513in}{6.680588in}}{\pgfqpoint{3.240395in}{6.684707in}}%
\pgfpathcurveto{\pgfqpoint{3.236277in}{6.688825in}}{\pgfqpoint{3.230691in}{6.691139in}}{\pgfqpoint{3.224867in}{6.691139in}}%
\pgfpathcurveto{\pgfqpoint{3.219043in}{6.691139in}}{\pgfqpoint{3.213457in}{6.688825in}}{\pgfqpoint{3.209339in}{6.684707in}}%
\pgfpathcurveto{\pgfqpoint{3.205220in}{6.680588in}}{\pgfqpoint{3.202907in}{6.675002in}}{\pgfqpoint{3.202907in}{6.669178in}}%
\pgfpathcurveto{\pgfqpoint{3.202907in}{6.663354in}}{\pgfqpoint{3.205220in}{6.657768in}}{\pgfqpoint{3.209339in}{6.653650in}}%
\pgfpathcurveto{\pgfqpoint{3.213457in}{6.649532in}}{\pgfqpoint{3.219043in}{6.647218in}}{\pgfqpoint{3.224867in}{6.647218in}}%
\pgfpathlineto{\pgfqpoint{3.224867in}{6.647218in}}%
\pgfpathclose%
\pgfusepath{stroke,fill}%
\end{pgfscope}%
\begin{pgfscope}%
\pgfpathrectangle{\pgfqpoint{1.000000in}{0.979904in}}{\pgfqpoint{6.200000in}{5.960192in}}%
\pgfusepath{clip}%
\pgfsetbuttcap%
\pgfsetroundjoin%
\definecolor{currentfill}{rgb}{0.200000,0.200000,0.800000}%
\pgfsetfillcolor{currentfill}%
\pgfsetlinewidth{1.003750pt}%
\definecolor{currentstroke}{rgb}{0.200000,0.200000,0.800000}%
\pgfsetstrokecolor{currentstroke}%
\pgfsetdash{}{0pt}%
\pgfpathmoveto{\pgfqpoint{3.206327in}{6.541948in}}%
\pgfpathcurveto{\pgfqpoint{3.212151in}{6.541948in}}{\pgfqpoint{3.217737in}{6.544262in}}{\pgfqpoint{3.221855in}{6.548380in}}%
\pgfpathcurveto{\pgfqpoint{3.225973in}{6.552498in}}{\pgfqpoint{3.228287in}{6.558084in}}{\pgfqpoint{3.228287in}{6.563908in}}%
\pgfpathcurveto{\pgfqpoint{3.228287in}{6.569732in}}{\pgfqpoint{3.225973in}{6.575318in}}{\pgfqpoint{3.221855in}{6.579436in}}%
\pgfpathcurveto{\pgfqpoint{3.217737in}{6.583554in}}{\pgfqpoint{3.212151in}{6.585868in}}{\pgfqpoint{3.206327in}{6.585868in}}%
\pgfpathcurveto{\pgfqpoint{3.200503in}{6.585868in}}{\pgfqpoint{3.194917in}{6.583554in}}{\pgfqpoint{3.190799in}{6.579436in}}%
\pgfpathcurveto{\pgfqpoint{3.186681in}{6.575318in}}{\pgfqpoint{3.184367in}{6.569732in}}{\pgfqpoint{3.184367in}{6.563908in}}%
\pgfpathcurveto{\pgfqpoint{3.184367in}{6.558084in}}{\pgfqpoint{3.186681in}{6.552498in}}{\pgfqpoint{3.190799in}{6.548380in}}%
\pgfpathcurveto{\pgfqpoint{3.194917in}{6.544262in}}{\pgfqpoint{3.200503in}{6.541948in}}{\pgfqpoint{3.206327in}{6.541948in}}%
\pgfpathlineto{\pgfqpoint{3.206327in}{6.541948in}}%
\pgfpathclose%
\pgfusepath{stroke,fill}%
\end{pgfscope}%
\begin{pgfscope}%
\pgfpathrectangle{\pgfqpoint{1.000000in}{0.979904in}}{\pgfqpoint{6.200000in}{5.960192in}}%
\pgfusepath{clip}%
\pgfsetbuttcap%
\pgfsetroundjoin%
\definecolor{currentfill}{rgb}{0.200000,0.200000,0.800000}%
\pgfsetfillcolor{currentfill}%
\pgfsetlinewidth{1.003750pt}%
\definecolor{currentstroke}{rgb}{0.200000,0.200000,0.800000}%
\pgfsetstrokecolor{currentstroke}%
\pgfsetdash{}{0pt}%
\pgfpathmoveto{\pgfqpoint{3.181117in}{6.465457in}}%
\pgfpathcurveto{\pgfqpoint{3.186941in}{6.465457in}}{\pgfqpoint{3.192527in}{6.467771in}}{\pgfqpoint{3.196645in}{6.471889in}}%
\pgfpathcurveto{\pgfqpoint{3.200763in}{6.476008in}}{\pgfqpoint{3.203077in}{6.481594in}}{\pgfqpoint{3.203077in}{6.487418in}}%
\pgfpathcurveto{\pgfqpoint{3.203077in}{6.493242in}}{\pgfqpoint{3.200763in}{6.498828in}}{\pgfqpoint{3.196645in}{6.502946in}}%
\pgfpathcurveto{\pgfqpoint{3.192527in}{6.507064in}}{\pgfqpoint{3.186941in}{6.509378in}}{\pgfqpoint{3.181117in}{6.509378in}}%
\pgfpathcurveto{\pgfqpoint{3.175293in}{6.509378in}}{\pgfqpoint{3.169707in}{6.507064in}}{\pgfqpoint{3.165589in}{6.502946in}}%
\pgfpathcurveto{\pgfqpoint{3.161471in}{6.498828in}}{\pgfqpoint{3.159157in}{6.493242in}}{\pgfqpoint{3.159157in}{6.487418in}}%
\pgfpathcurveto{\pgfqpoint{3.159157in}{6.481594in}}{\pgfqpoint{3.161471in}{6.476008in}}{\pgfqpoint{3.165589in}{6.471889in}}%
\pgfpathcurveto{\pgfqpoint{3.169707in}{6.467771in}}{\pgfqpoint{3.175293in}{6.465457in}}{\pgfqpoint{3.181117in}{6.465457in}}%
\pgfpathlineto{\pgfqpoint{3.181117in}{6.465457in}}%
\pgfpathclose%
\pgfusepath{stroke,fill}%
\end{pgfscope}%
\begin{pgfscope}%
\pgfpathrectangle{\pgfqpoint{1.000000in}{0.979904in}}{\pgfqpoint{6.200000in}{5.960192in}}%
\pgfusepath{clip}%
\pgfsetbuttcap%
\pgfsetroundjoin%
\definecolor{currentfill}{rgb}{0.200000,0.200000,0.800000}%
\pgfsetfillcolor{currentfill}%
\pgfsetlinewidth{1.003750pt}%
\definecolor{currentstroke}{rgb}{0.200000,0.200000,0.800000}%
\pgfsetstrokecolor{currentstroke}%
\pgfsetdash{}{0pt}%
\pgfpathmoveto{\pgfqpoint{3.110847in}{6.458589in}}%
\pgfpathcurveto{\pgfqpoint{3.116671in}{6.458589in}}{\pgfqpoint{3.122257in}{6.460903in}}{\pgfqpoint{3.126376in}{6.465021in}}%
\pgfpathcurveto{\pgfqpoint{3.130494in}{6.469139in}}{\pgfqpoint{3.132808in}{6.474725in}}{\pgfqpoint{3.132808in}{6.480549in}}%
\pgfpathcurveto{\pgfqpoint{3.132808in}{6.486373in}}{\pgfqpoint{3.130494in}{6.491959in}}{\pgfqpoint{3.126376in}{6.496077in}}%
\pgfpathcurveto{\pgfqpoint{3.122257in}{6.500195in}}{\pgfqpoint{3.116671in}{6.502509in}}{\pgfqpoint{3.110847in}{6.502509in}}%
\pgfpathcurveto{\pgfqpoint{3.105023in}{6.502509in}}{\pgfqpoint{3.099437in}{6.500195in}}{\pgfqpoint{3.095319in}{6.496077in}}%
\pgfpathcurveto{\pgfqpoint{3.091201in}{6.491959in}}{\pgfqpoint{3.088887in}{6.486373in}}{\pgfqpoint{3.088887in}{6.480549in}}%
\pgfpathcurveto{\pgfqpoint{3.088887in}{6.474725in}}{\pgfqpoint{3.091201in}{6.469139in}}{\pgfqpoint{3.095319in}{6.465021in}}%
\pgfpathcurveto{\pgfqpoint{3.099437in}{6.460903in}}{\pgfqpoint{3.105023in}{6.458589in}}{\pgfqpoint{3.110847in}{6.458589in}}%
\pgfpathlineto{\pgfqpoint{3.110847in}{6.458589in}}%
\pgfpathclose%
\pgfusepath{stroke,fill}%
\end{pgfscope}%
\begin{pgfscope}%
\pgfpathrectangle{\pgfqpoint{1.000000in}{0.979904in}}{\pgfqpoint{6.200000in}{5.960192in}}%
\pgfusepath{clip}%
\pgfsetbuttcap%
\pgfsetroundjoin%
\definecolor{currentfill}{rgb}{0.200000,0.200000,0.800000}%
\pgfsetfillcolor{currentfill}%
\pgfsetlinewidth{1.003750pt}%
\definecolor{currentstroke}{rgb}{0.200000,0.200000,0.800000}%
\pgfsetstrokecolor{currentstroke}%
\pgfsetdash{}{0pt}%
\pgfpathmoveto{\pgfqpoint{3.128159in}{6.343462in}}%
\pgfpathcurveto{\pgfqpoint{3.133983in}{6.343462in}}{\pgfqpoint{3.139569in}{6.345775in}}{\pgfqpoint{3.143687in}{6.349894in}}%
\pgfpathcurveto{\pgfqpoint{3.147805in}{6.354012in}}{\pgfqpoint{3.150119in}{6.359598in}}{\pgfqpoint{3.150119in}{6.365422in}}%
\pgfpathcurveto{\pgfqpoint{3.150119in}{6.371246in}}{\pgfqpoint{3.147805in}{6.376832in}}{\pgfqpoint{3.143687in}{6.380950in}}%
\pgfpathcurveto{\pgfqpoint{3.139569in}{6.385068in}}{\pgfqpoint{3.133983in}{6.387382in}}{\pgfqpoint{3.128159in}{6.387382in}}%
\pgfpathcurveto{\pgfqpoint{3.122335in}{6.387382in}}{\pgfqpoint{3.116749in}{6.385068in}}{\pgfqpoint{3.112631in}{6.380950in}}%
\pgfpathcurveto{\pgfqpoint{3.108512in}{6.376832in}}{\pgfqpoint{3.106199in}{6.371246in}}{\pgfqpoint{3.106199in}{6.365422in}}%
\pgfpathcurveto{\pgfqpoint{3.106199in}{6.359598in}}{\pgfqpoint{3.108512in}{6.354012in}}{\pgfqpoint{3.112631in}{6.349894in}}%
\pgfpathcurveto{\pgfqpoint{3.116749in}{6.345775in}}{\pgfqpoint{3.122335in}{6.343462in}}{\pgfqpoint{3.128159in}{6.343462in}}%
\pgfpathlineto{\pgfqpoint{3.128159in}{6.343462in}}%
\pgfpathclose%
\pgfusepath{stroke,fill}%
\end{pgfscope}%
\begin{pgfscope}%
\pgfpathrectangle{\pgfqpoint{1.000000in}{0.979904in}}{\pgfqpoint{6.200000in}{5.960192in}}%
\pgfusepath{clip}%
\pgfsetbuttcap%
\pgfsetroundjoin%
\definecolor{currentfill}{rgb}{0.200000,0.200000,0.800000}%
\pgfsetfillcolor{currentfill}%
\pgfsetlinewidth{1.003750pt}%
\definecolor{currentstroke}{rgb}{0.200000,0.200000,0.800000}%
\pgfsetstrokecolor{currentstroke}%
\pgfsetdash{}{0pt}%
\pgfpathmoveto{\pgfqpoint{3.108411in}{6.283770in}}%
\pgfpathcurveto{\pgfqpoint{3.114235in}{6.283770in}}{\pgfqpoint{3.119821in}{6.286084in}}{\pgfqpoint{3.123940in}{6.290202in}}%
\pgfpathcurveto{\pgfqpoint{3.128058in}{6.294320in}}{\pgfqpoint{3.130372in}{6.299906in}}{\pgfqpoint{3.130372in}{6.305730in}}%
\pgfpathcurveto{\pgfqpoint{3.130372in}{6.311554in}}{\pgfqpoint{3.128058in}{6.317140in}}{\pgfqpoint{3.123940in}{6.321258in}}%
\pgfpathcurveto{\pgfqpoint{3.119821in}{6.325376in}}{\pgfqpoint{3.114235in}{6.327690in}}{\pgfqpoint{3.108411in}{6.327690in}}%
\pgfpathcurveto{\pgfqpoint{3.102587in}{6.327690in}}{\pgfqpoint{3.097001in}{6.325376in}}{\pgfqpoint{3.092883in}{6.321258in}}%
\pgfpathcurveto{\pgfqpoint{3.088765in}{6.317140in}}{\pgfqpoint{3.086451in}{6.311554in}}{\pgfqpoint{3.086451in}{6.305730in}}%
\pgfpathcurveto{\pgfqpoint{3.086451in}{6.299906in}}{\pgfqpoint{3.088765in}{6.294320in}}{\pgfqpoint{3.092883in}{6.290202in}}%
\pgfpathcurveto{\pgfqpoint{3.097001in}{6.286084in}}{\pgfqpoint{3.102587in}{6.283770in}}{\pgfqpoint{3.108411in}{6.283770in}}%
\pgfpathlineto{\pgfqpoint{3.108411in}{6.283770in}}%
\pgfpathclose%
\pgfusepath{stroke,fill}%
\end{pgfscope}%
\begin{pgfscope}%
\pgfpathrectangle{\pgfqpoint{1.000000in}{0.979904in}}{\pgfqpoint{6.200000in}{5.960192in}}%
\pgfusepath{clip}%
\pgfsetbuttcap%
\pgfsetroundjoin%
\definecolor{currentfill}{rgb}{0.200000,0.200000,0.800000}%
\pgfsetfillcolor{currentfill}%
\pgfsetlinewidth{1.003750pt}%
\definecolor{currentstroke}{rgb}{0.200000,0.200000,0.800000}%
\pgfsetstrokecolor{currentstroke}%
\pgfsetdash{}{0pt}%
\pgfpathmoveto{\pgfqpoint{2.977920in}{6.328040in}}%
\pgfpathcurveto{\pgfqpoint{2.983744in}{6.328040in}}{\pgfqpoint{2.989330in}{6.330354in}}{\pgfqpoint{2.993448in}{6.334472in}}%
\pgfpathcurveto{\pgfqpoint{2.997566in}{6.338591in}}{\pgfqpoint{2.999880in}{6.344177in}}{\pgfqpoint{2.999880in}{6.350001in}}%
\pgfpathcurveto{\pgfqpoint{2.999880in}{6.355825in}}{\pgfqpoint{2.997566in}{6.361411in}}{\pgfqpoint{2.993448in}{6.365529in}}%
\pgfpathcurveto{\pgfqpoint{2.989330in}{6.369647in}}{\pgfqpoint{2.983744in}{6.371961in}}{\pgfqpoint{2.977920in}{6.371961in}}%
\pgfpathcurveto{\pgfqpoint{2.972096in}{6.371961in}}{\pgfqpoint{2.966510in}{6.369647in}}{\pgfqpoint{2.962392in}{6.365529in}}%
\pgfpathcurveto{\pgfqpoint{2.958274in}{6.361411in}}{\pgfqpoint{2.955960in}{6.355825in}}{\pgfqpoint{2.955960in}{6.350001in}}%
\pgfpathcurveto{\pgfqpoint{2.955960in}{6.344177in}}{\pgfqpoint{2.958274in}{6.338591in}}{\pgfqpoint{2.962392in}{6.334472in}}%
\pgfpathcurveto{\pgfqpoint{2.966510in}{6.330354in}}{\pgfqpoint{2.972096in}{6.328040in}}{\pgfqpoint{2.977920in}{6.328040in}}%
\pgfpathlineto{\pgfqpoint{2.977920in}{6.328040in}}%
\pgfpathclose%
\pgfusepath{stroke,fill}%
\end{pgfscope}%
\begin{pgfscope}%
\pgfpathrectangle{\pgfqpoint{1.000000in}{0.979904in}}{\pgfqpoint{6.200000in}{5.960192in}}%
\pgfusepath{clip}%
\pgfsetbuttcap%
\pgfsetroundjoin%
\definecolor{currentfill}{rgb}{0.200000,0.200000,0.800000}%
\pgfsetfillcolor{currentfill}%
\pgfsetlinewidth{1.003750pt}%
\definecolor{currentstroke}{rgb}{0.200000,0.200000,0.800000}%
\pgfsetstrokecolor{currentstroke}%
\pgfsetdash{}{0pt}%
\pgfpathmoveto{\pgfqpoint{2.992793in}{6.237538in}}%
\pgfpathcurveto{\pgfqpoint{2.998617in}{6.237538in}}{\pgfqpoint{3.004203in}{6.239852in}}{\pgfqpoint{3.008321in}{6.243970in}}%
\pgfpathcurveto{\pgfqpoint{3.012439in}{6.248089in}}{\pgfqpoint{3.014753in}{6.253675in}}{\pgfqpoint{3.014753in}{6.259499in}}%
\pgfpathcurveto{\pgfqpoint{3.014753in}{6.265323in}}{\pgfqpoint{3.012439in}{6.270909in}}{\pgfqpoint{3.008321in}{6.275027in}}%
\pgfpathcurveto{\pgfqpoint{3.004203in}{6.279145in}}{\pgfqpoint{2.998617in}{6.281459in}}{\pgfqpoint{2.992793in}{6.281459in}}%
\pgfpathcurveto{\pgfqpoint{2.986969in}{6.281459in}}{\pgfqpoint{2.981383in}{6.279145in}}{\pgfqpoint{2.977265in}{6.275027in}}%
\pgfpathcurveto{\pgfqpoint{2.973147in}{6.270909in}}{\pgfqpoint{2.970833in}{6.265323in}}{\pgfqpoint{2.970833in}{6.259499in}}%
\pgfpathcurveto{\pgfqpoint{2.970833in}{6.253675in}}{\pgfqpoint{2.973147in}{6.248089in}}{\pgfqpoint{2.977265in}{6.243970in}}%
\pgfpathcurveto{\pgfqpoint{2.981383in}{6.239852in}}{\pgfqpoint{2.986969in}{6.237538in}}{\pgfqpoint{2.992793in}{6.237538in}}%
\pgfpathlineto{\pgfqpoint{2.992793in}{6.237538in}}%
\pgfpathclose%
\pgfusepath{stroke,fill}%
\end{pgfscope}%
\begin{pgfscope}%
\pgfpathrectangle{\pgfqpoint{1.000000in}{0.979904in}}{\pgfqpoint{6.200000in}{5.960192in}}%
\pgfusepath{clip}%
\pgfsetbuttcap%
\pgfsetroundjoin%
\definecolor{currentfill}{rgb}{0.200000,0.200000,0.800000}%
\pgfsetfillcolor{currentfill}%
\pgfsetlinewidth{1.003750pt}%
\definecolor{currentstroke}{rgb}{0.200000,0.200000,0.800000}%
\pgfsetstrokecolor{currentstroke}%
\pgfsetdash{}{0pt}%
\pgfpathmoveto{\pgfqpoint{2.994000in}{6.166861in}}%
\pgfpathcurveto{\pgfqpoint{2.999823in}{6.166861in}}{\pgfqpoint{3.005410in}{6.169175in}}{\pgfqpoint{3.009528in}{6.173293in}}%
\pgfpathcurveto{\pgfqpoint{3.013646in}{6.177411in}}{\pgfqpoint{3.015960in}{6.182998in}}{\pgfqpoint{3.015960in}{6.188821in}}%
\pgfpathcurveto{\pgfqpoint{3.015960in}{6.194645in}}{\pgfqpoint{3.013646in}{6.200232in}}{\pgfqpoint{3.009528in}{6.204350in}}%
\pgfpathcurveto{\pgfqpoint{3.005410in}{6.208468in}}{\pgfqpoint{2.999823in}{6.210782in}}{\pgfqpoint{2.994000in}{6.210782in}}%
\pgfpathcurveto{\pgfqpoint{2.988176in}{6.210782in}}{\pgfqpoint{2.982589in}{6.208468in}}{\pgfqpoint{2.978471in}{6.204350in}}%
\pgfpathcurveto{\pgfqpoint{2.974353in}{6.200232in}}{\pgfqpoint{2.972039in}{6.194645in}}{\pgfqpoint{2.972039in}{6.188821in}}%
\pgfpathcurveto{\pgfqpoint{2.972039in}{6.182998in}}{\pgfqpoint{2.974353in}{6.177411in}}{\pgfqpoint{2.978471in}{6.173293in}}%
\pgfpathcurveto{\pgfqpoint{2.982589in}{6.169175in}}{\pgfqpoint{2.988176in}{6.166861in}}{\pgfqpoint{2.994000in}{6.166861in}}%
\pgfpathlineto{\pgfqpoint{2.994000in}{6.166861in}}%
\pgfpathclose%
\pgfusepath{stroke,fill}%
\end{pgfscope}%
\begin{pgfscope}%
\pgfpathrectangle{\pgfqpoint{1.000000in}{0.979904in}}{\pgfqpoint{6.200000in}{5.960192in}}%
\pgfusepath{clip}%
\pgfsetbuttcap%
\pgfsetroundjoin%
\definecolor{currentfill}{rgb}{0.200000,0.200000,0.800000}%
\pgfsetfillcolor{currentfill}%
\pgfsetlinewidth{1.003750pt}%
\definecolor{currentstroke}{rgb}{0.200000,0.200000,0.800000}%
\pgfsetstrokecolor{currentstroke}%
\pgfsetdash{}{0pt}%
\pgfpathmoveto{\pgfqpoint{2.943135in}{6.133431in}}%
\pgfpathcurveto{\pgfqpoint{2.948959in}{6.133431in}}{\pgfqpoint{2.954545in}{6.135745in}}{\pgfqpoint{2.958663in}{6.139863in}}%
\pgfpathcurveto{\pgfqpoint{2.962781in}{6.143982in}}{\pgfqpoint{2.965095in}{6.149568in}}{\pgfqpoint{2.965095in}{6.155392in}}%
\pgfpathcurveto{\pgfqpoint{2.965095in}{6.161216in}}{\pgfqpoint{2.962781in}{6.166802in}}{\pgfqpoint{2.958663in}{6.170920in}}%
\pgfpathcurveto{\pgfqpoint{2.954545in}{6.175038in}}{\pgfqpoint{2.948959in}{6.177352in}}{\pgfqpoint{2.943135in}{6.177352in}}%
\pgfpathcurveto{\pgfqpoint{2.937311in}{6.177352in}}{\pgfqpoint{2.931725in}{6.175038in}}{\pgfqpoint{2.927607in}{6.170920in}}%
\pgfpathcurveto{\pgfqpoint{2.923488in}{6.166802in}}{\pgfqpoint{2.921175in}{6.161216in}}{\pgfqpoint{2.921175in}{6.155392in}}%
\pgfpathcurveto{\pgfqpoint{2.921175in}{6.149568in}}{\pgfqpoint{2.923488in}{6.143982in}}{\pgfqpoint{2.927607in}{6.139863in}}%
\pgfpathcurveto{\pgfqpoint{2.931725in}{6.135745in}}{\pgfqpoint{2.937311in}{6.133431in}}{\pgfqpoint{2.943135in}{6.133431in}}%
\pgfpathlineto{\pgfqpoint{2.943135in}{6.133431in}}%
\pgfpathclose%
\pgfusepath{stroke,fill}%
\end{pgfscope}%
\begin{pgfscope}%
\pgfpathrectangle{\pgfqpoint{1.000000in}{0.979904in}}{\pgfqpoint{6.200000in}{5.960192in}}%
\pgfusepath{clip}%
\pgfsetbuttcap%
\pgfsetroundjoin%
\definecolor{currentfill}{rgb}{0.200000,0.200000,0.800000}%
\pgfsetfillcolor{currentfill}%
\pgfsetlinewidth{1.003750pt}%
\definecolor{currentstroke}{rgb}{0.200000,0.200000,0.800000}%
\pgfsetstrokecolor{currentstroke}%
\pgfsetdash{}{0pt}%
\pgfpathmoveto{\pgfqpoint{2.927498in}{6.078063in}}%
\pgfpathcurveto{\pgfqpoint{2.933322in}{6.078063in}}{\pgfqpoint{2.938908in}{6.080377in}}{\pgfqpoint{2.943026in}{6.084495in}}%
\pgfpathcurveto{\pgfqpoint{2.947144in}{6.088613in}}{\pgfqpoint{2.949458in}{6.094199in}}{\pgfqpoint{2.949458in}{6.100023in}}%
\pgfpathcurveto{\pgfqpoint{2.949458in}{6.105847in}}{\pgfqpoint{2.947144in}{6.111433in}}{\pgfqpoint{2.943026in}{6.115552in}}%
\pgfpathcurveto{\pgfqpoint{2.938908in}{6.119670in}}{\pgfqpoint{2.933322in}{6.121984in}}{\pgfqpoint{2.927498in}{6.121984in}}%
\pgfpathcurveto{\pgfqpoint{2.921674in}{6.121984in}}{\pgfqpoint{2.916088in}{6.119670in}}{\pgfqpoint{2.911969in}{6.115552in}}%
\pgfpathcurveto{\pgfqpoint{2.907851in}{6.111433in}}{\pgfqpoint{2.905537in}{6.105847in}}{\pgfqpoint{2.905537in}{6.100023in}}%
\pgfpathcurveto{\pgfqpoint{2.905537in}{6.094199in}}{\pgfqpoint{2.907851in}{6.088613in}}{\pgfqpoint{2.911969in}{6.084495in}}%
\pgfpathcurveto{\pgfqpoint{2.916088in}{6.080377in}}{\pgfqpoint{2.921674in}{6.078063in}}{\pgfqpoint{2.927498in}{6.078063in}}%
\pgfpathlineto{\pgfqpoint{2.927498in}{6.078063in}}%
\pgfpathclose%
\pgfusepath{stroke,fill}%
\end{pgfscope}%
\begin{pgfscope}%
\pgfpathrectangle{\pgfqpoint{1.000000in}{0.979904in}}{\pgfqpoint{6.200000in}{5.960192in}}%
\pgfusepath{clip}%
\pgfsetbuttcap%
\pgfsetroundjoin%
\definecolor{currentfill}{rgb}{0.200000,0.200000,0.800000}%
\pgfsetfillcolor{currentfill}%
\pgfsetlinewidth{1.003750pt}%
\definecolor{currentstroke}{rgb}{0.200000,0.200000,0.800000}%
\pgfsetstrokecolor{currentstroke}%
\pgfsetdash{}{0pt}%
\pgfpathmoveto{\pgfqpoint{2.842131in}{6.054800in}}%
\pgfpathcurveto{\pgfqpoint{2.847955in}{6.054800in}}{\pgfqpoint{2.853541in}{6.057114in}}{\pgfqpoint{2.857659in}{6.061232in}}%
\pgfpathcurveto{\pgfqpoint{2.861777in}{6.065350in}}{\pgfqpoint{2.864091in}{6.070937in}}{\pgfqpoint{2.864091in}{6.076760in}}%
\pgfpathcurveto{\pgfqpoint{2.864091in}{6.082584in}}{\pgfqpoint{2.861777in}{6.088171in}}{\pgfqpoint{2.857659in}{6.092289in}}%
\pgfpathcurveto{\pgfqpoint{2.853541in}{6.096407in}}{\pgfqpoint{2.847955in}{6.098721in}}{\pgfqpoint{2.842131in}{6.098721in}}%
\pgfpathcurveto{\pgfqpoint{2.836307in}{6.098721in}}{\pgfqpoint{2.830721in}{6.096407in}}{\pgfqpoint{2.826602in}{6.092289in}}%
\pgfpathcurveto{\pgfqpoint{2.822484in}{6.088171in}}{\pgfqpoint{2.820170in}{6.082584in}}{\pgfqpoint{2.820170in}{6.076760in}}%
\pgfpathcurveto{\pgfqpoint{2.820170in}{6.070937in}}{\pgfqpoint{2.822484in}{6.065350in}}{\pgfqpoint{2.826602in}{6.061232in}}%
\pgfpathcurveto{\pgfqpoint{2.830721in}{6.057114in}}{\pgfqpoint{2.836307in}{6.054800in}}{\pgfqpoint{2.842131in}{6.054800in}}%
\pgfpathlineto{\pgfqpoint{2.842131in}{6.054800in}}%
\pgfpathclose%
\pgfusepath{stroke,fill}%
\end{pgfscope}%
\begin{pgfscope}%
\pgfpathrectangle{\pgfqpoint{1.000000in}{0.979904in}}{\pgfqpoint{6.200000in}{5.960192in}}%
\pgfusepath{clip}%
\pgfsetbuttcap%
\pgfsetroundjoin%
\definecolor{currentfill}{rgb}{0.200000,0.200000,0.800000}%
\pgfsetfillcolor{currentfill}%
\pgfsetlinewidth{1.003750pt}%
\definecolor{currentstroke}{rgb}{0.200000,0.200000,0.800000}%
\pgfsetstrokecolor{currentstroke}%
\pgfsetdash{}{0pt}%
\pgfpathmoveto{\pgfqpoint{2.797576in}{6.007385in}}%
\pgfpathcurveto{\pgfqpoint{2.803399in}{6.007385in}}{\pgfqpoint{2.808986in}{6.009699in}}{\pgfqpoint{2.813104in}{6.013817in}}%
\pgfpathcurveto{\pgfqpoint{2.817222in}{6.017935in}}{\pgfqpoint{2.819536in}{6.023521in}}{\pgfqpoint{2.819536in}{6.029345in}}%
\pgfpathcurveto{\pgfqpoint{2.819536in}{6.035169in}}{\pgfqpoint{2.817222in}{6.040756in}}{\pgfqpoint{2.813104in}{6.044874in}}%
\pgfpathcurveto{\pgfqpoint{2.808986in}{6.048992in}}{\pgfqpoint{2.803399in}{6.051306in}}{\pgfqpoint{2.797576in}{6.051306in}}%
\pgfpathcurveto{\pgfqpoint{2.791752in}{6.051306in}}{\pgfqpoint{2.786165in}{6.048992in}}{\pgfqpoint{2.782047in}{6.044874in}}%
\pgfpathcurveto{\pgfqpoint{2.777929in}{6.040756in}}{\pgfqpoint{2.775615in}{6.035169in}}{\pgfqpoint{2.775615in}{6.029345in}}%
\pgfpathcurveto{\pgfqpoint{2.775615in}{6.023521in}}{\pgfqpoint{2.777929in}{6.017935in}}{\pgfqpoint{2.782047in}{6.013817in}}%
\pgfpathcurveto{\pgfqpoint{2.786165in}{6.009699in}}{\pgfqpoint{2.791752in}{6.007385in}}{\pgfqpoint{2.797576in}{6.007385in}}%
\pgfpathlineto{\pgfqpoint{2.797576in}{6.007385in}}%
\pgfpathclose%
\pgfusepath{stroke,fill}%
\end{pgfscope}%
\begin{pgfscope}%
\pgfpathrectangle{\pgfqpoint{1.000000in}{0.979904in}}{\pgfqpoint{6.200000in}{5.960192in}}%
\pgfusepath{clip}%
\pgfsetbuttcap%
\pgfsetroundjoin%
\definecolor{currentfill}{rgb}{0.200000,0.200000,0.800000}%
\pgfsetfillcolor{currentfill}%
\pgfsetlinewidth{1.003750pt}%
\definecolor{currentstroke}{rgb}{0.200000,0.200000,0.800000}%
\pgfsetstrokecolor{currentstroke}%
\pgfsetdash{}{0pt}%
\pgfpathmoveto{\pgfqpoint{2.762879in}{5.953684in}}%
\pgfpathcurveto{\pgfqpoint{2.768703in}{5.953684in}}{\pgfqpoint{2.774289in}{5.955998in}}{\pgfqpoint{2.778408in}{5.960116in}}%
\pgfpathcurveto{\pgfqpoint{2.782526in}{5.964234in}}{\pgfqpoint{2.784840in}{5.969820in}}{\pgfqpoint{2.784840in}{5.975644in}}%
\pgfpathcurveto{\pgfqpoint{2.784840in}{5.981468in}}{\pgfqpoint{2.782526in}{5.987054in}}{\pgfqpoint{2.778408in}{5.991173in}}%
\pgfpathcurveto{\pgfqpoint{2.774289in}{5.995291in}}{\pgfqpoint{2.768703in}{5.997605in}}{\pgfqpoint{2.762879in}{5.997605in}}%
\pgfpathcurveto{\pgfqpoint{2.757055in}{5.997605in}}{\pgfqpoint{2.751469in}{5.995291in}}{\pgfqpoint{2.747351in}{5.991173in}}%
\pgfpathcurveto{\pgfqpoint{2.743233in}{5.987054in}}{\pgfqpoint{2.740919in}{5.981468in}}{\pgfqpoint{2.740919in}{5.975644in}}%
\pgfpathcurveto{\pgfqpoint{2.740919in}{5.969820in}}{\pgfqpoint{2.743233in}{5.964234in}}{\pgfqpoint{2.747351in}{5.960116in}}%
\pgfpathcurveto{\pgfqpoint{2.751469in}{5.955998in}}{\pgfqpoint{2.757055in}{5.953684in}}{\pgfqpoint{2.762879in}{5.953684in}}%
\pgfpathlineto{\pgfqpoint{2.762879in}{5.953684in}}%
\pgfpathclose%
\pgfusepath{stroke,fill}%
\end{pgfscope}%
\begin{pgfscope}%
\pgfpathrectangle{\pgfqpoint{1.000000in}{0.979904in}}{\pgfqpoint{6.200000in}{5.960192in}}%
\pgfusepath{clip}%
\pgfsetbuttcap%
\pgfsetroundjoin%
\definecolor{currentfill}{rgb}{0.200000,0.200000,0.800000}%
\pgfsetfillcolor{currentfill}%
\pgfsetlinewidth{1.003750pt}%
\definecolor{currentstroke}{rgb}{0.200000,0.200000,0.800000}%
\pgfsetstrokecolor{currentstroke}%
\pgfsetdash{}{0pt}%
\pgfpathmoveto{\pgfqpoint{2.746877in}{5.893399in}}%
\pgfpathcurveto{\pgfqpoint{2.752701in}{5.893399in}}{\pgfqpoint{2.758287in}{5.895713in}}{\pgfqpoint{2.762405in}{5.899831in}}%
\pgfpathcurveto{\pgfqpoint{2.766523in}{5.903949in}}{\pgfqpoint{2.768837in}{5.909535in}}{\pgfqpoint{2.768837in}{5.915359in}}%
\pgfpathcurveto{\pgfqpoint{2.768837in}{5.921183in}}{\pgfqpoint{2.766523in}{5.926770in}}{\pgfqpoint{2.762405in}{5.930888in}}%
\pgfpathcurveto{\pgfqpoint{2.758287in}{5.935006in}}{\pgfqpoint{2.752701in}{5.937320in}}{\pgfqpoint{2.746877in}{5.937320in}}%
\pgfpathcurveto{\pgfqpoint{2.741053in}{5.937320in}}{\pgfqpoint{2.735467in}{5.935006in}}{\pgfqpoint{2.731349in}{5.930888in}}%
\pgfpathcurveto{\pgfqpoint{2.727231in}{5.926770in}}{\pgfqpoint{2.724917in}{5.921183in}}{\pgfqpoint{2.724917in}{5.915359in}}%
\pgfpathcurveto{\pgfqpoint{2.724917in}{5.909535in}}{\pgfqpoint{2.727231in}{5.903949in}}{\pgfqpoint{2.731349in}{5.899831in}}%
\pgfpathcurveto{\pgfqpoint{2.735467in}{5.895713in}}{\pgfqpoint{2.741053in}{5.893399in}}{\pgfqpoint{2.746877in}{5.893399in}}%
\pgfpathlineto{\pgfqpoint{2.746877in}{5.893399in}}%
\pgfpathclose%
\pgfusepath{stroke,fill}%
\end{pgfscope}%
\begin{pgfscope}%
\pgfpathrectangle{\pgfqpoint{1.000000in}{0.979904in}}{\pgfqpoint{6.200000in}{5.960192in}}%
\pgfusepath{clip}%
\pgfsetbuttcap%
\pgfsetroundjoin%
\definecolor{currentfill}{rgb}{0.200000,0.200000,0.800000}%
\pgfsetfillcolor{currentfill}%
\pgfsetlinewidth{1.003750pt}%
\definecolor{currentstroke}{rgb}{0.200000,0.200000,0.800000}%
\pgfsetstrokecolor{currentstroke}%
\pgfsetdash{}{0pt}%
\pgfpathmoveto{\pgfqpoint{2.742371in}{5.830976in}}%
\pgfpathcurveto{\pgfqpoint{2.748195in}{5.830976in}}{\pgfqpoint{2.753781in}{5.833289in}}{\pgfqpoint{2.757899in}{5.837408in}}%
\pgfpathcurveto{\pgfqpoint{2.762017in}{5.841526in}}{\pgfqpoint{2.764331in}{5.847112in}}{\pgfqpoint{2.764331in}{5.852936in}}%
\pgfpathcurveto{\pgfqpoint{2.764331in}{5.858760in}}{\pgfqpoint{2.762017in}{5.864346in}}{\pgfqpoint{2.757899in}{5.868464in}}%
\pgfpathcurveto{\pgfqpoint{2.753781in}{5.872582in}}{\pgfqpoint{2.748195in}{5.874896in}}{\pgfqpoint{2.742371in}{5.874896in}}%
\pgfpathcurveto{\pgfqpoint{2.736547in}{5.874896in}}{\pgfqpoint{2.730961in}{5.872582in}}{\pgfqpoint{2.726842in}{5.868464in}}%
\pgfpathcurveto{\pgfqpoint{2.722724in}{5.864346in}}{\pgfqpoint{2.720410in}{5.858760in}}{\pgfqpoint{2.720410in}{5.852936in}}%
\pgfpathcurveto{\pgfqpoint{2.720410in}{5.847112in}}{\pgfqpoint{2.722724in}{5.841526in}}{\pgfqpoint{2.726842in}{5.837408in}}%
\pgfpathcurveto{\pgfqpoint{2.730961in}{5.833289in}}{\pgfqpoint{2.736547in}{5.830976in}}{\pgfqpoint{2.742371in}{5.830976in}}%
\pgfpathlineto{\pgfqpoint{2.742371in}{5.830976in}}%
\pgfpathclose%
\pgfusepath{stroke,fill}%
\end{pgfscope}%
\begin{pgfscope}%
\pgfpathrectangle{\pgfqpoint{1.000000in}{0.979904in}}{\pgfqpoint{6.200000in}{5.960192in}}%
\pgfusepath{clip}%
\pgfsetbuttcap%
\pgfsetroundjoin%
\definecolor{currentfill}{rgb}{0.200000,0.200000,0.800000}%
\pgfsetfillcolor{currentfill}%
\pgfsetlinewidth{1.003750pt}%
\definecolor{currentstroke}{rgb}{0.200000,0.200000,0.800000}%
\pgfsetstrokecolor{currentstroke}%
\pgfsetdash{}{0pt}%
\pgfpathmoveto{\pgfqpoint{2.785351in}{5.764707in}}%
\pgfpathcurveto{\pgfqpoint{2.791175in}{5.764707in}}{\pgfqpoint{2.796761in}{5.767021in}}{\pgfqpoint{2.800879in}{5.771139in}}%
\pgfpathcurveto{\pgfqpoint{2.804997in}{5.775258in}}{\pgfqpoint{2.807311in}{5.780844in}}{\pgfqpoint{2.807311in}{5.786668in}}%
\pgfpathcurveto{\pgfqpoint{2.807311in}{5.792492in}}{\pgfqpoint{2.804997in}{5.798078in}}{\pgfqpoint{2.800879in}{5.802196in}}%
\pgfpathcurveto{\pgfqpoint{2.796761in}{5.806314in}}{\pgfqpoint{2.791175in}{5.808628in}}{\pgfqpoint{2.785351in}{5.808628in}}%
\pgfpathcurveto{\pgfqpoint{2.779527in}{5.808628in}}{\pgfqpoint{2.773941in}{5.806314in}}{\pgfqpoint{2.769823in}{5.802196in}}%
\pgfpathcurveto{\pgfqpoint{2.765704in}{5.798078in}}{\pgfqpoint{2.763391in}{5.792492in}}{\pgfqpoint{2.763391in}{5.786668in}}%
\pgfpathcurveto{\pgfqpoint{2.763391in}{5.780844in}}{\pgfqpoint{2.765704in}{5.775258in}}{\pgfqpoint{2.769823in}{5.771139in}}%
\pgfpathcurveto{\pgfqpoint{2.773941in}{5.767021in}}{\pgfqpoint{2.779527in}{5.764707in}}{\pgfqpoint{2.785351in}{5.764707in}}%
\pgfpathlineto{\pgfqpoint{2.785351in}{5.764707in}}%
\pgfpathclose%
\pgfusepath{stroke,fill}%
\end{pgfscope}%
\begin{pgfscope}%
\pgfpathrectangle{\pgfqpoint{1.000000in}{0.979904in}}{\pgfqpoint{6.200000in}{5.960192in}}%
\pgfusepath{clip}%
\pgfsetbuttcap%
\pgfsetroundjoin%
\definecolor{currentfill}{rgb}{0.200000,0.200000,0.800000}%
\pgfsetfillcolor{currentfill}%
\pgfsetlinewidth{1.003750pt}%
\definecolor{currentstroke}{rgb}{0.200000,0.200000,0.800000}%
\pgfsetstrokecolor{currentstroke}%
\pgfsetdash{}{0pt}%
\pgfpathmoveto{\pgfqpoint{2.803706in}{5.705450in}}%
\pgfpathcurveto{\pgfqpoint{2.809530in}{5.705450in}}{\pgfqpoint{2.815117in}{5.707764in}}{\pgfqpoint{2.819235in}{5.711882in}}%
\pgfpathcurveto{\pgfqpoint{2.823353in}{5.716001in}}{\pgfqpoint{2.825667in}{5.721587in}}{\pgfqpoint{2.825667in}{5.727411in}}%
\pgfpathcurveto{\pgfqpoint{2.825667in}{5.733235in}}{\pgfqpoint{2.823353in}{5.738821in}}{\pgfqpoint{2.819235in}{5.742939in}}%
\pgfpathcurveto{\pgfqpoint{2.815117in}{5.747057in}}{\pgfqpoint{2.809530in}{5.749371in}}{\pgfqpoint{2.803706in}{5.749371in}}%
\pgfpathcurveto{\pgfqpoint{2.797883in}{5.749371in}}{\pgfqpoint{2.792296in}{5.747057in}}{\pgfqpoint{2.788178in}{5.742939in}}%
\pgfpathcurveto{\pgfqpoint{2.784060in}{5.738821in}}{\pgfqpoint{2.781746in}{5.733235in}}{\pgfqpoint{2.781746in}{5.727411in}}%
\pgfpathcurveto{\pgfqpoint{2.781746in}{5.721587in}}{\pgfqpoint{2.784060in}{5.716001in}}{\pgfqpoint{2.788178in}{5.711882in}}%
\pgfpathcurveto{\pgfqpoint{2.792296in}{5.707764in}}{\pgfqpoint{2.797883in}{5.705450in}}{\pgfqpoint{2.803706in}{5.705450in}}%
\pgfpathlineto{\pgfqpoint{2.803706in}{5.705450in}}%
\pgfpathclose%
\pgfusepath{stroke,fill}%
\end{pgfscope}%
\begin{pgfscope}%
\pgfpathrectangle{\pgfqpoint{1.000000in}{0.979904in}}{\pgfqpoint{6.200000in}{5.960192in}}%
\pgfusepath{clip}%
\pgfsetbuttcap%
\pgfsetroundjoin%
\definecolor{currentfill}{rgb}{0.200000,0.200000,0.800000}%
\pgfsetfillcolor{currentfill}%
\pgfsetlinewidth{1.003750pt}%
\definecolor{currentstroke}{rgb}{0.200000,0.200000,0.800000}%
\pgfsetstrokecolor{currentstroke}%
\pgfsetdash{}{0pt}%
\pgfpathmoveto{\pgfqpoint{2.850621in}{5.649667in}}%
\pgfpathcurveto{\pgfqpoint{2.856445in}{5.649667in}}{\pgfqpoint{2.862031in}{5.651981in}}{\pgfqpoint{2.866149in}{5.656099in}}%
\pgfpathcurveto{\pgfqpoint{2.870267in}{5.660217in}}{\pgfqpoint{2.872581in}{5.665803in}}{\pgfqpoint{2.872581in}{5.671627in}}%
\pgfpathcurveto{\pgfqpoint{2.872581in}{5.677451in}}{\pgfqpoint{2.870267in}{5.683037in}}{\pgfqpoint{2.866149in}{5.687155in}}%
\pgfpathcurveto{\pgfqpoint{2.862031in}{5.691274in}}{\pgfqpoint{2.856445in}{5.693587in}}{\pgfqpoint{2.850621in}{5.693587in}}%
\pgfpathcurveto{\pgfqpoint{2.844797in}{5.693587in}}{\pgfqpoint{2.839211in}{5.691274in}}{\pgfqpoint{2.835093in}{5.687155in}}%
\pgfpathcurveto{\pgfqpoint{2.830974in}{5.683037in}}{\pgfqpoint{2.828661in}{5.677451in}}{\pgfqpoint{2.828661in}{5.671627in}}%
\pgfpathcurveto{\pgfqpoint{2.828661in}{5.665803in}}{\pgfqpoint{2.830974in}{5.660217in}}{\pgfqpoint{2.835093in}{5.656099in}}%
\pgfpathcurveto{\pgfqpoint{2.839211in}{5.651981in}}{\pgfqpoint{2.844797in}{5.649667in}}{\pgfqpoint{2.850621in}{5.649667in}}%
\pgfpathlineto{\pgfqpoint{2.850621in}{5.649667in}}%
\pgfpathclose%
\pgfusepath{stroke,fill}%
\end{pgfscope}%
\begin{pgfscope}%
\pgfpathrectangle{\pgfqpoint{1.000000in}{0.979904in}}{\pgfqpoint{6.200000in}{5.960192in}}%
\pgfusepath{clip}%
\pgfsetbuttcap%
\pgfsetroundjoin%
\definecolor{currentfill}{rgb}{0.200000,0.200000,0.800000}%
\pgfsetfillcolor{currentfill}%
\pgfsetlinewidth{1.003750pt}%
\definecolor{currentstroke}{rgb}{0.200000,0.200000,0.800000}%
\pgfsetstrokecolor{currentstroke}%
\pgfsetdash{}{0pt}%
\pgfpathmoveto{\pgfqpoint{2.716330in}{5.582330in}}%
\pgfpathcurveto{\pgfqpoint{2.722154in}{5.582330in}}{\pgfqpoint{2.727740in}{5.584644in}}{\pgfqpoint{2.731858in}{5.588762in}}%
\pgfpathcurveto{\pgfqpoint{2.735976in}{5.592880in}}{\pgfqpoint{2.738290in}{5.598466in}}{\pgfqpoint{2.738290in}{5.604290in}}%
\pgfpathcurveto{\pgfqpoint{2.738290in}{5.610114in}}{\pgfqpoint{2.735976in}{5.615700in}}{\pgfqpoint{2.731858in}{5.619818in}}%
\pgfpathcurveto{\pgfqpoint{2.727740in}{5.623937in}}{\pgfqpoint{2.722154in}{5.626250in}}{\pgfqpoint{2.716330in}{5.626250in}}%
\pgfpathcurveto{\pgfqpoint{2.710506in}{5.626250in}}{\pgfqpoint{2.704920in}{5.623937in}}{\pgfqpoint{2.700801in}{5.619818in}}%
\pgfpathcurveto{\pgfqpoint{2.696683in}{5.615700in}}{\pgfqpoint{2.694369in}{5.610114in}}{\pgfqpoint{2.694369in}{5.604290in}}%
\pgfpathcurveto{\pgfqpoint{2.694369in}{5.598466in}}{\pgfqpoint{2.696683in}{5.592880in}}{\pgfqpoint{2.700801in}{5.588762in}}%
\pgfpathcurveto{\pgfqpoint{2.704920in}{5.584644in}}{\pgfqpoint{2.710506in}{5.582330in}}{\pgfqpoint{2.716330in}{5.582330in}}%
\pgfpathlineto{\pgfqpoint{2.716330in}{5.582330in}}%
\pgfpathclose%
\pgfusepath{stroke,fill}%
\end{pgfscope}%
\begin{pgfscope}%
\pgfpathrectangle{\pgfqpoint{1.000000in}{0.979904in}}{\pgfqpoint{6.200000in}{5.960192in}}%
\pgfusepath{clip}%
\pgfsetbuttcap%
\pgfsetroundjoin%
\definecolor{currentfill}{rgb}{0.200000,0.200000,0.800000}%
\pgfsetfillcolor{currentfill}%
\pgfsetlinewidth{1.003750pt}%
\definecolor{currentstroke}{rgb}{0.200000,0.200000,0.800000}%
\pgfsetstrokecolor{currentstroke}%
\pgfsetdash{}{0pt}%
\pgfpathmoveto{\pgfqpoint{2.791237in}{5.530472in}}%
\pgfpathcurveto{\pgfqpoint{2.797060in}{5.530472in}}{\pgfqpoint{2.802647in}{5.532785in}}{\pgfqpoint{2.806765in}{5.536904in}}%
\pgfpathcurveto{\pgfqpoint{2.810883in}{5.541022in}}{\pgfqpoint{2.813197in}{5.546608in}}{\pgfqpoint{2.813197in}{5.552432in}}%
\pgfpathcurveto{\pgfqpoint{2.813197in}{5.558256in}}{\pgfqpoint{2.810883in}{5.563842in}}{\pgfqpoint{2.806765in}{5.567960in}}%
\pgfpathcurveto{\pgfqpoint{2.802647in}{5.572078in}}{\pgfqpoint{2.797060in}{5.574392in}}{\pgfqpoint{2.791237in}{5.574392in}}%
\pgfpathcurveto{\pgfqpoint{2.785413in}{5.574392in}}{\pgfqpoint{2.779826in}{5.572078in}}{\pgfqpoint{2.775708in}{5.567960in}}%
\pgfpathcurveto{\pgfqpoint{2.771590in}{5.563842in}}{\pgfqpoint{2.769276in}{5.558256in}}{\pgfqpoint{2.769276in}{5.552432in}}%
\pgfpathcurveto{\pgfqpoint{2.769276in}{5.546608in}}{\pgfqpoint{2.771590in}{5.541022in}}{\pgfqpoint{2.775708in}{5.536904in}}%
\pgfpathcurveto{\pgfqpoint{2.779826in}{5.532785in}}{\pgfqpoint{2.785413in}{5.530472in}}{\pgfqpoint{2.791237in}{5.530472in}}%
\pgfpathlineto{\pgfqpoint{2.791237in}{5.530472in}}%
\pgfpathclose%
\pgfusepath{stroke,fill}%
\end{pgfscope}%
\begin{pgfscope}%
\pgfpathrectangle{\pgfqpoint{1.000000in}{0.979904in}}{\pgfqpoint{6.200000in}{5.960192in}}%
\pgfusepath{clip}%
\pgfsetbuttcap%
\pgfsetroundjoin%
\definecolor{currentfill}{rgb}{0.200000,0.200000,0.800000}%
\pgfsetfillcolor{currentfill}%
\pgfsetlinewidth{1.003750pt}%
\definecolor{currentstroke}{rgb}{0.200000,0.200000,0.800000}%
\pgfsetstrokecolor{currentstroke}%
\pgfsetdash{}{0pt}%
\pgfpathmoveto{\pgfqpoint{2.814400in}{5.475480in}}%
\pgfpathcurveto{\pgfqpoint{2.820224in}{5.475480in}}{\pgfqpoint{2.825810in}{5.477794in}}{\pgfqpoint{2.829928in}{5.481912in}}%
\pgfpathcurveto{\pgfqpoint{2.834046in}{5.486030in}}{\pgfqpoint{2.836360in}{5.491616in}}{\pgfqpoint{2.836360in}{5.497440in}}%
\pgfpathcurveto{\pgfqpoint{2.836360in}{5.503264in}}{\pgfqpoint{2.834046in}{5.508850in}}{\pgfqpoint{2.829928in}{5.512968in}}%
\pgfpathcurveto{\pgfqpoint{2.825810in}{5.517086in}}{\pgfqpoint{2.820224in}{5.519400in}}{\pgfqpoint{2.814400in}{5.519400in}}%
\pgfpathcurveto{\pgfqpoint{2.808576in}{5.519400in}}{\pgfqpoint{2.802990in}{5.517086in}}{\pgfqpoint{2.798872in}{5.512968in}}%
\pgfpathcurveto{\pgfqpoint{2.794754in}{5.508850in}}{\pgfqpoint{2.792440in}{5.503264in}}{\pgfqpoint{2.792440in}{5.497440in}}%
\pgfpathcurveto{\pgfqpoint{2.792440in}{5.491616in}}{\pgfqpoint{2.794754in}{5.486030in}}{\pgfqpoint{2.798872in}{5.481912in}}%
\pgfpathcurveto{\pgfqpoint{2.802990in}{5.477794in}}{\pgfqpoint{2.808576in}{5.475480in}}{\pgfqpoint{2.814400in}{5.475480in}}%
\pgfpathlineto{\pgfqpoint{2.814400in}{5.475480in}}%
\pgfpathclose%
\pgfusepath{stroke,fill}%
\end{pgfscope}%
\begin{pgfscope}%
\pgfpathrectangle{\pgfqpoint{1.000000in}{0.979904in}}{\pgfqpoint{6.200000in}{5.960192in}}%
\pgfusepath{clip}%
\pgfsetbuttcap%
\pgfsetroundjoin%
\definecolor{currentfill}{rgb}{0.200000,0.200000,0.800000}%
\pgfsetfillcolor{currentfill}%
\pgfsetlinewidth{1.003750pt}%
\definecolor{currentstroke}{rgb}{0.200000,0.200000,0.800000}%
\pgfsetstrokecolor{currentstroke}%
\pgfsetdash{}{0pt}%
\pgfpathmoveto{\pgfqpoint{2.748016in}{5.395580in}}%
\pgfpathcurveto{\pgfqpoint{2.753840in}{5.395580in}}{\pgfqpoint{2.759426in}{5.397894in}}{\pgfqpoint{2.763544in}{5.402012in}}%
\pgfpathcurveto{\pgfqpoint{2.767662in}{5.406130in}}{\pgfqpoint{2.769976in}{5.411716in}}{\pgfqpoint{2.769976in}{5.417540in}}%
\pgfpathcurveto{\pgfqpoint{2.769976in}{5.423364in}}{\pgfqpoint{2.767662in}{5.428950in}}{\pgfqpoint{2.763544in}{5.433068in}}%
\pgfpathcurveto{\pgfqpoint{2.759426in}{5.437186in}}{\pgfqpoint{2.753840in}{5.439500in}}{\pgfqpoint{2.748016in}{5.439500in}}%
\pgfpathcurveto{\pgfqpoint{2.742192in}{5.439500in}}{\pgfqpoint{2.736606in}{5.437186in}}{\pgfqpoint{2.732488in}{5.433068in}}%
\pgfpathcurveto{\pgfqpoint{2.728370in}{5.428950in}}{\pgfqpoint{2.726056in}{5.423364in}}{\pgfqpoint{2.726056in}{5.417540in}}%
\pgfpathcurveto{\pgfqpoint{2.726056in}{5.411716in}}{\pgfqpoint{2.728370in}{5.406130in}}{\pgfqpoint{2.732488in}{5.402012in}}%
\pgfpathcurveto{\pgfqpoint{2.736606in}{5.397894in}}{\pgfqpoint{2.742192in}{5.395580in}}{\pgfqpoint{2.748016in}{5.395580in}}%
\pgfpathlineto{\pgfqpoint{2.748016in}{5.395580in}}%
\pgfpathclose%
\pgfusepath{stroke,fill}%
\end{pgfscope}%
\begin{pgfscope}%
\pgfpathrectangle{\pgfqpoint{1.000000in}{0.979904in}}{\pgfqpoint{6.200000in}{5.960192in}}%
\pgfusepath{clip}%
\pgfsetbuttcap%
\pgfsetroundjoin%
\definecolor{currentfill}{rgb}{0.200000,0.200000,0.800000}%
\pgfsetfillcolor{currentfill}%
\pgfsetlinewidth{1.003750pt}%
\definecolor{currentstroke}{rgb}{0.200000,0.200000,0.800000}%
\pgfsetstrokecolor{currentstroke}%
\pgfsetdash{}{0pt}%
\pgfpathmoveto{\pgfqpoint{2.871380in}{5.373106in}}%
\pgfpathcurveto{\pgfqpoint{2.877204in}{5.373106in}}{\pgfqpoint{2.882790in}{5.375420in}}{\pgfqpoint{2.886909in}{5.379538in}}%
\pgfpathcurveto{\pgfqpoint{2.891027in}{5.383656in}}{\pgfqpoint{2.893341in}{5.389242in}}{\pgfqpoint{2.893341in}{5.395066in}}%
\pgfpathcurveto{\pgfqpoint{2.893341in}{5.400890in}}{\pgfqpoint{2.891027in}{5.406476in}}{\pgfqpoint{2.886909in}{5.410594in}}%
\pgfpathcurveto{\pgfqpoint{2.882790in}{5.414712in}}{\pgfqpoint{2.877204in}{5.417026in}}{\pgfqpoint{2.871380in}{5.417026in}}%
\pgfpathcurveto{\pgfqpoint{2.865556in}{5.417026in}}{\pgfqpoint{2.859970in}{5.414712in}}{\pgfqpoint{2.855852in}{5.410594in}}%
\pgfpathcurveto{\pgfqpoint{2.851734in}{5.406476in}}{\pgfqpoint{2.849420in}{5.400890in}}{\pgfqpoint{2.849420in}{5.395066in}}%
\pgfpathcurveto{\pgfqpoint{2.849420in}{5.389242in}}{\pgfqpoint{2.851734in}{5.383656in}}{\pgfqpoint{2.855852in}{5.379538in}}%
\pgfpathcurveto{\pgfqpoint{2.859970in}{5.375420in}}{\pgfqpoint{2.865556in}{5.373106in}}{\pgfqpoint{2.871380in}{5.373106in}}%
\pgfpathlineto{\pgfqpoint{2.871380in}{5.373106in}}%
\pgfpathclose%
\pgfusepath{stroke,fill}%
\end{pgfscope}%
\begin{pgfscope}%
\pgfpathrectangle{\pgfqpoint{1.000000in}{0.979904in}}{\pgfqpoint{6.200000in}{5.960192in}}%
\pgfusepath{clip}%
\pgfsetbuttcap%
\pgfsetroundjoin%
\definecolor{currentfill}{rgb}{0.200000,0.200000,0.800000}%
\pgfsetfillcolor{currentfill}%
\pgfsetlinewidth{1.003750pt}%
\definecolor{currentstroke}{rgb}{0.200000,0.200000,0.800000}%
\pgfsetstrokecolor{currentstroke}%
\pgfsetdash{}{0pt}%
\pgfpathmoveto{\pgfqpoint{2.808089in}{5.283930in}}%
\pgfpathcurveto{\pgfqpoint{2.813913in}{5.283930in}}{\pgfqpoint{2.819499in}{5.286244in}}{\pgfqpoint{2.823617in}{5.290362in}}%
\pgfpathcurveto{\pgfqpoint{2.827735in}{5.294480in}}{\pgfqpoint{2.830049in}{5.300066in}}{\pgfqpoint{2.830049in}{5.305890in}}%
\pgfpathcurveto{\pgfqpoint{2.830049in}{5.311714in}}{\pgfqpoint{2.827735in}{5.317300in}}{\pgfqpoint{2.823617in}{5.321418in}}%
\pgfpathcurveto{\pgfqpoint{2.819499in}{5.325536in}}{\pgfqpoint{2.813913in}{5.327850in}}{\pgfqpoint{2.808089in}{5.327850in}}%
\pgfpathcurveto{\pgfqpoint{2.802265in}{5.327850in}}{\pgfqpoint{2.796679in}{5.325536in}}{\pgfqpoint{2.792561in}{5.321418in}}%
\pgfpathcurveto{\pgfqpoint{2.788443in}{5.317300in}}{\pgfqpoint{2.786129in}{5.311714in}}{\pgfqpoint{2.786129in}{5.305890in}}%
\pgfpathcurveto{\pgfqpoint{2.786129in}{5.300066in}}{\pgfqpoint{2.788443in}{5.294480in}}{\pgfqpoint{2.792561in}{5.290362in}}%
\pgfpathcurveto{\pgfqpoint{2.796679in}{5.286244in}}{\pgfqpoint{2.802265in}{5.283930in}}{\pgfqpoint{2.808089in}{5.283930in}}%
\pgfpathlineto{\pgfqpoint{2.808089in}{5.283930in}}%
\pgfpathclose%
\pgfusepath{stroke,fill}%
\end{pgfscope}%
\begin{pgfscope}%
\pgfpathrectangle{\pgfqpoint{1.000000in}{0.979904in}}{\pgfqpoint{6.200000in}{5.960192in}}%
\pgfusepath{clip}%
\pgfsetbuttcap%
\pgfsetroundjoin%
\definecolor{currentfill}{rgb}{0.200000,0.200000,0.800000}%
\pgfsetfillcolor{currentfill}%
\pgfsetlinewidth{1.003750pt}%
\definecolor{currentstroke}{rgb}{0.200000,0.200000,0.800000}%
\pgfsetstrokecolor{currentstroke}%
\pgfsetdash{}{0pt}%
\pgfpathmoveto{\pgfqpoint{2.860197in}{5.240869in}}%
\pgfpathcurveto{\pgfqpoint{2.866021in}{5.240869in}}{\pgfqpoint{2.871607in}{5.243183in}}{\pgfqpoint{2.875725in}{5.247301in}}%
\pgfpathcurveto{\pgfqpoint{2.879843in}{5.251419in}}{\pgfqpoint{2.882157in}{5.257005in}}{\pgfqpoint{2.882157in}{5.262829in}}%
\pgfpathcurveto{\pgfqpoint{2.882157in}{5.268653in}}{\pgfqpoint{2.879843in}{5.274240in}}{\pgfqpoint{2.875725in}{5.278358in}}%
\pgfpathcurveto{\pgfqpoint{2.871607in}{5.282476in}}{\pgfqpoint{2.866021in}{5.284790in}}{\pgfqpoint{2.860197in}{5.284790in}}%
\pgfpathcurveto{\pgfqpoint{2.854373in}{5.284790in}}{\pgfqpoint{2.848787in}{5.282476in}}{\pgfqpoint{2.844669in}{5.278358in}}%
\pgfpathcurveto{\pgfqpoint{2.840551in}{5.274240in}}{\pgfqpoint{2.838237in}{5.268653in}}{\pgfqpoint{2.838237in}{5.262829in}}%
\pgfpathcurveto{\pgfqpoint{2.838237in}{5.257005in}}{\pgfqpoint{2.840551in}{5.251419in}}{\pgfqpoint{2.844669in}{5.247301in}}%
\pgfpathcurveto{\pgfqpoint{2.848787in}{5.243183in}}{\pgfqpoint{2.854373in}{5.240869in}}{\pgfqpoint{2.860197in}{5.240869in}}%
\pgfpathlineto{\pgfqpoint{2.860197in}{5.240869in}}%
\pgfpathclose%
\pgfusepath{stroke,fill}%
\end{pgfscope}%
\begin{pgfscope}%
\pgfpathrectangle{\pgfqpoint{1.000000in}{0.979904in}}{\pgfqpoint{6.200000in}{5.960192in}}%
\pgfusepath{clip}%
\pgfsetbuttcap%
\pgfsetroundjoin%
\definecolor{currentfill}{rgb}{0.200000,0.200000,0.800000}%
\pgfsetfillcolor{currentfill}%
\pgfsetlinewidth{1.003750pt}%
\definecolor{currentstroke}{rgb}{0.200000,0.200000,0.800000}%
\pgfsetstrokecolor{currentstroke}%
\pgfsetdash{}{0pt}%
\pgfpathmoveto{\pgfqpoint{2.950293in}{5.224482in}}%
\pgfpathcurveto{\pgfqpoint{2.956117in}{5.224482in}}{\pgfqpoint{2.961703in}{5.226796in}}{\pgfqpoint{2.965821in}{5.230914in}}%
\pgfpathcurveto{\pgfqpoint{2.969939in}{5.235032in}}{\pgfqpoint{2.972253in}{5.240618in}}{\pgfqpoint{2.972253in}{5.246442in}}%
\pgfpathcurveto{\pgfqpoint{2.972253in}{5.252266in}}{\pgfqpoint{2.969939in}{5.257852in}}{\pgfqpoint{2.965821in}{5.261971in}}%
\pgfpathcurveto{\pgfqpoint{2.961703in}{5.266089in}}{\pgfqpoint{2.956117in}{5.268403in}}{\pgfqpoint{2.950293in}{5.268403in}}%
\pgfpathcurveto{\pgfqpoint{2.944469in}{5.268403in}}{\pgfqpoint{2.938883in}{5.266089in}}{\pgfqpoint{2.934764in}{5.261971in}}%
\pgfpathcurveto{\pgfqpoint{2.930646in}{5.257852in}}{\pgfqpoint{2.928332in}{5.252266in}}{\pgfqpoint{2.928332in}{5.246442in}}%
\pgfpathcurveto{\pgfqpoint{2.928332in}{5.240618in}}{\pgfqpoint{2.930646in}{5.235032in}}{\pgfqpoint{2.934764in}{5.230914in}}%
\pgfpathcurveto{\pgfqpoint{2.938883in}{5.226796in}}{\pgfqpoint{2.944469in}{5.224482in}}{\pgfqpoint{2.950293in}{5.224482in}}%
\pgfpathlineto{\pgfqpoint{2.950293in}{5.224482in}}%
\pgfpathclose%
\pgfusepath{stroke,fill}%
\end{pgfscope}%
\begin{pgfscope}%
\pgfpathrectangle{\pgfqpoint{1.000000in}{0.979904in}}{\pgfqpoint{6.200000in}{5.960192in}}%
\pgfusepath{clip}%
\pgfsetbuttcap%
\pgfsetroundjoin%
\definecolor{currentfill}{rgb}{0.200000,0.200000,0.800000}%
\pgfsetfillcolor{currentfill}%
\pgfsetlinewidth{1.003750pt}%
\definecolor{currentstroke}{rgb}{0.200000,0.200000,0.800000}%
\pgfsetstrokecolor{currentstroke}%
\pgfsetdash{}{0pt}%
\pgfpathmoveto{\pgfqpoint{2.938460in}{5.148531in}}%
\pgfpathcurveto{\pgfqpoint{2.944284in}{5.148531in}}{\pgfqpoint{2.949870in}{5.150845in}}{\pgfqpoint{2.953989in}{5.154963in}}%
\pgfpathcurveto{\pgfqpoint{2.958107in}{5.159081in}}{\pgfqpoint{2.960421in}{5.164667in}}{\pgfqpoint{2.960421in}{5.170491in}}%
\pgfpathcurveto{\pgfqpoint{2.960421in}{5.176315in}}{\pgfqpoint{2.958107in}{5.181901in}}{\pgfqpoint{2.953989in}{5.186020in}}%
\pgfpathcurveto{\pgfqpoint{2.949870in}{5.190138in}}{\pgfqpoint{2.944284in}{5.192452in}}{\pgfqpoint{2.938460in}{5.192452in}}%
\pgfpathcurveto{\pgfqpoint{2.932636in}{5.192452in}}{\pgfqpoint{2.927050in}{5.190138in}}{\pgfqpoint{2.922932in}{5.186020in}}%
\pgfpathcurveto{\pgfqpoint{2.918814in}{5.181901in}}{\pgfqpoint{2.916500in}{5.176315in}}{\pgfqpoint{2.916500in}{5.170491in}}%
\pgfpathcurveto{\pgfqpoint{2.916500in}{5.164667in}}{\pgfqpoint{2.918814in}{5.159081in}}{\pgfqpoint{2.922932in}{5.154963in}}%
\pgfpathcurveto{\pgfqpoint{2.927050in}{5.150845in}}{\pgfqpoint{2.932636in}{5.148531in}}{\pgfqpoint{2.938460in}{5.148531in}}%
\pgfpathlineto{\pgfqpoint{2.938460in}{5.148531in}}%
\pgfpathclose%
\pgfusepath{stroke,fill}%
\end{pgfscope}%
\begin{pgfscope}%
\pgfpathrectangle{\pgfqpoint{1.000000in}{0.979904in}}{\pgfqpoint{6.200000in}{5.960192in}}%
\pgfusepath{clip}%
\pgfsetbuttcap%
\pgfsetroundjoin%
\definecolor{currentfill}{rgb}{0.200000,0.200000,0.800000}%
\pgfsetfillcolor{currentfill}%
\pgfsetlinewidth{1.003750pt}%
\definecolor{currentstroke}{rgb}{0.200000,0.200000,0.800000}%
\pgfsetstrokecolor{currentstroke}%
\pgfsetdash{}{0pt}%
\pgfpathmoveto{\pgfqpoint{2.929772in}{5.066529in}}%
\pgfpathcurveto{\pgfqpoint{2.935596in}{5.066529in}}{\pgfqpoint{2.941182in}{5.068843in}}{\pgfqpoint{2.945300in}{5.072961in}}%
\pgfpathcurveto{\pgfqpoint{2.949418in}{5.077079in}}{\pgfqpoint{2.951732in}{5.082666in}}{\pgfqpoint{2.951732in}{5.088490in}}%
\pgfpathcurveto{\pgfqpoint{2.951732in}{5.094313in}}{\pgfqpoint{2.949418in}{5.099900in}}{\pgfqpoint{2.945300in}{5.104018in}}%
\pgfpathcurveto{\pgfqpoint{2.941182in}{5.108136in}}{\pgfqpoint{2.935596in}{5.110450in}}{\pgfqpoint{2.929772in}{5.110450in}}%
\pgfpathcurveto{\pgfqpoint{2.923948in}{5.110450in}}{\pgfqpoint{2.918362in}{5.108136in}}{\pgfqpoint{2.914244in}{5.104018in}}%
\pgfpathcurveto{\pgfqpoint{2.910125in}{5.099900in}}{\pgfqpoint{2.907812in}{5.094313in}}{\pgfqpoint{2.907812in}{5.088490in}}%
\pgfpathcurveto{\pgfqpoint{2.907812in}{5.082666in}}{\pgfqpoint{2.910125in}{5.077079in}}{\pgfqpoint{2.914244in}{5.072961in}}%
\pgfpathcurveto{\pgfqpoint{2.918362in}{5.068843in}}{\pgfqpoint{2.923948in}{5.066529in}}{\pgfqpoint{2.929772in}{5.066529in}}%
\pgfpathlineto{\pgfqpoint{2.929772in}{5.066529in}}%
\pgfpathclose%
\pgfusepath{stroke,fill}%
\end{pgfscope}%
\begin{pgfscope}%
\pgfpathrectangle{\pgfqpoint{1.000000in}{0.979904in}}{\pgfqpoint{6.200000in}{5.960192in}}%
\pgfusepath{clip}%
\pgfsetbuttcap%
\pgfsetroundjoin%
\definecolor{currentfill}{rgb}{0.200000,0.200000,0.800000}%
\pgfsetfillcolor{currentfill}%
\pgfsetlinewidth{1.003750pt}%
\definecolor{currentstroke}{rgb}{0.200000,0.200000,0.800000}%
\pgfsetstrokecolor{currentstroke}%
\pgfsetdash{}{0pt}%
\pgfpathmoveto{\pgfqpoint{2.977054in}{5.024813in}}%
\pgfpathcurveto{\pgfqpoint{2.982878in}{5.024813in}}{\pgfqpoint{2.988464in}{5.027127in}}{\pgfqpoint{2.992583in}{5.031245in}}%
\pgfpathcurveto{\pgfqpoint{2.996701in}{5.035363in}}{\pgfqpoint{2.999015in}{5.040949in}}{\pgfqpoint{2.999015in}{5.046773in}}%
\pgfpathcurveto{\pgfqpoint{2.999015in}{5.052597in}}{\pgfqpoint{2.996701in}{5.058183in}}{\pgfqpoint{2.992583in}{5.062302in}}%
\pgfpathcurveto{\pgfqpoint{2.988464in}{5.066420in}}{\pgfqpoint{2.982878in}{5.068734in}}{\pgfqpoint{2.977054in}{5.068734in}}%
\pgfpathcurveto{\pgfqpoint{2.971230in}{5.068734in}}{\pgfqpoint{2.965644in}{5.066420in}}{\pgfqpoint{2.961526in}{5.062302in}}%
\pgfpathcurveto{\pgfqpoint{2.957408in}{5.058183in}}{\pgfqpoint{2.955094in}{5.052597in}}{\pgfqpoint{2.955094in}{5.046773in}}%
\pgfpathcurveto{\pgfqpoint{2.955094in}{5.040949in}}{\pgfqpoint{2.957408in}{5.035363in}}{\pgfqpoint{2.961526in}{5.031245in}}%
\pgfpathcurveto{\pgfqpoint{2.965644in}{5.027127in}}{\pgfqpoint{2.971230in}{5.024813in}}{\pgfqpoint{2.977054in}{5.024813in}}%
\pgfpathlineto{\pgfqpoint{2.977054in}{5.024813in}}%
\pgfpathclose%
\pgfusepath{stroke,fill}%
\end{pgfscope}%
\begin{pgfscope}%
\pgfpathrectangle{\pgfqpoint{1.000000in}{0.979904in}}{\pgfqpoint{6.200000in}{5.960192in}}%
\pgfusepath{clip}%
\pgfsetbuttcap%
\pgfsetroundjoin%
\definecolor{currentfill}{rgb}{0.200000,0.200000,0.800000}%
\pgfsetfillcolor{currentfill}%
\pgfsetlinewidth{1.003750pt}%
\definecolor{currentstroke}{rgb}{0.200000,0.200000,0.800000}%
\pgfsetstrokecolor{currentstroke}%
\pgfsetdash{}{0pt}%
\pgfpathmoveto{\pgfqpoint{2.993827in}{4.953442in}}%
\pgfpathcurveto{\pgfqpoint{2.999651in}{4.953442in}}{\pgfqpoint{3.005238in}{4.955756in}}{\pgfqpoint{3.009356in}{4.959874in}}%
\pgfpathcurveto{\pgfqpoint{3.013474in}{4.963992in}}{\pgfqpoint{3.015788in}{4.969578in}}{\pgfqpoint{3.015788in}{4.975402in}}%
\pgfpathcurveto{\pgfqpoint{3.015788in}{4.981226in}}{\pgfqpoint{3.013474in}{4.986812in}}{\pgfqpoint{3.009356in}{4.990930in}}%
\pgfpathcurveto{\pgfqpoint{3.005238in}{4.995049in}}{\pgfqpoint{2.999651in}{4.997363in}}{\pgfqpoint{2.993827in}{4.997363in}}%
\pgfpathcurveto{\pgfqpoint{2.988004in}{4.997363in}}{\pgfqpoint{2.982417in}{4.995049in}}{\pgfqpoint{2.978299in}{4.990930in}}%
\pgfpathcurveto{\pgfqpoint{2.974181in}{4.986812in}}{\pgfqpoint{2.971867in}{4.981226in}}{\pgfqpoint{2.971867in}{4.975402in}}%
\pgfpathcurveto{\pgfqpoint{2.971867in}{4.969578in}}{\pgfqpoint{2.974181in}{4.963992in}}{\pgfqpoint{2.978299in}{4.959874in}}%
\pgfpathcurveto{\pgfqpoint{2.982417in}{4.955756in}}{\pgfqpoint{2.988004in}{4.953442in}}{\pgfqpoint{2.993827in}{4.953442in}}%
\pgfpathlineto{\pgfqpoint{2.993827in}{4.953442in}}%
\pgfpathclose%
\pgfusepath{stroke,fill}%
\end{pgfscope}%
\begin{pgfscope}%
\pgfpathrectangle{\pgfqpoint{1.000000in}{0.979904in}}{\pgfqpoint{6.200000in}{5.960192in}}%
\pgfusepath{clip}%
\pgfsetbuttcap%
\pgfsetroundjoin%
\definecolor{currentfill}{rgb}{0.200000,0.200000,0.800000}%
\pgfsetfillcolor{currentfill}%
\pgfsetlinewidth{1.003750pt}%
\definecolor{currentstroke}{rgb}{0.200000,0.200000,0.800000}%
\pgfsetstrokecolor{currentstroke}%
\pgfsetdash{}{0pt}%
\pgfpathmoveto{\pgfqpoint{3.081348in}{4.956145in}}%
\pgfpathcurveto{\pgfqpoint{3.087172in}{4.956145in}}{\pgfqpoint{3.092759in}{4.958458in}}{\pgfqpoint{3.096877in}{4.962577in}}%
\pgfpathcurveto{\pgfqpoint{3.100995in}{4.966695in}}{\pgfqpoint{3.103309in}{4.972281in}}{\pgfqpoint{3.103309in}{4.978105in}}%
\pgfpathcurveto{\pgfqpoint{3.103309in}{4.983929in}}{\pgfqpoint{3.100995in}{4.989515in}}{\pgfqpoint{3.096877in}{4.993633in}}%
\pgfpathcurveto{\pgfqpoint{3.092759in}{4.997751in}}{\pgfqpoint{3.087172in}{5.000065in}}{\pgfqpoint{3.081348in}{5.000065in}}%
\pgfpathcurveto{\pgfqpoint{3.075525in}{5.000065in}}{\pgfqpoint{3.069938in}{4.997751in}}{\pgfqpoint{3.065820in}{4.993633in}}%
\pgfpathcurveto{\pgfqpoint{3.061702in}{4.989515in}}{\pgfqpoint{3.059388in}{4.983929in}}{\pgfqpoint{3.059388in}{4.978105in}}%
\pgfpathcurveto{\pgfqpoint{3.059388in}{4.972281in}}{\pgfqpoint{3.061702in}{4.966695in}}{\pgfqpoint{3.065820in}{4.962577in}}%
\pgfpathcurveto{\pgfqpoint{3.069938in}{4.958458in}}{\pgfqpoint{3.075525in}{4.956145in}}{\pgfqpoint{3.081348in}{4.956145in}}%
\pgfpathlineto{\pgfqpoint{3.081348in}{4.956145in}}%
\pgfpathclose%
\pgfusepath{stroke,fill}%
\end{pgfscope}%
\begin{pgfscope}%
\pgfpathrectangle{\pgfqpoint{1.000000in}{0.979904in}}{\pgfqpoint{6.200000in}{5.960192in}}%
\pgfusepath{clip}%
\pgfsetbuttcap%
\pgfsetroundjoin%
\definecolor{currentfill}{rgb}{0.200000,0.200000,0.800000}%
\pgfsetfillcolor{currentfill}%
\pgfsetlinewidth{1.003750pt}%
\definecolor{currentstroke}{rgb}{0.200000,0.200000,0.800000}%
\pgfsetstrokecolor{currentstroke}%
\pgfsetdash{}{0pt}%
\pgfpathmoveto{\pgfqpoint{3.030822in}{4.789888in}}%
\pgfpathcurveto{\pgfqpoint{3.036646in}{4.789888in}}{\pgfqpoint{3.042232in}{4.792202in}}{\pgfqpoint{3.046351in}{4.796320in}}%
\pgfpathcurveto{\pgfqpoint{3.050469in}{4.800438in}}{\pgfqpoint{3.052783in}{4.806024in}}{\pgfqpoint{3.052783in}{4.811848in}}%
\pgfpathcurveto{\pgfqpoint{3.052783in}{4.817672in}}{\pgfqpoint{3.050469in}{4.823258in}}{\pgfqpoint{3.046351in}{4.827376in}}%
\pgfpathcurveto{\pgfqpoint{3.042232in}{4.831494in}}{\pgfqpoint{3.036646in}{4.833808in}}{\pgfqpoint{3.030822in}{4.833808in}}%
\pgfpathcurveto{\pgfqpoint{3.024998in}{4.833808in}}{\pgfqpoint{3.019412in}{4.831494in}}{\pgfqpoint{3.015294in}{4.827376in}}%
\pgfpathcurveto{\pgfqpoint{3.011176in}{4.823258in}}{\pgfqpoint{3.008862in}{4.817672in}}{\pgfqpoint{3.008862in}{4.811848in}}%
\pgfpathcurveto{\pgfqpoint{3.008862in}{4.806024in}}{\pgfqpoint{3.011176in}{4.800438in}}{\pgfqpoint{3.015294in}{4.796320in}}%
\pgfpathcurveto{\pgfqpoint{3.019412in}{4.792202in}}{\pgfqpoint{3.024998in}{4.789888in}}{\pgfqpoint{3.030822in}{4.789888in}}%
\pgfpathlineto{\pgfqpoint{3.030822in}{4.789888in}}%
\pgfpathclose%
\pgfusepath{stroke,fill}%
\end{pgfscope}%
\begin{pgfscope}%
\pgfpathrectangle{\pgfqpoint{1.000000in}{0.979904in}}{\pgfqpoint{6.200000in}{5.960192in}}%
\pgfusepath{clip}%
\pgfsetbuttcap%
\pgfsetroundjoin%
\definecolor{currentfill}{rgb}{0.200000,0.200000,0.800000}%
\pgfsetfillcolor{currentfill}%
\pgfsetlinewidth{1.003750pt}%
\definecolor{currentstroke}{rgb}{0.200000,0.200000,0.800000}%
\pgfsetstrokecolor{currentstroke}%
\pgfsetdash{}{0pt}%
\pgfpathmoveto{\pgfqpoint{3.214592in}{4.938485in}}%
\pgfpathcurveto{\pgfqpoint{3.220416in}{4.938485in}}{\pgfqpoint{3.226002in}{4.940799in}}{\pgfqpoint{3.230120in}{4.944917in}}%
\pgfpathcurveto{\pgfqpoint{3.234238in}{4.949036in}}{\pgfqpoint{3.236552in}{4.954622in}}{\pgfqpoint{3.236552in}{4.960446in}}%
\pgfpathcurveto{\pgfqpoint{3.236552in}{4.966270in}}{\pgfqpoint{3.234238in}{4.971856in}}{\pgfqpoint{3.230120in}{4.975974in}}%
\pgfpathcurveto{\pgfqpoint{3.226002in}{4.980092in}}{\pgfqpoint{3.220416in}{4.982406in}}{\pgfqpoint{3.214592in}{4.982406in}}%
\pgfpathcurveto{\pgfqpoint{3.208768in}{4.982406in}}{\pgfqpoint{3.203182in}{4.980092in}}{\pgfqpoint{3.199064in}{4.975974in}}%
\pgfpathcurveto{\pgfqpoint{3.194946in}{4.971856in}}{\pgfqpoint{3.192632in}{4.966270in}}{\pgfqpoint{3.192632in}{4.960446in}}%
\pgfpathcurveto{\pgfqpoint{3.192632in}{4.954622in}}{\pgfqpoint{3.194946in}{4.949036in}}{\pgfqpoint{3.199064in}{4.944917in}}%
\pgfpathcurveto{\pgfqpoint{3.203182in}{4.940799in}}{\pgfqpoint{3.208768in}{4.938485in}}{\pgfqpoint{3.214592in}{4.938485in}}%
\pgfpathlineto{\pgfqpoint{3.214592in}{4.938485in}}%
\pgfpathclose%
\pgfusepath{stroke,fill}%
\end{pgfscope}%
\begin{pgfscope}%
\pgfpathrectangle{\pgfqpoint{1.000000in}{0.979904in}}{\pgfqpoint{6.200000in}{5.960192in}}%
\pgfusepath{clip}%
\pgfsetbuttcap%
\pgfsetroundjoin%
\definecolor{currentfill}{rgb}{0.200000,0.800000,0.200000}%
\pgfsetfillcolor{currentfill}%
\pgfsetlinewidth{1.003750pt}%
\definecolor{currentstroke}{rgb}{0.200000,0.800000,0.200000}%
\pgfsetstrokecolor{currentstroke}%
\pgfsetdash{}{0pt}%
\pgfpathmoveto{\pgfqpoint{3.248568in}{4.884843in}}%
\pgfpathcurveto{\pgfqpoint{3.254391in}{4.884843in}}{\pgfqpoint{3.259978in}{4.887157in}}{\pgfqpoint{3.264096in}{4.891275in}}%
\pgfpathcurveto{\pgfqpoint{3.268214in}{4.895393in}}{\pgfqpoint{3.270528in}{4.900979in}}{\pgfqpoint{3.270528in}{4.906803in}}%
\pgfpathcurveto{\pgfqpoint{3.270528in}{4.912627in}}{\pgfqpoint{3.268214in}{4.918213in}}{\pgfqpoint{3.264096in}{4.922332in}}%
\pgfpathcurveto{\pgfqpoint{3.259978in}{4.926450in}}{\pgfqpoint{3.254391in}{4.928764in}}{\pgfqpoint{3.248568in}{4.928764in}}%
\pgfpathcurveto{\pgfqpoint{3.242744in}{4.928764in}}{\pgfqpoint{3.237157in}{4.926450in}}{\pgfqpoint{3.233039in}{4.922332in}}%
\pgfpathcurveto{\pgfqpoint{3.228921in}{4.918213in}}{\pgfqpoint{3.226607in}{4.912627in}}{\pgfqpoint{3.226607in}{4.906803in}}%
\pgfpathcurveto{\pgfqpoint{3.226607in}{4.900979in}}{\pgfqpoint{3.228921in}{4.895393in}}{\pgfqpoint{3.233039in}{4.891275in}}%
\pgfpathcurveto{\pgfqpoint{3.237157in}{4.887157in}}{\pgfqpoint{3.242744in}{4.884843in}}{\pgfqpoint{3.248568in}{4.884843in}}%
\pgfpathlineto{\pgfqpoint{3.248568in}{4.884843in}}%
\pgfpathclose%
\pgfusepath{stroke,fill}%
\end{pgfscope}%
\begin{pgfscope}%
\pgfpathrectangle{\pgfqpoint{1.000000in}{0.979904in}}{\pgfqpoint{6.200000in}{5.960192in}}%
\pgfusepath{clip}%
\pgfsetbuttcap%
\pgfsetroundjoin%
\definecolor{currentfill}{rgb}{0.200000,0.200000,0.800000}%
\pgfsetfillcolor{currentfill}%
\pgfsetlinewidth{1.003750pt}%
\definecolor{currentstroke}{rgb}{0.200000,0.200000,0.800000}%
\pgfsetstrokecolor{currentstroke}%
\pgfsetdash{}{0pt}%
\pgfpathmoveto{\pgfqpoint{3.291760in}{4.841380in}}%
\pgfpathcurveto{\pgfqpoint{3.297584in}{4.841380in}}{\pgfqpoint{3.303170in}{4.843694in}}{\pgfqpoint{3.307288in}{4.847812in}}%
\pgfpathcurveto{\pgfqpoint{3.311406in}{4.851930in}}{\pgfqpoint{3.313720in}{4.857516in}}{\pgfqpoint{3.313720in}{4.863340in}}%
\pgfpathcurveto{\pgfqpoint{3.313720in}{4.869164in}}{\pgfqpoint{3.311406in}{4.874750in}}{\pgfqpoint{3.307288in}{4.878869in}}%
\pgfpathcurveto{\pgfqpoint{3.303170in}{4.882987in}}{\pgfqpoint{3.297584in}{4.885301in}}{\pgfqpoint{3.291760in}{4.885301in}}%
\pgfpathcurveto{\pgfqpoint{3.285936in}{4.885301in}}{\pgfqpoint{3.280350in}{4.882987in}}{\pgfqpoint{3.276232in}{4.878869in}}%
\pgfpathcurveto{\pgfqpoint{3.272113in}{4.874750in}}{\pgfqpoint{3.269800in}{4.869164in}}{\pgfqpoint{3.269800in}{4.863340in}}%
\pgfpathcurveto{\pgfqpoint{3.269800in}{4.857516in}}{\pgfqpoint{3.272113in}{4.851930in}}{\pgfqpoint{3.276232in}{4.847812in}}%
\pgfpathcurveto{\pgfqpoint{3.280350in}{4.843694in}}{\pgfqpoint{3.285936in}{4.841380in}}{\pgfqpoint{3.291760in}{4.841380in}}%
\pgfpathlineto{\pgfqpoint{3.291760in}{4.841380in}}%
\pgfpathclose%
\pgfusepath{stroke,fill}%
\end{pgfscope}%
\begin{pgfscope}%
\pgfpathrectangle{\pgfqpoint{1.000000in}{0.979904in}}{\pgfqpoint{6.200000in}{5.960192in}}%
\pgfusepath{clip}%
\pgfsetbuttcap%
\pgfsetroundjoin%
\definecolor{currentfill}{rgb}{0.200000,0.200000,0.800000}%
\pgfsetfillcolor{currentfill}%
\pgfsetlinewidth{1.003750pt}%
\definecolor{currentstroke}{rgb}{0.200000,0.200000,0.800000}%
\pgfsetstrokecolor{currentstroke}%
\pgfsetdash{}{0pt}%
\pgfpathmoveto{\pgfqpoint{3.373414in}{4.883252in}}%
\pgfpathcurveto{\pgfqpoint{3.379238in}{4.883252in}}{\pgfqpoint{3.384824in}{4.885566in}}{\pgfqpoint{3.388942in}{4.889684in}}%
\pgfpathcurveto{\pgfqpoint{3.393061in}{4.893802in}}{\pgfqpoint{3.395374in}{4.899389in}}{\pgfqpoint{3.395374in}{4.905212in}}%
\pgfpathcurveto{\pgfqpoint{3.395374in}{4.911036in}}{\pgfqpoint{3.393061in}{4.916623in}}{\pgfqpoint{3.388942in}{4.920741in}}%
\pgfpathcurveto{\pgfqpoint{3.384824in}{4.924859in}}{\pgfqpoint{3.379238in}{4.927173in}}{\pgfqpoint{3.373414in}{4.927173in}}%
\pgfpathcurveto{\pgfqpoint{3.367590in}{4.927173in}}{\pgfqpoint{3.362004in}{4.924859in}}{\pgfqpoint{3.357886in}{4.920741in}}%
\pgfpathcurveto{\pgfqpoint{3.353768in}{4.916623in}}{\pgfqpoint{3.351454in}{4.911036in}}{\pgfqpoint{3.351454in}{4.905212in}}%
\pgfpathcurveto{\pgfqpoint{3.351454in}{4.899389in}}{\pgfqpoint{3.353768in}{4.893802in}}{\pgfqpoint{3.357886in}{4.889684in}}%
\pgfpathcurveto{\pgfqpoint{3.362004in}{4.885566in}}{\pgfqpoint{3.367590in}{4.883252in}}{\pgfqpoint{3.373414in}{4.883252in}}%
\pgfpathlineto{\pgfqpoint{3.373414in}{4.883252in}}%
\pgfpathclose%
\pgfusepath{stroke,fill}%
\end{pgfscope}%
\begin{pgfscope}%
\pgfpathrectangle{\pgfqpoint{1.000000in}{0.979904in}}{\pgfqpoint{6.200000in}{5.960192in}}%
\pgfusepath{clip}%
\pgfsetbuttcap%
\pgfsetroundjoin%
\definecolor{currentfill}{rgb}{0.200000,0.200000,0.800000}%
\pgfsetfillcolor{currentfill}%
\pgfsetlinewidth{1.003750pt}%
\definecolor{currentstroke}{rgb}{0.200000,0.200000,0.800000}%
\pgfsetstrokecolor{currentstroke}%
\pgfsetdash{}{0pt}%
\pgfpathmoveto{\pgfqpoint{3.381704in}{4.740365in}}%
\pgfpathcurveto{\pgfqpoint{3.387528in}{4.740365in}}{\pgfqpoint{3.393114in}{4.742679in}}{\pgfqpoint{3.397232in}{4.746797in}}%
\pgfpathcurveto{\pgfqpoint{3.401350in}{4.750915in}}{\pgfqpoint{3.403664in}{4.756501in}}{\pgfqpoint{3.403664in}{4.762325in}}%
\pgfpathcurveto{\pgfqpoint{3.403664in}{4.768149in}}{\pgfqpoint{3.401350in}{4.773735in}}{\pgfqpoint{3.397232in}{4.777853in}}%
\pgfpathcurveto{\pgfqpoint{3.393114in}{4.781971in}}{\pgfqpoint{3.387528in}{4.784285in}}{\pgfqpoint{3.381704in}{4.784285in}}%
\pgfpathcurveto{\pgfqpoint{3.375880in}{4.784285in}}{\pgfqpoint{3.370294in}{4.781971in}}{\pgfqpoint{3.366176in}{4.777853in}}%
\pgfpathcurveto{\pgfqpoint{3.362057in}{4.773735in}}{\pgfqpoint{3.359744in}{4.768149in}}{\pgfqpoint{3.359744in}{4.762325in}}%
\pgfpathcurveto{\pgfqpoint{3.359744in}{4.756501in}}{\pgfqpoint{3.362057in}{4.750915in}}{\pgfqpoint{3.366176in}{4.746797in}}%
\pgfpathcurveto{\pgfqpoint{3.370294in}{4.742679in}}{\pgfqpoint{3.375880in}{4.740365in}}{\pgfqpoint{3.381704in}{4.740365in}}%
\pgfpathlineto{\pgfqpoint{3.381704in}{4.740365in}}%
\pgfpathclose%
\pgfusepath{stroke,fill}%
\end{pgfscope}%
\begin{pgfscope}%
\pgfpathrectangle{\pgfqpoint{1.000000in}{0.979904in}}{\pgfqpoint{6.200000in}{5.960192in}}%
\pgfusepath{clip}%
\pgfsetbuttcap%
\pgfsetroundjoin%
\definecolor{currentfill}{rgb}{0.200000,0.200000,0.800000}%
\pgfsetfillcolor{currentfill}%
\pgfsetlinewidth{1.003750pt}%
\definecolor{currentstroke}{rgb}{0.200000,0.200000,0.800000}%
\pgfsetstrokecolor{currentstroke}%
\pgfsetdash{}{0pt}%
\pgfpathmoveto{\pgfqpoint{3.443714in}{4.728800in}}%
\pgfpathcurveto{\pgfqpoint{3.449538in}{4.728800in}}{\pgfqpoint{3.455124in}{4.731114in}}{\pgfqpoint{3.459243in}{4.735232in}}%
\pgfpathcurveto{\pgfqpoint{3.463361in}{4.739350in}}{\pgfqpoint{3.465675in}{4.744936in}}{\pgfqpoint{3.465675in}{4.750760in}}%
\pgfpathcurveto{\pgfqpoint{3.465675in}{4.756584in}}{\pgfqpoint{3.463361in}{4.762170in}}{\pgfqpoint{3.459243in}{4.766289in}}%
\pgfpathcurveto{\pgfqpoint{3.455124in}{4.770407in}}{\pgfqpoint{3.449538in}{4.772721in}}{\pgfqpoint{3.443714in}{4.772721in}}%
\pgfpathcurveto{\pgfqpoint{3.437890in}{4.772721in}}{\pgfqpoint{3.432304in}{4.770407in}}{\pgfqpoint{3.428186in}{4.766289in}}%
\pgfpathcurveto{\pgfqpoint{3.424068in}{4.762170in}}{\pgfqpoint{3.421754in}{4.756584in}}{\pgfqpoint{3.421754in}{4.750760in}}%
\pgfpathcurveto{\pgfqpoint{3.421754in}{4.744936in}}{\pgfqpoint{3.424068in}{4.739350in}}{\pgfqpoint{3.428186in}{4.735232in}}%
\pgfpathcurveto{\pgfqpoint{3.432304in}{4.731114in}}{\pgfqpoint{3.437890in}{4.728800in}}{\pgfqpoint{3.443714in}{4.728800in}}%
\pgfpathlineto{\pgfqpoint{3.443714in}{4.728800in}}%
\pgfpathclose%
\pgfusepath{stroke,fill}%
\end{pgfscope}%
\begin{pgfscope}%
\pgfpathrectangle{\pgfqpoint{1.000000in}{0.979904in}}{\pgfqpoint{6.200000in}{5.960192in}}%
\pgfusepath{clip}%
\pgfsetbuttcap%
\pgfsetroundjoin%
\definecolor{currentfill}{rgb}{0.200000,0.200000,0.800000}%
\pgfsetfillcolor{currentfill}%
\pgfsetlinewidth{1.003750pt}%
\definecolor{currentstroke}{rgb}{0.200000,0.200000,0.800000}%
\pgfsetstrokecolor{currentstroke}%
\pgfsetdash{}{0pt}%
\pgfpathmoveto{\pgfqpoint{3.513980in}{4.760021in}}%
\pgfpathcurveto{\pgfqpoint{3.519804in}{4.760021in}}{\pgfqpoint{3.525391in}{4.762335in}}{\pgfqpoint{3.529509in}{4.766453in}}%
\pgfpathcurveto{\pgfqpoint{3.533627in}{4.770571in}}{\pgfqpoint{3.535941in}{4.776157in}}{\pgfqpoint{3.535941in}{4.781981in}}%
\pgfpathcurveto{\pgfqpoint{3.535941in}{4.787805in}}{\pgfqpoint{3.533627in}{4.793391in}}{\pgfqpoint{3.529509in}{4.797509in}}%
\pgfpathcurveto{\pgfqpoint{3.525391in}{4.801628in}}{\pgfqpoint{3.519804in}{4.803941in}}{\pgfqpoint{3.513980in}{4.803941in}}%
\pgfpathcurveto{\pgfqpoint{3.508157in}{4.803941in}}{\pgfqpoint{3.502570in}{4.801628in}}{\pgfqpoint{3.498452in}{4.797509in}}%
\pgfpathcurveto{\pgfqpoint{3.494334in}{4.793391in}}{\pgfqpoint{3.492020in}{4.787805in}}{\pgfqpoint{3.492020in}{4.781981in}}%
\pgfpathcurveto{\pgfqpoint{3.492020in}{4.776157in}}{\pgfqpoint{3.494334in}{4.770571in}}{\pgfqpoint{3.498452in}{4.766453in}}%
\pgfpathcurveto{\pgfqpoint{3.502570in}{4.762335in}}{\pgfqpoint{3.508157in}{4.760021in}}{\pgfqpoint{3.513980in}{4.760021in}}%
\pgfpathlineto{\pgfqpoint{3.513980in}{4.760021in}}%
\pgfpathclose%
\pgfusepath{stroke,fill}%
\end{pgfscope}%
\begin{pgfscope}%
\pgfpathrectangle{\pgfqpoint{1.000000in}{0.979904in}}{\pgfqpoint{6.200000in}{5.960192in}}%
\pgfusepath{clip}%
\pgfsetbuttcap%
\pgfsetroundjoin%
\definecolor{currentfill}{rgb}{0.200000,0.200000,0.800000}%
\pgfsetfillcolor{currentfill}%
\pgfsetlinewidth{1.003750pt}%
\definecolor{currentstroke}{rgb}{0.200000,0.200000,0.800000}%
\pgfsetstrokecolor{currentstroke}%
\pgfsetdash{}{0pt}%
\pgfpathmoveto{\pgfqpoint{3.576692in}{4.778764in}}%
\pgfpathcurveto{\pgfqpoint{3.582516in}{4.778764in}}{\pgfqpoint{3.588102in}{4.781077in}}{\pgfqpoint{3.592220in}{4.785196in}}%
\pgfpathcurveto{\pgfqpoint{3.596338in}{4.789314in}}{\pgfqpoint{3.598652in}{4.794900in}}{\pgfqpoint{3.598652in}{4.800724in}}%
\pgfpathcurveto{\pgfqpoint{3.598652in}{4.806548in}}{\pgfqpoint{3.596338in}{4.812134in}}{\pgfqpoint{3.592220in}{4.816252in}}%
\pgfpathcurveto{\pgfqpoint{3.588102in}{4.820370in}}{\pgfqpoint{3.582516in}{4.822684in}}{\pgfqpoint{3.576692in}{4.822684in}}%
\pgfpathcurveto{\pgfqpoint{3.570868in}{4.822684in}}{\pgfqpoint{3.565282in}{4.820370in}}{\pgfqpoint{3.561164in}{4.816252in}}%
\pgfpathcurveto{\pgfqpoint{3.557046in}{4.812134in}}{\pgfqpoint{3.554732in}{4.806548in}}{\pgfqpoint{3.554732in}{4.800724in}}%
\pgfpathcurveto{\pgfqpoint{3.554732in}{4.794900in}}{\pgfqpoint{3.557046in}{4.789314in}}{\pgfqpoint{3.561164in}{4.785196in}}%
\pgfpathcurveto{\pgfqpoint{3.565282in}{4.781077in}}{\pgfqpoint{3.570868in}{4.778764in}}{\pgfqpoint{3.576692in}{4.778764in}}%
\pgfpathlineto{\pgfqpoint{3.576692in}{4.778764in}}%
\pgfpathclose%
\pgfusepath{stroke,fill}%
\end{pgfscope}%
\begin{pgfscope}%
\pgfpathrectangle{\pgfqpoint{1.000000in}{0.979904in}}{\pgfqpoint{6.200000in}{5.960192in}}%
\pgfusepath{clip}%
\pgfsetbuttcap%
\pgfsetroundjoin%
\definecolor{currentfill}{rgb}{0.200000,0.200000,0.800000}%
\pgfsetfillcolor{currentfill}%
\pgfsetlinewidth{1.003750pt}%
\definecolor{currentstroke}{rgb}{0.200000,0.200000,0.800000}%
\pgfsetstrokecolor{currentstroke}%
\pgfsetdash{}{0pt}%
\pgfpathmoveto{\pgfqpoint{3.637949in}{4.823190in}}%
\pgfpathcurveto{\pgfqpoint{3.643773in}{4.823190in}}{\pgfqpoint{3.649359in}{4.825504in}}{\pgfqpoint{3.653477in}{4.829622in}}%
\pgfpathcurveto{\pgfqpoint{3.657595in}{4.833740in}}{\pgfqpoint{3.659909in}{4.839327in}}{\pgfqpoint{3.659909in}{4.845150in}}%
\pgfpathcurveto{\pgfqpoint{3.659909in}{4.850974in}}{\pgfqpoint{3.657595in}{4.856561in}}{\pgfqpoint{3.653477in}{4.860679in}}%
\pgfpathcurveto{\pgfqpoint{3.649359in}{4.864797in}}{\pgfqpoint{3.643773in}{4.867111in}}{\pgfqpoint{3.637949in}{4.867111in}}%
\pgfpathcurveto{\pgfqpoint{3.632125in}{4.867111in}}{\pgfqpoint{3.626539in}{4.864797in}}{\pgfqpoint{3.622421in}{4.860679in}}%
\pgfpathcurveto{\pgfqpoint{3.618303in}{4.856561in}}{\pgfqpoint{3.615989in}{4.850974in}}{\pgfqpoint{3.615989in}{4.845150in}}%
\pgfpathcurveto{\pgfqpoint{3.615989in}{4.839327in}}{\pgfqpoint{3.618303in}{4.833740in}}{\pgfqpoint{3.622421in}{4.829622in}}%
\pgfpathcurveto{\pgfqpoint{3.626539in}{4.825504in}}{\pgfqpoint{3.632125in}{4.823190in}}{\pgfqpoint{3.637949in}{4.823190in}}%
\pgfpathlineto{\pgfqpoint{3.637949in}{4.823190in}}%
\pgfpathclose%
\pgfusepath{stroke,fill}%
\end{pgfscope}%
\begin{pgfscope}%
\pgfpathrectangle{\pgfqpoint{1.000000in}{0.979904in}}{\pgfqpoint{6.200000in}{5.960192in}}%
\pgfusepath{clip}%
\pgfsetbuttcap%
\pgfsetroundjoin%
\definecolor{currentfill}{rgb}{0.200000,0.200000,0.800000}%
\pgfsetfillcolor{currentfill}%
\pgfsetlinewidth{1.003750pt}%
\definecolor{currentstroke}{rgb}{0.200000,0.200000,0.800000}%
\pgfsetstrokecolor{currentstroke}%
\pgfsetdash{}{0pt}%
\pgfpathmoveto{\pgfqpoint{3.691632in}{4.783162in}}%
\pgfpathcurveto{\pgfqpoint{3.697456in}{4.783162in}}{\pgfqpoint{3.703042in}{4.785476in}}{\pgfqpoint{3.707160in}{4.789594in}}%
\pgfpathcurveto{\pgfqpoint{3.711278in}{4.793712in}}{\pgfqpoint{3.713592in}{4.799299in}}{\pgfqpoint{3.713592in}{4.805123in}}%
\pgfpathcurveto{\pgfqpoint{3.713592in}{4.810946in}}{\pgfqpoint{3.711278in}{4.816533in}}{\pgfqpoint{3.707160in}{4.820651in}}%
\pgfpathcurveto{\pgfqpoint{3.703042in}{4.824769in}}{\pgfqpoint{3.697456in}{4.827083in}}{\pgfqpoint{3.691632in}{4.827083in}}%
\pgfpathcurveto{\pgfqpoint{3.685808in}{4.827083in}}{\pgfqpoint{3.680222in}{4.824769in}}{\pgfqpoint{3.676103in}{4.820651in}}%
\pgfpathcurveto{\pgfqpoint{3.671985in}{4.816533in}}{\pgfqpoint{3.669671in}{4.810946in}}{\pgfqpoint{3.669671in}{4.805123in}}%
\pgfpathcurveto{\pgfqpoint{3.669671in}{4.799299in}}{\pgfqpoint{3.671985in}{4.793712in}}{\pgfqpoint{3.676103in}{4.789594in}}%
\pgfpathcurveto{\pgfqpoint{3.680222in}{4.785476in}}{\pgfqpoint{3.685808in}{4.783162in}}{\pgfqpoint{3.691632in}{4.783162in}}%
\pgfpathlineto{\pgfqpoint{3.691632in}{4.783162in}}%
\pgfpathclose%
\pgfusepath{stroke,fill}%
\end{pgfscope}%
\begin{pgfscope}%
\pgfpathrectangle{\pgfqpoint{1.000000in}{0.979904in}}{\pgfqpoint{6.200000in}{5.960192in}}%
\pgfusepath{clip}%
\pgfsetbuttcap%
\pgfsetroundjoin%
\definecolor{currentfill}{rgb}{0.200000,0.200000,0.800000}%
\pgfsetfillcolor{currentfill}%
\pgfsetlinewidth{1.003750pt}%
\definecolor{currentstroke}{rgb}{0.200000,0.200000,0.800000}%
\pgfsetstrokecolor{currentstroke}%
\pgfsetdash{}{0pt}%
\pgfpathmoveto{\pgfqpoint{3.752209in}{4.702820in}}%
\pgfpathcurveto{\pgfqpoint{3.758033in}{4.702820in}}{\pgfqpoint{3.763619in}{4.705134in}}{\pgfqpoint{3.767737in}{4.709252in}}%
\pgfpathcurveto{\pgfqpoint{3.771856in}{4.713370in}}{\pgfqpoint{3.774169in}{4.718956in}}{\pgfqpoint{3.774169in}{4.724780in}}%
\pgfpathcurveto{\pgfqpoint{3.774169in}{4.730604in}}{\pgfqpoint{3.771856in}{4.736190in}}{\pgfqpoint{3.767737in}{4.740308in}}%
\pgfpathcurveto{\pgfqpoint{3.763619in}{4.744427in}}{\pgfqpoint{3.758033in}{4.746741in}}{\pgfqpoint{3.752209in}{4.746741in}}%
\pgfpathcurveto{\pgfqpoint{3.746385in}{4.746741in}}{\pgfqpoint{3.740799in}{4.744427in}}{\pgfqpoint{3.736681in}{4.740308in}}%
\pgfpathcurveto{\pgfqpoint{3.732563in}{4.736190in}}{\pgfqpoint{3.730249in}{4.730604in}}{\pgfqpoint{3.730249in}{4.724780in}}%
\pgfpathcurveto{\pgfqpoint{3.730249in}{4.718956in}}{\pgfqpoint{3.732563in}{4.713370in}}{\pgfqpoint{3.736681in}{4.709252in}}%
\pgfpathcurveto{\pgfqpoint{3.740799in}{4.705134in}}{\pgfqpoint{3.746385in}{4.702820in}}{\pgfqpoint{3.752209in}{4.702820in}}%
\pgfpathlineto{\pgfqpoint{3.752209in}{4.702820in}}%
\pgfpathclose%
\pgfusepath{stroke,fill}%
\end{pgfscope}%
\begin{pgfscope}%
\pgfpathrectangle{\pgfqpoint{1.000000in}{0.979904in}}{\pgfqpoint{6.200000in}{5.960192in}}%
\pgfusepath{clip}%
\pgfsetbuttcap%
\pgfsetroundjoin%
\definecolor{currentfill}{rgb}{0.200000,0.200000,0.800000}%
\pgfsetfillcolor{currentfill}%
\pgfsetlinewidth{1.003750pt}%
\definecolor{currentstroke}{rgb}{0.200000,0.200000,0.800000}%
\pgfsetstrokecolor{currentstroke}%
\pgfsetdash{}{0pt}%
\pgfpathmoveto{\pgfqpoint{3.808269in}{4.758127in}}%
\pgfpathcurveto{\pgfqpoint{3.814093in}{4.758127in}}{\pgfqpoint{3.819679in}{4.760440in}}{\pgfqpoint{3.823797in}{4.764559in}}%
\pgfpathcurveto{\pgfqpoint{3.827916in}{4.768677in}}{\pgfqpoint{3.830229in}{4.774263in}}{\pgfqpoint{3.830229in}{4.780087in}}%
\pgfpathcurveto{\pgfqpoint{3.830229in}{4.785911in}}{\pgfqpoint{3.827916in}{4.791497in}}{\pgfqpoint{3.823797in}{4.795615in}}%
\pgfpathcurveto{\pgfqpoint{3.819679in}{4.799733in}}{\pgfqpoint{3.814093in}{4.802047in}}{\pgfqpoint{3.808269in}{4.802047in}}%
\pgfpathcurveto{\pgfqpoint{3.802445in}{4.802047in}}{\pgfqpoint{3.796859in}{4.799733in}}{\pgfqpoint{3.792741in}{4.795615in}}%
\pgfpathcurveto{\pgfqpoint{3.788623in}{4.791497in}}{\pgfqpoint{3.786309in}{4.785911in}}{\pgfqpoint{3.786309in}{4.780087in}}%
\pgfpathcurveto{\pgfqpoint{3.786309in}{4.774263in}}{\pgfqpoint{3.788623in}{4.768677in}}{\pgfqpoint{3.792741in}{4.764559in}}%
\pgfpathcurveto{\pgfqpoint{3.796859in}{4.760440in}}{\pgfqpoint{3.802445in}{4.758127in}}{\pgfqpoint{3.808269in}{4.758127in}}%
\pgfpathlineto{\pgfqpoint{3.808269in}{4.758127in}}%
\pgfpathclose%
\pgfusepath{stroke,fill}%
\end{pgfscope}%
\begin{pgfscope}%
\pgfpathrectangle{\pgfqpoint{1.000000in}{0.979904in}}{\pgfqpoint{6.200000in}{5.960192in}}%
\pgfusepath{clip}%
\pgfsetbuttcap%
\pgfsetroundjoin%
\definecolor{currentfill}{rgb}{0.200000,0.200000,0.800000}%
\pgfsetfillcolor{currentfill}%
\pgfsetlinewidth{1.003750pt}%
\definecolor{currentstroke}{rgb}{0.200000,0.200000,0.800000}%
\pgfsetstrokecolor{currentstroke}%
\pgfsetdash{}{0pt}%
\pgfpathmoveto{\pgfqpoint{3.864351in}{4.777693in}}%
\pgfpathcurveto{\pgfqpoint{3.870175in}{4.777693in}}{\pgfqpoint{3.875762in}{4.780007in}}{\pgfqpoint{3.879880in}{4.784125in}}%
\pgfpathcurveto{\pgfqpoint{3.883998in}{4.788243in}}{\pgfqpoint{3.886312in}{4.793829in}}{\pgfqpoint{3.886312in}{4.799653in}}%
\pgfpathcurveto{\pgfqpoint{3.886312in}{4.805477in}}{\pgfqpoint{3.883998in}{4.811063in}}{\pgfqpoint{3.879880in}{4.815181in}}%
\pgfpathcurveto{\pgfqpoint{3.875762in}{4.819300in}}{\pgfqpoint{3.870175in}{4.821614in}}{\pgfqpoint{3.864351in}{4.821614in}}%
\pgfpathcurveto{\pgfqpoint{3.858528in}{4.821614in}}{\pgfqpoint{3.852941in}{4.819300in}}{\pgfqpoint{3.848823in}{4.815181in}}%
\pgfpathcurveto{\pgfqpoint{3.844705in}{4.811063in}}{\pgfqpoint{3.842391in}{4.805477in}}{\pgfqpoint{3.842391in}{4.799653in}}%
\pgfpathcurveto{\pgfqpoint{3.842391in}{4.793829in}}{\pgfqpoint{3.844705in}{4.788243in}}{\pgfqpoint{3.848823in}{4.784125in}}%
\pgfpathcurveto{\pgfqpoint{3.852941in}{4.780007in}}{\pgfqpoint{3.858528in}{4.777693in}}{\pgfqpoint{3.864351in}{4.777693in}}%
\pgfpathlineto{\pgfqpoint{3.864351in}{4.777693in}}%
\pgfpathclose%
\pgfusepath{stroke,fill}%
\end{pgfscope}%
\begin{pgfscope}%
\pgfpathrectangle{\pgfqpoint{1.000000in}{0.979904in}}{\pgfqpoint{6.200000in}{5.960192in}}%
\pgfusepath{clip}%
\pgfsetbuttcap%
\pgfsetroundjoin%
\definecolor{currentfill}{rgb}{0.200000,0.200000,0.800000}%
\pgfsetfillcolor{currentfill}%
\pgfsetlinewidth{1.003750pt}%
\definecolor{currentstroke}{rgb}{0.200000,0.200000,0.800000}%
\pgfsetstrokecolor{currentstroke}%
\pgfsetdash{}{0pt}%
\pgfpathmoveto{\pgfqpoint{3.932670in}{4.741693in}}%
\pgfpathcurveto{\pgfqpoint{3.938494in}{4.741693in}}{\pgfqpoint{3.944080in}{4.744007in}}{\pgfqpoint{3.948198in}{4.748125in}}%
\pgfpathcurveto{\pgfqpoint{3.952316in}{4.752244in}}{\pgfqpoint{3.954630in}{4.757830in}}{\pgfqpoint{3.954630in}{4.763654in}}%
\pgfpathcurveto{\pgfqpoint{3.954630in}{4.769478in}}{\pgfqpoint{3.952316in}{4.775064in}}{\pgfqpoint{3.948198in}{4.779182in}}%
\pgfpathcurveto{\pgfqpoint{3.944080in}{4.783300in}}{\pgfqpoint{3.938494in}{4.785614in}}{\pgfqpoint{3.932670in}{4.785614in}}%
\pgfpathcurveto{\pgfqpoint{3.926846in}{4.785614in}}{\pgfqpoint{3.921260in}{4.783300in}}{\pgfqpoint{3.917142in}{4.779182in}}%
\pgfpathcurveto{\pgfqpoint{3.913024in}{4.775064in}}{\pgfqpoint{3.910710in}{4.769478in}}{\pgfqpoint{3.910710in}{4.763654in}}%
\pgfpathcurveto{\pgfqpoint{3.910710in}{4.757830in}}{\pgfqpoint{3.913024in}{4.752244in}}{\pgfqpoint{3.917142in}{4.748125in}}%
\pgfpathcurveto{\pgfqpoint{3.921260in}{4.744007in}}{\pgfqpoint{3.926846in}{4.741693in}}{\pgfqpoint{3.932670in}{4.741693in}}%
\pgfpathlineto{\pgfqpoint{3.932670in}{4.741693in}}%
\pgfpathclose%
\pgfusepath{stroke,fill}%
\end{pgfscope}%
\begin{pgfscope}%
\pgfpathrectangle{\pgfqpoint{1.000000in}{0.979904in}}{\pgfqpoint{6.200000in}{5.960192in}}%
\pgfusepath{clip}%
\pgfsetbuttcap%
\pgfsetroundjoin%
\definecolor{currentfill}{rgb}{0.200000,0.200000,0.800000}%
\pgfsetfillcolor{currentfill}%
\pgfsetlinewidth{1.003750pt}%
\definecolor{currentstroke}{rgb}{0.200000,0.200000,0.800000}%
\pgfsetstrokecolor{currentstroke}%
\pgfsetdash{}{0pt}%
\pgfpathmoveto{\pgfqpoint{4.001868in}{4.724691in}}%
\pgfpathcurveto{\pgfqpoint{4.007692in}{4.724691in}}{\pgfqpoint{4.013278in}{4.727005in}}{\pgfqpoint{4.017397in}{4.731123in}}%
\pgfpathcurveto{\pgfqpoint{4.021515in}{4.735241in}}{\pgfqpoint{4.023829in}{4.740828in}}{\pgfqpoint{4.023829in}{4.746652in}}%
\pgfpathcurveto{\pgfqpoint{4.023829in}{4.752476in}}{\pgfqpoint{4.021515in}{4.758062in}}{\pgfqpoint{4.017397in}{4.762180in}}%
\pgfpathcurveto{\pgfqpoint{4.013278in}{4.766298in}}{\pgfqpoint{4.007692in}{4.768612in}}{\pgfqpoint{4.001868in}{4.768612in}}%
\pgfpathcurveto{\pgfqpoint{3.996044in}{4.768612in}}{\pgfqpoint{3.990458in}{4.766298in}}{\pgfqpoint{3.986340in}{4.762180in}}%
\pgfpathcurveto{\pgfqpoint{3.982222in}{4.758062in}}{\pgfqpoint{3.979908in}{4.752476in}}{\pgfqpoint{3.979908in}{4.746652in}}%
\pgfpathcurveto{\pgfqpoint{3.979908in}{4.740828in}}{\pgfqpoint{3.982222in}{4.735241in}}{\pgfqpoint{3.986340in}{4.731123in}}%
\pgfpathcurveto{\pgfqpoint{3.990458in}{4.727005in}}{\pgfqpoint{3.996044in}{4.724691in}}{\pgfqpoint{4.001868in}{4.724691in}}%
\pgfpathlineto{\pgfqpoint{4.001868in}{4.724691in}}%
\pgfpathclose%
\pgfusepath{stroke,fill}%
\end{pgfscope}%
\begin{pgfscope}%
\pgfpathrectangle{\pgfqpoint{1.000000in}{0.979904in}}{\pgfqpoint{6.200000in}{5.960192in}}%
\pgfusepath{clip}%
\pgfsetbuttcap%
\pgfsetroundjoin%
\definecolor{currentfill}{rgb}{0.200000,0.200000,0.800000}%
\pgfsetfillcolor{currentfill}%
\pgfsetlinewidth{1.003750pt}%
\definecolor{currentstroke}{rgb}{0.200000,0.200000,0.800000}%
\pgfsetstrokecolor{currentstroke}%
\pgfsetdash{}{0pt}%
\pgfpathmoveto{\pgfqpoint{4.044598in}{4.790045in}}%
\pgfpathcurveto{\pgfqpoint{4.050422in}{4.790045in}}{\pgfqpoint{4.056008in}{4.792359in}}{\pgfqpoint{4.060126in}{4.796477in}}%
\pgfpathcurveto{\pgfqpoint{4.064244in}{4.800595in}}{\pgfqpoint{4.066558in}{4.806181in}}{\pgfqpoint{4.066558in}{4.812005in}}%
\pgfpathcurveto{\pgfqpoint{4.066558in}{4.817829in}}{\pgfqpoint{4.064244in}{4.823415in}}{\pgfqpoint{4.060126in}{4.827533in}}%
\pgfpathcurveto{\pgfqpoint{4.056008in}{4.831651in}}{\pgfqpoint{4.050422in}{4.833965in}}{\pgfqpoint{4.044598in}{4.833965in}}%
\pgfpathcurveto{\pgfqpoint{4.038774in}{4.833965in}}{\pgfqpoint{4.033188in}{4.831651in}}{\pgfqpoint{4.029069in}{4.827533in}}%
\pgfpathcurveto{\pgfqpoint{4.024951in}{4.823415in}}{\pgfqpoint{4.022637in}{4.817829in}}{\pgfqpoint{4.022637in}{4.812005in}}%
\pgfpathcurveto{\pgfqpoint{4.022637in}{4.806181in}}{\pgfqpoint{4.024951in}{4.800595in}}{\pgfqpoint{4.029069in}{4.796477in}}%
\pgfpathcurveto{\pgfqpoint{4.033188in}{4.792359in}}{\pgfqpoint{4.038774in}{4.790045in}}{\pgfqpoint{4.044598in}{4.790045in}}%
\pgfpathlineto{\pgfqpoint{4.044598in}{4.790045in}}%
\pgfpathclose%
\pgfusepath{stroke,fill}%
\end{pgfscope}%
\begin{pgfscope}%
\pgfpathrectangle{\pgfqpoint{1.000000in}{0.979904in}}{\pgfqpoint{6.200000in}{5.960192in}}%
\pgfusepath{clip}%
\pgfsetbuttcap%
\pgfsetroundjoin%
\definecolor{currentfill}{rgb}{0.200000,0.200000,0.800000}%
\pgfsetfillcolor{currentfill}%
\pgfsetlinewidth{1.003750pt}%
\definecolor{currentstroke}{rgb}{0.200000,0.200000,0.800000}%
\pgfsetstrokecolor{currentstroke}%
\pgfsetdash{}{0pt}%
\pgfpathmoveto{\pgfqpoint{4.101423in}{4.810547in}}%
\pgfpathcurveto{\pgfqpoint{4.107247in}{4.810547in}}{\pgfqpoint{4.112833in}{4.812861in}}{\pgfqpoint{4.116951in}{4.816979in}}%
\pgfpathcurveto{\pgfqpoint{4.121069in}{4.821097in}}{\pgfqpoint{4.123383in}{4.826684in}}{\pgfqpoint{4.123383in}{4.832507in}}%
\pgfpathcurveto{\pgfqpoint{4.123383in}{4.838331in}}{\pgfqpoint{4.121069in}{4.843918in}}{\pgfqpoint{4.116951in}{4.848036in}}%
\pgfpathcurveto{\pgfqpoint{4.112833in}{4.852154in}}{\pgfqpoint{4.107247in}{4.854468in}}{\pgfqpoint{4.101423in}{4.854468in}}%
\pgfpathcurveto{\pgfqpoint{4.095599in}{4.854468in}}{\pgfqpoint{4.090013in}{4.852154in}}{\pgfqpoint{4.085895in}{4.848036in}}%
\pgfpathcurveto{\pgfqpoint{4.081777in}{4.843918in}}{\pgfqpoint{4.079463in}{4.838331in}}{\pgfqpoint{4.079463in}{4.832507in}}%
\pgfpathcurveto{\pgfqpoint{4.079463in}{4.826684in}}{\pgfqpoint{4.081777in}{4.821097in}}{\pgfqpoint{4.085895in}{4.816979in}}%
\pgfpathcurveto{\pgfqpoint{4.090013in}{4.812861in}}{\pgfqpoint{4.095599in}{4.810547in}}{\pgfqpoint{4.101423in}{4.810547in}}%
\pgfpathlineto{\pgfqpoint{4.101423in}{4.810547in}}%
\pgfpathclose%
\pgfusepath{stroke,fill}%
\end{pgfscope}%
\begin{pgfscope}%
\pgfpathrectangle{\pgfqpoint{1.000000in}{0.979904in}}{\pgfqpoint{6.200000in}{5.960192in}}%
\pgfusepath{clip}%
\pgfsetbuttcap%
\pgfsetroundjoin%
\definecolor{currentfill}{rgb}{0.200000,0.200000,0.800000}%
\pgfsetfillcolor{currentfill}%
\pgfsetlinewidth{1.003750pt}%
\definecolor{currentstroke}{rgb}{0.200000,0.200000,0.800000}%
\pgfsetstrokecolor{currentstroke}%
\pgfsetdash{}{0pt}%
\pgfpathmoveto{\pgfqpoint{4.182063in}{4.787933in}}%
\pgfpathcurveto{\pgfqpoint{4.187887in}{4.787933in}}{\pgfqpoint{4.193473in}{4.790247in}}{\pgfqpoint{4.197591in}{4.794365in}}%
\pgfpathcurveto{\pgfqpoint{4.201709in}{4.798484in}}{\pgfqpoint{4.204023in}{4.804070in}}{\pgfqpoint{4.204023in}{4.809894in}}%
\pgfpathcurveto{\pgfqpoint{4.204023in}{4.815718in}}{\pgfqpoint{4.201709in}{4.821304in}}{\pgfqpoint{4.197591in}{4.825422in}}%
\pgfpathcurveto{\pgfqpoint{4.193473in}{4.829540in}}{\pgfqpoint{4.187887in}{4.831854in}}{\pgfqpoint{4.182063in}{4.831854in}}%
\pgfpathcurveto{\pgfqpoint{4.176239in}{4.831854in}}{\pgfqpoint{4.170653in}{4.829540in}}{\pgfqpoint{4.166534in}{4.825422in}}%
\pgfpathcurveto{\pgfqpoint{4.162416in}{4.821304in}}{\pgfqpoint{4.160102in}{4.815718in}}{\pgfqpoint{4.160102in}{4.809894in}}%
\pgfpathcurveto{\pgfqpoint{4.160102in}{4.804070in}}{\pgfqpoint{4.162416in}{4.798484in}}{\pgfqpoint{4.166534in}{4.794365in}}%
\pgfpathcurveto{\pgfqpoint{4.170653in}{4.790247in}}{\pgfqpoint{4.176239in}{4.787933in}}{\pgfqpoint{4.182063in}{4.787933in}}%
\pgfpathlineto{\pgfqpoint{4.182063in}{4.787933in}}%
\pgfpathclose%
\pgfusepath{stroke,fill}%
\end{pgfscope}%
\begin{pgfscope}%
\pgfpathrectangle{\pgfqpoint{1.000000in}{0.979904in}}{\pgfqpoint{6.200000in}{5.960192in}}%
\pgfusepath{clip}%
\pgfsetbuttcap%
\pgfsetroundjoin%
\definecolor{currentfill}{rgb}{0.200000,0.200000,0.800000}%
\pgfsetfillcolor{currentfill}%
\pgfsetlinewidth{1.003750pt}%
\definecolor{currentstroke}{rgb}{0.200000,0.200000,0.800000}%
\pgfsetstrokecolor{currentstroke}%
\pgfsetdash{}{0pt}%
\pgfpathmoveto{\pgfqpoint{4.204016in}{4.873863in}}%
\pgfpathcurveto{\pgfqpoint{4.209840in}{4.873863in}}{\pgfqpoint{4.215426in}{4.876177in}}{\pgfqpoint{4.219544in}{4.880295in}}%
\pgfpathcurveto{\pgfqpoint{4.223662in}{4.884413in}}{\pgfqpoint{4.225976in}{4.889999in}}{\pgfqpoint{4.225976in}{4.895823in}}%
\pgfpathcurveto{\pgfqpoint{4.225976in}{4.901647in}}{\pgfqpoint{4.223662in}{4.907233in}}{\pgfqpoint{4.219544in}{4.911351in}}%
\pgfpathcurveto{\pgfqpoint{4.215426in}{4.915469in}}{\pgfqpoint{4.209840in}{4.917783in}}{\pgfqpoint{4.204016in}{4.917783in}}%
\pgfpathcurveto{\pgfqpoint{4.198192in}{4.917783in}}{\pgfqpoint{4.192606in}{4.915469in}}{\pgfqpoint{4.188488in}{4.911351in}}%
\pgfpathcurveto{\pgfqpoint{4.184370in}{4.907233in}}{\pgfqpoint{4.182056in}{4.901647in}}{\pgfqpoint{4.182056in}{4.895823in}}%
\pgfpathcurveto{\pgfqpoint{4.182056in}{4.889999in}}{\pgfqpoint{4.184370in}{4.884413in}}{\pgfqpoint{4.188488in}{4.880295in}}%
\pgfpathcurveto{\pgfqpoint{4.192606in}{4.876177in}}{\pgfqpoint{4.198192in}{4.873863in}}{\pgfqpoint{4.204016in}{4.873863in}}%
\pgfpathlineto{\pgfqpoint{4.204016in}{4.873863in}}%
\pgfpathclose%
\pgfusepath{stroke,fill}%
\end{pgfscope}%
\begin{pgfscope}%
\pgfpathrectangle{\pgfqpoint{1.000000in}{0.979904in}}{\pgfqpoint{6.200000in}{5.960192in}}%
\pgfusepath{clip}%
\pgfsetbuttcap%
\pgfsetroundjoin%
\definecolor{currentfill}{rgb}{0.200000,0.200000,0.800000}%
\pgfsetfillcolor{currentfill}%
\pgfsetlinewidth{1.003750pt}%
\definecolor{currentstroke}{rgb}{0.200000,0.200000,0.800000}%
\pgfsetstrokecolor{currentstroke}%
\pgfsetdash{}{0pt}%
\pgfpathmoveto{\pgfqpoint{4.239018in}{4.928032in}}%
\pgfpathcurveto{\pgfqpoint{4.244842in}{4.928032in}}{\pgfqpoint{4.250428in}{4.930345in}}{\pgfqpoint{4.254546in}{4.934464in}}%
\pgfpathcurveto{\pgfqpoint{4.258664in}{4.938582in}}{\pgfqpoint{4.260978in}{4.944168in}}{\pgfqpoint{4.260978in}{4.949992in}}%
\pgfpathcurveto{\pgfqpoint{4.260978in}{4.955816in}}{\pgfqpoint{4.258664in}{4.961402in}}{\pgfqpoint{4.254546in}{4.965520in}}%
\pgfpathcurveto{\pgfqpoint{4.250428in}{4.969638in}}{\pgfqpoint{4.244842in}{4.971952in}}{\pgfqpoint{4.239018in}{4.971952in}}%
\pgfpathcurveto{\pgfqpoint{4.233194in}{4.971952in}}{\pgfqpoint{4.227608in}{4.969638in}}{\pgfqpoint{4.223490in}{4.965520in}}%
\pgfpathcurveto{\pgfqpoint{4.219371in}{4.961402in}}{\pgfqpoint{4.217058in}{4.955816in}}{\pgfqpoint{4.217058in}{4.949992in}}%
\pgfpathcurveto{\pgfqpoint{4.217058in}{4.944168in}}{\pgfqpoint{4.219371in}{4.938582in}}{\pgfqpoint{4.223490in}{4.934464in}}%
\pgfpathcurveto{\pgfqpoint{4.227608in}{4.930345in}}{\pgfqpoint{4.233194in}{4.928032in}}{\pgfqpoint{4.239018in}{4.928032in}}%
\pgfpathlineto{\pgfqpoint{4.239018in}{4.928032in}}%
\pgfpathclose%
\pgfusepath{stroke,fill}%
\end{pgfscope}%
\begin{pgfscope}%
\pgfpathrectangle{\pgfqpoint{1.000000in}{0.979904in}}{\pgfqpoint{6.200000in}{5.960192in}}%
\pgfusepath{clip}%
\pgfsetbuttcap%
\pgfsetroundjoin%
\definecolor{currentfill}{rgb}{0.200000,0.200000,0.800000}%
\pgfsetfillcolor{currentfill}%
\pgfsetlinewidth{1.003750pt}%
\definecolor{currentstroke}{rgb}{0.200000,0.200000,0.800000}%
\pgfsetstrokecolor{currentstroke}%
\pgfsetdash{}{0pt}%
\pgfpathmoveto{\pgfqpoint{4.318302in}{4.922901in}}%
\pgfpathcurveto{\pgfqpoint{4.324126in}{4.922901in}}{\pgfqpoint{4.329712in}{4.925215in}}{\pgfqpoint{4.333830in}{4.929333in}}%
\pgfpathcurveto{\pgfqpoint{4.337948in}{4.933452in}}{\pgfqpoint{4.340262in}{4.939038in}}{\pgfqpoint{4.340262in}{4.944862in}}%
\pgfpathcurveto{\pgfqpoint{4.340262in}{4.950686in}}{\pgfqpoint{4.337948in}{4.956272in}}{\pgfqpoint{4.333830in}{4.960390in}}%
\pgfpathcurveto{\pgfqpoint{4.329712in}{4.964508in}}{\pgfqpoint{4.324126in}{4.966822in}}{\pgfqpoint{4.318302in}{4.966822in}}%
\pgfpathcurveto{\pgfqpoint{4.312478in}{4.966822in}}{\pgfqpoint{4.306892in}{4.964508in}}{\pgfqpoint{4.302773in}{4.960390in}}%
\pgfpathcurveto{\pgfqpoint{4.298655in}{4.956272in}}{\pgfqpoint{4.296341in}{4.950686in}}{\pgfqpoint{4.296341in}{4.944862in}}%
\pgfpathcurveto{\pgfqpoint{4.296341in}{4.939038in}}{\pgfqpoint{4.298655in}{4.933452in}}{\pgfqpoint{4.302773in}{4.929333in}}%
\pgfpathcurveto{\pgfqpoint{4.306892in}{4.925215in}}{\pgfqpoint{4.312478in}{4.922901in}}{\pgfqpoint{4.318302in}{4.922901in}}%
\pgfpathlineto{\pgfqpoint{4.318302in}{4.922901in}}%
\pgfpathclose%
\pgfusepath{stroke,fill}%
\end{pgfscope}%
\begin{pgfscope}%
\pgfpathrectangle{\pgfqpoint{1.000000in}{0.979904in}}{\pgfqpoint{6.200000in}{5.960192in}}%
\pgfusepath{clip}%
\pgfsetbuttcap%
\pgfsetroundjoin%
\definecolor{currentfill}{rgb}{0.200000,0.200000,0.800000}%
\pgfsetfillcolor{currentfill}%
\pgfsetlinewidth{1.003750pt}%
\definecolor{currentstroke}{rgb}{0.200000,0.200000,0.800000}%
\pgfsetstrokecolor{currentstroke}%
\pgfsetdash{}{0pt}%
\pgfpathmoveto{\pgfqpoint{4.331414in}{4.999503in}}%
\pgfpathcurveto{\pgfqpoint{4.337238in}{4.999503in}}{\pgfqpoint{4.342824in}{5.001817in}}{\pgfqpoint{4.346942in}{5.005935in}}%
\pgfpathcurveto{\pgfqpoint{4.351060in}{5.010053in}}{\pgfqpoint{4.353374in}{5.015639in}}{\pgfqpoint{4.353374in}{5.021463in}}%
\pgfpathcurveto{\pgfqpoint{4.353374in}{5.027287in}}{\pgfqpoint{4.351060in}{5.032873in}}{\pgfqpoint{4.346942in}{5.036992in}}%
\pgfpathcurveto{\pgfqpoint{4.342824in}{5.041110in}}{\pgfqpoint{4.337238in}{5.043424in}}{\pgfqpoint{4.331414in}{5.043424in}}%
\pgfpathcurveto{\pgfqpoint{4.325590in}{5.043424in}}{\pgfqpoint{4.320004in}{5.041110in}}{\pgfqpoint{4.315886in}{5.036992in}}%
\pgfpathcurveto{\pgfqpoint{4.311768in}{5.032873in}}{\pgfqpoint{4.309454in}{5.027287in}}{\pgfqpoint{4.309454in}{5.021463in}}%
\pgfpathcurveto{\pgfqpoint{4.309454in}{5.015639in}}{\pgfqpoint{4.311768in}{5.010053in}}{\pgfqpoint{4.315886in}{5.005935in}}%
\pgfpathcurveto{\pgfqpoint{4.320004in}{5.001817in}}{\pgfqpoint{4.325590in}{4.999503in}}{\pgfqpoint{4.331414in}{4.999503in}}%
\pgfpathlineto{\pgfqpoint{4.331414in}{4.999503in}}%
\pgfpathclose%
\pgfusepath{stroke,fill}%
\end{pgfscope}%
\begin{pgfscope}%
\pgfpathrectangle{\pgfqpoint{1.000000in}{0.979904in}}{\pgfqpoint{6.200000in}{5.960192in}}%
\pgfusepath{clip}%
\pgfsetbuttcap%
\pgfsetroundjoin%
\definecolor{currentfill}{rgb}{0.200000,0.200000,0.800000}%
\pgfsetfillcolor{currentfill}%
\pgfsetlinewidth{1.003750pt}%
\definecolor{currentstroke}{rgb}{0.200000,0.200000,0.800000}%
\pgfsetstrokecolor{currentstroke}%
\pgfsetdash{}{0pt}%
\pgfpathmoveto{\pgfqpoint{4.368842in}{5.044616in}}%
\pgfpathcurveto{\pgfqpoint{4.374666in}{5.044616in}}{\pgfqpoint{4.380252in}{5.046930in}}{\pgfqpoint{4.384371in}{5.051048in}}%
\pgfpathcurveto{\pgfqpoint{4.388489in}{5.055166in}}{\pgfqpoint{4.390803in}{5.060752in}}{\pgfqpoint{4.390803in}{5.066576in}}%
\pgfpathcurveto{\pgfqpoint{4.390803in}{5.072400in}}{\pgfqpoint{4.388489in}{5.077986in}}{\pgfqpoint{4.384371in}{5.082105in}}%
\pgfpathcurveto{\pgfqpoint{4.380252in}{5.086223in}}{\pgfqpoint{4.374666in}{5.088537in}}{\pgfqpoint{4.368842in}{5.088537in}}%
\pgfpathcurveto{\pgfqpoint{4.363018in}{5.088537in}}{\pgfqpoint{4.357432in}{5.086223in}}{\pgfqpoint{4.353314in}{5.082105in}}%
\pgfpathcurveto{\pgfqpoint{4.349196in}{5.077986in}}{\pgfqpoint{4.346882in}{5.072400in}}{\pgfqpoint{4.346882in}{5.066576in}}%
\pgfpathcurveto{\pgfqpoint{4.346882in}{5.060752in}}{\pgfqpoint{4.349196in}{5.055166in}}{\pgfqpoint{4.353314in}{5.051048in}}%
\pgfpathcurveto{\pgfqpoint{4.357432in}{5.046930in}}{\pgfqpoint{4.363018in}{5.044616in}}{\pgfqpoint{4.368842in}{5.044616in}}%
\pgfpathlineto{\pgfqpoint{4.368842in}{5.044616in}}%
\pgfpathclose%
\pgfusepath{stroke,fill}%
\end{pgfscope}%
\begin{pgfscope}%
\pgfpathrectangle{\pgfqpoint{1.000000in}{0.979904in}}{\pgfqpoint{6.200000in}{5.960192in}}%
\pgfusepath{clip}%
\pgfsetbuttcap%
\pgfsetroundjoin%
\definecolor{currentfill}{rgb}{0.200000,0.200000,0.800000}%
\pgfsetfillcolor{currentfill}%
\pgfsetlinewidth{1.003750pt}%
\definecolor{currentstroke}{rgb}{0.200000,0.200000,0.800000}%
\pgfsetstrokecolor{currentstroke}%
\pgfsetdash{}{0pt}%
\pgfpathmoveto{\pgfqpoint{4.474323in}{5.031957in}}%
\pgfpathcurveto{\pgfqpoint{4.480147in}{5.031957in}}{\pgfqpoint{4.485733in}{5.034271in}}{\pgfqpoint{4.489851in}{5.038389in}}%
\pgfpathcurveto{\pgfqpoint{4.493969in}{5.042507in}}{\pgfqpoint{4.496283in}{5.048093in}}{\pgfqpoint{4.496283in}{5.053917in}}%
\pgfpathcurveto{\pgfqpoint{4.496283in}{5.059741in}}{\pgfqpoint{4.493969in}{5.065327in}}{\pgfqpoint{4.489851in}{5.069445in}}%
\pgfpathcurveto{\pgfqpoint{4.485733in}{5.073563in}}{\pgfqpoint{4.480147in}{5.075877in}}{\pgfqpoint{4.474323in}{5.075877in}}%
\pgfpathcurveto{\pgfqpoint{4.468499in}{5.075877in}}{\pgfqpoint{4.462913in}{5.073563in}}{\pgfqpoint{4.458795in}{5.069445in}}%
\pgfpathcurveto{\pgfqpoint{4.454677in}{5.065327in}}{\pgfqpoint{4.452363in}{5.059741in}}{\pgfqpoint{4.452363in}{5.053917in}}%
\pgfpathcurveto{\pgfqpoint{4.452363in}{5.048093in}}{\pgfqpoint{4.454677in}{5.042507in}}{\pgfqpoint{4.458795in}{5.038389in}}%
\pgfpathcurveto{\pgfqpoint{4.462913in}{5.034271in}}{\pgfqpoint{4.468499in}{5.031957in}}{\pgfqpoint{4.474323in}{5.031957in}}%
\pgfpathlineto{\pgfqpoint{4.474323in}{5.031957in}}%
\pgfpathclose%
\pgfusepath{stroke,fill}%
\end{pgfscope}%
\begin{pgfscope}%
\pgfpathrectangle{\pgfqpoint{1.000000in}{0.979904in}}{\pgfqpoint{6.200000in}{5.960192in}}%
\pgfusepath{clip}%
\pgfsetbuttcap%
\pgfsetroundjoin%
\definecolor{currentfill}{rgb}{0.200000,0.200000,0.800000}%
\pgfsetfillcolor{currentfill}%
\pgfsetlinewidth{1.003750pt}%
\definecolor{currentstroke}{rgb}{0.200000,0.200000,0.800000}%
\pgfsetstrokecolor{currentstroke}%
\pgfsetdash{}{0pt}%
\pgfpathmoveto{\pgfqpoint{4.446781in}{5.131250in}}%
\pgfpathcurveto{\pgfqpoint{4.452605in}{5.131250in}}{\pgfqpoint{4.458191in}{5.133564in}}{\pgfqpoint{4.462310in}{5.137682in}}%
\pgfpathcurveto{\pgfqpoint{4.466428in}{5.141800in}}{\pgfqpoint{4.468742in}{5.147386in}}{\pgfqpoint{4.468742in}{5.153210in}}%
\pgfpathcurveto{\pgfqpoint{4.468742in}{5.159034in}}{\pgfqpoint{4.466428in}{5.164620in}}{\pgfqpoint{4.462310in}{5.168739in}}%
\pgfpathcurveto{\pgfqpoint{4.458191in}{5.172857in}}{\pgfqpoint{4.452605in}{5.175171in}}{\pgfqpoint{4.446781in}{5.175171in}}%
\pgfpathcurveto{\pgfqpoint{4.440957in}{5.175171in}}{\pgfqpoint{4.435371in}{5.172857in}}{\pgfqpoint{4.431253in}{5.168739in}}%
\pgfpathcurveto{\pgfqpoint{4.427135in}{5.164620in}}{\pgfqpoint{4.424821in}{5.159034in}}{\pgfqpoint{4.424821in}{5.153210in}}%
\pgfpathcurveto{\pgfqpoint{4.424821in}{5.147386in}}{\pgfqpoint{4.427135in}{5.141800in}}{\pgfqpoint{4.431253in}{5.137682in}}%
\pgfpathcurveto{\pgfqpoint{4.435371in}{5.133564in}}{\pgfqpoint{4.440957in}{5.131250in}}{\pgfqpoint{4.446781in}{5.131250in}}%
\pgfpathlineto{\pgfqpoint{4.446781in}{5.131250in}}%
\pgfpathclose%
\pgfusepath{stroke,fill}%
\end{pgfscope}%
\begin{pgfscope}%
\pgfpathrectangle{\pgfqpoint{1.000000in}{0.979904in}}{\pgfqpoint{6.200000in}{5.960192in}}%
\pgfusepath{clip}%
\pgfsetbuttcap%
\pgfsetroundjoin%
\definecolor{currentfill}{rgb}{0.200000,0.200000,0.800000}%
\pgfsetfillcolor{currentfill}%
\pgfsetlinewidth{1.003750pt}%
\definecolor{currentstroke}{rgb}{0.200000,0.200000,0.800000}%
\pgfsetstrokecolor{currentstroke}%
\pgfsetdash{}{0pt}%
\pgfpathmoveto{\pgfqpoint{4.374411in}{5.247132in}}%
\pgfpathcurveto{\pgfqpoint{4.380235in}{5.247132in}}{\pgfqpoint{4.385821in}{5.249446in}}{\pgfqpoint{4.389939in}{5.253564in}}%
\pgfpathcurveto{\pgfqpoint{4.394057in}{5.257682in}}{\pgfqpoint{4.396371in}{5.263268in}}{\pgfqpoint{4.396371in}{5.269092in}}%
\pgfpathcurveto{\pgfqpoint{4.396371in}{5.274916in}}{\pgfqpoint{4.394057in}{5.280502in}}{\pgfqpoint{4.389939in}{5.284620in}}%
\pgfpathcurveto{\pgfqpoint{4.385821in}{5.288738in}}{\pgfqpoint{4.380235in}{5.291052in}}{\pgfqpoint{4.374411in}{5.291052in}}%
\pgfpathcurveto{\pgfqpoint{4.368587in}{5.291052in}}{\pgfqpoint{4.363001in}{5.288738in}}{\pgfqpoint{4.358883in}{5.284620in}}%
\pgfpathcurveto{\pgfqpoint{4.354765in}{5.280502in}}{\pgfqpoint{4.352451in}{5.274916in}}{\pgfqpoint{4.352451in}{5.269092in}}%
\pgfpathcurveto{\pgfqpoint{4.352451in}{5.263268in}}{\pgfqpoint{4.354765in}{5.257682in}}{\pgfqpoint{4.358883in}{5.253564in}}%
\pgfpathcurveto{\pgfqpoint{4.363001in}{5.249446in}}{\pgfqpoint{4.368587in}{5.247132in}}{\pgfqpoint{4.374411in}{5.247132in}}%
\pgfpathlineto{\pgfqpoint{4.374411in}{5.247132in}}%
\pgfpathclose%
\pgfusepath{stroke,fill}%
\end{pgfscope}%
\begin{pgfscope}%
\pgfpathrectangle{\pgfqpoint{1.000000in}{0.979904in}}{\pgfqpoint{6.200000in}{5.960192in}}%
\pgfusepath{clip}%
\pgfsetbuttcap%
\pgfsetroundjoin%
\definecolor{currentfill}{rgb}{0.200000,0.200000,0.800000}%
\pgfsetfillcolor{currentfill}%
\pgfsetlinewidth{1.003750pt}%
\definecolor{currentstroke}{rgb}{0.200000,0.200000,0.800000}%
\pgfsetstrokecolor{currentstroke}%
\pgfsetdash{}{0pt}%
\pgfpathmoveto{\pgfqpoint{4.539440in}{5.212996in}}%
\pgfpathcurveto{\pgfqpoint{4.545264in}{5.212996in}}{\pgfqpoint{4.550850in}{5.215310in}}{\pgfqpoint{4.554968in}{5.219428in}}%
\pgfpathcurveto{\pgfqpoint{4.559086in}{5.223546in}}{\pgfqpoint{4.561400in}{5.229132in}}{\pgfqpoint{4.561400in}{5.234956in}}%
\pgfpathcurveto{\pgfqpoint{4.561400in}{5.240780in}}{\pgfqpoint{4.559086in}{5.246366in}}{\pgfqpoint{4.554968in}{5.250485in}}%
\pgfpathcurveto{\pgfqpoint{4.550850in}{5.254603in}}{\pgfqpoint{4.545264in}{5.256917in}}{\pgfqpoint{4.539440in}{5.256917in}}%
\pgfpathcurveto{\pgfqpoint{4.533616in}{5.256917in}}{\pgfqpoint{4.528030in}{5.254603in}}{\pgfqpoint{4.523912in}{5.250485in}}%
\pgfpathcurveto{\pgfqpoint{4.519793in}{5.246366in}}{\pgfqpoint{4.517480in}{5.240780in}}{\pgfqpoint{4.517480in}{5.234956in}}%
\pgfpathcurveto{\pgfqpoint{4.517480in}{5.229132in}}{\pgfqpoint{4.519793in}{5.223546in}}{\pgfqpoint{4.523912in}{5.219428in}}%
\pgfpathcurveto{\pgfqpoint{4.528030in}{5.215310in}}{\pgfqpoint{4.533616in}{5.212996in}}{\pgfqpoint{4.539440in}{5.212996in}}%
\pgfpathlineto{\pgfqpoint{4.539440in}{5.212996in}}%
\pgfpathclose%
\pgfusepath{stroke,fill}%
\end{pgfscope}%
\begin{pgfscope}%
\pgfpathrectangle{\pgfqpoint{1.000000in}{0.979904in}}{\pgfqpoint{6.200000in}{5.960192in}}%
\pgfusepath{clip}%
\pgfsetbuttcap%
\pgfsetroundjoin%
\definecolor{currentfill}{rgb}{0.200000,0.200000,0.800000}%
\pgfsetfillcolor{currentfill}%
\pgfsetlinewidth{1.003750pt}%
\definecolor{currentstroke}{rgb}{0.200000,0.200000,0.800000}%
\pgfsetstrokecolor{currentstroke}%
\pgfsetdash{}{0pt}%
\pgfpathmoveto{\pgfqpoint{4.531678in}{5.283700in}}%
\pgfpathcurveto{\pgfqpoint{4.537502in}{5.283700in}}{\pgfqpoint{4.543088in}{5.286014in}}{\pgfqpoint{4.547207in}{5.290132in}}%
\pgfpathcurveto{\pgfqpoint{4.551325in}{5.294250in}}{\pgfqpoint{4.553639in}{5.299836in}}{\pgfqpoint{4.553639in}{5.305660in}}%
\pgfpathcurveto{\pgfqpoint{4.553639in}{5.311484in}}{\pgfqpoint{4.551325in}{5.317070in}}{\pgfqpoint{4.547207in}{5.321189in}}%
\pgfpathcurveto{\pgfqpoint{4.543088in}{5.325307in}}{\pgfqpoint{4.537502in}{5.327621in}}{\pgfqpoint{4.531678in}{5.327621in}}%
\pgfpathcurveto{\pgfqpoint{4.525854in}{5.327621in}}{\pgfqpoint{4.520268in}{5.325307in}}{\pgfqpoint{4.516150in}{5.321189in}}%
\pgfpathcurveto{\pgfqpoint{4.512032in}{5.317070in}}{\pgfqpoint{4.509718in}{5.311484in}}{\pgfqpoint{4.509718in}{5.305660in}}%
\pgfpathcurveto{\pgfqpoint{4.509718in}{5.299836in}}{\pgfqpoint{4.512032in}{5.294250in}}{\pgfqpoint{4.516150in}{5.290132in}}%
\pgfpathcurveto{\pgfqpoint{4.520268in}{5.286014in}}{\pgfqpoint{4.525854in}{5.283700in}}{\pgfqpoint{4.531678in}{5.283700in}}%
\pgfpathlineto{\pgfqpoint{4.531678in}{5.283700in}}%
\pgfpathclose%
\pgfusepath{stroke,fill}%
\end{pgfscope}%
\begin{pgfscope}%
\pgfpathrectangle{\pgfqpoint{1.000000in}{0.979904in}}{\pgfqpoint{6.200000in}{5.960192in}}%
\pgfusepath{clip}%
\pgfsetbuttcap%
\pgfsetroundjoin%
\definecolor{currentfill}{rgb}{0.200000,0.200000,0.800000}%
\pgfsetfillcolor{currentfill}%
\pgfsetlinewidth{1.003750pt}%
\definecolor{currentstroke}{rgb}{0.200000,0.200000,0.800000}%
\pgfsetstrokecolor{currentstroke}%
\pgfsetdash{}{0pt}%
\pgfpathmoveto{\pgfqpoint{4.560188in}{5.334773in}}%
\pgfpathcurveto{\pgfqpoint{4.566012in}{5.334773in}}{\pgfqpoint{4.571598in}{5.337087in}}{\pgfqpoint{4.575717in}{5.341205in}}%
\pgfpathcurveto{\pgfqpoint{4.579835in}{5.345324in}}{\pgfqpoint{4.582149in}{5.350910in}}{\pgfqpoint{4.582149in}{5.356734in}}%
\pgfpathcurveto{\pgfqpoint{4.582149in}{5.362558in}}{\pgfqpoint{4.579835in}{5.368144in}}{\pgfqpoint{4.575717in}{5.372262in}}%
\pgfpathcurveto{\pgfqpoint{4.571598in}{5.376380in}}{\pgfqpoint{4.566012in}{5.378694in}}{\pgfqpoint{4.560188in}{5.378694in}}%
\pgfpathcurveto{\pgfqpoint{4.554364in}{5.378694in}}{\pgfqpoint{4.548778in}{5.376380in}}{\pgfqpoint{4.544660in}{5.372262in}}%
\pgfpathcurveto{\pgfqpoint{4.540542in}{5.368144in}}{\pgfqpoint{4.538228in}{5.362558in}}{\pgfqpoint{4.538228in}{5.356734in}}%
\pgfpathcurveto{\pgfqpoint{4.538228in}{5.350910in}}{\pgfqpoint{4.540542in}{5.345324in}}{\pgfqpoint{4.544660in}{5.341205in}}%
\pgfpathcurveto{\pgfqpoint{4.548778in}{5.337087in}}{\pgfqpoint{4.554364in}{5.334773in}}{\pgfqpoint{4.560188in}{5.334773in}}%
\pgfpathlineto{\pgfqpoint{4.560188in}{5.334773in}}%
\pgfpathclose%
\pgfusepath{stroke,fill}%
\end{pgfscope}%
\begin{pgfscope}%
\pgfpathrectangle{\pgfqpoint{1.000000in}{0.979904in}}{\pgfqpoint{6.200000in}{5.960192in}}%
\pgfusepath{clip}%
\pgfsetbuttcap%
\pgfsetroundjoin%
\definecolor{currentfill}{rgb}{0.200000,0.200000,0.800000}%
\pgfsetfillcolor{currentfill}%
\pgfsetlinewidth{1.003750pt}%
\definecolor{currentstroke}{rgb}{0.200000,0.200000,0.800000}%
\pgfsetstrokecolor{currentstroke}%
\pgfsetdash{}{0pt}%
\pgfpathmoveto{\pgfqpoint{4.617568in}{5.377365in}}%
\pgfpathcurveto{\pgfqpoint{4.623392in}{5.377365in}}{\pgfqpoint{4.628978in}{5.379678in}}{\pgfqpoint{4.633096in}{5.383797in}}%
\pgfpathcurveto{\pgfqpoint{4.637214in}{5.387915in}}{\pgfqpoint{4.639528in}{5.393501in}}{\pgfqpoint{4.639528in}{5.399325in}}%
\pgfpathcurveto{\pgfqpoint{4.639528in}{5.405149in}}{\pgfqpoint{4.637214in}{5.410735in}}{\pgfqpoint{4.633096in}{5.414853in}}%
\pgfpathcurveto{\pgfqpoint{4.628978in}{5.418971in}}{\pgfqpoint{4.623392in}{5.421285in}}{\pgfqpoint{4.617568in}{5.421285in}}%
\pgfpathcurveto{\pgfqpoint{4.611744in}{5.421285in}}{\pgfqpoint{4.606158in}{5.418971in}}{\pgfqpoint{4.602040in}{5.414853in}}%
\pgfpathcurveto{\pgfqpoint{4.597921in}{5.410735in}}{\pgfqpoint{4.595608in}{5.405149in}}{\pgfqpoint{4.595608in}{5.399325in}}%
\pgfpathcurveto{\pgfqpoint{4.595608in}{5.393501in}}{\pgfqpoint{4.597921in}{5.387915in}}{\pgfqpoint{4.602040in}{5.383797in}}%
\pgfpathcurveto{\pgfqpoint{4.606158in}{5.379678in}}{\pgfqpoint{4.611744in}{5.377365in}}{\pgfqpoint{4.617568in}{5.377365in}}%
\pgfpathlineto{\pgfqpoint{4.617568in}{5.377365in}}%
\pgfpathclose%
\pgfusepath{stroke,fill}%
\end{pgfscope}%
\begin{pgfscope}%
\pgfpathrectangle{\pgfqpoint{1.000000in}{0.979904in}}{\pgfqpoint{6.200000in}{5.960192in}}%
\pgfusepath{clip}%
\pgfsetbuttcap%
\pgfsetroundjoin%
\definecolor{currentfill}{rgb}{0.200000,0.200000,0.800000}%
\pgfsetfillcolor{currentfill}%
\pgfsetlinewidth{1.003750pt}%
\definecolor{currentstroke}{rgb}{0.200000,0.200000,0.800000}%
\pgfsetstrokecolor{currentstroke}%
\pgfsetdash{}{0pt}%
\pgfpathmoveto{\pgfqpoint{4.602935in}{5.444043in}}%
\pgfpathcurveto{\pgfqpoint{4.608759in}{5.444043in}}{\pgfqpoint{4.614345in}{5.446357in}}{\pgfqpoint{4.618463in}{5.450475in}}%
\pgfpathcurveto{\pgfqpoint{4.622582in}{5.454593in}}{\pgfqpoint{4.624895in}{5.460179in}}{\pgfqpoint{4.624895in}{5.466003in}}%
\pgfpathcurveto{\pgfqpoint{4.624895in}{5.471827in}}{\pgfqpoint{4.622582in}{5.477413in}}{\pgfqpoint{4.618463in}{5.481531in}}%
\pgfpathcurveto{\pgfqpoint{4.614345in}{5.485649in}}{\pgfqpoint{4.608759in}{5.487963in}}{\pgfqpoint{4.602935in}{5.487963in}}%
\pgfpathcurveto{\pgfqpoint{4.597111in}{5.487963in}}{\pgfqpoint{4.591525in}{5.485649in}}{\pgfqpoint{4.587407in}{5.481531in}}%
\pgfpathcurveto{\pgfqpoint{4.583289in}{5.477413in}}{\pgfqpoint{4.580975in}{5.471827in}}{\pgfqpoint{4.580975in}{5.466003in}}%
\pgfpathcurveto{\pgfqpoint{4.580975in}{5.460179in}}{\pgfqpoint{4.583289in}{5.454593in}}{\pgfqpoint{4.587407in}{5.450475in}}%
\pgfpathcurveto{\pgfqpoint{4.591525in}{5.446357in}}{\pgfqpoint{4.597111in}{5.444043in}}{\pgfqpoint{4.602935in}{5.444043in}}%
\pgfpathlineto{\pgfqpoint{4.602935in}{5.444043in}}%
\pgfpathclose%
\pgfusepath{stroke,fill}%
\end{pgfscope}%
\begin{pgfscope}%
\pgfpathrectangle{\pgfqpoint{1.000000in}{0.979904in}}{\pgfqpoint{6.200000in}{5.960192in}}%
\pgfusepath{clip}%
\pgfsetbuttcap%
\pgfsetroundjoin%
\definecolor{currentfill}{rgb}{0.200000,0.200000,0.800000}%
\pgfsetfillcolor{currentfill}%
\pgfsetlinewidth{1.003750pt}%
\definecolor{currentstroke}{rgb}{0.200000,0.200000,0.800000}%
\pgfsetstrokecolor{currentstroke}%
\pgfsetdash{}{0pt}%
\pgfpathmoveto{\pgfqpoint{4.635741in}{5.497585in}}%
\pgfpathcurveto{\pgfqpoint{4.641565in}{5.497585in}}{\pgfqpoint{4.647152in}{5.499899in}}{\pgfqpoint{4.651270in}{5.504017in}}%
\pgfpathcurveto{\pgfqpoint{4.655388in}{5.508135in}}{\pgfqpoint{4.657702in}{5.513721in}}{\pgfqpoint{4.657702in}{5.519545in}}%
\pgfpathcurveto{\pgfqpoint{4.657702in}{5.525369in}}{\pgfqpoint{4.655388in}{5.530955in}}{\pgfqpoint{4.651270in}{5.535073in}}%
\pgfpathcurveto{\pgfqpoint{4.647152in}{5.539192in}}{\pgfqpoint{4.641565in}{5.541505in}}{\pgfqpoint{4.635741in}{5.541505in}}%
\pgfpathcurveto{\pgfqpoint{4.629918in}{5.541505in}}{\pgfqpoint{4.624331in}{5.539192in}}{\pgfqpoint{4.620213in}{5.535073in}}%
\pgfpathcurveto{\pgfqpoint{4.616095in}{5.530955in}}{\pgfqpoint{4.613781in}{5.525369in}}{\pgfqpoint{4.613781in}{5.519545in}}%
\pgfpathcurveto{\pgfqpoint{4.613781in}{5.513721in}}{\pgfqpoint{4.616095in}{5.508135in}}{\pgfqpoint{4.620213in}{5.504017in}}%
\pgfpathcurveto{\pgfqpoint{4.624331in}{5.499899in}}{\pgfqpoint{4.629918in}{5.497585in}}{\pgfqpoint{4.635741in}{5.497585in}}%
\pgfpathlineto{\pgfqpoint{4.635741in}{5.497585in}}%
\pgfpathclose%
\pgfusepath{stroke,fill}%
\end{pgfscope}%
\begin{pgfscope}%
\pgfpathrectangle{\pgfqpoint{1.000000in}{0.979904in}}{\pgfqpoint{6.200000in}{5.960192in}}%
\pgfusepath{clip}%
\pgfsetbuttcap%
\pgfsetroundjoin%
\definecolor{currentfill}{rgb}{0.200000,0.200000,0.800000}%
\pgfsetfillcolor{currentfill}%
\pgfsetlinewidth{1.003750pt}%
\definecolor{currentstroke}{rgb}{0.200000,0.200000,0.800000}%
\pgfsetstrokecolor{currentstroke}%
\pgfsetdash{}{0pt}%
\pgfpathmoveto{\pgfqpoint{4.527490in}{5.571952in}}%
\pgfpathcurveto{\pgfqpoint{4.533314in}{5.571952in}}{\pgfqpoint{4.538901in}{5.574266in}}{\pgfqpoint{4.543019in}{5.578384in}}%
\pgfpathcurveto{\pgfqpoint{4.547137in}{5.582502in}}{\pgfqpoint{4.549451in}{5.588088in}}{\pgfqpoint{4.549451in}{5.593912in}}%
\pgfpathcurveto{\pgfqpoint{4.549451in}{5.599736in}}{\pgfqpoint{4.547137in}{5.605323in}}{\pgfqpoint{4.543019in}{5.609441in}}%
\pgfpathcurveto{\pgfqpoint{4.538901in}{5.613559in}}{\pgfqpoint{4.533314in}{5.615873in}}{\pgfqpoint{4.527490in}{5.615873in}}%
\pgfpathcurveto{\pgfqpoint{4.521666in}{5.615873in}}{\pgfqpoint{4.516080in}{5.613559in}}{\pgfqpoint{4.511962in}{5.609441in}}%
\pgfpathcurveto{\pgfqpoint{4.507844in}{5.605323in}}{\pgfqpoint{4.505530in}{5.599736in}}{\pgfqpoint{4.505530in}{5.593912in}}%
\pgfpathcurveto{\pgfqpoint{4.505530in}{5.588088in}}{\pgfqpoint{4.507844in}{5.582502in}}{\pgfqpoint{4.511962in}{5.578384in}}%
\pgfpathcurveto{\pgfqpoint{4.516080in}{5.574266in}}{\pgfqpoint{4.521666in}{5.571952in}}{\pgfqpoint{4.527490in}{5.571952in}}%
\pgfpathlineto{\pgfqpoint{4.527490in}{5.571952in}}%
\pgfpathclose%
\pgfusepath{stroke,fill}%
\end{pgfscope}%
\begin{pgfscope}%
\pgfpathrectangle{\pgfqpoint{1.000000in}{0.979904in}}{\pgfqpoint{6.200000in}{5.960192in}}%
\pgfusepath{clip}%
\pgfsetbuttcap%
\pgfsetroundjoin%
\definecolor{currentfill}{rgb}{0.200000,0.200000,0.800000}%
\pgfsetfillcolor{currentfill}%
\pgfsetlinewidth{1.003750pt}%
\definecolor{currentstroke}{rgb}{0.200000,0.200000,0.800000}%
\pgfsetstrokecolor{currentstroke}%
\pgfsetdash{}{0pt}%
\pgfpathmoveto{\pgfqpoint{4.589633in}{5.620646in}}%
\pgfpathcurveto{\pgfqpoint{4.595457in}{5.620646in}}{\pgfqpoint{4.601044in}{5.622959in}}{\pgfqpoint{4.605162in}{5.627078in}}%
\pgfpathcurveto{\pgfqpoint{4.609280in}{5.631196in}}{\pgfqpoint{4.611594in}{5.636782in}}{\pgfqpoint{4.611594in}{5.642606in}}%
\pgfpathcurveto{\pgfqpoint{4.611594in}{5.648430in}}{\pgfqpoint{4.609280in}{5.654016in}}{\pgfqpoint{4.605162in}{5.658134in}}%
\pgfpathcurveto{\pgfqpoint{4.601044in}{5.662252in}}{\pgfqpoint{4.595457in}{5.664566in}}{\pgfqpoint{4.589633in}{5.664566in}}%
\pgfpathcurveto{\pgfqpoint{4.583809in}{5.664566in}}{\pgfqpoint{4.578223in}{5.662252in}}{\pgfqpoint{4.574105in}{5.658134in}}%
\pgfpathcurveto{\pgfqpoint{4.569987in}{5.654016in}}{\pgfqpoint{4.567673in}{5.648430in}}{\pgfqpoint{4.567673in}{5.642606in}}%
\pgfpathcurveto{\pgfqpoint{4.567673in}{5.636782in}}{\pgfqpoint{4.569987in}{5.631196in}}{\pgfqpoint{4.574105in}{5.627078in}}%
\pgfpathcurveto{\pgfqpoint{4.578223in}{5.622959in}}{\pgfqpoint{4.583809in}{5.620646in}}{\pgfqpoint{4.589633in}{5.620646in}}%
\pgfpathlineto{\pgfqpoint{4.589633in}{5.620646in}}%
\pgfpathclose%
\pgfusepath{stroke,fill}%
\end{pgfscope}%
\begin{pgfscope}%
\pgfpathrectangle{\pgfqpoint{1.000000in}{0.979904in}}{\pgfqpoint{6.200000in}{5.960192in}}%
\pgfusepath{clip}%
\pgfsetbuttcap%
\pgfsetroundjoin%
\definecolor{currentfill}{rgb}{0.200000,0.200000,0.800000}%
\pgfsetfillcolor{currentfill}%
\pgfsetlinewidth{1.003750pt}%
\definecolor{currentstroke}{rgb}{0.200000,0.200000,0.800000}%
\pgfsetstrokecolor{currentstroke}%
\pgfsetdash{}{0pt}%
\pgfpathmoveto{\pgfqpoint{4.715941in}{5.676814in}}%
\pgfpathcurveto{\pgfqpoint{4.721765in}{5.676814in}}{\pgfqpoint{4.727351in}{5.679128in}}{\pgfqpoint{4.731469in}{5.683246in}}%
\pgfpathcurveto{\pgfqpoint{4.735588in}{5.687364in}}{\pgfqpoint{4.737901in}{5.692950in}}{\pgfqpoint{4.737901in}{5.698774in}}%
\pgfpathcurveto{\pgfqpoint{4.737901in}{5.704598in}}{\pgfqpoint{4.735588in}{5.710184in}}{\pgfqpoint{4.731469in}{5.714303in}}%
\pgfpathcurveto{\pgfqpoint{4.727351in}{5.718421in}}{\pgfqpoint{4.721765in}{5.720735in}}{\pgfqpoint{4.715941in}{5.720735in}}%
\pgfpathcurveto{\pgfqpoint{4.710117in}{5.720735in}}{\pgfqpoint{4.704531in}{5.718421in}}{\pgfqpoint{4.700413in}{5.714303in}}%
\pgfpathcurveto{\pgfqpoint{4.696295in}{5.710184in}}{\pgfqpoint{4.693981in}{5.704598in}}{\pgfqpoint{4.693981in}{5.698774in}}%
\pgfpathcurveto{\pgfqpoint{4.693981in}{5.692950in}}{\pgfqpoint{4.696295in}{5.687364in}}{\pgfqpoint{4.700413in}{5.683246in}}%
\pgfpathcurveto{\pgfqpoint{4.704531in}{5.679128in}}{\pgfqpoint{4.710117in}{5.676814in}}{\pgfqpoint{4.715941in}{5.676814in}}%
\pgfpathlineto{\pgfqpoint{4.715941in}{5.676814in}}%
\pgfpathclose%
\pgfusepath{stroke,fill}%
\end{pgfscope}%
\begin{pgfscope}%
\pgfpathrectangle{\pgfqpoint{1.000000in}{0.979904in}}{\pgfqpoint{6.200000in}{5.960192in}}%
\pgfusepath{clip}%
\pgfsetbuttcap%
\pgfsetroundjoin%
\definecolor{currentfill}{rgb}{0.200000,0.800000,0.200000}%
\pgfsetfillcolor{currentfill}%
\pgfsetlinewidth{1.003750pt}%
\definecolor{currentstroke}{rgb}{0.200000,0.800000,0.200000}%
\pgfsetstrokecolor{currentstroke}%
\pgfsetdash{}{0pt}%
\pgfpathmoveto{\pgfqpoint{6.611965in}{4.431554in}}%
\pgfpathcurveto{\pgfqpoint{6.617789in}{4.431554in}}{\pgfqpoint{6.623375in}{4.433868in}}{\pgfqpoint{6.627493in}{4.437986in}}%
\pgfpathcurveto{\pgfqpoint{6.631612in}{4.442104in}}{\pgfqpoint{6.633925in}{4.447690in}}{\pgfqpoint{6.633925in}{4.453514in}}%
\pgfpathcurveto{\pgfqpoint{6.633925in}{4.459338in}}{\pgfqpoint{6.631612in}{4.464924in}}{\pgfqpoint{6.627493in}{4.469042in}}%
\pgfpathcurveto{\pgfqpoint{6.623375in}{4.473160in}}{\pgfqpoint{6.617789in}{4.475474in}}{\pgfqpoint{6.611965in}{4.475474in}}%
\pgfpathcurveto{\pgfqpoint{6.606141in}{4.475474in}}{\pgfqpoint{6.600555in}{4.473160in}}{\pgfqpoint{6.596437in}{4.469042in}}%
\pgfpathcurveto{\pgfqpoint{6.592319in}{4.464924in}}{\pgfqpoint{6.590005in}{4.459338in}}{\pgfqpoint{6.590005in}{4.453514in}}%
\pgfpathcurveto{\pgfqpoint{6.590005in}{4.447690in}}{\pgfqpoint{6.592319in}{4.442104in}}{\pgfqpoint{6.596437in}{4.437986in}}%
\pgfpathcurveto{\pgfqpoint{6.600555in}{4.433868in}}{\pgfqpoint{6.606141in}{4.431554in}}{\pgfqpoint{6.611965in}{4.431554in}}%
\pgfpathlineto{\pgfqpoint{6.611965in}{4.431554in}}%
\pgfpathclose%
\pgfusepath{stroke,fill}%
\end{pgfscope}%
\begin{pgfscope}%
\pgfpathrectangle{\pgfqpoint{1.000000in}{0.979904in}}{\pgfqpoint{6.200000in}{5.960192in}}%
\pgfusepath{clip}%
\pgfsetbuttcap%
\pgfsetroundjoin%
\definecolor{currentfill}{rgb}{0.200000,0.800000,0.200000}%
\pgfsetfillcolor{currentfill}%
\pgfsetlinewidth{1.003750pt}%
\definecolor{currentstroke}{rgb}{0.200000,0.800000,0.200000}%
\pgfsetstrokecolor{currentstroke}%
\pgfsetdash{}{0pt}%
\pgfpathmoveto{\pgfqpoint{6.708854in}{4.542051in}}%
\pgfpathcurveto{\pgfqpoint{6.714678in}{4.542051in}}{\pgfqpoint{6.720265in}{4.544364in}}{\pgfqpoint{6.724383in}{4.548483in}}%
\pgfpathcurveto{\pgfqpoint{6.728501in}{4.552601in}}{\pgfqpoint{6.730815in}{4.558187in}}{\pgfqpoint{6.730815in}{4.564011in}}%
\pgfpathcurveto{\pgfqpoint{6.730815in}{4.569835in}}{\pgfqpoint{6.728501in}{4.575421in}}{\pgfqpoint{6.724383in}{4.579539in}}%
\pgfpathcurveto{\pgfqpoint{6.720265in}{4.583657in}}{\pgfqpoint{6.714678in}{4.585971in}}{\pgfqpoint{6.708854in}{4.585971in}}%
\pgfpathcurveto{\pgfqpoint{6.703030in}{4.585971in}}{\pgfqpoint{6.697444in}{4.583657in}}{\pgfqpoint{6.693326in}{4.579539in}}%
\pgfpathcurveto{\pgfqpoint{6.689208in}{4.575421in}}{\pgfqpoint{6.686894in}{4.569835in}}{\pgfqpoint{6.686894in}{4.564011in}}%
\pgfpathcurveto{\pgfqpoint{6.686894in}{4.558187in}}{\pgfqpoint{6.689208in}{4.552601in}}{\pgfqpoint{6.693326in}{4.548483in}}%
\pgfpathcurveto{\pgfqpoint{6.697444in}{4.544364in}}{\pgfqpoint{6.703030in}{4.542051in}}{\pgfqpoint{6.708854in}{4.542051in}}%
\pgfpathlineto{\pgfqpoint{6.708854in}{4.542051in}}%
\pgfpathclose%
\pgfusepath{stroke,fill}%
\end{pgfscope}%
\begin{pgfscope}%
\pgfpathrectangle{\pgfqpoint{1.000000in}{0.979904in}}{\pgfqpoint{6.200000in}{5.960192in}}%
\pgfusepath{clip}%
\pgfsetbuttcap%
\pgfsetroundjoin%
\definecolor{currentfill}{rgb}{0.200000,0.800000,0.200000}%
\pgfsetfillcolor{currentfill}%
\pgfsetlinewidth{1.003750pt}%
\definecolor{currentstroke}{rgb}{0.200000,0.800000,0.200000}%
\pgfsetstrokecolor{currentstroke}%
\pgfsetdash{}{0pt}%
\pgfpathmoveto{\pgfqpoint{6.702911in}{4.652685in}}%
\pgfpathcurveto{\pgfqpoint{6.708735in}{4.652685in}}{\pgfqpoint{6.714321in}{4.654999in}}{\pgfqpoint{6.718439in}{4.659117in}}%
\pgfpathcurveto{\pgfqpoint{6.722557in}{4.663235in}}{\pgfqpoint{6.724871in}{4.668821in}}{\pgfqpoint{6.724871in}{4.674645in}}%
\pgfpathcurveto{\pgfqpoint{6.724871in}{4.680469in}}{\pgfqpoint{6.722557in}{4.686056in}}{\pgfqpoint{6.718439in}{4.690174in}}%
\pgfpathcurveto{\pgfqpoint{6.714321in}{4.694292in}}{\pgfqpoint{6.708735in}{4.696606in}}{\pgfqpoint{6.702911in}{4.696606in}}%
\pgfpathcurveto{\pgfqpoint{6.697087in}{4.696606in}}{\pgfqpoint{6.691501in}{4.694292in}}{\pgfqpoint{6.687383in}{4.690174in}}%
\pgfpathcurveto{\pgfqpoint{6.683265in}{4.686056in}}{\pgfqpoint{6.680951in}{4.680469in}}{\pgfqpoint{6.680951in}{4.674645in}}%
\pgfpathcurveto{\pgfqpoint{6.680951in}{4.668821in}}{\pgfqpoint{6.683265in}{4.663235in}}{\pgfqpoint{6.687383in}{4.659117in}}%
\pgfpathcurveto{\pgfqpoint{6.691501in}{4.654999in}}{\pgfqpoint{6.697087in}{4.652685in}}{\pgfqpoint{6.702911in}{4.652685in}}%
\pgfpathlineto{\pgfqpoint{6.702911in}{4.652685in}}%
\pgfpathclose%
\pgfusepath{stroke,fill}%
\end{pgfscope}%
\begin{pgfscope}%
\pgfpathrectangle{\pgfqpoint{1.000000in}{0.979904in}}{\pgfqpoint{6.200000in}{5.960192in}}%
\pgfusepath{clip}%
\pgfsetbuttcap%
\pgfsetroundjoin%
\definecolor{currentfill}{rgb}{0.200000,0.800000,0.200000}%
\pgfsetfillcolor{currentfill}%
\pgfsetlinewidth{1.003750pt}%
\definecolor{currentstroke}{rgb}{0.200000,0.800000,0.200000}%
\pgfsetstrokecolor{currentstroke}%
\pgfsetdash{}{0pt}%
\pgfpathmoveto{\pgfqpoint{6.810589in}{4.786266in}}%
\pgfpathcurveto{\pgfqpoint{6.816413in}{4.786266in}}{\pgfqpoint{6.821999in}{4.788580in}}{\pgfqpoint{6.826117in}{4.792698in}}%
\pgfpathcurveto{\pgfqpoint{6.830236in}{4.796816in}}{\pgfqpoint{6.832549in}{4.802403in}}{\pgfqpoint{6.832549in}{4.808226in}}%
\pgfpathcurveto{\pgfqpoint{6.832549in}{4.814050in}}{\pgfqpoint{6.830236in}{4.819637in}}{\pgfqpoint{6.826117in}{4.823755in}}%
\pgfpathcurveto{\pgfqpoint{6.821999in}{4.827873in}}{\pgfqpoint{6.816413in}{4.830187in}}{\pgfqpoint{6.810589in}{4.830187in}}%
\pgfpathcurveto{\pgfqpoint{6.804765in}{4.830187in}}{\pgfqpoint{6.799179in}{4.827873in}}{\pgfqpoint{6.795061in}{4.823755in}}%
\pgfpathcurveto{\pgfqpoint{6.790943in}{4.819637in}}{\pgfqpoint{6.788629in}{4.814050in}}{\pgfqpoint{6.788629in}{4.808226in}}%
\pgfpathcurveto{\pgfqpoint{6.788629in}{4.802403in}}{\pgfqpoint{6.790943in}{4.796816in}}{\pgfqpoint{6.795061in}{4.792698in}}%
\pgfpathcurveto{\pgfqpoint{6.799179in}{4.788580in}}{\pgfqpoint{6.804765in}{4.786266in}}{\pgfqpoint{6.810589in}{4.786266in}}%
\pgfpathlineto{\pgfqpoint{6.810589in}{4.786266in}}%
\pgfpathclose%
\pgfusepath{stroke,fill}%
\end{pgfscope}%
\begin{pgfscope}%
\pgfpathrectangle{\pgfqpoint{1.000000in}{0.979904in}}{\pgfqpoint{6.200000in}{5.960192in}}%
\pgfusepath{clip}%
\pgfsetbuttcap%
\pgfsetroundjoin%
\definecolor{currentfill}{rgb}{0.200000,0.800000,0.200000}%
\pgfsetfillcolor{currentfill}%
\pgfsetlinewidth{1.003750pt}%
\definecolor{currentstroke}{rgb}{0.200000,0.800000,0.200000}%
\pgfsetstrokecolor{currentstroke}%
\pgfsetdash{}{0pt}%
\pgfpathmoveto{\pgfqpoint{6.648005in}{4.866893in}}%
\pgfpathcurveto{\pgfqpoint{6.653829in}{4.866893in}}{\pgfqpoint{6.659415in}{4.869207in}}{\pgfqpoint{6.663533in}{4.873325in}}%
\pgfpathcurveto{\pgfqpoint{6.667651in}{4.877443in}}{\pgfqpoint{6.669965in}{4.883029in}}{\pgfqpoint{6.669965in}{4.888853in}}%
\pgfpathcurveto{\pgfqpoint{6.669965in}{4.894677in}}{\pgfqpoint{6.667651in}{4.900263in}}{\pgfqpoint{6.663533in}{4.904381in}}%
\pgfpathcurveto{\pgfqpoint{6.659415in}{4.908499in}}{\pgfqpoint{6.653829in}{4.910813in}}{\pgfqpoint{6.648005in}{4.910813in}}%
\pgfpathcurveto{\pgfqpoint{6.642181in}{4.910813in}}{\pgfqpoint{6.636595in}{4.908499in}}{\pgfqpoint{6.632477in}{4.904381in}}%
\pgfpathcurveto{\pgfqpoint{6.628359in}{4.900263in}}{\pgfqpoint{6.626045in}{4.894677in}}{\pgfqpoint{6.626045in}{4.888853in}}%
\pgfpathcurveto{\pgfqpoint{6.626045in}{4.883029in}}{\pgfqpoint{6.628359in}{4.877443in}}{\pgfqpoint{6.632477in}{4.873325in}}%
\pgfpathcurveto{\pgfqpoint{6.636595in}{4.869207in}}{\pgfqpoint{6.642181in}{4.866893in}}{\pgfqpoint{6.648005in}{4.866893in}}%
\pgfpathlineto{\pgfqpoint{6.648005in}{4.866893in}}%
\pgfpathclose%
\pgfusepath{stroke,fill}%
\end{pgfscope}%
\begin{pgfscope}%
\pgfpathrectangle{\pgfqpoint{1.000000in}{0.979904in}}{\pgfqpoint{6.200000in}{5.960192in}}%
\pgfusepath{clip}%
\pgfsetbuttcap%
\pgfsetroundjoin%
\definecolor{currentfill}{rgb}{0.200000,0.800000,0.200000}%
\pgfsetfillcolor{currentfill}%
\pgfsetlinewidth{1.003750pt}%
\definecolor{currentstroke}{rgb}{0.200000,0.800000,0.200000}%
\pgfsetstrokecolor{currentstroke}%
\pgfsetdash{}{0pt}%
\pgfpathmoveto{\pgfqpoint{6.641913in}{4.980608in}}%
\pgfpathcurveto{\pgfqpoint{6.647737in}{4.980608in}}{\pgfqpoint{6.653323in}{4.982922in}}{\pgfqpoint{6.657441in}{4.987040in}}%
\pgfpathcurveto{\pgfqpoint{6.661559in}{4.991158in}}{\pgfqpoint{6.663873in}{4.996745in}}{\pgfqpoint{6.663873in}{5.002569in}}%
\pgfpathcurveto{\pgfqpoint{6.663873in}{5.008393in}}{\pgfqpoint{6.661559in}{5.013979in}}{\pgfqpoint{6.657441in}{5.018097in}}%
\pgfpathcurveto{\pgfqpoint{6.653323in}{5.022215in}}{\pgfqpoint{6.647737in}{5.024529in}}{\pgfqpoint{6.641913in}{5.024529in}}%
\pgfpathcurveto{\pgfqpoint{6.636089in}{5.024529in}}{\pgfqpoint{6.630503in}{5.022215in}}{\pgfqpoint{6.626385in}{5.018097in}}%
\pgfpathcurveto{\pgfqpoint{6.622266in}{5.013979in}}{\pgfqpoint{6.619953in}{5.008393in}}{\pgfqpoint{6.619953in}{5.002569in}}%
\pgfpathcurveto{\pgfqpoint{6.619953in}{4.996745in}}{\pgfqpoint{6.622266in}{4.991158in}}{\pgfqpoint{6.626385in}{4.987040in}}%
\pgfpathcurveto{\pgfqpoint{6.630503in}{4.982922in}}{\pgfqpoint{6.636089in}{4.980608in}}{\pgfqpoint{6.641913in}{4.980608in}}%
\pgfpathlineto{\pgfqpoint{6.641913in}{4.980608in}}%
\pgfpathclose%
\pgfusepath{stroke,fill}%
\end{pgfscope}%
\begin{pgfscope}%
\pgfpathrectangle{\pgfqpoint{1.000000in}{0.979904in}}{\pgfqpoint{6.200000in}{5.960192in}}%
\pgfusepath{clip}%
\pgfsetbuttcap%
\pgfsetroundjoin%
\definecolor{currentfill}{rgb}{0.200000,0.800000,0.200000}%
\pgfsetfillcolor{currentfill}%
\pgfsetlinewidth{1.003750pt}%
\definecolor{currentstroke}{rgb}{0.200000,0.800000,0.200000}%
\pgfsetstrokecolor{currentstroke}%
\pgfsetdash{}{0pt}%
\pgfpathmoveto{\pgfqpoint{6.587161in}{5.078901in}}%
\pgfpathcurveto{\pgfqpoint{6.592985in}{5.078901in}}{\pgfqpoint{6.598571in}{5.081215in}}{\pgfqpoint{6.602689in}{5.085333in}}%
\pgfpathcurveto{\pgfqpoint{6.606807in}{5.089451in}}{\pgfqpoint{6.609121in}{5.095037in}}{\pgfqpoint{6.609121in}{5.100861in}}%
\pgfpathcurveto{\pgfqpoint{6.609121in}{5.106685in}}{\pgfqpoint{6.606807in}{5.112271in}}{\pgfqpoint{6.602689in}{5.116389in}}%
\pgfpathcurveto{\pgfqpoint{6.598571in}{5.120508in}}{\pgfqpoint{6.592985in}{5.122821in}}{\pgfqpoint{6.587161in}{5.122821in}}%
\pgfpathcurveto{\pgfqpoint{6.581337in}{5.122821in}}{\pgfqpoint{6.575751in}{5.120508in}}{\pgfqpoint{6.571633in}{5.116389in}}%
\pgfpathcurveto{\pgfqpoint{6.567515in}{5.112271in}}{\pgfqpoint{6.565201in}{5.106685in}}{\pgfqpoint{6.565201in}{5.100861in}}%
\pgfpathcurveto{\pgfqpoint{6.565201in}{5.095037in}}{\pgfqpoint{6.567515in}{5.089451in}}{\pgfqpoint{6.571633in}{5.085333in}}%
\pgfpathcurveto{\pgfqpoint{6.575751in}{5.081215in}}{\pgfqpoint{6.581337in}{5.078901in}}{\pgfqpoint{6.587161in}{5.078901in}}%
\pgfpathlineto{\pgfqpoint{6.587161in}{5.078901in}}%
\pgfpathclose%
\pgfusepath{stroke,fill}%
\end{pgfscope}%
\begin{pgfscope}%
\pgfpathrectangle{\pgfqpoint{1.000000in}{0.979904in}}{\pgfqpoint{6.200000in}{5.960192in}}%
\pgfusepath{clip}%
\pgfsetbuttcap%
\pgfsetroundjoin%
\definecolor{currentfill}{rgb}{0.200000,0.800000,0.200000}%
\pgfsetfillcolor{currentfill}%
\pgfsetlinewidth{1.003750pt}%
\definecolor{currentstroke}{rgb}{0.200000,0.800000,0.200000}%
\pgfsetstrokecolor{currentstroke}%
\pgfsetdash{}{0pt}%
\pgfpathmoveto{\pgfqpoint{6.488524in}{5.154295in}}%
\pgfpathcurveto{\pgfqpoint{6.494348in}{5.154295in}}{\pgfqpoint{6.499934in}{5.156609in}}{\pgfqpoint{6.504052in}{5.160727in}}%
\pgfpathcurveto{\pgfqpoint{6.508171in}{5.164846in}}{\pgfqpoint{6.510484in}{5.170432in}}{\pgfqpoint{6.510484in}{5.176256in}}%
\pgfpathcurveto{\pgfqpoint{6.510484in}{5.182080in}}{\pgfqpoint{6.508171in}{5.187666in}}{\pgfqpoint{6.504052in}{5.191784in}}%
\pgfpathcurveto{\pgfqpoint{6.499934in}{5.195902in}}{\pgfqpoint{6.494348in}{5.198216in}}{\pgfqpoint{6.488524in}{5.198216in}}%
\pgfpathcurveto{\pgfqpoint{6.482700in}{5.198216in}}{\pgfqpoint{6.477114in}{5.195902in}}{\pgfqpoint{6.472996in}{5.191784in}}%
\pgfpathcurveto{\pgfqpoint{6.468878in}{5.187666in}}{\pgfqpoint{6.466564in}{5.182080in}}{\pgfqpoint{6.466564in}{5.176256in}}%
\pgfpathcurveto{\pgfqpoint{6.466564in}{5.170432in}}{\pgfqpoint{6.468878in}{5.164846in}}{\pgfqpoint{6.472996in}{5.160727in}}%
\pgfpathcurveto{\pgfqpoint{6.477114in}{5.156609in}}{\pgfqpoint{6.482700in}{5.154295in}}{\pgfqpoint{6.488524in}{5.154295in}}%
\pgfpathlineto{\pgfqpoint{6.488524in}{5.154295in}}%
\pgfpathclose%
\pgfusepath{stroke,fill}%
\end{pgfscope}%
\begin{pgfscope}%
\pgfpathrectangle{\pgfqpoint{1.000000in}{0.979904in}}{\pgfqpoint{6.200000in}{5.960192in}}%
\pgfusepath{clip}%
\pgfsetbuttcap%
\pgfsetroundjoin%
\definecolor{currentfill}{rgb}{0.200000,0.800000,0.200000}%
\pgfsetfillcolor{currentfill}%
\pgfsetlinewidth{1.003750pt}%
\definecolor{currentstroke}{rgb}{0.200000,0.800000,0.200000}%
\pgfsetstrokecolor{currentstroke}%
\pgfsetdash{}{0pt}%
\pgfpathmoveto{\pgfqpoint{6.546797in}{5.308768in}}%
\pgfpathcurveto{\pgfqpoint{6.552621in}{5.308768in}}{\pgfqpoint{6.558208in}{5.311082in}}{\pgfqpoint{6.562326in}{5.315200in}}%
\pgfpathcurveto{\pgfqpoint{6.566444in}{5.319318in}}{\pgfqpoint{6.568758in}{5.324904in}}{\pgfqpoint{6.568758in}{5.330728in}}%
\pgfpathcurveto{\pgfqpoint{6.568758in}{5.336552in}}{\pgfqpoint{6.566444in}{5.342138in}}{\pgfqpoint{6.562326in}{5.346256in}}%
\pgfpathcurveto{\pgfqpoint{6.558208in}{5.350375in}}{\pgfqpoint{6.552621in}{5.352688in}}{\pgfqpoint{6.546797in}{5.352688in}}%
\pgfpathcurveto{\pgfqpoint{6.540974in}{5.352688in}}{\pgfqpoint{6.535387in}{5.350375in}}{\pgfqpoint{6.531269in}{5.346256in}}%
\pgfpathcurveto{\pgfqpoint{6.527151in}{5.342138in}}{\pgfqpoint{6.524837in}{5.336552in}}{\pgfqpoint{6.524837in}{5.330728in}}%
\pgfpathcurveto{\pgfqpoint{6.524837in}{5.324904in}}{\pgfqpoint{6.527151in}{5.319318in}}{\pgfqpoint{6.531269in}{5.315200in}}%
\pgfpathcurveto{\pgfqpoint{6.535387in}{5.311082in}}{\pgfqpoint{6.540974in}{5.308768in}}{\pgfqpoint{6.546797in}{5.308768in}}%
\pgfpathlineto{\pgfqpoint{6.546797in}{5.308768in}}%
\pgfpathclose%
\pgfusepath{stroke,fill}%
\end{pgfscope}%
\begin{pgfscope}%
\pgfpathrectangle{\pgfqpoint{1.000000in}{0.979904in}}{\pgfqpoint{6.200000in}{5.960192in}}%
\pgfusepath{clip}%
\pgfsetbuttcap%
\pgfsetroundjoin%
\definecolor{currentfill}{rgb}{0.200000,0.800000,0.200000}%
\pgfsetfillcolor{currentfill}%
\pgfsetlinewidth{1.003750pt}%
\definecolor{currentstroke}{rgb}{0.200000,0.800000,0.200000}%
\pgfsetstrokecolor{currentstroke}%
\pgfsetdash{}{0pt}%
\pgfpathmoveto{\pgfqpoint{6.470540in}{5.395785in}}%
\pgfpathcurveto{\pgfqpoint{6.476364in}{5.395785in}}{\pgfqpoint{6.481950in}{5.398099in}}{\pgfqpoint{6.486068in}{5.402217in}}%
\pgfpathcurveto{\pgfqpoint{6.490186in}{5.406336in}}{\pgfqpoint{6.492500in}{5.411922in}}{\pgfqpoint{6.492500in}{5.417746in}}%
\pgfpathcurveto{\pgfqpoint{6.492500in}{5.423570in}}{\pgfqpoint{6.490186in}{5.429156in}}{\pgfqpoint{6.486068in}{5.433274in}}%
\pgfpathcurveto{\pgfqpoint{6.481950in}{5.437392in}}{\pgfqpoint{6.476364in}{5.439706in}}{\pgfqpoint{6.470540in}{5.439706in}}%
\pgfpathcurveto{\pgfqpoint{6.464716in}{5.439706in}}{\pgfqpoint{6.459130in}{5.437392in}}{\pgfqpoint{6.455012in}{5.433274in}}%
\pgfpathcurveto{\pgfqpoint{6.450893in}{5.429156in}}{\pgfqpoint{6.448580in}{5.423570in}}{\pgfqpoint{6.448580in}{5.417746in}}%
\pgfpathcurveto{\pgfqpoint{6.448580in}{5.411922in}}{\pgfqpoint{6.450893in}{5.406336in}}{\pgfqpoint{6.455012in}{5.402217in}}%
\pgfpathcurveto{\pgfqpoint{6.459130in}{5.398099in}}{\pgfqpoint{6.464716in}{5.395785in}}{\pgfqpoint{6.470540in}{5.395785in}}%
\pgfpathlineto{\pgfqpoint{6.470540in}{5.395785in}}%
\pgfpathclose%
\pgfusepath{stroke,fill}%
\end{pgfscope}%
\begin{pgfscope}%
\pgfpathrectangle{\pgfqpoint{1.000000in}{0.979904in}}{\pgfqpoint{6.200000in}{5.960192in}}%
\pgfusepath{clip}%
\pgfsetbuttcap%
\pgfsetroundjoin%
\definecolor{currentfill}{rgb}{0.200000,0.800000,0.200000}%
\pgfsetfillcolor{currentfill}%
\pgfsetlinewidth{1.003750pt}%
\definecolor{currentstroke}{rgb}{0.200000,0.800000,0.200000}%
\pgfsetstrokecolor{currentstroke}%
\pgfsetdash{}{0pt}%
\pgfpathmoveto{\pgfqpoint{6.408034in}{5.490234in}}%
\pgfpathcurveto{\pgfqpoint{6.413858in}{5.490234in}}{\pgfqpoint{6.419444in}{5.492548in}}{\pgfqpoint{6.423562in}{5.496666in}}%
\pgfpathcurveto{\pgfqpoint{6.427681in}{5.500784in}}{\pgfqpoint{6.429994in}{5.506370in}}{\pgfqpoint{6.429994in}{5.512194in}}%
\pgfpathcurveto{\pgfqpoint{6.429994in}{5.518018in}}{\pgfqpoint{6.427681in}{5.523604in}}{\pgfqpoint{6.423562in}{5.527722in}}%
\pgfpathcurveto{\pgfqpoint{6.419444in}{5.531840in}}{\pgfqpoint{6.413858in}{5.534154in}}{\pgfqpoint{6.408034in}{5.534154in}}%
\pgfpathcurveto{\pgfqpoint{6.402210in}{5.534154in}}{\pgfqpoint{6.396624in}{5.531840in}}{\pgfqpoint{6.392506in}{5.527722in}}%
\pgfpathcurveto{\pgfqpoint{6.388388in}{5.523604in}}{\pgfqpoint{6.386074in}{5.518018in}}{\pgfqpoint{6.386074in}{5.512194in}}%
\pgfpathcurveto{\pgfqpoint{6.386074in}{5.506370in}}{\pgfqpoint{6.388388in}{5.500784in}}{\pgfqpoint{6.392506in}{5.496666in}}%
\pgfpathcurveto{\pgfqpoint{6.396624in}{5.492548in}}{\pgfqpoint{6.402210in}{5.490234in}}{\pgfqpoint{6.408034in}{5.490234in}}%
\pgfpathlineto{\pgfqpoint{6.408034in}{5.490234in}}%
\pgfpathclose%
\pgfusepath{stroke,fill}%
\end{pgfscope}%
\begin{pgfscope}%
\pgfpathrectangle{\pgfqpoint{1.000000in}{0.979904in}}{\pgfqpoint{6.200000in}{5.960192in}}%
\pgfusepath{clip}%
\pgfsetbuttcap%
\pgfsetroundjoin%
\definecolor{currentfill}{rgb}{0.200000,0.800000,0.200000}%
\pgfsetfillcolor{currentfill}%
\pgfsetlinewidth{1.003750pt}%
\definecolor{currentstroke}{rgb}{0.200000,0.800000,0.200000}%
\pgfsetstrokecolor{currentstroke}%
\pgfsetdash{}{0pt}%
\pgfpathmoveto{\pgfqpoint{6.268882in}{5.521306in}}%
\pgfpathcurveto{\pgfqpoint{6.274706in}{5.521306in}}{\pgfqpoint{6.280292in}{5.523620in}}{\pgfqpoint{6.284410in}{5.527738in}}%
\pgfpathcurveto{\pgfqpoint{6.288529in}{5.531856in}}{\pgfqpoint{6.290843in}{5.537442in}}{\pgfqpoint{6.290843in}{5.543266in}}%
\pgfpathcurveto{\pgfqpoint{6.290843in}{5.549090in}}{\pgfqpoint{6.288529in}{5.554676in}}{\pgfqpoint{6.284410in}{5.558795in}}%
\pgfpathcurveto{\pgfqpoint{6.280292in}{5.562913in}}{\pgfqpoint{6.274706in}{5.565227in}}{\pgfqpoint{6.268882in}{5.565227in}}%
\pgfpathcurveto{\pgfqpoint{6.263058in}{5.565227in}}{\pgfqpoint{6.257472in}{5.562913in}}{\pgfqpoint{6.253354in}{5.558795in}}%
\pgfpathcurveto{\pgfqpoint{6.249236in}{5.554676in}}{\pgfqpoint{6.246922in}{5.549090in}}{\pgfqpoint{6.246922in}{5.543266in}}%
\pgfpathcurveto{\pgfqpoint{6.246922in}{5.537442in}}{\pgfqpoint{6.249236in}{5.531856in}}{\pgfqpoint{6.253354in}{5.527738in}}%
\pgfpathcurveto{\pgfqpoint{6.257472in}{5.523620in}}{\pgfqpoint{6.263058in}{5.521306in}}{\pgfqpoint{6.268882in}{5.521306in}}%
\pgfpathlineto{\pgfqpoint{6.268882in}{5.521306in}}%
\pgfpathclose%
\pgfusepath{stroke,fill}%
\end{pgfscope}%
\begin{pgfscope}%
\pgfpathrectangle{\pgfqpoint{1.000000in}{0.979904in}}{\pgfqpoint{6.200000in}{5.960192in}}%
\pgfusepath{clip}%
\pgfsetbuttcap%
\pgfsetroundjoin%
\definecolor{currentfill}{rgb}{0.200000,0.800000,0.200000}%
\pgfsetfillcolor{currentfill}%
\pgfsetlinewidth{1.003750pt}%
\definecolor{currentstroke}{rgb}{0.200000,0.800000,0.200000}%
\pgfsetstrokecolor{currentstroke}%
\pgfsetdash{}{0pt}%
\pgfpathmoveto{\pgfqpoint{6.285936in}{5.686138in}}%
\pgfpathcurveto{\pgfqpoint{6.291760in}{5.686138in}}{\pgfqpoint{6.297347in}{5.688452in}}{\pgfqpoint{6.301465in}{5.692570in}}%
\pgfpathcurveto{\pgfqpoint{6.305583in}{5.696688in}}{\pgfqpoint{6.307897in}{5.702275in}}{\pgfqpoint{6.307897in}{5.708099in}}%
\pgfpathcurveto{\pgfqpoint{6.307897in}{5.713923in}}{\pgfqpoint{6.305583in}{5.719509in}}{\pgfqpoint{6.301465in}{5.723627in}}%
\pgfpathcurveto{\pgfqpoint{6.297347in}{5.727745in}}{\pgfqpoint{6.291760in}{5.730059in}}{\pgfqpoint{6.285936in}{5.730059in}}%
\pgfpathcurveto{\pgfqpoint{6.280112in}{5.730059in}}{\pgfqpoint{6.274526in}{5.727745in}}{\pgfqpoint{6.270408in}{5.723627in}}%
\pgfpathcurveto{\pgfqpoint{6.266290in}{5.719509in}}{\pgfqpoint{6.263976in}{5.713923in}}{\pgfqpoint{6.263976in}{5.708099in}}%
\pgfpathcurveto{\pgfqpoint{6.263976in}{5.702275in}}{\pgfqpoint{6.266290in}{5.696688in}}{\pgfqpoint{6.270408in}{5.692570in}}%
\pgfpathcurveto{\pgfqpoint{6.274526in}{5.688452in}}{\pgfqpoint{6.280112in}{5.686138in}}{\pgfqpoint{6.285936in}{5.686138in}}%
\pgfpathlineto{\pgfqpoint{6.285936in}{5.686138in}}%
\pgfpathclose%
\pgfusepath{stroke,fill}%
\end{pgfscope}%
\begin{pgfscope}%
\pgfpathrectangle{\pgfqpoint{1.000000in}{0.979904in}}{\pgfqpoint{6.200000in}{5.960192in}}%
\pgfusepath{clip}%
\pgfsetbuttcap%
\pgfsetroundjoin%
\definecolor{currentfill}{rgb}{0.200000,0.800000,0.200000}%
\pgfsetfillcolor{currentfill}%
\pgfsetlinewidth{1.003750pt}%
\definecolor{currentstroke}{rgb}{0.200000,0.800000,0.200000}%
\pgfsetstrokecolor{currentstroke}%
\pgfsetdash{}{0pt}%
\pgfpathmoveto{\pgfqpoint{6.102751in}{5.657753in}}%
\pgfpathcurveto{\pgfqpoint{6.108575in}{5.657753in}}{\pgfqpoint{6.114161in}{5.660067in}}{\pgfqpoint{6.118279in}{5.664185in}}%
\pgfpathcurveto{\pgfqpoint{6.122397in}{5.668303in}}{\pgfqpoint{6.124711in}{5.673889in}}{\pgfqpoint{6.124711in}{5.679713in}}%
\pgfpathcurveto{\pgfqpoint{6.124711in}{5.685537in}}{\pgfqpoint{6.122397in}{5.691123in}}{\pgfqpoint{6.118279in}{5.695241in}}%
\pgfpathcurveto{\pgfqpoint{6.114161in}{5.699359in}}{\pgfqpoint{6.108575in}{5.701673in}}{\pgfqpoint{6.102751in}{5.701673in}}%
\pgfpathcurveto{\pgfqpoint{6.096927in}{5.701673in}}{\pgfqpoint{6.091341in}{5.699359in}}{\pgfqpoint{6.087223in}{5.695241in}}%
\pgfpathcurveto{\pgfqpoint{6.083105in}{5.691123in}}{\pgfqpoint{6.080791in}{5.685537in}}{\pgfqpoint{6.080791in}{5.679713in}}%
\pgfpathcurveto{\pgfqpoint{6.080791in}{5.673889in}}{\pgfqpoint{6.083105in}{5.668303in}}{\pgfqpoint{6.087223in}{5.664185in}}%
\pgfpathcurveto{\pgfqpoint{6.091341in}{5.660067in}}{\pgfqpoint{6.096927in}{5.657753in}}{\pgfqpoint{6.102751in}{5.657753in}}%
\pgfpathlineto{\pgfqpoint{6.102751in}{5.657753in}}%
\pgfpathclose%
\pgfusepath{stroke,fill}%
\end{pgfscope}%
\begin{pgfscope}%
\pgfpathrectangle{\pgfqpoint{1.000000in}{0.979904in}}{\pgfqpoint{6.200000in}{5.960192in}}%
\pgfusepath{clip}%
\pgfsetbuttcap%
\pgfsetroundjoin%
\definecolor{currentfill}{rgb}{0.200000,0.800000,0.200000}%
\pgfsetfillcolor{currentfill}%
\pgfsetlinewidth{1.003750pt}%
\definecolor{currentstroke}{rgb}{0.200000,0.800000,0.200000}%
\pgfsetstrokecolor{currentstroke}%
\pgfsetdash{}{0pt}%
\pgfpathmoveto{\pgfqpoint{6.089233in}{5.809011in}}%
\pgfpathcurveto{\pgfqpoint{6.095057in}{5.809011in}}{\pgfqpoint{6.100643in}{5.811325in}}{\pgfqpoint{6.104761in}{5.815443in}}%
\pgfpathcurveto{\pgfqpoint{6.108879in}{5.819562in}}{\pgfqpoint{6.111193in}{5.825148in}}{\pgfqpoint{6.111193in}{5.830972in}}%
\pgfpathcurveto{\pgfqpoint{6.111193in}{5.836796in}}{\pgfqpoint{6.108879in}{5.842382in}}{\pgfqpoint{6.104761in}{5.846500in}}%
\pgfpathcurveto{\pgfqpoint{6.100643in}{5.850618in}}{\pgfqpoint{6.095057in}{5.852932in}}{\pgfqpoint{6.089233in}{5.852932in}}%
\pgfpathcurveto{\pgfqpoint{6.083409in}{5.852932in}}{\pgfqpoint{6.077823in}{5.850618in}}{\pgfqpoint{6.073704in}{5.846500in}}%
\pgfpathcurveto{\pgfqpoint{6.069586in}{5.842382in}}{\pgfqpoint{6.067272in}{5.836796in}}{\pgfqpoint{6.067272in}{5.830972in}}%
\pgfpathcurveto{\pgfqpoint{6.067272in}{5.825148in}}{\pgfqpoint{6.069586in}{5.819562in}}{\pgfqpoint{6.073704in}{5.815443in}}%
\pgfpathcurveto{\pgfqpoint{6.077823in}{5.811325in}}{\pgfqpoint{6.083409in}{5.809011in}}{\pgfqpoint{6.089233in}{5.809011in}}%
\pgfpathlineto{\pgfqpoint{6.089233in}{5.809011in}}%
\pgfpathclose%
\pgfusepath{stroke,fill}%
\end{pgfscope}%
\begin{pgfscope}%
\pgfpathrectangle{\pgfqpoint{1.000000in}{0.979904in}}{\pgfqpoint{6.200000in}{5.960192in}}%
\pgfusepath{clip}%
\pgfsetbuttcap%
\pgfsetroundjoin%
\definecolor{currentfill}{rgb}{0.200000,0.800000,0.200000}%
\pgfsetfillcolor{currentfill}%
\pgfsetlinewidth{1.003750pt}%
\definecolor{currentstroke}{rgb}{0.200000,0.800000,0.200000}%
\pgfsetstrokecolor{currentstroke}%
\pgfsetdash{}{0pt}%
\pgfpathmoveto{\pgfqpoint{6.048968in}{5.946519in}}%
\pgfpathcurveto{\pgfqpoint{6.054792in}{5.946519in}}{\pgfqpoint{6.060378in}{5.948833in}}{\pgfqpoint{6.064496in}{5.952951in}}%
\pgfpathcurveto{\pgfqpoint{6.068614in}{5.957069in}}{\pgfqpoint{6.070928in}{5.962656in}}{\pgfqpoint{6.070928in}{5.968479in}}%
\pgfpathcurveto{\pgfqpoint{6.070928in}{5.974303in}}{\pgfqpoint{6.068614in}{5.979890in}}{\pgfqpoint{6.064496in}{5.984008in}}%
\pgfpathcurveto{\pgfqpoint{6.060378in}{5.988126in}}{\pgfqpoint{6.054792in}{5.990440in}}{\pgfqpoint{6.048968in}{5.990440in}}%
\pgfpathcurveto{\pgfqpoint{6.043144in}{5.990440in}}{\pgfqpoint{6.037558in}{5.988126in}}{\pgfqpoint{6.033440in}{5.984008in}}%
\pgfpathcurveto{\pgfqpoint{6.029322in}{5.979890in}}{\pgfqpoint{6.027008in}{5.974303in}}{\pgfqpoint{6.027008in}{5.968479in}}%
\pgfpathcurveto{\pgfqpoint{6.027008in}{5.962656in}}{\pgfqpoint{6.029322in}{5.957069in}}{\pgfqpoint{6.033440in}{5.952951in}}%
\pgfpathcurveto{\pgfqpoint{6.037558in}{5.948833in}}{\pgfqpoint{6.043144in}{5.946519in}}{\pgfqpoint{6.048968in}{5.946519in}}%
\pgfpathlineto{\pgfqpoint{6.048968in}{5.946519in}}%
\pgfpathclose%
\pgfusepath{stroke,fill}%
\end{pgfscope}%
\begin{pgfscope}%
\pgfpathrectangle{\pgfqpoint{1.000000in}{0.979904in}}{\pgfqpoint{6.200000in}{5.960192in}}%
\pgfusepath{clip}%
\pgfsetbuttcap%
\pgfsetroundjoin%
\definecolor{currentfill}{rgb}{0.200000,0.800000,0.200000}%
\pgfsetfillcolor{currentfill}%
\pgfsetlinewidth{1.003750pt}%
\definecolor{currentstroke}{rgb}{0.200000,0.800000,0.200000}%
\pgfsetstrokecolor{currentstroke}%
\pgfsetdash{}{0pt}%
\pgfpathmoveto{\pgfqpoint{5.861730in}{5.868480in}}%
\pgfpathcurveto{\pgfqpoint{5.867554in}{5.868480in}}{\pgfqpoint{5.873140in}{5.870794in}}{\pgfqpoint{5.877258in}{5.874912in}}%
\pgfpathcurveto{\pgfqpoint{5.881376in}{5.879031in}}{\pgfqpoint{5.883690in}{5.884617in}}{\pgfqpoint{5.883690in}{5.890441in}}%
\pgfpathcurveto{\pgfqpoint{5.883690in}{5.896265in}}{\pgfqpoint{5.881376in}{5.901851in}}{\pgfqpoint{5.877258in}{5.905969in}}%
\pgfpathcurveto{\pgfqpoint{5.873140in}{5.910087in}}{\pgfqpoint{5.867554in}{5.912401in}}{\pgfqpoint{5.861730in}{5.912401in}}%
\pgfpathcurveto{\pgfqpoint{5.855906in}{5.912401in}}{\pgfqpoint{5.850320in}{5.910087in}}{\pgfqpoint{5.846201in}{5.905969in}}%
\pgfpathcurveto{\pgfqpoint{5.842083in}{5.901851in}}{\pgfqpoint{5.839769in}{5.896265in}}{\pgfqpoint{5.839769in}{5.890441in}}%
\pgfpathcurveto{\pgfqpoint{5.839769in}{5.884617in}}{\pgfqpoint{5.842083in}{5.879031in}}{\pgfqpoint{5.846201in}{5.874912in}}%
\pgfpathcurveto{\pgfqpoint{5.850320in}{5.870794in}}{\pgfqpoint{5.855906in}{5.868480in}}{\pgfqpoint{5.861730in}{5.868480in}}%
\pgfpathlineto{\pgfqpoint{5.861730in}{5.868480in}}%
\pgfpathclose%
\pgfusepath{stroke,fill}%
\end{pgfscope}%
\begin{pgfscope}%
\pgfpathrectangle{\pgfqpoint{1.000000in}{0.979904in}}{\pgfqpoint{6.200000in}{5.960192in}}%
\pgfusepath{clip}%
\pgfsetbuttcap%
\pgfsetroundjoin%
\definecolor{currentfill}{rgb}{0.200000,0.800000,0.200000}%
\pgfsetfillcolor{currentfill}%
\pgfsetlinewidth{1.003750pt}%
\definecolor{currentstroke}{rgb}{0.200000,0.800000,0.200000}%
\pgfsetstrokecolor{currentstroke}%
\pgfsetdash{}{0pt}%
\pgfpathmoveto{\pgfqpoint{5.779429in}{5.941974in}}%
\pgfpathcurveto{\pgfqpoint{5.785253in}{5.941974in}}{\pgfqpoint{5.790839in}{5.944288in}}{\pgfqpoint{5.794958in}{5.948406in}}%
\pgfpathcurveto{\pgfqpoint{5.799076in}{5.952524in}}{\pgfqpoint{5.801390in}{5.958111in}}{\pgfqpoint{5.801390in}{5.963935in}}%
\pgfpathcurveto{\pgfqpoint{5.801390in}{5.969758in}}{\pgfqpoint{5.799076in}{5.975345in}}{\pgfqpoint{5.794958in}{5.979463in}}%
\pgfpathcurveto{\pgfqpoint{5.790839in}{5.983581in}}{\pgfqpoint{5.785253in}{5.985895in}}{\pgfqpoint{5.779429in}{5.985895in}}%
\pgfpathcurveto{\pgfqpoint{5.773605in}{5.985895in}}{\pgfqpoint{5.768019in}{5.983581in}}{\pgfqpoint{5.763901in}{5.979463in}}%
\pgfpathcurveto{\pgfqpoint{5.759783in}{5.975345in}}{\pgfqpoint{5.757469in}{5.969758in}}{\pgfqpoint{5.757469in}{5.963935in}}%
\pgfpathcurveto{\pgfqpoint{5.757469in}{5.958111in}}{\pgfqpoint{5.759783in}{5.952524in}}{\pgfqpoint{5.763901in}{5.948406in}}%
\pgfpathcurveto{\pgfqpoint{5.768019in}{5.944288in}}{\pgfqpoint{5.773605in}{5.941974in}}{\pgfqpoint{5.779429in}{5.941974in}}%
\pgfpathlineto{\pgfqpoint{5.779429in}{5.941974in}}%
\pgfpathclose%
\pgfusepath{stroke,fill}%
\end{pgfscope}%
\begin{pgfscope}%
\pgfpathrectangle{\pgfqpoint{1.000000in}{0.979904in}}{\pgfqpoint{6.200000in}{5.960192in}}%
\pgfusepath{clip}%
\pgfsetbuttcap%
\pgfsetroundjoin%
\definecolor{currentfill}{rgb}{0.200000,0.800000,0.200000}%
\pgfsetfillcolor{currentfill}%
\pgfsetlinewidth{1.003750pt}%
\definecolor{currentstroke}{rgb}{0.200000,0.800000,0.200000}%
\pgfsetstrokecolor{currentstroke}%
\pgfsetdash{}{0pt}%
\pgfpathmoveto{\pgfqpoint{5.696681in}{6.022400in}}%
\pgfpathcurveto{\pgfqpoint{5.702505in}{6.022400in}}{\pgfqpoint{5.708091in}{6.024714in}}{\pgfqpoint{5.712209in}{6.028832in}}%
\pgfpathcurveto{\pgfqpoint{5.716327in}{6.032950in}}{\pgfqpoint{5.718641in}{6.038536in}}{\pgfqpoint{5.718641in}{6.044360in}}%
\pgfpathcurveto{\pgfqpoint{5.718641in}{6.050184in}}{\pgfqpoint{5.716327in}{6.055770in}}{\pgfqpoint{5.712209in}{6.059889in}}%
\pgfpathcurveto{\pgfqpoint{5.708091in}{6.064007in}}{\pgfqpoint{5.702505in}{6.066321in}}{\pgfqpoint{5.696681in}{6.066321in}}%
\pgfpathcurveto{\pgfqpoint{5.690857in}{6.066321in}}{\pgfqpoint{5.685271in}{6.064007in}}{\pgfqpoint{5.681153in}{6.059889in}}%
\pgfpathcurveto{\pgfqpoint{5.677035in}{6.055770in}}{\pgfqpoint{5.674721in}{6.050184in}}{\pgfqpoint{5.674721in}{6.044360in}}%
\pgfpathcurveto{\pgfqpoint{5.674721in}{6.038536in}}{\pgfqpoint{5.677035in}{6.032950in}}{\pgfqpoint{5.681153in}{6.028832in}}%
\pgfpathcurveto{\pgfqpoint{5.685271in}{6.024714in}}{\pgfqpoint{5.690857in}{6.022400in}}{\pgfqpoint{5.696681in}{6.022400in}}%
\pgfpathlineto{\pgfqpoint{5.696681in}{6.022400in}}%
\pgfpathclose%
\pgfusepath{stroke,fill}%
\end{pgfscope}%
\begin{pgfscope}%
\pgfpathrectangle{\pgfqpoint{1.000000in}{0.979904in}}{\pgfqpoint{6.200000in}{5.960192in}}%
\pgfusepath{clip}%
\pgfsetbuttcap%
\pgfsetroundjoin%
\definecolor{currentfill}{rgb}{0.200000,0.800000,0.200000}%
\pgfsetfillcolor{currentfill}%
\pgfsetlinewidth{1.003750pt}%
\definecolor{currentstroke}{rgb}{0.200000,0.800000,0.200000}%
\pgfsetstrokecolor{currentstroke}%
\pgfsetdash{}{0pt}%
\pgfpathmoveto{\pgfqpoint{5.582900in}{6.035383in}}%
\pgfpathcurveto{\pgfqpoint{5.588724in}{6.035383in}}{\pgfqpoint{5.594310in}{6.037697in}}{\pgfqpoint{5.598429in}{6.041815in}}%
\pgfpathcurveto{\pgfqpoint{5.602547in}{6.045933in}}{\pgfqpoint{5.604861in}{6.051519in}}{\pgfqpoint{5.604861in}{6.057343in}}%
\pgfpathcurveto{\pgfqpoint{5.604861in}{6.063167in}}{\pgfqpoint{5.602547in}{6.068753in}}{\pgfqpoint{5.598429in}{6.072871in}}%
\pgfpathcurveto{\pgfqpoint{5.594310in}{6.076990in}}{\pgfqpoint{5.588724in}{6.079303in}}{\pgfqpoint{5.582900in}{6.079303in}}%
\pgfpathcurveto{\pgfqpoint{5.577076in}{6.079303in}}{\pgfqpoint{5.571490in}{6.076990in}}{\pgfqpoint{5.567372in}{6.072871in}}%
\pgfpathcurveto{\pgfqpoint{5.563254in}{6.068753in}}{\pgfqpoint{5.560940in}{6.063167in}}{\pgfqpoint{5.560940in}{6.057343in}}%
\pgfpathcurveto{\pgfqpoint{5.560940in}{6.051519in}}{\pgfqpoint{5.563254in}{6.045933in}}{\pgfqpoint{5.567372in}{6.041815in}}%
\pgfpathcurveto{\pgfqpoint{5.571490in}{6.037697in}}{\pgfqpoint{5.577076in}{6.035383in}}{\pgfqpoint{5.582900in}{6.035383in}}%
\pgfpathlineto{\pgfqpoint{5.582900in}{6.035383in}}%
\pgfpathclose%
\pgfusepath{stroke,fill}%
\end{pgfscope}%
\begin{pgfscope}%
\pgfpathrectangle{\pgfqpoint{1.000000in}{0.979904in}}{\pgfqpoint{6.200000in}{5.960192in}}%
\pgfusepath{clip}%
\pgfsetbuttcap%
\pgfsetroundjoin%
\definecolor{currentfill}{rgb}{0.200000,0.800000,0.200000}%
\pgfsetfillcolor{currentfill}%
\pgfsetlinewidth{1.003750pt}%
\definecolor{currentstroke}{rgb}{0.200000,0.800000,0.200000}%
\pgfsetstrokecolor{currentstroke}%
\pgfsetdash{}{0pt}%
\pgfpathmoveto{\pgfqpoint{5.463449in}{6.017961in}}%
\pgfpathcurveto{\pgfqpoint{5.469272in}{6.017961in}}{\pgfqpoint{5.474859in}{6.020275in}}{\pgfqpoint{5.478977in}{6.024393in}}%
\pgfpathcurveto{\pgfqpoint{5.483095in}{6.028511in}}{\pgfqpoint{5.485409in}{6.034098in}}{\pgfqpoint{5.485409in}{6.039922in}}%
\pgfpathcurveto{\pgfqpoint{5.485409in}{6.045745in}}{\pgfqpoint{5.483095in}{6.051332in}}{\pgfqpoint{5.478977in}{6.055450in}}%
\pgfpathcurveto{\pgfqpoint{5.474859in}{6.059568in}}{\pgfqpoint{5.469272in}{6.061882in}}{\pgfqpoint{5.463449in}{6.061882in}}%
\pgfpathcurveto{\pgfqpoint{5.457625in}{6.061882in}}{\pgfqpoint{5.452038in}{6.059568in}}{\pgfqpoint{5.447920in}{6.055450in}}%
\pgfpathcurveto{\pgfqpoint{5.443802in}{6.051332in}}{\pgfqpoint{5.441488in}{6.045745in}}{\pgfqpoint{5.441488in}{6.039922in}}%
\pgfpathcurveto{\pgfqpoint{5.441488in}{6.034098in}}{\pgfqpoint{5.443802in}{6.028511in}}{\pgfqpoint{5.447920in}{6.024393in}}%
\pgfpathcurveto{\pgfqpoint{5.452038in}{6.020275in}}{\pgfqpoint{5.457625in}{6.017961in}}{\pgfqpoint{5.463449in}{6.017961in}}%
\pgfpathlineto{\pgfqpoint{5.463449in}{6.017961in}}%
\pgfpathclose%
\pgfusepath{stroke,fill}%
\end{pgfscope}%
\begin{pgfscope}%
\pgfpathrectangle{\pgfqpoint{1.000000in}{0.979904in}}{\pgfqpoint{6.200000in}{5.960192in}}%
\pgfusepath{clip}%
\pgfsetbuttcap%
\pgfsetroundjoin%
\definecolor{currentfill}{rgb}{0.200000,0.800000,0.200000}%
\pgfsetfillcolor{currentfill}%
\pgfsetlinewidth{1.003750pt}%
\definecolor{currentstroke}{rgb}{0.200000,0.800000,0.200000}%
\pgfsetstrokecolor{currentstroke}%
\pgfsetdash{}{0pt}%
\pgfpathmoveto{\pgfqpoint{5.382093in}{6.129542in}}%
\pgfpathcurveto{\pgfqpoint{5.387917in}{6.129542in}}{\pgfqpoint{5.393504in}{6.131856in}}{\pgfqpoint{5.397622in}{6.135974in}}%
\pgfpathcurveto{\pgfqpoint{5.401740in}{6.140092in}}{\pgfqpoint{5.404054in}{6.145679in}}{\pgfqpoint{5.404054in}{6.151503in}}%
\pgfpathcurveto{\pgfqpoint{5.404054in}{6.157326in}}{\pgfqpoint{5.401740in}{6.162913in}}{\pgfqpoint{5.397622in}{6.167031in}}%
\pgfpathcurveto{\pgfqpoint{5.393504in}{6.171149in}}{\pgfqpoint{5.387917in}{6.173463in}}{\pgfqpoint{5.382093in}{6.173463in}}%
\pgfpathcurveto{\pgfqpoint{5.376270in}{6.173463in}}{\pgfqpoint{5.370683in}{6.171149in}}{\pgfqpoint{5.366565in}{6.167031in}}%
\pgfpathcurveto{\pgfqpoint{5.362447in}{6.162913in}}{\pgfqpoint{5.360133in}{6.157326in}}{\pgfqpoint{5.360133in}{6.151503in}}%
\pgfpathcurveto{\pgfqpoint{5.360133in}{6.145679in}}{\pgfqpoint{5.362447in}{6.140092in}}{\pgfqpoint{5.366565in}{6.135974in}}%
\pgfpathcurveto{\pgfqpoint{5.370683in}{6.131856in}}{\pgfqpoint{5.376270in}{6.129542in}}{\pgfqpoint{5.382093in}{6.129542in}}%
\pgfpathlineto{\pgfqpoint{5.382093in}{6.129542in}}%
\pgfpathclose%
\pgfusepath{stroke,fill}%
\end{pgfscope}%
\begin{pgfscope}%
\pgfpathrectangle{\pgfqpoint{1.000000in}{0.979904in}}{\pgfqpoint{6.200000in}{5.960192in}}%
\pgfusepath{clip}%
\pgfsetbuttcap%
\pgfsetroundjoin%
\definecolor{currentfill}{rgb}{0.200000,0.800000,0.200000}%
\pgfsetfillcolor{currentfill}%
\pgfsetlinewidth{1.003750pt}%
\definecolor{currentstroke}{rgb}{0.200000,0.800000,0.200000}%
\pgfsetstrokecolor{currentstroke}%
\pgfsetdash{}{0pt}%
\pgfpathmoveto{\pgfqpoint{5.263152in}{6.093161in}}%
\pgfpathcurveto{\pgfqpoint{5.268976in}{6.093161in}}{\pgfqpoint{5.274562in}{6.095475in}}{\pgfqpoint{5.278680in}{6.099593in}}%
\pgfpathcurveto{\pgfqpoint{5.282799in}{6.103711in}}{\pgfqpoint{5.285112in}{6.109298in}}{\pgfqpoint{5.285112in}{6.115122in}}%
\pgfpathcurveto{\pgfqpoint{5.285112in}{6.120945in}}{\pgfqpoint{5.282799in}{6.126532in}}{\pgfqpoint{5.278680in}{6.130650in}}%
\pgfpathcurveto{\pgfqpoint{5.274562in}{6.134768in}}{\pgfqpoint{5.268976in}{6.137082in}}{\pgfqpoint{5.263152in}{6.137082in}}%
\pgfpathcurveto{\pgfqpoint{5.257328in}{6.137082in}}{\pgfqpoint{5.251742in}{6.134768in}}{\pgfqpoint{5.247624in}{6.130650in}}%
\pgfpathcurveto{\pgfqpoint{5.243506in}{6.126532in}}{\pgfqpoint{5.241192in}{6.120945in}}{\pgfqpoint{5.241192in}{6.115122in}}%
\pgfpathcurveto{\pgfqpoint{5.241192in}{6.109298in}}{\pgfqpoint{5.243506in}{6.103711in}}{\pgfqpoint{5.247624in}{6.099593in}}%
\pgfpathcurveto{\pgfqpoint{5.251742in}{6.095475in}}{\pgfqpoint{5.257328in}{6.093161in}}{\pgfqpoint{5.263152in}{6.093161in}}%
\pgfpathlineto{\pgfqpoint{5.263152in}{6.093161in}}%
\pgfpathclose%
\pgfusepath{stroke,fill}%
\end{pgfscope}%
\begin{pgfscope}%
\pgfpathrectangle{\pgfqpoint{1.000000in}{0.979904in}}{\pgfqpoint{6.200000in}{5.960192in}}%
\pgfusepath{clip}%
\pgfsetbuttcap%
\pgfsetroundjoin%
\definecolor{currentfill}{rgb}{0.200000,0.800000,0.200000}%
\pgfsetfillcolor{currentfill}%
\pgfsetlinewidth{1.003750pt}%
\definecolor{currentstroke}{rgb}{0.200000,0.800000,0.200000}%
\pgfsetstrokecolor{currentstroke}%
\pgfsetdash{}{0pt}%
\pgfpathmoveto{\pgfqpoint{5.166209in}{6.189386in}}%
\pgfpathcurveto{\pgfqpoint{5.172033in}{6.189386in}}{\pgfqpoint{5.177619in}{6.191700in}}{\pgfqpoint{5.181737in}{6.195818in}}%
\pgfpathcurveto{\pgfqpoint{5.185855in}{6.199936in}}{\pgfqpoint{5.188169in}{6.205522in}}{\pgfqpoint{5.188169in}{6.211346in}}%
\pgfpathcurveto{\pgfqpoint{5.188169in}{6.217170in}}{\pgfqpoint{5.185855in}{6.222756in}}{\pgfqpoint{5.181737in}{6.226874in}}%
\pgfpathcurveto{\pgfqpoint{5.177619in}{6.230992in}}{\pgfqpoint{5.172033in}{6.233306in}}{\pgfqpoint{5.166209in}{6.233306in}}%
\pgfpathcurveto{\pgfqpoint{5.160385in}{6.233306in}}{\pgfqpoint{5.154799in}{6.230992in}}{\pgfqpoint{5.150681in}{6.226874in}}%
\pgfpathcurveto{\pgfqpoint{5.146562in}{6.222756in}}{\pgfqpoint{5.144248in}{6.217170in}}{\pgfqpoint{5.144248in}{6.211346in}}%
\pgfpathcurveto{\pgfqpoint{5.144248in}{6.205522in}}{\pgfqpoint{5.146562in}{6.199936in}}{\pgfqpoint{5.150681in}{6.195818in}}%
\pgfpathcurveto{\pgfqpoint{5.154799in}{6.191700in}}{\pgfqpoint{5.160385in}{6.189386in}}{\pgfqpoint{5.166209in}{6.189386in}}%
\pgfpathlineto{\pgfqpoint{5.166209in}{6.189386in}}%
\pgfpathclose%
\pgfusepath{stroke,fill}%
\end{pgfscope}%
\begin{pgfscope}%
\pgfpathrectangle{\pgfqpoint{1.000000in}{0.979904in}}{\pgfqpoint{6.200000in}{5.960192in}}%
\pgfusepath{clip}%
\pgfsetbuttcap%
\pgfsetroundjoin%
\definecolor{currentfill}{rgb}{0.200000,0.800000,0.200000}%
\pgfsetfillcolor{currentfill}%
\pgfsetlinewidth{1.003750pt}%
\definecolor{currentstroke}{rgb}{0.200000,0.800000,0.200000}%
\pgfsetstrokecolor{currentstroke}%
\pgfsetdash{}{0pt}%
\pgfpathmoveto{\pgfqpoint{5.050809in}{6.124455in}}%
\pgfpathcurveto{\pgfqpoint{5.056633in}{6.124455in}}{\pgfqpoint{5.062219in}{6.126769in}}{\pgfqpoint{5.066337in}{6.130887in}}%
\pgfpathcurveto{\pgfqpoint{5.070455in}{6.135005in}}{\pgfqpoint{5.072769in}{6.140592in}}{\pgfqpoint{5.072769in}{6.146416in}}%
\pgfpathcurveto{\pgfqpoint{5.072769in}{6.152239in}}{\pgfqpoint{5.070455in}{6.157826in}}{\pgfqpoint{5.066337in}{6.161944in}}%
\pgfpathcurveto{\pgfqpoint{5.062219in}{6.166062in}}{\pgfqpoint{5.056633in}{6.168376in}}{\pgfqpoint{5.050809in}{6.168376in}}%
\pgfpathcurveto{\pgfqpoint{5.044985in}{6.168376in}}{\pgfqpoint{5.039399in}{6.166062in}}{\pgfqpoint{5.035280in}{6.161944in}}%
\pgfpathcurveto{\pgfqpoint{5.031162in}{6.157826in}}{\pgfqpoint{5.028848in}{6.152239in}}{\pgfqpoint{5.028848in}{6.146416in}}%
\pgfpathcurveto{\pgfqpoint{5.028848in}{6.140592in}}{\pgfqpoint{5.031162in}{6.135005in}}{\pgfqpoint{5.035280in}{6.130887in}}%
\pgfpathcurveto{\pgfqpoint{5.039399in}{6.126769in}}{\pgfqpoint{5.044985in}{6.124455in}}{\pgfqpoint{5.050809in}{6.124455in}}%
\pgfpathlineto{\pgfqpoint{5.050809in}{6.124455in}}%
\pgfpathclose%
\pgfusepath{stroke,fill}%
\end{pgfscope}%
\begin{pgfscope}%
\pgfpathrectangle{\pgfqpoint{1.000000in}{0.979904in}}{\pgfqpoint{6.200000in}{5.960192in}}%
\pgfusepath{clip}%
\pgfsetbuttcap%
\pgfsetroundjoin%
\definecolor{currentfill}{rgb}{0.200000,0.800000,0.200000}%
\pgfsetfillcolor{currentfill}%
\pgfsetlinewidth{1.003750pt}%
\definecolor{currentstroke}{rgb}{0.200000,0.800000,0.200000}%
\pgfsetstrokecolor{currentstroke}%
\pgfsetdash{}{0pt}%
\pgfpathmoveto{\pgfqpoint{4.941197in}{6.257161in}}%
\pgfpathcurveto{\pgfqpoint{4.947021in}{6.257161in}}{\pgfqpoint{4.952607in}{6.259475in}}{\pgfqpoint{4.956725in}{6.263593in}}%
\pgfpathcurveto{\pgfqpoint{4.960844in}{6.267711in}}{\pgfqpoint{4.963157in}{6.273297in}}{\pgfqpoint{4.963157in}{6.279121in}}%
\pgfpathcurveto{\pgfqpoint{4.963157in}{6.284945in}}{\pgfqpoint{4.960844in}{6.290531in}}{\pgfqpoint{4.956725in}{6.294650in}}%
\pgfpathcurveto{\pgfqpoint{4.952607in}{6.298768in}}{\pgfqpoint{4.947021in}{6.301082in}}{\pgfqpoint{4.941197in}{6.301082in}}%
\pgfpathcurveto{\pgfqpoint{4.935373in}{6.301082in}}{\pgfqpoint{4.929787in}{6.298768in}}{\pgfqpoint{4.925669in}{6.294650in}}%
\pgfpathcurveto{\pgfqpoint{4.921551in}{6.290531in}}{\pgfqpoint{4.919237in}{6.284945in}}{\pgfqpoint{4.919237in}{6.279121in}}%
\pgfpathcurveto{\pgfqpoint{4.919237in}{6.273297in}}{\pgfqpoint{4.921551in}{6.267711in}}{\pgfqpoint{4.925669in}{6.263593in}}%
\pgfpathcurveto{\pgfqpoint{4.929787in}{6.259475in}}{\pgfqpoint{4.935373in}{6.257161in}}{\pgfqpoint{4.941197in}{6.257161in}}%
\pgfpathlineto{\pgfqpoint{4.941197in}{6.257161in}}%
\pgfpathclose%
\pgfusepath{stroke,fill}%
\end{pgfscope}%
\begin{pgfscope}%
\pgfpathrectangle{\pgfqpoint{1.000000in}{0.979904in}}{\pgfqpoint{6.200000in}{5.960192in}}%
\pgfusepath{clip}%
\pgfsetbuttcap%
\pgfsetroundjoin%
\definecolor{currentfill}{rgb}{0.200000,0.800000,0.200000}%
\pgfsetfillcolor{currentfill}%
\pgfsetlinewidth{1.003750pt}%
\definecolor{currentstroke}{rgb}{0.200000,0.800000,0.200000}%
\pgfsetstrokecolor{currentstroke}%
\pgfsetdash{}{0pt}%
\pgfpathmoveto{\pgfqpoint{4.835279in}{6.128246in}}%
\pgfpathcurveto{\pgfqpoint{4.841103in}{6.128246in}}{\pgfqpoint{4.846689in}{6.130560in}}{\pgfqpoint{4.850807in}{6.134678in}}%
\pgfpathcurveto{\pgfqpoint{4.854925in}{6.138797in}}{\pgfqpoint{4.857239in}{6.144383in}}{\pgfqpoint{4.857239in}{6.150207in}}%
\pgfpathcurveto{\pgfqpoint{4.857239in}{6.156031in}}{\pgfqpoint{4.854925in}{6.161617in}}{\pgfqpoint{4.850807in}{6.165735in}}%
\pgfpathcurveto{\pgfqpoint{4.846689in}{6.169853in}}{\pgfqpoint{4.841103in}{6.172167in}}{\pgfqpoint{4.835279in}{6.172167in}}%
\pgfpathcurveto{\pgfqpoint{4.829455in}{6.172167in}}{\pgfqpoint{4.823869in}{6.169853in}}{\pgfqpoint{4.819751in}{6.165735in}}%
\pgfpathcurveto{\pgfqpoint{4.815632in}{6.161617in}}{\pgfqpoint{4.813319in}{6.156031in}}{\pgfqpoint{4.813319in}{6.150207in}}%
\pgfpathcurveto{\pgfqpoint{4.813319in}{6.144383in}}{\pgfqpoint{4.815632in}{6.138797in}}{\pgfqpoint{4.819751in}{6.134678in}}%
\pgfpathcurveto{\pgfqpoint{4.823869in}{6.130560in}}{\pgfqpoint{4.829455in}{6.128246in}}{\pgfqpoint{4.835279in}{6.128246in}}%
\pgfpathlineto{\pgfqpoint{4.835279in}{6.128246in}}%
\pgfpathclose%
\pgfusepath{stroke,fill}%
\end{pgfscope}%
\begin{pgfscope}%
\pgfpathrectangle{\pgfqpoint{1.000000in}{0.979904in}}{\pgfqpoint{6.200000in}{5.960192in}}%
\pgfusepath{clip}%
\pgfsetbuttcap%
\pgfsetroundjoin%
\definecolor{currentfill}{rgb}{0.200000,0.800000,0.200000}%
\pgfsetfillcolor{currentfill}%
\pgfsetlinewidth{1.003750pt}%
\definecolor{currentstroke}{rgb}{0.200000,0.800000,0.200000}%
\pgfsetstrokecolor{currentstroke}%
\pgfsetdash{}{0pt}%
\pgfpathmoveto{\pgfqpoint{4.724457in}{6.140496in}}%
\pgfpathcurveto{\pgfqpoint{4.730281in}{6.140496in}}{\pgfqpoint{4.735867in}{6.142809in}}{\pgfqpoint{4.739985in}{6.146928in}}%
\pgfpathcurveto{\pgfqpoint{4.744104in}{6.151046in}}{\pgfqpoint{4.746417in}{6.156632in}}{\pgfqpoint{4.746417in}{6.162456in}}%
\pgfpathcurveto{\pgfqpoint{4.746417in}{6.168280in}}{\pgfqpoint{4.744104in}{6.173866in}}{\pgfqpoint{4.739985in}{6.177984in}}%
\pgfpathcurveto{\pgfqpoint{4.735867in}{6.182102in}}{\pgfqpoint{4.730281in}{6.184416in}}{\pgfqpoint{4.724457in}{6.184416in}}%
\pgfpathcurveto{\pgfqpoint{4.718633in}{6.184416in}}{\pgfqpoint{4.713047in}{6.182102in}}{\pgfqpoint{4.708929in}{6.177984in}}%
\pgfpathcurveto{\pgfqpoint{4.704811in}{6.173866in}}{\pgfqpoint{4.702497in}{6.168280in}}{\pgfqpoint{4.702497in}{6.162456in}}%
\pgfpathcurveto{\pgfqpoint{4.702497in}{6.156632in}}{\pgfqpoint{4.704811in}{6.151046in}}{\pgfqpoint{4.708929in}{6.146928in}}%
\pgfpathcurveto{\pgfqpoint{4.713047in}{6.142809in}}{\pgfqpoint{4.718633in}{6.140496in}}{\pgfqpoint{4.724457in}{6.140496in}}%
\pgfpathlineto{\pgfqpoint{4.724457in}{6.140496in}}%
\pgfpathclose%
\pgfusepath{stroke,fill}%
\end{pgfscope}%
\begin{pgfscope}%
\pgfpathrectangle{\pgfqpoint{1.000000in}{0.979904in}}{\pgfqpoint{6.200000in}{5.960192in}}%
\pgfusepath{clip}%
\pgfsetbuttcap%
\pgfsetroundjoin%
\definecolor{currentfill}{rgb}{0.200000,0.800000,0.200000}%
\pgfsetfillcolor{currentfill}%
\pgfsetlinewidth{1.003750pt}%
\definecolor{currentstroke}{rgb}{0.200000,0.800000,0.200000}%
\pgfsetstrokecolor{currentstroke}%
\pgfsetdash{}{0pt}%
\pgfpathmoveto{\pgfqpoint{4.604224in}{6.180444in}}%
\pgfpathcurveto{\pgfqpoint{4.610048in}{6.180444in}}{\pgfqpoint{4.615635in}{6.182758in}}{\pgfqpoint{4.619753in}{6.186876in}}%
\pgfpathcurveto{\pgfqpoint{4.623871in}{6.190994in}}{\pgfqpoint{4.626185in}{6.196580in}}{\pgfqpoint{4.626185in}{6.202404in}}%
\pgfpathcurveto{\pgfqpoint{4.626185in}{6.208228in}}{\pgfqpoint{4.623871in}{6.213814in}}{\pgfqpoint{4.619753in}{6.217932in}}%
\pgfpathcurveto{\pgfqpoint{4.615635in}{6.222051in}}{\pgfqpoint{4.610048in}{6.224364in}}{\pgfqpoint{4.604224in}{6.224364in}}%
\pgfpathcurveto{\pgfqpoint{4.598400in}{6.224364in}}{\pgfqpoint{4.592814in}{6.222051in}}{\pgfqpoint{4.588696in}{6.217932in}}%
\pgfpathcurveto{\pgfqpoint{4.584578in}{6.213814in}}{\pgfqpoint{4.582264in}{6.208228in}}{\pgfqpoint{4.582264in}{6.202404in}}%
\pgfpathcurveto{\pgfqpoint{4.582264in}{6.196580in}}{\pgfqpoint{4.584578in}{6.190994in}}{\pgfqpoint{4.588696in}{6.186876in}}%
\pgfpathcurveto{\pgfqpoint{4.592814in}{6.182758in}}{\pgfqpoint{4.598400in}{6.180444in}}{\pgfqpoint{4.604224in}{6.180444in}}%
\pgfpathlineto{\pgfqpoint{4.604224in}{6.180444in}}%
\pgfpathclose%
\pgfusepath{stroke,fill}%
\end{pgfscope}%
\begin{pgfscope}%
\pgfpathrectangle{\pgfqpoint{1.000000in}{0.979904in}}{\pgfqpoint{6.200000in}{5.960192in}}%
\pgfusepath{clip}%
\pgfsetbuttcap%
\pgfsetroundjoin%
\definecolor{currentfill}{rgb}{0.200000,0.200000,0.800000}%
\pgfsetfillcolor{currentfill}%
\pgfsetlinewidth{1.003750pt}%
\definecolor{currentstroke}{rgb}{0.200000,0.200000,0.800000}%
\pgfsetstrokecolor{currentstroke}%
\pgfsetdash{}{0pt}%
\pgfpathmoveto{\pgfqpoint{4.522944in}{6.049167in}}%
\pgfpathcurveto{\pgfqpoint{4.528768in}{6.049167in}}{\pgfqpoint{4.534354in}{6.051481in}}{\pgfqpoint{4.538472in}{6.055599in}}%
\pgfpathcurveto{\pgfqpoint{4.542590in}{6.059718in}}{\pgfqpoint{4.544904in}{6.065304in}}{\pgfqpoint{4.544904in}{6.071128in}}%
\pgfpathcurveto{\pgfqpoint{4.544904in}{6.076952in}}{\pgfqpoint{4.542590in}{6.082538in}}{\pgfqpoint{4.538472in}{6.086656in}}%
\pgfpathcurveto{\pgfqpoint{4.534354in}{6.090774in}}{\pgfqpoint{4.528768in}{6.093088in}}{\pgfqpoint{4.522944in}{6.093088in}}%
\pgfpathcurveto{\pgfqpoint{4.517120in}{6.093088in}}{\pgfqpoint{4.511533in}{6.090774in}}{\pgfqpoint{4.507415in}{6.086656in}}%
\pgfpathcurveto{\pgfqpoint{4.503297in}{6.082538in}}{\pgfqpoint{4.500983in}{6.076952in}}{\pgfqpoint{4.500983in}{6.071128in}}%
\pgfpathcurveto{\pgfqpoint{4.500983in}{6.065304in}}{\pgfqpoint{4.503297in}{6.059718in}}{\pgfqpoint{4.507415in}{6.055599in}}%
\pgfpathcurveto{\pgfqpoint{4.511533in}{6.051481in}}{\pgfqpoint{4.517120in}{6.049167in}}{\pgfqpoint{4.522944in}{6.049167in}}%
\pgfpathlineto{\pgfqpoint{4.522944in}{6.049167in}}%
\pgfpathclose%
\pgfusepath{stroke,fill}%
\end{pgfscope}%
\begin{pgfscope}%
\pgfpathrectangle{\pgfqpoint{1.000000in}{0.979904in}}{\pgfqpoint{6.200000in}{5.960192in}}%
\pgfusepath{clip}%
\pgfsetbuttcap%
\pgfsetroundjoin%
\definecolor{currentfill}{rgb}{0.200000,0.800000,0.200000}%
\pgfsetfillcolor{currentfill}%
\pgfsetlinewidth{1.003750pt}%
\definecolor{currentstroke}{rgb}{0.200000,0.800000,0.200000}%
\pgfsetstrokecolor{currentstroke}%
\pgfsetdash{}{0pt}%
\pgfpathmoveto{\pgfqpoint{4.395942in}{6.090664in}}%
\pgfpathcurveto{\pgfqpoint{4.401766in}{6.090664in}}{\pgfqpoint{4.407352in}{6.092978in}}{\pgfqpoint{4.411470in}{6.097096in}}%
\pgfpathcurveto{\pgfqpoint{4.415589in}{6.101214in}}{\pgfqpoint{4.417902in}{6.106801in}}{\pgfqpoint{4.417902in}{6.112625in}}%
\pgfpathcurveto{\pgfqpoint{4.417902in}{6.118449in}}{\pgfqpoint{4.415589in}{6.124035in}}{\pgfqpoint{4.411470in}{6.128153in}}%
\pgfpathcurveto{\pgfqpoint{4.407352in}{6.132271in}}{\pgfqpoint{4.401766in}{6.134585in}}{\pgfqpoint{4.395942in}{6.134585in}}%
\pgfpathcurveto{\pgfqpoint{4.390118in}{6.134585in}}{\pgfqpoint{4.384532in}{6.132271in}}{\pgfqpoint{4.380414in}{6.128153in}}%
\pgfpathcurveto{\pgfqpoint{4.376296in}{6.124035in}}{\pgfqpoint{4.373982in}{6.118449in}}{\pgfqpoint{4.373982in}{6.112625in}}%
\pgfpathcurveto{\pgfqpoint{4.373982in}{6.106801in}}{\pgfqpoint{4.376296in}{6.101214in}}{\pgfqpoint{4.380414in}{6.097096in}}%
\pgfpathcurveto{\pgfqpoint{4.384532in}{6.092978in}}{\pgfqpoint{4.390118in}{6.090664in}}{\pgfqpoint{4.395942in}{6.090664in}}%
\pgfpathlineto{\pgfqpoint{4.395942in}{6.090664in}}%
\pgfpathclose%
\pgfusepath{stroke,fill}%
\end{pgfscope}%
\begin{pgfscope}%
\pgfpathrectangle{\pgfqpoint{1.000000in}{0.979904in}}{\pgfqpoint{6.200000in}{5.960192in}}%
\pgfusepath{clip}%
\pgfsetbuttcap%
\pgfsetroundjoin%
\definecolor{currentfill}{rgb}{0.200000,0.800000,0.200000}%
\pgfsetfillcolor{currentfill}%
\pgfsetlinewidth{1.003750pt}%
\definecolor{currentstroke}{rgb}{0.200000,0.800000,0.200000}%
\pgfsetstrokecolor{currentstroke}%
\pgfsetdash{}{0pt}%
\pgfpathmoveto{\pgfqpoint{4.269739in}{6.103741in}}%
\pgfpathcurveto{\pgfqpoint{4.275563in}{6.103741in}}{\pgfqpoint{4.281149in}{6.106055in}}{\pgfqpoint{4.285268in}{6.110173in}}%
\pgfpathcurveto{\pgfqpoint{4.289386in}{6.114291in}}{\pgfqpoint{4.291700in}{6.119877in}}{\pgfqpoint{4.291700in}{6.125701in}}%
\pgfpathcurveto{\pgfqpoint{4.291700in}{6.131525in}}{\pgfqpoint{4.289386in}{6.137111in}}{\pgfqpoint{4.285268in}{6.141229in}}%
\pgfpathcurveto{\pgfqpoint{4.281149in}{6.145348in}}{\pgfqpoint{4.275563in}{6.147661in}}{\pgfqpoint{4.269739in}{6.147661in}}%
\pgfpathcurveto{\pgfqpoint{4.263915in}{6.147661in}}{\pgfqpoint{4.258329in}{6.145348in}}{\pgfqpoint{4.254211in}{6.141229in}}%
\pgfpathcurveto{\pgfqpoint{4.250093in}{6.137111in}}{\pgfqpoint{4.247779in}{6.131525in}}{\pgfqpoint{4.247779in}{6.125701in}}%
\pgfpathcurveto{\pgfqpoint{4.247779in}{6.119877in}}{\pgfqpoint{4.250093in}{6.114291in}}{\pgfqpoint{4.254211in}{6.110173in}}%
\pgfpathcurveto{\pgfqpoint{4.258329in}{6.106055in}}{\pgfqpoint{4.263915in}{6.103741in}}{\pgfqpoint{4.269739in}{6.103741in}}%
\pgfpathlineto{\pgfqpoint{4.269739in}{6.103741in}}%
\pgfpathclose%
\pgfusepath{stroke,fill}%
\end{pgfscope}%
\begin{pgfscope}%
\pgfpathrectangle{\pgfqpoint{1.000000in}{0.979904in}}{\pgfqpoint{6.200000in}{5.960192in}}%
\pgfusepath{clip}%
\pgfsetbuttcap%
\pgfsetroundjoin%
\definecolor{currentfill}{rgb}{0.200000,0.800000,0.200000}%
\pgfsetfillcolor{currentfill}%
\pgfsetlinewidth{1.003750pt}%
\definecolor{currentstroke}{rgb}{0.200000,0.800000,0.200000}%
\pgfsetstrokecolor{currentstroke}%
\pgfsetdash{}{0pt}%
\pgfpathmoveto{\pgfqpoint{4.233763in}{5.917394in}}%
\pgfpathcurveto{\pgfqpoint{4.239587in}{5.917394in}}{\pgfqpoint{4.245174in}{5.919708in}}{\pgfqpoint{4.249292in}{5.923826in}}%
\pgfpathcurveto{\pgfqpoint{4.253410in}{5.927944in}}{\pgfqpoint{4.255724in}{5.933530in}}{\pgfqpoint{4.255724in}{5.939354in}}%
\pgfpathcurveto{\pgfqpoint{4.255724in}{5.945178in}}{\pgfqpoint{4.253410in}{5.950764in}}{\pgfqpoint{4.249292in}{5.954882in}}%
\pgfpathcurveto{\pgfqpoint{4.245174in}{5.959000in}}{\pgfqpoint{4.239587in}{5.961314in}}{\pgfqpoint{4.233763in}{5.961314in}}%
\pgfpathcurveto{\pgfqpoint{4.227940in}{5.961314in}}{\pgfqpoint{4.222353in}{5.959000in}}{\pgfqpoint{4.218235in}{5.954882in}}%
\pgfpathcurveto{\pgfqpoint{4.214117in}{5.950764in}}{\pgfqpoint{4.211803in}{5.945178in}}{\pgfqpoint{4.211803in}{5.939354in}}%
\pgfpathcurveto{\pgfqpoint{4.211803in}{5.933530in}}{\pgfqpoint{4.214117in}{5.927944in}}{\pgfqpoint{4.218235in}{5.923826in}}%
\pgfpathcurveto{\pgfqpoint{4.222353in}{5.919708in}}{\pgfqpoint{4.227940in}{5.917394in}}{\pgfqpoint{4.233763in}{5.917394in}}%
\pgfpathlineto{\pgfqpoint{4.233763in}{5.917394in}}%
\pgfpathclose%
\pgfusepath{stroke,fill}%
\end{pgfscope}%
\begin{pgfscope}%
\pgfpathrectangle{\pgfqpoint{1.000000in}{0.979904in}}{\pgfqpoint{6.200000in}{5.960192in}}%
\pgfusepath{clip}%
\pgfsetbuttcap%
\pgfsetroundjoin%
\definecolor{currentfill}{rgb}{0.200000,0.800000,0.200000}%
\pgfsetfillcolor{currentfill}%
\pgfsetlinewidth{1.003750pt}%
\definecolor{currentstroke}{rgb}{0.200000,0.800000,0.200000}%
\pgfsetstrokecolor{currentstroke}%
\pgfsetdash{}{0pt}%
\pgfpathmoveto{\pgfqpoint{4.093466in}{5.950042in}}%
\pgfpathcurveto{\pgfqpoint{4.099290in}{5.950042in}}{\pgfqpoint{4.104877in}{5.952356in}}{\pgfqpoint{4.108995in}{5.956474in}}%
\pgfpathcurveto{\pgfqpoint{4.113113in}{5.960592in}}{\pgfqpoint{4.115427in}{5.966178in}}{\pgfqpoint{4.115427in}{5.972002in}}%
\pgfpathcurveto{\pgfqpoint{4.115427in}{5.977826in}}{\pgfqpoint{4.113113in}{5.983412in}}{\pgfqpoint{4.108995in}{5.987530in}}%
\pgfpathcurveto{\pgfqpoint{4.104877in}{5.991648in}}{\pgfqpoint{4.099290in}{5.993962in}}{\pgfqpoint{4.093466in}{5.993962in}}%
\pgfpathcurveto{\pgfqpoint{4.087643in}{5.993962in}}{\pgfqpoint{4.082056in}{5.991648in}}{\pgfqpoint{4.077938in}{5.987530in}}%
\pgfpathcurveto{\pgfqpoint{4.073820in}{5.983412in}}{\pgfqpoint{4.071506in}{5.977826in}}{\pgfqpoint{4.071506in}{5.972002in}}%
\pgfpathcurveto{\pgfqpoint{4.071506in}{5.966178in}}{\pgfqpoint{4.073820in}{5.960592in}}{\pgfqpoint{4.077938in}{5.956474in}}%
\pgfpathcurveto{\pgfqpoint{4.082056in}{5.952356in}}{\pgfqpoint{4.087643in}{5.950042in}}{\pgfqpoint{4.093466in}{5.950042in}}%
\pgfpathlineto{\pgfqpoint{4.093466in}{5.950042in}}%
\pgfpathclose%
\pgfusepath{stroke,fill}%
\end{pgfscope}%
\begin{pgfscope}%
\pgfpathrectangle{\pgfqpoint{1.000000in}{0.979904in}}{\pgfqpoint{6.200000in}{5.960192in}}%
\pgfusepath{clip}%
\pgfsetbuttcap%
\pgfsetroundjoin%
\definecolor{currentfill}{rgb}{0.200000,0.800000,0.200000}%
\pgfsetfillcolor{currentfill}%
\pgfsetlinewidth{1.003750pt}%
\definecolor{currentstroke}{rgb}{0.200000,0.800000,0.200000}%
\pgfsetstrokecolor{currentstroke}%
\pgfsetdash{}{0pt}%
\pgfpathmoveto{\pgfqpoint{3.996712in}{5.894705in}}%
\pgfpathcurveto{\pgfqpoint{4.002535in}{5.894705in}}{\pgfqpoint{4.008122in}{5.897019in}}{\pgfqpoint{4.012240in}{5.901137in}}%
\pgfpathcurveto{\pgfqpoint{4.016358in}{5.905255in}}{\pgfqpoint{4.018672in}{5.910841in}}{\pgfqpoint{4.018672in}{5.916665in}}%
\pgfpathcurveto{\pgfqpoint{4.018672in}{5.922489in}}{\pgfqpoint{4.016358in}{5.928075in}}{\pgfqpoint{4.012240in}{5.932193in}}%
\pgfpathcurveto{\pgfqpoint{4.008122in}{5.936311in}}{\pgfqpoint{4.002535in}{5.938625in}}{\pgfqpoint{3.996712in}{5.938625in}}%
\pgfpathcurveto{\pgfqpoint{3.990888in}{5.938625in}}{\pgfqpoint{3.985301in}{5.936311in}}{\pgfqpoint{3.981183in}{5.932193in}}%
\pgfpathcurveto{\pgfqpoint{3.977065in}{5.928075in}}{\pgfqpoint{3.974751in}{5.922489in}}{\pgfqpoint{3.974751in}{5.916665in}}%
\pgfpathcurveto{\pgfqpoint{3.974751in}{5.910841in}}{\pgfqpoint{3.977065in}{5.905255in}}{\pgfqpoint{3.981183in}{5.901137in}}%
\pgfpathcurveto{\pgfqpoint{3.985301in}{5.897019in}}{\pgfqpoint{3.990888in}{5.894705in}}{\pgfqpoint{3.996712in}{5.894705in}}%
\pgfpathlineto{\pgfqpoint{3.996712in}{5.894705in}}%
\pgfpathclose%
\pgfusepath{stroke,fill}%
\end{pgfscope}%
\begin{pgfscope}%
\pgfpathrectangle{\pgfqpoint{1.000000in}{0.979904in}}{\pgfqpoint{6.200000in}{5.960192in}}%
\pgfusepath{clip}%
\pgfsetbuttcap%
\pgfsetroundjoin%
\definecolor{currentfill}{rgb}{0.200000,0.800000,0.200000}%
\pgfsetfillcolor{currentfill}%
\pgfsetlinewidth{1.003750pt}%
\definecolor{currentstroke}{rgb}{0.200000,0.800000,0.200000}%
\pgfsetstrokecolor{currentstroke}%
\pgfsetdash{}{0pt}%
\pgfpathmoveto{\pgfqpoint{3.889102in}{5.852055in}}%
\pgfpathcurveto{\pgfqpoint{3.894926in}{5.852055in}}{\pgfqpoint{3.900512in}{5.854369in}}{\pgfqpoint{3.904630in}{5.858487in}}%
\pgfpathcurveto{\pgfqpoint{3.908748in}{5.862605in}}{\pgfqpoint{3.911062in}{5.868192in}}{\pgfqpoint{3.911062in}{5.874015in}}%
\pgfpathcurveto{\pgfqpoint{3.911062in}{5.879839in}}{\pgfqpoint{3.908748in}{5.885426in}}{\pgfqpoint{3.904630in}{5.889544in}}%
\pgfpathcurveto{\pgfqpoint{3.900512in}{5.893662in}}{\pgfqpoint{3.894926in}{5.895976in}}{\pgfqpoint{3.889102in}{5.895976in}}%
\pgfpathcurveto{\pgfqpoint{3.883278in}{5.895976in}}{\pgfqpoint{3.877692in}{5.893662in}}{\pgfqpoint{3.873574in}{5.889544in}}%
\pgfpathcurveto{\pgfqpoint{3.869455in}{5.885426in}}{\pgfqpoint{3.867142in}{5.879839in}}{\pgfqpoint{3.867142in}{5.874015in}}%
\pgfpathcurveto{\pgfqpoint{3.867142in}{5.868192in}}{\pgfqpoint{3.869455in}{5.862605in}}{\pgfqpoint{3.873574in}{5.858487in}}%
\pgfpathcurveto{\pgfqpoint{3.877692in}{5.854369in}}{\pgfqpoint{3.883278in}{5.852055in}}{\pgfqpoint{3.889102in}{5.852055in}}%
\pgfpathlineto{\pgfqpoint{3.889102in}{5.852055in}}%
\pgfpathclose%
\pgfusepath{stroke,fill}%
\end{pgfscope}%
\begin{pgfscope}%
\pgfpathrectangle{\pgfqpoint{1.000000in}{0.979904in}}{\pgfqpoint{6.200000in}{5.960192in}}%
\pgfusepath{clip}%
\pgfsetbuttcap%
\pgfsetroundjoin%
\definecolor{currentfill}{rgb}{0.200000,0.800000,0.200000}%
\pgfsetfillcolor{currentfill}%
\pgfsetlinewidth{1.003750pt}%
\definecolor{currentstroke}{rgb}{0.200000,0.800000,0.200000}%
\pgfsetstrokecolor{currentstroke}%
\pgfsetdash{}{0pt}%
\pgfpathmoveto{\pgfqpoint{3.871326in}{5.699682in}}%
\pgfpathcurveto{\pgfqpoint{3.877150in}{5.699682in}}{\pgfqpoint{3.882736in}{5.701996in}}{\pgfqpoint{3.886855in}{5.706114in}}%
\pgfpathcurveto{\pgfqpoint{3.890973in}{5.710232in}}{\pgfqpoint{3.893287in}{5.715818in}}{\pgfqpoint{3.893287in}{5.721642in}}%
\pgfpathcurveto{\pgfqpoint{3.893287in}{5.727466in}}{\pgfqpoint{3.890973in}{5.733052in}}{\pgfqpoint{3.886855in}{5.737171in}}%
\pgfpathcurveto{\pgfqpoint{3.882736in}{5.741289in}}{\pgfqpoint{3.877150in}{5.743603in}}{\pgfqpoint{3.871326in}{5.743603in}}%
\pgfpathcurveto{\pgfqpoint{3.865502in}{5.743603in}}{\pgfqpoint{3.859916in}{5.741289in}}{\pgfqpoint{3.855798in}{5.737171in}}%
\pgfpathcurveto{\pgfqpoint{3.851680in}{5.733052in}}{\pgfqpoint{3.849366in}{5.727466in}}{\pgfqpoint{3.849366in}{5.721642in}}%
\pgfpathcurveto{\pgfqpoint{3.849366in}{5.715818in}}{\pgfqpoint{3.851680in}{5.710232in}}{\pgfqpoint{3.855798in}{5.706114in}}%
\pgfpathcurveto{\pgfqpoint{3.859916in}{5.701996in}}{\pgfqpoint{3.865502in}{5.699682in}}{\pgfqpoint{3.871326in}{5.699682in}}%
\pgfpathlineto{\pgfqpoint{3.871326in}{5.699682in}}%
\pgfpathclose%
\pgfusepath{stroke,fill}%
\end{pgfscope}%
\begin{pgfscope}%
\pgfpathrectangle{\pgfqpoint{1.000000in}{0.979904in}}{\pgfqpoint{6.200000in}{5.960192in}}%
\pgfusepath{clip}%
\pgfsetbuttcap%
\pgfsetroundjoin%
\definecolor{currentfill}{rgb}{0.200000,0.800000,0.200000}%
\pgfsetfillcolor{currentfill}%
\pgfsetlinewidth{1.003750pt}%
\definecolor{currentstroke}{rgb}{0.200000,0.800000,0.200000}%
\pgfsetstrokecolor{currentstroke}%
\pgfsetdash{}{0pt}%
\pgfpathmoveto{\pgfqpoint{3.796217in}{5.624279in}}%
\pgfpathcurveto{\pgfqpoint{3.802041in}{5.624279in}}{\pgfqpoint{3.807627in}{5.626592in}}{\pgfqpoint{3.811745in}{5.630711in}}%
\pgfpathcurveto{\pgfqpoint{3.815863in}{5.634829in}}{\pgfqpoint{3.818177in}{5.640415in}}{\pgfqpoint{3.818177in}{5.646239in}}%
\pgfpathcurveto{\pgfqpoint{3.818177in}{5.652063in}}{\pgfqpoint{3.815863in}{5.657649in}}{\pgfqpoint{3.811745in}{5.661767in}}%
\pgfpathcurveto{\pgfqpoint{3.807627in}{5.665885in}}{\pgfqpoint{3.802041in}{5.668199in}}{\pgfqpoint{3.796217in}{5.668199in}}%
\pgfpathcurveto{\pgfqpoint{3.790393in}{5.668199in}}{\pgfqpoint{3.784807in}{5.665885in}}{\pgfqpoint{3.780689in}{5.661767in}}%
\pgfpathcurveto{\pgfqpoint{3.776571in}{5.657649in}}{\pgfqpoint{3.774257in}{5.652063in}}{\pgfqpoint{3.774257in}{5.646239in}}%
\pgfpathcurveto{\pgfqpoint{3.774257in}{5.640415in}}{\pgfqpoint{3.776571in}{5.634829in}}{\pgfqpoint{3.780689in}{5.630711in}}%
\pgfpathcurveto{\pgfqpoint{3.784807in}{5.626592in}}{\pgfqpoint{3.790393in}{5.624279in}}{\pgfqpoint{3.796217in}{5.624279in}}%
\pgfpathlineto{\pgfqpoint{3.796217in}{5.624279in}}%
\pgfpathclose%
\pgfusepath{stroke,fill}%
\end{pgfscope}%
\begin{pgfscope}%
\pgfpathrectangle{\pgfqpoint{1.000000in}{0.979904in}}{\pgfqpoint{6.200000in}{5.960192in}}%
\pgfusepath{clip}%
\pgfsetbuttcap%
\pgfsetroundjoin%
\definecolor{currentfill}{rgb}{0.200000,0.800000,0.200000}%
\pgfsetfillcolor{currentfill}%
\pgfsetlinewidth{1.003750pt}%
\definecolor{currentstroke}{rgb}{0.200000,0.800000,0.200000}%
\pgfsetstrokecolor{currentstroke}%
\pgfsetdash{}{0pt}%
\pgfpathmoveto{\pgfqpoint{3.650082in}{5.612598in}}%
\pgfpathcurveto{\pgfqpoint{3.655906in}{5.612598in}}{\pgfqpoint{3.661493in}{5.614912in}}{\pgfqpoint{3.665611in}{5.619030in}}%
\pgfpathcurveto{\pgfqpoint{3.669729in}{5.623149in}}{\pgfqpoint{3.672043in}{5.628735in}}{\pgfqpoint{3.672043in}{5.634559in}}%
\pgfpathcurveto{\pgfqpoint{3.672043in}{5.640383in}}{\pgfqpoint{3.669729in}{5.645969in}}{\pgfqpoint{3.665611in}{5.650087in}}%
\pgfpathcurveto{\pgfqpoint{3.661493in}{5.654205in}}{\pgfqpoint{3.655906in}{5.656519in}}{\pgfqpoint{3.650082in}{5.656519in}}%
\pgfpathcurveto{\pgfqpoint{3.644259in}{5.656519in}}{\pgfqpoint{3.638672in}{5.654205in}}{\pgfqpoint{3.634554in}{5.650087in}}%
\pgfpathcurveto{\pgfqpoint{3.630436in}{5.645969in}}{\pgfqpoint{3.628122in}{5.640383in}}{\pgfqpoint{3.628122in}{5.634559in}}%
\pgfpathcurveto{\pgfqpoint{3.628122in}{5.628735in}}{\pgfqpoint{3.630436in}{5.623149in}}{\pgfqpoint{3.634554in}{5.619030in}}%
\pgfpathcurveto{\pgfqpoint{3.638672in}{5.614912in}}{\pgfqpoint{3.644259in}{5.612598in}}{\pgfqpoint{3.650082in}{5.612598in}}%
\pgfpathlineto{\pgfqpoint{3.650082in}{5.612598in}}%
\pgfpathclose%
\pgfusepath{stroke,fill}%
\end{pgfscope}%
\begin{pgfscope}%
\pgfpathrectangle{\pgfqpoint{1.000000in}{0.979904in}}{\pgfqpoint{6.200000in}{5.960192in}}%
\pgfusepath{clip}%
\pgfsetbuttcap%
\pgfsetroundjoin%
\definecolor{currentfill}{rgb}{0.200000,0.800000,0.200000}%
\pgfsetfillcolor{currentfill}%
\pgfsetlinewidth{1.003750pt}%
\definecolor{currentstroke}{rgb}{0.200000,0.800000,0.200000}%
\pgfsetstrokecolor{currentstroke}%
\pgfsetdash{}{0pt}%
\pgfpathmoveto{\pgfqpoint{3.587216in}{5.519118in}}%
\pgfpathcurveto{\pgfqpoint{3.593040in}{5.519118in}}{\pgfqpoint{3.598626in}{5.521432in}}{\pgfqpoint{3.602744in}{5.525550in}}%
\pgfpathcurveto{\pgfqpoint{3.606862in}{5.529668in}}{\pgfqpoint{3.609176in}{5.535254in}}{\pgfqpoint{3.609176in}{5.541078in}}%
\pgfpathcurveto{\pgfqpoint{3.609176in}{5.546902in}}{\pgfqpoint{3.606862in}{5.552488in}}{\pgfqpoint{3.602744in}{5.556606in}}%
\pgfpathcurveto{\pgfqpoint{3.598626in}{5.560724in}}{\pgfqpoint{3.593040in}{5.563038in}}{\pgfqpoint{3.587216in}{5.563038in}}%
\pgfpathcurveto{\pgfqpoint{3.581392in}{5.563038in}}{\pgfqpoint{3.575806in}{5.560724in}}{\pgfqpoint{3.571687in}{5.556606in}}%
\pgfpathcurveto{\pgfqpoint{3.567569in}{5.552488in}}{\pgfqpoint{3.565255in}{5.546902in}}{\pgfqpoint{3.565255in}{5.541078in}}%
\pgfpathcurveto{\pgfqpoint{3.565255in}{5.535254in}}{\pgfqpoint{3.567569in}{5.529668in}}{\pgfqpoint{3.571687in}{5.525550in}}%
\pgfpathcurveto{\pgfqpoint{3.575806in}{5.521432in}}{\pgfqpoint{3.581392in}{5.519118in}}{\pgfqpoint{3.587216in}{5.519118in}}%
\pgfpathlineto{\pgfqpoint{3.587216in}{5.519118in}}%
\pgfpathclose%
\pgfusepath{stroke,fill}%
\end{pgfscope}%
\begin{pgfscope}%
\pgfpathrectangle{\pgfqpoint{1.000000in}{0.979904in}}{\pgfqpoint{6.200000in}{5.960192in}}%
\pgfusepath{clip}%
\pgfsetbuttcap%
\pgfsetroundjoin%
\definecolor{currentfill}{rgb}{0.200000,0.800000,0.200000}%
\pgfsetfillcolor{currentfill}%
\pgfsetlinewidth{1.003750pt}%
\definecolor{currentstroke}{rgb}{0.200000,0.800000,0.200000}%
\pgfsetstrokecolor{currentstroke}%
\pgfsetdash{}{0pt}%
\pgfpathmoveto{\pgfqpoint{3.523054in}{5.427817in}}%
\pgfpathcurveto{\pgfqpoint{3.528878in}{5.427817in}}{\pgfqpoint{3.534464in}{5.430131in}}{\pgfqpoint{3.538582in}{5.434249in}}%
\pgfpathcurveto{\pgfqpoint{3.542701in}{5.438367in}}{\pgfqpoint{3.545014in}{5.443954in}}{\pgfqpoint{3.545014in}{5.449778in}}%
\pgfpathcurveto{\pgfqpoint{3.545014in}{5.455601in}}{\pgfqpoint{3.542701in}{5.461188in}}{\pgfqpoint{3.538582in}{5.465306in}}%
\pgfpathcurveto{\pgfqpoint{3.534464in}{5.469424in}}{\pgfqpoint{3.528878in}{5.471738in}}{\pgfqpoint{3.523054in}{5.471738in}}%
\pgfpathcurveto{\pgfqpoint{3.517230in}{5.471738in}}{\pgfqpoint{3.511644in}{5.469424in}}{\pgfqpoint{3.507526in}{5.465306in}}%
\pgfpathcurveto{\pgfqpoint{3.503408in}{5.461188in}}{\pgfqpoint{3.501094in}{5.455601in}}{\pgfqpoint{3.501094in}{5.449778in}}%
\pgfpathcurveto{\pgfqpoint{3.501094in}{5.443954in}}{\pgfqpoint{3.503408in}{5.438367in}}{\pgfqpoint{3.507526in}{5.434249in}}%
\pgfpathcurveto{\pgfqpoint{3.511644in}{5.430131in}}{\pgfqpoint{3.517230in}{5.427817in}}{\pgfqpoint{3.523054in}{5.427817in}}%
\pgfpathlineto{\pgfqpoint{3.523054in}{5.427817in}}%
\pgfpathclose%
\pgfusepath{stroke,fill}%
\end{pgfscope}%
\begin{pgfscope}%
\pgfpathrectangle{\pgfqpoint{1.000000in}{0.979904in}}{\pgfqpoint{6.200000in}{5.960192in}}%
\pgfusepath{clip}%
\pgfsetbuttcap%
\pgfsetroundjoin%
\definecolor{currentfill}{rgb}{0.200000,0.800000,0.200000}%
\pgfsetfillcolor{currentfill}%
\pgfsetlinewidth{1.003750pt}%
\definecolor{currentstroke}{rgb}{0.200000,0.800000,0.200000}%
\pgfsetstrokecolor{currentstroke}%
\pgfsetdash{}{0pt}%
\pgfpathmoveto{\pgfqpoint{3.458413in}{5.336645in}}%
\pgfpathcurveto{\pgfqpoint{3.464237in}{5.336645in}}{\pgfqpoint{3.469823in}{5.338959in}}{\pgfqpoint{3.473941in}{5.343077in}}%
\pgfpathcurveto{\pgfqpoint{3.478059in}{5.347195in}}{\pgfqpoint{3.480373in}{5.352781in}}{\pgfqpoint{3.480373in}{5.358605in}}%
\pgfpathcurveto{\pgfqpoint{3.480373in}{5.364429in}}{\pgfqpoint{3.478059in}{5.370016in}}{\pgfqpoint{3.473941in}{5.374134in}}%
\pgfpathcurveto{\pgfqpoint{3.469823in}{5.378252in}}{\pgfqpoint{3.464237in}{5.380566in}}{\pgfqpoint{3.458413in}{5.380566in}}%
\pgfpathcurveto{\pgfqpoint{3.452589in}{5.380566in}}{\pgfqpoint{3.447003in}{5.378252in}}{\pgfqpoint{3.442884in}{5.374134in}}%
\pgfpathcurveto{\pgfqpoint{3.438766in}{5.370016in}}{\pgfqpoint{3.436452in}{5.364429in}}{\pgfqpoint{3.436452in}{5.358605in}}%
\pgfpathcurveto{\pgfqpoint{3.436452in}{5.352781in}}{\pgfqpoint{3.438766in}{5.347195in}}{\pgfqpoint{3.442884in}{5.343077in}}%
\pgfpathcurveto{\pgfqpoint{3.447003in}{5.338959in}}{\pgfqpoint{3.452589in}{5.336645in}}{\pgfqpoint{3.458413in}{5.336645in}}%
\pgfpathlineto{\pgfqpoint{3.458413in}{5.336645in}}%
\pgfpathclose%
\pgfusepath{stroke,fill}%
\end{pgfscope}%
\begin{pgfscope}%
\pgfpathrectangle{\pgfqpoint{1.000000in}{0.979904in}}{\pgfqpoint{6.200000in}{5.960192in}}%
\pgfusepath{clip}%
\pgfsetbuttcap%
\pgfsetroundjoin%
\definecolor{currentfill}{rgb}{0.200000,0.800000,0.200000}%
\pgfsetfillcolor{currentfill}%
\pgfsetlinewidth{1.003750pt}%
\definecolor{currentstroke}{rgb}{0.200000,0.800000,0.200000}%
\pgfsetstrokecolor{currentstroke}%
\pgfsetdash{}{0pt}%
\pgfpathmoveto{\pgfqpoint{3.452459in}{5.213986in}}%
\pgfpathcurveto{\pgfqpoint{3.458283in}{5.213986in}}{\pgfqpoint{3.463869in}{5.216300in}}{\pgfqpoint{3.467987in}{5.220418in}}%
\pgfpathcurveto{\pgfqpoint{3.472105in}{5.224536in}}{\pgfqpoint{3.474419in}{5.230122in}}{\pgfqpoint{3.474419in}{5.235946in}}%
\pgfpathcurveto{\pgfqpoint{3.474419in}{5.241770in}}{\pgfqpoint{3.472105in}{5.247356in}}{\pgfqpoint{3.467987in}{5.251475in}}%
\pgfpathcurveto{\pgfqpoint{3.463869in}{5.255593in}}{\pgfqpoint{3.458283in}{5.257907in}}{\pgfqpoint{3.452459in}{5.257907in}}%
\pgfpathcurveto{\pgfqpoint{3.446635in}{5.257907in}}{\pgfqpoint{3.441049in}{5.255593in}}{\pgfqpoint{3.436931in}{5.251475in}}%
\pgfpathcurveto{\pgfqpoint{3.432813in}{5.247356in}}{\pgfqpoint{3.430499in}{5.241770in}}{\pgfqpoint{3.430499in}{5.235946in}}%
\pgfpathcurveto{\pgfqpoint{3.430499in}{5.230122in}}{\pgfqpoint{3.432813in}{5.224536in}}{\pgfqpoint{3.436931in}{5.220418in}}%
\pgfpathcurveto{\pgfqpoint{3.441049in}{5.216300in}}{\pgfqpoint{3.446635in}{5.213986in}}{\pgfqpoint{3.452459in}{5.213986in}}%
\pgfpathlineto{\pgfqpoint{3.452459in}{5.213986in}}%
\pgfpathclose%
\pgfusepath{stroke,fill}%
\end{pgfscope}%
\begin{pgfscope}%
\pgfpathrectangle{\pgfqpoint{1.000000in}{0.979904in}}{\pgfqpoint{6.200000in}{5.960192in}}%
\pgfusepath{clip}%
\pgfsetbuttcap%
\pgfsetroundjoin%
\definecolor{currentfill}{rgb}{0.200000,0.800000,0.200000}%
\pgfsetfillcolor{currentfill}%
\pgfsetlinewidth{1.003750pt}%
\definecolor{currentstroke}{rgb}{0.200000,0.800000,0.200000}%
\pgfsetstrokecolor{currentstroke}%
\pgfsetdash{}{0pt}%
\pgfpathmoveto{\pgfqpoint{3.361548in}{5.135558in}}%
\pgfpathcurveto{\pgfqpoint{3.367372in}{5.135558in}}{\pgfqpoint{3.372958in}{5.137872in}}{\pgfqpoint{3.377076in}{5.141990in}}%
\pgfpathcurveto{\pgfqpoint{3.381194in}{5.146108in}}{\pgfqpoint{3.383508in}{5.151694in}}{\pgfqpoint{3.383508in}{5.157518in}}%
\pgfpathcurveto{\pgfqpoint{3.383508in}{5.163342in}}{\pgfqpoint{3.381194in}{5.168928in}}{\pgfqpoint{3.377076in}{5.173046in}}%
\pgfpathcurveto{\pgfqpoint{3.372958in}{5.177164in}}{\pgfqpoint{3.367372in}{5.179478in}}{\pgfqpoint{3.361548in}{5.179478in}}%
\pgfpathcurveto{\pgfqpoint{3.355724in}{5.179478in}}{\pgfqpoint{3.350138in}{5.177164in}}{\pgfqpoint{3.346019in}{5.173046in}}%
\pgfpathcurveto{\pgfqpoint{3.341901in}{5.168928in}}{\pgfqpoint{3.339587in}{5.163342in}}{\pgfqpoint{3.339587in}{5.157518in}}%
\pgfpathcurveto{\pgfqpoint{3.339587in}{5.151694in}}{\pgfqpoint{3.341901in}{5.146108in}}{\pgfqpoint{3.346019in}{5.141990in}}%
\pgfpathcurveto{\pgfqpoint{3.350138in}{5.137872in}}{\pgfqpoint{3.355724in}{5.135558in}}{\pgfqpoint{3.361548in}{5.135558in}}%
\pgfpathlineto{\pgfqpoint{3.361548in}{5.135558in}}%
\pgfpathclose%
\pgfusepath{stroke,fill}%
\end{pgfscope}%
\begin{pgfscope}%
\pgfpathrectangle{\pgfqpoint{1.000000in}{0.979904in}}{\pgfqpoint{6.200000in}{5.960192in}}%
\pgfusepath{clip}%
\pgfsetbuttcap%
\pgfsetroundjoin%
\definecolor{currentfill}{rgb}{0.200000,0.800000,0.200000}%
\pgfsetfillcolor{currentfill}%
\pgfsetlinewidth{1.003750pt}%
\definecolor{currentstroke}{rgb}{0.200000,0.800000,0.200000}%
\pgfsetstrokecolor{currentstroke}%
\pgfsetdash{}{0pt}%
\pgfpathmoveto{\pgfqpoint{3.317103in}{5.033219in}}%
\pgfpathcurveto{\pgfqpoint{3.322927in}{5.033219in}}{\pgfqpoint{3.328514in}{5.035533in}}{\pgfqpoint{3.332632in}{5.039651in}}%
\pgfpathcurveto{\pgfqpoint{3.336750in}{5.043769in}}{\pgfqpoint{3.339064in}{5.049356in}}{\pgfqpoint{3.339064in}{5.055179in}}%
\pgfpathcurveto{\pgfqpoint{3.339064in}{5.061003in}}{\pgfqpoint{3.336750in}{5.066590in}}{\pgfqpoint{3.332632in}{5.070708in}}%
\pgfpathcurveto{\pgfqpoint{3.328514in}{5.074826in}}{\pgfqpoint{3.322927in}{5.077140in}}{\pgfqpoint{3.317103in}{5.077140in}}%
\pgfpathcurveto{\pgfqpoint{3.311280in}{5.077140in}}{\pgfqpoint{3.305693in}{5.074826in}}{\pgfqpoint{3.301575in}{5.070708in}}%
\pgfpathcurveto{\pgfqpoint{3.297457in}{5.066590in}}{\pgfqpoint{3.295143in}{5.061003in}}{\pgfqpoint{3.295143in}{5.055179in}}%
\pgfpathcurveto{\pgfqpoint{3.295143in}{5.049356in}}{\pgfqpoint{3.297457in}{5.043769in}}{\pgfqpoint{3.301575in}{5.039651in}}%
\pgfpathcurveto{\pgfqpoint{3.305693in}{5.035533in}}{\pgfqpoint{3.311280in}{5.033219in}}{\pgfqpoint{3.317103in}{5.033219in}}%
\pgfpathlineto{\pgfqpoint{3.317103in}{5.033219in}}%
\pgfpathclose%
\pgfusepath{stroke,fill}%
\end{pgfscope}%
\begin{pgfscope}%
\pgfpathrectangle{\pgfqpoint{1.000000in}{0.979904in}}{\pgfqpoint{6.200000in}{5.960192in}}%
\pgfusepath{clip}%
\pgfsetbuttcap%
\pgfsetroundjoin%
\definecolor{currentfill}{rgb}{0.200000,0.800000,0.200000}%
\pgfsetfillcolor{currentfill}%
\pgfsetlinewidth{1.003750pt}%
\definecolor{currentstroke}{rgb}{0.200000,0.800000,0.200000}%
\pgfsetstrokecolor{currentstroke}%
\pgfsetdash{}{0pt}%
\pgfpathmoveto{\pgfqpoint{3.333557in}{4.912105in}}%
\pgfpathcurveto{\pgfqpoint{3.339381in}{4.912105in}}{\pgfqpoint{3.344967in}{4.914419in}}{\pgfqpoint{3.349085in}{4.918537in}}%
\pgfpathcurveto{\pgfqpoint{3.353203in}{4.922656in}}{\pgfqpoint{3.355517in}{4.928242in}}{\pgfqpoint{3.355517in}{4.934066in}}%
\pgfpathcurveto{\pgfqpoint{3.355517in}{4.939890in}}{\pgfqpoint{3.353203in}{4.945476in}}{\pgfqpoint{3.349085in}{4.949594in}}%
\pgfpathcurveto{\pgfqpoint{3.344967in}{4.953712in}}{\pgfqpoint{3.339381in}{4.956026in}}{\pgfqpoint{3.333557in}{4.956026in}}%
\pgfpathcurveto{\pgfqpoint{3.327733in}{4.956026in}}{\pgfqpoint{3.322147in}{4.953712in}}{\pgfqpoint{3.318029in}{4.949594in}}%
\pgfpathcurveto{\pgfqpoint{3.313911in}{4.945476in}}{\pgfqpoint{3.311597in}{4.939890in}}{\pgfqpoint{3.311597in}{4.934066in}}%
\pgfpathcurveto{\pgfqpoint{3.311597in}{4.928242in}}{\pgfqpoint{3.313911in}{4.922656in}}{\pgfqpoint{3.318029in}{4.918537in}}%
\pgfpathcurveto{\pgfqpoint{3.322147in}{4.914419in}}{\pgfqpoint{3.327733in}{4.912105in}}{\pgfqpoint{3.333557in}{4.912105in}}%
\pgfpathlineto{\pgfqpoint{3.333557in}{4.912105in}}%
\pgfpathclose%
\pgfusepath{stroke,fill}%
\end{pgfscope}%
\begin{pgfscope}%
\pgfpathrectangle{\pgfqpoint{1.000000in}{0.979904in}}{\pgfqpoint{6.200000in}{5.960192in}}%
\pgfusepath{clip}%
\pgfsetbuttcap%
\pgfsetroundjoin%
\definecolor{currentfill}{rgb}{0.200000,0.200000,0.800000}%
\pgfsetfillcolor{currentfill}%
\pgfsetlinewidth{1.003750pt}%
\definecolor{currentstroke}{rgb}{0.200000,0.200000,0.800000}%
\pgfsetstrokecolor{currentstroke}%
\pgfsetdash{}{0pt}%
\pgfpathmoveto{\pgfqpoint{3.172844in}{4.837497in}}%
\pgfpathcurveto{\pgfqpoint{3.178668in}{4.837497in}}{\pgfqpoint{3.184254in}{4.839811in}}{\pgfqpoint{3.188372in}{4.843929in}}%
\pgfpathcurveto{\pgfqpoint{3.192491in}{4.848047in}}{\pgfqpoint{3.194804in}{4.853633in}}{\pgfqpoint{3.194804in}{4.859457in}}%
\pgfpathcurveto{\pgfqpoint{3.194804in}{4.865281in}}{\pgfqpoint{3.192491in}{4.870867in}}{\pgfqpoint{3.188372in}{4.874985in}}%
\pgfpathcurveto{\pgfqpoint{3.184254in}{4.879103in}}{\pgfqpoint{3.178668in}{4.881417in}}{\pgfqpoint{3.172844in}{4.881417in}}%
\pgfpathcurveto{\pgfqpoint{3.167020in}{4.881417in}}{\pgfqpoint{3.161434in}{4.879103in}}{\pgfqpoint{3.157316in}{4.874985in}}%
\pgfpathcurveto{\pgfqpoint{3.153198in}{4.870867in}}{\pgfqpoint{3.150884in}{4.865281in}}{\pgfqpoint{3.150884in}{4.859457in}}%
\pgfpathcurveto{\pgfqpoint{3.150884in}{4.853633in}}{\pgfqpoint{3.153198in}{4.848047in}}{\pgfqpoint{3.157316in}{4.843929in}}%
\pgfpathcurveto{\pgfqpoint{3.161434in}{4.839811in}}{\pgfqpoint{3.167020in}{4.837497in}}{\pgfqpoint{3.172844in}{4.837497in}}%
\pgfpathlineto{\pgfqpoint{3.172844in}{4.837497in}}%
\pgfpathclose%
\pgfusepath{stroke,fill}%
\end{pgfscope}%
\begin{pgfscope}%
\pgfpathrectangle{\pgfqpoint{1.000000in}{0.979904in}}{\pgfqpoint{6.200000in}{5.960192in}}%
\pgfusepath{clip}%
\pgfsetbuttcap%
\pgfsetroundjoin%
\definecolor{currentfill}{rgb}{0.200000,0.800000,0.200000}%
\pgfsetfillcolor{currentfill}%
\pgfsetlinewidth{1.003750pt}%
\definecolor{currentstroke}{rgb}{0.200000,0.800000,0.200000}%
\pgfsetstrokecolor{currentstroke}%
\pgfsetdash{}{0pt}%
\pgfpathmoveto{\pgfqpoint{3.223640in}{4.711018in}}%
\pgfpathcurveto{\pgfqpoint{3.229464in}{4.711018in}}{\pgfqpoint{3.235051in}{4.713331in}}{\pgfqpoint{3.239169in}{4.717450in}}%
\pgfpathcurveto{\pgfqpoint{3.243287in}{4.721568in}}{\pgfqpoint{3.245601in}{4.727154in}}{\pgfqpoint{3.245601in}{4.732978in}}%
\pgfpathcurveto{\pgfqpoint{3.245601in}{4.738802in}}{\pgfqpoint{3.243287in}{4.744388in}}{\pgfqpoint{3.239169in}{4.748506in}}%
\pgfpathcurveto{\pgfqpoint{3.235051in}{4.752624in}}{\pgfqpoint{3.229464in}{4.754938in}}{\pgfqpoint{3.223640in}{4.754938in}}%
\pgfpathcurveto{\pgfqpoint{3.217816in}{4.754938in}}{\pgfqpoint{3.212230in}{4.752624in}}{\pgfqpoint{3.208112in}{4.748506in}}%
\pgfpathcurveto{\pgfqpoint{3.203994in}{4.744388in}}{\pgfqpoint{3.201680in}{4.738802in}}{\pgfqpoint{3.201680in}{4.732978in}}%
\pgfpathcurveto{\pgfqpoint{3.201680in}{4.727154in}}{\pgfqpoint{3.203994in}{4.721568in}}{\pgfqpoint{3.208112in}{4.717450in}}%
\pgfpathcurveto{\pgfqpoint{3.212230in}{4.713331in}}{\pgfqpoint{3.217816in}{4.711018in}}{\pgfqpoint{3.223640in}{4.711018in}}%
\pgfpathlineto{\pgfqpoint{3.223640in}{4.711018in}}%
\pgfpathclose%
\pgfusepath{stroke,fill}%
\end{pgfscope}%
\begin{pgfscope}%
\pgfpathrectangle{\pgfqpoint{1.000000in}{0.979904in}}{\pgfqpoint{6.200000in}{5.960192in}}%
\pgfusepath{clip}%
\pgfsetbuttcap%
\pgfsetroundjoin%
\definecolor{currentfill}{rgb}{0.200000,0.800000,0.200000}%
\pgfsetfillcolor{currentfill}%
\pgfsetlinewidth{1.003750pt}%
\definecolor{currentstroke}{rgb}{0.200000,0.800000,0.200000}%
\pgfsetstrokecolor{currentstroke}%
\pgfsetdash{}{0pt}%
\pgfpathmoveto{\pgfqpoint{3.297379in}{4.591285in}}%
\pgfpathcurveto{\pgfqpoint{3.303203in}{4.591285in}}{\pgfqpoint{3.308789in}{4.593599in}}{\pgfqpoint{3.312907in}{4.597717in}}%
\pgfpathcurveto{\pgfqpoint{3.317025in}{4.601836in}}{\pgfqpoint{3.319339in}{4.607422in}}{\pgfqpoint{3.319339in}{4.613246in}}%
\pgfpathcurveto{\pgfqpoint{3.319339in}{4.619070in}}{\pgfqpoint{3.317025in}{4.624656in}}{\pgfqpoint{3.312907in}{4.628774in}}%
\pgfpathcurveto{\pgfqpoint{3.308789in}{4.632892in}}{\pgfqpoint{3.303203in}{4.635206in}}{\pgfqpoint{3.297379in}{4.635206in}}%
\pgfpathcurveto{\pgfqpoint{3.291555in}{4.635206in}}{\pgfqpoint{3.285968in}{4.632892in}}{\pgfqpoint{3.281850in}{4.628774in}}%
\pgfpathcurveto{\pgfqpoint{3.277732in}{4.624656in}}{\pgfqpoint{3.275418in}{4.619070in}}{\pgfqpoint{3.275418in}{4.613246in}}%
\pgfpathcurveto{\pgfqpoint{3.275418in}{4.607422in}}{\pgfqpoint{3.277732in}{4.601836in}}{\pgfqpoint{3.281850in}{4.597717in}}%
\pgfpathcurveto{\pgfqpoint{3.285968in}{4.593599in}}{\pgfqpoint{3.291555in}{4.591285in}}{\pgfqpoint{3.297379in}{4.591285in}}%
\pgfpathlineto{\pgfqpoint{3.297379in}{4.591285in}}%
\pgfpathclose%
\pgfusepath{stroke,fill}%
\end{pgfscope}%
\begin{pgfscope}%
\pgfpathrectangle{\pgfqpoint{1.000000in}{0.979904in}}{\pgfqpoint{6.200000in}{5.960192in}}%
\pgfusepath{clip}%
\pgfsetbuttcap%
\pgfsetroundjoin%
\definecolor{currentfill}{rgb}{0.200000,0.800000,0.200000}%
\pgfsetfillcolor{currentfill}%
\pgfsetlinewidth{1.003750pt}%
\definecolor{currentstroke}{rgb}{0.200000,0.800000,0.200000}%
\pgfsetstrokecolor{currentstroke}%
\pgfsetdash{}{0pt}%
\pgfpathmoveto{\pgfqpoint{3.192624in}{4.487980in}}%
\pgfpathcurveto{\pgfqpoint{3.198448in}{4.487980in}}{\pgfqpoint{3.204034in}{4.490294in}}{\pgfqpoint{3.208152in}{4.494412in}}%
\pgfpathcurveto{\pgfqpoint{3.212270in}{4.498530in}}{\pgfqpoint{3.214584in}{4.504116in}}{\pgfqpoint{3.214584in}{4.509940in}}%
\pgfpathcurveto{\pgfqpoint{3.214584in}{4.515764in}}{\pgfqpoint{3.212270in}{4.521350in}}{\pgfqpoint{3.208152in}{4.525468in}}%
\pgfpathcurveto{\pgfqpoint{3.204034in}{4.529586in}}{\pgfqpoint{3.198448in}{4.531900in}}{\pgfqpoint{3.192624in}{4.531900in}}%
\pgfpathcurveto{\pgfqpoint{3.186800in}{4.531900in}}{\pgfqpoint{3.181214in}{4.529586in}}{\pgfqpoint{3.177096in}{4.525468in}}%
\pgfpathcurveto{\pgfqpoint{3.172977in}{4.521350in}}{\pgfqpoint{3.170664in}{4.515764in}}{\pgfqpoint{3.170664in}{4.509940in}}%
\pgfpathcurveto{\pgfqpoint{3.170664in}{4.504116in}}{\pgfqpoint{3.172977in}{4.498530in}}{\pgfqpoint{3.177096in}{4.494412in}}%
\pgfpathcurveto{\pgfqpoint{3.181214in}{4.490294in}}{\pgfqpoint{3.186800in}{4.487980in}}{\pgfqpoint{3.192624in}{4.487980in}}%
\pgfpathlineto{\pgfqpoint{3.192624in}{4.487980in}}%
\pgfpathclose%
\pgfusepath{stroke,fill}%
\end{pgfscope}%
\begin{pgfscope}%
\pgfpathrectangle{\pgfqpoint{1.000000in}{0.979904in}}{\pgfqpoint{6.200000in}{5.960192in}}%
\pgfusepath{clip}%
\pgfsetbuttcap%
\pgfsetroundjoin%
\definecolor{currentfill}{rgb}{0.800000,0.200000,0.200000}%
\pgfsetfillcolor{currentfill}%
\pgfsetlinewidth{1.003750pt}%
\definecolor{currentstroke}{rgb}{0.800000,0.200000,0.200000}%
\pgfsetstrokecolor{currentstroke}%
\pgfsetdash{}{0pt}%
\pgfpathmoveto{\pgfqpoint{3.279128in}{4.377874in}}%
\pgfpathcurveto{\pgfqpoint{3.284952in}{4.377874in}}{\pgfqpoint{3.290538in}{4.380187in}}{\pgfqpoint{3.294656in}{4.384306in}}%
\pgfpathcurveto{\pgfqpoint{3.298775in}{4.388424in}}{\pgfqpoint{3.301088in}{4.394010in}}{\pgfqpoint{3.301088in}{4.399834in}}%
\pgfpathcurveto{\pgfqpoint{3.301088in}{4.405658in}}{\pgfqpoint{3.298775in}{4.411244in}}{\pgfqpoint{3.294656in}{4.415362in}}%
\pgfpathcurveto{\pgfqpoint{3.290538in}{4.419480in}}{\pgfqpoint{3.284952in}{4.421794in}}{\pgfqpoint{3.279128in}{4.421794in}}%
\pgfpathcurveto{\pgfqpoint{3.273304in}{4.421794in}}{\pgfqpoint{3.267718in}{4.419480in}}{\pgfqpoint{3.263600in}{4.415362in}}%
\pgfpathcurveto{\pgfqpoint{3.259482in}{4.411244in}}{\pgfqpoint{3.257168in}{4.405658in}}{\pgfqpoint{3.257168in}{4.399834in}}%
\pgfpathcurveto{\pgfqpoint{3.257168in}{4.394010in}}{\pgfqpoint{3.259482in}{4.388424in}}{\pgfqpoint{3.263600in}{4.384306in}}%
\pgfpathcurveto{\pgfqpoint{3.267718in}{4.380187in}}{\pgfqpoint{3.273304in}{4.377874in}}{\pgfqpoint{3.279128in}{4.377874in}}%
\pgfpathlineto{\pgfqpoint{3.279128in}{4.377874in}}%
\pgfpathclose%
\pgfusepath{stroke,fill}%
\end{pgfscope}%
\begin{pgfscope}%
\pgfpathrectangle{\pgfqpoint{1.000000in}{0.979904in}}{\pgfqpoint{6.200000in}{5.960192in}}%
\pgfusepath{clip}%
\pgfsetbuttcap%
\pgfsetroundjoin%
\definecolor{currentfill}{rgb}{0.200000,0.800000,0.200000}%
\pgfsetfillcolor{currentfill}%
\pgfsetlinewidth{1.003750pt}%
\definecolor{currentstroke}{rgb}{0.200000,0.800000,0.200000}%
\pgfsetstrokecolor{currentstroke}%
\pgfsetdash{}{0pt}%
\pgfpathmoveto{\pgfqpoint{3.192887in}{4.261844in}}%
\pgfpathcurveto{\pgfqpoint{3.198711in}{4.261844in}}{\pgfqpoint{3.204297in}{4.264158in}}{\pgfqpoint{3.208415in}{4.268276in}}%
\pgfpathcurveto{\pgfqpoint{3.212534in}{4.272394in}}{\pgfqpoint{3.214847in}{4.277980in}}{\pgfqpoint{3.214847in}{4.283804in}}%
\pgfpathcurveto{\pgfqpoint{3.214847in}{4.289628in}}{\pgfqpoint{3.212534in}{4.295214in}}{\pgfqpoint{3.208415in}{4.299333in}}%
\pgfpathcurveto{\pgfqpoint{3.204297in}{4.303451in}}{\pgfqpoint{3.198711in}{4.305765in}}{\pgfqpoint{3.192887in}{4.305765in}}%
\pgfpathcurveto{\pgfqpoint{3.187063in}{4.305765in}}{\pgfqpoint{3.181477in}{4.303451in}}{\pgfqpoint{3.177359in}{4.299333in}}%
\pgfpathcurveto{\pgfqpoint{3.173241in}{4.295214in}}{\pgfqpoint{3.170927in}{4.289628in}}{\pgfqpoint{3.170927in}{4.283804in}}%
\pgfpathcurveto{\pgfqpoint{3.170927in}{4.277980in}}{\pgfqpoint{3.173241in}{4.272394in}}{\pgfqpoint{3.177359in}{4.268276in}}%
\pgfpathcurveto{\pgfqpoint{3.181477in}{4.264158in}}{\pgfqpoint{3.187063in}{4.261844in}}{\pgfqpoint{3.192887in}{4.261844in}}%
\pgfpathlineto{\pgfqpoint{3.192887in}{4.261844in}}%
\pgfpathclose%
\pgfusepath{stroke,fill}%
\end{pgfscope}%
\begin{pgfscope}%
\pgfpathrectangle{\pgfqpoint{1.000000in}{0.979904in}}{\pgfqpoint{6.200000in}{5.960192in}}%
\pgfusepath{clip}%
\pgfsetbuttcap%
\pgfsetroundjoin%
\definecolor{currentfill}{rgb}{0.200000,0.800000,0.200000}%
\pgfsetfillcolor{currentfill}%
\pgfsetlinewidth{1.003750pt}%
\definecolor{currentstroke}{rgb}{0.200000,0.800000,0.200000}%
\pgfsetstrokecolor{currentstroke}%
\pgfsetdash{}{0pt}%
\pgfpathmoveto{\pgfqpoint{3.187766in}{4.146349in}}%
\pgfpathcurveto{\pgfqpoint{3.193589in}{4.146349in}}{\pgfqpoint{3.199176in}{4.148663in}}{\pgfqpoint{3.203294in}{4.152781in}}%
\pgfpathcurveto{\pgfqpoint{3.207412in}{4.156900in}}{\pgfqpoint{3.209726in}{4.162486in}}{\pgfqpoint{3.209726in}{4.168310in}}%
\pgfpathcurveto{\pgfqpoint{3.209726in}{4.174134in}}{\pgfqpoint{3.207412in}{4.179720in}}{\pgfqpoint{3.203294in}{4.183838in}}%
\pgfpathcurveto{\pgfqpoint{3.199176in}{4.187956in}}{\pgfqpoint{3.193589in}{4.190270in}}{\pgfqpoint{3.187766in}{4.190270in}}%
\pgfpathcurveto{\pgfqpoint{3.181942in}{4.190270in}}{\pgfqpoint{3.176355in}{4.187956in}}{\pgfqpoint{3.172237in}{4.183838in}}%
\pgfpathcurveto{\pgfqpoint{3.168119in}{4.179720in}}{\pgfqpoint{3.165805in}{4.174134in}}{\pgfqpoint{3.165805in}{4.168310in}}%
\pgfpathcurveto{\pgfqpoint{3.165805in}{4.162486in}}{\pgfqpoint{3.168119in}{4.156900in}}{\pgfqpoint{3.172237in}{4.152781in}}%
\pgfpathcurveto{\pgfqpoint{3.176355in}{4.148663in}}{\pgfqpoint{3.181942in}{4.146349in}}{\pgfqpoint{3.187766in}{4.146349in}}%
\pgfpathlineto{\pgfqpoint{3.187766in}{4.146349in}}%
\pgfpathclose%
\pgfusepath{stroke,fill}%
\end{pgfscope}%
\begin{pgfscope}%
\pgfpathrectangle{\pgfqpoint{1.000000in}{0.979904in}}{\pgfqpoint{6.200000in}{5.960192in}}%
\pgfusepath{clip}%
\pgfsetbuttcap%
\pgfsetroundjoin%
\definecolor{currentfill}{rgb}{0.200000,0.800000,0.200000}%
\pgfsetfillcolor{currentfill}%
\pgfsetlinewidth{1.003750pt}%
\definecolor{currentstroke}{rgb}{0.200000,0.800000,0.200000}%
\pgfsetstrokecolor{currentstroke}%
\pgfsetdash{}{0pt}%
\pgfpathmoveto{\pgfqpoint{3.311877in}{4.057012in}}%
\pgfpathcurveto{\pgfqpoint{3.317700in}{4.057012in}}{\pgfqpoint{3.323287in}{4.059326in}}{\pgfqpoint{3.327405in}{4.063445in}}%
\pgfpathcurveto{\pgfqpoint{3.331523in}{4.067563in}}{\pgfqpoint{3.333837in}{4.073149in}}{\pgfqpoint{3.333837in}{4.078973in}}%
\pgfpathcurveto{\pgfqpoint{3.333837in}{4.084797in}}{\pgfqpoint{3.331523in}{4.090383in}}{\pgfqpoint{3.327405in}{4.094501in}}%
\pgfpathcurveto{\pgfqpoint{3.323287in}{4.098619in}}{\pgfqpoint{3.317700in}{4.100933in}}{\pgfqpoint{3.311877in}{4.100933in}}%
\pgfpathcurveto{\pgfqpoint{3.306053in}{4.100933in}}{\pgfqpoint{3.300466in}{4.098619in}}{\pgfqpoint{3.296348in}{4.094501in}}%
\pgfpathcurveto{\pgfqpoint{3.292230in}{4.090383in}}{\pgfqpoint{3.289916in}{4.084797in}}{\pgfqpoint{3.289916in}{4.078973in}}%
\pgfpathcurveto{\pgfqpoint{3.289916in}{4.073149in}}{\pgfqpoint{3.292230in}{4.067563in}}{\pgfqpoint{3.296348in}{4.063445in}}%
\pgfpathcurveto{\pgfqpoint{3.300466in}{4.059326in}}{\pgfqpoint{3.306053in}{4.057012in}}{\pgfqpoint{3.311877in}{4.057012in}}%
\pgfpathlineto{\pgfqpoint{3.311877in}{4.057012in}}%
\pgfpathclose%
\pgfusepath{stroke,fill}%
\end{pgfscope}%
\begin{pgfscope}%
\pgfpathrectangle{\pgfqpoint{1.000000in}{0.979904in}}{\pgfqpoint{6.200000in}{5.960192in}}%
\pgfusepath{clip}%
\pgfsetbuttcap%
\pgfsetroundjoin%
\definecolor{currentfill}{rgb}{0.200000,0.800000,0.200000}%
\pgfsetfillcolor{currentfill}%
\pgfsetlinewidth{1.003750pt}%
\definecolor{currentstroke}{rgb}{0.200000,0.800000,0.200000}%
\pgfsetstrokecolor{currentstroke}%
\pgfsetdash{}{0pt}%
\pgfpathmoveto{\pgfqpoint{3.248148in}{3.925924in}}%
\pgfpathcurveto{\pgfqpoint{3.253972in}{3.925924in}}{\pgfqpoint{3.259558in}{3.928237in}}{\pgfqpoint{3.263676in}{3.932356in}}%
\pgfpathcurveto{\pgfqpoint{3.267795in}{3.936474in}}{\pgfqpoint{3.270108in}{3.942060in}}{\pgfqpoint{3.270108in}{3.947884in}}%
\pgfpathcurveto{\pgfqpoint{3.270108in}{3.953708in}}{\pgfqpoint{3.267795in}{3.959294in}}{\pgfqpoint{3.263676in}{3.963412in}}%
\pgfpathcurveto{\pgfqpoint{3.259558in}{3.967530in}}{\pgfqpoint{3.253972in}{3.969844in}}{\pgfqpoint{3.248148in}{3.969844in}}%
\pgfpathcurveto{\pgfqpoint{3.242324in}{3.969844in}}{\pgfqpoint{3.236738in}{3.967530in}}{\pgfqpoint{3.232620in}{3.963412in}}%
\pgfpathcurveto{\pgfqpoint{3.228502in}{3.959294in}}{\pgfqpoint{3.226188in}{3.953708in}}{\pgfqpoint{3.226188in}{3.947884in}}%
\pgfpathcurveto{\pgfqpoint{3.226188in}{3.942060in}}{\pgfqpoint{3.228502in}{3.936474in}}{\pgfqpoint{3.232620in}{3.932356in}}%
\pgfpathcurveto{\pgfqpoint{3.236738in}{3.928237in}}{\pgfqpoint{3.242324in}{3.925924in}}{\pgfqpoint{3.248148in}{3.925924in}}%
\pgfpathlineto{\pgfqpoint{3.248148in}{3.925924in}}%
\pgfpathclose%
\pgfusepath{stroke,fill}%
\end{pgfscope}%
\begin{pgfscope}%
\pgfpathrectangle{\pgfqpoint{1.000000in}{0.979904in}}{\pgfqpoint{6.200000in}{5.960192in}}%
\pgfusepath{clip}%
\pgfsetbuttcap%
\pgfsetroundjoin%
\definecolor{currentfill}{rgb}{0.200000,0.800000,0.200000}%
\pgfsetfillcolor{currentfill}%
\pgfsetlinewidth{1.003750pt}%
\definecolor{currentstroke}{rgb}{0.200000,0.800000,0.200000}%
\pgfsetstrokecolor{currentstroke}%
\pgfsetdash{}{0pt}%
\pgfpathmoveto{\pgfqpoint{3.311135in}{3.827716in}}%
\pgfpathcurveto{\pgfqpoint{3.316959in}{3.827716in}}{\pgfqpoint{3.322545in}{3.830030in}}{\pgfqpoint{3.326663in}{3.834148in}}%
\pgfpathcurveto{\pgfqpoint{3.330782in}{3.838266in}}{\pgfqpoint{3.333095in}{3.843852in}}{\pgfqpoint{3.333095in}{3.849676in}}%
\pgfpathcurveto{\pgfqpoint{3.333095in}{3.855500in}}{\pgfqpoint{3.330782in}{3.861086in}}{\pgfqpoint{3.326663in}{3.865204in}}%
\pgfpathcurveto{\pgfqpoint{3.322545in}{3.869323in}}{\pgfqpoint{3.316959in}{3.871636in}}{\pgfqpoint{3.311135in}{3.871636in}}%
\pgfpathcurveto{\pgfqpoint{3.305311in}{3.871636in}}{\pgfqpoint{3.299725in}{3.869323in}}{\pgfqpoint{3.295607in}{3.865204in}}%
\pgfpathcurveto{\pgfqpoint{3.291489in}{3.861086in}}{\pgfqpoint{3.289175in}{3.855500in}}{\pgfqpoint{3.289175in}{3.849676in}}%
\pgfpathcurveto{\pgfqpoint{3.289175in}{3.843852in}}{\pgfqpoint{3.291489in}{3.838266in}}{\pgfqpoint{3.295607in}{3.834148in}}%
\pgfpathcurveto{\pgfqpoint{3.299725in}{3.830030in}}{\pgfqpoint{3.305311in}{3.827716in}}{\pgfqpoint{3.311135in}{3.827716in}}%
\pgfpathlineto{\pgfqpoint{3.311135in}{3.827716in}}%
\pgfpathclose%
\pgfusepath{stroke,fill}%
\end{pgfscope}%
\begin{pgfscope}%
\pgfpathrectangle{\pgfqpoint{1.000000in}{0.979904in}}{\pgfqpoint{6.200000in}{5.960192in}}%
\pgfusepath{clip}%
\pgfsetbuttcap%
\pgfsetroundjoin%
\definecolor{currentfill}{rgb}{0.200000,0.800000,0.200000}%
\pgfsetfillcolor{currentfill}%
\pgfsetlinewidth{1.003750pt}%
\definecolor{currentstroke}{rgb}{0.200000,0.800000,0.200000}%
\pgfsetstrokecolor{currentstroke}%
\pgfsetdash{}{0pt}%
\pgfpathmoveto{\pgfqpoint{3.332880in}{3.715003in}}%
\pgfpathcurveto{\pgfqpoint{3.338704in}{3.715003in}}{\pgfqpoint{3.344290in}{3.717317in}}{\pgfqpoint{3.348408in}{3.721435in}}%
\pgfpathcurveto{\pgfqpoint{3.352526in}{3.725553in}}{\pgfqpoint{3.354840in}{3.731139in}}{\pgfqpoint{3.354840in}{3.736963in}}%
\pgfpathcurveto{\pgfqpoint{3.354840in}{3.742787in}}{\pgfqpoint{3.352526in}{3.748374in}}{\pgfqpoint{3.348408in}{3.752492in}}%
\pgfpathcurveto{\pgfqpoint{3.344290in}{3.756610in}}{\pgfqpoint{3.338704in}{3.758924in}}{\pgfqpoint{3.332880in}{3.758924in}}%
\pgfpathcurveto{\pgfqpoint{3.327056in}{3.758924in}}{\pgfqpoint{3.321470in}{3.756610in}}{\pgfqpoint{3.317352in}{3.752492in}}%
\pgfpathcurveto{\pgfqpoint{3.313233in}{3.748374in}}{\pgfqpoint{3.310920in}{3.742787in}}{\pgfqpoint{3.310920in}{3.736963in}}%
\pgfpathcurveto{\pgfqpoint{3.310920in}{3.731139in}}{\pgfqpoint{3.313233in}{3.725553in}}{\pgfqpoint{3.317352in}{3.721435in}}%
\pgfpathcurveto{\pgfqpoint{3.321470in}{3.717317in}}{\pgfqpoint{3.327056in}{3.715003in}}{\pgfqpoint{3.332880in}{3.715003in}}%
\pgfpathlineto{\pgfqpoint{3.332880in}{3.715003in}}%
\pgfpathclose%
\pgfusepath{stroke,fill}%
\end{pgfscope}%
\begin{pgfscope}%
\pgfpathrectangle{\pgfqpoint{1.000000in}{0.979904in}}{\pgfqpoint{6.200000in}{5.960192in}}%
\pgfusepath{clip}%
\pgfsetbuttcap%
\pgfsetroundjoin%
\definecolor{currentfill}{rgb}{0.200000,0.800000,0.200000}%
\pgfsetfillcolor{currentfill}%
\pgfsetlinewidth{1.003750pt}%
\definecolor{currentstroke}{rgb}{0.200000,0.800000,0.200000}%
\pgfsetstrokecolor{currentstroke}%
\pgfsetdash{}{0pt}%
\pgfpathmoveto{\pgfqpoint{3.330940in}{3.586474in}}%
\pgfpathcurveto{\pgfqpoint{3.336764in}{3.586474in}}{\pgfqpoint{3.342350in}{3.588788in}}{\pgfqpoint{3.346468in}{3.592906in}}%
\pgfpathcurveto{\pgfqpoint{3.350586in}{3.597024in}}{\pgfqpoint{3.352900in}{3.602610in}}{\pgfqpoint{3.352900in}{3.608434in}}%
\pgfpathcurveto{\pgfqpoint{3.352900in}{3.614258in}}{\pgfqpoint{3.350586in}{3.619844in}}{\pgfqpoint{3.346468in}{3.623962in}}%
\pgfpathcurveto{\pgfqpoint{3.342350in}{3.628080in}}{\pgfqpoint{3.336764in}{3.630394in}}{\pgfqpoint{3.330940in}{3.630394in}}%
\pgfpathcurveto{\pgfqpoint{3.325116in}{3.630394in}}{\pgfqpoint{3.319530in}{3.628080in}}{\pgfqpoint{3.315411in}{3.623962in}}%
\pgfpathcurveto{\pgfqpoint{3.311293in}{3.619844in}}{\pgfqpoint{3.308979in}{3.614258in}}{\pgfqpoint{3.308979in}{3.608434in}}%
\pgfpathcurveto{\pgfqpoint{3.308979in}{3.602610in}}{\pgfqpoint{3.311293in}{3.597024in}}{\pgfqpoint{3.315411in}{3.592906in}}%
\pgfpathcurveto{\pgfqpoint{3.319530in}{3.588788in}}{\pgfqpoint{3.325116in}{3.586474in}}{\pgfqpoint{3.330940in}{3.586474in}}%
\pgfpathlineto{\pgfqpoint{3.330940in}{3.586474in}}%
\pgfpathclose%
\pgfusepath{stroke,fill}%
\end{pgfscope}%
\begin{pgfscope}%
\pgfpathrectangle{\pgfqpoint{1.000000in}{0.979904in}}{\pgfqpoint{6.200000in}{5.960192in}}%
\pgfusepath{clip}%
\pgfsetbuttcap%
\pgfsetroundjoin%
\definecolor{currentfill}{rgb}{0.200000,0.800000,0.200000}%
\pgfsetfillcolor{currentfill}%
\pgfsetlinewidth{1.003750pt}%
\definecolor{currentstroke}{rgb}{0.200000,0.800000,0.200000}%
\pgfsetstrokecolor{currentstroke}%
\pgfsetdash{}{0pt}%
\pgfpathmoveto{\pgfqpoint{3.485354in}{3.542592in}}%
\pgfpathcurveto{\pgfqpoint{3.491178in}{3.542592in}}{\pgfqpoint{3.496764in}{3.544906in}}{\pgfqpoint{3.500882in}{3.549024in}}%
\pgfpathcurveto{\pgfqpoint{3.505000in}{3.553142in}}{\pgfqpoint{3.507314in}{3.558728in}}{\pgfqpoint{3.507314in}{3.564552in}}%
\pgfpathcurveto{\pgfqpoint{3.507314in}{3.570376in}}{\pgfqpoint{3.505000in}{3.575962in}}{\pgfqpoint{3.500882in}{3.580081in}}%
\pgfpathcurveto{\pgfqpoint{3.496764in}{3.584199in}}{\pgfqpoint{3.491178in}{3.586513in}}{\pgfqpoint{3.485354in}{3.586513in}}%
\pgfpathcurveto{\pgfqpoint{3.479530in}{3.586513in}}{\pgfqpoint{3.473944in}{3.584199in}}{\pgfqpoint{3.469826in}{3.580081in}}%
\pgfpathcurveto{\pgfqpoint{3.465708in}{3.575962in}}{\pgfqpoint{3.463394in}{3.570376in}}{\pgfqpoint{3.463394in}{3.564552in}}%
\pgfpathcurveto{\pgfqpoint{3.463394in}{3.558728in}}{\pgfqpoint{3.465708in}{3.553142in}}{\pgfqpoint{3.469826in}{3.549024in}}%
\pgfpathcurveto{\pgfqpoint{3.473944in}{3.544906in}}{\pgfqpoint{3.479530in}{3.542592in}}{\pgfqpoint{3.485354in}{3.542592in}}%
\pgfpathlineto{\pgfqpoint{3.485354in}{3.542592in}}%
\pgfpathclose%
\pgfusepath{stroke,fill}%
\end{pgfscope}%
\begin{pgfscope}%
\pgfpathrectangle{\pgfqpoint{1.000000in}{0.979904in}}{\pgfqpoint{6.200000in}{5.960192in}}%
\pgfusepath{clip}%
\pgfsetbuttcap%
\pgfsetroundjoin%
\definecolor{currentfill}{rgb}{0.200000,0.800000,0.200000}%
\pgfsetfillcolor{currentfill}%
\pgfsetlinewidth{1.003750pt}%
\definecolor{currentstroke}{rgb}{0.200000,0.800000,0.200000}%
\pgfsetstrokecolor{currentstroke}%
\pgfsetdash{}{0pt}%
\pgfpathmoveto{\pgfqpoint{3.566940in}{3.465504in}}%
\pgfpathcurveto{\pgfqpoint{3.572764in}{3.465504in}}{\pgfqpoint{3.578351in}{3.467817in}}{\pgfqpoint{3.582469in}{3.471936in}}%
\pgfpathcurveto{\pgfqpoint{3.586587in}{3.476054in}}{\pgfqpoint{3.588901in}{3.481640in}}{\pgfqpoint{3.588901in}{3.487464in}}%
\pgfpathcurveto{\pgfqpoint{3.588901in}{3.493288in}}{\pgfqpoint{3.586587in}{3.498874in}}{\pgfqpoint{3.582469in}{3.502992in}}%
\pgfpathcurveto{\pgfqpoint{3.578351in}{3.507110in}}{\pgfqpoint{3.572764in}{3.509424in}}{\pgfqpoint{3.566940in}{3.509424in}}%
\pgfpathcurveto{\pgfqpoint{3.561117in}{3.509424in}}{\pgfqpoint{3.555530in}{3.507110in}}{\pgfqpoint{3.551412in}{3.502992in}}%
\pgfpathcurveto{\pgfqpoint{3.547294in}{3.498874in}}{\pgfqpoint{3.544980in}{3.493288in}}{\pgfqpoint{3.544980in}{3.487464in}}%
\pgfpathcurveto{\pgfqpoint{3.544980in}{3.481640in}}{\pgfqpoint{3.547294in}{3.476054in}}{\pgfqpoint{3.551412in}{3.471936in}}%
\pgfpathcurveto{\pgfqpoint{3.555530in}{3.467817in}}{\pgfqpoint{3.561117in}{3.465504in}}{\pgfqpoint{3.566940in}{3.465504in}}%
\pgfpathlineto{\pgfqpoint{3.566940in}{3.465504in}}%
\pgfpathclose%
\pgfusepath{stroke,fill}%
\end{pgfscope}%
\begin{pgfscope}%
\pgfpathrectangle{\pgfqpoint{1.000000in}{0.979904in}}{\pgfqpoint{6.200000in}{5.960192in}}%
\pgfusepath{clip}%
\pgfsetbuttcap%
\pgfsetroundjoin%
\definecolor{currentfill}{rgb}{0.200000,0.800000,0.200000}%
\pgfsetfillcolor{currentfill}%
\pgfsetlinewidth{1.003750pt}%
\definecolor{currentstroke}{rgb}{0.200000,0.800000,0.200000}%
\pgfsetstrokecolor{currentstroke}%
\pgfsetdash{}{0pt}%
\pgfpathmoveto{\pgfqpoint{3.636699in}{3.382904in}}%
\pgfpathcurveto{\pgfqpoint{3.642523in}{3.382904in}}{\pgfqpoint{3.648109in}{3.385218in}}{\pgfqpoint{3.652227in}{3.389336in}}%
\pgfpathcurveto{\pgfqpoint{3.656345in}{3.393454in}}{\pgfqpoint{3.658659in}{3.399040in}}{\pgfqpoint{3.658659in}{3.404864in}}%
\pgfpathcurveto{\pgfqpoint{3.658659in}{3.410688in}}{\pgfqpoint{3.656345in}{3.416274in}}{\pgfqpoint{3.652227in}{3.420392in}}%
\pgfpathcurveto{\pgfqpoint{3.648109in}{3.424510in}}{\pgfqpoint{3.642523in}{3.426824in}}{\pgfqpoint{3.636699in}{3.426824in}}%
\pgfpathcurveto{\pgfqpoint{3.630875in}{3.426824in}}{\pgfqpoint{3.625289in}{3.424510in}}{\pgfqpoint{3.621171in}{3.420392in}}%
\pgfpathcurveto{\pgfqpoint{3.617053in}{3.416274in}}{\pgfqpoint{3.614739in}{3.410688in}}{\pgfqpoint{3.614739in}{3.404864in}}%
\pgfpathcurveto{\pgfqpoint{3.614739in}{3.399040in}}{\pgfqpoint{3.617053in}{3.393454in}}{\pgfqpoint{3.621171in}{3.389336in}}%
\pgfpathcurveto{\pgfqpoint{3.625289in}{3.385218in}}{\pgfqpoint{3.630875in}{3.382904in}}{\pgfqpoint{3.636699in}{3.382904in}}%
\pgfpathlineto{\pgfqpoint{3.636699in}{3.382904in}}%
\pgfpathclose%
\pgfusepath{stroke,fill}%
\end{pgfscope}%
\begin{pgfscope}%
\pgfpathrectangle{\pgfqpoint{1.000000in}{0.979904in}}{\pgfqpoint{6.200000in}{5.960192in}}%
\pgfusepath{clip}%
\pgfsetbuttcap%
\pgfsetroundjoin%
\definecolor{currentfill}{rgb}{0.200000,0.800000,0.200000}%
\pgfsetfillcolor{currentfill}%
\pgfsetlinewidth{1.003750pt}%
\definecolor{currentstroke}{rgb}{0.200000,0.800000,0.200000}%
\pgfsetstrokecolor{currentstroke}%
\pgfsetdash{}{0pt}%
\pgfpathmoveto{\pgfqpoint{3.702979in}{3.297834in}}%
\pgfpathcurveto{\pgfqpoint{3.708803in}{3.297834in}}{\pgfqpoint{3.714389in}{3.300148in}}{\pgfqpoint{3.718507in}{3.304266in}}%
\pgfpathcurveto{\pgfqpoint{3.722625in}{3.308384in}}{\pgfqpoint{3.724939in}{3.313970in}}{\pgfqpoint{3.724939in}{3.319794in}}%
\pgfpathcurveto{\pgfqpoint{3.724939in}{3.325618in}}{\pgfqpoint{3.722625in}{3.331204in}}{\pgfqpoint{3.718507in}{3.335322in}}%
\pgfpathcurveto{\pgfqpoint{3.714389in}{3.339441in}}{\pgfqpoint{3.708803in}{3.341754in}}{\pgfqpoint{3.702979in}{3.341754in}}%
\pgfpathcurveto{\pgfqpoint{3.697155in}{3.341754in}}{\pgfqpoint{3.691569in}{3.339441in}}{\pgfqpoint{3.687451in}{3.335322in}}%
\pgfpathcurveto{\pgfqpoint{3.683332in}{3.331204in}}{\pgfqpoint{3.681019in}{3.325618in}}{\pgfqpoint{3.681019in}{3.319794in}}%
\pgfpathcurveto{\pgfqpoint{3.681019in}{3.313970in}}{\pgfqpoint{3.683332in}{3.308384in}}{\pgfqpoint{3.687451in}{3.304266in}}%
\pgfpathcurveto{\pgfqpoint{3.691569in}{3.300148in}}{\pgfqpoint{3.697155in}{3.297834in}}{\pgfqpoint{3.702979in}{3.297834in}}%
\pgfpathlineto{\pgfqpoint{3.702979in}{3.297834in}}%
\pgfpathclose%
\pgfusepath{stroke,fill}%
\end{pgfscope}%
\begin{pgfscope}%
\pgfpathrectangle{\pgfqpoint{1.000000in}{0.979904in}}{\pgfqpoint{6.200000in}{5.960192in}}%
\pgfusepath{clip}%
\pgfsetbuttcap%
\pgfsetroundjoin%
\definecolor{currentfill}{rgb}{0.200000,0.800000,0.200000}%
\pgfsetfillcolor{currentfill}%
\pgfsetlinewidth{1.003750pt}%
\definecolor{currentstroke}{rgb}{0.200000,0.800000,0.200000}%
\pgfsetstrokecolor{currentstroke}%
\pgfsetdash{}{0pt}%
\pgfpathmoveto{\pgfqpoint{3.749942in}{3.191814in}}%
\pgfpathcurveto{\pgfqpoint{3.755766in}{3.191814in}}{\pgfqpoint{3.761352in}{3.194127in}}{\pgfqpoint{3.765470in}{3.198246in}}%
\pgfpathcurveto{\pgfqpoint{3.769588in}{3.202364in}}{\pgfqpoint{3.771902in}{3.207950in}}{\pgfqpoint{3.771902in}{3.213774in}}%
\pgfpathcurveto{\pgfqpoint{3.771902in}{3.219598in}}{\pgfqpoint{3.769588in}{3.225184in}}{\pgfqpoint{3.765470in}{3.229302in}}%
\pgfpathcurveto{\pgfqpoint{3.761352in}{3.233420in}}{\pgfqpoint{3.755766in}{3.235734in}}{\pgfqpoint{3.749942in}{3.235734in}}%
\pgfpathcurveto{\pgfqpoint{3.744118in}{3.235734in}}{\pgfqpoint{3.738532in}{3.233420in}}{\pgfqpoint{3.734414in}{3.229302in}}%
\pgfpathcurveto{\pgfqpoint{3.730295in}{3.225184in}}{\pgfqpoint{3.727982in}{3.219598in}}{\pgfqpoint{3.727982in}{3.213774in}}%
\pgfpathcurveto{\pgfqpoint{3.727982in}{3.207950in}}{\pgfqpoint{3.730295in}{3.202364in}}{\pgfqpoint{3.734414in}{3.198246in}}%
\pgfpathcurveto{\pgfqpoint{3.738532in}{3.194127in}}{\pgfqpoint{3.744118in}{3.191814in}}{\pgfqpoint{3.749942in}{3.191814in}}%
\pgfpathlineto{\pgfqpoint{3.749942in}{3.191814in}}%
\pgfpathclose%
\pgfusepath{stroke,fill}%
\end{pgfscope}%
\begin{pgfscope}%
\pgfpathrectangle{\pgfqpoint{1.000000in}{0.979904in}}{\pgfqpoint{6.200000in}{5.960192in}}%
\pgfusepath{clip}%
\pgfsetbuttcap%
\pgfsetroundjoin%
\definecolor{currentfill}{rgb}{0.200000,0.800000,0.200000}%
\pgfsetfillcolor{currentfill}%
\pgfsetlinewidth{1.003750pt}%
\definecolor{currentstroke}{rgb}{0.200000,0.800000,0.200000}%
\pgfsetstrokecolor{currentstroke}%
\pgfsetdash{}{0pt}%
\pgfpathmoveto{\pgfqpoint{3.816496in}{3.100147in}}%
\pgfpathcurveto{\pgfqpoint{3.822320in}{3.100147in}}{\pgfqpoint{3.827906in}{3.102461in}}{\pgfqpoint{3.832024in}{3.106579in}}%
\pgfpathcurveto{\pgfqpoint{3.836142in}{3.110697in}}{\pgfqpoint{3.838456in}{3.116284in}}{\pgfqpoint{3.838456in}{3.122107in}}%
\pgfpathcurveto{\pgfqpoint{3.838456in}{3.127931in}}{\pgfqpoint{3.836142in}{3.133518in}}{\pgfqpoint{3.832024in}{3.137636in}}%
\pgfpathcurveto{\pgfqpoint{3.827906in}{3.141754in}}{\pgfqpoint{3.822320in}{3.144068in}}{\pgfqpoint{3.816496in}{3.144068in}}%
\pgfpathcurveto{\pgfqpoint{3.810672in}{3.144068in}}{\pgfqpoint{3.805085in}{3.141754in}}{\pgfqpoint{3.800967in}{3.137636in}}%
\pgfpathcurveto{\pgfqpoint{3.796849in}{3.133518in}}{\pgfqpoint{3.794535in}{3.127931in}}{\pgfqpoint{3.794535in}{3.122107in}}%
\pgfpathcurveto{\pgfqpoint{3.794535in}{3.116284in}}{\pgfqpoint{3.796849in}{3.110697in}}{\pgfqpoint{3.800967in}{3.106579in}}%
\pgfpathcurveto{\pgfqpoint{3.805085in}{3.102461in}}{\pgfqpoint{3.810672in}{3.100147in}}{\pgfqpoint{3.816496in}{3.100147in}}%
\pgfpathlineto{\pgfqpoint{3.816496in}{3.100147in}}%
\pgfpathclose%
\pgfusepath{stroke,fill}%
\end{pgfscope}%
\begin{pgfscope}%
\pgfpathrectangle{\pgfqpoint{1.000000in}{0.979904in}}{\pgfqpoint{6.200000in}{5.960192in}}%
\pgfusepath{clip}%
\pgfsetbuttcap%
\pgfsetroundjoin%
\definecolor{currentfill}{rgb}{0.200000,0.800000,0.200000}%
\pgfsetfillcolor{currentfill}%
\pgfsetlinewidth{1.003750pt}%
\definecolor{currentstroke}{rgb}{0.200000,0.800000,0.200000}%
\pgfsetstrokecolor{currentstroke}%
\pgfsetdash{}{0pt}%
\pgfpathmoveto{\pgfqpoint{3.905333in}{3.032379in}}%
\pgfpathcurveto{\pgfqpoint{3.911157in}{3.032379in}}{\pgfqpoint{3.916743in}{3.034693in}}{\pgfqpoint{3.920861in}{3.038811in}}%
\pgfpathcurveto{\pgfqpoint{3.924979in}{3.042929in}}{\pgfqpoint{3.927293in}{3.048515in}}{\pgfqpoint{3.927293in}{3.054339in}}%
\pgfpathcurveto{\pgfqpoint{3.927293in}{3.060163in}}{\pgfqpoint{3.924979in}{3.065749in}}{\pgfqpoint{3.920861in}{3.069868in}}%
\pgfpathcurveto{\pgfqpoint{3.916743in}{3.073986in}}{\pgfqpoint{3.911157in}{3.076300in}}{\pgfqpoint{3.905333in}{3.076300in}}%
\pgfpathcurveto{\pgfqpoint{3.899509in}{3.076300in}}{\pgfqpoint{3.893922in}{3.073986in}}{\pgfqpoint{3.889804in}{3.069868in}}%
\pgfpathcurveto{\pgfqpoint{3.885686in}{3.065749in}}{\pgfqpoint{3.883372in}{3.060163in}}{\pgfqpoint{3.883372in}{3.054339in}}%
\pgfpathcurveto{\pgfqpoint{3.883372in}{3.048515in}}{\pgfqpoint{3.885686in}{3.042929in}}{\pgfqpoint{3.889804in}{3.038811in}}%
\pgfpathcurveto{\pgfqpoint{3.893922in}{3.034693in}}{\pgfqpoint{3.899509in}{3.032379in}}{\pgfqpoint{3.905333in}{3.032379in}}%
\pgfpathlineto{\pgfqpoint{3.905333in}{3.032379in}}%
\pgfpathclose%
\pgfusepath{stroke,fill}%
\end{pgfscope}%
\begin{pgfscope}%
\pgfpathrectangle{\pgfqpoint{1.000000in}{0.979904in}}{\pgfqpoint{6.200000in}{5.960192in}}%
\pgfusepath{clip}%
\pgfsetbuttcap%
\pgfsetroundjoin%
\definecolor{currentfill}{rgb}{0.200000,0.800000,0.200000}%
\pgfsetfillcolor{currentfill}%
\pgfsetlinewidth{1.003750pt}%
\definecolor{currentstroke}{rgb}{0.200000,0.800000,0.200000}%
\pgfsetstrokecolor{currentstroke}%
\pgfsetdash{}{0pt}%
\pgfpathmoveto{\pgfqpoint{3.980039in}{2.943343in}}%
\pgfpathcurveto{\pgfqpoint{3.985863in}{2.943343in}}{\pgfqpoint{3.991449in}{2.945657in}}{\pgfqpoint{3.995567in}{2.949775in}}%
\pgfpathcurveto{\pgfqpoint{3.999685in}{2.953893in}}{\pgfqpoint{4.001999in}{2.959479in}}{\pgfqpoint{4.001999in}{2.965303in}}%
\pgfpathcurveto{\pgfqpoint{4.001999in}{2.971127in}}{\pgfqpoint{3.999685in}{2.976713in}}{\pgfqpoint{3.995567in}{2.980832in}}%
\pgfpathcurveto{\pgfqpoint{3.991449in}{2.984950in}}{\pgfqpoint{3.985863in}{2.987264in}}{\pgfqpoint{3.980039in}{2.987264in}}%
\pgfpathcurveto{\pgfqpoint{3.974215in}{2.987264in}}{\pgfqpoint{3.968629in}{2.984950in}}{\pgfqpoint{3.964511in}{2.980832in}}%
\pgfpathcurveto{\pgfqpoint{3.960393in}{2.976713in}}{\pgfqpoint{3.958079in}{2.971127in}}{\pgfqpoint{3.958079in}{2.965303in}}%
\pgfpathcurveto{\pgfqpoint{3.958079in}{2.959479in}}{\pgfqpoint{3.960393in}{2.953893in}}{\pgfqpoint{3.964511in}{2.949775in}}%
\pgfpathcurveto{\pgfqpoint{3.968629in}{2.945657in}}{\pgfqpoint{3.974215in}{2.943343in}}{\pgfqpoint{3.980039in}{2.943343in}}%
\pgfpathlineto{\pgfqpoint{3.980039in}{2.943343in}}%
\pgfpathclose%
\pgfusepath{stroke,fill}%
\end{pgfscope}%
\begin{pgfscope}%
\pgfpathrectangle{\pgfqpoint{1.000000in}{0.979904in}}{\pgfqpoint{6.200000in}{5.960192in}}%
\pgfusepath{clip}%
\pgfsetbuttcap%
\pgfsetroundjoin%
\definecolor{currentfill}{rgb}{0.200000,0.800000,0.200000}%
\pgfsetfillcolor{currentfill}%
\pgfsetlinewidth{1.003750pt}%
\definecolor{currentstroke}{rgb}{0.200000,0.800000,0.200000}%
\pgfsetstrokecolor{currentstroke}%
\pgfsetdash{}{0pt}%
\pgfpathmoveto{\pgfqpoint{4.098020in}{2.920952in}}%
\pgfpathcurveto{\pgfqpoint{4.103843in}{2.920952in}}{\pgfqpoint{4.109430in}{2.923266in}}{\pgfqpoint{4.113548in}{2.927384in}}%
\pgfpathcurveto{\pgfqpoint{4.117666in}{2.931502in}}{\pgfqpoint{4.119980in}{2.937088in}}{\pgfqpoint{4.119980in}{2.942912in}}%
\pgfpathcurveto{\pgfqpoint{4.119980in}{2.948736in}}{\pgfqpoint{4.117666in}{2.954322in}}{\pgfqpoint{4.113548in}{2.958440in}}%
\pgfpathcurveto{\pgfqpoint{4.109430in}{2.962559in}}{\pgfqpoint{4.103843in}{2.964872in}}{\pgfqpoint{4.098020in}{2.964872in}}%
\pgfpathcurveto{\pgfqpoint{4.092196in}{2.964872in}}{\pgfqpoint{4.086609in}{2.962559in}}{\pgfqpoint{4.082491in}{2.958440in}}%
\pgfpathcurveto{\pgfqpoint{4.078373in}{2.954322in}}{\pgfqpoint{4.076059in}{2.948736in}}{\pgfqpoint{4.076059in}{2.942912in}}%
\pgfpathcurveto{\pgfqpoint{4.076059in}{2.937088in}}{\pgfqpoint{4.078373in}{2.931502in}}{\pgfqpoint{4.082491in}{2.927384in}}%
\pgfpathcurveto{\pgfqpoint{4.086609in}{2.923266in}}{\pgfqpoint{4.092196in}{2.920952in}}{\pgfqpoint{4.098020in}{2.920952in}}%
\pgfpathlineto{\pgfqpoint{4.098020in}{2.920952in}}%
\pgfpathclose%
\pgfusepath{stroke,fill}%
\end{pgfscope}%
\begin{pgfscope}%
\pgfpathrectangle{\pgfqpoint{1.000000in}{0.979904in}}{\pgfqpoint{6.200000in}{5.960192in}}%
\pgfusepath{clip}%
\pgfsetbuttcap%
\pgfsetroundjoin%
\definecolor{currentfill}{rgb}{0.200000,0.800000,0.200000}%
\pgfsetfillcolor{currentfill}%
\pgfsetlinewidth{1.003750pt}%
\definecolor{currentstroke}{rgb}{0.200000,0.800000,0.200000}%
\pgfsetstrokecolor{currentstroke}%
\pgfsetdash{}{0pt}%
\pgfpathmoveto{\pgfqpoint{4.161644in}{2.800198in}}%
\pgfpathcurveto{\pgfqpoint{4.167468in}{2.800198in}}{\pgfqpoint{4.173054in}{2.802511in}}{\pgfqpoint{4.177172in}{2.806630in}}%
\pgfpathcurveto{\pgfqpoint{4.181290in}{2.810748in}}{\pgfqpoint{4.183604in}{2.816334in}}{\pgfqpoint{4.183604in}{2.822158in}}%
\pgfpathcurveto{\pgfqpoint{4.183604in}{2.827982in}}{\pgfqpoint{4.181290in}{2.833568in}}{\pgfqpoint{4.177172in}{2.837686in}}%
\pgfpathcurveto{\pgfqpoint{4.173054in}{2.841804in}}{\pgfqpoint{4.167468in}{2.844118in}}{\pgfqpoint{4.161644in}{2.844118in}}%
\pgfpathcurveto{\pgfqpoint{4.155820in}{2.844118in}}{\pgfqpoint{4.150234in}{2.841804in}}{\pgfqpoint{4.146116in}{2.837686in}}%
\pgfpathcurveto{\pgfqpoint{4.141997in}{2.833568in}}{\pgfqpoint{4.139684in}{2.827982in}}{\pgfqpoint{4.139684in}{2.822158in}}%
\pgfpathcurveto{\pgfqpoint{4.139684in}{2.816334in}}{\pgfqpoint{4.141997in}{2.810748in}}{\pgfqpoint{4.146116in}{2.806630in}}%
\pgfpathcurveto{\pgfqpoint{4.150234in}{2.802511in}}{\pgfqpoint{4.155820in}{2.800198in}}{\pgfqpoint{4.161644in}{2.800198in}}%
\pgfpathlineto{\pgfqpoint{4.161644in}{2.800198in}}%
\pgfpathclose%
\pgfusepath{stroke,fill}%
\end{pgfscope}%
\begin{pgfscope}%
\pgfpathrectangle{\pgfqpoint{1.000000in}{0.979904in}}{\pgfqpoint{6.200000in}{5.960192in}}%
\pgfusepath{clip}%
\pgfsetbuttcap%
\pgfsetroundjoin%
\definecolor{currentfill}{rgb}{0.200000,0.800000,0.200000}%
\pgfsetfillcolor{currentfill}%
\pgfsetlinewidth{1.003750pt}%
\definecolor{currentstroke}{rgb}{0.200000,0.800000,0.200000}%
\pgfsetstrokecolor{currentstroke}%
\pgfsetdash{}{0pt}%
\pgfpathmoveto{\pgfqpoint{4.301958in}{2.836286in}}%
\pgfpathcurveto{\pgfqpoint{4.307782in}{2.836286in}}{\pgfqpoint{4.313369in}{2.838599in}}{\pgfqpoint{4.317487in}{2.842718in}}%
\pgfpathcurveto{\pgfqpoint{4.321605in}{2.846836in}}{\pgfqpoint{4.323919in}{2.852422in}}{\pgfqpoint{4.323919in}{2.858246in}}%
\pgfpathcurveto{\pgfqpoint{4.323919in}{2.864070in}}{\pgfqpoint{4.321605in}{2.869656in}}{\pgfqpoint{4.317487in}{2.873774in}}%
\pgfpathcurveto{\pgfqpoint{4.313369in}{2.877892in}}{\pgfqpoint{4.307782in}{2.880206in}}{\pgfqpoint{4.301958in}{2.880206in}}%
\pgfpathcurveto{\pgfqpoint{4.296134in}{2.880206in}}{\pgfqpoint{4.290548in}{2.877892in}}{\pgfqpoint{4.286430in}{2.873774in}}%
\pgfpathcurveto{\pgfqpoint{4.282312in}{2.869656in}}{\pgfqpoint{4.279998in}{2.864070in}}{\pgfqpoint{4.279998in}{2.858246in}}%
\pgfpathcurveto{\pgfqpoint{4.279998in}{2.852422in}}{\pgfqpoint{4.282312in}{2.846836in}}{\pgfqpoint{4.286430in}{2.842718in}}%
\pgfpathcurveto{\pgfqpoint{4.290548in}{2.838599in}}{\pgfqpoint{4.296134in}{2.836286in}}{\pgfqpoint{4.301958in}{2.836286in}}%
\pgfpathlineto{\pgfqpoint{4.301958in}{2.836286in}}%
\pgfpathclose%
\pgfusepath{stroke,fill}%
\end{pgfscope}%
\begin{pgfscope}%
\pgfpathrectangle{\pgfqpoint{1.000000in}{0.979904in}}{\pgfqpoint{6.200000in}{5.960192in}}%
\pgfusepath{clip}%
\pgfsetbuttcap%
\pgfsetroundjoin%
\definecolor{currentfill}{rgb}{0.800000,0.200000,0.200000}%
\pgfsetfillcolor{currentfill}%
\pgfsetlinewidth{1.003750pt}%
\definecolor{currentstroke}{rgb}{0.800000,0.200000,0.200000}%
\pgfsetstrokecolor{currentstroke}%
\pgfsetdash{}{0pt}%
\pgfpathmoveto{\pgfqpoint{4.352686in}{2.647462in}}%
\pgfpathcurveto{\pgfqpoint{4.358510in}{2.647462in}}{\pgfqpoint{4.364096in}{2.649776in}}{\pgfqpoint{4.368214in}{2.653894in}}%
\pgfpathcurveto{\pgfqpoint{4.372332in}{2.658012in}}{\pgfqpoint{4.374646in}{2.663598in}}{\pgfqpoint{4.374646in}{2.669422in}}%
\pgfpathcurveto{\pgfqpoint{4.374646in}{2.675246in}}{\pgfqpoint{4.372332in}{2.680832in}}{\pgfqpoint{4.368214in}{2.684951in}}%
\pgfpathcurveto{\pgfqpoint{4.364096in}{2.689069in}}{\pgfqpoint{4.358510in}{2.691383in}}{\pgfqpoint{4.352686in}{2.691383in}}%
\pgfpathcurveto{\pgfqpoint{4.346862in}{2.691383in}}{\pgfqpoint{4.341276in}{2.689069in}}{\pgfqpoint{4.337158in}{2.684951in}}%
\pgfpathcurveto{\pgfqpoint{4.333039in}{2.680832in}}{\pgfqpoint{4.330726in}{2.675246in}}{\pgfqpoint{4.330726in}{2.669422in}}%
\pgfpathcurveto{\pgfqpoint{4.330726in}{2.663598in}}{\pgfqpoint{4.333039in}{2.658012in}}{\pgfqpoint{4.337158in}{2.653894in}}%
\pgfpathcurveto{\pgfqpoint{4.341276in}{2.649776in}}{\pgfqpoint{4.346862in}{2.647462in}}{\pgfqpoint{4.352686in}{2.647462in}}%
\pgfpathlineto{\pgfqpoint{4.352686in}{2.647462in}}%
\pgfpathclose%
\pgfusepath{stroke,fill}%
\end{pgfscope}%
\begin{pgfscope}%
\pgfpathrectangle{\pgfqpoint{1.000000in}{0.979904in}}{\pgfqpoint{6.200000in}{5.960192in}}%
\pgfusepath{clip}%
\pgfsetbuttcap%
\pgfsetroundjoin%
\definecolor{currentfill}{rgb}{0.800000,0.200000,0.200000}%
\pgfsetfillcolor{currentfill}%
\pgfsetlinewidth{1.003750pt}%
\definecolor{currentstroke}{rgb}{0.800000,0.200000,0.200000}%
\pgfsetstrokecolor{currentstroke}%
\pgfsetdash{}{0pt}%
\pgfpathmoveto{\pgfqpoint{4.496696in}{2.719000in}}%
\pgfpathcurveto{\pgfqpoint{4.502519in}{2.719000in}}{\pgfqpoint{4.508106in}{2.721314in}}{\pgfqpoint{4.512224in}{2.725432in}}%
\pgfpathcurveto{\pgfqpoint{4.516342in}{2.729550in}}{\pgfqpoint{4.518656in}{2.735137in}}{\pgfqpoint{4.518656in}{2.740960in}}%
\pgfpathcurveto{\pgfqpoint{4.518656in}{2.746784in}}{\pgfqpoint{4.516342in}{2.752371in}}{\pgfqpoint{4.512224in}{2.756489in}}%
\pgfpathcurveto{\pgfqpoint{4.508106in}{2.760607in}}{\pgfqpoint{4.502519in}{2.762921in}}{\pgfqpoint{4.496696in}{2.762921in}}%
\pgfpathcurveto{\pgfqpoint{4.490872in}{2.762921in}}{\pgfqpoint{4.485285in}{2.760607in}}{\pgfqpoint{4.481167in}{2.756489in}}%
\pgfpathcurveto{\pgfqpoint{4.477049in}{2.752371in}}{\pgfqpoint{4.474735in}{2.746784in}}{\pgfqpoint{4.474735in}{2.740960in}}%
\pgfpathcurveto{\pgfqpoint{4.474735in}{2.735137in}}{\pgfqpoint{4.477049in}{2.729550in}}{\pgfqpoint{4.481167in}{2.725432in}}%
\pgfpathcurveto{\pgfqpoint{4.485285in}{2.721314in}}{\pgfqpoint{4.490872in}{2.719000in}}{\pgfqpoint{4.496696in}{2.719000in}}%
\pgfpathlineto{\pgfqpoint{4.496696in}{2.719000in}}%
\pgfpathclose%
\pgfusepath{stroke,fill}%
\end{pgfscope}%
\begin{pgfscope}%
\pgfpathrectangle{\pgfqpoint{1.000000in}{0.979904in}}{\pgfqpoint{6.200000in}{5.960192in}}%
\pgfusepath{clip}%
\pgfsetbuttcap%
\pgfsetroundjoin%
\definecolor{currentfill}{rgb}{0.200000,0.800000,0.200000}%
\pgfsetfillcolor{currentfill}%
\pgfsetlinewidth{1.003750pt}%
\definecolor{currentstroke}{rgb}{0.200000,0.800000,0.200000}%
\pgfsetstrokecolor{currentstroke}%
\pgfsetdash{}{0pt}%
\pgfpathmoveto{\pgfqpoint{4.622379in}{2.769425in}}%
\pgfpathcurveto{\pgfqpoint{4.628203in}{2.769425in}}{\pgfqpoint{4.633789in}{2.771739in}}{\pgfqpoint{4.637907in}{2.775857in}}%
\pgfpathcurveto{\pgfqpoint{4.642025in}{2.779975in}}{\pgfqpoint{4.644339in}{2.785561in}}{\pgfqpoint{4.644339in}{2.791385in}}%
\pgfpathcurveto{\pgfqpoint{4.644339in}{2.797209in}}{\pgfqpoint{4.642025in}{2.802795in}}{\pgfqpoint{4.637907in}{2.806913in}}%
\pgfpathcurveto{\pgfqpoint{4.633789in}{2.811031in}}{\pgfqpoint{4.628203in}{2.813345in}}{\pgfqpoint{4.622379in}{2.813345in}}%
\pgfpathcurveto{\pgfqpoint{4.616555in}{2.813345in}}{\pgfqpoint{4.610968in}{2.811031in}}{\pgfqpoint{4.606850in}{2.806913in}}%
\pgfpathcurveto{\pgfqpoint{4.602732in}{2.802795in}}{\pgfqpoint{4.600418in}{2.797209in}}{\pgfqpoint{4.600418in}{2.791385in}}%
\pgfpathcurveto{\pgfqpoint{4.600418in}{2.785561in}}{\pgfqpoint{4.602732in}{2.779975in}}{\pgfqpoint{4.606850in}{2.775857in}}%
\pgfpathcurveto{\pgfqpoint{4.610968in}{2.771739in}}{\pgfqpoint{4.616555in}{2.769425in}}{\pgfqpoint{4.622379in}{2.769425in}}%
\pgfpathlineto{\pgfqpoint{4.622379in}{2.769425in}}%
\pgfpathclose%
\pgfusepath{stroke,fill}%
\end{pgfscope}%
\begin{pgfscope}%
\pgfpathrectangle{\pgfqpoint{1.000000in}{0.979904in}}{\pgfqpoint{6.200000in}{5.960192in}}%
\pgfusepath{clip}%
\pgfsetbuttcap%
\pgfsetroundjoin%
\definecolor{currentfill}{rgb}{0.200000,0.800000,0.200000}%
\pgfsetfillcolor{currentfill}%
\pgfsetlinewidth{1.003750pt}%
\definecolor{currentstroke}{rgb}{0.200000,0.800000,0.200000}%
\pgfsetstrokecolor{currentstroke}%
\pgfsetdash{}{0pt}%
\pgfpathmoveto{\pgfqpoint{4.730111in}{2.761938in}}%
\pgfpathcurveto{\pgfqpoint{4.735935in}{2.761938in}}{\pgfqpoint{4.741521in}{2.764252in}}{\pgfqpoint{4.745640in}{2.768370in}}%
\pgfpathcurveto{\pgfqpoint{4.749758in}{2.772488in}}{\pgfqpoint{4.752072in}{2.778074in}}{\pgfqpoint{4.752072in}{2.783898in}}%
\pgfpathcurveto{\pgfqpoint{4.752072in}{2.789722in}}{\pgfqpoint{4.749758in}{2.795308in}}{\pgfqpoint{4.745640in}{2.799426in}}%
\pgfpathcurveto{\pgfqpoint{4.741521in}{2.803544in}}{\pgfqpoint{4.735935in}{2.805858in}}{\pgfqpoint{4.730111in}{2.805858in}}%
\pgfpathcurveto{\pgfqpoint{4.724287in}{2.805858in}}{\pgfqpoint{4.718701in}{2.803544in}}{\pgfqpoint{4.714583in}{2.799426in}}%
\pgfpathcurveto{\pgfqpoint{4.710465in}{2.795308in}}{\pgfqpoint{4.708151in}{2.789722in}}{\pgfqpoint{4.708151in}{2.783898in}}%
\pgfpathcurveto{\pgfqpoint{4.708151in}{2.778074in}}{\pgfqpoint{4.710465in}{2.772488in}}{\pgfqpoint{4.714583in}{2.768370in}}%
\pgfpathcurveto{\pgfqpoint{4.718701in}{2.764252in}}{\pgfqpoint{4.724287in}{2.761938in}}{\pgfqpoint{4.730111in}{2.761938in}}%
\pgfpathlineto{\pgfqpoint{4.730111in}{2.761938in}}%
\pgfpathclose%
\pgfusepath{stroke,fill}%
\end{pgfscope}%
\begin{pgfscope}%
\pgfpathrectangle{\pgfqpoint{1.000000in}{0.979904in}}{\pgfqpoint{6.200000in}{5.960192in}}%
\pgfusepath{clip}%
\pgfsetbuttcap%
\pgfsetroundjoin%
\definecolor{currentfill}{rgb}{0.200000,0.800000,0.200000}%
\pgfsetfillcolor{currentfill}%
\pgfsetlinewidth{1.003750pt}%
\definecolor{currentstroke}{rgb}{0.200000,0.800000,0.200000}%
\pgfsetstrokecolor{currentstroke}%
\pgfsetdash{}{0pt}%
\pgfpathmoveto{\pgfqpoint{4.829957in}{2.667917in}}%
\pgfpathcurveto{\pgfqpoint{4.835781in}{2.667917in}}{\pgfqpoint{4.841367in}{2.670231in}}{\pgfqpoint{4.845485in}{2.674349in}}%
\pgfpathcurveto{\pgfqpoint{4.849603in}{2.678467in}}{\pgfqpoint{4.851917in}{2.684053in}}{\pgfqpoint{4.851917in}{2.689877in}}%
\pgfpathcurveto{\pgfqpoint{4.851917in}{2.695701in}}{\pgfqpoint{4.849603in}{2.701287in}}{\pgfqpoint{4.845485in}{2.705405in}}%
\pgfpathcurveto{\pgfqpoint{4.841367in}{2.709524in}}{\pgfqpoint{4.835781in}{2.711837in}}{\pgfqpoint{4.829957in}{2.711837in}}%
\pgfpathcurveto{\pgfqpoint{4.824133in}{2.711837in}}{\pgfqpoint{4.818547in}{2.709524in}}{\pgfqpoint{4.814429in}{2.705405in}}%
\pgfpathcurveto{\pgfqpoint{4.810310in}{2.701287in}}{\pgfqpoint{4.807997in}{2.695701in}}{\pgfqpoint{4.807997in}{2.689877in}}%
\pgfpathcurveto{\pgfqpoint{4.807997in}{2.684053in}}{\pgfqpoint{4.810310in}{2.678467in}}{\pgfqpoint{4.814429in}{2.674349in}}%
\pgfpathcurveto{\pgfqpoint{4.818547in}{2.670231in}}{\pgfqpoint{4.824133in}{2.667917in}}{\pgfqpoint{4.829957in}{2.667917in}}%
\pgfpathlineto{\pgfqpoint{4.829957in}{2.667917in}}%
\pgfpathclose%
\pgfusepath{stroke,fill}%
\end{pgfscope}%
\begin{pgfscope}%
\pgfpathrectangle{\pgfqpoint{1.000000in}{0.979904in}}{\pgfqpoint{6.200000in}{5.960192in}}%
\pgfusepath{clip}%
\pgfsetbuttcap%
\pgfsetroundjoin%
\definecolor{currentfill}{rgb}{0.200000,0.800000,0.200000}%
\pgfsetfillcolor{currentfill}%
\pgfsetlinewidth{1.003750pt}%
\definecolor{currentstroke}{rgb}{0.200000,0.800000,0.200000}%
\pgfsetstrokecolor{currentstroke}%
\pgfsetdash{}{0pt}%
\pgfpathmoveto{\pgfqpoint{4.942690in}{2.700025in}}%
\pgfpathcurveto{\pgfqpoint{4.948514in}{2.700025in}}{\pgfqpoint{4.954100in}{2.702339in}}{\pgfqpoint{4.958218in}{2.706457in}}%
\pgfpathcurveto{\pgfqpoint{4.962336in}{2.710575in}}{\pgfqpoint{4.964650in}{2.716161in}}{\pgfqpoint{4.964650in}{2.721985in}}%
\pgfpathcurveto{\pgfqpoint{4.964650in}{2.727809in}}{\pgfqpoint{4.962336in}{2.733395in}}{\pgfqpoint{4.958218in}{2.737513in}}%
\pgfpathcurveto{\pgfqpoint{4.954100in}{2.741632in}}{\pgfqpoint{4.948514in}{2.743945in}}{\pgfqpoint{4.942690in}{2.743945in}}%
\pgfpathcurveto{\pgfqpoint{4.936866in}{2.743945in}}{\pgfqpoint{4.931280in}{2.741632in}}{\pgfqpoint{4.927162in}{2.737513in}}%
\pgfpathcurveto{\pgfqpoint{4.923044in}{2.733395in}}{\pgfqpoint{4.920730in}{2.727809in}}{\pgfqpoint{4.920730in}{2.721985in}}%
\pgfpathcurveto{\pgfqpoint{4.920730in}{2.716161in}}{\pgfqpoint{4.923044in}{2.710575in}}{\pgfqpoint{4.927162in}{2.706457in}}%
\pgfpathcurveto{\pgfqpoint{4.931280in}{2.702339in}}{\pgfqpoint{4.936866in}{2.700025in}}{\pgfqpoint{4.942690in}{2.700025in}}%
\pgfpathlineto{\pgfqpoint{4.942690in}{2.700025in}}%
\pgfpathclose%
\pgfusepath{stroke,fill}%
\end{pgfscope}%
\begin{pgfscope}%
\pgfpathrectangle{\pgfqpoint{1.000000in}{0.979904in}}{\pgfqpoint{6.200000in}{5.960192in}}%
\pgfusepath{clip}%
\pgfsetbuttcap%
\pgfsetroundjoin%
\definecolor{currentfill}{rgb}{0.200000,0.800000,0.200000}%
\pgfsetfillcolor{currentfill}%
\pgfsetlinewidth{1.003750pt}%
\definecolor{currentstroke}{rgb}{0.200000,0.800000,0.200000}%
\pgfsetstrokecolor{currentstroke}%
\pgfsetdash{}{0pt}%
\pgfpathmoveto{\pgfqpoint{5.054932in}{2.652086in}}%
\pgfpathcurveto{\pgfqpoint{5.060756in}{2.652086in}}{\pgfqpoint{5.066342in}{2.654400in}}{\pgfqpoint{5.070461in}{2.658518in}}%
\pgfpathcurveto{\pgfqpoint{5.074579in}{2.662636in}}{\pgfqpoint{5.076893in}{2.668222in}}{\pgfqpoint{5.076893in}{2.674046in}}%
\pgfpathcurveto{\pgfqpoint{5.076893in}{2.679870in}}{\pgfqpoint{5.074579in}{2.685457in}}{\pgfqpoint{5.070461in}{2.689575in}}%
\pgfpathcurveto{\pgfqpoint{5.066342in}{2.693693in}}{\pgfqpoint{5.060756in}{2.696007in}}{\pgfqpoint{5.054932in}{2.696007in}}%
\pgfpathcurveto{\pgfqpoint{5.049108in}{2.696007in}}{\pgfqpoint{5.043522in}{2.693693in}}{\pgfqpoint{5.039404in}{2.689575in}}%
\pgfpathcurveto{\pgfqpoint{5.035286in}{2.685457in}}{\pgfqpoint{5.032972in}{2.679870in}}{\pgfqpoint{5.032972in}{2.674046in}}%
\pgfpathcurveto{\pgfqpoint{5.032972in}{2.668222in}}{\pgfqpoint{5.035286in}{2.662636in}}{\pgfqpoint{5.039404in}{2.658518in}}%
\pgfpathcurveto{\pgfqpoint{5.043522in}{2.654400in}}{\pgfqpoint{5.049108in}{2.652086in}}{\pgfqpoint{5.054932in}{2.652086in}}%
\pgfpathlineto{\pgfqpoint{5.054932in}{2.652086in}}%
\pgfpathclose%
\pgfusepath{stroke,fill}%
\end{pgfscope}%
\begin{pgfscope}%
\pgfpathrectangle{\pgfqpoint{1.000000in}{0.979904in}}{\pgfqpoint{6.200000in}{5.960192in}}%
\pgfusepath{clip}%
\pgfsetbuttcap%
\pgfsetroundjoin%
\definecolor{currentfill}{rgb}{0.200000,0.800000,0.200000}%
\pgfsetfillcolor{currentfill}%
\pgfsetlinewidth{1.003750pt}%
\definecolor{currentstroke}{rgb}{0.200000,0.800000,0.200000}%
\pgfsetstrokecolor{currentstroke}%
\pgfsetdash{}{0pt}%
\pgfpathmoveto{\pgfqpoint{5.166207in}{2.673740in}}%
\pgfpathcurveto{\pgfqpoint{5.172031in}{2.673740in}}{\pgfqpoint{5.177617in}{2.676054in}}{\pgfqpoint{5.181735in}{2.680172in}}%
\pgfpathcurveto{\pgfqpoint{5.185853in}{2.684291in}}{\pgfqpoint{5.188167in}{2.689877in}}{\pgfqpoint{5.188167in}{2.695701in}}%
\pgfpathcurveto{\pgfqpoint{5.188167in}{2.701525in}}{\pgfqpoint{5.185853in}{2.707111in}}{\pgfqpoint{5.181735in}{2.711229in}}%
\pgfpathcurveto{\pgfqpoint{5.177617in}{2.715347in}}{\pgfqpoint{5.172031in}{2.717661in}}{\pgfqpoint{5.166207in}{2.717661in}}%
\pgfpathcurveto{\pgfqpoint{5.160383in}{2.717661in}}{\pgfqpoint{5.154797in}{2.715347in}}{\pgfqpoint{5.150678in}{2.711229in}}%
\pgfpathcurveto{\pgfqpoint{5.146560in}{2.707111in}}{\pgfqpoint{5.144246in}{2.701525in}}{\pgfqpoint{5.144246in}{2.695701in}}%
\pgfpathcurveto{\pgfqpoint{5.144246in}{2.689877in}}{\pgfqpoint{5.146560in}{2.684291in}}{\pgfqpoint{5.150678in}{2.680172in}}%
\pgfpathcurveto{\pgfqpoint{5.154797in}{2.676054in}}{\pgfqpoint{5.160383in}{2.673740in}}{\pgfqpoint{5.166207in}{2.673740in}}%
\pgfpathlineto{\pgfqpoint{5.166207in}{2.673740in}}%
\pgfpathclose%
\pgfusepath{stroke,fill}%
\end{pgfscope}%
\begin{pgfscope}%
\pgfpathrectangle{\pgfqpoint{1.000000in}{0.979904in}}{\pgfqpoint{6.200000in}{5.960192in}}%
\pgfusepath{clip}%
\pgfsetbuttcap%
\pgfsetroundjoin%
\definecolor{currentfill}{rgb}{0.200000,0.800000,0.200000}%
\pgfsetfillcolor{currentfill}%
\pgfsetlinewidth{1.003750pt}%
\definecolor{currentstroke}{rgb}{0.200000,0.800000,0.200000}%
\pgfsetstrokecolor{currentstroke}%
\pgfsetdash{}{0pt}%
\pgfpathmoveto{\pgfqpoint{5.272268in}{2.718245in}}%
\pgfpathcurveto{\pgfqpoint{5.278092in}{2.718245in}}{\pgfqpoint{5.283678in}{2.720559in}}{\pgfqpoint{5.287797in}{2.724677in}}%
\pgfpathcurveto{\pgfqpoint{5.291915in}{2.728796in}}{\pgfqpoint{5.294229in}{2.734382in}}{\pgfqpoint{5.294229in}{2.740206in}}%
\pgfpathcurveto{\pgfqpoint{5.294229in}{2.746030in}}{\pgfqpoint{5.291915in}{2.751616in}}{\pgfqpoint{5.287797in}{2.755734in}}%
\pgfpathcurveto{\pgfqpoint{5.283678in}{2.759852in}}{\pgfqpoint{5.278092in}{2.762166in}}{\pgfqpoint{5.272268in}{2.762166in}}%
\pgfpathcurveto{\pgfqpoint{5.266444in}{2.762166in}}{\pgfqpoint{5.260858in}{2.759852in}}{\pgfqpoint{5.256740in}{2.755734in}}%
\pgfpathcurveto{\pgfqpoint{5.252622in}{2.751616in}}{\pgfqpoint{5.250308in}{2.746030in}}{\pgfqpoint{5.250308in}{2.740206in}}%
\pgfpathcurveto{\pgfqpoint{5.250308in}{2.734382in}}{\pgfqpoint{5.252622in}{2.728796in}}{\pgfqpoint{5.256740in}{2.724677in}}%
\pgfpathcurveto{\pgfqpoint{5.260858in}{2.720559in}}{\pgfqpoint{5.266444in}{2.718245in}}{\pgfqpoint{5.272268in}{2.718245in}}%
\pgfpathlineto{\pgfqpoint{5.272268in}{2.718245in}}%
\pgfpathclose%
\pgfusepath{stroke,fill}%
\end{pgfscope}%
\begin{pgfscope}%
\pgfpathrectangle{\pgfqpoint{1.000000in}{0.979904in}}{\pgfqpoint{6.200000in}{5.960192in}}%
\pgfusepath{clip}%
\pgfsetbuttcap%
\pgfsetroundjoin%
\definecolor{currentfill}{rgb}{0.200000,0.800000,0.200000}%
\pgfsetfillcolor{currentfill}%
\pgfsetlinewidth{1.003750pt}%
\definecolor{currentstroke}{rgb}{0.200000,0.800000,0.200000}%
\pgfsetstrokecolor{currentstroke}%
\pgfsetdash{}{0pt}%
\pgfpathmoveto{\pgfqpoint{5.370644in}{2.780762in}}%
\pgfpathcurveto{\pgfqpoint{5.376468in}{2.780762in}}{\pgfqpoint{5.382054in}{2.783076in}}{\pgfqpoint{5.386172in}{2.787194in}}%
\pgfpathcurveto{\pgfqpoint{5.390290in}{2.791312in}}{\pgfqpoint{5.392604in}{2.796898in}}{\pgfqpoint{5.392604in}{2.802722in}}%
\pgfpathcurveto{\pgfqpoint{5.392604in}{2.808546in}}{\pgfqpoint{5.390290in}{2.814132in}}{\pgfqpoint{5.386172in}{2.818250in}}%
\pgfpathcurveto{\pgfqpoint{5.382054in}{2.822368in}}{\pgfqpoint{5.376468in}{2.824682in}}{\pgfqpoint{5.370644in}{2.824682in}}%
\pgfpathcurveto{\pgfqpoint{5.364820in}{2.824682in}}{\pgfqpoint{5.359234in}{2.822368in}}{\pgfqpoint{5.355115in}{2.818250in}}%
\pgfpathcurveto{\pgfqpoint{5.350997in}{2.814132in}}{\pgfqpoint{5.348683in}{2.808546in}}{\pgfqpoint{5.348683in}{2.802722in}}%
\pgfpathcurveto{\pgfqpoint{5.348683in}{2.796898in}}{\pgfqpoint{5.350997in}{2.791312in}}{\pgfqpoint{5.355115in}{2.787194in}}%
\pgfpathcurveto{\pgfqpoint{5.359234in}{2.783076in}}{\pgfqpoint{5.364820in}{2.780762in}}{\pgfqpoint{5.370644in}{2.780762in}}%
\pgfpathlineto{\pgfqpoint{5.370644in}{2.780762in}}%
\pgfpathclose%
\pgfusepath{stroke,fill}%
\end{pgfscope}%
\begin{pgfscope}%
\pgfpathrectangle{\pgfqpoint{1.000000in}{0.979904in}}{\pgfqpoint{6.200000in}{5.960192in}}%
\pgfusepath{clip}%
\pgfsetbuttcap%
\pgfsetroundjoin%
\definecolor{currentfill}{rgb}{0.200000,0.800000,0.200000}%
\pgfsetfillcolor{currentfill}%
\pgfsetlinewidth{1.003750pt}%
\definecolor{currentstroke}{rgb}{0.200000,0.800000,0.200000}%
\pgfsetstrokecolor{currentstroke}%
\pgfsetdash{}{0pt}%
\pgfpathmoveto{\pgfqpoint{5.464013in}{2.843331in}}%
\pgfpathcurveto{\pgfqpoint{5.469837in}{2.843331in}}{\pgfqpoint{5.475423in}{2.845645in}}{\pgfqpoint{5.479541in}{2.849763in}}%
\pgfpathcurveto{\pgfqpoint{5.483659in}{2.853882in}}{\pgfqpoint{5.485973in}{2.859468in}}{\pgfqpoint{5.485973in}{2.865292in}}%
\pgfpathcurveto{\pgfqpoint{5.485973in}{2.871116in}}{\pgfqpoint{5.483659in}{2.876702in}}{\pgfqpoint{5.479541in}{2.880820in}}%
\pgfpathcurveto{\pgfqpoint{5.475423in}{2.884938in}}{\pgfqpoint{5.469837in}{2.887252in}}{\pgfqpoint{5.464013in}{2.887252in}}%
\pgfpathcurveto{\pgfqpoint{5.458189in}{2.887252in}}{\pgfqpoint{5.452603in}{2.884938in}}{\pgfqpoint{5.448485in}{2.880820in}}%
\pgfpathcurveto{\pgfqpoint{5.444366in}{2.876702in}}{\pgfqpoint{5.442053in}{2.871116in}}{\pgfqpoint{5.442053in}{2.865292in}}%
\pgfpathcurveto{\pgfqpoint{5.442053in}{2.859468in}}{\pgfqpoint{5.444366in}{2.853882in}}{\pgfqpoint{5.448485in}{2.849763in}}%
\pgfpathcurveto{\pgfqpoint{5.452603in}{2.845645in}}{\pgfqpoint{5.458189in}{2.843331in}}{\pgfqpoint{5.464013in}{2.843331in}}%
\pgfpathlineto{\pgfqpoint{5.464013in}{2.843331in}}%
\pgfpathclose%
\pgfusepath{stroke,fill}%
\end{pgfscope}%
\begin{pgfscope}%
\pgfpathrectangle{\pgfqpoint{1.000000in}{0.979904in}}{\pgfqpoint{6.200000in}{5.960192in}}%
\pgfusepath{clip}%
\pgfsetbuttcap%
\pgfsetroundjoin%
\definecolor{currentfill}{rgb}{0.200000,0.800000,0.200000}%
\pgfsetfillcolor{currentfill}%
\pgfsetlinewidth{1.003750pt}%
\definecolor{currentstroke}{rgb}{0.200000,0.800000,0.200000}%
\pgfsetstrokecolor{currentstroke}%
\pgfsetdash{}{0pt}%
\pgfpathmoveto{\pgfqpoint{5.583979in}{2.824901in}}%
\pgfpathcurveto{\pgfqpoint{5.589803in}{2.824901in}}{\pgfqpoint{5.595389in}{2.827215in}}{\pgfqpoint{5.599507in}{2.831333in}}%
\pgfpathcurveto{\pgfqpoint{5.603625in}{2.835451in}}{\pgfqpoint{5.605939in}{2.841038in}}{\pgfqpoint{5.605939in}{2.846862in}}%
\pgfpathcurveto{\pgfqpoint{5.605939in}{2.852685in}}{\pgfqpoint{5.603625in}{2.858272in}}{\pgfqpoint{5.599507in}{2.862390in}}%
\pgfpathcurveto{\pgfqpoint{5.595389in}{2.866508in}}{\pgfqpoint{5.589803in}{2.868822in}}{\pgfqpoint{5.583979in}{2.868822in}}%
\pgfpathcurveto{\pgfqpoint{5.578155in}{2.868822in}}{\pgfqpoint{5.572569in}{2.866508in}}{\pgfqpoint{5.568451in}{2.862390in}}%
\pgfpathcurveto{\pgfqpoint{5.564332in}{2.858272in}}{\pgfqpoint{5.562019in}{2.852685in}}{\pgfqpoint{5.562019in}{2.846862in}}%
\pgfpathcurveto{\pgfqpoint{5.562019in}{2.841038in}}{\pgfqpoint{5.564332in}{2.835451in}}{\pgfqpoint{5.568451in}{2.831333in}}%
\pgfpathcurveto{\pgfqpoint{5.572569in}{2.827215in}}{\pgfqpoint{5.578155in}{2.824901in}}{\pgfqpoint{5.583979in}{2.824901in}}%
\pgfpathlineto{\pgfqpoint{5.583979in}{2.824901in}}%
\pgfpathclose%
\pgfusepath{stroke,fill}%
\end{pgfscope}%
\begin{pgfscope}%
\pgfpathrectangle{\pgfqpoint{1.000000in}{0.979904in}}{\pgfqpoint{6.200000in}{5.960192in}}%
\pgfusepath{clip}%
\pgfsetbuttcap%
\pgfsetroundjoin%
\definecolor{currentfill}{rgb}{0.200000,0.800000,0.200000}%
\pgfsetfillcolor{currentfill}%
\pgfsetlinewidth{1.003750pt}%
\definecolor{currentstroke}{rgb}{0.200000,0.800000,0.200000}%
\pgfsetstrokecolor{currentstroke}%
\pgfsetdash{}{0pt}%
\pgfpathmoveto{\pgfqpoint{5.713360in}{2.804185in}}%
\pgfpathcurveto{\pgfqpoint{5.719184in}{2.804185in}}{\pgfqpoint{5.724770in}{2.806499in}}{\pgfqpoint{5.728888in}{2.810617in}}%
\pgfpathcurveto{\pgfqpoint{5.733007in}{2.814735in}}{\pgfqpoint{5.735320in}{2.820321in}}{\pgfqpoint{5.735320in}{2.826145in}}%
\pgfpathcurveto{\pgfqpoint{5.735320in}{2.831969in}}{\pgfqpoint{5.733007in}{2.837556in}}{\pgfqpoint{5.728888in}{2.841674in}}%
\pgfpathcurveto{\pgfqpoint{5.724770in}{2.845792in}}{\pgfqpoint{5.719184in}{2.848106in}}{\pgfqpoint{5.713360in}{2.848106in}}%
\pgfpathcurveto{\pgfqpoint{5.707536in}{2.848106in}}{\pgfqpoint{5.701950in}{2.845792in}}{\pgfqpoint{5.697832in}{2.841674in}}%
\pgfpathcurveto{\pgfqpoint{5.693714in}{2.837556in}}{\pgfqpoint{5.691400in}{2.831969in}}{\pgfqpoint{5.691400in}{2.826145in}}%
\pgfpathcurveto{\pgfqpoint{5.691400in}{2.820321in}}{\pgfqpoint{5.693714in}{2.814735in}}{\pgfqpoint{5.697832in}{2.810617in}}%
\pgfpathcurveto{\pgfqpoint{5.701950in}{2.806499in}}{\pgfqpoint{5.707536in}{2.804185in}}{\pgfqpoint{5.713360in}{2.804185in}}%
\pgfpathlineto{\pgfqpoint{5.713360in}{2.804185in}}%
\pgfpathclose%
\pgfusepath{stroke,fill}%
\end{pgfscope}%
\begin{pgfscope}%
\pgfpathrectangle{\pgfqpoint{1.000000in}{0.979904in}}{\pgfqpoint{6.200000in}{5.960192in}}%
\pgfusepath{clip}%
\pgfsetbuttcap%
\pgfsetroundjoin%
\definecolor{currentfill}{rgb}{0.200000,0.800000,0.200000}%
\pgfsetfillcolor{currentfill}%
\pgfsetlinewidth{1.003750pt}%
\definecolor{currentstroke}{rgb}{0.200000,0.800000,0.200000}%
\pgfsetstrokecolor{currentstroke}%
\pgfsetdash{}{0pt}%
\pgfpathmoveto{\pgfqpoint{5.798984in}{2.884637in}}%
\pgfpathcurveto{\pgfqpoint{5.804808in}{2.884637in}}{\pgfqpoint{5.810394in}{2.886950in}}{\pgfqpoint{5.814512in}{2.891069in}}%
\pgfpathcurveto{\pgfqpoint{5.818630in}{2.895187in}}{\pgfqpoint{5.820944in}{2.900773in}}{\pgfqpoint{5.820944in}{2.906597in}}%
\pgfpathcurveto{\pgfqpoint{5.820944in}{2.912421in}}{\pgfqpoint{5.818630in}{2.918007in}}{\pgfqpoint{5.814512in}{2.922125in}}%
\pgfpathcurveto{\pgfqpoint{5.810394in}{2.926243in}}{\pgfqpoint{5.804808in}{2.928557in}}{\pgfqpoint{5.798984in}{2.928557in}}%
\pgfpathcurveto{\pgfqpoint{5.793160in}{2.928557in}}{\pgfqpoint{5.787574in}{2.926243in}}{\pgfqpoint{5.783455in}{2.922125in}}%
\pgfpathcurveto{\pgfqpoint{5.779337in}{2.918007in}}{\pgfqpoint{5.777023in}{2.912421in}}{\pgfqpoint{5.777023in}{2.906597in}}%
\pgfpathcurveto{\pgfqpoint{5.777023in}{2.900773in}}{\pgfqpoint{5.779337in}{2.895187in}}{\pgfqpoint{5.783455in}{2.891069in}}%
\pgfpathcurveto{\pgfqpoint{5.787574in}{2.886950in}}{\pgfqpoint{5.793160in}{2.884637in}}{\pgfqpoint{5.798984in}{2.884637in}}%
\pgfpathlineto{\pgfqpoint{5.798984in}{2.884637in}}%
\pgfpathclose%
\pgfusepath{stroke,fill}%
\end{pgfscope}%
\begin{pgfscope}%
\pgfpathrectangle{\pgfqpoint{1.000000in}{0.979904in}}{\pgfqpoint{6.200000in}{5.960192in}}%
\pgfusepath{clip}%
\pgfsetbuttcap%
\pgfsetroundjoin%
\definecolor{currentfill}{rgb}{0.200000,0.800000,0.200000}%
\pgfsetfillcolor{currentfill}%
\pgfsetlinewidth{1.003750pt}%
\definecolor{currentstroke}{rgb}{0.200000,0.800000,0.200000}%
\pgfsetstrokecolor{currentstroke}%
\pgfsetdash{}{0pt}%
\pgfpathmoveto{\pgfqpoint{5.854591in}{3.006132in}}%
\pgfpathcurveto{\pgfqpoint{5.860415in}{3.006132in}}{\pgfqpoint{5.866001in}{3.008446in}}{\pgfqpoint{5.870119in}{3.012564in}}%
\pgfpathcurveto{\pgfqpoint{5.874237in}{3.016683in}}{\pgfqpoint{5.876551in}{3.022269in}}{\pgfqpoint{5.876551in}{3.028093in}}%
\pgfpathcurveto{\pgfqpoint{5.876551in}{3.033917in}}{\pgfqpoint{5.874237in}{3.039503in}}{\pgfqpoint{5.870119in}{3.043621in}}%
\pgfpathcurveto{\pgfqpoint{5.866001in}{3.047739in}}{\pgfqpoint{5.860415in}{3.050053in}}{\pgfqpoint{5.854591in}{3.050053in}}%
\pgfpathcurveto{\pgfqpoint{5.848767in}{3.050053in}}{\pgfqpoint{5.843181in}{3.047739in}}{\pgfqpoint{5.839063in}{3.043621in}}%
\pgfpathcurveto{\pgfqpoint{5.834945in}{3.039503in}}{\pgfqpoint{5.832631in}{3.033917in}}{\pgfqpoint{5.832631in}{3.028093in}}%
\pgfpathcurveto{\pgfqpoint{5.832631in}{3.022269in}}{\pgfqpoint{5.834945in}{3.016683in}}{\pgfqpoint{5.839063in}{3.012564in}}%
\pgfpathcurveto{\pgfqpoint{5.843181in}{3.008446in}}{\pgfqpoint{5.848767in}{3.006132in}}{\pgfqpoint{5.854591in}{3.006132in}}%
\pgfpathlineto{\pgfqpoint{5.854591in}{3.006132in}}%
\pgfpathclose%
\pgfusepath{stroke,fill}%
\end{pgfscope}%
\begin{pgfscope}%
\pgfpathrectangle{\pgfqpoint{1.000000in}{0.979904in}}{\pgfqpoint{6.200000in}{5.960192in}}%
\pgfusepath{clip}%
\pgfsetbuttcap%
\pgfsetroundjoin%
\definecolor{currentfill}{rgb}{0.200000,0.800000,0.200000}%
\pgfsetfillcolor{currentfill}%
\pgfsetlinewidth{1.003750pt}%
\definecolor{currentstroke}{rgb}{0.200000,0.800000,0.200000}%
\pgfsetstrokecolor{currentstroke}%
\pgfsetdash{}{0pt}%
\pgfpathmoveto{\pgfqpoint{5.986885in}{3.003772in}}%
\pgfpathcurveto{\pgfqpoint{5.992709in}{3.003772in}}{\pgfqpoint{5.998295in}{3.006086in}}{\pgfqpoint{6.002413in}{3.010204in}}%
\pgfpathcurveto{\pgfqpoint{6.006531in}{3.014322in}}{\pgfqpoint{6.008845in}{3.019908in}}{\pgfqpoint{6.008845in}{3.025732in}}%
\pgfpathcurveto{\pgfqpoint{6.008845in}{3.031556in}}{\pgfqpoint{6.006531in}{3.037142in}}{\pgfqpoint{6.002413in}{3.041261in}}%
\pgfpathcurveto{\pgfqpoint{5.998295in}{3.045379in}}{\pgfqpoint{5.992709in}{3.047693in}}{\pgfqpoint{5.986885in}{3.047693in}}%
\pgfpathcurveto{\pgfqpoint{5.981061in}{3.047693in}}{\pgfqpoint{5.975474in}{3.045379in}}{\pgfqpoint{5.971356in}{3.041261in}}%
\pgfpathcurveto{\pgfqpoint{5.967238in}{3.037142in}}{\pgfqpoint{5.964924in}{3.031556in}}{\pgfqpoint{5.964924in}{3.025732in}}%
\pgfpathcurveto{\pgfqpoint{5.964924in}{3.019908in}}{\pgfqpoint{5.967238in}{3.014322in}}{\pgfqpoint{5.971356in}{3.010204in}}%
\pgfpathcurveto{\pgfqpoint{5.975474in}{3.006086in}}{\pgfqpoint{5.981061in}{3.003772in}}{\pgfqpoint{5.986885in}{3.003772in}}%
\pgfpathlineto{\pgfqpoint{5.986885in}{3.003772in}}%
\pgfpathclose%
\pgfusepath{stroke,fill}%
\end{pgfscope}%
\begin{pgfscope}%
\pgfpathrectangle{\pgfqpoint{1.000000in}{0.979904in}}{\pgfqpoint{6.200000in}{5.960192in}}%
\pgfusepath{clip}%
\pgfsetbuttcap%
\pgfsetroundjoin%
\definecolor{currentfill}{rgb}{0.200000,0.800000,0.200000}%
\pgfsetfillcolor{currentfill}%
\pgfsetlinewidth{1.003750pt}%
\definecolor{currentstroke}{rgb}{0.200000,0.800000,0.200000}%
\pgfsetstrokecolor{currentstroke}%
\pgfsetdash{}{0pt}%
\pgfpathmoveto{\pgfqpoint{6.082490in}{3.062395in}}%
\pgfpathcurveto{\pgfqpoint{6.088314in}{3.062395in}}{\pgfqpoint{6.093900in}{3.064709in}}{\pgfqpoint{6.098018in}{3.068827in}}%
\pgfpathcurveto{\pgfqpoint{6.102137in}{3.072945in}}{\pgfqpoint{6.104450in}{3.078532in}}{\pgfqpoint{6.104450in}{3.084355in}}%
\pgfpathcurveto{\pgfqpoint{6.104450in}{3.090179in}}{\pgfqpoint{6.102137in}{3.095766in}}{\pgfqpoint{6.098018in}{3.099884in}}%
\pgfpathcurveto{\pgfqpoint{6.093900in}{3.104002in}}{\pgfqpoint{6.088314in}{3.106316in}}{\pgfqpoint{6.082490in}{3.106316in}}%
\pgfpathcurveto{\pgfqpoint{6.076666in}{3.106316in}}{\pgfqpoint{6.071080in}{3.104002in}}{\pgfqpoint{6.066962in}{3.099884in}}%
\pgfpathcurveto{\pgfqpoint{6.062844in}{3.095766in}}{\pgfqpoint{6.060530in}{3.090179in}}{\pgfqpoint{6.060530in}{3.084355in}}%
\pgfpathcurveto{\pgfqpoint{6.060530in}{3.078532in}}{\pgfqpoint{6.062844in}{3.072945in}}{\pgfqpoint{6.066962in}{3.068827in}}%
\pgfpathcurveto{\pgfqpoint{6.071080in}{3.064709in}}{\pgfqpoint{6.076666in}{3.062395in}}{\pgfqpoint{6.082490in}{3.062395in}}%
\pgfpathlineto{\pgfqpoint{6.082490in}{3.062395in}}%
\pgfpathclose%
\pgfusepath{stroke,fill}%
\end{pgfscope}%
\begin{pgfscope}%
\pgfpathrectangle{\pgfqpoint{1.000000in}{0.979904in}}{\pgfqpoint{6.200000in}{5.960192in}}%
\pgfusepath{clip}%
\pgfsetbuttcap%
\pgfsetroundjoin%
\definecolor{currentfill}{rgb}{0.200000,0.800000,0.200000}%
\pgfsetfillcolor{currentfill}%
\pgfsetlinewidth{1.003750pt}%
\definecolor{currentstroke}{rgb}{0.200000,0.800000,0.200000}%
\pgfsetstrokecolor{currentstroke}%
\pgfsetdash{}{0pt}%
\pgfpathmoveto{\pgfqpoint{6.121398in}{3.185166in}}%
\pgfpathcurveto{\pgfqpoint{6.127222in}{3.185166in}}{\pgfqpoint{6.132809in}{3.187480in}}{\pgfqpoint{6.136927in}{3.191598in}}%
\pgfpathcurveto{\pgfqpoint{6.141045in}{3.195716in}}{\pgfqpoint{6.143359in}{3.201302in}}{\pgfqpoint{6.143359in}{3.207126in}}%
\pgfpathcurveto{\pgfqpoint{6.143359in}{3.212950in}}{\pgfqpoint{6.141045in}{3.218536in}}{\pgfqpoint{6.136927in}{3.222654in}}%
\pgfpathcurveto{\pgfqpoint{6.132809in}{3.226772in}}{\pgfqpoint{6.127222in}{3.229086in}}{\pgfqpoint{6.121398in}{3.229086in}}%
\pgfpathcurveto{\pgfqpoint{6.115575in}{3.229086in}}{\pgfqpoint{6.109988in}{3.226772in}}{\pgfqpoint{6.105870in}{3.222654in}}%
\pgfpathcurveto{\pgfqpoint{6.101752in}{3.218536in}}{\pgfqpoint{6.099438in}{3.212950in}}{\pgfqpoint{6.099438in}{3.207126in}}%
\pgfpathcurveto{\pgfqpoint{6.099438in}{3.201302in}}{\pgfqpoint{6.101752in}{3.195716in}}{\pgfqpoint{6.105870in}{3.191598in}}%
\pgfpathcurveto{\pgfqpoint{6.109988in}{3.187480in}}{\pgfqpoint{6.115575in}{3.185166in}}{\pgfqpoint{6.121398in}{3.185166in}}%
\pgfpathlineto{\pgfqpoint{6.121398in}{3.185166in}}%
\pgfpathclose%
\pgfusepath{stroke,fill}%
\end{pgfscope}%
\begin{pgfscope}%
\pgfpathrectangle{\pgfqpoint{1.000000in}{0.979904in}}{\pgfqpoint{6.200000in}{5.960192in}}%
\pgfusepath{clip}%
\pgfsetbuttcap%
\pgfsetroundjoin%
\definecolor{currentfill}{rgb}{0.200000,0.800000,0.200000}%
\pgfsetfillcolor{currentfill}%
\pgfsetlinewidth{1.003750pt}%
\definecolor{currentstroke}{rgb}{0.200000,0.800000,0.200000}%
\pgfsetstrokecolor{currentstroke}%
\pgfsetdash{}{0pt}%
\pgfpathmoveto{\pgfqpoint{6.237974in}{3.222702in}}%
\pgfpathcurveto{\pgfqpoint{6.243797in}{3.222702in}}{\pgfqpoint{6.249384in}{3.225015in}}{\pgfqpoint{6.253502in}{3.229134in}}%
\pgfpathcurveto{\pgfqpoint{6.257620in}{3.233252in}}{\pgfqpoint{6.259934in}{3.238838in}}{\pgfqpoint{6.259934in}{3.244662in}}%
\pgfpathcurveto{\pgfqpoint{6.259934in}{3.250486in}}{\pgfqpoint{6.257620in}{3.256072in}}{\pgfqpoint{6.253502in}{3.260190in}}%
\pgfpathcurveto{\pgfqpoint{6.249384in}{3.264308in}}{\pgfqpoint{6.243797in}{3.266622in}}{\pgfqpoint{6.237974in}{3.266622in}}%
\pgfpathcurveto{\pgfqpoint{6.232150in}{3.266622in}}{\pgfqpoint{6.226563in}{3.264308in}}{\pgfqpoint{6.222445in}{3.260190in}}%
\pgfpathcurveto{\pgfqpoint{6.218327in}{3.256072in}}{\pgfqpoint{6.216013in}{3.250486in}}{\pgfqpoint{6.216013in}{3.244662in}}%
\pgfpathcurveto{\pgfqpoint{6.216013in}{3.238838in}}{\pgfqpoint{6.218327in}{3.233252in}}{\pgfqpoint{6.222445in}{3.229134in}}%
\pgfpathcurveto{\pgfqpoint{6.226563in}{3.225015in}}{\pgfqpoint{6.232150in}{3.222702in}}{\pgfqpoint{6.237974in}{3.222702in}}%
\pgfpathlineto{\pgfqpoint{6.237974in}{3.222702in}}%
\pgfpathclose%
\pgfusepath{stroke,fill}%
\end{pgfscope}%
\begin{pgfscope}%
\pgfpathrectangle{\pgfqpoint{1.000000in}{0.979904in}}{\pgfqpoint{6.200000in}{5.960192in}}%
\pgfusepath{clip}%
\pgfsetbuttcap%
\pgfsetroundjoin%
\definecolor{currentfill}{rgb}{0.200000,0.800000,0.200000}%
\pgfsetfillcolor{currentfill}%
\pgfsetlinewidth{1.003750pt}%
\definecolor{currentstroke}{rgb}{0.200000,0.800000,0.200000}%
\pgfsetstrokecolor{currentstroke}%
\pgfsetdash{}{0pt}%
\pgfpathmoveto{\pgfqpoint{6.308209in}{3.308802in}}%
\pgfpathcurveto{\pgfqpoint{6.314033in}{3.308802in}}{\pgfqpoint{6.319619in}{3.311116in}}{\pgfqpoint{6.323737in}{3.315234in}}%
\pgfpathcurveto{\pgfqpoint{6.327855in}{3.319352in}}{\pgfqpoint{6.330169in}{3.324939in}}{\pgfqpoint{6.330169in}{3.330762in}}%
\pgfpathcurveto{\pgfqpoint{6.330169in}{3.336586in}}{\pgfqpoint{6.327855in}{3.342173in}}{\pgfqpoint{6.323737in}{3.346291in}}%
\pgfpathcurveto{\pgfqpoint{6.319619in}{3.350409in}}{\pgfqpoint{6.314033in}{3.352723in}}{\pgfqpoint{6.308209in}{3.352723in}}%
\pgfpathcurveto{\pgfqpoint{6.302385in}{3.352723in}}{\pgfqpoint{6.296799in}{3.350409in}}{\pgfqpoint{6.292681in}{3.346291in}}%
\pgfpathcurveto{\pgfqpoint{6.288563in}{3.342173in}}{\pgfqpoint{6.286249in}{3.336586in}}{\pgfqpoint{6.286249in}{3.330762in}}%
\pgfpathcurveto{\pgfqpoint{6.286249in}{3.324939in}}{\pgfqpoint{6.288563in}{3.319352in}}{\pgfqpoint{6.292681in}{3.315234in}}%
\pgfpathcurveto{\pgfqpoint{6.296799in}{3.311116in}}{\pgfqpoint{6.302385in}{3.308802in}}{\pgfqpoint{6.308209in}{3.308802in}}%
\pgfpathlineto{\pgfqpoint{6.308209in}{3.308802in}}%
\pgfpathclose%
\pgfusepath{stroke,fill}%
\end{pgfscope}%
\begin{pgfscope}%
\pgfpathrectangle{\pgfqpoint{1.000000in}{0.979904in}}{\pgfqpoint{6.200000in}{5.960192in}}%
\pgfusepath{clip}%
\pgfsetbuttcap%
\pgfsetroundjoin%
\definecolor{currentfill}{rgb}{0.200000,0.800000,0.200000}%
\pgfsetfillcolor{currentfill}%
\pgfsetlinewidth{1.003750pt}%
\definecolor{currentstroke}{rgb}{0.200000,0.800000,0.200000}%
\pgfsetstrokecolor{currentstroke}%
\pgfsetdash{}{0pt}%
\pgfpathmoveto{\pgfqpoint{6.325404in}{3.433713in}}%
\pgfpathcurveto{\pgfqpoint{6.331228in}{3.433713in}}{\pgfqpoint{6.336814in}{3.436027in}}{\pgfqpoint{6.340932in}{3.440145in}}%
\pgfpathcurveto{\pgfqpoint{6.345050in}{3.444263in}}{\pgfqpoint{6.347364in}{3.449849in}}{\pgfqpoint{6.347364in}{3.455673in}}%
\pgfpathcurveto{\pgfqpoint{6.347364in}{3.461497in}}{\pgfqpoint{6.345050in}{3.467083in}}{\pgfqpoint{6.340932in}{3.471201in}}%
\pgfpathcurveto{\pgfqpoint{6.336814in}{3.475320in}}{\pgfqpoint{6.331228in}{3.477633in}}{\pgfqpoint{6.325404in}{3.477633in}}%
\pgfpathcurveto{\pgfqpoint{6.319580in}{3.477633in}}{\pgfqpoint{6.313994in}{3.475320in}}{\pgfqpoint{6.309876in}{3.471201in}}%
\pgfpathcurveto{\pgfqpoint{6.305758in}{3.467083in}}{\pgfqpoint{6.303444in}{3.461497in}}{\pgfqpoint{6.303444in}{3.455673in}}%
\pgfpathcurveto{\pgfqpoint{6.303444in}{3.449849in}}{\pgfqpoint{6.305758in}{3.444263in}}{\pgfqpoint{6.309876in}{3.440145in}}%
\pgfpathcurveto{\pgfqpoint{6.313994in}{3.436027in}}{\pgfqpoint{6.319580in}{3.433713in}}{\pgfqpoint{6.325404in}{3.433713in}}%
\pgfpathlineto{\pgfqpoint{6.325404in}{3.433713in}}%
\pgfpathclose%
\pgfusepath{stroke,fill}%
\end{pgfscope}%
\begin{pgfscope}%
\pgfpathrectangle{\pgfqpoint{1.000000in}{0.979904in}}{\pgfqpoint{6.200000in}{5.960192in}}%
\pgfusepath{clip}%
\pgfsetbuttcap%
\pgfsetroundjoin%
\definecolor{currentfill}{rgb}{0.200000,0.800000,0.200000}%
\pgfsetfillcolor{currentfill}%
\pgfsetlinewidth{1.003750pt}%
\definecolor{currentstroke}{rgb}{0.200000,0.800000,0.200000}%
\pgfsetstrokecolor{currentstroke}%
\pgfsetdash{}{0pt}%
\pgfpathmoveto{\pgfqpoint{6.452947in}{3.478628in}}%
\pgfpathcurveto{\pgfqpoint{6.458771in}{3.478628in}}{\pgfqpoint{6.464358in}{3.480942in}}{\pgfqpoint{6.468476in}{3.485060in}}%
\pgfpathcurveto{\pgfqpoint{6.472594in}{3.489178in}}{\pgfqpoint{6.474908in}{3.494764in}}{\pgfqpoint{6.474908in}{3.500588in}}%
\pgfpathcurveto{\pgfqpoint{6.474908in}{3.506412in}}{\pgfqpoint{6.472594in}{3.511998in}}{\pgfqpoint{6.468476in}{3.516116in}}%
\pgfpathcurveto{\pgfqpoint{6.464358in}{3.520234in}}{\pgfqpoint{6.458771in}{3.522548in}}{\pgfqpoint{6.452947in}{3.522548in}}%
\pgfpathcurveto{\pgfqpoint{6.447124in}{3.522548in}}{\pgfqpoint{6.441537in}{3.520234in}}{\pgfqpoint{6.437419in}{3.516116in}}%
\pgfpathcurveto{\pgfqpoint{6.433301in}{3.511998in}}{\pgfqpoint{6.430987in}{3.506412in}}{\pgfqpoint{6.430987in}{3.500588in}}%
\pgfpathcurveto{\pgfqpoint{6.430987in}{3.494764in}}{\pgfqpoint{6.433301in}{3.489178in}}{\pgfqpoint{6.437419in}{3.485060in}}%
\pgfpathcurveto{\pgfqpoint{6.441537in}{3.480942in}}{\pgfqpoint{6.447124in}{3.478628in}}{\pgfqpoint{6.452947in}{3.478628in}}%
\pgfpathlineto{\pgfqpoint{6.452947in}{3.478628in}}%
\pgfpathclose%
\pgfusepath{stroke,fill}%
\end{pgfscope}%
\begin{pgfscope}%
\pgfpathrectangle{\pgfqpoint{1.000000in}{0.979904in}}{\pgfqpoint{6.200000in}{5.960192in}}%
\pgfusepath{clip}%
\pgfsetbuttcap%
\pgfsetroundjoin%
\definecolor{currentfill}{rgb}{0.200000,0.800000,0.200000}%
\pgfsetfillcolor{currentfill}%
\pgfsetlinewidth{1.003750pt}%
\definecolor{currentstroke}{rgb}{0.200000,0.800000,0.200000}%
\pgfsetstrokecolor{currentstroke}%
\pgfsetdash{}{0pt}%
\pgfpathmoveto{\pgfqpoint{6.519812in}{3.569354in}}%
\pgfpathcurveto{\pgfqpoint{6.525636in}{3.569354in}}{\pgfqpoint{6.531222in}{3.571668in}}{\pgfqpoint{6.535340in}{3.575786in}}%
\pgfpathcurveto{\pgfqpoint{6.539459in}{3.579904in}}{\pgfqpoint{6.541772in}{3.585490in}}{\pgfqpoint{6.541772in}{3.591314in}}%
\pgfpathcurveto{\pgfqpoint{6.541772in}{3.597138in}}{\pgfqpoint{6.539459in}{3.602724in}}{\pgfqpoint{6.535340in}{3.606842in}}%
\pgfpathcurveto{\pgfqpoint{6.531222in}{3.610960in}}{\pgfqpoint{6.525636in}{3.613274in}}{\pgfqpoint{6.519812in}{3.613274in}}%
\pgfpathcurveto{\pgfqpoint{6.513988in}{3.613274in}}{\pgfqpoint{6.508402in}{3.610960in}}{\pgfqpoint{6.504284in}{3.606842in}}%
\pgfpathcurveto{\pgfqpoint{6.500166in}{3.602724in}}{\pgfqpoint{6.497852in}{3.597138in}}{\pgfqpoint{6.497852in}{3.591314in}}%
\pgfpathcurveto{\pgfqpoint{6.497852in}{3.585490in}}{\pgfqpoint{6.500166in}{3.579904in}}{\pgfqpoint{6.504284in}{3.575786in}}%
\pgfpathcurveto{\pgfqpoint{6.508402in}{3.571668in}}{\pgfqpoint{6.513988in}{3.569354in}}{\pgfqpoint{6.519812in}{3.569354in}}%
\pgfpathlineto{\pgfqpoint{6.519812in}{3.569354in}}%
\pgfpathclose%
\pgfusepath{stroke,fill}%
\end{pgfscope}%
\begin{pgfscope}%
\pgfpathrectangle{\pgfqpoint{1.000000in}{0.979904in}}{\pgfqpoint{6.200000in}{5.960192in}}%
\pgfusepath{clip}%
\pgfsetbuttcap%
\pgfsetroundjoin%
\definecolor{currentfill}{rgb}{0.200000,0.800000,0.200000}%
\pgfsetfillcolor{currentfill}%
\pgfsetlinewidth{1.003750pt}%
\definecolor{currentstroke}{rgb}{0.200000,0.800000,0.200000}%
\pgfsetstrokecolor{currentstroke}%
\pgfsetdash{}{0pt}%
\pgfpathmoveto{\pgfqpoint{6.596396in}{3.657465in}}%
\pgfpathcurveto{\pgfqpoint{6.602220in}{3.657465in}}{\pgfqpoint{6.607806in}{3.659779in}}{\pgfqpoint{6.611924in}{3.663897in}}%
\pgfpathcurveto{\pgfqpoint{6.616043in}{3.668015in}}{\pgfqpoint{6.618356in}{3.673601in}}{\pgfqpoint{6.618356in}{3.679425in}}%
\pgfpathcurveto{\pgfqpoint{6.618356in}{3.685249in}}{\pgfqpoint{6.616043in}{3.690835in}}{\pgfqpoint{6.611924in}{3.694953in}}%
\pgfpathcurveto{\pgfqpoint{6.607806in}{3.699071in}}{\pgfqpoint{6.602220in}{3.701385in}}{\pgfqpoint{6.596396in}{3.701385in}}%
\pgfpathcurveto{\pgfqpoint{6.590572in}{3.701385in}}{\pgfqpoint{6.584986in}{3.699071in}}{\pgfqpoint{6.580868in}{3.694953in}}%
\pgfpathcurveto{\pgfqpoint{6.576750in}{3.690835in}}{\pgfqpoint{6.574436in}{3.685249in}}{\pgfqpoint{6.574436in}{3.679425in}}%
\pgfpathcurveto{\pgfqpoint{6.574436in}{3.673601in}}{\pgfqpoint{6.576750in}{3.668015in}}{\pgfqpoint{6.580868in}{3.663897in}}%
\pgfpathcurveto{\pgfqpoint{6.584986in}{3.659779in}}{\pgfqpoint{6.590572in}{3.657465in}}{\pgfqpoint{6.596396in}{3.657465in}}%
\pgfpathlineto{\pgfqpoint{6.596396in}{3.657465in}}%
\pgfpathclose%
\pgfusepath{stroke,fill}%
\end{pgfscope}%
\begin{pgfscope}%
\pgfpathrectangle{\pgfqpoint{1.000000in}{0.979904in}}{\pgfqpoint{6.200000in}{5.960192in}}%
\pgfusepath{clip}%
\pgfsetbuttcap%
\pgfsetroundjoin%
\definecolor{currentfill}{rgb}{0.200000,0.800000,0.200000}%
\pgfsetfillcolor{currentfill}%
\pgfsetlinewidth{1.003750pt}%
\definecolor{currentstroke}{rgb}{0.200000,0.800000,0.200000}%
\pgfsetstrokecolor{currentstroke}%
\pgfsetdash{}{0pt}%
\pgfpathmoveto{\pgfqpoint{6.640777in}{3.762742in}}%
\pgfpathcurveto{\pgfqpoint{6.646601in}{3.762742in}}{\pgfqpoint{6.652187in}{3.765056in}}{\pgfqpoint{6.656305in}{3.769174in}}%
\pgfpathcurveto{\pgfqpoint{6.660423in}{3.773292in}}{\pgfqpoint{6.662737in}{3.778878in}}{\pgfqpoint{6.662737in}{3.784702in}}%
\pgfpathcurveto{\pgfqpoint{6.662737in}{3.790526in}}{\pgfqpoint{6.660423in}{3.796112in}}{\pgfqpoint{6.656305in}{3.800230in}}%
\pgfpathcurveto{\pgfqpoint{6.652187in}{3.804348in}}{\pgfqpoint{6.646601in}{3.806662in}}{\pgfqpoint{6.640777in}{3.806662in}}%
\pgfpathcurveto{\pgfqpoint{6.634953in}{3.806662in}}{\pgfqpoint{6.629367in}{3.804348in}}{\pgfqpoint{6.625249in}{3.800230in}}%
\pgfpathcurveto{\pgfqpoint{6.621131in}{3.796112in}}{\pgfqpoint{6.618817in}{3.790526in}}{\pgfqpoint{6.618817in}{3.784702in}}%
\pgfpathcurveto{\pgfqpoint{6.618817in}{3.778878in}}{\pgfqpoint{6.621131in}{3.773292in}}{\pgfqpoint{6.625249in}{3.769174in}}%
\pgfpathcurveto{\pgfqpoint{6.629367in}{3.765056in}}{\pgfqpoint{6.634953in}{3.762742in}}{\pgfqpoint{6.640777in}{3.762742in}}%
\pgfpathlineto{\pgfqpoint{6.640777in}{3.762742in}}%
\pgfpathclose%
\pgfusepath{stroke,fill}%
\end{pgfscope}%
\begin{pgfscope}%
\pgfpathrectangle{\pgfqpoint{1.000000in}{0.979904in}}{\pgfqpoint{6.200000in}{5.960192in}}%
\pgfusepath{clip}%
\pgfsetbuttcap%
\pgfsetroundjoin%
\definecolor{currentfill}{rgb}{0.200000,0.800000,0.200000}%
\pgfsetfillcolor{currentfill}%
\pgfsetlinewidth{1.003750pt}%
\definecolor{currentstroke}{rgb}{0.200000,0.800000,0.200000}%
\pgfsetstrokecolor{currentstroke}%
\pgfsetdash{}{0pt}%
\pgfpathmoveto{\pgfqpoint{6.557268in}{3.910299in}}%
\pgfpathcurveto{\pgfqpoint{6.563092in}{3.910299in}}{\pgfqpoint{6.568678in}{3.912613in}}{\pgfqpoint{6.572796in}{3.916731in}}%
\pgfpathcurveto{\pgfqpoint{6.576914in}{3.920849in}}{\pgfqpoint{6.579228in}{3.926436in}}{\pgfqpoint{6.579228in}{3.932259in}}%
\pgfpathcurveto{\pgfqpoint{6.579228in}{3.938083in}}{\pgfqpoint{6.576914in}{3.943670in}}{\pgfqpoint{6.572796in}{3.947788in}}%
\pgfpathcurveto{\pgfqpoint{6.568678in}{3.951906in}}{\pgfqpoint{6.563092in}{3.954220in}}{\pgfqpoint{6.557268in}{3.954220in}}%
\pgfpathcurveto{\pgfqpoint{6.551444in}{3.954220in}}{\pgfqpoint{6.545857in}{3.951906in}}{\pgfqpoint{6.541739in}{3.947788in}}%
\pgfpathcurveto{\pgfqpoint{6.537621in}{3.943670in}}{\pgfqpoint{6.535307in}{3.938083in}}{\pgfqpoint{6.535307in}{3.932259in}}%
\pgfpathcurveto{\pgfqpoint{6.535307in}{3.926436in}}{\pgfqpoint{6.537621in}{3.920849in}}{\pgfqpoint{6.541739in}{3.916731in}}%
\pgfpathcurveto{\pgfqpoint{6.545857in}{3.912613in}}{\pgfqpoint{6.551444in}{3.910299in}}{\pgfqpoint{6.557268in}{3.910299in}}%
\pgfpathlineto{\pgfqpoint{6.557268in}{3.910299in}}%
\pgfpathclose%
\pgfusepath{stroke,fill}%
\end{pgfscope}%
\begin{pgfscope}%
\pgfpathrectangle{\pgfqpoint{1.000000in}{0.979904in}}{\pgfqpoint{6.200000in}{5.960192in}}%
\pgfusepath{clip}%
\pgfsetbuttcap%
\pgfsetroundjoin%
\definecolor{currentfill}{rgb}{0.200000,0.800000,0.200000}%
\pgfsetfillcolor{currentfill}%
\pgfsetlinewidth{1.003750pt}%
\definecolor{currentstroke}{rgb}{0.200000,0.800000,0.200000}%
\pgfsetstrokecolor{currentstroke}%
\pgfsetdash{}{0pt}%
\pgfpathmoveto{\pgfqpoint{6.609721in}{4.006148in}}%
\pgfpathcurveto{\pgfqpoint{6.615545in}{4.006148in}}{\pgfqpoint{6.621131in}{4.008462in}}{\pgfqpoint{6.625249in}{4.012580in}}%
\pgfpathcurveto{\pgfqpoint{6.629367in}{4.016698in}}{\pgfqpoint{6.631681in}{4.022284in}}{\pgfqpoint{6.631681in}{4.028108in}}%
\pgfpathcurveto{\pgfqpoint{6.631681in}{4.033932in}}{\pgfqpoint{6.629367in}{4.039518in}}{\pgfqpoint{6.625249in}{4.043636in}}%
\pgfpathcurveto{\pgfqpoint{6.621131in}{4.047755in}}{\pgfqpoint{6.615545in}{4.050068in}}{\pgfqpoint{6.609721in}{4.050068in}}%
\pgfpathcurveto{\pgfqpoint{6.603897in}{4.050068in}}{\pgfqpoint{6.598311in}{4.047755in}}{\pgfqpoint{6.594193in}{4.043636in}}%
\pgfpathcurveto{\pgfqpoint{6.590075in}{4.039518in}}{\pgfqpoint{6.587761in}{4.033932in}}{\pgfqpoint{6.587761in}{4.028108in}}%
\pgfpathcurveto{\pgfqpoint{6.587761in}{4.022284in}}{\pgfqpoint{6.590075in}{4.016698in}}{\pgfqpoint{6.594193in}{4.012580in}}%
\pgfpathcurveto{\pgfqpoint{6.598311in}{4.008462in}}{\pgfqpoint{6.603897in}{4.006148in}}{\pgfqpoint{6.609721in}{4.006148in}}%
\pgfpathlineto{\pgfqpoint{6.609721in}{4.006148in}}%
\pgfpathclose%
\pgfusepath{stroke,fill}%
\end{pgfscope}%
\begin{pgfscope}%
\pgfpathrectangle{\pgfqpoint{1.000000in}{0.979904in}}{\pgfqpoint{6.200000in}{5.960192in}}%
\pgfusepath{clip}%
\pgfsetbuttcap%
\pgfsetroundjoin%
\definecolor{currentfill}{rgb}{0.200000,0.800000,0.200000}%
\pgfsetfillcolor{currentfill}%
\pgfsetlinewidth{1.003750pt}%
\definecolor{currentstroke}{rgb}{0.200000,0.800000,0.200000}%
\pgfsetstrokecolor{currentstroke}%
\pgfsetdash{}{0pt}%
\pgfpathmoveto{\pgfqpoint{6.709092in}{4.096403in}}%
\pgfpathcurveto{\pgfqpoint{6.714916in}{4.096403in}}{\pgfqpoint{6.720502in}{4.098717in}}{\pgfqpoint{6.724620in}{4.102835in}}%
\pgfpathcurveto{\pgfqpoint{6.728738in}{4.106953in}}{\pgfqpoint{6.731052in}{4.112539in}}{\pgfqpoint{6.731052in}{4.118363in}}%
\pgfpathcurveto{\pgfqpoint{6.731052in}{4.124187in}}{\pgfqpoint{6.728738in}{4.129774in}}{\pgfqpoint{6.724620in}{4.133892in}}%
\pgfpathcurveto{\pgfqpoint{6.720502in}{4.138010in}}{\pgfqpoint{6.714916in}{4.140324in}}{\pgfqpoint{6.709092in}{4.140324in}}%
\pgfpathcurveto{\pgfqpoint{6.703268in}{4.140324in}}{\pgfqpoint{6.697682in}{4.138010in}}{\pgfqpoint{6.693564in}{4.133892in}}%
\pgfpathcurveto{\pgfqpoint{6.689445in}{4.129774in}}{\pgfqpoint{6.687132in}{4.124187in}}{\pgfqpoint{6.687132in}{4.118363in}}%
\pgfpathcurveto{\pgfqpoint{6.687132in}{4.112539in}}{\pgfqpoint{6.689445in}{4.106953in}}{\pgfqpoint{6.693564in}{4.102835in}}%
\pgfpathcurveto{\pgfqpoint{6.697682in}{4.098717in}}{\pgfqpoint{6.703268in}{4.096403in}}{\pgfqpoint{6.709092in}{4.096403in}}%
\pgfpathlineto{\pgfqpoint{6.709092in}{4.096403in}}%
\pgfpathclose%
\pgfusepath{stroke,fill}%
\end{pgfscope}%
\begin{pgfscope}%
\pgfpathrectangle{\pgfqpoint{1.000000in}{0.979904in}}{\pgfqpoint{6.200000in}{5.960192in}}%
\pgfusepath{clip}%
\pgfsetbuttcap%
\pgfsetroundjoin%
\definecolor{currentfill}{rgb}{0.200000,0.800000,0.200000}%
\pgfsetfillcolor{currentfill}%
\pgfsetlinewidth{1.003750pt}%
\definecolor{currentstroke}{rgb}{0.200000,0.800000,0.200000}%
\pgfsetstrokecolor{currentstroke}%
\pgfsetdash{}{0pt}%
\pgfpathmoveto{\pgfqpoint{6.726782in}{4.207376in}}%
\pgfpathcurveto{\pgfqpoint{6.732606in}{4.207376in}}{\pgfqpoint{6.738192in}{4.209690in}}{\pgfqpoint{6.742310in}{4.213808in}}%
\pgfpathcurveto{\pgfqpoint{6.746428in}{4.217926in}}{\pgfqpoint{6.748742in}{4.223512in}}{\pgfqpoint{6.748742in}{4.229336in}}%
\pgfpathcurveto{\pgfqpoint{6.748742in}{4.235160in}}{\pgfqpoint{6.746428in}{4.240746in}}{\pgfqpoint{6.742310in}{4.244864in}}%
\pgfpathcurveto{\pgfqpoint{6.738192in}{4.248982in}}{\pgfqpoint{6.732606in}{4.251296in}}{\pgfqpoint{6.726782in}{4.251296in}}%
\pgfpathcurveto{\pgfqpoint{6.720958in}{4.251296in}}{\pgfqpoint{6.715371in}{4.248982in}}{\pgfqpoint{6.711253in}{4.244864in}}%
\pgfpathcurveto{\pgfqpoint{6.707135in}{4.240746in}}{\pgfqpoint{6.704821in}{4.235160in}}{\pgfqpoint{6.704821in}{4.229336in}}%
\pgfpathcurveto{\pgfqpoint{6.704821in}{4.223512in}}{\pgfqpoint{6.707135in}{4.217926in}}{\pgfqpoint{6.711253in}{4.213808in}}%
\pgfpathcurveto{\pgfqpoint{6.715371in}{4.209690in}}{\pgfqpoint{6.720958in}{4.207376in}}{\pgfqpoint{6.726782in}{4.207376in}}%
\pgfpathlineto{\pgfqpoint{6.726782in}{4.207376in}}%
\pgfpathclose%
\pgfusepath{stroke,fill}%
\end{pgfscope}%
\begin{pgfscope}%
\pgfpathrectangle{\pgfqpoint{1.000000in}{0.979904in}}{\pgfqpoint{6.200000in}{5.960192in}}%
\pgfusepath{clip}%
\pgfsetbuttcap%
\pgfsetroundjoin%
\definecolor{currentfill}{rgb}{0.200000,0.800000,0.200000}%
\pgfsetfillcolor{currentfill}%
\pgfsetlinewidth{1.003750pt}%
\definecolor{currentstroke}{rgb}{0.200000,0.800000,0.200000}%
\pgfsetstrokecolor{currentstroke}%
\pgfsetdash{}{0pt}%
\pgfpathmoveto{\pgfqpoint{6.737530in}{4.319234in}}%
\pgfpathcurveto{\pgfqpoint{6.743354in}{4.319234in}}{\pgfqpoint{6.748940in}{4.321548in}}{\pgfqpoint{6.753058in}{4.325666in}}%
\pgfpathcurveto{\pgfqpoint{6.757176in}{4.329785in}}{\pgfqpoint{6.759490in}{4.335371in}}{\pgfqpoint{6.759490in}{4.341195in}}%
\pgfpathcurveto{\pgfqpoint{6.759490in}{4.347019in}}{\pgfqpoint{6.757176in}{4.352605in}}{\pgfqpoint{6.753058in}{4.356723in}}%
\pgfpathcurveto{\pgfqpoint{6.748940in}{4.360841in}}{\pgfqpoint{6.743354in}{4.363155in}}{\pgfqpoint{6.737530in}{4.363155in}}%
\pgfpathcurveto{\pgfqpoint{6.731706in}{4.363155in}}{\pgfqpoint{6.726120in}{4.360841in}}{\pgfqpoint{6.722002in}{4.356723in}}%
\pgfpathcurveto{\pgfqpoint{6.717884in}{4.352605in}}{\pgfqpoint{6.715570in}{4.347019in}}{\pgfqpoint{6.715570in}{4.341195in}}%
\pgfpathcurveto{\pgfqpoint{6.715570in}{4.335371in}}{\pgfqpoint{6.717884in}{4.329785in}}{\pgfqpoint{6.722002in}{4.325666in}}%
\pgfpathcurveto{\pgfqpoint{6.726120in}{4.321548in}}{\pgfqpoint{6.731706in}{4.319234in}}{\pgfqpoint{6.737530in}{4.319234in}}%
\pgfpathlineto{\pgfqpoint{6.737530in}{4.319234in}}%
\pgfpathclose%
\pgfusepath{stroke,fill}%
\end{pgfscope}%
\begin{pgfscope}%
\pgfpathrectangle{\pgfqpoint{1.000000in}{0.979904in}}{\pgfqpoint{6.200000in}{5.960192in}}%
\pgfusepath{clip}%
\pgfsetbuttcap%
\pgfsetroundjoin%
\definecolor{currentfill}{rgb}{0.200000,0.800000,0.200000}%
\pgfsetfillcolor{currentfill}%
\pgfsetlinewidth{1.003750pt}%
\definecolor{currentstroke}{rgb}{0.200000,0.800000,0.200000}%
\pgfsetstrokecolor{currentstroke}%
\pgfsetdash{}{0pt}%
\pgfpathmoveto{\pgfqpoint{6.918182in}{4.431554in}}%
\pgfpathcurveto{\pgfqpoint{6.924006in}{4.431554in}}{\pgfqpoint{6.929592in}{4.433868in}}{\pgfqpoint{6.933710in}{4.437986in}}%
\pgfpathcurveto{\pgfqpoint{6.937828in}{4.442104in}}{\pgfqpoint{6.940142in}{4.447690in}}{\pgfqpoint{6.940142in}{4.453514in}}%
\pgfpathcurveto{\pgfqpoint{6.940142in}{4.459338in}}{\pgfqpoint{6.937828in}{4.464924in}}{\pgfqpoint{6.933710in}{4.469042in}}%
\pgfpathcurveto{\pgfqpoint{6.929592in}{4.473160in}}{\pgfqpoint{6.924006in}{4.475474in}}{\pgfqpoint{6.918182in}{4.475474in}}%
\pgfpathcurveto{\pgfqpoint{6.912358in}{4.475474in}}{\pgfqpoint{6.906772in}{4.473160in}}{\pgfqpoint{6.902654in}{4.469042in}}%
\pgfpathcurveto{\pgfqpoint{6.898535in}{4.464924in}}{\pgfqpoint{6.896222in}{4.459338in}}{\pgfqpoint{6.896222in}{4.453514in}}%
\pgfpathcurveto{\pgfqpoint{6.896222in}{4.447690in}}{\pgfqpoint{6.898535in}{4.442104in}}{\pgfqpoint{6.902654in}{4.437986in}}%
\pgfpathcurveto{\pgfqpoint{6.906772in}{4.433868in}}{\pgfqpoint{6.912358in}{4.431554in}}{\pgfqpoint{6.918182in}{4.431554in}}%
\pgfpathlineto{\pgfqpoint{6.918182in}{4.431554in}}%
\pgfpathclose%
\pgfusepath{stroke,fill}%
\end{pgfscope}%
\begin{pgfscope}%
\pgfpathrectangle{\pgfqpoint{1.000000in}{0.979904in}}{\pgfqpoint{6.200000in}{5.960192in}}%
\pgfusepath{clip}%
\pgfsetbuttcap%
\pgfsetroundjoin%
\definecolor{currentfill}{rgb}{0.800000,0.200000,0.200000}%
\pgfsetfillcolor{currentfill}%
\pgfsetlinewidth{1.003750pt}%
\definecolor{currentstroke}{rgb}{0.800000,0.200000,0.200000}%
\pgfsetstrokecolor{currentstroke}%
\pgfsetdash{}{0pt}%
\pgfpathmoveto{\pgfqpoint{4.593682in}{2.865802in}}%
\pgfpathcurveto{\pgfqpoint{4.599506in}{2.865802in}}{\pgfqpoint{4.605092in}{2.868116in}}{\pgfqpoint{4.609210in}{2.872234in}}%
\pgfpathcurveto{\pgfqpoint{4.613328in}{2.876352in}}{\pgfqpoint{4.615642in}{2.881939in}}{\pgfqpoint{4.615642in}{2.887762in}}%
\pgfpathcurveto{\pgfqpoint{4.615642in}{2.893586in}}{\pgfqpoint{4.613328in}{2.899173in}}{\pgfqpoint{4.609210in}{2.903291in}}%
\pgfpathcurveto{\pgfqpoint{4.605092in}{2.907409in}}{\pgfqpoint{4.599506in}{2.909723in}}{\pgfqpoint{4.593682in}{2.909723in}}%
\pgfpathcurveto{\pgfqpoint{4.587858in}{2.909723in}}{\pgfqpoint{4.582272in}{2.907409in}}{\pgfqpoint{4.578154in}{2.903291in}}%
\pgfpathcurveto{\pgfqpoint{4.574036in}{2.899173in}}{\pgfqpoint{4.571722in}{2.893586in}}{\pgfqpoint{4.571722in}{2.887762in}}%
\pgfpathcurveto{\pgfqpoint{4.571722in}{2.881939in}}{\pgfqpoint{4.574036in}{2.876352in}}{\pgfqpoint{4.578154in}{2.872234in}}%
\pgfpathcurveto{\pgfqpoint{4.582272in}{2.868116in}}{\pgfqpoint{4.587858in}{2.865802in}}{\pgfqpoint{4.593682in}{2.865802in}}%
\pgfpathlineto{\pgfqpoint{4.593682in}{2.865802in}}%
\pgfpathclose%
\pgfusepath{stroke,fill}%
\end{pgfscope}%
\begin{pgfscope}%
\pgfpathrectangle{\pgfqpoint{1.000000in}{0.979904in}}{\pgfqpoint{6.200000in}{5.960192in}}%
\pgfusepath{clip}%
\pgfsetbuttcap%
\pgfsetroundjoin%
\definecolor{currentfill}{rgb}{0.800000,0.200000,0.200000}%
\pgfsetfillcolor{currentfill}%
\pgfsetlinewidth{1.003750pt}%
\definecolor{currentstroke}{rgb}{0.800000,0.200000,0.200000}%
\pgfsetstrokecolor{currentstroke}%
\pgfsetdash{}{0pt}%
\pgfpathmoveto{\pgfqpoint{4.572193in}{2.969984in}}%
\pgfpathcurveto{\pgfqpoint{4.578017in}{2.969984in}}{\pgfqpoint{4.583603in}{2.972298in}}{\pgfqpoint{4.587722in}{2.976416in}}%
\pgfpathcurveto{\pgfqpoint{4.591840in}{2.980534in}}{\pgfqpoint{4.594154in}{2.986120in}}{\pgfqpoint{4.594154in}{2.991944in}}%
\pgfpathcurveto{\pgfqpoint{4.594154in}{2.997768in}}{\pgfqpoint{4.591840in}{3.003354in}}{\pgfqpoint{4.587722in}{3.007472in}}%
\pgfpathcurveto{\pgfqpoint{4.583603in}{3.011590in}}{\pgfqpoint{4.578017in}{3.013904in}}{\pgfqpoint{4.572193in}{3.013904in}}%
\pgfpathcurveto{\pgfqpoint{4.566369in}{3.013904in}}{\pgfqpoint{4.560783in}{3.011590in}}{\pgfqpoint{4.556665in}{3.007472in}}%
\pgfpathcurveto{\pgfqpoint{4.552547in}{3.003354in}}{\pgfqpoint{4.550233in}{2.997768in}}{\pgfqpoint{4.550233in}{2.991944in}}%
\pgfpathcurveto{\pgfqpoint{4.550233in}{2.986120in}}{\pgfqpoint{4.552547in}{2.980534in}}{\pgfqpoint{4.556665in}{2.976416in}}%
\pgfpathcurveto{\pgfqpoint{4.560783in}{2.972298in}}{\pgfqpoint{4.566369in}{2.969984in}}{\pgfqpoint{4.572193in}{2.969984in}}%
\pgfpathlineto{\pgfqpoint{4.572193in}{2.969984in}}%
\pgfpathclose%
\pgfusepath{stroke,fill}%
\end{pgfscope}%
\begin{pgfscope}%
\pgfpathrectangle{\pgfqpoint{1.000000in}{0.979904in}}{\pgfqpoint{6.200000in}{5.960192in}}%
\pgfusepath{clip}%
\pgfsetbuttcap%
\pgfsetroundjoin%
\definecolor{currentfill}{rgb}{0.800000,0.200000,0.200000}%
\pgfsetfillcolor{currentfill}%
\pgfsetlinewidth{1.003750pt}%
\definecolor{currentstroke}{rgb}{0.800000,0.200000,0.200000}%
\pgfsetstrokecolor{currentstroke}%
\pgfsetdash{}{0pt}%
\pgfpathmoveto{\pgfqpoint{4.522502in}{3.068669in}}%
\pgfpathcurveto{\pgfqpoint{4.528326in}{3.068669in}}{\pgfqpoint{4.533912in}{3.070983in}}{\pgfqpoint{4.538030in}{3.075101in}}%
\pgfpathcurveto{\pgfqpoint{4.542148in}{3.079219in}}{\pgfqpoint{4.544462in}{3.084805in}}{\pgfqpoint{4.544462in}{3.090629in}}%
\pgfpathcurveto{\pgfqpoint{4.544462in}{3.096453in}}{\pgfqpoint{4.542148in}{3.102039in}}{\pgfqpoint{4.538030in}{3.106157in}}%
\pgfpathcurveto{\pgfqpoint{4.533912in}{3.110275in}}{\pgfqpoint{4.528326in}{3.112589in}}{\pgfqpoint{4.522502in}{3.112589in}}%
\pgfpathcurveto{\pgfqpoint{4.516678in}{3.112589in}}{\pgfqpoint{4.511092in}{3.110275in}}{\pgfqpoint{4.506974in}{3.106157in}}%
\pgfpathcurveto{\pgfqpoint{4.502856in}{3.102039in}}{\pgfqpoint{4.500542in}{3.096453in}}{\pgfqpoint{4.500542in}{3.090629in}}%
\pgfpathcurveto{\pgfqpoint{4.500542in}{3.084805in}}{\pgfqpoint{4.502856in}{3.079219in}}{\pgfqpoint{4.506974in}{3.075101in}}%
\pgfpathcurveto{\pgfqpoint{4.511092in}{3.070983in}}{\pgfqpoint{4.516678in}{3.068669in}}{\pgfqpoint{4.522502in}{3.068669in}}%
\pgfpathlineto{\pgfqpoint{4.522502in}{3.068669in}}%
\pgfpathclose%
\pgfusepath{stroke,fill}%
\end{pgfscope}%
\begin{pgfscope}%
\pgfpathrectangle{\pgfqpoint{1.000000in}{0.979904in}}{\pgfqpoint{6.200000in}{5.960192in}}%
\pgfusepath{clip}%
\pgfsetbuttcap%
\pgfsetroundjoin%
\definecolor{currentfill}{rgb}{0.800000,0.200000,0.200000}%
\pgfsetfillcolor{currentfill}%
\pgfsetlinewidth{1.003750pt}%
\definecolor{currentstroke}{rgb}{0.800000,0.200000,0.200000}%
\pgfsetstrokecolor{currentstroke}%
\pgfsetdash{}{0pt}%
\pgfpathmoveto{\pgfqpoint{4.503512in}{3.168517in}}%
\pgfpathcurveto{\pgfqpoint{4.509336in}{3.168517in}}{\pgfqpoint{4.514922in}{3.170831in}}{\pgfqpoint{4.519040in}{3.174949in}}%
\pgfpathcurveto{\pgfqpoint{4.523158in}{3.179067in}}{\pgfqpoint{4.525472in}{3.184653in}}{\pgfqpoint{4.525472in}{3.190477in}}%
\pgfpathcurveto{\pgfqpoint{4.525472in}{3.196301in}}{\pgfqpoint{4.523158in}{3.201888in}}{\pgfqpoint{4.519040in}{3.206006in}}%
\pgfpathcurveto{\pgfqpoint{4.514922in}{3.210124in}}{\pgfqpoint{4.509336in}{3.212438in}}{\pgfqpoint{4.503512in}{3.212438in}}%
\pgfpathcurveto{\pgfqpoint{4.497688in}{3.212438in}}{\pgfqpoint{4.492101in}{3.210124in}}{\pgfqpoint{4.487983in}{3.206006in}}%
\pgfpathcurveto{\pgfqpoint{4.483865in}{3.201888in}}{\pgfqpoint{4.481551in}{3.196301in}}{\pgfqpoint{4.481551in}{3.190477in}}%
\pgfpathcurveto{\pgfqpoint{4.481551in}{3.184653in}}{\pgfqpoint{4.483865in}{3.179067in}}{\pgfqpoint{4.487983in}{3.174949in}}%
\pgfpathcurveto{\pgfqpoint{4.492101in}{3.170831in}}{\pgfqpoint{4.497688in}{3.168517in}}{\pgfqpoint{4.503512in}{3.168517in}}%
\pgfpathlineto{\pgfqpoint{4.503512in}{3.168517in}}%
\pgfpathclose%
\pgfusepath{stroke,fill}%
\end{pgfscope}%
\begin{pgfscope}%
\pgfpathrectangle{\pgfqpoint{1.000000in}{0.979904in}}{\pgfqpoint{6.200000in}{5.960192in}}%
\pgfusepath{clip}%
\pgfsetbuttcap%
\pgfsetroundjoin%
\definecolor{currentfill}{rgb}{0.800000,0.200000,0.200000}%
\pgfsetfillcolor{currentfill}%
\pgfsetlinewidth{1.003750pt}%
\definecolor{currentstroke}{rgb}{0.800000,0.200000,0.200000}%
\pgfsetstrokecolor{currentstroke}%
\pgfsetdash{}{0pt}%
\pgfpathmoveto{\pgfqpoint{4.412911in}{3.249818in}}%
\pgfpathcurveto{\pgfqpoint{4.418735in}{3.249818in}}{\pgfqpoint{4.424321in}{3.252132in}}{\pgfqpoint{4.428439in}{3.256250in}}%
\pgfpathcurveto{\pgfqpoint{4.432557in}{3.260368in}}{\pgfqpoint{4.434871in}{3.265954in}}{\pgfqpoint{4.434871in}{3.271778in}}%
\pgfpathcurveto{\pgfqpoint{4.434871in}{3.277602in}}{\pgfqpoint{4.432557in}{3.283188in}}{\pgfqpoint{4.428439in}{3.287306in}}%
\pgfpathcurveto{\pgfqpoint{4.424321in}{3.291424in}}{\pgfqpoint{4.418735in}{3.293738in}}{\pgfqpoint{4.412911in}{3.293738in}}%
\pgfpathcurveto{\pgfqpoint{4.407087in}{3.293738in}}{\pgfqpoint{4.401501in}{3.291424in}}{\pgfqpoint{4.397383in}{3.287306in}}%
\pgfpathcurveto{\pgfqpoint{4.393264in}{3.283188in}}{\pgfqpoint{4.390950in}{3.277602in}}{\pgfqpoint{4.390950in}{3.271778in}}%
\pgfpathcurveto{\pgfqpoint{4.390950in}{3.265954in}}{\pgfqpoint{4.393264in}{3.260368in}}{\pgfqpoint{4.397383in}{3.256250in}}%
\pgfpathcurveto{\pgfqpoint{4.401501in}{3.252132in}}{\pgfqpoint{4.407087in}{3.249818in}}{\pgfqpoint{4.412911in}{3.249818in}}%
\pgfpathlineto{\pgfqpoint{4.412911in}{3.249818in}}%
\pgfpathclose%
\pgfusepath{stroke,fill}%
\end{pgfscope}%
\begin{pgfscope}%
\pgfpathrectangle{\pgfqpoint{1.000000in}{0.979904in}}{\pgfqpoint{6.200000in}{5.960192in}}%
\pgfusepath{clip}%
\pgfsetbuttcap%
\pgfsetroundjoin%
\definecolor{currentfill}{rgb}{0.800000,0.200000,0.200000}%
\pgfsetfillcolor{currentfill}%
\pgfsetlinewidth{1.003750pt}%
\definecolor{currentstroke}{rgb}{0.800000,0.200000,0.200000}%
\pgfsetstrokecolor{currentstroke}%
\pgfsetdash{}{0pt}%
\pgfpathmoveto{\pgfqpoint{4.368023in}{3.337150in}}%
\pgfpathcurveto{\pgfqpoint{4.373847in}{3.337150in}}{\pgfqpoint{4.379433in}{3.339463in}}{\pgfqpoint{4.383551in}{3.343582in}}%
\pgfpathcurveto{\pgfqpoint{4.387670in}{3.347700in}}{\pgfqpoint{4.389983in}{3.353286in}}{\pgfqpoint{4.389983in}{3.359110in}}%
\pgfpathcurveto{\pgfqpoint{4.389983in}{3.364934in}}{\pgfqpoint{4.387670in}{3.370520in}}{\pgfqpoint{4.383551in}{3.374638in}}%
\pgfpathcurveto{\pgfqpoint{4.379433in}{3.378756in}}{\pgfqpoint{4.373847in}{3.381070in}}{\pgfqpoint{4.368023in}{3.381070in}}%
\pgfpathcurveto{\pgfqpoint{4.362199in}{3.381070in}}{\pgfqpoint{4.356613in}{3.378756in}}{\pgfqpoint{4.352495in}{3.374638in}}%
\pgfpathcurveto{\pgfqpoint{4.348377in}{3.370520in}}{\pgfqpoint{4.346063in}{3.364934in}}{\pgfqpoint{4.346063in}{3.359110in}}%
\pgfpathcurveto{\pgfqpoint{4.346063in}{3.353286in}}{\pgfqpoint{4.348377in}{3.347700in}}{\pgfqpoint{4.352495in}{3.343582in}}%
\pgfpathcurveto{\pgfqpoint{4.356613in}{3.339463in}}{\pgfqpoint{4.362199in}{3.337150in}}{\pgfqpoint{4.368023in}{3.337150in}}%
\pgfpathlineto{\pgfqpoint{4.368023in}{3.337150in}}%
\pgfpathclose%
\pgfusepath{stroke,fill}%
\end{pgfscope}%
\begin{pgfscope}%
\pgfpathrectangle{\pgfqpoint{1.000000in}{0.979904in}}{\pgfqpoint{6.200000in}{5.960192in}}%
\pgfusepath{clip}%
\pgfsetbuttcap%
\pgfsetroundjoin%
\definecolor{currentfill}{rgb}{0.800000,0.200000,0.200000}%
\pgfsetfillcolor{currentfill}%
\pgfsetlinewidth{1.003750pt}%
\definecolor{currentstroke}{rgb}{0.800000,0.200000,0.200000}%
\pgfsetstrokecolor{currentstroke}%
\pgfsetdash{}{0pt}%
\pgfpathmoveto{\pgfqpoint{4.417289in}{3.460071in}}%
\pgfpathcurveto{\pgfqpoint{4.423113in}{3.460071in}}{\pgfqpoint{4.428700in}{3.462385in}}{\pgfqpoint{4.432818in}{3.466503in}}%
\pgfpathcurveto{\pgfqpoint{4.436936in}{3.470621in}}{\pgfqpoint{4.439250in}{3.476207in}}{\pgfqpoint{4.439250in}{3.482031in}}%
\pgfpathcurveto{\pgfqpoint{4.439250in}{3.487855in}}{\pgfqpoint{4.436936in}{3.493441in}}{\pgfqpoint{4.432818in}{3.497560in}}%
\pgfpathcurveto{\pgfqpoint{4.428700in}{3.501678in}}{\pgfqpoint{4.423113in}{3.503992in}}{\pgfqpoint{4.417289in}{3.503992in}}%
\pgfpathcurveto{\pgfqpoint{4.411465in}{3.503992in}}{\pgfqpoint{4.405879in}{3.501678in}}{\pgfqpoint{4.401761in}{3.497560in}}%
\pgfpathcurveto{\pgfqpoint{4.397643in}{3.493441in}}{\pgfqpoint{4.395329in}{3.487855in}}{\pgfqpoint{4.395329in}{3.482031in}}%
\pgfpathcurveto{\pgfqpoint{4.395329in}{3.476207in}}{\pgfqpoint{4.397643in}{3.470621in}}{\pgfqpoint{4.401761in}{3.466503in}}%
\pgfpathcurveto{\pgfqpoint{4.405879in}{3.462385in}}{\pgfqpoint{4.411465in}{3.460071in}}{\pgfqpoint{4.417289in}{3.460071in}}%
\pgfpathlineto{\pgfqpoint{4.417289in}{3.460071in}}%
\pgfpathclose%
\pgfusepath{stroke,fill}%
\end{pgfscope}%
\begin{pgfscope}%
\pgfpathrectangle{\pgfqpoint{1.000000in}{0.979904in}}{\pgfqpoint{6.200000in}{5.960192in}}%
\pgfusepath{clip}%
\pgfsetbuttcap%
\pgfsetroundjoin%
\definecolor{currentfill}{rgb}{0.800000,0.200000,0.200000}%
\pgfsetfillcolor{currentfill}%
\pgfsetlinewidth{1.003750pt}%
\definecolor{currentstroke}{rgb}{0.800000,0.200000,0.200000}%
\pgfsetstrokecolor{currentstroke}%
\pgfsetdash{}{0pt}%
\pgfpathmoveto{\pgfqpoint{4.327700in}{3.529741in}}%
\pgfpathcurveto{\pgfqpoint{4.333524in}{3.529741in}}{\pgfqpoint{4.339110in}{3.532054in}}{\pgfqpoint{4.343228in}{3.536173in}}%
\pgfpathcurveto{\pgfqpoint{4.347346in}{3.540291in}}{\pgfqpoint{4.349660in}{3.545877in}}{\pgfqpoint{4.349660in}{3.551701in}}%
\pgfpathcurveto{\pgfqpoint{4.349660in}{3.557525in}}{\pgfqpoint{4.347346in}{3.563111in}}{\pgfqpoint{4.343228in}{3.567229in}}%
\pgfpathcurveto{\pgfqpoint{4.339110in}{3.571347in}}{\pgfqpoint{4.333524in}{3.573661in}}{\pgfqpoint{4.327700in}{3.573661in}}%
\pgfpathcurveto{\pgfqpoint{4.321876in}{3.573661in}}{\pgfqpoint{4.316290in}{3.571347in}}{\pgfqpoint{4.312172in}{3.567229in}}%
\pgfpathcurveto{\pgfqpoint{4.308054in}{3.563111in}}{\pgfqpoint{4.305740in}{3.557525in}}{\pgfqpoint{4.305740in}{3.551701in}}%
\pgfpathcurveto{\pgfqpoint{4.305740in}{3.545877in}}{\pgfqpoint{4.308054in}{3.540291in}}{\pgfqpoint{4.312172in}{3.536173in}}%
\pgfpathcurveto{\pgfqpoint{4.316290in}{3.532054in}}{\pgfqpoint{4.321876in}{3.529741in}}{\pgfqpoint{4.327700in}{3.529741in}}%
\pgfpathlineto{\pgfqpoint{4.327700in}{3.529741in}}%
\pgfpathclose%
\pgfusepath{stroke,fill}%
\end{pgfscope}%
\begin{pgfscope}%
\pgfpathrectangle{\pgfqpoint{1.000000in}{0.979904in}}{\pgfqpoint{6.200000in}{5.960192in}}%
\pgfusepath{clip}%
\pgfsetbuttcap%
\pgfsetroundjoin%
\definecolor{currentfill}{rgb}{0.800000,0.200000,0.200000}%
\pgfsetfillcolor{currentfill}%
\pgfsetlinewidth{1.003750pt}%
\definecolor{currentstroke}{rgb}{0.800000,0.200000,0.200000}%
\pgfsetstrokecolor{currentstroke}%
\pgfsetdash{}{0pt}%
\pgfpathmoveto{\pgfqpoint{4.320628in}{3.637926in}}%
\pgfpathcurveto{\pgfqpoint{4.326452in}{3.637926in}}{\pgfqpoint{4.332038in}{3.640240in}}{\pgfqpoint{4.336157in}{3.644358in}}%
\pgfpathcurveto{\pgfqpoint{4.340275in}{3.648476in}}{\pgfqpoint{4.342589in}{3.654062in}}{\pgfqpoint{4.342589in}{3.659886in}}%
\pgfpathcurveto{\pgfqpoint{4.342589in}{3.665710in}}{\pgfqpoint{4.340275in}{3.671296in}}{\pgfqpoint{4.336157in}{3.675415in}}%
\pgfpathcurveto{\pgfqpoint{4.332038in}{3.679533in}}{\pgfqpoint{4.326452in}{3.681847in}}{\pgfqpoint{4.320628in}{3.681847in}}%
\pgfpathcurveto{\pgfqpoint{4.314804in}{3.681847in}}{\pgfqpoint{4.309218in}{3.679533in}}{\pgfqpoint{4.305100in}{3.675415in}}%
\pgfpathcurveto{\pgfqpoint{4.300982in}{3.671296in}}{\pgfqpoint{4.298668in}{3.665710in}}{\pgfqpoint{4.298668in}{3.659886in}}%
\pgfpathcurveto{\pgfqpoint{4.298668in}{3.654062in}}{\pgfqpoint{4.300982in}{3.648476in}}{\pgfqpoint{4.305100in}{3.644358in}}%
\pgfpathcurveto{\pgfqpoint{4.309218in}{3.640240in}}{\pgfqpoint{4.314804in}{3.637926in}}{\pgfqpoint{4.320628in}{3.637926in}}%
\pgfpathlineto{\pgfqpoint{4.320628in}{3.637926in}}%
\pgfpathclose%
\pgfusepath{stroke,fill}%
\end{pgfscope}%
\begin{pgfscope}%
\pgfpathrectangle{\pgfqpoint{1.000000in}{0.979904in}}{\pgfqpoint{6.200000in}{5.960192in}}%
\pgfusepath{clip}%
\pgfsetbuttcap%
\pgfsetroundjoin%
\definecolor{currentfill}{rgb}{0.800000,0.200000,0.200000}%
\pgfsetfillcolor{currentfill}%
\pgfsetlinewidth{1.003750pt}%
\definecolor{currentstroke}{rgb}{0.800000,0.200000,0.200000}%
\pgfsetstrokecolor{currentstroke}%
\pgfsetdash{}{0pt}%
\pgfpathmoveto{\pgfqpoint{4.246768in}{3.710188in}}%
\pgfpathcurveto{\pgfqpoint{4.252592in}{3.710188in}}{\pgfqpoint{4.258178in}{3.712502in}}{\pgfqpoint{4.262296in}{3.716621in}}%
\pgfpathcurveto{\pgfqpoint{4.266415in}{3.720739in}}{\pgfqpoint{4.268728in}{3.726325in}}{\pgfqpoint{4.268728in}{3.732149in}}%
\pgfpathcurveto{\pgfqpoint{4.268728in}{3.737973in}}{\pgfqpoint{4.266415in}{3.743559in}}{\pgfqpoint{4.262296in}{3.747677in}}%
\pgfpathcurveto{\pgfqpoint{4.258178in}{3.751795in}}{\pgfqpoint{4.252592in}{3.754109in}}{\pgfqpoint{4.246768in}{3.754109in}}%
\pgfpathcurveto{\pgfqpoint{4.240944in}{3.754109in}}{\pgfqpoint{4.235358in}{3.751795in}}{\pgfqpoint{4.231240in}{3.747677in}}%
\pgfpathcurveto{\pgfqpoint{4.227122in}{3.743559in}}{\pgfqpoint{4.224808in}{3.737973in}}{\pgfqpoint{4.224808in}{3.732149in}}%
\pgfpathcurveto{\pgfqpoint{4.224808in}{3.726325in}}{\pgfqpoint{4.227122in}{3.720739in}}{\pgfqpoint{4.231240in}{3.716621in}}%
\pgfpathcurveto{\pgfqpoint{4.235358in}{3.712502in}}{\pgfqpoint{4.240944in}{3.710188in}}{\pgfqpoint{4.246768in}{3.710188in}}%
\pgfpathlineto{\pgfqpoint{4.246768in}{3.710188in}}%
\pgfpathclose%
\pgfusepath{stroke,fill}%
\end{pgfscope}%
\begin{pgfscope}%
\pgfpathrectangle{\pgfqpoint{1.000000in}{0.979904in}}{\pgfqpoint{6.200000in}{5.960192in}}%
\pgfusepath{clip}%
\pgfsetbuttcap%
\pgfsetroundjoin%
\definecolor{currentfill}{rgb}{0.800000,0.200000,0.200000}%
\pgfsetfillcolor{currentfill}%
\pgfsetlinewidth{1.003750pt}%
\definecolor{currentstroke}{rgb}{0.800000,0.200000,0.200000}%
\pgfsetstrokecolor{currentstroke}%
\pgfsetdash{}{0pt}%
\pgfpathmoveto{\pgfqpoint{4.136999in}{3.752378in}}%
\pgfpathcurveto{\pgfqpoint{4.142823in}{3.752378in}}{\pgfqpoint{4.148409in}{3.754692in}}{\pgfqpoint{4.152527in}{3.758810in}}%
\pgfpathcurveto{\pgfqpoint{4.156645in}{3.762928in}}{\pgfqpoint{4.158959in}{3.768514in}}{\pgfqpoint{4.158959in}{3.774338in}}%
\pgfpathcurveto{\pgfqpoint{4.158959in}{3.780162in}}{\pgfqpoint{4.156645in}{3.785748in}}{\pgfqpoint{4.152527in}{3.789866in}}%
\pgfpathcurveto{\pgfqpoint{4.148409in}{3.793984in}}{\pgfqpoint{4.142823in}{3.796298in}}{\pgfqpoint{4.136999in}{3.796298in}}%
\pgfpathcurveto{\pgfqpoint{4.131175in}{3.796298in}}{\pgfqpoint{4.125589in}{3.793984in}}{\pgfqpoint{4.121470in}{3.789866in}}%
\pgfpathcurveto{\pgfqpoint{4.117352in}{3.785748in}}{\pgfqpoint{4.115038in}{3.780162in}}{\pgfqpoint{4.115038in}{3.774338in}}%
\pgfpathcurveto{\pgfqpoint{4.115038in}{3.768514in}}{\pgfqpoint{4.117352in}{3.762928in}}{\pgfqpoint{4.121470in}{3.758810in}}%
\pgfpathcurveto{\pgfqpoint{4.125589in}{3.754692in}}{\pgfqpoint{4.131175in}{3.752378in}}{\pgfqpoint{4.136999in}{3.752378in}}%
\pgfpathlineto{\pgfqpoint{4.136999in}{3.752378in}}%
\pgfpathclose%
\pgfusepath{stroke,fill}%
\end{pgfscope}%
\begin{pgfscope}%
\pgfpathrectangle{\pgfqpoint{1.000000in}{0.979904in}}{\pgfqpoint{6.200000in}{5.960192in}}%
\pgfusepath{clip}%
\pgfsetbuttcap%
\pgfsetroundjoin%
\definecolor{currentfill}{rgb}{0.800000,0.200000,0.200000}%
\pgfsetfillcolor{currentfill}%
\pgfsetlinewidth{1.003750pt}%
\definecolor{currentstroke}{rgb}{0.800000,0.200000,0.200000}%
\pgfsetstrokecolor{currentstroke}%
\pgfsetdash{}{0pt}%
\pgfpathmoveto{\pgfqpoint{4.165558in}{3.900144in}}%
\pgfpathcurveto{\pgfqpoint{4.171382in}{3.900144in}}{\pgfqpoint{4.176968in}{3.902458in}}{\pgfqpoint{4.181086in}{3.906576in}}%
\pgfpathcurveto{\pgfqpoint{4.185204in}{3.910694in}}{\pgfqpoint{4.187518in}{3.916280in}}{\pgfqpoint{4.187518in}{3.922104in}}%
\pgfpathcurveto{\pgfqpoint{4.187518in}{3.927928in}}{\pgfqpoint{4.185204in}{3.933514in}}{\pgfqpoint{4.181086in}{3.937632in}}%
\pgfpathcurveto{\pgfqpoint{4.176968in}{3.941750in}}{\pgfqpoint{4.171382in}{3.944064in}}{\pgfqpoint{4.165558in}{3.944064in}}%
\pgfpathcurveto{\pgfqpoint{4.159734in}{3.944064in}}{\pgfqpoint{4.154148in}{3.941750in}}{\pgfqpoint{4.150029in}{3.937632in}}%
\pgfpathcurveto{\pgfqpoint{4.145911in}{3.933514in}}{\pgfqpoint{4.143597in}{3.927928in}}{\pgfqpoint{4.143597in}{3.922104in}}%
\pgfpathcurveto{\pgfqpoint{4.143597in}{3.916280in}}{\pgfqpoint{4.145911in}{3.910694in}}{\pgfqpoint{4.150029in}{3.906576in}}%
\pgfpathcurveto{\pgfqpoint{4.154148in}{3.902458in}}{\pgfqpoint{4.159734in}{3.900144in}}{\pgfqpoint{4.165558in}{3.900144in}}%
\pgfpathlineto{\pgfqpoint{4.165558in}{3.900144in}}%
\pgfpathclose%
\pgfusepath{stroke,fill}%
\end{pgfscope}%
\begin{pgfscope}%
\pgfpathrectangle{\pgfqpoint{1.000000in}{0.979904in}}{\pgfqpoint{6.200000in}{5.960192in}}%
\pgfusepath{clip}%
\pgfsetbuttcap%
\pgfsetroundjoin%
\definecolor{currentfill}{rgb}{0.800000,0.200000,0.200000}%
\pgfsetfillcolor{currentfill}%
\pgfsetlinewidth{1.003750pt}%
\definecolor{currentstroke}{rgb}{0.800000,0.200000,0.200000}%
\pgfsetstrokecolor{currentstroke}%
\pgfsetdash{}{0pt}%
\pgfpathmoveto{\pgfqpoint{4.067836in}{3.947983in}}%
\pgfpathcurveto{\pgfqpoint{4.073660in}{3.947983in}}{\pgfqpoint{4.079246in}{3.950297in}}{\pgfqpoint{4.083364in}{3.954415in}}%
\pgfpathcurveto{\pgfqpoint{4.087483in}{3.958533in}}{\pgfqpoint{4.089796in}{3.964119in}}{\pgfqpoint{4.089796in}{3.969943in}}%
\pgfpathcurveto{\pgfqpoint{4.089796in}{3.975767in}}{\pgfqpoint{4.087483in}{3.981354in}}{\pgfqpoint{4.083364in}{3.985472in}}%
\pgfpathcurveto{\pgfqpoint{4.079246in}{3.989590in}}{\pgfqpoint{4.073660in}{3.991904in}}{\pgfqpoint{4.067836in}{3.991904in}}%
\pgfpathcurveto{\pgfqpoint{4.062012in}{3.991904in}}{\pgfqpoint{4.056426in}{3.989590in}}{\pgfqpoint{4.052308in}{3.985472in}}%
\pgfpathcurveto{\pgfqpoint{4.048190in}{3.981354in}}{\pgfqpoint{4.045876in}{3.975767in}}{\pgfqpoint{4.045876in}{3.969943in}}%
\pgfpathcurveto{\pgfqpoint{4.045876in}{3.964119in}}{\pgfqpoint{4.048190in}{3.958533in}}{\pgfqpoint{4.052308in}{3.954415in}}%
\pgfpathcurveto{\pgfqpoint{4.056426in}{3.950297in}}{\pgfqpoint{4.062012in}{3.947983in}}{\pgfqpoint{4.067836in}{3.947983in}}%
\pgfpathlineto{\pgfqpoint{4.067836in}{3.947983in}}%
\pgfpathclose%
\pgfusepath{stroke,fill}%
\end{pgfscope}%
\begin{pgfscope}%
\pgfpathrectangle{\pgfqpoint{1.000000in}{0.979904in}}{\pgfqpoint{6.200000in}{5.960192in}}%
\pgfusepath{clip}%
\pgfsetbuttcap%
\pgfsetroundjoin%
\definecolor{currentfill}{rgb}{0.800000,0.200000,0.200000}%
\pgfsetfillcolor{currentfill}%
\pgfsetlinewidth{1.003750pt}%
\definecolor{currentstroke}{rgb}{0.800000,0.200000,0.200000}%
\pgfsetstrokecolor{currentstroke}%
\pgfsetdash{}{0pt}%
\pgfpathmoveto{\pgfqpoint{4.014725in}{4.037070in}}%
\pgfpathcurveto{\pgfqpoint{4.020549in}{4.037070in}}{\pgfqpoint{4.026135in}{4.039384in}}{\pgfqpoint{4.030253in}{4.043502in}}%
\pgfpathcurveto{\pgfqpoint{4.034371in}{4.047620in}}{\pgfqpoint{4.036685in}{4.053206in}}{\pgfqpoint{4.036685in}{4.059030in}}%
\pgfpathcurveto{\pgfqpoint{4.036685in}{4.064854in}}{\pgfqpoint{4.034371in}{4.070440in}}{\pgfqpoint{4.030253in}{4.074559in}}%
\pgfpathcurveto{\pgfqpoint{4.026135in}{4.078677in}}{\pgfqpoint{4.020549in}{4.080991in}}{\pgfqpoint{4.014725in}{4.080991in}}%
\pgfpathcurveto{\pgfqpoint{4.008901in}{4.080991in}}{\pgfqpoint{4.003315in}{4.078677in}}{\pgfqpoint{3.999197in}{4.074559in}}%
\pgfpathcurveto{\pgfqpoint{3.995079in}{4.070440in}}{\pgfqpoint{3.992765in}{4.064854in}}{\pgfqpoint{3.992765in}{4.059030in}}%
\pgfpathcurveto{\pgfqpoint{3.992765in}{4.053206in}}{\pgfqpoint{3.995079in}{4.047620in}}{\pgfqpoint{3.999197in}{4.043502in}}%
\pgfpathcurveto{\pgfqpoint{4.003315in}{4.039384in}}{\pgfqpoint{4.008901in}{4.037070in}}{\pgfqpoint{4.014725in}{4.037070in}}%
\pgfpathlineto{\pgfqpoint{4.014725in}{4.037070in}}%
\pgfpathclose%
\pgfusepath{stroke,fill}%
\end{pgfscope}%
\begin{pgfscope}%
\pgfpathrectangle{\pgfqpoint{1.000000in}{0.979904in}}{\pgfqpoint{6.200000in}{5.960192in}}%
\pgfusepath{clip}%
\pgfsetbuttcap%
\pgfsetroundjoin%
\definecolor{currentfill}{rgb}{0.800000,0.200000,0.200000}%
\pgfsetfillcolor{currentfill}%
\pgfsetlinewidth{1.003750pt}%
\definecolor{currentstroke}{rgb}{0.800000,0.200000,0.200000}%
\pgfsetstrokecolor{currentstroke}%
\pgfsetdash{}{0pt}%
\pgfpathmoveto{\pgfqpoint{3.905698in}{4.063246in}}%
\pgfpathcurveto{\pgfqpoint{3.911522in}{4.063246in}}{\pgfqpoint{3.917108in}{4.065560in}}{\pgfqpoint{3.921226in}{4.069678in}}%
\pgfpathcurveto{\pgfqpoint{3.925345in}{4.073796in}}{\pgfqpoint{3.927658in}{4.079382in}}{\pgfqpoint{3.927658in}{4.085206in}}%
\pgfpathcurveto{\pgfqpoint{3.927658in}{4.091030in}}{\pgfqpoint{3.925345in}{4.096616in}}{\pgfqpoint{3.921226in}{4.100734in}}%
\pgfpathcurveto{\pgfqpoint{3.917108in}{4.104853in}}{\pgfqpoint{3.911522in}{4.107166in}}{\pgfqpoint{3.905698in}{4.107166in}}%
\pgfpathcurveto{\pgfqpoint{3.899874in}{4.107166in}}{\pgfqpoint{3.894288in}{4.104853in}}{\pgfqpoint{3.890170in}{4.100734in}}%
\pgfpathcurveto{\pgfqpoint{3.886052in}{4.096616in}}{\pgfqpoint{3.883738in}{4.091030in}}{\pgfqpoint{3.883738in}{4.085206in}}%
\pgfpathcurveto{\pgfqpoint{3.883738in}{4.079382in}}{\pgfqpoint{3.886052in}{4.073796in}}{\pgfqpoint{3.890170in}{4.069678in}}%
\pgfpathcurveto{\pgfqpoint{3.894288in}{4.065560in}}{\pgfqpoint{3.899874in}{4.063246in}}{\pgfqpoint{3.905698in}{4.063246in}}%
\pgfpathlineto{\pgfqpoint{3.905698in}{4.063246in}}%
\pgfpathclose%
\pgfusepath{stroke,fill}%
\end{pgfscope}%
\begin{pgfscope}%
\pgfpathrectangle{\pgfqpoint{1.000000in}{0.979904in}}{\pgfqpoint{6.200000in}{5.960192in}}%
\pgfusepath{clip}%
\pgfsetbuttcap%
\pgfsetroundjoin%
\definecolor{currentfill}{rgb}{0.800000,0.200000,0.200000}%
\pgfsetfillcolor{currentfill}%
\pgfsetlinewidth{1.003750pt}%
\definecolor{currentstroke}{rgb}{0.800000,0.200000,0.200000}%
\pgfsetstrokecolor{currentstroke}%
\pgfsetdash{}{0pt}%
\pgfpathmoveto{\pgfqpoint{3.844975in}{4.146664in}}%
\pgfpathcurveto{\pgfqpoint{3.850799in}{4.146664in}}{\pgfqpoint{3.856385in}{4.148978in}}{\pgfqpoint{3.860503in}{4.153096in}}%
\pgfpathcurveto{\pgfqpoint{3.864622in}{4.157214in}}{\pgfqpoint{3.866935in}{4.162801in}}{\pgfqpoint{3.866935in}{4.168625in}}%
\pgfpathcurveto{\pgfqpoint{3.866935in}{4.174448in}}{\pgfqpoint{3.864622in}{4.180035in}}{\pgfqpoint{3.860503in}{4.184153in}}%
\pgfpathcurveto{\pgfqpoint{3.856385in}{4.188271in}}{\pgfqpoint{3.850799in}{4.190585in}}{\pgfqpoint{3.844975in}{4.190585in}}%
\pgfpathcurveto{\pgfqpoint{3.839151in}{4.190585in}}{\pgfqpoint{3.833565in}{4.188271in}}{\pgfqpoint{3.829447in}{4.184153in}}%
\pgfpathcurveto{\pgfqpoint{3.825329in}{4.180035in}}{\pgfqpoint{3.823015in}{4.174448in}}{\pgfqpoint{3.823015in}{4.168625in}}%
\pgfpathcurveto{\pgfqpoint{3.823015in}{4.162801in}}{\pgfqpoint{3.825329in}{4.157214in}}{\pgfqpoint{3.829447in}{4.153096in}}%
\pgfpathcurveto{\pgfqpoint{3.833565in}{4.148978in}}{\pgfqpoint{3.839151in}{4.146664in}}{\pgfqpoint{3.844975in}{4.146664in}}%
\pgfpathlineto{\pgfqpoint{3.844975in}{4.146664in}}%
\pgfpathclose%
\pgfusepath{stroke,fill}%
\end{pgfscope}%
\begin{pgfscope}%
\pgfpathrectangle{\pgfqpoint{1.000000in}{0.979904in}}{\pgfqpoint{6.200000in}{5.960192in}}%
\pgfusepath{clip}%
\pgfsetbuttcap%
\pgfsetroundjoin%
\definecolor{currentfill}{rgb}{0.800000,0.200000,0.200000}%
\pgfsetfillcolor{currentfill}%
\pgfsetlinewidth{1.003750pt}%
\definecolor{currentstroke}{rgb}{0.800000,0.200000,0.200000}%
\pgfsetstrokecolor{currentstroke}%
\pgfsetdash{}{0pt}%
\pgfpathmoveto{\pgfqpoint{3.814916in}{4.287377in}}%
\pgfpathcurveto{\pgfqpoint{3.820740in}{4.287377in}}{\pgfqpoint{3.826326in}{4.289690in}}{\pgfqpoint{3.830444in}{4.293809in}}%
\pgfpathcurveto{\pgfqpoint{3.834562in}{4.297927in}}{\pgfqpoint{3.836876in}{4.303513in}}{\pgfqpoint{3.836876in}{4.309337in}}%
\pgfpathcurveto{\pgfqpoint{3.836876in}{4.315161in}}{\pgfqpoint{3.834562in}{4.320747in}}{\pgfqpoint{3.830444in}{4.324865in}}%
\pgfpathcurveto{\pgfqpoint{3.826326in}{4.328983in}}{\pgfqpoint{3.820740in}{4.331297in}}{\pgfqpoint{3.814916in}{4.331297in}}%
\pgfpathcurveto{\pgfqpoint{3.809092in}{4.331297in}}{\pgfqpoint{3.803505in}{4.328983in}}{\pgfqpoint{3.799387in}{4.324865in}}%
\pgfpathcurveto{\pgfqpoint{3.795269in}{4.320747in}}{\pgfqpoint{3.792955in}{4.315161in}}{\pgfqpoint{3.792955in}{4.309337in}}%
\pgfpathcurveto{\pgfqpoint{3.792955in}{4.303513in}}{\pgfqpoint{3.795269in}{4.297927in}}{\pgfqpoint{3.799387in}{4.293809in}}%
\pgfpathcurveto{\pgfqpoint{3.803505in}{4.289690in}}{\pgfqpoint{3.809092in}{4.287377in}}{\pgfqpoint{3.814916in}{4.287377in}}%
\pgfpathlineto{\pgfqpoint{3.814916in}{4.287377in}}%
\pgfpathclose%
\pgfusepath{stroke,fill}%
\end{pgfscope}%
\begin{pgfscope}%
\pgfpathrectangle{\pgfqpoint{1.000000in}{0.979904in}}{\pgfqpoint{6.200000in}{5.960192in}}%
\pgfusepath{clip}%
\pgfsetbuttcap%
\pgfsetroundjoin%
\definecolor{currentfill}{rgb}{0.800000,0.200000,0.200000}%
\pgfsetfillcolor{currentfill}%
\pgfsetlinewidth{1.003750pt}%
\definecolor{currentstroke}{rgb}{0.800000,0.200000,0.200000}%
\pgfsetstrokecolor{currentstroke}%
\pgfsetdash{}{0pt}%
\pgfpathmoveto{\pgfqpoint{3.687100in}{4.273494in}}%
\pgfpathcurveto{\pgfqpoint{3.692924in}{4.273494in}}{\pgfqpoint{3.698511in}{4.275808in}}{\pgfqpoint{3.702629in}{4.279927in}}%
\pgfpathcurveto{\pgfqpoint{3.706747in}{4.284045in}}{\pgfqpoint{3.709061in}{4.289631in}}{\pgfqpoint{3.709061in}{4.295455in}}%
\pgfpathcurveto{\pgfqpoint{3.709061in}{4.301279in}}{\pgfqpoint{3.706747in}{4.306865in}}{\pgfqpoint{3.702629in}{4.310983in}}%
\pgfpathcurveto{\pgfqpoint{3.698511in}{4.315101in}}{\pgfqpoint{3.692924in}{4.317415in}}{\pgfqpoint{3.687100in}{4.317415in}}%
\pgfpathcurveto{\pgfqpoint{3.681277in}{4.317415in}}{\pgfqpoint{3.675690in}{4.315101in}}{\pgfqpoint{3.671572in}{4.310983in}}%
\pgfpathcurveto{\pgfqpoint{3.667454in}{4.306865in}}{\pgfqpoint{3.665140in}{4.301279in}}{\pgfqpoint{3.665140in}{4.295455in}}%
\pgfpathcurveto{\pgfqpoint{3.665140in}{4.289631in}}{\pgfqpoint{3.667454in}{4.284045in}}{\pgfqpoint{3.671572in}{4.279927in}}%
\pgfpathcurveto{\pgfqpoint{3.675690in}{4.275808in}}{\pgfqpoint{3.681277in}{4.273494in}}{\pgfqpoint{3.687100in}{4.273494in}}%
\pgfpathlineto{\pgfqpoint{3.687100in}{4.273494in}}%
\pgfpathclose%
\pgfusepath{stroke,fill}%
\end{pgfscope}%
\begin{pgfscope}%
\pgfpathrectangle{\pgfqpoint{1.000000in}{0.979904in}}{\pgfqpoint{6.200000in}{5.960192in}}%
\pgfusepath{clip}%
\pgfsetbuttcap%
\pgfsetroundjoin%
\definecolor{currentfill}{rgb}{0.800000,0.200000,0.200000}%
\pgfsetfillcolor{currentfill}%
\pgfsetlinewidth{1.003750pt}%
\definecolor{currentstroke}{rgb}{0.800000,0.200000,0.200000}%
\pgfsetstrokecolor{currentstroke}%
\pgfsetdash{}{0pt}%
\pgfpathmoveto{\pgfqpoint{3.605580in}{4.338816in}}%
\pgfpathcurveto{\pgfqpoint{3.611404in}{4.338816in}}{\pgfqpoint{3.616990in}{4.341130in}}{\pgfqpoint{3.621108in}{4.345248in}}%
\pgfpathcurveto{\pgfqpoint{3.625227in}{4.349366in}}{\pgfqpoint{3.627540in}{4.354952in}}{\pgfqpoint{3.627540in}{4.360776in}}%
\pgfpathcurveto{\pgfqpoint{3.627540in}{4.366600in}}{\pgfqpoint{3.625227in}{4.372186in}}{\pgfqpoint{3.621108in}{4.376304in}}%
\pgfpathcurveto{\pgfqpoint{3.616990in}{4.380422in}}{\pgfqpoint{3.611404in}{4.382736in}}{\pgfqpoint{3.605580in}{4.382736in}}%
\pgfpathcurveto{\pgfqpoint{3.599756in}{4.382736in}}{\pgfqpoint{3.594170in}{4.380422in}}{\pgfqpoint{3.590052in}{4.376304in}}%
\pgfpathcurveto{\pgfqpoint{3.585934in}{4.372186in}}{\pgfqpoint{3.583620in}{4.366600in}}{\pgfqpoint{3.583620in}{4.360776in}}%
\pgfpathcurveto{\pgfqpoint{3.583620in}{4.354952in}}{\pgfqpoint{3.585934in}{4.349366in}}{\pgfqpoint{3.590052in}{4.345248in}}%
\pgfpathcurveto{\pgfqpoint{3.594170in}{4.341130in}}{\pgfqpoint{3.599756in}{4.338816in}}{\pgfqpoint{3.605580in}{4.338816in}}%
\pgfpathlineto{\pgfqpoint{3.605580in}{4.338816in}}%
\pgfpathclose%
\pgfusepath{stroke,fill}%
\end{pgfscope}%
\begin{pgfscope}%
\pgfpathrectangle{\pgfqpoint{1.000000in}{0.979904in}}{\pgfqpoint{6.200000in}{5.960192in}}%
\pgfusepath{clip}%
\pgfsetbuttcap%
\pgfsetroundjoin%
\definecolor{currentfill}{rgb}{0.800000,0.200000,0.200000}%
\pgfsetfillcolor{currentfill}%
\pgfsetlinewidth{1.003750pt}%
\definecolor{currentstroke}{rgb}{0.800000,0.200000,0.200000}%
\pgfsetstrokecolor{currentstroke}%
\pgfsetdash{}{0pt}%
\pgfpathmoveto{\pgfqpoint{3.476331in}{4.288289in}}%
\pgfpathcurveto{\pgfqpoint{3.482155in}{4.288289in}}{\pgfqpoint{3.487741in}{4.290603in}}{\pgfqpoint{3.491859in}{4.294721in}}%
\pgfpathcurveto{\pgfqpoint{3.495977in}{4.298839in}}{\pgfqpoint{3.498291in}{4.304425in}}{\pgfqpoint{3.498291in}{4.310249in}}%
\pgfpathcurveto{\pgfqpoint{3.498291in}{4.316073in}}{\pgfqpoint{3.495977in}{4.321659in}}{\pgfqpoint{3.491859in}{4.325778in}}%
\pgfpathcurveto{\pgfqpoint{3.487741in}{4.329896in}}{\pgfqpoint{3.482155in}{4.332210in}}{\pgfqpoint{3.476331in}{4.332210in}}%
\pgfpathcurveto{\pgfqpoint{3.470507in}{4.332210in}}{\pgfqpoint{3.464921in}{4.329896in}}{\pgfqpoint{3.460803in}{4.325778in}}%
\pgfpathcurveto{\pgfqpoint{3.456685in}{4.321659in}}{\pgfqpoint{3.454371in}{4.316073in}}{\pgfqpoint{3.454371in}{4.310249in}}%
\pgfpathcurveto{\pgfqpoint{3.454371in}{4.304425in}}{\pgfqpoint{3.456685in}{4.298839in}}{\pgfqpoint{3.460803in}{4.294721in}}%
\pgfpathcurveto{\pgfqpoint{3.464921in}{4.290603in}}{\pgfqpoint{3.470507in}{4.288289in}}{\pgfqpoint{3.476331in}{4.288289in}}%
\pgfpathlineto{\pgfqpoint{3.476331in}{4.288289in}}%
\pgfpathclose%
\pgfusepath{stroke,fill}%
\end{pgfscope}%
\begin{pgfscope}%
\pgfpathrectangle{\pgfqpoint{1.000000in}{0.979904in}}{\pgfqpoint{6.200000in}{5.960192in}}%
\pgfusepath{clip}%
\pgfsetbuttcap%
\pgfsetroundjoin%
\definecolor{currentfill}{rgb}{0.800000,0.200000,0.200000}%
\pgfsetfillcolor{currentfill}%
\pgfsetlinewidth{1.003750pt}%
\definecolor{currentstroke}{rgb}{0.800000,0.200000,0.200000}%
\pgfsetstrokecolor{currentstroke}%
\pgfsetdash{}{0pt}%
\pgfpathmoveto{\pgfqpoint{3.442674in}{4.505318in}}%
\pgfpathcurveto{\pgfqpoint{3.448498in}{4.505318in}}{\pgfqpoint{3.454084in}{4.507632in}}{\pgfqpoint{3.458202in}{4.511750in}}%
\pgfpathcurveto{\pgfqpoint{3.462320in}{4.515868in}}{\pgfqpoint{3.464634in}{4.521455in}}{\pgfqpoint{3.464634in}{4.527279in}}%
\pgfpathcurveto{\pgfqpoint{3.464634in}{4.533103in}}{\pgfqpoint{3.462320in}{4.538689in}}{\pgfqpoint{3.458202in}{4.542807in}}%
\pgfpathcurveto{\pgfqpoint{3.454084in}{4.546925in}}{\pgfqpoint{3.448498in}{4.549239in}}{\pgfqpoint{3.442674in}{4.549239in}}%
\pgfpathcurveto{\pgfqpoint{3.436850in}{4.549239in}}{\pgfqpoint{3.431264in}{4.546925in}}{\pgfqpoint{3.427146in}{4.542807in}}%
\pgfpathcurveto{\pgfqpoint{3.423028in}{4.538689in}}{\pgfqpoint{3.420714in}{4.533103in}}{\pgfqpoint{3.420714in}{4.527279in}}%
\pgfpathcurveto{\pgfqpoint{3.420714in}{4.521455in}}{\pgfqpoint{3.423028in}{4.515868in}}{\pgfqpoint{3.427146in}{4.511750in}}%
\pgfpathcurveto{\pgfqpoint{3.431264in}{4.507632in}}{\pgfqpoint{3.436850in}{4.505318in}}{\pgfqpoint{3.442674in}{4.505318in}}%
\pgfpathlineto{\pgfqpoint{3.442674in}{4.505318in}}%
\pgfpathclose%
\pgfusepath{stroke,fill}%
\end{pgfscope}%
\begin{pgfscope}%
\pgfpathrectangle{\pgfqpoint{1.000000in}{0.979904in}}{\pgfqpoint{6.200000in}{5.960192in}}%
\pgfusepath{clip}%
\pgfsetbuttcap%
\pgfsetroundjoin%
\definecolor{currentfill}{rgb}{0.200000,0.800000,0.200000}%
\pgfsetfillcolor{currentfill}%
\pgfsetlinewidth{1.003750pt}%
\definecolor{currentstroke}{rgb}{0.200000,0.800000,0.200000}%
\pgfsetstrokecolor{currentstroke}%
\pgfsetdash{}{0pt}%
\pgfpathmoveto{\pgfqpoint{3.330838in}{4.506218in}}%
\pgfpathcurveto{\pgfqpoint{3.336662in}{4.506218in}}{\pgfqpoint{3.342248in}{4.508532in}}{\pgfqpoint{3.346366in}{4.512650in}}%
\pgfpathcurveto{\pgfqpoint{3.350484in}{4.516768in}}{\pgfqpoint{3.352798in}{4.522354in}}{\pgfqpoint{3.352798in}{4.528178in}}%
\pgfpathcurveto{\pgfqpoint{3.352798in}{4.534002in}}{\pgfqpoint{3.350484in}{4.539588in}}{\pgfqpoint{3.346366in}{4.543706in}}%
\pgfpathcurveto{\pgfqpoint{3.342248in}{4.547825in}}{\pgfqpoint{3.336662in}{4.550138in}}{\pgfqpoint{3.330838in}{4.550138in}}%
\pgfpathcurveto{\pgfqpoint{3.325014in}{4.550138in}}{\pgfqpoint{3.319428in}{4.547825in}}{\pgfqpoint{3.315310in}{4.543706in}}%
\pgfpathcurveto{\pgfqpoint{3.311192in}{4.539588in}}{\pgfqpoint{3.308878in}{4.534002in}}{\pgfqpoint{3.308878in}{4.528178in}}%
\pgfpathcurveto{\pgfqpoint{3.308878in}{4.522354in}}{\pgfqpoint{3.311192in}{4.516768in}}{\pgfqpoint{3.315310in}{4.512650in}}%
\pgfpathcurveto{\pgfqpoint{3.319428in}{4.508532in}}{\pgfqpoint{3.325014in}{4.506218in}}{\pgfqpoint{3.330838in}{4.506218in}}%
\pgfpathlineto{\pgfqpoint{3.330838in}{4.506218in}}%
\pgfpathclose%
\pgfusepath{stroke,fill}%
\end{pgfscope}%
\begin{pgfscope}%
\pgfpathrectangle{\pgfqpoint{1.000000in}{0.979904in}}{\pgfqpoint{6.200000in}{5.960192in}}%
\pgfusepath{clip}%
\pgfsetbuttcap%
\pgfsetroundjoin%
\definecolor{currentfill}{rgb}{0.200000,0.800000,0.200000}%
\pgfsetfillcolor{currentfill}%
\pgfsetlinewidth{1.003750pt}%
\definecolor{currentstroke}{rgb}{0.200000,0.800000,0.200000}%
\pgfsetstrokecolor{currentstroke}%
\pgfsetdash{}{0pt}%
\pgfpathmoveto{\pgfqpoint{3.221573in}{4.503076in}}%
\pgfpathcurveto{\pgfqpoint{3.227397in}{4.503076in}}{\pgfqpoint{3.232983in}{4.505390in}}{\pgfqpoint{3.237101in}{4.509508in}}%
\pgfpathcurveto{\pgfqpoint{3.241219in}{4.513626in}}{\pgfqpoint{3.243533in}{4.519213in}}{\pgfqpoint{3.243533in}{4.525037in}}%
\pgfpathcurveto{\pgfqpoint{3.243533in}{4.530860in}}{\pgfqpoint{3.241219in}{4.536447in}}{\pgfqpoint{3.237101in}{4.540565in}}%
\pgfpathcurveto{\pgfqpoint{3.232983in}{4.544683in}}{\pgfqpoint{3.227397in}{4.546997in}}{\pgfqpoint{3.221573in}{4.546997in}}%
\pgfpathcurveto{\pgfqpoint{3.215749in}{4.546997in}}{\pgfqpoint{3.210163in}{4.544683in}}{\pgfqpoint{3.206045in}{4.540565in}}%
\pgfpathcurveto{\pgfqpoint{3.201927in}{4.536447in}}{\pgfqpoint{3.199613in}{4.530860in}}{\pgfqpoint{3.199613in}{4.525037in}}%
\pgfpathcurveto{\pgfqpoint{3.199613in}{4.519213in}}{\pgfqpoint{3.201927in}{4.513626in}}{\pgfqpoint{3.206045in}{4.509508in}}%
\pgfpathcurveto{\pgfqpoint{3.210163in}{4.505390in}}{\pgfqpoint{3.215749in}{4.503076in}}{\pgfqpoint{3.221573in}{4.503076in}}%
\pgfpathlineto{\pgfqpoint{3.221573in}{4.503076in}}%
\pgfpathclose%
\pgfusepath{stroke,fill}%
\end{pgfscope}%
\begin{pgfscope}%
\pgfpathrectangle{\pgfqpoint{1.000000in}{0.979904in}}{\pgfqpoint{6.200000in}{5.960192in}}%
\pgfusepath{clip}%
\pgfsetbuttcap%
\pgfsetroundjoin%
\definecolor{currentfill}{rgb}{0.800000,0.200000,0.200000}%
\pgfsetfillcolor{currentfill}%
\pgfsetlinewidth{1.003750pt}%
\definecolor{currentstroke}{rgb}{0.800000,0.200000,0.200000}%
\pgfsetstrokecolor{currentstroke}%
\pgfsetdash{}{0pt}%
\pgfpathmoveto{\pgfqpoint{3.114005in}{4.489897in}}%
\pgfpathcurveto{\pgfqpoint{3.119829in}{4.489897in}}{\pgfqpoint{3.125415in}{4.492211in}}{\pgfqpoint{3.129534in}{4.496329in}}%
\pgfpathcurveto{\pgfqpoint{3.133652in}{4.500447in}}{\pgfqpoint{3.135966in}{4.506033in}}{\pgfqpoint{3.135966in}{4.511857in}}%
\pgfpathcurveto{\pgfqpoint{3.135966in}{4.517681in}}{\pgfqpoint{3.133652in}{4.523267in}}{\pgfqpoint{3.129534in}{4.527386in}}%
\pgfpathcurveto{\pgfqpoint{3.125415in}{4.531504in}}{\pgfqpoint{3.119829in}{4.533818in}}{\pgfqpoint{3.114005in}{4.533818in}}%
\pgfpathcurveto{\pgfqpoint{3.108181in}{4.533818in}}{\pgfqpoint{3.102595in}{4.531504in}}{\pgfqpoint{3.098477in}{4.527386in}}%
\pgfpathcurveto{\pgfqpoint{3.094359in}{4.523267in}}{\pgfqpoint{3.092045in}{4.517681in}}{\pgfqpoint{3.092045in}{4.511857in}}%
\pgfpathcurveto{\pgfqpoint{3.092045in}{4.506033in}}{\pgfqpoint{3.094359in}{4.500447in}}{\pgfqpoint{3.098477in}{4.496329in}}%
\pgfpathcurveto{\pgfqpoint{3.102595in}{4.492211in}}{\pgfqpoint{3.108181in}{4.489897in}}{\pgfqpoint{3.114005in}{4.489897in}}%
\pgfpathlineto{\pgfqpoint{3.114005in}{4.489897in}}%
\pgfpathclose%
\pgfusepath{stroke,fill}%
\end{pgfscope}%
\begin{pgfscope}%
\pgfpathrectangle{\pgfqpoint{1.000000in}{0.979904in}}{\pgfqpoint{6.200000in}{5.960192in}}%
\pgfusepath{clip}%
\pgfsetbuttcap%
\pgfsetroundjoin%
\definecolor{currentfill}{rgb}{0.800000,0.200000,0.200000}%
\pgfsetfillcolor{currentfill}%
\pgfsetlinewidth{1.003750pt}%
\definecolor{currentstroke}{rgb}{0.800000,0.200000,0.200000}%
\pgfsetstrokecolor{currentstroke}%
\pgfsetdash{}{0pt}%
\pgfpathmoveto{\pgfqpoint{3.008580in}{4.454996in}}%
\pgfpathcurveto{\pgfqpoint{3.014404in}{4.454996in}}{\pgfqpoint{3.019990in}{4.457310in}}{\pgfqpoint{3.024108in}{4.461428in}}%
\pgfpathcurveto{\pgfqpoint{3.028227in}{4.465546in}}{\pgfqpoint{3.030540in}{4.471132in}}{\pgfqpoint{3.030540in}{4.476956in}}%
\pgfpathcurveto{\pgfqpoint{3.030540in}{4.482780in}}{\pgfqpoint{3.028227in}{4.488366in}}{\pgfqpoint{3.024108in}{4.492484in}}%
\pgfpathcurveto{\pgfqpoint{3.019990in}{4.496603in}}{\pgfqpoint{3.014404in}{4.498916in}}{\pgfqpoint{3.008580in}{4.498916in}}%
\pgfpathcurveto{\pgfqpoint{3.002756in}{4.498916in}}{\pgfqpoint{2.997170in}{4.496603in}}{\pgfqpoint{2.993052in}{4.492484in}}%
\pgfpathcurveto{\pgfqpoint{2.988934in}{4.488366in}}{\pgfqpoint{2.986620in}{4.482780in}}{\pgfqpoint{2.986620in}{4.476956in}}%
\pgfpathcurveto{\pgfqpoint{2.986620in}{4.471132in}}{\pgfqpoint{2.988934in}{4.465546in}}{\pgfqpoint{2.993052in}{4.461428in}}%
\pgfpathcurveto{\pgfqpoint{2.997170in}{4.457310in}}{\pgfqpoint{3.002756in}{4.454996in}}{\pgfqpoint{3.008580in}{4.454996in}}%
\pgfpathlineto{\pgfqpoint{3.008580in}{4.454996in}}%
\pgfpathclose%
\pgfusepath{stroke,fill}%
\end{pgfscope}%
\begin{pgfscope}%
\pgfpathrectangle{\pgfqpoint{1.000000in}{0.979904in}}{\pgfqpoint{6.200000in}{5.960192in}}%
\pgfusepath{clip}%
\pgfsetbuttcap%
\pgfsetroundjoin%
\definecolor{currentfill}{rgb}{0.800000,0.200000,0.200000}%
\pgfsetfillcolor{currentfill}%
\pgfsetlinewidth{1.003750pt}%
\definecolor{currentstroke}{rgb}{0.800000,0.200000,0.200000}%
\pgfsetstrokecolor{currentstroke}%
\pgfsetdash{}{0pt}%
\pgfpathmoveto{\pgfqpoint{2.907275in}{4.479246in}}%
\pgfpathcurveto{\pgfqpoint{2.913099in}{4.479246in}}{\pgfqpoint{2.918686in}{4.481560in}}{\pgfqpoint{2.922804in}{4.485678in}}%
\pgfpathcurveto{\pgfqpoint{2.926922in}{4.489797in}}{\pgfqpoint{2.929236in}{4.495383in}}{\pgfqpoint{2.929236in}{4.501207in}}%
\pgfpathcurveto{\pgfqpoint{2.929236in}{4.507031in}}{\pgfqpoint{2.926922in}{4.512617in}}{\pgfqpoint{2.922804in}{4.516735in}}%
\pgfpathcurveto{\pgfqpoint{2.918686in}{4.520853in}}{\pgfqpoint{2.913099in}{4.523167in}}{\pgfqpoint{2.907275in}{4.523167in}}%
\pgfpathcurveto{\pgfqpoint{2.901452in}{4.523167in}}{\pgfqpoint{2.895865in}{4.520853in}}{\pgfqpoint{2.891747in}{4.516735in}}%
\pgfpathcurveto{\pgfqpoint{2.887629in}{4.512617in}}{\pgfqpoint{2.885315in}{4.507031in}}{\pgfqpoint{2.885315in}{4.501207in}}%
\pgfpathcurveto{\pgfqpoint{2.885315in}{4.495383in}}{\pgfqpoint{2.887629in}{4.489797in}}{\pgfqpoint{2.891747in}{4.485678in}}%
\pgfpathcurveto{\pgfqpoint{2.895865in}{4.481560in}}{\pgfqpoint{2.901452in}{4.479246in}}{\pgfqpoint{2.907275in}{4.479246in}}%
\pgfpathlineto{\pgfqpoint{2.907275in}{4.479246in}}%
\pgfpathclose%
\pgfusepath{stroke,fill}%
\end{pgfscope}%
\begin{pgfscope}%
\pgfpathrectangle{\pgfqpoint{1.000000in}{0.979904in}}{\pgfqpoint{6.200000in}{5.960192in}}%
\pgfusepath{clip}%
\pgfsetbuttcap%
\pgfsetroundjoin%
\definecolor{currentfill}{rgb}{0.800000,0.200000,0.200000}%
\pgfsetfillcolor{currentfill}%
\pgfsetlinewidth{1.003750pt}%
\definecolor{currentstroke}{rgb}{0.800000,0.200000,0.200000}%
\pgfsetstrokecolor{currentstroke}%
\pgfsetdash{}{0pt}%
\pgfpathmoveto{\pgfqpoint{2.802448in}{4.506423in}}%
\pgfpathcurveto{\pgfqpoint{2.808272in}{4.506423in}}{\pgfqpoint{2.813858in}{4.508736in}}{\pgfqpoint{2.817976in}{4.512855in}}%
\pgfpathcurveto{\pgfqpoint{2.822095in}{4.516973in}}{\pgfqpoint{2.824408in}{4.522559in}}{\pgfqpoint{2.824408in}{4.528383in}}%
\pgfpathcurveto{\pgfqpoint{2.824408in}{4.534207in}}{\pgfqpoint{2.822095in}{4.539793in}}{\pgfqpoint{2.817976in}{4.543911in}}%
\pgfpathcurveto{\pgfqpoint{2.813858in}{4.548029in}}{\pgfqpoint{2.808272in}{4.550343in}}{\pgfqpoint{2.802448in}{4.550343in}}%
\pgfpathcurveto{\pgfqpoint{2.796624in}{4.550343in}}{\pgfqpoint{2.791038in}{4.548029in}}{\pgfqpoint{2.786920in}{4.543911in}}%
\pgfpathcurveto{\pgfqpoint{2.782802in}{4.539793in}}{\pgfqpoint{2.780488in}{4.534207in}}{\pgfqpoint{2.780488in}{4.528383in}}%
\pgfpathcurveto{\pgfqpoint{2.780488in}{4.522559in}}{\pgfqpoint{2.782802in}{4.516973in}}{\pgfqpoint{2.786920in}{4.512855in}}%
\pgfpathcurveto{\pgfqpoint{2.791038in}{4.508736in}}{\pgfqpoint{2.796624in}{4.506423in}}{\pgfqpoint{2.802448in}{4.506423in}}%
\pgfpathlineto{\pgfqpoint{2.802448in}{4.506423in}}%
\pgfpathclose%
\pgfusepath{stroke,fill}%
\end{pgfscope}%
\begin{pgfscope}%
\pgfpathrectangle{\pgfqpoint{1.000000in}{0.979904in}}{\pgfqpoint{6.200000in}{5.960192in}}%
\pgfusepath{clip}%
\pgfsetbuttcap%
\pgfsetroundjoin%
\definecolor{currentfill}{rgb}{0.800000,0.200000,0.200000}%
\pgfsetfillcolor{currentfill}%
\pgfsetlinewidth{1.003750pt}%
\definecolor{currentstroke}{rgb}{0.800000,0.200000,0.200000}%
\pgfsetstrokecolor{currentstroke}%
\pgfsetdash{}{0pt}%
\pgfpathmoveto{\pgfqpoint{2.711316in}{4.406798in}}%
\pgfpathcurveto{\pgfqpoint{2.717140in}{4.406798in}}{\pgfqpoint{2.722726in}{4.409112in}}{\pgfqpoint{2.726844in}{4.413230in}}%
\pgfpathcurveto{\pgfqpoint{2.730962in}{4.417348in}}{\pgfqpoint{2.733276in}{4.422935in}}{\pgfqpoint{2.733276in}{4.428759in}}%
\pgfpathcurveto{\pgfqpoint{2.733276in}{4.434582in}}{\pgfqpoint{2.730962in}{4.440169in}}{\pgfqpoint{2.726844in}{4.444287in}}%
\pgfpathcurveto{\pgfqpoint{2.722726in}{4.448405in}}{\pgfqpoint{2.717140in}{4.450719in}}{\pgfqpoint{2.711316in}{4.450719in}}%
\pgfpathcurveto{\pgfqpoint{2.705492in}{4.450719in}}{\pgfqpoint{2.699906in}{4.448405in}}{\pgfqpoint{2.695787in}{4.444287in}}%
\pgfpathcurveto{\pgfqpoint{2.691669in}{4.440169in}}{\pgfqpoint{2.689355in}{4.434582in}}{\pgfqpoint{2.689355in}{4.428759in}}%
\pgfpathcurveto{\pgfqpoint{2.689355in}{4.422935in}}{\pgfqpoint{2.691669in}{4.417348in}}{\pgfqpoint{2.695787in}{4.413230in}}%
\pgfpathcurveto{\pgfqpoint{2.699906in}{4.409112in}}{\pgfqpoint{2.705492in}{4.406798in}}{\pgfqpoint{2.711316in}{4.406798in}}%
\pgfpathlineto{\pgfqpoint{2.711316in}{4.406798in}}%
\pgfpathclose%
\pgfusepath{stroke,fill}%
\end{pgfscope}%
\begin{pgfscope}%
\pgfpathrectangle{\pgfqpoint{1.000000in}{0.979904in}}{\pgfqpoint{6.200000in}{5.960192in}}%
\pgfusepath{clip}%
\pgfsetbuttcap%
\pgfsetroundjoin%
\definecolor{currentfill}{rgb}{0.800000,0.200000,0.200000}%
\pgfsetfillcolor{currentfill}%
\pgfsetlinewidth{1.003750pt}%
\definecolor{currentstroke}{rgb}{0.800000,0.200000,0.200000}%
\pgfsetstrokecolor{currentstroke}%
\pgfsetdash{}{0pt}%
\pgfpathmoveto{\pgfqpoint{2.612026in}{4.399203in}}%
\pgfpathcurveto{\pgfqpoint{2.617850in}{4.399203in}}{\pgfqpoint{2.623436in}{4.401516in}}{\pgfqpoint{2.627554in}{4.405635in}}%
\pgfpathcurveto{\pgfqpoint{2.631672in}{4.409753in}}{\pgfqpoint{2.633986in}{4.415339in}}{\pgfqpoint{2.633986in}{4.421163in}}%
\pgfpathcurveto{\pgfqpoint{2.633986in}{4.426987in}}{\pgfqpoint{2.631672in}{4.432573in}}{\pgfqpoint{2.627554in}{4.436691in}}%
\pgfpathcurveto{\pgfqpoint{2.623436in}{4.440809in}}{\pgfqpoint{2.617850in}{4.443123in}}{\pgfqpoint{2.612026in}{4.443123in}}%
\pgfpathcurveto{\pgfqpoint{2.606202in}{4.443123in}}{\pgfqpoint{2.600615in}{4.440809in}}{\pgfqpoint{2.596497in}{4.436691in}}%
\pgfpathcurveto{\pgfqpoint{2.592379in}{4.432573in}}{\pgfqpoint{2.590065in}{4.426987in}}{\pgfqpoint{2.590065in}{4.421163in}}%
\pgfpathcurveto{\pgfqpoint{2.590065in}{4.415339in}}{\pgfqpoint{2.592379in}{4.409753in}}{\pgfqpoint{2.596497in}{4.405635in}}%
\pgfpathcurveto{\pgfqpoint{2.600615in}{4.401516in}}{\pgfqpoint{2.606202in}{4.399203in}}{\pgfqpoint{2.612026in}{4.399203in}}%
\pgfpathlineto{\pgfqpoint{2.612026in}{4.399203in}}%
\pgfpathclose%
\pgfusepath{stroke,fill}%
\end{pgfscope}%
\begin{pgfscope}%
\pgfpathrectangle{\pgfqpoint{1.000000in}{0.979904in}}{\pgfqpoint{6.200000in}{5.960192in}}%
\pgfusepath{clip}%
\pgfsetbuttcap%
\pgfsetroundjoin%
\definecolor{currentfill}{rgb}{0.800000,0.200000,0.200000}%
\pgfsetfillcolor{currentfill}%
\pgfsetlinewidth{1.003750pt}%
\definecolor{currentstroke}{rgb}{0.800000,0.200000,0.200000}%
\pgfsetstrokecolor{currentstroke}%
\pgfsetdash{}{0pt}%
\pgfpathmoveto{\pgfqpoint{2.532731in}{4.313140in}}%
\pgfpathcurveto{\pgfqpoint{2.538555in}{4.313140in}}{\pgfqpoint{2.544142in}{4.315454in}}{\pgfqpoint{2.548260in}{4.319572in}}%
\pgfpathcurveto{\pgfqpoint{2.552378in}{4.323690in}}{\pgfqpoint{2.554692in}{4.329276in}}{\pgfqpoint{2.554692in}{4.335100in}}%
\pgfpathcurveto{\pgfqpoint{2.554692in}{4.340924in}}{\pgfqpoint{2.552378in}{4.346510in}}{\pgfqpoint{2.548260in}{4.350629in}}%
\pgfpathcurveto{\pgfqpoint{2.544142in}{4.354747in}}{\pgfqpoint{2.538555in}{4.357061in}}{\pgfqpoint{2.532731in}{4.357061in}}%
\pgfpathcurveto{\pgfqpoint{2.526908in}{4.357061in}}{\pgfqpoint{2.521321in}{4.354747in}}{\pgfqpoint{2.517203in}{4.350629in}}%
\pgfpathcurveto{\pgfqpoint{2.513085in}{4.346510in}}{\pgfqpoint{2.510771in}{4.340924in}}{\pgfqpoint{2.510771in}{4.335100in}}%
\pgfpathcurveto{\pgfqpoint{2.510771in}{4.329276in}}{\pgfqpoint{2.513085in}{4.323690in}}{\pgfqpoint{2.517203in}{4.319572in}}%
\pgfpathcurveto{\pgfqpoint{2.521321in}{4.315454in}}{\pgfqpoint{2.526908in}{4.313140in}}{\pgfqpoint{2.532731in}{4.313140in}}%
\pgfpathlineto{\pgfqpoint{2.532731in}{4.313140in}}%
\pgfpathclose%
\pgfusepath{stroke,fill}%
\end{pgfscope}%
\begin{pgfscope}%
\pgfpathrectangle{\pgfqpoint{1.000000in}{0.979904in}}{\pgfqpoint{6.200000in}{5.960192in}}%
\pgfusepath{clip}%
\pgfsetbuttcap%
\pgfsetroundjoin%
\definecolor{currentfill}{rgb}{0.800000,0.200000,0.200000}%
\pgfsetfillcolor{currentfill}%
\pgfsetlinewidth{1.003750pt}%
\definecolor{currentstroke}{rgb}{0.800000,0.200000,0.200000}%
\pgfsetstrokecolor{currentstroke}%
\pgfsetdash{}{0pt}%
\pgfpathmoveto{\pgfqpoint{2.426170in}{4.329837in}}%
\pgfpathcurveto{\pgfqpoint{2.431994in}{4.329837in}}{\pgfqpoint{2.437580in}{4.332151in}}{\pgfqpoint{2.441698in}{4.336269in}}%
\pgfpathcurveto{\pgfqpoint{2.445816in}{4.340387in}}{\pgfqpoint{2.448130in}{4.345974in}}{\pgfqpoint{2.448130in}{4.351798in}}%
\pgfpathcurveto{\pgfqpoint{2.448130in}{4.357621in}}{\pgfqpoint{2.445816in}{4.363208in}}{\pgfqpoint{2.441698in}{4.367326in}}%
\pgfpathcurveto{\pgfqpoint{2.437580in}{4.371444in}}{\pgfqpoint{2.431994in}{4.373758in}}{\pgfqpoint{2.426170in}{4.373758in}}%
\pgfpathcurveto{\pgfqpoint{2.420346in}{4.373758in}}{\pgfqpoint{2.414760in}{4.371444in}}{\pgfqpoint{2.410642in}{4.367326in}}%
\pgfpathcurveto{\pgfqpoint{2.406524in}{4.363208in}}{\pgfqpoint{2.404210in}{4.357621in}}{\pgfqpoint{2.404210in}{4.351798in}}%
\pgfpathcurveto{\pgfqpoint{2.404210in}{4.345974in}}{\pgfqpoint{2.406524in}{4.340387in}}{\pgfqpoint{2.410642in}{4.336269in}}%
\pgfpathcurveto{\pgfqpoint{2.414760in}{4.332151in}}{\pgfqpoint{2.420346in}{4.329837in}}{\pgfqpoint{2.426170in}{4.329837in}}%
\pgfpathlineto{\pgfqpoint{2.426170in}{4.329837in}}%
\pgfpathclose%
\pgfusepath{stroke,fill}%
\end{pgfscope}%
\begin{pgfscope}%
\pgfpathrectangle{\pgfqpoint{1.000000in}{0.979904in}}{\pgfqpoint{6.200000in}{5.960192in}}%
\pgfusepath{clip}%
\pgfsetbuttcap%
\pgfsetroundjoin%
\definecolor{currentfill}{rgb}{0.800000,0.200000,0.200000}%
\pgfsetfillcolor{currentfill}%
\pgfsetlinewidth{1.003750pt}%
\definecolor{currentstroke}{rgb}{0.800000,0.200000,0.200000}%
\pgfsetstrokecolor{currentstroke}%
\pgfsetdash{}{0pt}%
\pgfpathmoveto{\pgfqpoint{2.284461in}{4.413822in}}%
\pgfpathcurveto{\pgfqpoint{2.290285in}{4.413822in}}{\pgfqpoint{2.295871in}{4.416136in}}{\pgfqpoint{2.299989in}{4.420254in}}%
\pgfpathcurveto{\pgfqpoint{2.304107in}{4.424372in}}{\pgfqpoint{2.306421in}{4.429958in}}{\pgfqpoint{2.306421in}{4.435782in}}%
\pgfpathcurveto{\pgfqpoint{2.306421in}{4.441606in}}{\pgfqpoint{2.304107in}{4.447192in}}{\pgfqpoint{2.299989in}{4.451311in}}%
\pgfpathcurveto{\pgfqpoint{2.295871in}{4.455429in}}{\pgfqpoint{2.290285in}{4.457743in}}{\pgfqpoint{2.284461in}{4.457743in}}%
\pgfpathcurveto{\pgfqpoint{2.278637in}{4.457743in}}{\pgfqpoint{2.273051in}{4.455429in}}{\pgfqpoint{2.268933in}{4.451311in}}%
\pgfpathcurveto{\pgfqpoint{2.264814in}{4.447192in}}{\pgfqpoint{2.262501in}{4.441606in}}{\pgfqpoint{2.262501in}{4.435782in}}%
\pgfpathcurveto{\pgfqpoint{2.262501in}{4.429958in}}{\pgfqpoint{2.264814in}{4.424372in}}{\pgfqpoint{2.268933in}{4.420254in}}%
\pgfpathcurveto{\pgfqpoint{2.273051in}{4.416136in}}{\pgfqpoint{2.278637in}{4.413822in}}{\pgfqpoint{2.284461in}{4.413822in}}%
\pgfpathlineto{\pgfqpoint{2.284461in}{4.413822in}}%
\pgfpathclose%
\pgfusepath{stroke,fill}%
\end{pgfscope}%
\begin{pgfscope}%
\pgfpathrectangle{\pgfqpoint{1.000000in}{0.979904in}}{\pgfqpoint{6.200000in}{5.960192in}}%
\pgfusepath{clip}%
\pgfsetbuttcap%
\pgfsetroundjoin%
\definecolor{currentfill}{rgb}{0.800000,0.200000,0.200000}%
\pgfsetfillcolor{currentfill}%
\pgfsetlinewidth{1.003750pt}%
\definecolor{currentstroke}{rgb}{0.800000,0.200000,0.200000}%
\pgfsetstrokecolor{currentstroke}%
\pgfsetdash{}{0pt}%
\pgfpathmoveto{\pgfqpoint{2.234601in}{4.274715in}}%
\pgfpathcurveto{\pgfqpoint{2.240425in}{4.274715in}}{\pgfqpoint{2.246011in}{4.277029in}}{\pgfqpoint{2.250130in}{4.281147in}}%
\pgfpathcurveto{\pgfqpoint{2.254248in}{4.285265in}}{\pgfqpoint{2.256562in}{4.290851in}}{\pgfqpoint{2.256562in}{4.296675in}}%
\pgfpathcurveto{\pgfqpoint{2.256562in}{4.302499in}}{\pgfqpoint{2.254248in}{4.308085in}}{\pgfqpoint{2.250130in}{4.312204in}}%
\pgfpathcurveto{\pgfqpoint{2.246011in}{4.316322in}}{\pgfqpoint{2.240425in}{4.318636in}}{\pgfqpoint{2.234601in}{4.318636in}}%
\pgfpathcurveto{\pgfqpoint{2.228777in}{4.318636in}}{\pgfqpoint{2.223191in}{4.316322in}}{\pgfqpoint{2.219073in}{4.312204in}}%
\pgfpathcurveto{\pgfqpoint{2.214955in}{4.308085in}}{\pgfqpoint{2.212641in}{4.302499in}}{\pgfqpoint{2.212641in}{4.296675in}}%
\pgfpathcurveto{\pgfqpoint{2.212641in}{4.290851in}}{\pgfqpoint{2.214955in}{4.285265in}}{\pgfqpoint{2.219073in}{4.281147in}}%
\pgfpathcurveto{\pgfqpoint{2.223191in}{4.277029in}}{\pgfqpoint{2.228777in}{4.274715in}}{\pgfqpoint{2.234601in}{4.274715in}}%
\pgfpathlineto{\pgfqpoint{2.234601in}{4.274715in}}%
\pgfpathclose%
\pgfusepath{stroke,fill}%
\end{pgfscope}%
\begin{pgfscope}%
\pgfpathrectangle{\pgfqpoint{1.000000in}{0.979904in}}{\pgfqpoint{6.200000in}{5.960192in}}%
\pgfusepath{clip}%
\pgfsetbuttcap%
\pgfsetroundjoin%
\definecolor{currentfill}{rgb}{0.800000,0.200000,0.200000}%
\pgfsetfillcolor{currentfill}%
\pgfsetlinewidth{1.003750pt}%
\definecolor{currentstroke}{rgb}{0.800000,0.200000,0.200000}%
\pgfsetstrokecolor{currentstroke}%
\pgfsetdash{}{0pt}%
\pgfpathmoveto{\pgfqpoint{2.146213in}{4.228345in}}%
\pgfpathcurveto{\pgfqpoint{2.152037in}{4.228345in}}{\pgfqpoint{2.157623in}{4.230658in}}{\pgfqpoint{2.161742in}{4.234777in}}%
\pgfpathcurveto{\pgfqpoint{2.165860in}{4.238895in}}{\pgfqpoint{2.168174in}{4.244481in}}{\pgfqpoint{2.168174in}{4.250305in}}%
\pgfpathcurveto{\pgfqpoint{2.168174in}{4.256129in}}{\pgfqpoint{2.165860in}{4.261715in}}{\pgfqpoint{2.161742in}{4.265833in}}%
\pgfpathcurveto{\pgfqpoint{2.157623in}{4.269951in}}{\pgfqpoint{2.152037in}{4.272265in}}{\pgfqpoint{2.146213in}{4.272265in}}%
\pgfpathcurveto{\pgfqpoint{2.140389in}{4.272265in}}{\pgfqpoint{2.134803in}{4.269951in}}{\pgfqpoint{2.130685in}{4.265833in}}%
\pgfpathcurveto{\pgfqpoint{2.126567in}{4.261715in}}{\pgfqpoint{2.124253in}{4.256129in}}{\pgfqpoint{2.124253in}{4.250305in}}%
\pgfpathcurveto{\pgfqpoint{2.124253in}{4.244481in}}{\pgfqpoint{2.126567in}{4.238895in}}{\pgfqpoint{2.130685in}{4.234777in}}%
\pgfpathcurveto{\pgfqpoint{2.134803in}{4.230658in}}{\pgfqpoint{2.140389in}{4.228345in}}{\pgfqpoint{2.146213in}{4.228345in}}%
\pgfpathlineto{\pgfqpoint{2.146213in}{4.228345in}}%
\pgfpathclose%
\pgfusepath{stroke,fill}%
\end{pgfscope}%
\begin{pgfscope}%
\pgfpathrectangle{\pgfqpoint{1.000000in}{0.979904in}}{\pgfqpoint{6.200000in}{5.960192in}}%
\pgfusepath{clip}%
\pgfsetbuttcap%
\pgfsetroundjoin%
\definecolor{currentfill}{rgb}{0.800000,0.200000,0.200000}%
\pgfsetfillcolor{currentfill}%
\pgfsetlinewidth{1.003750pt}%
\definecolor{currentstroke}{rgb}{0.800000,0.200000,0.200000}%
\pgfsetstrokecolor{currentstroke}%
\pgfsetdash{}{0pt}%
\pgfpathmoveto{\pgfqpoint{2.042699in}{4.203785in}}%
\pgfpathcurveto{\pgfqpoint{2.048523in}{4.203785in}}{\pgfqpoint{2.054109in}{4.206099in}}{\pgfqpoint{2.058227in}{4.210217in}}%
\pgfpathcurveto{\pgfqpoint{2.062346in}{4.214335in}}{\pgfqpoint{2.064659in}{4.219922in}}{\pgfqpoint{2.064659in}{4.225745in}}%
\pgfpathcurveto{\pgfqpoint{2.064659in}{4.231569in}}{\pgfqpoint{2.062346in}{4.237156in}}{\pgfqpoint{2.058227in}{4.241274in}}%
\pgfpathcurveto{\pgfqpoint{2.054109in}{4.245392in}}{\pgfqpoint{2.048523in}{4.247706in}}{\pgfqpoint{2.042699in}{4.247706in}}%
\pgfpathcurveto{\pgfqpoint{2.036875in}{4.247706in}}{\pgfqpoint{2.031289in}{4.245392in}}{\pgfqpoint{2.027171in}{4.241274in}}%
\pgfpathcurveto{\pgfqpoint{2.023053in}{4.237156in}}{\pgfqpoint{2.020739in}{4.231569in}}{\pgfqpoint{2.020739in}{4.225745in}}%
\pgfpathcurveto{\pgfqpoint{2.020739in}{4.219922in}}{\pgfqpoint{2.023053in}{4.214335in}}{\pgfqpoint{2.027171in}{4.210217in}}%
\pgfpathcurveto{\pgfqpoint{2.031289in}{4.206099in}}{\pgfqpoint{2.036875in}{4.203785in}}{\pgfqpoint{2.042699in}{4.203785in}}%
\pgfpathlineto{\pgfqpoint{2.042699in}{4.203785in}}%
\pgfpathclose%
\pgfusepath{stroke,fill}%
\end{pgfscope}%
\begin{pgfscope}%
\pgfpathrectangle{\pgfqpoint{1.000000in}{0.979904in}}{\pgfqpoint{6.200000in}{5.960192in}}%
\pgfusepath{clip}%
\pgfsetbuttcap%
\pgfsetroundjoin%
\definecolor{currentfill}{rgb}{0.800000,0.200000,0.200000}%
\pgfsetfillcolor{currentfill}%
\pgfsetlinewidth{1.003750pt}%
\definecolor{currentstroke}{rgb}{0.800000,0.200000,0.200000}%
\pgfsetstrokecolor{currentstroke}%
\pgfsetdash{}{0pt}%
\pgfpathmoveto{\pgfqpoint{2.021645in}{4.063148in}}%
\pgfpathcurveto{\pgfqpoint{2.027469in}{4.063148in}}{\pgfqpoint{2.033055in}{4.065461in}}{\pgfqpoint{2.037173in}{4.069580in}}%
\pgfpathcurveto{\pgfqpoint{2.041292in}{4.073698in}}{\pgfqpoint{2.043605in}{4.079284in}}{\pgfqpoint{2.043605in}{4.085108in}}%
\pgfpathcurveto{\pgfqpoint{2.043605in}{4.090932in}}{\pgfqpoint{2.041292in}{4.096518in}}{\pgfqpoint{2.037173in}{4.100636in}}%
\pgfpathcurveto{\pgfqpoint{2.033055in}{4.104754in}}{\pgfqpoint{2.027469in}{4.107068in}}{\pgfqpoint{2.021645in}{4.107068in}}%
\pgfpathcurveto{\pgfqpoint{2.015821in}{4.107068in}}{\pgfqpoint{2.010235in}{4.104754in}}{\pgfqpoint{2.006117in}{4.100636in}}%
\pgfpathcurveto{\pgfqpoint{2.001999in}{4.096518in}}{\pgfqpoint{1.999685in}{4.090932in}}{\pgfqpoint{1.999685in}{4.085108in}}%
\pgfpathcurveto{\pgfqpoint{1.999685in}{4.079284in}}{\pgfqpoint{2.001999in}{4.073698in}}{\pgfqpoint{2.006117in}{4.069580in}}%
\pgfpathcurveto{\pgfqpoint{2.010235in}{4.065461in}}{\pgfqpoint{2.015821in}{4.063148in}}{\pgfqpoint{2.021645in}{4.063148in}}%
\pgfpathlineto{\pgfqpoint{2.021645in}{4.063148in}}%
\pgfpathclose%
\pgfusepath{stroke,fill}%
\end{pgfscope}%
\begin{pgfscope}%
\pgfpathrectangle{\pgfqpoint{1.000000in}{0.979904in}}{\pgfqpoint{6.200000in}{5.960192in}}%
\pgfusepath{clip}%
\pgfsetbuttcap%
\pgfsetroundjoin%
\definecolor{currentfill}{rgb}{0.800000,0.200000,0.200000}%
\pgfsetfillcolor{currentfill}%
\pgfsetlinewidth{1.003750pt}%
\definecolor{currentstroke}{rgb}{0.800000,0.200000,0.200000}%
\pgfsetstrokecolor{currentstroke}%
\pgfsetdash{}{0pt}%
\pgfpathmoveto{\pgfqpoint{1.858676in}{4.105497in}}%
\pgfpathcurveto{\pgfqpoint{1.864500in}{4.105497in}}{\pgfqpoint{1.870086in}{4.107811in}}{\pgfqpoint{1.874204in}{4.111929in}}%
\pgfpathcurveto{\pgfqpoint{1.878322in}{4.116047in}}{\pgfqpoint{1.880636in}{4.121633in}}{\pgfqpoint{1.880636in}{4.127457in}}%
\pgfpathcurveto{\pgfqpoint{1.880636in}{4.133281in}}{\pgfqpoint{1.878322in}{4.138867in}}{\pgfqpoint{1.874204in}{4.142985in}}%
\pgfpathcurveto{\pgfqpoint{1.870086in}{4.147103in}}{\pgfqpoint{1.864500in}{4.149417in}}{\pgfqpoint{1.858676in}{4.149417in}}%
\pgfpathcurveto{\pgfqpoint{1.852852in}{4.149417in}}{\pgfqpoint{1.847266in}{4.147103in}}{\pgfqpoint{1.843148in}{4.142985in}}%
\pgfpathcurveto{\pgfqpoint{1.839030in}{4.138867in}}{\pgfqpoint{1.836716in}{4.133281in}}{\pgfqpoint{1.836716in}{4.127457in}}%
\pgfpathcurveto{\pgfqpoint{1.836716in}{4.121633in}}{\pgfqpoint{1.839030in}{4.116047in}}{\pgfqpoint{1.843148in}{4.111929in}}%
\pgfpathcurveto{\pgfqpoint{1.847266in}{4.107811in}}{\pgfqpoint{1.852852in}{4.105497in}}{\pgfqpoint{1.858676in}{4.105497in}}%
\pgfpathlineto{\pgfqpoint{1.858676in}{4.105497in}}%
\pgfpathclose%
\pgfusepath{stroke,fill}%
\end{pgfscope}%
\begin{pgfscope}%
\pgfpathrectangle{\pgfqpoint{1.000000in}{0.979904in}}{\pgfqpoint{6.200000in}{5.960192in}}%
\pgfusepath{clip}%
\pgfsetbuttcap%
\pgfsetroundjoin%
\definecolor{currentfill}{rgb}{0.800000,0.200000,0.200000}%
\pgfsetfillcolor{currentfill}%
\pgfsetlinewidth{1.003750pt}%
\definecolor{currentstroke}{rgb}{0.800000,0.200000,0.200000}%
\pgfsetstrokecolor{currentstroke}%
\pgfsetdash{}{0pt}%
\pgfpathmoveto{\pgfqpoint{1.770265in}{4.047010in}}%
\pgfpathcurveto{\pgfqpoint{1.776088in}{4.047010in}}{\pgfqpoint{1.781675in}{4.049324in}}{\pgfqpoint{1.785793in}{4.053442in}}%
\pgfpathcurveto{\pgfqpoint{1.789911in}{4.057560in}}{\pgfqpoint{1.792225in}{4.063146in}}{\pgfqpoint{1.792225in}{4.068970in}}%
\pgfpathcurveto{\pgfqpoint{1.792225in}{4.074794in}}{\pgfqpoint{1.789911in}{4.080380in}}{\pgfqpoint{1.785793in}{4.084498in}}%
\pgfpathcurveto{\pgfqpoint{1.781675in}{4.088617in}}{\pgfqpoint{1.776088in}{4.090930in}}{\pgfqpoint{1.770265in}{4.090930in}}%
\pgfpathcurveto{\pgfqpoint{1.764441in}{4.090930in}}{\pgfqpoint{1.758854in}{4.088617in}}{\pgfqpoint{1.754736in}{4.084498in}}%
\pgfpathcurveto{\pgfqpoint{1.750618in}{4.080380in}}{\pgfqpoint{1.748304in}{4.074794in}}{\pgfqpoint{1.748304in}{4.068970in}}%
\pgfpathcurveto{\pgfqpoint{1.748304in}{4.063146in}}{\pgfqpoint{1.750618in}{4.057560in}}{\pgfqpoint{1.754736in}{4.053442in}}%
\pgfpathcurveto{\pgfqpoint{1.758854in}{4.049324in}}{\pgfqpoint{1.764441in}{4.047010in}}{\pgfqpoint{1.770265in}{4.047010in}}%
\pgfpathlineto{\pgfqpoint{1.770265in}{4.047010in}}%
\pgfpathclose%
\pgfusepath{stroke,fill}%
\end{pgfscope}%
\begin{pgfscope}%
\pgfpathrectangle{\pgfqpoint{1.000000in}{0.979904in}}{\pgfqpoint{6.200000in}{5.960192in}}%
\pgfusepath{clip}%
\pgfsetbuttcap%
\pgfsetroundjoin%
\definecolor{currentfill}{rgb}{0.800000,0.200000,0.200000}%
\pgfsetfillcolor{currentfill}%
\pgfsetlinewidth{1.003750pt}%
\definecolor{currentstroke}{rgb}{0.800000,0.200000,0.200000}%
\pgfsetstrokecolor{currentstroke}%
\pgfsetdash{}{0pt}%
\pgfpathmoveto{\pgfqpoint{1.732845in}{3.939441in}}%
\pgfpathcurveto{\pgfqpoint{1.738669in}{3.939441in}}{\pgfqpoint{1.744255in}{3.941754in}}{\pgfqpoint{1.748373in}{3.945873in}}%
\pgfpathcurveto{\pgfqpoint{1.752491in}{3.949991in}}{\pgfqpoint{1.754805in}{3.955577in}}{\pgfqpoint{1.754805in}{3.961401in}}%
\pgfpathcurveto{\pgfqpoint{1.754805in}{3.967225in}}{\pgfqpoint{1.752491in}{3.972811in}}{\pgfqpoint{1.748373in}{3.976929in}}%
\pgfpathcurveto{\pgfqpoint{1.744255in}{3.981047in}}{\pgfqpoint{1.738669in}{3.983361in}}{\pgfqpoint{1.732845in}{3.983361in}}%
\pgfpathcurveto{\pgfqpoint{1.727021in}{3.983361in}}{\pgfqpoint{1.721435in}{3.981047in}}{\pgfqpoint{1.717317in}{3.976929in}}%
\pgfpathcurveto{\pgfqpoint{1.713199in}{3.972811in}}{\pgfqpoint{1.710885in}{3.967225in}}{\pgfqpoint{1.710885in}{3.961401in}}%
\pgfpathcurveto{\pgfqpoint{1.710885in}{3.955577in}}{\pgfqpoint{1.713199in}{3.949991in}}{\pgfqpoint{1.717317in}{3.945873in}}%
\pgfpathcurveto{\pgfqpoint{1.721435in}{3.941754in}}{\pgfqpoint{1.727021in}{3.939441in}}{\pgfqpoint{1.732845in}{3.939441in}}%
\pgfpathlineto{\pgfqpoint{1.732845in}{3.939441in}}%
\pgfpathclose%
\pgfusepath{stroke,fill}%
\end{pgfscope}%
\begin{pgfscope}%
\pgfpathrectangle{\pgfqpoint{1.000000in}{0.979904in}}{\pgfqpoint{6.200000in}{5.960192in}}%
\pgfusepath{clip}%
\pgfsetbuttcap%
\pgfsetroundjoin%
\definecolor{currentfill}{rgb}{0.800000,0.200000,0.200000}%
\pgfsetfillcolor{currentfill}%
\pgfsetlinewidth{1.003750pt}%
\definecolor{currentstroke}{rgb}{0.800000,0.200000,0.200000}%
\pgfsetstrokecolor{currentstroke}%
\pgfsetdash{}{0pt}%
\pgfpathmoveto{\pgfqpoint{1.614032in}{3.902954in}}%
\pgfpathcurveto{\pgfqpoint{1.619856in}{3.902954in}}{\pgfqpoint{1.625442in}{3.905268in}}{\pgfqpoint{1.629560in}{3.909386in}}%
\pgfpathcurveto{\pgfqpoint{1.633678in}{3.913504in}}{\pgfqpoint{1.635992in}{3.919090in}}{\pgfqpoint{1.635992in}{3.924914in}}%
\pgfpathcurveto{\pgfqpoint{1.635992in}{3.930738in}}{\pgfqpoint{1.633678in}{3.936324in}}{\pgfqpoint{1.629560in}{3.940442in}}%
\pgfpathcurveto{\pgfqpoint{1.625442in}{3.944561in}}{\pgfqpoint{1.619856in}{3.946874in}}{\pgfqpoint{1.614032in}{3.946874in}}%
\pgfpathcurveto{\pgfqpoint{1.608208in}{3.946874in}}{\pgfqpoint{1.602622in}{3.944561in}}{\pgfqpoint{1.598504in}{3.940442in}}%
\pgfpathcurveto{\pgfqpoint{1.594385in}{3.936324in}}{\pgfqpoint{1.592072in}{3.930738in}}{\pgfqpoint{1.592072in}{3.924914in}}%
\pgfpathcurveto{\pgfqpoint{1.592072in}{3.919090in}}{\pgfqpoint{1.594385in}{3.913504in}}{\pgfqpoint{1.598504in}{3.909386in}}%
\pgfpathcurveto{\pgfqpoint{1.602622in}{3.905268in}}{\pgfqpoint{1.608208in}{3.902954in}}{\pgfqpoint{1.614032in}{3.902954in}}%
\pgfpathlineto{\pgfqpoint{1.614032in}{3.902954in}}%
\pgfpathclose%
\pgfusepath{stroke,fill}%
\end{pgfscope}%
\begin{pgfscope}%
\pgfpathrectangle{\pgfqpoint{1.000000in}{0.979904in}}{\pgfqpoint{6.200000in}{5.960192in}}%
\pgfusepath{clip}%
\pgfsetbuttcap%
\pgfsetroundjoin%
\definecolor{currentfill}{rgb}{0.800000,0.200000,0.200000}%
\pgfsetfillcolor{currentfill}%
\pgfsetlinewidth{1.003750pt}%
\definecolor{currentstroke}{rgb}{0.800000,0.200000,0.200000}%
\pgfsetstrokecolor{currentstroke}%
\pgfsetdash{}{0pt}%
\pgfpathmoveto{\pgfqpoint{1.681058in}{3.727617in}}%
\pgfpathcurveto{\pgfqpoint{1.686882in}{3.727617in}}{\pgfqpoint{1.692468in}{3.729931in}}{\pgfqpoint{1.696586in}{3.734049in}}%
\pgfpathcurveto{\pgfqpoint{1.700704in}{3.738167in}}{\pgfqpoint{1.703018in}{3.743753in}}{\pgfqpoint{1.703018in}{3.749577in}}%
\pgfpathcurveto{\pgfqpoint{1.703018in}{3.755401in}}{\pgfqpoint{1.700704in}{3.760987in}}{\pgfqpoint{1.696586in}{3.765105in}}%
\pgfpathcurveto{\pgfqpoint{1.692468in}{3.769224in}}{\pgfqpoint{1.686882in}{3.771538in}}{\pgfqpoint{1.681058in}{3.771538in}}%
\pgfpathcurveto{\pgfqpoint{1.675234in}{3.771538in}}{\pgfqpoint{1.669648in}{3.769224in}}{\pgfqpoint{1.665530in}{3.765105in}}%
\pgfpathcurveto{\pgfqpoint{1.661412in}{3.760987in}}{\pgfqpoint{1.659098in}{3.755401in}}{\pgfqpoint{1.659098in}{3.749577in}}%
\pgfpathcurveto{\pgfqpoint{1.659098in}{3.743753in}}{\pgfqpoint{1.661412in}{3.738167in}}{\pgfqpoint{1.665530in}{3.734049in}}%
\pgfpathcurveto{\pgfqpoint{1.669648in}{3.729931in}}{\pgfqpoint{1.675234in}{3.727617in}}{\pgfqpoint{1.681058in}{3.727617in}}%
\pgfpathlineto{\pgfqpoint{1.681058in}{3.727617in}}%
\pgfpathclose%
\pgfusepath{stroke,fill}%
\end{pgfscope}%
\begin{pgfscope}%
\pgfpathrectangle{\pgfqpoint{1.000000in}{0.979904in}}{\pgfqpoint{6.200000in}{5.960192in}}%
\pgfusepath{clip}%
\pgfsetbuttcap%
\pgfsetroundjoin%
\definecolor{currentfill}{rgb}{0.800000,0.200000,0.200000}%
\pgfsetfillcolor{currentfill}%
\pgfsetlinewidth{1.003750pt}%
\definecolor{currentstroke}{rgb}{0.800000,0.200000,0.200000}%
\pgfsetstrokecolor{currentstroke}%
\pgfsetdash{}{0pt}%
\pgfpathmoveto{\pgfqpoint{1.484590in}{3.732897in}}%
\pgfpathcurveto{\pgfqpoint{1.490414in}{3.732897in}}{\pgfqpoint{1.496000in}{3.735211in}}{\pgfqpoint{1.500118in}{3.739329in}}%
\pgfpathcurveto{\pgfqpoint{1.504236in}{3.743447in}}{\pgfqpoint{1.506550in}{3.749033in}}{\pgfqpoint{1.506550in}{3.754857in}}%
\pgfpathcurveto{\pgfqpoint{1.506550in}{3.760681in}}{\pgfqpoint{1.504236in}{3.766267in}}{\pgfqpoint{1.500118in}{3.770385in}}%
\pgfpathcurveto{\pgfqpoint{1.496000in}{3.774503in}}{\pgfqpoint{1.490414in}{3.776817in}}{\pgfqpoint{1.484590in}{3.776817in}}%
\pgfpathcurveto{\pgfqpoint{1.478766in}{3.776817in}}{\pgfqpoint{1.473180in}{3.774503in}}{\pgfqpoint{1.469062in}{3.770385in}}%
\pgfpathcurveto{\pgfqpoint{1.464943in}{3.766267in}}{\pgfqpoint{1.462630in}{3.760681in}}{\pgfqpoint{1.462630in}{3.754857in}}%
\pgfpathcurveto{\pgfqpoint{1.462630in}{3.749033in}}{\pgfqpoint{1.464943in}{3.743447in}}{\pgfqpoint{1.469062in}{3.739329in}}%
\pgfpathcurveto{\pgfqpoint{1.473180in}{3.735211in}}{\pgfqpoint{1.478766in}{3.732897in}}{\pgfqpoint{1.484590in}{3.732897in}}%
\pgfpathlineto{\pgfqpoint{1.484590in}{3.732897in}}%
\pgfpathclose%
\pgfusepath{stroke,fill}%
\end{pgfscope}%
\begin{pgfscope}%
\pgfpathrectangle{\pgfqpoint{1.000000in}{0.979904in}}{\pgfqpoint{6.200000in}{5.960192in}}%
\pgfusepath{clip}%
\pgfsetbuttcap%
\pgfsetroundjoin%
\definecolor{currentfill}{rgb}{0.800000,0.200000,0.200000}%
\pgfsetfillcolor{currentfill}%
\pgfsetlinewidth{1.003750pt}%
\definecolor{currentstroke}{rgb}{0.800000,0.200000,0.200000}%
\pgfsetstrokecolor{currentstroke}%
\pgfsetdash{}{0pt}%
\pgfpathmoveto{\pgfqpoint{1.553919in}{3.576705in}}%
\pgfpathcurveto{\pgfqpoint{1.559743in}{3.576705in}}{\pgfqpoint{1.565329in}{3.579019in}}{\pgfqpoint{1.569447in}{3.583137in}}%
\pgfpathcurveto{\pgfqpoint{1.573565in}{3.587255in}}{\pgfqpoint{1.575879in}{3.592841in}}{\pgfqpoint{1.575879in}{3.598665in}}%
\pgfpathcurveto{\pgfqpoint{1.575879in}{3.604489in}}{\pgfqpoint{1.573565in}{3.610075in}}{\pgfqpoint{1.569447in}{3.614193in}}%
\pgfpathcurveto{\pgfqpoint{1.565329in}{3.618311in}}{\pgfqpoint{1.559743in}{3.620625in}}{\pgfqpoint{1.553919in}{3.620625in}}%
\pgfpathcurveto{\pgfqpoint{1.548095in}{3.620625in}}{\pgfqpoint{1.542509in}{3.618311in}}{\pgfqpoint{1.538391in}{3.614193in}}%
\pgfpathcurveto{\pgfqpoint{1.534273in}{3.610075in}}{\pgfqpoint{1.531959in}{3.604489in}}{\pgfqpoint{1.531959in}{3.598665in}}%
\pgfpathcurveto{\pgfqpoint{1.531959in}{3.592841in}}{\pgfqpoint{1.534273in}{3.587255in}}{\pgfqpoint{1.538391in}{3.583137in}}%
\pgfpathcurveto{\pgfqpoint{1.542509in}{3.579019in}}{\pgfqpoint{1.548095in}{3.576705in}}{\pgfqpoint{1.553919in}{3.576705in}}%
\pgfpathlineto{\pgfqpoint{1.553919in}{3.576705in}}%
\pgfpathclose%
\pgfusepath{stroke,fill}%
\end{pgfscope}%
\begin{pgfscope}%
\pgfpathrectangle{\pgfqpoint{1.000000in}{0.979904in}}{\pgfqpoint{6.200000in}{5.960192in}}%
\pgfusepath{clip}%
\pgfsetbuttcap%
\pgfsetroundjoin%
\definecolor{currentfill}{rgb}{0.800000,0.200000,0.200000}%
\pgfsetfillcolor{currentfill}%
\pgfsetlinewidth{1.003750pt}%
\definecolor{currentstroke}{rgb}{0.800000,0.200000,0.200000}%
\pgfsetstrokecolor{currentstroke}%
\pgfsetdash{}{0pt}%
\pgfpathmoveto{\pgfqpoint{1.538300in}{3.476132in}}%
\pgfpathcurveto{\pgfqpoint{1.544124in}{3.476132in}}{\pgfqpoint{1.549710in}{3.478446in}}{\pgfqpoint{1.553829in}{3.482564in}}%
\pgfpathcurveto{\pgfqpoint{1.557947in}{3.486683in}}{\pgfqpoint{1.560261in}{3.492269in}}{\pgfqpoint{1.560261in}{3.498093in}}%
\pgfpathcurveto{\pgfqpoint{1.560261in}{3.503917in}}{\pgfqpoint{1.557947in}{3.509503in}}{\pgfqpoint{1.553829in}{3.513621in}}%
\pgfpathcurveto{\pgfqpoint{1.549710in}{3.517739in}}{\pgfqpoint{1.544124in}{3.520053in}}{\pgfqpoint{1.538300in}{3.520053in}}%
\pgfpathcurveto{\pgfqpoint{1.532476in}{3.520053in}}{\pgfqpoint{1.526890in}{3.517739in}}{\pgfqpoint{1.522772in}{3.513621in}}%
\pgfpathcurveto{\pgfqpoint{1.518654in}{3.509503in}}{\pgfqpoint{1.516340in}{3.503917in}}{\pgfqpoint{1.516340in}{3.498093in}}%
\pgfpathcurveto{\pgfqpoint{1.516340in}{3.492269in}}{\pgfqpoint{1.518654in}{3.486683in}}{\pgfqpoint{1.522772in}{3.482564in}}%
\pgfpathcurveto{\pgfqpoint{1.526890in}{3.478446in}}{\pgfqpoint{1.532476in}{3.476132in}}{\pgfqpoint{1.538300in}{3.476132in}}%
\pgfpathlineto{\pgfqpoint{1.538300in}{3.476132in}}%
\pgfpathclose%
\pgfusepath{stroke,fill}%
\end{pgfscope}%
\begin{pgfscope}%
\pgfpathrectangle{\pgfqpoint{1.000000in}{0.979904in}}{\pgfqpoint{6.200000in}{5.960192in}}%
\pgfusepath{clip}%
\pgfsetbuttcap%
\pgfsetroundjoin%
\definecolor{currentfill}{rgb}{0.800000,0.200000,0.200000}%
\pgfsetfillcolor{currentfill}%
\pgfsetlinewidth{1.003750pt}%
\definecolor{currentstroke}{rgb}{0.800000,0.200000,0.200000}%
\pgfsetstrokecolor{currentstroke}%
\pgfsetdash{}{0pt}%
\pgfpathmoveto{\pgfqpoint{1.450421in}{3.405372in}}%
\pgfpathcurveto{\pgfqpoint{1.456245in}{3.405372in}}{\pgfqpoint{1.461831in}{3.407686in}}{\pgfqpoint{1.465949in}{3.411804in}}%
\pgfpathcurveto{\pgfqpoint{1.470068in}{3.415922in}}{\pgfqpoint{1.472381in}{3.421509in}}{\pgfqpoint{1.472381in}{3.427332in}}%
\pgfpathcurveto{\pgfqpoint{1.472381in}{3.433156in}}{\pgfqpoint{1.470068in}{3.438743in}}{\pgfqpoint{1.465949in}{3.442861in}}%
\pgfpathcurveto{\pgfqpoint{1.461831in}{3.446979in}}{\pgfqpoint{1.456245in}{3.449293in}}{\pgfqpoint{1.450421in}{3.449293in}}%
\pgfpathcurveto{\pgfqpoint{1.444597in}{3.449293in}}{\pgfqpoint{1.439011in}{3.446979in}}{\pgfqpoint{1.434893in}{3.442861in}}%
\pgfpathcurveto{\pgfqpoint{1.430775in}{3.438743in}}{\pgfqpoint{1.428461in}{3.433156in}}{\pgfqpoint{1.428461in}{3.427332in}}%
\pgfpathcurveto{\pgfqpoint{1.428461in}{3.421509in}}{\pgfqpoint{1.430775in}{3.415922in}}{\pgfqpoint{1.434893in}{3.411804in}}%
\pgfpathcurveto{\pgfqpoint{1.439011in}{3.407686in}}{\pgfqpoint{1.444597in}{3.405372in}}{\pgfqpoint{1.450421in}{3.405372in}}%
\pgfpathlineto{\pgfqpoint{1.450421in}{3.405372in}}%
\pgfpathclose%
\pgfusepath{stroke,fill}%
\end{pgfscope}%
\begin{pgfscope}%
\pgfpathrectangle{\pgfqpoint{1.000000in}{0.979904in}}{\pgfqpoint{6.200000in}{5.960192in}}%
\pgfusepath{clip}%
\pgfsetbuttcap%
\pgfsetroundjoin%
\definecolor{currentfill}{rgb}{0.800000,0.200000,0.200000}%
\pgfsetfillcolor{currentfill}%
\pgfsetlinewidth{1.003750pt}%
\definecolor{currentstroke}{rgb}{0.800000,0.200000,0.200000}%
\pgfsetstrokecolor{currentstroke}%
\pgfsetdash{}{0pt}%
\pgfpathmoveto{\pgfqpoint{1.311800in}{3.341794in}}%
\pgfpathcurveto{\pgfqpoint{1.317624in}{3.341794in}}{\pgfqpoint{1.323210in}{3.344107in}}{\pgfqpoint{1.327328in}{3.348226in}}%
\pgfpathcurveto{\pgfqpoint{1.331446in}{3.352344in}}{\pgfqpoint{1.333760in}{3.357930in}}{\pgfqpoint{1.333760in}{3.363754in}}%
\pgfpathcurveto{\pgfqpoint{1.333760in}{3.369578in}}{\pgfqpoint{1.331446in}{3.375164in}}{\pgfqpoint{1.327328in}{3.379282in}}%
\pgfpathcurveto{\pgfqpoint{1.323210in}{3.383400in}}{\pgfqpoint{1.317624in}{3.385714in}}{\pgfqpoint{1.311800in}{3.385714in}}%
\pgfpathcurveto{\pgfqpoint{1.305976in}{3.385714in}}{\pgfqpoint{1.300390in}{3.383400in}}{\pgfqpoint{1.296272in}{3.379282in}}%
\pgfpathcurveto{\pgfqpoint{1.292154in}{3.375164in}}{\pgfqpoint{1.289840in}{3.369578in}}{\pgfqpoint{1.289840in}{3.363754in}}%
\pgfpathcurveto{\pgfqpoint{1.289840in}{3.357930in}}{\pgfqpoint{1.292154in}{3.352344in}}{\pgfqpoint{1.296272in}{3.348226in}}%
\pgfpathcurveto{\pgfqpoint{1.300390in}{3.344107in}}{\pgfqpoint{1.305976in}{3.341794in}}{\pgfqpoint{1.311800in}{3.341794in}}%
\pgfpathlineto{\pgfqpoint{1.311800in}{3.341794in}}%
\pgfpathclose%
\pgfusepath{stroke,fill}%
\end{pgfscope}%
\begin{pgfscope}%
\pgfpathrectangle{\pgfqpoint{1.000000in}{0.979904in}}{\pgfqpoint{6.200000in}{5.960192in}}%
\pgfusepath{clip}%
\pgfsetbuttcap%
\pgfsetroundjoin%
\definecolor{currentfill}{rgb}{0.800000,0.200000,0.200000}%
\pgfsetfillcolor{currentfill}%
\pgfsetlinewidth{1.003750pt}%
\definecolor{currentstroke}{rgb}{0.800000,0.200000,0.200000}%
\pgfsetstrokecolor{currentstroke}%
\pgfsetdash{}{0pt}%
\pgfpathmoveto{\pgfqpoint{1.382763in}{3.215911in}}%
\pgfpathcurveto{\pgfqpoint{1.388587in}{3.215911in}}{\pgfqpoint{1.394173in}{3.218225in}}{\pgfqpoint{1.398291in}{3.222343in}}%
\pgfpathcurveto{\pgfqpoint{1.402409in}{3.226461in}}{\pgfqpoint{1.404723in}{3.232047in}}{\pgfqpoint{1.404723in}{3.237871in}}%
\pgfpathcurveto{\pgfqpoint{1.404723in}{3.243695in}}{\pgfqpoint{1.402409in}{3.249281in}}{\pgfqpoint{1.398291in}{3.253400in}}%
\pgfpathcurveto{\pgfqpoint{1.394173in}{3.257518in}}{\pgfqpoint{1.388587in}{3.259832in}}{\pgfqpoint{1.382763in}{3.259832in}}%
\pgfpathcurveto{\pgfqpoint{1.376939in}{3.259832in}}{\pgfqpoint{1.371353in}{3.257518in}}{\pgfqpoint{1.367235in}{3.253400in}}%
\pgfpathcurveto{\pgfqpoint{1.363116in}{3.249281in}}{\pgfqpoint{1.360803in}{3.243695in}}{\pgfqpoint{1.360803in}{3.237871in}}%
\pgfpathcurveto{\pgfqpoint{1.360803in}{3.232047in}}{\pgfqpoint{1.363116in}{3.226461in}}{\pgfqpoint{1.367235in}{3.222343in}}%
\pgfpathcurveto{\pgfqpoint{1.371353in}{3.218225in}}{\pgfqpoint{1.376939in}{3.215911in}}{\pgfqpoint{1.382763in}{3.215911in}}%
\pgfpathlineto{\pgfqpoint{1.382763in}{3.215911in}}%
\pgfpathclose%
\pgfusepath{stroke,fill}%
\end{pgfscope}%
\begin{pgfscope}%
\pgfpathrectangle{\pgfqpoint{1.000000in}{0.979904in}}{\pgfqpoint{6.200000in}{5.960192in}}%
\pgfusepath{clip}%
\pgfsetbuttcap%
\pgfsetroundjoin%
\definecolor{currentfill}{rgb}{0.800000,0.200000,0.200000}%
\pgfsetfillcolor{currentfill}%
\pgfsetlinewidth{1.003750pt}%
\definecolor{currentstroke}{rgb}{0.800000,0.200000,0.200000}%
\pgfsetstrokecolor{currentstroke}%
\pgfsetdash{}{0pt}%
\pgfpathmoveto{\pgfqpoint{1.411518in}{3.109237in}}%
\pgfpathcurveto{\pgfqpoint{1.417342in}{3.109237in}}{\pgfqpoint{1.422929in}{3.111551in}}{\pgfqpoint{1.427047in}{3.115669in}}%
\pgfpathcurveto{\pgfqpoint{1.431165in}{3.119787in}}{\pgfqpoint{1.433479in}{3.125373in}}{\pgfqpoint{1.433479in}{3.131197in}}%
\pgfpathcurveto{\pgfqpoint{1.433479in}{3.137021in}}{\pgfqpoint{1.431165in}{3.142607in}}{\pgfqpoint{1.427047in}{3.146725in}}%
\pgfpathcurveto{\pgfqpoint{1.422929in}{3.150844in}}{\pgfqpoint{1.417342in}{3.153157in}}{\pgfqpoint{1.411518in}{3.153157in}}%
\pgfpathcurveto{\pgfqpoint{1.405695in}{3.153157in}}{\pgfqpoint{1.400108in}{3.150844in}}{\pgfqpoint{1.395990in}{3.146725in}}%
\pgfpathcurveto{\pgfqpoint{1.391872in}{3.142607in}}{\pgfqpoint{1.389558in}{3.137021in}}{\pgfqpoint{1.389558in}{3.131197in}}%
\pgfpathcurveto{\pgfqpoint{1.389558in}{3.125373in}}{\pgfqpoint{1.391872in}{3.119787in}}{\pgfqpoint{1.395990in}{3.115669in}}%
\pgfpathcurveto{\pgfqpoint{1.400108in}{3.111551in}}{\pgfqpoint{1.405695in}{3.109237in}}{\pgfqpoint{1.411518in}{3.109237in}}%
\pgfpathlineto{\pgfqpoint{1.411518in}{3.109237in}}%
\pgfpathclose%
\pgfusepath{stroke,fill}%
\end{pgfscope}%
\begin{pgfscope}%
\pgfpathrectangle{\pgfqpoint{1.000000in}{0.979904in}}{\pgfqpoint{6.200000in}{5.960192in}}%
\pgfusepath{clip}%
\pgfsetbuttcap%
\pgfsetroundjoin%
\definecolor{currentfill}{rgb}{0.800000,0.200000,0.200000}%
\pgfsetfillcolor{currentfill}%
\pgfsetlinewidth{1.003750pt}%
\definecolor{currentstroke}{rgb}{0.800000,0.200000,0.200000}%
\pgfsetstrokecolor{currentstroke}%
\pgfsetdash{}{0pt}%
\pgfpathmoveto{\pgfqpoint{1.418280in}{3.010429in}}%
\pgfpathcurveto{\pgfqpoint{1.424104in}{3.010429in}}{\pgfqpoint{1.429690in}{3.012743in}}{\pgfqpoint{1.433808in}{3.016861in}}%
\pgfpathcurveto{\pgfqpoint{1.437926in}{3.020979in}}{\pgfqpoint{1.440240in}{3.026565in}}{\pgfqpoint{1.440240in}{3.032389in}}%
\pgfpathcurveto{\pgfqpoint{1.440240in}{3.038213in}}{\pgfqpoint{1.437926in}{3.043799in}}{\pgfqpoint{1.433808in}{3.047917in}}%
\pgfpathcurveto{\pgfqpoint{1.429690in}{3.052035in}}{\pgfqpoint{1.424104in}{3.054349in}}{\pgfqpoint{1.418280in}{3.054349in}}%
\pgfpathcurveto{\pgfqpoint{1.412456in}{3.054349in}}{\pgfqpoint{1.406870in}{3.052035in}}{\pgfqpoint{1.402752in}{3.047917in}}%
\pgfpathcurveto{\pgfqpoint{1.398634in}{3.043799in}}{\pgfqpoint{1.396320in}{3.038213in}}{\pgfqpoint{1.396320in}{3.032389in}}%
\pgfpathcurveto{\pgfqpoint{1.396320in}{3.026565in}}{\pgfqpoint{1.398634in}{3.020979in}}{\pgfqpoint{1.402752in}{3.016861in}}%
\pgfpathcurveto{\pgfqpoint{1.406870in}{3.012743in}}{\pgfqpoint{1.412456in}{3.010429in}}{\pgfqpoint{1.418280in}{3.010429in}}%
\pgfpathlineto{\pgfqpoint{1.418280in}{3.010429in}}%
\pgfpathclose%
\pgfusepath{stroke,fill}%
\end{pgfscope}%
\begin{pgfscope}%
\pgfpathrectangle{\pgfqpoint{1.000000in}{0.979904in}}{\pgfqpoint{6.200000in}{5.960192in}}%
\pgfusepath{clip}%
\pgfsetbuttcap%
\pgfsetroundjoin%
\definecolor{currentfill}{rgb}{0.800000,0.200000,0.200000}%
\pgfsetfillcolor{currentfill}%
\pgfsetlinewidth{1.003750pt}%
\definecolor{currentstroke}{rgb}{0.800000,0.200000,0.200000}%
\pgfsetstrokecolor{currentstroke}%
\pgfsetdash{}{0pt}%
\pgfpathmoveto{\pgfqpoint{1.349060in}{2.916079in}}%
\pgfpathcurveto{\pgfqpoint{1.354884in}{2.916079in}}{\pgfqpoint{1.360470in}{2.918393in}}{\pgfqpoint{1.364589in}{2.922511in}}%
\pgfpathcurveto{\pgfqpoint{1.368707in}{2.926629in}}{\pgfqpoint{1.371021in}{2.932215in}}{\pgfqpoint{1.371021in}{2.938039in}}%
\pgfpathcurveto{\pgfqpoint{1.371021in}{2.943863in}}{\pgfqpoint{1.368707in}{2.949449in}}{\pgfqpoint{1.364589in}{2.953567in}}%
\pgfpathcurveto{\pgfqpoint{1.360470in}{2.957685in}}{\pgfqpoint{1.354884in}{2.959999in}}{\pgfqpoint{1.349060in}{2.959999in}}%
\pgfpathcurveto{\pgfqpoint{1.343236in}{2.959999in}}{\pgfqpoint{1.337650in}{2.957685in}}{\pgfqpoint{1.333532in}{2.953567in}}%
\pgfpathcurveto{\pgfqpoint{1.329414in}{2.949449in}}{\pgfqpoint{1.327100in}{2.943863in}}{\pgfqpoint{1.327100in}{2.938039in}}%
\pgfpathcurveto{\pgfqpoint{1.327100in}{2.932215in}}{\pgfqpoint{1.329414in}{2.926629in}}{\pgfqpoint{1.333532in}{2.922511in}}%
\pgfpathcurveto{\pgfqpoint{1.337650in}{2.918393in}}{\pgfqpoint{1.343236in}{2.916079in}}{\pgfqpoint{1.349060in}{2.916079in}}%
\pgfpathlineto{\pgfqpoint{1.349060in}{2.916079in}}%
\pgfpathclose%
\pgfusepath{stroke,fill}%
\end{pgfscope}%
\begin{pgfscope}%
\pgfpathrectangle{\pgfqpoint{1.000000in}{0.979904in}}{\pgfqpoint{6.200000in}{5.960192in}}%
\pgfusepath{clip}%
\pgfsetbuttcap%
\pgfsetroundjoin%
\definecolor{currentfill}{rgb}{0.800000,0.200000,0.200000}%
\pgfsetfillcolor{currentfill}%
\pgfsetlinewidth{1.003750pt}%
\definecolor{currentstroke}{rgb}{0.800000,0.200000,0.200000}%
\pgfsetstrokecolor{currentstroke}%
\pgfsetdash{}{0pt}%
\pgfpathmoveto{\pgfqpoint{1.449279in}{2.818707in}}%
\pgfpathcurveto{\pgfqpoint{1.455102in}{2.818707in}}{\pgfqpoint{1.460689in}{2.821021in}}{\pgfqpoint{1.464807in}{2.825139in}}%
\pgfpathcurveto{\pgfqpoint{1.468925in}{2.829257in}}{\pgfqpoint{1.471239in}{2.834843in}}{\pgfqpoint{1.471239in}{2.840667in}}%
\pgfpathcurveto{\pgfqpoint{1.471239in}{2.846491in}}{\pgfqpoint{1.468925in}{2.852077in}}{\pgfqpoint{1.464807in}{2.856196in}}%
\pgfpathcurveto{\pgfqpoint{1.460689in}{2.860314in}}{\pgfqpoint{1.455102in}{2.862628in}}{\pgfqpoint{1.449279in}{2.862628in}}%
\pgfpathcurveto{\pgfqpoint{1.443455in}{2.862628in}}{\pgfqpoint{1.437868in}{2.860314in}}{\pgfqpoint{1.433750in}{2.856196in}}%
\pgfpathcurveto{\pgfqpoint{1.429632in}{2.852077in}}{\pgfqpoint{1.427318in}{2.846491in}}{\pgfqpoint{1.427318in}{2.840667in}}%
\pgfpathcurveto{\pgfqpoint{1.427318in}{2.834843in}}{\pgfqpoint{1.429632in}{2.829257in}}{\pgfqpoint{1.433750in}{2.825139in}}%
\pgfpathcurveto{\pgfqpoint{1.437868in}{2.821021in}}{\pgfqpoint{1.443455in}{2.818707in}}{\pgfqpoint{1.449279in}{2.818707in}}%
\pgfpathlineto{\pgfqpoint{1.449279in}{2.818707in}}%
\pgfpathclose%
\pgfusepath{stroke,fill}%
\end{pgfscope}%
\begin{pgfscope}%
\pgfpathrectangle{\pgfqpoint{1.000000in}{0.979904in}}{\pgfqpoint{6.200000in}{5.960192in}}%
\pgfusepath{clip}%
\pgfsetbuttcap%
\pgfsetroundjoin%
\definecolor{currentfill}{rgb}{0.800000,0.200000,0.200000}%
\pgfsetfillcolor{currentfill}%
\pgfsetlinewidth{1.003750pt}%
\definecolor{currentstroke}{rgb}{0.800000,0.200000,0.200000}%
\pgfsetstrokecolor{currentstroke}%
\pgfsetdash{}{0pt}%
\pgfpathmoveto{\pgfqpoint{1.418557in}{2.721202in}}%
\pgfpathcurveto{\pgfqpoint{1.424381in}{2.721202in}}{\pgfqpoint{1.429967in}{2.723516in}}{\pgfqpoint{1.434086in}{2.727634in}}%
\pgfpathcurveto{\pgfqpoint{1.438204in}{2.731752in}}{\pgfqpoint{1.440518in}{2.737339in}}{\pgfqpoint{1.440518in}{2.743162in}}%
\pgfpathcurveto{\pgfqpoint{1.440518in}{2.748986in}}{\pgfqpoint{1.438204in}{2.754573in}}{\pgfqpoint{1.434086in}{2.758691in}}%
\pgfpathcurveto{\pgfqpoint{1.429967in}{2.762809in}}{\pgfqpoint{1.424381in}{2.765123in}}{\pgfqpoint{1.418557in}{2.765123in}}%
\pgfpathcurveto{\pgfqpoint{1.412733in}{2.765123in}}{\pgfqpoint{1.407147in}{2.762809in}}{\pgfqpoint{1.403029in}{2.758691in}}%
\pgfpathcurveto{\pgfqpoint{1.398911in}{2.754573in}}{\pgfqpoint{1.396597in}{2.748986in}}{\pgfqpoint{1.396597in}{2.743162in}}%
\pgfpathcurveto{\pgfqpoint{1.396597in}{2.737339in}}{\pgfqpoint{1.398911in}{2.731752in}}{\pgfqpoint{1.403029in}{2.727634in}}%
\pgfpathcurveto{\pgfqpoint{1.407147in}{2.723516in}}{\pgfqpoint{1.412733in}{2.721202in}}{\pgfqpoint{1.418557in}{2.721202in}}%
\pgfpathlineto{\pgfqpoint{1.418557in}{2.721202in}}%
\pgfpathclose%
\pgfusepath{stroke,fill}%
\end{pgfscope}%
\begin{pgfscope}%
\pgfpathrectangle{\pgfqpoint{1.000000in}{0.979904in}}{\pgfqpoint{6.200000in}{5.960192in}}%
\pgfusepath{clip}%
\pgfsetbuttcap%
\pgfsetroundjoin%
\definecolor{currentfill}{rgb}{0.800000,0.200000,0.200000}%
\pgfsetfillcolor{currentfill}%
\pgfsetlinewidth{1.003750pt}%
\definecolor{currentstroke}{rgb}{0.800000,0.200000,0.200000}%
\pgfsetstrokecolor{currentstroke}%
\pgfsetdash{}{0pt}%
\pgfpathmoveto{\pgfqpoint{1.281818in}{2.601614in}}%
\pgfpathcurveto{\pgfqpoint{1.287642in}{2.601614in}}{\pgfqpoint{1.293228in}{2.603928in}}{\pgfqpoint{1.297346in}{2.608046in}}%
\pgfpathcurveto{\pgfqpoint{1.301465in}{2.612164in}}{\pgfqpoint{1.303778in}{2.617750in}}{\pgfqpoint{1.303778in}{2.623574in}}%
\pgfpathcurveto{\pgfqpoint{1.303778in}{2.629398in}}{\pgfqpoint{1.301465in}{2.634984in}}{\pgfqpoint{1.297346in}{2.639103in}}%
\pgfpathcurveto{\pgfqpoint{1.293228in}{2.643221in}}{\pgfqpoint{1.287642in}{2.645535in}}{\pgfqpoint{1.281818in}{2.645535in}}%
\pgfpathcurveto{\pgfqpoint{1.275994in}{2.645535in}}{\pgfqpoint{1.270408in}{2.643221in}}{\pgfqpoint{1.266290in}{2.639103in}}%
\pgfpathcurveto{\pgfqpoint{1.262172in}{2.634984in}}{\pgfqpoint{1.259858in}{2.629398in}}{\pgfqpoint{1.259858in}{2.623574in}}%
\pgfpathcurveto{\pgfqpoint{1.259858in}{2.617750in}}{\pgfqpoint{1.262172in}{2.612164in}}{\pgfqpoint{1.266290in}{2.608046in}}%
\pgfpathcurveto{\pgfqpoint{1.270408in}{2.603928in}}{\pgfqpoint{1.275994in}{2.601614in}}{\pgfqpoint{1.281818in}{2.601614in}}%
\pgfpathlineto{\pgfqpoint{1.281818in}{2.601614in}}%
\pgfpathclose%
\pgfusepath{stroke,fill}%
\end{pgfscope}%
\begin{pgfscope}%
\pgfpathrectangle{\pgfqpoint{1.000000in}{0.979904in}}{\pgfqpoint{6.200000in}{5.960192in}}%
\pgfusepath{clip}%
\pgfsetbuttcap%
\pgfsetroundjoin%
\definecolor{currentfill}{rgb}{0.800000,0.200000,0.200000}%
\pgfsetfillcolor{currentfill}%
\pgfsetlinewidth{1.003750pt}%
\definecolor{currentstroke}{rgb}{0.800000,0.200000,0.200000}%
\pgfsetstrokecolor{currentstroke}%
\pgfsetdash{}{0pt}%
\pgfpathmoveto{\pgfqpoint{1.313274in}{2.499999in}}%
\pgfpathcurveto{\pgfqpoint{1.319098in}{2.499999in}}{\pgfqpoint{1.324684in}{2.502312in}}{\pgfqpoint{1.328802in}{2.506431in}}%
\pgfpathcurveto{\pgfqpoint{1.332920in}{2.510549in}}{\pgfqpoint{1.335234in}{2.516135in}}{\pgfqpoint{1.335234in}{2.521959in}}%
\pgfpathcurveto{\pgfqpoint{1.335234in}{2.527783in}}{\pgfqpoint{1.332920in}{2.533369in}}{\pgfqpoint{1.328802in}{2.537487in}}%
\pgfpathcurveto{\pgfqpoint{1.324684in}{2.541605in}}{\pgfqpoint{1.319098in}{2.543919in}}{\pgfqpoint{1.313274in}{2.543919in}}%
\pgfpathcurveto{\pgfqpoint{1.307450in}{2.543919in}}{\pgfqpoint{1.301864in}{2.541605in}}{\pgfqpoint{1.297745in}{2.537487in}}%
\pgfpathcurveto{\pgfqpoint{1.293627in}{2.533369in}}{\pgfqpoint{1.291313in}{2.527783in}}{\pgfqpoint{1.291313in}{2.521959in}}%
\pgfpathcurveto{\pgfqpoint{1.291313in}{2.516135in}}{\pgfqpoint{1.293627in}{2.510549in}}{\pgfqpoint{1.297745in}{2.506431in}}%
\pgfpathcurveto{\pgfqpoint{1.301864in}{2.502312in}}{\pgfqpoint{1.307450in}{2.499999in}}{\pgfqpoint{1.313274in}{2.499999in}}%
\pgfpathlineto{\pgfqpoint{1.313274in}{2.499999in}}%
\pgfpathclose%
\pgfusepath{stroke,fill}%
\end{pgfscope}%
\begin{pgfscope}%
\pgfpathrectangle{\pgfqpoint{1.000000in}{0.979904in}}{\pgfqpoint{6.200000in}{5.960192in}}%
\pgfusepath{clip}%
\pgfsetbuttcap%
\pgfsetroundjoin%
\definecolor{currentfill}{rgb}{0.800000,0.200000,0.200000}%
\pgfsetfillcolor{currentfill}%
\pgfsetlinewidth{1.003750pt}%
\definecolor{currentstroke}{rgb}{0.800000,0.200000,0.200000}%
\pgfsetstrokecolor{currentstroke}%
\pgfsetdash{}{0pt}%
\pgfpathmoveto{\pgfqpoint{1.413636in}{2.419713in}}%
\pgfpathcurveto{\pgfqpoint{1.419460in}{2.419713in}}{\pgfqpoint{1.425046in}{2.422027in}}{\pgfqpoint{1.429165in}{2.426145in}}%
\pgfpathcurveto{\pgfqpoint{1.433283in}{2.430263in}}{\pgfqpoint{1.435597in}{2.435849in}}{\pgfqpoint{1.435597in}{2.441673in}}%
\pgfpathcurveto{\pgfqpoint{1.435597in}{2.447497in}}{\pgfqpoint{1.433283in}{2.453083in}}{\pgfqpoint{1.429165in}{2.457201in}}%
\pgfpathcurveto{\pgfqpoint{1.425046in}{2.461319in}}{\pgfqpoint{1.419460in}{2.463633in}}{\pgfqpoint{1.413636in}{2.463633in}}%
\pgfpathcurveto{\pgfqpoint{1.407812in}{2.463633in}}{\pgfqpoint{1.402226in}{2.461319in}}{\pgfqpoint{1.398108in}{2.457201in}}%
\pgfpathcurveto{\pgfqpoint{1.393990in}{2.453083in}}{\pgfqpoint{1.391676in}{2.447497in}}{\pgfqpoint{1.391676in}{2.441673in}}%
\pgfpathcurveto{\pgfqpoint{1.391676in}{2.435849in}}{\pgfqpoint{1.393990in}{2.430263in}}{\pgfqpoint{1.398108in}{2.426145in}}%
\pgfpathcurveto{\pgfqpoint{1.402226in}{2.422027in}}{\pgfqpoint{1.407812in}{2.419713in}}{\pgfqpoint{1.413636in}{2.419713in}}%
\pgfpathlineto{\pgfqpoint{1.413636in}{2.419713in}}%
\pgfpathclose%
\pgfusepath{stroke,fill}%
\end{pgfscope}%
\begin{pgfscope}%
\pgfpathrectangle{\pgfqpoint{1.000000in}{0.979904in}}{\pgfqpoint{6.200000in}{5.960192in}}%
\pgfusepath{clip}%
\pgfsetbuttcap%
\pgfsetroundjoin%
\definecolor{currentfill}{rgb}{0.800000,0.200000,0.200000}%
\pgfsetfillcolor{currentfill}%
\pgfsetlinewidth{1.003750pt}%
\definecolor{currentstroke}{rgb}{0.800000,0.200000,0.200000}%
\pgfsetstrokecolor{currentstroke}%
\pgfsetdash{}{0pt}%
\pgfpathmoveto{\pgfqpoint{1.461802in}{2.330375in}}%
\pgfpathcurveto{\pgfqpoint{1.467626in}{2.330375in}}{\pgfqpoint{1.473212in}{2.332688in}}{\pgfqpoint{1.477330in}{2.336807in}}%
\pgfpathcurveto{\pgfqpoint{1.481449in}{2.340925in}}{\pgfqpoint{1.483762in}{2.346511in}}{\pgfqpoint{1.483762in}{2.352335in}}%
\pgfpathcurveto{\pgfqpoint{1.483762in}{2.358159in}}{\pgfqpoint{1.481449in}{2.363745in}}{\pgfqpoint{1.477330in}{2.367863in}}%
\pgfpathcurveto{\pgfqpoint{1.473212in}{2.371981in}}{\pgfqpoint{1.467626in}{2.374295in}}{\pgfqpoint{1.461802in}{2.374295in}}%
\pgfpathcurveto{\pgfqpoint{1.455978in}{2.374295in}}{\pgfqpoint{1.450392in}{2.371981in}}{\pgfqpoint{1.446274in}{2.367863in}}%
\pgfpathcurveto{\pgfqpoint{1.442156in}{2.363745in}}{\pgfqpoint{1.439842in}{2.358159in}}{\pgfqpoint{1.439842in}{2.352335in}}%
\pgfpathcurveto{\pgfqpoint{1.439842in}{2.346511in}}{\pgfqpoint{1.442156in}{2.340925in}}{\pgfqpoint{1.446274in}{2.336807in}}%
\pgfpathcurveto{\pgfqpoint{1.450392in}{2.332688in}}{\pgfqpoint{1.455978in}{2.330375in}}{\pgfqpoint{1.461802in}{2.330375in}}%
\pgfpathlineto{\pgfqpoint{1.461802in}{2.330375in}}%
\pgfpathclose%
\pgfusepath{stroke,fill}%
\end{pgfscope}%
\begin{pgfscope}%
\pgfpathrectangle{\pgfqpoint{1.000000in}{0.979904in}}{\pgfqpoint{6.200000in}{5.960192in}}%
\pgfusepath{clip}%
\pgfsetbuttcap%
\pgfsetroundjoin%
\definecolor{currentfill}{rgb}{0.800000,0.200000,0.200000}%
\pgfsetfillcolor{currentfill}%
\pgfsetlinewidth{1.003750pt}%
\definecolor{currentstroke}{rgb}{0.800000,0.200000,0.200000}%
\pgfsetstrokecolor{currentstroke}%
\pgfsetdash{}{0pt}%
\pgfpathmoveto{\pgfqpoint{1.473245in}{2.227001in}}%
\pgfpathcurveto{\pgfqpoint{1.479069in}{2.227001in}}{\pgfqpoint{1.484655in}{2.229315in}}{\pgfqpoint{1.488773in}{2.233433in}}%
\pgfpathcurveto{\pgfqpoint{1.492891in}{2.237551in}}{\pgfqpoint{1.495205in}{2.243137in}}{\pgfqpoint{1.495205in}{2.248961in}}%
\pgfpathcurveto{\pgfqpoint{1.495205in}{2.254785in}}{\pgfqpoint{1.492891in}{2.260371in}}{\pgfqpoint{1.488773in}{2.264490in}}%
\pgfpathcurveto{\pgfqpoint{1.484655in}{2.268608in}}{\pgfqpoint{1.479069in}{2.270922in}}{\pgfqpoint{1.473245in}{2.270922in}}%
\pgfpathcurveto{\pgfqpoint{1.467421in}{2.270922in}}{\pgfqpoint{1.461835in}{2.268608in}}{\pgfqpoint{1.457717in}{2.264490in}}%
\pgfpathcurveto{\pgfqpoint{1.453599in}{2.260371in}}{\pgfqpoint{1.451285in}{2.254785in}}{\pgfqpoint{1.451285in}{2.248961in}}%
\pgfpathcurveto{\pgfqpoint{1.451285in}{2.243137in}}{\pgfqpoint{1.453599in}{2.237551in}}{\pgfqpoint{1.457717in}{2.233433in}}%
\pgfpathcurveto{\pgfqpoint{1.461835in}{2.229315in}}{\pgfqpoint{1.467421in}{2.227001in}}{\pgfqpoint{1.473245in}{2.227001in}}%
\pgfpathlineto{\pgfqpoint{1.473245in}{2.227001in}}%
\pgfpathclose%
\pgfusepath{stroke,fill}%
\end{pgfscope}%
\begin{pgfscope}%
\pgfpathrectangle{\pgfqpoint{1.000000in}{0.979904in}}{\pgfqpoint{6.200000in}{5.960192in}}%
\pgfusepath{clip}%
\pgfsetbuttcap%
\pgfsetroundjoin%
\definecolor{currentfill}{rgb}{0.800000,0.200000,0.200000}%
\pgfsetfillcolor{currentfill}%
\pgfsetlinewidth{1.003750pt}%
\definecolor{currentstroke}{rgb}{0.800000,0.200000,0.200000}%
\pgfsetstrokecolor{currentstroke}%
\pgfsetdash{}{0pt}%
\pgfpathmoveto{\pgfqpoint{1.408730in}{2.080050in}}%
\pgfpathcurveto{\pgfqpoint{1.414554in}{2.080050in}}{\pgfqpoint{1.420140in}{2.082363in}}{\pgfqpoint{1.424258in}{2.086482in}}%
\pgfpathcurveto{\pgfqpoint{1.428376in}{2.090600in}}{\pgfqpoint{1.430690in}{2.096186in}}{\pgfqpoint{1.430690in}{2.102010in}}%
\pgfpathcurveto{\pgfqpoint{1.430690in}{2.107834in}}{\pgfqpoint{1.428376in}{2.113420in}}{\pgfqpoint{1.424258in}{2.117538in}}%
\pgfpathcurveto{\pgfqpoint{1.420140in}{2.121656in}}{\pgfqpoint{1.414554in}{2.123970in}}{\pgfqpoint{1.408730in}{2.123970in}}%
\pgfpathcurveto{\pgfqpoint{1.402906in}{2.123970in}}{\pgfqpoint{1.397320in}{2.121656in}}{\pgfqpoint{1.393202in}{2.117538in}}%
\pgfpathcurveto{\pgfqpoint{1.389084in}{2.113420in}}{\pgfqpoint{1.386770in}{2.107834in}}{\pgfqpoint{1.386770in}{2.102010in}}%
\pgfpathcurveto{\pgfqpoint{1.386770in}{2.096186in}}{\pgfqpoint{1.389084in}{2.090600in}}{\pgfqpoint{1.393202in}{2.086482in}}%
\pgfpathcurveto{\pgfqpoint{1.397320in}{2.082363in}}{\pgfqpoint{1.402906in}{2.080050in}}{\pgfqpoint{1.408730in}{2.080050in}}%
\pgfpathlineto{\pgfqpoint{1.408730in}{2.080050in}}%
\pgfpathclose%
\pgfusepath{stroke,fill}%
\end{pgfscope}%
\begin{pgfscope}%
\pgfpathrectangle{\pgfqpoint{1.000000in}{0.979904in}}{\pgfqpoint{6.200000in}{5.960192in}}%
\pgfusepath{clip}%
\pgfsetbuttcap%
\pgfsetroundjoin%
\definecolor{currentfill}{rgb}{0.800000,0.200000,0.200000}%
\pgfsetfillcolor{currentfill}%
\pgfsetlinewidth{1.003750pt}%
\definecolor{currentstroke}{rgb}{0.800000,0.200000,0.200000}%
\pgfsetstrokecolor{currentstroke}%
\pgfsetdash{}{0pt}%
\pgfpathmoveto{\pgfqpoint{1.550084in}{2.037919in}}%
\pgfpathcurveto{\pgfqpoint{1.555908in}{2.037919in}}{\pgfqpoint{1.561494in}{2.040233in}}{\pgfqpoint{1.565612in}{2.044351in}}%
\pgfpathcurveto{\pgfqpoint{1.569730in}{2.048469in}}{\pgfqpoint{1.572044in}{2.054056in}}{\pgfqpoint{1.572044in}{2.059879in}}%
\pgfpathcurveto{\pgfqpoint{1.572044in}{2.065703in}}{\pgfqpoint{1.569730in}{2.071290in}}{\pgfqpoint{1.565612in}{2.075408in}}%
\pgfpathcurveto{\pgfqpoint{1.561494in}{2.079526in}}{\pgfqpoint{1.555908in}{2.081840in}}{\pgfqpoint{1.550084in}{2.081840in}}%
\pgfpathcurveto{\pgfqpoint{1.544260in}{2.081840in}}{\pgfqpoint{1.538674in}{2.079526in}}{\pgfqpoint{1.534555in}{2.075408in}}%
\pgfpathcurveto{\pgfqpoint{1.530437in}{2.071290in}}{\pgfqpoint{1.528123in}{2.065703in}}{\pgfqpoint{1.528123in}{2.059879in}}%
\pgfpathcurveto{\pgfqpoint{1.528123in}{2.054056in}}{\pgfqpoint{1.530437in}{2.048469in}}{\pgfqpoint{1.534555in}{2.044351in}}%
\pgfpathcurveto{\pgfqpoint{1.538674in}{2.040233in}}{\pgfqpoint{1.544260in}{2.037919in}}{\pgfqpoint{1.550084in}{2.037919in}}%
\pgfpathlineto{\pgfqpoint{1.550084in}{2.037919in}}%
\pgfpathclose%
\pgfusepath{stroke,fill}%
\end{pgfscope}%
\begin{pgfscope}%
\pgfpathrectangle{\pgfqpoint{1.000000in}{0.979904in}}{\pgfqpoint{6.200000in}{5.960192in}}%
\pgfusepath{clip}%
\pgfsetbuttcap%
\pgfsetroundjoin%
\definecolor{currentfill}{rgb}{0.800000,0.200000,0.200000}%
\pgfsetfillcolor{currentfill}%
\pgfsetlinewidth{1.003750pt}%
\definecolor{currentstroke}{rgb}{0.800000,0.200000,0.200000}%
\pgfsetstrokecolor{currentstroke}%
\pgfsetdash{}{0pt}%
\pgfpathmoveto{\pgfqpoint{1.583316in}{1.936697in}}%
\pgfpathcurveto{\pgfqpoint{1.589140in}{1.936697in}}{\pgfqpoint{1.594727in}{1.939011in}}{\pgfqpoint{1.598845in}{1.943129in}}%
\pgfpathcurveto{\pgfqpoint{1.602963in}{1.947248in}}{\pgfqpoint{1.605277in}{1.952834in}}{\pgfqpoint{1.605277in}{1.958658in}}%
\pgfpathcurveto{\pgfqpoint{1.605277in}{1.964482in}}{\pgfqpoint{1.602963in}{1.970068in}}{\pgfqpoint{1.598845in}{1.974186in}}%
\pgfpathcurveto{\pgfqpoint{1.594727in}{1.978304in}}{\pgfqpoint{1.589140in}{1.980618in}}{\pgfqpoint{1.583316in}{1.980618in}}%
\pgfpathcurveto{\pgfqpoint{1.577492in}{1.980618in}}{\pgfqpoint{1.571906in}{1.978304in}}{\pgfqpoint{1.567788in}{1.974186in}}%
\pgfpathcurveto{\pgfqpoint{1.563670in}{1.970068in}}{\pgfqpoint{1.561356in}{1.964482in}}{\pgfqpoint{1.561356in}{1.958658in}}%
\pgfpathcurveto{\pgfqpoint{1.561356in}{1.952834in}}{\pgfqpoint{1.563670in}{1.947248in}}{\pgfqpoint{1.567788in}{1.943129in}}%
\pgfpathcurveto{\pgfqpoint{1.571906in}{1.939011in}}{\pgfqpoint{1.577492in}{1.936697in}}{\pgfqpoint{1.583316in}{1.936697in}}%
\pgfpathlineto{\pgfqpoint{1.583316in}{1.936697in}}%
\pgfpathclose%
\pgfusepath{stroke,fill}%
\end{pgfscope}%
\begin{pgfscope}%
\pgfpathrectangle{\pgfqpoint{1.000000in}{0.979904in}}{\pgfqpoint{6.200000in}{5.960192in}}%
\pgfusepath{clip}%
\pgfsetbuttcap%
\pgfsetroundjoin%
\definecolor{currentfill}{rgb}{0.800000,0.200000,0.200000}%
\pgfsetfillcolor{currentfill}%
\pgfsetlinewidth{1.003750pt}%
\definecolor{currentstroke}{rgb}{0.800000,0.200000,0.200000}%
\pgfsetstrokecolor{currentstroke}%
\pgfsetdash{}{0pt}%
\pgfpathmoveto{\pgfqpoint{1.691346in}{1.889451in}}%
\pgfpathcurveto{\pgfqpoint{1.697169in}{1.889451in}}{\pgfqpoint{1.702756in}{1.891765in}}{\pgfqpoint{1.706874in}{1.895883in}}%
\pgfpathcurveto{\pgfqpoint{1.710992in}{1.900001in}}{\pgfqpoint{1.713306in}{1.905587in}}{\pgfqpoint{1.713306in}{1.911411in}}%
\pgfpathcurveto{\pgfqpoint{1.713306in}{1.917235in}}{\pgfqpoint{1.710992in}{1.922821in}}{\pgfqpoint{1.706874in}{1.926939in}}%
\pgfpathcurveto{\pgfqpoint{1.702756in}{1.931057in}}{\pgfqpoint{1.697169in}{1.933371in}}{\pgfqpoint{1.691346in}{1.933371in}}%
\pgfpathcurveto{\pgfqpoint{1.685522in}{1.933371in}}{\pgfqpoint{1.679935in}{1.931057in}}{\pgfqpoint{1.675817in}{1.926939in}}%
\pgfpathcurveto{\pgfqpoint{1.671699in}{1.922821in}}{\pgfqpoint{1.669385in}{1.917235in}}{\pgfqpoint{1.669385in}{1.911411in}}%
\pgfpathcurveto{\pgfqpoint{1.669385in}{1.905587in}}{\pgfqpoint{1.671699in}{1.900001in}}{\pgfqpoint{1.675817in}{1.895883in}}%
\pgfpathcurveto{\pgfqpoint{1.679935in}{1.891765in}}{\pgfqpoint{1.685522in}{1.889451in}}{\pgfqpoint{1.691346in}{1.889451in}}%
\pgfpathlineto{\pgfqpoint{1.691346in}{1.889451in}}%
\pgfpathclose%
\pgfusepath{stroke,fill}%
\end{pgfscope}%
\begin{pgfscope}%
\pgfpathrectangle{\pgfqpoint{1.000000in}{0.979904in}}{\pgfqpoint{6.200000in}{5.960192in}}%
\pgfusepath{clip}%
\pgfsetbuttcap%
\pgfsetroundjoin%
\definecolor{currentfill}{rgb}{0.800000,0.200000,0.200000}%
\pgfsetfillcolor{currentfill}%
\pgfsetlinewidth{1.003750pt}%
\definecolor{currentstroke}{rgb}{0.800000,0.200000,0.200000}%
\pgfsetstrokecolor{currentstroke}%
\pgfsetdash{}{0pt}%
\pgfpathmoveto{\pgfqpoint{1.745356in}{1.803357in}}%
\pgfpathcurveto{\pgfqpoint{1.751180in}{1.803357in}}{\pgfqpoint{1.756766in}{1.805671in}}{\pgfqpoint{1.760884in}{1.809789in}}%
\pgfpathcurveto{\pgfqpoint{1.765002in}{1.813907in}}{\pgfqpoint{1.767316in}{1.819493in}}{\pgfqpoint{1.767316in}{1.825317in}}%
\pgfpathcurveto{\pgfqpoint{1.767316in}{1.831141in}}{\pgfqpoint{1.765002in}{1.836727in}}{\pgfqpoint{1.760884in}{1.840846in}}%
\pgfpathcurveto{\pgfqpoint{1.756766in}{1.844964in}}{\pgfqpoint{1.751180in}{1.847278in}}{\pgfqpoint{1.745356in}{1.847278in}}%
\pgfpathcurveto{\pgfqpoint{1.739532in}{1.847278in}}{\pgfqpoint{1.733946in}{1.844964in}}{\pgfqpoint{1.729828in}{1.840846in}}%
\pgfpathcurveto{\pgfqpoint{1.725709in}{1.836727in}}{\pgfqpoint{1.723396in}{1.831141in}}{\pgfqpoint{1.723396in}{1.825317in}}%
\pgfpathcurveto{\pgfqpoint{1.723396in}{1.819493in}}{\pgfqpoint{1.725709in}{1.813907in}}{\pgfqpoint{1.729828in}{1.809789in}}%
\pgfpathcurveto{\pgfqpoint{1.733946in}{1.805671in}}{\pgfqpoint{1.739532in}{1.803357in}}{\pgfqpoint{1.745356in}{1.803357in}}%
\pgfpathlineto{\pgfqpoint{1.745356in}{1.803357in}}%
\pgfpathclose%
\pgfusepath{stroke,fill}%
\end{pgfscope}%
\begin{pgfscope}%
\pgfpathrectangle{\pgfqpoint{1.000000in}{0.979904in}}{\pgfqpoint{6.200000in}{5.960192in}}%
\pgfusepath{clip}%
\pgfsetbuttcap%
\pgfsetroundjoin%
\definecolor{currentfill}{rgb}{0.800000,0.200000,0.200000}%
\pgfsetfillcolor{currentfill}%
\pgfsetlinewidth{1.003750pt}%
\definecolor{currentstroke}{rgb}{0.800000,0.200000,0.200000}%
\pgfsetstrokecolor{currentstroke}%
\pgfsetdash{}{0pt}%
\pgfpathmoveto{\pgfqpoint{1.793312in}{1.708010in}}%
\pgfpathcurveto{\pgfqpoint{1.799135in}{1.708010in}}{\pgfqpoint{1.804722in}{1.710324in}}{\pgfqpoint{1.808840in}{1.714442in}}%
\pgfpathcurveto{\pgfqpoint{1.812958in}{1.718560in}}{\pgfqpoint{1.815272in}{1.724147in}}{\pgfqpoint{1.815272in}{1.729970in}}%
\pgfpathcurveto{\pgfqpoint{1.815272in}{1.735794in}}{\pgfqpoint{1.812958in}{1.741381in}}{\pgfqpoint{1.808840in}{1.745499in}}%
\pgfpathcurveto{\pgfqpoint{1.804722in}{1.749617in}}{\pgfqpoint{1.799135in}{1.751931in}}{\pgfqpoint{1.793312in}{1.751931in}}%
\pgfpathcurveto{\pgfqpoint{1.787488in}{1.751931in}}{\pgfqpoint{1.781901in}{1.749617in}}{\pgfqpoint{1.777783in}{1.745499in}}%
\pgfpathcurveto{\pgfqpoint{1.773665in}{1.741381in}}{\pgfqpoint{1.771351in}{1.735794in}}{\pgfqpoint{1.771351in}{1.729970in}}%
\pgfpathcurveto{\pgfqpoint{1.771351in}{1.724147in}}{\pgfqpoint{1.773665in}{1.718560in}}{\pgfqpoint{1.777783in}{1.714442in}}%
\pgfpathcurveto{\pgfqpoint{1.781901in}{1.710324in}}{\pgfqpoint{1.787488in}{1.708010in}}{\pgfqpoint{1.793312in}{1.708010in}}%
\pgfpathlineto{\pgfqpoint{1.793312in}{1.708010in}}%
\pgfpathclose%
\pgfusepath{stroke,fill}%
\end{pgfscope}%
\begin{pgfscope}%
\pgfpathrectangle{\pgfqpoint{1.000000in}{0.979904in}}{\pgfqpoint{6.200000in}{5.960192in}}%
\pgfusepath{clip}%
\pgfsetbuttcap%
\pgfsetroundjoin%
\definecolor{currentfill}{rgb}{0.800000,0.200000,0.200000}%
\pgfsetfillcolor{currentfill}%
\pgfsetlinewidth{1.003750pt}%
\definecolor{currentstroke}{rgb}{0.800000,0.200000,0.200000}%
\pgfsetstrokecolor{currentstroke}%
\pgfsetdash{}{0pt}%
\pgfpathmoveto{\pgfqpoint{1.887142in}{1.658959in}}%
\pgfpathcurveto{\pgfqpoint{1.892966in}{1.658959in}}{\pgfqpoint{1.898552in}{1.661273in}}{\pgfqpoint{1.902670in}{1.665391in}}%
\pgfpathcurveto{\pgfqpoint{1.906788in}{1.669509in}}{\pgfqpoint{1.909102in}{1.675095in}}{\pgfqpoint{1.909102in}{1.680919in}}%
\pgfpathcurveto{\pgfqpoint{1.909102in}{1.686743in}}{\pgfqpoint{1.906788in}{1.692329in}}{\pgfqpoint{1.902670in}{1.696448in}}%
\pgfpathcurveto{\pgfqpoint{1.898552in}{1.700566in}}{\pgfqpoint{1.892966in}{1.702880in}}{\pgfqpoint{1.887142in}{1.702880in}}%
\pgfpathcurveto{\pgfqpoint{1.881318in}{1.702880in}}{\pgfqpoint{1.875732in}{1.700566in}}{\pgfqpoint{1.871614in}{1.696448in}}%
\pgfpathcurveto{\pgfqpoint{1.867496in}{1.692329in}}{\pgfqpoint{1.865182in}{1.686743in}}{\pgfqpoint{1.865182in}{1.680919in}}%
\pgfpathcurveto{\pgfqpoint{1.865182in}{1.675095in}}{\pgfqpoint{1.867496in}{1.669509in}}{\pgfqpoint{1.871614in}{1.665391in}}%
\pgfpathcurveto{\pgfqpoint{1.875732in}{1.661273in}}{\pgfqpoint{1.881318in}{1.658959in}}{\pgfqpoint{1.887142in}{1.658959in}}%
\pgfpathlineto{\pgfqpoint{1.887142in}{1.658959in}}%
\pgfpathclose%
\pgfusepath{stroke,fill}%
\end{pgfscope}%
\begin{pgfscope}%
\pgfpathrectangle{\pgfqpoint{1.000000in}{0.979904in}}{\pgfqpoint{6.200000in}{5.960192in}}%
\pgfusepath{clip}%
\pgfsetbuttcap%
\pgfsetroundjoin%
\definecolor{currentfill}{rgb}{0.800000,0.200000,0.200000}%
\pgfsetfillcolor{currentfill}%
\pgfsetlinewidth{1.003750pt}%
\definecolor{currentstroke}{rgb}{0.800000,0.200000,0.200000}%
\pgfsetstrokecolor{currentstroke}%
\pgfsetdash{}{0pt}%
\pgfpathmoveto{\pgfqpoint{2.030807in}{1.680496in}}%
\pgfpathcurveto{\pgfqpoint{2.036631in}{1.680496in}}{\pgfqpoint{2.042218in}{1.682810in}}{\pgfqpoint{2.046336in}{1.686928in}}%
\pgfpathcurveto{\pgfqpoint{2.050454in}{1.691046in}}{\pgfqpoint{2.052768in}{1.696632in}}{\pgfqpoint{2.052768in}{1.702456in}}%
\pgfpathcurveto{\pgfqpoint{2.052768in}{1.708280in}}{\pgfqpoint{2.050454in}{1.713866in}}{\pgfqpoint{2.046336in}{1.717985in}}%
\pgfpathcurveto{\pgfqpoint{2.042218in}{1.722103in}}{\pgfqpoint{2.036631in}{1.724417in}}{\pgfqpoint{2.030807in}{1.724417in}}%
\pgfpathcurveto{\pgfqpoint{2.024984in}{1.724417in}}{\pgfqpoint{2.019397in}{1.722103in}}{\pgfqpoint{2.015279in}{1.717985in}}%
\pgfpathcurveto{\pgfqpoint{2.011161in}{1.713866in}}{\pgfqpoint{2.008847in}{1.708280in}}{\pgfqpoint{2.008847in}{1.702456in}}%
\pgfpathcurveto{\pgfqpoint{2.008847in}{1.696632in}}{\pgfqpoint{2.011161in}{1.691046in}}{\pgfqpoint{2.015279in}{1.686928in}}%
\pgfpathcurveto{\pgfqpoint{2.019397in}{1.682810in}}{\pgfqpoint{2.024984in}{1.680496in}}{\pgfqpoint{2.030807in}{1.680496in}}%
\pgfpathlineto{\pgfqpoint{2.030807in}{1.680496in}}%
\pgfpathclose%
\pgfusepath{stroke,fill}%
\end{pgfscope}%
\begin{pgfscope}%
\pgfpathrectangle{\pgfqpoint{1.000000in}{0.979904in}}{\pgfqpoint{6.200000in}{5.960192in}}%
\pgfusepath{clip}%
\pgfsetbuttcap%
\pgfsetroundjoin%
\definecolor{currentfill}{rgb}{0.800000,0.200000,0.200000}%
\pgfsetfillcolor{currentfill}%
\pgfsetlinewidth{1.003750pt}%
\definecolor{currentstroke}{rgb}{0.800000,0.200000,0.200000}%
\pgfsetstrokecolor{currentstroke}%
\pgfsetdash{}{0pt}%
\pgfpathmoveto{\pgfqpoint{2.014404in}{1.485290in}}%
\pgfpathcurveto{\pgfqpoint{2.020228in}{1.485290in}}{\pgfqpoint{2.025814in}{1.487604in}}{\pgfqpoint{2.029932in}{1.491722in}}%
\pgfpathcurveto{\pgfqpoint{2.034050in}{1.495840in}}{\pgfqpoint{2.036364in}{1.501427in}}{\pgfqpoint{2.036364in}{1.507251in}}%
\pgfpathcurveto{\pgfqpoint{2.036364in}{1.513074in}}{\pgfqpoint{2.034050in}{1.518661in}}{\pgfqpoint{2.029932in}{1.522779in}}%
\pgfpathcurveto{\pgfqpoint{2.025814in}{1.526897in}}{\pgfqpoint{2.020228in}{1.529211in}}{\pgfqpoint{2.014404in}{1.529211in}}%
\pgfpathcurveto{\pgfqpoint{2.008580in}{1.529211in}}{\pgfqpoint{2.002994in}{1.526897in}}{\pgfqpoint{1.998876in}{1.522779in}}%
\pgfpathcurveto{\pgfqpoint{1.994758in}{1.518661in}}{\pgfqpoint{1.992444in}{1.513074in}}{\pgfqpoint{1.992444in}{1.507251in}}%
\pgfpathcurveto{\pgfqpoint{1.992444in}{1.501427in}}{\pgfqpoint{1.994758in}{1.495840in}}{\pgfqpoint{1.998876in}{1.491722in}}%
\pgfpathcurveto{\pgfqpoint{2.002994in}{1.487604in}}{\pgfqpoint{2.008580in}{1.485290in}}{\pgfqpoint{2.014404in}{1.485290in}}%
\pgfpathlineto{\pgfqpoint{2.014404in}{1.485290in}}%
\pgfpathclose%
\pgfusepath{stroke,fill}%
\end{pgfscope}%
\begin{pgfscope}%
\pgfpathrectangle{\pgfqpoint{1.000000in}{0.979904in}}{\pgfqpoint{6.200000in}{5.960192in}}%
\pgfusepath{clip}%
\pgfsetbuttcap%
\pgfsetroundjoin%
\definecolor{currentfill}{rgb}{0.800000,0.200000,0.200000}%
\pgfsetfillcolor{currentfill}%
\pgfsetlinewidth{1.003750pt}%
\definecolor{currentstroke}{rgb}{0.800000,0.200000,0.200000}%
\pgfsetstrokecolor{currentstroke}%
\pgfsetdash{}{0pt}%
\pgfpathmoveto{\pgfqpoint{2.154100in}{1.516920in}}%
\pgfpathcurveto{\pgfqpoint{2.159924in}{1.516920in}}{\pgfqpoint{2.165510in}{1.519234in}}{\pgfqpoint{2.169628in}{1.523352in}}%
\pgfpathcurveto{\pgfqpoint{2.173746in}{1.527470in}}{\pgfqpoint{2.176060in}{1.533056in}}{\pgfqpoint{2.176060in}{1.538880in}}%
\pgfpathcurveto{\pgfqpoint{2.176060in}{1.544704in}}{\pgfqpoint{2.173746in}{1.550290in}}{\pgfqpoint{2.169628in}{1.554408in}}%
\pgfpathcurveto{\pgfqpoint{2.165510in}{1.558526in}}{\pgfqpoint{2.159924in}{1.560840in}}{\pgfqpoint{2.154100in}{1.560840in}}%
\pgfpathcurveto{\pgfqpoint{2.148276in}{1.560840in}}{\pgfqpoint{2.142690in}{1.558526in}}{\pgfqpoint{2.138572in}{1.554408in}}%
\pgfpathcurveto{\pgfqpoint{2.134453in}{1.550290in}}{\pgfqpoint{2.132140in}{1.544704in}}{\pgfqpoint{2.132140in}{1.538880in}}%
\pgfpathcurveto{\pgfqpoint{2.132140in}{1.533056in}}{\pgfqpoint{2.134453in}{1.527470in}}{\pgfqpoint{2.138572in}{1.523352in}}%
\pgfpathcurveto{\pgfqpoint{2.142690in}{1.519234in}}{\pgfqpoint{2.148276in}{1.516920in}}{\pgfqpoint{2.154100in}{1.516920in}}%
\pgfpathlineto{\pgfqpoint{2.154100in}{1.516920in}}%
\pgfpathclose%
\pgfusepath{stroke,fill}%
\end{pgfscope}%
\begin{pgfscope}%
\pgfpathrectangle{\pgfqpoint{1.000000in}{0.979904in}}{\pgfqpoint{6.200000in}{5.960192in}}%
\pgfusepath{clip}%
\pgfsetbuttcap%
\pgfsetroundjoin%
\definecolor{currentfill}{rgb}{0.800000,0.200000,0.200000}%
\pgfsetfillcolor{currentfill}%
\pgfsetlinewidth{1.003750pt}%
\definecolor{currentstroke}{rgb}{0.800000,0.200000,0.200000}%
\pgfsetstrokecolor{currentstroke}%
\pgfsetdash{}{0pt}%
\pgfpathmoveto{\pgfqpoint{2.176982in}{1.340631in}}%
\pgfpathcurveto{\pgfqpoint{2.182806in}{1.340631in}}{\pgfqpoint{2.188392in}{1.342945in}}{\pgfqpoint{2.192510in}{1.347063in}}%
\pgfpathcurveto{\pgfqpoint{2.196628in}{1.351181in}}{\pgfqpoint{2.198942in}{1.356767in}}{\pgfqpoint{2.198942in}{1.362591in}}%
\pgfpathcurveto{\pgfqpoint{2.198942in}{1.368415in}}{\pgfqpoint{2.196628in}{1.374001in}}{\pgfqpoint{2.192510in}{1.378120in}}%
\pgfpathcurveto{\pgfqpoint{2.188392in}{1.382238in}}{\pgfqpoint{2.182806in}{1.384552in}}{\pgfqpoint{2.176982in}{1.384552in}}%
\pgfpathcurveto{\pgfqpoint{2.171158in}{1.384552in}}{\pgfqpoint{2.165572in}{1.382238in}}{\pgfqpoint{2.161454in}{1.378120in}}%
\pgfpathcurveto{\pgfqpoint{2.157336in}{1.374001in}}{\pgfqpoint{2.155022in}{1.368415in}}{\pgfqpoint{2.155022in}{1.362591in}}%
\pgfpathcurveto{\pgfqpoint{2.155022in}{1.356767in}}{\pgfqpoint{2.157336in}{1.351181in}}{\pgfqpoint{2.161454in}{1.347063in}}%
\pgfpathcurveto{\pgfqpoint{2.165572in}{1.342945in}}{\pgfqpoint{2.171158in}{1.340631in}}{\pgfqpoint{2.176982in}{1.340631in}}%
\pgfpathlineto{\pgfqpoint{2.176982in}{1.340631in}}%
\pgfpathclose%
\pgfusepath{stroke,fill}%
\end{pgfscope}%
\begin{pgfscope}%
\pgfpathrectangle{\pgfqpoint{1.000000in}{0.979904in}}{\pgfqpoint{6.200000in}{5.960192in}}%
\pgfusepath{clip}%
\pgfsetbuttcap%
\pgfsetroundjoin%
\definecolor{currentfill}{rgb}{0.800000,0.200000,0.200000}%
\pgfsetfillcolor{currentfill}%
\pgfsetlinewidth{1.003750pt}%
\definecolor{currentstroke}{rgb}{0.800000,0.200000,0.200000}%
\pgfsetstrokecolor{currentstroke}%
\pgfsetdash{}{0pt}%
\pgfpathmoveto{\pgfqpoint{2.304435in}{1.365469in}}%
\pgfpathcurveto{\pgfqpoint{2.310259in}{1.365469in}}{\pgfqpoint{2.315845in}{1.367783in}}{\pgfqpoint{2.319964in}{1.371901in}}%
\pgfpathcurveto{\pgfqpoint{2.324082in}{1.376019in}}{\pgfqpoint{2.326396in}{1.381606in}}{\pgfqpoint{2.326396in}{1.387429in}}%
\pgfpathcurveto{\pgfqpoint{2.326396in}{1.393253in}}{\pgfqpoint{2.324082in}{1.398840in}}{\pgfqpoint{2.319964in}{1.402958in}}%
\pgfpathcurveto{\pgfqpoint{2.315845in}{1.407076in}}{\pgfqpoint{2.310259in}{1.409390in}}{\pgfqpoint{2.304435in}{1.409390in}}%
\pgfpathcurveto{\pgfqpoint{2.298611in}{1.409390in}}{\pgfqpoint{2.293025in}{1.407076in}}{\pgfqpoint{2.288907in}{1.402958in}}%
\pgfpathcurveto{\pgfqpoint{2.284789in}{1.398840in}}{\pgfqpoint{2.282475in}{1.393253in}}{\pgfqpoint{2.282475in}{1.387429in}}%
\pgfpathcurveto{\pgfqpoint{2.282475in}{1.381606in}}{\pgfqpoint{2.284789in}{1.376019in}}{\pgfqpoint{2.288907in}{1.371901in}}%
\pgfpathcurveto{\pgfqpoint{2.293025in}{1.367783in}}{\pgfqpoint{2.298611in}{1.365469in}}{\pgfqpoint{2.304435in}{1.365469in}}%
\pgfpathlineto{\pgfqpoint{2.304435in}{1.365469in}}%
\pgfpathclose%
\pgfusepath{stroke,fill}%
\end{pgfscope}%
\begin{pgfscope}%
\pgfpathrectangle{\pgfqpoint{1.000000in}{0.979904in}}{\pgfqpoint{6.200000in}{5.960192in}}%
\pgfusepath{clip}%
\pgfsetbuttcap%
\pgfsetroundjoin%
\definecolor{currentfill}{rgb}{0.800000,0.200000,0.200000}%
\pgfsetfillcolor{currentfill}%
\pgfsetlinewidth{1.003750pt}%
\definecolor{currentstroke}{rgb}{0.800000,0.200000,0.200000}%
\pgfsetstrokecolor{currentstroke}%
\pgfsetdash{}{0pt}%
\pgfpathmoveto{\pgfqpoint{2.411545in}{1.359511in}}%
\pgfpathcurveto{\pgfqpoint{2.417369in}{1.359511in}}{\pgfqpoint{2.422955in}{1.361825in}}{\pgfqpoint{2.427073in}{1.365943in}}%
\pgfpathcurveto{\pgfqpoint{2.431191in}{1.370061in}}{\pgfqpoint{2.433505in}{1.375647in}}{\pgfqpoint{2.433505in}{1.381471in}}%
\pgfpathcurveto{\pgfqpoint{2.433505in}{1.387295in}}{\pgfqpoint{2.431191in}{1.392881in}}{\pgfqpoint{2.427073in}{1.396999in}}%
\pgfpathcurveto{\pgfqpoint{2.422955in}{1.401117in}}{\pgfqpoint{2.417369in}{1.403431in}}{\pgfqpoint{2.411545in}{1.403431in}}%
\pgfpathcurveto{\pgfqpoint{2.405721in}{1.403431in}}{\pgfqpoint{2.400135in}{1.401117in}}{\pgfqpoint{2.396017in}{1.396999in}}%
\pgfpathcurveto{\pgfqpoint{2.391899in}{1.392881in}}{\pgfqpoint{2.389585in}{1.387295in}}{\pgfqpoint{2.389585in}{1.381471in}}%
\pgfpathcurveto{\pgfqpoint{2.389585in}{1.375647in}}{\pgfqpoint{2.391899in}{1.370061in}}{\pgfqpoint{2.396017in}{1.365943in}}%
\pgfpathcurveto{\pgfqpoint{2.400135in}{1.361825in}}{\pgfqpoint{2.405721in}{1.359511in}}{\pgfqpoint{2.411545in}{1.359511in}}%
\pgfpathlineto{\pgfqpoint{2.411545in}{1.359511in}}%
\pgfpathclose%
\pgfusepath{stroke,fill}%
\end{pgfscope}%
\begin{pgfscope}%
\pgfpathrectangle{\pgfqpoint{1.000000in}{0.979904in}}{\pgfqpoint{6.200000in}{5.960192in}}%
\pgfusepath{clip}%
\pgfsetbuttcap%
\pgfsetroundjoin%
\definecolor{currentfill}{rgb}{0.800000,0.200000,0.200000}%
\pgfsetfillcolor{currentfill}%
\pgfsetlinewidth{1.003750pt}%
\definecolor{currentstroke}{rgb}{0.800000,0.200000,0.200000}%
\pgfsetstrokecolor{currentstroke}%
\pgfsetdash{}{0pt}%
\pgfpathmoveto{\pgfqpoint{2.502922in}{1.310641in}}%
\pgfpathcurveto{\pgfqpoint{2.508746in}{1.310641in}}{\pgfqpoint{2.514332in}{1.312955in}}{\pgfqpoint{2.518450in}{1.317073in}}%
\pgfpathcurveto{\pgfqpoint{2.522568in}{1.321192in}}{\pgfqpoint{2.524882in}{1.326778in}}{\pgfqpoint{2.524882in}{1.332602in}}%
\pgfpathcurveto{\pgfqpoint{2.524882in}{1.338426in}}{\pgfqpoint{2.522568in}{1.344012in}}{\pgfqpoint{2.518450in}{1.348130in}}%
\pgfpathcurveto{\pgfqpoint{2.514332in}{1.352248in}}{\pgfqpoint{2.508746in}{1.354562in}}{\pgfqpoint{2.502922in}{1.354562in}}%
\pgfpathcurveto{\pgfqpoint{2.497098in}{1.354562in}}{\pgfqpoint{2.491511in}{1.352248in}}{\pgfqpoint{2.487393in}{1.348130in}}%
\pgfpathcurveto{\pgfqpoint{2.483275in}{1.344012in}}{\pgfqpoint{2.480961in}{1.338426in}}{\pgfqpoint{2.480961in}{1.332602in}}%
\pgfpathcurveto{\pgfqpoint{2.480961in}{1.326778in}}{\pgfqpoint{2.483275in}{1.321192in}}{\pgfqpoint{2.487393in}{1.317073in}}%
\pgfpathcurveto{\pgfqpoint{2.491511in}{1.312955in}}{\pgfqpoint{2.497098in}{1.310641in}}{\pgfqpoint{2.502922in}{1.310641in}}%
\pgfpathlineto{\pgfqpoint{2.502922in}{1.310641in}}%
\pgfpathclose%
\pgfusepath{stroke,fill}%
\end{pgfscope}%
\begin{pgfscope}%
\pgfpathrectangle{\pgfqpoint{1.000000in}{0.979904in}}{\pgfqpoint{6.200000in}{5.960192in}}%
\pgfusepath{clip}%
\pgfsetbuttcap%
\pgfsetroundjoin%
\definecolor{currentfill}{rgb}{0.800000,0.200000,0.200000}%
\pgfsetfillcolor{currentfill}%
\pgfsetlinewidth{1.003750pt}%
\definecolor{currentstroke}{rgb}{0.800000,0.200000,0.200000}%
\pgfsetstrokecolor{currentstroke}%
\pgfsetdash{}{0pt}%
\pgfpathmoveto{\pgfqpoint{2.608375in}{1.314953in}}%
\pgfpathcurveto{\pgfqpoint{2.614198in}{1.314953in}}{\pgfqpoint{2.619785in}{1.317267in}}{\pgfqpoint{2.623903in}{1.321385in}}%
\pgfpathcurveto{\pgfqpoint{2.628021in}{1.325503in}}{\pgfqpoint{2.630335in}{1.331089in}}{\pgfqpoint{2.630335in}{1.336913in}}%
\pgfpathcurveto{\pgfqpoint{2.630335in}{1.342737in}}{\pgfqpoint{2.628021in}{1.348323in}}{\pgfqpoint{2.623903in}{1.352441in}}%
\pgfpathcurveto{\pgfqpoint{2.619785in}{1.356559in}}{\pgfqpoint{2.614198in}{1.358873in}}{\pgfqpoint{2.608375in}{1.358873in}}%
\pgfpathcurveto{\pgfqpoint{2.602551in}{1.358873in}}{\pgfqpoint{2.596964in}{1.356559in}}{\pgfqpoint{2.592846in}{1.352441in}}%
\pgfpathcurveto{\pgfqpoint{2.588728in}{1.348323in}}{\pgfqpoint{2.586414in}{1.342737in}}{\pgfqpoint{2.586414in}{1.336913in}}%
\pgfpathcurveto{\pgfqpoint{2.586414in}{1.331089in}}{\pgfqpoint{2.588728in}{1.325503in}}{\pgfqpoint{2.592846in}{1.321385in}}%
\pgfpathcurveto{\pgfqpoint{2.596964in}{1.317267in}}{\pgfqpoint{2.602551in}{1.314953in}}{\pgfqpoint{2.608375in}{1.314953in}}%
\pgfpathlineto{\pgfqpoint{2.608375in}{1.314953in}}%
\pgfpathclose%
\pgfusepath{stroke,fill}%
\end{pgfscope}%
\begin{pgfscope}%
\pgfpathrectangle{\pgfqpoint{1.000000in}{0.979904in}}{\pgfqpoint{6.200000in}{5.960192in}}%
\pgfusepath{clip}%
\pgfsetbuttcap%
\pgfsetroundjoin%
\definecolor{currentfill}{rgb}{0.800000,0.200000,0.200000}%
\pgfsetfillcolor{currentfill}%
\pgfsetlinewidth{1.003750pt}%
\definecolor{currentstroke}{rgb}{0.800000,0.200000,0.200000}%
\pgfsetstrokecolor{currentstroke}%
\pgfsetdash{}{0pt}%
\pgfpathmoveto{\pgfqpoint{2.710242in}{1.317342in}}%
\pgfpathcurveto{\pgfqpoint{2.716066in}{1.317342in}}{\pgfqpoint{2.721653in}{1.319656in}}{\pgfqpoint{2.725771in}{1.323774in}}%
\pgfpathcurveto{\pgfqpoint{2.729889in}{1.327892in}}{\pgfqpoint{2.732203in}{1.333478in}}{\pgfqpoint{2.732203in}{1.339302in}}%
\pgfpathcurveto{\pgfqpoint{2.732203in}{1.345126in}}{\pgfqpoint{2.729889in}{1.350712in}}{\pgfqpoint{2.725771in}{1.354830in}}%
\pgfpathcurveto{\pgfqpoint{2.721653in}{1.358948in}}{\pgfqpoint{2.716066in}{1.361262in}}{\pgfqpoint{2.710242in}{1.361262in}}%
\pgfpathcurveto{\pgfqpoint{2.704419in}{1.361262in}}{\pgfqpoint{2.698832in}{1.358948in}}{\pgfqpoint{2.694714in}{1.354830in}}%
\pgfpathcurveto{\pgfqpoint{2.690596in}{1.350712in}}{\pgfqpoint{2.688282in}{1.345126in}}{\pgfqpoint{2.688282in}{1.339302in}}%
\pgfpathcurveto{\pgfqpoint{2.688282in}{1.333478in}}{\pgfqpoint{2.690596in}{1.327892in}}{\pgfqpoint{2.694714in}{1.323774in}}%
\pgfpathcurveto{\pgfqpoint{2.698832in}{1.319656in}}{\pgfqpoint{2.704419in}{1.317342in}}{\pgfqpoint{2.710242in}{1.317342in}}%
\pgfpathlineto{\pgfqpoint{2.710242in}{1.317342in}}%
\pgfpathclose%
\pgfusepath{stroke,fill}%
\end{pgfscope}%
\begin{pgfscope}%
\pgfpathrectangle{\pgfqpoint{1.000000in}{0.979904in}}{\pgfqpoint{6.200000in}{5.960192in}}%
\pgfusepath{clip}%
\pgfsetbuttcap%
\pgfsetroundjoin%
\definecolor{currentfill}{rgb}{0.800000,0.200000,0.200000}%
\pgfsetfillcolor{currentfill}%
\pgfsetlinewidth{1.003750pt}%
\definecolor{currentstroke}{rgb}{0.800000,0.200000,0.200000}%
\pgfsetstrokecolor{currentstroke}%
\pgfsetdash{}{0pt}%
\pgfpathmoveto{\pgfqpoint{2.805982in}{1.269632in}}%
\pgfpathcurveto{\pgfqpoint{2.811806in}{1.269632in}}{\pgfqpoint{2.817392in}{1.271946in}}{\pgfqpoint{2.821510in}{1.276064in}}%
\pgfpathcurveto{\pgfqpoint{2.825628in}{1.280182in}}{\pgfqpoint{2.827942in}{1.285768in}}{\pgfqpoint{2.827942in}{1.291592in}}%
\pgfpathcurveto{\pgfqpoint{2.827942in}{1.297416in}}{\pgfqpoint{2.825628in}{1.303002in}}{\pgfqpoint{2.821510in}{1.307120in}}%
\pgfpathcurveto{\pgfqpoint{2.817392in}{1.311238in}}{\pgfqpoint{2.811806in}{1.313552in}}{\pgfqpoint{2.805982in}{1.313552in}}%
\pgfpathcurveto{\pgfqpoint{2.800158in}{1.313552in}}{\pgfqpoint{2.794572in}{1.311238in}}{\pgfqpoint{2.790454in}{1.307120in}}%
\pgfpathcurveto{\pgfqpoint{2.786336in}{1.303002in}}{\pgfqpoint{2.784022in}{1.297416in}}{\pgfqpoint{2.784022in}{1.291592in}}%
\pgfpathcurveto{\pgfqpoint{2.784022in}{1.285768in}}{\pgfqpoint{2.786336in}{1.280182in}}{\pgfqpoint{2.790454in}{1.276064in}}%
\pgfpathcurveto{\pgfqpoint{2.794572in}{1.271946in}}{\pgfqpoint{2.800158in}{1.269632in}}{\pgfqpoint{2.805982in}{1.269632in}}%
\pgfpathlineto{\pgfqpoint{2.805982in}{1.269632in}}%
\pgfpathclose%
\pgfusepath{stroke,fill}%
\end{pgfscope}%
\begin{pgfscope}%
\pgfpathrectangle{\pgfqpoint{1.000000in}{0.979904in}}{\pgfqpoint{6.200000in}{5.960192in}}%
\pgfusepath{clip}%
\pgfsetbuttcap%
\pgfsetroundjoin%
\definecolor{currentfill}{rgb}{0.800000,0.200000,0.200000}%
\pgfsetfillcolor{currentfill}%
\pgfsetlinewidth{1.003750pt}%
\definecolor{currentstroke}{rgb}{0.800000,0.200000,0.200000}%
\pgfsetstrokecolor{currentstroke}%
\pgfsetdash{}{0pt}%
\pgfpathmoveto{\pgfqpoint{2.906903in}{1.228861in}}%
\pgfpathcurveto{\pgfqpoint{2.912727in}{1.228861in}}{\pgfqpoint{2.918313in}{1.231175in}}{\pgfqpoint{2.922431in}{1.235293in}}%
\pgfpathcurveto{\pgfqpoint{2.926549in}{1.239412in}}{\pgfqpoint{2.928863in}{1.244998in}}{\pgfqpoint{2.928863in}{1.250822in}}%
\pgfpathcurveto{\pgfqpoint{2.928863in}{1.256646in}}{\pgfqpoint{2.926549in}{1.262232in}}{\pgfqpoint{2.922431in}{1.266350in}}%
\pgfpathcurveto{\pgfqpoint{2.918313in}{1.270468in}}{\pgfqpoint{2.912727in}{1.272782in}}{\pgfqpoint{2.906903in}{1.272782in}}%
\pgfpathcurveto{\pgfqpoint{2.901079in}{1.272782in}}{\pgfqpoint{2.895492in}{1.270468in}}{\pgfqpoint{2.891374in}{1.266350in}}%
\pgfpathcurveto{\pgfqpoint{2.887256in}{1.262232in}}{\pgfqpoint{2.884942in}{1.256646in}}{\pgfqpoint{2.884942in}{1.250822in}}%
\pgfpathcurveto{\pgfqpoint{2.884942in}{1.244998in}}{\pgfqpoint{2.887256in}{1.239412in}}{\pgfqpoint{2.891374in}{1.235293in}}%
\pgfpathcurveto{\pgfqpoint{2.895492in}{1.231175in}}{\pgfqpoint{2.901079in}{1.228861in}}{\pgfqpoint{2.906903in}{1.228861in}}%
\pgfpathlineto{\pgfqpoint{2.906903in}{1.228861in}}%
\pgfpathclose%
\pgfusepath{stroke,fill}%
\end{pgfscope}%
\begin{pgfscope}%
\pgfpathrectangle{\pgfqpoint{1.000000in}{0.979904in}}{\pgfqpoint{6.200000in}{5.960192in}}%
\pgfusepath{clip}%
\pgfsetbuttcap%
\pgfsetroundjoin%
\definecolor{currentfill}{rgb}{0.800000,0.200000,0.200000}%
\pgfsetfillcolor{currentfill}%
\pgfsetlinewidth{1.003750pt}%
\definecolor{currentstroke}{rgb}{0.800000,0.200000,0.200000}%
\pgfsetstrokecolor{currentstroke}%
\pgfsetdash{}{0pt}%
\pgfpathmoveto{\pgfqpoint{3.007369in}{1.302030in}}%
\pgfpathcurveto{\pgfqpoint{3.013193in}{1.302030in}}{\pgfqpoint{3.018779in}{1.304343in}}{\pgfqpoint{3.022897in}{1.308462in}}%
\pgfpathcurveto{\pgfqpoint{3.027016in}{1.312580in}}{\pgfqpoint{3.029329in}{1.318166in}}{\pgfqpoint{3.029329in}{1.323990in}}%
\pgfpathcurveto{\pgfqpoint{3.029329in}{1.329814in}}{\pgfqpoint{3.027016in}{1.335400in}}{\pgfqpoint{3.022897in}{1.339518in}}%
\pgfpathcurveto{\pgfqpoint{3.018779in}{1.343636in}}{\pgfqpoint{3.013193in}{1.345950in}}{\pgfqpoint{3.007369in}{1.345950in}}%
\pgfpathcurveto{\pgfqpoint{3.001545in}{1.345950in}}{\pgfqpoint{2.995959in}{1.343636in}}{\pgfqpoint{2.991841in}{1.339518in}}%
\pgfpathcurveto{\pgfqpoint{2.987723in}{1.335400in}}{\pgfqpoint{2.985409in}{1.329814in}}{\pgfqpoint{2.985409in}{1.323990in}}%
\pgfpathcurveto{\pgfqpoint{2.985409in}{1.318166in}}{\pgfqpoint{2.987723in}{1.312580in}}{\pgfqpoint{2.991841in}{1.308462in}}%
\pgfpathcurveto{\pgfqpoint{2.995959in}{1.304343in}}{\pgfqpoint{3.001545in}{1.302030in}}{\pgfqpoint{3.007369in}{1.302030in}}%
\pgfpathlineto{\pgfqpoint{3.007369in}{1.302030in}}%
\pgfpathclose%
\pgfusepath{stroke,fill}%
\end{pgfscope}%
\begin{pgfscope}%
\pgfpathrectangle{\pgfqpoint{1.000000in}{0.979904in}}{\pgfqpoint{6.200000in}{5.960192in}}%
\pgfusepath{clip}%
\pgfsetbuttcap%
\pgfsetroundjoin%
\definecolor{currentfill}{rgb}{0.800000,0.200000,0.200000}%
\pgfsetfillcolor{currentfill}%
\pgfsetlinewidth{1.003750pt}%
\definecolor{currentstroke}{rgb}{0.800000,0.200000,0.200000}%
\pgfsetstrokecolor{currentstroke}%
\pgfsetdash{}{0pt}%
\pgfpathmoveto{\pgfqpoint{3.105472in}{1.318218in}}%
\pgfpathcurveto{\pgfqpoint{3.111296in}{1.318218in}}{\pgfqpoint{3.116882in}{1.320532in}}{\pgfqpoint{3.121001in}{1.324650in}}%
\pgfpathcurveto{\pgfqpoint{3.125119in}{1.328769in}}{\pgfqpoint{3.127433in}{1.334355in}}{\pgfqpoint{3.127433in}{1.340179in}}%
\pgfpathcurveto{\pgfqpoint{3.127433in}{1.346003in}}{\pgfqpoint{3.125119in}{1.351589in}}{\pgfqpoint{3.121001in}{1.355707in}}%
\pgfpathcurveto{\pgfqpoint{3.116882in}{1.359825in}}{\pgfqpoint{3.111296in}{1.362139in}}{\pgfqpoint{3.105472in}{1.362139in}}%
\pgfpathcurveto{\pgfqpoint{3.099648in}{1.362139in}}{\pgfqpoint{3.094062in}{1.359825in}}{\pgfqpoint{3.089944in}{1.355707in}}%
\pgfpathcurveto{\pgfqpoint{3.085826in}{1.351589in}}{\pgfqpoint{3.083512in}{1.346003in}}{\pgfqpoint{3.083512in}{1.340179in}}%
\pgfpathcurveto{\pgfqpoint{3.083512in}{1.334355in}}{\pgfqpoint{3.085826in}{1.328769in}}{\pgfqpoint{3.089944in}{1.324650in}}%
\pgfpathcurveto{\pgfqpoint{3.094062in}{1.320532in}}{\pgfqpoint{3.099648in}{1.318218in}}{\pgfqpoint{3.105472in}{1.318218in}}%
\pgfpathlineto{\pgfqpoint{3.105472in}{1.318218in}}%
\pgfpathclose%
\pgfusepath{stroke,fill}%
\end{pgfscope}%
\begin{pgfscope}%
\pgfpathrectangle{\pgfqpoint{1.000000in}{0.979904in}}{\pgfqpoint{6.200000in}{5.960192in}}%
\pgfusepath{clip}%
\pgfsetbuttcap%
\pgfsetroundjoin%
\definecolor{currentfill}{rgb}{0.800000,0.200000,0.200000}%
\pgfsetfillcolor{currentfill}%
\pgfsetlinewidth{1.003750pt}%
\definecolor{currentstroke}{rgb}{0.800000,0.200000,0.200000}%
\pgfsetstrokecolor{currentstroke}%
\pgfsetdash{}{0pt}%
\pgfpathmoveto{\pgfqpoint{3.205736in}{1.318343in}}%
\pgfpathcurveto{\pgfqpoint{3.211560in}{1.318343in}}{\pgfqpoint{3.217146in}{1.320657in}}{\pgfqpoint{3.221265in}{1.324775in}}%
\pgfpathcurveto{\pgfqpoint{3.225383in}{1.328893in}}{\pgfqpoint{3.227697in}{1.334479in}}{\pgfqpoint{3.227697in}{1.340303in}}%
\pgfpathcurveto{\pgfqpoint{3.227697in}{1.346127in}}{\pgfqpoint{3.225383in}{1.351713in}}{\pgfqpoint{3.221265in}{1.355832in}}%
\pgfpathcurveto{\pgfqpoint{3.217146in}{1.359950in}}{\pgfqpoint{3.211560in}{1.362264in}}{\pgfqpoint{3.205736in}{1.362264in}}%
\pgfpathcurveto{\pgfqpoint{3.199912in}{1.362264in}}{\pgfqpoint{3.194326in}{1.359950in}}{\pgfqpoint{3.190208in}{1.355832in}}%
\pgfpathcurveto{\pgfqpoint{3.186090in}{1.351713in}}{\pgfqpoint{3.183776in}{1.346127in}}{\pgfqpoint{3.183776in}{1.340303in}}%
\pgfpathcurveto{\pgfqpoint{3.183776in}{1.334479in}}{\pgfqpoint{3.186090in}{1.328893in}}{\pgfqpoint{3.190208in}{1.324775in}}%
\pgfpathcurveto{\pgfqpoint{3.194326in}{1.320657in}}{\pgfqpoint{3.199912in}{1.318343in}}{\pgfqpoint{3.205736in}{1.318343in}}%
\pgfpathlineto{\pgfqpoint{3.205736in}{1.318343in}}%
\pgfpathclose%
\pgfusepath{stroke,fill}%
\end{pgfscope}%
\begin{pgfscope}%
\pgfpathrectangle{\pgfqpoint{1.000000in}{0.979904in}}{\pgfqpoint{6.200000in}{5.960192in}}%
\pgfusepath{clip}%
\pgfsetbuttcap%
\pgfsetroundjoin%
\definecolor{currentfill}{rgb}{0.800000,0.200000,0.200000}%
\pgfsetfillcolor{currentfill}%
\pgfsetlinewidth{1.003750pt}%
\definecolor{currentstroke}{rgb}{0.800000,0.200000,0.200000}%
\pgfsetstrokecolor{currentstroke}%
\pgfsetdash{}{0pt}%
\pgfpathmoveto{\pgfqpoint{3.319688in}{1.271348in}}%
\pgfpathcurveto{\pgfqpoint{3.325512in}{1.271348in}}{\pgfqpoint{3.331098in}{1.273662in}}{\pgfqpoint{3.335216in}{1.277780in}}%
\pgfpathcurveto{\pgfqpoint{3.339334in}{1.281898in}}{\pgfqpoint{3.341648in}{1.287485in}}{\pgfqpoint{3.341648in}{1.293309in}}%
\pgfpathcurveto{\pgfqpoint{3.341648in}{1.299132in}}{\pgfqpoint{3.339334in}{1.304719in}}{\pgfqpoint{3.335216in}{1.308837in}}%
\pgfpathcurveto{\pgfqpoint{3.331098in}{1.312955in}}{\pgfqpoint{3.325512in}{1.315269in}}{\pgfqpoint{3.319688in}{1.315269in}}%
\pgfpathcurveto{\pgfqpoint{3.313864in}{1.315269in}}{\pgfqpoint{3.308278in}{1.312955in}}{\pgfqpoint{3.304160in}{1.308837in}}%
\pgfpathcurveto{\pgfqpoint{3.300041in}{1.304719in}}{\pgfqpoint{3.297728in}{1.299132in}}{\pgfqpoint{3.297728in}{1.293309in}}%
\pgfpathcurveto{\pgfqpoint{3.297728in}{1.287485in}}{\pgfqpoint{3.300041in}{1.281898in}}{\pgfqpoint{3.304160in}{1.277780in}}%
\pgfpathcurveto{\pgfqpoint{3.308278in}{1.273662in}}{\pgfqpoint{3.313864in}{1.271348in}}{\pgfqpoint{3.319688in}{1.271348in}}%
\pgfpathlineto{\pgfqpoint{3.319688in}{1.271348in}}%
\pgfpathclose%
\pgfusepath{stroke,fill}%
\end{pgfscope}%
\begin{pgfscope}%
\pgfpathrectangle{\pgfqpoint{1.000000in}{0.979904in}}{\pgfqpoint{6.200000in}{5.960192in}}%
\pgfusepath{clip}%
\pgfsetbuttcap%
\pgfsetroundjoin%
\definecolor{currentfill}{rgb}{0.800000,0.200000,0.200000}%
\pgfsetfillcolor{currentfill}%
\pgfsetlinewidth{1.003750pt}%
\definecolor{currentstroke}{rgb}{0.800000,0.200000,0.200000}%
\pgfsetstrokecolor{currentstroke}%
\pgfsetdash{}{0pt}%
\pgfpathmoveto{\pgfqpoint{3.427921in}{1.273732in}}%
\pgfpathcurveto{\pgfqpoint{3.433745in}{1.273732in}}{\pgfqpoint{3.439331in}{1.276046in}}{\pgfqpoint{3.443449in}{1.280164in}}%
\pgfpathcurveto{\pgfqpoint{3.447567in}{1.284282in}}{\pgfqpoint{3.449881in}{1.289869in}}{\pgfqpoint{3.449881in}{1.295692in}}%
\pgfpathcurveto{\pgfqpoint{3.449881in}{1.301516in}}{\pgfqpoint{3.447567in}{1.307103in}}{\pgfqpoint{3.443449in}{1.311221in}}%
\pgfpathcurveto{\pgfqpoint{3.439331in}{1.315339in}}{\pgfqpoint{3.433745in}{1.317653in}}{\pgfqpoint{3.427921in}{1.317653in}}%
\pgfpathcurveto{\pgfqpoint{3.422097in}{1.317653in}}{\pgfqpoint{3.416511in}{1.315339in}}{\pgfqpoint{3.412393in}{1.311221in}}%
\pgfpathcurveto{\pgfqpoint{3.408275in}{1.307103in}}{\pgfqpoint{3.405961in}{1.301516in}}{\pgfqpoint{3.405961in}{1.295692in}}%
\pgfpathcurveto{\pgfqpoint{3.405961in}{1.289869in}}{\pgfqpoint{3.408275in}{1.284282in}}{\pgfqpoint{3.412393in}{1.280164in}}%
\pgfpathcurveto{\pgfqpoint{3.416511in}{1.276046in}}{\pgfqpoint{3.422097in}{1.273732in}}{\pgfqpoint{3.427921in}{1.273732in}}%
\pgfpathlineto{\pgfqpoint{3.427921in}{1.273732in}}%
\pgfpathclose%
\pgfusepath{stroke,fill}%
\end{pgfscope}%
\begin{pgfscope}%
\pgfpathrectangle{\pgfqpoint{1.000000in}{0.979904in}}{\pgfqpoint{6.200000in}{5.960192in}}%
\pgfusepath{clip}%
\pgfsetbuttcap%
\pgfsetroundjoin%
\definecolor{currentfill}{rgb}{0.800000,0.200000,0.200000}%
\pgfsetfillcolor{currentfill}%
\pgfsetlinewidth{1.003750pt}%
\definecolor{currentstroke}{rgb}{0.800000,0.200000,0.200000}%
\pgfsetstrokecolor{currentstroke}%
\pgfsetdash{}{0pt}%
\pgfpathmoveto{\pgfqpoint{3.507250in}{1.362385in}}%
\pgfpathcurveto{\pgfqpoint{3.513074in}{1.362385in}}{\pgfqpoint{3.518660in}{1.364699in}}{\pgfqpoint{3.522778in}{1.368817in}}%
\pgfpathcurveto{\pgfqpoint{3.526896in}{1.372935in}}{\pgfqpoint{3.529210in}{1.378521in}}{\pgfqpoint{3.529210in}{1.384345in}}%
\pgfpathcurveto{\pgfqpoint{3.529210in}{1.390169in}}{\pgfqpoint{3.526896in}{1.395755in}}{\pgfqpoint{3.522778in}{1.399874in}}%
\pgfpathcurveto{\pgfqpoint{3.518660in}{1.403992in}}{\pgfqpoint{3.513074in}{1.406306in}}{\pgfqpoint{3.507250in}{1.406306in}}%
\pgfpathcurveto{\pgfqpoint{3.501426in}{1.406306in}}{\pgfqpoint{3.495840in}{1.403992in}}{\pgfqpoint{3.491722in}{1.399874in}}%
\pgfpathcurveto{\pgfqpoint{3.487604in}{1.395755in}}{\pgfqpoint{3.485290in}{1.390169in}}{\pgfqpoint{3.485290in}{1.384345in}}%
\pgfpathcurveto{\pgfqpoint{3.485290in}{1.378521in}}{\pgfqpoint{3.487604in}{1.372935in}}{\pgfqpoint{3.491722in}{1.368817in}}%
\pgfpathcurveto{\pgfqpoint{3.495840in}{1.364699in}}{\pgfqpoint{3.501426in}{1.362385in}}{\pgfqpoint{3.507250in}{1.362385in}}%
\pgfpathlineto{\pgfqpoint{3.507250in}{1.362385in}}%
\pgfpathclose%
\pgfusepath{stroke,fill}%
\end{pgfscope}%
\begin{pgfscope}%
\pgfpathrectangle{\pgfqpoint{1.000000in}{0.979904in}}{\pgfqpoint{6.200000in}{5.960192in}}%
\pgfusepath{clip}%
\pgfsetbuttcap%
\pgfsetroundjoin%
\definecolor{currentfill}{rgb}{0.800000,0.200000,0.200000}%
\pgfsetfillcolor{currentfill}%
\pgfsetlinewidth{1.003750pt}%
\definecolor{currentstroke}{rgb}{0.800000,0.200000,0.200000}%
\pgfsetstrokecolor{currentstroke}%
\pgfsetdash{}{0pt}%
\pgfpathmoveto{\pgfqpoint{3.606280in}{1.391257in}}%
\pgfpathcurveto{\pgfqpoint{3.612104in}{1.391257in}}{\pgfqpoint{3.617690in}{1.393571in}}{\pgfqpoint{3.621808in}{1.397689in}}%
\pgfpathcurveto{\pgfqpoint{3.625926in}{1.401807in}}{\pgfqpoint{3.628240in}{1.407393in}}{\pgfqpoint{3.628240in}{1.413217in}}%
\pgfpathcurveto{\pgfqpoint{3.628240in}{1.419041in}}{\pgfqpoint{3.625926in}{1.424627in}}{\pgfqpoint{3.621808in}{1.428746in}}%
\pgfpathcurveto{\pgfqpoint{3.617690in}{1.432864in}}{\pgfqpoint{3.612104in}{1.435178in}}{\pgfqpoint{3.606280in}{1.435178in}}%
\pgfpathcurveto{\pgfqpoint{3.600456in}{1.435178in}}{\pgfqpoint{3.594870in}{1.432864in}}{\pgfqpoint{3.590751in}{1.428746in}}%
\pgfpathcurveto{\pgfqpoint{3.586633in}{1.424627in}}{\pgfqpoint{3.584319in}{1.419041in}}{\pgfqpoint{3.584319in}{1.413217in}}%
\pgfpathcurveto{\pgfqpoint{3.584319in}{1.407393in}}{\pgfqpoint{3.586633in}{1.401807in}}{\pgfqpoint{3.590751in}{1.397689in}}%
\pgfpathcurveto{\pgfqpoint{3.594870in}{1.393571in}}{\pgfqpoint{3.600456in}{1.391257in}}{\pgfqpoint{3.606280in}{1.391257in}}%
\pgfpathlineto{\pgfqpoint{3.606280in}{1.391257in}}%
\pgfpathclose%
\pgfusepath{stroke,fill}%
\end{pgfscope}%
\begin{pgfscope}%
\pgfpathrectangle{\pgfqpoint{1.000000in}{0.979904in}}{\pgfqpoint{6.200000in}{5.960192in}}%
\pgfusepath{clip}%
\pgfsetbuttcap%
\pgfsetroundjoin%
\definecolor{currentfill}{rgb}{0.800000,0.200000,0.200000}%
\pgfsetfillcolor{currentfill}%
\pgfsetlinewidth{1.003750pt}%
\definecolor{currentstroke}{rgb}{0.800000,0.200000,0.200000}%
\pgfsetstrokecolor{currentstroke}%
\pgfsetdash{}{0pt}%
\pgfpathmoveto{\pgfqpoint{3.668101in}{1.493571in}}%
\pgfpathcurveto{\pgfqpoint{3.673925in}{1.493571in}}{\pgfqpoint{3.679511in}{1.495885in}}{\pgfqpoint{3.683629in}{1.500003in}}%
\pgfpathcurveto{\pgfqpoint{3.687747in}{1.504121in}}{\pgfqpoint{3.690061in}{1.509707in}}{\pgfqpoint{3.690061in}{1.515531in}}%
\pgfpathcurveto{\pgfqpoint{3.690061in}{1.521355in}}{\pgfqpoint{3.687747in}{1.526941in}}{\pgfqpoint{3.683629in}{1.531060in}}%
\pgfpathcurveto{\pgfqpoint{3.679511in}{1.535178in}}{\pgfqpoint{3.673925in}{1.537492in}}{\pgfqpoint{3.668101in}{1.537492in}}%
\pgfpathcurveto{\pgfqpoint{3.662277in}{1.537492in}}{\pgfqpoint{3.656691in}{1.535178in}}{\pgfqpoint{3.652573in}{1.531060in}}%
\pgfpathcurveto{\pgfqpoint{3.648454in}{1.526941in}}{\pgfqpoint{3.646141in}{1.521355in}}{\pgfqpoint{3.646141in}{1.515531in}}%
\pgfpathcurveto{\pgfqpoint{3.646141in}{1.509707in}}{\pgfqpoint{3.648454in}{1.504121in}}{\pgfqpoint{3.652573in}{1.500003in}}%
\pgfpathcurveto{\pgfqpoint{3.656691in}{1.495885in}}{\pgfqpoint{3.662277in}{1.493571in}}{\pgfqpoint{3.668101in}{1.493571in}}%
\pgfpathlineto{\pgfqpoint{3.668101in}{1.493571in}}%
\pgfpathclose%
\pgfusepath{stroke,fill}%
\end{pgfscope}%
\begin{pgfscope}%
\pgfpathrectangle{\pgfqpoint{1.000000in}{0.979904in}}{\pgfqpoint{6.200000in}{5.960192in}}%
\pgfusepath{clip}%
\pgfsetbuttcap%
\pgfsetroundjoin%
\definecolor{currentfill}{rgb}{0.800000,0.200000,0.200000}%
\pgfsetfillcolor{currentfill}%
\pgfsetlinewidth{1.003750pt}%
\definecolor{currentstroke}{rgb}{0.800000,0.200000,0.200000}%
\pgfsetstrokecolor{currentstroke}%
\pgfsetdash{}{0pt}%
\pgfpathmoveto{\pgfqpoint{3.795793in}{1.475048in}}%
\pgfpathcurveto{\pgfqpoint{3.801617in}{1.475048in}}{\pgfqpoint{3.807203in}{1.477362in}}{\pgfqpoint{3.811321in}{1.481480in}}%
\pgfpathcurveto{\pgfqpoint{3.815439in}{1.485598in}}{\pgfqpoint{3.817753in}{1.491185in}}{\pgfqpoint{3.817753in}{1.497008in}}%
\pgfpathcurveto{\pgfqpoint{3.817753in}{1.502832in}}{\pgfqpoint{3.815439in}{1.508419in}}{\pgfqpoint{3.811321in}{1.512537in}}%
\pgfpathcurveto{\pgfqpoint{3.807203in}{1.516655in}}{\pgfqpoint{3.801617in}{1.518969in}}{\pgfqpoint{3.795793in}{1.518969in}}%
\pgfpathcurveto{\pgfqpoint{3.789969in}{1.518969in}}{\pgfqpoint{3.784383in}{1.516655in}}{\pgfqpoint{3.780264in}{1.512537in}}%
\pgfpathcurveto{\pgfqpoint{3.776146in}{1.508419in}}{\pgfqpoint{3.773832in}{1.502832in}}{\pgfqpoint{3.773832in}{1.497008in}}%
\pgfpathcurveto{\pgfqpoint{3.773832in}{1.491185in}}{\pgfqpoint{3.776146in}{1.485598in}}{\pgfqpoint{3.780264in}{1.481480in}}%
\pgfpathcurveto{\pgfqpoint{3.784383in}{1.477362in}}{\pgfqpoint{3.789969in}{1.475048in}}{\pgfqpoint{3.795793in}{1.475048in}}%
\pgfpathlineto{\pgfqpoint{3.795793in}{1.475048in}}%
\pgfpathclose%
\pgfusepath{stroke,fill}%
\end{pgfscope}%
\begin{pgfscope}%
\pgfpathrectangle{\pgfqpoint{1.000000in}{0.979904in}}{\pgfqpoint{6.200000in}{5.960192in}}%
\pgfusepath{clip}%
\pgfsetbuttcap%
\pgfsetroundjoin%
\definecolor{currentfill}{rgb}{0.800000,0.200000,0.200000}%
\pgfsetfillcolor{currentfill}%
\pgfsetlinewidth{1.003750pt}%
\definecolor{currentstroke}{rgb}{0.800000,0.200000,0.200000}%
\pgfsetstrokecolor{currentstroke}%
\pgfsetdash{}{0pt}%
\pgfpathmoveto{\pgfqpoint{3.872260in}{1.546625in}}%
\pgfpathcurveto{\pgfqpoint{3.878084in}{1.546625in}}{\pgfqpoint{3.883670in}{1.548938in}}{\pgfqpoint{3.887788in}{1.553057in}}%
\pgfpathcurveto{\pgfqpoint{3.891906in}{1.557175in}}{\pgfqpoint{3.894220in}{1.562761in}}{\pgfqpoint{3.894220in}{1.568585in}}%
\pgfpathcurveto{\pgfqpoint{3.894220in}{1.574409in}}{\pgfqpoint{3.891906in}{1.579995in}}{\pgfqpoint{3.887788in}{1.584113in}}%
\pgfpathcurveto{\pgfqpoint{3.883670in}{1.588231in}}{\pgfqpoint{3.878084in}{1.590545in}}{\pgfqpoint{3.872260in}{1.590545in}}%
\pgfpathcurveto{\pgfqpoint{3.866436in}{1.590545in}}{\pgfqpoint{3.860850in}{1.588231in}}{\pgfqpoint{3.856731in}{1.584113in}}%
\pgfpathcurveto{\pgfqpoint{3.852613in}{1.579995in}}{\pgfqpoint{3.850299in}{1.574409in}}{\pgfqpoint{3.850299in}{1.568585in}}%
\pgfpathcurveto{\pgfqpoint{3.850299in}{1.562761in}}{\pgfqpoint{3.852613in}{1.557175in}}{\pgfqpoint{3.856731in}{1.553057in}}%
\pgfpathcurveto{\pgfqpoint{3.860850in}{1.548938in}}{\pgfqpoint{3.866436in}{1.546625in}}{\pgfqpoint{3.872260in}{1.546625in}}%
\pgfpathlineto{\pgfqpoint{3.872260in}{1.546625in}}%
\pgfpathclose%
\pgfusepath{stroke,fill}%
\end{pgfscope}%
\begin{pgfscope}%
\pgfpathrectangle{\pgfqpoint{1.000000in}{0.979904in}}{\pgfqpoint{6.200000in}{5.960192in}}%
\pgfusepath{clip}%
\pgfsetbuttcap%
\pgfsetroundjoin%
\definecolor{currentfill}{rgb}{0.800000,0.200000,0.200000}%
\pgfsetfillcolor{currentfill}%
\pgfsetlinewidth{1.003750pt}%
\definecolor{currentstroke}{rgb}{0.800000,0.200000,0.200000}%
\pgfsetstrokecolor{currentstroke}%
\pgfsetdash{}{0pt}%
\pgfpathmoveto{\pgfqpoint{3.911820in}{1.660823in}}%
\pgfpathcurveto{\pgfqpoint{3.917644in}{1.660823in}}{\pgfqpoint{3.923230in}{1.663137in}}{\pgfqpoint{3.927348in}{1.667255in}}%
\pgfpathcurveto{\pgfqpoint{3.931467in}{1.671373in}}{\pgfqpoint{3.933780in}{1.676959in}}{\pgfqpoint{3.933780in}{1.682783in}}%
\pgfpathcurveto{\pgfqpoint{3.933780in}{1.688607in}}{\pgfqpoint{3.931467in}{1.694193in}}{\pgfqpoint{3.927348in}{1.698312in}}%
\pgfpathcurveto{\pgfqpoint{3.923230in}{1.702430in}}{\pgfqpoint{3.917644in}{1.704744in}}{\pgfqpoint{3.911820in}{1.704744in}}%
\pgfpathcurveto{\pgfqpoint{3.905996in}{1.704744in}}{\pgfqpoint{3.900410in}{1.702430in}}{\pgfqpoint{3.896292in}{1.698312in}}%
\pgfpathcurveto{\pgfqpoint{3.892174in}{1.694193in}}{\pgfqpoint{3.889860in}{1.688607in}}{\pgfqpoint{3.889860in}{1.682783in}}%
\pgfpathcurveto{\pgfqpoint{3.889860in}{1.676959in}}{\pgfqpoint{3.892174in}{1.671373in}}{\pgfqpoint{3.896292in}{1.667255in}}%
\pgfpathcurveto{\pgfqpoint{3.900410in}{1.663137in}}{\pgfqpoint{3.905996in}{1.660823in}}{\pgfqpoint{3.911820in}{1.660823in}}%
\pgfpathlineto{\pgfqpoint{3.911820in}{1.660823in}}%
\pgfpathclose%
\pgfusepath{stroke,fill}%
\end{pgfscope}%
\begin{pgfscope}%
\pgfpathrectangle{\pgfqpoint{1.000000in}{0.979904in}}{\pgfqpoint{6.200000in}{5.960192in}}%
\pgfusepath{clip}%
\pgfsetbuttcap%
\pgfsetroundjoin%
\definecolor{currentfill}{rgb}{0.800000,0.200000,0.200000}%
\pgfsetfillcolor{currentfill}%
\pgfsetlinewidth{1.003750pt}%
\definecolor{currentstroke}{rgb}{0.800000,0.200000,0.200000}%
\pgfsetstrokecolor{currentstroke}%
\pgfsetdash{}{0pt}%
\pgfpathmoveto{\pgfqpoint{3.947483in}{1.767335in}}%
\pgfpathcurveto{\pgfqpoint{3.953307in}{1.767335in}}{\pgfqpoint{3.958893in}{1.769649in}}{\pgfqpoint{3.963011in}{1.773767in}}%
\pgfpathcurveto{\pgfqpoint{3.967129in}{1.777885in}}{\pgfqpoint{3.969443in}{1.783471in}}{\pgfqpoint{3.969443in}{1.789295in}}%
\pgfpathcurveto{\pgfqpoint{3.969443in}{1.795119in}}{\pgfqpoint{3.967129in}{1.800705in}}{\pgfqpoint{3.963011in}{1.804823in}}%
\pgfpathcurveto{\pgfqpoint{3.958893in}{1.808941in}}{\pgfqpoint{3.953307in}{1.811255in}}{\pgfqpoint{3.947483in}{1.811255in}}%
\pgfpathcurveto{\pgfqpoint{3.941659in}{1.811255in}}{\pgfqpoint{3.936073in}{1.808941in}}{\pgfqpoint{3.931954in}{1.804823in}}%
\pgfpathcurveto{\pgfqpoint{3.927836in}{1.800705in}}{\pgfqpoint{3.925522in}{1.795119in}}{\pgfqpoint{3.925522in}{1.789295in}}%
\pgfpathcurveto{\pgfqpoint{3.925522in}{1.783471in}}{\pgfqpoint{3.927836in}{1.777885in}}{\pgfqpoint{3.931954in}{1.773767in}}%
\pgfpathcurveto{\pgfqpoint{3.936073in}{1.769649in}}{\pgfqpoint{3.941659in}{1.767335in}}{\pgfqpoint{3.947483in}{1.767335in}}%
\pgfpathlineto{\pgfqpoint{3.947483in}{1.767335in}}%
\pgfpathclose%
\pgfusepath{stroke,fill}%
\end{pgfscope}%
\begin{pgfscope}%
\pgfpathrectangle{\pgfqpoint{1.000000in}{0.979904in}}{\pgfqpoint{6.200000in}{5.960192in}}%
\pgfusepath{clip}%
\pgfsetbuttcap%
\pgfsetroundjoin%
\definecolor{currentfill}{rgb}{0.800000,0.200000,0.200000}%
\pgfsetfillcolor{currentfill}%
\pgfsetlinewidth{1.003750pt}%
\definecolor{currentstroke}{rgb}{0.800000,0.200000,0.200000}%
\pgfsetstrokecolor{currentstroke}%
\pgfsetdash{}{0pt}%
\pgfpathmoveto{\pgfqpoint{4.036885in}{1.813133in}}%
\pgfpathcurveto{\pgfqpoint{4.042709in}{1.813133in}}{\pgfqpoint{4.048295in}{1.815447in}}{\pgfqpoint{4.052413in}{1.819565in}}%
\pgfpathcurveto{\pgfqpoint{4.056532in}{1.823683in}}{\pgfqpoint{4.058845in}{1.829269in}}{\pgfqpoint{4.058845in}{1.835093in}}%
\pgfpathcurveto{\pgfqpoint{4.058845in}{1.840917in}}{\pgfqpoint{4.056532in}{1.846503in}}{\pgfqpoint{4.052413in}{1.850622in}}%
\pgfpathcurveto{\pgfqpoint{4.048295in}{1.854740in}}{\pgfqpoint{4.042709in}{1.857054in}}{\pgfqpoint{4.036885in}{1.857054in}}%
\pgfpathcurveto{\pgfqpoint{4.031061in}{1.857054in}}{\pgfqpoint{4.025475in}{1.854740in}}{\pgfqpoint{4.021357in}{1.850622in}}%
\pgfpathcurveto{\pgfqpoint{4.017239in}{1.846503in}}{\pgfqpoint{4.014925in}{1.840917in}}{\pgfqpoint{4.014925in}{1.835093in}}%
\pgfpathcurveto{\pgfqpoint{4.014925in}{1.829269in}}{\pgfqpoint{4.017239in}{1.823683in}}{\pgfqpoint{4.021357in}{1.819565in}}%
\pgfpathcurveto{\pgfqpoint{4.025475in}{1.815447in}}{\pgfqpoint{4.031061in}{1.813133in}}{\pgfqpoint{4.036885in}{1.813133in}}%
\pgfpathlineto{\pgfqpoint{4.036885in}{1.813133in}}%
\pgfpathclose%
\pgfusepath{stroke,fill}%
\end{pgfscope}%
\begin{pgfscope}%
\pgfpathrectangle{\pgfqpoint{1.000000in}{0.979904in}}{\pgfqpoint{6.200000in}{5.960192in}}%
\pgfusepath{clip}%
\pgfsetbuttcap%
\pgfsetroundjoin%
\definecolor{currentfill}{rgb}{0.800000,0.200000,0.200000}%
\pgfsetfillcolor{currentfill}%
\pgfsetlinewidth{1.003750pt}%
\definecolor{currentstroke}{rgb}{0.800000,0.200000,0.200000}%
\pgfsetstrokecolor{currentstroke}%
\pgfsetdash{}{0pt}%
\pgfpathmoveto{\pgfqpoint{4.123837in}{1.866469in}}%
\pgfpathcurveto{\pgfqpoint{4.129660in}{1.866469in}}{\pgfqpoint{4.135247in}{1.868783in}}{\pgfqpoint{4.139365in}{1.872901in}}%
\pgfpathcurveto{\pgfqpoint{4.143483in}{1.877019in}}{\pgfqpoint{4.145797in}{1.882605in}}{\pgfqpoint{4.145797in}{1.888429in}}%
\pgfpathcurveto{\pgfqpoint{4.145797in}{1.894253in}}{\pgfqpoint{4.143483in}{1.899839in}}{\pgfqpoint{4.139365in}{1.903957in}}%
\pgfpathcurveto{\pgfqpoint{4.135247in}{1.908076in}}{\pgfqpoint{4.129660in}{1.910389in}}{\pgfqpoint{4.123837in}{1.910389in}}%
\pgfpathcurveto{\pgfqpoint{4.118013in}{1.910389in}}{\pgfqpoint{4.112426in}{1.908076in}}{\pgfqpoint{4.108308in}{1.903957in}}%
\pgfpathcurveto{\pgfqpoint{4.104190in}{1.899839in}}{\pgfqpoint{4.101876in}{1.894253in}}{\pgfqpoint{4.101876in}{1.888429in}}%
\pgfpathcurveto{\pgfqpoint{4.101876in}{1.882605in}}{\pgfqpoint{4.104190in}{1.877019in}}{\pgfqpoint{4.108308in}{1.872901in}}%
\pgfpathcurveto{\pgfqpoint{4.112426in}{1.868783in}}{\pgfqpoint{4.118013in}{1.866469in}}{\pgfqpoint{4.123837in}{1.866469in}}%
\pgfpathlineto{\pgfqpoint{4.123837in}{1.866469in}}%
\pgfpathclose%
\pgfusepath{stroke,fill}%
\end{pgfscope}%
\begin{pgfscope}%
\pgfpathrectangle{\pgfqpoint{1.000000in}{0.979904in}}{\pgfqpoint{6.200000in}{5.960192in}}%
\pgfusepath{clip}%
\pgfsetbuttcap%
\pgfsetroundjoin%
\definecolor{currentfill}{rgb}{0.800000,0.200000,0.200000}%
\pgfsetfillcolor{currentfill}%
\pgfsetlinewidth{1.003750pt}%
\definecolor{currentstroke}{rgb}{0.800000,0.200000,0.200000}%
\pgfsetstrokecolor{currentstroke}%
\pgfsetdash{}{0pt}%
\pgfpathmoveto{\pgfqpoint{4.263056in}{1.886413in}}%
\pgfpathcurveto{\pgfqpoint{4.268880in}{1.886413in}}{\pgfqpoint{4.274466in}{1.888727in}}{\pgfqpoint{4.278584in}{1.892845in}}%
\pgfpathcurveto{\pgfqpoint{4.282702in}{1.896963in}}{\pgfqpoint{4.285016in}{1.902549in}}{\pgfqpoint{4.285016in}{1.908373in}}%
\pgfpathcurveto{\pgfqpoint{4.285016in}{1.914197in}}{\pgfqpoint{4.282702in}{1.919783in}}{\pgfqpoint{4.278584in}{1.923901in}}%
\pgfpathcurveto{\pgfqpoint{4.274466in}{1.928019in}}{\pgfqpoint{4.268880in}{1.930333in}}{\pgfqpoint{4.263056in}{1.930333in}}%
\pgfpathcurveto{\pgfqpoint{4.257232in}{1.930333in}}{\pgfqpoint{4.251646in}{1.928019in}}{\pgfqpoint{4.247528in}{1.923901in}}%
\pgfpathcurveto{\pgfqpoint{4.243409in}{1.919783in}}{\pgfqpoint{4.241095in}{1.914197in}}{\pgfqpoint{4.241095in}{1.908373in}}%
\pgfpathcurveto{\pgfqpoint{4.241095in}{1.902549in}}{\pgfqpoint{4.243409in}{1.896963in}}{\pgfqpoint{4.247528in}{1.892845in}}%
\pgfpathcurveto{\pgfqpoint{4.251646in}{1.888727in}}{\pgfqpoint{4.257232in}{1.886413in}}{\pgfqpoint{4.263056in}{1.886413in}}%
\pgfpathlineto{\pgfqpoint{4.263056in}{1.886413in}}%
\pgfpathclose%
\pgfusepath{stroke,fill}%
\end{pgfscope}%
\begin{pgfscope}%
\pgfpathrectangle{\pgfqpoint{1.000000in}{0.979904in}}{\pgfqpoint{6.200000in}{5.960192in}}%
\pgfusepath{clip}%
\pgfsetbuttcap%
\pgfsetroundjoin%
\definecolor{currentfill}{rgb}{0.800000,0.200000,0.200000}%
\pgfsetfillcolor{currentfill}%
\pgfsetlinewidth{1.003750pt}%
\definecolor{currentstroke}{rgb}{0.800000,0.200000,0.200000}%
\pgfsetstrokecolor{currentstroke}%
\pgfsetdash{}{0pt}%
\pgfpathmoveto{\pgfqpoint{4.339943in}{1.961536in}}%
\pgfpathcurveto{\pgfqpoint{4.345767in}{1.961536in}}{\pgfqpoint{4.351353in}{1.963850in}}{\pgfqpoint{4.355471in}{1.967968in}}%
\pgfpathcurveto{\pgfqpoint{4.359589in}{1.972086in}}{\pgfqpoint{4.361903in}{1.977673in}}{\pgfqpoint{4.361903in}{1.983496in}}%
\pgfpathcurveto{\pgfqpoint{4.361903in}{1.989320in}}{\pgfqpoint{4.359589in}{1.994907in}}{\pgfqpoint{4.355471in}{1.999025in}}%
\pgfpathcurveto{\pgfqpoint{4.351353in}{2.003143in}}{\pgfqpoint{4.345767in}{2.005457in}}{\pgfqpoint{4.339943in}{2.005457in}}%
\pgfpathcurveto{\pgfqpoint{4.334119in}{2.005457in}}{\pgfqpoint{4.328533in}{2.003143in}}{\pgfqpoint{4.324415in}{1.999025in}}%
\pgfpathcurveto{\pgfqpoint{4.320296in}{1.994907in}}{\pgfqpoint{4.317983in}{1.989320in}}{\pgfqpoint{4.317983in}{1.983496in}}%
\pgfpathcurveto{\pgfqpoint{4.317983in}{1.977673in}}{\pgfqpoint{4.320296in}{1.972086in}}{\pgfqpoint{4.324415in}{1.967968in}}%
\pgfpathcurveto{\pgfqpoint{4.328533in}{1.963850in}}{\pgfqpoint{4.334119in}{1.961536in}}{\pgfqpoint{4.339943in}{1.961536in}}%
\pgfpathlineto{\pgfqpoint{4.339943in}{1.961536in}}%
\pgfpathclose%
\pgfusepath{stroke,fill}%
\end{pgfscope}%
\begin{pgfscope}%
\pgfpathrectangle{\pgfqpoint{1.000000in}{0.979904in}}{\pgfqpoint{6.200000in}{5.960192in}}%
\pgfusepath{clip}%
\pgfsetbuttcap%
\pgfsetroundjoin%
\definecolor{currentfill}{rgb}{0.800000,0.200000,0.200000}%
\pgfsetfillcolor{currentfill}%
\pgfsetlinewidth{1.003750pt}%
\definecolor{currentstroke}{rgb}{0.800000,0.200000,0.200000}%
\pgfsetstrokecolor{currentstroke}%
\pgfsetdash{}{0pt}%
\pgfpathmoveto{\pgfqpoint{4.302088in}{2.103994in}}%
\pgfpathcurveto{\pgfqpoint{4.307912in}{2.103994in}}{\pgfqpoint{4.313499in}{2.106308in}}{\pgfqpoint{4.317617in}{2.110426in}}%
\pgfpathcurveto{\pgfqpoint{4.321735in}{2.114544in}}{\pgfqpoint{4.324049in}{2.120130in}}{\pgfqpoint{4.324049in}{2.125954in}}%
\pgfpathcurveto{\pgfqpoint{4.324049in}{2.131778in}}{\pgfqpoint{4.321735in}{2.137364in}}{\pgfqpoint{4.317617in}{2.141482in}}%
\pgfpathcurveto{\pgfqpoint{4.313499in}{2.145600in}}{\pgfqpoint{4.307912in}{2.147914in}}{\pgfqpoint{4.302088in}{2.147914in}}%
\pgfpathcurveto{\pgfqpoint{4.296264in}{2.147914in}}{\pgfqpoint{4.290678in}{2.145600in}}{\pgfqpoint{4.286560in}{2.141482in}}%
\pgfpathcurveto{\pgfqpoint{4.282442in}{2.137364in}}{\pgfqpoint{4.280128in}{2.131778in}}{\pgfqpoint{4.280128in}{2.125954in}}%
\pgfpathcurveto{\pgfqpoint{4.280128in}{2.120130in}}{\pgfqpoint{4.282442in}{2.114544in}}{\pgfqpoint{4.286560in}{2.110426in}}%
\pgfpathcurveto{\pgfqpoint{4.290678in}{2.106308in}}{\pgfqpoint{4.296264in}{2.103994in}}{\pgfqpoint{4.302088in}{2.103994in}}%
\pgfpathlineto{\pgfqpoint{4.302088in}{2.103994in}}%
\pgfpathclose%
\pgfusepath{stroke,fill}%
\end{pgfscope}%
\begin{pgfscope}%
\pgfpathrectangle{\pgfqpoint{1.000000in}{0.979904in}}{\pgfqpoint{6.200000in}{5.960192in}}%
\pgfusepath{clip}%
\pgfsetbuttcap%
\pgfsetroundjoin%
\definecolor{currentfill}{rgb}{0.800000,0.200000,0.200000}%
\pgfsetfillcolor{currentfill}%
\pgfsetlinewidth{1.003750pt}%
\definecolor{currentstroke}{rgb}{0.800000,0.200000,0.200000}%
\pgfsetstrokecolor{currentstroke}%
\pgfsetdash{}{0pt}%
\pgfpathmoveto{\pgfqpoint{4.349163in}{2.191648in}}%
\pgfpathcurveto{\pgfqpoint{4.354987in}{2.191648in}}{\pgfqpoint{4.360573in}{2.193961in}}{\pgfqpoint{4.364691in}{2.198080in}}%
\pgfpathcurveto{\pgfqpoint{4.368809in}{2.202198in}}{\pgfqpoint{4.371123in}{2.207784in}}{\pgfqpoint{4.371123in}{2.213608in}}%
\pgfpathcurveto{\pgfqpoint{4.371123in}{2.219432in}}{\pgfqpoint{4.368809in}{2.225018in}}{\pgfqpoint{4.364691in}{2.229136in}}%
\pgfpathcurveto{\pgfqpoint{4.360573in}{2.233254in}}{\pgfqpoint{4.354987in}{2.235568in}}{\pgfqpoint{4.349163in}{2.235568in}}%
\pgfpathcurveto{\pgfqpoint{4.343339in}{2.235568in}}{\pgfqpoint{4.337752in}{2.233254in}}{\pgfqpoint{4.333634in}{2.229136in}}%
\pgfpathcurveto{\pgfqpoint{4.329516in}{2.225018in}}{\pgfqpoint{4.327202in}{2.219432in}}{\pgfqpoint{4.327202in}{2.213608in}}%
\pgfpathcurveto{\pgfqpoint{4.327202in}{2.207784in}}{\pgfqpoint{4.329516in}{2.202198in}}{\pgfqpoint{4.333634in}{2.198080in}}%
\pgfpathcurveto{\pgfqpoint{4.337752in}{2.193961in}}{\pgfqpoint{4.343339in}{2.191648in}}{\pgfqpoint{4.349163in}{2.191648in}}%
\pgfpathlineto{\pgfqpoint{4.349163in}{2.191648in}}%
\pgfpathclose%
\pgfusepath{stroke,fill}%
\end{pgfscope}%
\begin{pgfscope}%
\pgfpathrectangle{\pgfqpoint{1.000000in}{0.979904in}}{\pgfqpoint{6.200000in}{5.960192in}}%
\pgfusepath{clip}%
\pgfsetbuttcap%
\pgfsetroundjoin%
\definecolor{currentfill}{rgb}{0.800000,0.200000,0.200000}%
\pgfsetfillcolor{currentfill}%
\pgfsetlinewidth{1.003750pt}%
\definecolor{currentstroke}{rgb}{0.800000,0.200000,0.200000}%
\pgfsetstrokecolor{currentstroke}%
\pgfsetdash{}{0pt}%
\pgfpathmoveto{\pgfqpoint{4.422048in}{2.269628in}}%
\pgfpathcurveto{\pgfqpoint{4.427871in}{2.269628in}}{\pgfqpoint{4.433458in}{2.271942in}}{\pgfqpoint{4.437576in}{2.276060in}}%
\pgfpathcurveto{\pgfqpoint{4.441694in}{2.280179in}}{\pgfqpoint{4.444008in}{2.285765in}}{\pgfqpoint{4.444008in}{2.291589in}}%
\pgfpathcurveto{\pgfqpoint{4.444008in}{2.297413in}}{\pgfqpoint{4.441694in}{2.302999in}}{\pgfqpoint{4.437576in}{2.307117in}}%
\pgfpathcurveto{\pgfqpoint{4.433458in}{2.311235in}}{\pgfqpoint{4.427871in}{2.313549in}}{\pgfqpoint{4.422048in}{2.313549in}}%
\pgfpathcurveto{\pgfqpoint{4.416224in}{2.313549in}}{\pgfqpoint{4.410637in}{2.311235in}}{\pgfqpoint{4.406519in}{2.307117in}}%
\pgfpathcurveto{\pgfqpoint{4.402401in}{2.302999in}}{\pgfqpoint{4.400087in}{2.297413in}}{\pgfqpoint{4.400087in}{2.291589in}}%
\pgfpathcurveto{\pgfqpoint{4.400087in}{2.285765in}}{\pgfqpoint{4.402401in}{2.280179in}}{\pgfqpoint{4.406519in}{2.276060in}}%
\pgfpathcurveto{\pgfqpoint{4.410637in}{2.271942in}}{\pgfqpoint{4.416224in}{2.269628in}}{\pgfqpoint{4.422048in}{2.269628in}}%
\pgfpathlineto{\pgfqpoint{4.422048in}{2.269628in}}%
\pgfpathclose%
\pgfusepath{stroke,fill}%
\end{pgfscope}%
\begin{pgfscope}%
\pgfpathrectangle{\pgfqpoint{1.000000in}{0.979904in}}{\pgfqpoint{6.200000in}{5.960192in}}%
\pgfusepath{clip}%
\pgfsetbuttcap%
\pgfsetroundjoin%
\definecolor{currentfill}{rgb}{0.800000,0.200000,0.200000}%
\pgfsetfillcolor{currentfill}%
\pgfsetlinewidth{1.003750pt}%
\definecolor{currentstroke}{rgb}{0.800000,0.200000,0.200000}%
\pgfsetstrokecolor{currentstroke}%
\pgfsetdash{}{0pt}%
\pgfpathmoveto{\pgfqpoint{4.444286in}{2.369408in}}%
\pgfpathcurveto{\pgfqpoint{4.450109in}{2.369408in}}{\pgfqpoint{4.455696in}{2.371722in}}{\pgfqpoint{4.459814in}{2.375840in}}%
\pgfpathcurveto{\pgfqpoint{4.463932in}{2.379958in}}{\pgfqpoint{4.466246in}{2.385544in}}{\pgfqpoint{4.466246in}{2.391368in}}%
\pgfpathcurveto{\pgfqpoint{4.466246in}{2.397192in}}{\pgfqpoint{4.463932in}{2.402778in}}{\pgfqpoint{4.459814in}{2.406896in}}%
\pgfpathcurveto{\pgfqpoint{4.455696in}{2.411015in}}{\pgfqpoint{4.450109in}{2.413328in}}{\pgfqpoint{4.444286in}{2.413328in}}%
\pgfpathcurveto{\pgfqpoint{4.438462in}{2.413328in}}{\pgfqpoint{4.432875in}{2.411015in}}{\pgfqpoint{4.428757in}{2.406896in}}%
\pgfpathcurveto{\pgfqpoint{4.424639in}{2.402778in}}{\pgfqpoint{4.422325in}{2.397192in}}{\pgfqpoint{4.422325in}{2.391368in}}%
\pgfpathcurveto{\pgfqpoint{4.422325in}{2.385544in}}{\pgfqpoint{4.424639in}{2.379958in}}{\pgfqpoint{4.428757in}{2.375840in}}%
\pgfpathcurveto{\pgfqpoint{4.432875in}{2.371722in}}{\pgfqpoint{4.438462in}{2.369408in}}{\pgfqpoint{4.444286in}{2.369408in}}%
\pgfpathlineto{\pgfqpoint{4.444286in}{2.369408in}}%
\pgfpathclose%
\pgfusepath{stroke,fill}%
\end{pgfscope}%
\begin{pgfscope}%
\pgfpathrectangle{\pgfqpoint{1.000000in}{0.979904in}}{\pgfqpoint{6.200000in}{5.960192in}}%
\pgfusepath{clip}%
\pgfsetbuttcap%
\pgfsetroundjoin%
\definecolor{currentfill}{rgb}{0.800000,0.200000,0.200000}%
\pgfsetfillcolor{currentfill}%
\pgfsetlinewidth{1.003750pt}%
\definecolor{currentstroke}{rgb}{0.800000,0.200000,0.200000}%
\pgfsetstrokecolor{currentstroke}%
\pgfsetdash{}{0pt}%
\pgfpathmoveto{\pgfqpoint{4.519911in}{2.454024in}}%
\pgfpathcurveto{\pgfqpoint{4.525735in}{2.454024in}}{\pgfqpoint{4.531321in}{2.456338in}}{\pgfqpoint{4.535439in}{2.460456in}}%
\pgfpathcurveto{\pgfqpoint{4.539558in}{2.464574in}}{\pgfqpoint{4.541871in}{2.470160in}}{\pgfqpoint{4.541871in}{2.475984in}}%
\pgfpathcurveto{\pgfqpoint{4.541871in}{2.481808in}}{\pgfqpoint{4.539558in}{2.487394in}}{\pgfqpoint{4.535439in}{2.491512in}}%
\pgfpathcurveto{\pgfqpoint{4.531321in}{2.495631in}}{\pgfqpoint{4.525735in}{2.497944in}}{\pgfqpoint{4.519911in}{2.497944in}}%
\pgfpathcurveto{\pgfqpoint{4.514087in}{2.497944in}}{\pgfqpoint{4.508501in}{2.495631in}}{\pgfqpoint{4.504383in}{2.491512in}}%
\pgfpathcurveto{\pgfqpoint{4.500265in}{2.487394in}}{\pgfqpoint{4.497951in}{2.481808in}}{\pgfqpoint{4.497951in}{2.475984in}}%
\pgfpathcurveto{\pgfqpoint{4.497951in}{2.470160in}}{\pgfqpoint{4.500265in}{2.464574in}}{\pgfqpoint{4.504383in}{2.460456in}}%
\pgfpathcurveto{\pgfqpoint{4.508501in}{2.456338in}}{\pgfqpoint{4.514087in}{2.454024in}}{\pgfqpoint{4.519911in}{2.454024in}}%
\pgfpathlineto{\pgfqpoint{4.519911in}{2.454024in}}%
\pgfpathclose%
\pgfusepath{stroke,fill}%
\end{pgfscope}%
\begin{pgfscope}%
\pgfpathrectangle{\pgfqpoint{1.000000in}{0.979904in}}{\pgfqpoint{6.200000in}{5.960192in}}%
\pgfusepath{clip}%
\pgfsetbuttcap%
\pgfsetroundjoin%
\definecolor{currentfill}{rgb}{0.800000,0.200000,0.200000}%
\pgfsetfillcolor{currentfill}%
\pgfsetlinewidth{1.003750pt}%
\definecolor{currentstroke}{rgb}{0.800000,0.200000,0.200000}%
\pgfsetstrokecolor{currentstroke}%
\pgfsetdash{}{0pt}%
\pgfpathmoveto{\pgfqpoint{4.467076in}{2.570110in}}%
\pgfpathcurveto{\pgfqpoint{4.472900in}{2.570110in}}{\pgfqpoint{4.478486in}{2.572424in}}{\pgfqpoint{4.482604in}{2.576542in}}%
\pgfpathcurveto{\pgfqpoint{4.486722in}{2.580660in}}{\pgfqpoint{4.489036in}{2.586246in}}{\pgfqpoint{4.489036in}{2.592070in}}%
\pgfpathcurveto{\pgfqpoint{4.489036in}{2.597894in}}{\pgfqpoint{4.486722in}{2.603480in}}{\pgfqpoint{4.482604in}{2.607598in}}%
\pgfpathcurveto{\pgfqpoint{4.478486in}{2.611716in}}{\pgfqpoint{4.472900in}{2.614030in}}{\pgfqpoint{4.467076in}{2.614030in}}%
\pgfpathcurveto{\pgfqpoint{4.461252in}{2.614030in}}{\pgfqpoint{4.455666in}{2.611716in}}{\pgfqpoint{4.451548in}{2.607598in}}%
\pgfpathcurveto{\pgfqpoint{4.447430in}{2.603480in}}{\pgfqpoint{4.445116in}{2.597894in}}{\pgfqpoint{4.445116in}{2.592070in}}%
\pgfpathcurveto{\pgfqpoint{4.445116in}{2.586246in}}{\pgfqpoint{4.447430in}{2.580660in}}{\pgfqpoint{4.451548in}{2.576542in}}%
\pgfpathcurveto{\pgfqpoint{4.455666in}{2.572424in}}{\pgfqpoint{4.461252in}{2.570110in}}{\pgfqpoint{4.467076in}{2.570110in}}%
\pgfpathlineto{\pgfqpoint{4.467076in}{2.570110in}}%
\pgfpathclose%
\pgfusepath{stroke,fill}%
\end{pgfscope}%
\begin{pgfscope}%
\pgfpathrectangle{\pgfqpoint{1.000000in}{0.979904in}}{\pgfqpoint{6.200000in}{5.960192in}}%
\pgfusepath{clip}%
\pgfsetbuttcap%
\pgfsetroundjoin%
\definecolor{currentfill}{rgb}{0.800000,0.200000,0.200000}%
\pgfsetfillcolor{currentfill}%
\pgfsetlinewidth{1.003750pt}%
\definecolor{currentstroke}{rgb}{0.800000,0.200000,0.200000}%
\pgfsetstrokecolor{currentstroke}%
\pgfsetdash{}{0pt}%
\pgfpathmoveto{\pgfqpoint{4.560230in}{2.658121in}}%
\pgfpathcurveto{\pgfqpoint{4.566054in}{2.658121in}}{\pgfqpoint{4.571640in}{2.660435in}}{\pgfqpoint{4.575758in}{2.664553in}}%
\pgfpathcurveto{\pgfqpoint{4.579876in}{2.668671in}}{\pgfqpoint{4.582190in}{2.674257in}}{\pgfqpoint{4.582190in}{2.680081in}}%
\pgfpathcurveto{\pgfqpoint{4.582190in}{2.685905in}}{\pgfqpoint{4.579876in}{2.691491in}}{\pgfqpoint{4.575758in}{2.695609in}}%
\pgfpathcurveto{\pgfqpoint{4.571640in}{2.699728in}}{\pgfqpoint{4.566054in}{2.702041in}}{\pgfqpoint{4.560230in}{2.702041in}}%
\pgfpathcurveto{\pgfqpoint{4.554406in}{2.702041in}}{\pgfqpoint{4.548819in}{2.699728in}}{\pgfqpoint{4.544701in}{2.695609in}}%
\pgfpathcurveto{\pgfqpoint{4.540583in}{2.691491in}}{\pgfqpoint{4.538269in}{2.685905in}}{\pgfqpoint{4.538269in}{2.680081in}}%
\pgfpathcurveto{\pgfqpoint{4.538269in}{2.674257in}}{\pgfqpoint{4.540583in}{2.668671in}}{\pgfqpoint{4.544701in}{2.664553in}}%
\pgfpathcurveto{\pgfqpoint{4.548819in}{2.660435in}}{\pgfqpoint{4.554406in}{2.658121in}}{\pgfqpoint{4.560230in}{2.658121in}}%
\pgfpathlineto{\pgfqpoint{4.560230in}{2.658121in}}%
\pgfpathclose%
\pgfusepath{stroke,fill}%
\end{pgfscope}%
\begin{pgfscope}%
\pgfpathrectangle{\pgfqpoint{1.000000in}{0.979904in}}{\pgfqpoint{6.200000in}{5.960192in}}%
\pgfusepath{clip}%
\pgfsetbuttcap%
\pgfsetroundjoin%
\definecolor{currentfill}{rgb}{0.200000,0.800000,0.200000}%
\pgfsetfillcolor{currentfill}%
\pgfsetlinewidth{1.003750pt}%
\definecolor{currentstroke}{rgb}{0.200000,0.800000,0.200000}%
\pgfsetstrokecolor{currentstroke}%
\pgfsetdash{}{0pt}%
\pgfpathmoveto{\pgfqpoint{4.507201in}{2.765751in}}%
\pgfpathcurveto{\pgfqpoint{4.513025in}{2.765751in}}{\pgfqpoint{4.518611in}{2.768065in}}{\pgfqpoint{4.522730in}{2.772183in}}%
\pgfpathcurveto{\pgfqpoint{4.526848in}{2.776301in}}{\pgfqpoint{4.529162in}{2.781887in}}{\pgfqpoint{4.529162in}{2.787711in}}%
\pgfpathcurveto{\pgfqpoint{4.529162in}{2.793535in}}{\pgfqpoint{4.526848in}{2.799121in}}{\pgfqpoint{4.522730in}{2.803240in}}%
\pgfpathcurveto{\pgfqpoint{4.518611in}{2.807358in}}{\pgfqpoint{4.513025in}{2.809672in}}{\pgfqpoint{4.507201in}{2.809672in}}%
\pgfpathcurveto{\pgfqpoint{4.501377in}{2.809672in}}{\pgfqpoint{4.495791in}{2.807358in}}{\pgfqpoint{4.491673in}{2.803240in}}%
\pgfpathcurveto{\pgfqpoint{4.487555in}{2.799121in}}{\pgfqpoint{4.485241in}{2.793535in}}{\pgfqpoint{4.485241in}{2.787711in}}%
\pgfpathcurveto{\pgfqpoint{4.485241in}{2.781887in}}{\pgfqpoint{4.487555in}{2.776301in}}{\pgfqpoint{4.491673in}{2.772183in}}%
\pgfpathcurveto{\pgfqpoint{4.495791in}{2.768065in}}{\pgfqpoint{4.501377in}{2.765751in}}{\pgfqpoint{4.507201in}{2.765751in}}%
\pgfpathlineto{\pgfqpoint{4.507201in}{2.765751in}}%
\pgfpathclose%
\pgfusepath{stroke,fill}%
\end{pgfscope}%
\begin{pgfscope}%
\pgfpathrectangle{\pgfqpoint{1.000000in}{0.979904in}}{\pgfqpoint{6.200000in}{5.960192in}}%
\pgfusepath{clip}%
\pgfsetbuttcap%
\pgfsetroundjoin%
\definecolor{currentfill}{rgb}{0.800000,0.200000,0.200000}%
\pgfsetfillcolor{currentfill}%
\pgfsetlinewidth{1.003750pt}%
\definecolor{currentstroke}{rgb}{0.800000,0.200000,0.200000}%
\pgfsetstrokecolor{currentstroke}%
\pgfsetdash{}{0pt}%
\pgfpathmoveto{\pgfqpoint{4.568379in}{2.865802in}}%
\pgfpathcurveto{\pgfqpoint{4.574203in}{2.865802in}}{\pgfqpoint{4.579789in}{2.868116in}}{\pgfqpoint{4.583907in}{2.872234in}}%
\pgfpathcurveto{\pgfqpoint{4.588025in}{2.876352in}}{\pgfqpoint{4.590339in}{2.881939in}}{\pgfqpoint{4.590339in}{2.887762in}}%
\pgfpathcurveto{\pgfqpoint{4.590339in}{2.893586in}}{\pgfqpoint{4.588025in}{2.899173in}}{\pgfqpoint{4.583907in}{2.903291in}}%
\pgfpathcurveto{\pgfqpoint{4.579789in}{2.907409in}}{\pgfqpoint{4.574203in}{2.909723in}}{\pgfqpoint{4.568379in}{2.909723in}}%
\pgfpathcurveto{\pgfqpoint{4.562555in}{2.909723in}}{\pgfqpoint{4.556969in}{2.907409in}}{\pgfqpoint{4.552851in}{2.903291in}}%
\pgfpathcurveto{\pgfqpoint{4.548732in}{2.899173in}}{\pgfqpoint{4.546419in}{2.893586in}}{\pgfqpoint{4.546419in}{2.887762in}}%
\pgfpathcurveto{\pgfqpoint{4.546419in}{2.881939in}}{\pgfqpoint{4.548732in}{2.876352in}}{\pgfqpoint{4.552851in}{2.872234in}}%
\pgfpathcurveto{\pgfqpoint{4.556969in}{2.868116in}}{\pgfqpoint{4.562555in}{2.865802in}}{\pgfqpoint{4.568379in}{2.865802in}}%
\pgfpathlineto{\pgfqpoint{4.568379in}{2.865802in}}%
\pgfpathclose%
\pgfusepath{stroke,fill}%
\end{pgfscope}%
\begin{pgfscope}%
\pgfpathrectangle{\pgfqpoint{1.000000in}{0.979904in}}{\pgfqpoint{6.200000in}{5.960192in}}%
\pgfusepath{clip}%
\pgfsetbuttcap%
\pgfsetmiterjoin%
\pgfsetlinewidth{1.003750pt}%
\definecolor{currentstroke}{rgb}{0.800000,0.200000,0.200000}%
\pgfsetstrokecolor{currentstroke}%
\pgfsetdash{}{0pt}%
\pgfpathmoveto{\pgfqpoint{2.949351in}{1.277768in}}%
\pgfpathcurveto{\pgfqpoint{3.371917in}{1.277768in}}{\pgfqpoint{3.777233in}{1.445655in}}{\pgfqpoint{4.076032in}{1.744454in}}%
\pgfpathcurveto{\pgfqpoint{4.374831in}{2.043254in}}{\pgfqpoint{4.542719in}{2.448569in}}{\pgfqpoint{4.542719in}{2.871135in}}%
\pgfpathcurveto{\pgfqpoint{4.542719in}{3.293701in}}{\pgfqpoint{4.374831in}{3.699017in}}{\pgfqpoint{4.076032in}{3.997816in}}%
\pgfpathcurveto{\pgfqpoint{3.777233in}{4.296615in}}{\pgfqpoint{3.371917in}{4.464503in}}{\pgfqpoint{2.949351in}{4.464503in}}%
\pgfpathcurveto{\pgfqpoint{2.526785in}{4.464503in}}{\pgfqpoint{2.121470in}{4.296615in}}{\pgfqpoint{1.822670in}{3.997816in}}%
\pgfpathcurveto{\pgfqpoint{1.523871in}{3.699017in}}{\pgfqpoint{1.355984in}{3.293701in}}{\pgfqpoint{1.355984in}{2.871135in}}%
\pgfpathcurveto{\pgfqpoint{1.355984in}{2.448569in}}{\pgfqpoint{1.523871in}{2.043254in}}{\pgfqpoint{1.822670in}{1.744454in}}%
\pgfpathcurveto{\pgfqpoint{2.121470in}{1.445655in}}{\pgfqpoint{2.526785in}{1.277768in}}{\pgfqpoint{2.949351in}{1.277768in}}%
\pgfpathlineto{\pgfqpoint{2.949351in}{1.277768in}}%
\pgfpathclose%
\pgfusepath{stroke}%
\end{pgfscope}%
\begin{pgfscope}%
\pgfpathrectangle{\pgfqpoint{1.000000in}{0.979904in}}{\pgfqpoint{6.200000in}{5.960192in}}%
\pgfusepath{clip}%
\pgfsetbuttcap%
\pgfsetroundjoin%
\definecolor{currentfill}{rgb}{0.000000,0.000000,0.000000}%
\pgfsetfillcolor{currentfill}%
\pgfsetlinewidth{1.003750pt}%
\definecolor{currentstroke}{rgb}{0.000000,0.000000,0.000000}%
\pgfsetstrokecolor{currentstroke}%
\pgfsetdash{}{0pt}%
\pgfsys@defobject{currentmarker}{\pgfqpoint{-0.021960in}{-0.021960in}}{\pgfqpoint{0.021960in}{0.021960in}}{%
\pgfpathmoveto{\pgfqpoint{0.000000in}{-0.021960in}}%
\pgfpathcurveto{\pgfqpoint{0.005824in}{-0.021960in}}{\pgfqpoint{0.011410in}{-0.019646in}}{\pgfqpoint{0.015528in}{-0.015528in}}%
\pgfpathcurveto{\pgfqpoint{0.019646in}{-0.011410in}}{\pgfqpoint{0.021960in}{-0.005824in}}{\pgfqpoint{0.021960in}{0.000000in}}%
\pgfpathcurveto{\pgfqpoint{0.021960in}{0.005824in}}{\pgfqpoint{0.019646in}{0.011410in}}{\pgfqpoint{0.015528in}{0.015528in}}%
\pgfpathcurveto{\pgfqpoint{0.011410in}{0.019646in}}{\pgfqpoint{0.005824in}{0.021960in}}{\pgfqpoint{0.000000in}{0.021960in}}%
\pgfpathcurveto{\pgfqpoint{-0.005824in}{0.021960in}}{\pgfqpoint{-0.011410in}{0.019646in}}{\pgfqpoint{-0.015528in}{0.015528in}}%
\pgfpathcurveto{\pgfqpoint{-0.019646in}{0.011410in}}{\pgfqpoint{-0.021960in}{0.005824in}}{\pgfqpoint{-0.021960in}{0.000000in}}%
\pgfpathcurveto{\pgfqpoint{-0.021960in}{-0.005824in}}{\pgfqpoint{-0.019646in}{-0.011410in}}{\pgfqpoint{-0.015528in}{-0.015528in}}%
\pgfpathcurveto{\pgfqpoint{-0.011410in}{-0.019646in}}{\pgfqpoint{-0.005824in}{-0.021960in}}{\pgfqpoint{0.000000in}{-0.021960in}}%
\pgfpathlineto{\pgfqpoint{0.000000in}{-0.021960in}}%
\pgfpathclose%
\pgfusepath{stroke,fill}%
}%
\begin{pgfscope}%
\pgfsys@transformshift{2.949351in}{2.871135in}%
\pgfsys@useobject{currentmarker}{}%
\end{pgfscope}%
\end{pgfscope}%
\begin{pgfscope}%
\pgfpathrectangle{\pgfqpoint{1.000000in}{0.979904in}}{\pgfqpoint{6.200000in}{5.960192in}}%
\pgfusepath{clip}%
\pgfsetbuttcap%
\pgfsetmiterjoin%
\pgfsetlinewidth{1.003750pt}%
\definecolor{currentstroke}{rgb}{0.200000,0.800000,0.200000}%
\pgfsetstrokecolor{currentstroke}%
\pgfsetdash{}{0pt}%
\pgfpathmoveto{\pgfqpoint{4.972638in}{2.743016in}}%
\pgfpathcurveto{\pgfqpoint{5.435940in}{2.743016in}}{\pgfqpoint{5.880329in}{2.927088in}}{\pgfqpoint{6.207933in}{3.254692in}}%
\pgfpathcurveto{\pgfqpoint{6.535537in}{3.582296in}}{\pgfqpoint{6.719609in}{4.026685in}}{\pgfqpoint{6.719609in}{4.489987in}}%
\pgfpathcurveto{\pgfqpoint{6.719609in}{4.953289in}}{\pgfqpoint{6.535537in}{5.397678in}}{\pgfqpoint{6.207933in}{5.725282in}}%
\pgfpathcurveto{\pgfqpoint{5.880329in}{6.052886in}}{\pgfqpoint{5.435940in}{6.236958in}}{\pgfqpoint{4.972638in}{6.236958in}}%
\pgfpathcurveto{\pgfqpoint{4.509336in}{6.236958in}}{\pgfqpoint{4.064947in}{6.052886in}}{\pgfqpoint{3.737343in}{5.725282in}}%
\pgfpathcurveto{\pgfqpoint{3.409739in}{5.397678in}}{\pgfqpoint{3.225667in}{4.953289in}}{\pgfqpoint{3.225667in}{4.489987in}}%
\pgfpathcurveto{\pgfqpoint{3.225667in}{4.026685in}}{\pgfqpoint{3.409739in}{3.582296in}}{\pgfqpoint{3.737343in}{3.254692in}}%
\pgfpathcurveto{\pgfqpoint{4.064947in}{2.927088in}}{\pgfqpoint{4.509336in}{2.743016in}}{\pgfqpoint{4.972638in}{2.743016in}}%
\pgfpathlineto{\pgfqpoint{4.972638in}{2.743016in}}%
\pgfpathclose%
\pgfusepath{stroke}%
\end{pgfscope}%
\begin{pgfscope}%
\pgfpathrectangle{\pgfqpoint{1.000000in}{0.979904in}}{\pgfqpoint{6.200000in}{5.960192in}}%
\pgfusepath{clip}%
\pgfsetbuttcap%
\pgfsetroundjoin%
\definecolor{currentfill}{rgb}{0.000000,0.000000,0.000000}%
\pgfsetfillcolor{currentfill}%
\pgfsetlinewidth{1.003750pt}%
\definecolor{currentstroke}{rgb}{0.000000,0.000000,0.000000}%
\pgfsetstrokecolor{currentstroke}%
\pgfsetdash{}{0pt}%
\pgfsys@defobject{currentmarker}{\pgfqpoint{-0.021960in}{-0.021960in}}{\pgfqpoint{0.021960in}{0.021960in}}{%
\pgfpathmoveto{\pgfqpoint{0.000000in}{-0.021960in}}%
\pgfpathcurveto{\pgfqpoint{0.005824in}{-0.021960in}}{\pgfqpoint{0.011410in}{-0.019646in}}{\pgfqpoint{0.015528in}{-0.015528in}}%
\pgfpathcurveto{\pgfqpoint{0.019646in}{-0.011410in}}{\pgfqpoint{0.021960in}{-0.005824in}}{\pgfqpoint{0.021960in}{0.000000in}}%
\pgfpathcurveto{\pgfqpoint{0.021960in}{0.005824in}}{\pgfqpoint{0.019646in}{0.011410in}}{\pgfqpoint{0.015528in}{0.015528in}}%
\pgfpathcurveto{\pgfqpoint{0.011410in}{0.019646in}}{\pgfqpoint{0.005824in}{0.021960in}}{\pgfqpoint{0.000000in}{0.021960in}}%
\pgfpathcurveto{\pgfqpoint{-0.005824in}{0.021960in}}{\pgfqpoint{-0.011410in}{0.019646in}}{\pgfqpoint{-0.015528in}{0.015528in}}%
\pgfpathcurveto{\pgfqpoint{-0.019646in}{0.011410in}}{\pgfqpoint{-0.021960in}{0.005824in}}{\pgfqpoint{-0.021960in}{0.000000in}}%
\pgfpathcurveto{\pgfqpoint{-0.021960in}{-0.005824in}}{\pgfqpoint{-0.019646in}{-0.011410in}}{\pgfqpoint{-0.015528in}{-0.015528in}}%
\pgfpathcurveto{\pgfqpoint{-0.011410in}{-0.019646in}}{\pgfqpoint{-0.005824in}{-0.021960in}}{\pgfqpoint{0.000000in}{-0.021960in}}%
\pgfpathlineto{\pgfqpoint{0.000000in}{-0.021960in}}%
\pgfpathclose%
\pgfusepath{stroke,fill}%
}%
\begin{pgfscope}%
\pgfsys@transformshift{4.972638in}{4.489987in}%
\pgfsys@useobject{currentmarker}{}%
\end{pgfscope}%
\end{pgfscope}%
\begin{pgfscope}%
\pgfpathrectangle{\pgfqpoint{1.000000in}{0.979904in}}{\pgfqpoint{6.200000in}{5.960192in}}%
\pgfusepath{clip}%
\pgfsetbuttcap%
\pgfsetmiterjoin%
\pgfsetlinewidth{1.003750pt}%
\definecolor{currentstroke}{rgb}{0.200000,0.200000,0.800000}%
\pgfsetstrokecolor{currentstroke}%
\pgfsetdash{}{0pt}%
\pgfpathmoveto{\pgfqpoint{3.699003in}{4.756614in}}%
\pgfpathcurveto{\pgfqpoint{3.947152in}{4.756614in}}{\pgfqpoint{4.185171in}{4.855204in}}{\pgfqpoint{4.360638in}{5.030672in}}%
\pgfpathcurveto{\pgfqpoint{4.536106in}{5.206140in}}{\pgfqpoint{4.634697in}{5.444159in}}{\pgfqpoint{4.634697in}{5.692307in}}%
\pgfpathcurveto{\pgfqpoint{4.634697in}{5.940456in}}{\pgfqpoint{4.536106in}{6.178475in}}{\pgfqpoint{4.360638in}{6.353943in}}%
\pgfpathcurveto{\pgfqpoint{4.185171in}{6.529411in}}{\pgfqpoint{3.947152in}{6.628001in}}{\pgfqpoint{3.699003in}{6.628001in}}%
\pgfpathcurveto{\pgfqpoint{3.450854in}{6.628001in}}{\pgfqpoint{3.212835in}{6.529411in}}{\pgfqpoint{3.037367in}{6.353943in}}%
\pgfpathcurveto{\pgfqpoint{2.861900in}{6.178475in}}{\pgfqpoint{2.763309in}{5.940456in}}{\pgfqpoint{2.763309in}{5.692307in}}%
\pgfpathcurveto{\pgfqpoint{2.763309in}{5.444159in}}{\pgfqpoint{2.861900in}{5.206140in}}{\pgfqpoint{3.037367in}{5.030672in}}%
\pgfpathcurveto{\pgfqpoint{3.212835in}{4.855204in}}{\pgfqpoint{3.450854in}{4.756614in}}{\pgfqpoint{3.699003in}{4.756614in}}%
\pgfpathlineto{\pgfqpoint{3.699003in}{4.756614in}}%
\pgfpathclose%
\pgfusepath{stroke}%
\end{pgfscope}%
\begin{pgfscope}%
\pgfpathrectangle{\pgfqpoint{1.000000in}{0.979904in}}{\pgfqpoint{6.200000in}{5.960192in}}%
\pgfusepath{clip}%
\pgfsetbuttcap%
\pgfsetroundjoin%
\definecolor{currentfill}{rgb}{0.000000,0.000000,0.000000}%
\pgfsetfillcolor{currentfill}%
\pgfsetlinewidth{1.003750pt}%
\definecolor{currentstroke}{rgb}{0.000000,0.000000,0.000000}%
\pgfsetstrokecolor{currentstroke}%
\pgfsetdash{}{0pt}%
\pgfsys@defobject{currentmarker}{\pgfqpoint{-0.021960in}{-0.021960in}}{\pgfqpoint{0.021960in}{0.021960in}}{%
\pgfpathmoveto{\pgfqpoint{0.000000in}{-0.021960in}}%
\pgfpathcurveto{\pgfqpoint{0.005824in}{-0.021960in}}{\pgfqpoint{0.011410in}{-0.019646in}}{\pgfqpoint{0.015528in}{-0.015528in}}%
\pgfpathcurveto{\pgfqpoint{0.019646in}{-0.011410in}}{\pgfqpoint{0.021960in}{-0.005824in}}{\pgfqpoint{0.021960in}{0.000000in}}%
\pgfpathcurveto{\pgfqpoint{0.021960in}{0.005824in}}{\pgfqpoint{0.019646in}{0.011410in}}{\pgfqpoint{0.015528in}{0.015528in}}%
\pgfpathcurveto{\pgfqpoint{0.011410in}{0.019646in}}{\pgfqpoint{0.005824in}{0.021960in}}{\pgfqpoint{0.000000in}{0.021960in}}%
\pgfpathcurveto{\pgfqpoint{-0.005824in}{0.021960in}}{\pgfqpoint{-0.011410in}{0.019646in}}{\pgfqpoint{-0.015528in}{0.015528in}}%
\pgfpathcurveto{\pgfqpoint{-0.019646in}{0.011410in}}{\pgfqpoint{-0.021960in}{0.005824in}}{\pgfqpoint{-0.021960in}{0.000000in}}%
\pgfpathcurveto{\pgfqpoint{-0.021960in}{-0.005824in}}{\pgfqpoint{-0.019646in}{-0.011410in}}{\pgfqpoint{-0.015528in}{-0.015528in}}%
\pgfpathcurveto{\pgfqpoint{-0.011410in}{-0.019646in}}{\pgfqpoint{-0.005824in}{-0.021960in}}{\pgfqpoint{0.000000in}{-0.021960in}}%
\pgfpathlineto{\pgfqpoint{0.000000in}{-0.021960in}}%
\pgfpathclose%
\pgfusepath{stroke,fill}%
}%
\begin{pgfscope}%
\pgfsys@transformshift{3.699003in}{5.692307in}%
\pgfsys@useobject{currentmarker}{}%
\end{pgfscope}%
\end{pgfscope}%
\begin{pgfscope}%
\pgfsetbuttcap%
\pgfsetroundjoin%
\definecolor{currentfill}{rgb}{0.000000,0.000000,0.000000}%
\pgfsetfillcolor{currentfill}%
\pgfsetlinewidth{0.803000pt}%
\definecolor{currentstroke}{rgb}{0.000000,0.000000,0.000000}%
\pgfsetstrokecolor{currentstroke}%
\pgfsetdash{}{0pt}%
\pgfsys@defobject{currentmarker}{\pgfqpoint{0.000000in}{-0.048611in}}{\pgfqpoint{0.000000in}{0.000000in}}{%
\pgfpathmoveto{\pgfqpoint{0.000000in}{0.000000in}}%
\pgfpathlineto{\pgfqpoint{0.000000in}{-0.048611in}}%
\pgfusepath{stroke,fill}%
}%
\begin{pgfscope}%
\pgfsys@transformshift{1.496745in}{0.979904in}%
\pgfsys@useobject{currentmarker}{}%
\end{pgfscope}%
\end{pgfscope}%
\begin{pgfscope}%
\definecolor{textcolor}{rgb}{0.000000,0.000000,0.000000}%
\pgfsetstrokecolor{textcolor}%
\pgfsetfillcolor{textcolor}%
\pgftext[x=1.496745in,y=0.882682in,,top]{\color{textcolor}{\sffamily\fontsize{10.000000}{12.000000}\selectfont\catcode`\^=\active\def^{\ifmmode\sp\else\^{}\fi}\catcode`\%=\active\def%{\%}\ensuremath{-}150}}%
\end{pgfscope}%
\begin{pgfscope}%
\pgfsetbuttcap%
\pgfsetroundjoin%
\definecolor{currentfill}{rgb}{0.000000,0.000000,0.000000}%
\pgfsetfillcolor{currentfill}%
\pgfsetlinewidth{0.803000pt}%
\definecolor{currentstroke}{rgb}{0.000000,0.000000,0.000000}%
\pgfsetstrokecolor{currentstroke}%
\pgfsetdash{}{0pt}%
\pgfsys@defobject{currentmarker}{\pgfqpoint{0.000000in}{-0.048611in}}{\pgfqpoint{0.000000in}{0.000000in}}{%
\pgfpathmoveto{\pgfqpoint{0.000000in}{0.000000in}}%
\pgfpathlineto{\pgfqpoint{0.000000in}{-0.048611in}}%
\pgfusepath{stroke,fill}%
}%
\begin{pgfscope}%
\pgfsys@transformshift{2.381022in}{0.979904in}%
\pgfsys@useobject{currentmarker}{}%
\end{pgfscope}%
\end{pgfscope}%
\begin{pgfscope}%
\definecolor{textcolor}{rgb}{0.000000,0.000000,0.000000}%
\pgfsetstrokecolor{textcolor}%
\pgfsetfillcolor{textcolor}%
\pgftext[x=2.381022in,y=0.882682in,,top]{\color{textcolor}{\sffamily\fontsize{10.000000}{12.000000}\selectfont\catcode`\^=\active\def^{\ifmmode\sp\else\^{}\fi}\catcode`\%=\active\def%{\%}\ensuremath{-}100}}%
\end{pgfscope}%
\begin{pgfscope}%
\pgfsetbuttcap%
\pgfsetroundjoin%
\definecolor{currentfill}{rgb}{0.000000,0.000000,0.000000}%
\pgfsetfillcolor{currentfill}%
\pgfsetlinewidth{0.803000pt}%
\definecolor{currentstroke}{rgb}{0.000000,0.000000,0.000000}%
\pgfsetstrokecolor{currentstroke}%
\pgfsetdash{}{0pt}%
\pgfsys@defobject{currentmarker}{\pgfqpoint{0.000000in}{-0.048611in}}{\pgfqpoint{0.000000in}{0.000000in}}{%
\pgfpathmoveto{\pgfqpoint{0.000000in}{0.000000in}}%
\pgfpathlineto{\pgfqpoint{0.000000in}{-0.048611in}}%
\pgfusepath{stroke,fill}%
}%
\begin{pgfscope}%
\pgfsys@transformshift{3.265300in}{0.979904in}%
\pgfsys@useobject{currentmarker}{}%
\end{pgfscope}%
\end{pgfscope}%
\begin{pgfscope}%
\definecolor{textcolor}{rgb}{0.000000,0.000000,0.000000}%
\pgfsetstrokecolor{textcolor}%
\pgfsetfillcolor{textcolor}%
\pgftext[x=3.265300in,y=0.882682in,,top]{\color{textcolor}{\sffamily\fontsize{10.000000}{12.000000}\selectfont\catcode`\^=\active\def^{\ifmmode\sp\else\^{}\fi}\catcode`\%=\active\def%{\%}\ensuremath{-}50}}%
\end{pgfscope}%
\begin{pgfscope}%
\pgfsetbuttcap%
\pgfsetroundjoin%
\definecolor{currentfill}{rgb}{0.000000,0.000000,0.000000}%
\pgfsetfillcolor{currentfill}%
\pgfsetlinewidth{0.803000pt}%
\definecolor{currentstroke}{rgb}{0.000000,0.000000,0.000000}%
\pgfsetstrokecolor{currentstroke}%
\pgfsetdash{}{0pt}%
\pgfsys@defobject{currentmarker}{\pgfqpoint{0.000000in}{-0.048611in}}{\pgfqpoint{0.000000in}{0.000000in}}{%
\pgfpathmoveto{\pgfqpoint{0.000000in}{0.000000in}}%
\pgfpathlineto{\pgfqpoint{0.000000in}{-0.048611in}}%
\pgfusepath{stroke,fill}%
}%
\begin{pgfscope}%
\pgfsys@transformshift{4.149578in}{0.979904in}%
\pgfsys@useobject{currentmarker}{}%
\end{pgfscope}%
\end{pgfscope}%
\begin{pgfscope}%
\definecolor{textcolor}{rgb}{0.000000,0.000000,0.000000}%
\pgfsetstrokecolor{textcolor}%
\pgfsetfillcolor{textcolor}%
\pgftext[x=4.149578in,y=0.882682in,,top]{\color{textcolor}{\sffamily\fontsize{10.000000}{12.000000}\selectfont\catcode`\^=\active\def^{\ifmmode\sp\else\^{}\fi}\catcode`\%=\active\def%{\%}0}}%
\end{pgfscope}%
\begin{pgfscope}%
\pgfsetbuttcap%
\pgfsetroundjoin%
\definecolor{currentfill}{rgb}{0.000000,0.000000,0.000000}%
\pgfsetfillcolor{currentfill}%
\pgfsetlinewidth{0.803000pt}%
\definecolor{currentstroke}{rgb}{0.000000,0.000000,0.000000}%
\pgfsetstrokecolor{currentstroke}%
\pgfsetdash{}{0pt}%
\pgfsys@defobject{currentmarker}{\pgfqpoint{0.000000in}{-0.048611in}}{\pgfqpoint{0.000000in}{0.000000in}}{%
\pgfpathmoveto{\pgfqpoint{0.000000in}{0.000000in}}%
\pgfpathlineto{\pgfqpoint{0.000000in}{-0.048611in}}%
\pgfusepath{stroke,fill}%
}%
\begin{pgfscope}%
\pgfsys@transformshift{5.033855in}{0.979904in}%
\pgfsys@useobject{currentmarker}{}%
\end{pgfscope}%
\end{pgfscope}%
\begin{pgfscope}%
\definecolor{textcolor}{rgb}{0.000000,0.000000,0.000000}%
\pgfsetstrokecolor{textcolor}%
\pgfsetfillcolor{textcolor}%
\pgftext[x=5.033855in,y=0.882682in,,top]{\color{textcolor}{\sffamily\fontsize{10.000000}{12.000000}\selectfont\catcode`\^=\active\def^{\ifmmode\sp\else\^{}\fi}\catcode`\%=\active\def%{\%}50}}%
\end{pgfscope}%
\begin{pgfscope}%
\pgfsetbuttcap%
\pgfsetroundjoin%
\definecolor{currentfill}{rgb}{0.000000,0.000000,0.000000}%
\pgfsetfillcolor{currentfill}%
\pgfsetlinewidth{0.803000pt}%
\definecolor{currentstroke}{rgb}{0.000000,0.000000,0.000000}%
\pgfsetstrokecolor{currentstroke}%
\pgfsetdash{}{0pt}%
\pgfsys@defobject{currentmarker}{\pgfqpoint{0.000000in}{-0.048611in}}{\pgfqpoint{0.000000in}{0.000000in}}{%
\pgfpathmoveto{\pgfqpoint{0.000000in}{0.000000in}}%
\pgfpathlineto{\pgfqpoint{0.000000in}{-0.048611in}}%
\pgfusepath{stroke,fill}%
}%
\begin{pgfscope}%
\pgfsys@transformshift{5.918133in}{0.979904in}%
\pgfsys@useobject{currentmarker}{}%
\end{pgfscope}%
\end{pgfscope}%
\begin{pgfscope}%
\definecolor{textcolor}{rgb}{0.000000,0.000000,0.000000}%
\pgfsetstrokecolor{textcolor}%
\pgfsetfillcolor{textcolor}%
\pgftext[x=5.918133in,y=0.882682in,,top]{\color{textcolor}{\sffamily\fontsize{10.000000}{12.000000}\selectfont\catcode`\^=\active\def^{\ifmmode\sp\else\^{}\fi}\catcode`\%=\active\def%{\%}100}}%
\end{pgfscope}%
\begin{pgfscope}%
\pgfsetbuttcap%
\pgfsetroundjoin%
\definecolor{currentfill}{rgb}{0.000000,0.000000,0.000000}%
\pgfsetfillcolor{currentfill}%
\pgfsetlinewidth{0.803000pt}%
\definecolor{currentstroke}{rgb}{0.000000,0.000000,0.000000}%
\pgfsetstrokecolor{currentstroke}%
\pgfsetdash{}{0pt}%
\pgfsys@defobject{currentmarker}{\pgfqpoint{0.000000in}{-0.048611in}}{\pgfqpoint{0.000000in}{0.000000in}}{%
\pgfpathmoveto{\pgfqpoint{0.000000in}{0.000000in}}%
\pgfpathlineto{\pgfqpoint{0.000000in}{-0.048611in}}%
\pgfusepath{stroke,fill}%
}%
\begin{pgfscope}%
\pgfsys@transformshift{6.802411in}{0.979904in}%
\pgfsys@useobject{currentmarker}{}%
\end{pgfscope}%
\end{pgfscope}%
\begin{pgfscope}%
\definecolor{textcolor}{rgb}{0.000000,0.000000,0.000000}%
\pgfsetstrokecolor{textcolor}%
\pgfsetfillcolor{textcolor}%
\pgftext[x=6.802411in,y=0.882682in,,top]{\color{textcolor}{\sffamily\fontsize{10.000000}{12.000000}\selectfont\catcode`\^=\active\def^{\ifmmode\sp\else\^{}\fi}\catcode`\%=\active\def%{\%}150}}%
\end{pgfscope}%
\begin{pgfscope}%
\pgfsetbuttcap%
\pgfsetroundjoin%
\definecolor{currentfill}{rgb}{0.000000,0.000000,0.000000}%
\pgfsetfillcolor{currentfill}%
\pgfsetlinewidth{0.803000pt}%
\definecolor{currentstroke}{rgb}{0.000000,0.000000,0.000000}%
\pgfsetstrokecolor{currentstroke}%
\pgfsetdash{}{0pt}%
\pgfsys@defobject{currentmarker}{\pgfqpoint{-0.048611in}{0.000000in}}{\pgfqpoint{-0.000000in}{0.000000in}}{%
\pgfpathmoveto{\pgfqpoint{-0.000000in}{0.000000in}}%
\pgfpathlineto{\pgfqpoint{-0.048611in}{0.000000in}}%
\pgfusepath{stroke,fill}%
}%
\begin{pgfscope}%
\pgfsys@transformshift{1.000000in}{1.451715in}%
\pgfsys@useobject{currentmarker}{}%
\end{pgfscope}%
\end{pgfscope}%
\begin{pgfscope}%
\definecolor{textcolor}{rgb}{0.000000,0.000000,0.000000}%
\pgfsetstrokecolor{textcolor}%
\pgfsetfillcolor{textcolor}%
\pgftext[x=0.529657in, y=1.398953in, left, base]{\color{textcolor}{\sffamily\fontsize{10.000000}{12.000000}\selectfont\catcode`\^=\active\def^{\ifmmode\sp\else\^{}\fi}\catcode`\%=\active\def%{\%}\ensuremath{-}150}}%
\end{pgfscope}%
\begin{pgfscope}%
\pgfsetbuttcap%
\pgfsetroundjoin%
\definecolor{currentfill}{rgb}{0.000000,0.000000,0.000000}%
\pgfsetfillcolor{currentfill}%
\pgfsetlinewidth{0.803000pt}%
\definecolor{currentstroke}{rgb}{0.000000,0.000000,0.000000}%
\pgfsetstrokecolor{currentstroke}%
\pgfsetdash{}{0pt}%
\pgfsys@defobject{currentmarker}{\pgfqpoint{-0.048611in}{0.000000in}}{\pgfqpoint{-0.000000in}{0.000000in}}{%
\pgfpathmoveto{\pgfqpoint{-0.000000in}{0.000000in}}%
\pgfpathlineto{\pgfqpoint{-0.048611in}{0.000000in}}%
\pgfusepath{stroke,fill}%
}%
\begin{pgfscope}%
\pgfsys@transformshift{1.000000in}{2.335993in}%
\pgfsys@useobject{currentmarker}{}%
\end{pgfscope}%
\end{pgfscope}%
\begin{pgfscope}%
\definecolor{textcolor}{rgb}{0.000000,0.000000,0.000000}%
\pgfsetstrokecolor{textcolor}%
\pgfsetfillcolor{textcolor}%
\pgftext[x=0.529657in, y=2.283231in, left, base]{\color{textcolor}{\sffamily\fontsize{10.000000}{12.000000}\selectfont\catcode`\^=\active\def^{\ifmmode\sp\else\^{}\fi}\catcode`\%=\active\def%{\%}\ensuremath{-}100}}%
\end{pgfscope}%
\begin{pgfscope}%
\pgfsetbuttcap%
\pgfsetroundjoin%
\definecolor{currentfill}{rgb}{0.000000,0.000000,0.000000}%
\pgfsetfillcolor{currentfill}%
\pgfsetlinewidth{0.803000pt}%
\definecolor{currentstroke}{rgb}{0.000000,0.000000,0.000000}%
\pgfsetstrokecolor{currentstroke}%
\pgfsetdash{}{0pt}%
\pgfsys@defobject{currentmarker}{\pgfqpoint{-0.048611in}{0.000000in}}{\pgfqpoint{-0.000000in}{0.000000in}}{%
\pgfpathmoveto{\pgfqpoint{-0.000000in}{0.000000in}}%
\pgfpathlineto{\pgfqpoint{-0.048611in}{0.000000in}}%
\pgfusepath{stroke,fill}%
}%
\begin{pgfscope}%
\pgfsys@transformshift{1.000000in}{3.220270in}%
\pgfsys@useobject{currentmarker}{}%
\end{pgfscope}%
\end{pgfscope}%
\begin{pgfscope}%
\definecolor{textcolor}{rgb}{0.000000,0.000000,0.000000}%
\pgfsetstrokecolor{textcolor}%
\pgfsetfillcolor{textcolor}%
\pgftext[x=0.618022in, y=3.167509in, left, base]{\color{textcolor}{\sffamily\fontsize{10.000000}{12.000000}\selectfont\catcode`\^=\active\def^{\ifmmode\sp\else\^{}\fi}\catcode`\%=\active\def%{\%}\ensuremath{-}50}}%
\end{pgfscope}%
\begin{pgfscope}%
\pgfsetbuttcap%
\pgfsetroundjoin%
\definecolor{currentfill}{rgb}{0.000000,0.000000,0.000000}%
\pgfsetfillcolor{currentfill}%
\pgfsetlinewidth{0.803000pt}%
\definecolor{currentstroke}{rgb}{0.000000,0.000000,0.000000}%
\pgfsetstrokecolor{currentstroke}%
\pgfsetdash{}{0pt}%
\pgfsys@defobject{currentmarker}{\pgfqpoint{-0.048611in}{0.000000in}}{\pgfqpoint{-0.000000in}{0.000000in}}{%
\pgfpathmoveto{\pgfqpoint{-0.000000in}{0.000000in}}%
\pgfpathlineto{\pgfqpoint{-0.048611in}{0.000000in}}%
\pgfusepath{stroke,fill}%
}%
\begin{pgfscope}%
\pgfsys@transformshift{1.000000in}{4.104548in}%
\pgfsys@useobject{currentmarker}{}%
\end{pgfscope}%
\end{pgfscope}%
\begin{pgfscope}%
\definecolor{textcolor}{rgb}{0.000000,0.000000,0.000000}%
\pgfsetstrokecolor{textcolor}%
\pgfsetfillcolor{textcolor}%
\pgftext[x=0.814412in, y=4.051786in, left, base]{\color{textcolor}{\sffamily\fontsize{10.000000}{12.000000}\selectfont\catcode`\^=\active\def^{\ifmmode\sp\else\^{}\fi}\catcode`\%=\active\def%{\%}0}}%
\end{pgfscope}%
\begin{pgfscope}%
\pgfsetbuttcap%
\pgfsetroundjoin%
\definecolor{currentfill}{rgb}{0.000000,0.000000,0.000000}%
\pgfsetfillcolor{currentfill}%
\pgfsetlinewidth{0.803000pt}%
\definecolor{currentstroke}{rgb}{0.000000,0.000000,0.000000}%
\pgfsetstrokecolor{currentstroke}%
\pgfsetdash{}{0pt}%
\pgfsys@defobject{currentmarker}{\pgfqpoint{-0.048611in}{0.000000in}}{\pgfqpoint{-0.000000in}{0.000000in}}{%
\pgfpathmoveto{\pgfqpoint{-0.000000in}{0.000000in}}%
\pgfpathlineto{\pgfqpoint{-0.048611in}{0.000000in}}%
\pgfusepath{stroke,fill}%
}%
\begin{pgfscope}%
\pgfsys@transformshift{1.000000in}{4.988826in}%
\pgfsys@useobject{currentmarker}{}%
\end{pgfscope}%
\end{pgfscope}%
\begin{pgfscope}%
\definecolor{textcolor}{rgb}{0.000000,0.000000,0.000000}%
\pgfsetstrokecolor{textcolor}%
\pgfsetfillcolor{textcolor}%
\pgftext[x=0.726047in, y=4.936064in, left, base]{\color{textcolor}{\sffamily\fontsize{10.000000}{12.000000}\selectfont\catcode`\^=\active\def^{\ifmmode\sp\else\^{}\fi}\catcode`\%=\active\def%{\%}50}}%
\end{pgfscope}%
\begin{pgfscope}%
\pgfsetbuttcap%
\pgfsetroundjoin%
\definecolor{currentfill}{rgb}{0.000000,0.000000,0.000000}%
\pgfsetfillcolor{currentfill}%
\pgfsetlinewidth{0.803000pt}%
\definecolor{currentstroke}{rgb}{0.000000,0.000000,0.000000}%
\pgfsetstrokecolor{currentstroke}%
\pgfsetdash{}{0pt}%
\pgfsys@defobject{currentmarker}{\pgfqpoint{-0.048611in}{0.000000in}}{\pgfqpoint{-0.000000in}{0.000000in}}{%
\pgfpathmoveto{\pgfqpoint{-0.000000in}{0.000000in}}%
\pgfpathlineto{\pgfqpoint{-0.048611in}{0.000000in}}%
\pgfusepath{stroke,fill}%
}%
\begin{pgfscope}%
\pgfsys@transformshift{1.000000in}{5.873103in}%
\pgfsys@useobject{currentmarker}{}%
\end{pgfscope}%
\end{pgfscope}%
\begin{pgfscope}%
\definecolor{textcolor}{rgb}{0.000000,0.000000,0.000000}%
\pgfsetstrokecolor{textcolor}%
\pgfsetfillcolor{textcolor}%
\pgftext[x=0.637682in, y=5.820342in, left, base]{\color{textcolor}{\sffamily\fontsize{10.000000}{12.000000}\selectfont\catcode`\^=\active\def^{\ifmmode\sp\else\^{}\fi}\catcode`\%=\active\def%{\%}100}}%
\end{pgfscope}%
\begin{pgfscope}%
\pgfsetbuttcap%
\pgfsetroundjoin%
\definecolor{currentfill}{rgb}{0.000000,0.000000,0.000000}%
\pgfsetfillcolor{currentfill}%
\pgfsetlinewidth{0.803000pt}%
\definecolor{currentstroke}{rgb}{0.000000,0.000000,0.000000}%
\pgfsetstrokecolor{currentstroke}%
\pgfsetdash{}{0pt}%
\pgfsys@defobject{currentmarker}{\pgfqpoint{-0.048611in}{0.000000in}}{\pgfqpoint{-0.000000in}{0.000000in}}{%
\pgfpathmoveto{\pgfqpoint{-0.000000in}{0.000000in}}%
\pgfpathlineto{\pgfqpoint{-0.048611in}{0.000000in}}%
\pgfusepath{stroke,fill}%
}%
\begin{pgfscope}%
\pgfsys@transformshift{1.000000in}{6.757381in}%
\pgfsys@useobject{currentmarker}{}%
\end{pgfscope}%
\end{pgfscope}%
\begin{pgfscope}%
\definecolor{textcolor}{rgb}{0.000000,0.000000,0.000000}%
\pgfsetstrokecolor{textcolor}%
\pgfsetfillcolor{textcolor}%
\pgftext[x=0.637682in, y=6.704619in, left, base]{\color{textcolor}{\sffamily\fontsize{10.000000}{12.000000}\selectfont\catcode`\^=\active\def^{\ifmmode\sp\else\^{}\fi}\catcode`\%=\active\def%{\%}150}}%
\end{pgfscope}%
\begin{pgfscope}%
\pgfsetrectcap%
\pgfsetmiterjoin%
\pgfsetlinewidth{0.803000pt}%
\definecolor{currentstroke}{rgb}{0.000000,0.000000,0.000000}%
\pgfsetstrokecolor{currentstroke}%
\pgfsetdash{}{0pt}%
\pgfpathmoveto{\pgfqpoint{1.000000in}{0.979904in}}%
\pgfpathlineto{\pgfqpoint{1.000000in}{6.940096in}}%
\pgfusepath{stroke}%
\end{pgfscope}%
\begin{pgfscope}%
\pgfsetrectcap%
\pgfsetmiterjoin%
\pgfsetlinewidth{0.803000pt}%
\definecolor{currentstroke}{rgb}{0.000000,0.000000,0.000000}%
\pgfsetstrokecolor{currentstroke}%
\pgfsetdash{}{0pt}%
\pgfpathmoveto{\pgfqpoint{7.200000in}{0.979904in}}%
\pgfpathlineto{\pgfqpoint{7.200000in}{6.940096in}}%
\pgfusepath{stroke}%
\end{pgfscope}%
\begin{pgfscope}%
\pgfsetrectcap%
\pgfsetmiterjoin%
\pgfsetlinewidth{0.803000pt}%
\definecolor{currentstroke}{rgb}{0.000000,0.000000,0.000000}%
\pgfsetstrokecolor{currentstroke}%
\pgfsetdash{}{0pt}%
\pgfpathmoveto{\pgfqpoint{1.000000in}{0.979904in}}%
\pgfpathlineto{\pgfqpoint{7.200000in}{0.979904in}}%
\pgfusepath{stroke}%
\end{pgfscope}%
\begin{pgfscope}%
\pgfsetrectcap%
\pgfsetmiterjoin%
\pgfsetlinewidth{0.803000pt}%
\definecolor{currentstroke}{rgb}{0.000000,0.000000,0.000000}%
\pgfsetstrokecolor{currentstroke}%
\pgfsetdash{}{0pt}%
\pgfpathmoveto{\pgfqpoint{1.000000in}{6.940096in}}%
\pgfpathlineto{\pgfqpoint{7.200000in}{6.940096in}}%
\pgfusepath{stroke}%
\end{pgfscope}%
\end{pgfpicture}%
\makeatother%
\endgroup%
}
    \label{fig:noisy_rings}
    \caption{Example of a dataset with different rings and noise, and their correct classification.}
\end{figure}


\section{Fuzzy Clustering Algorithms}
In hard clustering algorithms, each data point is assigned to a single cluster. In contrast, fuzzy algorithms assign a membership degree to each data point for each cluster.
One of the most popular ones is the Fuzzy C-Means algorithm. A formal description can be found in \cite{bookpatternrecognition} and \cite{BEZDEK1984191}. It can be seen as an optimization problem,
where the objective function is to minimize the following equation:
\begin{equation}
J(U, V) = \sum_{i=1}^{n} \sum_{j=1}^{k} (u_{ij})^q d_{ij}^2
\end{equation}
where $k$ is the number of clusters, $n$ is the number of data samples, $u_{ij}$ is the membership degree of cluster $i$ to data sample $j$,
$d_{ij}$ is the eucledian distance between data point $i$ and the center of cluster $j$, and $q$ is a parameter that controls the fuzziness of the membership degrees.
That is, higher values of $q$ will make the membership degrees more 'fuzzy', and lower values will make them harder, that is, closer to regular K-Means.
$u$ is a matrix of size $n \times k$, and can be interpreted as 'how much data point $i$ belongs to cluster $j$'.
It is important to note that the following conditions must be met, as described in \cite{BEZDEK1984191}:
\begin{enumerate}
    \item $u_{ij} \in [0, 1]$
    \item $\sum_{j=1}^{k} u_{ij} = 1$
\end{enumerate}
The higher the value of $q$, the more 'fuzzy' the algorithm will be. In the limit, when $q \rightarrow 1$, the algorithm will be equivalent to K-Means.


\section{The Fuzzy K-Rings Algorithm}
The Fuzzy K-Rings algorithm is a clustering algorithm that is able to cluster data points in a ring-shaped dataset.
The algorithm is inspired on the K-Means algorithm, and described in \cite{DAVE1992713} and \cite{308484}, altough in different variations.
It is described as an optimization problem, where the objective function is to minimize the following equation:
\begin{equation}\label{eq:objective}
J_q(U, V) = \sum_{i=1}^{n} \sum_{j=1}^{k} u_{ij}^q (d_{ij} - r_i)^2
\end{equation}
where $k$ is the number of clusters, $n$ is the number of data samples, $u_{ij}$ is the membership degree of cluster $i$ to data sample $j$, $d_{ij}$ is the eucledian distance between data point $i$ and the center of cluster $j$, $r_i$ is the radius of the cluster $i$, and $q$ is a parameter that controls the fuzziness of the membership degrees.
That is, higher values of $q$ will make the membership degrees more fuzzy, and lower values will make them harder.
From now on, we'll refer to $$(d_{ij} - r_i)^2$$ as $d'_ij$.
Now, we'll describe the ways to update the different parameters in the algorithm, and then we'll describe the initialization and convergence criteria, as well as the concrete steps.

\subsection{Updating the Membership Degrees}
The membership degrees are updated using the following equation, as described in both \cite{DAVE1992713} and \cite{308484}. It's the same as the one used in the Fuzzy C-Means algorithm, but with $d_{ij}$ replaced by $d'_{ij}$.
\begin{equation}
u_{ij} = \frac{d'^2(X_j, V_i)^{\frac{-1}{q-1}}}{\sum_{k=1}^{K} d'^2(X_j, V_k)^{\frac{-1}{q-1}}}
\end{equation}

\subsection{Updating the Cluster Radii and Centers}
As mentioned, we can define the algorith as an optimization problem after fixing $U$. We can then obtain the optimal (minimum) values for the objective function
by setting the partial derivatives with respect to $r_i$ and $V_i$ to zero.
First, we have:
\begin{equation}\label{eq:d_dr}
\frac{\partial}{\partial r_i}(J_q) = \sum_{j=1}^{n} u_{ij}^q\frac{\partial}{\partial r_i} (d_{ij} - r_i)^2 = \sum_{j=1}^{n} u_{ij}^q (r_i - d_{ij}) = 0
\end{equation}

Here, we take a different approach to the one taken in \cite{308484}, and similar to the one in \cite{DAVE1992713}. It allows vectorized computations, and
automatic extension to higher dimensions, unlike \cite{308484}.

\begin{tikzpicture}\label{fig:circle}
    % Define radius
    \def\radius{3cm}
    \def\angle{45}

    % Draw the circle
    \draw (0,0) circle (\radius);

    % Define points
    \coordinate (V_i) at (0,0); % Center of the circle
    \coordinate (A) at ({\radius*cos(\angle)},0); % Point on the circle at the specified angle
    \coordinate (V'_i) at (\angle:\radius);
    \coordinate (X_j) at (3.5, 3.5);

    % Draw the triangle
    \draw (V_i) -- (A) -- (V'_i) -- cycle;
    \draw[dotted] (V'_i) -- (X_j);

    % Draw points
    \fill (V_i) circle (2pt);
    \fill (V'_i) circle (2pt);
    \fill (X_j) circle (2pt);

    \node[below] at (V_i) {$V_i$};
    \node[above right] at (V'_i) {$V'_i$};
    \node[above right] at (X_j) {$X_j$};

    % label on hypotenuse (line that connects V_i and V'_i). Not on the point, on the line
    \node[above=2pt] at ($(V_i)!0.5!(V'_i)$) {$r_i$};
    % explanation
    \node[below] at (0,-\radius-0.5) {
        \begin{tabular}{c}
            $V'_i = \frac{r_i}{d_{ij}}X_j + (1 - \frac{r_i}{d_{ij}})V_i$ \\
        \end{tabular}
    };
\end{tikzpicture}
\begin{center}
\textbf{Figure \ref{fig:circle}:} Illustration of the update of the cluster centers equation when $X_j$ is outside the circle.
\end{center}


Let $X_j$ be a data point, and $V_i$ be the center of cluster $i$. Let $d_{ij}$ be the eucledian distance between $X_j$ and $V_i$,
and $r_i$ be the radius of cluster $i$.
Let $d'_{ij}$ be the distance between $X_j$ and the circle with center $V_i$ and radius $r_i$.

Then, let the following be true:
\begin{equation}
V'_i = \frac{r_i}{d_{ij}}X_j + (1 - \frac{r_i}{d_{ij}})V_i
\end{equation}

Differentiating \eqref{eq:objective} with respect to $V_i$ and setting it to zero, we get:
\begin{equation}\label{eq:d_dV}
\frac{\partial}{\partial V_i}(J_q) = \sum_{j=1}^{n} u_{ij}^q\frac{\partial}{\partial V_i} (d'_{ij})^2 = 0
\end{equation}
Note that we can rewrite $(d'_{ij})^2$ as $(X_j - V'_i)^T(X_j - V'_i)$.
Then, we can solve:
\begin{equation}
\begin{aligned}
\frac{\partial}{\partial V_i} (d_{ij} - r_i)^2 &= \left(\frac{\partial}{\partial V_i} (|X_j - V_i| - r_i)^2\right) \\
&= -2 \left( (X_j - V_i) - \frac{r_i}{|X_j - V_i|} (X_j - V_i) \right) \\
&= -2 \left( (X_j - V_i) - \frac{r_i}{d_{ij}} (X_j - V_i) \right) \\
&= -2 \left( (1 - \frac{r_i}{d_{ij}})X_J - (1 - \frac{r_i}{d_{ij}})V_i \right)
\end{aligned}
\end{equation}
Let's define $(1 - \frac{r_i}{d_{ij}})$ as $\alpha_{ij}$.
Then, we can rewrite the equation as:
\begin{equation}
\frac{\partial}{\partial V_i} (d_{ij} - r_i)^2 = -2 \alpha_{ij} (X_j - V_i)
\end{equation}
Plugging that into \eqref{eq:d_dV}, we get:
\begin{equation}
\sum_{j=1}^{n} u_{ij}^q (d_{ij} - r_i) (X_j - V_i) = 0
\end{equation}
We now have a system of equations that we can solve for $V_i$ and $r_i$ to obtain the critical points of the objective function. It is important that since
the equations are coupled, they must be solved together, since it yields better results. \cite{DAVE1992713} mentions (and cites the proof) that, indeed, the critical
points are minima. The experimental results also back this up.
One solution, as noted in \cite{DAVE1992713}, is:
\begin{equation}
V_i = \frac{\sum_{j=1}^{n} u_{ij}^q X_j}{\sum_{j=1}^{n} u_{ij}^q}
\end{equation}
\begin{equation}\label{eq:r_i}
r_i = \frac{\sum_{j=1}^{n} u_{ij}^q d_{ij}}{\sum_{j=1}^{n} u_{ij}^q}
\end{equation}
On the other hand, \cite{308484} proposes a different solution, which is to solve the equations separately. In my experiments, I found that solution not to work very
well in practice, and the one proposed by \cite{DAVE1992713} to work better.
It is also easily checkable that the solution proposed by \cite{DAVE1992713} is a critical point by substituting it into the equations.


\subsection{Intuition}
After having given a formal description of the algorithm, we can give an intuitive explanation of the algorithm.

First, for the membership degrees, we can see that the algorithm is trying to assign higher membership degrees to points that are closer to the ring contour.
This is done by averaging the distances of the points to the different rings, and assigning them proportionally.

As for the cluster centers, the algorithm is just using a weighted average of the points, with the weights being the membership degrees.
This is similar to the K-Means algorithm, but with the weights being the membership degrees instead of a binary value, and exactly the same as the Fuzzy C-Means algorithm.

Finally, for the radii, the algorithm is trying to assign the radii by using a weighted average of the distances of the points to the cluster centers.
This is done by using the membership degrees as weights, and the distances as the values to be averaged.


\subsection{Initializing the Parameters and Nature of the Data}
\cite{308484} proposes two initialization methods, depending on the nature of the data. We adopt both:
\begin{itemize}
    \item Concentric datasets: In this case, since all rings share the same center, but have different radii, the procedure is as follows.
    For the centers, we simply compute the baricenter of the dataset:
    \begin{equation}
        V_i = \frac{1}{n} \sum_{j=1}^{n} X_j
    \end{equation}
    Then, for the radii we define max and min as follows:
    \begin{equation}
        r_{\text{max}} = \max_{j} d(X_j, V_i)
    \end{equation}
    \begin{equation}
        r_{\text{min}} = \min_{j} d(X_j, V_i)
    \end{equation}
    They denote the maximum and minimum distance of a point to the center. Then, we can initialize the radii as sampling from a uniform distribution:
    \begin{equation}
        r_i = r_{\text{min}} + (r_{\text{max}} - r_{\text{min}}) \cdot \text{rand()}
    \end{equation}
    Which will generate a random radius between the minimum and maximum distance of a point to the center.
    \item Non-concentric datasets: In this case, the rings do not share the same center. The rings can also interlock, and their radii are usually different.
    As described in \cite{308484}, we first run the Fuzzy C-Means algorithm on the dataset. After that, we directly use the membership degrees and centers as our initial
    stato. As for the radius, we obtain it with equation \eqref{eq:r_i}.
\end{itemize}

\subsection{Convergence Criterion}
We use the following convergence criterion:
\begin{equation}
|\hat{u_{ij}} - u_{ij}| < \epsilon \quad \forall i, j
\end{equation}
Where $\hat{u_{ij}}$ is the membership degree of the previous iteration, and $u_{ij}$ is the membership degree of the current iteration,
and $\epsilon$ is a small value, usually $10^{-3}$, given as a hyperparameter.
That is, after each update, we check for the difference between the membership degrees of the current and previous iteration, and if the difference is smaller than $\epsilon$,
we break the loop. We do not stop it completely, but we will get to that later.

\subsection{Background noise detection}
In the case of noisy data, the algorithm can be sensitive to noise. To mitigate this, we propose an aditional step.
\begin{figure}[H]
    \centering
    \resizebox{0.9\linewidth}{!}{%% Creator: Matplotlib, PGF backend
%%
%% To include the figure in your LaTeX document, write
%%   \input{<filename>.pgf}
%%
%% Make sure the required packages are loaded in your preamble
%%   \usepackage{pgf}
%%
%% Also ensure that all the required font packages are loaded; for instance,
%% the lmodern package is sometimes necessary when using math font.
%%   \usepackage{lmodern}
%%
%% Figures using additional raster images can only be included by \input if
%% they are in the same directory as the main LaTeX file. For loading figures
%% from other directories you can use the `import` package
%%   \usepackage{import}
%%
%% and then include the figures with
%%   \import{<path to file>}{<filename>.pgf}
%%
%% Matplotlib used the following preamble
%%   \def\mathdefault#1{#1}
%%   \everymath=\expandafter{\the\everymath\displaystyle}
%%   
%%   \usepackage{fontspec}
%%   \setmainfont{DejaVuSerif.ttf}[Path=\detokenize{C:/Users/dagom/anaconda3/envs/pytorch/lib/site-packages/matplotlib/mpl-data/fonts/ttf/}]
%%   \setsansfont{DejaVuSans.ttf}[Path=\detokenize{C:/Users/dagom/anaconda3/envs/pytorch/lib/site-packages/matplotlib/mpl-data/fonts/ttf/}]
%%   \setmonofont{DejaVuSansMono.ttf}[Path=\detokenize{C:/Users/dagom/anaconda3/envs/pytorch/lib/site-packages/matplotlib/mpl-data/fonts/ttf/}]
%%   \makeatletter\@ifpackageloaded{underscore}{}{\usepackage[strings]{underscore}}\makeatother
%%
\begingroup%
\makeatletter%
\begin{pgfpicture}%
\pgfpathrectangle{\pgfpointorigin}{\pgfqpoint{8.000000in}{8.000000in}}%
\pgfusepath{use as bounding box, clip}%
\begin{pgfscope}%
\pgfsetbuttcap%
\pgfsetmiterjoin%
\definecolor{currentfill}{rgb}{1.000000,1.000000,1.000000}%
\pgfsetfillcolor{currentfill}%
\pgfsetlinewidth{0.000000pt}%
\definecolor{currentstroke}{rgb}{1.000000,1.000000,1.000000}%
\pgfsetstrokecolor{currentstroke}%
\pgfsetdash{}{0pt}%
\pgfpathmoveto{\pgfqpoint{0.000000in}{0.000000in}}%
\pgfpathlineto{\pgfqpoint{8.000000in}{0.000000in}}%
\pgfpathlineto{\pgfqpoint{8.000000in}{8.000000in}}%
\pgfpathlineto{\pgfqpoint{0.000000in}{8.000000in}}%
\pgfpathlineto{\pgfqpoint{0.000000in}{0.000000in}}%
\pgfpathclose%
\pgfusepath{fill}%
\end{pgfscope}%
\begin{pgfscope}%
\pgfsetbuttcap%
\pgfsetmiterjoin%
\definecolor{currentfill}{rgb}{1.000000,1.000000,1.000000}%
\pgfsetfillcolor{currentfill}%
\pgfsetlinewidth{0.000000pt}%
\definecolor{currentstroke}{rgb}{0.000000,0.000000,0.000000}%
\pgfsetstrokecolor{currentstroke}%
\pgfsetstrokeopacity{0.000000}%
\pgfsetdash{}{0pt}%
\pgfpathmoveto{\pgfqpoint{1.542338in}{0.880000in}}%
\pgfpathlineto{\pgfqpoint{6.657662in}{0.880000in}}%
\pgfpathlineto{\pgfqpoint{6.657662in}{7.040000in}}%
\pgfpathlineto{\pgfqpoint{1.542338in}{7.040000in}}%
\pgfpathlineto{\pgfqpoint{1.542338in}{0.880000in}}%
\pgfpathclose%
\pgfusepath{fill}%
\end{pgfscope}%
\begin{pgfscope}%
\pgfpathrectangle{\pgfqpoint{1.542338in}{0.880000in}}{\pgfqpoint{5.115323in}{6.160000in}}%
\pgfusepath{clip}%
\pgfsetbuttcap%
\pgfsetroundjoin%
\definecolor{currentfill}{rgb}{0.800000,0.200000,0.200000}%
\pgfsetfillcolor{currentfill}%
\pgfsetlinewidth{1.003750pt}%
\definecolor{currentstroke}{rgb}{0.800000,0.200000,0.200000}%
\pgfsetstrokecolor{currentstroke}%
\pgfsetdash{}{0pt}%
\pgfpathmoveto{\pgfqpoint{6.425147in}{2.153901in}}%
\pgfpathcurveto{\pgfqpoint{6.430971in}{2.153901in}}{\pgfqpoint{6.436557in}{2.156215in}}{\pgfqpoint{6.440675in}{2.160333in}}%
\pgfpathcurveto{\pgfqpoint{6.444793in}{2.164451in}}{\pgfqpoint{6.447107in}{2.170037in}}{\pgfqpoint{6.447107in}{2.175861in}}%
\pgfpathcurveto{\pgfqpoint{6.447107in}{2.181685in}}{\pgfqpoint{6.444793in}{2.187272in}}{\pgfqpoint{6.440675in}{2.191390in}}%
\pgfpathcurveto{\pgfqpoint{6.436557in}{2.195508in}}{\pgfqpoint{6.430971in}{2.197822in}}{\pgfqpoint{6.425147in}{2.197822in}}%
\pgfpathcurveto{\pgfqpoint{6.419323in}{2.197822in}}{\pgfqpoint{6.413737in}{2.195508in}}{\pgfqpoint{6.409619in}{2.191390in}}%
\pgfpathcurveto{\pgfqpoint{6.405501in}{2.187272in}}{\pgfqpoint{6.403187in}{2.181685in}}{\pgfqpoint{6.403187in}{2.175861in}}%
\pgfpathcurveto{\pgfqpoint{6.403187in}{2.170037in}}{\pgfqpoint{6.405501in}{2.164451in}}{\pgfqpoint{6.409619in}{2.160333in}}%
\pgfpathcurveto{\pgfqpoint{6.413737in}{2.156215in}}{\pgfqpoint{6.419323in}{2.153901in}}{\pgfqpoint{6.425147in}{2.153901in}}%
\pgfpathlineto{\pgfqpoint{6.425147in}{2.153901in}}%
\pgfpathclose%
\pgfusepath{stroke,fill}%
\end{pgfscope}%
\begin{pgfscope}%
\pgfpathrectangle{\pgfqpoint{1.542338in}{0.880000in}}{\pgfqpoint{5.115323in}{6.160000in}}%
\pgfusepath{clip}%
\pgfsetbuttcap%
\pgfsetroundjoin%
\definecolor{currentfill}{rgb}{0.800000,0.200000,0.200000}%
\pgfsetfillcolor{currentfill}%
\pgfsetlinewidth{1.003750pt}%
\definecolor{currentstroke}{rgb}{0.800000,0.200000,0.200000}%
\pgfsetstrokecolor{currentstroke}%
\pgfsetdash{}{0pt}%
\pgfpathmoveto{\pgfqpoint{6.413756in}{2.283604in}}%
\pgfpathcurveto{\pgfqpoint{6.419580in}{2.283604in}}{\pgfqpoint{6.425166in}{2.285918in}}{\pgfqpoint{6.429284in}{2.290036in}}%
\pgfpathcurveto{\pgfqpoint{6.433402in}{2.294154in}}{\pgfqpoint{6.435716in}{2.299740in}}{\pgfqpoint{6.435716in}{2.305564in}}%
\pgfpathcurveto{\pgfqpoint{6.435716in}{2.311388in}}{\pgfqpoint{6.433402in}{2.316975in}}{\pgfqpoint{6.429284in}{2.321093in}}%
\pgfpathcurveto{\pgfqpoint{6.425166in}{2.325211in}}{\pgfqpoint{6.419580in}{2.327525in}}{\pgfqpoint{6.413756in}{2.327525in}}%
\pgfpathcurveto{\pgfqpoint{6.407932in}{2.327525in}}{\pgfqpoint{6.402346in}{2.325211in}}{\pgfqpoint{6.398228in}{2.321093in}}%
\pgfpathcurveto{\pgfqpoint{6.394110in}{2.316975in}}{\pgfqpoint{6.391796in}{2.311388in}}{\pgfqpoint{6.391796in}{2.305564in}}%
\pgfpathcurveto{\pgfqpoint{6.391796in}{2.299740in}}{\pgfqpoint{6.394110in}{2.294154in}}{\pgfqpoint{6.398228in}{2.290036in}}%
\pgfpathcurveto{\pgfqpoint{6.402346in}{2.285918in}}{\pgfqpoint{6.407932in}{2.283604in}}{\pgfqpoint{6.413756in}{2.283604in}}%
\pgfpathlineto{\pgfqpoint{6.413756in}{2.283604in}}%
\pgfpathclose%
\pgfusepath{stroke,fill}%
\end{pgfscope}%
\begin{pgfscope}%
\pgfpathrectangle{\pgfqpoint{1.542338in}{0.880000in}}{\pgfqpoint{5.115323in}{6.160000in}}%
\pgfusepath{clip}%
\pgfsetbuttcap%
\pgfsetroundjoin%
\definecolor{currentfill}{rgb}{0.800000,0.200000,0.200000}%
\pgfsetfillcolor{currentfill}%
\pgfsetlinewidth{1.003750pt}%
\definecolor{currentstroke}{rgb}{0.800000,0.200000,0.200000}%
\pgfsetstrokecolor{currentstroke}%
\pgfsetdash{}{0pt}%
\pgfpathmoveto{\pgfqpoint{6.390083in}{2.411485in}}%
\pgfpathcurveto{\pgfqpoint{6.395907in}{2.411485in}}{\pgfqpoint{6.401493in}{2.413799in}}{\pgfqpoint{6.405611in}{2.417917in}}%
\pgfpathcurveto{\pgfqpoint{6.409729in}{2.422035in}}{\pgfqpoint{6.412043in}{2.427621in}}{\pgfqpoint{6.412043in}{2.433445in}}%
\pgfpathcurveto{\pgfqpoint{6.412043in}{2.439269in}}{\pgfqpoint{6.409729in}{2.444855in}}{\pgfqpoint{6.405611in}{2.448973in}}%
\pgfpathcurveto{\pgfqpoint{6.401493in}{2.453091in}}{\pgfqpoint{6.395907in}{2.455405in}}{\pgfqpoint{6.390083in}{2.455405in}}%
\pgfpathcurveto{\pgfqpoint{6.384259in}{2.455405in}}{\pgfqpoint{6.378673in}{2.453091in}}{\pgfqpoint{6.374555in}{2.448973in}}%
\pgfpathcurveto{\pgfqpoint{6.370437in}{2.444855in}}{\pgfqpoint{6.368123in}{2.439269in}}{\pgfqpoint{6.368123in}{2.433445in}}%
\pgfpathcurveto{\pgfqpoint{6.368123in}{2.427621in}}{\pgfqpoint{6.370437in}{2.422035in}}{\pgfqpoint{6.374555in}{2.417917in}}%
\pgfpathcurveto{\pgfqpoint{6.378673in}{2.413799in}}{\pgfqpoint{6.384259in}{2.411485in}}{\pgfqpoint{6.390083in}{2.411485in}}%
\pgfpathlineto{\pgfqpoint{6.390083in}{2.411485in}}%
\pgfpathclose%
\pgfusepath{stroke,fill}%
\end{pgfscope}%
\begin{pgfscope}%
\pgfpathrectangle{\pgfqpoint{1.542338in}{0.880000in}}{\pgfqpoint{5.115323in}{6.160000in}}%
\pgfusepath{clip}%
\pgfsetbuttcap%
\pgfsetroundjoin%
\definecolor{currentfill}{rgb}{0.800000,0.200000,0.200000}%
\pgfsetfillcolor{currentfill}%
\pgfsetlinewidth{1.003750pt}%
\definecolor{currentstroke}{rgb}{0.800000,0.200000,0.200000}%
\pgfsetstrokecolor{currentstroke}%
\pgfsetdash{}{0pt}%
\pgfpathmoveto{\pgfqpoint{6.351541in}{2.535977in}}%
\pgfpathcurveto{\pgfqpoint{6.357364in}{2.535977in}}{\pgfqpoint{6.362951in}{2.538291in}}{\pgfqpoint{6.367069in}{2.542409in}}%
\pgfpathcurveto{\pgfqpoint{6.371187in}{2.546527in}}{\pgfqpoint{6.373501in}{2.552114in}}{\pgfqpoint{6.373501in}{2.557937in}}%
\pgfpathcurveto{\pgfqpoint{6.373501in}{2.563761in}}{\pgfqpoint{6.371187in}{2.569348in}}{\pgfqpoint{6.367069in}{2.573466in}}%
\pgfpathcurveto{\pgfqpoint{6.362951in}{2.577584in}}{\pgfqpoint{6.357364in}{2.579898in}}{\pgfqpoint{6.351541in}{2.579898in}}%
\pgfpathcurveto{\pgfqpoint{6.345717in}{2.579898in}}{\pgfqpoint{6.340130in}{2.577584in}}{\pgfqpoint{6.336012in}{2.573466in}}%
\pgfpathcurveto{\pgfqpoint{6.331894in}{2.569348in}}{\pgfqpoint{6.329580in}{2.563761in}}{\pgfqpoint{6.329580in}{2.557937in}}%
\pgfpathcurveto{\pgfqpoint{6.329580in}{2.552114in}}{\pgfqpoint{6.331894in}{2.546527in}}{\pgfqpoint{6.336012in}{2.542409in}}%
\pgfpathcurveto{\pgfqpoint{6.340130in}{2.538291in}}{\pgfqpoint{6.345717in}{2.535977in}}{\pgfqpoint{6.351541in}{2.535977in}}%
\pgfpathlineto{\pgfqpoint{6.351541in}{2.535977in}}%
\pgfpathclose%
\pgfusepath{stroke,fill}%
\end{pgfscope}%
\begin{pgfscope}%
\pgfpathrectangle{\pgfqpoint{1.542338in}{0.880000in}}{\pgfqpoint{5.115323in}{6.160000in}}%
\pgfusepath{clip}%
\pgfsetbuttcap%
\pgfsetroundjoin%
\definecolor{currentfill}{rgb}{0.800000,0.200000,0.200000}%
\pgfsetfillcolor{currentfill}%
\pgfsetlinewidth{1.003750pt}%
\definecolor{currentstroke}{rgb}{0.800000,0.200000,0.200000}%
\pgfsetstrokecolor{currentstroke}%
\pgfsetdash{}{0pt}%
\pgfpathmoveto{\pgfqpoint{6.295859in}{2.654044in}}%
\pgfpathcurveto{\pgfqpoint{6.301683in}{2.654044in}}{\pgfqpoint{6.307269in}{2.656358in}}{\pgfqpoint{6.311387in}{2.660476in}}%
\pgfpathcurveto{\pgfqpoint{6.315505in}{2.664594in}}{\pgfqpoint{6.317819in}{2.670180in}}{\pgfqpoint{6.317819in}{2.676004in}}%
\pgfpathcurveto{\pgfqpoint{6.317819in}{2.681828in}}{\pgfqpoint{6.315505in}{2.687414in}}{\pgfqpoint{6.311387in}{2.691532in}}%
\pgfpathcurveto{\pgfqpoint{6.307269in}{2.695650in}}{\pgfqpoint{6.301683in}{2.697964in}}{\pgfqpoint{6.295859in}{2.697964in}}%
\pgfpathcurveto{\pgfqpoint{6.290035in}{2.697964in}}{\pgfqpoint{6.284449in}{2.695650in}}{\pgfqpoint{6.280331in}{2.691532in}}%
\pgfpathcurveto{\pgfqpoint{6.276213in}{2.687414in}}{\pgfqpoint{6.273899in}{2.681828in}}{\pgfqpoint{6.273899in}{2.676004in}}%
\pgfpathcurveto{\pgfqpoint{6.273899in}{2.670180in}}{\pgfqpoint{6.276213in}{2.664594in}}{\pgfqpoint{6.280331in}{2.660476in}}%
\pgfpathcurveto{\pgfqpoint{6.284449in}{2.656358in}}{\pgfqpoint{6.290035in}{2.654044in}}{\pgfqpoint{6.295859in}{2.654044in}}%
\pgfpathlineto{\pgfqpoint{6.295859in}{2.654044in}}%
\pgfpathclose%
\pgfusepath{stroke,fill}%
\end{pgfscope}%
\begin{pgfscope}%
\pgfpathrectangle{\pgfqpoint{1.542338in}{0.880000in}}{\pgfqpoint{5.115323in}{6.160000in}}%
\pgfusepath{clip}%
\pgfsetbuttcap%
\pgfsetroundjoin%
\definecolor{currentfill}{rgb}{0.800000,0.200000,0.200000}%
\pgfsetfillcolor{currentfill}%
\pgfsetlinewidth{1.003750pt}%
\definecolor{currentstroke}{rgb}{0.800000,0.200000,0.200000}%
\pgfsetstrokecolor{currentstroke}%
\pgfsetdash{}{0pt}%
\pgfpathmoveto{\pgfqpoint{6.227647in}{2.765765in}}%
\pgfpathcurveto{\pgfqpoint{6.233471in}{2.765765in}}{\pgfqpoint{6.239057in}{2.768079in}}{\pgfqpoint{6.243175in}{2.772197in}}%
\pgfpathcurveto{\pgfqpoint{6.247294in}{2.776315in}}{\pgfqpoint{6.249607in}{2.781901in}}{\pgfqpoint{6.249607in}{2.787725in}}%
\pgfpathcurveto{\pgfqpoint{6.249607in}{2.793549in}}{\pgfqpoint{6.247294in}{2.799135in}}{\pgfqpoint{6.243175in}{2.803254in}}%
\pgfpathcurveto{\pgfqpoint{6.239057in}{2.807372in}}{\pgfqpoint{6.233471in}{2.809686in}}{\pgfqpoint{6.227647in}{2.809686in}}%
\pgfpathcurveto{\pgfqpoint{6.221823in}{2.809686in}}{\pgfqpoint{6.216237in}{2.807372in}}{\pgfqpoint{6.212119in}{2.803254in}}%
\pgfpathcurveto{\pgfqpoint{6.208001in}{2.799135in}}{\pgfqpoint{6.205687in}{2.793549in}}{\pgfqpoint{6.205687in}{2.787725in}}%
\pgfpathcurveto{\pgfqpoint{6.205687in}{2.781901in}}{\pgfqpoint{6.208001in}{2.776315in}}{\pgfqpoint{6.212119in}{2.772197in}}%
\pgfpathcurveto{\pgfqpoint{6.216237in}{2.768079in}}{\pgfqpoint{6.221823in}{2.765765in}}{\pgfqpoint{6.227647in}{2.765765in}}%
\pgfpathlineto{\pgfqpoint{6.227647in}{2.765765in}}%
\pgfpathclose%
\pgfusepath{stroke,fill}%
\end{pgfscope}%
\begin{pgfscope}%
\pgfpathrectangle{\pgfqpoint{1.542338in}{0.880000in}}{\pgfqpoint{5.115323in}{6.160000in}}%
\pgfusepath{clip}%
\pgfsetbuttcap%
\pgfsetroundjoin%
\definecolor{currentfill}{rgb}{0.800000,0.200000,0.200000}%
\pgfsetfillcolor{currentfill}%
\pgfsetlinewidth{1.003750pt}%
\definecolor{currentstroke}{rgb}{0.800000,0.200000,0.200000}%
\pgfsetstrokecolor{currentstroke}%
\pgfsetdash{}{0pt}%
\pgfpathmoveto{\pgfqpoint{6.133648in}{2.856840in}}%
\pgfpathcurveto{\pgfqpoint{6.139472in}{2.856840in}}{\pgfqpoint{6.145058in}{2.859154in}}{\pgfqpoint{6.149176in}{2.863272in}}%
\pgfpathcurveto{\pgfqpoint{6.153294in}{2.867390in}}{\pgfqpoint{6.155608in}{2.872977in}}{\pgfqpoint{6.155608in}{2.878801in}}%
\pgfpathcurveto{\pgfqpoint{6.155608in}{2.884625in}}{\pgfqpoint{6.153294in}{2.890211in}}{\pgfqpoint{6.149176in}{2.894329in}}%
\pgfpathcurveto{\pgfqpoint{6.145058in}{2.898447in}}{\pgfqpoint{6.139472in}{2.900761in}}{\pgfqpoint{6.133648in}{2.900761in}}%
\pgfpathcurveto{\pgfqpoint{6.127824in}{2.900761in}}{\pgfqpoint{6.122238in}{2.898447in}}{\pgfqpoint{6.118119in}{2.894329in}}%
\pgfpathcurveto{\pgfqpoint{6.114001in}{2.890211in}}{\pgfqpoint{6.111687in}{2.884625in}}{\pgfqpoint{6.111687in}{2.878801in}}%
\pgfpathcurveto{\pgfqpoint{6.111687in}{2.872977in}}{\pgfqpoint{6.114001in}{2.867390in}}{\pgfqpoint{6.118119in}{2.863272in}}%
\pgfpathcurveto{\pgfqpoint{6.122238in}{2.859154in}}{\pgfqpoint{6.127824in}{2.856840in}}{\pgfqpoint{6.133648in}{2.856840in}}%
\pgfpathlineto{\pgfqpoint{6.133648in}{2.856840in}}%
\pgfpathclose%
\pgfusepath{stroke,fill}%
\end{pgfscope}%
\begin{pgfscope}%
\pgfpathrectangle{\pgfqpoint{1.542338in}{0.880000in}}{\pgfqpoint{5.115323in}{6.160000in}}%
\pgfusepath{clip}%
\pgfsetbuttcap%
\pgfsetroundjoin%
\definecolor{currentfill}{rgb}{0.800000,0.200000,0.200000}%
\pgfsetfillcolor{currentfill}%
\pgfsetlinewidth{1.003750pt}%
\definecolor{currentstroke}{rgb}{0.800000,0.200000,0.200000}%
\pgfsetstrokecolor{currentstroke}%
\pgfsetdash{}{0pt}%
\pgfpathmoveto{\pgfqpoint{6.042527in}{2.949817in}}%
\pgfpathcurveto{\pgfqpoint{6.048351in}{2.949817in}}{\pgfqpoint{6.053937in}{2.952131in}}{\pgfqpoint{6.058055in}{2.956249in}}%
\pgfpathcurveto{\pgfqpoint{6.062174in}{2.960367in}}{\pgfqpoint{6.064487in}{2.965954in}}{\pgfqpoint{6.064487in}{2.971778in}}%
\pgfpathcurveto{\pgfqpoint{6.064487in}{2.977601in}}{\pgfqpoint{6.062174in}{2.983188in}}{\pgfqpoint{6.058055in}{2.987306in}}%
\pgfpathcurveto{\pgfqpoint{6.053937in}{2.991424in}}{\pgfqpoint{6.048351in}{2.993738in}}{\pgfqpoint{6.042527in}{2.993738in}}%
\pgfpathcurveto{\pgfqpoint{6.036703in}{2.993738in}}{\pgfqpoint{6.031117in}{2.991424in}}{\pgfqpoint{6.026999in}{2.987306in}}%
\pgfpathcurveto{\pgfqpoint{6.022881in}{2.983188in}}{\pgfqpoint{6.020567in}{2.977601in}}{\pgfqpoint{6.020567in}{2.971778in}}%
\pgfpathcurveto{\pgfqpoint{6.020567in}{2.965954in}}{\pgfqpoint{6.022881in}{2.960367in}}{\pgfqpoint{6.026999in}{2.956249in}}%
\pgfpathcurveto{\pgfqpoint{6.031117in}{2.952131in}}{\pgfqpoint{6.036703in}{2.949817in}}{\pgfqpoint{6.042527in}{2.949817in}}%
\pgfpathlineto{\pgfqpoint{6.042527in}{2.949817in}}%
\pgfpathclose%
\pgfusepath{stroke,fill}%
\end{pgfscope}%
\begin{pgfscope}%
\pgfpathrectangle{\pgfqpoint{1.542338in}{0.880000in}}{\pgfqpoint{5.115323in}{6.160000in}}%
\pgfusepath{clip}%
\pgfsetbuttcap%
\pgfsetroundjoin%
\definecolor{currentfill}{rgb}{0.800000,0.200000,0.200000}%
\pgfsetfillcolor{currentfill}%
\pgfsetlinewidth{1.003750pt}%
\definecolor{currentstroke}{rgb}{0.800000,0.200000,0.200000}%
\pgfsetstrokecolor{currentstroke}%
\pgfsetdash{}{0pt}%
\pgfpathmoveto{\pgfqpoint{5.935374in}{3.024181in}}%
\pgfpathcurveto{\pgfqpoint{5.941198in}{3.024181in}}{\pgfqpoint{5.946784in}{3.026495in}}{\pgfqpoint{5.950902in}{3.030613in}}%
\pgfpathcurveto{\pgfqpoint{5.955021in}{3.034731in}}{\pgfqpoint{5.957334in}{3.040318in}}{\pgfqpoint{5.957334in}{3.046142in}}%
\pgfpathcurveto{\pgfqpoint{5.957334in}{3.051965in}}{\pgfqpoint{5.955021in}{3.057552in}}{\pgfqpoint{5.950902in}{3.061670in}}%
\pgfpathcurveto{\pgfqpoint{5.946784in}{3.065788in}}{\pgfqpoint{5.941198in}{3.068102in}}{\pgfqpoint{5.935374in}{3.068102in}}%
\pgfpathcurveto{\pgfqpoint{5.929550in}{3.068102in}}{\pgfqpoint{5.923964in}{3.065788in}}{\pgfqpoint{5.919846in}{3.061670in}}%
\pgfpathcurveto{\pgfqpoint{5.915728in}{3.057552in}}{\pgfqpoint{5.913414in}{3.051965in}}{\pgfqpoint{5.913414in}{3.046142in}}%
\pgfpathcurveto{\pgfqpoint{5.913414in}{3.040318in}}{\pgfqpoint{5.915728in}{3.034731in}}{\pgfqpoint{5.919846in}{3.030613in}}%
\pgfpathcurveto{\pgfqpoint{5.923964in}{3.026495in}}{\pgfqpoint{5.929550in}{3.024181in}}{\pgfqpoint{5.935374in}{3.024181in}}%
\pgfpathlineto{\pgfqpoint{5.935374in}{3.024181in}}%
\pgfpathclose%
\pgfusepath{stroke,fill}%
\end{pgfscope}%
\begin{pgfscope}%
\pgfpathrectangle{\pgfqpoint{1.542338in}{0.880000in}}{\pgfqpoint{5.115323in}{6.160000in}}%
\pgfusepath{clip}%
\pgfsetbuttcap%
\pgfsetroundjoin%
\definecolor{currentfill}{rgb}{0.800000,0.200000,0.200000}%
\pgfsetfillcolor{currentfill}%
\pgfsetlinewidth{1.003750pt}%
\definecolor{currentstroke}{rgb}{0.800000,0.200000,0.200000}%
\pgfsetstrokecolor{currentstroke}%
\pgfsetdash{}{0pt}%
\pgfpathmoveto{\pgfqpoint{5.819344in}{3.083575in}}%
\pgfpathcurveto{\pgfqpoint{5.825168in}{3.083575in}}{\pgfqpoint{5.830754in}{3.085889in}}{\pgfqpoint{5.834872in}{3.090007in}}%
\pgfpathcurveto{\pgfqpoint{5.838991in}{3.094125in}}{\pgfqpoint{5.841304in}{3.099711in}}{\pgfqpoint{5.841304in}{3.105535in}}%
\pgfpathcurveto{\pgfqpoint{5.841304in}{3.111359in}}{\pgfqpoint{5.838991in}{3.116945in}}{\pgfqpoint{5.834872in}{3.121063in}}%
\pgfpathcurveto{\pgfqpoint{5.830754in}{3.125181in}}{\pgfqpoint{5.825168in}{3.127495in}}{\pgfqpoint{5.819344in}{3.127495in}}%
\pgfpathcurveto{\pgfqpoint{5.813520in}{3.127495in}}{\pgfqpoint{5.807934in}{3.125181in}}{\pgfqpoint{5.803816in}{3.121063in}}%
\pgfpathcurveto{\pgfqpoint{5.799698in}{3.116945in}}{\pgfqpoint{5.797384in}{3.111359in}}{\pgfqpoint{5.797384in}{3.105535in}}%
\pgfpathcurveto{\pgfqpoint{5.797384in}{3.099711in}}{\pgfqpoint{5.799698in}{3.094125in}}{\pgfqpoint{5.803816in}{3.090007in}}%
\pgfpathcurveto{\pgfqpoint{5.807934in}{3.085889in}}{\pgfqpoint{5.813520in}{3.083575in}}{\pgfqpoint{5.819344in}{3.083575in}}%
\pgfpathlineto{\pgfqpoint{5.819344in}{3.083575in}}%
\pgfpathclose%
\pgfusepath{stroke,fill}%
\end{pgfscope}%
\begin{pgfscope}%
\pgfpathrectangle{\pgfqpoint{1.542338in}{0.880000in}}{\pgfqpoint{5.115323in}{6.160000in}}%
\pgfusepath{clip}%
\pgfsetbuttcap%
\pgfsetroundjoin%
\definecolor{currentfill}{rgb}{0.800000,0.200000,0.200000}%
\pgfsetfillcolor{currentfill}%
\pgfsetlinewidth{1.003750pt}%
\definecolor{currentstroke}{rgb}{0.800000,0.200000,0.200000}%
\pgfsetstrokecolor{currentstroke}%
\pgfsetdash{}{0pt}%
\pgfpathmoveto{\pgfqpoint{5.697564in}{3.130193in}}%
\pgfpathcurveto{\pgfqpoint{5.703388in}{3.130193in}}{\pgfqpoint{5.708974in}{3.132507in}}{\pgfqpoint{5.713092in}{3.136625in}}%
\pgfpathcurveto{\pgfqpoint{5.717210in}{3.140744in}}{\pgfqpoint{5.719524in}{3.146330in}}{\pgfqpoint{5.719524in}{3.152154in}}%
\pgfpathcurveto{\pgfqpoint{5.719524in}{3.157978in}}{\pgfqpoint{5.717210in}{3.163564in}}{\pgfqpoint{5.713092in}{3.167682in}}%
\pgfpathcurveto{\pgfqpoint{5.708974in}{3.171800in}}{\pgfqpoint{5.703388in}{3.174114in}}{\pgfqpoint{5.697564in}{3.174114in}}%
\pgfpathcurveto{\pgfqpoint{5.691740in}{3.174114in}}{\pgfqpoint{5.686154in}{3.171800in}}{\pgfqpoint{5.682035in}{3.167682in}}%
\pgfpathcurveto{\pgfqpoint{5.677917in}{3.163564in}}{\pgfqpoint{5.675603in}{3.157978in}}{\pgfqpoint{5.675603in}{3.152154in}}%
\pgfpathcurveto{\pgfqpoint{5.675603in}{3.146330in}}{\pgfqpoint{5.677917in}{3.140744in}}{\pgfqpoint{5.682035in}{3.136625in}}%
\pgfpathcurveto{\pgfqpoint{5.686154in}{3.132507in}}{\pgfqpoint{5.691740in}{3.130193in}}{\pgfqpoint{5.697564in}{3.130193in}}%
\pgfpathlineto{\pgfqpoint{5.697564in}{3.130193in}}%
\pgfpathclose%
\pgfusepath{stroke,fill}%
\end{pgfscope}%
\begin{pgfscope}%
\pgfpathrectangle{\pgfqpoint{1.542338in}{0.880000in}}{\pgfqpoint{5.115323in}{6.160000in}}%
\pgfusepath{clip}%
\pgfsetbuttcap%
\pgfsetroundjoin%
\definecolor{currentfill}{rgb}{0.800000,0.200000,0.200000}%
\pgfsetfillcolor{currentfill}%
\pgfsetlinewidth{1.003750pt}%
\definecolor{currentstroke}{rgb}{0.800000,0.200000,0.200000}%
\pgfsetstrokecolor{currentstroke}%
\pgfsetdash{}{0pt}%
\pgfpathmoveto{\pgfqpoint{5.569964in}{3.156913in}}%
\pgfpathcurveto{\pgfqpoint{5.575788in}{3.156913in}}{\pgfqpoint{5.581375in}{3.159227in}}{\pgfqpoint{5.585493in}{3.163345in}}%
\pgfpathcurveto{\pgfqpoint{5.589611in}{3.167463in}}{\pgfqpoint{5.591925in}{3.173050in}}{\pgfqpoint{5.591925in}{3.178874in}}%
\pgfpathcurveto{\pgfqpoint{5.591925in}{3.184697in}}{\pgfqpoint{5.589611in}{3.190284in}}{\pgfqpoint{5.585493in}{3.194402in}}%
\pgfpathcurveto{\pgfqpoint{5.581375in}{3.198520in}}{\pgfqpoint{5.575788in}{3.200834in}}{\pgfqpoint{5.569964in}{3.200834in}}%
\pgfpathcurveto{\pgfqpoint{5.564141in}{3.200834in}}{\pgfqpoint{5.558554in}{3.198520in}}{\pgfqpoint{5.554436in}{3.194402in}}%
\pgfpathcurveto{\pgfqpoint{5.550318in}{3.190284in}}{\pgfqpoint{5.548004in}{3.184697in}}{\pgfqpoint{5.548004in}{3.178874in}}%
\pgfpathcurveto{\pgfqpoint{5.548004in}{3.173050in}}{\pgfqpoint{5.550318in}{3.167463in}}{\pgfqpoint{5.554436in}{3.163345in}}%
\pgfpathcurveto{\pgfqpoint{5.558554in}{3.159227in}}{\pgfqpoint{5.564141in}{3.156913in}}{\pgfqpoint{5.569964in}{3.156913in}}%
\pgfpathlineto{\pgfqpoint{5.569964in}{3.156913in}}%
\pgfpathclose%
\pgfusepath{stroke,fill}%
\end{pgfscope}%
\begin{pgfscope}%
\pgfpathrectangle{\pgfqpoint{1.542338in}{0.880000in}}{\pgfqpoint{5.115323in}{6.160000in}}%
\pgfusepath{clip}%
\pgfsetbuttcap%
\pgfsetroundjoin%
\definecolor{currentfill}{rgb}{0.800000,0.200000,0.200000}%
\pgfsetfillcolor{currentfill}%
\pgfsetlinewidth{1.003750pt}%
\definecolor{currentstroke}{rgb}{0.800000,0.200000,0.200000}%
\pgfsetstrokecolor{currentstroke}%
\pgfsetdash{}{0pt}%
\pgfpathmoveto{\pgfqpoint{5.440272in}{3.166328in}}%
\pgfpathcurveto{\pgfqpoint{5.446096in}{3.166328in}}{\pgfqpoint{5.451682in}{3.168642in}}{\pgfqpoint{5.455800in}{3.172760in}}%
\pgfpathcurveto{\pgfqpoint{5.459918in}{3.176879in}}{\pgfqpoint{5.462232in}{3.182465in}}{\pgfqpoint{5.462232in}{3.188289in}}%
\pgfpathcurveto{\pgfqpoint{5.462232in}{3.194113in}}{\pgfqpoint{5.459918in}{3.199699in}}{\pgfqpoint{5.455800in}{3.203817in}}%
\pgfpathcurveto{\pgfqpoint{5.451682in}{3.207935in}}{\pgfqpoint{5.446096in}{3.210249in}}{\pgfqpoint{5.440272in}{3.210249in}}%
\pgfpathcurveto{\pgfqpoint{5.434448in}{3.210249in}}{\pgfqpoint{5.428862in}{3.207935in}}{\pgfqpoint{5.424744in}{3.203817in}}%
\pgfpathcurveto{\pgfqpoint{5.420625in}{3.199699in}}{\pgfqpoint{5.418312in}{3.194113in}}{\pgfqpoint{5.418312in}{3.188289in}}%
\pgfpathcurveto{\pgfqpoint{5.418312in}{3.182465in}}{\pgfqpoint{5.420625in}{3.176879in}}{\pgfqpoint{5.424744in}{3.172760in}}%
\pgfpathcurveto{\pgfqpoint{5.428862in}{3.168642in}}{\pgfqpoint{5.434448in}{3.166328in}}{\pgfqpoint{5.440272in}{3.166328in}}%
\pgfpathlineto{\pgfqpoint{5.440272in}{3.166328in}}%
\pgfpathclose%
\pgfusepath{stroke,fill}%
\end{pgfscope}%
\begin{pgfscope}%
\pgfpathrectangle{\pgfqpoint{1.542338in}{0.880000in}}{\pgfqpoint{5.115323in}{6.160000in}}%
\pgfusepath{clip}%
\pgfsetbuttcap%
\pgfsetroundjoin%
\definecolor{currentfill}{rgb}{0.800000,0.200000,0.200000}%
\pgfsetfillcolor{currentfill}%
\pgfsetlinewidth{1.003750pt}%
\definecolor{currentstroke}{rgb}{0.800000,0.200000,0.200000}%
\pgfsetstrokecolor{currentstroke}%
\pgfsetdash{}{0pt}%
\pgfpathmoveto{\pgfqpoint{5.310072in}{3.167012in}}%
\pgfpathcurveto{\pgfqpoint{5.315896in}{3.167012in}}{\pgfqpoint{5.321482in}{3.169326in}}{\pgfqpoint{5.325600in}{3.173444in}}%
\pgfpathcurveto{\pgfqpoint{5.329718in}{3.177563in}}{\pgfqpoint{5.332032in}{3.183149in}}{\pgfqpoint{5.332032in}{3.188973in}}%
\pgfpathcurveto{\pgfqpoint{5.332032in}{3.194797in}}{\pgfqpoint{5.329718in}{3.200383in}}{\pgfqpoint{5.325600in}{3.204501in}}%
\pgfpathcurveto{\pgfqpoint{5.321482in}{3.208619in}}{\pgfqpoint{5.315896in}{3.210933in}}{\pgfqpoint{5.310072in}{3.210933in}}%
\pgfpathcurveto{\pgfqpoint{5.304248in}{3.210933in}}{\pgfqpoint{5.298661in}{3.208619in}}{\pgfqpoint{5.294543in}{3.204501in}}%
\pgfpathcurveto{\pgfqpoint{5.290425in}{3.200383in}}{\pgfqpoint{5.288111in}{3.194797in}}{\pgfqpoint{5.288111in}{3.188973in}}%
\pgfpathcurveto{\pgfqpoint{5.288111in}{3.183149in}}{\pgfqpoint{5.290425in}{3.177563in}}{\pgfqpoint{5.294543in}{3.173444in}}%
\pgfpathcurveto{\pgfqpoint{5.298661in}{3.169326in}}{\pgfqpoint{5.304248in}{3.167012in}}{\pgfqpoint{5.310072in}{3.167012in}}%
\pgfpathlineto{\pgfqpoint{5.310072in}{3.167012in}}%
\pgfpathclose%
\pgfusepath{stroke,fill}%
\end{pgfscope}%
\begin{pgfscope}%
\pgfpathrectangle{\pgfqpoint{1.542338in}{0.880000in}}{\pgfqpoint{5.115323in}{6.160000in}}%
\pgfusepath{clip}%
\pgfsetbuttcap%
\pgfsetroundjoin%
\definecolor{currentfill}{rgb}{0.800000,0.200000,0.200000}%
\pgfsetfillcolor{currentfill}%
\pgfsetlinewidth{1.003750pt}%
\definecolor{currentstroke}{rgb}{0.800000,0.200000,0.200000}%
\pgfsetstrokecolor{currentstroke}%
\pgfsetdash{}{0pt}%
\pgfpathmoveto{\pgfqpoint{5.180931in}{3.147902in}}%
\pgfpathcurveto{\pgfqpoint{5.186755in}{3.147902in}}{\pgfqpoint{5.192341in}{3.150216in}}{\pgfqpoint{5.196459in}{3.154334in}}%
\pgfpathcurveto{\pgfqpoint{5.200577in}{3.158452in}}{\pgfqpoint{5.202891in}{3.164039in}}{\pgfqpoint{5.202891in}{3.169863in}}%
\pgfpathcurveto{\pgfqpoint{5.202891in}{3.175686in}}{\pgfqpoint{5.200577in}{3.181273in}}{\pgfqpoint{5.196459in}{3.185391in}}%
\pgfpathcurveto{\pgfqpoint{5.192341in}{3.189509in}}{\pgfqpoint{5.186755in}{3.191823in}}{\pgfqpoint{5.180931in}{3.191823in}}%
\pgfpathcurveto{\pgfqpoint{5.175107in}{3.191823in}}{\pgfqpoint{5.169521in}{3.189509in}}{\pgfqpoint{5.165403in}{3.185391in}}%
\pgfpathcurveto{\pgfqpoint{5.161285in}{3.181273in}}{\pgfqpoint{5.158971in}{3.175686in}}{\pgfqpoint{5.158971in}{3.169863in}}%
\pgfpathcurveto{\pgfqpoint{5.158971in}{3.164039in}}{\pgfqpoint{5.161285in}{3.158452in}}{\pgfqpoint{5.165403in}{3.154334in}}%
\pgfpathcurveto{\pgfqpoint{5.169521in}{3.150216in}}{\pgfqpoint{5.175107in}{3.147902in}}{\pgfqpoint{5.180931in}{3.147902in}}%
\pgfpathlineto{\pgfqpoint{5.180931in}{3.147902in}}%
\pgfpathclose%
\pgfusepath{stroke,fill}%
\end{pgfscope}%
\begin{pgfscope}%
\pgfpathrectangle{\pgfqpoint{1.542338in}{0.880000in}}{\pgfqpoint{5.115323in}{6.160000in}}%
\pgfusepath{clip}%
\pgfsetbuttcap%
\pgfsetroundjoin%
\definecolor{currentfill}{rgb}{0.800000,0.200000,0.200000}%
\pgfsetfillcolor{currentfill}%
\pgfsetlinewidth{1.003750pt}%
\definecolor{currentstroke}{rgb}{0.800000,0.200000,0.200000}%
\pgfsetstrokecolor{currentstroke}%
\pgfsetdash{}{0pt}%
\pgfpathmoveto{\pgfqpoint{5.056649in}{3.108105in}}%
\pgfpathcurveto{\pgfqpoint{5.062473in}{3.108105in}}{\pgfqpoint{5.068059in}{3.110419in}}{\pgfqpoint{5.072178in}{3.114537in}}%
\pgfpathcurveto{\pgfqpoint{5.076296in}{3.118655in}}{\pgfqpoint{5.078610in}{3.124242in}}{\pgfqpoint{5.078610in}{3.130066in}}%
\pgfpathcurveto{\pgfqpoint{5.078610in}{3.135889in}}{\pgfqpoint{5.076296in}{3.141476in}}{\pgfqpoint{5.072178in}{3.145594in}}%
\pgfpathcurveto{\pgfqpoint{5.068059in}{3.149712in}}{\pgfqpoint{5.062473in}{3.152026in}}{\pgfqpoint{5.056649in}{3.152026in}}%
\pgfpathcurveto{\pgfqpoint{5.050825in}{3.152026in}}{\pgfqpoint{5.045239in}{3.149712in}}{\pgfqpoint{5.041121in}{3.145594in}}%
\pgfpathcurveto{\pgfqpoint{5.037003in}{3.141476in}}{\pgfqpoint{5.034689in}{3.135889in}}{\pgfqpoint{5.034689in}{3.130066in}}%
\pgfpathcurveto{\pgfqpoint{5.034689in}{3.124242in}}{\pgfqpoint{5.037003in}{3.118655in}}{\pgfqpoint{5.041121in}{3.114537in}}%
\pgfpathcurveto{\pgfqpoint{5.045239in}{3.110419in}}{\pgfqpoint{5.050825in}{3.108105in}}{\pgfqpoint{5.056649in}{3.108105in}}%
\pgfpathlineto{\pgfqpoint{5.056649in}{3.108105in}}%
\pgfpathclose%
\pgfusepath{stroke,fill}%
\end{pgfscope}%
\begin{pgfscope}%
\pgfpathrectangle{\pgfqpoint{1.542338in}{0.880000in}}{\pgfqpoint{5.115323in}{6.160000in}}%
\pgfusepath{clip}%
\pgfsetbuttcap%
\pgfsetroundjoin%
\definecolor{currentfill}{rgb}{0.800000,0.200000,0.200000}%
\pgfsetfillcolor{currentfill}%
\pgfsetlinewidth{1.003750pt}%
\definecolor{currentstroke}{rgb}{0.800000,0.200000,0.200000}%
\pgfsetstrokecolor{currentstroke}%
\pgfsetdash{}{0pt}%
\pgfpathmoveto{\pgfqpoint{4.935456in}{3.059305in}}%
\pgfpathcurveto{\pgfqpoint{4.941280in}{3.059305in}}{\pgfqpoint{4.946867in}{3.061619in}}{\pgfqpoint{4.950985in}{3.065737in}}%
\pgfpathcurveto{\pgfqpoint{4.955103in}{3.069855in}}{\pgfqpoint{4.957417in}{3.075442in}}{\pgfqpoint{4.957417in}{3.081265in}}%
\pgfpathcurveto{\pgfqpoint{4.957417in}{3.087089in}}{\pgfqpoint{4.955103in}{3.092676in}}{\pgfqpoint{4.950985in}{3.096794in}}%
\pgfpathcurveto{\pgfqpoint{4.946867in}{3.100912in}}{\pgfqpoint{4.941280in}{3.103226in}}{\pgfqpoint{4.935456in}{3.103226in}}%
\pgfpathcurveto{\pgfqpoint{4.929633in}{3.103226in}}{\pgfqpoint{4.924046in}{3.100912in}}{\pgfqpoint{4.919928in}{3.096794in}}%
\pgfpathcurveto{\pgfqpoint{4.915810in}{3.092676in}}{\pgfqpoint{4.913496in}{3.087089in}}{\pgfqpoint{4.913496in}{3.081265in}}%
\pgfpathcurveto{\pgfqpoint{4.913496in}{3.075442in}}{\pgfqpoint{4.915810in}{3.069855in}}{\pgfqpoint{4.919928in}{3.065737in}}%
\pgfpathcurveto{\pgfqpoint{4.924046in}{3.061619in}}{\pgfqpoint{4.929633in}{3.059305in}}{\pgfqpoint{4.935456in}{3.059305in}}%
\pgfpathlineto{\pgfqpoint{4.935456in}{3.059305in}}%
\pgfpathclose%
\pgfusepath{stroke,fill}%
\end{pgfscope}%
\begin{pgfscope}%
\pgfpathrectangle{\pgfqpoint{1.542338in}{0.880000in}}{\pgfqpoint{5.115323in}{6.160000in}}%
\pgfusepath{clip}%
\pgfsetbuttcap%
\pgfsetroundjoin%
\definecolor{currentfill}{rgb}{0.800000,0.200000,0.200000}%
\pgfsetfillcolor{currentfill}%
\pgfsetlinewidth{1.003750pt}%
\definecolor{currentstroke}{rgb}{0.800000,0.200000,0.200000}%
\pgfsetstrokecolor{currentstroke}%
\pgfsetdash{}{0pt}%
\pgfpathmoveto{\pgfqpoint{4.826532in}{2.987200in}}%
\pgfpathcurveto{\pgfqpoint{4.832356in}{2.987200in}}{\pgfqpoint{4.837942in}{2.989514in}}{\pgfqpoint{4.842060in}{2.993632in}}%
\pgfpathcurveto{\pgfqpoint{4.846179in}{2.997750in}}{\pgfqpoint{4.848492in}{3.003336in}}{\pgfqpoint{4.848492in}{3.009160in}}%
\pgfpathcurveto{\pgfqpoint{4.848492in}{3.014984in}}{\pgfqpoint{4.846179in}{3.020570in}}{\pgfqpoint{4.842060in}{3.024688in}}%
\pgfpathcurveto{\pgfqpoint{4.837942in}{3.028806in}}{\pgfqpoint{4.832356in}{3.031120in}}{\pgfqpoint{4.826532in}{3.031120in}}%
\pgfpathcurveto{\pgfqpoint{4.820708in}{3.031120in}}{\pgfqpoint{4.815122in}{3.028806in}}{\pgfqpoint{4.811004in}{3.024688in}}%
\pgfpathcurveto{\pgfqpoint{4.806886in}{3.020570in}}{\pgfqpoint{4.804572in}{3.014984in}}{\pgfqpoint{4.804572in}{3.009160in}}%
\pgfpathcurveto{\pgfqpoint{4.804572in}{3.003336in}}{\pgfqpoint{4.806886in}{2.997750in}}{\pgfqpoint{4.811004in}{2.993632in}}%
\pgfpathcurveto{\pgfqpoint{4.815122in}{2.989514in}}{\pgfqpoint{4.820708in}{2.987200in}}{\pgfqpoint{4.826532in}{2.987200in}}%
\pgfpathlineto{\pgfqpoint{4.826532in}{2.987200in}}%
\pgfpathclose%
\pgfusepath{stroke,fill}%
\end{pgfscope}%
\begin{pgfscope}%
\pgfpathrectangle{\pgfqpoint{1.542338in}{0.880000in}}{\pgfqpoint{5.115323in}{6.160000in}}%
\pgfusepath{clip}%
\pgfsetbuttcap%
\pgfsetroundjoin%
\definecolor{currentfill}{rgb}{0.800000,0.200000,0.200000}%
\pgfsetfillcolor{currentfill}%
\pgfsetlinewidth{1.003750pt}%
\definecolor{currentstroke}{rgb}{0.800000,0.200000,0.200000}%
\pgfsetstrokecolor{currentstroke}%
\pgfsetdash{}{0pt}%
\pgfpathmoveto{\pgfqpoint{4.722861in}{2.908101in}}%
\pgfpathcurveto{\pgfqpoint{4.728685in}{2.908101in}}{\pgfqpoint{4.734271in}{2.910415in}}{\pgfqpoint{4.738389in}{2.914533in}}%
\pgfpathcurveto{\pgfqpoint{4.742507in}{2.918651in}}{\pgfqpoint{4.744821in}{2.924237in}}{\pgfqpoint{4.744821in}{2.930061in}}%
\pgfpathcurveto{\pgfqpoint{4.744821in}{2.935885in}}{\pgfqpoint{4.742507in}{2.941471in}}{\pgfqpoint{4.738389in}{2.945590in}}%
\pgfpathcurveto{\pgfqpoint{4.734271in}{2.949708in}}{\pgfqpoint{4.728685in}{2.952022in}}{\pgfqpoint{4.722861in}{2.952022in}}%
\pgfpathcurveto{\pgfqpoint{4.717037in}{2.952022in}}{\pgfqpoint{4.711451in}{2.949708in}}{\pgfqpoint{4.707333in}{2.945590in}}%
\pgfpathcurveto{\pgfqpoint{4.703214in}{2.941471in}}{\pgfqpoint{4.700901in}{2.935885in}}{\pgfqpoint{4.700901in}{2.930061in}}%
\pgfpathcurveto{\pgfqpoint{4.700901in}{2.924237in}}{\pgfqpoint{4.703214in}{2.918651in}}{\pgfqpoint{4.707333in}{2.914533in}}%
\pgfpathcurveto{\pgfqpoint{4.711451in}{2.910415in}}{\pgfqpoint{4.717037in}{2.908101in}}{\pgfqpoint{4.722861in}{2.908101in}}%
\pgfpathlineto{\pgfqpoint{4.722861in}{2.908101in}}%
\pgfpathclose%
\pgfusepath{stroke,fill}%
\end{pgfscope}%
\begin{pgfscope}%
\pgfpathrectangle{\pgfqpoint{1.542338in}{0.880000in}}{\pgfqpoint{5.115323in}{6.160000in}}%
\pgfusepath{clip}%
\pgfsetbuttcap%
\pgfsetroundjoin%
\definecolor{currentfill}{rgb}{0.800000,0.200000,0.200000}%
\pgfsetfillcolor{currentfill}%
\pgfsetlinewidth{1.003750pt}%
\definecolor{currentstroke}{rgb}{0.800000,0.200000,0.200000}%
\pgfsetstrokecolor{currentstroke}%
\pgfsetdash{}{0pt}%
\pgfpathmoveto{\pgfqpoint{4.638282in}{2.809006in}}%
\pgfpathcurveto{\pgfqpoint{4.644106in}{2.809006in}}{\pgfqpoint{4.649692in}{2.811320in}}{\pgfqpoint{4.653810in}{2.815438in}}%
\pgfpathcurveto{\pgfqpoint{4.657929in}{2.819556in}}{\pgfqpoint{4.660242in}{2.825142in}}{\pgfqpoint{4.660242in}{2.830966in}}%
\pgfpathcurveto{\pgfqpoint{4.660242in}{2.836790in}}{\pgfqpoint{4.657929in}{2.842376in}}{\pgfqpoint{4.653810in}{2.846494in}}%
\pgfpathcurveto{\pgfqpoint{4.649692in}{2.850612in}}{\pgfqpoint{4.644106in}{2.852926in}}{\pgfqpoint{4.638282in}{2.852926in}}%
\pgfpathcurveto{\pgfqpoint{4.632458in}{2.852926in}}{\pgfqpoint{4.626872in}{2.850612in}}{\pgfqpoint{4.622754in}{2.846494in}}%
\pgfpathcurveto{\pgfqpoint{4.618636in}{2.842376in}}{\pgfqpoint{4.616322in}{2.836790in}}{\pgfqpoint{4.616322in}{2.830966in}}%
\pgfpathcurveto{\pgfqpoint{4.616322in}{2.825142in}}{\pgfqpoint{4.618636in}{2.819556in}}{\pgfqpoint{4.622754in}{2.815438in}}%
\pgfpathcurveto{\pgfqpoint{4.626872in}{2.811320in}}{\pgfqpoint{4.632458in}{2.809006in}}{\pgfqpoint{4.638282in}{2.809006in}}%
\pgfpathlineto{\pgfqpoint{4.638282in}{2.809006in}}%
\pgfpathclose%
\pgfusepath{stroke,fill}%
\end{pgfscope}%
\begin{pgfscope}%
\pgfpathrectangle{\pgfqpoint{1.542338in}{0.880000in}}{\pgfqpoint{5.115323in}{6.160000in}}%
\pgfusepath{clip}%
\pgfsetbuttcap%
\pgfsetroundjoin%
\definecolor{currentfill}{rgb}{0.800000,0.200000,0.200000}%
\pgfsetfillcolor{currentfill}%
\pgfsetlinewidth{1.003750pt}%
\definecolor{currentstroke}{rgb}{0.800000,0.200000,0.200000}%
\pgfsetstrokecolor{currentstroke}%
\pgfsetdash{}{0pt}%
\pgfpathmoveto{\pgfqpoint{4.547961in}{2.713598in}}%
\pgfpathcurveto{\pgfqpoint{4.553785in}{2.713598in}}{\pgfqpoint{4.559371in}{2.715912in}}{\pgfqpoint{4.563489in}{2.720030in}}%
\pgfpathcurveto{\pgfqpoint{4.567607in}{2.724148in}}{\pgfqpoint{4.569921in}{2.729734in}}{\pgfqpoint{4.569921in}{2.735558in}}%
\pgfpathcurveto{\pgfqpoint{4.569921in}{2.741382in}}{\pgfqpoint{4.567607in}{2.746968in}}{\pgfqpoint{4.563489in}{2.751086in}}%
\pgfpathcurveto{\pgfqpoint{4.559371in}{2.755204in}}{\pgfqpoint{4.553785in}{2.757518in}}{\pgfqpoint{4.547961in}{2.757518in}}%
\pgfpathcurveto{\pgfqpoint{4.542137in}{2.757518in}}{\pgfqpoint{4.536551in}{2.755204in}}{\pgfqpoint{4.532433in}{2.751086in}}%
\pgfpathcurveto{\pgfqpoint{4.528314in}{2.746968in}}{\pgfqpoint{4.526001in}{2.741382in}}{\pgfqpoint{4.526001in}{2.735558in}}%
\pgfpathcurveto{\pgfqpoint{4.526001in}{2.729734in}}{\pgfqpoint{4.528314in}{2.724148in}}{\pgfqpoint{4.532433in}{2.720030in}}%
\pgfpathcurveto{\pgfqpoint{4.536551in}{2.715912in}}{\pgfqpoint{4.542137in}{2.713598in}}{\pgfqpoint{4.547961in}{2.713598in}}%
\pgfpathlineto{\pgfqpoint{4.547961in}{2.713598in}}%
\pgfpathclose%
\pgfusepath{stroke,fill}%
\end{pgfscope}%
\begin{pgfscope}%
\pgfpathrectangle{\pgfqpoint{1.542338in}{0.880000in}}{\pgfqpoint{5.115323in}{6.160000in}}%
\pgfusepath{clip}%
\pgfsetbuttcap%
\pgfsetroundjoin%
\definecolor{currentfill}{rgb}{0.800000,0.200000,0.200000}%
\pgfsetfillcolor{currentfill}%
\pgfsetlinewidth{1.003750pt}%
\definecolor{currentstroke}{rgb}{0.800000,0.200000,0.200000}%
\pgfsetstrokecolor{currentstroke}%
\pgfsetdash{}{0pt}%
\pgfpathmoveto{\pgfqpoint{4.492381in}{2.594746in}}%
\pgfpathcurveto{\pgfqpoint{4.498205in}{2.594746in}}{\pgfqpoint{4.503792in}{2.597060in}}{\pgfqpoint{4.507910in}{2.601178in}}%
\pgfpathcurveto{\pgfqpoint{4.512028in}{2.605296in}}{\pgfqpoint{4.514342in}{2.610882in}}{\pgfqpoint{4.514342in}{2.616706in}}%
\pgfpathcurveto{\pgfqpoint{4.514342in}{2.622530in}}{\pgfqpoint{4.512028in}{2.628116in}}{\pgfqpoint{4.507910in}{2.632235in}}%
\pgfpathcurveto{\pgfqpoint{4.503792in}{2.636353in}}{\pgfqpoint{4.498205in}{2.638667in}}{\pgfqpoint{4.492381in}{2.638667in}}%
\pgfpathcurveto{\pgfqpoint{4.486557in}{2.638667in}}{\pgfqpoint{4.480971in}{2.636353in}}{\pgfqpoint{4.476853in}{2.632235in}}%
\pgfpathcurveto{\pgfqpoint{4.472735in}{2.628116in}}{\pgfqpoint{4.470421in}{2.622530in}}{\pgfqpoint{4.470421in}{2.616706in}}%
\pgfpathcurveto{\pgfqpoint{4.470421in}{2.610882in}}{\pgfqpoint{4.472735in}{2.605296in}}{\pgfqpoint{4.476853in}{2.601178in}}%
\pgfpathcurveto{\pgfqpoint{4.480971in}{2.597060in}}{\pgfqpoint{4.486557in}{2.594746in}}{\pgfqpoint{4.492381in}{2.594746in}}%
\pgfpathlineto{\pgfqpoint{4.492381in}{2.594746in}}%
\pgfpathclose%
\pgfusepath{stroke,fill}%
\end{pgfscope}%
\begin{pgfscope}%
\pgfpathrectangle{\pgfqpoint{1.542338in}{0.880000in}}{\pgfqpoint{5.115323in}{6.160000in}}%
\pgfusepath{clip}%
\pgfsetbuttcap%
\pgfsetroundjoin%
\definecolor{currentfill}{rgb}{0.800000,0.200000,0.200000}%
\pgfsetfillcolor{currentfill}%
\pgfsetlinewidth{1.003750pt}%
\definecolor{currentstroke}{rgb}{0.800000,0.200000,0.200000}%
\pgfsetstrokecolor{currentstroke}%
\pgfsetdash{}{0pt}%
\pgfpathmoveto{\pgfqpoint{4.444516in}{2.473735in}}%
\pgfpathcurveto{\pgfqpoint{4.450340in}{2.473735in}}{\pgfqpoint{4.455926in}{2.476049in}}{\pgfqpoint{4.460044in}{2.480167in}}%
\pgfpathcurveto{\pgfqpoint{4.464162in}{2.484285in}}{\pgfqpoint{4.466476in}{2.489872in}}{\pgfqpoint{4.466476in}{2.495695in}}%
\pgfpathcurveto{\pgfqpoint{4.466476in}{2.501519in}}{\pgfqpoint{4.464162in}{2.507106in}}{\pgfqpoint{4.460044in}{2.511224in}}%
\pgfpathcurveto{\pgfqpoint{4.455926in}{2.515342in}}{\pgfqpoint{4.450340in}{2.517656in}}{\pgfqpoint{4.444516in}{2.517656in}}%
\pgfpathcurveto{\pgfqpoint{4.438692in}{2.517656in}}{\pgfqpoint{4.433106in}{2.515342in}}{\pgfqpoint{4.428988in}{2.511224in}}%
\pgfpathcurveto{\pgfqpoint{4.424870in}{2.507106in}}{\pgfqpoint{4.422556in}{2.501519in}}{\pgfqpoint{4.422556in}{2.495695in}}%
\pgfpathcurveto{\pgfqpoint{4.422556in}{2.489872in}}{\pgfqpoint{4.424870in}{2.484285in}}{\pgfqpoint{4.428988in}{2.480167in}}%
\pgfpathcurveto{\pgfqpoint{4.433106in}{2.476049in}}{\pgfqpoint{4.438692in}{2.473735in}}{\pgfqpoint{4.444516in}{2.473735in}}%
\pgfpathlineto{\pgfqpoint{4.444516in}{2.473735in}}%
\pgfpathclose%
\pgfusepath{stroke,fill}%
\end{pgfscope}%
\begin{pgfscope}%
\pgfpathrectangle{\pgfqpoint{1.542338in}{0.880000in}}{\pgfqpoint{5.115323in}{6.160000in}}%
\pgfusepath{clip}%
\pgfsetbuttcap%
\pgfsetroundjoin%
\definecolor{currentfill}{rgb}{0.800000,0.200000,0.200000}%
\pgfsetfillcolor{currentfill}%
\pgfsetlinewidth{1.003750pt}%
\definecolor{currentstroke}{rgb}{0.800000,0.200000,0.200000}%
\pgfsetstrokecolor{currentstroke}%
\pgfsetdash{}{0pt}%
\pgfpathmoveto{\pgfqpoint{4.410649in}{2.348097in}}%
\pgfpathcurveto{\pgfqpoint{4.416473in}{2.348097in}}{\pgfqpoint{4.422059in}{2.350411in}}{\pgfqpoint{4.426177in}{2.354529in}}%
\pgfpathcurveto{\pgfqpoint{4.430295in}{2.358648in}}{\pgfqpoint{4.432609in}{2.364234in}}{\pgfqpoint{4.432609in}{2.370058in}}%
\pgfpathcurveto{\pgfqpoint{4.432609in}{2.375882in}}{\pgfqpoint{4.430295in}{2.381468in}}{\pgfqpoint{4.426177in}{2.385586in}}%
\pgfpathcurveto{\pgfqpoint{4.422059in}{2.389704in}}{\pgfqpoint{4.416473in}{2.392018in}}{\pgfqpoint{4.410649in}{2.392018in}}%
\pgfpathcurveto{\pgfqpoint{4.404825in}{2.392018in}}{\pgfqpoint{4.399238in}{2.389704in}}{\pgfqpoint{4.395120in}{2.385586in}}%
\pgfpathcurveto{\pgfqpoint{4.391002in}{2.381468in}}{\pgfqpoint{4.388688in}{2.375882in}}{\pgfqpoint{4.388688in}{2.370058in}}%
\pgfpathcurveto{\pgfqpoint{4.388688in}{2.364234in}}{\pgfqpoint{4.391002in}{2.358648in}}{\pgfqpoint{4.395120in}{2.354529in}}%
\pgfpathcurveto{\pgfqpoint{4.399238in}{2.350411in}}{\pgfqpoint{4.404825in}{2.348097in}}{\pgfqpoint{4.410649in}{2.348097in}}%
\pgfpathlineto{\pgfqpoint{4.410649in}{2.348097in}}%
\pgfpathclose%
\pgfusepath{stroke,fill}%
\end{pgfscope}%
\begin{pgfscope}%
\pgfpathrectangle{\pgfqpoint{1.542338in}{0.880000in}}{\pgfqpoint{5.115323in}{6.160000in}}%
\pgfusepath{clip}%
\pgfsetbuttcap%
\pgfsetroundjoin%
\definecolor{currentfill}{rgb}{0.800000,0.200000,0.200000}%
\pgfsetfillcolor{currentfill}%
\pgfsetlinewidth{1.003750pt}%
\definecolor{currentstroke}{rgb}{0.800000,0.200000,0.200000}%
\pgfsetstrokecolor{currentstroke}%
\pgfsetdash{}{0pt}%
\pgfpathmoveto{\pgfqpoint{4.396738in}{2.218814in}}%
\pgfpathcurveto{\pgfqpoint{4.402562in}{2.218814in}}{\pgfqpoint{4.408148in}{2.221128in}}{\pgfqpoint{4.412267in}{2.225246in}}%
\pgfpathcurveto{\pgfqpoint{4.416385in}{2.229364in}}{\pgfqpoint{4.418699in}{2.234950in}}{\pgfqpoint{4.418699in}{2.240774in}}%
\pgfpathcurveto{\pgfqpoint{4.418699in}{2.246598in}}{\pgfqpoint{4.416385in}{2.252184in}}{\pgfqpoint{4.412267in}{2.256302in}}%
\pgfpathcurveto{\pgfqpoint{4.408148in}{2.260420in}}{\pgfqpoint{4.402562in}{2.262734in}}{\pgfqpoint{4.396738in}{2.262734in}}%
\pgfpathcurveto{\pgfqpoint{4.390914in}{2.262734in}}{\pgfqpoint{4.385328in}{2.260420in}}{\pgfqpoint{4.381210in}{2.256302in}}%
\pgfpathcurveto{\pgfqpoint{4.377092in}{2.252184in}}{\pgfqpoint{4.374778in}{2.246598in}}{\pgfqpoint{4.374778in}{2.240774in}}%
\pgfpathcurveto{\pgfqpoint{4.374778in}{2.234950in}}{\pgfqpoint{4.377092in}{2.229364in}}{\pgfqpoint{4.381210in}{2.225246in}}%
\pgfpathcurveto{\pgfqpoint{4.385328in}{2.221128in}}{\pgfqpoint{4.390914in}{2.218814in}}{\pgfqpoint{4.396738in}{2.218814in}}%
\pgfpathlineto{\pgfqpoint{4.396738in}{2.218814in}}%
\pgfpathclose%
\pgfusepath{stroke,fill}%
\end{pgfscope}%
\begin{pgfscope}%
\pgfpathrectangle{\pgfqpoint{1.542338in}{0.880000in}}{\pgfqpoint{5.115323in}{6.160000in}}%
\pgfusepath{clip}%
\pgfsetbuttcap%
\pgfsetroundjoin%
\definecolor{currentfill}{rgb}{0.800000,0.200000,0.200000}%
\pgfsetfillcolor{currentfill}%
\pgfsetlinewidth{1.003750pt}%
\definecolor{currentstroke}{rgb}{0.800000,0.200000,0.200000}%
\pgfsetstrokecolor{currentstroke}%
\pgfsetdash{}{0pt}%
\pgfpathmoveto{\pgfqpoint{4.397619in}{2.089045in}}%
\pgfpathcurveto{\pgfqpoint{4.403443in}{2.089045in}}{\pgfqpoint{4.409029in}{2.091359in}}{\pgfqpoint{4.413147in}{2.095477in}}%
\pgfpathcurveto{\pgfqpoint{4.417265in}{2.099595in}}{\pgfqpoint{4.419579in}{2.105181in}}{\pgfqpoint{4.419579in}{2.111005in}}%
\pgfpathcurveto{\pgfqpoint{4.419579in}{2.116829in}}{\pgfqpoint{4.417265in}{2.122415in}}{\pgfqpoint{4.413147in}{2.126534in}}%
\pgfpathcurveto{\pgfqpoint{4.409029in}{2.130652in}}{\pgfqpoint{4.403443in}{2.132966in}}{\pgfqpoint{4.397619in}{2.132966in}}%
\pgfpathcurveto{\pgfqpoint{4.391795in}{2.132966in}}{\pgfqpoint{4.386208in}{2.130652in}}{\pgfqpoint{4.382090in}{2.126534in}}%
\pgfpathcurveto{\pgfqpoint{4.377972in}{2.122415in}}{\pgfqpoint{4.375658in}{2.116829in}}{\pgfqpoint{4.375658in}{2.111005in}}%
\pgfpathcurveto{\pgfqpoint{4.375658in}{2.105181in}}{\pgfqpoint{4.377972in}{2.099595in}}{\pgfqpoint{4.382090in}{2.095477in}}%
\pgfpathcurveto{\pgfqpoint{4.386208in}{2.091359in}}{\pgfqpoint{4.391795in}{2.089045in}}{\pgfqpoint{4.397619in}{2.089045in}}%
\pgfpathlineto{\pgfqpoint{4.397619in}{2.089045in}}%
\pgfpathclose%
\pgfusepath{stroke,fill}%
\end{pgfscope}%
\begin{pgfscope}%
\pgfpathrectangle{\pgfqpoint{1.542338in}{0.880000in}}{\pgfqpoint{5.115323in}{6.160000in}}%
\pgfusepath{clip}%
\pgfsetbuttcap%
\pgfsetroundjoin%
\definecolor{currentfill}{rgb}{0.800000,0.200000,0.200000}%
\pgfsetfillcolor{currentfill}%
\pgfsetlinewidth{1.003750pt}%
\definecolor{currentstroke}{rgb}{0.800000,0.200000,0.200000}%
\pgfsetstrokecolor{currentstroke}%
\pgfsetdash{}{0pt}%
\pgfpathmoveto{\pgfqpoint{4.409440in}{1.959469in}}%
\pgfpathcurveto{\pgfqpoint{4.415264in}{1.959469in}}{\pgfqpoint{4.420850in}{1.961783in}}{\pgfqpoint{4.424968in}{1.965901in}}%
\pgfpathcurveto{\pgfqpoint{4.429086in}{1.970020in}}{\pgfqpoint{4.431400in}{1.975606in}}{\pgfqpoint{4.431400in}{1.981430in}}%
\pgfpathcurveto{\pgfqpoint{4.431400in}{1.987254in}}{\pgfqpoint{4.429086in}{1.992840in}}{\pgfqpoint{4.424968in}{1.996958in}}%
\pgfpathcurveto{\pgfqpoint{4.420850in}{2.001076in}}{\pgfqpoint{4.415264in}{2.003390in}}{\pgfqpoint{4.409440in}{2.003390in}}%
\pgfpathcurveto{\pgfqpoint{4.403616in}{2.003390in}}{\pgfqpoint{4.398030in}{2.001076in}}{\pgfqpoint{4.393911in}{1.996958in}}%
\pgfpathcurveto{\pgfqpoint{4.389793in}{1.992840in}}{\pgfqpoint{4.387479in}{1.987254in}}{\pgfqpoint{4.387479in}{1.981430in}}%
\pgfpathcurveto{\pgfqpoint{4.387479in}{1.975606in}}{\pgfqpoint{4.389793in}{1.970020in}}{\pgfqpoint{4.393911in}{1.965901in}}%
\pgfpathcurveto{\pgfqpoint{4.398030in}{1.961783in}}{\pgfqpoint{4.403616in}{1.959469in}}{\pgfqpoint{4.409440in}{1.959469in}}%
\pgfpathlineto{\pgfqpoint{4.409440in}{1.959469in}}%
\pgfpathclose%
\pgfusepath{stroke,fill}%
\end{pgfscope}%
\begin{pgfscope}%
\pgfpathrectangle{\pgfqpoint{1.542338in}{0.880000in}}{\pgfqpoint{5.115323in}{6.160000in}}%
\pgfusepath{clip}%
\pgfsetbuttcap%
\pgfsetroundjoin%
\definecolor{currentfill}{rgb}{0.800000,0.200000,0.200000}%
\pgfsetfillcolor{currentfill}%
\pgfsetlinewidth{1.003750pt}%
\definecolor{currentstroke}{rgb}{0.800000,0.200000,0.200000}%
\pgfsetstrokecolor{currentstroke}%
\pgfsetdash{}{0pt}%
\pgfpathmoveto{\pgfqpoint{4.442221in}{1.833305in}}%
\pgfpathcurveto{\pgfqpoint{4.448045in}{1.833305in}}{\pgfqpoint{4.453631in}{1.835619in}}{\pgfqpoint{4.457749in}{1.839737in}}%
\pgfpathcurveto{\pgfqpoint{4.461868in}{1.843855in}}{\pgfqpoint{4.464181in}{1.849442in}}{\pgfqpoint{4.464181in}{1.855265in}}%
\pgfpathcurveto{\pgfqpoint{4.464181in}{1.861089in}}{\pgfqpoint{4.461868in}{1.866676in}}{\pgfqpoint{4.457749in}{1.870794in}}%
\pgfpathcurveto{\pgfqpoint{4.453631in}{1.874912in}}{\pgfqpoint{4.448045in}{1.877226in}}{\pgfqpoint{4.442221in}{1.877226in}}%
\pgfpathcurveto{\pgfqpoint{4.436397in}{1.877226in}}{\pgfqpoint{4.430811in}{1.874912in}}{\pgfqpoint{4.426693in}{1.870794in}}%
\pgfpathcurveto{\pgfqpoint{4.422575in}{1.866676in}}{\pgfqpoint{4.420261in}{1.861089in}}{\pgfqpoint{4.420261in}{1.855265in}}%
\pgfpathcurveto{\pgfqpoint{4.420261in}{1.849442in}}{\pgfqpoint{4.422575in}{1.843855in}}{\pgfqpoint{4.426693in}{1.839737in}}%
\pgfpathcurveto{\pgfqpoint{4.430811in}{1.835619in}}{\pgfqpoint{4.436397in}{1.833305in}}{\pgfqpoint{4.442221in}{1.833305in}}%
\pgfpathlineto{\pgfqpoint{4.442221in}{1.833305in}}%
\pgfpathclose%
\pgfusepath{stroke,fill}%
\end{pgfscope}%
\begin{pgfscope}%
\pgfpathrectangle{\pgfqpoint{1.542338in}{0.880000in}}{\pgfqpoint{5.115323in}{6.160000in}}%
\pgfusepath{clip}%
\pgfsetbuttcap%
\pgfsetroundjoin%
\definecolor{currentfill}{rgb}{0.800000,0.200000,0.200000}%
\pgfsetfillcolor{currentfill}%
\pgfsetlinewidth{1.003750pt}%
\definecolor{currentstroke}{rgb}{0.800000,0.200000,0.200000}%
\pgfsetstrokecolor{currentstroke}%
\pgfsetdash{}{0pt}%
\pgfpathmoveto{\pgfqpoint{4.490749in}{1.712270in}}%
\pgfpathcurveto{\pgfqpoint{4.496573in}{1.712270in}}{\pgfqpoint{4.502159in}{1.714584in}}{\pgfqpoint{4.506277in}{1.718702in}}%
\pgfpathcurveto{\pgfqpoint{4.510395in}{1.722820in}}{\pgfqpoint{4.512709in}{1.728406in}}{\pgfqpoint{4.512709in}{1.734230in}}%
\pgfpathcurveto{\pgfqpoint{4.512709in}{1.740054in}}{\pgfqpoint{4.510395in}{1.745640in}}{\pgfqpoint{4.506277in}{1.749758in}}%
\pgfpathcurveto{\pgfqpoint{4.502159in}{1.753877in}}{\pgfqpoint{4.496573in}{1.756190in}}{\pgfqpoint{4.490749in}{1.756190in}}%
\pgfpathcurveto{\pgfqpoint{4.484925in}{1.756190in}}{\pgfqpoint{4.479338in}{1.753877in}}{\pgfqpoint{4.475220in}{1.749758in}}%
\pgfpathcurveto{\pgfqpoint{4.471102in}{1.745640in}}{\pgfqpoint{4.468788in}{1.740054in}}{\pgfqpoint{4.468788in}{1.734230in}}%
\pgfpathcurveto{\pgfqpoint{4.468788in}{1.728406in}}{\pgfqpoint{4.471102in}{1.722820in}}{\pgfqpoint{4.475220in}{1.718702in}}%
\pgfpathcurveto{\pgfqpoint{4.479338in}{1.714584in}}{\pgfqpoint{4.484925in}{1.712270in}}{\pgfqpoint{4.490749in}{1.712270in}}%
\pgfpathlineto{\pgfqpoint{4.490749in}{1.712270in}}%
\pgfpathclose%
\pgfusepath{stroke,fill}%
\end{pgfscope}%
\begin{pgfscope}%
\pgfpathrectangle{\pgfqpoint{1.542338in}{0.880000in}}{\pgfqpoint{5.115323in}{6.160000in}}%
\pgfusepath{clip}%
\pgfsetbuttcap%
\pgfsetroundjoin%
\definecolor{currentfill}{rgb}{0.800000,0.200000,0.200000}%
\pgfsetfillcolor{currentfill}%
\pgfsetlinewidth{1.003750pt}%
\definecolor{currentstroke}{rgb}{0.800000,0.200000,0.200000}%
\pgfsetstrokecolor{currentstroke}%
\pgfsetdash{}{0pt}%
\pgfpathmoveto{\pgfqpoint{4.555620in}{1.599190in}}%
\pgfpathcurveto{\pgfqpoint{4.561443in}{1.599190in}}{\pgfqpoint{4.567030in}{1.601504in}}{\pgfqpoint{4.571148in}{1.605622in}}%
\pgfpathcurveto{\pgfqpoint{4.575266in}{1.609740in}}{\pgfqpoint{4.577580in}{1.615326in}}{\pgfqpoint{4.577580in}{1.621150in}}%
\pgfpathcurveto{\pgfqpoint{4.577580in}{1.626974in}}{\pgfqpoint{4.575266in}{1.632560in}}{\pgfqpoint{4.571148in}{1.636678in}}%
\pgfpathcurveto{\pgfqpoint{4.567030in}{1.640796in}}{\pgfqpoint{4.561443in}{1.643110in}}{\pgfqpoint{4.555620in}{1.643110in}}%
\pgfpathcurveto{\pgfqpoint{4.549796in}{1.643110in}}{\pgfqpoint{4.544209in}{1.640796in}}{\pgfqpoint{4.540091in}{1.636678in}}%
\pgfpathcurveto{\pgfqpoint{4.535973in}{1.632560in}}{\pgfqpoint{4.533659in}{1.626974in}}{\pgfqpoint{4.533659in}{1.621150in}}%
\pgfpathcurveto{\pgfqpoint{4.533659in}{1.615326in}}{\pgfqpoint{4.535973in}{1.609740in}}{\pgfqpoint{4.540091in}{1.605622in}}%
\pgfpathcurveto{\pgfqpoint{4.544209in}{1.601504in}}{\pgfqpoint{4.549796in}{1.599190in}}{\pgfqpoint{4.555620in}{1.599190in}}%
\pgfpathlineto{\pgfqpoint{4.555620in}{1.599190in}}%
\pgfpathclose%
\pgfusepath{stroke,fill}%
\end{pgfscope}%
\begin{pgfscope}%
\pgfpathrectangle{\pgfqpoint{1.542338in}{0.880000in}}{\pgfqpoint{5.115323in}{6.160000in}}%
\pgfusepath{clip}%
\pgfsetbuttcap%
\pgfsetroundjoin%
\definecolor{currentfill}{rgb}{0.800000,0.200000,0.200000}%
\pgfsetfillcolor{currentfill}%
\pgfsetlinewidth{1.003750pt}%
\definecolor{currentstroke}{rgb}{0.800000,0.200000,0.200000}%
\pgfsetstrokecolor{currentstroke}%
\pgfsetdash{}{0pt}%
\pgfpathmoveto{\pgfqpoint{4.633701in}{1.494897in}}%
\pgfpathcurveto{\pgfqpoint{4.639525in}{1.494897in}}{\pgfqpoint{4.645111in}{1.497211in}}{\pgfqpoint{4.649229in}{1.501329in}}%
\pgfpathcurveto{\pgfqpoint{4.653348in}{1.505447in}}{\pgfqpoint{4.655661in}{1.511033in}}{\pgfqpoint{4.655661in}{1.516857in}}%
\pgfpathcurveto{\pgfqpoint{4.655661in}{1.522681in}}{\pgfqpoint{4.653348in}{1.528267in}}{\pgfqpoint{4.649229in}{1.532385in}}%
\pgfpathcurveto{\pgfqpoint{4.645111in}{1.536503in}}{\pgfqpoint{4.639525in}{1.538817in}}{\pgfqpoint{4.633701in}{1.538817in}}%
\pgfpathcurveto{\pgfqpoint{4.627877in}{1.538817in}}{\pgfqpoint{4.622291in}{1.536503in}}{\pgfqpoint{4.618173in}{1.532385in}}%
\pgfpathcurveto{\pgfqpoint{4.614055in}{1.528267in}}{\pgfqpoint{4.611741in}{1.522681in}}{\pgfqpoint{4.611741in}{1.516857in}}%
\pgfpathcurveto{\pgfqpoint{4.611741in}{1.511033in}}{\pgfqpoint{4.614055in}{1.505447in}}{\pgfqpoint{4.618173in}{1.501329in}}%
\pgfpathcurveto{\pgfqpoint{4.622291in}{1.497211in}}{\pgfqpoint{4.627877in}{1.494897in}}{\pgfqpoint{4.633701in}{1.494897in}}%
\pgfpathlineto{\pgfqpoint{4.633701in}{1.494897in}}%
\pgfpathclose%
\pgfusepath{stroke,fill}%
\end{pgfscope}%
\begin{pgfscope}%
\pgfpathrectangle{\pgfqpoint{1.542338in}{0.880000in}}{\pgfqpoint{5.115323in}{6.160000in}}%
\pgfusepath{clip}%
\pgfsetbuttcap%
\pgfsetroundjoin%
\definecolor{currentfill}{rgb}{0.800000,0.200000,0.200000}%
\pgfsetfillcolor{currentfill}%
\pgfsetlinewidth{1.003750pt}%
\definecolor{currentstroke}{rgb}{0.800000,0.200000,0.200000}%
\pgfsetstrokecolor{currentstroke}%
\pgfsetdash{}{0pt}%
\pgfpathmoveto{\pgfqpoint{4.727351in}{1.404645in}}%
\pgfpathcurveto{\pgfqpoint{4.733175in}{1.404645in}}{\pgfqpoint{4.738761in}{1.406959in}}{\pgfqpoint{4.742879in}{1.411077in}}%
\pgfpathcurveto{\pgfqpoint{4.746997in}{1.415195in}}{\pgfqpoint{4.749311in}{1.420782in}}{\pgfqpoint{4.749311in}{1.426606in}}%
\pgfpathcurveto{\pgfqpoint{4.749311in}{1.432429in}}{\pgfqpoint{4.746997in}{1.438016in}}{\pgfqpoint{4.742879in}{1.442134in}}%
\pgfpathcurveto{\pgfqpoint{4.738761in}{1.446252in}}{\pgfqpoint{4.733175in}{1.448566in}}{\pgfqpoint{4.727351in}{1.448566in}}%
\pgfpathcurveto{\pgfqpoint{4.721527in}{1.448566in}}{\pgfqpoint{4.715941in}{1.446252in}}{\pgfqpoint{4.711823in}{1.442134in}}%
\pgfpathcurveto{\pgfqpoint{4.707704in}{1.438016in}}{\pgfqpoint{4.705391in}{1.432429in}}{\pgfqpoint{4.705391in}{1.426606in}}%
\pgfpathcurveto{\pgfqpoint{4.705391in}{1.420782in}}{\pgfqpoint{4.707704in}{1.415195in}}{\pgfqpoint{4.711823in}{1.411077in}}%
\pgfpathcurveto{\pgfqpoint{4.715941in}{1.406959in}}{\pgfqpoint{4.721527in}{1.404645in}}{\pgfqpoint{4.727351in}{1.404645in}}%
\pgfpathlineto{\pgfqpoint{4.727351in}{1.404645in}}%
\pgfpathclose%
\pgfusepath{stroke,fill}%
\end{pgfscope}%
\begin{pgfscope}%
\pgfpathrectangle{\pgfqpoint{1.542338in}{0.880000in}}{\pgfqpoint{5.115323in}{6.160000in}}%
\pgfusepath{clip}%
\pgfsetbuttcap%
\pgfsetroundjoin%
\definecolor{currentfill}{rgb}{0.800000,0.200000,0.200000}%
\pgfsetfillcolor{currentfill}%
\pgfsetlinewidth{1.003750pt}%
\definecolor{currentstroke}{rgb}{0.800000,0.200000,0.200000}%
\pgfsetstrokecolor{currentstroke}%
\pgfsetdash{}{0pt}%
\pgfpathmoveto{\pgfqpoint{4.827106in}{1.321426in}}%
\pgfpathcurveto{\pgfqpoint{4.832930in}{1.321426in}}{\pgfqpoint{4.838516in}{1.323739in}}{\pgfqpoint{4.842635in}{1.327858in}}%
\pgfpathcurveto{\pgfqpoint{4.846753in}{1.331976in}}{\pgfqpoint{4.849067in}{1.337562in}}{\pgfqpoint{4.849067in}{1.343386in}}%
\pgfpathcurveto{\pgfqpoint{4.849067in}{1.349210in}}{\pgfqpoint{4.846753in}{1.354796in}}{\pgfqpoint{4.842635in}{1.358914in}}%
\pgfpathcurveto{\pgfqpoint{4.838516in}{1.363032in}}{\pgfqpoint{4.832930in}{1.365346in}}{\pgfqpoint{4.827106in}{1.365346in}}%
\pgfpathcurveto{\pgfqpoint{4.821282in}{1.365346in}}{\pgfqpoint{4.815696in}{1.363032in}}{\pgfqpoint{4.811578in}{1.358914in}}%
\pgfpathcurveto{\pgfqpoint{4.807460in}{1.354796in}}{\pgfqpoint{4.805146in}{1.349210in}}{\pgfqpoint{4.805146in}{1.343386in}}%
\pgfpathcurveto{\pgfqpoint{4.805146in}{1.337562in}}{\pgfqpoint{4.807460in}{1.331976in}}{\pgfqpoint{4.811578in}{1.327858in}}%
\pgfpathcurveto{\pgfqpoint{4.815696in}{1.323739in}}{\pgfqpoint{4.821282in}{1.321426in}}{\pgfqpoint{4.827106in}{1.321426in}}%
\pgfpathlineto{\pgfqpoint{4.827106in}{1.321426in}}%
\pgfpathclose%
\pgfusepath{stroke,fill}%
\end{pgfscope}%
\begin{pgfscope}%
\pgfpathrectangle{\pgfqpoint{1.542338in}{0.880000in}}{\pgfqpoint{5.115323in}{6.160000in}}%
\pgfusepath{clip}%
\pgfsetbuttcap%
\pgfsetroundjoin%
\definecolor{currentfill}{rgb}{0.800000,0.200000,0.200000}%
\pgfsetfillcolor{currentfill}%
\pgfsetlinewidth{1.003750pt}%
\definecolor{currentstroke}{rgb}{0.800000,0.200000,0.200000}%
\pgfsetstrokecolor{currentstroke}%
\pgfsetdash{}{0pt}%
\pgfpathmoveto{\pgfqpoint{4.937124in}{1.251693in}}%
\pgfpathcurveto{\pgfqpoint{4.942948in}{1.251693in}}{\pgfqpoint{4.948534in}{1.254007in}}{\pgfqpoint{4.952652in}{1.258125in}}%
\pgfpathcurveto{\pgfqpoint{4.956770in}{1.262244in}}{\pgfqpoint{4.959084in}{1.267830in}}{\pgfqpoint{4.959084in}{1.273654in}}%
\pgfpathcurveto{\pgfqpoint{4.959084in}{1.279478in}}{\pgfqpoint{4.956770in}{1.285064in}}{\pgfqpoint{4.952652in}{1.289182in}}%
\pgfpathcurveto{\pgfqpoint{4.948534in}{1.293300in}}{\pgfqpoint{4.942948in}{1.295614in}}{\pgfqpoint{4.937124in}{1.295614in}}%
\pgfpathcurveto{\pgfqpoint{4.931300in}{1.295614in}}{\pgfqpoint{4.925714in}{1.293300in}}{\pgfqpoint{4.921596in}{1.289182in}}%
\pgfpathcurveto{\pgfqpoint{4.917478in}{1.285064in}}{\pgfqpoint{4.915164in}{1.279478in}}{\pgfqpoint{4.915164in}{1.273654in}}%
\pgfpathcurveto{\pgfqpoint{4.915164in}{1.267830in}}{\pgfqpoint{4.917478in}{1.262244in}}{\pgfqpoint{4.921596in}{1.258125in}}%
\pgfpathcurveto{\pgfqpoint{4.925714in}{1.254007in}}{\pgfqpoint{4.931300in}{1.251693in}}{\pgfqpoint{4.937124in}{1.251693in}}%
\pgfpathlineto{\pgfqpoint{4.937124in}{1.251693in}}%
\pgfpathclose%
\pgfusepath{stroke,fill}%
\end{pgfscope}%
\begin{pgfscope}%
\pgfpathrectangle{\pgfqpoint{1.542338in}{0.880000in}}{\pgfqpoint{5.115323in}{6.160000in}}%
\pgfusepath{clip}%
\pgfsetbuttcap%
\pgfsetroundjoin%
\definecolor{currentfill}{rgb}{0.800000,0.200000,0.200000}%
\pgfsetfillcolor{currentfill}%
\pgfsetlinewidth{1.003750pt}%
\definecolor{currentstroke}{rgb}{0.800000,0.200000,0.200000}%
\pgfsetstrokecolor{currentstroke}%
\pgfsetdash{}{0pt}%
\pgfpathmoveto{\pgfqpoint{5.057331in}{1.201548in}}%
\pgfpathcurveto{\pgfqpoint{5.063155in}{1.201548in}}{\pgfqpoint{5.068741in}{1.203862in}}{\pgfqpoint{5.072859in}{1.207980in}}%
\pgfpathcurveto{\pgfqpoint{5.076977in}{1.212099in}}{\pgfqpoint{5.079291in}{1.217685in}}{\pgfqpoint{5.079291in}{1.223509in}}%
\pgfpathcurveto{\pgfqpoint{5.079291in}{1.229333in}}{\pgfqpoint{5.076977in}{1.234919in}}{\pgfqpoint{5.072859in}{1.239037in}}%
\pgfpathcurveto{\pgfqpoint{5.068741in}{1.243155in}}{\pgfqpoint{5.063155in}{1.245469in}}{\pgfqpoint{5.057331in}{1.245469in}}%
\pgfpathcurveto{\pgfqpoint{5.051507in}{1.245469in}}{\pgfqpoint{5.045921in}{1.243155in}}{\pgfqpoint{5.041802in}{1.239037in}}%
\pgfpathcurveto{\pgfqpoint{5.037684in}{1.234919in}}{\pgfqpoint{5.035370in}{1.229333in}}{\pgfqpoint{5.035370in}{1.223509in}}%
\pgfpathcurveto{\pgfqpoint{5.035370in}{1.217685in}}{\pgfqpoint{5.037684in}{1.212099in}}{\pgfqpoint{5.041802in}{1.207980in}}%
\pgfpathcurveto{\pgfqpoint{5.045921in}{1.203862in}}{\pgfqpoint{5.051507in}{1.201548in}}{\pgfqpoint{5.057331in}{1.201548in}}%
\pgfpathlineto{\pgfqpoint{5.057331in}{1.201548in}}%
\pgfpathclose%
\pgfusepath{stroke,fill}%
\end{pgfscope}%
\begin{pgfscope}%
\pgfpathrectangle{\pgfqpoint{1.542338in}{0.880000in}}{\pgfqpoint{5.115323in}{6.160000in}}%
\pgfusepath{clip}%
\pgfsetbuttcap%
\pgfsetroundjoin%
\definecolor{currentfill}{rgb}{0.800000,0.200000,0.200000}%
\pgfsetfillcolor{currentfill}%
\pgfsetlinewidth{1.003750pt}%
\definecolor{currentstroke}{rgb}{0.800000,0.200000,0.200000}%
\pgfsetstrokecolor{currentstroke}%
\pgfsetdash{}{0pt}%
\pgfpathmoveto{\pgfqpoint{5.181250in}{1.161298in}}%
\pgfpathcurveto{\pgfqpoint{5.187074in}{1.161298in}}{\pgfqpoint{5.192660in}{1.163611in}}{\pgfqpoint{5.196778in}{1.167730in}}%
\pgfpathcurveto{\pgfqpoint{5.200896in}{1.171848in}}{\pgfqpoint{5.203210in}{1.177434in}}{\pgfqpoint{5.203210in}{1.183258in}}%
\pgfpathcurveto{\pgfqpoint{5.203210in}{1.189082in}}{\pgfqpoint{5.200896in}{1.194668in}}{\pgfqpoint{5.196778in}{1.198786in}}%
\pgfpathcurveto{\pgfqpoint{5.192660in}{1.202904in}}{\pgfqpoint{5.187074in}{1.205218in}}{\pgfqpoint{5.181250in}{1.205218in}}%
\pgfpathcurveto{\pgfqpoint{5.175426in}{1.205218in}}{\pgfqpoint{5.169840in}{1.202904in}}{\pgfqpoint{5.165722in}{1.198786in}}%
\pgfpathcurveto{\pgfqpoint{5.161604in}{1.194668in}}{\pgfqpoint{5.159290in}{1.189082in}}{\pgfqpoint{5.159290in}{1.183258in}}%
\pgfpathcurveto{\pgfqpoint{5.159290in}{1.177434in}}{\pgfqpoint{5.161604in}{1.171848in}}{\pgfqpoint{5.165722in}{1.167730in}}%
\pgfpathcurveto{\pgfqpoint{5.169840in}{1.163611in}}{\pgfqpoint{5.175426in}{1.161298in}}{\pgfqpoint{5.181250in}{1.161298in}}%
\pgfpathlineto{\pgfqpoint{5.181250in}{1.161298in}}%
\pgfpathclose%
\pgfusepath{stroke,fill}%
\end{pgfscope}%
\begin{pgfscope}%
\pgfpathrectangle{\pgfqpoint{1.542338in}{0.880000in}}{\pgfqpoint{5.115323in}{6.160000in}}%
\pgfusepath{clip}%
\pgfsetbuttcap%
\pgfsetroundjoin%
\definecolor{currentfill}{rgb}{0.800000,0.200000,0.200000}%
\pgfsetfillcolor{currentfill}%
\pgfsetlinewidth{1.003750pt}%
\definecolor{currentstroke}{rgb}{0.800000,0.200000,0.200000}%
\pgfsetstrokecolor{currentstroke}%
\pgfsetdash{}{0pt}%
\pgfpathmoveto{\pgfqpoint{5.310206in}{1.142183in}}%
\pgfpathcurveto{\pgfqpoint{5.316030in}{1.142183in}}{\pgfqpoint{5.321616in}{1.144497in}}{\pgfqpoint{5.325734in}{1.148615in}}%
\pgfpathcurveto{\pgfqpoint{5.329852in}{1.152733in}}{\pgfqpoint{5.332166in}{1.158320in}}{\pgfqpoint{5.332166in}{1.164144in}}%
\pgfpathcurveto{\pgfqpoint{5.332166in}{1.169968in}}{\pgfqpoint{5.329852in}{1.175554in}}{\pgfqpoint{5.325734in}{1.179672in}}%
\pgfpathcurveto{\pgfqpoint{5.321616in}{1.183790in}}{\pgfqpoint{5.316030in}{1.186104in}}{\pgfqpoint{5.310206in}{1.186104in}}%
\pgfpathcurveto{\pgfqpoint{5.304382in}{1.186104in}}{\pgfqpoint{5.298796in}{1.183790in}}{\pgfqpoint{5.294678in}{1.179672in}}%
\pgfpathcurveto{\pgfqpoint{5.290560in}{1.175554in}}{\pgfqpoint{5.288246in}{1.169968in}}{\pgfqpoint{5.288246in}{1.164144in}}%
\pgfpathcurveto{\pgfqpoint{5.288246in}{1.158320in}}{\pgfqpoint{5.290560in}{1.152733in}}{\pgfqpoint{5.294678in}{1.148615in}}%
\pgfpathcurveto{\pgfqpoint{5.298796in}{1.144497in}}{\pgfqpoint{5.304382in}{1.142183in}}{\pgfqpoint{5.310206in}{1.142183in}}%
\pgfpathlineto{\pgfqpoint{5.310206in}{1.142183in}}%
\pgfpathclose%
\pgfusepath{stroke,fill}%
\end{pgfscope}%
\begin{pgfscope}%
\pgfpathrectangle{\pgfqpoint{1.542338in}{0.880000in}}{\pgfqpoint{5.115323in}{6.160000in}}%
\pgfusepath{clip}%
\pgfsetbuttcap%
\pgfsetroundjoin%
\definecolor{currentfill}{rgb}{0.800000,0.200000,0.200000}%
\pgfsetfillcolor{currentfill}%
\pgfsetlinewidth{1.003750pt}%
\definecolor{currentstroke}{rgb}{0.800000,0.200000,0.200000}%
\pgfsetstrokecolor{currentstroke}%
\pgfsetdash{}{0pt}%
\pgfpathmoveto{\pgfqpoint{5.440382in}{1.138040in}}%
\pgfpathcurveto{\pgfqpoint{5.446206in}{1.138040in}}{\pgfqpoint{5.451792in}{1.140354in}}{\pgfqpoint{5.455910in}{1.144472in}}%
\pgfpathcurveto{\pgfqpoint{5.460028in}{1.148590in}}{\pgfqpoint{5.462342in}{1.154176in}}{\pgfqpoint{5.462342in}{1.160000in}}%
\pgfpathcurveto{\pgfqpoint{5.462342in}{1.165824in}}{\pgfqpoint{5.460028in}{1.171410in}}{\pgfqpoint{5.455910in}{1.175528in}}%
\pgfpathcurveto{\pgfqpoint{5.451792in}{1.179646in}}{\pgfqpoint{5.446206in}{1.181960in}}{\pgfqpoint{5.440382in}{1.181960in}}%
\pgfpathcurveto{\pgfqpoint{5.434558in}{1.181960in}}{\pgfqpoint{5.428972in}{1.179646in}}{\pgfqpoint{5.424854in}{1.175528in}}%
\pgfpathcurveto{\pgfqpoint{5.420736in}{1.171410in}}{\pgfqpoint{5.418422in}{1.165824in}}{\pgfqpoint{5.418422in}{1.160000in}}%
\pgfpathcurveto{\pgfqpoint{5.418422in}{1.154176in}}{\pgfqpoint{5.420736in}{1.148590in}}{\pgfqpoint{5.424854in}{1.144472in}}%
\pgfpathcurveto{\pgfqpoint{5.428972in}{1.140354in}}{\pgfqpoint{5.434558in}{1.138040in}}{\pgfqpoint{5.440382in}{1.138040in}}%
\pgfpathlineto{\pgfqpoint{5.440382in}{1.138040in}}%
\pgfpathclose%
\pgfusepath{stroke,fill}%
\end{pgfscope}%
\begin{pgfscope}%
\pgfpathrectangle{\pgfqpoint{1.542338in}{0.880000in}}{\pgfqpoint{5.115323in}{6.160000in}}%
\pgfusepath{clip}%
\pgfsetbuttcap%
\pgfsetroundjoin%
\definecolor{currentfill}{rgb}{0.800000,0.200000,0.200000}%
\pgfsetfillcolor{currentfill}%
\pgfsetlinewidth{1.003750pt}%
\definecolor{currentstroke}{rgb}{0.800000,0.200000,0.200000}%
\pgfsetstrokecolor{currentstroke}%
\pgfsetdash{}{0pt}%
\pgfpathmoveto{\pgfqpoint{5.570244in}{1.149158in}}%
\pgfpathcurveto{\pgfqpoint{5.576068in}{1.149158in}}{\pgfqpoint{5.581654in}{1.151472in}}{\pgfqpoint{5.585773in}{1.155590in}}%
\pgfpathcurveto{\pgfqpoint{5.589891in}{1.159708in}}{\pgfqpoint{5.592205in}{1.165294in}}{\pgfqpoint{5.592205in}{1.171118in}}%
\pgfpathcurveto{\pgfqpoint{5.592205in}{1.176942in}}{\pgfqpoint{5.589891in}{1.182528in}}{\pgfqpoint{5.585773in}{1.186646in}}%
\pgfpathcurveto{\pgfqpoint{5.581654in}{1.190765in}}{\pgfqpoint{5.576068in}{1.193079in}}{\pgfqpoint{5.570244in}{1.193079in}}%
\pgfpathcurveto{\pgfqpoint{5.564420in}{1.193079in}}{\pgfqpoint{5.558834in}{1.190765in}}{\pgfqpoint{5.554716in}{1.186646in}}%
\pgfpathcurveto{\pgfqpoint{5.550598in}{1.182528in}}{\pgfqpoint{5.548284in}{1.176942in}}{\pgfqpoint{5.548284in}{1.171118in}}%
\pgfpathcurveto{\pgfqpoint{5.548284in}{1.165294in}}{\pgfqpoint{5.550598in}{1.159708in}}{\pgfqpoint{5.554716in}{1.155590in}}%
\pgfpathcurveto{\pgfqpoint{5.558834in}{1.151472in}}{\pgfqpoint{5.564420in}{1.149158in}}{\pgfqpoint{5.570244in}{1.149158in}}%
\pgfpathlineto{\pgfqpoint{5.570244in}{1.149158in}}%
\pgfpathclose%
\pgfusepath{stroke,fill}%
\end{pgfscope}%
\begin{pgfscope}%
\pgfpathrectangle{\pgfqpoint{1.542338in}{0.880000in}}{\pgfqpoint{5.115323in}{6.160000in}}%
\pgfusepath{clip}%
\pgfsetbuttcap%
\pgfsetroundjoin%
\definecolor{currentfill}{rgb}{0.800000,0.200000,0.200000}%
\pgfsetfillcolor{currentfill}%
\pgfsetlinewidth{1.003750pt}%
\definecolor{currentstroke}{rgb}{0.800000,0.200000,0.200000}%
\pgfsetstrokecolor{currentstroke}%
\pgfsetdash{}{0pt}%
\pgfpathmoveto{\pgfqpoint{5.697712in}{1.177110in}}%
\pgfpathcurveto{\pgfqpoint{5.703536in}{1.177110in}}{\pgfqpoint{5.709122in}{1.179424in}}{\pgfqpoint{5.713240in}{1.183542in}}%
\pgfpathcurveto{\pgfqpoint{5.717358in}{1.187660in}}{\pgfqpoint{5.719672in}{1.193246in}}{\pgfqpoint{5.719672in}{1.199070in}}%
\pgfpathcurveto{\pgfqpoint{5.719672in}{1.204894in}}{\pgfqpoint{5.717358in}{1.210480in}}{\pgfqpoint{5.713240in}{1.214598in}}%
\pgfpathcurveto{\pgfqpoint{5.709122in}{1.218716in}}{\pgfqpoint{5.703536in}{1.221030in}}{\pgfqpoint{5.697712in}{1.221030in}}%
\pgfpathcurveto{\pgfqpoint{5.691888in}{1.221030in}}{\pgfqpoint{5.686302in}{1.218716in}}{\pgfqpoint{5.682184in}{1.214598in}}%
\pgfpathcurveto{\pgfqpoint{5.678065in}{1.210480in}}{\pgfqpoint{5.675752in}{1.204894in}}{\pgfqpoint{5.675752in}{1.199070in}}%
\pgfpathcurveto{\pgfqpoint{5.675752in}{1.193246in}}{\pgfqpoint{5.678065in}{1.187660in}}{\pgfqpoint{5.682184in}{1.183542in}}%
\pgfpathcurveto{\pgfqpoint{5.686302in}{1.179424in}}{\pgfqpoint{5.691888in}{1.177110in}}{\pgfqpoint{5.697712in}{1.177110in}}%
\pgfpathlineto{\pgfqpoint{5.697712in}{1.177110in}}%
\pgfpathclose%
\pgfusepath{stroke,fill}%
\end{pgfscope}%
\begin{pgfscope}%
\pgfpathrectangle{\pgfqpoint{1.542338in}{0.880000in}}{\pgfqpoint{5.115323in}{6.160000in}}%
\pgfusepath{clip}%
\pgfsetbuttcap%
\pgfsetroundjoin%
\definecolor{currentfill}{rgb}{0.800000,0.200000,0.200000}%
\pgfsetfillcolor{currentfill}%
\pgfsetlinewidth{1.003750pt}%
\definecolor{currentstroke}{rgb}{0.800000,0.200000,0.200000}%
\pgfsetstrokecolor{currentstroke}%
\pgfsetdash{}{0pt}%
\pgfpathmoveto{\pgfqpoint{5.820603in}{1.221383in}}%
\pgfpathcurveto{\pgfqpoint{5.826427in}{1.221383in}}{\pgfqpoint{5.832013in}{1.223697in}}{\pgfqpoint{5.836131in}{1.227816in}}%
\pgfpathcurveto{\pgfqpoint{5.840250in}{1.231934in}}{\pgfqpoint{5.842563in}{1.237520in}}{\pgfqpoint{5.842563in}{1.243344in}}%
\pgfpathcurveto{\pgfqpoint{5.842563in}{1.249168in}}{\pgfqpoint{5.840250in}{1.254754in}}{\pgfqpoint{5.836131in}{1.258872in}}%
\pgfpathcurveto{\pgfqpoint{5.832013in}{1.262990in}}{\pgfqpoint{5.826427in}{1.265304in}}{\pgfqpoint{5.820603in}{1.265304in}}%
\pgfpathcurveto{\pgfqpoint{5.814779in}{1.265304in}}{\pgfqpoint{5.809193in}{1.262990in}}{\pgfqpoint{5.805075in}{1.258872in}}%
\pgfpathcurveto{\pgfqpoint{5.800957in}{1.254754in}}{\pgfqpoint{5.798643in}{1.249168in}}{\pgfqpoint{5.798643in}{1.243344in}}%
\pgfpathcurveto{\pgfqpoint{5.798643in}{1.237520in}}{\pgfqpoint{5.800957in}{1.231934in}}{\pgfqpoint{5.805075in}{1.227816in}}%
\pgfpathcurveto{\pgfqpoint{5.809193in}{1.223697in}}{\pgfqpoint{5.814779in}{1.221383in}}{\pgfqpoint{5.820603in}{1.221383in}}%
\pgfpathlineto{\pgfqpoint{5.820603in}{1.221383in}}%
\pgfpathclose%
\pgfusepath{stroke,fill}%
\end{pgfscope}%
\begin{pgfscope}%
\pgfpathrectangle{\pgfqpoint{1.542338in}{0.880000in}}{\pgfqpoint{5.115323in}{6.160000in}}%
\pgfusepath{clip}%
\pgfsetbuttcap%
\pgfsetroundjoin%
\definecolor{currentfill}{rgb}{0.800000,0.200000,0.200000}%
\pgfsetfillcolor{currentfill}%
\pgfsetlinewidth{1.003750pt}%
\definecolor{currentstroke}{rgb}{0.800000,0.200000,0.200000}%
\pgfsetstrokecolor{currentstroke}%
\pgfsetdash{}{0pt}%
\pgfpathmoveto{\pgfqpoint{5.939115in}{1.277450in}}%
\pgfpathcurveto{\pgfqpoint{5.944939in}{1.277450in}}{\pgfqpoint{5.950525in}{1.279764in}}{\pgfqpoint{5.954643in}{1.283882in}}%
\pgfpathcurveto{\pgfqpoint{5.958762in}{1.288000in}}{\pgfqpoint{5.961075in}{1.293586in}}{\pgfqpoint{5.961075in}{1.299410in}}%
\pgfpathcurveto{\pgfqpoint{5.961075in}{1.305234in}}{\pgfqpoint{5.958762in}{1.310820in}}{\pgfqpoint{5.954643in}{1.314938in}}%
\pgfpathcurveto{\pgfqpoint{5.950525in}{1.319057in}}{\pgfqpoint{5.944939in}{1.321370in}}{\pgfqpoint{5.939115in}{1.321370in}}%
\pgfpathcurveto{\pgfqpoint{5.933291in}{1.321370in}}{\pgfqpoint{5.927705in}{1.319057in}}{\pgfqpoint{5.923587in}{1.314938in}}%
\pgfpathcurveto{\pgfqpoint{5.919469in}{1.310820in}}{\pgfqpoint{5.917155in}{1.305234in}}{\pgfqpoint{5.917155in}{1.299410in}}%
\pgfpathcurveto{\pgfqpoint{5.917155in}{1.293586in}}{\pgfqpoint{5.919469in}{1.288000in}}{\pgfqpoint{5.923587in}{1.283882in}}%
\pgfpathcurveto{\pgfqpoint{5.927705in}{1.279764in}}{\pgfqpoint{5.933291in}{1.277450in}}{\pgfqpoint{5.939115in}{1.277450in}}%
\pgfpathlineto{\pgfqpoint{5.939115in}{1.277450in}}%
\pgfpathclose%
\pgfusepath{stroke,fill}%
\end{pgfscope}%
\begin{pgfscope}%
\pgfpathrectangle{\pgfqpoint{1.542338in}{0.880000in}}{\pgfqpoint{5.115323in}{6.160000in}}%
\pgfusepath{clip}%
\pgfsetbuttcap%
\pgfsetroundjoin%
\definecolor{currentfill}{rgb}{0.800000,0.200000,0.200000}%
\pgfsetfillcolor{currentfill}%
\pgfsetlinewidth{1.003750pt}%
\definecolor{currentstroke}{rgb}{0.800000,0.200000,0.200000}%
\pgfsetstrokecolor{currentstroke}%
\pgfsetdash{}{0pt}%
\pgfpathmoveto{\pgfqpoint{6.044030in}{1.356101in}}%
\pgfpathcurveto{\pgfqpoint{6.049854in}{1.356101in}}{\pgfqpoint{6.055440in}{1.358415in}}{\pgfqpoint{6.059558in}{1.362533in}}%
\pgfpathcurveto{\pgfqpoint{6.063676in}{1.366651in}}{\pgfqpoint{6.065990in}{1.372237in}}{\pgfqpoint{6.065990in}{1.378061in}}%
\pgfpathcurveto{\pgfqpoint{6.065990in}{1.383885in}}{\pgfqpoint{6.063676in}{1.389471in}}{\pgfqpoint{6.059558in}{1.393589in}}%
\pgfpathcurveto{\pgfqpoint{6.055440in}{1.397707in}}{\pgfqpoint{6.049854in}{1.400021in}}{\pgfqpoint{6.044030in}{1.400021in}}%
\pgfpathcurveto{\pgfqpoint{6.038206in}{1.400021in}}{\pgfqpoint{6.032620in}{1.397707in}}{\pgfqpoint{6.028501in}{1.393589in}}%
\pgfpathcurveto{\pgfqpoint{6.024383in}{1.389471in}}{\pgfqpoint{6.022069in}{1.383885in}}{\pgfqpoint{6.022069in}{1.378061in}}%
\pgfpathcurveto{\pgfqpoint{6.022069in}{1.372237in}}{\pgfqpoint{6.024383in}{1.366651in}}{\pgfqpoint{6.028501in}{1.362533in}}%
\pgfpathcurveto{\pgfqpoint{6.032620in}{1.358415in}}{\pgfqpoint{6.038206in}{1.356101in}}{\pgfqpoint{6.044030in}{1.356101in}}%
\pgfpathlineto{\pgfqpoint{6.044030in}{1.356101in}}%
\pgfpathclose%
\pgfusepath{stroke,fill}%
\end{pgfscope}%
\begin{pgfscope}%
\pgfpathrectangle{\pgfqpoint{1.542338in}{0.880000in}}{\pgfqpoint{5.115323in}{6.160000in}}%
\pgfusepath{clip}%
\pgfsetbuttcap%
\pgfsetroundjoin%
\definecolor{currentfill}{rgb}{0.800000,0.200000,0.200000}%
\pgfsetfillcolor{currentfill}%
\pgfsetlinewidth{1.003750pt}%
\definecolor{currentstroke}{rgb}{0.800000,0.200000,0.200000}%
\pgfsetstrokecolor{currentstroke}%
\pgfsetdash{}{0pt}%
\pgfpathmoveto{\pgfqpoint{6.134067in}{1.450556in}}%
\pgfpathcurveto{\pgfqpoint{6.139891in}{1.450556in}}{\pgfqpoint{6.145477in}{1.452870in}}{\pgfqpoint{6.149595in}{1.456988in}}%
\pgfpathcurveto{\pgfqpoint{6.153713in}{1.461106in}}{\pgfqpoint{6.156027in}{1.466693in}}{\pgfqpoint{6.156027in}{1.472516in}}%
\pgfpathcurveto{\pgfqpoint{6.156027in}{1.478340in}}{\pgfqpoint{6.153713in}{1.483927in}}{\pgfqpoint{6.149595in}{1.488045in}}%
\pgfpathcurveto{\pgfqpoint{6.145477in}{1.492163in}}{\pgfqpoint{6.139891in}{1.494477in}}{\pgfqpoint{6.134067in}{1.494477in}}%
\pgfpathcurveto{\pgfqpoint{6.128243in}{1.494477in}}{\pgfqpoint{6.122657in}{1.492163in}}{\pgfqpoint{6.118538in}{1.488045in}}%
\pgfpathcurveto{\pgfqpoint{6.114420in}{1.483927in}}{\pgfqpoint{6.112106in}{1.478340in}}{\pgfqpoint{6.112106in}{1.472516in}}%
\pgfpathcurveto{\pgfqpoint{6.112106in}{1.466693in}}{\pgfqpoint{6.114420in}{1.461106in}}{\pgfqpoint{6.118538in}{1.456988in}}%
\pgfpathcurveto{\pgfqpoint{6.122657in}{1.452870in}}{\pgfqpoint{6.128243in}{1.450556in}}{\pgfqpoint{6.134067in}{1.450556in}}%
\pgfpathlineto{\pgfqpoint{6.134067in}{1.450556in}}%
\pgfpathclose%
\pgfusepath{stroke,fill}%
\end{pgfscope}%
\begin{pgfscope}%
\pgfpathrectangle{\pgfqpoint{1.542338in}{0.880000in}}{\pgfqpoint{5.115323in}{6.160000in}}%
\pgfusepath{clip}%
\pgfsetbuttcap%
\pgfsetroundjoin%
\definecolor{currentfill}{rgb}{0.800000,0.200000,0.200000}%
\pgfsetfillcolor{currentfill}%
\pgfsetlinewidth{1.003750pt}%
\definecolor{currentstroke}{rgb}{0.800000,0.200000,0.200000}%
\pgfsetstrokecolor{currentstroke}%
\pgfsetdash{}{0pt}%
\pgfpathmoveto{\pgfqpoint{6.219322in}{1.548251in}}%
\pgfpathcurveto{\pgfqpoint{6.225146in}{1.548251in}}{\pgfqpoint{6.230732in}{1.550565in}}{\pgfqpoint{6.234850in}{1.554683in}}%
\pgfpathcurveto{\pgfqpoint{6.238968in}{1.558801in}}{\pgfqpoint{6.241282in}{1.564387in}}{\pgfqpoint{6.241282in}{1.570211in}}%
\pgfpathcurveto{\pgfqpoint{6.241282in}{1.576035in}}{\pgfqpoint{6.238968in}{1.581621in}}{\pgfqpoint{6.234850in}{1.585739in}}%
\pgfpathcurveto{\pgfqpoint{6.230732in}{1.589857in}}{\pgfqpoint{6.225146in}{1.592171in}}{\pgfqpoint{6.219322in}{1.592171in}}%
\pgfpathcurveto{\pgfqpoint{6.213498in}{1.592171in}}{\pgfqpoint{6.207912in}{1.589857in}}{\pgfqpoint{6.203793in}{1.585739in}}%
\pgfpathcurveto{\pgfqpoint{6.199675in}{1.581621in}}{\pgfqpoint{6.197361in}{1.576035in}}{\pgfqpoint{6.197361in}{1.570211in}}%
\pgfpathcurveto{\pgfqpoint{6.197361in}{1.564387in}}{\pgfqpoint{6.199675in}{1.558801in}}{\pgfqpoint{6.203793in}{1.554683in}}%
\pgfpathcurveto{\pgfqpoint{6.207912in}{1.550565in}}{\pgfqpoint{6.213498in}{1.548251in}}{\pgfqpoint{6.219322in}{1.548251in}}%
\pgfpathlineto{\pgfqpoint{6.219322in}{1.548251in}}%
\pgfpathclose%
\pgfusepath{stroke,fill}%
\end{pgfscope}%
\begin{pgfscope}%
\pgfpathrectangle{\pgfqpoint{1.542338in}{0.880000in}}{\pgfqpoint{5.115323in}{6.160000in}}%
\pgfusepath{clip}%
\pgfsetbuttcap%
\pgfsetroundjoin%
\definecolor{currentfill}{rgb}{0.800000,0.200000,0.200000}%
\pgfsetfillcolor{currentfill}%
\pgfsetlinewidth{1.003750pt}%
\definecolor{currentstroke}{rgb}{0.800000,0.200000,0.200000}%
\pgfsetstrokecolor{currentstroke}%
\pgfsetdash{}{0pt}%
\pgfpathmoveto{\pgfqpoint{6.292939in}{1.655403in}}%
\pgfpathcurveto{\pgfqpoint{6.298762in}{1.655403in}}{\pgfqpoint{6.304349in}{1.657717in}}{\pgfqpoint{6.308467in}{1.661835in}}%
\pgfpathcurveto{\pgfqpoint{6.312585in}{1.665954in}}{\pgfqpoint{6.314899in}{1.671540in}}{\pgfqpoint{6.314899in}{1.677364in}}%
\pgfpathcurveto{\pgfqpoint{6.314899in}{1.683188in}}{\pgfqpoint{6.312585in}{1.688774in}}{\pgfqpoint{6.308467in}{1.692892in}}%
\pgfpathcurveto{\pgfqpoint{6.304349in}{1.697010in}}{\pgfqpoint{6.298762in}{1.699324in}}{\pgfqpoint{6.292939in}{1.699324in}}%
\pgfpathcurveto{\pgfqpoint{6.287115in}{1.699324in}}{\pgfqpoint{6.281528in}{1.697010in}}{\pgfqpoint{6.277410in}{1.692892in}}%
\pgfpathcurveto{\pgfqpoint{6.273292in}{1.688774in}}{\pgfqpoint{6.270978in}{1.683188in}}{\pgfqpoint{6.270978in}{1.677364in}}%
\pgfpathcurveto{\pgfqpoint{6.270978in}{1.671540in}}{\pgfqpoint{6.273292in}{1.665954in}}{\pgfqpoint{6.277410in}{1.661835in}}%
\pgfpathcurveto{\pgfqpoint{6.281528in}{1.657717in}}{\pgfqpoint{6.287115in}{1.655403in}}{\pgfqpoint{6.292939in}{1.655403in}}%
\pgfpathlineto{\pgfqpoint{6.292939in}{1.655403in}}%
\pgfpathclose%
\pgfusepath{stroke,fill}%
\end{pgfscope}%
\begin{pgfscope}%
\pgfpathrectangle{\pgfqpoint{1.542338in}{0.880000in}}{\pgfqpoint{5.115323in}{6.160000in}}%
\pgfusepath{clip}%
\pgfsetbuttcap%
\pgfsetroundjoin%
\definecolor{currentfill}{rgb}{0.800000,0.200000,0.200000}%
\pgfsetfillcolor{currentfill}%
\pgfsetlinewidth{1.003750pt}%
\definecolor{currentstroke}{rgb}{0.800000,0.200000,0.200000}%
\pgfsetstrokecolor{currentstroke}%
\pgfsetdash{}{0pt}%
\pgfpathmoveto{\pgfqpoint{6.347601in}{1.773420in}}%
\pgfpathcurveto{\pgfqpoint{6.353425in}{1.773420in}}{\pgfqpoint{6.359011in}{1.775734in}}{\pgfqpoint{6.363129in}{1.779852in}}%
\pgfpathcurveto{\pgfqpoint{6.367248in}{1.783970in}}{\pgfqpoint{6.369561in}{1.789556in}}{\pgfqpoint{6.369561in}{1.795380in}}%
\pgfpathcurveto{\pgfqpoint{6.369561in}{1.801204in}}{\pgfqpoint{6.367248in}{1.806790in}}{\pgfqpoint{6.363129in}{1.810908in}}%
\pgfpathcurveto{\pgfqpoint{6.359011in}{1.815027in}}{\pgfqpoint{6.353425in}{1.817340in}}{\pgfqpoint{6.347601in}{1.817340in}}%
\pgfpathcurveto{\pgfqpoint{6.341777in}{1.817340in}}{\pgfqpoint{6.336191in}{1.815027in}}{\pgfqpoint{6.332073in}{1.810908in}}%
\pgfpathcurveto{\pgfqpoint{6.327955in}{1.806790in}}{\pgfqpoint{6.325641in}{1.801204in}}{\pgfqpoint{6.325641in}{1.795380in}}%
\pgfpathcurveto{\pgfqpoint{6.325641in}{1.789556in}}{\pgfqpoint{6.327955in}{1.783970in}}{\pgfqpoint{6.332073in}{1.779852in}}%
\pgfpathcurveto{\pgfqpoint{6.336191in}{1.775734in}}{\pgfqpoint{6.341777in}{1.773420in}}{\pgfqpoint{6.347601in}{1.773420in}}%
\pgfpathlineto{\pgfqpoint{6.347601in}{1.773420in}}%
\pgfpathclose%
\pgfusepath{stroke,fill}%
\end{pgfscope}%
\begin{pgfscope}%
\pgfpathrectangle{\pgfqpoint{1.542338in}{0.880000in}}{\pgfqpoint{5.115323in}{6.160000in}}%
\pgfusepath{clip}%
\pgfsetbuttcap%
\pgfsetroundjoin%
\definecolor{currentfill}{rgb}{0.800000,0.200000,0.200000}%
\pgfsetfillcolor{currentfill}%
\pgfsetlinewidth{1.003750pt}%
\definecolor{currentstroke}{rgb}{0.800000,0.200000,0.200000}%
\pgfsetstrokecolor{currentstroke}%
\pgfsetdash{}{0pt}%
\pgfpathmoveto{\pgfqpoint{6.393019in}{1.895548in}}%
\pgfpathcurveto{\pgfqpoint{6.398843in}{1.895548in}}{\pgfqpoint{6.404429in}{1.897861in}}{\pgfqpoint{6.408547in}{1.901980in}}%
\pgfpathcurveto{\pgfqpoint{6.412666in}{1.906098in}}{\pgfqpoint{6.414979in}{1.911684in}}{\pgfqpoint{6.414979in}{1.917508in}}%
\pgfpathcurveto{\pgfqpoint{6.414979in}{1.923332in}}{\pgfqpoint{6.412666in}{1.928918in}}{\pgfqpoint{6.408547in}{1.933036in}}%
\pgfpathcurveto{\pgfqpoint{6.404429in}{1.937154in}}{\pgfqpoint{6.398843in}{1.939468in}}{\pgfqpoint{6.393019in}{1.939468in}}%
\pgfpathcurveto{\pgfqpoint{6.387195in}{1.939468in}}{\pgfqpoint{6.381609in}{1.937154in}}{\pgfqpoint{6.377491in}{1.933036in}}%
\pgfpathcurveto{\pgfqpoint{6.373373in}{1.928918in}}{\pgfqpoint{6.371059in}{1.923332in}}{\pgfqpoint{6.371059in}{1.917508in}}%
\pgfpathcurveto{\pgfqpoint{6.371059in}{1.911684in}}{\pgfqpoint{6.373373in}{1.906098in}}{\pgfqpoint{6.377491in}{1.901980in}}%
\pgfpathcurveto{\pgfqpoint{6.381609in}{1.897861in}}{\pgfqpoint{6.387195in}{1.895548in}}{\pgfqpoint{6.393019in}{1.895548in}}%
\pgfpathlineto{\pgfqpoint{6.393019in}{1.895548in}}%
\pgfpathclose%
\pgfusepath{stroke,fill}%
\end{pgfscope}%
\begin{pgfscope}%
\pgfpathrectangle{\pgfqpoint{1.542338in}{0.880000in}}{\pgfqpoint{5.115323in}{6.160000in}}%
\pgfusepath{clip}%
\pgfsetbuttcap%
\pgfsetroundjoin%
\definecolor{currentfill}{rgb}{0.800000,0.200000,0.200000}%
\pgfsetfillcolor{currentfill}%
\pgfsetlinewidth{1.003750pt}%
\definecolor{currentstroke}{rgb}{0.800000,0.200000,0.200000}%
\pgfsetstrokecolor{currentstroke}%
\pgfsetdash{}{0pt}%
\pgfpathmoveto{\pgfqpoint{6.418391in}{2.023601in}}%
\pgfpathcurveto{\pgfqpoint{6.424214in}{2.023601in}}{\pgfqpoint{6.429801in}{2.025914in}}{\pgfqpoint{6.433919in}{2.030033in}}%
\pgfpathcurveto{\pgfqpoint{6.438037in}{2.034151in}}{\pgfqpoint{6.440351in}{2.039737in}}{\pgfqpoint{6.440351in}{2.045561in}}%
\pgfpathcurveto{\pgfqpoint{6.440351in}{2.051385in}}{\pgfqpoint{6.438037in}{2.056971in}}{\pgfqpoint{6.433919in}{2.061089in}}%
\pgfpathcurveto{\pgfqpoint{6.429801in}{2.065207in}}{\pgfqpoint{6.424214in}{2.067521in}}{\pgfqpoint{6.418391in}{2.067521in}}%
\pgfpathcurveto{\pgfqpoint{6.412567in}{2.067521in}}{\pgfqpoint{6.406980in}{2.065207in}}{\pgfqpoint{6.402862in}{2.061089in}}%
\pgfpathcurveto{\pgfqpoint{6.398744in}{2.056971in}}{\pgfqpoint{6.396430in}{2.051385in}}{\pgfqpoint{6.396430in}{2.045561in}}%
\pgfpathcurveto{\pgfqpoint{6.396430in}{2.039737in}}{\pgfqpoint{6.398744in}{2.034151in}}{\pgfqpoint{6.402862in}{2.030033in}}%
\pgfpathcurveto{\pgfqpoint{6.406980in}{2.025914in}}{\pgfqpoint{6.412567in}{2.023601in}}{\pgfqpoint{6.418391in}{2.023601in}}%
\pgfpathlineto{\pgfqpoint{6.418391in}{2.023601in}}%
\pgfpathclose%
\pgfusepath{stroke,fill}%
\end{pgfscope}%
\begin{pgfscope}%
\pgfpathrectangle{\pgfqpoint{1.542338in}{0.880000in}}{\pgfqpoint{5.115323in}{6.160000in}}%
\pgfusepath{clip}%
\pgfsetbuttcap%
\pgfsetroundjoin%
\definecolor{currentfill}{rgb}{0.800000,0.200000,0.200000}%
\pgfsetfillcolor{currentfill}%
\pgfsetlinewidth{1.003750pt}%
\definecolor{currentstroke}{rgb}{0.800000,0.200000,0.200000}%
\pgfsetstrokecolor{currentstroke}%
\pgfsetdash{}{0pt}%
\pgfpathmoveto{\pgfqpoint{6.422822in}{2.153901in}}%
\pgfpathcurveto{\pgfqpoint{6.428645in}{2.153901in}}{\pgfqpoint{6.434232in}{2.156215in}}{\pgfqpoint{6.438350in}{2.160333in}}%
\pgfpathcurveto{\pgfqpoint{6.442468in}{2.164451in}}{\pgfqpoint{6.444782in}{2.170037in}}{\pgfqpoint{6.444782in}{2.175861in}}%
\pgfpathcurveto{\pgfqpoint{6.444782in}{2.181685in}}{\pgfqpoint{6.442468in}{2.187272in}}{\pgfqpoint{6.438350in}{2.191390in}}%
\pgfpathcurveto{\pgfqpoint{6.434232in}{2.195508in}}{\pgfqpoint{6.428645in}{2.197822in}}{\pgfqpoint{6.422822in}{2.197822in}}%
\pgfpathcurveto{\pgfqpoint{6.416998in}{2.197822in}}{\pgfqpoint{6.411411in}{2.195508in}}{\pgfqpoint{6.407293in}{2.191390in}}%
\pgfpathcurveto{\pgfqpoint{6.403175in}{2.187272in}}{\pgfqpoint{6.400861in}{2.181685in}}{\pgfqpoint{6.400861in}{2.175861in}}%
\pgfpathcurveto{\pgfqpoint{6.400861in}{2.170037in}}{\pgfqpoint{6.403175in}{2.164451in}}{\pgfqpoint{6.407293in}{2.160333in}}%
\pgfpathcurveto{\pgfqpoint{6.411411in}{2.156215in}}{\pgfqpoint{6.416998in}{2.153901in}}{\pgfqpoint{6.422822in}{2.153901in}}%
\pgfpathlineto{\pgfqpoint{6.422822in}{2.153901in}}%
\pgfpathclose%
\pgfusepath{stroke,fill}%
\end{pgfscope}%
\begin{pgfscope}%
\pgfpathrectangle{\pgfqpoint{1.542338in}{0.880000in}}{\pgfqpoint{5.115323in}{6.160000in}}%
\pgfusepath{clip}%
\pgfsetbuttcap%
\pgfsetroundjoin%
\definecolor{currentfill}{rgb}{0.200000,0.200000,0.800000}%
\pgfsetfillcolor{currentfill}%
\pgfsetlinewidth{1.003750pt}%
\definecolor{currentstroke}{rgb}{0.200000,0.200000,0.800000}%
\pgfsetstrokecolor{currentstroke}%
\pgfsetdash{}{0pt}%
\pgfpathmoveto{\pgfqpoint{4.165962in}{2.531321in}}%
\pgfpathcurveto{\pgfqpoint{4.171786in}{2.531321in}}{\pgfqpoint{4.177372in}{2.533635in}}{\pgfqpoint{4.181490in}{2.537753in}}%
\pgfpathcurveto{\pgfqpoint{4.185608in}{2.541872in}}{\pgfqpoint{4.187922in}{2.547458in}}{\pgfqpoint{4.187922in}{2.553282in}}%
\pgfpathcurveto{\pgfqpoint{4.187922in}{2.559106in}}{\pgfqpoint{4.185608in}{2.564692in}}{\pgfqpoint{4.181490in}{2.568810in}}%
\pgfpathcurveto{\pgfqpoint{4.177372in}{2.572928in}}{\pgfqpoint{4.171786in}{2.575242in}}{\pgfqpoint{4.165962in}{2.575242in}}%
\pgfpathcurveto{\pgfqpoint{4.160138in}{2.575242in}}{\pgfqpoint{4.154552in}{2.572928in}}{\pgfqpoint{4.150434in}{2.568810in}}%
\pgfpathcurveto{\pgfqpoint{4.146315in}{2.564692in}}{\pgfqpoint{4.144002in}{2.559106in}}{\pgfqpoint{4.144002in}{2.553282in}}%
\pgfpathcurveto{\pgfqpoint{4.144002in}{2.547458in}}{\pgfqpoint{4.146315in}{2.541872in}}{\pgfqpoint{4.150434in}{2.537753in}}%
\pgfpathcurveto{\pgfqpoint{4.154552in}{2.533635in}}{\pgfqpoint{4.160138in}{2.531321in}}{\pgfqpoint{4.165962in}{2.531321in}}%
\pgfpathlineto{\pgfqpoint{4.165962in}{2.531321in}}%
\pgfpathclose%
\pgfusepath{stroke,fill}%
\end{pgfscope}%
\begin{pgfscope}%
\pgfpathrectangle{\pgfqpoint{1.542338in}{0.880000in}}{\pgfqpoint{5.115323in}{6.160000in}}%
\pgfusepath{clip}%
\pgfsetbuttcap%
\pgfsetroundjoin%
\definecolor{currentfill}{rgb}{0.200000,0.200000,0.800000}%
\pgfsetfillcolor{currentfill}%
\pgfsetlinewidth{1.003750pt}%
\definecolor{currentstroke}{rgb}{0.200000,0.200000,0.800000}%
\pgfsetstrokecolor{currentstroke}%
\pgfsetdash{}{0pt}%
\pgfpathmoveto{\pgfqpoint{4.168069in}{2.685262in}}%
\pgfpathcurveto{\pgfqpoint{4.173893in}{2.685262in}}{\pgfqpoint{4.179480in}{2.687576in}}{\pgfqpoint{4.183598in}{2.691694in}}%
\pgfpathcurveto{\pgfqpoint{4.187716in}{2.695812in}}{\pgfqpoint{4.190030in}{2.701398in}}{\pgfqpoint{4.190030in}{2.707222in}}%
\pgfpathcurveto{\pgfqpoint{4.190030in}{2.713046in}}{\pgfqpoint{4.187716in}{2.718632in}}{\pgfqpoint{4.183598in}{2.722750in}}%
\pgfpathcurveto{\pgfqpoint{4.179480in}{2.726868in}}{\pgfqpoint{4.173893in}{2.729182in}}{\pgfqpoint{4.168069in}{2.729182in}}%
\pgfpathcurveto{\pgfqpoint{4.162246in}{2.729182in}}{\pgfqpoint{4.156659in}{2.726868in}}{\pgfqpoint{4.152541in}{2.722750in}}%
\pgfpathcurveto{\pgfqpoint{4.148423in}{2.718632in}}{\pgfqpoint{4.146109in}{2.713046in}}{\pgfqpoint{4.146109in}{2.707222in}}%
\pgfpathcurveto{\pgfqpoint{4.146109in}{2.701398in}}{\pgfqpoint{4.148423in}{2.695812in}}{\pgfqpoint{4.152541in}{2.691694in}}%
\pgfpathcurveto{\pgfqpoint{4.156659in}{2.687576in}}{\pgfqpoint{4.162246in}{2.685262in}}{\pgfqpoint{4.168069in}{2.685262in}}%
\pgfpathlineto{\pgfqpoint{4.168069in}{2.685262in}}%
\pgfpathclose%
\pgfusepath{stroke,fill}%
\end{pgfscope}%
\begin{pgfscope}%
\pgfpathrectangle{\pgfqpoint{1.542338in}{0.880000in}}{\pgfqpoint{5.115323in}{6.160000in}}%
\pgfusepath{clip}%
\pgfsetbuttcap%
\pgfsetroundjoin%
\definecolor{currentfill}{rgb}{0.200000,0.200000,0.800000}%
\pgfsetfillcolor{currentfill}%
\pgfsetlinewidth{1.003750pt}%
\definecolor{currentstroke}{rgb}{0.200000,0.200000,0.800000}%
\pgfsetstrokecolor{currentstroke}%
\pgfsetdash{}{0pt}%
\pgfpathmoveto{\pgfqpoint{4.137677in}{2.836437in}}%
\pgfpathcurveto{\pgfqpoint{4.143501in}{2.836437in}}{\pgfqpoint{4.149087in}{2.838751in}}{\pgfqpoint{4.153205in}{2.842869in}}%
\pgfpathcurveto{\pgfqpoint{4.157323in}{2.846987in}}{\pgfqpoint{4.159637in}{2.852574in}}{\pgfqpoint{4.159637in}{2.858398in}}%
\pgfpathcurveto{\pgfqpoint{4.159637in}{2.864221in}}{\pgfqpoint{4.157323in}{2.869808in}}{\pgfqpoint{4.153205in}{2.873926in}}%
\pgfpathcurveto{\pgfqpoint{4.149087in}{2.878044in}}{\pgfqpoint{4.143501in}{2.880358in}}{\pgfqpoint{4.137677in}{2.880358in}}%
\pgfpathcurveto{\pgfqpoint{4.131853in}{2.880358in}}{\pgfqpoint{4.126267in}{2.878044in}}{\pgfqpoint{4.122149in}{2.873926in}}%
\pgfpathcurveto{\pgfqpoint{4.118030in}{2.869808in}}{\pgfqpoint{4.115717in}{2.864221in}}{\pgfqpoint{4.115717in}{2.858398in}}%
\pgfpathcurveto{\pgfqpoint{4.115717in}{2.852574in}}{\pgfqpoint{4.118030in}{2.846987in}}{\pgfqpoint{4.122149in}{2.842869in}}%
\pgfpathcurveto{\pgfqpoint{4.126267in}{2.838751in}}{\pgfqpoint{4.131853in}{2.836437in}}{\pgfqpoint{4.137677in}{2.836437in}}%
\pgfpathlineto{\pgfqpoint{4.137677in}{2.836437in}}%
\pgfpathclose%
\pgfusepath{stroke,fill}%
\end{pgfscope}%
\begin{pgfscope}%
\pgfpathrectangle{\pgfqpoint{1.542338in}{0.880000in}}{\pgfqpoint{5.115323in}{6.160000in}}%
\pgfusepath{clip}%
\pgfsetbuttcap%
\pgfsetroundjoin%
\definecolor{currentfill}{rgb}{0.200000,0.200000,0.800000}%
\pgfsetfillcolor{currentfill}%
\pgfsetlinewidth{1.003750pt}%
\definecolor{currentstroke}{rgb}{0.200000,0.200000,0.800000}%
\pgfsetstrokecolor{currentstroke}%
\pgfsetdash{}{0pt}%
\pgfpathmoveto{\pgfqpoint{4.079875in}{2.978985in}}%
\pgfpathcurveto{\pgfqpoint{4.085699in}{2.978985in}}{\pgfqpoint{4.091285in}{2.981299in}}{\pgfqpoint{4.095403in}{2.985417in}}%
\pgfpathcurveto{\pgfqpoint{4.099521in}{2.989535in}}{\pgfqpoint{4.101835in}{2.995121in}}{\pgfqpoint{4.101835in}{3.000945in}}%
\pgfpathcurveto{\pgfqpoint{4.101835in}{3.006769in}}{\pgfqpoint{4.099521in}{3.012355in}}{\pgfqpoint{4.095403in}{3.016473in}}%
\pgfpathcurveto{\pgfqpoint{4.091285in}{3.020591in}}{\pgfqpoint{4.085699in}{3.022905in}}{\pgfqpoint{4.079875in}{3.022905in}}%
\pgfpathcurveto{\pgfqpoint{4.074051in}{3.022905in}}{\pgfqpoint{4.068465in}{3.020591in}}{\pgfqpoint{4.064347in}{3.016473in}}%
\pgfpathcurveto{\pgfqpoint{4.060229in}{3.012355in}}{\pgfqpoint{4.057915in}{3.006769in}}{\pgfqpoint{4.057915in}{3.000945in}}%
\pgfpathcurveto{\pgfqpoint{4.057915in}{2.995121in}}{\pgfqpoint{4.060229in}{2.989535in}}{\pgfqpoint{4.064347in}{2.985417in}}%
\pgfpathcurveto{\pgfqpoint{4.068465in}{2.981299in}}{\pgfqpoint{4.074051in}{2.978985in}}{\pgfqpoint{4.079875in}{2.978985in}}%
\pgfpathlineto{\pgfqpoint{4.079875in}{2.978985in}}%
\pgfpathclose%
\pgfusepath{stroke,fill}%
\end{pgfscope}%
\begin{pgfscope}%
\pgfpathrectangle{\pgfqpoint{1.542338in}{0.880000in}}{\pgfqpoint{5.115323in}{6.160000in}}%
\pgfusepath{clip}%
\pgfsetbuttcap%
\pgfsetroundjoin%
\definecolor{currentfill}{rgb}{0.200000,0.200000,0.800000}%
\pgfsetfillcolor{currentfill}%
\pgfsetlinewidth{1.003750pt}%
\definecolor{currentstroke}{rgb}{0.200000,0.200000,0.800000}%
\pgfsetstrokecolor{currentstroke}%
\pgfsetdash{}{0pt}%
\pgfpathmoveto{\pgfqpoint{4.016867in}{3.118576in}}%
\pgfpathcurveto{\pgfqpoint{4.022691in}{3.118576in}}{\pgfqpoint{4.028277in}{3.120890in}}{\pgfqpoint{4.032396in}{3.125008in}}%
\pgfpathcurveto{\pgfqpoint{4.036514in}{3.129126in}}{\pgfqpoint{4.038828in}{3.134712in}}{\pgfqpoint{4.038828in}{3.140536in}}%
\pgfpathcurveto{\pgfqpoint{4.038828in}{3.146360in}}{\pgfqpoint{4.036514in}{3.151946in}}{\pgfqpoint{4.032396in}{3.156064in}}%
\pgfpathcurveto{\pgfqpoint{4.028277in}{3.160182in}}{\pgfqpoint{4.022691in}{3.162496in}}{\pgfqpoint{4.016867in}{3.162496in}}%
\pgfpathcurveto{\pgfqpoint{4.011043in}{3.162496in}}{\pgfqpoint{4.005457in}{3.160182in}}{\pgfqpoint{4.001339in}{3.156064in}}%
\pgfpathcurveto{\pgfqpoint{3.997221in}{3.151946in}}{\pgfqpoint{3.994907in}{3.146360in}}{\pgfqpoint{3.994907in}{3.140536in}}%
\pgfpathcurveto{\pgfqpoint{3.994907in}{3.134712in}}{\pgfqpoint{3.997221in}{3.129126in}}{\pgfqpoint{4.001339in}{3.125008in}}%
\pgfpathcurveto{\pgfqpoint{4.005457in}{3.120890in}}{\pgfqpoint{4.011043in}{3.118576in}}{\pgfqpoint{4.016867in}{3.118576in}}%
\pgfpathlineto{\pgfqpoint{4.016867in}{3.118576in}}%
\pgfpathclose%
\pgfusepath{stroke,fill}%
\end{pgfscope}%
\begin{pgfscope}%
\pgfpathrectangle{\pgfqpoint{1.542338in}{0.880000in}}{\pgfqpoint{5.115323in}{6.160000in}}%
\pgfusepath{clip}%
\pgfsetbuttcap%
\pgfsetroundjoin%
\definecolor{currentfill}{rgb}{0.200000,0.200000,0.800000}%
\pgfsetfillcolor{currentfill}%
\pgfsetlinewidth{1.003750pt}%
\definecolor{currentstroke}{rgb}{0.200000,0.200000,0.800000}%
\pgfsetstrokecolor{currentstroke}%
\pgfsetdash{}{0pt}%
\pgfpathmoveto{\pgfqpoint{3.938480in}{3.251029in}}%
\pgfpathcurveto{\pgfqpoint{3.944304in}{3.251029in}}{\pgfqpoint{3.949890in}{3.253343in}}{\pgfqpoint{3.954008in}{3.257461in}}%
\pgfpathcurveto{\pgfqpoint{3.958126in}{3.261579in}}{\pgfqpoint{3.960440in}{3.267165in}}{\pgfqpoint{3.960440in}{3.272989in}}%
\pgfpathcurveto{\pgfqpoint{3.960440in}{3.278813in}}{\pgfqpoint{3.958126in}{3.284399in}}{\pgfqpoint{3.954008in}{3.288517in}}%
\pgfpathcurveto{\pgfqpoint{3.949890in}{3.292635in}}{\pgfqpoint{3.944304in}{3.294949in}}{\pgfqpoint{3.938480in}{3.294949in}}%
\pgfpathcurveto{\pgfqpoint{3.932656in}{3.294949in}}{\pgfqpoint{3.927070in}{3.292635in}}{\pgfqpoint{3.922952in}{3.288517in}}%
\pgfpathcurveto{\pgfqpoint{3.918833in}{3.284399in}}{\pgfqpoint{3.916520in}{3.278813in}}{\pgfqpoint{3.916520in}{3.272989in}}%
\pgfpathcurveto{\pgfqpoint{3.916520in}{3.267165in}}{\pgfqpoint{3.918833in}{3.261579in}}{\pgfqpoint{3.922952in}{3.257461in}}%
\pgfpathcurveto{\pgfqpoint{3.927070in}{3.253343in}}{\pgfqpoint{3.932656in}{3.251029in}}{\pgfqpoint{3.938480in}{3.251029in}}%
\pgfpathlineto{\pgfqpoint{3.938480in}{3.251029in}}%
\pgfpathclose%
\pgfusepath{stroke,fill}%
\end{pgfscope}%
\begin{pgfscope}%
\pgfpathrectangle{\pgfqpoint{1.542338in}{0.880000in}}{\pgfqpoint{5.115323in}{6.160000in}}%
\pgfusepath{clip}%
\pgfsetbuttcap%
\pgfsetroundjoin%
\definecolor{currentfill}{rgb}{0.200000,0.200000,0.800000}%
\pgfsetfillcolor{currentfill}%
\pgfsetlinewidth{1.003750pt}%
\definecolor{currentstroke}{rgb}{0.200000,0.200000,0.800000}%
\pgfsetstrokecolor{currentstroke}%
\pgfsetdash{}{0pt}%
\pgfpathmoveto{\pgfqpoint{3.837150in}{3.367103in}}%
\pgfpathcurveto{\pgfqpoint{3.842974in}{3.367103in}}{\pgfqpoint{3.848560in}{3.369416in}}{\pgfqpoint{3.852679in}{3.373535in}}%
\pgfpathcurveto{\pgfqpoint{3.856797in}{3.377653in}}{\pgfqpoint{3.859111in}{3.383239in}}{\pgfqpoint{3.859111in}{3.389063in}}%
\pgfpathcurveto{\pgfqpoint{3.859111in}{3.394887in}}{\pgfqpoint{3.856797in}{3.400473in}}{\pgfqpoint{3.852679in}{3.404591in}}%
\pgfpathcurveto{\pgfqpoint{3.848560in}{3.408709in}}{\pgfqpoint{3.842974in}{3.411023in}}{\pgfqpoint{3.837150in}{3.411023in}}%
\pgfpathcurveto{\pgfqpoint{3.831326in}{3.411023in}}{\pgfqpoint{3.825740in}{3.408709in}}{\pgfqpoint{3.821622in}{3.404591in}}%
\pgfpathcurveto{\pgfqpoint{3.817504in}{3.400473in}}{\pgfqpoint{3.815190in}{3.394887in}}{\pgfqpoint{3.815190in}{3.389063in}}%
\pgfpathcurveto{\pgfqpoint{3.815190in}{3.383239in}}{\pgfqpoint{3.817504in}{3.377653in}}{\pgfqpoint{3.821622in}{3.373535in}}%
\pgfpathcurveto{\pgfqpoint{3.825740in}{3.369416in}}{\pgfqpoint{3.831326in}{3.367103in}}{\pgfqpoint{3.837150in}{3.367103in}}%
\pgfpathlineto{\pgfqpoint{3.837150in}{3.367103in}}%
\pgfpathclose%
\pgfusepath{stroke,fill}%
\end{pgfscope}%
\begin{pgfscope}%
\pgfpathrectangle{\pgfqpoint{1.542338in}{0.880000in}}{\pgfqpoint{5.115323in}{6.160000in}}%
\pgfusepath{clip}%
\pgfsetbuttcap%
\pgfsetroundjoin%
\definecolor{currentfill}{rgb}{0.200000,0.200000,0.800000}%
\pgfsetfillcolor{currentfill}%
\pgfsetlinewidth{1.003750pt}%
\definecolor{currentstroke}{rgb}{0.200000,0.200000,0.800000}%
\pgfsetstrokecolor{currentstroke}%
\pgfsetdash{}{0pt}%
\pgfpathmoveto{\pgfqpoint{3.719259in}{3.465674in}}%
\pgfpathcurveto{\pgfqpoint{3.725083in}{3.465674in}}{\pgfqpoint{3.730669in}{3.467988in}}{\pgfqpoint{3.734787in}{3.472106in}}%
\pgfpathcurveto{\pgfqpoint{3.738905in}{3.476224in}}{\pgfqpoint{3.741219in}{3.481810in}}{\pgfqpoint{3.741219in}{3.487634in}}%
\pgfpathcurveto{\pgfqpoint{3.741219in}{3.493458in}}{\pgfqpoint{3.738905in}{3.499044in}}{\pgfqpoint{3.734787in}{3.503162in}}%
\pgfpathcurveto{\pgfqpoint{3.730669in}{3.507281in}}{\pgfqpoint{3.725083in}{3.509594in}}{\pgfqpoint{3.719259in}{3.509594in}}%
\pgfpathcurveto{\pgfqpoint{3.713435in}{3.509594in}}{\pgfqpoint{3.707849in}{3.507281in}}{\pgfqpoint{3.703731in}{3.503162in}}%
\pgfpathcurveto{\pgfqpoint{3.699613in}{3.499044in}}{\pgfqpoint{3.697299in}{3.493458in}}{\pgfqpoint{3.697299in}{3.487634in}}%
\pgfpathcurveto{\pgfqpoint{3.697299in}{3.481810in}}{\pgfqpoint{3.699613in}{3.476224in}}{\pgfqpoint{3.703731in}{3.472106in}}%
\pgfpathcurveto{\pgfqpoint{3.707849in}{3.467988in}}{\pgfqpoint{3.713435in}{3.465674in}}{\pgfqpoint{3.719259in}{3.465674in}}%
\pgfpathlineto{\pgfqpoint{3.719259in}{3.465674in}}%
\pgfpathclose%
\pgfusepath{stroke,fill}%
\end{pgfscope}%
\begin{pgfscope}%
\pgfpathrectangle{\pgfqpoint{1.542338in}{0.880000in}}{\pgfqpoint{5.115323in}{6.160000in}}%
\pgfusepath{clip}%
\pgfsetbuttcap%
\pgfsetroundjoin%
\definecolor{currentfill}{rgb}{0.200000,0.200000,0.800000}%
\pgfsetfillcolor{currentfill}%
\pgfsetlinewidth{1.003750pt}%
\definecolor{currentstroke}{rgb}{0.200000,0.200000,0.800000}%
\pgfsetstrokecolor{currentstroke}%
\pgfsetdash{}{0pt}%
\pgfpathmoveto{\pgfqpoint{3.596055in}{3.557238in}}%
\pgfpathcurveto{\pgfqpoint{3.601879in}{3.557238in}}{\pgfqpoint{3.607465in}{3.559552in}}{\pgfqpoint{3.611583in}{3.563670in}}%
\pgfpathcurveto{\pgfqpoint{3.615701in}{3.567789in}}{\pgfqpoint{3.618015in}{3.573375in}}{\pgfqpoint{3.618015in}{3.579199in}}%
\pgfpathcurveto{\pgfqpoint{3.618015in}{3.585023in}}{\pgfqpoint{3.615701in}{3.590609in}}{\pgfqpoint{3.611583in}{3.594727in}}%
\pgfpathcurveto{\pgfqpoint{3.607465in}{3.598845in}}{\pgfqpoint{3.601879in}{3.601159in}}{\pgfqpoint{3.596055in}{3.601159in}}%
\pgfpathcurveto{\pgfqpoint{3.590231in}{3.601159in}}{\pgfqpoint{3.584644in}{3.598845in}}{\pgfqpoint{3.580526in}{3.594727in}}%
\pgfpathcurveto{\pgfqpoint{3.576408in}{3.590609in}}{\pgfqpoint{3.574094in}{3.585023in}}{\pgfqpoint{3.574094in}{3.579199in}}%
\pgfpathcurveto{\pgfqpoint{3.574094in}{3.573375in}}{\pgfqpoint{3.576408in}{3.567789in}}{\pgfqpoint{3.580526in}{3.563670in}}%
\pgfpathcurveto{\pgfqpoint{3.584644in}{3.559552in}}{\pgfqpoint{3.590231in}{3.557238in}}{\pgfqpoint{3.596055in}{3.557238in}}%
\pgfpathlineto{\pgfqpoint{3.596055in}{3.557238in}}%
\pgfpathclose%
\pgfusepath{stroke,fill}%
\end{pgfscope}%
\begin{pgfscope}%
\pgfpathrectangle{\pgfqpoint{1.542338in}{0.880000in}}{\pgfqpoint{5.115323in}{6.160000in}}%
\pgfusepath{clip}%
\pgfsetbuttcap%
\pgfsetroundjoin%
\definecolor{currentfill}{rgb}{0.200000,0.200000,0.800000}%
\pgfsetfillcolor{currentfill}%
\pgfsetlinewidth{1.003750pt}%
\definecolor{currentstroke}{rgb}{0.200000,0.200000,0.800000}%
\pgfsetstrokecolor{currentstroke}%
\pgfsetdash{}{0pt}%
\pgfpathmoveto{\pgfqpoint{3.463304in}{3.636358in}}%
\pgfpathcurveto{\pgfqpoint{3.469128in}{3.636358in}}{\pgfqpoint{3.474714in}{3.638671in}}{\pgfqpoint{3.478833in}{3.642790in}}%
\pgfpathcurveto{\pgfqpoint{3.482951in}{3.646908in}}{\pgfqpoint{3.485265in}{3.652494in}}{\pgfqpoint{3.485265in}{3.658318in}}%
\pgfpathcurveto{\pgfqpoint{3.485265in}{3.664142in}}{\pgfqpoint{3.482951in}{3.669728in}}{\pgfqpoint{3.478833in}{3.673846in}}%
\pgfpathcurveto{\pgfqpoint{3.474714in}{3.677964in}}{\pgfqpoint{3.469128in}{3.680278in}}{\pgfqpoint{3.463304in}{3.680278in}}%
\pgfpathcurveto{\pgfqpoint{3.457480in}{3.680278in}}{\pgfqpoint{3.451894in}{3.677964in}}{\pgfqpoint{3.447776in}{3.673846in}}%
\pgfpathcurveto{\pgfqpoint{3.443658in}{3.669728in}}{\pgfqpoint{3.441344in}{3.664142in}}{\pgfqpoint{3.441344in}{3.658318in}}%
\pgfpathcurveto{\pgfqpoint{3.441344in}{3.652494in}}{\pgfqpoint{3.443658in}{3.646908in}}{\pgfqpoint{3.447776in}{3.642790in}}%
\pgfpathcurveto{\pgfqpoint{3.451894in}{3.638671in}}{\pgfqpoint{3.457480in}{3.636358in}}{\pgfqpoint{3.463304in}{3.636358in}}%
\pgfpathlineto{\pgfqpoint{3.463304in}{3.636358in}}%
\pgfpathclose%
\pgfusepath{stroke,fill}%
\end{pgfscope}%
\begin{pgfscope}%
\pgfpathrectangle{\pgfqpoint{1.542338in}{0.880000in}}{\pgfqpoint{5.115323in}{6.160000in}}%
\pgfusepath{clip}%
\pgfsetbuttcap%
\pgfsetroundjoin%
\definecolor{currentfill}{rgb}{0.200000,0.200000,0.800000}%
\pgfsetfillcolor{currentfill}%
\pgfsetlinewidth{1.003750pt}%
\definecolor{currentstroke}{rgb}{0.200000,0.200000,0.800000}%
\pgfsetstrokecolor{currentstroke}%
\pgfsetdash{}{0pt}%
\pgfpathmoveto{\pgfqpoint{3.314676in}{3.678709in}}%
\pgfpathcurveto{\pgfqpoint{3.320500in}{3.678709in}}{\pgfqpoint{3.326086in}{3.681022in}}{\pgfqpoint{3.330204in}{3.685141in}}%
\pgfpathcurveto{\pgfqpoint{3.334322in}{3.689259in}}{\pgfqpoint{3.336636in}{3.694845in}}{\pgfqpoint{3.336636in}{3.700669in}}%
\pgfpathcurveto{\pgfqpoint{3.336636in}{3.706493in}}{\pgfqpoint{3.334322in}{3.712079in}}{\pgfqpoint{3.330204in}{3.716197in}}%
\pgfpathcurveto{\pgfqpoint{3.326086in}{3.720315in}}{\pgfqpoint{3.320500in}{3.722629in}}{\pgfqpoint{3.314676in}{3.722629in}}%
\pgfpathcurveto{\pgfqpoint{3.308852in}{3.722629in}}{\pgfqpoint{3.303266in}{3.720315in}}{\pgfqpoint{3.299148in}{3.716197in}}%
\pgfpathcurveto{\pgfqpoint{3.295030in}{3.712079in}}{\pgfqpoint{3.292716in}{3.706493in}}{\pgfqpoint{3.292716in}{3.700669in}}%
\pgfpathcurveto{\pgfqpoint{3.292716in}{3.694845in}}{\pgfqpoint{3.295030in}{3.689259in}}{\pgfqpoint{3.299148in}{3.685141in}}%
\pgfpathcurveto{\pgfqpoint{3.303266in}{3.681022in}}{\pgfqpoint{3.308852in}{3.678709in}}{\pgfqpoint{3.314676in}{3.678709in}}%
\pgfpathlineto{\pgfqpoint{3.314676in}{3.678709in}}%
\pgfpathclose%
\pgfusepath{stroke,fill}%
\end{pgfscope}%
\begin{pgfscope}%
\pgfpathrectangle{\pgfqpoint{1.542338in}{0.880000in}}{\pgfqpoint{5.115323in}{6.160000in}}%
\pgfusepath{clip}%
\pgfsetbuttcap%
\pgfsetroundjoin%
\definecolor{currentfill}{rgb}{0.200000,0.200000,0.800000}%
\pgfsetfillcolor{currentfill}%
\pgfsetlinewidth{1.003750pt}%
\definecolor{currentstroke}{rgb}{0.200000,0.200000,0.800000}%
\pgfsetstrokecolor{currentstroke}%
\pgfsetdash{}{0pt}%
\pgfpathmoveto{\pgfqpoint{3.164683in}{3.709912in}}%
\pgfpathcurveto{\pgfqpoint{3.170507in}{3.709912in}}{\pgfqpoint{3.176093in}{3.712226in}}{\pgfqpoint{3.180211in}{3.716344in}}%
\pgfpathcurveto{\pgfqpoint{3.184329in}{3.720462in}}{\pgfqpoint{3.186643in}{3.726048in}}{\pgfqpoint{3.186643in}{3.731872in}}%
\pgfpathcurveto{\pgfqpoint{3.186643in}{3.737696in}}{\pgfqpoint{3.184329in}{3.743282in}}{\pgfqpoint{3.180211in}{3.747400in}}%
\pgfpathcurveto{\pgfqpoint{3.176093in}{3.751518in}}{\pgfqpoint{3.170507in}{3.753832in}}{\pgfqpoint{3.164683in}{3.753832in}}%
\pgfpathcurveto{\pgfqpoint{3.158859in}{3.753832in}}{\pgfqpoint{3.153273in}{3.751518in}}{\pgfqpoint{3.149155in}{3.747400in}}%
\pgfpathcurveto{\pgfqpoint{3.145037in}{3.743282in}}{\pgfqpoint{3.142723in}{3.737696in}}{\pgfqpoint{3.142723in}{3.731872in}}%
\pgfpathcurveto{\pgfqpoint{3.142723in}{3.726048in}}{\pgfqpoint{3.145037in}{3.720462in}}{\pgfqpoint{3.149155in}{3.716344in}}%
\pgfpathcurveto{\pgfqpoint{3.153273in}{3.712226in}}{\pgfqpoint{3.158859in}{3.709912in}}{\pgfqpoint{3.164683in}{3.709912in}}%
\pgfpathlineto{\pgfqpoint{3.164683in}{3.709912in}}%
\pgfpathclose%
\pgfusepath{stroke,fill}%
\end{pgfscope}%
\begin{pgfscope}%
\pgfpathrectangle{\pgfqpoint{1.542338in}{0.880000in}}{\pgfqpoint{5.115323in}{6.160000in}}%
\pgfusepath{clip}%
\pgfsetbuttcap%
\pgfsetroundjoin%
\definecolor{currentfill}{rgb}{0.200000,0.200000,0.800000}%
\pgfsetfillcolor{currentfill}%
\pgfsetlinewidth{1.003750pt}%
\definecolor{currentstroke}{rgb}{0.200000,0.200000,0.800000}%
\pgfsetstrokecolor{currentstroke}%
\pgfsetdash{}{0pt}%
\pgfpathmoveto{\pgfqpoint{3.012507in}{3.727832in}}%
\pgfpathcurveto{\pgfqpoint{3.018331in}{3.727832in}}{\pgfqpoint{3.023917in}{3.730146in}}{\pgfqpoint{3.028036in}{3.734264in}}%
\pgfpathcurveto{\pgfqpoint{3.032154in}{3.738382in}}{\pgfqpoint{3.034468in}{3.743968in}}{\pgfqpoint{3.034468in}{3.749792in}}%
\pgfpathcurveto{\pgfqpoint{3.034468in}{3.755616in}}{\pgfqpoint{3.032154in}{3.761202in}}{\pgfqpoint{3.028036in}{3.765321in}}%
\pgfpathcurveto{\pgfqpoint{3.023917in}{3.769439in}}{\pgfqpoint{3.018331in}{3.771753in}}{\pgfqpoint{3.012507in}{3.771753in}}%
\pgfpathcurveto{\pgfqpoint{3.006683in}{3.771753in}}{\pgfqpoint{3.001097in}{3.769439in}}{\pgfqpoint{2.996979in}{3.765321in}}%
\pgfpathcurveto{\pgfqpoint{2.992861in}{3.761202in}}{\pgfqpoint{2.990547in}{3.755616in}}{\pgfqpoint{2.990547in}{3.749792in}}%
\pgfpathcurveto{\pgfqpoint{2.990547in}{3.743968in}}{\pgfqpoint{2.992861in}{3.738382in}}{\pgfqpoint{2.996979in}{3.734264in}}%
\pgfpathcurveto{\pgfqpoint{3.001097in}{3.730146in}}{\pgfqpoint{3.006683in}{3.727832in}}{\pgfqpoint{3.012507in}{3.727832in}}%
\pgfpathlineto{\pgfqpoint{3.012507in}{3.727832in}}%
\pgfpathclose%
\pgfusepath{stroke,fill}%
\end{pgfscope}%
\begin{pgfscope}%
\pgfpathrectangle{\pgfqpoint{1.542338in}{0.880000in}}{\pgfqpoint{5.115323in}{6.160000in}}%
\pgfusepath{clip}%
\pgfsetbuttcap%
\pgfsetroundjoin%
\definecolor{currentfill}{rgb}{0.200000,0.200000,0.800000}%
\pgfsetfillcolor{currentfill}%
\pgfsetlinewidth{1.003750pt}%
\definecolor{currentstroke}{rgb}{0.200000,0.200000,0.800000}%
\pgfsetstrokecolor{currentstroke}%
\pgfsetdash{}{0pt}%
\pgfpathmoveto{\pgfqpoint{2.859157in}{3.723212in}}%
\pgfpathcurveto{\pgfqpoint{2.864981in}{3.723212in}}{\pgfqpoint{2.870568in}{3.725525in}}{\pgfqpoint{2.874686in}{3.729644in}}%
\pgfpathcurveto{\pgfqpoint{2.878804in}{3.733762in}}{\pgfqpoint{2.881118in}{3.739348in}}{\pgfqpoint{2.881118in}{3.745172in}}%
\pgfpathcurveto{\pgfqpoint{2.881118in}{3.750996in}}{\pgfqpoint{2.878804in}{3.756582in}}{\pgfqpoint{2.874686in}{3.760700in}}%
\pgfpathcurveto{\pgfqpoint{2.870568in}{3.764818in}}{\pgfqpoint{2.864981in}{3.767132in}}{\pgfqpoint{2.859157in}{3.767132in}}%
\pgfpathcurveto{\pgfqpoint{2.853333in}{3.767132in}}{\pgfqpoint{2.847747in}{3.764818in}}{\pgfqpoint{2.843629in}{3.760700in}}%
\pgfpathcurveto{\pgfqpoint{2.839511in}{3.756582in}}{\pgfqpoint{2.837197in}{3.750996in}}{\pgfqpoint{2.837197in}{3.745172in}}%
\pgfpathcurveto{\pgfqpoint{2.837197in}{3.739348in}}{\pgfqpoint{2.839511in}{3.733762in}}{\pgfqpoint{2.843629in}{3.729644in}}%
\pgfpathcurveto{\pgfqpoint{2.847747in}{3.725525in}}{\pgfqpoint{2.853333in}{3.723212in}}{\pgfqpoint{2.859157in}{3.723212in}}%
\pgfpathlineto{\pgfqpoint{2.859157in}{3.723212in}}%
\pgfpathclose%
\pgfusepath{stroke,fill}%
\end{pgfscope}%
\begin{pgfscope}%
\pgfpathrectangle{\pgfqpoint{1.542338in}{0.880000in}}{\pgfqpoint{5.115323in}{6.160000in}}%
\pgfusepath{clip}%
\pgfsetbuttcap%
\pgfsetroundjoin%
\definecolor{currentfill}{rgb}{0.200000,0.200000,0.800000}%
\pgfsetfillcolor{currentfill}%
\pgfsetlinewidth{1.003750pt}%
\definecolor{currentstroke}{rgb}{0.200000,0.200000,0.800000}%
\pgfsetstrokecolor{currentstroke}%
\pgfsetdash{}{0pt}%
\pgfpathmoveto{\pgfqpoint{2.708375in}{3.695705in}}%
\pgfpathcurveto{\pgfqpoint{2.714199in}{3.695705in}}{\pgfqpoint{2.719785in}{3.698019in}}{\pgfqpoint{2.723903in}{3.702137in}}%
\pgfpathcurveto{\pgfqpoint{2.728021in}{3.706255in}}{\pgfqpoint{2.730335in}{3.711841in}}{\pgfqpoint{2.730335in}{3.717665in}}%
\pgfpathcurveto{\pgfqpoint{2.730335in}{3.723489in}}{\pgfqpoint{2.728021in}{3.729075in}}{\pgfqpoint{2.723903in}{3.733193in}}%
\pgfpathcurveto{\pgfqpoint{2.719785in}{3.737312in}}{\pgfqpoint{2.714199in}{3.739626in}}{\pgfqpoint{2.708375in}{3.739626in}}%
\pgfpathcurveto{\pgfqpoint{2.702551in}{3.739626in}}{\pgfqpoint{2.696965in}{3.737312in}}{\pgfqpoint{2.692846in}{3.733193in}}%
\pgfpathcurveto{\pgfqpoint{2.688728in}{3.729075in}}{\pgfqpoint{2.686414in}{3.723489in}}{\pgfqpoint{2.686414in}{3.717665in}}%
\pgfpathcurveto{\pgfqpoint{2.686414in}{3.711841in}}{\pgfqpoint{2.688728in}{3.706255in}}{\pgfqpoint{2.692846in}{3.702137in}}%
\pgfpathcurveto{\pgfqpoint{2.696965in}{3.698019in}}{\pgfqpoint{2.702551in}{3.695705in}}{\pgfqpoint{2.708375in}{3.695705in}}%
\pgfpathlineto{\pgfqpoint{2.708375in}{3.695705in}}%
\pgfpathclose%
\pgfusepath{stroke,fill}%
\end{pgfscope}%
\begin{pgfscope}%
\pgfpathrectangle{\pgfqpoint{1.542338in}{0.880000in}}{\pgfqpoint{5.115323in}{6.160000in}}%
\pgfusepath{clip}%
\pgfsetbuttcap%
\pgfsetroundjoin%
\definecolor{currentfill}{rgb}{0.200000,0.200000,0.800000}%
\pgfsetfillcolor{currentfill}%
\pgfsetlinewidth{1.003750pt}%
\definecolor{currentstroke}{rgb}{0.200000,0.200000,0.800000}%
\pgfsetstrokecolor{currentstroke}%
\pgfsetdash{}{0pt}%
\pgfpathmoveto{\pgfqpoint{2.560319in}{3.655800in}}%
\pgfpathcurveto{\pgfqpoint{2.566143in}{3.655800in}}{\pgfqpoint{2.571729in}{3.658114in}}{\pgfqpoint{2.575847in}{3.662232in}}%
\pgfpathcurveto{\pgfqpoint{2.579965in}{3.666350in}}{\pgfqpoint{2.582279in}{3.671937in}}{\pgfqpoint{2.582279in}{3.677761in}}%
\pgfpathcurveto{\pgfqpoint{2.582279in}{3.683585in}}{\pgfqpoint{2.579965in}{3.689171in}}{\pgfqpoint{2.575847in}{3.693289in}}%
\pgfpathcurveto{\pgfqpoint{2.571729in}{3.697407in}}{\pgfqpoint{2.566143in}{3.699721in}}{\pgfqpoint{2.560319in}{3.699721in}}%
\pgfpathcurveto{\pgfqpoint{2.554495in}{3.699721in}}{\pgfqpoint{2.548909in}{3.697407in}}{\pgfqpoint{2.544791in}{3.693289in}}%
\pgfpathcurveto{\pgfqpoint{2.540673in}{3.689171in}}{\pgfqpoint{2.538359in}{3.683585in}}{\pgfqpoint{2.538359in}{3.677761in}}%
\pgfpathcurveto{\pgfqpoint{2.538359in}{3.671937in}}{\pgfqpoint{2.540673in}{3.666350in}}{\pgfqpoint{2.544791in}{3.662232in}}%
\pgfpathcurveto{\pgfqpoint{2.548909in}{3.658114in}}{\pgfqpoint{2.554495in}{3.655800in}}{\pgfqpoint{2.560319in}{3.655800in}}%
\pgfpathlineto{\pgfqpoint{2.560319in}{3.655800in}}%
\pgfpathclose%
\pgfusepath{stroke,fill}%
\end{pgfscope}%
\begin{pgfscope}%
\pgfpathrectangle{\pgfqpoint{1.542338in}{0.880000in}}{\pgfqpoint{5.115323in}{6.160000in}}%
\pgfusepath{clip}%
\pgfsetbuttcap%
\pgfsetroundjoin%
\definecolor{currentfill}{rgb}{0.200000,0.200000,0.800000}%
\pgfsetfillcolor{currentfill}%
\pgfsetlinewidth{1.003750pt}%
\definecolor{currentstroke}{rgb}{0.200000,0.200000,0.800000}%
\pgfsetstrokecolor{currentstroke}%
\pgfsetdash{}{0pt}%
\pgfpathmoveto{\pgfqpoint{2.423070in}{3.587614in}}%
\pgfpathcurveto{\pgfqpoint{2.428894in}{3.587614in}}{\pgfqpoint{2.434481in}{3.589928in}}{\pgfqpoint{2.438599in}{3.594046in}}%
\pgfpathcurveto{\pgfqpoint{2.442717in}{3.598164in}}{\pgfqpoint{2.445031in}{3.603750in}}{\pgfqpoint{2.445031in}{3.609574in}}%
\pgfpathcurveto{\pgfqpoint{2.445031in}{3.615398in}}{\pgfqpoint{2.442717in}{3.620984in}}{\pgfqpoint{2.438599in}{3.625102in}}%
\pgfpathcurveto{\pgfqpoint{2.434481in}{3.629221in}}{\pgfqpoint{2.428894in}{3.631534in}}{\pgfqpoint{2.423070in}{3.631534in}}%
\pgfpathcurveto{\pgfqpoint{2.417247in}{3.631534in}}{\pgfqpoint{2.411660in}{3.629221in}}{\pgfqpoint{2.407542in}{3.625102in}}%
\pgfpathcurveto{\pgfqpoint{2.403424in}{3.620984in}}{\pgfqpoint{2.401110in}{3.615398in}}{\pgfqpoint{2.401110in}{3.609574in}}%
\pgfpathcurveto{\pgfqpoint{2.401110in}{3.603750in}}{\pgfqpoint{2.403424in}{3.598164in}}{\pgfqpoint{2.407542in}{3.594046in}}%
\pgfpathcurveto{\pgfqpoint{2.411660in}{3.589928in}}{\pgfqpoint{2.417247in}{3.587614in}}{\pgfqpoint{2.423070in}{3.587614in}}%
\pgfpathlineto{\pgfqpoint{2.423070in}{3.587614in}}%
\pgfpathclose%
\pgfusepath{stroke,fill}%
\end{pgfscope}%
\begin{pgfscope}%
\pgfpathrectangle{\pgfqpoint{1.542338in}{0.880000in}}{\pgfqpoint{5.115323in}{6.160000in}}%
\pgfusepath{clip}%
\pgfsetbuttcap%
\pgfsetroundjoin%
\definecolor{currentfill}{rgb}{0.200000,0.200000,0.800000}%
\pgfsetfillcolor{currentfill}%
\pgfsetlinewidth{1.003750pt}%
\definecolor{currentstroke}{rgb}{0.200000,0.200000,0.800000}%
\pgfsetstrokecolor{currentstroke}%
\pgfsetdash{}{0pt}%
\pgfpathmoveto{\pgfqpoint{2.291167in}{3.510412in}}%
\pgfpathcurveto{\pgfqpoint{2.296990in}{3.510412in}}{\pgfqpoint{2.302577in}{3.512726in}}{\pgfqpoint{2.306695in}{3.516844in}}%
\pgfpathcurveto{\pgfqpoint{2.310813in}{3.520962in}}{\pgfqpoint{2.313127in}{3.526548in}}{\pgfqpoint{2.313127in}{3.532372in}}%
\pgfpathcurveto{\pgfqpoint{2.313127in}{3.538196in}}{\pgfqpoint{2.310813in}{3.543782in}}{\pgfqpoint{2.306695in}{3.547900in}}%
\pgfpathcurveto{\pgfqpoint{2.302577in}{3.552018in}}{\pgfqpoint{2.296990in}{3.554332in}}{\pgfqpoint{2.291167in}{3.554332in}}%
\pgfpathcurveto{\pgfqpoint{2.285343in}{3.554332in}}{\pgfqpoint{2.279756in}{3.552018in}}{\pgfqpoint{2.275638in}{3.547900in}}%
\pgfpathcurveto{\pgfqpoint{2.271520in}{3.543782in}}{\pgfqpoint{2.269206in}{3.538196in}}{\pgfqpoint{2.269206in}{3.532372in}}%
\pgfpathcurveto{\pgfqpoint{2.269206in}{3.526548in}}{\pgfqpoint{2.271520in}{3.520962in}}{\pgfqpoint{2.275638in}{3.516844in}}%
\pgfpathcurveto{\pgfqpoint{2.279756in}{3.512726in}}{\pgfqpoint{2.285343in}{3.510412in}}{\pgfqpoint{2.291167in}{3.510412in}}%
\pgfpathlineto{\pgfqpoint{2.291167in}{3.510412in}}%
\pgfpathclose%
\pgfusepath{stroke,fill}%
\end{pgfscope}%
\begin{pgfscope}%
\pgfpathrectangle{\pgfqpoint{1.542338in}{0.880000in}}{\pgfqpoint{5.115323in}{6.160000in}}%
\pgfusepath{clip}%
\pgfsetbuttcap%
\pgfsetroundjoin%
\definecolor{currentfill}{rgb}{0.200000,0.200000,0.800000}%
\pgfsetfillcolor{currentfill}%
\pgfsetlinewidth{1.003750pt}%
\definecolor{currentstroke}{rgb}{0.200000,0.200000,0.800000}%
\pgfsetstrokecolor{currentstroke}%
\pgfsetdash{}{0pt}%
\pgfpathmoveto{\pgfqpoint{2.167728in}{3.419268in}}%
\pgfpathcurveto{\pgfqpoint{2.173552in}{3.419268in}}{\pgfqpoint{2.179138in}{3.421582in}}{\pgfqpoint{2.183256in}{3.425700in}}%
\pgfpathcurveto{\pgfqpoint{2.187374in}{3.429818in}}{\pgfqpoint{2.189688in}{3.435405in}}{\pgfqpoint{2.189688in}{3.441228in}}%
\pgfpathcurveto{\pgfqpoint{2.189688in}{3.447052in}}{\pgfqpoint{2.187374in}{3.452639in}}{\pgfqpoint{2.183256in}{3.456757in}}%
\pgfpathcurveto{\pgfqpoint{2.179138in}{3.460875in}}{\pgfqpoint{2.173552in}{3.463189in}}{\pgfqpoint{2.167728in}{3.463189in}}%
\pgfpathcurveto{\pgfqpoint{2.161904in}{3.463189in}}{\pgfqpoint{2.156318in}{3.460875in}}{\pgfqpoint{2.152200in}{3.456757in}}%
\pgfpathcurveto{\pgfqpoint{2.148081in}{3.452639in}}{\pgfqpoint{2.145768in}{3.447052in}}{\pgfqpoint{2.145768in}{3.441228in}}%
\pgfpathcurveto{\pgfqpoint{2.145768in}{3.435405in}}{\pgfqpoint{2.148081in}{3.429818in}}{\pgfqpoint{2.152200in}{3.425700in}}%
\pgfpathcurveto{\pgfqpoint{2.156318in}{3.421582in}}{\pgfqpoint{2.161904in}{3.419268in}}{\pgfqpoint{2.167728in}{3.419268in}}%
\pgfpathlineto{\pgfqpoint{2.167728in}{3.419268in}}%
\pgfpathclose%
\pgfusepath{stroke,fill}%
\end{pgfscope}%
\begin{pgfscope}%
\pgfpathrectangle{\pgfqpoint{1.542338in}{0.880000in}}{\pgfqpoint{5.115323in}{6.160000in}}%
\pgfusepath{clip}%
\pgfsetbuttcap%
\pgfsetroundjoin%
\definecolor{currentfill}{rgb}{0.200000,0.200000,0.800000}%
\pgfsetfillcolor{currentfill}%
\pgfsetlinewidth{1.003750pt}%
\definecolor{currentstroke}{rgb}{0.200000,0.200000,0.800000}%
\pgfsetstrokecolor{currentstroke}%
\pgfsetdash{}{0pt}%
\pgfpathmoveto{\pgfqpoint{2.068482in}{3.302317in}}%
\pgfpathcurveto{\pgfqpoint{2.074306in}{3.302317in}}{\pgfqpoint{2.079892in}{3.304631in}}{\pgfqpoint{2.084011in}{3.308749in}}%
\pgfpathcurveto{\pgfqpoint{2.088129in}{3.312867in}}{\pgfqpoint{2.090443in}{3.318453in}}{\pgfqpoint{2.090443in}{3.324277in}}%
\pgfpathcurveto{\pgfqpoint{2.090443in}{3.330101in}}{\pgfqpoint{2.088129in}{3.335687in}}{\pgfqpoint{2.084011in}{3.339806in}}%
\pgfpathcurveto{\pgfqpoint{2.079892in}{3.343924in}}{\pgfqpoint{2.074306in}{3.346238in}}{\pgfqpoint{2.068482in}{3.346238in}}%
\pgfpathcurveto{\pgfqpoint{2.062658in}{3.346238in}}{\pgfqpoint{2.057072in}{3.343924in}}{\pgfqpoint{2.052954in}{3.339806in}}%
\pgfpathcurveto{\pgfqpoint{2.048836in}{3.335687in}}{\pgfqpoint{2.046522in}{3.330101in}}{\pgfqpoint{2.046522in}{3.324277in}}%
\pgfpathcurveto{\pgfqpoint{2.046522in}{3.318453in}}{\pgfqpoint{2.048836in}{3.312867in}}{\pgfqpoint{2.052954in}{3.308749in}}%
\pgfpathcurveto{\pgfqpoint{2.057072in}{3.304631in}}{\pgfqpoint{2.062658in}{3.302317in}}{\pgfqpoint{2.068482in}{3.302317in}}%
\pgfpathlineto{\pgfqpoint{2.068482in}{3.302317in}}%
\pgfpathclose%
\pgfusepath{stroke,fill}%
\end{pgfscope}%
\begin{pgfscope}%
\pgfpathrectangle{\pgfqpoint{1.542338in}{0.880000in}}{\pgfqpoint{5.115323in}{6.160000in}}%
\pgfusepath{clip}%
\pgfsetbuttcap%
\pgfsetroundjoin%
\definecolor{currentfill}{rgb}{0.200000,0.200000,0.800000}%
\pgfsetfillcolor{currentfill}%
\pgfsetlinewidth{1.003750pt}%
\definecolor{currentstroke}{rgb}{0.200000,0.200000,0.800000}%
\pgfsetstrokecolor{currentstroke}%
\pgfsetdash{}{0pt}%
\pgfpathmoveto{\pgfqpoint{1.968382in}{3.185996in}}%
\pgfpathcurveto{\pgfqpoint{1.974206in}{3.185996in}}{\pgfqpoint{1.979792in}{3.188309in}}{\pgfqpoint{1.983910in}{3.192428in}}%
\pgfpathcurveto{\pgfqpoint{1.988029in}{3.196546in}}{\pgfqpoint{1.990342in}{3.202132in}}{\pgfqpoint{1.990342in}{3.207956in}}%
\pgfpathcurveto{\pgfqpoint{1.990342in}{3.213780in}}{\pgfqpoint{1.988029in}{3.219366in}}{\pgfqpoint{1.983910in}{3.223484in}}%
\pgfpathcurveto{\pgfqpoint{1.979792in}{3.227602in}}{\pgfqpoint{1.974206in}{3.229916in}}{\pgfqpoint{1.968382in}{3.229916in}}%
\pgfpathcurveto{\pgfqpoint{1.962558in}{3.229916in}}{\pgfqpoint{1.956972in}{3.227602in}}{\pgfqpoint{1.952854in}{3.223484in}}%
\pgfpathcurveto{\pgfqpoint{1.948736in}{3.219366in}}{\pgfqpoint{1.946422in}{3.213780in}}{\pgfqpoint{1.946422in}{3.207956in}}%
\pgfpathcurveto{\pgfqpoint{1.946422in}{3.202132in}}{\pgfqpoint{1.948736in}{3.196546in}}{\pgfqpoint{1.952854in}{3.192428in}}%
\pgfpathcurveto{\pgfqpoint{1.956972in}{3.188309in}}{\pgfqpoint{1.962558in}{3.185996in}}{\pgfqpoint{1.968382in}{3.185996in}}%
\pgfpathlineto{\pgfqpoint{1.968382in}{3.185996in}}%
\pgfpathclose%
\pgfusepath{stroke,fill}%
\end{pgfscope}%
\begin{pgfscope}%
\pgfpathrectangle{\pgfqpoint{1.542338in}{0.880000in}}{\pgfqpoint{5.115323in}{6.160000in}}%
\pgfusepath{clip}%
\pgfsetbuttcap%
\pgfsetroundjoin%
\definecolor{currentfill}{rgb}{0.200000,0.200000,0.800000}%
\pgfsetfillcolor{currentfill}%
\pgfsetlinewidth{1.003750pt}%
\definecolor{currentstroke}{rgb}{0.200000,0.200000,0.800000}%
\pgfsetstrokecolor{currentstroke}%
\pgfsetdash{}{0pt}%
\pgfpathmoveto{\pgfqpoint{1.903465in}{3.046930in}}%
\pgfpathcurveto{\pgfqpoint{1.909289in}{3.046930in}}{\pgfqpoint{1.914875in}{3.049244in}}{\pgfqpoint{1.918993in}{3.053362in}}%
\pgfpathcurveto{\pgfqpoint{1.923112in}{3.057480in}}{\pgfqpoint{1.925425in}{3.063066in}}{\pgfqpoint{1.925425in}{3.068890in}}%
\pgfpathcurveto{\pgfqpoint{1.925425in}{3.074714in}}{\pgfqpoint{1.923112in}{3.080300in}}{\pgfqpoint{1.918993in}{3.084419in}}%
\pgfpathcurveto{\pgfqpoint{1.914875in}{3.088537in}}{\pgfqpoint{1.909289in}{3.090851in}}{\pgfqpoint{1.903465in}{3.090851in}}%
\pgfpathcurveto{\pgfqpoint{1.897641in}{3.090851in}}{\pgfqpoint{1.892055in}{3.088537in}}{\pgfqpoint{1.887937in}{3.084419in}}%
\pgfpathcurveto{\pgfqpoint{1.883819in}{3.080300in}}{\pgfqpoint{1.881505in}{3.074714in}}{\pgfqpoint{1.881505in}{3.068890in}}%
\pgfpathcurveto{\pgfqpoint{1.881505in}{3.063066in}}{\pgfqpoint{1.883819in}{3.057480in}}{\pgfqpoint{1.887937in}{3.053362in}}%
\pgfpathcurveto{\pgfqpoint{1.892055in}{3.049244in}}{\pgfqpoint{1.897641in}{3.046930in}}{\pgfqpoint{1.903465in}{3.046930in}}%
\pgfpathlineto{\pgfqpoint{1.903465in}{3.046930in}}%
\pgfpathclose%
\pgfusepath{stroke,fill}%
\end{pgfscope}%
\begin{pgfscope}%
\pgfpathrectangle{\pgfqpoint{1.542338in}{0.880000in}}{\pgfqpoint{5.115323in}{6.160000in}}%
\pgfusepath{clip}%
\pgfsetbuttcap%
\pgfsetroundjoin%
\definecolor{currentfill}{rgb}{0.200000,0.200000,0.800000}%
\pgfsetfillcolor{currentfill}%
\pgfsetlinewidth{1.003750pt}%
\definecolor{currentstroke}{rgb}{0.200000,0.200000,0.800000}%
\pgfsetstrokecolor{currentstroke}%
\pgfsetdash{}{0pt}%
\pgfpathmoveto{\pgfqpoint{1.845252in}{2.906137in}}%
\pgfpathcurveto{\pgfqpoint{1.851075in}{2.906137in}}{\pgfqpoint{1.856662in}{2.908451in}}{\pgfqpoint{1.860780in}{2.912569in}}%
\pgfpathcurveto{\pgfqpoint{1.864898in}{2.916687in}}{\pgfqpoint{1.867212in}{2.922274in}}{\pgfqpoint{1.867212in}{2.928098in}}%
\pgfpathcurveto{\pgfqpoint{1.867212in}{2.933922in}}{\pgfqpoint{1.864898in}{2.939508in}}{\pgfqpoint{1.860780in}{2.943626in}}%
\pgfpathcurveto{\pgfqpoint{1.856662in}{2.947744in}}{\pgfqpoint{1.851075in}{2.950058in}}{\pgfqpoint{1.845252in}{2.950058in}}%
\pgfpathcurveto{\pgfqpoint{1.839428in}{2.950058in}}{\pgfqpoint{1.833841in}{2.947744in}}{\pgfqpoint{1.829723in}{2.943626in}}%
\pgfpathcurveto{\pgfqpoint{1.825605in}{2.939508in}}{\pgfqpoint{1.823291in}{2.933922in}}{\pgfqpoint{1.823291in}{2.928098in}}%
\pgfpathcurveto{\pgfqpoint{1.823291in}{2.922274in}}{\pgfqpoint{1.825605in}{2.916687in}}{\pgfqpoint{1.829723in}{2.912569in}}%
\pgfpathcurveto{\pgfqpoint{1.833841in}{2.908451in}}{\pgfqpoint{1.839428in}{2.906137in}}{\pgfqpoint{1.845252in}{2.906137in}}%
\pgfpathlineto{\pgfqpoint{1.845252in}{2.906137in}}%
\pgfpathclose%
\pgfusepath{stroke,fill}%
\end{pgfscope}%
\begin{pgfscope}%
\pgfpathrectangle{\pgfqpoint{1.542338in}{0.880000in}}{\pgfqpoint{5.115323in}{6.160000in}}%
\pgfusepath{clip}%
\pgfsetbuttcap%
\pgfsetroundjoin%
\definecolor{currentfill}{rgb}{0.200000,0.200000,0.800000}%
\pgfsetfillcolor{currentfill}%
\pgfsetlinewidth{1.003750pt}%
\definecolor{currentstroke}{rgb}{0.200000,0.200000,0.800000}%
\pgfsetstrokecolor{currentstroke}%
\pgfsetdash{}{0pt}%
\pgfpathmoveto{\pgfqpoint{1.801733in}{2.759647in}}%
\pgfpathcurveto{\pgfqpoint{1.807557in}{2.759647in}}{\pgfqpoint{1.813144in}{2.761961in}}{\pgfqpoint{1.817262in}{2.766079in}}%
\pgfpathcurveto{\pgfqpoint{1.821380in}{2.770197in}}{\pgfqpoint{1.823694in}{2.775783in}}{\pgfqpoint{1.823694in}{2.781607in}}%
\pgfpathcurveto{\pgfqpoint{1.823694in}{2.787431in}}{\pgfqpoint{1.821380in}{2.793017in}}{\pgfqpoint{1.817262in}{2.797136in}}%
\pgfpathcurveto{\pgfqpoint{1.813144in}{2.801254in}}{\pgfqpoint{1.807557in}{2.803568in}}{\pgfqpoint{1.801733in}{2.803568in}}%
\pgfpathcurveto{\pgfqpoint{1.795909in}{2.803568in}}{\pgfqpoint{1.790323in}{2.801254in}}{\pgfqpoint{1.786205in}{2.797136in}}%
\pgfpathcurveto{\pgfqpoint{1.782087in}{2.793017in}}{\pgfqpoint{1.779773in}{2.787431in}}{\pgfqpoint{1.779773in}{2.781607in}}%
\pgfpathcurveto{\pgfqpoint{1.779773in}{2.775783in}}{\pgfqpoint{1.782087in}{2.770197in}}{\pgfqpoint{1.786205in}{2.766079in}}%
\pgfpathcurveto{\pgfqpoint{1.790323in}{2.761961in}}{\pgfqpoint{1.795909in}{2.759647in}}{\pgfqpoint{1.801733in}{2.759647in}}%
\pgfpathlineto{\pgfqpoint{1.801733in}{2.759647in}}%
\pgfpathclose%
\pgfusepath{stroke,fill}%
\end{pgfscope}%
\begin{pgfscope}%
\pgfpathrectangle{\pgfqpoint{1.542338in}{0.880000in}}{\pgfqpoint{5.115323in}{6.160000in}}%
\pgfusepath{clip}%
\pgfsetbuttcap%
\pgfsetroundjoin%
\definecolor{currentfill}{rgb}{0.200000,0.200000,0.800000}%
\pgfsetfillcolor{currentfill}%
\pgfsetlinewidth{1.003750pt}%
\definecolor{currentstroke}{rgb}{0.200000,0.200000,0.800000}%
\pgfsetstrokecolor{currentstroke}%
\pgfsetdash{}{0pt}%
\pgfpathmoveto{\pgfqpoint{1.779013in}{2.608051in}}%
\pgfpathcurveto{\pgfqpoint{1.784837in}{2.608051in}}{\pgfqpoint{1.790423in}{2.610365in}}{\pgfqpoint{1.794541in}{2.614483in}}%
\pgfpathcurveto{\pgfqpoint{1.798659in}{2.618601in}}{\pgfqpoint{1.800973in}{2.624187in}}{\pgfqpoint{1.800973in}{2.630011in}}%
\pgfpathcurveto{\pgfqpoint{1.800973in}{2.635835in}}{\pgfqpoint{1.798659in}{2.641421in}}{\pgfqpoint{1.794541in}{2.645539in}}%
\pgfpathcurveto{\pgfqpoint{1.790423in}{2.649658in}}{\pgfqpoint{1.784837in}{2.651971in}}{\pgfqpoint{1.779013in}{2.651971in}}%
\pgfpathcurveto{\pgfqpoint{1.773189in}{2.651971in}}{\pgfqpoint{1.767603in}{2.649658in}}{\pgfqpoint{1.763485in}{2.645539in}}%
\pgfpathcurveto{\pgfqpoint{1.759367in}{2.641421in}}{\pgfqpoint{1.757053in}{2.635835in}}{\pgfqpoint{1.757053in}{2.630011in}}%
\pgfpathcurveto{\pgfqpoint{1.757053in}{2.624187in}}{\pgfqpoint{1.759367in}{2.618601in}}{\pgfqpoint{1.763485in}{2.614483in}}%
\pgfpathcurveto{\pgfqpoint{1.767603in}{2.610365in}}{\pgfqpoint{1.773189in}{2.608051in}}{\pgfqpoint{1.779013in}{2.608051in}}%
\pgfpathlineto{\pgfqpoint{1.779013in}{2.608051in}}%
\pgfpathclose%
\pgfusepath{stroke,fill}%
\end{pgfscope}%
\begin{pgfscope}%
\pgfpathrectangle{\pgfqpoint{1.542338in}{0.880000in}}{\pgfqpoint{5.115323in}{6.160000in}}%
\pgfusepath{clip}%
\pgfsetbuttcap%
\pgfsetroundjoin%
\definecolor{currentfill}{rgb}{0.200000,0.200000,0.800000}%
\pgfsetfillcolor{currentfill}%
\pgfsetlinewidth{1.003750pt}%
\definecolor{currentstroke}{rgb}{0.200000,0.200000,0.800000}%
\pgfsetstrokecolor{currentstroke}%
\pgfsetdash{}{0pt}%
\pgfpathmoveto{\pgfqpoint{1.774853in}{2.454325in}}%
\pgfpathcurveto{\pgfqpoint{1.780677in}{2.454325in}}{\pgfqpoint{1.786263in}{2.456639in}}{\pgfqpoint{1.790381in}{2.460757in}}%
\pgfpathcurveto{\pgfqpoint{1.794499in}{2.464875in}}{\pgfqpoint{1.796813in}{2.470461in}}{\pgfqpoint{1.796813in}{2.476285in}}%
\pgfpathcurveto{\pgfqpoint{1.796813in}{2.482109in}}{\pgfqpoint{1.794499in}{2.487695in}}{\pgfqpoint{1.790381in}{2.491813in}}%
\pgfpathcurveto{\pgfqpoint{1.786263in}{2.495931in}}{\pgfqpoint{1.780677in}{2.498245in}}{\pgfqpoint{1.774853in}{2.498245in}}%
\pgfpathcurveto{\pgfqpoint{1.769029in}{2.498245in}}{\pgfqpoint{1.763443in}{2.495931in}}{\pgfqpoint{1.759325in}{2.491813in}}%
\pgfpathcurveto{\pgfqpoint{1.755207in}{2.487695in}}{\pgfqpoint{1.752893in}{2.482109in}}{\pgfqpoint{1.752893in}{2.476285in}}%
\pgfpathcurveto{\pgfqpoint{1.752893in}{2.470461in}}{\pgfqpoint{1.755207in}{2.464875in}}{\pgfqpoint{1.759325in}{2.460757in}}%
\pgfpathcurveto{\pgfqpoint{1.763443in}{2.456639in}}{\pgfqpoint{1.769029in}{2.454325in}}{\pgfqpoint{1.774853in}{2.454325in}}%
\pgfpathlineto{\pgfqpoint{1.774853in}{2.454325in}}%
\pgfpathclose%
\pgfusepath{stroke,fill}%
\end{pgfscope}%
\begin{pgfscope}%
\pgfpathrectangle{\pgfqpoint{1.542338in}{0.880000in}}{\pgfqpoint{5.115323in}{6.160000in}}%
\pgfusepath{clip}%
\pgfsetbuttcap%
\pgfsetroundjoin%
\definecolor{currentfill}{rgb}{0.200000,0.200000,0.800000}%
\pgfsetfillcolor{currentfill}%
\pgfsetlinewidth{1.003750pt}%
\definecolor{currentstroke}{rgb}{0.200000,0.200000,0.800000}%
\pgfsetstrokecolor{currentstroke}%
\pgfsetdash{}{0pt}%
\pgfpathmoveto{\pgfqpoint{1.799540in}{2.302569in}}%
\pgfpathcurveto{\pgfqpoint{1.805364in}{2.302569in}}{\pgfqpoint{1.810950in}{2.304882in}}{\pgfqpoint{1.815068in}{2.309001in}}%
\pgfpathcurveto{\pgfqpoint{1.819187in}{2.313119in}}{\pgfqpoint{1.821500in}{2.318705in}}{\pgfqpoint{1.821500in}{2.324529in}}%
\pgfpathcurveto{\pgfqpoint{1.821500in}{2.330353in}}{\pgfqpoint{1.819187in}{2.335939in}}{\pgfqpoint{1.815068in}{2.340057in}}%
\pgfpathcurveto{\pgfqpoint{1.810950in}{2.344175in}}{\pgfqpoint{1.805364in}{2.346489in}}{\pgfqpoint{1.799540in}{2.346489in}}%
\pgfpathcurveto{\pgfqpoint{1.793716in}{2.346489in}}{\pgfqpoint{1.788130in}{2.344175in}}{\pgfqpoint{1.784012in}{2.340057in}}%
\pgfpathcurveto{\pgfqpoint{1.779894in}{2.335939in}}{\pgfqpoint{1.777580in}{2.330353in}}{\pgfqpoint{1.777580in}{2.324529in}}%
\pgfpathcurveto{\pgfqpoint{1.777580in}{2.318705in}}{\pgfqpoint{1.779894in}{2.313119in}}{\pgfqpoint{1.784012in}{2.309001in}}%
\pgfpathcurveto{\pgfqpoint{1.788130in}{2.304882in}}{\pgfqpoint{1.793716in}{2.302569in}}{\pgfqpoint{1.799540in}{2.302569in}}%
\pgfpathlineto{\pgfqpoint{1.799540in}{2.302569in}}%
\pgfpathclose%
\pgfusepath{stroke,fill}%
\end{pgfscope}%
\begin{pgfscope}%
\pgfpathrectangle{\pgfqpoint{1.542338in}{0.880000in}}{\pgfqpoint{5.115323in}{6.160000in}}%
\pgfusepath{clip}%
\pgfsetbuttcap%
\pgfsetroundjoin%
\definecolor{currentfill}{rgb}{0.200000,0.200000,0.800000}%
\pgfsetfillcolor{currentfill}%
\pgfsetlinewidth{1.003750pt}%
\definecolor{currentstroke}{rgb}{0.200000,0.200000,0.800000}%
\pgfsetstrokecolor{currentstroke}%
\pgfsetdash{}{0pt}%
\pgfpathmoveto{\pgfqpoint{1.839595in}{2.154627in}}%
\pgfpathcurveto{\pgfqpoint{1.845419in}{2.154627in}}{\pgfqpoint{1.851005in}{2.156941in}}{\pgfqpoint{1.855123in}{2.161059in}}%
\pgfpathcurveto{\pgfqpoint{1.859241in}{2.165177in}}{\pgfqpoint{1.861555in}{2.170764in}}{\pgfqpoint{1.861555in}{2.176587in}}%
\pgfpathcurveto{\pgfqpoint{1.861555in}{2.182411in}}{\pgfqpoint{1.859241in}{2.187998in}}{\pgfqpoint{1.855123in}{2.192116in}}%
\pgfpathcurveto{\pgfqpoint{1.851005in}{2.196234in}}{\pgfqpoint{1.845419in}{2.198548in}}{\pgfqpoint{1.839595in}{2.198548in}}%
\pgfpathcurveto{\pgfqpoint{1.833771in}{2.198548in}}{\pgfqpoint{1.828185in}{2.196234in}}{\pgfqpoint{1.824066in}{2.192116in}}%
\pgfpathcurveto{\pgfqpoint{1.819948in}{2.187998in}}{\pgfqpoint{1.817634in}{2.182411in}}{\pgfqpoint{1.817634in}{2.176587in}}%
\pgfpathcurveto{\pgfqpoint{1.817634in}{2.170764in}}{\pgfqpoint{1.819948in}{2.165177in}}{\pgfqpoint{1.824066in}{2.161059in}}%
\pgfpathcurveto{\pgfqpoint{1.828185in}{2.156941in}}{\pgfqpoint{1.833771in}{2.154627in}}{\pgfqpoint{1.839595in}{2.154627in}}%
\pgfpathlineto{\pgfqpoint{1.839595in}{2.154627in}}%
\pgfpathclose%
\pgfusepath{stroke,fill}%
\end{pgfscope}%
\begin{pgfscope}%
\pgfpathrectangle{\pgfqpoint{1.542338in}{0.880000in}}{\pgfqpoint{5.115323in}{6.160000in}}%
\pgfusepath{clip}%
\pgfsetbuttcap%
\pgfsetroundjoin%
\definecolor{currentfill}{rgb}{0.200000,0.200000,0.800000}%
\pgfsetfillcolor{currentfill}%
\pgfsetlinewidth{1.003750pt}%
\definecolor{currentstroke}{rgb}{0.200000,0.200000,0.800000}%
\pgfsetstrokecolor{currentstroke}%
\pgfsetdash{}{0pt}%
\pgfpathmoveto{\pgfqpoint{1.894092in}{2.011199in}}%
\pgfpathcurveto{\pgfqpoint{1.899916in}{2.011199in}}{\pgfqpoint{1.905502in}{2.013513in}}{\pgfqpoint{1.909620in}{2.017631in}}%
\pgfpathcurveto{\pgfqpoint{1.913738in}{2.021749in}}{\pgfqpoint{1.916052in}{2.027335in}}{\pgfqpoint{1.916052in}{2.033159in}}%
\pgfpathcurveto{\pgfqpoint{1.916052in}{2.038983in}}{\pgfqpoint{1.913738in}{2.044569in}}{\pgfqpoint{1.909620in}{2.048687in}}%
\pgfpathcurveto{\pgfqpoint{1.905502in}{2.052805in}}{\pgfqpoint{1.899916in}{2.055119in}}{\pgfqpoint{1.894092in}{2.055119in}}%
\pgfpathcurveto{\pgfqpoint{1.888268in}{2.055119in}}{\pgfqpoint{1.882682in}{2.052805in}}{\pgfqpoint{1.878564in}{2.048687in}}%
\pgfpathcurveto{\pgfqpoint{1.874446in}{2.044569in}}{\pgfqpoint{1.872132in}{2.038983in}}{\pgfqpoint{1.872132in}{2.033159in}}%
\pgfpathcurveto{\pgfqpoint{1.872132in}{2.027335in}}{\pgfqpoint{1.874446in}{2.021749in}}{\pgfqpoint{1.878564in}{2.017631in}}%
\pgfpathcurveto{\pgfqpoint{1.882682in}{2.013513in}}{\pgfqpoint{1.888268in}{2.011199in}}{\pgfqpoint{1.894092in}{2.011199in}}%
\pgfpathlineto{\pgfqpoint{1.894092in}{2.011199in}}%
\pgfpathclose%
\pgfusepath{stroke,fill}%
\end{pgfscope}%
\begin{pgfscope}%
\pgfpathrectangle{\pgfqpoint{1.542338in}{0.880000in}}{\pgfqpoint{5.115323in}{6.160000in}}%
\pgfusepath{clip}%
\pgfsetbuttcap%
\pgfsetroundjoin%
\definecolor{currentfill}{rgb}{0.200000,0.200000,0.800000}%
\pgfsetfillcolor{currentfill}%
\pgfsetlinewidth{1.003750pt}%
\definecolor{currentstroke}{rgb}{0.200000,0.200000,0.800000}%
\pgfsetstrokecolor{currentstroke}%
\pgfsetdash{}{0pt}%
\pgfpathmoveto{\pgfqpoint{1.974351in}{1.880532in}}%
\pgfpathcurveto{\pgfqpoint{1.980175in}{1.880532in}}{\pgfqpoint{1.985761in}{1.882846in}}{\pgfqpoint{1.989879in}{1.886964in}}%
\pgfpathcurveto{\pgfqpoint{1.993997in}{1.891082in}}{\pgfqpoint{1.996311in}{1.896669in}}{\pgfqpoint{1.996311in}{1.902493in}}%
\pgfpathcurveto{\pgfqpoint{1.996311in}{1.908317in}}{\pgfqpoint{1.993997in}{1.913903in}}{\pgfqpoint{1.989879in}{1.918021in}}%
\pgfpathcurveto{\pgfqpoint{1.985761in}{1.922139in}}{\pgfqpoint{1.980175in}{1.924453in}}{\pgfqpoint{1.974351in}{1.924453in}}%
\pgfpathcurveto{\pgfqpoint{1.968527in}{1.924453in}}{\pgfqpoint{1.962941in}{1.922139in}}{\pgfqpoint{1.958823in}{1.918021in}}%
\pgfpathcurveto{\pgfqpoint{1.954705in}{1.913903in}}{\pgfqpoint{1.952391in}{1.908317in}}{\pgfqpoint{1.952391in}{1.902493in}}%
\pgfpathcurveto{\pgfqpoint{1.952391in}{1.896669in}}{\pgfqpoint{1.954705in}{1.891082in}}{\pgfqpoint{1.958823in}{1.886964in}}%
\pgfpathcurveto{\pgfqpoint{1.962941in}{1.882846in}}{\pgfqpoint{1.968527in}{1.880532in}}{\pgfqpoint{1.974351in}{1.880532in}}%
\pgfpathlineto{\pgfqpoint{1.974351in}{1.880532in}}%
\pgfpathclose%
\pgfusepath{stroke,fill}%
\end{pgfscope}%
\begin{pgfscope}%
\pgfpathrectangle{\pgfqpoint{1.542338in}{0.880000in}}{\pgfqpoint{5.115323in}{6.160000in}}%
\pgfusepath{clip}%
\pgfsetbuttcap%
\pgfsetroundjoin%
\definecolor{currentfill}{rgb}{0.200000,0.200000,0.800000}%
\pgfsetfillcolor{currentfill}%
\pgfsetlinewidth{1.003750pt}%
\definecolor{currentstroke}{rgb}{0.200000,0.200000,0.800000}%
\pgfsetstrokecolor{currentstroke}%
\pgfsetdash{}{0pt}%
\pgfpathmoveto{\pgfqpoint{2.062692in}{1.755396in}}%
\pgfpathcurveto{\pgfqpoint{2.068516in}{1.755396in}}{\pgfqpoint{2.074102in}{1.757710in}}{\pgfqpoint{2.078220in}{1.761828in}}%
\pgfpathcurveto{\pgfqpoint{2.082338in}{1.765946in}}{\pgfqpoint{2.084652in}{1.771533in}}{\pgfqpoint{2.084652in}{1.777357in}}%
\pgfpathcurveto{\pgfqpoint{2.084652in}{1.783180in}}{\pgfqpoint{2.082338in}{1.788767in}}{\pgfqpoint{2.078220in}{1.792885in}}%
\pgfpathcurveto{\pgfqpoint{2.074102in}{1.797003in}}{\pgfqpoint{2.068516in}{1.799317in}}{\pgfqpoint{2.062692in}{1.799317in}}%
\pgfpathcurveto{\pgfqpoint{2.056868in}{1.799317in}}{\pgfqpoint{2.051282in}{1.797003in}}{\pgfqpoint{2.047164in}{1.792885in}}%
\pgfpathcurveto{\pgfqpoint{2.043046in}{1.788767in}}{\pgfqpoint{2.040732in}{1.783180in}}{\pgfqpoint{2.040732in}{1.777357in}}%
\pgfpathcurveto{\pgfqpoint{2.040732in}{1.771533in}}{\pgfqpoint{2.043046in}{1.765946in}}{\pgfqpoint{2.047164in}{1.761828in}}%
\pgfpathcurveto{\pgfqpoint{2.051282in}{1.757710in}}{\pgfqpoint{2.056868in}{1.755396in}}{\pgfqpoint{2.062692in}{1.755396in}}%
\pgfpathlineto{\pgfqpoint{2.062692in}{1.755396in}}%
\pgfpathclose%
\pgfusepath{stroke,fill}%
\end{pgfscope}%
\begin{pgfscope}%
\pgfpathrectangle{\pgfqpoint{1.542338in}{0.880000in}}{\pgfqpoint{5.115323in}{6.160000in}}%
\pgfusepath{clip}%
\pgfsetbuttcap%
\pgfsetroundjoin%
\definecolor{currentfill}{rgb}{0.200000,0.200000,0.800000}%
\pgfsetfillcolor{currentfill}%
\pgfsetlinewidth{1.003750pt}%
\definecolor{currentstroke}{rgb}{0.200000,0.200000,0.800000}%
\pgfsetstrokecolor{currentstroke}%
\pgfsetdash{}{0pt}%
\pgfpathmoveto{\pgfqpoint{2.173011in}{1.649192in}}%
\pgfpathcurveto{\pgfqpoint{2.178835in}{1.649192in}}{\pgfqpoint{2.184421in}{1.651506in}}{\pgfqpoint{2.188540in}{1.655624in}}%
\pgfpathcurveto{\pgfqpoint{2.192658in}{1.659742in}}{\pgfqpoint{2.194972in}{1.665329in}}{\pgfqpoint{2.194972in}{1.671153in}}%
\pgfpathcurveto{\pgfqpoint{2.194972in}{1.676977in}}{\pgfqpoint{2.192658in}{1.682563in}}{\pgfqpoint{2.188540in}{1.686681in}}%
\pgfpathcurveto{\pgfqpoint{2.184421in}{1.690799in}}{\pgfqpoint{2.178835in}{1.693113in}}{\pgfqpoint{2.173011in}{1.693113in}}%
\pgfpathcurveto{\pgfqpoint{2.167187in}{1.693113in}}{\pgfqpoint{2.161601in}{1.690799in}}{\pgfqpoint{2.157483in}{1.686681in}}%
\pgfpathcurveto{\pgfqpoint{2.153365in}{1.682563in}}{\pgfqpoint{2.151051in}{1.676977in}}{\pgfqpoint{2.151051in}{1.671153in}}%
\pgfpathcurveto{\pgfqpoint{2.151051in}{1.665329in}}{\pgfqpoint{2.153365in}{1.659742in}}{\pgfqpoint{2.157483in}{1.655624in}}%
\pgfpathcurveto{\pgfqpoint{2.161601in}{1.651506in}}{\pgfqpoint{2.167187in}{1.649192in}}{\pgfqpoint{2.173011in}{1.649192in}}%
\pgfpathlineto{\pgfqpoint{2.173011in}{1.649192in}}%
\pgfpathclose%
\pgfusepath{stroke,fill}%
\end{pgfscope}%
\begin{pgfscope}%
\pgfpathrectangle{\pgfqpoint{1.542338in}{0.880000in}}{\pgfqpoint{5.115323in}{6.160000in}}%
\pgfusepath{clip}%
\pgfsetbuttcap%
\pgfsetroundjoin%
\definecolor{currentfill}{rgb}{0.200000,0.200000,0.800000}%
\pgfsetfillcolor{currentfill}%
\pgfsetlinewidth{1.003750pt}%
\definecolor{currentstroke}{rgb}{0.200000,0.200000,0.800000}%
\pgfsetstrokecolor{currentstroke}%
\pgfsetdash{}{0pt}%
\pgfpathmoveto{\pgfqpoint{2.289448in}{1.549768in}}%
\pgfpathcurveto{\pgfqpoint{2.295272in}{1.549768in}}{\pgfqpoint{2.300858in}{1.552082in}}{\pgfqpoint{2.304977in}{1.556200in}}%
\pgfpathcurveto{\pgfqpoint{2.309095in}{1.560318in}}{\pgfqpoint{2.311409in}{1.565904in}}{\pgfqpoint{2.311409in}{1.571728in}}%
\pgfpathcurveto{\pgfqpoint{2.311409in}{1.577552in}}{\pgfqpoint{2.309095in}{1.583138in}}{\pgfqpoint{2.304977in}{1.587256in}}%
\pgfpathcurveto{\pgfqpoint{2.300858in}{1.591374in}}{\pgfqpoint{2.295272in}{1.593688in}}{\pgfqpoint{2.289448in}{1.593688in}}%
\pgfpathcurveto{\pgfqpoint{2.283624in}{1.593688in}}{\pgfqpoint{2.278038in}{1.591374in}}{\pgfqpoint{2.273920in}{1.587256in}}%
\pgfpathcurveto{\pgfqpoint{2.269802in}{1.583138in}}{\pgfqpoint{2.267488in}{1.577552in}}{\pgfqpoint{2.267488in}{1.571728in}}%
\pgfpathcurveto{\pgfqpoint{2.267488in}{1.565904in}}{\pgfqpoint{2.269802in}{1.560318in}}{\pgfqpoint{2.273920in}{1.556200in}}%
\pgfpathcurveto{\pgfqpoint{2.278038in}{1.552082in}}{\pgfqpoint{2.283624in}{1.549768in}}{\pgfqpoint{2.289448in}{1.549768in}}%
\pgfpathlineto{\pgfqpoint{2.289448in}{1.549768in}}%
\pgfpathclose%
\pgfusepath{stroke,fill}%
\end{pgfscope}%
\begin{pgfscope}%
\pgfpathrectangle{\pgfqpoint{1.542338in}{0.880000in}}{\pgfqpoint{5.115323in}{6.160000in}}%
\pgfusepath{clip}%
\pgfsetbuttcap%
\pgfsetroundjoin%
\definecolor{currentfill}{rgb}{0.200000,0.200000,0.800000}%
\pgfsetfillcolor{currentfill}%
\pgfsetlinewidth{1.003750pt}%
\definecolor{currentstroke}{rgb}{0.200000,0.200000,0.800000}%
\pgfsetstrokecolor{currentstroke}%
\pgfsetdash{}{0pt}%
\pgfpathmoveto{\pgfqpoint{2.420982in}{1.471026in}}%
\pgfpathcurveto{\pgfqpoint{2.426806in}{1.471026in}}{\pgfqpoint{2.432392in}{1.473340in}}{\pgfqpoint{2.436511in}{1.477458in}}%
\pgfpathcurveto{\pgfqpoint{2.440629in}{1.481576in}}{\pgfqpoint{2.442943in}{1.487163in}}{\pgfqpoint{2.442943in}{1.492986in}}%
\pgfpathcurveto{\pgfqpoint{2.442943in}{1.498810in}}{\pgfqpoint{2.440629in}{1.504397in}}{\pgfqpoint{2.436511in}{1.508515in}}%
\pgfpathcurveto{\pgfqpoint{2.432392in}{1.512633in}}{\pgfqpoint{2.426806in}{1.514947in}}{\pgfqpoint{2.420982in}{1.514947in}}%
\pgfpathcurveto{\pgfqpoint{2.415158in}{1.514947in}}{\pgfqpoint{2.409572in}{1.512633in}}{\pgfqpoint{2.405454in}{1.508515in}}%
\pgfpathcurveto{\pgfqpoint{2.401336in}{1.504397in}}{\pgfqpoint{2.399022in}{1.498810in}}{\pgfqpoint{2.399022in}{1.492986in}}%
\pgfpathcurveto{\pgfqpoint{2.399022in}{1.487163in}}{\pgfqpoint{2.401336in}{1.481576in}}{\pgfqpoint{2.405454in}{1.477458in}}%
\pgfpathcurveto{\pgfqpoint{2.409572in}{1.473340in}}{\pgfqpoint{2.415158in}{1.471026in}}{\pgfqpoint{2.420982in}{1.471026in}}%
\pgfpathlineto{\pgfqpoint{2.420982in}{1.471026in}}%
\pgfpathclose%
\pgfusepath{stroke,fill}%
\end{pgfscope}%
\begin{pgfscope}%
\pgfpathrectangle{\pgfqpoint{1.542338in}{0.880000in}}{\pgfqpoint{5.115323in}{6.160000in}}%
\pgfusepath{clip}%
\pgfsetbuttcap%
\pgfsetroundjoin%
\definecolor{currentfill}{rgb}{0.200000,0.200000,0.800000}%
\pgfsetfillcolor{currentfill}%
\pgfsetlinewidth{1.003750pt}%
\definecolor{currentstroke}{rgb}{0.200000,0.200000,0.800000}%
\pgfsetstrokecolor{currentstroke}%
\pgfsetdash{}{0pt}%
\pgfpathmoveto{\pgfqpoint{2.562046in}{1.411536in}}%
\pgfpathcurveto{\pgfqpoint{2.567870in}{1.411536in}}{\pgfqpoint{2.573456in}{1.413850in}}{\pgfqpoint{2.577575in}{1.417968in}}%
\pgfpathcurveto{\pgfqpoint{2.581693in}{1.422086in}}{\pgfqpoint{2.584007in}{1.427672in}}{\pgfqpoint{2.584007in}{1.433496in}}%
\pgfpathcurveto{\pgfqpoint{2.584007in}{1.439320in}}{\pgfqpoint{2.581693in}{1.444907in}}{\pgfqpoint{2.577575in}{1.449025in}}%
\pgfpathcurveto{\pgfqpoint{2.573456in}{1.453143in}}{\pgfqpoint{2.567870in}{1.455457in}}{\pgfqpoint{2.562046in}{1.455457in}}%
\pgfpathcurveto{\pgfqpoint{2.556222in}{1.455457in}}{\pgfqpoint{2.550636in}{1.453143in}}{\pgfqpoint{2.546518in}{1.449025in}}%
\pgfpathcurveto{\pgfqpoint{2.542400in}{1.444907in}}{\pgfqpoint{2.540086in}{1.439320in}}{\pgfqpoint{2.540086in}{1.433496in}}%
\pgfpathcurveto{\pgfqpoint{2.540086in}{1.427672in}}{\pgfqpoint{2.542400in}{1.422086in}}{\pgfqpoint{2.546518in}{1.417968in}}%
\pgfpathcurveto{\pgfqpoint{2.550636in}{1.413850in}}{\pgfqpoint{2.556222in}{1.411536in}}{\pgfqpoint{2.562046in}{1.411536in}}%
\pgfpathlineto{\pgfqpoint{2.562046in}{1.411536in}}%
\pgfpathclose%
\pgfusepath{stroke,fill}%
\end{pgfscope}%
\begin{pgfscope}%
\pgfpathrectangle{\pgfqpoint{1.542338in}{0.880000in}}{\pgfqpoint{5.115323in}{6.160000in}}%
\pgfusepath{clip}%
\pgfsetbuttcap%
\pgfsetroundjoin%
\definecolor{currentfill}{rgb}{0.200000,0.200000,0.800000}%
\pgfsetfillcolor{currentfill}%
\pgfsetlinewidth{1.003750pt}%
\definecolor{currentstroke}{rgb}{0.200000,0.200000,0.800000}%
\pgfsetstrokecolor{currentstroke}%
\pgfsetdash{}{0pt}%
\pgfpathmoveto{\pgfqpoint{2.707759in}{1.364239in}}%
\pgfpathcurveto{\pgfqpoint{2.713583in}{1.364239in}}{\pgfqpoint{2.719169in}{1.366553in}}{\pgfqpoint{2.723287in}{1.370671in}}%
\pgfpathcurveto{\pgfqpoint{2.727405in}{1.374789in}}{\pgfqpoint{2.729719in}{1.380375in}}{\pgfqpoint{2.729719in}{1.386199in}}%
\pgfpathcurveto{\pgfqpoint{2.729719in}{1.392023in}}{\pgfqpoint{2.727405in}{1.397609in}}{\pgfqpoint{2.723287in}{1.401727in}}%
\pgfpathcurveto{\pgfqpoint{2.719169in}{1.405846in}}{\pgfqpoint{2.713583in}{1.408159in}}{\pgfqpoint{2.707759in}{1.408159in}}%
\pgfpathcurveto{\pgfqpoint{2.701935in}{1.408159in}}{\pgfqpoint{2.696349in}{1.405846in}}{\pgfqpoint{2.692230in}{1.401727in}}%
\pgfpathcurveto{\pgfqpoint{2.688112in}{1.397609in}}{\pgfqpoint{2.685798in}{1.392023in}}{\pgfqpoint{2.685798in}{1.386199in}}%
\pgfpathcurveto{\pgfqpoint{2.685798in}{1.380375in}}{\pgfqpoint{2.688112in}{1.374789in}}{\pgfqpoint{2.692230in}{1.370671in}}%
\pgfpathcurveto{\pgfqpoint{2.696349in}{1.366553in}}{\pgfqpoint{2.701935in}{1.364239in}}{\pgfqpoint{2.707759in}{1.364239in}}%
\pgfpathlineto{\pgfqpoint{2.707759in}{1.364239in}}%
\pgfpathclose%
\pgfusepath{stroke,fill}%
\end{pgfscope}%
\begin{pgfscope}%
\pgfpathrectangle{\pgfqpoint{1.542338in}{0.880000in}}{\pgfqpoint{5.115323in}{6.160000in}}%
\pgfusepath{clip}%
\pgfsetbuttcap%
\pgfsetroundjoin%
\definecolor{currentfill}{rgb}{0.200000,0.200000,0.800000}%
\pgfsetfillcolor{currentfill}%
\pgfsetlinewidth{1.003750pt}%
\definecolor{currentstroke}{rgb}{0.200000,0.200000,0.800000}%
\pgfsetstrokecolor{currentstroke}%
\pgfsetdash{}{0pt}%
\pgfpathmoveto{\pgfqpoint{2.859191in}{1.339783in}}%
\pgfpathcurveto{\pgfqpoint{2.865015in}{1.339783in}}{\pgfqpoint{2.870601in}{1.342097in}}{\pgfqpoint{2.874720in}{1.346215in}}%
\pgfpathcurveto{\pgfqpoint{2.878838in}{1.350333in}}{\pgfqpoint{2.881152in}{1.355919in}}{\pgfqpoint{2.881152in}{1.361743in}}%
\pgfpathcurveto{\pgfqpoint{2.881152in}{1.367567in}}{\pgfqpoint{2.878838in}{1.373153in}}{\pgfqpoint{2.874720in}{1.377271in}}%
\pgfpathcurveto{\pgfqpoint{2.870601in}{1.381389in}}{\pgfqpoint{2.865015in}{1.383703in}}{\pgfqpoint{2.859191in}{1.383703in}}%
\pgfpathcurveto{\pgfqpoint{2.853367in}{1.383703in}}{\pgfqpoint{2.847781in}{1.381389in}}{\pgfqpoint{2.843663in}{1.377271in}}%
\pgfpathcurveto{\pgfqpoint{2.839545in}{1.373153in}}{\pgfqpoint{2.837231in}{1.367567in}}{\pgfqpoint{2.837231in}{1.361743in}}%
\pgfpathcurveto{\pgfqpoint{2.837231in}{1.355919in}}{\pgfqpoint{2.839545in}{1.350333in}}{\pgfqpoint{2.843663in}{1.346215in}}%
\pgfpathcurveto{\pgfqpoint{2.847781in}{1.342097in}}{\pgfqpoint{2.853367in}{1.339783in}}{\pgfqpoint{2.859191in}{1.339783in}}%
\pgfpathlineto{\pgfqpoint{2.859191in}{1.339783in}}%
\pgfpathclose%
\pgfusepath{stroke,fill}%
\end{pgfscope}%
\begin{pgfscope}%
\pgfpathrectangle{\pgfqpoint{1.542338in}{0.880000in}}{\pgfqpoint{5.115323in}{6.160000in}}%
\pgfusepath{clip}%
\pgfsetbuttcap%
\pgfsetroundjoin%
\definecolor{currentfill}{rgb}{0.200000,0.200000,0.800000}%
\pgfsetfillcolor{currentfill}%
\pgfsetlinewidth{1.003750pt}%
\definecolor{currentstroke}{rgb}{0.200000,0.200000,0.800000}%
\pgfsetstrokecolor{currentstroke}%
\pgfsetdash{}{0pt}%
\pgfpathmoveto{\pgfqpoint{3.012636in}{1.330786in}}%
\pgfpathcurveto{\pgfqpoint{3.018460in}{1.330786in}}{\pgfqpoint{3.024047in}{1.333100in}}{\pgfqpoint{3.028165in}{1.337218in}}%
\pgfpathcurveto{\pgfqpoint{3.032283in}{1.341336in}}{\pgfqpoint{3.034597in}{1.346922in}}{\pgfqpoint{3.034597in}{1.352746in}}%
\pgfpathcurveto{\pgfqpoint{3.034597in}{1.358570in}}{\pgfqpoint{3.032283in}{1.364156in}}{\pgfqpoint{3.028165in}{1.368275in}}%
\pgfpathcurveto{\pgfqpoint{3.024047in}{1.372393in}}{\pgfqpoint{3.018460in}{1.374707in}}{\pgfqpoint{3.012636in}{1.374707in}}%
\pgfpathcurveto{\pgfqpoint{3.006813in}{1.374707in}}{\pgfqpoint{3.001226in}{1.372393in}}{\pgfqpoint{2.997108in}{1.368275in}}%
\pgfpathcurveto{\pgfqpoint{2.992990in}{1.364156in}}{\pgfqpoint{2.990676in}{1.358570in}}{\pgfqpoint{2.990676in}{1.352746in}}%
\pgfpathcurveto{\pgfqpoint{2.990676in}{1.346922in}}{\pgfqpoint{2.992990in}{1.341336in}}{\pgfqpoint{2.997108in}{1.337218in}}%
\pgfpathcurveto{\pgfqpoint{3.001226in}{1.333100in}}{\pgfqpoint{3.006813in}{1.330786in}}{\pgfqpoint{3.012636in}{1.330786in}}%
\pgfpathlineto{\pgfqpoint{3.012636in}{1.330786in}}%
\pgfpathclose%
\pgfusepath{stroke,fill}%
\end{pgfscope}%
\begin{pgfscope}%
\pgfpathrectangle{\pgfqpoint{1.542338in}{0.880000in}}{\pgfqpoint{5.115323in}{6.160000in}}%
\pgfusepath{clip}%
\pgfsetbuttcap%
\pgfsetroundjoin%
\definecolor{currentfill}{rgb}{0.200000,0.200000,0.800000}%
\pgfsetfillcolor{currentfill}%
\pgfsetlinewidth{1.003750pt}%
\definecolor{currentstroke}{rgb}{0.200000,0.200000,0.800000}%
\pgfsetstrokecolor{currentstroke}%
\pgfsetdash{}{0pt}%
\pgfpathmoveto{\pgfqpoint{3.164060in}{1.356587in}}%
\pgfpathcurveto{\pgfqpoint{3.169883in}{1.356587in}}{\pgfqpoint{3.175470in}{1.358901in}}{\pgfqpoint{3.179588in}{1.363019in}}%
\pgfpathcurveto{\pgfqpoint{3.183706in}{1.367137in}}{\pgfqpoint{3.186020in}{1.372723in}}{\pgfqpoint{3.186020in}{1.378547in}}%
\pgfpathcurveto{\pgfqpoint{3.186020in}{1.384371in}}{\pgfqpoint{3.183706in}{1.389957in}}{\pgfqpoint{3.179588in}{1.394075in}}%
\pgfpathcurveto{\pgfqpoint{3.175470in}{1.398194in}}{\pgfqpoint{3.169883in}{1.400507in}}{\pgfqpoint{3.164060in}{1.400507in}}%
\pgfpathcurveto{\pgfqpoint{3.158236in}{1.400507in}}{\pgfqpoint{3.152649in}{1.398194in}}{\pgfqpoint{3.148531in}{1.394075in}}%
\pgfpathcurveto{\pgfqpoint{3.144413in}{1.389957in}}{\pgfqpoint{3.142099in}{1.384371in}}{\pgfqpoint{3.142099in}{1.378547in}}%
\pgfpathcurveto{\pgfqpoint{3.142099in}{1.372723in}}{\pgfqpoint{3.144413in}{1.367137in}}{\pgfqpoint{3.148531in}{1.363019in}}%
\pgfpathcurveto{\pgfqpoint{3.152649in}{1.358901in}}{\pgfqpoint{3.158236in}{1.356587in}}{\pgfqpoint{3.164060in}{1.356587in}}%
\pgfpathlineto{\pgfqpoint{3.164060in}{1.356587in}}%
\pgfpathclose%
\pgfusepath{stroke,fill}%
\end{pgfscope}%
\begin{pgfscope}%
\pgfpathrectangle{\pgfqpoint{1.542338in}{0.880000in}}{\pgfqpoint{5.115323in}{6.160000in}}%
\pgfusepath{clip}%
\pgfsetbuttcap%
\pgfsetroundjoin%
\definecolor{currentfill}{rgb}{0.200000,0.200000,0.800000}%
\pgfsetfillcolor{currentfill}%
\pgfsetlinewidth{1.003750pt}%
\definecolor{currentstroke}{rgb}{0.200000,0.200000,0.800000}%
\pgfsetstrokecolor{currentstroke}%
\pgfsetdash{}{0pt}%
\pgfpathmoveto{\pgfqpoint{3.314469in}{1.384632in}}%
\pgfpathcurveto{\pgfqpoint{3.320293in}{1.384632in}}{\pgfqpoint{3.325879in}{1.386945in}}{\pgfqpoint{3.329997in}{1.391064in}}%
\pgfpathcurveto{\pgfqpoint{3.334116in}{1.395182in}}{\pgfqpoint{3.336429in}{1.400768in}}{\pgfqpoint{3.336429in}{1.406592in}}%
\pgfpathcurveto{\pgfqpoint{3.336429in}{1.412416in}}{\pgfqpoint{3.334116in}{1.418002in}}{\pgfqpoint{3.329997in}{1.422120in}}%
\pgfpathcurveto{\pgfqpoint{3.325879in}{1.426238in}}{\pgfqpoint{3.320293in}{1.428552in}}{\pgfqpoint{3.314469in}{1.428552in}}%
\pgfpathcurveto{\pgfqpoint{3.308645in}{1.428552in}}{\pgfqpoint{3.303059in}{1.426238in}}{\pgfqpoint{3.298941in}{1.422120in}}%
\pgfpathcurveto{\pgfqpoint{3.294823in}{1.418002in}}{\pgfqpoint{3.292509in}{1.412416in}}{\pgfqpoint{3.292509in}{1.406592in}}%
\pgfpathcurveto{\pgfqpoint{3.292509in}{1.400768in}}{\pgfqpoint{3.294823in}{1.395182in}}{\pgfqpoint{3.298941in}{1.391064in}}%
\pgfpathcurveto{\pgfqpoint{3.303059in}{1.386945in}}{\pgfqpoint{3.308645in}{1.384632in}}{\pgfqpoint{3.314469in}{1.384632in}}%
\pgfpathlineto{\pgfqpoint{3.314469in}{1.384632in}}%
\pgfpathclose%
\pgfusepath{stroke,fill}%
\end{pgfscope}%
\begin{pgfscope}%
\pgfpathrectangle{\pgfqpoint{1.542338in}{0.880000in}}{\pgfqpoint{5.115323in}{6.160000in}}%
\pgfusepath{clip}%
\pgfsetbuttcap%
\pgfsetroundjoin%
\definecolor{currentfill}{rgb}{0.200000,0.200000,0.800000}%
\pgfsetfillcolor{currentfill}%
\pgfsetlinewidth{1.003750pt}%
\definecolor{currentstroke}{rgb}{0.200000,0.200000,0.800000}%
\pgfsetstrokecolor{currentstroke}%
\pgfsetdash{}{0pt}%
\pgfpathmoveto{\pgfqpoint{3.457199in}{1.440078in}}%
\pgfpathcurveto{\pgfqpoint{3.463023in}{1.440078in}}{\pgfqpoint{3.468609in}{1.442392in}}{\pgfqpoint{3.472727in}{1.446510in}}%
\pgfpathcurveto{\pgfqpoint{3.476845in}{1.450628in}}{\pgfqpoint{3.479159in}{1.456214in}}{\pgfqpoint{3.479159in}{1.462038in}}%
\pgfpathcurveto{\pgfqpoint{3.479159in}{1.467862in}}{\pgfqpoint{3.476845in}{1.473448in}}{\pgfqpoint{3.472727in}{1.477567in}}%
\pgfpathcurveto{\pgfqpoint{3.468609in}{1.481685in}}{\pgfqpoint{3.463023in}{1.483999in}}{\pgfqpoint{3.457199in}{1.483999in}}%
\pgfpathcurveto{\pgfqpoint{3.451375in}{1.483999in}}{\pgfqpoint{3.445789in}{1.481685in}}{\pgfqpoint{3.441670in}{1.477567in}}%
\pgfpathcurveto{\pgfqpoint{3.437552in}{1.473448in}}{\pgfqpoint{3.435238in}{1.467862in}}{\pgfqpoint{3.435238in}{1.462038in}}%
\pgfpathcurveto{\pgfqpoint{3.435238in}{1.456214in}}{\pgfqpoint{3.437552in}{1.450628in}}{\pgfqpoint{3.441670in}{1.446510in}}%
\pgfpathcurveto{\pgfqpoint{3.445789in}{1.442392in}}{\pgfqpoint{3.451375in}{1.440078in}}{\pgfqpoint{3.457199in}{1.440078in}}%
\pgfpathlineto{\pgfqpoint{3.457199in}{1.440078in}}%
\pgfpathclose%
\pgfusepath{stroke,fill}%
\end{pgfscope}%
\begin{pgfscope}%
\pgfpathrectangle{\pgfqpoint{1.542338in}{0.880000in}}{\pgfqpoint{5.115323in}{6.160000in}}%
\pgfusepath{clip}%
\pgfsetbuttcap%
\pgfsetroundjoin%
\definecolor{currentfill}{rgb}{0.200000,0.200000,0.800000}%
\pgfsetfillcolor{currentfill}%
\pgfsetlinewidth{1.003750pt}%
\definecolor{currentstroke}{rgb}{0.200000,0.200000,0.800000}%
\pgfsetstrokecolor{currentstroke}%
\pgfsetdash{}{0pt}%
\pgfpathmoveto{\pgfqpoint{3.596170in}{1.505214in}}%
\pgfpathcurveto{\pgfqpoint{3.601994in}{1.505214in}}{\pgfqpoint{3.607580in}{1.507528in}}{\pgfqpoint{3.611698in}{1.511646in}}%
\pgfpathcurveto{\pgfqpoint{3.615816in}{1.515764in}}{\pgfqpoint{3.618130in}{1.521350in}}{\pgfqpoint{3.618130in}{1.527174in}}%
\pgfpathcurveto{\pgfqpoint{3.618130in}{1.532998in}}{\pgfqpoint{3.615816in}{1.538584in}}{\pgfqpoint{3.611698in}{1.542703in}}%
\pgfpathcurveto{\pgfqpoint{3.607580in}{1.546821in}}{\pgfqpoint{3.601994in}{1.549135in}}{\pgfqpoint{3.596170in}{1.549135in}}%
\pgfpathcurveto{\pgfqpoint{3.590346in}{1.549135in}}{\pgfqpoint{3.584760in}{1.546821in}}{\pgfqpoint{3.580642in}{1.542703in}}%
\pgfpathcurveto{\pgfqpoint{3.576524in}{1.538584in}}{\pgfqpoint{3.574210in}{1.532998in}}{\pgfqpoint{3.574210in}{1.527174in}}%
\pgfpathcurveto{\pgfqpoint{3.574210in}{1.521350in}}{\pgfqpoint{3.576524in}{1.515764in}}{\pgfqpoint{3.580642in}{1.511646in}}%
\pgfpathcurveto{\pgfqpoint{3.584760in}{1.507528in}}{\pgfqpoint{3.590346in}{1.505214in}}{\pgfqpoint{3.596170in}{1.505214in}}%
\pgfpathlineto{\pgfqpoint{3.596170in}{1.505214in}}%
\pgfpathclose%
\pgfusepath{stroke,fill}%
\end{pgfscope}%
\begin{pgfscope}%
\pgfpathrectangle{\pgfqpoint{1.542338in}{0.880000in}}{\pgfqpoint{5.115323in}{6.160000in}}%
\pgfusepath{clip}%
\pgfsetbuttcap%
\pgfsetroundjoin%
\definecolor{currentfill}{rgb}{0.200000,0.200000,0.800000}%
\pgfsetfillcolor{currentfill}%
\pgfsetlinewidth{1.003750pt}%
\definecolor{currentstroke}{rgb}{0.200000,0.200000,0.800000}%
\pgfsetstrokecolor{currentstroke}%
\pgfsetdash{}{0pt}%
\pgfpathmoveto{\pgfqpoint{3.721989in}{1.593546in}}%
\pgfpathcurveto{\pgfqpoint{3.727813in}{1.593546in}}{\pgfqpoint{3.733399in}{1.595860in}}{\pgfqpoint{3.737517in}{1.599978in}}%
\pgfpathcurveto{\pgfqpoint{3.741635in}{1.604096in}}{\pgfqpoint{3.743949in}{1.609682in}}{\pgfqpoint{3.743949in}{1.615506in}}%
\pgfpathcurveto{\pgfqpoint{3.743949in}{1.621330in}}{\pgfqpoint{3.741635in}{1.626916in}}{\pgfqpoint{3.737517in}{1.631034in}}%
\pgfpathcurveto{\pgfqpoint{3.733399in}{1.635152in}}{\pgfqpoint{3.727813in}{1.637466in}}{\pgfqpoint{3.721989in}{1.637466in}}%
\pgfpathcurveto{\pgfqpoint{3.716165in}{1.637466in}}{\pgfqpoint{3.710579in}{1.635152in}}{\pgfqpoint{3.706461in}{1.631034in}}%
\pgfpathcurveto{\pgfqpoint{3.702342in}{1.626916in}}{\pgfqpoint{3.700029in}{1.621330in}}{\pgfqpoint{3.700029in}{1.615506in}}%
\pgfpathcurveto{\pgfqpoint{3.700029in}{1.609682in}}{\pgfqpoint{3.702342in}{1.604096in}}{\pgfqpoint{3.706461in}{1.599978in}}%
\pgfpathcurveto{\pgfqpoint{3.710579in}{1.595860in}}{\pgfqpoint{3.716165in}{1.593546in}}{\pgfqpoint{3.721989in}{1.593546in}}%
\pgfpathlineto{\pgfqpoint{3.721989in}{1.593546in}}%
\pgfpathclose%
\pgfusepath{stroke,fill}%
\end{pgfscope}%
\begin{pgfscope}%
\pgfpathrectangle{\pgfqpoint{1.542338in}{0.880000in}}{\pgfqpoint{5.115323in}{6.160000in}}%
\pgfusepath{clip}%
\pgfsetbuttcap%
\pgfsetroundjoin%
\definecolor{currentfill}{rgb}{0.200000,0.200000,0.800000}%
\pgfsetfillcolor{currentfill}%
\pgfsetlinewidth{1.003750pt}%
\definecolor{currentstroke}{rgb}{0.200000,0.200000,0.800000}%
\pgfsetstrokecolor{currentstroke}%
\pgfsetdash{}{0pt}%
\pgfpathmoveto{\pgfqpoint{3.835060in}{1.697565in}}%
\pgfpathcurveto{\pgfqpoint{3.840883in}{1.697565in}}{\pgfqpoint{3.846470in}{1.699879in}}{\pgfqpoint{3.850588in}{1.703997in}}%
\pgfpathcurveto{\pgfqpoint{3.854706in}{1.708115in}}{\pgfqpoint{3.857020in}{1.713701in}}{\pgfqpoint{3.857020in}{1.719525in}}%
\pgfpathcurveto{\pgfqpoint{3.857020in}{1.725349in}}{\pgfqpoint{3.854706in}{1.730935in}}{\pgfqpoint{3.850588in}{1.735053in}}%
\pgfpathcurveto{\pgfqpoint{3.846470in}{1.739172in}}{\pgfqpoint{3.840883in}{1.741485in}}{\pgfqpoint{3.835060in}{1.741485in}}%
\pgfpathcurveto{\pgfqpoint{3.829236in}{1.741485in}}{\pgfqpoint{3.823649in}{1.739172in}}{\pgfqpoint{3.819531in}{1.735053in}}%
\pgfpathcurveto{\pgfqpoint{3.815413in}{1.730935in}}{\pgfqpoint{3.813099in}{1.725349in}}{\pgfqpoint{3.813099in}{1.719525in}}%
\pgfpathcurveto{\pgfqpoint{3.813099in}{1.713701in}}{\pgfqpoint{3.815413in}{1.708115in}}{\pgfqpoint{3.819531in}{1.703997in}}%
\pgfpathcurveto{\pgfqpoint{3.823649in}{1.699879in}}{\pgfqpoint{3.829236in}{1.697565in}}{\pgfqpoint{3.835060in}{1.697565in}}%
\pgfpathlineto{\pgfqpoint{3.835060in}{1.697565in}}%
\pgfpathclose%
\pgfusepath{stroke,fill}%
\end{pgfscope}%
\begin{pgfscope}%
\pgfpathrectangle{\pgfqpoint{1.542338in}{0.880000in}}{\pgfqpoint{5.115323in}{6.160000in}}%
\pgfusepath{clip}%
\pgfsetbuttcap%
\pgfsetroundjoin%
\definecolor{currentfill}{rgb}{0.200000,0.200000,0.800000}%
\pgfsetfillcolor{currentfill}%
\pgfsetlinewidth{1.003750pt}%
\definecolor{currentstroke}{rgb}{0.200000,0.200000,0.800000}%
\pgfsetstrokecolor{currentstroke}%
\pgfsetdash{}{0pt}%
\pgfpathmoveto{\pgfqpoint{3.930711in}{1.817412in}}%
\pgfpathcurveto{\pgfqpoint{3.936535in}{1.817412in}}{\pgfqpoint{3.942122in}{1.819726in}}{\pgfqpoint{3.946240in}{1.823844in}}%
\pgfpathcurveto{\pgfqpoint{3.950358in}{1.827962in}}{\pgfqpoint{3.952672in}{1.833548in}}{\pgfqpoint{3.952672in}{1.839372in}}%
\pgfpathcurveto{\pgfqpoint{3.952672in}{1.845196in}}{\pgfqpoint{3.950358in}{1.850782in}}{\pgfqpoint{3.946240in}{1.854900in}}%
\pgfpathcurveto{\pgfqpoint{3.942122in}{1.859018in}}{\pgfqpoint{3.936535in}{1.861332in}}{\pgfqpoint{3.930711in}{1.861332in}}%
\pgfpathcurveto{\pgfqpoint{3.924888in}{1.861332in}}{\pgfqpoint{3.919301in}{1.859018in}}{\pgfqpoint{3.915183in}{1.854900in}}%
\pgfpathcurveto{\pgfqpoint{3.911065in}{1.850782in}}{\pgfqpoint{3.908751in}{1.845196in}}{\pgfqpoint{3.908751in}{1.839372in}}%
\pgfpathcurveto{\pgfqpoint{3.908751in}{1.833548in}}{\pgfqpoint{3.911065in}{1.827962in}}{\pgfqpoint{3.915183in}{1.823844in}}%
\pgfpathcurveto{\pgfqpoint{3.919301in}{1.819726in}}{\pgfqpoint{3.924888in}{1.817412in}}{\pgfqpoint{3.930711in}{1.817412in}}%
\pgfpathlineto{\pgfqpoint{3.930711in}{1.817412in}}%
\pgfpathclose%
\pgfusepath{stroke,fill}%
\end{pgfscope}%
\begin{pgfscope}%
\pgfpathrectangle{\pgfqpoint{1.542338in}{0.880000in}}{\pgfqpoint{5.115323in}{6.160000in}}%
\pgfusepath{clip}%
\pgfsetbuttcap%
\pgfsetroundjoin%
\definecolor{currentfill}{rgb}{0.200000,0.200000,0.800000}%
\pgfsetfillcolor{currentfill}%
\pgfsetlinewidth{1.003750pt}%
\definecolor{currentstroke}{rgb}{0.200000,0.200000,0.800000}%
\pgfsetstrokecolor{currentstroke}%
\pgfsetdash{}{0pt}%
\pgfpathmoveto{\pgfqpoint{4.019710in}{1.942466in}}%
\pgfpathcurveto{\pgfqpoint{4.025534in}{1.942466in}}{\pgfqpoint{4.031120in}{1.944780in}}{\pgfqpoint{4.035238in}{1.948898in}}%
\pgfpathcurveto{\pgfqpoint{4.039356in}{1.953016in}}{\pgfqpoint{4.041670in}{1.958602in}}{\pgfqpoint{4.041670in}{1.964426in}}%
\pgfpathcurveto{\pgfqpoint{4.041670in}{1.970250in}}{\pgfqpoint{4.039356in}{1.975836in}}{\pgfqpoint{4.035238in}{1.979955in}}%
\pgfpathcurveto{\pgfqpoint{4.031120in}{1.984073in}}{\pgfqpoint{4.025534in}{1.986387in}}{\pgfqpoint{4.019710in}{1.986387in}}%
\pgfpathcurveto{\pgfqpoint{4.013886in}{1.986387in}}{\pgfqpoint{4.008300in}{1.984073in}}{\pgfqpoint{4.004182in}{1.979955in}}%
\pgfpathcurveto{\pgfqpoint{4.000063in}{1.975836in}}{\pgfqpoint{3.997750in}{1.970250in}}{\pgfqpoint{3.997750in}{1.964426in}}%
\pgfpathcurveto{\pgfqpoint{3.997750in}{1.958602in}}{\pgfqpoint{4.000063in}{1.953016in}}{\pgfqpoint{4.004182in}{1.948898in}}%
\pgfpathcurveto{\pgfqpoint{4.008300in}{1.944780in}}{\pgfqpoint{4.013886in}{1.942466in}}{\pgfqpoint{4.019710in}{1.942466in}}%
\pgfpathlineto{\pgfqpoint{4.019710in}{1.942466in}}%
\pgfpathclose%
\pgfusepath{stroke,fill}%
\end{pgfscope}%
\begin{pgfscope}%
\pgfpathrectangle{\pgfqpoint{1.542338in}{0.880000in}}{\pgfqpoint{5.115323in}{6.160000in}}%
\pgfusepath{clip}%
\pgfsetbuttcap%
\pgfsetroundjoin%
\definecolor{currentfill}{rgb}{0.200000,0.200000,0.800000}%
\pgfsetfillcolor{currentfill}%
\pgfsetlinewidth{1.003750pt}%
\definecolor{currentstroke}{rgb}{0.200000,0.200000,0.800000}%
\pgfsetstrokecolor{currentstroke}%
\pgfsetdash{}{0pt}%
\pgfpathmoveto{\pgfqpoint{4.077438in}{2.084644in}}%
\pgfpathcurveto{\pgfqpoint{4.083262in}{2.084644in}}{\pgfqpoint{4.088848in}{2.086958in}}{\pgfqpoint{4.092967in}{2.091076in}}%
\pgfpathcurveto{\pgfqpoint{4.097085in}{2.095195in}}{\pgfqpoint{4.099399in}{2.100781in}}{\pgfqpoint{4.099399in}{2.106605in}}%
\pgfpathcurveto{\pgfqpoint{4.099399in}{2.112429in}}{\pgfqpoint{4.097085in}{2.118015in}}{\pgfqpoint{4.092967in}{2.122133in}}%
\pgfpathcurveto{\pgfqpoint{4.088848in}{2.126251in}}{\pgfqpoint{4.083262in}{2.128565in}}{\pgfqpoint{4.077438in}{2.128565in}}%
\pgfpathcurveto{\pgfqpoint{4.071614in}{2.128565in}}{\pgfqpoint{4.066028in}{2.126251in}}{\pgfqpoint{4.061910in}{2.122133in}}%
\pgfpathcurveto{\pgfqpoint{4.057792in}{2.118015in}}{\pgfqpoint{4.055478in}{2.112429in}}{\pgfqpoint{4.055478in}{2.106605in}}%
\pgfpathcurveto{\pgfqpoint{4.055478in}{2.100781in}}{\pgfqpoint{4.057792in}{2.095195in}}{\pgfqpoint{4.061910in}{2.091076in}}%
\pgfpathcurveto{\pgfqpoint{4.066028in}{2.086958in}}{\pgfqpoint{4.071614in}{2.084644in}}{\pgfqpoint{4.077438in}{2.084644in}}%
\pgfpathlineto{\pgfqpoint{4.077438in}{2.084644in}}%
\pgfpathclose%
\pgfusepath{stroke,fill}%
\end{pgfscope}%
\begin{pgfscope}%
\pgfpathrectangle{\pgfqpoint{1.542338in}{0.880000in}}{\pgfqpoint{5.115323in}{6.160000in}}%
\pgfusepath{clip}%
\pgfsetbuttcap%
\pgfsetroundjoin%
\definecolor{currentfill}{rgb}{0.200000,0.200000,0.800000}%
\pgfsetfillcolor{currentfill}%
\pgfsetlinewidth{1.003750pt}%
\definecolor{currentstroke}{rgb}{0.200000,0.200000,0.800000}%
\pgfsetstrokecolor{currentstroke}%
\pgfsetdash{}{0pt}%
\pgfpathmoveto{\pgfqpoint{4.124703in}{2.229607in}}%
\pgfpathcurveto{\pgfqpoint{4.130527in}{2.229607in}}{\pgfqpoint{4.136113in}{2.231921in}}{\pgfqpoint{4.140232in}{2.236040in}}%
\pgfpathcurveto{\pgfqpoint{4.144350in}{2.240158in}}{\pgfqpoint{4.146664in}{2.245744in}}{\pgfqpoint{4.146664in}{2.251568in}}%
\pgfpathcurveto{\pgfqpoint{4.146664in}{2.257392in}}{\pgfqpoint{4.144350in}{2.262978in}}{\pgfqpoint{4.140232in}{2.267096in}}%
\pgfpathcurveto{\pgfqpoint{4.136113in}{2.271214in}}{\pgfqpoint{4.130527in}{2.273528in}}{\pgfqpoint{4.124703in}{2.273528in}}%
\pgfpathcurveto{\pgfqpoint{4.118879in}{2.273528in}}{\pgfqpoint{4.113293in}{2.271214in}}{\pgfqpoint{4.109175in}{2.267096in}}%
\pgfpathcurveto{\pgfqpoint{4.105057in}{2.262978in}}{\pgfqpoint{4.102743in}{2.257392in}}{\pgfqpoint{4.102743in}{2.251568in}}%
\pgfpathcurveto{\pgfqpoint{4.102743in}{2.245744in}}{\pgfqpoint{4.105057in}{2.240158in}}{\pgfqpoint{4.109175in}{2.236040in}}%
\pgfpathcurveto{\pgfqpoint{4.113293in}{2.231921in}}{\pgfqpoint{4.118879in}{2.229607in}}{\pgfqpoint{4.124703in}{2.229607in}}%
\pgfpathlineto{\pgfqpoint{4.124703in}{2.229607in}}%
\pgfpathclose%
\pgfusepath{stroke,fill}%
\end{pgfscope}%
\begin{pgfscope}%
\pgfpathrectangle{\pgfqpoint{1.542338in}{0.880000in}}{\pgfqpoint{5.115323in}{6.160000in}}%
\pgfusepath{clip}%
\pgfsetbuttcap%
\pgfsetroundjoin%
\definecolor{currentfill}{rgb}{0.200000,0.200000,0.800000}%
\pgfsetfillcolor{currentfill}%
\pgfsetlinewidth{1.003750pt}%
\definecolor{currentstroke}{rgb}{0.200000,0.200000,0.800000}%
\pgfsetstrokecolor{currentstroke}%
\pgfsetdash{}{0pt}%
\pgfpathmoveto{\pgfqpoint{4.155669in}{2.378980in}}%
\pgfpathcurveto{\pgfqpoint{4.161493in}{2.378980in}}{\pgfqpoint{4.167079in}{2.381294in}}{\pgfqpoint{4.171197in}{2.385412in}}%
\pgfpathcurveto{\pgfqpoint{4.175315in}{2.389530in}}{\pgfqpoint{4.177629in}{2.395116in}}{\pgfqpoint{4.177629in}{2.400940in}}%
\pgfpathcurveto{\pgfqpoint{4.177629in}{2.406764in}}{\pgfqpoint{4.175315in}{2.412350in}}{\pgfqpoint{4.171197in}{2.416468in}}%
\pgfpathcurveto{\pgfqpoint{4.167079in}{2.420586in}}{\pgfqpoint{4.161493in}{2.422900in}}{\pgfqpoint{4.155669in}{2.422900in}}%
\pgfpathcurveto{\pgfqpoint{4.149845in}{2.422900in}}{\pgfqpoint{4.144259in}{2.420586in}}{\pgfqpoint{4.140140in}{2.416468in}}%
\pgfpathcurveto{\pgfqpoint{4.136022in}{2.412350in}}{\pgfqpoint{4.133708in}{2.406764in}}{\pgfqpoint{4.133708in}{2.400940in}}%
\pgfpathcurveto{\pgfqpoint{4.133708in}{2.395116in}}{\pgfqpoint{4.136022in}{2.389530in}}{\pgfqpoint{4.140140in}{2.385412in}}%
\pgfpathcurveto{\pgfqpoint{4.144259in}{2.381294in}}{\pgfqpoint{4.149845in}{2.378980in}}{\pgfqpoint{4.155669in}{2.378980in}}%
\pgfpathlineto{\pgfqpoint{4.155669in}{2.378980in}}%
\pgfpathclose%
\pgfusepath{stroke,fill}%
\end{pgfscope}%
\begin{pgfscope}%
\pgfpathrectangle{\pgfqpoint{1.542338in}{0.880000in}}{\pgfqpoint{5.115323in}{6.160000in}}%
\pgfusepath{clip}%
\pgfsetbuttcap%
\pgfsetroundjoin%
\definecolor{currentfill}{rgb}{0.200000,0.200000,0.800000}%
\pgfsetfillcolor{currentfill}%
\pgfsetlinewidth{1.003750pt}%
\definecolor{currentstroke}{rgb}{0.200000,0.200000,0.800000}%
\pgfsetstrokecolor{currentstroke}%
\pgfsetdash{}{0pt}%
\pgfpathmoveto{\pgfqpoint{4.170437in}{2.531321in}}%
\pgfpathcurveto{\pgfqpoint{4.176261in}{2.531321in}}{\pgfqpoint{4.181847in}{2.533635in}}{\pgfqpoint{4.185965in}{2.537753in}}%
\pgfpathcurveto{\pgfqpoint{4.190083in}{2.541872in}}{\pgfqpoint{4.192397in}{2.547458in}}{\pgfqpoint{4.192397in}{2.553282in}}%
\pgfpathcurveto{\pgfqpoint{4.192397in}{2.559106in}}{\pgfqpoint{4.190083in}{2.564692in}}{\pgfqpoint{4.185965in}{2.568810in}}%
\pgfpathcurveto{\pgfqpoint{4.181847in}{2.572928in}}{\pgfqpoint{4.176261in}{2.575242in}}{\pgfqpoint{4.170437in}{2.575242in}}%
\pgfpathcurveto{\pgfqpoint{4.164613in}{2.575242in}}{\pgfqpoint{4.159027in}{2.572928in}}{\pgfqpoint{4.154909in}{2.568810in}}%
\pgfpathcurveto{\pgfqpoint{4.150790in}{2.564692in}}{\pgfqpoint{4.148477in}{2.559106in}}{\pgfqpoint{4.148477in}{2.553282in}}%
\pgfpathcurveto{\pgfqpoint{4.148477in}{2.547458in}}{\pgfqpoint{4.150790in}{2.541872in}}{\pgfqpoint{4.154909in}{2.537753in}}%
\pgfpathcurveto{\pgfqpoint{4.159027in}{2.533635in}}{\pgfqpoint{4.164613in}{2.531321in}}{\pgfqpoint{4.170437in}{2.531321in}}%
\pgfpathlineto{\pgfqpoint{4.170437in}{2.531321in}}%
\pgfpathclose%
\pgfusepath{stroke,fill}%
\end{pgfscope}%
\begin{pgfscope}%
\pgfpathrectangle{\pgfqpoint{1.542338in}{0.880000in}}{\pgfqpoint{5.115323in}{6.160000in}}%
\pgfusepath{clip}%
\pgfsetbuttcap%
\pgfsetroundjoin%
\definecolor{currentfill}{rgb}{0.200000,0.800000,0.200000}%
\pgfsetfillcolor{currentfill}%
\pgfsetlinewidth{1.003750pt}%
\definecolor{currentstroke}{rgb}{0.200000,0.800000,0.200000}%
\pgfsetstrokecolor{currentstroke}%
\pgfsetdash{}{0pt}%
\pgfpathmoveto{\pgfqpoint{5.439255in}{5.791555in}}%
\pgfpathcurveto{\pgfqpoint{5.445079in}{5.791555in}}{\pgfqpoint{5.450665in}{5.793869in}}{\pgfqpoint{5.454783in}{5.797987in}}%
\pgfpathcurveto{\pgfqpoint{5.458901in}{5.802105in}}{\pgfqpoint{5.461215in}{5.807691in}}{\pgfqpoint{5.461215in}{5.813515in}}%
\pgfpathcurveto{\pgfqpoint{5.461215in}{5.819339in}}{\pgfqpoint{5.458901in}{5.824925in}}{\pgfqpoint{5.454783in}{5.829043in}}%
\pgfpathcurveto{\pgfqpoint{5.450665in}{5.833162in}}{\pgfqpoint{5.445079in}{5.835475in}}{\pgfqpoint{5.439255in}{5.835475in}}%
\pgfpathcurveto{\pgfqpoint{5.433431in}{5.835475in}}{\pgfqpoint{5.427845in}{5.833162in}}{\pgfqpoint{5.423726in}{5.829043in}}%
\pgfpathcurveto{\pgfqpoint{5.419608in}{5.824925in}}{\pgfqpoint{5.417294in}{5.819339in}}{\pgfqpoint{5.417294in}{5.813515in}}%
\pgfpathcurveto{\pgfqpoint{5.417294in}{5.807691in}}{\pgfqpoint{5.419608in}{5.802105in}}{\pgfqpoint{5.423726in}{5.797987in}}%
\pgfpathcurveto{\pgfqpoint{5.427845in}{5.793869in}}{\pgfqpoint{5.433431in}{5.791555in}}{\pgfqpoint{5.439255in}{5.791555in}}%
\pgfpathlineto{\pgfqpoint{5.439255in}{5.791555in}}%
\pgfpathclose%
\pgfusepath{stroke,fill}%
\end{pgfscope}%
\begin{pgfscope}%
\pgfpathrectangle{\pgfqpoint{1.542338in}{0.880000in}}{\pgfqpoint{5.115323in}{6.160000in}}%
\pgfusepath{clip}%
\pgfsetbuttcap%
\pgfsetroundjoin%
\definecolor{currentfill}{rgb}{0.200000,0.800000,0.200000}%
\pgfsetfillcolor{currentfill}%
\pgfsetlinewidth{1.003750pt}%
\definecolor{currentstroke}{rgb}{0.200000,0.800000,0.200000}%
\pgfsetstrokecolor{currentstroke}%
\pgfsetdash{}{0pt}%
\pgfpathmoveto{\pgfqpoint{5.423202in}{5.913116in}}%
\pgfpathcurveto{\pgfqpoint{5.429026in}{5.913116in}}{\pgfqpoint{5.434613in}{5.915430in}}{\pgfqpoint{5.438731in}{5.919548in}}%
\pgfpathcurveto{\pgfqpoint{5.442849in}{5.923666in}}{\pgfqpoint{5.445163in}{5.929253in}}{\pgfqpoint{5.445163in}{5.935076in}}%
\pgfpathcurveto{\pgfqpoint{5.445163in}{5.940900in}}{\pgfqpoint{5.442849in}{5.946487in}}{\pgfqpoint{5.438731in}{5.950605in}}%
\pgfpathcurveto{\pgfqpoint{5.434613in}{5.954723in}}{\pgfqpoint{5.429026in}{5.957037in}}{\pgfqpoint{5.423202in}{5.957037in}}%
\pgfpathcurveto{\pgfqpoint{5.417378in}{5.957037in}}{\pgfqpoint{5.411792in}{5.954723in}}{\pgfqpoint{5.407674in}{5.950605in}}%
\pgfpathcurveto{\pgfqpoint{5.403556in}{5.946487in}}{\pgfqpoint{5.401242in}{5.940900in}}{\pgfqpoint{5.401242in}{5.935076in}}%
\pgfpathcurveto{\pgfqpoint{5.401242in}{5.929253in}}{\pgfqpoint{5.403556in}{5.923666in}}{\pgfqpoint{5.407674in}{5.919548in}}%
\pgfpathcurveto{\pgfqpoint{5.411792in}{5.915430in}}{\pgfqpoint{5.417378in}{5.913116in}}{\pgfqpoint{5.423202in}{5.913116in}}%
\pgfpathlineto{\pgfqpoint{5.423202in}{5.913116in}}%
\pgfpathclose%
\pgfusepath{stroke,fill}%
\end{pgfscope}%
\begin{pgfscope}%
\pgfpathrectangle{\pgfqpoint{1.542338in}{0.880000in}}{\pgfqpoint{5.115323in}{6.160000in}}%
\pgfusepath{clip}%
\pgfsetbuttcap%
\pgfsetroundjoin%
\definecolor{currentfill}{rgb}{0.200000,0.800000,0.200000}%
\pgfsetfillcolor{currentfill}%
\pgfsetlinewidth{1.003750pt}%
\definecolor{currentstroke}{rgb}{0.200000,0.800000,0.200000}%
\pgfsetstrokecolor{currentstroke}%
\pgfsetdash{}{0pt}%
\pgfpathmoveto{\pgfqpoint{5.396086in}{6.031677in}}%
\pgfpathcurveto{\pgfqpoint{5.401910in}{6.031677in}}{\pgfqpoint{5.407497in}{6.033991in}}{\pgfqpoint{5.411615in}{6.038109in}}%
\pgfpathcurveto{\pgfqpoint{5.415733in}{6.042227in}}{\pgfqpoint{5.418047in}{6.047813in}}{\pgfqpoint{5.418047in}{6.053637in}}%
\pgfpathcurveto{\pgfqpoint{5.418047in}{6.059461in}}{\pgfqpoint{5.415733in}{6.065047in}}{\pgfqpoint{5.411615in}{6.069165in}}%
\pgfpathcurveto{\pgfqpoint{5.407497in}{6.073284in}}{\pgfqpoint{5.401910in}{6.075597in}}{\pgfqpoint{5.396086in}{6.075597in}}%
\pgfpathcurveto{\pgfqpoint{5.390262in}{6.075597in}}{\pgfqpoint{5.384676in}{6.073284in}}{\pgfqpoint{5.380558in}{6.069165in}}%
\pgfpathcurveto{\pgfqpoint{5.376440in}{6.065047in}}{\pgfqpoint{5.374126in}{6.059461in}}{\pgfqpoint{5.374126in}{6.053637in}}%
\pgfpathcurveto{\pgfqpoint{5.374126in}{6.047813in}}{\pgfqpoint{5.376440in}{6.042227in}}{\pgfqpoint{5.380558in}{6.038109in}}%
\pgfpathcurveto{\pgfqpoint{5.384676in}{6.033991in}}{\pgfqpoint{5.390262in}{6.031677in}}{\pgfqpoint{5.396086in}{6.031677in}}%
\pgfpathlineto{\pgfqpoint{5.396086in}{6.031677in}}%
\pgfpathclose%
\pgfusepath{stroke,fill}%
\end{pgfscope}%
\begin{pgfscope}%
\pgfpathrectangle{\pgfqpoint{1.542338in}{0.880000in}}{\pgfqpoint{5.115323in}{6.160000in}}%
\pgfusepath{clip}%
\pgfsetbuttcap%
\pgfsetroundjoin%
\definecolor{currentfill}{rgb}{0.200000,0.800000,0.200000}%
\pgfsetfillcolor{currentfill}%
\pgfsetlinewidth{1.003750pt}%
\definecolor{currentstroke}{rgb}{0.200000,0.800000,0.200000}%
\pgfsetstrokecolor{currentstroke}%
\pgfsetdash{}{0pt}%
\pgfpathmoveto{\pgfqpoint{5.362552in}{6.148700in}}%
\pgfpathcurveto{\pgfqpoint{5.368376in}{6.148700in}}{\pgfqpoint{5.373962in}{6.151014in}}{\pgfqpoint{5.378080in}{6.155132in}}%
\pgfpathcurveto{\pgfqpoint{5.382198in}{6.159250in}}{\pgfqpoint{5.384512in}{6.164836in}}{\pgfqpoint{5.384512in}{6.170660in}}%
\pgfpathcurveto{\pgfqpoint{5.384512in}{6.176484in}}{\pgfqpoint{5.382198in}{6.182070in}}{\pgfqpoint{5.378080in}{6.186188in}}%
\pgfpathcurveto{\pgfqpoint{5.373962in}{6.190306in}}{\pgfqpoint{5.368376in}{6.192620in}}{\pgfqpoint{5.362552in}{6.192620in}}%
\pgfpathcurveto{\pgfqpoint{5.356728in}{6.192620in}}{\pgfqpoint{5.351142in}{6.190306in}}{\pgfqpoint{5.347023in}{6.186188in}}%
\pgfpathcurveto{\pgfqpoint{5.342905in}{6.182070in}}{\pgfqpoint{5.340591in}{6.176484in}}{\pgfqpoint{5.340591in}{6.170660in}}%
\pgfpathcurveto{\pgfqpoint{5.340591in}{6.164836in}}{\pgfqpoint{5.342905in}{6.159250in}}{\pgfqpoint{5.347023in}{6.155132in}}%
\pgfpathcurveto{\pgfqpoint{5.351142in}{6.151014in}}{\pgfqpoint{5.356728in}{6.148700in}}{\pgfqpoint{5.362552in}{6.148700in}}%
\pgfpathlineto{\pgfqpoint{5.362552in}{6.148700in}}%
\pgfpathclose%
\pgfusepath{stroke,fill}%
\end{pgfscope}%
\begin{pgfscope}%
\pgfpathrectangle{\pgfqpoint{1.542338in}{0.880000in}}{\pgfqpoint{5.115323in}{6.160000in}}%
\pgfusepath{clip}%
\pgfsetbuttcap%
\pgfsetroundjoin%
\definecolor{currentfill}{rgb}{0.200000,0.800000,0.200000}%
\pgfsetfillcolor{currentfill}%
\pgfsetlinewidth{1.003750pt}%
\definecolor{currentstroke}{rgb}{0.200000,0.800000,0.200000}%
\pgfsetstrokecolor{currentstroke}%
\pgfsetdash{}{0pt}%
\pgfpathmoveto{\pgfqpoint{5.307444in}{6.257339in}}%
\pgfpathcurveto{\pgfqpoint{5.313268in}{6.257339in}}{\pgfqpoint{5.318854in}{6.259653in}}{\pgfqpoint{5.322972in}{6.263771in}}%
\pgfpathcurveto{\pgfqpoint{5.327091in}{6.267889in}}{\pgfqpoint{5.329404in}{6.273476in}}{\pgfqpoint{5.329404in}{6.279299in}}%
\pgfpathcurveto{\pgfqpoint{5.329404in}{6.285123in}}{\pgfqpoint{5.327091in}{6.290710in}}{\pgfqpoint{5.322972in}{6.294828in}}%
\pgfpathcurveto{\pgfqpoint{5.318854in}{6.298946in}}{\pgfqpoint{5.313268in}{6.301260in}}{\pgfqpoint{5.307444in}{6.301260in}}%
\pgfpathcurveto{\pgfqpoint{5.301620in}{6.301260in}}{\pgfqpoint{5.296034in}{6.298946in}}{\pgfqpoint{5.291916in}{6.294828in}}%
\pgfpathcurveto{\pgfqpoint{5.287798in}{6.290710in}}{\pgfqpoint{5.285484in}{6.285123in}}{\pgfqpoint{5.285484in}{6.279299in}}%
\pgfpathcurveto{\pgfqpoint{5.285484in}{6.273476in}}{\pgfqpoint{5.287798in}{6.267889in}}{\pgfqpoint{5.291916in}{6.263771in}}%
\pgfpathcurveto{\pgfqpoint{5.296034in}{6.259653in}}{\pgfqpoint{5.301620in}{6.257339in}}{\pgfqpoint{5.307444in}{6.257339in}}%
\pgfpathlineto{\pgfqpoint{5.307444in}{6.257339in}}%
\pgfpathclose%
\pgfusepath{stroke,fill}%
\end{pgfscope}%
\begin{pgfscope}%
\pgfpathrectangle{\pgfqpoint{1.542338in}{0.880000in}}{\pgfqpoint{5.115323in}{6.160000in}}%
\pgfusepath{clip}%
\pgfsetbuttcap%
\pgfsetroundjoin%
\definecolor{currentfill}{rgb}{0.200000,0.800000,0.200000}%
\pgfsetfillcolor{currentfill}%
\pgfsetlinewidth{1.003750pt}%
\definecolor{currentstroke}{rgb}{0.200000,0.800000,0.200000}%
\pgfsetstrokecolor{currentstroke}%
\pgfsetdash{}{0pt}%
\pgfpathmoveto{\pgfqpoint{5.235079in}{6.354789in}}%
\pgfpathcurveto{\pgfqpoint{5.240903in}{6.354789in}}{\pgfqpoint{5.246489in}{6.357103in}}{\pgfqpoint{5.250608in}{6.361221in}}%
\pgfpathcurveto{\pgfqpoint{5.254726in}{6.365339in}}{\pgfqpoint{5.257040in}{6.370925in}}{\pgfqpoint{5.257040in}{6.376749in}}%
\pgfpathcurveto{\pgfqpoint{5.257040in}{6.382573in}}{\pgfqpoint{5.254726in}{6.388159in}}{\pgfqpoint{5.250608in}{6.392277in}}%
\pgfpathcurveto{\pgfqpoint{5.246489in}{6.396396in}}{\pgfqpoint{5.240903in}{6.398710in}}{\pgfqpoint{5.235079in}{6.398710in}}%
\pgfpathcurveto{\pgfqpoint{5.229255in}{6.398710in}}{\pgfqpoint{5.223669in}{6.396396in}}{\pgfqpoint{5.219551in}{6.392277in}}%
\pgfpathcurveto{\pgfqpoint{5.215433in}{6.388159in}}{\pgfqpoint{5.213119in}{6.382573in}}{\pgfqpoint{5.213119in}{6.376749in}}%
\pgfpathcurveto{\pgfqpoint{5.213119in}{6.370925in}}{\pgfqpoint{5.215433in}{6.365339in}}{\pgfqpoint{5.219551in}{6.361221in}}%
\pgfpathcurveto{\pgfqpoint{5.223669in}{6.357103in}}{\pgfqpoint{5.229255in}{6.354789in}}{\pgfqpoint{5.235079in}{6.354789in}}%
\pgfpathlineto{\pgfqpoint{5.235079in}{6.354789in}}%
\pgfpathclose%
\pgfusepath{stroke,fill}%
\end{pgfscope}%
\begin{pgfscope}%
\pgfpathrectangle{\pgfqpoint{1.542338in}{0.880000in}}{\pgfqpoint{5.115323in}{6.160000in}}%
\pgfusepath{clip}%
\pgfsetbuttcap%
\pgfsetroundjoin%
\definecolor{currentfill}{rgb}{0.200000,0.800000,0.200000}%
\pgfsetfillcolor{currentfill}%
\pgfsetlinewidth{1.003750pt}%
\definecolor{currentstroke}{rgb}{0.200000,0.800000,0.200000}%
\pgfsetstrokecolor{currentstroke}%
\pgfsetdash{}{0pt}%
\pgfpathmoveto{\pgfqpoint{5.158410in}{6.448174in}}%
\pgfpathcurveto{\pgfqpoint{5.164234in}{6.448174in}}{\pgfqpoint{5.169821in}{6.450488in}}{\pgfqpoint{5.173939in}{6.454606in}}%
\pgfpathcurveto{\pgfqpoint{5.178057in}{6.458724in}}{\pgfqpoint{5.180371in}{6.464311in}}{\pgfqpoint{5.180371in}{6.470135in}}%
\pgfpathcurveto{\pgfqpoint{5.180371in}{6.475958in}}{\pgfqpoint{5.178057in}{6.481545in}}{\pgfqpoint{5.173939in}{6.485663in}}%
\pgfpathcurveto{\pgfqpoint{5.169821in}{6.489781in}}{\pgfqpoint{5.164234in}{6.492095in}}{\pgfqpoint{5.158410in}{6.492095in}}%
\pgfpathcurveto{\pgfqpoint{5.152586in}{6.492095in}}{\pgfqpoint{5.147000in}{6.489781in}}{\pgfqpoint{5.142882in}{6.485663in}}%
\pgfpathcurveto{\pgfqpoint{5.138764in}{6.481545in}}{\pgfqpoint{5.136450in}{6.475958in}}{\pgfqpoint{5.136450in}{6.470135in}}%
\pgfpathcurveto{\pgfqpoint{5.136450in}{6.464311in}}{\pgfqpoint{5.138764in}{6.458724in}}{\pgfqpoint{5.142882in}{6.454606in}}%
\pgfpathcurveto{\pgfqpoint{5.147000in}{6.450488in}}{\pgfqpoint{5.152586in}{6.448174in}}{\pgfqpoint{5.158410in}{6.448174in}}%
\pgfpathlineto{\pgfqpoint{5.158410in}{6.448174in}}%
\pgfpathclose%
\pgfusepath{stroke,fill}%
\end{pgfscope}%
\begin{pgfscope}%
\pgfpathrectangle{\pgfqpoint{1.542338in}{0.880000in}}{\pgfqpoint{5.115323in}{6.160000in}}%
\pgfusepath{clip}%
\pgfsetbuttcap%
\pgfsetroundjoin%
\definecolor{currentfill}{rgb}{0.200000,0.800000,0.200000}%
\pgfsetfillcolor{currentfill}%
\pgfsetlinewidth{1.003750pt}%
\definecolor{currentstroke}{rgb}{0.200000,0.800000,0.200000}%
\pgfsetstrokecolor{currentstroke}%
\pgfsetdash{}{0pt}%
\pgfpathmoveto{\pgfqpoint{5.073019in}{6.534680in}}%
\pgfpathcurveto{\pgfqpoint{5.078843in}{6.534680in}}{\pgfqpoint{5.084430in}{6.536994in}}{\pgfqpoint{5.088548in}{6.541112in}}%
\pgfpathcurveto{\pgfqpoint{5.092666in}{6.545230in}}{\pgfqpoint{5.094980in}{6.550816in}}{\pgfqpoint{5.094980in}{6.556640in}}%
\pgfpathcurveto{\pgfqpoint{5.094980in}{6.562464in}}{\pgfqpoint{5.092666in}{6.568050in}}{\pgfqpoint{5.088548in}{6.572168in}}%
\pgfpathcurveto{\pgfqpoint{5.084430in}{6.576287in}}{\pgfqpoint{5.078843in}{6.578600in}}{\pgfqpoint{5.073019in}{6.578600in}}%
\pgfpathcurveto{\pgfqpoint{5.067195in}{6.578600in}}{\pgfqpoint{5.061609in}{6.576287in}}{\pgfqpoint{5.057491in}{6.572168in}}%
\pgfpathcurveto{\pgfqpoint{5.053373in}{6.568050in}}{\pgfqpoint{5.051059in}{6.562464in}}{\pgfqpoint{5.051059in}{6.556640in}}%
\pgfpathcurveto{\pgfqpoint{5.051059in}{6.550816in}}{\pgfqpoint{5.053373in}{6.545230in}}{\pgfqpoint{5.057491in}{6.541112in}}%
\pgfpathcurveto{\pgfqpoint{5.061609in}{6.536994in}}{\pgfqpoint{5.067195in}{6.534680in}}{\pgfqpoint{5.073019in}{6.534680in}}%
\pgfpathlineto{\pgfqpoint{5.073019in}{6.534680in}}%
\pgfpathclose%
\pgfusepath{stroke,fill}%
\end{pgfscope}%
\begin{pgfscope}%
\pgfpathrectangle{\pgfqpoint{1.542338in}{0.880000in}}{\pgfqpoint{5.115323in}{6.160000in}}%
\pgfusepath{clip}%
\pgfsetbuttcap%
\pgfsetroundjoin%
\definecolor{currentfill}{rgb}{0.200000,0.800000,0.200000}%
\pgfsetfillcolor{currentfill}%
\pgfsetlinewidth{1.003750pt}%
\definecolor{currentstroke}{rgb}{0.200000,0.800000,0.200000}%
\pgfsetstrokecolor{currentstroke}%
\pgfsetdash{}{0pt}%
\pgfpathmoveto{\pgfqpoint{4.968984in}{6.597530in}}%
\pgfpathcurveto{\pgfqpoint{4.974808in}{6.597530in}}{\pgfqpoint{4.980394in}{6.599844in}}{\pgfqpoint{4.984512in}{6.603962in}}%
\pgfpathcurveto{\pgfqpoint{4.988630in}{6.608080in}}{\pgfqpoint{4.990944in}{6.613667in}}{\pgfqpoint{4.990944in}{6.619491in}}%
\pgfpathcurveto{\pgfqpoint{4.990944in}{6.625314in}}{\pgfqpoint{4.988630in}{6.630901in}}{\pgfqpoint{4.984512in}{6.635019in}}%
\pgfpathcurveto{\pgfqpoint{4.980394in}{6.639137in}}{\pgfqpoint{4.974808in}{6.641451in}}{\pgfqpoint{4.968984in}{6.641451in}}%
\pgfpathcurveto{\pgfqpoint{4.963160in}{6.641451in}}{\pgfqpoint{4.957574in}{6.639137in}}{\pgfqpoint{4.953456in}{6.635019in}}%
\pgfpathcurveto{\pgfqpoint{4.949338in}{6.630901in}}{\pgfqpoint{4.947024in}{6.625314in}}{\pgfqpoint{4.947024in}{6.619491in}}%
\pgfpathcurveto{\pgfqpoint{4.947024in}{6.613667in}}{\pgfqpoint{4.949338in}{6.608080in}}{\pgfqpoint{4.953456in}{6.603962in}}%
\pgfpathcurveto{\pgfqpoint{4.957574in}{6.599844in}}{\pgfqpoint{4.963160in}{6.597530in}}{\pgfqpoint{4.968984in}{6.597530in}}%
\pgfpathlineto{\pgfqpoint{4.968984in}{6.597530in}}%
\pgfpathclose%
\pgfusepath{stroke,fill}%
\end{pgfscope}%
\begin{pgfscope}%
\pgfpathrectangle{\pgfqpoint{1.542338in}{0.880000in}}{\pgfqpoint{5.115323in}{6.160000in}}%
\pgfusepath{clip}%
\pgfsetbuttcap%
\pgfsetroundjoin%
\definecolor{currentfill}{rgb}{0.200000,0.800000,0.200000}%
\pgfsetfillcolor{currentfill}%
\pgfsetlinewidth{1.003750pt}%
\definecolor{currentstroke}{rgb}{0.200000,0.800000,0.200000}%
\pgfsetstrokecolor{currentstroke}%
\pgfsetdash{}{0pt}%
\pgfpathmoveto{\pgfqpoint{4.864110in}{6.658369in}}%
\pgfpathcurveto{\pgfqpoint{4.869934in}{6.658369in}}{\pgfqpoint{4.875520in}{6.660683in}}{\pgfqpoint{4.879638in}{6.664801in}}%
\pgfpathcurveto{\pgfqpoint{4.883756in}{6.668920in}}{\pgfqpoint{4.886070in}{6.674506in}}{\pgfqpoint{4.886070in}{6.680330in}}%
\pgfpathcurveto{\pgfqpoint{4.886070in}{6.686154in}}{\pgfqpoint{4.883756in}{6.691740in}}{\pgfqpoint{4.879638in}{6.695858in}}%
\pgfpathcurveto{\pgfqpoint{4.875520in}{6.699976in}}{\pgfqpoint{4.869934in}{6.702290in}}{\pgfqpoint{4.864110in}{6.702290in}}%
\pgfpathcurveto{\pgfqpoint{4.858286in}{6.702290in}}{\pgfqpoint{4.852700in}{6.699976in}}{\pgfqpoint{4.848582in}{6.695858in}}%
\pgfpathcurveto{\pgfqpoint{4.844464in}{6.691740in}}{\pgfqpoint{4.842150in}{6.686154in}}{\pgfqpoint{4.842150in}{6.680330in}}%
\pgfpathcurveto{\pgfqpoint{4.842150in}{6.674506in}}{\pgfqpoint{4.844464in}{6.668920in}}{\pgfqpoint{4.848582in}{6.664801in}}%
\pgfpathcurveto{\pgfqpoint{4.852700in}{6.660683in}}{\pgfqpoint{4.858286in}{6.658369in}}{\pgfqpoint{4.864110in}{6.658369in}}%
\pgfpathlineto{\pgfqpoint{4.864110in}{6.658369in}}%
\pgfpathclose%
\pgfusepath{stroke,fill}%
\end{pgfscope}%
\begin{pgfscope}%
\pgfpathrectangle{\pgfqpoint{1.542338in}{0.880000in}}{\pgfqpoint{5.115323in}{6.160000in}}%
\pgfusepath{clip}%
\pgfsetbuttcap%
\pgfsetroundjoin%
\definecolor{currentfill}{rgb}{0.200000,0.800000,0.200000}%
\pgfsetfillcolor{currentfill}%
\pgfsetlinewidth{1.003750pt}%
\definecolor{currentstroke}{rgb}{0.200000,0.800000,0.200000}%
\pgfsetstrokecolor{currentstroke}%
\pgfsetdash{}{0pt}%
\pgfpathmoveto{\pgfqpoint{4.750433in}{6.701396in}}%
\pgfpathcurveto{\pgfqpoint{4.756257in}{6.701396in}}{\pgfqpoint{4.761843in}{6.703710in}}{\pgfqpoint{4.765961in}{6.707828in}}%
\pgfpathcurveto{\pgfqpoint{4.770080in}{6.711947in}}{\pgfqpoint{4.772393in}{6.717533in}}{\pgfqpoint{4.772393in}{6.723357in}}%
\pgfpathcurveto{\pgfqpoint{4.772393in}{6.729181in}}{\pgfqpoint{4.770080in}{6.734767in}}{\pgfqpoint{4.765961in}{6.738885in}}%
\pgfpathcurveto{\pgfqpoint{4.761843in}{6.743003in}}{\pgfqpoint{4.756257in}{6.745317in}}{\pgfqpoint{4.750433in}{6.745317in}}%
\pgfpathcurveto{\pgfqpoint{4.744609in}{6.745317in}}{\pgfqpoint{4.739023in}{6.743003in}}{\pgfqpoint{4.734905in}{6.738885in}}%
\pgfpathcurveto{\pgfqpoint{4.730787in}{6.734767in}}{\pgfqpoint{4.728473in}{6.729181in}}{\pgfqpoint{4.728473in}{6.723357in}}%
\pgfpathcurveto{\pgfqpoint{4.728473in}{6.717533in}}{\pgfqpoint{4.730787in}{6.711947in}}{\pgfqpoint{4.734905in}{6.707828in}}%
\pgfpathcurveto{\pgfqpoint{4.739023in}{6.703710in}}{\pgfqpoint{4.744609in}{6.701396in}}{\pgfqpoint{4.750433in}{6.701396in}}%
\pgfpathlineto{\pgfqpoint{4.750433in}{6.701396in}}%
\pgfpathclose%
\pgfusepath{stroke,fill}%
\end{pgfscope}%
\begin{pgfscope}%
\pgfpathrectangle{\pgfqpoint{1.542338in}{0.880000in}}{\pgfqpoint{5.115323in}{6.160000in}}%
\pgfusepath{clip}%
\pgfsetbuttcap%
\pgfsetroundjoin%
\definecolor{currentfill}{rgb}{0.200000,0.800000,0.200000}%
\pgfsetfillcolor{currentfill}%
\pgfsetlinewidth{1.003750pt}%
\definecolor{currentstroke}{rgb}{0.200000,0.800000,0.200000}%
\pgfsetstrokecolor{currentstroke}%
\pgfsetdash{}{0pt}%
\pgfpathmoveto{\pgfqpoint{4.631269in}{6.724751in}}%
\pgfpathcurveto{\pgfqpoint{4.637093in}{6.724751in}}{\pgfqpoint{4.642679in}{6.727065in}}{\pgfqpoint{4.646797in}{6.731183in}}%
\pgfpathcurveto{\pgfqpoint{4.650915in}{6.735301in}}{\pgfqpoint{4.653229in}{6.740888in}}{\pgfqpoint{4.653229in}{6.746711in}}%
\pgfpathcurveto{\pgfqpoint{4.653229in}{6.752535in}}{\pgfqpoint{4.650915in}{6.758122in}}{\pgfqpoint{4.646797in}{6.762240in}}%
\pgfpathcurveto{\pgfqpoint{4.642679in}{6.766358in}}{\pgfqpoint{4.637093in}{6.768672in}}{\pgfqpoint{4.631269in}{6.768672in}}%
\pgfpathcurveto{\pgfqpoint{4.625445in}{6.768672in}}{\pgfqpoint{4.619859in}{6.766358in}}{\pgfqpoint{4.615741in}{6.762240in}}%
\pgfpathcurveto{\pgfqpoint{4.611623in}{6.758122in}}{\pgfqpoint{4.609309in}{6.752535in}}{\pgfqpoint{4.609309in}{6.746711in}}%
\pgfpathcurveto{\pgfqpoint{4.609309in}{6.740888in}}{\pgfqpoint{4.611623in}{6.735301in}}{\pgfqpoint{4.615741in}{6.731183in}}%
\pgfpathcurveto{\pgfqpoint{4.619859in}{6.727065in}}{\pgfqpoint{4.625445in}{6.724751in}}{\pgfqpoint{4.631269in}{6.724751in}}%
\pgfpathlineto{\pgfqpoint{4.631269in}{6.724751in}}%
\pgfpathclose%
\pgfusepath{stroke,fill}%
\end{pgfscope}%
\begin{pgfscope}%
\pgfpathrectangle{\pgfqpoint{1.542338in}{0.880000in}}{\pgfqpoint{5.115323in}{6.160000in}}%
\pgfusepath{clip}%
\pgfsetbuttcap%
\pgfsetroundjoin%
\definecolor{currentfill}{rgb}{0.200000,0.800000,0.200000}%
\pgfsetfillcolor{currentfill}%
\pgfsetlinewidth{1.003750pt}%
\definecolor{currentstroke}{rgb}{0.200000,0.800000,0.200000}%
\pgfsetstrokecolor{currentstroke}%
\pgfsetdash{}{0pt}%
\pgfpathmoveto{\pgfqpoint{4.510749in}{6.738040in}}%
\pgfpathcurveto{\pgfqpoint{4.516573in}{6.738040in}}{\pgfqpoint{4.522159in}{6.740354in}}{\pgfqpoint{4.526277in}{6.744472in}}%
\pgfpathcurveto{\pgfqpoint{4.530395in}{6.748590in}}{\pgfqpoint{4.532709in}{6.754176in}}{\pgfqpoint{4.532709in}{6.760000in}}%
\pgfpathcurveto{\pgfqpoint{4.532709in}{6.765824in}}{\pgfqpoint{4.530395in}{6.771410in}}{\pgfqpoint{4.526277in}{6.775528in}}%
\pgfpathcurveto{\pgfqpoint{4.522159in}{6.779646in}}{\pgfqpoint{4.516573in}{6.781960in}}{\pgfqpoint{4.510749in}{6.781960in}}%
\pgfpathcurveto{\pgfqpoint{4.504925in}{6.781960in}}{\pgfqpoint{4.499339in}{6.779646in}}{\pgfqpoint{4.495221in}{6.775528in}}%
\pgfpathcurveto{\pgfqpoint{4.491103in}{6.771410in}}{\pgfqpoint{4.488789in}{6.765824in}}{\pgfqpoint{4.488789in}{6.760000in}}%
\pgfpathcurveto{\pgfqpoint{4.488789in}{6.754176in}}{\pgfqpoint{4.491103in}{6.748590in}}{\pgfqpoint{4.495221in}{6.744472in}}%
\pgfpathcurveto{\pgfqpoint{4.499339in}{6.740354in}}{\pgfqpoint{4.504925in}{6.738040in}}{\pgfqpoint{4.510749in}{6.738040in}}%
\pgfpathlineto{\pgfqpoint{4.510749in}{6.738040in}}%
\pgfpathclose%
\pgfusepath{stroke,fill}%
\end{pgfscope}%
\begin{pgfscope}%
\pgfpathrectangle{\pgfqpoint{1.542338in}{0.880000in}}{\pgfqpoint{5.115323in}{6.160000in}}%
\pgfusepath{clip}%
\pgfsetbuttcap%
\pgfsetroundjoin%
\definecolor{currentfill}{rgb}{0.200000,0.800000,0.200000}%
\pgfsetfillcolor{currentfill}%
\pgfsetlinewidth{1.003750pt}%
\definecolor{currentstroke}{rgb}{0.200000,0.800000,0.200000}%
\pgfsetstrokecolor{currentstroke}%
\pgfsetdash{}{0pt}%
\pgfpathmoveto{\pgfqpoint{4.389204in}{6.736865in}}%
\pgfpathcurveto{\pgfqpoint{4.395028in}{6.736865in}}{\pgfqpoint{4.400614in}{6.739179in}}{\pgfqpoint{4.404732in}{6.743297in}}%
\pgfpathcurveto{\pgfqpoint{4.408850in}{6.747416in}}{\pgfqpoint{4.411164in}{6.753002in}}{\pgfqpoint{4.411164in}{6.758826in}}%
\pgfpathcurveto{\pgfqpoint{4.411164in}{6.764650in}}{\pgfqpoint{4.408850in}{6.770236in}}{\pgfqpoint{4.404732in}{6.774354in}}%
\pgfpathcurveto{\pgfqpoint{4.400614in}{6.778472in}}{\pgfqpoint{4.395028in}{6.780786in}}{\pgfqpoint{4.389204in}{6.780786in}}%
\pgfpathcurveto{\pgfqpoint{4.383380in}{6.780786in}}{\pgfqpoint{4.377794in}{6.778472in}}{\pgfqpoint{4.373676in}{6.774354in}}%
\pgfpathcurveto{\pgfqpoint{4.369558in}{6.770236in}}{\pgfqpoint{4.367244in}{6.764650in}}{\pgfqpoint{4.367244in}{6.758826in}}%
\pgfpathcurveto{\pgfqpoint{4.367244in}{6.753002in}}{\pgfqpoint{4.369558in}{6.747416in}}{\pgfqpoint{4.373676in}{6.743297in}}%
\pgfpathcurveto{\pgfqpoint{4.377794in}{6.739179in}}{\pgfqpoint{4.383380in}{6.736865in}}{\pgfqpoint{4.389204in}{6.736865in}}%
\pgfpathlineto{\pgfqpoint{4.389204in}{6.736865in}}%
\pgfpathclose%
\pgfusepath{stroke,fill}%
\end{pgfscope}%
\begin{pgfscope}%
\pgfpathrectangle{\pgfqpoint{1.542338in}{0.880000in}}{\pgfqpoint{5.115323in}{6.160000in}}%
\pgfusepath{clip}%
\pgfsetbuttcap%
\pgfsetroundjoin%
\definecolor{currentfill}{rgb}{0.200000,0.800000,0.200000}%
\pgfsetfillcolor{currentfill}%
\pgfsetlinewidth{1.003750pt}%
\definecolor{currentstroke}{rgb}{0.200000,0.800000,0.200000}%
\pgfsetstrokecolor{currentstroke}%
\pgfsetdash{}{0pt}%
\pgfpathmoveto{\pgfqpoint{4.269112in}{6.717256in}}%
\pgfpathcurveto{\pgfqpoint{4.274936in}{6.717256in}}{\pgfqpoint{4.280522in}{6.719570in}}{\pgfqpoint{4.284640in}{6.723688in}}%
\pgfpathcurveto{\pgfqpoint{4.288758in}{6.727806in}}{\pgfqpoint{4.291072in}{6.733393in}}{\pgfqpoint{4.291072in}{6.739217in}}%
\pgfpathcurveto{\pgfqpoint{4.291072in}{6.745040in}}{\pgfqpoint{4.288758in}{6.750627in}}{\pgfqpoint{4.284640in}{6.754745in}}%
\pgfpathcurveto{\pgfqpoint{4.280522in}{6.758863in}}{\pgfqpoint{4.274936in}{6.761177in}}{\pgfqpoint{4.269112in}{6.761177in}}%
\pgfpathcurveto{\pgfqpoint{4.263288in}{6.761177in}}{\pgfqpoint{4.257702in}{6.758863in}}{\pgfqpoint{4.253583in}{6.754745in}}%
\pgfpathcurveto{\pgfqpoint{4.249465in}{6.750627in}}{\pgfqpoint{4.247151in}{6.745040in}}{\pgfqpoint{4.247151in}{6.739217in}}%
\pgfpathcurveto{\pgfqpoint{4.247151in}{6.733393in}}{\pgfqpoint{4.249465in}{6.727806in}}{\pgfqpoint{4.253583in}{6.723688in}}%
\pgfpathcurveto{\pgfqpoint{4.257702in}{6.719570in}}{\pgfqpoint{4.263288in}{6.717256in}}{\pgfqpoint{4.269112in}{6.717256in}}%
\pgfpathlineto{\pgfqpoint{4.269112in}{6.717256in}}%
\pgfpathclose%
\pgfusepath{stroke,fill}%
\end{pgfscope}%
\begin{pgfscope}%
\pgfpathrectangle{\pgfqpoint{1.542338in}{0.880000in}}{\pgfqpoint{5.115323in}{6.160000in}}%
\pgfusepath{clip}%
\pgfsetbuttcap%
\pgfsetroundjoin%
\definecolor{currentfill}{rgb}{0.200000,0.800000,0.200000}%
\pgfsetfillcolor{currentfill}%
\pgfsetlinewidth{1.003750pt}%
\definecolor{currentstroke}{rgb}{0.200000,0.800000,0.200000}%
\pgfsetstrokecolor{currentstroke}%
\pgfsetdash{}{0pt}%
\pgfpathmoveto{\pgfqpoint{4.152936in}{6.681372in}}%
\pgfpathcurveto{\pgfqpoint{4.158760in}{6.681372in}}{\pgfqpoint{4.164346in}{6.683686in}}{\pgfqpoint{4.168464in}{6.687804in}}%
\pgfpathcurveto{\pgfqpoint{4.172582in}{6.691922in}}{\pgfqpoint{4.174896in}{6.697509in}}{\pgfqpoint{4.174896in}{6.703332in}}%
\pgfpathcurveto{\pgfqpoint{4.174896in}{6.709156in}}{\pgfqpoint{4.172582in}{6.714743in}}{\pgfqpoint{4.168464in}{6.718861in}}%
\pgfpathcurveto{\pgfqpoint{4.164346in}{6.722979in}}{\pgfqpoint{4.158760in}{6.725293in}}{\pgfqpoint{4.152936in}{6.725293in}}%
\pgfpathcurveto{\pgfqpoint{4.147112in}{6.725293in}}{\pgfqpoint{4.141526in}{6.722979in}}{\pgfqpoint{4.137408in}{6.718861in}}%
\pgfpathcurveto{\pgfqpoint{4.133290in}{6.714743in}}{\pgfqpoint{4.130976in}{6.709156in}}{\pgfqpoint{4.130976in}{6.703332in}}%
\pgfpathcurveto{\pgfqpoint{4.130976in}{6.697509in}}{\pgfqpoint{4.133290in}{6.691922in}}{\pgfqpoint{4.137408in}{6.687804in}}%
\pgfpathcurveto{\pgfqpoint{4.141526in}{6.683686in}}{\pgfqpoint{4.147112in}{6.681372in}}{\pgfqpoint{4.152936in}{6.681372in}}%
\pgfpathlineto{\pgfqpoint{4.152936in}{6.681372in}}%
\pgfpathclose%
\pgfusepath{stroke,fill}%
\end{pgfscope}%
\begin{pgfscope}%
\pgfpathrectangle{\pgfqpoint{1.542338in}{0.880000in}}{\pgfqpoint{5.115323in}{6.160000in}}%
\pgfusepath{clip}%
\pgfsetbuttcap%
\pgfsetroundjoin%
\definecolor{currentfill}{rgb}{0.200000,0.800000,0.200000}%
\pgfsetfillcolor{currentfill}%
\pgfsetlinewidth{1.003750pt}%
\definecolor{currentstroke}{rgb}{0.200000,0.800000,0.200000}%
\pgfsetstrokecolor{currentstroke}%
\pgfsetdash{}{0pt}%
\pgfpathmoveto{\pgfqpoint{4.042693in}{6.630551in}}%
\pgfpathcurveto{\pgfqpoint{4.048517in}{6.630551in}}{\pgfqpoint{4.054104in}{6.632865in}}{\pgfqpoint{4.058222in}{6.636983in}}%
\pgfpathcurveto{\pgfqpoint{4.062340in}{6.641101in}}{\pgfqpoint{4.064654in}{6.646687in}}{\pgfqpoint{4.064654in}{6.652511in}}%
\pgfpathcurveto{\pgfqpoint{4.064654in}{6.658335in}}{\pgfqpoint{4.062340in}{6.663921in}}{\pgfqpoint{4.058222in}{6.668039in}}%
\pgfpathcurveto{\pgfqpoint{4.054104in}{6.672157in}}{\pgfqpoint{4.048517in}{6.674471in}}{\pgfqpoint{4.042693in}{6.674471in}}%
\pgfpathcurveto{\pgfqpoint{4.036869in}{6.674471in}}{\pgfqpoint{4.031283in}{6.672157in}}{\pgfqpoint{4.027165in}{6.668039in}}%
\pgfpathcurveto{\pgfqpoint{4.023047in}{6.663921in}}{\pgfqpoint{4.020733in}{6.658335in}}{\pgfqpoint{4.020733in}{6.652511in}}%
\pgfpathcurveto{\pgfqpoint{4.020733in}{6.646687in}}{\pgfqpoint{4.023047in}{6.641101in}}{\pgfqpoint{4.027165in}{6.636983in}}%
\pgfpathcurveto{\pgfqpoint{4.031283in}{6.632865in}}{\pgfqpoint{4.036869in}{6.630551in}}{\pgfqpoint{4.042693in}{6.630551in}}%
\pgfpathlineto{\pgfqpoint{4.042693in}{6.630551in}}%
\pgfpathclose%
\pgfusepath{stroke,fill}%
\end{pgfscope}%
\begin{pgfscope}%
\pgfpathrectangle{\pgfqpoint{1.542338in}{0.880000in}}{\pgfqpoint{5.115323in}{6.160000in}}%
\pgfusepath{clip}%
\pgfsetbuttcap%
\pgfsetroundjoin%
\definecolor{currentfill}{rgb}{0.200000,0.800000,0.200000}%
\pgfsetfillcolor{currentfill}%
\pgfsetlinewidth{1.003750pt}%
\definecolor{currentstroke}{rgb}{0.200000,0.800000,0.200000}%
\pgfsetstrokecolor{currentstroke}%
\pgfsetdash{}{0pt}%
\pgfpathmoveto{\pgfqpoint{3.937620in}{6.569667in}}%
\pgfpathcurveto{\pgfqpoint{3.943444in}{6.569667in}}{\pgfqpoint{3.949030in}{6.571981in}}{\pgfqpoint{3.953148in}{6.576099in}}%
\pgfpathcurveto{\pgfqpoint{3.957266in}{6.580217in}}{\pgfqpoint{3.959580in}{6.585803in}}{\pgfqpoint{3.959580in}{6.591627in}}%
\pgfpathcurveto{\pgfqpoint{3.959580in}{6.597451in}}{\pgfqpoint{3.957266in}{6.603037in}}{\pgfqpoint{3.953148in}{6.607155in}}%
\pgfpathcurveto{\pgfqpoint{3.949030in}{6.611274in}}{\pgfqpoint{3.943444in}{6.613587in}}{\pgfqpoint{3.937620in}{6.613587in}}%
\pgfpathcurveto{\pgfqpoint{3.931796in}{6.613587in}}{\pgfqpoint{3.926210in}{6.611274in}}{\pgfqpoint{3.922092in}{6.607155in}}%
\pgfpathcurveto{\pgfqpoint{3.917973in}{6.603037in}}{\pgfqpoint{3.915660in}{6.597451in}}{\pgfqpoint{3.915660in}{6.591627in}}%
\pgfpathcurveto{\pgfqpoint{3.915660in}{6.585803in}}{\pgfqpoint{3.917973in}{6.580217in}}{\pgfqpoint{3.922092in}{6.576099in}}%
\pgfpathcurveto{\pgfqpoint{3.926210in}{6.571981in}}{\pgfqpoint{3.931796in}{6.569667in}}{\pgfqpoint{3.937620in}{6.569667in}}%
\pgfpathlineto{\pgfqpoint{3.937620in}{6.569667in}}%
\pgfpathclose%
\pgfusepath{stroke,fill}%
\end{pgfscope}%
\begin{pgfscope}%
\pgfpathrectangle{\pgfqpoint{1.542338in}{0.880000in}}{\pgfqpoint{5.115323in}{6.160000in}}%
\pgfusepath{clip}%
\pgfsetbuttcap%
\pgfsetroundjoin%
\definecolor{currentfill}{rgb}{0.200000,0.800000,0.200000}%
\pgfsetfillcolor{currentfill}%
\pgfsetlinewidth{1.003750pt}%
\definecolor{currentstroke}{rgb}{0.200000,0.800000,0.200000}%
\pgfsetstrokecolor{currentstroke}%
\pgfsetdash{}{0pt}%
\pgfpathmoveto{\pgfqpoint{3.849614in}{6.486117in}}%
\pgfpathcurveto{\pgfqpoint{3.855438in}{6.486117in}}{\pgfqpoint{3.861024in}{6.488431in}}{\pgfqpoint{3.865143in}{6.492549in}}%
\pgfpathcurveto{\pgfqpoint{3.869261in}{6.496667in}}{\pgfqpoint{3.871575in}{6.502253in}}{\pgfqpoint{3.871575in}{6.508077in}}%
\pgfpathcurveto{\pgfqpoint{3.871575in}{6.513901in}}{\pgfqpoint{3.869261in}{6.519487in}}{\pgfqpoint{3.865143in}{6.523605in}}%
\pgfpathcurveto{\pgfqpoint{3.861024in}{6.527723in}}{\pgfqpoint{3.855438in}{6.530037in}}{\pgfqpoint{3.849614in}{6.530037in}}%
\pgfpathcurveto{\pgfqpoint{3.843790in}{6.530037in}}{\pgfqpoint{3.838204in}{6.527723in}}{\pgfqpoint{3.834086in}{6.523605in}}%
\pgfpathcurveto{\pgfqpoint{3.829968in}{6.519487in}}{\pgfqpoint{3.827654in}{6.513901in}}{\pgfqpoint{3.827654in}{6.508077in}}%
\pgfpathcurveto{\pgfqpoint{3.827654in}{6.502253in}}{\pgfqpoint{3.829968in}{6.496667in}}{\pgfqpoint{3.834086in}{6.492549in}}%
\pgfpathcurveto{\pgfqpoint{3.838204in}{6.488431in}}{\pgfqpoint{3.843790in}{6.486117in}}{\pgfqpoint{3.849614in}{6.486117in}}%
\pgfpathlineto{\pgfqpoint{3.849614in}{6.486117in}}%
\pgfpathclose%
\pgfusepath{stroke,fill}%
\end{pgfscope}%
\begin{pgfscope}%
\pgfpathrectangle{\pgfqpoint{1.542338in}{0.880000in}}{\pgfqpoint{5.115323in}{6.160000in}}%
\pgfusepath{clip}%
\pgfsetbuttcap%
\pgfsetroundjoin%
\definecolor{currentfill}{rgb}{0.200000,0.800000,0.200000}%
\pgfsetfillcolor{currentfill}%
\pgfsetlinewidth{1.003750pt}%
\definecolor{currentstroke}{rgb}{0.200000,0.800000,0.200000}%
\pgfsetstrokecolor{currentstroke}%
\pgfsetdash{}{0pt}%
\pgfpathmoveto{\pgfqpoint{3.755285in}{6.408852in}}%
\pgfpathcurveto{\pgfqpoint{3.761109in}{6.408852in}}{\pgfqpoint{3.766695in}{6.411166in}}{\pgfqpoint{3.770813in}{6.415284in}}%
\pgfpathcurveto{\pgfqpoint{3.774931in}{6.419402in}}{\pgfqpoint{3.777245in}{6.424988in}}{\pgfqpoint{3.777245in}{6.430812in}}%
\pgfpathcurveto{\pgfqpoint{3.777245in}{6.436636in}}{\pgfqpoint{3.774931in}{6.442222in}}{\pgfqpoint{3.770813in}{6.446341in}}%
\pgfpathcurveto{\pgfqpoint{3.766695in}{6.450459in}}{\pgfqpoint{3.761109in}{6.452773in}}{\pgfqpoint{3.755285in}{6.452773in}}%
\pgfpathcurveto{\pgfqpoint{3.749461in}{6.452773in}}{\pgfqpoint{3.743875in}{6.450459in}}{\pgfqpoint{3.739756in}{6.446341in}}%
\pgfpathcurveto{\pgfqpoint{3.735638in}{6.442222in}}{\pgfqpoint{3.733324in}{6.436636in}}{\pgfqpoint{3.733324in}{6.430812in}}%
\pgfpathcurveto{\pgfqpoint{3.733324in}{6.424988in}}{\pgfqpoint{3.735638in}{6.419402in}}{\pgfqpoint{3.739756in}{6.415284in}}%
\pgfpathcurveto{\pgfqpoint{3.743875in}{6.411166in}}{\pgfqpoint{3.749461in}{6.408852in}}{\pgfqpoint{3.755285in}{6.408852in}}%
\pgfpathlineto{\pgfqpoint{3.755285in}{6.408852in}}%
\pgfpathclose%
\pgfusepath{stroke,fill}%
\end{pgfscope}%
\begin{pgfscope}%
\pgfpathrectangle{\pgfqpoint{1.542338in}{0.880000in}}{\pgfqpoint{5.115323in}{6.160000in}}%
\pgfusepath{clip}%
\pgfsetbuttcap%
\pgfsetroundjoin%
\definecolor{currentfill}{rgb}{0.200000,0.800000,0.200000}%
\pgfsetfillcolor{currentfill}%
\pgfsetlinewidth{1.003750pt}%
\definecolor{currentstroke}{rgb}{0.200000,0.800000,0.200000}%
\pgfsetstrokecolor{currentstroke}%
\pgfsetdash{}{0pt}%
\pgfpathmoveto{\pgfqpoint{3.687710in}{6.307537in}}%
\pgfpathcurveto{\pgfqpoint{3.693534in}{6.307537in}}{\pgfqpoint{3.699120in}{6.309851in}}{\pgfqpoint{3.703238in}{6.313969in}}%
\pgfpathcurveto{\pgfqpoint{3.707356in}{6.318087in}}{\pgfqpoint{3.709670in}{6.323673in}}{\pgfqpoint{3.709670in}{6.329497in}}%
\pgfpathcurveto{\pgfqpoint{3.709670in}{6.335321in}}{\pgfqpoint{3.707356in}{6.340907in}}{\pgfqpoint{3.703238in}{6.345025in}}%
\pgfpathcurveto{\pgfqpoint{3.699120in}{6.349144in}}{\pgfqpoint{3.693534in}{6.351457in}}{\pgfqpoint{3.687710in}{6.351457in}}%
\pgfpathcurveto{\pgfqpoint{3.681886in}{6.351457in}}{\pgfqpoint{3.676300in}{6.349144in}}{\pgfqpoint{3.672182in}{6.345025in}}%
\pgfpathcurveto{\pgfqpoint{3.668064in}{6.340907in}}{\pgfqpoint{3.665750in}{6.335321in}}{\pgfqpoint{3.665750in}{6.329497in}}%
\pgfpathcurveto{\pgfqpoint{3.665750in}{6.323673in}}{\pgfqpoint{3.668064in}{6.318087in}}{\pgfqpoint{3.672182in}{6.313969in}}%
\pgfpathcurveto{\pgfqpoint{3.676300in}{6.309851in}}{\pgfqpoint{3.681886in}{6.307537in}}{\pgfqpoint{3.687710in}{6.307537in}}%
\pgfpathlineto{\pgfqpoint{3.687710in}{6.307537in}}%
\pgfpathclose%
\pgfusepath{stroke,fill}%
\end{pgfscope}%
\begin{pgfscope}%
\pgfpathrectangle{\pgfqpoint{1.542338in}{0.880000in}}{\pgfqpoint{5.115323in}{6.160000in}}%
\pgfusepath{clip}%
\pgfsetbuttcap%
\pgfsetroundjoin%
\definecolor{currentfill}{rgb}{0.200000,0.800000,0.200000}%
\pgfsetfillcolor{currentfill}%
\pgfsetlinewidth{1.003750pt}%
\definecolor{currentstroke}{rgb}{0.200000,0.800000,0.200000}%
\pgfsetstrokecolor{currentstroke}%
\pgfsetdash{}{0pt}%
\pgfpathmoveto{\pgfqpoint{3.630415in}{6.200885in}}%
\pgfpathcurveto{\pgfqpoint{3.636238in}{6.200885in}}{\pgfqpoint{3.641825in}{6.203199in}}{\pgfqpoint{3.645943in}{6.207317in}}%
\pgfpathcurveto{\pgfqpoint{3.650061in}{6.211435in}}{\pgfqpoint{3.652375in}{6.217021in}}{\pgfqpoint{3.652375in}{6.222845in}}%
\pgfpathcurveto{\pgfqpoint{3.652375in}{6.228669in}}{\pgfqpoint{3.650061in}{6.234255in}}{\pgfqpoint{3.645943in}{6.238373in}}%
\pgfpathcurveto{\pgfqpoint{3.641825in}{6.242492in}}{\pgfqpoint{3.636238in}{6.244805in}}{\pgfqpoint{3.630415in}{6.244805in}}%
\pgfpathcurveto{\pgfqpoint{3.624591in}{6.244805in}}{\pgfqpoint{3.619004in}{6.242492in}}{\pgfqpoint{3.614886in}{6.238373in}}%
\pgfpathcurveto{\pgfqpoint{3.610768in}{6.234255in}}{\pgfqpoint{3.608454in}{6.228669in}}{\pgfqpoint{3.608454in}{6.222845in}}%
\pgfpathcurveto{\pgfqpoint{3.608454in}{6.217021in}}{\pgfqpoint{3.610768in}{6.211435in}}{\pgfqpoint{3.614886in}{6.207317in}}%
\pgfpathcurveto{\pgfqpoint{3.619004in}{6.203199in}}{\pgfqpoint{3.624591in}{6.200885in}}{\pgfqpoint{3.630415in}{6.200885in}}%
\pgfpathlineto{\pgfqpoint{3.630415in}{6.200885in}}%
\pgfpathclose%
\pgfusepath{stroke,fill}%
\end{pgfscope}%
\begin{pgfscope}%
\pgfpathrectangle{\pgfqpoint{1.542338in}{0.880000in}}{\pgfqpoint{5.115323in}{6.160000in}}%
\pgfusepath{clip}%
\pgfsetbuttcap%
\pgfsetroundjoin%
\definecolor{currentfill}{rgb}{0.200000,0.800000,0.200000}%
\pgfsetfillcolor{currentfill}%
\pgfsetlinewidth{1.003750pt}%
\definecolor{currentstroke}{rgb}{0.200000,0.800000,0.200000}%
\pgfsetstrokecolor{currentstroke}%
\pgfsetdash{}{0pt}%
\pgfpathmoveto{\pgfqpoint{3.579965in}{6.090519in}}%
\pgfpathcurveto{\pgfqpoint{3.585789in}{6.090519in}}{\pgfqpoint{3.591375in}{6.092833in}}{\pgfqpoint{3.595493in}{6.096951in}}%
\pgfpathcurveto{\pgfqpoint{3.599611in}{6.101069in}}{\pgfqpoint{3.601925in}{6.106655in}}{\pgfqpoint{3.601925in}{6.112479in}}%
\pgfpathcurveto{\pgfqpoint{3.601925in}{6.118303in}}{\pgfqpoint{3.599611in}{6.123889in}}{\pgfqpoint{3.595493in}{6.128007in}}%
\pgfpathcurveto{\pgfqpoint{3.591375in}{6.132126in}}{\pgfqpoint{3.585789in}{6.134440in}}{\pgfqpoint{3.579965in}{6.134440in}}%
\pgfpathcurveto{\pgfqpoint{3.574141in}{6.134440in}}{\pgfqpoint{3.568555in}{6.132126in}}{\pgfqpoint{3.564437in}{6.128007in}}%
\pgfpathcurveto{\pgfqpoint{3.560318in}{6.123889in}}{\pgfqpoint{3.558005in}{6.118303in}}{\pgfqpoint{3.558005in}{6.112479in}}%
\pgfpathcurveto{\pgfqpoint{3.558005in}{6.106655in}}{\pgfqpoint{3.560318in}{6.101069in}}{\pgfqpoint{3.564437in}{6.096951in}}%
\pgfpathcurveto{\pgfqpoint{3.568555in}{6.092833in}}{\pgfqpoint{3.574141in}{6.090519in}}{\pgfqpoint{3.579965in}{6.090519in}}%
\pgfpathlineto{\pgfqpoint{3.579965in}{6.090519in}}%
\pgfpathclose%
\pgfusepath{stroke,fill}%
\end{pgfscope}%
\begin{pgfscope}%
\pgfpathrectangle{\pgfqpoint{1.542338in}{0.880000in}}{\pgfqpoint{5.115323in}{6.160000in}}%
\pgfusepath{clip}%
\pgfsetbuttcap%
\pgfsetroundjoin%
\definecolor{currentfill}{rgb}{0.200000,0.800000,0.200000}%
\pgfsetfillcolor{currentfill}%
\pgfsetlinewidth{1.003750pt}%
\definecolor{currentstroke}{rgb}{0.200000,0.800000,0.200000}%
\pgfsetstrokecolor{currentstroke}%
\pgfsetdash{}{0pt}%
\pgfpathmoveto{\pgfqpoint{3.549237in}{5.972898in}}%
\pgfpathcurveto{\pgfqpoint{3.555061in}{5.972898in}}{\pgfqpoint{3.560647in}{5.975212in}}{\pgfqpoint{3.564765in}{5.979330in}}%
\pgfpathcurveto{\pgfqpoint{3.568884in}{5.983448in}}{\pgfqpoint{3.571197in}{5.989035in}}{\pgfqpoint{3.571197in}{5.994859in}}%
\pgfpathcurveto{\pgfqpoint{3.571197in}{6.000682in}}{\pgfqpoint{3.568884in}{6.006269in}}{\pgfqpoint{3.564765in}{6.010387in}}%
\pgfpathcurveto{\pgfqpoint{3.560647in}{6.014505in}}{\pgfqpoint{3.555061in}{6.016819in}}{\pgfqpoint{3.549237in}{6.016819in}}%
\pgfpathcurveto{\pgfqpoint{3.543413in}{6.016819in}}{\pgfqpoint{3.537827in}{6.014505in}}{\pgfqpoint{3.533709in}{6.010387in}}%
\pgfpathcurveto{\pgfqpoint{3.529591in}{6.006269in}}{\pgfqpoint{3.527277in}{6.000682in}}{\pgfqpoint{3.527277in}{5.994859in}}%
\pgfpathcurveto{\pgfqpoint{3.527277in}{5.989035in}}{\pgfqpoint{3.529591in}{5.983448in}}{\pgfqpoint{3.533709in}{5.979330in}}%
\pgfpathcurveto{\pgfqpoint{3.537827in}{5.975212in}}{\pgfqpoint{3.543413in}{5.972898in}}{\pgfqpoint{3.549237in}{5.972898in}}%
\pgfpathlineto{\pgfqpoint{3.549237in}{5.972898in}}%
\pgfpathclose%
\pgfusepath{stroke,fill}%
\end{pgfscope}%
\begin{pgfscope}%
\pgfpathrectangle{\pgfqpoint{1.542338in}{0.880000in}}{\pgfqpoint{5.115323in}{6.160000in}}%
\pgfusepath{clip}%
\pgfsetbuttcap%
\pgfsetroundjoin%
\definecolor{currentfill}{rgb}{0.200000,0.800000,0.200000}%
\pgfsetfillcolor{currentfill}%
\pgfsetlinewidth{1.003750pt}%
\definecolor{currentstroke}{rgb}{0.200000,0.800000,0.200000}%
\pgfsetstrokecolor{currentstroke}%
\pgfsetdash{}{0pt}%
\pgfpathmoveto{\pgfqpoint{3.539893in}{5.851937in}}%
\pgfpathcurveto{\pgfqpoint{3.545717in}{5.851937in}}{\pgfqpoint{3.551303in}{5.854251in}}{\pgfqpoint{3.555422in}{5.858369in}}%
\pgfpathcurveto{\pgfqpoint{3.559540in}{5.862487in}}{\pgfqpoint{3.561854in}{5.868074in}}{\pgfqpoint{3.561854in}{5.873898in}}%
\pgfpathcurveto{\pgfqpoint{3.561854in}{5.879721in}}{\pgfqpoint{3.559540in}{5.885308in}}{\pgfqpoint{3.555422in}{5.889426in}}%
\pgfpathcurveto{\pgfqpoint{3.551303in}{5.893544in}}{\pgfqpoint{3.545717in}{5.895858in}}{\pgfqpoint{3.539893in}{5.895858in}}%
\pgfpathcurveto{\pgfqpoint{3.534069in}{5.895858in}}{\pgfqpoint{3.528483in}{5.893544in}}{\pgfqpoint{3.524365in}{5.889426in}}%
\pgfpathcurveto{\pgfqpoint{3.520247in}{5.885308in}}{\pgfqpoint{3.517933in}{5.879721in}}{\pgfqpoint{3.517933in}{5.873898in}}%
\pgfpathcurveto{\pgfqpoint{3.517933in}{5.868074in}}{\pgfqpoint{3.520247in}{5.862487in}}{\pgfqpoint{3.524365in}{5.858369in}}%
\pgfpathcurveto{\pgfqpoint{3.528483in}{5.854251in}}{\pgfqpoint{3.534069in}{5.851937in}}{\pgfqpoint{3.539893in}{5.851937in}}%
\pgfpathlineto{\pgfqpoint{3.539893in}{5.851937in}}%
\pgfpathclose%
\pgfusepath{stroke,fill}%
\end{pgfscope}%
\begin{pgfscope}%
\pgfpathrectangle{\pgfqpoint{1.542338in}{0.880000in}}{\pgfqpoint{5.115323in}{6.160000in}}%
\pgfusepath{clip}%
\pgfsetbuttcap%
\pgfsetroundjoin%
\definecolor{currentfill}{rgb}{0.200000,0.800000,0.200000}%
\pgfsetfillcolor{currentfill}%
\pgfsetlinewidth{1.003750pt}%
\definecolor{currentstroke}{rgb}{0.200000,0.800000,0.200000}%
\pgfsetstrokecolor{currentstroke}%
\pgfsetdash{}{0pt}%
\pgfpathmoveto{\pgfqpoint{3.535711in}{5.730904in}}%
\pgfpathcurveto{\pgfqpoint{3.541535in}{5.730904in}}{\pgfqpoint{3.547121in}{5.733218in}}{\pgfqpoint{3.551239in}{5.737336in}}%
\pgfpathcurveto{\pgfqpoint{3.555357in}{5.741454in}}{\pgfqpoint{3.557671in}{5.747040in}}{\pgfqpoint{3.557671in}{5.752864in}}%
\pgfpathcurveto{\pgfqpoint{3.557671in}{5.758688in}}{\pgfqpoint{3.555357in}{5.764274in}}{\pgfqpoint{3.551239in}{5.768393in}}%
\pgfpathcurveto{\pgfqpoint{3.547121in}{5.772511in}}{\pgfqpoint{3.541535in}{5.774825in}}{\pgfqpoint{3.535711in}{5.774825in}}%
\pgfpathcurveto{\pgfqpoint{3.529887in}{5.774825in}}{\pgfqpoint{3.524301in}{5.772511in}}{\pgfqpoint{3.520183in}{5.768393in}}%
\pgfpathcurveto{\pgfqpoint{3.516065in}{5.764274in}}{\pgfqpoint{3.513751in}{5.758688in}}{\pgfqpoint{3.513751in}{5.752864in}}%
\pgfpathcurveto{\pgfqpoint{3.513751in}{5.747040in}}{\pgfqpoint{3.516065in}{5.741454in}}{\pgfqpoint{3.520183in}{5.737336in}}%
\pgfpathcurveto{\pgfqpoint{3.524301in}{5.733218in}}{\pgfqpoint{3.529887in}{5.730904in}}{\pgfqpoint{3.535711in}{5.730904in}}%
\pgfpathlineto{\pgfqpoint{3.535711in}{5.730904in}}%
\pgfpathclose%
\pgfusepath{stroke,fill}%
\end{pgfscope}%
\begin{pgfscope}%
\pgfpathrectangle{\pgfqpoint{1.542338in}{0.880000in}}{\pgfqpoint{5.115323in}{6.160000in}}%
\pgfusepath{clip}%
\pgfsetbuttcap%
\pgfsetroundjoin%
\definecolor{currentfill}{rgb}{0.200000,0.800000,0.200000}%
\pgfsetfillcolor{currentfill}%
\pgfsetlinewidth{1.003750pt}%
\definecolor{currentstroke}{rgb}{0.200000,0.800000,0.200000}%
\pgfsetstrokecolor{currentstroke}%
\pgfsetdash{}{0pt}%
\pgfpathmoveto{\pgfqpoint{3.551342in}{5.610621in}}%
\pgfpathcurveto{\pgfqpoint{3.557166in}{5.610621in}}{\pgfqpoint{3.562752in}{5.612935in}}{\pgfqpoint{3.566870in}{5.617053in}}%
\pgfpathcurveto{\pgfqpoint{3.570988in}{5.621172in}}{\pgfqpoint{3.573302in}{5.626758in}}{\pgfqpoint{3.573302in}{5.632582in}}%
\pgfpathcurveto{\pgfqpoint{3.573302in}{5.638406in}}{\pgfqpoint{3.570988in}{5.643992in}}{\pgfqpoint{3.566870in}{5.648110in}}%
\pgfpathcurveto{\pgfqpoint{3.562752in}{5.652228in}}{\pgfqpoint{3.557166in}{5.654542in}}{\pgfqpoint{3.551342in}{5.654542in}}%
\pgfpathcurveto{\pgfqpoint{3.545518in}{5.654542in}}{\pgfqpoint{3.539932in}{5.652228in}}{\pgfqpoint{3.535813in}{5.648110in}}%
\pgfpathcurveto{\pgfqpoint{3.531695in}{5.643992in}}{\pgfqpoint{3.529381in}{5.638406in}}{\pgfqpoint{3.529381in}{5.632582in}}%
\pgfpathcurveto{\pgfqpoint{3.529381in}{5.626758in}}{\pgfqpoint{3.531695in}{5.621172in}}{\pgfqpoint{3.535813in}{5.617053in}}%
\pgfpathcurveto{\pgfqpoint{3.539932in}{5.612935in}}{\pgfqpoint{3.545518in}{5.610621in}}{\pgfqpoint{3.551342in}{5.610621in}}%
\pgfpathlineto{\pgfqpoint{3.551342in}{5.610621in}}%
\pgfpathclose%
\pgfusepath{stroke,fill}%
\end{pgfscope}%
\begin{pgfscope}%
\pgfpathrectangle{\pgfqpoint{1.542338in}{0.880000in}}{\pgfqpoint{5.115323in}{6.160000in}}%
\pgfusepath{clip}%
\pgfsetbuttcap%
\pgfsetroundjoin%
\definecolor{currentfill}{rgb}{0.200000,0.800000,0.200000}%
\pgfsetfillcolor{currentfill}%
\pgfsetlinewidth{1.003750pt}%
\definecolor{currentstroke}{rgb}{0.200000,0.800000,0.200000}%
\pgfsetstrokecolor{currentstroke}%
\pgfsetdash{}{0pt}%
\pgfpathmoveto{\pgfqpoint{3.582355in}{5.493384in}}%
\pgfpathcurveto{\pgfqpoint{3.588179in}{5.493384in}}{\pgfqpoint{3.593765in}{5.495698in}}{\pgfqpoint{3.597883in}{5.499816in}}%
\pgfpathcurveto{\pgfqpoint{3.602002in}{5.503935in}}{\pgfqpoint{3.604315in}{5.509521in}}{\pgfqpoint{3.604315in}{5.515345in}}%
\pgfpathcurveto{\pgfqpoint{3.604315in}{5.521169in}}{\pgfqpoint{3.602002in}{5.526755in}}{\pgfqpoint{3.597883in}{5.530873in}}%
\pgfpathcurveto{\pgfqpoint{3.593765in}{5.534991in}}{\pgfqpoint{3.588179in}{5.537305in}}{\pgfqpoint{3.582355in}{5.537305in}}%
\pgfpathcurveto{\pgfqpoint{3.576531in}{5.537305in}}{\pgfqpoint{3.570945in}{5.534991in}}{\pgfqpoint{3.566827in}{5.530873in}}%
\pgfpathcurveto{\pgfqpoint{3.562709in}{5.526755in}}{\pgfqpoint{3.560395in}{5.521169in}}{\pgfqpoint{3.560395in}{5.515345in}}%
\pgfpathcurveto{\pgfqpoint{3.560395in}{5.509521in}}{\pgfqpoint{3.562709in}{5.503935in}}{\pgfqpoint{3.566827in}{5.499816in}}%
\pgfpathcurveto{\pgfqpoint{3.570945in}{5.495698in}}{\pgfqpoint{3.576531in}{5.493384in}}{\pgfqpoint{3.582355in}{5.493384in}}%
\pgfpathlineto{\pgfqpoint{3.582355in}{5.493384in}}%
\pgfpathclose%
\pgfusepath{stroke,fill}%
\end{pgfscope}%
\begin{pgfscope}%
\pgfpathrectangle{\pgfqpoint{1.542338in}{0.880000in}}{\pgfqpoint{5.115323in}{6.160000in}}%
\pgfusepath{clip}%
\pgfsetbuttcap%
\pgfsetroundjoin%
\definecolor{currentfill}{rgb}{0.200000,0.800000,0.200000}%
\pgfsetfillcolor{currentfill}%
\pgfsetlinewidth{1.003750pt}%
\definecolor{currentstroke}{rgb}{0.200000,0.800000,0.200000}%
\pgfsetstrokecolor{currentstroke}%
\pgfsetdash{}{0pt}%
\pgfpathmoveto{\pgfqpoint{3.621882in}{5.378116in}}%
\pgfpathcurveto{\pgfqpoint{3.627706in}{5.378116in}}{\pgfqpoint{3.633293in}{5.380430in}}{\pgfqpoint{3.637411in}{5.384548in}}%
\pgfpathcurveto{\pgfqpoint{3.641529in}{5.388666in}}{\pgfqpoint{3.643843in}{5.394252in}}{\pgfqpoint{3.643843in}{5.400076in}}%
\pgfpathcurveto{\pgfqpoint{3.643843in}{5.405900in}}{\pgfqpoint{3.641529in}{5.411486in}}{\pgfqpoint{3.637411in}{5.415605in}}%
\pgfpathcurveto{\pgfqpoint{3.633293in}{5.419723in}}{\pgfqpoint{3.627706in}{5.422037in}}{\pgfqpoint{3.621882in}{5.422037in}}%
\pgfpathcurveto{\pgfqpoint{3.616058in}{5.422037in}}{\pgfqpoint{3.610472in}{5.419723in}}{\pgfqpoint{3.606354in}{5.415605in}}%
\pgfpathcurveto{\pgfqpoint{3.602236in}{5.411486in}}{\pgfqpoint{3.599922in}{5.405900in}}{\pgfqpoint{3.599922in}{5.400076in}}%
\pgfpathcurveto{\pgfqpoint{3.599922in}{5.394252in}}{\pgfqpoint{3.602236in}{5.388666in}}{\pgfqpoint{3.606354in}{5.384548in}}%
\pgfpathcurveto{\pgfqpoint{3.610472in}{5.380430in}}{\pgfqpoint{3.616058in}{5.378116in}}{\pgfqpoint{3.621882in}{5.378116in}}%
\pgfpathlineto{\pgfqpoint{3.621882in}{5.378116in}}%
\pgfpathclose%
\pgfusepath{stroke,fill}%
\end{pgfscope}%
\begin{pgfscope}%
\pgfpathrectangle{\pgfqpoint{1.542338in}{0.880000in}}{\pgfqpoint{5.115323in}{6.160000in}}%
\pgfusepath{clip}%
\pgfsetbuttcap%
\pgfsetroundjoin%
\definecolor{currentfill}{rgb}{0.200000,0.800000,0.200000}%
\pgfsetfillcolor{currentfill}%
\pgfsetlinewidth{1.003750pt}%
\definecolor{currentstroke}{rgb}{0.200000,0.800000,0.200000}%
\pgfsetstrokecolor{currentstroke}%
\pgfsetdash{}{0pt}%
\pgfpathmoveto{\pgfqpoint{3.687332in}{5.275327in}}%
\pgfpathcurveto{\pgfqpoint{3.693156in}{5.275327in}}{\pgfqpoint{3.698742in}{5.277641in}}{\pgfqpoint{3.702860in}{5.281759in}}%
\pgfpathcurveto{\pgfqpoint{3.706978in}{5.285877in}}{\pgfqpoint{3.709292in}{5.291463in}}{\pgfqpoint{3.709292in}{5.297287in}}%
\pgfpathcurveto{\pgfqpoint{3.709292in}{5.303111in}}{\pgfqpoint{3.706978in}{5.308697in}}{\pgfqpoint{3.702860in}{5.312815in}}%
\pgfpathcurveto{\pgfqpoint{3.698742in}{5.316933in}}{\pgfqpoint{3.693156in}{5.319247in}}{\pgfqpoint{3.687332in}{5.319247in}}%
\pgfpathcurveto{\pgfqpoint{3.681508in}{5.319247in}}{\pgfqpoint{3.675922in}{5.316933in}}{\pgfqpoint{3.671804in}{5.312815in}}%
\pgfpathcurveto{\pgfqpoint{3.667686in}{5.308697in}}{\pgfqpoint{3.665372in}{5.303111in}}{\pgfqpoint{3.665372in}{5.297287in}}%
\pgfpathcurveto{\pgfqpoint{3.665372in}{5.291463in}}{\pgfqpoint{3.667686in}{5.285877in}}{\pgfqpoint{3.671804in}{5.281759in}}%
\pgfpathcurveto{\pgfqpoint{3.675922in}{5.277641in}}{\pgfqpoint{3.681508in}{5.275327in}}{\pgfqpoint{3.687332in}{5.275327in}}%
\pgfpathlineto{\pgfqpoint{3.687332in}{5.275327in}}%
\pgfpathclose%
\pgfusepath{stroke,fill}%
\end{pgfscope}%
\begin{pgfscope}%
\pgfpathrectangle{\pgfqpoint{1.542338in}{0.880000in}}{\pgfqpoint{5.115323in}{6.160000in}}%
\pgfusepath{clip}%
\pgfsetbuttcap%
\pgfsetroundjoin%
\definecolor{currentfill}{rgb}{0.200000,0.800000,0.200000}%
\pgfsetfillcolor{currentfill}%
\pgfsetlinewidth{1.003750pt}%
\definecolor{currentstroke}{rgb}{0.200000,0.800000,0.200000}%
\pgfsetstrokecolor{currentstroke}%
\pgfsetdash{}{0pt}%
\pgfpathmoveto{\pgfqpoint{3.760192in}{5.178436in}}%
\pgfpathcurveto{\pgfqpoint{3.766016in}{5.178436in}}{\pgfqpoint{3.771602in}{5.180750in}}{\pgfqpoint{3.775720in}{5.184868in}}%
\pgfpathcurveto{\pgfqpoint{3.779839in}{5.188986in}}{\pgfqpoint{3.782152in}{5.194572in}}{\pgfqpoint{3.782152in}{5.200396in}}%
\pgfpathcurveto{\pgfqpoint{3.782152in}{5.206220in}}{\pgfqpoint{3.779839in}{5.211806in}}{\pgfqpoint{3.775720in}{5.215924in}}%
\pgfpathcurveto{\pgfqpoint{3.771602in}{5.220042in}}{\pgfqpoint{3.766016in}{5.222356in}}{\pgfqpoint{3.760192in}{5.222356in}}%
\pgfpathcurveto{\pgfqpoint{3.754368in}{5.222356in}}{\pgfqpoint{3.748782in}{5.220042in}}{\pgfqpoint{3.744664in}{5.215924in}}%
\pgfpathcurveto{\pgfqpoint{3.740546in}{5.211806in}}{\pgfqpoint{3.738232in}{5.206220in}}{\pgfqpoint{3.738232in}{5.200396in}}%
\pgfpathcurveto{\pgfqpoint{3.738232in}{5.194572in}}{\pgfqpoint{3.740546in}{5.188986in}}{\pgfqpoint{3.744664in}{5.184868in}}%
\pgfpathcurveto{\pgfqpoint{3.748782in}{5.180750in}}{\pgfqpoint{3.754368in}{5.178436in}}{\pgfqpoint{3.760192in}{5.178436in}}%
\pgfpathlineto{\pgfqpoint{3.760192in}{5.178436in}}%
\pgfpathclose%
\pgfusepath{stroke,fill}%
\end{pgfscope}%
\begin{pgfscope}%
\pgfpathrectangle{\pgfqpoint{1.542338in}{0.880000in}}{\pgfqpoint{5.115323in}{6.160000in}}%
\pgfusepath{clip}%
\pgfsetbuttcap%
\pgfsetroundjoin%
\definecolor{currentfill}{rgb}{0.200000,0.800000,0.200000}%
\pgfsetfillcolor{currentfill}%
\pgfsetlinewidth{1.003750pt}%
\definecolor{currentstroke}{rgb}{0.200000,0.800000,0.200000}%
\pgfsetstrokecolor{currentstroke}%
\pgfsetdash{}{0pt}%
\pgfpathmoveto{\pgfqpoint{3.845995in}{5.093007in}}%
\pgfpathcurveto{\pgfqpoint{3.851819in}{5.093007in}}{\pgfqpoint{3.857405in}{5.095321in}}{\pgfqpoint{3.861523in}{5.099439in}}%
\pgfpathcurveto{\pgfqpoint{3.865641in}{5.103558in}}{\pgfqpoint{3.867955in}{5.109144in}}{\pgfqpoint{3.867955in}{5.114968in}}%
\pgfpathcurveto{\pgfqpoint{3.867955in}{5.120792in}}{\pgfqpoint{3.865641in}{5.126378in}}{\pgfqpoint{3.861523in}{5.130496in}}%
\pgfpathcurveto{\pgfqpoint{3.857405in}{5.134614in}}{\pgfqpoint{3.851819in}{5.136928in}}{\pgfqpoint{3.845995in}{5.136928in}}%
\pgfpathcurveto{\pgfqpoint{3.840171in}{5.136928in}}{\pgfqpoint{3.834585in}{5.134614in}}{\pgfqpoint{3.830466in}{5.130496in}}%
\pgfpathcurveto{\pgfqpoint{3.826348in}{5.126378in}}{\pgfqpoint{3.824034in}{5.120792in}}{\pgfqpoint{3.824034in}{5.114968in}}%
\pgfpathcurveto{\pgfqpoint{3.824034in}{5.109144in}}{\pgfqpoint{3.826348in}{5.103558in}}{\pgfqpoint{3.830466in}{5.099439in}}%
\pgfpathcurveto{\pgfqpoint{3.834585in}{5.095321in}}{\pgfqpoint{3.840171in}{5.093007in}}{\pgfqpoint{3.845995in}{5.093007in}}%
\pgfpathlineto{\pgfqpoint{3.845995in}{5.093007in}}%
\pgfpathclose%
\pgfusepath{stroke,fill}%
\end{pgfscope}%
\begin{pgfscope}%
\pgfpathrectangle{\pgfqpoint{1.542338in}{0.880000in}}{\pgfqpoint{5.115323in}{6.160000in}}%
\pgfusepath{clip}%
\pgfsetbuttcap%
\pgfsetroundjoin%
\definecolor{currentfill}{rgb}{0.200000,0.800000,0.200000}%
\pgfsetfillcolor{currentfill}%
\pgfsetlinewidth{1.003750pt}%
\definecolor{currentstroke}{rgb}{0.200000,0.800000,0.200000}%
\pgfsetstrokecolor{currentstroke}%
\pgfsetdash{}{0pt}%
\pgfpathmoveto{\pgfqpoint{3.940863in}{5.018093in}}%
\pgfpathcurveto{\pgfqpoint{3.946687in}{5.018093in}}{\pgfqpoint{3.952273in}{5.020406in}}{\pgfqpoint{3.956391in}{5.024525in}}%
\pgfpathcurveto{\pgfqpoint{3.960510in}{5.028643in}}{\pgfqpoint{3.962823in}{5.034229in}}{\pgfqpoint{3.962823in}{5.040053in}}%
\pgfpathcurveto{\pgfqpoint{3.962823in}{5.045877in}}{\pgfqpoint{3.960510in}{5.051463in}}{\pgfqpoint{3.956391in}{5.055581in}}%
\pgfpathcurveto{\pgfqpoint{3.952273in}{5.059699in}}{\pgfqpoint{3.946687in}{5.062013in}}{\pgfqpoint{3.940863in}{5.062013in}}%
\pgfpathcurveto{\pgfqpoint{3.935039in}{5.062013in}}{\pgfqpoint{3.929453in}{5.059699in}}{\pgfqpoint{3.925335in}{5.055581in}}%
\pgfpathcurveto{\pgfqpoint{3.921217in}{5.051463in}}{\pgfqpoint{3.918903in}{5.045877in}}{\pgfqpoint{3.918903in}{5.040053in}}%
\pgfpathcurveto{\pgfqpoint{3.918903in}{5.034229in}}{\pgfqpoint{3.921217in}{5.028643in}}{\pgfqpoint{3.925335in}{5.024525in}}%
\pgfpathcurveto{\pgfqpoint{3.929453in}{5.020406in}}{\pgfqpoint{3.935039in}{5.018093in}}{\pgfqpoint{3.940863in}{5.018093in}}%
\pgfpathlineto{\pgfqpoint{3.940863in}{5.018093in}}%
\pgfpathclose%
\pgfusepath{stroke,fill}%
\end{pgfscope}%
\begin{pgfscope}%
\pgfpathrectangle{\pgfqpoint{1.542338in}{0.880000in}}{\pgfqpoint{5.115323in}{6.160000in}}%
\pgfusepath{clip}%
\pgfsetbuttcap%
\pgfsetroundjoin%
\definecolor{currentfill}{rgb}{0.200000,0.800000,0.200000}%
\pgfsetfillcolor{currentfill}%
\pgfsetlinewidth{1.003750pt}%
\definecolor{currentstroke}{rgb}{0.200000,0.800000,0.200000}%
\pgfsetstrokecolor{currentstroke}%
\pgfsetdash{}{0pt}%
\pgfpathmoveto{\pgfqpoint{4.041941in}{4.951118in}}%
\pgfpathcurveto{\pgfqpoint{4.047765in}{4.951118in}}{\pgfqpoint{4.053352in}{4.953431in}}{\pgfqpoint{4.057470in}{4.957550in}}%
\pgfpathcurveto{\pgfqpoint{4.061588in}{4.961668in}}{\pgfqpoint{4.063902in}{4.967254in}}{\pgfqpoint{4.063902in}{4.973078in}}%
\pgfpathcurveto{\pgfqpoint{4.063902in}{4.978902in}}{\pgfqpoint{4.061588in}{4.984488in}}{\pgfqpoint{4.057470in}{4.988606in}}%
\pgfpathcurveto{\pgfqpoint{4.053352in}{4.992724in}}{\pgfqpoint{4.047765in}{4.995038in}}{\pgfqpoint{4.041941in}{4.995038in}}%
\pgfpathcurveto{\pgfqpoint{4.036117in}{4.995038in}}{\pgfqpoint{4.030531in}{4.992724in}}{\pgfqpoint{4.026413in}{4.988606in}}%
\pgfpathcurveto{\pgfqpoint{4.022295in}{4.984488in}}{\pgfqpoint{4.019981in}{4.978902in}}{\pgfqpoint{4.019981in}{4.973078in}}%
\pgfpathcurveto{\pgfqpoint{4.019981in}{4.967254in}}{\pgfqpoint{4.022295in}{4.961668in}}{\pgfqpoint{4.026413in}{4.957550in}}%
\pgfpathcurveto{\pgfqpoint{4.030531in}{4.953431in}}{\pgfqpoint{4.036117in}{4.951118in}}{\pgfqpoint{4.041941in}{4.951118in}}%
\pgfpathlineto{\pgfqpoint{4.041941in}{4.951118in}}%
\pgfpathclose%
\pgfusepath{stroke,fill}%
\end{pgfscope}%
\begin{pgfscope}%
\pgfpathrectangle{\pgfqpoint{1.542338in}{0.880000in}}{\pgfqpoint{5.115323in}{6.160000in}}%
\pgfusepath{clip}%
\pgfsetbuttcap%
\pgfsetroundjoin%
\definecolor{currentfill}{rgb}{0.200000,0.800000,0.200000}%
\pgfsetfillcolor{currentfill}%
\pgfsetlinewidth{1.003750pt}%
\definecolor{currentstroke}{rgb}{0.200000,0.800000,0.200000}%
\pgfsetstrokecolor{currentstroke}%
\pgfsetdash{}{0pt}%
\pgfpathmoveto{\pgfqpoint{4.154209in}{4.905196in}}%
\pgfpathcurveto{\pgfqpoint{4.160033in}{4.905196in}}{\pgfqpoint{4.165619in}{4.907510in}}{\pgfqpoint{4.169737in}{4.911628in}}%
\pgfpathcurveto{\pgfqpoint{4.173855in}{4.915746in}}{\pgfqpoint{4.176169in}{4.921332in}}{\pgfqpoint{4.176169in}{4.927156in}}%
\pgfpathcurveto{\pgfqpoint{4.176169in}{4.932980in}}{\pgfqpoint{4.173855in}{4.938566in}}{\pgfqpoint{4.169737in}{4.942685in}}%
\pgfpathcurveto{\pgfqpoint{4.165619in}{4.946803in}}{\pgfqpoint{4.160033in}{4.949117in}}{\pgfqpoint{4.154209in}{4.949117in}}%
\pgfpathcurveto{\pgfqpoint{4.148385in}{4.949117in}}{\pgfqpoint{4.142799in}{4.946803in}}{\pgfqpoint{4.138681in}{4.942685in}}%
\pgfpathcurveto{\pgfqpoint{4.134562in}{4.938566in}}{\pgfqpoint{4.132249in}{4.932980in}}{\pgfqpoint{4.132249in}{4.927156in}}%
\pgfpathcurveto{\pgfqpoint{4.132249in}{4.921332in}}{\pgfqpoint{4.134562in}{4.915746in}}{\pgfqpoint{4.138681in}{4.911628in}}%
\pgfpathcurveto{\pgfqpoint{4.142799in}{4.907510in}}{\pgfqpoint{4.148385in}{4.905196in}}{\pgfqpoint{4.154209in}{4.905196in}}%
\pgfpathlineto{\pgfqpoint{4.154209in}{4.905196in}}%
\pgfpathclose%
\pgfusepath{stroke,fill}%
\end{pgfscope}%
\begin{pgfscope}%
\pgfpathrectangle{\pgfqpoint{1.542338in}{0.880000in}}{\pgfqpoint{5.115323in}{6.160000in}}%
\pgfusepath{clip}%
\pgfsetbuttcap%
\pgfsetroundjoin%
\definecolor{currentfill}{rgb}{0.200000,0.800000,0.200000}%
\pgfsetfillcolor{currentfill}%
\pgfsetlinewidth{1.003750pt}%
\definecolor{currentstroke}{rgb}{0.200000,0.800000,0.200000}%
\pgfsetstrokecolor{currentstroke}%
\pgfsetdash{}{0pt}%
\pgfpathmoveto{\pgfqpoint{4.269980in}{4.869658in}}%
\pgfpathcurveto{\pgfqpoint{4.275804in}{4.869658in}}{\pgfqpoint{4.281390in}{4.871972in}}{\pgfqpoint{4.285508in}{4.876090in}}%
\pgfpathcurveto{\pgfqpoint{4.289626in}{4.880208in}}{\pgfqpoint{4.291940in}{4.885794in}}{\pgfqpoint{4.291940in}{4.891618in}}%
\pgfpathcurveto{\pgfqpoint{4.291940in}{4.897442in}}{\pgfqpoint{4.289626in}{4.903028in}}{\pgfqpoint{4.285508in}{4.907146in}}%
\pgfpathcurveto{\pgfqpoint{4.281390in}{4.911264in}}{\pgfqpoint{4.275804in}{4.913578in}}{\pgfqpoint{4.269980in}{4.913578in}}%
\pgfpathcurveto{\pgfqpoint{4.264156in}{4.913578in}}{\pgfqpoint{4.258570in}{4.911264in}}{\pgfqpoint{4.254452in}{4.907146in}}%
\pgfpathcurveto{\pgfqpoint{4.250334in}{4.903028in}}{\pgfqpoint{4.248020in}{4.897442in}}{\pgfqpoint{4.248020in}{4.891618in}}%
\pgfpathcurveto{\pgfqpoint{4.248020in}{4.885794in}}{\pgfqpoint{4.250334in}{4.880208in}}{\pgfqpoint{4.254452in}{4.876090in}}%
\pgfpathcurveto{\pgfqpoint{4.258570in}{4.871972in}}{\pgfqpoint{4.264156in}{4.869658in}}{\pgfqpoint{4.269980in}{4.869658in}}%
\pgfpathlineto{\pgfqpoint{4.269980in}{4.869658in}}%
\pgfpathclose%
\pgfusepath{stroke,fill}%
\end{pgfscope}%
\begin{pgfscope}%
\pgfpathrectangle{\pgfqpoint{1.542338in}{0.880000in}}{\pgfqpoint{5.115323in}{6.160000in}}%
\pgfusepath{clip}%
\pgfsetbuttcap%
\pgfsetroundjoin%
\definecolor{currentfill}{rgb}{0.200000,0.800000,0.200000}%
\pgfsetfillcolor{currentfill}%
\pgfsetlinewidth{1.003750pt}%
\definecolor{currentstroke}{rgb}{0.200000,0.800000,0.200000}%
\pgfsetstrokecolor{currentstroke}%
\pgfsetdash{}{0pt}%
\pgfpathmoveto{\pgfqpoint{4.389448in}{4.848776in}}%
\pgfpathcurveto{\pgfqpoint{4.395272in}{4.848776in}}{\pgfqpoint{4.400858in}{4.851090in}}{\pgfqpoint{4.404977in}{4.855208in}}%
\pgfpathcurveto{\pgfqpoint{4.409095in}{4.859326in}}{\pgfqpoint{4.411409in}{4.864913in}}{\pgfqpoint{4.411409in}{4.870737in}}%
\pgfpathcurveto{\pgfqpoint{4.411409in}{4.876560in}}{\pgfqpoint{4.409095in}{4.882147in}}{\pgfqpoint{4.404977in}{4.886265in}}%
\pgfpathcurveto{\pgfqpoint{4.400858in}{4.890383in}}{\pgfqpoint{4.395272in}{4.892697in}}{\pgfqpoint{4.389448in}{4.892697in}}%
\pgfpathcurveto{\pgfqpoint{4.383624in}{4.892697in}}{\pgfqpoint{4.378038in}{4.890383in}}{\pgfqpoint{4.373920in}{4.886265in}}%
\pgfpathcurveto{\pgfqpoint{4.369802in}{4.882147in}}{\pgfqpoint{4.367488in}{4.876560in}}{\pgfqpoint{4.367488in}{4.870737in}}%
\pgfpathcurveto{\pgfqpoint{4.367488in}{4.864913in}}{\pgfqpoint{4.369802in}{4.859326in}}{\pgfqpoint{4.373920in}{4.855208in}}%
\pgfpathcurveto{\pgfqpoint{4.378038in}{4.851090in}}{\pgfqpoint{4.383624in}{4.848776in}}{\pgfqpoint{4.389448in}{4.848776in}}%
\pgfpathlineto{\pgfqpoint{4.389448in}{4.848776in}}%
\pgfpathclose%
\pgfusepath{stroke,fill}%
\end{pgfscope}%
\begin{pgfscope}%
\pgfpathrectangle{\pgfqpoint{1.542338in}{0.880000in}}{\pgfqpoint{5.115323in}{6.160000in}}%
\pgfusepath{clip}%
\pgfsetbuttcap%
\pgfsetroundjoin%
\definecolor{currentfill}{rgb}{0.200000,0.800000,0.200000}%
\pgfsetfillcolor{currentfill}%
\pgfsetlinewidth{1.003750pt}%
\definecolor{currentstroke}{rgb}{0.200000,0.800000,0.200000}%
\pgfsetstrokecolor{currentstroke}%
\pgfsetdash{}{0pt}%
\pgfpathmoveto{\pgfqpoint{4.510738in}{4.845426in}}%
\pgfpathcurveto{\pgfqpoint{4.516561in}{4.845426in}}{\pgfqpoint{4.522148in}{4.847740in}}{\pgfqpoint{4.526266in}{4.851858in}}%
\pgfpathcurveto{\pgfqpoint{4.530384in}{4.855976in}}{\pgfqpoint{4.532698in}{4.861562in}}{\pgfqpoint{4.532698in}{4.867386in}}%
\pgfpathcurveto{\pgfqpoint{4.532698in}{4.873210in}}{\pgfqpoint{4.530384in}{4.878796in}}{\pgfqpoint{4.526266in}{4.882914in}}%
\pgfpathcurveto{\pgfqpoint{4.522148in}{4.887032in}}{\pgfqpoint{4.516561in}{4.889346in}}{\pgfqpoint{4.510738in}{4.889346in}}%
\pgfpathcurveto{\pgfqpoint{4.504914in}{4.889346in}}{\pgfqpoint{4.499327in}{4.887032in}}{\pgfqpoint{4.495209in}{4.882914in}}%
\pgfpathcurveto{\pgfqpoint{4.491091in}{4.878796in}}{\pgfqpoint{4.488777in}{4.873210in}}{\pgfqpoint{4.488777in}{4.867386in}}%
\pgfpathcurveto{\pgfqpoint{4.488777in}{4.861562in}}{\pgfqpoint{4.491091in}{4.855976in}}{\pgfqpoint{4.495209in}{4.851858in}}%
\pgfpathcurveto{\pgfqpoint{4.499327in}{4.847740in}}{\pgfqpoint{4.504914in}{4.845426in}}{\pgfqpoint{4.510738in}{4.845426in}}%
\pgfpathlineto{\pgfqpoint{4.510738in}{4.845426in}}%
\pgfpathclose%
\pgfusepath{stroke,fill}%
\end{pgfscope}%
\begin{pgfscope}%
\pgfpathrectangle{\pgfqpoint{1.542338in}{0.880000in}}{\pgfqpoint{5.115323in}{6.160000in}}%
\pgfusepath{clip}%
\pgfsetbuttcap%
\pgfsetroundjoin%
\definecolor{currentfill}{rgb}{0.200000,0.800000,0.200000}%
\pgfsetfillcolor{currentfill}%
\pgfsetlinewidth{1.003750pt}%
\definecolor{currentstroke}{rgb}{0.200000,0.800000,0.200000}%
\pgfsetstrokecolor{currentstroke}%
\pgfsetdash{}{0pt}%
\pgfpathmoveto{\pgfqpoint{4.631522in}{4.856791in}}%
\pgfpathcurveto{\pgfqpoint{4.637346in}{4.856791in}}{\pgfqpoint{4.642933in}{4.859105in}}{\pgfqpoint{4.647051in}{4.863223in}}%
\pgfpathcurveto{\pgfqpoint{4.651169in}{4.867341in}}{\pgfqpoint{4.653483in}{4.872927in}}{\pgfqpoint{4.653483in}{4.878751in}}%
\pgfpathcurveto{\pgfqpoint{4.653483in}{4.884575in}}{\pgfqpoint{4.651169in}{4.890161in}}{\pgfqpoint{4.647051in}{4.894279in}}%
\pgfpathcurveto{\pgfqpoint{4.642933in}{4.898398in}}{\pgfqpoint{4.637346in}{4.900711in}}{\pgfqpoint{4.631522in}{4.900711in}}%
\pgfpathcurveto{\pgfqpoint{4.625698in}{4.900711in}}{\pgfqpoint{4.620112in}{4.898398in}}{\pgfqpoint{4.615994in}{4.894279in}}%
\pgfpathcurveto{\pgfqpoint{4.611876in}{4.890161in}}{\pgfqpoint{4.609562in}{4.884575in}}{\pgfqpoint{4.609562in}{4.878751in}}%
\pgfpathcurveto{\pgfqpoint{4.609562in}{4.872927in}}{\pgfqpoint{4.611876in}{4.867341in}}{\pgfqpoint{4.615994in}{4.863223in}}%
\pgfpathcurveto{\pgfqpoint{4.620112in}{4.859105in}}{\pgfqpoint{4.625698in}{4.856791in}}{\pgfqpoint{4.631522in}{4.856791in}}%
\pgfpathlineto{\pgfqpoint{4.631522in}{4.856791in}}%
\pgfpathclose%
\pgfusepath{stroke,fill}%
\end{pgfscope}%
\begin{pgfscope}%
\pgfpathrectangle{\pgfqpoint{1.542338in}{0.880000in}}{\pgfqpoint{5.115323in}{6.160000in}}%
\pgfusepath{clip}%
\pgfsetbuttcap%
\pgfsetroundjoin%
\definecolor{currentfill}{rgb}{0.200000,0.800000,0.200000}%
\pgfsetfillcolor{currentfill}%
\pgfsetlinewidth{1.003750pt}%
\definecolor{currentstroke}{rgb}{0.200000,0.800000,0.200000}%
\pgfsetstrokecolor{currentstroke}%
\pgfsetdash{}{0pt}%
\pgfpathmoveto{\pgfqpoint{4.749812in}{4.883805in}}%
\pgfpathcurveto{\pgfqpoint{4.755636in}{4.883805in}}{\pgfqpoint{4.761223in}{4.886119in}}{\pgfqpoint{4.765341in}{4.890237in}}%
\pgfpathcurveto{\pgfqpoint{4.769459in}{4.894355in}}{\pgfqpoint{4.771773in}{4.899941in}}{\pgfqpoint{4.771773in}{4.905765in}}%
\pgfpathcurveto{\pgfqpoint{4.771773in}{4.911589in}}{\pgfqpoint{4.769459in}{4.917175in}}{\pgfqpoint{4.765341in}{4.921293in}}%
\pgfpathcurveto{\pgfqpoint{4.761223in}{4.925411in}}{\pgfqpoint{4.755636in}{4.927725in}}{\pgfqpoint{4.749812in}{4.927725in}}%
\pgfpathcurveto{\pgfqpoint{4.743989in}{4.927725in}}{\pgfqpoint{4.738402in}{4.925411in}}{\pgfqpoint{4.734284in}{4.921293in}}%
\pgfpathcurveto{\pgfqpoint{4.730166in}{4.917175in}}{\pgfqpoint{4.727852in}{4.911589in}}{\pgfqpoint{4.727852in}{4.905765in}}%
\pgfpathcurveto{\pgfqpoint{4.727852in}{4.899941in}}{\pgfqpoint{4.730166in}{4.894355in}}{\pgfqpoint{4.734284in}{4.890237in}}%
\pgfpathcurveto{\pgfqpoint{4.738402in}{4.886119in}}{\pgfqpoint{4.743989in}{4.883805in}}{\pgfqpoint{4.749812in}{4.883805in}}%
\pgfpathlineto{\pgfqpoint{4.749812in}{4.883805in}}%
\pgfpathclose%
\pgfusepath{stroke,fill}%
\end{pgfscope}%
\begin{pgfscope}%
\pgfpathrectangle{\pgfqpoint{1.542338in}{0.880000in}}{\pgfqpoint{5.115323in}{6.160000in}}%
\pgfusepath{clip}%
\pgfsetbuttcap%
\pgfsetroundjoin%
\definecolor{currentfill}{rgb}{0.200000,0.800000,0.200000}%
\pgfsetfillcolor{currentfill}%
\pgfsetlinewidth{1.003750pt}%
\definecolor{currentstroke}{rgb}{0.200000,0.800000,0.200000}%
\pgfsetstrokecolor{currentstroke}%
\pgfsetdash{}{0pt}%
\pgfpathmoveto{\pgfqpoint{4.864483in}{4.923899in}}%
\pgfpathcurveto{\pgfqpoint{4.870307in}{4.923899in}}{\pgfqpoint{4.875893in}{4.926213in}}{\pgfqpoint{4.880011in}{4.930331in}}%
\pgfpathcurveto{\pgfqpoint{4.884129in}{4.934449in}}{\pgfqpoint{4.886443in}{4.940035in}}{\pgfqpoint{4.886443in}{4.945859in}}%
\pgfpathcurveto{\pgfqpoint{4.886443in}{4.951683in}}{\pgfqpoint{4.884129in}{4.957269in}}{\pgfqpoint{4.880011in}{4.961387in}}%
\pgfpathcurveto{\pgfqpoint{4.875893in}{4.965505in}}{\pgfqpoint{4.870307in}{4.967819in}}{\pgfqpoint{4.864483in}{4.967819in}}%
\pgfpathcurveto{\pgfqpoint{4.858659in}{4.967819in}}{\pgfqpoint{4.853073in}{4.965505in}}{\pgfqpoint{4.848954in}{4.961387in}}%
\pgfpathcurveto{\pgfqpoint{4.844836in}{4.957269in}}{\pgfqpoint{4.842522in}{4.951683in}}{\pgfqpoint{4.842522in}{4.945859in}}%
\pgfpathcurveto{\pgfqpoint{4.842522in}{4.940035in}}{\pgfqpoint{4.844836in}{4.934449in}}{\pgfqpoint{4.848954in}{4.930331in}}%
\pgfpathcurveto{\pgfqpoint{4.853073in}{4.926213in}}{\pgfqpoint{4.858659in}{4.923899in}}{\pgfqpoint{4.864483in}{4.923899in}}%
\pgfpathlineto{\pgfqpoint{4.864483in}{4.923899in}}%
\pgfpathclose%
\pgfusepath{stroke,fill}%
\end{pgfscope}%
\begin{pgfscope}%
\pgfpathrectangle{\pgfqpoint{1.542338in}{0.880000in}}{\pgfqpoint{5.115323in}{6.160000in}}%
\pgfusepath{clip}%
\pgfsetbuttcap%
\pgfsetroundjoin%
\definecolor{currentfill}{rgb}{0.200000,0.800000,0.200000}%
\pgfsetfillcolor{currentfill}%
\pgfsetlinewidth{1.003750pt}%
\definecolor{currentstroke}{rgb}{0.200000,0.800000,0.200000}%
\pgfsetstrokecolor{currentstroke}%
\pgfsetdash{}{0pt}%
\pgfpathmoveto{\pgfqpoint{4.976479in}{4.973216in}}%
\pgfpathcurveto{\pgfqpoint{4.982303in}{4.973216in}}{\pgfqpoint{4.987889in}{4.975530in}}{\pgfqpoint{4.992007in}{4.979648in}}%
\pgfpathcurveto{\pgfqpoint{4.996125in}{4.983766in}}{\pgfqpoint{4.998439in}{4.989352in}}{\pgfqpoint{4.998439in}{4.995176in}}%
\pgfpathcurveto{\pgfqpoint{4.998439in}{5.001000in}}{\pgfqpoint{4.996125in}{5.006586in}}{\pgfqpoint{4.992007in}{5.010704in}}%
\pgfpathcurveto{\pgfqpoint{4.987889in}{5.014822in}}{\pgfqpoint{4.982303in}{5.017136in}}{\pgfqpoint{4.976479in}{5.017136in}}%
\pgfpathcurveto{\pgfqpoint{4.970655in}{5.017136in}}{\pgfqpoint{4.965069in}{5.014822in}}{\pgfqpoint{4.960951in}{5.010704in}}%
\pgfpathcurveto{\pgfqpoint{4.956833in}{5.006586in}}{\pgfqpoint{4.954519in}{5.001000in}}{\pgfqpoint{4.954519in}{4.995176in}}%
\pgfpathcurveto{\pgfqpoint{4.954519in}{4.989352in}}{\pgfqpoint{4.956833in}{4.983766in}}{\pgfqpoint{4.960951in}{4.979648in}}%
\pgfpathcurveto{\pgfqpoint{4.965069in}{4.975530in}}{\pgfqpoint{4.970655in}{4.973216in}}{\pgfqpoint{4.976479in}{4.973216in}}%
\pgfpathlineto{\pgfqpoint{4.976479in}{4.973216in}}%
\pgfpathclose%
\pgfusepath{stroke,fill}%
\end{pgfscope}%
\begin{pgfscope}%
\pgfpathrectangle{\pgfqpoint{1.542338in}{0.880000in}}{\pgfqpoint{5.115323in}{6.160000in}}%
\pgfusepath{clip}%
\pgfsetbuttcap%
\pgfsetroundjoin%
\definecolor{currentfill}{rgb}{0.200000,0.800000,0.200000}%
\pgfsetfillcolor{currentfill}%
\pgfsetlinewidth{1.003750pt}%
\definecolor{currentstroke}{rgb}{0.200000,0.800000,0.200000}%
\pgfsetstrokecolor{currentstroke}%
\pgfsetdash{}{0pt}%
\pgfpathmoveto{\pgfqpoint{5.073520in}{5.047803in}}%
\pgfpathcurveto{\pgfqpoint{5.079344in}{5.047803in}}{\pgfqpoint{5.084930in}{5.050117in}}{\pgfqpoint{5.089048in}{5.054235in}}%
\pgfpathcurveto{\pgfqpoint{5.093166in}{5.058353in}}{\pgfqpoint{5.095480in}{5.063939in}}{\pgfqpoint{5.095480in}{5.069763in}}%
\pgfpathcurveto{\pgfqpoint{5.095480in}{5.075587in}}{\pgfqpoint{5.093166in}{5.081173in}}{\pgfqpoint{5.089048in}{5.085291in}}%
\pgfpathcurveto{\pgfqpoint{5.084930in}{5.089409in}}{\pgfqpoint{5.079344in}{5.091723in}}{\pgfqpoint{5.073520in}{5.091723in}}%
\pgfpathcurveto{\pgfqpoint{5.067696in}{5.091723in}}{\pgfqpoint{5.062110in}{5.089409in}}{\pgfqpoint{5.057991in}{5.085291in}}%
\pgfpathcurveto{\pgfqpoint{5.053873in}{5.081173in}}{\pgfqpoint{5.051559in}{5.075587in}}{\pgfqpoint{5.051559in}{5.069763in}}%
\pgfpathcurveto{\pgfqpoint{5.051559in}{5.063939in}}{\pgfqpoint{5.053873in}{5.058353in}}{\pgfqpoint{5.057991in}{5.054235in}}%
\pgfpathcurveto{\pgfqpoint{5.062110in}{5.050117in}}{\pgfqpoint{5.067696in}{5.047803in}}{\pgfqpoint{5.073520in}{5.047803in}}%
\pgfpathlineto{\pgfqpoint{5.073520in}{5.047803in}}%
\pgfpathclose%
\pgfusepath{stroke,fill}%
\end{pgfscope}%
\begin{pgfscope}%
\pgfpathrectangle{\pgfqpoint{1.542338in}{0.880000in}}{\pgfqpoint{5.115323in}{6.160000in}}%
\pgfusepath{clip}%
\pgfsetbuttcap%
\pgfsetroundjoin%
\definecolor{currentfill}{rgb}{0.200000,0.800000,0.200000}%
\pgfsetfillcolor{currentfill}%
\pgfsetlinewidth{1.003750pt}%
\definecolor{currentstroke}{rgb}{0.200000,0.800000,0.200000}%
\pgfsetstrokecolor{currentstroke}%
\pgfsetdash{}{0pt}%
\pgfpathmoveto{\pgfqpoint{5.159169in}{5.134201in}}%
\pgfpathcurveto{\pgfqpoint{5.164993in}{5.134201in}}{\pgfqpoint{5.170579in}{5.136515in}}{\pgfqpoint{5.174697in}{5.140633in}}%
\pgfpathcurveto{\pgfqpoint{5.178816in}{5.144751in}}{\pgfqpoint{5.181130in}{5.150337in}}{\pgfqpoint{5.181130in}{5.156161in}}%
\pgfpathcurveto{\pgfqpoint{5.181130in}{5.161985in}}{\pgfqpoint{5.178816in}{5.167571in}}{\pgfqpoint{5.174697in}{5.171689in}}%
\pgfpathcurveto{\pgfqpoint{5.170579in}{5.175807in}}{\pgfqpoint{5.164993in}{5.178121in}}{\pgfqpoint{5.159169in}{5.178121in}}%
\pgfpathcurveto{\pgfqpoint{5.153345in}{5.178121in}}{\pgfqpoint{5.147759in}{5.175807in}}{\pgfqpoint{5.143641in}{5.171689in}}%
\pgfpathcurveto{\pgfqpoint{5.139523in}{5.167571in}}{\pgfqpoint{5.137209in}{5.161985in}}{\pgfqpoint{5.137209in}{5.156161in}}%
\pgfpathcurveto{\pgfqpoint{5.137209in}{5.150337in}}{\pgfqpoint{5.139523in}{5.144751in}}{\pgfqpoint{5.143641in}{5.140633in}}%
\pgfpathcurveto{\pgfqpoint{5.147759in}{5.136515in}}{\pgfqpoint{5.153345in}{5.134201in}}{\pgfqpoint{5.159169in}{5.134201in}}%
\pgfpathlineto{\pgfqpoint{5.159169in}{5.134201in}}%
\pgfpathclose%
\pgfusepath{stroke,fill}%
\end{pgfscope}%
\begin{pgfscope}%
\pgfpathrectangle{\pgfqpoint{1.542338in}{0.880000in}}{\pgfqpoint{5.115323in}{6.160000in}}%
\pgfusepath{clip}%
\pgfsetbuttcap%
\pgfsetroundjoin%
\definecolor{currentfill}{rgb}{0.200000,0.800000,0.200000}%
\pgfsetfillcolor{currentfill}%
\pgfsetlinewidth{1.003750pt}%
\definecolor{currentstroke}{rgb}{0.200000,0.800000,0.200000}%
\pgfsetstrokecolor{currentstroke}%
\pgfsetdash{}{0pt}%
\pgfpathmoveto{\pgfqpoint{5.242733in}{5.222609in}}%
\pgfpathcurveto{\pgfqpoint{5.248557in}{5.222609in}}{\pgfqpoint{5.254143in}{5.224922in}}{\pgfqpoint{5.258261in}{5.229041in}}%
\pgfpathcurveto{\pgfqpoint{5.262380in}{5.233159in}}{\pgfqpoint{5.264693in}{5.238745in}}{\pgfqpoint{5.264693in}{5.244569in}}%
\pgfpathcurveto{\pgfqpoint{5.264693in}{5.250393in}}{\pgfqpoint{5.262380in}{5.255979in}}{\pgfqpoint{5.258261in}{5.260097in}}%
\pgfpathcurveto{\pgfqpoint{5.254143in}{5.264215in}}{\pgfqpoint{5.248557in}{5.266529in}}{\pgfqpoint{5.242733in}{5.266529in}}%
\pgfpathcurveto{\pgfqpoint{5.236909in}{5.266529in}}{\pgfqpoint{5.231323in}{5.264215in}}{\pgfqpoint{5.227205in}{5.260097in}}%
\pgfpathcurveto{\pgfqpoint{5.223087in}{5.255979in}}{\pgfqpoint{5.220773in}{5.250393in}}{\pgfqpoint{5.220773in}{5.244569in}}%
\pgfpathcurveto{\pgfqpoint{5.220773in}{5.238745in}}{\pgfqpoint{5.223087in}{5.233159in}}{\pgfqpoint{5.227205in}{5.229041in}}%
\pgfpathcurveto{\pgfqpoint{5.231323in}{5.224922in}}{\pgfqpoint{5.236909in}{5.222609in}}{\pgfqpoint{5.242733in}{5.222609in}}%
\pgfpathlineto{\pgfqpoint{5.242733in}{5.222609in}}%
\pgfpathclose%
\pgfusepath{stroke,fill}%
\end{pgfscope}%
\begin{pgfscope}%
\pgfpathrectangle{\pgfqpoint{1.542338in}{0.880000in}}{\pgfqpoint{5.115323in}{6.160000in}}%
\pgfusepath{clip}%
\pgfsetbuttcap%
\pgfsetroundjoin%
\definecolor{currentfill}{rgb}{0.200000,0.800000,0.200000}%
\pgfsetfillcolor{currentfill}%
\pgfsetlinewidth{1.003750pt}%
\definecolor{currentstroke}{rgb}{0.200000,0.800000,0.200000}%
\pgfsetstrokecolor{currentstroke}%
\pgfsetdash{}{0pt}%
\pgfpathmoveto{\pgfqpoint{5.301292in}{5.329235in}}%
\pgfpathcurveto{\pgfqpoint{5.307116in}{5.329235in}}{\pgfqpoint{5.312702in}{5.331549in}}{\pgfqpoint{5.316820in}{5.335667in}}%
\pgfpathcurveto{\pgfqpoint{5.320938in}{5.339786in}}{\pgfqpoint{5.323252in}{5.345372in}}{\pgfqpoint{5.323252in}{5.351196in}}%
\pgfpathcurveto{\pgfqpoint{5.323252in}{5.357020in}}{\pgfqpoint{5.320938in}{5.362606in}}{\pgfqpoint{5.316820in}{5.366724in}}%
\pgfpathcurveto{\pgfqpoint{5.312702in}{5.370842in}}{\pgfqpoint{5.307116in}{5.373156in}}{\pgfqpoint{5.301292in}{5.373156in}}%
\pgfpathcurveto{\pgfqpoint{5.295468in}{5.373156in}}{\pgfqpoint{5.289882in}{5.370842in}}{\pgfqpoint{5.285764in}{5.366724in}}%
\pgfpathcurveto{\pgfqpoint{5.281646in}{5.362606in}}{\pgfqpoint{5.279332in}{5.357020in}}{\pgfqpoint{5.279332in}{5.351196in}}%
\pgfpathcurveto{\pgfqpoint{5.279332in}{5.345372in}}{\pgfqpoint{5.281646in}{5.339786in}}{\pgfqpoint{5.285764in}{5.335667in}}%
\pgfpathcurveto{\pgfqpoint{5.289882in}{5.331549in}}{\pgfqpoint{5.295468in}{5.329235in}}{\pgfqpoint{5.301292in}{5.329235in}}%
\pgfpathlineto{\pgfqpoint{5.301292in}{5.329235in}}%
\pgfpathclose%
\pgfusepath{stroke,fill}%
\end{pgfscope}%
\begin{pgfscope}%
\pgfpathrectangle{\pgfqpoint{1.542338in}{0.880000in}}{\pgfqpoint{5.115323in}{6.160000in}}%
\pgfusepath{clip}%
\pgfsetbuttcap%
\pgfsetroundjoin%
\definecolor{currentfill}{rgb}{0.200000,0.800000,0.200000}%
\pgfsetfillcolor{currentfill}%
\pgfsetlinewidth{1.003750pt}%
\definecolor{currentstroke}{rgb}{0.200000,0.800000,0.200000}%
\pgfsetstrokecolor{currentstroke}%
\pgfsetdash{}{0pt}%
\pgfpathmoveto{\pgfqpoint{5.355487in}{5.437270in}}%
\pgfpathcurveto{\pgfqpoint{5.361311in}{5.437270in}}{\pgfqpoint{5.366897in}{5.439584in}}{\pgfqpoint{5.371015in}{5.443702in}}%
\pgfpathcurveto{\pgfqpoint{5.375133in}{5.447820in}}{\pgfqpoint{5.377447in}{5.453407in}}{\pgfqpoint{5.377447in}{5.459231in}}%
\pgfpathcurveto{\pgfqpoint{5.377447in}{5.465055in}}{\pgfqpoint{5.375133in}{5.470641in}}{\pgfqpoint{5.371015in}{5.474759in}}%
\pgfpathcurveto{\pgfqpoint{5.366897in}{5.478877in}}{\pgfqpoint{5.361311in}{5.481191in}}{\pgfqpoint{5.355487in}{5.481191in}}%
\pgfpathcurveto{\pgfqpoint{5.349663in}{5.481191in}}{\pgfqpoint{5.344077in}{5.478877in}}{\pgfqpoint{5.339959in}{5.474759in}}%
\pgfpathcurveto{\pgfqpoint{5.335840in}{5.470641in}}{\pgfqpoint{5.333526in}{5.465055in}}{\pgfqpoint{5.333526in}{5.459231in}}%
\pgfpathcurveto{\pgfqpoint{5.333526in}{5.453407in}}{\pgfqpoint{5.335840in}{5.447820in}}{\pgfqpoint{5.339959in}{5.443702in}}%
\pgfpathcurveto{\pgfqpoint{5.344077in}{5.439584in}}{\pgfqpoint{5.349663in}{5.437270in}}{\pgfqpoint{5.355487in}{5.437270in}}%
\pgfpathlineto{\pgfqpoint{5.355487in}{5.437270in}}%
\pgfpathclose%
\pgfusepath{stroke,fill}%
\end{pgfscope}%
\begin{pgfscope}%
\pgfpathrectangle{\pgfqpoint{1.542338in}{0.880000in}}{\pgfqpoint{5.115323in}{6.160000in}}%
\pgfusepath{clip}%
\pgfsetbuttcap%
\pgfsetroundjoin%
\definecolor{currentfill}{rgb}{0.200000,0.800000,0.200000}%
\pgfsetfillcolor{currentfill}%
\pgfsetlinewidth{1.003750pt}%
\definecolor{currentstroke}{rgb}{0.200000,0.800000,0.200000}%
\pgfsetstrokecolor{currentstroke}%
\pgfsetdash{}{0pt}%
\pgfpathmoveto{\pgfqpoint{5.398551in}{5.550787in}}%
\pgfpathcurveto{\pgfqpoint{5.404375in}{5.550787in}}{\pgfqpoint{5.409961in}{5.553100in}}{\pgfqpoint{5.414079in}{5.557219in}}%
\pgfpathcurveto{\pgfqpoint{5.418197in}{5.561337in}}{\pgfqpoint{5.420511in}{5.566923in}}{\pgfqpoint{5.420511in}{5.572747in}}%
\pgfpathcurveto{\pgfqpoint{5.420511in}{5.578571in}}{\pgfqpoint{5.418197in}{5.584157in}}{\pgfqpoint{5.414079in}{5.588275in}}%
\pgfpathcurveto{\pgfqpoint{5.409961in}{5.592393in}}{\pgfqpoint{5.404375in}{5.594707in}}{\pgfqpoint{5.398551in}{5.594707in}}%
\pgfpathcurveto{\pgfqpoint{5.392727in}{5.594707in}}{\pgfqpoint{5.387141in}{5.592393in}}{\pgfqpoint{5.383023in}{5.588275in}}%
\pgfpathcurveto{\pgfqpoint{5.378905in}{5.584157in}}{\pgfqpoint{5.376591in}{5.578571in}}{\pgfqpoint{5.376591in}{5.572747in}}%
\pgfpathcurveto{\pgfqpoint{5.376591in}{5.566923in}}{\pgfqpoint{5.378905in}{5.561337in}}{\pgfqpoint{5.383023in}{5.557219in}}%
\pgfpathcurveto{\pgfqpoint{5.387141in}{5.553100in}}{\pgfqpoint{5.392727in}{5.550787in}}{\pgfqpoint{5.398551in}{5.550787in}}%
\pgfpathlineto{\pgfqpoint{5.398551in}{5.550787in}}%
\pgfpathclose%
\pgfusepath{stroke,fill}%
\end{pgfscope}%
\begin{pgfscope}%
\pgfpathrectangle{\pgfqpoint{1.542338in}{0.880000in}}{\pgfqpoint{5.115323in}{6.160000in}}%
\pgfusepath{clip}%
\pgfsetbuttcap%
\pgfsetroundjoin%
\definecolor{currentfill}{rgb}{0.200000,0.800000,0.200000}%
\pgfsetfillcolor{currentfill}%
\pgfsetlinewidth{1.003750pt}%
\definecolor{currentstroke}{rgb}{0.200000,0.800000,0.200000}%
\pgfsetstrokecolor{currentstroke}%
\pgfsetdash{}{0pt}%
\pgfpathmoveto{\pgfqpoint{5.421678in}{5.670190in}}%
\pgfpathcurveto{\pgfqpoint{5.427502in}{5.670190in}}{\pgfqpoint{5.433088in}{5.672504in}}{\pgfqpoint{5.437206in}{5.676622in}}%
\pgfpathcurveto{\pgfqpoint{5.441325in}{5.680740in}}{\pgfqpoint{5.443638in}{5.686326in}}{\pgfqpoint{5.443638in}{5.692150in}}%
\pgfpathcurveto{\pgfqpoint{5.443638in}{5.697974in}}{\pgfqpoint{5.441325in}{5.703561in}}{\pgfqpoint{5.437206in}{5.707679in}}%
\pgfpathcurveto{\pgfqpoint{5.433088in}{5.711797in}}{\pgfqpoint{5.427502in}{5.714111in}}{\pgfqpoint{5.421678in}{5.714111in}}%
\pgfpathcurveto{\pgfqpoint{5.415854in}{5.714111in}}{\pgfqpoint{5.410268in}{5.711797in}}{\pgfqpoint{5.406150in}{5.707679in}}%
\pgfpathcurveto{\pgfqpoint{5.402032in}{5.703561in}}{\pgfqpoint{5.399718in}{5.697974in}}{\pgfqpoint{5.399718in}{5.692150in}}%
\pgfpathcurveto{\pgfqpoint{5.399718in}{5.686326in}}{\pgfqpoint{5.402032in}{5.680740in}}{\pgfqpoint{5.406150in}{5.676622in}}%
\pgfpathcurveto{\pgfqpoint{5.410268in}{5.672504in}}{\pgfqpoint{5.415854in}{5.670190in}}{\pgfqpoint{5.421678in}{5.670190in}}%
\pgfpathlineto{\pgfqpoint{5.421678in}{5.670190in}}%
\pgfpathclose%
\pgfusepath{stroke,fill}%
\end{pgfscope}%
\begin{pgfscope}%
\pgfpathrectangle{\pgfqpoint{1.542338in}{0.880000in}}{\pgfqpoint{5.115323in}{6.160000in}}%
\pgfusepath{clip}%
\pgfsetbuttcap%
\pgfsetroundjoin%
\definecolor{currentfill}{rgb}{0.200000,0.800000,0.200000}%
\pgfsetfillcolor{currentfill}%
\pgfsetlinewidth{1.003750pt}%
\definecolor{currentstroke}{rgb}{0.200000,0.800000,0.200000}%
\pgfsetstrokecolor{currentstroke}%
\pgfsetdash{}{0pt}%
\pgfpathmoveto{\pgfqpoint{5.429172in}{5.791555in}}%
\pgfpathcurveto{\pgfqpoint{5.434996in}{5.791555in}}{\pgfqpoint{5.440582in}{5.793869in}}{\pgfqpoint{5.444700in}{5.797987in}}%
\pgfpathcurveto{\pgfqpoint{5.448818in}{5.802105in}}{\pgfqpoint{5.451132in}{5.807691in}}{\pgfqpoint{5.451132in}{5.813515in}}%
\pgfpathcurveto{\pgfqpoint{5.451132in}{5.819339in}}{\pgfqpoint{5.448818in}{5.824925in}}{\pgfqpoint{5.444700in}{5.829043in}}%
\pgfpathcurveto{\pgfqpoint{5.440582in}{5.833162in}}{\pgfqpoint{5.434996in}{5.835475in}}{\pgfqpoint{5.429172in}{5.835475in}}%
\pgfpathcurveto{\pgfqpoint{5.423348in}{5.835475in}}{\pgfqpoint{5.417762in}{5.833162in}}{\pgfqpoint{5.413643in}{5.829043in}}%
\pgfpathcurveto{\pgfqpoint{5.409525in}{5.824925in}}{\pgfqpoint{5.407211in}{5.819339in}}{\pgfqpoint{5.407211in}{5.813515in}}%
\pgfpathcurveto{\pgfqpoint{5.407211in}{5.807691in}}{\pgfqpoint{5.409525in}{5.802105in}}{\pgfqpoint{5.413643in}{5.797987in}}%
\pgfpathcurveto{\pgfqpoint{5.417762in}{5.793869in}}{\pgfqpoint{5.423348in}{5.791555in}}{\pgfqpoint{5.429172in}{5.791555in}}%
\pgfpathlineto{\pgfqpoint{5.429172in}{5.791555in}}%
\pgfpathclose%
\pgfusepath{stroke,fill}%
\end{pgfscope}%
\begin{pgfscope}%
\pgfpathrectangle{\pgfqpoint{1.542338in}{0.880000in}}{\pgfqpoint{5.115323in}{6.160000in}}%
\pgfusepath{clip}%
\pgfsetbuttcap%
\pgfsetroundjoin%
\definecolor{currentfill}{rgb}{0.500000,0.000000,0.500000}%
\pgfsetfillcolor{currentfill}%
\pgfsetlinewidth{1.003750pt}%
\definecolor{currentstroke}{rgb}{0.500000,0.000000,0.500000}%
\pgfsetstrokecolor{currentstroke}%
\pgfsetdash{}{0pt}%
\pgfpathmoveto{\pgfqpoint{3.019964in}{5.535553in}}%
\pgfpathcurveto{\pgfqpoint{3.025788in}{5.535553in}}{\pgfqpoint{3.031374in}{5.537867in}}{\pgfqpoint{3.035492in}{5.541985in}}%
\pgfpathcurveto{\pgfqpoint{3.039610in}{5.546103in}}{\pgfqpoint{3.041924in}{5.551690in}}{\pgfqpoint{3.041924in}{5.557514in}}%
\pgfpathcurveto{\pgfqpoint{3.041924in}{5.563337in}}{\pgfqpoint{3.039610in}{5.568924in}}{\pgfqpoint{3.035492in}{5.573042in}}%
\pgfpathcurveto{\pgfqpoint{3.031374in}{5.577160in}}{\pgfqpoint{3.025788in}{5.579474in}}{\pgfqpoint{3.019964in}{5.579474in}}%
\pgfpathcurveto{\pgfqpoint{3.014140in}{5.579474in}}{\pgfqpoint{3.008554in}{5.577160in}}{\pgfqpoint{3.004436in}{5.573042in}}%
\pgfpathcurveto{\pgfqpoint{3.000317in}{5.568924in}}{\pgfqpoint{2.998004in}{5.563337in}}{\pgfqpoint{2.998004in}{5.557514in}}%
\pgfpathcurveto{\pgfqpoint{2.998004in}{5.551690in}}{\pgfqpoint{3.000317in}{5.546103in}}{\pgfqpoint{3.004436in}{5.541985in}}%
\pgfpathcurveto{\pgfqpoint{3.008554in}{5.537867in}}{\pgfqpoint{3.014140in}{5.535553in}}{\pgfqpoint{3.019964in}{5.535553in}}%
\pgfpathlineto{\pgfqpoint{3.019964in}{5.535553in}}%
\pgfpathclose%
\pgfusepath{stroke,fill}%
\end{pgfscope}%
\begin{pgfscope}%
\pgfpathrectangle{\pgfqpoint{1.542338in}{0.880000in}}{\pgfqpoint{5.115323in}{6.160000in}}%
\pgfusepath{clip}%
\pgfsetbuttcap%
\pgfsetroundjoin%
\definecolor{currentfill}{rgb}{0.500000,0.000000,0.500000}%
\pgfsetfillcolor{currentfill}%
\pgfsetlinewidth{1.003750pt}%
\definecolor{currentstroke}{rgb}{0.500000,0.000000,0.500000}%
\pgfsetstrokecolor{currentstroke}%
\pgfsetdash{}{0pt}%
\pgfpathmoveto{\pgfqpoint{2.020723in}{6.031441in}}%
\pgfpathcurveto{\pgfqpoint{2.026547in}{6.031441in}}{\pgfqpoint{2.032133in}{6.033755in}}{\pgfqpoint{2.036251in}{6.037873in}}%
\pgfpathcurveto{\pgfqpoint{2.040369in}{6.041992in}}{\pgfqpoint{2.042683in}{6.047578in}}{\pgfqpoint{2.042683in}{6.053402in}}%
\pgfpathcurveto{\pgfqpoint{2.042683in}{6.059226in}}{\pgfqpoint{2.040369in}{6.064812in}}{\pgfqpoint{2.036251in}{6.068930in}}%
\pgfpathcurveto{\pgfqpoint{2.032133in}{6.073048in}}{\pgfqpoint{2.026547in}{6.075362in}}{\pgfqpoint{2.020723in}{6.075362in}}%
\pgfpathcurveto{\pgfqpoint{2.014899in}{6.075362in}}{\pgfqpoint{2.009313in}{6.073048in}}{\pgfqpoint{2.005194in}{6.068930in}}%
\pgfpathcurveto{\pgfqpoint{2.001076in}{6.064812in}}{\pgfqpoint{1.998762in}{6.059226in}}{\pgfqpoint{1.998762in}{6.053402in}}%
\pgfpathcurveto{\pgfqpoint{1.998762in}{6.047578in}}{\pgfqpoint{2.001076in}{6.041992in}}{\pgfqpoint{2.005194in}{6.037873in}}%
\pgfpathcurveto{\pgfqpoint{2.009313in}{6.033755in}}{\pgfqpoint{2.014899in}{6.031441in}}{\pgfqpoint{2.020723in}{6.031441in}}%
\pgfpathlineto{\pgfqpoint{2.020723in}{6.031441in}}%
\pgfpathclose%
\pgfusepath{stroke,fill}%
\end{pgfscope}%
\begin{pgfscope}%
\pgfpathrectangle{\pgfqpoint{1.542338in}{0.880000in}}{\pgfqpoint{5.115323in}{6.160000in}}%
\pgfusepath{clip}%
\pgfsetbuttcap%
\pgfsetroundjoin%
\definecolor{currentfill}{rgb}{0.200000,0.200000,0.800000}%
\pgfsetfillcolor{currentfill}%
\pgfsetlinewidth{1.003750pt}%
\definecolor{currentstroke}{rgb}{0.200000,0.200000,0.800000}%
\pgfsetstrokecolor{currentstroke}%
\pgfsetdash{}{0pt}%
\pgfpathmoveto{\pgfqpoint{2.802067in}{3.661123in}}%
\pgfpathcurveto{\pgfqpoint{2.807891in}{3.661123in}}{\pgfqpoint{2.813477in}{3.663437in}}{\pgfqpoint{2.817595in}{3.667555in}}%
\pgfpathcurveto{\pgfqpoint{2.821713in}{3.671673in}}{\pgfqpoint{2.824027in}{3.677259in}}{\pgfqpoint{2.824027in}{3.683083in}}%
\pgfpathcurveto{\pgfqpoint{2.824027in}{3.688907in}}{\pgfqpoint{2.821713in}{3.694493in}}{\pgfqpoint{2.817595in}{3.698612in}}%
\pgfpathcurveto{\pgfqpoint{2.813477in}{3.702730in}}{\pgfqpoint{2.807891in}{3.705044in}}{\pgfqpoint{2.802067in}{3.705044in}}%
\pgfpathcurveto{\pgfqpoint{2.796243in}{3.705044in}}{\pgfqpoint{2.790657in}{3.702730in}}{\pgfqpoint{2.786539in}{3.698612in}}%
\pgfpathcurveto{\pgfqpoint{2.782420in}{3.694493in}}{\pgfqpoint{2.780106in}{3.688907in}}{\pgfqpoint{2.780106in}{3.683083in}}%
\pgfpathcurveto{\pgfqpoint{2.780106in}{3.677259in}}{\pgfqpoint{2.782420in}{3.671673in}}{\pgfqpoint{2.786539in}{3.667555in}}%
\pgfpathcurveto{\pgfqpoint{2.790657in}{3.663437in}}{\pgfqpoint{2.796243in}{3.661123in}}{\pgfqpoint{2.802067in}{3.661123in}}%
\pgfpathlineto{\pgfqpoint{2.802067in}{3.661123in}}%
\pgfpathclose%
\pgfusepath{stroke,fill}%
\end{pgfscope}%
\begin{pgfscope}%
\pgfpathrectangle{\pgfqpoint{1.542338in}{0.880000in}}{\pgfqpoint{5.115323in}{6.160000in}}%
\pgfusepath{clip}%
\pgfsetbuttcap%
\pgfsetroundjoin%
\definecolor{currentfill}{rgb}{0.500000,0.000000,0.500000}%
\pgfsetfillcolor{currentfill}%
\pgfsetlinewidth{1.003750pt}%
\definecolor{currentstroke}{rgb}{0.500000,0.000000,0.500000}%
\pgfsetstrokecolor{currentstroke}%
\pgfsetdash{}{0pt}%
\pgfpathmoveto{\pgfqpoint{2.785322in}{6.328746in}}%
\pgfpathcurveto{\pgfqpoint{2.791146in}{6.328746in}}{\pgfqpoint{2.796733in}{6.331060in}}{\pgfqpoint{2.800851in}{6.335178in}}%
\pgfpathcurveto{\pgfqpoint{2.804969in}{6.339296in}}{\pgfqpoint{2.807283in}{6.344883in}}{\pgfqpoint{2.807283in}{6.350707in}}%
\pgfpathcurveto{\pgfqpoint{2.807283in}{6.356531in}}{\pgfqpoint{2.804969in}{6.362117in}}{\pgfqpoint{2.800851in}{6.366235in}}%
\pgfpathcurveto{\pgfqpoint{2.796733in}{6.370353in}}{\pgfqpoint{2.791146in}{6.372667in}}{\pgfqpoint{2.785322in}{6.372667in}}%
\pgfpathcurveto{\pgfqpoint{2.779499in}{6.372667in}}{\pgfqpoint{2.773912in}{6.370353in}}{\pgfqpoint{2.769794in}{6.366235in}}%
\pgfpathcurveto{\pgfqpoint{2.765676in}{6.362117in}}{\pgfqpoint{2.763362in}{6.356531in}}{\pgfqpoint{2.763362in}{6.350707in}}%
\pgfpathcurveto{\pgfqpoint{2.763362in}{6.344883in}}{\pgfqpoint{2.765676in}{6.339296in}}{\pgfqpoint{2.769794in}{6.335178in}}%
\pgfpathcurveto{\pgfqpoint{2.773912in}{6.331060in}}{\pgfqpoint{2.779499in}{6.328746in}}{\pgfqpoint{2.785322in}{6.328746in}}%
\pgfpathlineto{\pgfqpoint{2.785322in}{6.328746in}}%
\pgfpathclose%
\pgfusepath{stroke,fill}%
\end{pgfscope}%
\begin{pgfscope}%
\pgfpathrectangle{\pgfqpoint{1.542338in}{0.880000in}}{\pgfqpoint{5.115323in}{6.160000in}}%
\pgfusepath{clip}%
\pgfsetbuttcap%
\pgfsetroundjoin%
\definecolor{currentfill}{rgb}{0.500000,0.000000,0.500000}%
\pgfsetfillcolor{currentfill}%
\pgfsetlinewidth{1.003750pt}%
\definecolor{currentstroke}{rgb}{0.500000,0.000000,0.500000}%
\pgfsetstrokecolor{currentstroke}%
\pgfsetdash{}{0pt}%
\pgfpathmoveto{\pgfqpoint{2.133033in}{2.624077in}}%
\pgfpathcurveto{\pgfqpoint{2.138857in}{2.624077in}}{\pgfqpoint{2.144443in}{2.626391in}}{\pgfqpoint{2.148562in}{2.630509in}}%
\pgfpathcurveto{\pgfqpoint{2.152680in}{2.634628in}}{\pgfqpoint{2.154994in}{2.640214in}}{\pgfqpoint{2.154994in}{2.646038in}}%
\pgfpathcurveto{\pgfqpoint{2.154994in}{2.651862in}}{\pgfqpoint{2.152680in}{2.657448in}}{\pgfqpoint{2.148562in}{2.661566in}}%
\pgfpathcurveto{\pgfqpoint{2.144443in}{2.665684in}}{\pgfqpoint{2.138857in}{2.667998in}}{\pgfqpoint{2.133033in}{2.667998in}}%
\pgfpathcurveto{\pgfqpoint{2.127209in}{2.667998in}}{\pgfqpoint{2.121623in}{2.665684in}}{\pgfqpoint{2.117505in}{2.661566in}}%
\pgfpathcurveto{\pgfqpoint{2.113387in}{2.657448in}}{\pgfqpoint{2.111073in}{2.651862in}}{\pgfqpoint{2.111073in}{2.646038in}}%
\pgfpathcurveto{\pgfqpoint{2.111073in}{2.640214in}}{\pgfqpoint{2.113387in}{2.634628in}}{\pgfqpoint{2.117505in}{2.630509in}}%
\pgfpathcurveto{\pgfqpoint{2.121623in}{2.626391in}}{\pgfqpoint{2.127209in}{2.624077in}}{\pgfqpoint{2.133033in}{2.624077in}}%
\pgfpathlineto{\pgfqpoint{2.133033in}{2.624077in}}%
\pgfpathclose%
\pgfusepath{stroke,fill}%
\end{pgfscope}%
\begin{pgfscope}%
\pgfpathrectangle{\pgfqpoint{1.542338in}{0.880000in}}{\pgfqpoint{5.115323in}{6.160000in}}%
\pgfusepath{clip}%
\pgfsetbuttcap%
\pgfsetroundjoin%
\definecolor{currentfill}{rgb}{0.200000,0.800000,0.200000}%
\pgfsetfillcolor{currentfill}%
\pgfsetlinewidth{1.003750pt}%
\definecolor{currentstroke}{rgb}{0.200000,0.800000,0.200000}%
\pgfsetstrokecolor{currentstroke}%
\pgfsetdash{}{0pt}%
\pgfpathmoveto{\pgfqpoint{5.316136in}{5.026479in}}%
\pgfpathcurveto{\pgfqpoint{5.321960in}{5.026479in}}{\pgfqpoint{5.327546in}{5.028793in}}{\pgfqpoint{5.331665in}{5.032911in}}%
\pgfpathcurveto{\pgfqpoint{5.335783in}{5.037029in}}{\pgfqpoint{5.338097in}{5.042615in}}{\pgfqpoint{5.338097in}{5.048439in}}%
\pgfpathcurveto{\pgfqpoint{5.338097in}{5.054263in}}{\pgfqpoint{5.335783in}{5.059849in}}{\pgfqpoint{5.331665in}{5.063967in}}%
\pgfpathcurveto{\pgfqpoint{5.327546in}{5.068085in}}{\pgfqpoint{5.321960in}{5.070399in}}{\pgfqpoint{5.316136in}{5.070399in}}%
\pgfpathcurveto{\pgfqpoint{5.310312in}{5.070399in}}{\pgfqpoint{5.304726in}{5.068085in}}{\pgfqpoint{5.300608in}{5.063967in}}%
\pgfpathcurveto{\pgfqpoint{5.296490in}{5.059849in}}{\pgfqpoint{5.294176in}{5.054263in}}{\pgfqpoint{5.294176in}{5.048439in}}%
\pgfpathcurveto{\pgfqpoint{5.294176in}{5.042615in}}{\pgfqpoint{5.296490in}{5.037029in}}{\pgfqpoint{5.300608in}{5.032911in}}%
\pgfpathcurveto{\pgfqpoint{5.304726in}{5.028793in}}{\pgfqpoint{5.310312in}{5.026479in}}{\pgfqpoint{5.316136in}{5.026479in}}%
\pgfpathlineto{\pgfqpoint{5.316136in}{5.026479in}}%
\pgfpathclose%
\pgfusepath{stroke,fill}%
\end{pgfscope}%
\begin{pgfscope}%
\pgfpathrectangle{\pgfqpoint{1.542338in}{0.880000in}}{\pgfqpoint{5.115323in}{6.160000in}}%
\pgfusepath{clip}%
\pgfsetbuttcap%
\pgfsetroundjoin%
\definecolor{currentfill}{rgb}{0.500000,0.000000,0.500000}%
\pgfsetfillcolor{currentfill}%
\pgfsetlinewidth{1.003750pt}%
\definecolor{currentstroke}{rgb}{0.500000,0.000000,0.500000}%
\pgfsetstrokecolor{currentstroke}%
\pgfsetdash{}{0pt}%
\pgfpathmoveto{\pgfqpoint{3.649199in}{4.157653in}}%
\pgfpathcurveto{\pgfqpoint{3.655023in}{4.157653in}}{\pgfqpoint{3.660609in}{4.159967in}}{\pgfqpoint{3.664727in}{4.164085in}}%
\pgfpathcurveto{\pgfqpoint{3.668846in}{4.168203in}}{\pgfqpoint{3.671159in}{4.173789in}}{\pgfqpoint{3.671159in}{4.179613in}}%
\pgfpathcurveto{\pgfqpoint{3.671159in}{4.185437in}}{\pgfqpoint{3.668846in}{4.191024in}}{\pgfqpoint{3.664727in}{4.195142in}}%
\pgfpathcurveto{\pgfqpoint{3.660609in}{4.199260in}}{\pgfqpoint{3.655023in}{4.201574in}}{\pgfqpoint{3.649199in}{4.201574in}}%
\pgfpathcurveto{\pgfqpoint{3.643375in}{4.201574in}}{\pgfqpoint{3.637789in}{4.199260in}}{\pgfqpoint{3.633671in}{4.195142in}}%
\pgfpathcurveto{\pgfqpoint{3.629553in}{4.191024in}}{\pgfqpoint{3.627239in}{4.185437in}}{\pgfqpoint{3.627239in}{4.179613in}}%
\pgfpathcurveto{\pgfqpoint{3.627239in}{4.173789in}}{\pgfqpoint{3.629553in}{4.168203in}}{\pgfqpoint{3.633671in}{4.164085in}}%
\pgfpathcurveto{\pgfqpoint{3.637789in}{4.159967in}}{\pgfqpoint{3.643375in}{4.157653in}}{\pgfqpoint{3.649199in}{4.157653in}}%
\pgfpathlineto{\pgfqpoint{3.649199in}{4.157653in}}%
\pgfpathclose%
\pgfusepath{stroke,fill}%
\end{pgfscope}%
\begin{pgfscope}%
\pgfpathrectangle{\pgfqpoint{1.542338in}{0.880000in}}{\pgfqpoint{5.115323in}{6.160000in}}%
\pgfusepath{clip}%
\pgfsetbuttcap%
\pgfsetroundjoin%
\definecolor{currentfill}{rgb}{0.500000,0.000000,0.500000}%
\pgfsetfillcolor{currentfill}%
\pgfsetlinewidth{1.003750pt}%
\definecolor{currentstroke}{rgb}{0.500000,0.000000,0.500000}%
\pgfsetstrokecolor{currentstroke}%
\pgfsetdash{}{0pt}%
\pgfpathmoveto{\pgfqpoint{6.079092in}{5.520421in}}%
\pgfpathcurveto{\pgfqpoint{6.084916in}{5.520421in}}{\pgfqpoint{6.090502in}{5.522735in}}{\pgfqpoint{6.094620in}{5.526853in}}%
\pgfpathcurveto{\pgfqpoint{6.098739in}{5.530971in}}{\pgfqpoint{6.101052in}{5.536558in}}{\pgfqpoint{6.101052in}{5.542381in}}%
\pgfpathcurveto{\pgfqpoint{6.101052in}{5.548205in}}{\pgfqpoint{6.098739in}{5.553792in}}{\pgfqpoint{6.094620in}{5.557910in}}%
\pgfpathcurveto{\pgfqpoint{6.090502in}{5.562028in}}{\pgfqpoint{6.084916in}{5.564342in}}{\pgfqpoint{6.079092in}{5.564342in}}%
\pgfpathcurveto{\pgfqpoint{6.073268in}{5.564342in}}{\pgfqpoint{6.067682in}{5.562028in}}{\pgfqpoint{6.063564in}{5.557910in}}%
\pgfpathcurveto{\pgfqpoint{6.059446in}{5.553792in}}{\pgfqpoint{6.057132in}{5.548205in}}{\pgfqpoint{6.057132in}{5.542381in}}%
\pgfpathcurveto{\pgfqpoint{6.057132in}{5.536558in}}{\pgfqpoint{6.059446in}{5.530971in}}{\pgfqpoint{6.063564in}{5.526853in}}%
\pgfpathcurveto{\pgfqpoint{6.067682in}{5.522735in}}{\pgfqpoint{6.073268in}{5.520421in}}{\pgfqpoint{6.079092in}{5.520421in}}%
\pgfpathlineto{\pgfqpoint{6.079092in}{5.520421in}}%
\pgfpathclose%
\pgfusepath{stroke,fill}%
\end{pgfscope}%
\begin{pgfscope}%
\pgfpathrectangle{\pgfqpoint{1.542338in}{0.880000in}}{\pgfqpoint{5.115323in}{6.160000in}}%
\pgfusepath{clip}%
\pgfsetbuttcap%
\pgfsetroundjoin%
\definecolor{currentfill}{rgb}{0.200000,0.200000,0.800000}%
\pgfsetfillcolor{currentfill}%
\pgfsetlinewidth{1.003750pt}%
\definecolor{currentstroke}{rgb}{0.200000,0.200000,0.800000}%
\pgfsetstrokecolor{currentstroke}%
\pgfsetdash{}{0pt}%
\pgfpathmoveto{\pgfqpoint{3.033532in}{3.881917in}}%
\pgfpathcurveto{\pgfqpoint{3.039356in}{3.881917in}}{\pgfqpoint{3.044942in}{3.884231in}}{\pgfqpoint{3.049060in}{3.888350in}}%
\pgfpathcurveto{\pgfqpoint{3.053178in}{3.892468in}}{\pgfqpoint{3.055492in}{3.898054in}}{\pgfqpoint{3.055492in}{3.903878in}}%
\pgfpathcurveto{\pgfqpoint{3.055492in}{3.909702in}}{\pgfqpoint{3.053178in}{3.915288in}}{\pgfqpoint{3.049060in}{3.919406in}}%
\pgfpathcurveto{\pgfqpoint{3.044942in}{3.923524in}}{\pgfqpoint{3.039356in}{3.925838in}}{\pgfqpoint{3.033532in}{3.925838in}}%
\pgfpathcurveto{\pgfqpoint{3.027708in}{3.925838in}}{\pgfqpoint{3.022122in}{3.923524in}}{\pgfqpoint{3.018003in}{3.919406in}}%
\pgfpathcurveto{\pgfqpoint{3.013885in}{3.915288in}}{\pgfqpoint{3.011571in}{3.909702in}}{\pgfqpoint{3.011571in}{3.903878in}}%
\pgfpathcurveto{\pgfqpoint{3.011571in}{3.898054in}}{\pgfqpoint{3.013885in}{3.892468in}}{\pgfqpoint{3.018003in}{3.888350in}}%
\pgfpathcurveto{\pgfqpoint{3.022122in}{3.884231in}}{\pgfqpoint{3.027708in}{3.881917in}}{\pgfqpoint{3.033532in}{3.881917in}}%
\pgfpathlineto{\pgfqpoint{3.033532in}{3.881917in}}%
\pgfpathclose%
\pgfusepath{stroke,fill}%
\end{pgfscope}%
\begin{pgfscope}%
\pgfpathrectangle{\pgfqpoint{1.542338in}{0.880000in}}{\pgfqpoint{5.115323in}{6.160000in}}%
\pgfusepath{clip}%
\pgfsetbuttcap%
\pgfsetroundjoin%
\definecolor{currentfill}{rgb}{0.500000,0.000000,0.500000}%
\pgfsetfillcolor{currentfill}%
\pgfsetlinewidth{1.003750pt}%
\definecolor{currentstroke}{rgb}{0.500000,0.000000,0.500000}%
\pgfsetstrokecolor{currentstroke}%
\pgfsetdash{}{0pt}%
\pgfpathmoveto{\pgfqpoint{4.002100in}{5.361634in}}%
\pgfpathcurveto{\pgfqpoint{4.007924in}{5.361634in}}{\pgfqpoint{4.013510in}{5.363948in}}{\pgfqpoint{4.017628in}{5.368066in}}%
\pgfpathcurveto{\pgfqpoint{4.021747in}{5.372184in}}{\pgfqpoint{4.024060in}{5.377770in}}{\pgfqpoint{4.024060in}{5.383594in}}%
\pgfpathcurveto{\pgfqpoint{4.024060in}{5.389418in}}{\pgfqpoint{4.021747in}{5.395004in}}{\pgfqpoint{4.017628in}{5.399123in}}%
\pgfpathcurveto{\pgfqpoint{4.013510in}{5.403241in}}{\pgfqpoint{4.007924in}{5.405555in}}{\pgfqpoint{4.002100in}{5.405555in}}%
\pgfpathcurveto{\pgfqpoint{3.996276in}{5.405555in}}{\pgfqpoint{3.990690in}{5.403241in}}{\pgfqpoint{3.986572in}{5.399123in}}%
\pgfpathcurveto{\pgfqpoint{3.982454in}{5.395004in}}{\pgfqpoint{3.980140in}{5.389418in}}{\pgfqpoint{3.980140in}{5.383594in}}%
\pgfpathcurveto{\pgfqpoint{3.980140in}{5.377770in}}{\pgfqpoint{3.982454in}{5.372184in}}{\pgfqpoint{3.986572in}{5.368066in}}%
\pgfpathcurveto{\pgfqpoint{3.990690in}{5.363948in}}{\pgfqpoint{3.996276in}{5.361634in}}{\pgfqpoint{4.002100in}{5.361634in}}%
\pgfpathlineto{\pgfqpoint{4.002100in}{5.361634in}}%
\pgfpathclose%
\pgfusepath{stroke,fill}%
\end{pgfscope}%
\begin{pgfscope}%
\pgfpathrectangle{\pgfqpoint{1.542338in}{0.880000in}}{\pgfqpoint{5.115323in}{6.160000in}}%
\pgfusepath{clip}%
\pgfsetbuttcap%
\pgfsetroundjoin%
\definecolor{currentfill}{rgb}{0.500000,0.000000,0.500000}%
\pgfsetfillcolor{currentfill}%
\pgfsetlinewidth{1.003750pt}%
\definecolor{currentstroke}{rgb}{0.500000,0.000000,0.500000}%
\pgfsetstrokecolor{currentstroke}%
\pgfsetdash{}{0pt}%
\pgfpathmoveto{\pgfqpoint{4.897081in}{6.066066in}}%
\pgfpathcurveto{\pgfqpoint{4.902905in}{6.066066in}}{\pgfqpoint{4.908491in}{6.068380in}}{\pgfqpoint{4.912609in}{6.072498in}}%
\pgfpathcurveto{\pgfqpoint{4.916727in}{6.076616in}}{\pgfqpoint{4.919041in}{6.082202in}}{\pgfqpoint{4.919041in}{6.088026in}}%
\pgfpathcurveto{\pgfqpoint{4.919041in}{6.093850in}}{\pgfqpoint{4.916727in}{6.099436in}}{\pgfqpoint{4.912609in}{6.103555in}}%
\pgfpathcurveto{\pgfqpoint{4.908491in}{6.107673in}}{\pgfqpoint{4.902905in}{6.109987in}}{\pgfqpoint{4.897081in}{6.109987in}}%
\pgfpathcurveto{\pgfqpoint{4.891257in}{6.109987in}}{\pgfqpoint{4.885671in}{6.107673in}}{\pgfqpoint{4.881553in}{6.103555in}}%
\pgfpathcurveto{\pgfqpoint{4.877435in}{6.099436in}}{\pgfqpoint{4.875121in}{6.093850in}}{\pgfqpoint{4.875121in}{6.088026in}}%
\pgfpathcurveto{\pgfqpoint{4.875121in}{6.082202in}}{\pgfqpoint{4.877435in}{6.076616in}}{\pgfqpoint{4.881553in}{6.072498in}}%
\pgfpathcurveto{\pgfqpoint{4.885671in}{6.068380in}}{\pgfqpoint{4.891257in}{6.066066in}}{\pgfqpoint{4.897081in}{6.066066in}}%
\pgfpathlineto{\pgfqpoint{4.897081in}{6.066066in}}%
\pgfpathclose%
\pgfusepath{stroke,fill}%
\end{pgfscope}%
\begin{pgfscope}%
\pgfpathrectangle{\pgfqpoint{1.542338in}{0.880000in}}{\pgfqpoint{5.115323in}{6.160000in}}%
\pgfusepath{clip}%
\pgfsetbuttcap%
\pgfsetroundjoin%
\definecolor{currentfill}{rgb}{0.500000,0.000000,0.500000}%
\pgfsetfillcolor{currentfill}%
\pgfsetlinewidth{1.003750pt}%
\definecolor{currentstroke}{rgb}{0.500000,0.000000,0.500000}%
\pgfsetstrokecolor{currentstroke}%
\pgfsetdash{}{0pt}%
\pgfpathmoveto{\pgfqpoint{6.045751in}{5.112848in}}%
\pgfpathcurveto{\pgfqpoint{6.051575in}{5.112848in}}{\pgfqpoint{6.057161in}{5.115162in}}{\pgfqpoint{6.061279in}{5.119280in}}%
\pgfpathcurveto{\pgfqpoint{6.065397in}{5.123399in}}{\pgfqpoint{6.067711in}{5.128985in}}{\pgfqpoint{6.067711in}{5.134809in}}%
\pgfpathcurveto{\pgfqpoint{6.067711in}{5.140633in}}{\pgfqpoint{6.065397in}{5.146219in}}{\pgfqpoint{6.061279in}{5.150337in}}%
\pgfpathcurveto{\pgfqpoint{6.057161in}{5.154455in}}{\pgfqpoint{6.051575in}{5.156769in}}{\pgfqpoint{6.045751in}{5.156769in}}%
\pgfpathcurveto{\pgfqpoint{6.039927in}{5.156769in}}{\pgfqpoint{6.034341in}{5.154455in}}{\pgfqpoint{6.030222in}{5.150337in}}%
\pgfpathcurveto{\pgfqpoint{6.026104in}{5.146219in}}{\pgfqpoint{6.023790in}{5.140633in}}{\pgfqpoint{6.023790in}{5.134809in}}%
\pgfpathcurveto{\pgfqpoint{6.023790in}{5.128985in}}{\pgfqpoint{6.026104in}{5.123399in}}{\pgfqpoint{6.030222in}{5.119280in}}%
\pgfpathcurveto{\pgfqpoint{6.034341in}{5.115162in}}{\pgfqpoint{6.039927in}{5.112848in}}{\pgfqpoint{6.045751in}{5.112848in}}%
\pgfpathlineto{\pgfqpoint{6.045751in}{5.112848in}}%
\pgfpathclose%
\pgfusepath{stroke,fill}%
\end{pgfscope}%
\begin{pgfscope}%
\pgfpathrectangle{\pgfqpoint{1.542338in}{0.880000in}}{\pgfqpoint{5.115323in}{6.160000in}}%
\pgfusepath{clip}%
\pgfsetbuttcap%
\pgfsetroundjoin%
\definecolor{currentfill}{rgb}{0.500000,0.000000,0.500000}%
\pgfsetfillcolor{currentfill}%
\pgfsetlinewidth{1.003750pt}%
\definecolor{currentstroke}{rgb}{0.500000,0.000000,0.500000}%
\pgfsetstrokecolor{currentstroke}%
\pgfsetdash{}{0pt}%
\pgfpathmoveto{\pgfqpoint{4.346001in}{5.323846in}}%
\pgfpathcurveto{\pgfqpoint{4.351825in}{5.323846in}}{\pgfqpoint{4.357411in}{5.326159in}}{\pgfqpoint{4.361529in}{5.330278in}}%
\pgfpathcurveto{\pgfqpoint{4.365648in}{5.334396in}}{\pgfqpoint{4.367961in}{5.339982in}}{\pgfqpoint{4.367961in}{5.345806in}}%
\pgfpathcurveto{\pgfqpoint{4.367961in}{5.351630in}}{\pgfqpoint{4.365648in}{5.357216in}}{\pgfqpoint{4.361529in}{5.361334in}}%
\pgfpathcurveto{\pgfqpoint{4.357411in}{5.365452in}}{\pgfqpoint{4.351825in}{5.367766in}}{\pgfqpoint{4.346001in}{5.367766in}}%
\pgfpathcurveto{\pgfqpoint{4.340177in}{5.367766in}}{\pgfqpoint{4.334591in}{5.365452in}}{\pgfqpoint{4.330473in}{5.361334in}}%
\pgfpathcurveto{\pgfqpoint{4.326355in}{5.357216in}}{\pgfqpoint{4.324041in}{5.351630in}}{\pgfqpoint{4.324041in}{5.345806in}}%
\pgfpathcurveto{\pgfqpoint{4.324041in}{5.339982in}}{\pgfqpoint{4.326355in}{5.334396in}}{\pgfqpoint{4.330473in}{5.330278in}}%
\pgfpathcurveto{\pgfqpoint{4.334591in}{5.326159in}}{\pgfqpoint{4.340177in}{5.323846in}}{\pgfqpoint{4.346001in}{5.323846in}}%
\pgfpathlineto{\pgfqpoint{4.346001in}{5.323846in}}%
\pgfpathclose%
\pgfusepath{stroke,fill}%
\end{pgfscope}%
\begin{pgfscope}%
\pgfpathrectangle{\pgfqpoint{1.542338in}{0.880000in}}{\pgfqpoint{5.115323in}{6.160000in}}%
\pgfusepath{clip}%
\pgfsetbuttcap%
\pgfsetroundjoin%
\definecolor{currentfill}{rgb}{0.500000,0.000000,0.500000}%
\pgfsetfillcolor{currentfill}%
\pgfsetlinewidth{1.003750pt}%
\definecolor{currentstroke}{rgb}{0.500000,0.000000,0.500000}%
\pgfsetstrokecolor{currentstroke}%
\pgfsetdash{}{0pt}%
\pgfpathmoveto{\pgfqpoint{3.618126in}{1.908781in}}%
\pgfpathcurveto{\pgfqpoint{3.623950in}{1.908781in}}{\pgfqpoint{3.629536in}{1.911095in}}{\pgfqpoint{3.633654in}{1.915213in}}%
\pgfpathcurveto{\pgfqpoint{3.637773in}{1.919331in}}{\pgfqpoint{3.640086in}{1.924917in}}{\pgfqpoint{3.640086in}{1.930741in}}%
\pgfpathcurveto{\pgfqpoint{3.640086in}{1.936565in}}{\pgfqpoint{3.637773in}{1.942151in}}{\pgfqpoint{3.633654in}{1.946269in}}%
\pgfpathcurveto{\pgfqpoint{3.629536in}{1.950388in}}{\pgfqpoint{3.623950in}{1.952701in}}{\pgfqpoint{3.618126in}{1.952701in}}%
\pgfpathcurveto{\pgfqpoint{3.612302in}{1.952701in}}{\pgfqpoint{3.606716in}{1.950388in}}{\pgfqpoint{3.602598in}{1.946269in}}%
\pgfpathcurveto{\pgfqpoint{3.598480in}{1.942151in}}{\pgfqpoint{3.596166in}{1.936565in}}{\pgfqpoint{3.596166in}{1.930741in}}%
\pgfpathcurveto{\pgfqpoint{3.596166in}{1.924917in}}{\pgfqpoint{3.598480in}{1.919331in}}{\pgfqpoint{3.602598in}{1.915213in}}%
\pgfpathcurveto{\pgfqpoint{3.606716in}{1.911095in}}{\pgfqpoint{3.612302in}{1.908781in}}{\pgfqpoint{3.618126in}{1.908781in}}%
\pgfpathlineto{\pgfqpoint{3.618126in}{1.908781in}}%
\pgfpathclose%
\pgfusepath{stroke,fill}%
\end{pgfscope}%
\begin{pgfscope}%
\pgfpathrectangle{\pgfqpoint{1.542338in}{0.880000in}}{\pgfqpoint{5.115323in}{6.160000in}}%
\pgfusepath{clip}%
\pgfsetbuttcap%
\pgfsetroundjoin%
\definecolor{currentfill}{rgb}{0.500000,0.000000,0.500000}%
\pgfsetfillcolor{currentfill}%
\pgfsetlinewidth{1.003750pt}%
\definecolor{currentstroke}{rgb}{0.500000,0.000000,0.500000}%
\pgfsetstrokecolor{currentstroke}%
\pgfsetdash{}{0pt}%
\pgfpathmoveto{\pgfqpoint{5.995411in}{2.572517in}}%
\pgfpathcurveto{\pgfqpoint{6.001235in}{2.572517in}}{\pgfqpoint{6.006821in}{2.574831in}}{\pgfqpoint{6.010939in}{2.578949in}}%
\pgfpathcurveto{\pgfqpoint{6.015057in}{2.583067in}}{\pgfqpoint{6.017371in}{2.588654in}}{\pgfqpoint{6.017371in}{2.594478in}}%
\pgfpathcurveto{\pgfqpoint{6.017371in}{2.600302in}}{\pgfqpoint{6.015057in}{2.605888in}}{\pgfqpoint{6.010939in}{2.610006in}}%
\pgfpathcurveto{\pgfqpoint{6.006821in}{2.614124in}}{\pgfqpoint{6.001235in}{2.616438in}}{\pgfqpoint{5.995411in}{2.616438in}}%
\pgfpathcurveto{\pgfqpoint{5.989587in}{2.616438in}}{\pgfqpoint{5.984001in}{2.614124in}}{\pgfqpoint{5.979882in}{2.610006in}}%
\pgfpathcurveto{\pgfqpoint{5.975764in}{2.605888in}}{\pgfqpoint{5.973450in}{2.600302in}}{\pgfqpoint{5.973450in}{2.594478in}}%
\pgfpathcurveto{\pgfqpoint{5.973450in}{2.588654in}}{\pgfqpoint{5.975764in}{2.583067in}}{\pgfqpoint{5.979882in}{2.578949in}}%
\pgfpathcurveto{\pgfqpoint{5.984001in}{2.574831in}}{\pgfqpoint{5.989587in}{2.572517in}}{\pgfqpoint{5.995411in}{2.572517in}}%
\pgfpathlineto{\pgfqpoint{5.995411in}{2.572517in}}%
\pgfpathclose%
\pgfusepath{stroke,fill}%
\end{pgfscope}%
\begin{pgfscope}%
\pgfpathrectangle{\pgfqpoint{1.542338in}{0.880000in}}{\pgfqpoint{5.115323in}{6.160000in}}%
\pgfusepath{clip}%
\pgfsetbuttcap%
\pgfsetroundjoin%
\definecolor{currentfill}{rgb}{0.500000,0.000000,0.500000}%
\pgfsetfillcolor{currentfill}%
\pgfsetlinewidth{1.003750pt}%
\definecolor{currentstroke}{rgb}{0.500000,0.000000,0.500000}%
\pgfsetstrokecolor{currentstroke}%
\pgfsetdash{}{0pt}%
\pgfpathmoveto{\pgfqpoint{3.313297in}{6.278677in}}%
\pgfpathcurveto{\pgfqpoint{3.319121in}{6.278677in}}{\pgfqpoint{3.324707in}{6.280991in}}{\pgfqpoint{3.328825in}{6.285109in}}%
\pgfpathcurveto{\pgfqpoint{3.332943in}{6.289227in}}{\pgfqpoint{3.335257in}{6.294813in}}{\pgfqpoint{3.335257in}{6.300637in}}%
\pgfpathcurveto{\pgfqpoint{3.335257in}{6.306461in}}{\pgfqpoint{3.332943in}{6.312048in}}{\pgfqpoint{3.328825in}{6.316166in}}%
\pgfpathcurveto{\pgfqpoint{3.324707in}{6.320284in}}{\pgfqpoint{3.319121in}{6.322598in}}{\pgfqpoint{3.313297in}{6.322598in}}%
\pgfpathcurveto{\pgfqpoint{3.307473in}{6.322598in}}{\pgfqpoint{3.301887in}{6.320284in}}{\pgfqpoint{3.297769in}{6.316166in}}%
\pgfpathcurveto{\pgfqpoint{3.293651in}{6.312048in}}{\pgfqpoint{3.291337in}{6.306461in}}{\pgfqpoint{3.291337in}{6.300637in}}%
\pgfpathcurveto{\pgfqpoint{3.291337in}{6.294813in}}{\pgfqpoint{3.293651in}{6.289227in}}{\pgfqpoint{3.297769in}{6.285109in}}%
\pgfpathcurveto{\pgfqpoint{3.301887in}{6.280991in}}{\pgfqpoint{3.307473in}{6.278677in}}{\pgfqpoint{3.313297in}{6.278677in}}%
\pgfpathlineto{\pgfqpoint{3.313297in}{6.278677in}}%
\pgfpathclose%
\pgfusepath{stroke,fill}%
\end{pgfscope}%
\begin{pgfscope}%
\pgfpathrectangle{\pgfqpoint{1.542338in}{0.880000in}}{\pgfqpoint{5.115323in}{6.160000in}}%
\pgfusepath{clip}%
\pgfsetbuttcap%
\pgfsetroundjoin%
\definecolor{currentfill}{rgb}{0.500000,0.000000,0.500000}%
\pgfsetfillcolor{currentfill}%
\pgfsetlinewidth{1.003750pt}%
\definecolor{currentstroke}{rgb}{0.500000,0.000000,0.500000}%
\pgfsetstrokecolor{currentstroke}%
\pgfsetdash{}{0pt}%
\pgfpathmoveto{\pgfqpoint{2.201358in}{4.924458in}}%
\pgfpathcurveto{\pgfqpoint{2.207182in}{4.924458in}}{\pgfqpoint{2.212768in}{4.926771in}}{\pgfqpoint{2.216886in}{4.930890in}}%
\pgfpathcurveto{\pgfqpoint{2.221004in}{4.935008in}}{\pgfqpoint{2.223318in}{4.940594in}}{\pgfqpoint{2.223318in}{4.946418in}}%
\pgfpathcurveto{\pgfqpoint{2.223318in}{4.952242in}}{\pgfqpoint{2.221004in}{4.957828in}}{\pgfqpoint{2.216886in}{4.961946in}}%
\pgfpathcurveto{\pgfqpoint{2.212768in}{4.966064in}}{\pgfqpoint{2.207182in}{4.968378in}}{\pgfqpoint{2.201358in}{4.968378in}}%
\pgfpathcurveto{\pgfqpoint{2.195534in}{4.968378in}}{\pgfqpoint{2.189948in}{4.966064in}}{\pgfqpoint{2.185830in}{4.961946in}}%
\pgfpathcurveto{\pgfqpoint{2.181712in}{4.957828in}}{\pgfqpoint{2.179398in}{4.952242in}}{\pgfqpoint{2.179398in}{4.946418in}}%
\pgfpathcurveto{\pgfqpoint{2.179398in}{4.940594in}}{\pgfqpoint{2.181712in}{4.935008in}}{\pgfqpoint{2.185830in}{4.930890in}}%
\pgfpathcurveto{\pgfqpoint{2.189948in}{4.926771in}}{\pgfqpoint{2.195534in}{4.924458in}}{\pgfqpoint{2.201358in}{4.924458in}}%
\pgfpathlineto{\pgfqpoint{2.201358in}{4.924458in}}%
\pgfpathclose%
\pgfusepath{stroke,fill}%
\end{pgfscope}%
\begin{pgfscope}%
\pgfpathrectangle{\pgfqpoint{1.542338in}{0.880000in}}{\pgfqpoint{5.115323in}{6.160000in}}%
\pgfusepath{clip}%
\pgfsetbuttcap%
\pgfsetroundjoin%
\definecolor{currentfill}{rgb}{0.500000,0.000000,0.500000}%
\pgfsetfillcolor{currentfill}%
\pgfsetlinewidth{1.003750pt}%
\definecolor{currentstroke}{rgb}{0.500000,0.000000,0.500000}%
\pgfsetstrokecolor{currentstroke}%
\pgfsetdash{}{0pt}%
\pgfpathmoveto{\pgfqpoint{3.962646in}{4.289838in}}%
\pgfpathcurveto{\pgfqpoint{3.968470in}{4.289838in}}{\pgfqpoint{3.974056in}{4.292152in}}{\pgfqpoint{3.978175in}{4.296270in}}%
\pgfpathcurveto{\pgfqpoint{3.982293in}{4.300388in}}{\pgfqpoint{3.984607in}{4.305974in}}{\pgfqpoint{3.984607in}{4.311798in}}%
\pgfpathcurveto{\pgfqpoint{3.984607in}{4.317622in}}{\pgfqpoint{3.982293in}{4.323208in}}{\pgfqpoint{3.978175in}{4.327327in}}%
\pgfpathcurveto{\pgfqpoint{3.974056in}{4.331445in}}{\pgfqpoint{3.968470in}{4.333759in}}{\pgfqpoint{3.962646in}{4.333759in}}%
\pgfpathcurveto{\pgfqpoint{3.956822in}{4.333759in}}{\pgfqpoint{3.951236in}{4.331445in}}{\pgfqpoint{3.947118in}{4.327327in}}%
\pgfpathcurveto{\pgfqpoint{3.943000in}{4.323208in}}{\pgfqpoint{3.940686in}{4.317622in}}{\pgfqpoint{3.940686in}{4.311798in}}%
\pgfpathcurveto{\pgfqpoint{3.940686in}{4.305974in}}{\pgfqpoint{3.943000in}{4.300388in}}{\pgfqpoint{3.947118in}{4.296270in}}%
\pgfpathcurveto{\pgfqpoint{3.951236in}{4.292152in}}{\pgfqpoint{3.956822in}{4.289838in}}{\pgfqpoint{3.962646in}{4.289838in}}%
\pgfpathlineto{\pgfqpoint{3.962646in}{4.289838in}}%
\pgfpathclose%
\pgfusepath{stroke,fill}%
\end{pgfscope}%
\begin{pgfscope}%
\pgfpathrectangle{\pgfqpoint{1.542338in}{0.880000in}}{\pgfqpoint{5.115323in}{6.160000in}}%
\pgfusepath{clip}%
\pgfsetbuttcap%
\pgfsetroundjoin%
\definecolor{currentfill}{rgb}{0.800000,0.200000,0.200000}%
\pgfsetfillcolor{currentfill}%
\pgfsetlinewidth{1.003750pt}%
\definecolor{currentstroke}{rgb}{0.800000,0.200000,0.200000}%
\pgfsetstrokecolor{currentstroke}%
\pgfsetdash{}{0pt}%
\pgfpathmoveto{\pgfqpoint{5.600770in}{3.353008in}}%
\pgfpathcurveto{\pgfqpoint{5.606594in}{3.353008in}}{\pgfqpoint{5.612180in}{3.355322in}}{\pgfqpoint{5.616298in}{3.359440in}}%
\pgfpathcurveto{\pgfqpoint{5.620417in}{3.363558in}}{\pgfqpoint{5.622730in}{3.369144in}}{\pgfqpoint{5.622730in}{3.374968in}}%
\pgfpathcurveto{\pgfqpoint{5.622730in}{3.380792in}}{\pgfqpoint{5.620417in}{3.386378in}}{\pgfqpoint{5.616298in}{3.390496in}}%
\pgfpathcurveto{\pgfqpoint{5.612180in}{3.394614in}}{\pgfqpoint{5.606594in}{3.396928in}}{\pgfqpoint{5.600770in}{3.396928in}}%
\pgfpathcurveto{\pgfqpoint{5.594946in}{3.396928in}}{\pgfqpoint{5.589360in}{3.394614in}}{\pgfqpoint{5.585242in}{3.390496in}}%
\pgfpathcurveto{\pgfqpoint{5.581124in}{3.386378in}}{\pgfqpoint{5.578810in}{3.380792in}}{\pgfqpoint{5.578810in}{3.374968in}}%
\pgfpathcurveto{\pgfqpoint{5.578810in}{3.369144in}}{\pgfqpoint{5.581124in}{3.363558in}}{\pgfqpoint{5.585242in}{3.359440in}}%
\pgfpathcurveto{\pgfqpoint{5.589360in}{3.355322in}}{\pgfqpoint{5.594946in}{3.353008in}}{\pgfqpoint{5.600770in}{3.353008in}}%
\pgfpathlineto{\pgfqpoint{5.600770in}{3.353008in}}%
\pgfpathclose%
\pgfusepath{stroke,fill}%
\end{pgfscope}%
\begin{pgfscope}%
\pgfpathrectangle{\pgfqpoint{1.542338in}{0.880000in}}{\pgfqpoint{5.115323in}{6.160000in}}%
\pgfusepath{clip}%
\pgfsetbuttcap%
\pgfsetroundjoin%
\definecolor{currentfill}{rgb}{0.200000,0.200000,0.800000}%
\pgfsetfillcolor{currentfill}%
\pgfsetlinewidth{1.003750pt}%
\definecolor{currentstroke}{rgb}{0.200000,0.200000,0.800000}%
\pgfsetstrokecolor{currentstroke}%
\pgfsetdash{}{0pt}%
\pgfpathmoveto{\pgfqpoint{1.967031in}{1.799012in}}%
\pgfpathcurveto{\pgfqpoint{1.972855in}{1.799012in}}{\pgfqpoint{1.978442in}{1.801326in}}{\pgfqpoint{1.982560in}{1.805444in}}%
\pgfpathcurveto{\pgfqpoint{1.986678in}{1.809562in}}{\pgfqpoint{1.988992in}{1.815149in}}{\pgfqpoint{1.988992in}{1.820972in}}%
\pgfpathcurveto{\pgfqpoint{1.988992in}{1.826796in}}{\pgfqpoint{1.986678in}{1.832383in}}{\pgfqpoint{1.982560in}{1.836501in}}%
\pgfpathcurveto{\pgfqpoint{1.978442in}{1.840619in}}{\pgfqpoint{1.972855in}{1.842933in}}{\pgfqpoint{1.967031in}{1.842933in}}%
\pgfpathcurveto{\pgfqpoint{1.961208in}{1.842933in}}{\pgfqpoint{1.955621in}{1.840619in}}{\pgfqpoint{1.951503in}{1.836501in}}%
\pgfpathcurveto{\pgfqpoint{1.947385in}{1.832383in}}{\pgfqpoint{1.945071in}{1.826796in}}{\pgfqpoint{1.945071in}{1.820972in}}%
\pgfpathcurveto{\pgfqpoint{1.945071in}{1.815149in}}{\pgfqpoint{1.947385in}{1.809562in}}{\pgfqpoint{1.951503in}{1.805444in}}%
\pgfpathcurveto{\pgfqpoint{1.955621in}{1.801326in}}{\pgfqpoint{1.961208in}{1.799012in}}{\pgfqpoint{1.967031in}{1.799012in}}%
\pgfpathlineto{\pgfqpoint{1.967031in}{1.799012in}}%
\pgfpathclose%
\pgfusepath{stroke,fill}%
\end{pgfscope}%
\begin{pgfscope}%
\pgfpathrectangle{\pgfqpoint{1.542338in}{0.880000in}}{\pgfqpoint{5.115323in}{6.160000in}}%
\pgfusepath{clip}%
\pgfsetbuttcap%
\pgfsetroundjoin%
\definecolor{currentfill}{rgb}{0.500000,0.000000,0.500000}%
\pgfsetfillcolor{currentfill}%
\pgfsetlinewidth{1.003750pt}%
\definecolor{currentstroke}{rgb}{0.500000,0.000000,0.500000}%
\pgfsetstrokecolor{currentstroke}%
\pgfsetdash{}{0pt}%
\pgfpathmoveto{\pgfqpoint{4.641114in}{3.363411in}}%
\pgfpathcurveto{\pgfqpoint{4.646937in}{3.363411in}}{\pgfqpoint{4.652524in}{3.365725in}}{\pgfqpoint{4.656642in}{3.369843in}}%
\pgfpathcurveto{\pgfqpoint{4.660760in}{3.373961in}}{\pgfqpoint{4.663074in}{3.379547in}}{\pgfqpoint{4.663074in}{3.385371in}}%
\pgfpathcurveto{\pgfqpoint{4.663074in}{3.391195in}}{\pgfqpoint{4.660760in}{3.396781in}}{\pgfqpoint{4.656642in}{3.400899in}}%
\pgfpathcurveto{\pgfqpoint{4.652524in}{3.405017in}}{\pgfqpoint{4.646937in}{3.407331in}}{\pgfqpoint{4.641114in}{3.407331in}}%
\pgfpathcurveto{\pgfqpoint{4.635290in}{3.407331in}}{\pgfqpoint{4.629703in}{3.405017in}}{\pgfqpoint{4.625585in}{3.400899in}}%
\pgfpathcurveto{\pgfqpoint{4.621467in}{3.396781in}}{\pgfqpoint{4.619153in}{3.391195in}}{\pgfqpoint{4.619153in}{3.385371in}}%
\pgfpathcurveto{\pgfqpoint{4.619153in}{3.379547in}}{\pgfqpoint{4.621467in}{3.373961in}}{\pgfqpoint{4.625585in}{3.369843in}}%
\pgfpathcurveto{\pgfqpoint{4.629703in}{3.365725in}}{\pgfqpoint{4.635290in}{3.363411in}}{\pgfqpoint{4.641114in}{3.363411in}}%
\pgfpathlineto{\pgfqpoint{4.641114in}{3.363411in}}%
\pgfpathclose%
\pgfusepath{stroke,fill}%
\end{pgfscope}%
\begin{pgfscope}%
\pgfpathrectangle{\pgfqpoint{1.542338in}{0.880000in}}{\pgfqpoint{5.115323in}{6.160000in}}%
\pgfusepath{clip}%
\pgfsetbuttcap%
\pgfsetroundjoin%
\definecolor{currentfill}{rgb}{0.500000,0.000000,0.500000}%
\pgfsetfillcolor{currentfill}%
\pgfsetlinewidth{1.003750pt}%
\definecolor{currentstroke}{rgb}{0.500000,0.000000,0.500000}%
\pgfsetstrokecolor{currentstroke}%
\pgfsetdash{}{0pt}%
\pgfpathmoveto{\pgfqpoint{4.682455in}{3.988584in}}%
\pgfpathcurveto{\pgfqpoint{4.688278in}{3.988584in}}{\pgfqpoint{4.693865in}{3.990897in}}{\pgfqpoint{4.697983in}{3.995016in}}%
\pgfpathcurveto{\pgfqpoint{4.702101in}{3.999134in}}{\pgfqpoint{4.704415in}{4.004720in}}{\pgfqpoint{4.704415in}{4.010544in}}%
\pgfpathcurveto{\pgfqpoint{4.704415in}{4.016368in}}{\pgfqpoint{4.702101in}{4.021954in}}{\pgfqpoint{4.697983in}{4.026072in}}%
\pgfpathcurveto{\pgfqpoint{4.693865in}{4.030190in}}{\pgfqpoint{4.688278in}{4.032504in}}{\pgfqpoint{4.682455in}{4.032504in}}%
\pgfpathcurveto{\pgfqpoint{4.676631in}{4.032504in}}{\pgfqpoint{4.671044in}{4.030190in}}{\pgfqpoint{4.666926in}{4.026072in}}%
\pgfpathcurveto{\pgfqpoint{4.662808in}{4.021954in}}{\pgfqpoint{4.660494in}{4.016368in}}{\pgfqpoint{4.660494in}{4.010544in}}%
\pgfpathcurveto{\pgfqpoint{4.660494in}{4.004720in}}{\pgfqpoint{4.662808in}{3.999134in}}{\pgfqpoint{4.666926in}{3.995016in}}%
\pgfpathcurveto{\pgfqpoint{4.671044in}{3.990897in}}{\pgfqpoint{4.676631in}{3.988584in}}{\pgfqpoint{4.682455in}{3.988584in}}%
\pgfpathlineto{\pgfqpoint{4.682455in}{3.988584in}}%
\pgfpathclose%
\pgfusepath{stroke,fill}%
\end{pgfscope}%
\begin{pgfscope}%
\pgfpathrectangle{\pgfqpoint{1.542338in}{0.880000in}}{\pgfqpoint{5.115323in}{6.160000in}}%
\pgfusepath{clip}%
\pgfsetbuttcap%
\pgfsetroundjoin%
\definecolor{currentfill}{rgb}{0.800000,0.200000,0.200000}%
\pgfsetfillcolor{currentfill}%
\pgfsetlinewidth{1.003750pt}%
\definecolor{currentstroke}{rgb}{0.800000,0.200000,0.200000}%
\pgfsetstrokecolor{currentstroke}%
\pgfsetdash{}{0pt}%
\pgfpathmoveto{\pgfqpoint{6.291802in}{2.603839in}}%
\pgfpathcurveto{\pgfqpoint{6.297626in}{2.603839in}}{\pgfqpoint{6.303212in}{2.606152in}}{\pgfqpoint{6.307331in}{2.610271in}}%
\pgfpathcurveto{\pgfqpoint{6.311449in}{2.614389in}}{\pgfqpoint{6.313763in}{2.619975in}}{\pgfqpoint{6.313763in}{2.625799in}}%
\pgfpathcurveto{\pgfqpoint{6.313763in}{2.631623in}}{\pgfqpoint{6.311449in}{2.637209in}}{\pgfqpoint{6.307331in}{2.641327in}}%
\pgfpathcurveto{\pgfqpoint{6.303212in}{2.645445in}}{\pgfqpoint{6.297626in}{2.647759in}}{\pgfqpoint{6.291802in}{2.647759in}}%
\pgfpathcurveto{\pgfqpoint{6.285978in}{2.647759in}}{\pgfqpoint{6.280392in}{2.645445in}}{\pgfqpoint{6.276274in}{2.641327in}}%
\pgfpathcurveto{\pgfqpoint{6.272156in}{2.637209in}}{\pgfqpoint{6.269842in}{2.631623in}}{\pgfqpoint{6.269842in}{2.625799in}}%
\pgfpathcurveto{\pgfqpoint{6.269842in}{2.619975in}}{\pgfqpoint{6.272156in}{2.614389in}}{\pgfqpoint{6.276274in}{2.610271in}}%
\pgfpathcurveto{\pgfqpoint{6.280392in}{2.606152in}}{\pgfqpoint{6.285978in}{2.603839in}}{\pgfqpoint{6.291802in}{2.603839in}}%
\pgfpathlineto{\pgfqpoint{6.291802in}{2.603839in}}%
\pgfpathclose%
\pgfusepath{stroke,fill}%
\end{pgfscope}%
\begin{pgfscope}%
\pgfpathrectangle{\pgfqpoint{1.542338in}{0.880000in}}{\pgfqpoint{5.115323in}{6.160000in}}%
\pgfusepath{clip}%
\pgfsetbuttcap%
\pgfsetroundjoin%
\definecolor{currentfill}{rgb}{0.500000,0.000000,0.500000}%
\pgfsetfillcolor{currentfill}%
\pgfsetlinewidth{1.003750pt}%
\definecolor{currentstroke}{rgb}{0.500000,0.000000,0.500000}%
\pgfsetstrokecolor{currentstroke}%
\pgfsetdash{}{0pt}%
\pgfpathmoveto{\pgfqpoint{3.198734in}{2.419227in}}%
\pgfpathcurveto{\pgfqpoint{3.204558in}{2.419227in}}{\pgfqpoint{3.210144in}{2.421541in}}{\pgfqpoint{3.214263in}{2.425659in}}%
\pgfpathcurveto{\pgfqpoint{3.218381in}{2.429777in}}{\pgfqpoint{3.220695in}{2.435363in}}{\pgfqpoint{3.220695in}{2.441187in}}%
\pgfpathcurveto{\pgfqpoint{3.220695in}{2.447011in}}{\pgfqpoint{3.218381in}{2.452597in}}{\pgfqpoint{3.214263in}{2.456716in}}%
\pgfpathcurveto{\pgfqpoint{3.210144in}{2.460834in}}{\pgfqpoint{3.204558in}{2.463148in}}{\pgfqpoint{3.198734in}{2.463148in}}%
\pgfpathcurveto{\pgfqpoint{3.192910in}{2.463148in}}{\pgfqpoint{3.187324in}{2.460834in}}{\pgfqpoint{3.183206in}{2.456716in}}%
\pgfpathcurveto{\pgfqpoint{3.179088in}{2.452597in}}{\pgfqpoint{3.176774in}{2.447011in}}{\pgfqpoint{3.176774in}{2.441187in}}%
\pgfpathcurveto{\pgfqpoint{3.176774in}{2.435363in}}{\pgfqpoint{3.179088in}{2.429777in}}{\pgfqpoint{3.183206in}{2.425659in}}%
\pgfpathcurveto{\pgfqpoint{3.187324in}{2.421541in}}{\pgfqpoint{3.192910in}{2.419227in}}{\pgfqpoint{3.198734in}{2.419227in}}%
\pgfpathlineto{\pgfqpoint{3.198734in}{2.419227in}}%
\pgfpathclose%
\pgfusepath{stroke,fill}%
\end{pgfscope}%
\begin{pgfscope}%
\pgfpathrectangle{\pgfqpoint{1.542338in}{0.880000in}}{\pgfqpoint{5.115323in}{6.160000in}}%
\pgfusepath{clip}%
\pgfsetbuttcap%
\pgfsetroundjoin%
\definecolor{currentfill}{rgb}{0.500000,0.000000,0.500000}%
\pgfsetfillcolor{currentfill}%
\pgfsetlinewidth{1.003750pt}%
\definecolor{currentstroke}{rgb}{0.500000,0.000000,0.500000}%
\pgfsetstrokecolor{currentstroke}%
\pgfsetdash{}{0pt}%
\pgfpathmoveto{\pgfqpoint{3.161987in}{4.034365in}}%
\pgfpathcurveto{\pgfqpoint{3.167811in}{4.034365in}}{\pgfqpoint{3.173397in}{4.036679in}}{\pgfqpoint{3.177515in}{4.040797in}}%
\pgfpathcurveto{\pgfqpoint{3.181633in}{4.044916in}}{\pgfqpoint{3.183947in}{4.050502in}}{\pgfqpoint{3.183947in}{4.056326in}}%
\pgfpathcurveto{\pgfqpoint{3.183947in}{4.062150in}}{\pgfqpoint{3.181633in}{4.067736in}}{\pgfqpoint{3.177515in}{4.071854in}}%
\pgfpathcurveto{\pgfqpoint{3.173397in}{4.075972in}}{\pgfqpoint{3.167811in}{4.078286in}}{\pgfqpoint{3.161987in}{4.078286in}}%
\pgfpathcurveto{\pgfqpoint{3.156163in}{4.078286in}}{\pgfqpoint{3.150577in}{4.075972in}}{\pgfqpoint{3.146459in}{4.071854in}}%
\pgfpathcurveto{\pgfqpoint{3.142341in}{4.067736in}}{\pgfqpoint{3.140027in}{4.062150in}}{\pgfqpoint{3.140027in}{4.056326in}}%
\pgfpathcurveto{\pgfqpoint{3.140027in}{4.050502in}}{\pgfqpoint{3.142341in}{4.044916in}}{\pgfqpoint{3.146459in}{4.040797in}}%
\pgfpathcurveto{\pgfqpoint{3.150577in}{4.036679in}}{\pgfqpoint{3.156163in}{4.034365in}}{\pgfqpoint{3.161987in}{4.034365in}}%
\pgfpathlineto{\pgfqpoint{3.161987in}{4.034365in}}%
\pgfpathclose%
\pgfusepath{stroke,fill}%
\end{pgfscope}%
\begin{pgfscope}%
\pgfpathrectangle{\pgfqpoint{1.542338in}{0.880000in}}{\pgfqpoint{5.115323in}{6.160000in}}%
\pgfusepath{clip}%
\pgfsetbuttcap%
\pgfsetroundjoin%
\definecolor{currentfill}{rgb}{0.500000,0.000000,0.500000}%
\pgfsetfillcolor{currentfill}%
\pgfsetlinewidth{1.003750pt}%
\definecolor{currentstroke}{rgb}{0.500000,0.000000,0.500000}%
\pgfsetstrokecolor{currentstroke}%
\pgfsetdash{}{0pt}%
\pgfpathmoveto{\pgfqpoint{3.940687in}{4.407646in}}%
\pgfpathcurveto{\pgfqpoint{3.946511in}{4.407646in}}{\pgfqpoint{3.952097in}{4.409960in}}{\pgfqpoint{3.956215in}{4.414078in}}%
\pgfpathcurveto{\pgfqpoint{3.960333in}{4.418196in}}{\pgfqpoint{3.962647in}{4.423782in}}{\pgfqpoint{3.962647in}{4.429606in}}%
\pgfpathcurveto{\pgfqpoint{3.962647in}{4.435430in}}{\pgfqpoint{3.960333in}{4.441016in}}{\pgfqpoint{3.956215in}{4.445134in}}%
\pgfpathcurveto{\pgfqpoint{3.952097in}{4.449252in}}{\pgfqpoint{3.946511in}{4.451566in}}{\pgfqpoint{3.940687in}{4.451566in}}%
\pgfpathcurveto{\pgfqpoint{3.934863in}{4.451566in}}{\pgfqpoint{3.929277in}{4.449252in}}{\pgfqpoint{3.925159in}{4.445134in}}%
\pgfpathcurveto{\pgfqpoint{3.921041in}{4.441016in}}{\pgfqpoint{3.918727in}{4.435430in}}{\pgfqpoint{3.918727in}{4.429606in}}%
\pgfpathcurveto{\pgfqpoint{3.918727in}{4.423782in}}{\pgfqpoint{3.921041in}{4.418196in}}{\pgfqpoint{3.925159in}{4.414078in}}%
\pgfpathcurveto{\pgfqpoint{3.929277in}{4.409960in}}{\pgfqpoint{3.934863in}{4.407646in}}{\pgfqpoint{3.940687in}{4.407646in}}%
\pgfpathlineto{\pgfqpoint{3.940687in}{4.407646in}}%
\pgfpathclose%
\pgfusepath{stroke,fill}%
\end{pgfscope}%
\begin{pgfscope}%
\pgfpathrectangle{\pgfqpoint{1.542338in}{0.880000in}}{\pgfqpoint{5.115323in}{6.160000in}}%
\pgfusepath{clip}%
\pgfsetbuttcap%
\pgfsetroundjoin%
\definecolor{currentfill}{rgb}{0.500000,0.000000,0.500000}%
\pgfsetfillcolor{currentfill}%
\pgfsetlinewidth{1.003750pt}%
\definecolor{currentstroke}{rgb}{0.500000,0.000000,0.500000}%
\pgfsetstrokecolor{currentstroke}%
\pgfsetdash{}{0pt}%
\pgfpathmoveto{\pgfqpoint{3.538112in}{2.740931in}}%
\pgfpathcurveto{\pgfqpoint{3.543936in}{2.740931in}}{\pgfqpoint{3.549523in}{2.743245in}}{\pgfqpoint{3.553641in}{2.747363in}}%
\pgfpathcurveto{\pgfqpoint{3.557759in}{2.751481in}}{\pgfqpoint{3.560073in}{2.757067in}}{\pgfqpoint{3.560073in}{2.762891in}}%
\pgfpathcurveto{\pgfqpoint{3.560073in}{2.768715in}}{\pgfqpoint{3.557759in}{2.774301in}}{\pgfqpoint{3.553641in}{2.778419in}}%
\pgfpathcurveto{\pgfqpoint{3.549523in}{2.782538in}}{\pgfqpoint{3.543936in}{2.784851in}}{\pgfqpoint{3.538112in}{2.784851in}}%
\pgfpathcurveto{\pgfqpoint{3.532289in}{2.784851in}}{\pgfqpoint{3.526702in}{2.782538in}}{\pgfqpoint{3.522584in}{2.778419in}}%
\pgfpathcurveto{\pgfqpoint{3.518466in}{2.774301in}}{\pgfqpoint{3.516152in}{2.768715in}}{\pgfqpoint{3.516152in}{2.762891in}}%
\pgfpathcurveto{\pgfqpoint{3.516152in}{2.757067in}}{\pgfqpoint{3.518466in}{2.751481in}}{\pgfqpoint{3.522584in}{2.747363in}}%
\pgfpathcurveto{\pgfqpoint{3.526702in}{2.743245in}}{\pgfqpoint{3.532289in}{2.740931in}}{\pgfqpoint{3.538112in}{2.740931in}}%
\pgfpathlineto{\pgfqpoint{3.538112in}{2.740931in}}%
\pgfpathclose%
\pgfusepath{stroke,fill}%
\end{pgfscope}%
\begin{pgfscope}%
\pgfpathrectangle{\pgfqpoint{1.542338in}{0.880000in}}{\pgfqpoint{5.115323in}{6.160000in}}%
\pgfusepath{clip}%
\pgfsetbuttcap%
\pgfsetroundjoin%
\definecolor{currentfill}{rgb}{0.500000,0.000000,0.500000}%
\pgfsetfillcolor{currentfill}%
\pgfsetlinewidth{1.003750pt}%
\definecolor{currentstroke}{rgb}{0.500000,0.000000,0.500000}%
\pgfsetstrokecolor{currentstroke}%
\pgfsetdash{}{0pt}%
\pgfpathmoveto{\pgfqpoint{5.365940in}{4.414801in}}%
\pgfpathcurveto{\pgfqpoint{5.371764in}{4.414801in}}{\pgfqpoint{5.377350in}{4.417114in}}{\pgfqpoint{5.381468in}{4.421233in}}%
\pgfpathcurveto{\pgfqpoint{5.385586in}{4.425351in}}{\pgfqpoint{5.387900in}{4.430937in}}{\pgfqpoint{5.387900in}{4.436761in}}%
\pgfpathcurveto{\pgfqpoint{5.387900in}{4.442585in}}{\pgfqpoint{5.385586in}{4.448171in}}{\pgfqpoint{5.381468in}{4.452289in}}%
\pgfpathcurveto{\pgfqpoint{5.377350in}{4.456407in}}{\pgfqpoint{5.371764in}{4.458721in}}{\pgfqpoint{5.365940in}{4.458721in}}%
\pgfpathcurveto{\pgfqpoint{5.360116in}{4.458721in}}{\pgfqpoint{5.354530in}{4.456407in}}{\pgfqpoint{5.350412in}{4.452289in}}%
\pgfpathcurveto{\pgfqpoint{5.346293in}{4.448171in}}{\pgfqpoint{5.343980in}{4.442585in}}{\pgfqpoint{5.343980in}{4.436761in}}%
\pgfpathcurveto{\pgfqpoint{5.343980in}{4.430937in}}{\pgfqpoint{5.346293in}{4.425351in}}{\pgfqpoint{5.350412in}{4.421233in}}%
\pgfpathcurveto{\pgfqpoint{5.354530in}{4.417114in}}{\pgfqpoint{5.360116in}{4.414801in}}{\pgfqpoint{5.365940in}{4.414801in}}%
\pgfpathlineto{\pgfqpoint{5.365940in}{4.414801in}}%
\pgfpathclose%
\pgfusepath{stroke,fill}%
\end{pgfscope}%
\begin{pgfscope}%
\pgfpathrectangle{\pgfqpoint{1.542338in}{0.880000in}}{\pgfqpoint{5.115323in}{6.160000in}}%
\pgfusepath{clip}%
\pgfsetbuttcap%
\pgfsetroundjoin%
\definecolor{currentfill}{rgb}{0.500000,0.000000,0.500000}%
\pgfsetfillcolor{currentfill}%
\pgfsetlinewidth{1.003750pt}%
\definecolor{currentstroke}{rgb}{0.500000,0.000000,0.500000}%
\pgfsetstrokecolor{currentstroke}%
\pgfsetdash{}{0pt}%
\pgfpathmoveto{\pgfqpoint{4.910254in}{5.922373in}}%
\pgfpathcurveto{\pgfqpoint{4.916078in}{5.922373in}}{\pgfqpoint{4.921665in}{5.924686in}}{\pgfqpoint{4.925783in}{5.928805in}}%
\pgfpathcurveto{\pgfqpoint{4.929901in}{5.932923in}}{\pgfqpoint{4.932215in}{5.938509in}}{\pgfqpoint{4.932215in}{5.944333in}}%
\pgfpathcurveto{\pgfqpoint{4.932215in}{5.950157in}}{\pgfqpoint{4.929901in}{5.955743in}}{\pgfqpoint{4.925783in}{5.959861in}}%
\pgfpathcurveto{\pgfqpoint{4.921665in}{5.963979in}}{\pgfqpoint{4.916078in}{5.966293in}}{\pgfqpoint{4.910254in}{5.966293in}}%
\pgfpathcurveto{\pgfqpoint{4.904431in}{5.966293in}}{\pgfqpoint{4.898844in}{5.963979in}}{\pgfqpoint{4.894726in}{5.959861in}}%
\pgfpathcurveto{\pgfqpoint{4.890608in}{5.955743in}}{\pgfqpoint{4.888294in}{5.950157in}}{\pgfqpoint{4.888294in}{5.944333in}}%
\pgfpathcurveto{\pgfqpoint{4.888294in}{5.938509in}}{\pgfqpoint{4.890608in}{5.932923in}}{\pgfqpoint{4.894726in}{5.928805in}}%
\pgfpathcurveto{\pgfqpoint{4.898844in}{5.924686in}}{\pgfqpoint{4.904431in}{5.922373in}}{\pgfqpoint{4.910254in}{5.922373in}}%
\pgfpathlineto{\pgfqpoint{4.910254in}{5.922373in}}%
\pgfpathclose%
\pgfusepath{stroke,fill}%
\end{pgfscope}%
\begin{pgfscope}%
\pgfpathrectangle{\pgfqpoint{1.542338in}{0.880000in}}{\pgfqpoint{5.115323in}{6.160000in}}%
\pgfusepath{clip}%
\pgfsetbuttcap%
\pgfsetroundjoin%
\definecolor{currentfill}{rgb}{0.500000,0.000000,0.500000}%
\pgfsetfillcolor{currentfill}%
\pgfsetlinewidth{1.003750pt}%
\definecolor{currentstroke}{rgb}{0.500000,0.000000,0.500000}%
\pgfsetstrokecolor{currentstroke}%
\pgfsetdash{}{0pt}%
\pgfpathmoveto{\pgfqpoint{5.474691in}{3.864588in}}%
\pgfpathcurveto{\pgfqpoint{5.480515in}{3.864588in}}{\pgfqpoint{5.486101in}{3.866902in}}{\pgfqpoint{5.490219in}{3.871020in}}%
\pgfpathcurveto{\pgfqpoint{5.494337in}{3.875139in}}{\pgfqpoint{5.496651in}{3.880725in}}{\pgfqpoint{5.496651in}{3.886549in}}%
\pgfpathcurveto{\pgfqpoint{5.496651in}{3.892373in}}{\pgfqpoint{5.494337in}{3.897959in}}{\pgfqpoint{5.490219in}{3.902077in}}%
\pgfpathcurveto{\pgfqpoint{5.486101in}{3.906195in}}{\pgfqpoint{5.480515in}{3.908509in}}{\pgfqpoint{5.474691in}{3.908509in}}%
\pgfpathcurveto{\pgfqpoint{5.468867in}{3.908509in}}{\pgfqpoint{5.463281in}{3.906195in}}{\pgfqpoint{5.459163in}{3.902077in}}%
\pgfpathcurveto{\pgfqpoint{5.455044in}{3.897959in}}{\pgfqpoint{5.452731in}{3.892373in}}{\pgfqpoint{5.452731in}{3.886549in}}%
\pgfpathcurveto{\pgfqpoint{5.452731in}{3.880725in}}{\pgfqpoint{5.455044in}{3.875139in}}{\pgfqpoint{5.459163in}{3.871020in}}%
\pgfpathcurveto{\pgfqpoint{5.463281in}{3.866902in}}{\pgfqpoint{5.468867in}{3.864588in}}{\pgfqpoint{5.474691in}{3.864588in}}%
\pgfpathlineto{\pgfqpoint{5.474691in}{3.864588in}}%
\pgfpathclose%
\pgfusepath{stroke,fill}%
\end{pgfscope}%
\begin{pgfscope}%
\pgfpathrectangle{\pgfqpoint{1.542338in}{0.880000in}}{\pgfqpoint{5.115323in}{6.160000in}}%
\pgfusepath{clip}%
\pgfsetbuttcap%
\pgfsetmiterjoin%
\pgfsetlinewidth{1.003750pt}%
\definecolor{currentstroke}{rgb}{0.800000,0.200000,0.200000}%
\pgfsetstrokecolor{currentstroke}%
\pgfsetdash{}{0pt}%
\pgfpathmoveto{\pgfqpoint{5.448380in}{1.188476in}}%
\pgfpathcurveto{\pgfqpoint{5.718627in}{1.188476in}}{\pgfqpoint{5.977841in}{1.295846in}}{\pgfqpoint{6.168935in}{1.486939in}}%
\pgfpathcurveto{\pgfqpoint{6.360028in}{1.678033in}}{\pgfqpoint{6.467398in}{1.937247in}}{\pgfqpoint{6.467398in}{2.207494in}}%
\pgfpathcurveto{\pgfqpoint{6.467398in}{2.477741in}}{\pgfqpoint{6.360028in}{2.736956in}}{\pgfqpoint{6.168935in}{2.928049in}}%
\pgfpathcurveto{\pgfqpoint{5.977841in}{3.119142in}}{\pgfqpoint{5.718627in}{3.226513in}}{\pgfqpoint{5.448380in}{3.226513in}}%
\pgfpathcurveto{\pgfqpoint{5.178133in}{3.226513in}}{\pgfqpoint{4.918918in}{3.119142in}}{\pgfqpoint{4.727825in}{2.928049in}}%
\pgfpathcurveto{\pgfqpoint{4.536732in}{2.736956in}}{\pgfqpoint{4.429361in}{2.477741in}}{\pgfqpoint{4.429361in}{2.207494in}}%
\pgfpathcurveto{\pgfqpoint{4.429361in}{1.937247in}}{\pgfqpoint{4.536732in}{1.678033in}}{\pgfqpoint{4.727825in}{1.486939in}}%
\pgfpathcurveto{\pgfqpoint{4.918918in}{1.295846in}}{\pgfqpoint{5.178133in}{1.188476in}}{\pgfqpoint{5.448380in}{1.188476in}}%
\pgfpathlineto{\pgfqpoint{5.448380in}{1.188476in}}%
\pgfpathclose%
\pgfusepath{stroke}%
\end{pgfscope}%
\begin{pgfscope}%
\pgfpathrectangle{\pgfqpoint{1.542338in}{0.880000in}}{\pgfqpoint{5.115323in}{6.160000in}}%
\pgfusepath{clip}%
\pgfsetbuttcap%
\pgfsetroundjoin%
\definecolor{currentfill}{rgb}{0.000000,0.000000,0.000000}%
\pgfsetfillcolor{currentfill}%
\pgfsetlinewidth{1.003750pt}%
\definecolor{currentstroke}{rgb}{0.000000,0.000000,0.000000}%
\pgfsetstrokecolor{currentstroke}%
\pgfsetdash{}{0pt}%
\pgfsys@defobject{currentmarker}{\pgfqpoint{-0.021960in}{-0.021960in}}{\pgfqpoint{0.021960in}{0.021960in}}{%
\pgfpathmoveto{\pgfqpoint{0.000000in}{-0.021960in}}%
\pgfpathcurveto{\pgfqpoint{0.005824in}{-0.021960in}}{\pgfqpoint{0.011410in}{-0.019646in}}{\pgfqpoint{0.015528in}{-0.015528in}}%
\pgfpathcurveto{\pgfqpoint{0.019646in}{-0.011410in}}{\pgfqpoint{0.021960in}{-0.005824in}}{\pgfqpoint{0.021960in}{0.000000in}}%
\pgfpathcurveto{\pgfqpoint{0.021960in}{0.005824in}}{\pgfqpoint{0.019646in}{0.011410in}}{\pgfqpoint{0.015528in}{0.015528in}}%
\pgfpathcurveto{\pgfqpoint{0.011410in}{0.019646in}}{\pgfqpoint{0.005824in}{0.021960in}}{\pgfqpoint{0.000000in}{0.021960in}}%
\pgfpathcurveto{\pgfqpoint{-0.005824in}{0.021960in}}{\pgfqpoint{-0.011410in}{0.019646in}}{\pgfqpoint{-0.015528in}{0.015528in}}%
\pgfpathcurveto{\pgfqpoint{-0.019646in}{0.011410in}}{\pgfqpoint{-0.021960in}{0.005824in}}{\pgfqpoint{-0.021960in}{0.000000in}}%
\pgfpathcurveto{\pgfqpoint{-0.021960in}{-0.005824in}}{\pgfqpoint{-0.019646in}{-0.011410in}}{\pgfqpoint{-0.015528in}{-0.015528in}}%
\pgfpathcurveto{\pgfqpoint{-0.011410in}{-0.019646in}}{\pgfqpoint{-0.005824in}{-0.021960in}}{\pgfqpoint{0.000000in}{-0.021960in}}%
\pgfpathlineto{\pgfqpoint{0.000000in}{-0.021960in}}%
\pgfpathclose%
\pgfusepath{stroke,fill}%
}%
\begin{pgfscope}%
\pgfsys@transformshift{5.448380in}{2.207494in}%
\pgfsys@useobject{currentmarker}{}%
\end{pgfscope}%
\end{pgfscope}%
\begin{pgfscope}%
\pgfpathrectangle{\pgfqpoint{1.542338in}{0.880000in}}{\pgfqpoint{5.115323in}{6.160000in}}%
\pgfusepath{clip}%
\pgfsetbuttcap%
\pgfsetmiterjoin%
\pgfsetlinewidth{1.003750pt}%
\definecolor{currentstroke}{rgb}{0.200000,0.800000,0.200000}%
\pgfsetstrokecolor{currentstroke}%
\pgfsetdash{}{0pt}%
\pgfpathmoveto{\pgfqpoint{4.516112in}{4.845237in}}%
\pgfpathcurveto{\pgfqpoint{4.768988in}{4.845237in}}{\pgfqpoint{5.011541in}{4.945706in}}{\pgfqpoint{5.190352in}{5.124516in}}%
\pgfpathcurveto{\pgfqpoint{5.369162in}{5.303327in}}{\pgfqpoint{5.469631in}{5.545880in}}{\pgfqpoint{5.469631in}{5.798756in}}%
\pgfpathcurveto{\pgfqpoint{5.469631in}{6.051633in}}{\pgfqpoint{5.369162in}{6.294186in}}{\pgfqpoint{5.190352in}{6.472996in}}%
\pgfpathcurveto{\pgfqpoint{5.011541in}{6.651807in}}{\pgfqpoint{4.768988in}{6.752276in}}{\pgfqpoint{4.516112in}{6.752276in}}%
\pgfpathcurveto{\pgfqpoint{4.263236in}{6.752276in}}{\pgfqpoint{4.020682in}{6.651807in}}{\pgfqpoint{3.841872in}{6.472996in}}%
\pgfpathcurveto{\pgfqpoint{3.663061in}{6.294186in}}{\pgfqpoint{3.562593in}{6.051633in}}{\pgfqpoint{3.562593in}{5.798756in}}%
\pgfpathcurveto{\pgfqpoint{3.562593in}{5.545880in}}{\pgfqpoint{3.663061in}{5.303327in}}{\pgfqpoint{3.841872in}{5.124516in}}%
\pgfpathcurveto{\pgfqpoint{4.020682in}{4.945706in}}{\pgfqpoint{4.263236in}{4.845237in}}{\pgfqpoint{4.516112in}{4.845237in}}%
\pgfpathlineto{\pgfqpoint{4.516112in}{4.845237in}}%
\pgfpathclose%
\pgfusepath{stroke}%
\end{pgfscope}%
\begin{pgfscope}%
\pgfpathrectangle{\pgfqpoint{1.542338in}{0.880000in}}{\pgfqpoint{5.115323in}{6.160000in}}%
\pgfusepath{clip}%
\pgfsetbuttcap%
\pgfsetroundjoin%
\definecolor{currentfill}{rgb}{0.000000,0.000000,0.000000}%
\pgfsetfillcolor{currentfill}%
\pgfsetlinewidth{1.003750pt}%
\definecolor{currentstroke}{rgb}{0.000000,0.000000,0.000000}%
\pgfsetstrokecolor{currentstroke}%
\pgfsetdash{}{0pt}%
\pgfsys@defobject{currentmarker}{\pgfqpoint{-0.021960in}{-0.021960in}}{\pgfqpoint{0.021960in}{0.021960in}}{%
\pgfpathmoveto{\pgfqpoint{0.000000in}{-0.021960in}}%
\pgfpathcurveto{\pgfqpoint{0.005824in}{-0.021960in}}{\pgfqpoint{0.011410in}{-0.019646in}}{\pgfqpoint{0.015528in}{-0.015528in}}%
\pgfpathcurveto{\pgfqpoint{0.019646in}{-0.011410in}}{\pgfqpoint{0.021960in}{-0.005824in}}{\pgfqpoint{0.021960in}{0.000000in}}%
\pgfpathcurveto{\pgfqpoint{0.021960in}{0.005824in}}{\pgfqpoint{0.019646in}{0.011410in}}{\pgfqpoint{0.015528in}{0.015528in}}%
\pgfpathcurveto{\pgfqpoint{0.011410in}{0.019646in}}{\pgfqpoint{0.005824in}{0.021960in}}{\pgfqpoint{0.000000in}{0.021960in}}%
\pgfpathcurveto{\pgfqpoint{-0.005824in}{0.021960in}}{\pgfqpoint{-0.011410in}{0.019646in}}{\pgfqpoint{-0.015528in}{0.015528in}}%
\pgfpathcurveto{\pgfqpoint{-0.019646in}{0.011410in}}{\pgfqpoint{-0.021960in}{0.005824in}}{\pgfqpoint{-0.021960in}{0.000000in}}%
\pgfpathcurveto{\pgfqpoint{-0.021960in}{-0.005824in}}{\pgfqpoint{-0.019646in}{-0.011410in}}{\pgfqpoint{-0.015528in}{-0.015528in}}%
\pgfpathcurveto{\pgfqpoint{-0.011410in}{-0.019646in}}{\pgfqpoint{-0.005824in}{-0.021960in}}{\pgfqpoint{0.000000in}{-0.021960in}}%
\pgfpathlineto{\pgfqpoint{0.000000in}{-0.021960in}}%
\pgfpathclose%
\pgfusepath{stroke,fill}%
}%
\begin{pgfscope}%
\pgfsys@transformshift{4.516112in}{5.798756in}%
\pgfsys@useobject{currentmarker}{}%
\end{pgfscope}%
\end{pgfscope}%
\begin{pgfscope}%
\pgfpathrectangle{\pgfqpoint{1.542338in}{0.880000in}}{\pgfqpoint{5.115323in}{6.160000in}}%
\pgfusepath{clip}%
\pgfsetbuttcap%
\pgfsetmiterjoin%
\pgfsetlinewidth{1.003750pt}%
\definecolor{currentstroke}{rgb}{0.200000,0.200000,0.800000}%
\pgfsetstrokecolor{currentstroke}%
\pgfsetdash{}{0pt}%
\pgfpathmoveto{\pgfqpoint{2.975988in}{1.387924in}}%
\pgfpathcurveto{\pgfqpoint{3.293874in}{1.387924in}}{\pgfqpoint{3.598782in}{1.514221in}}{\pgfqpoint{3.823561in}{1.739000in}}%
\pgfpathcurveto{\pgfqpoint{4.048340in}{1.963779in}}{\pgfqpoint{4.174637in}{2.268688in}}{\pgfqpoint{4.174637in}{2.586574in}}%
\pgfpathcurveto{\pgfqpoint{4.174637in}{2.904459in}}{\pgfqpoint{4.048340in}{3.209368in}}{\pgfqpoint{3.823561in}{3.434147in}}%
\pgfpathcurveto{\pgfqpoint{3.598782in}{3.658926in}}{\pgfqpoint{3.293874in}{3.785223in}}{\pgfqpoint{2.975988in}{3.785223in}}%
\pgfpathcurveto{\pgfqpoint{2.658103in}{3.785223in}}{\pgfqpoint{2.353194in}{3.658926in}}{\pgfqpoint{2.128415in}{3.434147in}}%
\pgfpathcurveto{\pgfqpoint{1.903636in}{3.209368in}}{\pgfqpoint{1.777339in}{2.904459in}}{\pgfqpoint{1.777339in}{2.586574in}}%
\pgfpathcurveto{\pgfqpoint{1.777339in}{2.268688in}}{\pgfqpoint{1.903636in}{1.963779in}}{\pgfqpoint{2.128415in}{1.739000in}}%
\pgfpathcurveto{\pgfqpoint{2.353194in}{1.514221in}}{\pgfqpoint{2.658103in}{1.387924in}}{\pgfqpoint{2.975988in}{1.387924in}}%
\pgfpathlineto{\pgfqpoint{2.975988in}{1.387924in}}%
\pgfpathclose%
\pgfusepath{stroke}%
\end{pgfscope}%
\begin{pgfscope}%
\pgfpathrectangle{\pgfqpoint{1.542338in}{0.880000in}}{\pgfqpoint{5.115323in}{6.160000in}}%
\pgfusepath{clip}%
\pgfsetbuttcap%
\pgfsetroundjoin%
\definecolor{currentfill}{rgb}{0.000000,0.000000,0.000000}%
\pgfsetfillcolor{currentfill}%
\pgfsetlinewidth{1.003750pt}%
\definecolor{currentstroke}{rgb}{0.000000,0.000000,0.000000}%
\pgfsetstrokecolor{currentstroke}%
\pgfsetdash{}{0pt}%
\pgfsys@defobject{currentmarker}{\pgfqpoint{-0.021960in}{-0.021960in}}{\pgfqpoint{0.021960in}{0.021960in}}{%
\pgfpathmoveto{\pgfqpoint{0.000000in}{-0.021960in}}%
\pgfpathcurveto{\pgfqpoint{0.005824in}{-0.021960in}}{\pgfqpoint{0.011410in}{-0.019646in}}{\pgfqpoint{0.015528in}{-0.015528in}}%
\pgfpathcurveto{\pgfqpoint{0.019646in}{-0.011410in}}{\pgfqpoint{0.021960in}{-0.005824in}}{\pgfqpoint{0.021960in}{0.000000in}}%
\pgfpathcurveto{\pgfqpoint{0.021960in}{0.005824in}}{\pgfqpoint{0.019646in}{0.011410in}}{\pgfqpoint{0.015528in}{0.015528in}}%
\pgfpathcurveto{\pgfqpoint{0.011410in}{0.019646in}}{\pgfqpoint{0.005824in}{0.021960in}}{\pgfqpoint{0.000000in}{0.021960in}}%
\pgfpathcurveto{\pgfqpoint{-0.005824in}{0.021960in}}{\pgfqpoint{-0.011410in}{0.019646in}}{\pgfqpoint{-0.015528in}{0.015528in}}%
\pgfpathcurveto{\pgfqpoint{-0.019646in}{0.011410in}}{\pgfqpoint{-0.021960in}{0.005824in}}{\pgfqpoint{-0.021960in}{0.000000in}}%
\pgfpathcurveto{\pgfqpoint{-0.021960in}{-0.005824in}}{\pgfqpoint{-0.019646in}{-0.011410in}}{\pgfqpoint{-0.015528in}{-0.015528in}}%
\pgfpathcurveto{\pgfqpoint{-0.011410in}{-0.019646in}}{\pgfqpoint{-0.005824in}{-0.021960in}}{\pgfqpoint{0.000000in}{-0.021960in}}%
\pgfpathlineto{\pgfqpoint{0.000000in}{-0.021960in}}%
\pgfpathclose%
\pgfusepath{stroke,fill}%
}%
\begin{pgfscope}%
\pgfsys@transformshift{2.975988in}{2.586574in}%
\pgfsys@useobject{currentmarker}{}%
\end{pgfscope}%
\end{pgfscope}%
\begin{pgfscope}%
\pgfsetbuttcap%
\pgfsetroundjoin%
\definecolor{currentfill}{rgb}{0.000000,0.000000,0.000000}%
\pgfsetfillcolor{currentfill}%
\pgfsetlinewidth{0.803000pt}%
\definecolor{currentstroke}{rgb}{0.000000,0.000000,0.000000}%
\pgfsetstrokecolor{currentstroke}%
\pgfsetdash{}{0pt}%
\pgfsys@defobject{currentmarker}{\pgfqpoint{0.000000in}{-0.048611in}}{\pgfqpoint{0.000000in}{0.000000in}}{%
\pgfpathmoveto{\pgfqpoint{0.000000in}{0.000000in}}%
\pgfpathlineto{\pgfqpoint{0.000000in}{-0.048611in}}%
\pgfusepath{stroke,fill}%
}%
\begin{pgfscope}%
\pgfsys@transformshift{1.818751in}{0.880000in}%
\pgfsys@useobject{currentmarker}{}%
\end{pgfscope}%
\end{pgfscope}%
\begin{pgfscope}%
\definecolor{textcolor}{rgb}{0.000000,0.000000,0.000000}%
\pgfsetstrokecolor{textcolor}%
\pgfsetfillcolor{textcolor}%
\pgftext[x=1.818751in,y=0.782778in,,top]{\color{textcolor}{\sffamily\fontsize{10.000000}{12.000000}\selectfont\catcode`\^=\active\def^{\ifmmode\sp\else\^{}\fi}\catcode`\%=\active\def%{\%}\ensuremath{-}600}}%
\end{pgfscope}%
\begin{pgfscope}%
\pgfsetbuttcap%
\pgfsetroundjoin%
\definecolor{currentfill}{rgb}{0.000000,0.000000,0.000000}%
\pgfsetfillcolor{currentfill}%
\pgfsetlinewidth{0.803000pt}%
\definecolor{currentstroke}{rgb}{0.000000,0.000000,0.000000}%
\pgfsetstrokecolor{currentstroke}%
\pgfsetdash{}{0pt}%
\pgfsys@defobject{currentmarker}{\pgfqpoint{0.000000in}{-0.048611in}}{\pgfqpoint{0.000000in}{0.000000in}}{%
\pgfpathmoveto{\pgfqpoint{0.000000in}{0.000000in}}%
\pgfpathlineto{\pgfqpoint{0.000000in}{-0.048611in}}%
\pgfusepath{stroke,fill}%
}%
\begin{pgfscope}%
\pgfsys@transformshift{2.621632in}{0.880000in}%
\pgfsys@useobject{currentmarker}{}%
\end{pgfscope}%
\end{pgfscope}%
\begin{pgfscope}%
\definecolor{textcolor}{rgb}{0.000000,0.000000,0.000000}%
\pgfsetstrokecolor{textcolor}%
\pgfsetfillcolor{textcolor}%
\pgftext[x=2.621632in,y=0.782778in,,top]{\color{textcolor}{\sffamily\fontsize{10.000000}{12.000000}\selectfont\catcode`\^=\active\def^{\ifmmode\sp\else\^{}\fi}\catcode`\%=\active\def%{\%}\ensuremath{-}400}}%
\end{pgfscope}%
\begin{pgfscope}%
\pgfsetbuttcap%
\pgfsetroundjoin%
\definecolor{currentfill}{rgb}{0.000000,0.000000,0.000000}%
\pgfsetfillcolor{currentfill}%
\pgfsetlinewidth{0.803000pt}%
\definecolor{currentstroke}{rgb}{0.000000,0.000000,0.000000}%
\pgfsetstrokecolor{currentstroke}%
\pgfsetdash{}{0pt}%
\pgfsys@defobject{currentmarker}{\pgfqpoint{0.000000in}{-0.048611in}}{\pgfqpoint{0.000000in}{0.000000in}}{%
\pgfpathmoveto{\pgfqpoint{0.000000in}{0.000000in}}%
\pgfpathlineto{\pgfqpoint{0.000000in}{-0.048611in}}%
\pgfusepath{stroke,fill}%
}%
\begin{pgfscope}%
\pgfsys@transformshift{3.424513in}{0.880000in}%
\pgfsys@useobject{currentmarker}{}%
\end{pgfscope}%
\end{pgfscope}%
\begin{pgfscope}%
\definecolor{textcolor}{rgb}{0.000000,0.000000,0.000000}%
\pgfsetstrokecolor{textcolor}%
\pgfsetfillcolor{textcolor}%
\pgftext[x=3.424513in,y=0.782778in,,top]{\color{textcolor}{\sffamily\fontsize{10.000000}{12.000000}\selectfont\catcode`\^=\active\def^{\ifmmode\sp\else\^{}\fi}\catcode`\%=\active\def%{\%}\ensuremath{-}200}}%
\end{pgfscope}%
\begin{pgfscope}%
\pgfsetbuttcap%
\pgfsetroundjoin%
\definecolor{currentfill}{rgb}{0.000000,0.000000,0.000000}%
\pgfsetfillcolor{currentfill}%
\pgfsetlinewidth{0.803000pt}%
\definecolor{currentstroke}{rgb}{0.000000,0.000000,0.000000}%
\pgfsetstrokecolor{currentstroke}%
\pgfsetdash{}{0pt}%
\pgfsys@defobject{currentmarker}{\pgfqpoint{0.000000in}{-0.048611in}}{\pgfqpoint{0.000000in}{0.000000in}}{%
\pgfpathmoveto{\pgfqpoint{0.000000in}{0.000000in}}%
\pgfpathlineto{\pgfqpoint{0.000000in}{-0.048611in}}%
\pgfusepath{stroke,fill}%
}%
\begin{pgfscope}%
\pgfsys@transformshift{4.227394in}{0.880000in}%
\pgfsys@useobject{currentmarker}{}%
\end{pgfscope}%
\end{pgfscope}%
\begin{pgfscope}%
\definecolor{textcolor}{rgb}{0.000000,0.000000,0.000000}%
\pgfsetstrokecolor{textcolor}%
\pgfsetfillcolor{textcolor}%
\pgftext[x=4.227394in,y=0.782778in,,top]{\color{textcolor}{\sffamily\fontsize{10.000000}{12.000000}\selectfont\catcode`\^=\active\def^{\ifmmode\sp\else\^{}\fi}\catcode`\%=\active\def%{\%}0}}%
\end{pgfscope}%
\begin{pgfscope}%
\pgfsetbuttcap%
\pgfsetroundjoin%
\definecolor{currentfill}{rgb}{0.000000,0.000000,0.000000}%
\pgfsetfillcolor{currentfill}%
\pgfsetlinewidth{0.803000pt}%
\definecolor{currentstroke}{rgb}{0.000000,0.000000,0.000000}%
\pgfsetstrokecolor{currentstroke}%
\pgfsetdash{}{0pt}%
\pgfsys@defobject{currentmarker}{\pgfqpoint{0.000000in}{-0.048611in}}{\pgfqpoint{0.000000in}{0.000000in}}{%
\pgfpathmoveto{\pgfqpoint{0.000000in}{0.000000in}}%
\pgfpathlineto{\pgfqpoint{0.000000in}{-0.048611in}}%
\pgfusepath{stroke,fill}%
}%
\begin{pgfscope}%
\pgfsys@transformshift{5.030275in}{0.880000in}%
\pgfsys@useobject{currentmarker}{}%
\end{pgfscope}%
\end{pgfscope}%
\begin{pgfscope}%
\definecolor{textcolor}{rgb}{0.000000,0.000000,0.000000}%
\pgfsetstrokecolor{textcolor}%
\pgfsetfillcolor{textcolor}%
\pgftext[x=5.030275in,y=0.782778in,,top]{\color{textcolor}{\sffamily\fontsize{10.000000}{12.000000}\selectfont\catcode`\^=\active\def^{\ifmmode\sp\else\^{}\fi}\catcode`\%=\active\def%{\%}200}}%
\end{pgfscope}%
\begin{pgfscope}%
\pgfsetbuttcap%
\pgfsetroundjoin%
\definecolor{currentfill}{rgb}{0.000000,0.000000,0.000000}%
\pgfsetfillcolor{currentfill}%
\pgfsetlinewidth{0.803000pt}%
\definecolor{currentstroke}{rgb}{0.000000,0.000000,0.000000}%
\pgfsetstrokecolor{currentstroke}%
\pgfsetdash{}{0pt}%
\pgfsys@defobject{currentmarker}{\pgfqpoint{0.000000in}{-0.048611in}}{\pgfqpoint{0.000000in}{0.000000in}}{%
\pgfpathmoveto{\pgfqpoint{0.000000in}{0.000000in}}%
\pgfpathlineto{\pgfqpoint{0.000000in}{-0.048611in}}%
\pgfusepath{stroke,fill}%
}%
\begin{pgfscope}%
\pgfsys@transformshift{5.833156in}{0.880000in}%
\pgfsys@useobject{currentmarker}{}%
\end{pgfscope}%
\end{pgfscope}%
\begin{pgfscope}%
\definecolor{textcolor}{rgb}{0.000000,0.000000,0.000000}%
\pgfsetstrokecolor{textcolor}%
\pgfsetfillcolor{textcolor}%
\pgftext[x=5.833156in,y=0.782778in,,top]{\color{textcolor}{\sffamily\fontsize{10.000000}{12.000000}\selectfont\catcode`\^=\active\def^{\ifmmode\sp\else\^{}\fi}\catcode`\%=\active\def%{\%}400}}%
\end{pgfscope}%
\begin{pgfscope}%
\pgfsetbuttcap%
\pgfsetroundjoin%
\definecolor{currentfill}{rgb}{0.000000,0.000000,0.000000}%
\pgfsetfillcolor{currentfill}%
\pgfsetlinewidth{0.803000pt}%
\definecolor{currentstroke}{rgb}{0.000000,0.000000,0.000000}%
\pgfsetstrokecolor{currentstroke}%
\pgfsetdash{}{0pt}%
\pgfsys@defobject{currentmarker}{\pgfqpoint{0.000000in}{-0.048611in}}{\pgfqpoint{0.000000in}{0.000000in}}{%
\pgfpathmoveto{\pgfqpoint{0.000000in}{0.000000in}}%
\pgfpathlineto{\pgfqpoint{0.000000in}{-0.048611in}}%
\pgfusepath{stroke,fill}%
}%
\begin{pgfscope}%
\pgfsys@transformshift{6.636037in}{0.880000in}%
\pgfsys@useobject{currentmarker}{}%
\end{pgfscope}%
\end{pgfscope}%
\begin{pgfscope}%
\definecolor{textcolor}{rgb}{0.000000,0.000000,0.000000}%
\pgfsetstrokecolor{textcolor}%
\pgfsetfillcolor{textcolor}%
\pgftext[x=6.636037in,y=0.782778in,,top]{\color{textcolor}{\sffamily\fontsize{10.000000}{12.000000}\selectfont\catcode`\^=\active\def^{\ifmmode\sp\else\^{}\fi}\catcode`\%=\active\def%{\%}600}}%
\end{pgfscope}%
\begin{pgfscope}%
\pgfsetbuttcap%
\pgfsetroundjoin%
\definecolor{currentfill}{rgb}{0.000000,0.000000,0.000000}%
\pgfsetfillcolor{currentfill}%
\pgfsetlinewidth{0.803000pt}%
\definecolor{currentstroke}{rgb}{0.000000,0.000000,0.000000}%
\pgfsetstrokecolor{currentstroke}%
\pgfsetdash{}{0pt}%
\pgfsys@defobject{currentmarker}{\pgfqpoint{-0.048611in}{0.000000in}}{\pgfqpoint{-0.000000in}{0.000000in}}{%
\pgfpathmoveto{\pgfqpoint{-0.000000in}{0.000000in}}%
\pgfpathlineto{\pgfqpoint{-0.048611in}{0.000000in}}%
\pgfusepath{stroke,fill}%
}%
\begin{pgfscope}%
\pgfsys@transformshift{1.542338in}{1.607138in}%
\pgfsys@useobject{currentmarker}{}%
\end{pgfscope}%
\end{pgfscope}%
\begin{pgfscope}%
\definecolor{textcolor}{rgb}{0.000000,0.000000,0.000000}%
\pgfsetstrokecolor{textcolor}%
\pgfsetfillcolor{textcolor}%
\pgftext[x=1.071995in, y=1.554376in, left, base]{\color{textcolor}{\sffamily\fontsize{10.000000}{12.000000}\selectfont\catcode`\^=\active\def^{\ifmmode\sp\else\^{}\fi}\catcode`\%=\active\def%{\%}\ensuremath{-}600}}%
\end{pgfscope}%
\begin{pgfscope}%
\pgfsetbuttcap%
\pgfsetroundjoin%
\definecolor{currentfill}{rgb}{0.000000,0.000000,0.000000}%
\pgfsetfillcolor{currentfill}%
\pgfsetlinewidth{0.803000pt}%
\definecolor{currentstroke}{rgb}{0.000000,0.000000,0.000000}%
\pgfsetstrokecolor{currentstroke}%
\pgfsetdash{}{0pt}%
\pgfsys@defobject{currentmarker}{\pgfqpoint{-0.048611in}{0.000000in}}{\pgfqpoint{-0.000000in}{0.000000in}}{%
\pgfpathmoveto{\pgfqpoint{-0.000000in}{0.000000in}}%
\pgfpathlineto{\pgfqpoint{-0.048611in}{0.000000in}}%
\pgfusepath{stroke,fill}%
}%
\begin{pgfscope}%
\pgfsys@transformshift{1.542338in}{2.410019in}%
\pgfsys@useobject{currentmarker}{}%
\end{pgfscope}%
\end{pgfscope}%
\begin{pgfscope}%
\definecolor{textcolor}{rgb}{0.000000,0.000000,0.000000}%
\pgfsetstrokecolor{textcolor}%
\pgfsetfillcolor{textcolor}%
\pgftext[x=1.071995in, y=2.357257in, left, base]{\color{textcolor}{\sffamily\fontsize{10.000000}{12.000000}\selectfont\catcode`\^=\active\def^{\ifmmode\sp\else\^{}\fi}\catcode`\%=\active\def%{\%}\ensuremath{-}400}}%
\end{pgfscope}%
\begin{pgfscope}%
\pgfsetbuttcap%
\pgfsetroundjoin%
\definecolor{currentfill}{rgb}{0.000000,0.000000,0.000000}%
\pgfsetfillcolor{currentfill}%
\pgfsetlinewidth{0.803000pt}%
\definecolor{currentstroke}{rgb}{0.000000,0.000000,0.000000}%
\pgfsetstrokecolor{currentstroke}%
\pgfsetdash{}{0pt}%
\pgfsys@defobject{currentmarker}{\pgfqpoint{-0.048611in}{0.000000in}}{\pgfqpoint{-0.000000in}{0.000000in}}{%
\pgfpathmoveto{\pgfqpoint{-0.000000in}{0.000000in}}%
\pgfpathlineto{\pgfqpoint{-0.048611in}{0.000000in}}%
\pgfusepath{stroke,fill}%
}%
\begin{pgfscope}%
\pgfsys@transformshift{1.542338in}{3.212900in}%
\pgfsys@useobject{currentmarker}{}%
\end{pgfscope}%
\end{pgfscope}%
\begin{pgfscope}%
\definecolor{textcolor}{rgb}{0.000000,0.000000,0.000000}%
\pgfsetstrokecolor{textcolor}%
\pgfsetfillcolor{textcolor}%
\pgftext[x=1.071995in, y=3.160138in, left, base]{\color{textcolor}{\sffamily\fontsize{10.000000}{12.000000}\selectfont\catcode`\^=\active\def^{\ifmmode\sp\else\^{}\fi}\catcode`\%=\active\def%{\%}\ensuremath{-}200}}%
\end{pgfscope}%
\begin{pgfscope}%
\pgfsetbuttcap%
\pgfsetroundjoin%
\definecolor{currentfill}{rgb}{0.000000,0.000000,0.000000}%
\pgfsetfillcolor{currentfill}%
\pgfsetlinewidth{0.803000pt}%
\definecolor{currentstroke}{rgb}{0.000000,0.000000,0.000000}%
\pgfsetstrokecolor{currentstroke}%
\pgfsetdash{}{0pt}%
\pgfsys@defobject{currentmarker}{\pgfqpoint{-0.048611in}{0.000000in}}{\pgfqpoint{-0.000000in}{0.000000in}}{%
\pgfpathmoveto{\pgfqpoint{-0.000000in}{0.000000in}}%
\pgfpathlineto{\pgfqpoint{-0.048611in}{0.000000in}}%
\pgfusepath{stroke,fill}%
}%
\begin{pgfscope}%
\pgfsys@transformshift{1.542338in}{4.015781in}%
\pgfsys@useobject{currentmarker}{}%
\end{pgfscope}%
\end{pgfscope}%
\begin{pgfscope}%
\definecolor{textcolor}{rgb}{0.000000,0.000000,0.000000}%
\pgfsetstrokecolor{textcolor}%
\pgfsetfillcolor{textcolor}%
\pgftext[x=1.356751in, y=3.963019in, left, base]{\color{textcolor}{\sffamily\fontsize{10.000000}{12.000000}\selectfont\catcode`\^=\active\def^{\ifmmode\sp\else\^{}\fi}\catcode`\%=\active\def%{\%}0}}%
\end{pgfscope}%
\begin{pgfscope}%
\pgfsetbuttcap%
\pgfsetroundjoin%
\definecolor{currentfill}{rgb}{0.000000,0.000000,0.000000}%
\pgfsetfillcolor{currentfill}%
\pgfsetlinewidth{0.803000pt}%
\definecolor{currentstroke}{rgb}{0.000000,0.000000,0.000000}%
\pgfsetstrokecolor{currentstroke}%
\pgfsetdash{}{0pt}%
\pgfsys@defobject{currentmarker}{\pgfqpoint{-0.048611in}{0.000000in}}{\pgfqpoint{-0.000000in}{0.000000in}}{%
\pgfpathmoveto{\pgfqpoint{-0.000000in}{0.000000in}}%
\pgfpathlineto{\pgfqpoint{-0.048611in}{0.000000in}}%
\pgfusepath{stroke,fill}%
}%
\begin{pgfscope}%
\pgfsys@transformshift{1.542338in}{4.818662in}%
\pgfsys@useobject{currentmarker}{}%
\end{pgfscope}%
\end{pgfscope}%
\begin{pgfscope}%
\definecolor{textcolor}{rgb}{0.000000,0.000000,0.000000}%
\pgfsetstrokecolor{textcolor}%
\pgfsetfillcolor{textcolor}%
\pgftext[x=1.180020in, y=4.765900in, left, base]{\color{textcolor}{\sffamily\fontsize{10.000000}{12.000000}\selectfont\catcode`\^=\active\def^{\ifmmode\sp\else\^{}\fi}\catcode`\%=\active\def%{\%}200}}%
\end{pgfscope}%
\begin{pgfscope}%
\pgfsetbuttcap%
\pgfsetroundjoin%
\definecolor{currentfill}{rgb}{0.000000,0.000000,0.000000}%
\pgfsetfillcolor{currentfill}%
\pgfsetlinewidth{0.803000pt}%
\definecolor{currentstroke}{rgb}{0.000000,0.000000,0.000000}%
\pgfsetstrokecolor{currentstroke}%
\pgfsetdash{}{0pt}%
\pgfsys@defobject{currentmarker}{\pgfqpoint{-0.048611in}{0.000000in}}{\pgfqpoint{-0.000000in}{0.000000in}}{%
\pgfpathmoveto{\pgfqpoint{-0.000000in}{0.000000in}}%
\pgfpathlineto{\pgfqpoint{-0.048611in}{0.000000in}}%
\pgfusepath{stroke,fill}%
}%
\begin{pgfscope}%
\pgfsys@transformshift{1.542338in}{5.621543in}%
\pgfsys@useobject{currentmarker}{}%
\end{pgfscope}%
\end{pgfscope}%
\begin{pgfscope}%
\definecolor{textcolor}{rgb}{0.000000,0.000000,0.000000}%
\pgfsetstrokecolor{textcolor}%
\pgfsetfillcolor{textcolor}%
\pgftext[x=1.180020in, y=5.568781in, left, base]{\color{textcolor}{\sffamily\fontsize{10.000000}{12.000000}\selectfont\catcode`\^=\active\def^{\ifmmode\sp\else\^{}\fi}\catcode`\%=\active\def%{\%}400}}%
\end{pgfscope}%
\begin{pgfscope}%
\pgfsetbuttcap%
\pgfsetroundjoin%
\definecolor{currentfill}{rgb}{0.000000,0.000000,0.000000}%
\pgfsetfillcolor{currentfill}%
\pgfsetlinewidth{0.803000pt}%
\definecolor{currentstroke}{rgb}{0.000000,0.000000,0.000000}%
\pgfsetstrokecolor{currentstroke}%
\pgfsetdash{}{0pt}%
\pgfsys@defobject{currentmarker}{\pgfqpoint{-0.048611in}{0.000000in}}{\pgfqpoint{-0.000000in}{0.000000in}}{%
\pgfpathmoveto{\pgfqpoint{-0.000000in}{0.000000in}}%
\pgfpathlineto{\pgfqpoint{-0.048611in}{0.000000in}}%
\pgfusepath{stroke,fill}%
}%
\begin{pgfscope}%
\pgfsys@transformshift{1.542338in}{6.424424in}%
\pgfsys@useobject{currentmarker}{}%
\end{pgfscope}%
\end{pgfscope}%
\begin{pgfscope}%
\definecolor{textcolor}{rgb}{0.000000,0.000000,0.000000}%
\pgfsetstrokecolor{textcolor}%
\pgfsetfillcolor{textcolor}%
\pgftext[x=1.180020in, y=6.371662in, left, base]{\color{textcolor}{\sffamily\fontsize{10.000000}{12.000000}\selectfont\catcode`\^=\active\def^{\ifmmode\sp\else\^{}\fi}\catcode`\%=\active\def%{\%}600}}%
\end{pgfscope}%
\begin{pgfscope}%
\pgfsetrectcap%
\pgfsetmiterjoin%
\pgfsetlinewidth{0.803000pt}%
\definecolor{currentstroke}{rgb}{0.000000,0.000000,0.000000}%
\pgfsetstrokecolor{currentstroke}%
\pgfsetdash{}{0pt}%
\pgfpathmoveto{\pgfqpoint{1.542338in}{0.880000in}}%
\pgfpathlineto{\pgfqpoint{1.542338in}{7.040000in}}%
\pgfusepath{stroke}%
\end{pgfscope}%
\begin{pgfscope}%
\pgfsetrectcap%
\pgfsetmiterjoin%
\pgfsetlinewidth{0.803000pt}%
\definecolor{currentstroke}{rgb}{0.000000,0.000000,0.000000}%
\pgfsetstrokecolor{currentstroke}%
\pgfsetdash{}{0pt}%
\pgfpathmoveto{\pgfqpoint{6.657662in}{0.880000in}}%
\pgfpathlineto{\pgfqpoint{6.657662in}{7.040000in}}%
\pgfusepath{stroke}%
\end{pgfscope}%
\begin{pgfscope}%
\pgfsetrectcap%
\pgfsetmiterjoin%
\pgfsetlinewidth{0.803000pt}%
\definecolor{currentstroke}{rgb}{0.000000,0.000000,0.000000}%
\pgfsetstrokecolor{currentstroke}%
\pgfsetdash{}{0pt}%
\pgfpathmoveto{\pgfqpoint{1.542338in}{0.880000in}}%
\pgfpathlineto{\pgfqpoint{6.657662in}{0.880000in}}%
\pgfusepath{stroke}%
\end{pgfscope}%
\begin{pgfscope}%
\pgfsetrectcap%
\pgfsetmiterjoin%
\pgfsetlinewidth{0.803000pt}%
\definecolor{currentstroke}{rgb}{0.000000,0.000000,0.000000}%
\pgfsetstrokecolor{currentstroke}%
\pgfsetdash{}{0pt}%
\pgfpathmoveto{\pgfqpoint{1.542338in}{7.040000in}}%
\pgfpathlineto{\pgfqpoint{6.657662in}{7.040000in}}%
\pgfusepath{stroke}%
\end{pgfscope}%
\end{pgfpicture}%
\makeatother%
\endgroup%
}
    \label{fig:noisy_bg}
    \caption{Example of a dataset with different rings and background noise, and their correct classification (purple means noise).}
\end{figure}
The algorithm takes an additional noise threshold, which is the maximum distance a point can have to a cluster center to be considered noise.
We can express the equation as:
\begin{equation}
    \text{is\_noise}(X_j) = \begin{cases}
        1 & \text{if } \min_{i} d_{ij} > \text{noise\_distance\_threshold} \\
        0 & \text{otherwise}
    \end{cases}
\end{equation}
When computing the centers and radii, points that are considered noise are ignored, that is, all their weights are set to zero.
It is important to note that it is only for the computation of the centers and radii. The stored membership degrees are not modified.
This is done with the use of a mask, where 1 means not noise, and 0 means noise, and then multiplying the mask by the membership degrees and the distances.


\subsection{Overall Steps}
The overall steps of the algorithm are as follows:
\begin{enumerate}
    \item Initialize the parameters $U$, $V$, and $R$.
    \item While iter < max\_iter:
    \begin{enumerate}
        \item Update the membership degrees $U$.
        \item Update the cluster radii $R$ and centers $V$ using the equations \eqref{eq:d_dr} and \eqref{eq:r_i}.
        \item If convergence criterion is met, compute the noise mask, and continue to the next step.
        \item If convergence criterion is met, and the noise mask is not the same as the old one, go to the start of the loop.
        \item Else, break the loop.
    \end{enumerate}
    \item Return the membership degrees $U$, the cluster centers $V$, and the cluster radii $R$.
\end{enumerate}

\subsection{Obtaining the results}
After the algorithm has converged or we have reached the maximum number of iterations, we can obtain the results.
Recall that the membership degree can be seen as 'how much a point belongs to a cluster', and each vector can be seen as a probability distribution.
Having this in mind, there are multiple ways we could obtain the results:
\begin{enumerate}
    \item We can assign each point to the cluster with the highest membership degree.
    \item We can sample from a multinomial distribution with the membership degrees as the probabilities.
    \item We can simply use the membership degrees directly.
\end{enumerate}
We chose the first option, that is, assigning each point to the cluster with the highest membership degree.
As for the radius and the center, obtaining them is a direct result of the algorithm, and we can use them directly.

\subsection{Implementation}
\subsubsection{Data Structures}
We implement the algorithm in Python, using the NumPy \cite{harris2020array} library. The state of the algorithm is stored in three matrices:
\begin{itemize}
    \item $U$: A matrix of size $k \times n$, where $n$ is the number of data samples, and $k$ is the number of clusters. It stores the membership degrees.
    \item $V$: A matrix of size $k \times d$, where $d$ is the number of dimensions of the data samples. It stores the cluster centers.
    \item $R$: A vector of size $k$. It stores the cluster radii.
\end{itemize}
\subsubsection{Pseudocode}
All of the above steps can be formulated as Tensor/NDArray operations, with no need for explicit loops,
by defining the different operations as element-wise operations, tensor multiplications and exploiting
expansion and broadcasting rules as needed. Code is shown at \cite{github}.
This allows to exploit the parallelism of modern computing frameworks, like NumPy, Eigen, or PyTorch.
The algorithm can be implemented with the following pseudocode:
\begin{verbatim}
fn fkr(X, k, q, epsilon):
    U, V, R = initialize(X, k)
    while True:
        U = update_membership(X, V, q)
        R = update_radii(X, U, q)
        V = update_centers(X, U, R, q)
        if converged(U, U_old, epsilon):
            break
    return U, V, R
\end{verbatim}
\begin{verbatim}
fn update_membership(X, V, q):
    D = distance_matrix(X, V)
    U = D ** (-1 / (q - 1))
    U = U / U.sum(axis=1)
    return u
\end{verbatim}

\section{Experiments}
We conducted different experiments to test the performance of the algorithm. Given a dataset, that is, a set of points, we could define, informally,
two types of points, those that belong to a ring, and those that don't belong to any. We call the second one 'background noise' or 'noise' from now on.
In order to generate the dataset, we generate N rings with n noise. The noise of the rings can be seen as 'imperfection' that would occur in a real dataset.
Moreover, we generate $N$ noise points randomly accross the space.
\subsection{Evaluation Metrics}
We use the following metrics to evaluate the performance of the algorithm:
\begin{itemize}
    \item Squared distance error (with hard labels)
    \begin{equation}
        \text{SDE} = \sum_{i=1}^{n} distance(X_i, ring_i)^2
    \end{equation}
    Where $distance(X_i, ring_i)$ is the distance between the point $X_i$ and the circunference of the ring $ring_i$.
    $ring_i$ denotes the classified ring of the point $X_i$.
    Recall that we could classify some points as noise. In the case of a noise point, distance returns 0, that is,
    we do not take noise points into account when computing the SDE.
\end{itemize}
On the other hand, we are also interested in getting the lowest runtime possible. We measure the runtime of the algorithm, as well as the total number of iterations it takes to converge.

\subsection{Results}
We conducted different experiments to test the performance of the algorithm.
\subsubsection{General test with excentric rings}
We performed a general test with excentric rings, with different levels of noise and different numbers of rings.
The hyperparameters were set as follows:

\begin{itemize}
    \item q: 1.1
    \item convergence\_eps: $10^{-5}$
    \item max\_iters: 10000
    \item noise\_distance\_threshold: 100
    \item max\_noise\_checks: 20
    \item apply\_noise\_removal: True
    \item init\_method: "fuzzycmeans"
\end{itemize}

Each circle had 100 samples.
The data was generated in the following way:
\begin{itemize}
    \item For each ring, we randomly select a center in a rect centered at $(0, 0)$ with sides of length 1200.
    \item Each circle had a radius between 100 and 400.
    \item For each ring, we randomly select 100 samples in the circunference. For each sample, we add a noise with the following equation:
    \begin{equation}
        X_{\text{noise}} = X_{\text{ring}} + \text{randn}(0, 1) \cdot \text{noise\_level}
    \end{equation}
    Where $X_{\text{ring}}$ is the point in the circunference, and $\text{noise\_level}$ is the noise level, and $\text{randn}(0, 1)$ is a random number from a normal distribution with mean 0 and variance 1.
    \item To add the background noise, we select N points in the rect, sampled from an uniform distribution.
\end{itemize}

It is noteworthy to say that the algorithm is sensible to the different hyperparametrs.
As we can see in the results, both the runtime and performance degrades with higher noise and higher number of rings.
% tik picture
\begin{figure}[H]
    \centering
    \resizebox{0.9\linewidth}{!}{%% Creator: Matplotlib, PGF backend
%%
%% To include the figure in your LaTeX document, write
%%   \input{<filename>.pgf}
%%
%% Make sure the required packages are loaded in your preamble
%%   \usepackage{pgf}
%%
%% Also ensure that all the required font packages are loaded; for instance,
%% the lmodern package is sometimes necessary when using math font.
%%   \usepackage{lmodern}
%%
%% Figures using additional raster images can only be included by \input if
%% they are in the same directory as the main LaTeX file. For loading figures
%% from other directories you can use the `import` package
%%   \usepackage{import}
%%
%% and then include the figures with
%%   \import{<path to file>}{<filename>.pgf}
%%
%% Matplotlib used the following preamble
%%   \def\mathdefault#1{#1}
%%   \everymath=\expandafter{\the\everymath\displaystyle}
%%   
%%   \usepackage{fontspec}
%%   \setmainfont{DejaVuSerif.ttf}[Path=\detokenize{C:/Users/dagom/anaconda3/envs/pytorch/lib/site-packages/matplotlib/mpl-data/fonts/ttf/}]
%%   \setsansfont{DejaVuSans.ttf}[Path=\detokenize{C:/Users/dagom/anaconda3/envs/pytorch/lib/site-packages/matplotlib/mpl-data/fonts/ttf/}]
%%   \setmonofont{DejaVuSansMono.ttf}[Path=\detokenize{C:/Users/dagom/anaconda3/envs/pytorch/lib/site-packages/matplotlib/mpl-data/fonts/ttf/}]
%%   \makeatletter\@ifpackageloaded{underscore}{}{\usepackage[strings]{underscore}}\makeatother
%%
\begingroup%
\makeatletter%
\begin{pgfpicture}%
\pgfpathrectangle{\pgfpointorigin}{\pgfqpoint{8.000000in}{8.000000in}}%
\pgfusepath{use as bounding box, clip}%
\begin{pgfscope}%
\pgfsetbuttcap%
\pgfsetmiterjoin%
\definecolor{currentfill}{rgb}{1.000000,1.000000,1.000000}%
\pgfsetfillcolor{currentfill}%
\pgfsetlinewidth{0.000000pt}%
\definecolor{currentstroke}{rgb}{1.000000,1.000000,1.000000}%
\pgfsetstrokecolor{currentstroke}%
\pgfsetdash{}{0pt}%
\pgfpathmoveto{\pgfqpoint{0.000000in}{0.000000in}}%
\pgfpathlineto{\pgfqpoint{8.000000in}{0.000000in}}%
\pgfpathlineto{\pgfqpoint{8.000000in}{8.000000in}}%
\pgfpathlineto{\pgfqpoint{0.000000in}{8.000000in}}%
\pgfpathlineto{\pgfqpoint{0.000000in}{0.000000in}}%
\pgfpathclose%
\pgfusepath{fill}%
\end{pgfscope}%
\begin{pgfscope}%
\pgfsetbuttcap%
\pgfsetmiterjoin%
\definecolor{currentfill}{rgb}{1.000000,1.000000,1.000000}%
\pgfsetfillcolor{currentfill}%
\pgfsetlinewidth{0.000000pt}%
\definecolor{currentstroke}{rgb}{0.000000,0.000000,0.000000}%
\pgfsetstrokecolor{currentstroke}%
\pgfsetstrokeopacity{0.000000}%
\pgfsetdash{}{0pt}%
\pgfpathmoveto{\pgfqpoint{1.000000in}{1.148311in}}%
\pgfpathlineto{\pgfqpoint{7.200000in}{1.148311in}}%
\pgfpathlineto{\pgfqpoint{7.200000in}{6.771689in}}%
\pgfpathlineto{\pgfqpoint{1.000000in}{6.771689in}}%
\pgfpathlineto{\pgfqpoint{1.000000in}{1.148311in}}%
\pgfpathclose%
\pgfusepath{fill}%
\end{pgfscope}%
\begin{pgfscope}%
\pgfpathrectangle{\pgfqpoint{1.000000in}{1.148311in}}{\pgfqpoint{6.200000in}{5.623377in}}%
\pgfusepath{clip}%
\pgfsetbuttcap%
\pgfsetroundjoin%
\definecolor{currentfill}{rgb}{0.200000,0.800000,0.200000}%
\pgfsetfillcolor{currentfill}%
\pgfsetlinewidth{1.003750pt}%
\definecolor{currentstroke}{rgb}{0.200000,0.800000,0.200000}%
\pgfsetstrokecolor{currentstroke}%
\pgfsetdash{}{0pt}%
\pgfpathmoveto{\pgfqpoint{4.444253in}{2.278586in}}%
\pgfpathcurveto{\pgfqpoint{4.450077in}{2.278586in}}{\pgfqpoint{4.455663in}{2.280900in}}{\pgfqpoint{4.459781in}{2.285018in}}%
\pgfpathcurveto{\pgfqpoint{4.463900in}{2.289136in}}{\pgfqpoint{4.466213in}{2.294722in}}{\pgfqpoint{4.466213in}{2.300546in}}%
\pgfpathcurveto{\pgfqpoint{4.466213in}{2.306370in}}{\pgfqpoint{4.463900in}{2.311956in}}{\pgfqpoint{4.459781in}{2.316074in}}%
\pgfpathcurveto{\pgfqpoint{4.455663in}{2.320193in}}{\pgfqpoint{4.450077in}{2.322506in}}{\pgfqpoint{4.444253in}{2.322506in}}%
\pgfpathcurveto{\pgfqpoint{4.438429in}{2.322506in}}{\pgfqpoint{4.432843in}{2.320193in}}{\pgfqpoint{4.428725in}{2.316074in}}%
\pgfpathcurveto{\pgfqpoint{4.424607in}{2.311956in}}{\pgfqpoint{4.422293in}{2.306370in}}{\pgfqpoint{4.422293in}{2.300546in}}%
\pgfpathcurveto{\pgfqpoint{4.422293in}{2.294722in}}{\pgfqpoint{4.424607in}{2.289136in}}{\pgfqpoint{4.428725in}{2.285018in}}%
\pgfpathcurveto{\pgfqpoint{4.432843in}{2.280900in}}{\pgfqpoint{4.438429in}{2.278586in}}{\pgfqpoint{4.444253in}{2.278586in}}%
\pgfpathlineto{\pgfqpoint{4.444253in}{2.278586in}}%
\pgfpathclose%
\pgfusepath{stroke,fill}%
\end{pgfscope}%
\begin{pgfscope}%
\pgfpathrectangle{\pgfqpoint{1.000000in}{1.148311in}}{\pgfqpoint{6.200000in}{5.623377in}}%
\pgfusepath{clip}%
\pgfsetbuttcap%
\pgfsetroundjoin%
\definecolor{currentfill}{rgb}{0.200000,0.800000,0.200000}%
\pgfsetfillcolor{currentfill}%
\pgfsetlinewidth{1.003750pt}%
\definecolor{currentstroke}{rgb}{0.200000,0.800000,0.200000}%
\pgfsetstrokecolor{currentstroke}%
\pgfsetdash{}{0pt}%
\pgfpathmoveto{\pgfqpoint{4.469389in}{2.308715in}}%
\pgfpathcurveto{\pgfqpoint{4.475213in}{2.308715in}}{\pgfqpoint{4.480799in}{2.311028in}}{\pgfqpoint{4.484917in}{2.315147in}}%
\pgfpathcurveto{\pgfqpoint{4.489035in}{2.319265in}}{\pgfqpoint{4.491349in}{2.324851in}}{\pgfqpoint{4.491349in}{2.330675in}}%
\pgfpathcurveto{\pgfqpoint{4.491349in}{2.336499in}}{\pgfqpoint{4.489035in}{2.342085in}}{\pgfqpoint{4.484917in}{2.346203in}}%
\pgfpathcurveto{\pgfqpoint{4.480799in}{2.350321in}}{\pgfqpoint{4.475213in}{2.352635in}}{\pgfqpoint{4.469389in}{2.352635in}}%
\pgfpathcurveto{\pgfqpoint{4.463565in}{2.352635in}}{\pgfqpoint{4.457979in}{2.350321in}}{\pgfqpoint{4.453861in}{2.346203in}}%
\pgfpathcurveto{\pgfqpoint{4.449743in}{2.342085in}}{\pgfqpoint{4.447429in}{2.336499in}}{\pgfqpoint{4.447429in}{2.330675in}}%
\pgfpathcurveto{\pgfqpoint{4.447429in}{2.324851in}}{\pgfqpoint{4.449743in}{2.319265in}}{\pgfqpoint{4.453861in}{2.315147in}}%
\pgfpathcurveto{\pgfqpoint{4.457979in}{2.311028in}}{\pgfqpoint{4.463565in}{2.308715in}}{\pgfqpoint{4.469389in}{2.308715in}}%
\pgfpathlineto{\pgfqpoint{4.469389in}{2.308715in}}%
\pgfpathclose%
\pgfusepath{stroke,fill}%
\end{pgfscope}%
\begin{pgfscope}%
\pgfpathrectangle{\pgfqpoint{1.000000in}{1.148311in}}{\pgfqpoint{6.200000in}{5.623377in}}%
\pgfusepath{clip}%
\pgfsetbuttcap%
\pgfsetroundjoin%
\definecolor{currentfill}{rgb}{0.200000,0.800000,0.200000}%
\pgfsetfillcolor{currentfill}%
\pgfsetlinewidth{1.003750pt}%
\definecolor{currentstroke}{rgb}{0.200000,0.800000,0.200000}%
\pgfsetstrokecolor{currentstroke}%
\pgfsetdash{}{0pt}%
\pgfpathmoveto{\pgfqpoint{4.427703in}{2.333768in}}%
\pgfpathcurveto{\pgfqpoint{4.433527in}{2.333768in}}{\pgfqpoint{4.439113in}{2.336082in}}{\pgfqpoint{4.443232in}{2.340200in}}%
\pgfpathcurveto{\pgfqpoint{4.447350in}{2.344318in}}{\pgfqpoint{4.449664in}{2.349904in}}{\pgfqpoint{4.449664in}{2.355728in}}%
\pgfpathcurveto{\pgfqpoint{4.449664in}{2.361552in}}{\pgfqpoint{4.447350in}{2.367138in}}{\pgfqpoint{4.443232in}{2.371256in}}%
\pgfpathcurveto{\pgfqpoint{4.439113in}{2.375374in}}{\pgfqpoint{4.433527in}{2.377688in}}{\pgfqpoint{4.427703in}{2.377688in}}%
\pgfpathcurveto{\pgfqpoint{4.421879in}{2.377688in}}{\pgfqpoint{4.416293in}{2.375374in}}{\pgfqpoint{4.412175in}{2.371256in}}%
\pgfpathcurveto{\pgfqpoint{4.408057in}{2.367138in}}{\pgfqpoint{4.405743in}{2.361552in}}{\pgfqpoint{4.405743in}{2.355728in}}%
\pgfpathcurveto{\pgfqpoint{4.405743in}{2.349904in}}{\pgfqpoint{4.408057in}{2.344318in}}{\pgfqpoint{4.412175in}{2.340200in}}%
\pgfpathcurveto{\pgfqpoint{4.416293in}{2.336082in}}{\pgfqpoint{4.421879in}{2.333768in}}{\pgfqpoint{4.427703in}{2.333768in}}%
\pgfpathlineto{\pgfqpoint{4.427703in}{2.333768in}}%
\pgfpathclose%
\pgfusepath{stroke,fill}%
\end{pgfscope}%
\begin{pgfscope}%
\pgfpathrectangle{\pgfqpoint{1.000000in}{1.148311in}}{\pgfqpoint{6.200000in}{5.623377in}}%
\pgfusepath{clip}%
\pgfsetbuttcap%
\pgfsetroundjoin%
\definecolor{currentfill}{rgb}{0.200000,0.800000,0.200000}%
\pgfsetfillcolor{currentfill}%
\pgfsetlinewidth{1.003750pt}%
\definecolor{currentstroke}{rgb}{0.200000,0.800000,0.200000}%
\pgfsetstrokecolor{currentstroke}%
\pgfsetdash{}{0pt}%
\pgfpathmoveto{\pgfqpoint{4.449479in}{2.366120in}}%
\pgfpathcurveto{\pgfqpoint{4.455303in}{2.366120in}}{\pgfqpoint{4.460889in}{2.368434in}}{\pgfqpoint{4.465007in}{2.372552in}}%
\pgfpathcurveto{\pgfqpoint{4.469125in}{2.376670in}}{\pgfqpoint{4.471439in}{2.382256in}}{\pgfqpoint{4.471439in}{2.388080in}}%
\pgfpathcurveto{\pgfqpoint{4.471439in}{2.393904in}}{\pgfqpoint{4.469125in}{2.399490in}}{\pgfqpoint{4.465007in}{2.403609in}}%
\pgfpathcurveto{\pgfqpoint{4.460889in}{2.407727in}}{\pgfqpoint{4.455303in}{2.410041in}}{\pgfqpoint{4.449479in}{2.410041in}}%
\pgfpathcurveto{\pgfqpoint{4.443655in}{2.410041in}}{\pgfqpoint{4.438069in}{2.407727in}}{\pgfqpoint{4.433951in}{2.403609in}}%
\pgfpathcurveto{\pgfqpoint{4.429833in}{2.399490in}}{\pgfqpoint{4.427519in}{2.393904in}}{\pgfqpoint{4.427519in}{2.388080in}}%
\pgfpathcurveto{\pgfqpoint{4.427519in}{2.382256in}}{\pgfqpoint{4.429833in}{2.376670in}}{\pgfqpoint{4.433951in}{2.372552in}}%
\pgfpathcurveto{\pgfqpoint{4.438069in}{2.368434in}}{\pgfqpoint{4.443655in}{2.366120in}}{\pgfqpoint{4.449479in}{2.366120in}}%
\pgfpathlineto{\pgfqpoint{4.449479in}{2.366120in}}%
\pgfpathclose%
\pgfusepath{stroke,fill}%
\end{pgfscope}%
\begin{pgfscope}%
\pgfpathrectangle{\pgfqpoint{1.000000in}{1.148311in}}{\pgfqpoint{6.200000in}{5.623377in}}%
\pgfusepath{clip}%
\pgfsetbuttcap%
\pgfsetroundjoin%
\definecolor{currentfill}{rgb}{0.200000,0.800000,0.200000}%
\pgfsetfillcolor{currentfill}%
\pgfsetlinewidth{1.003750pt}%
\definecolor{currentstroke}{rgb}{0.200000,0.800000,0.200000}%
\pgfsetstrokecolor{currentstroke}%
\pgfsetdash{}{0pt}%
\pgfpathmoveto{\pgfqpoint{4.484687in}{2.405562in}}%
\pgfpathcurveto{\pgfqpoint{4.490511in}{2.405562in}}{\pgfqpoint{4.496097in}{2.407876in}}{\pgfqpoint{4.500215in}{2.411994in}}%
\pgfpathcurveto{\pgfqpoint{4.504333in}{2.416112in}}{\pgfqpoint{4.506647in}{2.421698in}}{\pgfqpoint{4.506647in}{2.427522in}}%
\pgfpathcurveto{\pgfqpoint{4.506647in}{2.433346in}}{\pgfqpoint{4.504333in}{2.438932in}}{\pgfqpoint{4.500215in}{2.443051in}}%
\pgfpathcurveto{\pgfqpoint{4.496097in}{2.447169in}}{\pgfqpoint{4.490511in}{2.449483in}}{\pgfqpoint{4.484687in}{2.449483in}}%
\pgfpathcurveto{\pgfqpoint{4.478863in}{2.449483in}}{\pgfqpoint{4.473277in}{2.447169in}}{\pgfqpoint{4.469159in}{2.443051in}}%
\pgfpathcurveto{\pgfqpoint{4.465040in}{2.438932in}}{\pgfqpoint{4.462727in}{2.433346in}}{\pgfqpoint{4.462727in}{2.427522in}}%
\pgfpathcurveto{\pgfqpoint{4.462727in}{2.421698in}}{\pgfqpoint{4.465040in}{2.416112in}}{\pgfqpoint{4.469159in}{2.411994in}}%
\pgfpathcurveto{\pgfqpoint{4.473277in}{2.407876in}}{\pgfqpoint{4.478863in}{2.405562in}}{\pgfqpoint{4.484687in}{2.405562in}}%
\pgfpathlineto{\pgfqpoint{4.484687in}{2.405562in}}%
\pgfpathclose%
\pgfusepath{stroke,fill}%
\end{pgfscope}%
\begin{pgfscope}%
\pgfpathrectangle{\pgfqpoint{1.000000in}{1.148311in}}{\pgfqpoint{6.200000in}{5.623377in}}%
\pgfusepath{clip}%
\pgfsetbuttcap%
\pgfsetroundjoin%
\definecolor{currentfill}{rgb}{0.200000,0.800000,0.200000}%
\pgfsetfillcolor{currentfill}%
\pgfsetlinewidth{1.003750pt}%
\definecolor{currentstroke}{rgb}{0.200000,0.800000,0.200000}%
\pgfsetstrokecolor{currentstroke}%
\pgfsetdash{}{0pt}%
\pgfpathmoveto{\pgfqpoint{4.424601in}{2.419579in}}%
\pgfpathcurveto{\pgfqpoint{4.430424in}{2.419579in}}{\pgfqpoint{4.436011in}{2.421893in}}{\pgfqpoint{4.440129in}{2.426011in}}%
\pgfpathcurveto{\pgfqpoint{4.444247in}{2.430129in}}{\pgfqpoint{4.446561in}{2.435715in}}{\pgfqpoint{4.446561in}{2.441539in}}%
\pgfpathcurveto{\pgfqpoint{4.446561in}{2.447363in}}{\pgfqpoint{4.444247in}{2.452949in}}{\pgfqpoint{4.440129in}{2.457068in}}%
\pgfpathcurveto{\pgfqpoint{4.436011in}{2.461186in}}{\pgfqpoint{4.430424in}{2.463500in}}{\pgfqpoint{4.424601in}{2.463500in}}%
\pgfpathcurveto{\pgfqpoint{4.418777in}{2.463500in}}{\pgfqpoint{4.413190in}{2.461186in}}{\pgfqpoint{4.409072in}{2.457068in}}%
\pgfpathcurveto{\pgfqpoint{4.404954in}{2.452949in}}{\pgfqpoint{4.402640in}{2.447363in}}{\pgfqpoint{4.402640in}{2.441539in}}%
\pgfpathcurveto{\pgfqpoint{4.402640in}{2.435715in}}{\pgfqpoint{4.404954in}{2.430129in}}{\pgfqpoint{4.409072in}{2.426011in}}%
\pgfpathcurveto{\pgfqpoint{4.413190in}{2.421893in}}{\pgfqpoint{4.418777in}{2.419579in}}{\pgfqpoint{4.424601in}{2.419579in}}%
\pgfpathlineto{\pgfqpoint{4.424601in}{2.419579in}}%
\pgfpathclose%
\pgfusepath{stroke,fill}%
\end{pgfscope}%
\begin{pgfscope}%
\pgfpathrectangle{\pgfqpoint{1.000000in}{1.148311in}}{\pgfqpoint{6.200000in}{5.623377in}}%
\pgfusepath{clip}%
\pgfsetbuttcap%
\pgfsetroundjoin%
\definecolor{currentfill}{rgb}{0.200000,0.800000,0.200000}%
\pgfsetfillcolor{currentfill}%
\pgfsetlinewidth{1.003750pt}%
\definecolor{currentstroke}{rgb}{0.200000,0.800000,0.200000}%
\pgfsetstrokecolor{currentstroke}%
\pgfsetdash{}{0pt}%
\pgfpathmoveto{\pgfqpoint{4.388606in}{2.436039in}}%
\pgfpathcurveto{\pgfqpoint{4.394430in}{2.436039in}}{\pgfqpoint{4.400017in}{2.438353in}}{\pgfqpoint{4.404135in}{2.442471in}}%
\pgfpathcurveto{\pgfqpoint{4.408253in}{2.446589in}}{\pgfqpoint{4.410567in}{2.452175in}}{\pgfqpoint{4.410567in}{2.457999in}}%
\pgfpathcurveto{\pgfqpoint{4.410567in}{2.463823in}}{\pgfqpoint{4.408253in}{2.469409in}}{\pgfqpoint{4.404135in}{2.473527in}}%
\pgfpathcurveto{\pgfqpoint{4.400017in}{2.477645in}}{\pgfqpoint{4.394430in}{2.479959in}}{\pgfqpoint{4.388606in}{2.479959in}}%
\pgfpathcurveto{\pgfqpoint{4.382782in}{2.479959in}}{\pgfqpoint{4.377196in}{2.477645in}}{\pgfqpoint{4.373078in}{2.473527in}}%
\pgfpathcurveto{\pgfqpoint{4.368960in}{2.469409in}}{\pgfqpoint{4.366646in}{2.463823in}}{\pgfqpoint{4.366646in}{2.457999in}}%
\pgfpathcurveto{\pgfqpoint{4.366646in}{2.452175in}}{\pgfqpoint{4.368960in}{2.446589in}}{\pgfqpoint{4.373078in}{2.442471in}}%
\pgfpathcurveto{\pgfqpoint{4.377196in}{2.438353in}}{\pgfqpoint{4.382782in}{2.436039in}}{\pgfqpoint{4.388606in}{2.436039in}}%
\pgfpathlineto{\pgfqpoint{4.388606in}{2.436039in}}%
\pgfpathclose%
\pgfusepath{stroke,fill}%
\end{pgfscope}%
\begin{pgfscope}%
\pgfpathrectangle{\pgfqpoint{1.000000in}{1.148311in}}{\pgfqpoint{6.200000in}{5.623377in}}%
\pgfusepath{clip}%
\pgfsetbuttcap%
\pgfsetroundjoin%
\definecolor{currentfill}{rgb}{0.200000,0.800000,0.200000}%
\pgfsetfillcolor{currentfill}%
\pgfsetlinewidth{1.003750pt}%
\definecolor{currentstroke}{rgb}{0.200000,0.800000,0.200000}%
\pgfsetstrokecolor{currentstroke}%
\pgfsetdash{}{0pt}%
\pgfpathmoveto{\pgfqpoint{4.374527in}{2.459095in}}%
\pgfpathcurveto{\pgfqpoint{4.380351in}{2.459095in}}{\pgfqpoint{4.385938in}{2.461409in}}{\pgfqpoint{4.390056in}{2.465527in}}%
\pgfpathcurveto{\pgfqpoint{4.394174in}{2.469645in}}{\pgfqpoint{4.396488in}{2.475231in}}{\pgfqpoint{4.396488in}{2.481055in}}%
\pgfpathcurveto{\pgfqpoint{4.396488in}{2.486879in}}{\pgfqpoint{4.394174in}{2.492465in}}{\pgfqpoint{4.390056in}{2.496583in}}%
\pgfpathcurveto{\pgfqpoint{4.385938in}{2.500702in}}{\pgfqpoint{4.380351in}{2.503015in}}{\pgfqpoint{4.374527in}{2.503015in}}%
\pgfpathcurveto{\pgfqpoint{4.368703in}{2.503015in}}{\pgfqpoint{4.363117in}{2.500702in}}{\pgfqpoint{4.358999in}{2.496583in}}%
\pgfpathcurveto{\pgfqpoint{4.354881in}{2.492465in}}{\pgfqpoint{4.352567in}{2.486879in}}{\pgfqpoint{4.352567in}{2.481055in}}%
\pgfpathcurveto{\pgfqpoint{4.352567in}{2.475231in}}{\pgfqpoint{4.354881in}{2.469645in}}{\pgfqpoint{4.358999in}{2.465527in}}%
\pgfpathcurveto{\pgfqpoint{4.363117in}{2.461409in}}{\pgfqpoint{4.368703in}{2.459095in}}{\pgfqpoint{4.374527in}{2.459095in}}%
\pgfpathlineto{\pgfqpoint{4.374527in}{2.459095in}}%
\pgfpathclose%
\pgfusepath{stroke,fill}%
\end{pgfscope}%
\begin{pgfscope}%
\pgfpathrectangle{\pgfqpoint{1.000000in}{1.148311in}}{\pgfqpoint{6.200000in}{5.623377in}}%
\pgfusepath{clip}%
\pgfsetbuttcap%
\pgfsetroundjoin%
\definecolor{currentfill}{rgb}{0.200000,0.800000,0.200000}%
\pgfsetfillcolor{currentfill}%
\pgfsetlinewidth{1.003750pt}%
\definecolor{currentstroke}{rgb}{0.200000,0.800000,0.200000}%
\pgfsetstrokecolor{currentstroke}%
\pgfsetdash{}{0pt}%
\pgfpathmoveto{\pgfqpoint{4.398304in}{2.502807in}}%
\pgfpathcurveto{\pgfqpoint{4.404128in}{2.502807in}}{\pgfqpoint{4.409714in}{2.505121in}}{\pgfqpoint{4.413832in}{2.509239in}}%
\pgfpathcurveto{\pgfqpoint{4.417950in}{2.513357in}}{\pgfqpoint{4.420264in}{2.518943in}}{\pgfqpoint{4.420264in}{2.524767in}}%
\pgfpathcurveto{\pgfqpoint{4.420264in}{2.530591in}}{\pgfqpoint{4.417950in}{2.536177in}}{\pgfqpoint{4.413832in}{2.540295in}}%
\pgfpathcurveto{\pgfqpoint{4.409714in}{2.544413in}}{\pgfqpoint{4.404128in}{2.546727in}}{\pgfqpoint{4.398304in}{2.546727in}}%
\pgfpathcurveto{\pgfqpoint{4.392480in}{2.546727in}}{\pgfqpoint{4.386894in}{2.544413in}}{\pgfqpoint{4.382776in}{2.540295in}}%
\pgfpathcurveto{\pgfqpoint{4.378658in}{2.536177in}}{\pgfqpoint{4.376344in}{2.530591in}}{\pgfqpoint{4.376344in}{2.524767in}}%
\pgfpathcurveto{\pgfqpoint{4.376344in}{2.518943in}}{\pgfqpoint{4.378658in}{2.513357in}}{\pgfqpoint{4.382776in}{2.509239in}}%
\pgfpathcurveto{\pgfqpoint{4.386894in}{2.505121in}}{\pgfqpoint{4.392480in}{2.502807in}}{\pgfqpoint{4.398304in}{2.502807in}}%
\pgfpathlineto{\pgfqpoint{4.398304in}{2.502807in}}%
\pgfpathclose%
\pgfusepath{stroke,fill}%
\end{pgfscope}%
\begin{pgfscope}%
\pgfpathrectangle{\pgfqpoint{1.000000in}{1.148311in}}{\pgfqpoint{6.200000in}{5.623377in}}%
\pgfusepath{clip}%
\pgfsetbuttcap%
\pgfsetroundjoin%
\definecolor{currentfill}{rgb}{0.200000,0.800000,0.200000}%
\pgfsetfillcolor{currentfill}%
\pgfsetlinewidth{1.003750pt}%
\definecolor{currentstroke}{rgb}{0.200000,0.800000,0.200000}%
\pgfsetstrokecolor{currentstroke}%
\pgfsetdash{}{0pt}%
\pgfpathmoveto{\pgfqpoint{4.353931in}{2.509058in}}%
\pgfpathcurveto{\pgfqpoint{4.359755in}{2.509058in}}{\pgfqpoint{4.365341in}{2.511372in}}{\pgfqpoint{4.369459in}{2.515491in}}%
\pgfpathcurveto{\pgfqpoint{4.373577in}{2.519609in}}{\pgfqpoint{4.375891in}{2.525195in}}{\pgfqpoint{4.375891in}{2.531019in}}%
\pgfpathcurveto{\pgfqpoint{4.375891in}{2.536843in}}{\pgfqpoint{4.373577in}{2.542429in}}{\pgfqpoint{4.369459in}{2.546547in}}%
\pgfpathcurveto{\pgfqpoint{4.365341in}{2.550665in}}{\pgfqpoint{4.359755in}{2.552979in}}{\pgfqpoint{4.353931in}{2.552979in}}%
\pgfpathcurveto{\pgfqpoint{4.348107in}{2.552979in}}{\pgfqpoint{4.342521in}{2.550665in}}{\pgfqpoint{4.338402in}{2.546547in}}%
\pgfpathcurveto{\pgfqpoint{4.334284in}{2.542429in}}{\pgfqpoint{4.331970in}{2.536843in}}{\pgfqpoint{4.331970in}{2.531019in}}%
\pgfpathcurveto{\pgfqpoint{4.331970in}{2.525195in}}{\pgfqpoint{4.334284in}{2.519609in}}{\pgfqpoint{4.338402in}{2.515491in}}%
\pgfpathcurveto{\pgfqpoint{4.342521in}{2.511372in}}{\pgfqpoint{4.348107in}{2.509058in}}{\pgfqpoint{4.353931in}{2.509058in}}%
\pgfpathlineto{\pgfqpoint{4.353931in}{2.509058in}}%
\pgfpathclose%
\pgfusepath{stroke,fill}%
\end{pgfscope}%
\begin{pgfscope}%
\pgfpathrectangle{\pgfqpoint{1.000000in}{1.148311in}}{\pgfqpoint{6.200000in}{5.623377in}}%
\pgfusepath{clip}%
\pgfsetbuttcap%
\pgfsetroundjoin%
\definecolor{currentfill}{rgb}{0.200000,0.800000,0.200000}%
\pgfsetfillcolor{currentfill}%
\pgfsetlinewidth{1.003750pt}%
\definecolor{currentstroke}{rgb}{0.200000,0.800000,0.200000}%
\pgfsetstrokecolor{currentstroke}%
\pgfsetdash{}{0pt}%
\pgfpathmoveto{\pgfqpoint{4.318664in}{2.516668in}}%
\pgfpathcurveto{\pgfqpoint{4.324488in}{2.516668in}}{\pgfqpoint{4.330074in}{2.518982in}}{\pgfqpoint{4.334193in}{2.523100in}}%
\pgfpathcurveto{\pgfqpoint{4.338311in}{2.527218in}}{\pgfqpoint{4.340625in}{2.532804in}}{\pgfqpoint{4.340625in}{2.538628in}}%
\pgfpathcurveto{\pgfqpoint{4.340625in}{2.544452in}}{\pgfqpoint{4.338311in}{2.550038in}}{\pgfqpoint{4.334193in}{2.554156in}}%
\pgfpathcurveto{\pgfqpoint{4.330074in}{2.558275in}}{\pgfqpoint{4.324488in}{2.560588in}}{\pgfqpoint{4.318664in}{2.560588in}}%
\pgfpathcurveto{\pgfqpoint{4.312840in}{2.560588in}}{\pgfqpoint{4.307254in}{2.558275in}}{\pgfqpoint{4.303136in}{2.554156in}}%
\pgfpathcurveto{\pgfqpoint{4.299018in}{2.550038in}}{\pgfqpoint{4.296704in}{2.544452in}}{\pgfqpoint{4.296704in}{2.538628in}}%
\pgfpathcurveto{\pgfqpoint{4.296704in}{2.532804in}}{\pgfqpoint{4.299018in}{2.527218in}}{\pgfqpoint{4.303136in}{2.523100in}}%
\pgfpathcurveto{\pgfqpoint{4.307254in}{2.518982in}}{\pgfqpoint{4.312840in}{2.516668in}}{\pgfqpoint{4.318664in}{2.516668in}}%
\pgfpathlineto{\pgfqpoint{4.318664in}{2.516668in}}%
\pgfpathclose%
\pgfusepath{stroke,fill}%
\end{pgfscope}%
\begin{pgfscope}%
\pgfpathrectangle{\pgfqpoint{1.000000in}{1.148311in}}{\pgfqpoint{6.200000in}{5.623377in}}%
\pgfusepath{clip}%
\pgfsetbuttcap%
\pgfsetroundjoin%
\definecolor{currentfill}{rgb}{0.200000,0.800000,0.200000}%
\pgfsetfillcolor{currentfill}%
\pgfsetlinewidth{1.003750pt}%
\definecolor{currentstroke}{rgb}{0.200000,0.800000,0.200000}%
\pgfsetstrokecolor{currentstroke}%
\pgfsetdash{}{0pt}%
\pgfpathmoveto{\pgfqpoint{4.310013in}{2.542654in}}%
\pgfpathcurveto{\pgfqpoint{4.315837in}{2.542654in}}{\pgfqpoint{4.321423in}{2.544968in}}{\pgfqpoint{4.325541in}{2.549086in}}%
\pgfpathcurveto{\pgfqpoint{4.329659in}{2.553205in}}{\pgfqpoint{4.331973in}{2.558791in}}{\pgfqpoint{4.331973in}{2.564615in}}%
\pgfpathcurveto{\pgfqpoint{4.331973in}{2.570439in}}{\pgfqpoint{4.329659in}{2.576025in}}{\pgfqpoint{4.325541in}{2.580143in}}%
\pgfpathcurveto{\pgfqpoint{4.321423in}{2.584261in}}{\pgfqpoint{4.315837in}{2.586575in}}{\pgfqpoint{4.310013in}{2.586575in}}%
\pgfpathcurveto{\pgfqpoint{4.304189in}{2.586575in}}{\pgfqpoint{4.298603in}{2.584261in}}{\pgfqpoint{4.294485in}{2.580143in}}%
\pgfpathcurveto{\pgfqpoint{4.290367in}{2.576025in}}{\pgfqpoint{4.288053in}{2.570439in}}{\pgfqpoint{4.288053in}{2.564615in}}%
\pgfpathcurveto{\pgfqpoint{4.288053in}{2.558791in}}{\pgfqpoint{4.290367in}{2.553205in}}{\pgfqpoint{4.294485in}{2.549086in}}%
\pgfpathcurveto{\pgfqpoint{4.298603in}{2.544968in}}{\pgfqpoint{4.304189in}{2.542654in}}{\pgfqpoint{4.310013in}{2.542654in}}%
\pgfpathlineto{\pgfqpoint{4.310013in}{2.542654in}}%
\pgfpathclose%
\pgfusepath{stroke,fill}%
\end{pgfscope}%
\begin{pgfscope}%
\pgfpathrectangle{\pgfqpoint{1.000000in}{1.148311in}}{\pgfqpoint{6.200000in}{5.623377in}}%
\pgfusepath{clip}%
\pgfsetbuttcap%
\pgfsetroundjoin%
\definecolor{currentfill}{rgb}{0.200000,0.800000,0.200000}%
\pgfsetfillcolor{currentfill}%
\pgfsetlinewidth{1.003750pt}%
\definecolor{currentstroke}{rgb}{0.200000,0.800000,0.200000}%
\pgfsetstrokecolor{currentstroke}%
\pgfsetdash{}{0pt}%
\pgfpathmoveto{\pgfqpoint{4.360628in}{2.626918in}}%
\pgfpathcurveto{\pgfqpoint{4.366452in}{2.626918in}}{\pgfqpoint{4.372038in}{2.629231in}}{\pgfqpoint{4.376156in}{2.633350in}}%
\pgfpathcurveto{\pgfqpoint{4.380274in}{2.637468in}}{\pgfqpoint{4.382588in}{2.643054in}}{\pgfqpoint{4.382588in}{2.648878in}}%
\pgfpathcurveto{\pgfqpoint{4.382588in}{2.654702in}}{\pgfqpoint{4.380274in}{2.660288in}}{\pgfqpoint{4.376156in}{2.664406in}}%
\pgfpathcurveto{\pgfqpoint{4.372038in}{2.668524in}}{\pgfqpoint{4.366452in}{2.670838in}}{\pgfqpoint{4.360628in}{2.670838in}}%
\pgfpathcurveto{\pgfqpoint{4.354804in}{2.670838in}}{\pgfqpoint{4.349218in}{2.668524in}}{\pgfqpoint{4.345100in}{2.664406in}}%
\pgfpathcurveto{\pgfqpoint{4.340982in}{2.660288in}}{\pgfqpoint{4.338668in}{2.654702in}}{\pgfqpoint{4.338668in}{2.648878in}}%
\pgfpathcurveto{\pgfqpoint{4.338668in}{2.643054in}}{\pgfqpoint{4.340982in}{2.637468in}}{\pgfqpoint{4.345100in}{2.633350in}}%
\pgfpathcurveto{\pgfqpoint{4.349218in}{2.629231in}}{\pgfqpoint{4.354804in}{2.626918in}}{\pgfqpoint{4.360628in}{2.626918in}}%
\pgfpathlineto{\pgfqpoint{4.360628in}{2.626918in}}%
\pgfpathclose%
\pgfusepath{stroke,fill}%
\end{pgfscope}%
\begin{pgfscope}%
\pgfpathrectangle{\pgfqpoint{1.000000in}{1.148311in}}{\pgfqpoint{6.200000in}{5.623377in}}%
\pgfusepath{clip}%
\pgfsetbuttcap%
\pgfsetroundjoin%
\definecolor{currentfill}{rgb}{0.200000,0.800000,0.200000}%
\pgfsetfillcolor{currentfill}%
\pgfsetlinewidth{1.003750pt}%
\definecolor{currentstroke}{rgb}{0.200000,0.800000,0.200000}%
\pgfsetstrokecolor{currentstroke}%
\pgfsetdash{}{0pt}%
\pgfpathmoveto{\pgfqpoint{4.302418in}{2.611080in}}%
\pgfpathcurveto{\pgfqpoint{4.308242in}{2.611080in}}{\pgfqpoint{4.313828in}{2.613393in}}{\pgfqpoint{4.317946in}{2.617512in}}%
\pgfpathcurveto{\pgfqpoint{4.322064in}{2.621630in}}{\pgfqpoint{4.324378in}{2.627216in}}{\pgfqpoint{4.324378in}{2.633040in}}%
\pgfpathcurveto{\pgfqpoint{4.324378in}{2.638864in}}{\pgfqpoint{4.322064in}{2.644450in}}{\pgfqpoint{4.317946in}{2.648568in}}%
\pgfpathcurveto{\pgfqpoint{4.313828in}{2.652686in}}{\pgfqpoint{4.308242in}{2.655000in}}{\pgfqpoint{4.302418in}{2.655000in}}%
\pgfpathcurveto{\pgfqpoint{4.296594in}{2.655000in}}{\pgfqpoint{4.291008in}{2.652686in}}{\pgfqpoint{4.286890in}{2.648568in}}%
\pgfpathcurveto{\pgfqpoint{4.282771in}{2.644450in}}{\pgfqpoint{4.280458in}{2.638864in}}{\pgfqpoint{4.280458in}{2.633040in}}%
\pgfpathcurveto{\pgfqpoint{4.280458in}{2.627216in}}{\pgfqpoint{4.282771in}{2.621630in}}{\pgfqpoint{4.286890in}{2.617512in}}%
\pgfpathcurveto{\pgfqpoint{4.291008in}{2.613393in}}{\pgfqpoint{4.296594in}{2.611080in}}{\pgfqpoint{4.302418in}{2.611080in}}%
\pgfpathlineto{\pgfqpoint{4.302418in}{2.611080in}}%
\pgfpathclose%
\pgfusepath{stroke,fill}%
\end{pgfscope}%
\begin{pgfscope}%
\pgfpathrectangle{\pgfqpoint{1.000000in}{1.148311in}}{\pgfqpoint{6.200000in}{5.623377in}}%
\pgfusepath{clip}%
\pgfsetbuttcap%
\pgfsetroundjoin%
\definecolor{currentfill}{rgb}{0.200000,0.800000,0.200000}%
\pgfsetfillcolor{currentfill}%
\pgfsetlinewidth{1.003750pt}%
\definecolor{currentstroke}{rgb}{0.200000,0.800000,0.200000}%
\pgfsetstrokecolor{currentstroke}%
\pgfsetdash{}{0pt}%
\pgfpathmoveto{\pgfqpoint{4.285706in}{2.636037in}}%
\pgfpathcurveto{\pgfqpoint{4.291530in}{2.636037in}}{\pgfqpoint{4.297117in}{2.638350in}}{\pgfqpoint{4.301235in}{2.642469in}}%
\pgfpathcurveto{\pgfqpoint{4.305353in}{2.646587in}}{\pgfqpoint{4.307667in}{2.652173in}}{\pgfqpoint{4.307667in}{2.657997in}}%
\pgfpathcurveto{\pgfqpoint{4.307667in}{2.663821in}}{\pgfqpoint{4.305353in}{2.669407in}}{\pgfqpoint{4.301235in}{2.673525in}}%
\pgfpathcurveto{\pgfqpoint{4.297117in}{2.677643in}}{\pgfqpoint{4.291530in}{2.679957in}}{\pgfqpoint{4.285706in}{2.679957in}}%
\pgfpathcurveto{\pgfqpoint{4.279883in}{2.679957in}}{\pgfqpoint{4.274296in}{2.677643in}}{\pgfqpoint{4.270178in}{2.673525in}}%
\pgfpathcurveto{\pgfqpoint{4.266060in}{2.669407in}}{\pgfqpoint{4.263746in}{2.663821in}}{\pgfqpoint{4.263746in}{2.657997in}}%
\pgfpathcurveto{\pgfqpoint{4.263746in}{2.652173in}}{\pgfqpoint{4.266060in}{2.646587in}}{\pgfqpoint{4.270178in}{2.642469in}}%
\pgfpathcurveto{\pgfqpoint{4.274296in}{2.638350in}}{\pgfqpoint{4.279883in}{2.636037in}}{\pgfqpoint{4.285706in}{2.636037in}}%
\pgfpathlineto{\pgfqpoint{4.285706in}{2.636037in}}%
\pgfpathclose%
\pgfusepath{stroke,fill}%
\end{pgfscope}%
\begin{pgfscope}%
\pgfpathrectangle{\pgfqpoint{1.000000in}{1.148311in}}{\pgfqpoint{6.200000in}{5.623377in}}%
\pgfusepath{clip}%
\pgfsetbuttcap%
\pgfsetroundjoin%
\definecolor{currentfill}{rgb}{0.200000,0.800000,0.200000}%
\pgfsetfillcolor{currentfill}%
\pgfsetlinewidth{1.003750pt}%
\definecolor{currentstroke}{rgb}{0.200000,0.800000,0.200000}%
\pgfsetstrokecolor{currentstroke}%
\pgfsetdash{}{0pt}%
\pgfpathmoveto{\pgfqpoint{4.238932in}{2.620707in}}%
\pgfpathcurveto{\pgfqpoint{4.244756in}{2.620707in}}{\pgfqpoint{4.250342in}{2.623021in}}{\pgfqpoint{4.254460in}{2.627140in}}%
\pgfpathcurveto{\pgfqpoint{4.258578in}{2.631258in}}{\pgfqpoint{4.260892in}{2.636844in}}{\pgfqpoint{4.260892in}{2.642668in}}%
\pgfpathcurveto{\pgfqpoint{4.260892in}{2.648492in}}{\pgfqpoint{4.258578in}{2.654078in}}{\pgfqpoint{4.254460in}{2.658196in}}%
\pgfpathcurveto{\pgfqpoint{4.250342in}{2.662314in}}{\pgfqpoint{4.244756in}{2.664628in}}{\pgfqpoint{4.238932in}{2.664628in}}%
\pgfpathcurveto{\pgfqpoint{4.233108in}{2.664628in}}{\pgfqpoint{4.227522in}{2.662314in}}{\pgfqpoint{4.223404in}{2.658196in}}%
\pgfpathcurveto{\pgfqpoint{4.219286in}{2.654078in}}{\pgfqpoint{4.216972in}{2.648492in}}{\pgfqpoint{4.216972in}{2.642668in}}%
\pgfpathcurveto{\pgfqpoint{4.216972in}{2.636844in}}{\pgfqpoint{4.219286in}{2.631258in}}{\pgfqpoint{4.223404in}{2.627140in}}%
\pgfpathcurveto{\pgfqpoint{4.227522in}{2.623021in}}{\pgfqpoint{4.233108in}{2.620707in}}{\pgfqpoint{4.238932in}{2.620707in}}%
\pgfpathlineto{\pgfqpoint{4.238932in}{2.620707in}}%
\pgfpathclose%
\pgfusepath{stroke,fill}%
\end{pgfscope}%
\begin{pgfscope}%
\pgfpathrectangle{\pgfqpoint{1.000000in}{1.148311in}}{\pgfqpoint{6.200000in}{5.623377in}}%
\pgfusepath{clip}%
\pgfsetbuttcap%
\pgfsetroundjoin%
\definecolor{currentfill}{rgb}{0.200000,0.800000,0.200000}%
\pgfsetfillcolor{currentfill}%
\pgfsetlinewidth{1.003750pt}%
\definecolor{currentstroke}{rgb}{0.200000,0.800000,0.200000}%
\pgfsetstrokecolor{currentstroke}%
\pgfsetdash{}{0pt}%
\pgfpathmoveto{\pgfqpoint{4.224779in}{2.648423in}}%
\pgfpathcurveto{\pgfqpoint{4.230603in}{2.648423in}}{\pgfqpoint{4.236190in}{2.650737in}}{\pgfqpoint{4.240308in}{2.654855in}}%
\pgfpathcurveto{\pgfqpoint{4.244426in}{2.658973in}}{\pgfqpoint{4.246740in}{2.664559in}}{\pgfqpoint{4.246740in}{2.670383in}}%
\pgfpathcurveto{\pgfqpoint{4.246740in}{2.676207in}}{\pgfqpoint{4.244426in}{2.681793in}}{\pgfqpoint{4.240308in}{2.685911in}}%
\pgfpathcurveto{\pgfqpoint{4.236190in}{2.690030in}}{\pgfqpoint{4.230603in}{2.692343in}}{\pgfqpoint{4.224779in}{2.692343in}}%
\pgfpathcurveto{\pgfqpoint{4.218956in}{2.692343in}}{\pgfqpoint{4.213369in}{2.690030in}}{\pgfqpoint{4.209251in}{2.685911in}}%
\pgfpathcurveto{\pgfqpoint{4.205133in}{2.681793in}}{\pgfqpoint{4.202819in}{2.676207in}}{\pgfqpoint{4.202819in}{2.670383in}}%
\pgfpathcurveto{\pgfqpoint{4.202819in}{2.664559in}}{\pgfqpoint{4.205133in}{2.658973in}}{\pgfqpoint{4.209251in}{2.654855in}}%
\pgfpathcurveto{\pgfqpoint{4.213369in}{2.650737in}}{\pgfqpoint{4.218956in}{2.648423in}}{\pgfqpoint{4.224779in}{2.648423in}}%
\pgfpathlineto{\pgfqpoint{4.224779in}{2.648423in}}%
\pgfpathclose%
\pgfusepath{stroke,fill}%
\end{pgfscope}%
\begin{pgfscope}%
\pgfpathrectangle{\pgfqpoint{1.000000in}{1.148311in}}{\pgfqpoint{6.200000in}{5.623377in}}%
\pgfusepath{clip}%
\pgfsetbuttcap%
\pgfsetroundjoin%
\definecolor{currentfill}{rgb}{0.200000,0.800000,0.200000}%
\pgfsetfillcolor{currentfill}%
\pgfsetlinewidth{1.003750pt}%
\definecolor{currentstroke}{rgb}{0.200000,0.800000,0.200000}%
\pgfsetstrokecolor{currentstroke}%
\pgfsetdash{}{0pt}%
\pgfpathmoveto{\pgfqpoint{4.200258in}{2.661107in}}%
\pgfpathcurveto{\pgfqpoint{4.206082in}{2.661107in}}{\pgfqpoint{4.211668in}{2.663421in}}{\pgfqpoint{4.215786in}{2.667539in}}%
\pgfpathcurveto{\pgfqpoint{4.219904in}{2.671657in}}{\pgfqpoint{4.222218in}{2.677243in}}{\pgfqpoint{4.222218in}{2.683067in}}%
\pgfpathcurveto{\pgfqpoint{4.222218in}{2.688891in}}{\pgfqpoint{4.219904in}{2.694477in}}{\pgfqpoint{4.215786in}{2.698595in}}%
\pgfpathcurveto{\pgfqpoint{4.211668in}{2.702713in}}{\pgfqpoint{4.206082in}{2.705027in}}{\pgfqpoint{4.200258in}{2.705027in}}%
\pgfpathcurveto{\pgfqpoint{4.194434in}{2.705027in}}{\pgfqpoint{4.188848in}{2.702713in}}{\pgfqpoint{4.184730in}{2.698595in}}%
\pgfpathcurveto{\pgfqpoint{4.180612in}{2.694477in}}{\pgfqpoint{4.178298in}{2.688891in}}{\pgfqpoint{4.178298in}{2.683067in}}%
\pgfpathcurveto{\pgfqpoint{4.178298in}{2.677243in}}{\pgfqpoint{4.180612in}{2.671657in}}{\pgfqpoint{4.184730in}{2.667539in}}%
\pgfpathcurveto{\pgfqpoint{4.188848in}{2.663421in}}{\pgfqpoint{4.194434in}{2.661107in}}{\pgfqpoint{4.200258in}{2.661107in}}%
\pgfpathlineto{\pgfqpoint{4.200258in}{2.661107in}}%
\pgfpathclose%
\pgfusepath{stroke,fill}%
\end{pgfscope}%
\begin{pgfscope}%
\pgfpathrectangle{\pgfqpoint{1.000000in}{1.148311in}}{\pgfqpoint{6.200000in}{5.623377in}}%
\pgfusepath{clip}%
\pgfsetbuttcap%
\pgfsetroundjoin%
\definecolor{currentfill}{rgb}{0.200000,0.800000,0.200000}%
\pgfsetfillcolor{currentfill}%
\pgfsetlinewidth{1.003750pt}%
\definecolor{currentstroke}{rgb}{0.200000,0.800000,0.200000}%
\pgfsetstrokecolor{currentstroke}%
\pgfsetdash{}{0pt}%
\pgfpathmoveto{\pgfqpoint{4.181125in}{2.685468in}}%
\pgfpathcurveto{\pgfqpoint{4.186949in}{2.685468in}}{\pgfqpoint{4.192535in}{2.687782in}}{\pgfqpoint{4.196653in}{2.691900in}}%
\pgfpathcurveto{\pgfqpoint{4.200772in}{2.696018in}}{\pgfqpoint{4.203086in}{2.701604in}}{\pgfqpoint{4.203086in}{2.707428in}}%
\pgfpathcurveto{\pgfqpoint{4.203086in}{2.713252in}}{\pgfqpoint{4.200772in}{2.718839in}}{\pgfqpoint{4.196653in}{2.722957in}}%
\pgfpathcurveto{\pgfqpoint{4.192535in}{2.727075in}}{\pgfqpoint{4.186949in}{2.729389in}}{\pgfqpoint{4.181125in}{2.729389in}}%
\pgfpathcurveto{\pgfqpoint{4.175301in}{2.729389in}}{\pgfqpoint{4.169715in}{2.727075in}}{\pgfqpoint{4.165597in}{2.722957in}}%
\pgfpathcurveto{\pgfqpoint{4.161479in}{2.718839in}}{\pgfqpoint{4.159165in}{2.713252in}}{\pgfqpoint{4.159165in}{2.707428in}}%
\pgfpathcurveto{\pgfqpoint{4.159165in}{2.701604in}}{\pgfqpoint{4.161479in}{2.696018in}}{\pgfqpoint{4.165597in}{2.691900in}}%
\pgfpathcurveto{\pgfqpoint{4.169715in}{2.687782in}}{\pgfqpoint{4.175301in}{2.685468in}}{\pgfqpoint{4.181125in}{2.685468in}}%
\pgfpathlineto{\pgfqpoint{4.181125in}{2.685468in}}%
\pgfpathclose%
\pgfusepath{stroke,fill}%
\end{pgfscope}%
\begin{pgfscope}%
\pgfpathrectangle{\pgfqpoint{1.000000in}{1.148311in}}{\pgfqpoint{6.200000in}{5.623377in}}%
\pgfusepath{clip}%
\pgfsetbuttcap%
\pgfsetroundjoin%
\definecolor{currentfill}{rgb}{0.200000,0.800000,0.200000}%
\pgfsetfillcolor{currentfill}%
\pgfsetlinewidth{1.003750pt}%
\definecolor{currentstroke}{rgb}{0.200000,0.800000,0.200000}%
\pgfsetstrokecolor{currentstroke}%
\pgfsetdash{}{0pt}%
\pgfpathmoveto{\pgfqpoint{4.135428in}{2.645349in}}%
\pgfpathcurveto{\pgfqpoint{4.141252in}{2.645349in}}{\pgfqpoint{4.146838in}{2.647663in}}{\pgfqpoint{4.150957in}{2.651781in}}%
\pgfpathcurveto{\pgfqpoint{4.155075in}{2.655899in}}{\pgfqpoint{4.157389in}{2.661485in}}{\pgfqpoint{4.157389in}{2.667309in}}%
\pgfpathcurveto{\pgfqpoint{4.157389in}{2.673133in}}{\pgfqpoint{4.155075in}{2.678719in}}{\pgfqpoint{4.150957in}{2.682838in}}%
\pgfpathcurveto{\pgfqpoint{4.146838in}{2.686956in}}{\pgfqpoint{4.141252in}{2.689270in}}{\pgfqpoint{4.135428in}{2.689270in}}%
\pgfpathcurveto{\pgfqpoint{4.129604in}{2.689270in}}{\pgfqpoint{4.124018in}{2.686956in}}{\pgfqpoint{4.119900in}{2.682838in}}%
\pgfpathcurveto{\pgfqpoint{4.115782in}{2.678719in}}{\pgfqpoint{4.113468in}{2.673133in}}{\pgfqpoint{4.113468in}{2.667309in}}%
\pgfpathcurveto{\pgfqpoint{4.113468in}{2.661485in}}{\pgfqpoint{4.115782in}{2.655899in}}{\pgfqpoint{4.119900in}{2.651781in}}%
\pgfpathcurveto{\pgfqpoint{4.124018in}{2.647663in}}{\pgfqpoint{4.129604in}{2.645349in}}{\pgfqpoint{4.135428in}{2.645349in}}%
\pgfpathlineto{\pgfqpoint{4.135428in}{2.645349in}}%
\pgfpathclose%
\pgfusepath{stroke,fill}%
\end{pgfscope}%
\begin{pgfscope}%
\pgfpathrectangle{\pgfqpoint{1.000000in}{1.148311in}}{\pgfqpoint{6.200000in}{5.623377in}}%
\pgfusepath{clip}%
\pgfsetbuttcap%
\pgfsetroundjoin%
\definecolor{currentfill}{rgb}{0.200000,0.800000,0.200000}%
\pgfsetfillcolor{currentfill}%
\pgfsetlinewidth{1.003750pt}%
\definecolor{currentstroke}{rgb}{0.200000,0.800000,0.200000}%
\pgfsetstrokecolor{currentstroke}%
\pgfsetdash{}{0pt}%
\pgfpathmoveto{\pgfqpoint{4.144181in}{2.757364in}}%
\pgfpathcurveto{\pgfqpoint{4.150005in}{2.757364in}}{\pgfqpoint{4.155591in}{2.759678in}}{\pgfqpoint{4.159709in}{2.763796in}}%
\pgfpathcurveto{\pgfqpoint{4.163828in}{2.767914in}}{\pgfqpoint{4.166141in}{2.773500in}}{\pgfqpoint{4.166141in}{2.779324in}}%
\pgfpathcurveto{\pgfqpoint{4.166141in}{2.785148in}}{\pgfqpoint{4.163828in}{2.790734in}}{\pgfqpoint{4.159709in}{2.794852in}}%
\pgfpathcurveto{\pgfqpoint{4.155591in}{2.798971in}}{\pgfqpoint{4.150005in}{2.801284in}}{\pgfqpoint{4.144181in}{2.801284in}}%
\pgfpathcurveto{\pgfqpoint{4.138357in}{2.801284in}}{\pgfqpoint{4.132771in}{2.798971in}}{\pgfqpoint{4.128653in}{2.794852in}}%
\pgfpathcurveto{\pgfqpoint{4.124535in}{2.790734in}}{\pgfqpoint{4.122221in}{2.785148in}}{\pgfqpoint{4.122221in}{2.779324in}}%
\pgfpathcurveto{\pgfqpoint{4.122221in}{2.773500in}}{\pgfqpoint{4.124535in}{2.767914in}}{\pgfqpoint{4.128653in}{2.763796in}}%
\pgfpathcurveto{\pgfqpoint{4.132771in}{2.759678in}}{\pgfqpoint{4.138357in}{2.757364in}}{\pgfqpoint{4.144181in}{2.757364in}}%
\pgfpathlineto{\pgfqpoint{4.144181in}{2.757364in}}%
\pgfpathclose%
\pgfusepath{stroke,fill}%
\end{pgfscope}%
\begin{pgfscope}%
\pgfpathrectangle{\pgfqpoint{1.000000in}{1.148311in}}{\pgfqpoint{6.200000in}{5.623377in}}%
\pgfusepath{clip}%
\pgfsetbuttcap%
\pgfsetroundjoin%
\definecolor{currentfill}{rgb}{0.200000,0.800000,0.200000}%
\pgfsetfillcolor{currentfill}%
\pgfsetlinewidth{1.003750pt}%
\definecolor{currentstroke}{rgb}{0.200000,0.800000,0.200000}%
\pgfsetstrokecolor{currentstroke}%
\pgfsetdash{}{0pt}%
\pgfpathmoveto{\pgfqpoint{4.103194in}{2.723296in}}%
\pgfpathcurveto{\pgfqpoint{4.109018in}{2.723296in}}{\pgfqpoint{4.114604in}{2.725609in}}{\pgfqpoint{4.118722in}{2.729728in}}%
\pgfpathcurveto{\pgfqpoint{4.122840in}{2.733846in}}{\pgfqpoint{4.125154in}{2.739432in}}{\pgfqpoint{4.125154in}{2.745256in}}%
\pgfpathcurveto{\pgfqpoint{4.125154in}{2.751080in}}{\pgfqpoint{4.122840in}{2.756666in}}{\pgfqpoint{4.118722in}{2.760784in}}%
\pgfpathcurveto{\pgfqpoint{4.114604in}{2.764902in}}{\pgfqpoint{4.109018in}{2.767216in}}{\pgfqpoint{4.103194in}{2.767216in}}%
\pgfpathcurveto{\pgfqpoint{4.097370in}{2.767216in}}{\pgfqpoint{4.091784in}{2.764902in}}{\pgfqpoint{4.087665in}{2.760784in}}%
\pgfpathcurveto{\pgfqpoint{4.083547in}{2.756666in}}{\pgfqpoint{4.081233in}{2.751080in}}{\pgfqpoint{4.081233in}{2.745256in}}%
\pgfpathcurveto{\pgfqpoint{4.081233in}{2.739432in}}{\pgfqpoint{4.083547in}{2.733846in}}{\pgfqpoint{4.087665in}{2.729728in}}%
\pgfpathcurveto{\pgfqpoint{4.091784in}{2.725609in}}{\pgfqpoint{4.097370in}{2.723296in}}{\pgfqpoint{4.103194in}{2.723296in}}%
\pgfpathlineto{\pgfqpoint{4.103194in}{2.723296in}}%
\pgfpathclose%
\pgfusepath{stroke,fill}%
\end{pgfscope}%
\begin{pgfscope}%
\pgfpathrectangle{\pgfqpoint{1.000000in}{1.148311in}}{\pgfqpoint{6.200000in}{5.623377in}}%
\pgfusepath{clip}%
\pgfsetbuttcap%
\pgfsetroundjoin%
\definecolor{currentfill}{rgb}{0.200000,0.800000,0.200000}%
\pgfsetfillcolor{currentfill}%
\pgfsetlinewidth{1.003750pt}%
\definecolor{currentstroke}{rgb}{0.200000,0.800000,0.200000}%
\pgfsetstrokecolor{currentstroke}%
\pgfsetdash{}{0pt}%
\pgfpathmoveto{\pgfqpoint{4.076273in}{2.737760in}}%
\pgfpathcurveto{\pgfqpoint{4.082097in}{2.737760in}}{\pgfqpoint{4.087683in}{2.740074in}}{\pgfqpoint{4.091801in}{2.744192in}}%
\pgfpathcurveto{\pgfqpoint{4.095920in}{2.748310in}}{\pgfqpoint{4.098233in}{2.753897in}}{\pgfqpoint{4.098233in}{2.759720in}}%
\pgfpathcurveto{\pgfqpoint{4.098233in}{2.765544in}}{\pgfqpoint{4.095920in}{2.771131in}}{\pgfqpoint{4.091801in}{2.775249in}}%
\pgfpathcurveto{\pgfqpoint{4.087683in}{2.779367in}}{\pgfqpoint{4.082097in}{2.781681in}}{\pgfqpoint{4.076273in}{2.781681in}}%
\pgfpathcurveto{\pgfqpoint{4.070449in}{2.781681in}}{\pgfqpoint{4.064863in}{2.779367in}}{\pgfqpoint{4.060745in}{2.775249in}}%
\pgfpathcurveto{\pgfqpoint{4.056627in}{2.771131in}}{\pgfqpoint{4.054313in}{2.765544in}}{\pgfqpoint{4.054313in}{2.759720in}}%
\pgfpathcurveto{\pgfqpoint{4.054313in}{2.753897in}}{\pgfqpoint{4.056627in}{2.748310in}}{\pgfqpoint{4.060745in}{2.744192in}}%
\pgfpathcurveto{\pgfqpoint{4.064863in}{2.740074in}}{\pgfqpoint{4.070449in}{2.737760in}}{\pgfqpoint{4.076273in}{2.737760in}}%
\pgfpathlineto{\pgfqpoint{4.076273in}{2.737760in}}%
\pgfpathclose%
\pgfusepath{stroke,fill}%
\end{pgfscope}%
\begin{pgfscope}%
\pgfpathrectangle{\pgfqpoint{1.000000in}{1.148311in}}{\pgfqpoint{6.200000in}{5.623377in}}%
\pgfusepath{clip}%
\pgfsetbuttcap%
\pgfsetroundjoin%
\definecolor{currentfill}{rgb}{0.200000,0.800000,0.200000}%
\pgfsetfillcolor{currentfill}%
\pgfsetlinewidth{1.003750pt}%
\definecolor{currentstroke}{rgb}{0.200000,0.800000,0.200000}%
\pgfsetstrokecolor{currentstroke}%
\pgfsetdash{}{0pt}%
\pgfpathmoveto{\pgfqpoint{4.050112in}{2.769989in}}%
\pgfpathcurveto{\pgfqpoint{4.055936in}{2.769989in}}{\pgfqpoint{4.061522in}{2.772303in}}{\pgfqpoint{4.065640in}{2.776421in}}%
\pgfpathcurveto{\pgfqpoint{4.069759in}{2.780539in}}{\pgfqpoint{4.072073in}{2.786125in}}{\pgfqpoint{4.072073in}{2.791949in}}%
\pgfpathcurveto{\pgfqpoint{4.072073in}{2.797773in}}{\pgfqpoint{4.069759in}{2.803359in}}{\pgfqpoint{4.065640in}{2.807477in}}%
\pgfpathcurveto{\pgfqpoint{4.061522in}{2.811595in}}{\pgfqpoint{4.055936in}{2.813909in}}{\pgfqpoint{4.050112in}{2.813909in}}%
\pgfpathcurveto{\pgfqpoint{4.044288in}{2.813909in}}{\pgfqpoint{4.038702in}{2.811595in}}{\pgfqpoint{4.034584in}{2.807477in}}%
\pgfpathcurveto{\pgfqpoint{4.030466in}{2.803359in}}{\pgfqpoint{4.028152in}{2.797773in}}{\pgfqpoint{4.028152in}{2.791949in}}%
\pgfpathcurveto{\pgfqpoint{4.028152in}{2.786125in}}{\pgfqpoint{4.030466in}{2.780539in}}{\pgfqpoint{4.034584in}{2.776421in}}%
\pgfpathcurveto{\pgfqpoint{4.038702in}{2.772303in}}{\pgfqpoint{4.044288in}{2.769989in}}{\pgfqpoint{4.050112in}{2.769989in}}%
\pgfpathlineto{\pgfqpoint{4.050112in}{2.769989in}}%
\pgfpathclose%
\pgfusepath{stroke,fill}%
\end{pgfscope}%
\begin{pgfscope}%
\pgfpathrectangle{\pgfqpoint{1.000000in}{1.148311in}}{\pgfqpoint{6.200000in}{5.623377in}}%
\pgfusepath{clip}%
\pgfsetbuttcap%
\pgfsetroundjoin%
\definecolor{currentfill}{rgb}{0.200000,0.800000,0.200000}%
\pgfsetfillcolor{currentfill}%
\pgfsetlinewidth{1.003750pt}%
\definecolor{currentstroke}{rgb}{0.200000,0.800000,0.200000}%
\pgfsetstrokecolor{currentstroke}%
\pgfsetdash{}{0pt}%
\pgfpathmoveto{\pgfqpoint{4.018691in}{2.769447in}}%
\pgfpathcurveto{\pgfqpoint{4.024515in}{2.769447in}}{\pgfqpoint{4.030101in}{2.771761in}}{\pgfqpoint{4.034219in}{2.775879in}}%
\pgfpathcurveto{\pgfqpoint{4.038337in}{2.779997in}}{\pgfqpoint{4.040651in}{2.785583in}}{\pgfqpoint{4.040651in}{2.791407in}}%
\pgfpathcurveto{\pgfqpoint{4.040651in}{2.797231in}}{\pgfqpoint{4.038337in}{2.802817in}}{\pgfqpoint{4.034219in}{2.806936in}}%
\pgfpathcurveto{\pgfqpoint{4.030101in}{2.811054in}}{\pgfqpoint{4.024515in}{2.813368in}}{\pgfqpoint{4.018691in}{2.813368in}}%
\pgfpathcurveto{\pgfqpoint{4.012867in}{2.813368in}}{\pgfqpoint{4.007281in}{2.811054in}}{\pgfqpoint{4.003163in}{2.806936in}}%
\pgfpathcurveto{\pgfqpoint{3.999045in}{2.802817in}}{\pgfqpoint{3.996731in}{2.797231in}}{\pgfqpoint{3.996731in}{2.791407in}}%
\pgfpathcurveto{\pgfqpoint{3.996731in}{2.785583in}}{\pgfqpoint{3.999045in}{2.779997in}}{\pgfqpoint{4.003163in}{2.775879in}}%
\pgfpathcurveto{\pgfqpoint{4.007281in}{2.771761in}}{\pgfqpoint{4.012867in}{2.769447in}}{\pgfqpoint{4.018691in}{2.769447in}}%
\pgfpathlineto{\pgfqpoint{4.018691in}{2.769447in}}%
\pgfpathclose%
\pgfusepath{stroke,fill}%
\end{pgfscope}%
\begin{pgfscope}%
\pgfpathrectangle{\pgfqpoint{1.000000in}{1.148311in}}{\pgfqpoint{6.200000in}{5.623377in}}%
\pgfusepath{clip}%
\pgfsetbuttcap%
\pgfsetroundjoin%
\definecolor{currentfill}{rgb}{0.200000,0.800000,0.200000}%
\pgfsetfillcolor{currentfill}%
\pgfsetlinewidth{1.003750pt}%
\definecolor{currentstroke}{rgb}{0.200000,0.800000,0.200000}%
\pgfsetstrokecolor{currentstroke}%
\pgfsetdash{}{0pt}%
\pgfpathmoveto{\pgfqpoint{3.988399in}{2.714028in}}%
\pgfpathcurveto{\pgfqpoint{3.994223in}{2.714028in}}{\pgfqpoint{3.999809in}{2.716342in}}{\pgfqpoint{4.003927in}{2.720460in}}%
\pgfpathcurveto{\pgfqpoint{4.008045in}{2.724578in}}{\pgfqpoint{4.010359in}{2.730164in}}{\pgfqpoint{4.010359in}{2.735988in}}%
\pgfpathcurveto{\pgfqpoint{4.010359in}{2.741812in}}{\pgfqpoint{4.008045in}{2.747398in}}{\pgfqpoint{4.003927in}{2.751516in}}%
\pgfpathcurveto{\pgfqpoint{3.999809in}{2.755634in}}{\pgfqpoint{3.994223in}{2.757948in}}{\pgfqpoint{3.988399in}{2.757948in}}%
\pgfpathcurveto{\pgfqpoint{3.982575in}{2.757948in}}{\pgfqpoint{3.976989in}{2.755634in}}{\pgfqpoint{3.972871in}{2.751516in}}%
\pgfpathcurveto{\pgfqpoint{3.968752in}{2.747398in}}{\pgfqpoint{3.966439in}{2.741812in}}{\pgfqpoint{3.966439in}{2.735988in}}%
\pgfpathcurveto{\pgfqpoint{3.966439in}{2.730164in}}{\pgfqpoint{3.968752in}{2.724578in}}{\pgfqpoint{3.972871in}{2.720460in}}%
\pgfpathcurveto{\pgfqpoint{3.976989in}{2.716342in}}{\pgfqpoint{3.982575in}{2.714028in}}{\pgfqpoint{3.988399in}{2.714028in}}%
\pgfpathlineto{\pgfqpoint{3.988399in}{2.714028in}}%
\pgfpathclose%
\pgfusepath{stroke,fill}%
\end{pgfscope}%
\begin{pgfscope}%
\pgfpathrectangle{\pgfqpoint{1.000000in}{1.148311in}}{\pgfqpoint{6.200000in}{5.623377in}}%
\pgfusepath{clip}%
\pgfsetbuttcap%
\pgfsetroundjoin%
\definecolor{currentfill}{rgb}{0.200000,0.800000,0.200000}%
\pgfsetfillcolor{currentfill}%
\pgfsetlinewidth{1.003750pt}%
\definecolor{currentstroke}{rgb}{0.200000,0.800000,0.200000}%
\pgfsetstrokecolor{currentstroke}%
\pgfsetdash{}{0pt}%
\pgfpathmoveto{\pgfqpoint{3.957095in}{2.759254in}}%
\pgfpathcurveto{\pgfqpoint{3.962919in}{2.759254in}}{\pgfqpoint{3.968505in}{2.761568in}}{\pgfqpoint{3.972623in}{2.765686in}}%
\pgfpathcurveto{\pgfqpoint{3.976742in}{2.769804in}}{\pgfqpoint{3.979056in}{2.775390in}}{\pgfqpoint{3.979056in}{2.781214in}}%
\pgfpathcurveto{\pgfqpoint{3.979056in}{2.787038in}}{\pgfqpoint{3.976742in}{2.792624in}}{\pgfqpoint{3.972623in}{2.796742in}}%
\pgfpathcurveto{\pgfqpoint{3.968505in}{2.800860in}}{\pgfqpoint{3.962919in}{2.803174in}}{\pgfqpoint{3.957095in}{2.803174in}}%
\pgfpathcurveto{\pgfqpoint{3.951271in}{2.803174in}}{\pgfqpoint{3.945685in}{2.800860in}}{\pgfqpoint{3.941567in}{2.796742in}}%
\pgfpathcurveto{\pgfqpoint{3.937449in}{2.792624in}}{\pgfqpoint{3.935135in}{2.787038in}}{\pgfqpoint{3.935135in}{2.781214in}}%
\pgfpathcurveto{\pgfqpoint{3.935135in}{2.775390in}}{\pgfqpoint{3.937449in}{2.769804in}}{\pgfqpoint{3.941567in}{2.765686in}}%
\pgfpathcurveto{\pgfqpoint{3.945685in}{2.761568in}}{\pgfqpoint{3.951271in}{2.759254in}}{\pgfqpoint{3.957095in}{2.759254in}}%
\pgfpathlineto{\pgfqpoint{3.957095in}{2.759254in}}%
\pgfpathclose%
\pgfusepath{stroke,fill}%
\end{pgfscope}%
\begin{pgfscope}%
\pgfpathrectangle{\pgfqpoint{1.000000in}{1.148311in}}{\pgfqpoint{6.200000in}{5.623377in}}%
\pgfusepath{clip}%
\pgfsetbuttcap%
\pgfsetroundjoin%
\definecolor{currentfill}{rgb}{0.200000,0.800000,0.200000}%
\pgfsetfillcolor{currentfill}%
\pgfsetlinewidth{1.003750pt}%
\definecolor{currentstroke}{rgb}{0.200000,0.800000,0.200000}%
\pgfsetstrokecolor{currentstroke}%
\pgfsetdash{}{0pt}%
\pgfpathmoveto{\pgfqpoint{3.926744in}{2.755459in}}%
\pgfpathcurveto{\pgfqpoint{3.932568in}{2.755459in}}{\pgfqpoint{3.938154in}{2.757773in}}{\pgfqpoint{3.942273in}{2.761891in}}%
\pgfpathcurveto{\pgfqpoint{3.946391in}{2.766010in}}{\pgfqpoint{3.948705in}{2.771596in}}{\pgfqpoint{3.948705in}{2.777420in}}%
\pgfpathcurveto{\pgfqpoint{3.948705in}{2.783244in}}{\pgfqpoint{3.946391in}{2.788830in}}{\pgfqpoint{3.942273in}{2.792948in}}%
\pgfpathcurveto{\pgfqpoint{3.938154in}{2.797066in}}{\pgfqpoint{3.932568in}{2.799380in}}{\pgfqpoint{3.926744in}{2.799380in}}%
\pgfpathcurveto{\pgfqpoint{3.920920in}{2.799380in}}{\pgfqpoint{3.915334in}{2.797066in}}{\pgfqpoint{3.911216in}{2.792948in}}%
\pgfpathcurveto{\pgfqpoint{3.907098in}{2.788830in}}{\pgfqpoint{3.904784in}{2.783244in}}{\pgfqpoint{3.904784in}{2.777420in}}%
\pgfpathcurveto{\pgfqpoint{3.904784in}{2.771596in}}{\pgfqpoint{3.907098in}{2.766010in}}{\pgfqpoint{3.911216in}{2.761891in}}%
\pgfpathcurveto{\pgfqpoint{3.915334in}{2.757773in}}{\pgfqpoint{3.920920in}{2.755459in}}{\pgfqpoint{3.926744in}{2.755459in}}%
\pgfpathlineto{\pgfqpoint{3.926744in}{2.755459in}}%
\pgfpathclose%
\pgfusepath{stroke,fill}%
\end{pgfscope}%
\begin{pgfscope}%
\pgfpathrectangle{\pgfqpoint{1.000000in}{1.148311in}}{\pgfqpoint{6.200000in}{5.623377in}}%
\pgfusepath{clip}%
\pgfsetbuttcap%
\pgfsetroundjoin%
\definecolor{currentfill}{rgb}{0.200000,0.800000,0.200000}%
\pgfsetfillcolor{currentfill}%
\pgfsetlinewidth{1.003750pt}%
\definecolor{currentstroke}{rgb}{0.200000,0.800000,0.200000}%
\pgfsetstrokecolor{currentstroke}%
\pgfsetdash{}{0pt}%
\pgfpathmoveto{\pgfqpoint{3.898892in}{2.739373in}}%
\pgfpathcurveto{\pgfqpoint{3.904716in}{2.739373in}}{\pgfqpoint{3.910302in}{2.741687in}}{\pgfqpoint{3.914421in}{2.745805in}}%
\pgfpathcurveto{\pgfqpoint{3.918539in}{2.749923in}}{\pgfqpoint{3.920853in}{2.755509in}}{\pgfqpoint{3.920853in}{2.761333in}}%
\pgfpathcurveto{\pgfqpoint{3.920853in}{2.767157in}}{\pgfqpoint{3.918539in}{2.772743in}}{\pgfqpoint{3.914421in}{2.776861in}}%
\pgfpathcurveto{\pgfqpoint{3.910302in}{2.780979in}}{\pgfqpoint{3.904716in}{2.783293in}}{\pgfqpoint{3.898892in}{2.783293in}}%
\pgfpathcurveto{\pgfqpoint{3.893068in}{2.783293in}}{\pgfqpoint{3.887482in}{2.780979in}}{\pgfqpoint{3.883364in}{2.776861in}}%
\pgfpathcurveto{\pgfqpoint{3.879246in}{2.772743in}}{\pgfqpoint{3.876932in}{2.767157in}}{\pgfqpoint{3.876932in}{2.761333in}}%
\pgfpathcurveto{\pgfqpoint{3.876932in}{2.755509in}}{\pgfqpoint{3.879246in}{2.749923in}}{\pgfqpoint{3.883364in}{2.745805in}}%
\pgfpathcurveto{\pgfqpoint{3.887482in}{2.741687in}}{\pgfqpoint{3.893068in}{2.739373in}}{\pgfqpoint{3.898892in}{2.739373in}}%
\pgfpathlineto{\pgfqpoint{3.898892in}{2.739373in}}%
\pgfpathclose%
\pgfusepath{stroke,fill}%
\end{pgfscope}%
\begin{pgfscope}%
\pgfpathrectangle{\pgfqpoint{1.000000in}{1.148311in}}{\pgfqpoint{6.200000in}{5.623377in}}%
\pgfusepath{clip}%
\pgfsetbuttcap%
\pgfsetroundjoin%
\definecolor{currentfill}{rgb}{0.200000,0.800000,0.200000}%
\pgfsetfillcolor{currentfill}%
\pgfsetlinewidth{1.003750pt}%
\definecolor{currentstroke}{rgb}{0.200000,0.800000,0.200000}%
\pgfsetstrokecolor{currentstroke}%
\pgfsetdash{}{0pt}%
\pgfpathmoveto{\pgfqpoint{3.862002in}{2.760757in}}%
\pgfpathcurveto{\pgfqpoint{3.867826in}{2.760757in}}{\pgfqpoint{3.873412in}{2.763071in}}{\pgfqpoint{3.877531in}{2.767189in}}%
\pgfpathcurveto{\pgfqpoint{3.881649in}{2.771307in}}{\pgfqpoint{3.883963in}{2.776893in}}{\pgfqpoint{3.883963in}{2.782717in}}%
\pgfpathcurveto{\pgfqpoint{3.883963in}{2.788541in}}{\pgfqpoint{3.881649in}{2.794127in}}{\pgfqpoint{3.877531in}{2.798245in}}%
\pgfpathcurveto{\pgfqpoint{3.873412in}{2.802364in}}{\pgfqpoint{3.867826in}{2.804677in}}{\pgfqpoint{3.862002in}{2.804677in}}%
\pgfpathcurveto{\pgfqpoint{3.856178in}{2.804677in}}{\pgfqpoint{3.850592in}{2.802364in}}{\pgfqpoint{3.846474in}{2.798245in}}%
\pgfpathcurveto{\pgfqpoint{3.842356in}{2.794127in}}{\pgfqpoint{3.840042in}{2.788541in}}{\pgfqpoint{3.840042in}{2.782717in}}%
\pgfpathcurveto{\pgfqpoint{3.840042in}{2.776893in}}{\pgfqpoint{3.842356in}{2.771307in}}{\pgfqpoint{3.846474in}{2.767189in}}%
\pgfpathcurveto{\pgfqpoint{3.850592in}{2.763071in}}{\pgfqpoint{3.856178in}{2.760757in}}{\pgfqpoint{3.862002in}{2.760757in}}%
\pgfpathlineto{\pgfqpoint{3.862002in}{2.760757in}}%
\pgfpathclose%
\pgfusepath{stroke,fill}%
\end{pgfscope}%
\begin{pgfscope}%
\pgfpathrectangle{\pgfqpoint{1.000000in}{1.148311in}}{\pgfqpoint{6.200000in}{5.623377in}}%
\pgfusepath{clip}%
\pgfsetbuttcap%
\pgfsetroundjoin%
\definecolor{currentfill}{rgb}{0.200000,0.800000,0.200000}%
\pgfsetfillcolor{currentfill}%
\pgfsetlinewidth{1.003750pt}%
\definecolor{currentstroke}{rgb}{0.200000,0.800000,0.200000}%
\pgfsetstrokecolor{currentstroke}%
\pgfsetdash{}{0pt}%
\pgfpathmoveto{\pgfqpoint{3.835076in}{2.741547in}}%
\pgfpathcurveto{\pgfqpoint{3.840900in}{2.741547in}}{\pgfqpoint{3.846486in}{2.743861in}}{\pgfqpoint{3.850604in}{2.747979in}}%
\pgfpathcurveto{\pgfqpoint{3.854722in}{2.752097in}}{\pgfqpoint{3.857036in}{2.757683in}}{\pgfqpoint{3.857036in}{2.763507in}}%
\pgfpathcurveto{\pgfqpoint{3.857036in}{2.769331in}}{\pgfqpoint{3.854722in}{2.774917in}}{\pgfqpoint{3.850604in}{2.779035in}}%
\pgfpathcurveto{\pgfqpoint{3.846486in}{2.783154in}}{\pgfqpoint{3.840900in}{2.785467in}}{\pgfqpoint{3.835076in}{2.785467in}}%
\pgfpathcurveto{\pgfqpoint{3.829252in}{2.785467in}}{\pgfqpoint{3.823666in}{2.783154in}}{\pgfqpoint{3.819548in}{2.779035in}}%
\pgfpathcurveto{\pgfqpoint{3.815430in}{2.774917in}}{\pgfqpoint{3.813116in}{2.769331in}}{\pgfqpoint{3.813116in}{2.763507in}}%
\pgfpathcurveto{\pgfqpoint{3.813116in}{2.757683in}}{\pgfqpoint{3.815430in}{2.752097in}}{\pgfqpoint{3.819548in}{2.747979in}}%
\pgfpathcurveto{\pgfqpoint{3.823666in}{2.743861in}}{\pgfqpoint{3.829252in}{2.741547in}}{\pgfqpoint{3.835076in}{2.741547in}}%
\pgfpathlineto{\pgfqpoint{3.835076in}{2.741547in}}%
\pgfpathclose%
\pgfusepath{stroke,fill}%
\end{pgfscope}%
\begin{pgfscope}%
\pgfpathrectangle{\pgfqpoint{1.000000in}{1.148311in}}{\pgfqpoint{6.200000in}{5.623377in}}%
\pgfusepath{clip}%
\pgfsetbuttcap%
\pgfsetroundjoin%
\definecolor{currentfill}{rgb}{0.200000,0.800000,0.200000}%
\pgfsetfillcolor{currentfill}%
\pgfsetlinewidth{1.003750pt}%
\definecolor{currentstroke}{rgb}{0.200000,0.800000,0.200000}%
\pgfsetstrokecolor{currentstroke}%
\pgfsetdash{}{0pt}%
\pgfpathmoveto{\pgfqpoint{3.807139in}{2.727818in}}%
\pgfpathcurveto{\pgfqpoint{3.812963in}{2.727818in}}{\pgfqpoint{3.818549in}{2.730132in}}{\pgfqpoint{3.822668in}{2.734250in}}%
\pgfpathcurveto{\pgfqpoint{3.826786in}{2.738368in}}{\pgfqpoint{3.829100in}{2.743954in}}{\pgfqpoint{3.829100in}{2.749778in}}%
\pgfpathcurveto{\pgfqpoint{3.829100in}{2.755602in}}{\pgfqpoint{3.826786in}{2.761188in}}{\pgfqpoint{3.822668in}{2.765306in}}%
\pgfpathcurveto{\pgfqpoint{3.818549in}{2.769424in}}{\pgfqpoint{3.812963in}{2.771738in}}{\pgfqpoint{3.807139in}{2.771738in}}%
\pgfpathcurveto{\pgfqpoint{3.801315in}{2.771738in}}{\pgfqpoint{3.795729in}{2.769424in}}{\pgfqpoint{3.791611in}{2.765306in}}%
\pgfpathcurveto{\pgfqpoint{3.787493in}{2.761188in}}{\pgfqpoint{3.785179in}{2.755602in}}{\pgfqpoint{3.785179in}{2.749778in}}%
\pgfpathcurveto{\pgfqpoint{3.785179in}{2.743954in}}{\pgfqpoint{3.787493in}{2.738368in}}{\pgfqpoint{3.791611in}{2.734250in}}%
\pgfpathcurveto{\pgfqpoint{3.795729in}{2.730132in}}{\pgfqpoint{3.801315in}{2.727818in}}{\pgfqpoint{3.807139in}{2.727818in}}%
\pgfpathlineto{\pgfqpoint{3.807139in}{2.727818in}}%
\pgfpathclose%
\pgfusepath{stroke,fill}%
\end{pgfscope}%
\begin{pgfscope}%
\pgfpathrectangle{\pgfqpoint{1.000000in}{1.148311in}}{\pgfqpoint{6.200000in}{5.623377in}}%
\pgfusepath{clip}%
\pgfsetbuttcap%
\pgfsetroundjoin%
\definecolor{currentfill}{rgb}{0.200000,0.800000,0.200000}%
\pgfsetfillcolor{currentfill}%
\pgfsetlinewidth{1.003750pt}%
\definecolor{currentstroke}{rgb}{0.200000,0.800000,0.200000}%
\pgfsetstrokecolor{currentstroke}%
\pgfsetdash{}{0pt}%
\pgfpathmoveto{\pgfqpoint{3.767814in}{2.737602in}}%
\pgfpathcurveto{\pgfqpoint{3.773638in}{2.737602in}}{\pgfqpoint{3.779224in}{2.739916in}}{\pgfqpoint{3.783342in}{2.744034in}}%
\pgfpathcurveto{\pgfqpoint{3.787460in}{2.748152in}}{\pgfqpoint{3.789774in}{2.753738in}}{\pgfqpoint{3.789774in}{2.759562in}}%
\pgfpathcurveto{\pgfqpoint{3.789774in}{2.765386in}}{\pgfqpoint{3.787460in}{2.770972in}}{\pgfqpoint{3.783342in}{2.775090in}}%
\pgfpathcurveto{\pgfqpoint{3.779224in}{2.779209in}}{\pgfqpoint{3.773638in}{2.781523in}}{\pgfqpoint{3.767814in}{2.781523in}}%
\pgfpathcurveto{\pgfqpoint{3.761990in}{2.781523in}}{\pgfqpoint{3.756404in}{2.779209in}}{\pgfqpoint{3.752286in}{2.775090in}}%
\pgfpathcurveto{\pgfqpoint{3.748167in}{2.770972in}}{\pgfqpoint{3.745853in}{2.765386in}}{\pgfqpoint{3.745853in}{2.759562in}}%
\pgfpathcurveto{\pgfqpoint{3.745853in}{2.753738in}}{\pgfqpoint{3.748167in}{2.748152in}}{\pgfqpoint{3.752286in}{2.744034in}}%
\pgfpathcurveto{\pgfqpoint{3.756404in}{2.739916in}}{\pgfqpoint{3.761990in}{2.737602in}}{\pgfqpoint{3.767814in}{2.737602in}}%
\pgfpathlineto{\pgfqpoint{3.767814in}{2.737602in}}%
\pgfpathclose%
\pgfusepath{stroke,fill}%
\end{pgfscope}%
\begin{pgfscope}%
\pgfpathrectangle{\pgfqpoint{1.000000in}{1.148311in}}{\pgfqpoint{6.200000in}{5.623377in}}%
\pgfusepath{clip}%
\pgfsetbuttcap%
\pgfsetroundjoin%
\definecolor{currentfill}{rgb}{0.200000,0.800000,0.200000}%
\pgfsetfillcolor{currentfill}%
\pgfsetlinewidth{1.003750pt}%
\definecolor{currentstroke}{rgb}{0.200000,0.800000,0.200000}%
\pgfsetstrokecolor{currentstroke}%
\pgfsetdash{}{0pt}%
\pgfpathmoveto{\pgfqpoint{3.765612in}{2.676432in}}%
\pgfpathcurveto{\pgfqpoint{3.771436in}{2.676432in}}{\pgfqpoint{3.777022in}{2.678746in}}{\pgfqpoint{3.781140in}{2.682864in}}%
\pgfpathcurveto{\pgfqpoint{3.785258in}{2.686982in}}{\pgfqpoint{3.787572in}{2.692569in}}{\pgfqpoint{3.787572in}{2.698393in}}%
\pgfpathcurveto{\pgfqpoint{3.787572in}{2.704216in}}{\pgfqpoint{3.785258in}{2.709803in}}{\pgfqpoint{3.781140in}{2.713921in}}%
\pgfpathcurveto{\pgfqpoint{3.777022in}{2.718039in}}{\pgfqpoint{3.771436in}{2.720353in}}{\pgfqpoint{3.765612in}{2.720353in}}%
\pgfpathcurveto{\pgfqpoint{3.759788in}{2.720353in}}{\pgfqpoint{3.754202in}{2.718039in}}{\pgfqpoint{3.750083in}{2.713921in}}%
\pgfpathcurveto{\pgfqpoint{3.745965in}{2.709803in}}{\pgfqpoint{3.743651in}{2.704216in}}{\pgfqpoint{3.743651in}{2.698393in}}%
\pgfpathcurveto{\pgfqpoint{3.743651in}{2.692569in}}{\pgfqpoint{3.745965in}{2.686982in}}{\pgfqpoint{3.750083in}{2.682864in}}%
\pgfpathcurveto{\pgfqpoint{3.754202in}{2.678746in}}{\pgfqpoint{3.759788in}{2.676432in}}{\pgfqpoint{3.765612in}{2.676432in}}%
\pgfpathlineto{\pgfqpoint{3.765612in}{2.676432in}}%
\pgfpathclose%
\pgfusepath{stroke,fill}%
\end{pgfscope}%
\begin{pgfscope}%
\pgfpathrectangle{\pgfqpoint{1.000000in}{1.148311in}}{\pgfqpoint{6.200000in}{5.623377in}}%
\pgfusepath{clip}%
\pgfsetbuttcap%
\pgfsetroundjoin%
\definecolor{currentfill}{rgb}{0.200000,0.800000,0.200000}%
\pgfsetfillcolor{currentfill}%
\pgfsetlinewidth{1.003750pt}%
\definecolor{currentstroke}{rgb}{0.200000,0.800000,0.200000}%
\pgfsetstrokecolor{currentstroke}%
\pgfsetdash{}{0pt}%
\pgfpathmoveto{\pgfqpoint{3.740203in}{2.662022in}}%
\pgfpathcurveto{\pgfqpoint{3.746027in}{2.662022in}}{\pgfqpoint{3.751613in}{2.664336in}}{\pgfqpoint{3.755731in}{2.668454in}}%
\pgfpathcurveto{\pgfqpoint{3.759849in}{2.672572in}}{\pgfqpoint{3.762163in}{2.678159in}}{\pgfqpoint{3.762163in}{2.683983in}}%
\pgfpathcurveto{\pgfqpoint{3.762163in}{2.689806in}}{\pgfqpoint{3.759849in}{2.695393in}}{\pgfqpoint{3.755731in}{2.699511in}}%
\pgfpathcurveto{\pgfqpoint{3.751613in}{2.703629in}}{\pgfqpoint{3.746027in}{2.705943in}}{\pgfqpoint{3.740203in}{2.705943in}}%
\pgfpathcurveto{\pgfqpoint{3.734379in}{2.705943in}}{\pgfqpoint{3.728793in}{2.703629in}}{\pgfqpoint{3.724675in}{2.699511in}}%
\pgfpathcurveto{\pgfqpoint{3.720556in}{2.695393in}}{\pgfqpoint{3.718243in}{2.689806in}}{\pgfqpoint{3.718243in}{2.683983in}}%
\pgfpathcurveto{\pgfqpoint{3.718243in}{2.678159in}}{\pgfqpoint{3.720556in}{2.672572in}}{\pgfqpoint{3.724675in}{2.668454in}}%
\pgfpathcurveto{\pgfqpoint{3.728793in}{2.664336in}}{\pgfqpoint{3.734379in}{2.662022in}}{\pgfqpoint{3.740203in}{2.662022in}}%
\pgfpathlineto{\pgfqpoint{3.740203in}{2.662022in}}%
\pgfpathclose%
\pgfusepath{stroke,fill}%
\end{pgfscope}%
\begin{pgfscope}%
\pgfpathrectangle{\pgfqpoint{1.000000in}{1.148311in}}{\pgfqpoint{6.200000in}{5.623377in}}%
\pgfusepath{clip}%
\pgfsetbuttcap%
\pgfsetroundjoin%
\definecolor{currentfill}{rgb}{0.200000,0.800000,0.200000}%
\pgfsetfillcolor{currentfill}%
\pgfsetlinewidth{1.003750pt}%
\definecolor{currentstroke}{rgb}{0.200000,0.800000,0.200000}%
\pgfsetstrokecolor{currentstroke}%
\pgfsetdash{}{0pt}%
\pgfpathmoveto{\pgfqpoint{3.698145in}{2.669053in}}%
\pgfpathcurveto{\pgfqpoint{3.703969in}{2.669053in}}{\pgfqpoint{3.709555in}{2.671367in}}{\pgfqpoint{3.713674in}{2.675485in}}%
\pgfpathcurveto{\pgfqpoint{3.717792in}{2.679604in}}{\pgfqpoint{3.720106in}{2.685190in}}{\pgfqpoint{3.720106in}{2.691014in}}%
\pgfpathcurveto{\pgfqpoint{3.720106in}{2.696838in}}{\pgfqpoint{3.717792in}{2.702424in}}{\pgfqpoint{3.713674in}{2.706542in}}%
\pgfpathcurveto{\pgfqpoint{3.709555in}{2.710660in}}{\pgfqpoint{3.703969in}{2.712974in}}{\pgfqpoint{3.698145in}{2.712974in}}%
\pgfpathcurveto{\pgfqpoint{3.692321in}{2.712974in}}{\pgfqpoint{3.686735in}{2.710660in}}{\pgfqpoint{3.682617in}{2.706542in}}%
\pgfpathcurveto{\pgfqpoint{3.678499in}{2.702424in}}{\pgfqpoint{3.676185in}{2.696838in}}{\pgfqpoint{3.676185in}{2.691014in}}%
\pgfpathcurveto{\pgfqpoint{3.676185in}{2.685190in}}{\pgfqpoint{3.678499in}{2.679604in}}{\pgfqpoint{3.682617in}{2.675485in}}%
\pgfpathcurveto{\pgfqpoint{3.686735in}{2.671367in}}{\pgfqpoint{3.692321in}{2.669053in}}{\pgfqpoint{3.698145in}{2.669053in}}%
\pgfpathlineto{\pgfqpoint{3.698145in}{2.669053in}}%
\pgfpathclose%
\pgfusepath{stroke,fill}%
\end{pgfscope}%
\begin{pgfscope}%
\pgfpathrectangle{\pgfqpoint{1.000000in}{1.148311in}}{\pgfqpoint{6.200000in}{5.623377in}}%
\pgfusepath{clip}%
\pgfsetbuttcap%
\pgfsetroundjoin%
\definecolor{currentfill}{rgb}{0.200000,0.800000,0.200000}%
\pgfsetfillcolor{currentfill}%
\pgfsetlinewidth{1.003750pt}%
\definecolor{currentstroke}{rgb}{0.200000,0.800000,0.200000}%
\pgfsetstrokecolor{currentstroke}%
\pgfsetdash{}{0pt}%
\pgfpathmoveto{\pgfqpoint{3.699838in}{2.619577in}}%
\pgfpathcurveto{\pgfqpoint{3.705662in}{2.619577in}}{\pgfqpoint{3.711248in}{2.621891in}}{\pgfqpoint{3.715366in}{2.626009in}}%
\pgfpathcurveto{\pgfqpoint{3.719484in}{2.630127in}}{\pgfqpoint{3.721798in}{2.635713in}}{\pgfqpoint{3.721798in}{2.641537in}}%
\pgfpathcurveto{\pgfqpoint{3.721798in}{2.647361in}}{\pgfqpoint{3.719484in}{2.652947in}}{\pgfqpoint{3.715366in}{2.657065in}}%
\pgfpathcurveto{\pgfqpoint{3.711248in}{2.661183in}}{\pgfqpoint{3.705662in}{2.663497in}}{\pgfqpoint{3.699838in}{2.663497in}}%
\pgfpathcurveto{\pgfqpoint{3.694014in}{2.663497in}}{\pgfqpoint{3.688428in}{2.661183in}}{\pgfqpoint{3.684310in}{2.657065in}}%
\pgfpathcurveto{\pgfqpoint{3.680192in}{2.652947in}}{\pgfqpoint{3.677878in}{2.647361in}}{\pgfqpoint{3.677878in}{2.641537in}}%
\pgfpathcurveto{\pgfqpoint{3.677878in}{2.635713in}}{\pgfqpoint{3.680192in}{2.630127in}}{\pgfqpoint{3.684310in}{2.626009in}}%
\pgfpathcurveto{\pgfqpoint{3.688428in}{2.621891in}}{\pgfqpoint{3.694014in}{2.619577in}}{\pgfqpoint{3.699838in}{2.619577in}}%
\pgfpathlineto{\pgfqpoint{3.699838in}{2.619577in}}%
\pgfpathclose%
\pgfusepath{stroke,fill}%
\end{pgfscope}%
\begin{pgfscope}%
\pgfpathrectangle{\pgfqpoint{1.000000in}{1.148311in}}{\pgfqpoint{6.200000in}{5.623377in}}%
\pgfusepath{clip}%
\pgfsetbuttcap%
\pgfsetroundjoin%
\definecolor{currentfill}{rgb}{0.200000,0.800000,0.200000}%
\pgfsetfillcolor{currentfill}%
\pgfsetlinewidth{1.003750pt}%
\definecolor{currentstroke}{rgb}{0.200000,0.800000,0.200000}%
\pgfsetstrokecolor{currentstroke}%
\pgfsetdash{}{0pt}%
\pgfpathmoveto{\pgfqpoint{3.660479in}{2.618770in}}%
\pgfpathcurveto{\pgfqpoint{3.666303in}{2.618770in}}{\pgfqpoint{3.671889in}{2.621084in}}{\pgfqpoint{3.676007in}{2.625202in}}%
\pgfpathcurveto{\pgfqpoint{3.680126in}{2.629320in}}{\pgfqpoint{3.682439in}{2.634907in}}{\pgfqpoint{3.682439in}{2.640730in}}%
\pgfpathcurveto{\pgfqpoint{3.682439in}{2.646554in}}{\pgfqpoint{3.680126in}{2.652141in}}{\pgfqpoint{3.676007in}{2.656259in}}%
\pgfpathcurveto{\pgfqpoint{3.671889in}{2.660377in}}{\pgfqpoint{3.666303in}{2.662691in}}{\pgfqpoint{3.660479in}{2.662691in}}%
\pgfpathcurveto{\pgfqpoint{3.654655in}{2.662691in}}{\pgfqpoint{3.649069in}{2.660377in}}{\pgfqpoint{3.644951in}{2.656259in}}%
\pgfpathcurveto{\pgfqpoint{3.640833in}{2.652141in}}{\pgfqpoint{3.638519in}{2.646554in}}{\pgfqpoint{3.638519in}{2.640730in}}%
\pgfpathcurveto{\pgfqpoint{3.638519in}{2.634907in}}{\pgfqpoint{3.640833in}{2.629320in}}{\pgfqpoint{3.644951in}{2.625202in}}%
\pgfpathcurveto{\pgfqpoint{3.649069in}{2.621084in}}{\pgfqpoint{3.654655in}{2.618770in}}{\pgfqpoint{3.660479in}{2.618770in}}%
\pgfpathlineto{\pgfqpoint{3.660479in}{2.618770in}}%
\pgfpathclose%
\pgfusepath{stroke,fill}%
\end{pgfscope}%
\begin{pgfscope}%
\pgfpathrectangle{\pgfqpoint{1.000000in}{1.148311in}}{\pgfqpoint{6.200000in}{5.623377in}}%
\pgfusepath{clip}%
\pgfsetbuttcap%
\pgfsetroundjoin%
\definecolor{currentfill}{rgb}{0.200000,0.800000,0.200000}%
\pgfsetfillcolor{currentfill}%
\pgfsetlinewidth{1.003750pt}%
\definecolor{currentstroke}{rgb}{0.200000,0.800000,0.200000}%
\pgfsetstrokecolor{currentstroke}%
\pgfsetdash{}{0pt}%
\pgfpathmoveto{\pgfqpoint{3.670306in}{2.569357in}}%
\pgfpathcurveto{\pgfqpoint{3.676130in}{2.569357in}}{\pgfqpoint{3.681717in}{2.571671in}}{\pgfqpoint{3.685835in}{2.575789in}}%
\pgfpathcurveto{\pgfqpoint{3.689953in}{2.579907in}}{\pgfqpoint{3.692267in}{2.585493in}}{\pgfqpoint{3.692267in}{2.591317in}}%
\pgfpathcurveto{\pgfqpoint{3.692267in}{2.597141in}}{\pgfqpoint{3.689953in}{2.602727in}}{\pgfqpoint{3.685835in}{2.606845in}}%
\pgfpathcurveto{\pgfqpoint{3.681717in}{2.610963in}}{\pgfqpoint{3.676130in}{2.613277in}}{\pgfqpoint{3.670306in}{2.613277in}}%
\pgfpathcurveto{\pgfqpoint{3.664482in}{2.613277in}}{\pgfqpoint{3.658896in}{2.610963in}}{\pgfqpoint{3.654778in}{2.606845in}}%
\pgfpathcurveto{\pgfqpoint{3.650660in}{2.602727in}}{\pgfqpoint{3.648346in}{2.597141in}}{\pgfqpoint{3.648346in}{2.591317in}}%
\pgfpathcurveto{\pgfqpoint{3.648346in}{2.585493in}}{\pgfqpoint{3.650660in}{2.579907in}}{\pgfqpoint{3.654778in}{2.575789in}}%
\pgfpathcurveto{\pgfqpoint{3.658896in}{2.571671in}}{\pgfqpoint{3.664482in}{2.569357in}}{\pgfqpoint{3.670306in}{2.569357in}}%
\pgfpathlineto{\pgfqpoint{3.670306in}{2.569357in}}%
\pgfpathclose%
\pgfusepath{stroke,fill}%
\end{pgfscope}%
\begin{pgfscope}%
\pgfpathrectangle{\pgfqpoint{1.000000in}{1.148311in}}{\pgfqpoint{6.200000in}{5.623377in}}%
\pgfusepath{clip}%
\pgfsetbuttcap%
\pgfsetroundjoin%
\definecolor{currentfill}{rgb}{0.200000,0.800000,0.200000}%
\pgfsetfillcolor{currentfill}%
\pgfsetlinewidth{1.003750pt}%
\definecolor{currentstroke}{rgb}{0.200000,0.800000,0.200000}%
\pgfsetstrokecolor{currentstroke}%
\pgfsetdash{}{0pt}%
\pgfpathmoveto{\pgfqpoint{3.706014in}{2.506090in}}%
\pgfpathcurveto{\pgfqpoint{3.711838in}{2.506090in}}{\pgfqpoint{3.717424in}{2.508403in}}{\pgfqpoint{3.721542in}{2.512522in}}%
\pgfpathcurveto{\pgfqpoint{3.725660in}{2.516640in}}{\pgfqpoint{3.727974in}{2.522226in}}{\pgfqpoint{3.727974in}{2.528050in}}%
\pgfpathcurveto{\pgfqpoint{3.727974in}{2.533874in}}{\pgfqpoint{3.725660in}{2.539460in}}{\pgfqpoint{3.721542in}{2.543578in}}%
\pgfpathcurveto{\pgfqpoint{3.717424in}{2.547696in}}{\pgfqpoint{3.711838in}{2.550010in}}{\pgfqpoint{3.706014in}{2.550010in}}%
\pgfpathcurveto{\pgfqpoint{3.700190in}{2.550010in}}{\pgfqpoint{3.694604in}{2.547696in}}{\pgfqpoint{3.690486in}{2.543578in}}%
\pgfpathcurveto{\pgfqpoint{3.686368in}{2.539460in}}{\pgfqpoint{3.684054in}{2.533874in}}{\pgfqpoint{3.684054in}{2.528050in}}%
\pgfpathcurveto{\pgfqpoint{3.684054in}{2.522226in}}{\pgfqpoint{3.686368in}{2.516640in}}{\pgfqpoint{3.690486in}{2.512522in}}%
\pgfpathcurveto{\pgfqpoint{3.694604in}{2.508403in}}{\pgfqpoint{3.700190in}{2.506090in}}{\pgfqpoint{3.706014in}{2.506090in}}%
\pgfpathlineto{\pgfqpoint{3.706014in}{2.506090in}}%
\pgfpathclose%
\pgfusepath{stroke,fill}%
\end{pgfscope}%
\begin{pgfscope}%
\pgfpathrectangle{\pgfqpoint{1.000000in}{1.148311in}}{\pgfqpoint{6.200000in}{5.623377in}}%
\pgfusepath{clip}%
\pgfsetbuttcap%
\pgfsetroundjoin%
\definecolor{currentfill}{rgb}{0.200000,0.800000,0.200000}%
\pgfsetfillcolor{currentfill}%
\pgfsetlinewidth{1.003750pt}%
\definecolor{currentstroke}{rgb}{0.200000,0.800000,0.200000}%
\pgfsetstrokecolor{currentstroke}%
\pgfsetdash{}{0pt}%
\pgfpathmoveto{\pgfqpoint{3.598359in}{2.551865in}}%
\pgfpathcurveto{\pgfqpoint{3.604183in}{2.551865in}}{\pgfqpoint{3.609769in}{2.554179in}}{\pgfqpoint{3.613887in}{2.558297in}}%
\pgfpathcurveto{\pgfqpoint{3.618006in}{2.562416in}}{\pgfqpoint{3.620319in}{2.568002in}}{\pgfqpoint{3.620319in}{2.573826in}}%
\pgfpathcurveto{\pgfqpoint{3.620319in}{2.579650in}}{\pgfqpoint{3.618006in}{2.585236in}}{\pgfqpoint{3.613887in}{2.589354in}}%
\pgfpathcurveto{\pgfqpoint{3.609769in}{2.593472in}}{\pgfqpoint{3.604183in}{2.595786in}}{\pgfqpoint{3.598359in}{2.595786in}}%
\pgfpathcurveto{\pgfqpoint{3.592535in}{2.595786in}}{\pgfqpoint{3.586949in}{2.593472in}}{\pgfqpoint{3.582831in}{2.589354in}}%
\pgfpathcurveto{\pgfqpoint{3.578713in}{2.585236in}}{\pgfqpoint{3.576399in}{2.579650in}}{\pgfqpoint{3.576399in}{2.573826in}}%
\pgfpathcurveto{\pgfqpoint{3.576399in}{2.568002in}}{\pgfqpoint{3.578713in}{2.562416in}}{\pgfqpoint{3.582831in}{2.558297in}}%
\pgfpathcurveto{\pgfqpoint{3.586949in}{2.554179in}}{\pgfqpoint{3.592535in}{2.551865in}}{\pgfqpoint{3.598359in}{2.551865in}}%
\pgfpathlineto{\pgfqpoint{3.598359in}{2.551865in}}%
\pgfpathclose%
\pgfusepath{stroke,fill}%
\end{pgfscope}%
\begin{pgfscope}%
\pgfpathrectangle{\pgfqpoint{1.000000in}{1.148311in}}{\pgfqpoint{6.200000in}{5.623377in}}%
\pgfusepath{clip}%
\pgfsetbuttcap%
\pgfsetroundjoin%
\definecolor{currentfill}{rgb}{0.200000,0.800000,0.200000}%
\pgfsetfillcolor{currentfill}%
\pgfsetlinewidth{1.003750pt}%
\definecolor{currentstroke}{rgb}{0.200000,0.800000,0.200000}%
\pgfsetstrokecolor{currentstroke}%
\pgfsetdash{}{0pt}%
\pgfpathmoveto{\pgfqpoint{3.654723in}{2.482496in}}%
\pgfpathcurveto{\pgfqpoint{3.660547in}{2.482496in}}{\pgfqpoint{3.666133in}{2.484809in}}{\pgfqpoint{3.670251in}{2.488928in}}%
\pgfpathcurveto{\pgfqpoint{3.674369in}{2.493046in}}{\pgfqpoint{3.676683in}{2.498632in}}{\pgfqpoint{3.676683in}{2.504456in}}%
\pgfpathcurveto{\pgfqpoint{3.676683in}{2.510280in}}{\pgfqpoint{3.674369in}{2.515866in}}{\pgfqpoint{3.670251in}{2.519984in}}%
\pgfpathcurveto{\pgfqpoint{3.666133in}{2.524102in}}{\pgfqpoint{3.660547in}{2.526416in}}{\pgfqpoint{3.654723in}{2.526416in}}%
\pgfpathcurveto{\pgfqpoint{3.648899in}{2.526416in}}{\pgfqpoint{3.643313in}{2.524102in}}{\pgfqpoint{3.639194in}{2.519984in}}%
\pgfpathcurveto{\pgfqpoint{3.635076in}{2.515866in}}{\pgfqpoint{3.632762in}{2.510280in}}{\pgfqpoint{3.632762in}{2.504456in}}%
\pgfpathcurveto{\pgfqpoint{3.632762in}{2.498632in}}{\pgfqpoint{3.635076in}{2.493046in}}{\pgfqpoint{3.639194in}{2.488928in}}%
\pgfpathcurveto{\pgfqpoint{3.643313in}{2.484809in}}{\pgfqpoint{3.648899in}{2.482496in}}{\pgfqpoint{3.654723in}{2.482496in}}%
\pgfpathlineto{\pgfqpoint{3.654723in}{2.482496in}}%
\pgfpathclose%
\pgfusepath{stroke,fill}%
\end{pgfscope}%
\begin{pgfscope}%
\pgfpathrectangle{\pgfqpoint{1.000000in}{1.148311in}}{\pgfqpoint{6.200000in}{5.623377in}}%
\pgfusepath{clip}%
\pgfsetbuttcap%
\pgfsetroundjoin%
\definecolor{currentfill}{rgb}{0.200000,0.800000,0.200000}%
\pgfsetfillcolor{currentfill}%
\pgfsetlinewidth{1.003750pt}%
\definecolor{currentstroke}{rgb}{0.200000,0.800000,0.200000}%
\pgfsetstrokecolor{currentstroke}%
\pgfsetdash{}{0pt}%
\pgfpathmoveto{\pgfqpoint{3.612587in}{2.475892in}}%
\pgfpathcurveto{\pgfqpoint{3.618411in}{2.475892in}}{\pgfqpoint{3.623997in}{2.478206in}}{\pgfqpoint{3.628115in}{2.482324in}}%
\pgfpathcurveto{\pgfqpoint{3.632233in}{2.486443in}}{\pgfqpoint{3.634547in}{2.492029in}}{\pgfqpoint{3.634547in}{2.497853in}}%
\pgfpathcurveto{\pgfqpoint{3.634547in}{2.503677in}}{\pgfqpoint{3.632233in}{2.509263in}}{\pgfqpoint{3.628115in}{2.513381in}}%
\pgfpathcurveto{\pgfqpoint{3.623997in}{2.517499in}}{\pgfqpoint{3.618411in}{2.519813in}}{\pgfqpoint{3.612587in}{2.519813in}}%
\pgfpathcurveto{\pgfqpoint{3.606763in}{2.519813in}}{\pgfqpoint{3.601177in}{2.517499in}}{\pgfqpoint{3.597059in}{2.513381in}}%
\pgfpathcurveto{\pgfqpoint{3.592941in}{2.509263in}}{\pgfqpoint{3.590627in}{2.503677in}}{\pgfqpoint{3.590627in}{2.497853in}}%
\pgfpathcurveto{\pgfqpoint{3.590627in}{2.492029in}}{\pgfqpoint{3.592941in}{2.486443in}}{\pgfqpoint{3.597059in}{2.482324in}}%
\pgfpathcurveto{\pgfqpoint{3.601177in}{2.478206in}}{\pgfqpoint{3.606763in}{2.475892in}}{\pgfqpoint{3.612587in}{2.475892in}}%
\pgfpathlineto{\pgfqpoint{3.612587in}{2.475892in}}%
\pgfpathclose%
\pgfusepath{stroke,fill}%
\end{pgfscope}%
\begin{pgfscope}%
\pgfpathrectangle{\pgfqpoint{1.000000in}{1.148311in}}{\pgfqpoint{6.200000in}{5.623377in}}%
\pgfusepath{clip}%
\pgfsetbuttcap%
\pgfsetroundjoin%
\definecolor{currentfill}{rgb}{0.200000,0.800000,0.200000}%
\pgfsetfillcolor{currentfill}%
\pgfsetlinewidth{1.003750pt}%
\definecolor{currentstroke}{rgb}{0.200000,0.800000,0.200000}%
\pgfsetstrokecolor{currentstroke}%
\pgfsetdash{}{0pt}%
\pgfpathmoveto{\pgfqpoint{3.545759in}{2.475329in}}%
\pgfpathcurveto{\pgfqpoint{3.551583in}{2.475329in}}{\pgfqpoint{3.557169in}{2.477643in}}{\pgfqpoint{3.561288in}{2.481761in}}%
\pgfpathcurveto{\pgfqpoint{3.565406in}{2.485879in}}{\pgfqpoint{3.567720in}{2.491465in}}{\pgfqpoint{3.567720in}{2.497289in}}%
\pgfpathcurveto{\pgfqpoint{3.567720in}{2.503113in}}{\pgfqpoint{3.565406in}{2.508699in}}{\pgfqpoint{3.561288in}{2.512817in}}%
\pgfpathcurveto{\pgfqpoint{3.557169in}{2.516936in}}{\pgfqpoint{3.551583in}{2.519249in}}{\pgfqpoint{3.545759in}{2.519249in}}%
\pgfpathcurveto{\pgfqpoint{3.539935in}{2.519249in}}{\pgfqpoint{3.534349in}{2.516936in}}{\pgfqpoint{3.530231in}{2.512817in}}%
\pgfpathcurveto{\pgfqpoint{3.526113in}{2.508699in}}{\pgfqpoint{3.523799in}{2.503113in}}{\pgfqpoint{3.523799in}{2.497289in}}%
\pgfpathcurveto{\pgfqpoint{3.523799in}{2.491465in}}{\pgfqpoint{3.526113in}{2.485879in}}{\pgfqpoint{3.530231in}{2.481761in}}%
\pgfpathcurveto{\pgfqpoint{3.534349in}{2.477643in}}{\pgfqpoint{3.539935in}{2.475329in}}{\pgfqpoint{3.545759in}{2.475329in}}%
\pgfpathlineto{\pgfqpoint{3.545759in}{2.475329in}}%
\pgfpathclose%
\pgfusepath{stroke,fill}%
\end{pgfscope}%
\begin{pgfscope}%
\pgfpathrectangle{\pgfqpoint{1.000000in}{1.148311in}}{\pgfqpoint{6.200000in}{5.623377in}}%
\pgfusepath{clip}%
\pgfsetbuttcap%
\pgfsetroundjoin%
\definecolor{currentfill}{rgb}{0.200000,0.800000,0.200000}%
\pgfsetfillcolor{currentfill}%
\pgfsetlinewidth{1.003750pt}%
\definecolor{currentstroke}{rgb}{0.200000,0.800000,0.200000}%
\pgfsetstrokecolor{currentstroke}%
\pgfsetdash{}{0pt}%
\pgfpathmoveto{\pgfqpoint{3.533749in}{2.446580in}}%
\pgfpathcurveto{\pgfqpoint{3.539573in}{2.446580in}}{\pgfqpoint{3.545159in}{2.448894in}}{\pgfqpoint{3.549277in}{2.453012in}}%
\pgfpathcurveto{\pgfqpoint{3.553395in}{2.457130in}}{\pgfqpoint{3.555709in}{2.462716in}}{\pgfqpoint{3.555709in}{2.468540in}}%
\pgfpathcurveto{\pgfqpoint{3.555709in}{2.474364in}}{\pgfqpoint{3.553395in}{2.479950in}}{\pgfqpoint{3.549277in}{2.484068in}}%
\pgfpathcurveto{\pgfqpoint{3.545159in}{2.488186in}}{\pgfqpoint{3.539573in}{2.490500in}}{\pgfqpoint{3.533749in}{2.490500in}}%
\pgfpathcurveto{\pgfqpoint{3.527925in}{2.490500in}}{\pgfqpoint{3.522339in}{2.488186in}}{\pgfqpoint{3.518221in}{2.484068in}}%
\pgfpathcurveto{\pgfqpoint{3.514103in}{2.479950in}}{\pgfqpoint{3.511789in}{2.474364in}}{\pgfqpoint{3.511789in}{2.468540in}}%
\pgfpathcurveto{\pgfqpoint{3.511789in}{2.462716in}}{\pgfqpoint{3.514103in}{2.457130in}}{\pgfqpoint{3.518221in}{2.453012in}}%
\pgfpathcurveto{\pgfqpoint{3.522339in}{2.448894in}}{\pgfqpoint{3.527925in}{2.446580in}}{\pgfqpoint{3.533749in}{2.446580in}}%
\pgfpathlineto{\pgfqpoint{3.533749in}{2.446580in}}%
\pgfpathclose%
\pgfusepath{stroke,fill}%
\end{pgfscope}%
\begin{pgfscope}%
\pgfpathrectangle{\pgfqpoint{1.000000in}{1.148311in}}{\pgfqpoint{6.200000in}{5.623377in}}%
\pgfusepath{clip}%
\pgfsetbuttcap%
\pgfsetroundjoin%
\definecolor{currentfill}{rgb}{0.200000,0.800000,0.200000}%
\pgfsetfillcolor{currentfill}%
\pgfsetlinewidth{1.003750pt}%
\definecolor{currentstroke}{rgb}{0.200000,0.800000,0.200000}%
\pgfsetstrokecolor{currentstroke}%
\pgfsetdash{}{0pt}%
\pgfpathmoveto{\pgfqpoint{3.510978in}{2.420798in}}%
\pgfpathcurveto{\pgfqpoint{3.516802in}{2.420798in}}{\pgfqpoint{3.522388in}{2.423112in}}{\pgfqpoint{3.526506in}{2.427230in}}%
\pgfpathcurveto{\pgfqpoint{3.530624in}{2.431348in}}{\pgfqpoint{3.532938in}{2.436934in}}{\pgfqpoint{3.532938in}{2.442758in}}%
\pgfpathcurveto{\pgfqpoint{3.532938in}{2.448582in}}{\pgfqpoint{3.530624in}{2.454169in}}{\pgfqpoint{3.526506in}{2.458287in}}%
\pgfpathcurveto{\pgfqpoint{3.522388in}{2.462405in}}{\pgfqpoint{3.516802in}{2.464719in}}{\pgfqpoint{3.510978in}{2.464719in}}%
\pgfpathcurveto{\pgfqpoint{3.505154in}{2.464719in}}{\pgfqpoint{3.499568in}{2.462405in}}{\pgfqpoint{3.495450in}{2.458287in}}%
\pgfpathcurveto{\pgfqpoint{3.491331in}{2.454169in}}{\pgfqpoint{3.489018in}{2.448582in}}{\pgfqpoint{3.489018in}{2.442758in}}%
\pgfpathcurveto{\pgfqpoint{3.489018in}{2.436934in}}{\pgfqpoint{3.491331in}{2.431348in}}{\pgfqpoint{3.495450in}{2.427230in}}%
\pgfpathcurveto{\pgfqpoint{3.499568in}{2.423112in}}{\pgfqpoint{3.505154in}{2.420798in}}{\pgfqpoint{3.510978in}{2.420798in}}%
\pgfpathlineto{\pgfqpoint{3.510978in}{2.420798in}}%
\pgfpathclose%
\pgfusepath{stroke,fill}%
\end{pgfscope}%
\begin{pgfscope}%
\pgfpathrectangle{\pgfqpoint{1.000000in}{1.148311in}}{\pgfqpoint{6.200000in}{5.623377in}}%
\pgfusepath{clip}%
\pgfsetbuttcap%
\pgfsetroundjoin%
\definecolor{currentfill}{rgb}{0.200000,0.800000,0.200000}%
\pgfsetfillcolor{currentfill}%
\pgfsetlinewidth{1.003750pt}%
\definecolor{currentstroke}{rgb}{0.200000,0.800000,0.200000}%
\pgfsetstrokecolor{currentstroke}%
\pgfsetdash{}{0pt}%
\pgfpathmoveto{\pgfqpoint{3.626287in}{2.361933in}}%
\pgfpathcurveto{\pgfqpoint{3.632111in}{2.361933in}}{\pgfqpoint{3.637697in}{2.364247in}}{\pgfqpoint{3.641816in}{2.368365in}}%
\pgfpathcurveto{\pgfqpoint{3.645934in}{2.372483in}}{\pgfqpoint{3.648248in}{2.378069in}}{\pgfqpoint{3.648248in}{2.383893in}}%
\pgfpathcurveto{\pgfqpoint{3.648248in}{2.389717in}}{\pgfqpoint{3.645934in}{2.395303in}}{\pgfqpoint{3.641816in}{2.399421in}}%
\pgfpathcurveto{\pgfqpoint{3.637697in}{2.403540in}}{\pgfqpoint{3.632111in}{2.405853in}}{\pgfqpoint{3.626287in}{2.405853in}}%
\pgfpathcurveto{\pgfqpoint{3.620463in}{2.405853in}}{\pgfqpoint{3.614877in}{2.403540in}}{\pgfqpoint{3.610759in}{2.399421in}}%
\pgfpathcurveto{\pgfqpoint{3.606641in}{2.395303in}}{\pgfqpoint{3.604327in}{2.389717in}}{\pgfqpoint{3.604327in}{2.383893in}}%
\pgfpathcurveto{\pgfqpoint{3.604327in}{2.378069in}}{\pgfqpoint{3.606641in}{2.372483in}}{\pgfqpoint{3.610759in}{2.368365in}}%
\pgfpathcurveto{\pgfqpoint{3.614877in}{2.364247in}}{\pgfqpoint{3.620463in}{2.361933in}}{\pgfqpoint{3.626287in}{2.361933in}}%
\pgfpathlineto{\pgfqpoint{3.626287in}{2.361933in}}%
\pgfpathclose%
\pgfusepath{stroke,fill}%
\end{pgfscope}%
\begin{pgfscope}%
\pgfpathrectangle{\pgfqpoint{1.000000in}{1.148311in}}{\pgfqpoint{6.200000in}{5.623377in}}%
\pgfusepath{clip}%
\pgfsetbuttcap%
\pgfsetroundjoin%
\definecolor{currentfill}{rgb}{0.200000,0.800000,0.200000}%
\pgfsetfillcolor{currentfill}%
\pgfsetlinewidth{1.003750pt}%
\definecolor{currentstroke}{rgb}{0.200000,0.800000,0.200000}%
\pgfsetstrokecolor{currentstroke}%
\pgfsetdash{}{0pt}%
\pgfpathmoveto{\pgfqpoint{3.531236in}{2.352843in}}%
\pgfpathcurveto{\pgfqpoint{3.537060in}{2.352843in}}{\pgfqpoint{3.542646in}{2.355157in}}{\pgfqpoint{3.546764in}{2.359275in}}%
\pgfpathcurveto{\pgfqpoint{3.550883in}{2.363393in}}{\pgfqpoint{3.553196in}{2.368979in}}{\pgfqpoint{3.553196in}{2.374803in}}%
\pgfpathcurveto{\pgfqpoint{3.553196in}{2.380627in}}{\pgfqpoint{3.550883in}{2.386213in}}{\pgfqpoint{3.546764in}{2.390331in}}%
\pgfpathcurveto{\pgfqpoint{3.542646in}{2.394449in}}{\pgfqpoint{3.537060in}{2.396763in}}{\pgfqpoint{3.531236in}{2.396763in}}%
\pgfpathcurveto{\pgfqpoint{3.525412in}{2.396763in}}{\pgfqpoint{3.519826in}{2.394449in}}{\pgfqpoint{3.515708in}{2.390331in}}%
\pgfpathcurveto{\pgfqpoint{3.511590in}{2.386213in}}{\pgfqpoint{3.509276in}{2.380627in}}{\pgfqpoint{3.509276in}{2.374803in}}%
\pgfpathcurveto{\pgfqpoint{3.509276in}{2.368979in}}{\pgfqpoint{3.511590in}{2.363393in}}{\pgfqpoint{3.515708in}{2.359275in}}%
\pgfpathcurveto{\pgfqpoint{3.519826in}{2.355157in}}{\pgfqpoint{3.525412in}{2.352843in}}{\pgfqpoint{3.531236in}{2.352843in}}%
\pgfpathlineto{\pgfqpoint{3.531236in}{2.352843in}}%
\pgfpathclose%
\pgfusepath{stroke,fill}%
\end{pgfscope}%
\begin{pgfscope}%
\pgfpathrectangle{\pgfqpoint{1.000000in}{1.148311in}}{\pgfqpoint{6.200000in}{5.623377in}}%
\pgfusepath{clip}%
\pgfsetbuttcap%
\pgfsetroundjoin%
\definecolor{currentfill}{rgb}{0.200000,0.800000,0.200000}%
\pgfsetfillcolor{currentfill}%
\pgfsetlinewidth{1.003750pt}%
\definecolor{currentstroke}{rgb}{0.200000,0.800000,0.200000}%
\pgfsetstrokecolor{currentstroke}%
\pgfsetdash{}{0pt}%
\pgfpathmoveto{\pgfqpoint{3.549278in}{2.321177in}}%
\pgfpathcurveto{\pgfqpoint{3.555102in}{2.321177in}}{\pgfqpoint{3.560688in}{2.323491in}}{\pgfqpoint{3.564806in}{2.327609in}}%
\pgfpathcurveto{\pgfqpoint{3.568924in}{2.331727in}}{\pgfqpoint{3.571238in}{2.337313in}}{\pgfqpoint{3.571238in}{2.343137in}}%
\pgfpathcurveto{\pgfqpoint{3.571238in}{2.348961in}}{\pgfqpoint{3.568924in}{2.354547in}}{\pgfqpoint{3.564806in}{2.358665in}}%
\pgfpathcurveto{\pgfqpoint{3.560688in}{2.362783in}}{\pgfqpoint{3.555102in}{2.365097in}}{\pgfqpoint{3.549278in}{2.365097in}}%
\pgfpathcurveto{\pgfqpoint{3.543454in}{2.365097in}}{\pgfqpoint{3.537867in}{2.362783in}}{\pgfqpoint{3.533749in}{2.358665in}}%
\pgfpathcurveto{\pgfqpoint{3.529631in}{2.354547in}}{\pgfqpoint{3.527317in}{2.348961in}}{\pgfqpoint{3.527317in}{2.343137in}}%
\pgfpathcurveto{\pgfqpoint{3.527317in}{2.337313in}}{\pgfqpoint{3.529631in}{2.331727in}}{\pgfqpoint{3.533749in}{2.327609in}}%
\pgfpathcurveto{\pgfqpoint{3.537867in}{2.323491in}}{\pgfqpoint{3.543454in}{2.321177in}}{\pgfqpoint{3.549278in}{2.321177in}}%
\pgfpathlineto{\pgfqpoint{3.549278in}{2.321177in}}%
\pgfpathclose%
\pgfusepath{stroke,fill}%
\end{pgfscope}%
\begin{pgfscope}%
\pgfpathrectangle{\pgfqpoint{1.000000in}{1.148311in}}{\pgfqpoint{6.200000in}{5.623377in}}%
\pgfusepath{clip}%
\pgfsetbuttcap%
\pgfsetroundjoin%
\definecolor{currentfill}{rgb}{0.200000,0.800000,0.200000}%
\pgfsetfillcolor{currentfill}%
\pgfsetlinewidth{1.003750pt}%
\definecolor{currentstroke}{rgb}{0.200000,0.800000,0.200000}%
\pgfsetstrokecolor{currentstroke}%
\pgfsetdash{}{0pt}%
\pgfpathmoveto{\pgfqpoint{3.498185in}{2.294367in}}%
\pgfpathcurveto{\pgfqpoint{3.504009in}{2.294367in}}{\pgfqpoint{3.509595in}{2.296680in}}{\pgfqpoint{3.513713in}{2.300799in}}%
\pgfpathcurveto{\pgfqpoint{3.517831in}{2.304917in}}{\pgfqpoint{3.520145in}{2.310503in}}{\pgfqpoint{3.520145in}{2.316327in}}%
\pgfpathcurveto{\pgfqpoint{3.520145in}{2.322151in}}{\pgfqpoint{3.517831in}{2.327737in}}{\pgfqpoint{3.513713in}{2.331855in}}%
\pgfpathcurveto{\pgfqpoint{3.509595in}{2.335973in}}{\pgfqpoint{3.504009in}{2.338287in}}{\pgfqpoint{3.498185in}{2.338287in}}%
\pgfpathcurveto{\pgfqpoint{3.492361in}{2.338287in}}{\pgfqpoint{3.486775in}{2.335973in}}{\pgfqpoint{3.482657in}{2.331855in}}%
\pgfpathcurveto{\pgfqpoint{3.478539in}{2.327737in}}{\pgfqpoint{3.476225in}{2.322151in}}{\pgfqpoint{3.476225in}{2.316327in}}%
\pgfpathcurveto{\pgfqpoint{3.476225in}{2.310503in}}{\pgfqpoint{3.478539in}{2.304917in}}{\pgfqpoint{3.482657in}{2.300799in}}%
\pgfpathcurveto{\pgfqpoint{3.486775in}{2.296680in}}{\pgfqpoint{3.492361in}{2.294367in}}{\pgfqpoint{3.498185in}{2.294367in}}%
\pgfpathlineto{\pgfqpoint{3.498185in}{2.294367in}}%
\pgfpathclose%
\pgfusepath{stroke,fill}%
\end{pgfscope}%
\begin{pgfscope}%
\pgfpathrectangle{\pgfqpoint{1.000000in}{1.148311in}}{\pgfqpoint{6.200000in}{5.623377in}}%
\pgfusepath{clip}%
\pgfsetbuttcap%
\pgfsetroundjoin%
\definecolor{currentfill}{rgb}{0.200000,0.800000,0.200000}%
\pgfsetfillcolor{currentfill}%
\pgfsetlinewidth{1.003750pt}%
\definecolor{currentstroke}{rgb}{0.200000,0.800000,0.200000}%
\pgfsetstrokecolor{currentstroke}%
\pgfsetdash{}{0pt}%
\pgfpathmoveto{\pgfqpoint{3.478874in}{2.262192in}}%
\pgfpathcurveto{\pgfqpoint{3.484697in}{2.262192in}}{\pgfqpoint{3.490284in}{2.264506in}}{\pgfqpoint{3.494402in}{2.268624in}}%
\pgfpathcurveto{\pgfqpoint{3.498520in}{2.272742in}}{\pgfqpoint{3.500834in}{2.278329in}}{\pgfqpoint{3.500834in}{2.284152in}}%
\pgfpathcurveto{\pgfqpoint{3.500834in}{2.289976in}}{\pgfqpoint{3.498520in}{2.295563in}}{\pgfqpoint{3.494402in}{2.299681in}}%
\pgfpathcurveto{\pgfqpoint{3.490284in}{2.303799in}}{\pgfqpoint{3.484697in}{2.306113in}}{\pgfqpoint{3.478874in}{2.306113in}}%
\pgfpathcurveto{\pgfqpoint{3.473050in}{2.306113in}}{\pgfqpoint{3.467463in}{2.303799in}}{\pgfqpoint{3.463345in}{2.299681in}}%
\pgfpathcurveto{\pgfqpoint{3.459227in}{2.295563in}}{\pgfqpoint{3.456913in}{2.289976in}}{\pgfqpoint{3.456913in}{2.284152in}}%
\pgfpathcurveto{\pgfqpoint{3.456913in}{2.278329in}}{\pgfqpoint{3.459227in}{2.272742in}}{\pgfqpoint{3.463345in}{2.268624in}}%
\pgfpathcurveto{\pgfqpoint{3.467463in}{2.264506in}}{\pgfqpoint{3.473050in}{2.262192in}}{\pgfqpoint{3.478874in}{2.262192in}}%
\pgfpathlineto{\pgfqpoint{3.478874in}{2.262192in}}%
\pgfpathclose%
\pgfusepath{stroke,fill}%
\end{pgfscope}%
\begin{pgfscope}%
\pgfpathrectangle{\pgfqpoint{1.000000in}{1.148311in}}{\pgfqpoint{6.200000in}{5.623377in}}%
\pgfusepath{clip}%
\pgfsetbuttcap%
\pgfsetroundjoin%
\definecolor{currentfill}{rgb}{0.200000,0.800000,0.200000}%
\pgfsetfillcolor{currentfill}%
\pgfsetlinewidth{1.003750pt}%
\definecolor{currentstroke}{rgb}{0.200000,0.800000,0.200000}%
\pgfsetstrokecolor{currentstroke}%
\pgfsetdash{}{0pt}%
\pgfpathmoveto{\pgfqpoint{3.590698in}{2.239950in}}%
\pgfpathcurveto{\pgfqpoint{3.596522in}{2.239950in}}{\pgfqpoint{3.602108in}{2.242264in}}{\pgfqpoint{3.606226in}{2.246382in}}%
\pgfpathcurveto{\pgfqpoint{3.610344in}{2.250500in}}{\pgfqpoint{3.612658in}{2.256087in}}{\pgfqpoint{3.612658in}{2.261911in}}%
\pgfpathcurveto{\pgfqpoint{3.612658in}{2.267734in}}{\pgfqpoint{3.610344in}{2.273321in}}{\pgfqpoint{3.606226in}{2.277439in}}%
\pgfpathcurveto{\pgfqpoint{3.602108in}{2.281557in}}{\pgfqpoint{3.596522in}{2.283871in}}{\pgfqpoint{3.590698in}{2.283871in}}%
\pgfpathcurveto{\pgfqpoint{3.584874in}{2.283871in}}{\pgfqpoint{3.579288in}{2.281557in}}{\pgfqpoint{3.575170in}{2.277439in}}%
\pgfpathcurveto{\pgfqpoint{3.571052in}{2.273321in}}{\pgfqpoint{3.568738in}{2.267734in}}{\pgfqpoint{3.568738in}{2.261911in}}%
\pgfpathcurveto{\pgfqpoint{3.568738in}{2.256087in}}{\pgfqpoint{3.571052in}{2.250500in}}{\pgfqpoint{3.575170in}{2.246382in}}%
\pgfpathcurveto{\pgfqpoint{3.579288in}{2.242264in}}{\pgfqpoint{3.584874in}{2.239950in}}{\pgfqpoint{3.590698in}{2.239950in}}%
\pgfpathlineto{\pgfqpoint{3.590698in}{2.239950in}}%
\pgfpathclose%
\pgfusepath{stroke,fill}%
\end{pgfscope}%
\begin{pgfscope}%
\pgfpathrectangle{\pgfqpoint{1.000000in}{1.148311in}}{\pgfqpoint{6.200000in}{5.623377in}}%
\pgfusepath{clip}%
\pgfsetbuttcap%
\pgfsetroundjoin%
\definecolor{currentfill}{rgb}{0.200000,0.800000,0.200000}%
\pgfsetfillcolor{currentfill}%
\pgfsetlinewidth{1.003750pt}%
\definecolor{currentstroke}{rgb}{0.200000,0.800000,0.200000}%
\pgfsetstrokecolor{currentstroke}%
\pgfsetdash{}{0pt}%
\pgfpathmoveto{\pgfqpoint{3.542185in}{2.206081in}}%
\pgfpathcurveto{\pgfqpoint{3.548009in}{2.206081in}}{\pgfqpoint{3.553595in}{2.208395in}}{\pgfqpoint{3.557713in}{2.212513in}}%
\pgfpathcurveto{\pgfqpoint{3.561831in}{2.216631in}}{\pgfqpoint{3.564145in}{2.222217in}}{\pgfqpoint{3.564145in}{2.228041in}}%
\pgfpathcurveto{\pgfqpoint{3.564145in}{2.233865in}}{\pgfqpoint{3.561831in}{2.239451in}}{\pgfqpoint{3.557713in}{2.243569in}}%
\pgfpathcurveto{\pgfqpoint{3.553595in}{2.247688in}}{\pgfqpoint{3.548009in}{2.250001in}}{\pgfqpoint{3.542185in}{2.250001in}}%
\pgfpathcurveto{\pgfqpoint{3.536361in}{2.250001in}}{\pgfqpoint{3.530775in}{2.247688in}}{\pgfqpoint{3.526657in}{2.243569in}}%
\pgfpathcurveto{\pgfqpoint{3.522538in}{2.239451in}}{\pgfqpoint{3.520225in}{2.233865in}}{\pgfqpoint{3.520225in}{2.228041in}}%
\pgfpathcurveto{\pgfqpoint{3.520225in}{2.222217in}}{\pgfqpoint{3.522538in}{2.216631in}}{\pgfqpoint{3.526657in}{2.212513in}}%
\pgfpathcurveto{\pgfqpoint{3.530775in}{2.208395in}}{\pgfqpoint{3.536361in}{2.206081in}}{\pgfqpoint{3.542185in}{2.206081in}}%
\pgfpathlineto{\pgfqpoint{3.542185in}{2.206081in}}%
\pgfpathclose%
\pgfusepath{stroke,fill}%
\end{pgfscope}%
\begin{pgfscope}%
\pgfpathrectangle{\pgfqpoint{1.000000in}{1.148311in}}{\pgfqpoint{6.200000in}{5.623377in}}%
\pgfusepath{clip}%
\pgfsetbuttcap%
\pgfsetroundjoin%
\definecolor{currentfill}{rgb}{0.200000,0.800000,0.200000}%
\pgfsetfillcolor{currentfill}%
\pgfsetlinewidth{1.003750pt}%
\definecolor{currentstroke}{rgb}{0.200000,0.800000,0.200000}%
\pgfsetstrokecolor{currentstroke}%
\pgfsetdash{}{0pt}%
\pgfpathmoveto{\pgfqpoint{3.516162in}{2.170366in}}%
\pgfpathcurveto{\pgfqpoint{3.521986in}{2.170366in}}{\pgfqpoint{3.527572in}{2.172680in}}{\pgfqpoint{3.531690in}{2.176798in}}%
\pgfpathcurveto{\pgfqpoint{3.535808in}{2.180916in}}{\pgfqpoint{3.538122in}{2.186502in}}{\pgfqpoint{3.538122in}{2.192326in}}%
\pgfpathcurveto{\pgfqpoint{3.538122in}{2.198150in}}{\pgfqpoint{3.535808in}{2.203736in}}{\pgfqpoint{3.531690in}{2.207854in}}%
\pgfpathcurveto{\pgfqpoint{3.527572in}{2.211973in}}{\pgfqpoint{3.521986in}{2.214286in}}{\pgfqpoint{3.516162in}{2.214286in}}%
\pgfpathcurveto{\pgfqpoint{3.510338in}{2.214286in}}{\pgfqpoint{3.504752in}{2.211973in}}{\pgfqpoint{3.500634in}{2.207854in}}%
\pgfpathcurveto{\pgfqpoint{3.496516in}{2.203736in}}{\pgfqpoint{3.494202in}{2.198150in}}{\pgfqpoint{3.494202in}{2.192326in}}%
\pgfpathcurveto{\pgfqpoint{3.494202in}{2.186502in}}{\pgfqpoint{3.496516in}{2.180916in}}{\pgfqpoint{3.500634in}{2.176798in}}%
\pgfpathcurveto{\pgfqpoint{3.504752in}{2.172680in}}{\pgfqpoint{3.510338in}{2.170366in}}{\pgfqpoint{3.516162in}{2.170366in}}%
\pgfpathlineto{\pgfqpoint{3.516162in}{2.170366in}}%
\pgfpathclose%
\pgfusepath{stroke,fill}%
\end{pgfscope}%
\begin{pgfscope}%
\pgfpathrectangle{\pgfqpoint{1.000000in}{1.148311in}}{\pgfqpoint{6.200000in}{5.623377in}}%
\pgfusepath{clip}%
\pgfsetbuttcap%
\pgfsetroundjoin%
\definecolor{currentfill}{rgb}{0.200000,0.800000,0.200000}%
\pgfsetfillcolor{currentfill}%
\pgfsetlinewidth{1.003750pt}%
\definecolor{currentstroke}{rgb}{0.200000,0.800000,0.200000}%
\pgfsetstrokecolor{currentstroke}%
\pgfsetdash{}{0pt}%
\pgfpathmoveto{\pgfqpoint{3.568448in}{2.153248in}}%
\pgfpathcurveto{\pgfqpoint{3.574272in}{2.153248in}}{\pgfqpoint{3.579859in}{2.155562in}}{\pgfqpoint{3.583977in}{2.159680in}}%
\pgfpathcurveto{\pgfqpoint{3.588095in}{2.163799in}}{\pgfqpoint{3.590409in}{2.169385in}}{\pgfqpoint{3.590409in}{2.175209in}}%
\pgfpathcurveto{\pgfqpoint{3.590409in}{2.181033in}}{\pgfqpoint{3.588095in}{2.186619in}}{\pgfqpoint{3.583977in}{2.190737in}}%
\pgfpathcurveto{\pgfqpoint{3.579859in}{2.194855in}}{\pgfqpoint{3.574272in}{2.197169in}}{\pgfqpoint{3.568448in}{2.197169in}}%
\pgfpathcurveto{\pgfqpoint{3.562624in}{2.197169in}}{\pgfqpoint{3.557038in}{2.194855in}}{\pgfqpoint{3.552920in}{2.190737in}}%
\pgfpathcurveto{\pgfqpoint{3.548802in}{2.186619in}}{\pgfqpoint{3.546488in}{2.181033in}}{\pgfqpoint{3.546488in}{2.175209in}}%
\pgfpathcurveto{\pgfqpoint{3.546488in}{2.169385in}}{\pgfqpoint{3.548802in}{2.163799in}}{\pgfqpoint{3.552920in}{2.159680in}}%
\pgfpathcurveto{\pgfqpoint{3.557038in}{2.155562in}}{\pgfqpoint{3.562624in}{2.153248in}}{\pgfqpoint{3.568448in}{2.153248in}}%
\pgfpathlineto{\pgfqpoint{3.568448in}{2.153248in}}%
\pgfpathclose%
\pgfusepath{stroke,fill}%
\end{pgfscope}%
\begin{pgfscope}%
\pgfpathrectangle{\pgfqpoint{1.000000in}{1.148311in}}{\pgfqpoint{6.200000in}{5.623377in}}%
\pgfusepath{clip}%
\pgfsetbuttcap%
\pgfsetroundjoin%
\definecolor{currentfill}{rgb}{0.200000,0.800000,0.200000}%
\pgfsetfillcolor{currentfill}%
\pgfsetlinewidth{1.003750pt}%
\definecolor{currentstroke}{rgb}{0.200000,0.800000,0.200000}%
\pgfsetstrokecolor{currentstroke}%
\pgfsetdash{}{0pt}%
\pgfpathmoveto{\pgfqpoint{3.499756in}{2.098220in}}%
\pgfpathcurveto{\pgfqpoint{3.505580in}{2.098220in}}{\pgfqpoint{3.511166in}{2.100534in}}{\pgfqpoint{3.515285in}{2.104652in}}%
\pgfpathcurveto{\pgfqpoint{3.519403in}{2.108770in}}{\pgfqpoint{3.521717in}{2.114356in}}{\pgfqpoint{3.521717in}{2.120180in}}%
\pgfpathcurveto{\pgfqpoint{3.521717in}{2.126004in}}{\pgfqpoint{3.519403in}{2.131590in}}{\pgfqpoint{3.515285in}{2.135708in}}%
\pgfpathcurveto{\pgfqpoint{3.511166in}{2.139826in}}{\pgfqpoint{3.505580in}{2.142140in}}{\pgfqpoint{3.499756in}{2.142140in}}%
\pgfpathcurveto{\pgfqpoint{3.493932in}{2.142140in}}{\pgfqpoint{3.488346in}{2.139826in}}{\pgfqpoint{3.484228in}{2.135708in}}%
\pgfpathcurveto{\pgfqpoint{3.480110in}{2.131590in}}{\pgfqpoint{3.477796in}{2.126004in}}{\pgfqpoint{3.477796in}{2.120180in}}%
\pgfpathcurveto{\pgfqpoint{3.477796in}{2.114356in}}{\pgfqpoint{3.480110in}{2.108770in}}{\pgfqpoint{3.484228in}{2.104652in}}%
\pgfpathcurveto{\pgfqpoint{3.488346in}{2.100534in}}{\pgfqpoint{3.493932in}{2.098220in}}{\pgfqpoint{3.499756in}{2.098220in}}%
\pgfpathlineto{\pgfqpoint{3.499756in}{2.098220in}}%
\pgfpathclose%
\pgfusepath{stroke,fill}%
\end{pgfscope}%
\begin{pgfscope}%
\pgfpathrectangle{\pgfqpoint{1.000000in}{1.148311in}}{\pgfqpoint{6.200000in}{5.623377in}}%
\pgfusepath{clip}%
\pgfsetbuttcap%
\pgfsetroundjoin%
\definecolor{currentfill}{rgb}{0.200000,0.800000,0.200000}%
\pgfsetfillcolor{currentfill}%
\pgfsetlinewidth{1.003750pt}%
\definecolor{currentstroke}{rgb}{0.200000,0.800000,0.200000}%
\pgfsetstrokecolor{currentstroke}%
\pgfsetdash{}{0pt}%
\pgfpathmoveto{\pgfqpoint{3.590051in}{2.101227in}}%
\pgfpathcurveto{\pgfqpoint{3.595875in}{2.101227in}}{\pgfqpoint{3.601461in}{2.103541in}}{\pgfqpoint{3.605579in}{2.107659in}}%
\pgfpathcurveto{\pgfqpoint{3.609697in}{2.111777in}}{\pgfqpoint{3.612011in}{2.117363in}}{\pgfqpoint{3.612011in}{2.123187in}}%
\pgfpathcurveto{\pgfqpoint{3.612011in}{2.129011in}}{\pgfqpoint{3.609697in}{2.134597in}}{\pgfqpoint{3.605579in}{2.138715in}}%
\pgfpathcurveto{\pgfqpoint{3.601461in}{2.142833in}}{\pgfqpoint{3.595875in}{2.145147in}}{\pgfqpoint{3.590051in}{2.145147in}}%
\pgfpathcurveto{\pgfqpoint{3.584227in}{2.145147in}}{\pgfqpoint{3.578641in}{2.142833in}}{\pgfqpoint{3.574523in}{2.138715in}}%
\pgfpathcurveto{\pgfqpoint{3.570404in}{2.134597in}}{\pgfqpoint{3.568091in}{2.129011in}}{\pgfqpoint{3.568091in}{2.123187in}}%
\pgfpathcurveto{\pgfqpoint{3.568091in}{2.117363in}}{\pgfqpoint{3.570404in}{2.111777in}}{\pgfqpoint{3.574523in}{2.107659in}}%
\pgfpathcurveto{\pgfqpoint{3.578641in}{2.103541in}}{\pgfqpoint{3.584227in}{2.101227in}}{\pgfqpoint{3.590051in}{2.101227in}}%
\pgfpathlineto{\pgfqpoint{3.590051in}{2.101227in}}%
\pgfpathclose%
\pgfusepath{stroke,fill}%
\end{pgfscope}%
\begin{pgfscope}%
\pgfpathrectangle{\pgfqpoint{1.000000in}{1.148311in}}{\pgfqpoint{6.200000in}{5.623377in}}%
\pgfusepath{clip}%
\pgfsetbuttcap%
\pgfsetroundjoin%
\definecolor{currentfill}{rgb}{0.200000,0.800000,0.200000}%
\pgfsetfillcolor{currentfill}%
\pgfsetlinewidth{1.003750pt}%
\definecolor{currentstroke}{rgb}{0.200000,0.800000,0.200000}%
\pgfsetstrokecolor{currentstroke}%
\pgfsetdash{}{0pt}%
\pgfpathmoveto{\pgfqpoint{3.598553in}{2.074044in}}%
\pgfpathcurveto{\pgfqpoint{3.604377in}{2.074044in}}{\pgfqpoint{3.609963in}{2.076358in}}{\pgfqpoint{3.614081in}{2.080476in}}%
\pgfpathcurveto{\pgfqpoint{3.618199in}{2.084594in}}{\pgfqpoint{3.620513in}{2.090181in}}{\pgfqpoint{3.620513in}{2.096004in}}%
\pgfpathcurveto{\pgfqpoint{3.620513in}{2.101828in}}{\pgfqpoint{3.618199in}{2.107415in}}{\pgfqpoint{3.614081in}{2.111533in}}%
\pgfpathcurveto{\pgfqpoint{3.609963in}{2.115651in}}{\pgfqpoint{3.604377in}{2.117965in}}{\pgfqpoint{3.598553in}{2.117965in}}%
\pgfpathcurveto{\pgfqpoint{3.592729in}{2.117965in}}{\pgfqpoint{3.587143in}{2.115651in}}{\pgfqpoint{3.583025in}{2.111533in}}%
\pgfpathcurveto{\pgfqpoint{3.578906in}{2.107415in}}{\pgfqpoint{3.576593in}{2.101828in}}{\pgfqpoint{3.576593in}{2.096004in}}%
\pgfpathcurveto{\pgfqpoint{3.576593in}{2.090181in}}{\pgfqpoint{3.578906in}{2.084594in}}{\pgfqpoint{3.583025in}{2.080476in}}%
\pgfpathcurveto{\pgfqpoint{3.587143in}{2.076358in}}{\pgfqpoint{3.592729in}{2.074044in}}{\pgfqpoint{3.598553in}{2.074044in}}%
\pgfpathlineto{\pgfqpoint{3.598553in}{2.074044in}}%
\pgfpathclose%
\pgfusepath{stroke,fill}%
\end{pgfscope}%
\begin{pgfscope}%
\pgfpathrectangle{\pgfqpoint{1.000000in}{1.148311in}}{\pgfqpoint{6.200000in}{5.623377in}}%
\pgfusepath{clip}%
\pgfsetbuttcap%
\pgfsetroundjoin%
\definecolor{currentfill}{rgb}{0.200000,0.800000,0.200000}%
\pgfsetfillcolor{currentfill}%
\pgfsetlinewidth{1.003750pt}%
\definecolor{currentstroke}{rgb}{0.200000,0.800000,0.200000}%
\pgfsetstrokecolor{currentstroke}%
\pgfsetdash{}{0pt}%
\pgfpathmoveto{\pgfqpoint{3.638579in}{2.065011in}}%
\pgfpathcurveto{\pgfqpoint{3.644403in}{2.065011in}}{\pgfqpoint{3.649989in}{2.067325in}}{\pgfqpoint{3.654107in}{2.071443in}}%
\pgfpathcurveto{\pgfqpoint{3.658225in}{2.075561in}}{\pgfqpoint{3.660539in}{2.081147in}}{\pgfqpoint{3.660539in}{2.086971in}}%
\pgfpathcurveto{\pgfqpoint{3.660539in}{2.092795in}}{\pgfqpoint{3.658225in}{2.098381in}}{\pgfqpoint{3.654107in}{2.102499in}}%
\pgfpathcurveto{\pgfqpoint{3.649989in}{2.106618in}}{\pgfqpoint{3.644403in}{2.108931in}}{\pgfqpoint{3.638579in}{2.108931in}}%
\pgfpathcurveto{\pgfqpoint{3.632755in}{2.108931in}}{\pgfqpoint{3.627169in}{2.106618in}}{\pgfqpoint{3.623051in}{2.102499in}}%
\pgfpathcurveto{\pgfqpoint{3.618933in}{2.098381in}}{\pgfqpoint{3.616619in}{2.092795in}}{\pgfqpoint{3.616619in}{2.086971in}}%
\pgfpathcurveto{\pgfqpoint{3.616619in}{2.081147in}}{\pgfqpoint{3.618933in}{2.075561in}}{\pgfqpoint{3.623051in}{2.071443in}}%
\pgfpathcurveto{\pgfqpoint{3.627169in}{2.067325in}}{\pgfqpoint{3.632755in}{2.065011in}}{\pgfqpoint{3.638579in}{2.065011in}}%
\pgfpathlineto{\pgfqpoint{3.638579in}{2.065011in}}%
\pgfpathclose%
\pgfusepath{stroke,fill}%
\end{pgfscope}%
\begin{pgfscope}%
\pgfpathrectangle{\pgfqpoint{1.000000in}{1.148311in}}{\pgfqpoint{6.200000in}{5.623377in}}%
\pgfusepath{clip}%
\pgfsetbuttcap%
\pgfsetroundjoin%
\definecolor{currentfill}{rgb}{0.200000,0.800000,0.200000}%
\pgfsetfillcolor{currentfill}%
\pgfsetlinewidth{1.003750pt}%
\definecolor{currentstroke}{rgb}{0.200000,0.800000,0.200000}%
\pgfsetstrokecolor{currentstroke}%
\pgfsetdash{}{0pt}%
\pgfpathmoveto{\pgfqpoint{3.612833in}{2.015271in}}%
\pgfpathcurveto{\pgfqpoint{3.618657in}{2.015271in}}{\pgfqpoint{3.624243in}{2.017585in}}{\pgfqpoint{3.628361in}{2.021703in}}%
\pgfpathcurveto{\pgfqpoint{3.632479in}{2.025821in}}{\pgfqpoint{3.634793in}{2.031407in}}{\pgfqpoint{3.634793in}{2.037231in}}%
\pgfpathcurveto{\pgfqpoint{3.634793in}{2.043055in}}{\pgfqpoint{3.632479in}{2.048641in}}{\pgfqpoint{3.628361in}{2.052759in}}%
\pgfpathcurveto{\pgfqpoint{3.624243in}{2.056877in}}{\pgfqpoint{3.618657in}{2.059191in}}{\pgfqpoint{3.612833in}{2.059191in}}%
\pgfpathcurveto{\pgfqpoint{3.607009in}{2.059191in}}{\pgfqpoint{3.601423in}{2.056877in}}{\pgfqpoint{3.597305in}{2.052759in}}%
\pgfpathcurveto{\pgfqpoint{3.593187in}{2.048641in}}{\pgfqpoint{3.590873in}{2.043055in}}{\pgfqpoint{3.590873in}{2.037231in}}%
\pgfpathcurveto{\pgfqpoint{3.590873in}{2.031407in}}{\pgfqpoint{3.593187in}{2.025821in}}{\pgfqpoint{3.597305in}{2.021703in}}%
\pgfpathcurveto{\pgfqpoint{3.601423in}{2.017585in}}{\pgfqpoint{3.607009in}{2.015271in}}{\pgfqpoint{3.612833in}{2.015271in}}%
\pgfpathlineto{\pgfqpoint{3.612833in}{2.015271in}}%
\pgfpathclose%
\pgfusepath{stroke,fill}%
\end{pgfscope}%
\begin{pgfscope}%
\pgfpathrectangle{\pgfqpoint{1.000000in}{1.148311in}}{\pgfqpoint{6.200000in}{5.623377in}}%
\pgfusepath{clip}%
\pgfsetbuttcap%
\pgfsetroundjoin%
\definecolor{currentfill}{rgb}{0.200000,0.800000,0.200000}%
\pgfsetfillcolor{currentfill}%
\pgfsetlinewidth{1.003750pt}%
\definecolor{currentstroke}{rgb}{0.200000,0.800000,0.200000}%
\pgfsetstrokecolor{currentstroke}%
\pgfsetdash{}{0pt}%
\pgfpathmoveto{\pgfqpoint{3.728997in}{2.069156in}}%
\pgfpathcurveto{\pgfqpoint{3.734821in}{2.069156in}}{\pgfqpoint{3.740407in}{2.071470in}}{\pgfqpoint{3.744525in}{2.075588in}}%
\pgfpathcurveto{\pgfqpoint{3.748643in}{2.079706in}}{\pgfqpoint{3.750957in}{2.085293in}}{\pgfqpoint{3.750957in}{2.091117in}}%
\pgfpathcurveto{\pgfqpoint{3.750957in}{2.096941in}}{\pgfqpoint{3.748643in}{2.102527in}}{\pgfqpoint{3.744525in}{2.106645in}}%
\pgfpathcurveto{\pgfqpoint{3.740407in}{2.110763in}}{\pgfqpoint{3.734821in}{2.113077in}}{\pgfqpoint{3.728997in}{2.113077in}}%
\pgfpathcurveto{\pgfqpoint{3.723173in}{2.113077in}}{\pgfqpoint{3.717587in}{2.110763in}}{\pgfqpoint{3.713469in}{2.106645in}}%
\pgfpathcurveto{\pgfqpoint{3.709351in}{2.102527in}}{\pgfqpoint{3.707037in}{2.096941in}}{\pgfqpoint{3.707037in}{2.091117in}}%
\pgfpathcurveto{\pgfqpoint{3.707037in}{2.085293in}}{\pgfqpoint{3.709351in}{2.079706in}}{\pgfqpoint{3.713469in}{2.075588in}}%
\pgfpathcurveto{\pgfqpoint{3.717587in}{2.071470in}}{\pgfqpoint{3.723173in}{2.069156in}}{\pgfqpoint{3.728997in}{2.069156in}}%
\pgfpathlineto{\pgfqpoint{3.728997in}{2.069156in}}%
\pgfpathclose%
\pgfusepath{stroke,fill}%
\end{pgfscope}%
\begin{pgfscope}%
\pgfpathrectangle{\pgfqpoint{1.000000in}{1.148311in}}{\pgfqpoint{6.200000in}{5.623377in}}%
\pgfusepath{clip}%
\pgfsetbuttcap%
\pgfsetroundjoin%
\definecolor{currentfill}{rgb}{0.200000,0.800000,0.200000}%
\pgfsetfillcolor{currentfill}%
\pgfsetlinewidth{1.003750pt}%
\definecolor{currentstroke}{rgb}{0.200000,0.800000,0.200000}%
\pgfsetstrokecolor{currentstroke}%
\pgfsetdash{}{0pt}%
\pgfpathmoveto{\pgfqpoint{3.661093in}{1.979572in}}%
\pgfpathcurveto{\pgfqpoint{3.666917in}{1.979572in}}{\pgfqpoint{3.672503in}{1.981886in}}{\pgfqpoint{3.676621in}{1.986004in}}%
\pgfpathcurveto{\pgfqpoint{3.680739in}{1.990122in}}{\pgfqpoint{3.683053in}{1.995708in}}{\pgfqpoint{3.683053in}{2.001532in}}%
\pgfpathcurveto{\pgfqpoint{3.683053in}{2.007356in}}{\pgfqpoint{3.680739in}{2.012942in}}{\pgfqpoint{3.676621in}{2.017060in}}%
\pgfpathcurveto{\pgfqpoint{3.672503in}{2.021178in}}{\pgfqpoint{3.666917in}{2.023492in}}{\pgfqpoint{3.661093in}{2.023492in}}%
\pgfpathcurveto{\pgfqpoint{3.655269in}{2.023492in}}{\pgfqpoint{3.649683in}{2.021178in}}{\pgfqpoint{3.645565in}{2.017060in}}%
\pgfpathcurveto{\pgfqpoint{3.641446in}{2.012942in}}{\pgfqpoint{3.639133in}{2.007356in}}{\pgfqpoint{3.639133in}{2.001532in}}%
\pgfpathcurveto{\pgfqpoint{3.639133in}{1.995708in}}{\pgfqpoint{3.641446in}{1.990122in}}{\pgfqpoint{3.645565in}{1.986004in}}%
\pgfpathcurveto{\pgfqpoint{3.649683in}{1.981886in}}{\pgfqpoint{3.655269in}{1.979572in}}{\pgfqpoint{3.661093in}{1.979572in}}%
\pgfpathlineto{\pgfqpoint{3.661093in}{1.979572in}}%
\pgfpathclose%
\pgfusepath{stroke,fill}%
\end{pgfscope}%
\begin{pgfscope}%
\pgfpathrectangle{\pgfqpoint{1.000000in}{1.148311in}}{\pgfqpoint{6.200000in}{5.623377in}}%
\pgfusepath{clip}%
\pgfsetbuttcap%
\pgfsetroundjoin%
\definecolor{currentfill}{rgb}{0.200000,0.800000,0.200000}%
\pgfsetfillcolor{currentfill}%
\pgfsetlinewidth{1.003750pt}%
\definecolor{currentstroke}{rgb}{0.200000,0.800000,0.200000}%
\pgfsetstrokecolor{currentstroke}%
\pgfsetdash{}{0pt}%
\pgfpathmoveto{\pgfqpoint{3.638618in}{1.916191in}}%
\pgfpathcurveto{\pgfqpoint{3.644442in}{1.916191in}}{\pgfqpoint{3.650028in}{1.918505in}}{\pgfqpoint{3.654146in}{1.922623in}}%
\pgfpathcurveto{\pgfqpoint{3.658264in}{1.926741in}}{\pgfqpoint{3.660578in}{1.932327in}}{\pgfqpoint{3.660578in}{1.938151in}}%
\pgfpathcurveto{\pgfqpoint{3.660578in}{1.943975in}}{\pgfqpoint{3.658264in}{1.949561in}}{\pgfqpoint{3.654146in}{1.953679in}}%
\pgfpathcurveto{\pgfqpoint{3.650028in}{1.957797in}}{\pgfqpoint{3.644442in}{1.960111in}}{\pgfqpoint{3.638618in}{1.960111in}}%
\pgfpathcurveto{\pgfqpoint{3.632794in}{1.960111in}}{\pgfqpoint{3.627208in}{1.957797in}}{\pgfqpoint{3.623090in}{1.953679in}}%
\pgfpathcurveto{\pgfqpoint{3.618972in}{1.949561in}}{\pgfqpoint{3.616658in}{1.943975in}}{\pgfqpoint{3.616658in}{1.938151in}}%
\pgfpathcurveto{\pgfqpoint{3.616658in}{1.932327in}}{\pgfqpoint{3.618972in}{1.926741in}}{\pgfqpoint{3.623090in}{1.922623in}}%
\pgfpathcurveto{\pgfqpoint{3.627208in}{1.918505in}}{\pgfqpoint{3.632794in}{1.916191in}}{\pgfqpoint{3.638618in}{1.916191in}}%
\pgfpathlineto{\pgfqpoint{3.638618in}{1.916191in}}%
\pgfpathclose%
\pgfusepath{stroke,fill}%
\end{pgfscope}%
\begin{pgfscope}%
\pgfpathrectangle{\pgfqpoint{1.000000in}{1.148311in}}{\pgfqpoint{6.200000in}{5.623377in}}%
\pgfusepath{clip}%
\pgfsetbuttcap%
\pgfsetroundjoin%
\definecolor{currentfill}{rgb}{0.200000,0.800000,0.200000}%
\pgfsetfillcolor{currentfill}%
\pgfsetlinewidth{1.003750pt}%
\definecolor{currentstroke}{rgb}{0.200000,0.800000,0.200000}%
\pgfsetstrokecolor{currentstroke}%
\pgfsetdash{}{0pt}%
\pgfpathmoveto{\pgfqpoint{3.731417in}{1.974039in}}%
\pgfpathcurveto{\pgfqpoint{3.737241in}{1.974039in}}{\pgfqpoint{3.742827in}{1.976353in}}{\pgfqpoint{3.746945in}{1.980471in}}%
\pgfpathcurveto{\pgfqpoint{3.751063in}{1.984589in}}{\pgfqpoint{3.753377in}{1.990175in}}{\pgfqpoint{3.753377in}{1.995999in}}%
\pgfpathcurveto{\pgfqpoint{3.753377in}{2.001823in}}{\pgfqpoint{3.751063in}{2.007409in}}{\pgfqpoint{3.746945in}{2.011527in}}%
\pgfpathcurveto{\pgfqpoint{3.742827in}{2.015645in}}{\pgfqpoint{3.737241in}{2.017959in}}{\pgfqpoint{3.731417in}{2.017959in}}%
\pgfpathcurveto{\pgfqpoint{3.725593in}{2.017959in}}{\pgfqpoint{3.720007in}{2.015645in}}{\pgfqpoint{3.715889in}{2.011527in}}%
\pgfpathcurveto{\pgfqpoint{3.711770in}{2.007409in}}{\pgfqpoint{3.709456in}{2.001823in}}{\pgfqpoint{3.709456in}{1.995999in}}%
\pgfpathcurveto{\pgfqpoint{3.709456in}{1.990175in}}{\pgfqpoint{3.711770in}{1.984589in}}{\pgfqpoint{3.715889in}{1.980471in}}%
\pgfpathcurveto{\pgfqpoint{3.720007in}{1.976353in}}{\pgfqpoint{3.725593in}{1.974039in}}{\pgfqpoint{3.731417in}{1.974039in}}%
\pgfpathlineto{\pgfqpoint{3.731417in}{1.974039in}}%
\pgfpathclose%
\pgfusepath{stroke,fill}%
\end{pgfscope}%
\begin{pgfscope}%
\pgfpathrectangle{\pgfqpoint{1.000000in}{1.148311in}}{\pgfqpoint{6.200000in}{5.623377in}}%
\pgfusepath{clip}%
\pgfsetbuttcap%
\pgfsetroundjoin%
\definecolor{currentfill}{rgb}{0.200000,0.800000,0.200000}%
\pgfsetfillcolor{currentfill}%
\pgfsetlinewidth{1.003750pt}%
\definecolor{currentstroke}{rgb}{0.200000,0.800000,0.200000}%
\pgfsetstrokecolor{currentstroke}%
\pgfsetdash{}{0pt}%
\pgfpathmoveto{\pgfqpoint{3.756574in}{1.964893in}}%
\pgfpathcurveto{\pgfqpoint{3.762398in}{1.964893in}}{\pgfqpoint{3.767984in}{1.967206in}}{\pgfqpoint{3.772102in}{1.971325in}}%
\pgfpathcurveto{\pgfqpoint{3.776220in}{1.975443in}}{\pgfqpoint{3.778534in}{1.981029in}}{\pgfqpoint{3.778534in}{1.986853in}}%
\pgfpathcurveto{\pgfqpoint{3.778534in}{1.992677in}}{\pgfqpoint{3.776220in}{1.998263in}}{\pgfqpoint{3.772102in}{2.002381in}}%
\pgfpathcurveto{\pgfqpoint{3.767984in}{2.006499in}}{\pgfqpoint{3.762398in}{2.008813in}}{\pgfqpoint{3.756574in}{2.008813in}}%
\pgfpathcurveto{\pgfqpoint{3.750750in}{2.008813in}}{\pgfqpoint{3.745164in}{2.006499in}}{\pgfqpoint{3.741046in}{2.002381in}}%
\pgfpathcurveto{\pgfqpoint{3.736928in}{1.998263in}}{\pgfqpoint{3.734614in}{1.992677in}}{\pgfqpoint{3.734614in}{1.986853in}}%
\pgfpathcurveto{\pgfqpoint{3.734614in}{1.981029in}}{\pgfqpoint{3.736928in}{1.975443in}}{\pgfqpoint{3.741046in}{1.971325in}}%
\pgfpathcurveto{\pgfqpoint{3.745164in}{1.967206in}}{\pgfqpoint{3.750750in}{1.964893in}}{\pgfqpoint{3.756574in}{1.964893in}}%
\pgfpathlineto{\pgfqpoint{3.756574in}{1.964893in}}%
\pgfpathclose%
\pgfusepath{stroke,fill}%
\end{pgfscope}%
\begin{pgfscope}%
\pgfpathrectangle{\pgfqpoint{1.000000in}{1.148311in}}{\pgfqpoint{6.200000in}{5.623377in}}%
\pgfusepath{clip}%
\pgfsetbuttcap%
\pgfsetroundjoin%
\definecolor{currentfill}{rgb}{0.200000,0.800000,0.200000}%
\pgfsetfillcolor{currentfill}%
\pgfsetlinewidth{1.003750pt}%
\definecolor{currentstroke}{rgb}{0.200000,0.800000,0.200000}%
\pgfsetstrokecolor{currentstroke}%
\pgfsetdash{}{0pt}%
\pgfpathmoveto{\pgfqpoint{3.716997in}{1.860269in}}%
\pgfpathcurveto{\pgfqpoint{3.722820in}{1.860269in}}{\pgfqpoint{3.728407in}{1.862583in}}{\pgfqpoint{3.732525in}{1.866701in}}%
\pgfpathcurveto{\pgfqpoint{3.736643in}{1.870819in}}{\pgfqpoint{3.738957in}{1.876406in}}{\pgfqpoint{3.738957in}{1.882229in}}%
\pgfpathcurveto{\pgfqpoint{3.738957in}{1.888053in}}{\pgfqpoint{3.736643in}{1.893640in}}{\pgfqpoint{3.732525in}{1.897758in}}%
\pgfpathcurveto{\pgfqpoint{3.728407in}{1.901876in}}{\pgfqpoint{3.722820in}{1.904190in}}{\pgfqpoint{3.716997in}{1.904190in}}%
\pgfpathcurveto{\pgfqpoint{3.711173in}{1.904190in}}{\pgfqpoint{3.705586in}{1.901876in}}{\pgfqpoint{3.701468in}{1.897758in}}%
\pgfpathcurveto{\pgfqpoint{3.697350in}{1.893640in}}{\pgfqpoint{3.695036in}{1.888053in}}{\pgfqpoint{3.695036in}{1.882229in}}%
\pgfpathcurveto{\pgfqpoint{3.695036in}{1.876406in}}{\pgfqpoint{3.697350in}{1.870819in}}{\pgfqpoint{3.701468in}{1.866701in}}%
\pgfpathcurveto{\pgfqpoint{3.705586in}{1.862583in}}{\pgfqpoint{3.711173in}{1.860269in}}{\pgfqpoint{3.716997in}{1.860269in}}%
\pgfpathlineto{\pgfqpoint{3.716997in}{1.860269in}}%
\pgfpathclose%
\pgfusepath{stroke,fill}%
\end{pgfscope}%
\begin{pgfscope}%
\pgfpathrectangle{\pgfqpoint{1.000000in}{1.148311in}}{\pgfqpoint{6.200000in}{5.623377in}}%
\pgfusepath{clip}%
\pgfsetbuttcap%
\pgfsetroundjoin%
\definecolor{currentfill}{rgb}{0.200000,0.800000,0.200000}%
\pgfsetfillcolor{currentfill}%
\pgfsetlinewidth{1.003750pt}%
\definecolor{currentstroke}{rgb}{0.200000,0.800000,0.200000}%
\pgfsetstrokecolor{currentstroke}%
\pgfsetdash{}{0pt}%
\pgfpathmoveto{\pgfqpoint{3.741407in}{1.838816in}}%
\pgfpathcurveto{\pgfqpoint{3.747231in}{1.838816in}}{\pgfqpoint{3.752817in}{1.841130in}}{\pgfqpoint{3.756935in}{1.845248in}}%
\pgfpathcurveto{\pgfqpoint{3.761053in}{1.849366in}}{\pgfqpoint{3.763367in}{1.854952in}}{\pgfqpoint{3.763367in}{1.860776in}}%
\pgfpathcurveto{\pgfqpoint{3.763367in}{1.866600in}}{\pgfqpoint{3.761053in}{1.872186in}}{\pgfqpoint{3.756935in}{1.876304in}}%
\pgfpathcurveto{\pgfqpoint{3.752817in}{1.880422in}}{\pgfqpoint{3.747231in}{1.882736in}}{\pgfqpoint{3.741407in}{1.882736in}}%
\pgfpathcurveto{\pgfqpoint{3.735583in}{1.882736in}}{\pgfqpoint{3.729997in}{1.880422in}}{\pgfqpoint{3.725879in}{1.876304in}}%
\pgfpathcurveto{\pgfqpoint{3.721761in}{1.872186in}}{\pgfqpoint{3.719447in}{1.866600in}}{\pgfqpoint{3.719447in}{1.860776in}}%
\pgfpathcurveto{\pgfqpoint{3.719447in}{1.854952in}}{\pgfqpoint{3.721761in}{1.849366in}}{\pgfqpoint{3.725879in}{1.845248in}}%
\pgfpathcurveto{\pgfqpoint{3.729997in}{1.841130in}}{\pgfqpoint{3.735583in}{1.838816in}}{\pgfqpoint{3.741407in}{1.838816in}}%
\pgfpathlineto{\pgfqpoint{3.741407in}{1.838816in}}%
\pgfpathclose%
\pgfusepath{stroke,fill}%
\end{pgfscope}%
\begin{pgfscope}%
\pgfpathrectangle{\pgfqpoint{1.000000in}{1.148311in}}{\pgfqpoint{6.200000in}{5.623377in}}%
\pgfusepath{clip}%
\pgfsetbuttcap%
\pgfsetroundjoin%
\definecolor{currentfill}{rgb}{0.200000,0.800000,0.200000}%
\pgfsetfillcolor{currentfill}%
\pgfsetlinewidth{1.003750pt}%
\definecolor{currentstroke}{rgb}{0.200000,0.800000,0.200000}%
\pgfsetstrokecolor{currentstroke}%
\pgfsetdash{}{0pt}%
\pgfpathmoveto{\pgfqpoint{3.815639in}{1.916067in}}%
\pgfpathcurveto{\pgfqpoint{3.821463in}{1.916067in}}{\pgfqpoint{3.827049in}{1.918381in}}{\pgfqpoint{3.831167in}{1.922499in}}%
\pgfpathcurveto{\pgfqpoint{3.835286in}{1.926617in}}{\pgfqpoint{3.837599in}{1.932204in}}{\pgfqpoint{3.837599in}{1.938027in}}%
\pgfpathcurveto{\pgfqpoint{3.837599in}{1.943851in}}{\pgfqpoint{3.835286in}{1.949438in}}{\pgfqpoint{3.831167in}{1.953556in}}%
\pgfpathcurveto{\pgfqpoint{3.827049in}{1.957674in}}{\pgfqpoint{3.821463in}{1.959988in}}{\pgfqpoint{3.815639in}{1.959988in}}%
\pgfpathcurveto{\pgfqpoint{3.809815in}{1.959988in}}{\pgfqpoint{3.804229in}{1.957674in}}{\pgfqpoint{3.800111in}{1.953556in}}%
\pgfpathcurveto{\pgfqpoint{3.795993in}{1.949438in}}{\pgfqpoint{3.793679in}{1.943851in}}{\pgfqpoint{3.793679in}{1.938027in}}%
\pgfpathcurveto{\pgfqpoint{3.793679in}{1.932204in}}{\pgfqpoint{3.795993in}{1.926617in}}{\pgfqpoint{3.800111in}{1.922499in}}%
\pgfpathcurveto{\pgfqpoint{3.804229in}{1.918381in}}{\pgfqpoint{3.809815in}{1.916067in}}{\pgfqpoint{3.815639in}{1.916067in}}%
\pgfpathlineto{\pgfqpoint{3.815639in}{1.916067in}}%
\pgfpathclose%
\pgfusepath{stroke,fill}%
\end{pgfscope}%
\begin{pgfscope}%
\pgfpathrectangle{\pgfqpoint{1.000000in}{1.148311in}}{\pgfqpoint{6.200000in}{5.623377in}}%
\pgfusepath{clip}%
\pgfsetbuttcap%
\pgfsetroundjoin%
\definecolor{currentfill}{rgb}{0.200000,0.800000,0.200000}%
\pgfsetfillcolor{currentfill}%
\pgfsetlinewidth{1.003750pt}%
\definecolor{currentstroke}{rgb}{0.200000,0.800000,0.200000}%
\pgfsetstrokecolor{currentstroke}%
\pgfsetdash{}{0pt}%
\pgfpathmoveto{\pgfqpoint{3.791765in}{1.792650in}}%
\pgfpathcurveto{\pgfqpoint{3.797589in}{1.792650in}}{\pgfqpoint{3.803175in}{1.794964in}}{\pgfqpoint{3.807293in}{1.799082in}}%
\pgfpathcurveto{\pgfqpoint{3.811412in}{1.803200in}}{\pgfqpoint{3.813725in}{1.808786in}}{\pgfqpoint{3.813725in}{1.814610in}}%
\pgfpathcurveto{\pgfqpoint{3.813725in}{1.820434in}}{\pgfqpoint{3.811412in}{1.826020in}}{\pgfqpoint{3.807293in}{1.830139in}}%
\pgfpathcurveto{\pgfqpoint{3.803175in}{1.834257in}}{\pgfqpoint{3.797589in}{1.836571in}}{\pgfqpoint{3.791765in}{1.836571in}}%
\pgfpathcurveto{\pgfqpoint{3.785941in}{1.836571in}}{\pgfqpoint{3.780355in}{1.834257in}}{\pgfqpoint{3.776237in}{1.830139in}}%
\pgfpathcurveto{\pgfqpoint{3.772119in}{1.826020in}}{\pgfqpoint{3.769805in}{1.820434in}}{\pgfqpoint{3.769805in}{1.814610in}}%
\pgfpathcurveto{\pgfqpoint{3.769805in}{1.808786in}}{\pgfqpoint{3.772119in}{1.803200in}}{\pgfqpoint{3.776237in}{1.799082in}}%
\pgfpathcurveto{\pgfqpoint{3.780355in}{1.794964in}}{\pgfqpoint{3.785941in}{1.792650in}}{\pgfqpoint{3.791765in}{1.792650in}}%
\pgfpathlineto{\pgfqpoint{3.791765in}{1.792650in}}%
\pgfpathclose%
\pgfusepath{stroke,fill}%
\end{pgfscope}%
\begin{pgfscope}%
\pgfpathrectangle{\pgfqpoint{1.000000in}{1.148311in}}{\pgfqpoint{6.200000in}{5.623377in}}%
\pgfusepath{clip}%
\pgfsetbuttcap%
\pgfsetroundjoin%
\definecolor{currentfill}{rgb}{0.200000,0.800000,0.200000}%
\pgfsetfillcolor{currentfill}%
\pgfsetlinewidth{1.003750pt}%
\definecolor{currentstroke}{rgb}{0.200000,0.800000,0.200000}%
\pgfsetstrokecolor{currentstroke}%
\pgfsetdash{}{0pt}%
\pgfpathmoveto{\pgfqpoint{3.839828in}{1.829355in}}%
\pgfpathcurveto{\pgfqpoint{3.845652in}{1.829355in}}{\pgfqpoint{3.851238in}{1.831669in}}{\pgfqpoint{3.855356in}{1.835787in}}%
\pgfpathcurveto{\pgfqpoint{3.859474in}{1.839905in}}{\pgfqpoint{3.861788in}{1.845492in}}{\pgfqpoint{3.861788in}{1.851315in}}%
\pgfpathcurveto{\pgfqpoint{3.861788in}{1.857139in}}{\pgfqpoint{3.859474in}{1.862726in}}{\pgfqpoint{3.855356in}{1.866844in}}%
\pgfpathcurveto{\pgfqpoint{3.851238in}{1.870962in}}{\pgfqpoint{3.845652in}{1.873276in}}{\pgfqpoint{3.839828in}{1.873276in}}%
\pgfpathcurveto{\pgfqpoint{3.834004in}{1.873276in}}{\pgfqpoint{3.828418in}{1.870962in}}{\pgfqpoint{3.824300in}{1.866844in}}%
\pgfpathcurveto{\pgfqpoint{3.820182in}{1.862726in}}{\pgfqpoint{3.817868in}{1.857139in}}{\pgfqpoint{3.817868in}{1.851315in}}%
\pgfpathcurveto{\pgfqpoint{3.817868in}{1.845492in}}{\pgfqpoint{3.820182in}{1.839905in}}{\pgfqpoint{3.824300in}{1.835787in}}%
\pgfpathcurveto{\pgfqpoint{3.828418in}{1.831669in}}{\pgfqpoint{3.834004in}{1.829355in}}{\pgfqpoint{3.839828in}{1.829355in}}%
\pgfpathlineto{\pgfqpoint{3.839828in}{1.829355in}}%
\pgfpathclose%
\pgfusepath{stroke,fill}%
\end{pgfscope}%
\begin{pgfscope}%
\pgfpathrectangle{\pgfqpoint{1.000000in}{1.148311in}}{\pgfqpoint{6.200000in}{5.623377in}}%
\pgfusepath{clip}%
\pgfsetbuttcap%
\pgfsetroundjoin%
\definecolor{currentfill}{rgb}{0.200000,0.800000,0.200000}%
\pgfsetfillcolor{currentfill}%
\pgfsetlinewidth{1.003750pt}%
\definecolor{currentstroke}{rgb}{0.200000,0.800000,0.200000}%
\pgfsetstrokecolor{currentstroke}%
\pgfsetdash{}{0pt}%
\pgfpathmoveto{\pgfqpoint{3.862133in}{1.796887in}}%
\pgfpathcurveto{\pgfqpoint{3.867957in}{1.796887in}}{\pgfqpoint{3.873543in}{1.799201in}}{\pgfqpoint{3.877661in}{1.803319in}}%
\pgfpathcurveto{\pgfqpoint{3.881779in}{1.807437in}}{\pgfqpoint{3.884093in}{1.813023in}}{\pgfqpoint{3.884093in}{1.818847in}}%
\pgfpathcurveto{\pgfqpoint{3.884093in}{1.824671in}}{\pgfqpoint{3.881779in}{1.830257in}}{\pgfqpoint{3.877661in}{1.834376in}}%
\pgfpathcurveto{\pgfqpoint{3.873543in}{1.838494in}}{\pgfqpoint{3.867957in}{1.840808in}}{\pgfqpoint{3.862133in}{1.840808in}}%
\pgfpathcurveto{\pgfqpoint{3.856309in}{1.840808in}}{\pgfqpoint{3.850723in}{1.838494in}}{\pgfqpoint{3.846605in}{1.834376in}}%
\pgfpathcurveto{\pgfqpoint{3.842486in}{1.830257in}}{\pgfqpoint{3.840173in}{1.824671in}}{\pgfqpoint{3.840173in}{1.818847in}}%
\pgfpathcurveto{\pgfqpoint{3.840173in}{1.813023in}}{\pgfqpoint{3.842486in}{1.807437in}}{\pgfqpoint{3.846605in}{1.803319in}}%
\pgfpathcurveto{\pgfqpoint{3.850723in}{1.799201in}}{\pgfqpoint{3.856309in}{1.796887in}}{\pgfqpoint{3.862133in}{1.796887in}}%
\pgfpathlineto{\pgfqpoint{3.862133in}{1.796887in}}%
\pgfpathclose%
\pgfusepath{stroke,fill}%
\end{pgfscope}%
\begin{pgfscope}%
\pgfpathrectangle{\pgfqpoint{1.000000in}{1.148311in}}{\pgfqpoint{6.200000in}{5.623377in}}%
\pgfusepath{clip}%
\pgfsetbuttcap%
\pgfsetroundjoin%
\definecolor{currentfill}{rgb}{0.200000,0.800000,0.200000}%
\pgfsetfillcolor{currentfill}%
\pgfsetlinewidth{1.003750pt}%
\definecolor{currentstroke}{rgb}{0.200000,0.800000,0.200000}%
\pgfsetstrokecolor{currentstroke}%
\pgfsetdash{}{0pt}%
\pgfpathmoveto{\pgfqpoint{3.907873in}{1.860719in}}%
\pgfpathcurveto{\pgfqpoint{3.913697in}{1.860719in}}{\pgfqpoint{3.919283in}{1.863033in}}{\pgfqpoint{3.923401in}{1.867151in}}%
\pgfpathcurveto{\pgfqpoint{3.927519in}{1.871270in}}{\pgfqpoint{3.929833in}{1.876856in}}{\pgfqpoint{3.929833in}{1.882680in}}%
\pgfpathcurveto{\pgfqpoint{3.929833in}{1.888504in}}{\pgfqpoint{3.927519in}{1.894090in}}{\pgfqpoint{3.923401in}{1.898208in}}%
\pgfpathcurveto{\pgfqpoint{3.919283in}{1.902326in}}{\pgfqpoint{3.913697in}{1.904640in}}{\pgfqpoint{3.907873in}{1.904640in}}%
\pgfpathcurveto{\pgfqpoint{3.902049in}{1.904640in}}{\pgfqpoint{3.896463in}{1.902326in}}{\pgfqpoint{3.892345in}{1.898208in}}%
\pgfpathcurveto{\pgfqpoint{3.888227in}{1.894090in}}{\pgfqpoint{3.885913in}{1.888504in}}{\pgfqpoint{3.885913in}{1.882680in}}%
\pgfpathcurveto{\pgfqpoint{3.885913in}{1.876856in}}{\pgfqpoint{3.888227in}{1.871270in}}{\pgfqpoint{3.892345in}{1.867151in}}%
\pgfpathcurveto{\pgfqpoint{3.896463in}{1.863033in}}{\pgfqpoint{3.902049in}{1.860719in}}{\pgfqpoint{3.907873in}{1.860719in}}%
\pgfpathlineto{\pgfqpoint{3.907873in}{1.860719in}}%
\pgfpathclose%
\pgfusepath{stroke,fill}%
\end{pgfscope}%
\begin{pgfscope}%
\pgfpathrectangle{\pgfqpoint{1.000000in}{1.148311in}}{\pgfqpoint{6.200000in}{5.623377in}}%
\pgfusepath{clip}%
\pgfsetbuttcap%
\pgfsetroundjoin%
\definecolor{currentfill}{rgb}{0.200000,0.800000,0.200000}%
\pgfsetfillcolor{currentfill}%
\pgfsetlinewidth{1.003750pt}%
\definecolor{currentstroke}{rgb}{0.200000,0.800000,0.200000}%
\pgfsetstrokecolor{currentstroke}%
\pgfsetdash{}{0pt}%
\pgfpathmoveto{\pgfqpoint{3.921703in}{1.766653in}}%
\pgfpathcurveto{\pgfqpoint{3.927527in}{1.766653in}}{\pgfqpoint{3.933114in}{1.768966in}}{\pgfqpoint{3.937232in}{1.773085in}}%
\pgfpathcurveto{\pgfqpoint{3.941350in}{1.777203in}}{\pgfqpoint{3.943664in}{1.782789in}}{\pgfqpoint{3.943664in}{1.788613in}}%
\pgfpathcurveto{\pgfqpoint{3.943664in}{1.794437in}}{\pgfqpoint{3.941350in}{1.800023in}}{\pgfqpoint{3.937232in}{1.804141in}}%
\pgfpathcurveto{\pgfqpoint{3.933114in}{1.808259in}}{\pgfqpoint{3.927527in}{1.810573in}}{\pgfqpoint{3.921703in}{1.810573in}}%
\pgfpathcurveto{\pgfqpoint{3.915880in}{1.810573in}}{\pgfqpoint{3.910293in}{1.808259in}}{\pgfqpoint{3.906175in}{1.804141in}}%
\pgfpathcurveto{\pgfqpoint{3.902057in}{1.800023in}}{\pgfqpoint{3.899743in}{1.794437in}}{\pgfqpoint{3.899743in}{1.788613in}}%
\pgfpathcurveto{\pgfqpoint{3.899743in}{1.782789in}}{\pgfqpoint{3.902057in}{1.777203in}}{\pgfqpoint{3.906175in}{1.773085in}}%
\pgfpathcurveto{\pgfqpoint{3.910293in}{1.768966in}}{\pgfqpoint{3.915880in}{1.766653in}}{\pgfqpoint{3.921703in}{1.766653in}}%
\pgfpathlineto{\pgfqpoint{3.921703in}{1.766653in}}%
\pgfpathclose%
\pgfusepath{stroke,fill}%
\end{pgfscope}%
\begin{pgfscope}%
\pgfpathrectangle{\pgfqpoint{1.000000in}{1.148311in}}{\pgfqpoint{6.200000in}{5.623377in}}%
\pgfusepath{clip}%
\pgfsetbuttcap%
\pgfsetroundjoin%
\definecolor{currentfill}{rgb}{0.200000,0.800000,0.200000}%
\pgfsetfillcolor{currentfill}%
\pgfsetlinewidth{1.003750pt}%
\definecolor{currentstroke}{rgb}{0.200000,0.800000,0.200000}%
\pgfsetstrokecolor{currentstroke}%
\pgfsetdash{}{0pt}%
\pgfpathmoveto{\pgfqpoint{3.959832in}{1.832343in}}%
\pgfpathcurveto{\pgfqpoint{3.965656in}{1.832343in}}{\pgfqpoint{3.971242in}{1.834657in}}{\pgfqpoint{3.975360in}{1.838775in}}%
\pgfpathcurveto{\pgfqpoint{3.979478in}{1.842893in}}{\pgfqpoint{3.981792in}{1.848479in}}{\pgfqpoint{3.981792in}{1.854303in}}%
\pgfpathcurveto{\pgfqpoint{3.981792in}{1.860127in}}{\pgfqpoint{3.979478in}{1.865714in}}{\pgfqpoint{3.975360in}{1.869832in}}%
\pgfpathcurveto{\pgfqpoint{3.971242in}{1.873950in}}{\pgfqpoint{3.965656in}{1.876264in}}{\pgfqpoint{3.959832in}{1.876264in}}%
\pgfpathcurveto{\pgfqpoint{3.954008in}{1.876264in}}{\pgfqpoint{3.948422in}{1.873950in}}{\pgfqpoint{3.944304in}{1.869832in}}%
\pgfpathcurveto{\pgfqpoint{3.940186in}{1.865714in}}{\pgfqpoint{3.937872in}{1.860127in}}{\pgfqpoint{3.937872in}{1.854303in}}%
\pgfpathcurveto{\pgfqpoint{3.937872in}{1.848479in}}{\pgfqpoint{3.940186in}{1.842893in}}{\pgfqpoint{3.944304in}{1.838775in}}%
\pgfpathcurveto{\pgfqpoint{3.948422in}{1.834657in}}{\pgfqpoint{3.954008in}{1.832343in}}{\pgfqpoint{3.959832in}{1.832343in}}%
\pgfpathlineto{\pgfqpoint{3.959832in}{1.832343in}}%
\pgfpathclose%
\pgfusepath{stroke,fill}%
\end{pgfscope}%
\begin{pgfscope}%
\pgfpathrectangle{\pgfqpoint{1.000000in}{1.148311in}}{\pgfqpoint{6.200000in}{5.623377in}}%
\pgfusepath{clip}%
\pgfsetbuttcap%
\pgfsetroundjoin%
\definecolor{currentfill}{rgb}{0.200000,0.800000,0.200000}%
\pgfsetfillcolor{currentfill}%
\pgfsetlinewidth{1.003750pt}%
\definecolor{currentstroke}{rgb}{0.200000,0.800000,0.200000}%
\pgfsetstrokecolor{currentstroke}%
\pgfsetdash{}{0pt}%
\pgfpathmoveto{\pgfqpoint{3.988234in}{1.832734in}}%
\pgfpathcurveto{\pgfqpoint{3.994058in}{1.832734in}}{\pgfqpoint{3.999644in}{1.835047in}}{\pgfqpoint{4.003762in}{1.839166in}}%
\pgfpathcurveto{\pgfqpoint{4.007880in}{1.843284in}}{\pgfqpoint{4.010194in}{1.848870in}}{\pgfqpoint{4.010194in}{1.854694in}}%
\pgfpathcurveto{\pgfqpoint{4.010194in}{1.860518in}}{\pgfqpoint{4.007880in}{1.866104in}}{\pgfqpoint{4.003762in}{1.870222in}}%
\pgfpathcurveto{\pgfqpoint{3.999644in}{1.874340in}}{\pgfqpoint{3.994058in}{1.876654in}}{\pgfqpoint{3.988234in}{1.876654in}}%
\pgfpathcurveto{\pgfqpoint{3.982410in}{1.876654in}}{\pgfqpoint{3.976823in}{1.874340in}}{\pgfqpoint{3.972705in}{1.870222in}}%
\pgfpathcurveto{\pgfqpoint{3.968587in}{1.866104in}}{\pgfqpoint{3.966273in}{1.860518in}}{\pgfqpoint{3.966273in}{1.854694in}}%
\pgfpathcurveto{\pgfqpoint{3.966273in}{1.848870in}}{\pgfqpoint{3.968587in}{1.843284in}}{\pgfqpoint{3.972705in}{1.839166in}}%
\pgfpathcurveto{\pgfqpoint{3.976823in}{1.835047in}}{\pgfqpoint{3.982410in}{1.832734in}}{\pgfqpoint{3.988234in}{1.832734in}}%
\pgfpathlineto{\pgfqpoint{3.988234in}{1.832734in}}%
\pgfpathclose%
\pgfusepath{stroke,fill}%
\end{pgfscope}%
\begin{pgfscope}%
\pgfpathrectangle{\pgfqpoint{1.000000in}{1.148311in}}{\pgfqpoint{6.200000in}{5.623377in}}%
\pgfusepath{clip}%
\pgfsetbuttcap%
\pgfsetroundjoin%
\definecolor{currentfill}{rgb}{0.200000,0.800000,0.200000}%
\pgfsetfillcolor{currentfill}%
\pgfsetlinewidth{1.003750pt}%
\definecolor{currentstroke}{rgb}{0.200000,0.800000,0.200000}%
\pgfsetstrokecolor{currentstroke}%
\pgfsetdash{}{0pt}%
\pgfpathmoveto{\pgfqpoint{4.018022in}{1.801774in}}%
\pgfpathcurveto{\pgfqpoint{4.023846in}{1.801774in}}{\pgfqpoint{4.029432in}{1.804088in}}{\pgfqpoint{4.033550in}{1.808206in}}%
\pgfpathcurveto{\pgfqpoint{4.037668in}{1.812325in}}{\pgfqpoint{4.039982in}{1.817911in}}{\pgfqpoint{4.039982in}{1.823735in}}%
\pgfpathcurveto{\pgfqpoint{4.039982in}{1.829559in}}{\pgfqpoint{4.037668in}{1.835145in}}{\pgfqpoint{4.033550in}{1.839263in}}%
\pgfpathcurveto{\pgfqpoint{4.029432in}{1.843381in}}{\pgfqpoint{4.023846in}{1.845695in}}{\pgfqpoint{4.018022in}{1.845695in}}%
\pgfpathcurveto{\pgfqpoint{4.012198in}{1.845695in}}{\pgfqpoint{4.006612in}{1.843381in}}{\pgfqpoint{4.002493in}{1.839263in}}%
\pgfpathcurveto{\pgfqpoint{3.998375in}{1.835145in}}{\pgfqpoint{3.996061in}{1.829559in}}{\pgfqpoint{3.996061in}{1.823735in}}%
\pgfpathcurveto{\pgfqpoint{3.996061in}{1.817911in}}{\pgfqpoint{3.998375in}{1.812325in}}{\pgfqpoint{4.002493in}{1.808206in}}%
\pgfpathcurveto{\pgfqpoint{4.006612in}{1.804088in}}{\pgfqpoint{4.012198in}{1.801774in}}{\pgfqpoint{4.018022in}{1.801774in}}%
\pgfpathlineto{\pgfqpoint{4.018022in}{1.801774in}}%
\pgfpathclose%
\pgfusepath{stroke,fill}%
\end{pgfscope}%
\begin{pgfscope}%
\pgfpathrectangle{\pgfqpoint{1.000000in}{1.148311in}}{\pgfqpoint{6.200000in}{5.623377in}}%
\pgfusepath{clip}%
\pgfsetbuttcap%
\pgfsetroundjoin%
\definecolor{currentfill}{rgb}{0.200000,0.800000,0.200000}%
\pgfsetfillcolor{currentfill}%
\pgfsetlinewidth{1.003750pt}%
\definecolor{currentstroke}{rgb}{0.200000,0.800000,0.200000}%
\pgfsetstrokecolor{currentstroke}%
\pgfsetdash{}{0pt}%
\pgfpathmoveto{\pgfqpoint{4.044381in}{1.838572in}}%
\pgfpathcurveto{\pgfqpoint{4.050205in}{1.838572in}}{\pgfqpoint{4.055791in}{1.840885in}}{\pgfqpoint{4.059909in}{1.845004in}}%
\pgfpathcurveto{\pgfqpoint{4.064027in}{1.849122in}}{\pgfqpoint{4.066341in}{1.854708in}}{\pgfqpoint{4.066341in}{1.860532in}}%
\pgfpathcurveto{\pgfqpoint{4.066341in}{1.866356in}}{\pgfqpoint{4.064027in}{1.871942in}}{\pgfqpoint{4.059909in}{1.876060in}}%
\pgfpathcurveto{\pgfqpoint{4.055791in}{1.880178in}}{\pgfqpoint{4.050205in}{1.882492in}}{\pgfqpoint{4.044381in}{1.882492in}}%
\pgfpathcurveto{\pgfqpoint{4.038557in}{1.882492in}}{\pgfqpoint{4.032971in}{1.880178in}}{\pgfqpoint{4.028853in}{1.876060in}}%
\pgfpathcurveto{\pgfqpoint{4.024735in}{1.871942in}}{\pgfqpoint{4.022421in}{1.866356in}}{\pgfqpoint{4.022421in}{1.860532in}}%
\pgfpathcurveto{\pgfqpoint{4.022421in}{1.854708in}}{\pgfqpoint{4.024735in}{1.849122in}}{\pgfqpoint{4.028853in}{1.845004in}}%
\pgfpathcurveto{\pgfqpoint{4.032971in}{1.840885in}}{\pgfqpoint{4.038557in}{1.838572in}}{\pgfqpoint{4.044381in}{1.838572in}}%
\pgfpathlineto{\pgfqpoint{4.044381in}{1.838572in}}%
\pgfpathclose%
\pgfusepath{stroke,fill}%
\end{pgfscope}%
\begin{pgfscope}%
\pgfpathrectangle{\pgfqpoint{1.000000in}{1.148311in}}{\pgfqpoint{6.200000in}{5.623377in}}%
\pgfusepath{clip}%
\pgfsetbuttcap%
\pgfsetroundjoin%
\definecolor{currentfill}{rgb}{0.200000,0.800000,0.200000}%
\pgfsetfillcolor{currentfill}%
\pgfsetlinewidth{1.003750pt}%
\definecolor{currentstroke}{rgb}{0.200000,0.800000,0.200000}%
\pgfsetstrokecolor{currentstroke}%
\pgfsetdash{}{0pt}%
\pgfpathmoveto{\pgfqpoint{4.064432in}{1.886565in}}%
\pgfpathcurveto{\pgfqpoint{4.070256in}{1.886565in}}{\pgfqpoint{4.075842in}{1.888879in}}{\pgfqpoint{4.079960in}{1.892998in}}%
\pgfpathcurveto{\pgfqpoint{4.084079in}{1.897116in}}{\pgfqpoint{4.086392in}{1.902702in}}{\pgfqpoint{4.086392in}{1.908526in}}%
\pgfpathcurveto{\pgfqpoint{4.086392in}{1.914350in}}{\pgfqpoint{4.084079in}{1.919936in}}{\pgfqpoint{4.079960in}{1.924054in}}%
\pgfpathcurveto{\pgfqpoint{4.075842in}{1.928172in}}{\pgfqpoint{4.070256in}{1.930486in}}{\pgfqpoint{4.064432in}{1.930486in}}%
\pgfpathcurveto{\pgfqpoint{4.058608in}{1.930486in}}{\pgfqpoint{4.053022in}{1.928172in}}{\pgfqpoint{4.048904in}{1.924054in}}%
\pgfpathcurveto{\pgfqpoint{4.044786in}{1.919936in}}{\pgfqpoint{4.042472in}{1.914350in}}{\pgfqpoint{4.042472in}{1.908526in}}%
\pgfpathcurveto{\pgfqpoint{4.042472in}{1.902702in}}{\pgfqpoint{4.044786in}{1.897116in}}{\pgfqpoint{4.048904in}{1.892998in}}%
\pgfpathcurveto{\pgfqpoint{4.053022in}{1.888879in}}{\pgfqpoint{4.058608in}{1.886565in}}{\pgfqpoint{4.064432in}{1.886565in}}%
\pgfpathlineto{\pgfqpoint{4.064432in}{1.886565in}}%
\pgfpathclose%
\pgfusepath{stroke,fill}%
\end{pgfscope}%
\begin{pgfscope}%
\pgfpathrectangle{\pgfqpoint{1.000000in}{1.148311in}}{\pgfqpoint{6.200000in}{5.623377in}}%
\pgfusepath{clip}%
\pgfsetbuttcap%
\pgfsetroundjoin%
\definecolor{currentfill}{rgb}{0.200000,0.800000,0.200000}%
\pgfsetfillcolor{currentfill}%
\pgfsetlinewidth{1.003750pt}%
\definecolor{currentstroke}{rgb}{0.200000,0.800000,0.200000}%
\pgfsetstrokecolor{currentstroke}%
\pgfsetdash{}{0pt}%
\pgfpathmoveto{\pgfqpoint{4.104577in}{1.828176in}}%
\pgfpathcurveto{\pgfqpoint{4.110401in}{1.828176in}}{\pgfqpoint{4.115987in}{1.830490in}}{\pgfqpoint{4.120105in}{1.834608in}}%
\pgfpathcurveto{\pgfqpoint{4.124223in}{1.838726in}}{\pgfqpoint{4.126537in}{1.844312in}}{\pgfqpoint{4.126537in}{1.850136in}}%
\pgfpathcurveto{\pgfqpoint{4.126537in}{1.855960in}}{\pgfqpoint{4.124223in}{1.861546in}}{\pgfqpoint{4.120105in}{1.865664in}}%
\pgfpathcurveto{\pgfqpoint{4.115987in}{1.869783in}}{\pgfqpoint{4.110401in}{1.872096in}}{\pgfqpoint{4.104577in}{1.872096in}}%
\pgfpathcurveto{\pgfqpoint{4.098753in}{1.872096in}}{\pgfqpoint{4.093167in}{1.869783in}}{\pgfqpoint{4.089048in}{1.865664in}}%
\pgfpathcurveto{\pgfqpoint{4.084930in}{1.861546in}}{\pgfqpoint{4.082616in}{1.855960in}}{\pgfqpoint{4.082616in}{1.850136in}}%
\pgfpathcurveto{\pgfqpoint{4.082616in}{1.844312in}}{\pgfqpoint{4.084930in}{1.838726in}}{\pgfqpoint{4.089048in}{1.834608in}}%
\pgfpathcurveto{\pgfqpoint{4.093167in}{1.830490in}}{\pgfqpoint{4.098753in}{1.828176in}}{\pgfqpoint{4.104577in}{1.828176in}}%
\pgfpathlineto{\pgfqpoint{4.104577in}{1.828176in}}%
\pgfpathclose%
\pgfusepath{stroke,fill}%
\end{pgfscope}%
\begin{pgfscope}%
\pgfpathrectangle{\pgfqpoint{1.000000in}{1.148311in}}{\pgfqpoint{6.200000in}{5.623377in}}%
\pgfusepath{clip}%
\pgfsetbuttcap%
\pgfsetroundjoin%
\definecolor{currentfill}{rgb}{0.200000,0.800000,0.200000}%
\pgfsetfillcolor{currentfill}%
\pgfsetlinewidth{1.003750pt}%
\definecolor{currentstroke}{rgb}{0.200000,0.800000,0.200000}%
\pgfsetstrokecolor{currentstroke}%
\pgfsetdash{}{0pt}%
\pgfpathmoveto{\pgfqpoint{4.127635in}{1.853022in}}%
\pgfpathcurveto{\pgfqpoint{4.133459in}{1.853022in}}{\pgfqpoint{4.139045in}{1.855335in}}{\pgfqpoint{4.143163in}{1.859454in}}%
\pgfpathcurveto{\pgfqpoint{4.147281in}{1.863572in}}{\pgfqpoint{4.149595in}{1.869158in}}{\pgfqpoint{4.149595in}{1.874982in}}%
\pgfpathcurveto{\pgfqpoint{4.149595in}{1.880806in}}{\pgfqpoint{4.147281in}{1.886392in}}{\pgfqpoint{4.143163in}{1.890510in}}%
\pgfpathcurveto{\pgfqpoint{4.139045in}{1.894628in}}{\pgfqpoint{4.133459in}{1.896942in}}{\pgfqpoint{4.127635in}{1.896942in}}%
\pgfpathcurveto{\pgfqpoint{4.121811in}{1.896942in}}{\pgfqpoint{4.116225in}{1.894628in}}{\pgfqpoint{4.112106in}{1.890510in}}%
\pgfpathcurveto{\pgfqpoint{4.107988in}{1.886392in}}{\pgfqpoint{4.105674in}{1.880806in}}{\pgfqpoint{4.105674in}{1.874982in}}%
\pgfpathcurveto{\pgfqpoint{4.105674in}{1.869158in}}{\pgfqpoint{4.107988in}{1.863572in}}{\pgfqpoint{4.112106in}{1.859454in}}%
\pgfpathcurveto{\pgfqpoint{4.116225in}{1.855335in}}{\pgfqpoint{4.121811in}{1.853022in}}{\pgfqpoint{4.127635in}{1.853022in}}%
\pgfpathlineto{\pgfqpoint{4.127635in}{1.853022in}}%
\pgfpathclose%
\pgfusepath{stroke,fill}%
\end{pgfscope}%
\begin{pgfscope}%
\pgfpathrectangle{\pgfqpoint{1.000000in}{1.148311in}}{\pgfqpoint{6.200000in}{5.623377in}}%
\pgfusepath{clip}%
\pgfsetbuttcap%
\pgfsetroundjoin%
\definecolor{currentfill}{rgb}{0.200000,0.800000,0.200000}%
\pgfsetfillcolor{currentfill}%
\pgfsetlinewidth{1.003750pt}%
\definecolor{currentstroke}{rgb}{0.200000,0.800000,0.200000}%
\pgfsetstrokecolor{currentstroke}%
\pgfsetdash{}{0pt}%
\pgfpathmoveto{\pgfqpoint{4.148007in}{1.878897in}}%
\pgfpathcurveto{\pgfqpoint{4.153831in}{1.878897in}}{\pgfqpoint{4.159417in}{1.881211in}}{\pgfqpoint{4.163535in}{1.885329in}}%
\pgfpathcurveto{\pgfqpoint{4.167654in}{1.889447in}}{\pgfqpoint{4.169967in}{1.895034in}}{\pgfqpoint{4.169967in}{1.900858in}}%
\pgfpathcurveto{\pgfqpoint{4.169967in}{1.906682in}}{\pgfqpoint{4.167654in}{1.912268in}}{\pgfqpoint{4.163535in}{1.916386in}}%
\pgfpathcurveto{\pgfqpoint{4.159417in}{1.920504in}}{\pgfqpoint{4.153831in}{1.922818in}}{\pgfqpoint{4.148007in}{1.922818in}}%
\pgfpathcurveto{\pgfqpoint{4.142183in}{1.922818in}}{\pgfqpoint{4.136597in}{1.920504in}}{\pgfqpoint{4.132479in}{1.916386in}}%
\pgfpathcurveto{\pgfqpoint{4.128361in}{1.912268in}}{\pgfqpoint{4.126047in}{1.906682in}}{\pgfqpoint{4.126047in}{1.900858in}}%
\pgfpathcurveto{\pgfqpoint{4.126047in}{1.895034in}}{\pgfqpoint{4.128361in}{1.889447in}}{\pgfqpoint{4.132479in}{1.885329in}}%
\pgfpathcurveto{\pgfqpoint{4.136597in}{1.881211in}}{\pgfqpoint{4.142183in}{1.878897in}}{\pgfqpoint{4.148007in}{1.878897in}}%
\pgfpathlineto{\pgfqpoint{4.148007in}{1.878897in}}%
\pgfpathclose%
\pgfusepath{stroke,fill}%
\end{pgfscope}%
\begin{pgfscope}%
\pgfpathrectangle{\pgfqpoint{1.000000in}{1.148311in}}{\pgfqpoint{6.200000in}{5.623377in}}%
\pgfusepath{clip}%
\pgfsetbuttcap%
\pgfsetroundjoin%
\definecolor{currentfill}{rgb}{0.200000,0.800000,0.200000}%
\pgfsetfillcolor{currentfill}%
\pgfsetlinewidth{1.003750pt}%
\definecolor{currentstroke}{rgb}{0.200000,0.800000,0.200000}%
\pgfsetstrokecolor{currentstroke}%
\pgfsetdash{}{0pt}%
\pgfpathmoveto{\pgfqpoint{4.206073in}{1.817075in}}%
\pgfpathcurveto{\pgfqpoint{4.211897in}{1.817075in}}{\pgfqpoint{4.217484in}{1.819388in}}{\pgfqpoint{4.221602in}{1.823507in}}%
\pgfpathcurveto{\pgfqpoint{4.225720in}{1.827625in}}{\pgfqpoint{4.228034in}{1.833211in}}{\pgfqpoint{4.228034in}{1.839035in}}%
\pgfpathcurveto{\pgfqpoint{4.228034in}{1.844859in}}{\pgfqpoint{4.225720in}{1.850445in}}{\pgfqpoint{4.221602in}{1.854563in}}%
\pgfpathcurveto{\pgfqpoint{4.217484in}{1.858681in}}{\pgfqpoint{4.211897in}{1.860995in}}{\pgfqpoint{4.206073in}{1.860995in}}%
\pgfpathcurveto{\pgfqpoint{4.200250in}{1.860995in}}{\pgfqpoint{4.194663in}{1.858681in}}{\pgfqpoint{4.190545in}{1.854563in}}%
\pgfpathcurveto{\pgfqpoint{4.186427in}{1.850445in}}{\pgfqpoint{4.184113in}{1.844859in}}{\pgfqpoint{4.184113in}{1.839035in}}%
\pgfpathcurveto{\pgfqpoint{4.184113in}{1.833211in}}{\pgfqpoint{4.186427in}{1.827625in}}{\pgfqpoint{4.190545in}{1.823507in}}%
\pgfpathcurveto{\pgfqpoint{4.194663in}{1.819388in}}{\pgfqpoint{4.200250in}{1.817075in}}{\pgfqpoint{4.206073in}{1.817075in}}%
\pgfpathlineto{\pgfqpoint{4.206073in}{1.817075in}}%
\pgfpathclose%
\pgfusepath{stroke,fill}%
\end{pgfscope}%
\begin{pgfscope}%
\pgfpathrectangle{\pgfqpoint{1.000000in}{1.148311in}}{\pgfqpoint{6.200000in}{5.623377in}}%
\pgfusepath{clip}%
\pgfsetbuttcap%
\pgfsetroundjoin%
\definecolor{currentfill}{rgb}{0.200000,0.800000,0.200000}%
\pgfsetfillcolor{currentfill}%
\pgfsetlinewidth{1.003750pt}%
\definecolor{currentstroke}{rgb}{0.200000,0.800000,0.200000}%
\pgfsetstrokecolor{currentstroke}%
\pgfsetdash{}{0pt}%
\pgfpathmoveto{\pgfqpoint{4.256451in}{1.791186in}}%
\pgfpathcurveto{\pgfqpoint{4.262274in}{1.791186in}}{\pgfqpoint{4.267861in}{1.793500in}}{\pgfqpoint{4.271979in}{1.797618in}}%
\pgfpathcurveto{\pgfqpoint{4.276097in}{1.801737in}}{\pgfqpoint{4.278411in}{1.807323in}}{\pgfqpoint{4.278411in}{1.813147in}}%
\pgfpathcurveto{\pgfqpoint{4.278411in}{1.818971in}}{\pgfqpoint{4.276097in}{1.824557in}}{\pgfqpoint{4.271979in}{1.828675in}}%
\pgfpathcurveto{\pgfqpoint{4.267861in}{1.832793in}}{\pgfqpoint{4.262274in}{1.835107in}}{\pgfqpoint{4.256451in}{1.835107in}}%
\pgfpathcurveto{\pgfqpoint{4.250627in}{1.835107in}}{\pgfqpoint{4.245040in}{1.832793in}}{\pgfqpoint{4.240922in}{1.828675in}}%
\pgfpathcurveto{\pgfqpoint{4.236804in}{1.824557in}}{\pgfqpoint{4.234490in}{1.818971in}}{\pgfqpoint{4.234490in}{1.813147in}}%
\pgfpathcurveto{\pgfqpoint{4.234490in}{1.807323in}}{\pgfqpoint{4.236804in}{1.801737in}}{\pgfqpoint{4.240922in}{1.797618in}}%
\pgfpathcurveto{\pgfqpoint{4.245040in}{1.793500in}}{\pgfqpoint{4.250627in}{1.791186in}}{\pgfqpoint{4.256451in}{1.791186in}}%
\pgfpathlineto{\pgfqpoint{4.256451in}{1.791186in}}%
\pgfpathclose%
\pgfusepath{stroke,fill}%
\end{pgfscope}%
\begin{pgfscope}%
\pgfpathrectangle{\pgfqpoint{1.000000in}{1.148311in}}{\pgfqpoint{6.200000in}{5.623377in}}%
\pgfusepath{clip}%
\pgfsetbuttcap%
\pgfsetroundjoin%
\definecolor{currentfill}{rgb}{0.200000,0.800000,0.200000}%
\pgfsetfillcolor{currentfill}%
\pgfsetlinewidth{1.003750pt}%
\definecolor{currentstroke}{rgb}{0.200000,0.800000,0.200000}%
\pgfsetstrokecolor{currentstroke}%
\pgfsetdash{}{0pt}%
\pgfpathmoveto{\pgfqpoint{4.217734in}{1.920104in}}%
\pgfpathcurveto{\pgfqpoint{4.223558in}{1.920104in}}{\pgfqpoint{4.229144in}{1.922418in}}{\pgfqpoint{4.233262in}{1.926536in}}%
\pgfpathcurveto{\pgfqpoint{4.237380in}{1.930655in}}{\pgfqpoint{4.239694in}{1.936241in}}{\pgfqpoint{4.239694in}{1.942065in}}%
\pgfpathcurveto{\pgfqpoint{4.239694in}{1.947889in}}{\pgfqpoint{4.237380in}{1.953475in}}{\pgfqpoint{4.233262in}{1.957593in}}%
\pgfpathcurveto{\pgfqpoint{4.229144in}{1.961711in}}{\pgfqpoint{4.223558in}{1.964025in}}{\pgfqpoint{4.217734in}{1.964025in}}%
\pgfpathcurveto{\pgfqpoint{4.211910in}{1.964025in}}{\pgfqpoint{4.206324in}{1.961711in}}{\pgfqpoint{4.202206in}{1.957593in}}%
\pgfpathcurveto{\pgfqpoint{4.198087in}{1.953475in}}{\pgfqpoint{4.195774in}{1.947889in}}{\pgfqpoint{4.195774in}{1.942065in}}%
\pgfpathcurveto{\pgfqpoint{4.195774in}{1.936241in}}{\pgfqpoint{4.198087in}{1.930655in}}{\pgfqpoint{4.202206in}{1.926536in}}%
\pgfpathcurveto{\pgfqpoint{4.206324in}{1.922418in}}{\pgfqpoint{4.211910in}{1.920104in}}{\pgfqpoint{4.217734in}{1.920104in}}%
\pgfpathlineto{\pgfqpoint{4.217734in}{1.920104in}}%
\pgfpathclose%
\pgfusepath{stroke,fill}%
\end{pgfscope}%
\begin{pgfscope}%
\pgfpathrectangle{\pgfqpoint{1.000000in}{1.148311in}}{\pgfqpoint{6.200000in}{5.623377in}}%
\pgfusepath{clip}%
\pgfsetbuttcap%
\pgfsetroundjoin%
\definecolor{currentfill}{rgb}{0.200000,0.800000,0.200000}%
\pgfsetfillcolor{currentfill}%
\pgfsetlinewidth{1.003750pt}%
\definecolor{currentstroke}{rgb}{0.200000,0.800000,0.200000}%
\pgfsetstrokecolor{currentstroke}%
\pgfsetdash{}{0pt}%
\pgfpathmoveto{\pgfqpoint{4.265658in}{1.898933in}}%
\pgfpathcurveto{\pgfqpoint{4.271482in}{1.898933in}}{\pgfqpoint{4.277068in}{1.901247in}}{\pgfqpoint{4.281186in}{1.905365in}}%
\pgfpathcurveto{\pgfqpoint{4.285304in}{1.909483in}}{\pgfqpoint{4.287618in}{1.915070in}}{\pgfqpoint{4.287618in}{1.920893in}}%
\pgfpathcurveto{\pgfqpoint{4.287618in}{1.926717in}}{\pgfqpoint{4.285304in}{1.932304in}}{\pgfqpoint{4.281186in}{1.936422in}}%
\pgfpathcurveto{\pgfqpoint{4.277068in}{1.940540in}}{\pgfqpoint{4.271482in}{1.942854in}}{\pgfqpoint{4.265658in}{1.942854in}}%
\pgfpathcurveto{\pgfqpoint{4.259834in}{1.942854in}}{\pgfqpoint{4.254248in}{1.940540in}}{\pgfqpoint{4.250130in}{1.936422in}}%
\pgfpathcurveto{\pgfqpoint{4.246011in}{1.932304in}}{\pgfqpoint{4.243698in}{1.926717in}}{\pgfqpoint{4.243698in}{1.920893in}}%
\pgfpathcurveto{\pgfqpoint{4.243698in}{1.915070in}}{\pgfqpoint{4.246011in}{1.909483in}}{\pgfqpoint{4.250130in}{1.905365in}}%
\pgfpathcurveto{\pgfqpoint{4.254248in}{1.901247in}}{\pgfqpoint{4.259834in}{1.898933in}}{\pgfqpoint{4.265658in}{1.898933in}}%
\pgfpathlineto{\pgfqpoint{4.265658in}{1.898933in}}%
\pgfpathclose%
\pgfusepath{stroke,fill}%
\end{pgfscope}%
\begin{pgfscope}%
\pgfpathrectangle{\pgfqpoint{1.000000in}{1.148311in}}{\pgfqpoint{6.200000in}{5.623377in}}%
\pgfusepath{clip}%
\pgfsetbuttcap%
\pgfsetroundjoin%
\definecolor{currentfill}{rgb}{0.200000,0.800000,0.200000}%
\pgfsetfillcolor{currentfill}%
\pgfsetlinewidth{1.003750pt}%
\definecolor{currentstroke}{rgb}{0.200000,0.800000,0.200000}%
\pgfsetstrokecolor{currentstroke}%
\pgfsetdash{}{0pt}%
\pgfpathmoveto{\pgfqpoint{4.274321in}{1.935150in}}%
\pgfpathcurveto{\pgfqpoint{4.280145in}{1.935150in}}{\pgfqpoint{4.285731in}{1.937464in}}{\pgfqpoint{4.289849in}{1.941582in}}%
\pgfpathcurveto{\pgfqpoint{4.293967in}{1.945700in}}{\pgfqpoint{4.296281in}{1.951286in}}{\pgfqpoint{4.296281in}{1.957110in}}%
\pgfpathcurveto{\pgfqpoint{4.296281in}{1.962934in}}{\pgfqpoint{4.293967in}{1.968520in}}{\pgfqpoint{4.289849in}{1.972639in}}%
\pgfpathcurveto{\pgfqpoint{4.285731in}{1.976757in}}{\pgfqpoint{4.280145in}{1.979071in}}{\pgfqpoint{4.274321in}{1.979071in}}%
\pgfpathcurveto{\pgfqpoint{4.268497in}{1.979071in}}{\pgfqpoint{4.262910in}{1.976757in}}{\pgfqpoint{4.258792in}{1.972639in}}%
\pgfpathcurveto{\pgfqpoint{4.254674in}{1.968520in}}{\pgfqpoint{4.252360in}{1.962934in}}{\pgfqpoint{4.252360in}{1.957110in}}%
\pgfpathcurveto{\pgfqpoint{4.252360in}{1.951286in}}{\pgfqpoint{4.254674in}{1.945700in}}{\pgfqpoint{4.258792in}{1.941582in}}%
\pgfpathcurveto{\pgfqpoint{4.262910in}{1.937464in}}{\pgfqpoint{4.268497in}{1.935150in}}{\pgfqpoint{4.274321in}{1.935150in}}%
\pgfpathlineto{\pgfqpoint{4.274321in}{1.935150in}}%
\pgfpathclose%
\pgfusepath{stroke,fill}%
\end{pgfscope}%
\begin{pgfscope}%
\pgfpathrectangle{\pgfqpoint{1.000000in}{1.148311in}}{\pgfqpoint{6.200000in}{5.623377in}}%
\pgfusepath{clip}%
\pgfsetbuttcap%
\pgfsetroundjoin%
\definecolor{currentfill}{rgb}{0.200000,0.800000,0.200000}%
\pgfsetfillcolor{currentfill}%
\pgfsetlinewidth{1.003750pt}%
\definecolor{currentstroke}{rgb}{0.200000,0.800000,0.200000}%
\pgfsetstrokecolor{currentstroke}%
\pgfsetdash{}{0pt}%
\pgfpathmoveto{\pgfqpoint{4.367699in}{1.875415in}}%
\pgfpathcurveto{\pgfqpoint{4.373523in}{1.875415in}}{\pgfqpoint{4.379109in}{1.877729in}}{\pgfqpoint{4.383227in}{1.881847in}}%
\pgfpathcurveto{\pgfqpoint{4.387345in}{1.885965in}}{\pgfqpoint{4.389659in}{1.891552in}}{\pgfqpoint{4.389659in}{1.897376in}}%
\pgfpathcurveto{\pgfqpoint{4.389659in}{1.903199in}}{\pgfqpoint{4.387345in}{1.908786in}}{\pgfqpoint{4.383227in}{1.912904in}}%
\pgfpathcurveto{\pgfqpoint{4.379109in}{1.917022in}}{\pgfqpoint{4.373523in}{1.919336in}}{\pgfqpoint{4.367699in}{1.919336in}}%
\pgfpathcurveto{\pgfqpoint{4.361875in}{1.919336in}}{\pgfqpoint{4.356289in}{1.917022in}}{\pgfqpoint{4.352171in}{1.912904in}}%
\pgfpathcurveto{\pgfqpoint{4.348053in}{1.908786in}}{\pgfqpoint{4.345739in}{1.903199in}}{\pgfqpoint{4.345739in}{1.897376in}}%
\pgfpathcurveto{\pgfqpoint{4.345739in}{1.891552in}}{\pgfqpoint{4.348053in}{1.885965in}}{\pgfqpoint{4.352171in}{1.881847in}}%
\pgfpathcurveto{\pgfqpoint{4.356289in}{1.877729in}}{\pgfqpoint{4.361875in}{1.875415in}}{\pgfqpoint{4.367699in}{1.875415in}}%
\pgfpathlineto{\pgfqpoint{4.367699in}{1.875415in}}%
\pgfpathclose%
\pgfusepath{stroke,fill}%
\end{pgfscope}%
\begin{pgfscope}%
\pgfpathrectangle{\pgfqpoint{1.000000in}{1.148311in}}{\pgfqpoint{6.200000in}{5.623377in}}%
\pgfusepath{clip}%
\pgfsetbuttcap%
\pgfsetroundjoin%
\definecolor{currentfill}{rgb}{0.200000,0.800000,0.200000}%
\pgfsetfillcolor{currentfill}%
\pgfsetlinewidth{1.003750pt}%
\definecolor{currentstroke}{rgb}{0.200000,0.800000,0.200000}%
\pgfsetstrokecolor{currentstroke}%
\pgfsetdash{}{0pt}%
\pgfpathmoveto{\pgfqpoint{4.341052in}{1.948920in}}%
\pgfpathcurveto{\pgfqpoint{4.346876in}{1.948920in}}{\pgfqpoint{4.352462in}{1.951234in}}{\pgfqpoint{4.356581in}{1.955352in}}%
\pgfpathcurveto{\pgfqpoint{4.360699in}{1.959470in}}{\pgfqpoint{4.363013in}{1.965056in}}{\pgfqpoint{4.363013in}{1.970880in}}%
\pgfpathcurveto{\pgfqpoint{4.363013in}{1.976704in}}{\pgfqpoint{4.360699in}{1.982290in}}{\pgfqpoint{4.356581in}{1.986408in}}%
\pgfpathcurveto{\pgfqpoint{4.352462in}{1.990526in}}{\pgfqpoint{4.346876in}{1.992840in}}{\pgfqpoint{4.341052in}{1.992840in}}%
\pgfpathcurveto{\pgfqpoint{4.335228in}{1.992840in}}{\pgfqpoint{4.329642in}{1.990526in}}{\pgfqpoint{4.325524in}{1.986408in}}%
\pgfpathcurveto{\pgfqpoint{4.321406in}{1.982290in}}{\pgfqpoint{4.319092in}{1.976704in}}{\pgfqpoint{4.319092in}{1.970880in}}%
\pgfpathcurveto{\pgfqpoint{4.319092in}{1.965056in}}{\pgfqpoint{4.321406in}{1.959470in}}{\pgfqpoint{4.325524in}{1.955352in}}%
\pgfpathcurveto{\pgfqpoint{4.329642in}{1.951234in}}{\pgfqpoint{4.335228in}{1.948920in}}{\pgfqpoint{4.341052in}{1.948920in}}%
\pgfpathlineto{\pgfqpoint{4.341052in}{1.948920in}}%
\pgfpathclose%
\pgfusepath{stroke,fill}%
\end{pgfscope}%
\begin{pgfscope}%
\pgfpathrectangle{\pgfqpoint{1.000000in}{1.148311in}}{\pgfqpoint{6.200000in}{5.623377in}}%
\pgfusepath{clip}%
\pgfsetbuttcap%
\pgfsetroundjoin%
\definecolor{currentfill}{rgb}{0.200000,0.800000,0.200000}%
\pgfsetfillcolor{currentfill}%
\pgfsetlinewidth{1.003750pt}%
\definecolor{currentstroke}{rgb}{0.200000,0.800000,0.200000}%
\pgfsetstrokecolor{currentstroke}%
\pgfsetdash{}{0pt}%
\pgfpathmoveto{\pgfqpoint{4.374461in}{1.960439in}}%
\pgfpathcurveto{\pgfqpoint{4.380285in}{1.960439in}}{\pgfqpoint{4.385871in}{1.962753in}}{\pgfqpoint{4.389989in}{1.966871in}}%
\pgfpathcurveto{\pgfqpoint{4.394108in}{1.970989in}}{\pgfqpoint{4.396421in}{1.976575in}}{\pgfqpoint{4.396421in}{1.982399in}}%
\pgfpathcurveto{\pgfqpoint{4.396421in}{1.988223in}}{\pgfqpoint{4.394108in}{1.993809in}}{\pgfqpoint{4.389989in}{1.997927in}}%
\pgfpathcurveto{\pgfqpoint{4.385871in}{2.002045in}}{\pgfqpoint{4.380285in}{2.004359in}}{\pgfqpoint{4.374461in}{2.004359in}}%
\pgfpathcurveto{\pgfqpoint{4.368637in}{2.004359in}}{\pgfqpoint{4.363051in}{2.002045in}}{\pgfqpoint{4.358933in}{1.997927in}}%
\pgfpathcurveto{\pgfqpoint{4.354815in}{1.993809in}}{\pgfqpoint{4.352501in}{1.988223in}}{\pgfqpoint{4.352501in}{1.982399in}}%
\pgfpathcurveto{\pgfqpoint{4.352501in}{1.976575in}}{\pgfqpoint{4.354815in}{1.970989in}}{\pgfqpoint{4.358933in}{1.966871in}}%
\pgfpathcurveto{\pgfqpoint{4.363051in}{1.962753in}}{\pgfqpoint{4.368637in}{1.960439in}}{\pgfqpoint{4.374461in}{1.960439in}}%
\pgfpathlineto{\pgfqpoint{4.374461in}{1.960439in}}%
\pgfpathclose%
\pgfusepath{stroke,fill}%
\end{pgfscope}%
\begin{pgfscope}%
\pgfpathrectangle{\pgfqpoint{1.000000in}{1.148311in}}{\pgfqpoint{6.200000in}{5.623377in}}%
\pgfusepath{clip}%
\pgfsetbuttcap%
\pgfsetroundjoin%
\definecolor{currentfill}{rgb}{0.200000,0.800000,0.200000}%
\pgfsetfillcolor{currentfill}%
\pgfsetlinewidth{1.003750pt}%
\definecolor{currentstroke}{rgb}{0.200000,0.800000,0.200000}%
\pgfsetstrokecolor{currentstroke}%
\pgfsetdash{}{0pt}%
\pgfpathmoveto{\pgfqpoint{4.402000in}{1.979145in}}%
\pgfpathcurveto{\pgfqpoint{4.407824in}{1.979145in}}{\pgfqpoint{4.413410in}{1.981459in}}{\pgfqpoint{4.417528in}{1.985577in}}%
\pgfpathcurveto{\pgfqpoint{4.421646in}{1.989695in}}{\pgfqpoint{4.423960in}{1.995282in}}{\pgfqpoint{4.423960in}{2.001106in}}%
\pgfpathcurveto{\pgfqpoint{4.423960in}{2.006929in}}{\pgfqpoint{4.421646in}{2.012516in}}{\pgfqpoint{4.417528in}{2.016634in}}%
\pgfpathcurveto{\pgfqpoint{4.413410in}{2.020752in}}{\pgfqpoint{4.407824in}{2.023066in}}{\pgfqpoint{4.402000in}{2.023066in}}%
\pgfpathcurveto{\pgfqpoint{4.396176in}{2.023066in}}{\pgfqpoint{4.390590in}{2.020752in}}{\pgfqpoint{4.386472in}{2.016634in}}%
\pgfpathcurveto{\pgfqpoint{4.382353in}{2.012516in}}{\pgfqpoint{4.380040in}{2.006929in}}{\pgfqpoint{4.380040in}{2.001106in}}%
\pgfpathcurveto{\pgfqpoint{4.380040in}{1.995282in}}{\pgfqpoint{4.382353in}{1.989695in}}{\pgfqpoint{4.386472in}{1.985577in}}%
\pgfpathcurveto{\pgfqpoint{4.390590in}{1.981459in}}{\pgfqpoint{4.396176in}{1.979145in}}{\pgfqpoint{4.402000in}{1.979145in}}%
\pgfpathlineto{\pgfqpoint{4.402000in}{1.979145in}}%
\pgfpathclose%
\pgfusepath{stroke,fill}%
\end{pgfscope}%
\begin{pgfscope}%
\pgfpathrectangle{\pgfqpoint{1.000000in}{1.148311in}}{\pgfqpoint{6.200000in}{5.623377in}}%
\pgfusepath{clip}%
\pgfsetbuttcap%
\pgfsetroundjoin%
\definecolor{currentfill}{rgb}{0.200000,0.800000,0.200000}%
\pgfsetfillcolor{currentfill}%
\pgfsetlinewidth{1.003750pt}%
\definecolor{currentstroke}{rgb}{0.200000,0.800000,0.200000}%
\pgfsetstrokecolor{currentstroke}%
\pgfsetdash{}{0pt}%
\pgfpathmoveto{\pgfqpoint{4.385154in}{2.028047in}}%
\pgfpathcurveto{\pgfqpoint{4.390978in}{2.028047in}}{\pgfqpoint{4.396564in}{2.030361in}}{\pgfqpoint{4.400682in}{2.034479in}}%
\pgfpathcurveto{\pgfqpoint{4.404801in}{2.038597in}}{\pgfqpoint{4.407114in}{2.044183in}}{\pgfqpoint{4.407114in}{2.050007in}}%
\pgfpathcurveto{\pgfqpoint{4.407114in}{2.055831in}}{\pgfqpoint{4.404801in}{2.061418in}}{\pgfqpoint{4.400682in}{2.065536in}}%
\pgfpathcurveto{\pgfqpoint{4.396564in}{2.069654in}}{\pgfqpoint{4.390978in}{2.071968in}}{\pgfqpoint{4.385154in}{2.071968in}}%
\pgfpathcurveto{\pgfqpoint{4.379330in}{2.071968in}}{\pgfqpoint{4.373744in}{2.069654in}}{\pgfqpoint{4.369626in}{2.065536in}}%
\pgfpathcurveto{\pgfqpoint{4.365508in}{2.061418in}}{\pgfqpoint{4.363194in}{2.055831in}}{\pgfqpoint{4.363194in}{2.050007in}}%
\pgfpathcurveto{\pgfqpoint{4.363194in}{2.044183in}}{\pgfqpoint{4.365508in}{2.038597in}}{\pgfqpoint{4.369626in}{2.034479in}}%
\pgfpathcurveto{\pgfqpoint{4.373744in}{2.030361in}}{\pgfqpoint{4.379330in}{2.028047in}}{\pgfqpoint{4.385154in}{2.028047in}}%
\pgfpathlineto{\pgfqpoint{4.385154in}{2.028047in}}%
\pgfpathclose%
\pgfusepath{stroke,fill}%
\end{pgfscope}%
\begin{pgfscope}%
\pgfpathrectangle{\pgfqpoint{1.000000in}{1.148311in}}{\pgfqpoint{6.200000in}{5.623377in}}%
\pgfusepath{clip}%
\pgfsetbuttcap%
\pgfsetroundjoin%
\definecolor{currentfill}{rgb}{0.200000,0.800000,0.200000}%
\pgfsetfillcolor{currentfill}%
\pgfsetlinewidth{1.003750pt}%
\definecolor{currentstroke}{rgb}{0.200000,0.800000,0.200000}%
\pgfsetstrokecolor{currentstroke}%
\pgfsetdash{}{0pt}%
\pgfpathmoveto{\pgfqpoint{4.389311in}{2.059369in}}%
\pgfpathcurveto{\pgfqpoint{4.395135in}{2.059369in}}{\pgfqpoint{4.400721in}{2.061682in}}{\pgfqpoint{4.404839in}{2.065801in}}%
\pgfpathcurveto{\pgfqpoint{4.408958in}{2.069919in}}{\pgfqpoint{4.411271in}{2.075505in}}{\pgfqpoint{4.411271in}{2.081329in}}%
\pgfpathcurveto{\pgfqpoint{4.411271in}{2.087153in}}{\pgfqpoint{4.408958in}{2.092739in}}{\pgfqpoint{4.404839in}{2.096857in}}%
\pgfpathcurveto{\pgfqpoint{4.400721in}{2.100975in}}{\pgfqpoint{4.395135in}{2.103289in}}{\pgfqpoint{4.389311in}{2.103289in}}%
\pgfpathcurveto{\pgfqpoint{4.383487in}{2.103289in}}{\pgfqpoint{4.377901in}{2.100975in}}{\pgfqpoint{4.373783in}{2.096857in}}%
\pgfpathcurveto{\pgfqpoint{4.369665in}{2.092739in}}{\pgfqpoint{4.367351in}{2.087153in}}{\pgfqpoint{4.367351in}{2.081329in}}%
\pgfpathcurveto{\pgfqpoint{4.367351in}{2.075505in}}{\pgfqpoint{4.369665in}{2.069919in}}{\pgfqpoint{4.373783in}{2.065801in}}%
\pgfpathcurveto{\pgfqpoint{4.377901in}{2.061682in}}{\pgfqpoint{4.383487in}{2.059369in}}{\pgfqpoint{4.389311in}{2.059369in}}%
\pgfpathlineto{\pgfqpoint{4.389311in}{2.059369in}}%
\pgfpathclose%
\pgfusepath{stroke,fill}%
\end{pgfscope}%
\begin{pgfscope}%
\pgfpathrectangle{\pgfqpoint{1.000000in}{1.148311in}}{\pgfqpoint{6.200000in}{5.623377in}}%
\pgfusepath{clip}%
\pgfsetbuttcap%
\pgfsetroundjoin%
\definecolor{currentfill}{rgb}{0.200000,0.800000,0.200000}%
\pgfsetfillcolor{currentfill}%
\pgfsetlinewidth{1.003750pt}%
\definecolor{currentstroke}{rgb}{0.200000,0.800000,0.200000}%
\pgfsetstrokecolor{currentstroke}%
\pgfsetdash{}{0pt}%
\pgfpathmoveto{\pgfqpoint{4.463367in}{2.055789in}}%
\pgfpathcurveto{\pgfqpoint{4.469191in}{2.055789in}}{\pgfqpoint{4.474778in}{2.058103in}}{\pgfqpoint{4.478896in}{2.062221in}}%
\pgfpathcurveto{\pgfqpoint{4.483014in}{2.066339in}}{\pgfqpoint{4.485328in}{2.071925in}}{\pgfqpoint{4.485328in}{2.077749in}}%
\pgfpathcurveto{\pgfqpoint{4.485328in}{2.083573in}}{\pgfqpoint{4.483014in}{2.089159in}}{\pgfqpoint{4.478896in}{2.093277in}}%
\pgfpathcurveto{\pgfqpoint{4.474778in}{2.097395in}}{\pgfqpoint{4.469191in}{2.099709in}}{\pgfqpoint{4.463367in}{2.099709in}}%
\pgfpathcurveto{\pgfqpoint{4.457543in}{2.099709in}}{\pgfqpoint{4.451957in}{2.097395in}}{\pgfqpoint{4.447839in}{2.093277in}}%
\pgfpathcurveto{\pgfqpoint{4.443721in}{2.089159in}}{\pgfqpoint{4.441407in}{2.083573in}}{\pgfqpoint{4.441407in}{2.077749in}}%
\pgfpathcurveto{\pgfqpoint{4.441407in}{2.071925in}}{\pgfqpoint{4.443721in}{2.066339in}}{\pgfqpoint{4.447839in}{2.062221in}}%
\pgfpathcurveto{\pgfqpoint{4.451957in}{2.058103in}}{\pgfqpoint{4.457543in}{2.055789in}}{\pgfqpoint{4.463367in}{2.055789in}}%
\pgfpathlineto{\pgfqpoint{4.463367in}{2.055789in}}%
\pgfpathclose%
\pgfusepath{stroke,fill}%
\end{pgfscope}%
\begin{pgfscope}%
\pgfpathrectangle{\pgfqpoint{1.000000in}{1.148311in}}{\pgfqpoint{6.200000in}{5.623377in}}%
\pgfusepath{clip}%
\pgfsetbuttcap%
\pgfsetroundjoin%
\definecolor{currentfill}{rgb}{0.200000,0.800000,0.200000}%
\pgfsetfillcolor{currentfill}%
\pgfsetlinewidth{1.003750pt}%
\definecolor{currentstroke}{rgb}{0.200000,0.800000,0.200000}%
\pgfsetstrokecolor{currentstroke}%
\pgfsetdash{}{0pt}%
\pgfpathmoveto{\pgfqpoint{4.449325in}{2.096825in}}%
\pgfpathcurveto{\pgfqpoint{4.455149in}{2.096825in}}{\pgfqpoint{4.460736in}{2.099139in}}{\pgfqpoint{4.464854in}{2.103257in}}%
\pgfpathcurveto{\pgfqpoint{4.468972in}{2.107375in}}{\pgfqpoint{4.471286in}{2.112961in}}{\pgfqpoint{4.471286in}{2.118785in}}%
\pgfpathcurveto{\pgfqpoint{4.471286in}{2.124609in}}{\pgfqpoint{4.468972in}{2.130195in}}{\pgfqpoint{4.464854in}{2.134313in}}%
\pgfpathcurveto{\pgfqpoint{4.460736in}{2.138431in}}{\pgfqpoint{4.455149in}{2.140745in}}{\pgfqpoint{4.449325in}{2.140745in}}%
\pgfpathcurveto{\pgfqpoint{4.443502in}{2.140745in}}{\pgfqpoint{4.437915in}{2.138431in}}{\pgfqpoint{4.433797in}{2.134313in}}%
\pgfpathcurveto{\pgfqpoint{4.429679in}{2.130195in}}{\pgfqpoint{4.427365in}{2.124609in}}{\pgfqpoint{4.427365in}{2.118785in}}%
\pgfpathcurveto{\pgfqpoint{4.427365in}{2.112961in}}{\pgfqpoint{4.429679in}{2.107375in}}{\pgfqpoint{4.433797in}{2.103257in}}%
\pgfpathcurveto{\pgfqpoint{4.437915in}{2.099139in}}{\pgfqpoint{4.443502in}{2.096825in}}{\pgfqpoint{4.449325in}{2.096825in}}%
\pgfpathlineto{\pgfqpoint{4.449325in}{2.096825in}}%
\pgfpathclose%
\pgfusepath{stroke,fill}%
\end{pgfscope}%
\begin{pgfscope}%
\pgfpathrectangle{\pgfqpoint{1.000000in}{1.148311in}}{\pgfqpoint{6.200000in}{5.623377in}}%
\pgfusepath{clip}%
\pgfsetbuttcap%
\pgfsetroundjoin%
\definecolor{currentfill}{rgb}{0.200000,0.800000,0.200000}%
\pgfsetfillcolor{currentfill}%
\pgfsetlinewidth{1.003750pt}%
\definecolor{currentstroke}{rgb}{0.200000,0.800000,0.200000}%
\pgfsetstrokecolor{currentstroke}%
\pgfsetdash{}{0pt}%
\pgfpathmoveto{\pgfqpoint{4.398877in}{2.146041in}}%
\pgfpathcurveto{\pgfqpoint{4.404701in}{2.146041in}}{\pgfqpoint{4.410287in}{2.148355in}}{\pgfqpoint{4.414405in}{2.152473in}}%
\pgfpathcurveto{\pgfqpoint{4.418524in}{2.156591in}}{\pgfqpoint{4.420837in}{2.162177in}}{\pgfqpoint{4.420837in}{2.168001in}}%
\pgfpathcurveto{\pgfqpoint{4.420837in}{2.173825in}}{\pgfqpoint{4.418524in}{2.179411in}}{\pgfqpoint{4.414405in}{2.183530in}}%
\pgfpathcurveto{\pgfqpoint{4.410287in}{2.187648in}}{\pgfqpoint{4.404701in}{2.189962in}}{\pgfqpoint{4.398877in}{2.189962in}}%
\pgfpathcurveto{\pgfqpoint{4.393053in}{2.189962in}}{\pgfqpoint{4.387467in}{2.187648in}}{\pgfqpoint{4.383349in}{2.183530in}}%
\pgfpathcurveto{\pgfqpoint{4.379231in}{2.179411in}}{\pgfqpoint{4.376917in}{2.173825in}}{\pgfqpoint{4.376917in}{2.168001in}}%
\pgfpathcurveto{\pgfqpoint{4.376917in}{2.162177in}}{\pgfqpoint{4.379231in}{2.156591in}}{\pgfqpoint{4.383349in}{2.152473in}}%
\pgfpathcurveto{\pgfqpoint{4.387467in}{2.148355in}}{\pgfqpoint{4.393053in}{2.146041in}}{\pgfqpoint{4.398877in}{2.146041in}}%
\pgfpathlineto{\pgfqpoint{4.398877in}{2.146041in}}%
\pgfpathclose%
\pgfusepath{stroke,fill}%
\end{pgfscope}%
\begin{pgfscope}%
\pgfpathrectangle{\pgfqpoint{1.000000in}{1.148311in}}{\pgfqpoint{6.200000in}{5.623377in}}%
\pgfusepath{clip}%
\pgfsetbuttcap%
\pgfsetroundjoin%
\definecolor{currentfill}{rgb}{0.200000,0.800000,0.200000}%
\pgfsetfillcolor{currentfill}%
\pgfsetlinewidth{1.003750pt}%
\definecolor{currentstroke}{rgb}{0.200000,0.800000,0.200000}%
\pgfsetstrokecolor{currentstroke}%
\pgfsetdash{}{0pt}%
\pgfpathmoveto{\pgfqpoint{4.411133in}{2.170694in}}%
\pgfpathcurveto{\pgfqpoint{4.416957in}{2.170694in}}{\pgfqpoint{4.422543in}{2.173008in}}{\pgfqpoint{4.426662in}{2.177126in}}%
\pgfpathcurveto{\pgfqpoint{4.430780in}{2.181244in}}{\pgfqpoint{4.433094in}{2.186831in}}{\pgfqpoint{4.433094in}{2.192654in}}%
\pgfpathcurveto{\pgfqpoint{4.433094in}{2.198478in}}{\pgfqpoint{4.430780in}{2.204065in}}{\pgfqpoint{4.426662in}{2.208183in}}%
\pgfpathcurveto{\pgfqpoint{4.422543in}{2.212301in}}{\pgfqpoint{4.416957in}{2.214615in}}{\pgfqpoint{4.411133in}{2.214615in}}%
\pgfpathcurveto{\pgfqpoint{4.405309in}{2.214615in}}{\pgfqpoint{4.399723in}{2.212301in}}{\pgfqpoint{4.395605in}{2.208183in}}%
\pgfpathcurveto{\pgfqpoint{4.391487in}{2.204065in}}{\pgfqpoint{4.389173in}{2.198478in}}{\pgfqpoint{4.389173in}{2.192654in}}%
\pgfpathcurveto{\pgfqpoint{4.389173in}{2.186831in}}{\pgfqpoint{4.391487in}{2.181244in}}{\pgfqpoint{4.395605in}{2.177126in}}%
\pgfpathcurveto{\pgfqpoint{4.399723in}{2.173008in}}{\pgfqpoint{4.405309in}{2.170694in}}{\pgfqpoint{4.411133in}{2.170694in}}%
\pgfpathlineto{\pgfqpoint{4.411133in}{2.170694in}}%
\pgfpathclose%
\pgfusepath{stroke,fill}%
\end{pgfscope}%
\begin{pgfscope}%
\pgfpathrectangle{\pgfqpoint{1.000000in}{1.148311in}}{\pgfqpoint{6.200000in}{5.623377in}}%
\pgfusepath{clip}%
\pgfsetbuttcap%
\pgfsetroundjoin%
\definecolor{currentfill}{rgb}{0.200000,0.800000,0.200000}%
\pgfsetfillcolor{currentfill}%
\pgfsetlinewidth{1.003750pt}%
\definecolor{currentstroke}{rgb}{0.200000,0.800000,0.200000}%
\pgfsetstrokecolor{currentstroke}%
\pgfsetdash{}{0pt}%
\pgfpathmoveto{\pgfqpoint{4.430142in}{2.194779in}}%
\pgfpathcurveto{\pgfqpoint{4.435966in}{2.194779in}}{\pgfqpoint{4.441552in}{2.197092in}}{\pgfqpoint{4.445670in}{2.201211in}}%
\pgfpathcurveto{\pgfqpoint{4.449788in}{2.205329in}}{\pgfqpoint{4.452102in}{2.210915in}}{\pgfqpoint{4.452102in}{2.216739in}}%
\pgfpathcurveto{\pgfqpoint{4.452102in}{2.222563in}}{\pgfqpoint{4.449788in}{2.228149in}}{\pgfqpoint{4.445670in}{2.232267in}}%
\pgfpathcurveto{\pgfqpoint{4.441552in}{2.236385in}}{\pgfqpoint{4.435966in}{2.238699in}}{\pgfqpoint{4.430142in}{2.238699in}}%
\pgfpathcurveto{\pgfqpoint{4.424318in}{2.238699in}}{\pgfqpoint{4.418732in}{2.236385in}}{\pgfqpoint{4.414614in}{2.232267in}}%
\pgfpathcurveto{\pgfqpoint{4.410496in}{2.228149in}}{\pgfqpoint{4.408182in}{2.222563in}}{\pgfqpoint{4.408182in}{2.216739in}}%
\pgfpathcurveto{\pgfqpoint{4.408182in}{2.210915in}}{\pgfqpoint{4.410496in}{2.205329in}}{\pgfqpoint{4.414614in}{2.201211in}}%
\pgfpathcurveto{\pgfqpoint{4.418732in}{2.197092in}}{\pgfqpoint{4.424318in}{2.194779in}}{\pgfqpoint{4.430142in}{2.194779in}}%
\pgfpathlineto{\pgfqpoint{4.430142in}{2.194779in}}%
\pgfpathclose%
\pgfusepath{stroke,fill}%
\end{pgfscope}%
\begin{pgfscope}%
\pgfpathrectangle{\pgfqpoint{1.000000in}{1.148311in}}{\pgfqpoint{6.200000in}{5.623377in}}%
\pgfusepath{clip}%
\pgfsetbuttcap%
\pgfsetroundjoin%
\definecolor{currentfill}{rgb}{0.200000,0.800000,0.200000}%
\pgfsetfillcolor{currentfill}%
\pgfsetlinewidth{1.003750pt}%
\definecolor{currentstroke}{rgb}{0.200000,0.800000,0.200000}%
\pgfsetstrokecolor{currentstroke}%
\pgfsetdash{}{0pt}%
\pgfpathmoveto{\pgfqpoint{4.495395in}{2.214765in}}%
\pgfpathcurveto{\pgfqpoint{4.501219in}{2.214765in}}{\pgfqpoint{4.506805in}{2.217079in}}{\pgfqpoint{4.510923in}{2.221197in}}%
\pgfpathcurveto{\pgfqpoint{4.515041in}{2.225315in}}{\pgfqpoint{4.517355in}{2.230902in}}{\pgfqpoint{4.517355in}{2.236726in}}%
\pgfpathcurveto{\pgfqpoint{4.517355in}{2.242549in}}{\pgfqpoint{4.515041in}{2.248136in}}{\pgfqpoint{4.510923in}{2.252254in}}%
\pgfpathcurveto{\pgfqpoint{4.506805in}{2.256372in}}{\pgfqpoint{4.501219in}{2.258686in}}{\pgfqpoint{4.495395in}{2.258686in}}%
\pgfpathcurveto{\pgfqpoint{4.489571in}{2.258686in}}{\pgfqpoint{4.483984in}{2.256372in}}{\pgfqpoint{4.479866in}{2.252254in}}%
\pgfpathcurveto{\pgfqpoint{4.475748in}{2.248136in}}{\pgfqpoint{4.473434in}{2.242549in}}{\pgfqpoint{4.473434in}{2.236726in}}%
\pgfpathcurveto{\pgfqpoint{4.473434in}{2.230902in}}{\pgfqpoint{4.475748in}{2.225315in}}{\pgfqpoint{4.479866in}{2.221197in}}%
\pgfpathcurveto{\pgfqpoint{4.483984in}{2.217079in}}{\pgfqpoint{4.489571in}{2.214765in}}{\pgfqpoint{4.495395in}{2.214765in}}%
\pgfpathlineto{\pgfqpoint{4.495395in}{2.214765in}}%
\pgfpathclose%
\pgfusepath{stroke,fill}%
\end{pgfscope}%
\begin{pgfscope}%
\pgfpathrectangle{\pgfqpoint{1.000000in}{1.148311in}}{\pgfqpoint{6.200000in}{5.623377in}}%
\pgfusepath{clip}%
\pgfsetbuttcap%
\pgfsetroundjoin%
\definecolor{currentfill}{rgb}{0.200000,0.800000,0.200000}%
\pgfsetfillcolor{currentfill}%
\pgfsetlinewidth{1.003750pt}%
\definecolor{currentstroke}{rgb}{0.200000,0.800000,0.200000}%
\pgfsetstrokecolor{currentstroke}%
\pgfsetdash{}{0pt}%
\pgfpathmoveto{\pgfqpoint{4.510848in}{2.245822in}}%
\pgfpathcurveto{\pgfqpoint{4.516672in}{2.245822in}}{\pgfqpoint{4.522258in}{2.248136in}}{\pgfqpoint{4.526376in}{2.252254in}}%
\pgfpathcurveto{\pgfqpoint{4.530494in}{2.256373in}}{\pgfqpoint{4.532808in}{2.261959in}}{\pgfqpoint{4.532808in}{2.267783in}}%
\pgfpathcurveto{\pgfqpoint{4.532808in}{2.273607in}}{\pgfqpoint{4.530494in}{2.279193in}}{\pgfqpoint{4.526376in}{2.283311in}}%
\pgfpathcurveto{\pgfqpoint{4.522258in}{2.287429in}}{\pgfqpoint{4.516672in}{2.289743in}}{\pgfqpoint{4.510848in}{2.289743in}}%
\pgfpathcurveto{\pgfqpoint{4.505024in}{2.289743in}}{\pgfqpoint{4.499438in}{2.287429in}}{\pgfqpoint{4.495320in}{2.283311in}}%
\pgfpathcurveto{\pgfqpoint{4.491202in}{2.279193in}}{\pgfqpoint{4.488888in}{2.273607in}}{\pgfqpoint{4.488888in}{2.267783in}}%
\pgfpathcurveto{\pgfqpoint{4.488888in}{2.261959in}}{\pgfqpoint{4.491202in}{2.256373in}}{\pgfqpoint{4.495320in}{2.252254in}}%
\pgfpathcurveto{\pgfqpoint{4.499438in}{2.248136in}}{\pgfqpoint{4.505024in}{2.245822in}}{\pgfqpoint{4.510848in}{2.245822in}}%
\pgfpathlineto{\pgfqpoint{4.510848in}{2.245822in}}%
\pgfpathclose%
\pgfusepath{stroke,fill}%
\end{pgfscope}%
\begin{pgfscope}%
\pgfpathrectangle{\pgfqpoint{1.000000in}{1.148311in}}{\pgfqpoint{6.200000in}{5.623377in}}%
\pgfusepath{clip}%
\pgfsetbuttcap%
\pgfsetroundjoin%
\definecolor{currentfill}{rgb}{0.200000,0.800000,0.200000}%
\pgfsetfillcolor{currentfill}%
\pgfsetlinewidth{1.003750pt}%
\definecolor{currentstroke}{rgb}{0.200000,0.800000,0.200000}%
\pgfsetstrokecolor{currentstroke}%
\pgfsetdash{}{0pt}%
\pgfpathmoveto{\pgfqpoint{4.538735in}{2.278586in}}%
\pgfpathcurveto{\pgfqpoint{4.544559in}{2.278586in}}{\pgfqpoint{4.550145in}{2.280900in}}{\pgfqpoint{4.554263in}{2.285018in}}%
\pgfpathcurveto{\pgfqpoint{4.558381in}{2.289136in}}{\pgfqpoint{4.560695in}{2.294722in}}{\pgfqpoint{4.560695in}{2.300546in}}%
\pgfpathcurveto{\pgfqpoint{4.560695in}{2.306370in}}{\pgfqpoint{4.558381in}{2.311956in}}{\pgfqpoint{4.554263in}{2.316074in}}%
\pgfpathcurveto{\pgfqpoint{4.550145in}{2.320193in}}{\pgfqpoint{4.544559in}{2.322506in}}{\pgfqpoint{4.538735in}{2.322506in}}%
\pgfpathcurveto{\pgfqpoint{4.532911in}{2.322506in}}{\pgfqpoint{4.527325in}{2.320193in}}{\pgfqpoint{4.523207in}{2.316074in}}%
\pgfpathcurveto{\pgfqpoint{4.519089in}{2.311956in}}{\pgfqpoint{4.516775in}{2.306370in}}{\pgfqpoint{4.516775in}{2.300546in}}%
\pgfpathcurveto{\pgfqpoint{4.516775in}{2.294722in}}{\pgfqpoint{4.519089in}{2.289136in}}{\pgfqpoint{4.523207in}{2.285018in}}%
\pgfpathcurveto{\pgfqpoint{4.527325in}{2.280900in}}{\pgfqpoint{4.532911in}{2.278586in}}{\pgfqpoint{4.538735in}{2.278586in}}%
\pgfpathlineto{\pgfqpoint{4.538735in}{2.278586in}}%
\pgfpathclose%
\pgfusepath{stroke,fill}%
\end{pgfscope}%
\begin{pgfscope}%
\pgfpathrectangle{\pgfqpoint{1.000000in}{1.148311in}}{\pgfqpoint{6.200000in}{5.623377in}}%
\pgfusepath{clip}%
\pgfsetbuttcap%
\pgfsetroundjoin%
\definecolor{currentfill}{rgb}{0.200000,0.200000,0.800000}%
\pgfsetfillcolor{currentfill}%
\pgfsetlinewidth{1.003750pt}%
\definecolor{currentstroke}{rgb}{0.200000,0.200000,0.800000}%
\pgfsetstrokecolor{currentstroke}%
\pgfsetdash{}{0pt}%
\pgfpathmoveto{\pgfqpoint{6.530402in}{5.321136in}}%
\pgfpathcurveto{\pgfqpoint{6.536226in}{5.321136in}}{\pgfqpoint{6.541812in}{5.323450in}}{\pgfqpoint{6.545930in}{5.327568in}}%
\pgfpathcurveto{\pgfqpoint{6.550048in}{5.331687in}}{\pgfqpoint{6.552362in}{5.337273in}}{\pgfqpoint{6.552362in}{5.343097in}}%
\pgfpathcurveto{\pgfqpoint{6.552362in}{5.348921in}}{\pgfqpoint{6.550048in}{5.354507in}}{\pgfqpoint{6.545930in}{5.358625in}}%
\pgfpathcurveto{\pgfqpoint{6.541812in}{5.362743in}}{\pgfqpoint{6.536226in}{5.365057in}}{\pgfqpoint{6.530402in}{5.365057in}}%
\pgfpathcurveto{\pgfqpoint{6.524578in}{5.365057in}}{\pgfqpoint{6.518992in}{5.362743in}}{\pgfqpoint{6.514874in}{5.358625in}}%
\pgfpathcurveto{\pgfqpoint{6.510755in}{5.354507in}}{\pgfqpoint{6.508442in}{5.348921in}}{\pgfqpoint{6.508442in}{5.343097in}}%
\pgfpathcurveto{\pgfqpoint{6.508442in}{5.337273in}}{\pgfqpoint{6.510755in}{5.331687in}}{\pgfqpoint{6.514874in}{5.327568in}}%
\pgfpathcurveto{\pgfqpoint{6.518992in}{5.323450in}}{\pgfqpoint{6.524578in}{5.321136in}}{\pgfqpoint{6.530402in}{5.321136in}}%
\pgfpathlineto{\pgfqpoint{6.530402in}{5.321136in}}%
\pgfpathclose%
\pgfusepath{stroke,fill}%
\end{pgfscope}%
\begin{pgfscope}%
\pgfpathrectangle{\pgfqpoint{1.000000in}{1.148311in}}{\pgfqpoint{6.200000in}{5.623377in}}%
\pgfusepath{clip}%
\pgfsetbuttcap%
\pgfsetroundjoin%
\definecolor{currentfill}{rgb}{0.200000,0.200000,0.800000}%
\pgfsetfillcolor{currentfill}%
\pgfsetlinewidth{1.003750pt}%
\definecolor{currentstroke}{rgb}{0.200000,0.200000,0.800000}%
\pgfsetstrokecolor{currentstroke}%
\pgfsetdash{}{0pt}%
\pgfpathmoveto{\pgfqpoint{6.519179in}{5.389248in}}%
\pgfpathcurveto{\pgfqpoint{6.525003in}{5.389248in}}{\pgfqpoint{6.530590in}{5.391562in}}{\pgfqpoint{6.534708in}{5.395680in}}%
\pgfpathcurveto{\pgfqpoint{6.538826in}{5.399798in}}{\pgfqpoint{6.541140in}{5.405385in}}{\pgfqpoint{6.541140in}{5.411209in}}%
\pgfpathcurveto{\pgfqpoint{6.541140in}{5.417032in}}{\pgfqpoint{6.538826in}{5.422619in}}{\pgfqpoint{6.534708in}{5.426737in}}%
\pgfpathcurveto{\pgfqpoint{6.530590in}{5.430855in}}{\pgfqpoint{6.525003in}{5.433169in}}{\pgfqpoint{6.519179in}{5.433169in}}%
\pgfpathcurveto{\pgfqpoint{6.513355in}{5.433169in}}{\pgfqpoint{6.507769in}{5.430855in}}{\pgfqpoint{6.503651in}{5.426737in}}%
\pgfpathcurveto{\pgfqpoint{6.499533in}{5.422619in}}{\pgfqpoint{6.497219in}{5.417032in}}{\pgfqpoint{6.497219in}{5.411209in}}%
\pgfpathcurveto{\pgfqpoint{6.497219in}{5.405385in}}{\pgfqpoint{6.499533in}{5.399798in}}{\pgfqpoint{6.503651in}{5.395680in}}%
\pgfpathcurveto{\pgfqpoint{6.507769in}{5.391562in}}{\pgfqpoint{6.513355in}{5.389248in}}{\pgfqpoint{6.519179in}{5.389248in}}%
\pgfpathlineto{\pgfqpoint{6.519179in}{5.389248in}}%
\pgfpathclose%
\pgfusepath{stroke,fill}%
\end{pgfscope}%
\begin{pgfscope}%
\pgfpathrectangle{\pgfqpoint{1.000000in}{1.148311in}}{\pgfqpoint{6.200000in}{5.623377in}}%
\pgfusepath{clip}%
\pgfsetbuttcap%
\pgfsetroundjoin%
\definecolor{currentfill}{rgb}{0.200000,0.200000,0.800000}%
\pgfsetfillcolor{currentfill}%
\pgfsetlinewidth{1.003750pt}%
\definecolor{currentstroke}{rgb}{0.200000,0.200000,0.800000}%
\pgfsetstrokecolor{currentstroke}%
\pgfsetdash{}{0pt}%
\pgfpathmoveto{\pgfqpoint{6.527958in}{5.459033in}}%
\pgfpathcurveto{\pgfqpoint{6.533782in}{5.459033in}}{\pgfqpoint{6.539368in}{5.461347in}}{\pgfqpoint{6.543487in}{5.465465in}}%
\pgfpathcurveto{\pgfqpoint{6.547605in}{5.469583in}}{\pgfqpoint{6.549919in}{5.475169in}}{\pgfqpoint{6.549919in}{5.480993in}}%
\pgfpathcurveto{\pgfqpoint{6.549919in}{5.486817in}}{\pgfqpoint{6.547605in}{5.492403in}}{\pgfqpoint{6.543487in}{5.496521in}}%
\pgfpathcurveto{\pgfqpoint{6.539368in}{5.500640in}}{\pgfqpoint{6.533782in}{5.502953in}}{\pgfqpoint{6.527958in}{5.502953in}}%
\pgfpathcurveto{\pgfqpoint{6.522134in}{5.502953in}}{\pgfqpoint{6.516548in}{5.500640in}}{\pgfqpoint{6.512430in}{5.496521in}}%
\pgfpathcurveto{\pgfqpoint{6.508312in}{5.492403in}}{\pgfqpoint{6.505998in}{5.486817in}}{\pgfqpoint{6.505998in}{5.480993in}}%
\pgfpathcurveto{\pgfqpoint{6.505998in}{5.475169in}}{\pgfqpoint{6.508312in}{5.469583in}}{\pgfqpoint{6.512430in}{5.465465in}}%
\pgfpathcurveto{\pgfqpoint{6.516548in}{5.461347in}}{\pgfqpoint{6.522134in}{5.459033in}}{\pgfqpoint{6.527958in}{5.459033in}}%
\pgfpathlineto{\pgfqpoint{6.527958in}{5.459033in}}%
\pgfpathclose%
\pgfusepath{stroke,fill}%
\end{pgfscope}%
\begin{pgfscope}%
\pgfpathrectangle{\pgfqpoint{1.000000in}{1.148311in}}{\pgfqpoint{6.200000in}{5.623377in}}%
\pgfusepath{clip}%
\pgfsetbuttcap%
\pgfsetroundjoin%
\definecolor{currentfill}{rgb}{0.200000,0.200000,0.800000}%
\pgfsetfillcolor{currentfill}%
\pgfsetlinewidth{1.003750pt}%
\definecolor{currentstroke}{rgb}{0.200000,0.200000,0.800000}%
\pgfsetstrokecolor{currentstroke}%
\pgfsetdash{}{0pt}%
\pgfpathmoveto{\pgfqpoint{6.489152in}{5.521912in}}%
\pgfpathcurveto{\pgfqpoint{6.494976in}{5.521912in}}{\pgfqpoint{6.500562in}{5.524226in}}{\pgfqpoint{6.504680in}{5.528344in}}%
\pgfpathcurveto{\pgfqpoint{6.508798in}{5.532462in}}{\pgfqpoint{6.511112in}{5.538049in}}{\pgfqpoint{6.511112in}{5.543873in}}%
\pgfpathcurveto{\pgfqpoint{6.511112in}{5.549697in}}{\pgfqpoint{6.508798in}{5.555283in}}{\pgfqpoint{6.504680in}{5.559401in}}%
\pgfpathcurveto{\pgfqpoint{6.500562in}{5.563519in}}{\pgfqpoint{6.494976in}{5.565833in}}{\pgfqpoint{6.489152in}{5.565833in}}%
\pgfpathcurveto{\pgfqpoint{6.483328in}{5.565833in}}{\pgfqpoint{6.477742in}{5.563519in}}{\pgfqpoint{6.473624in}{5.559401in}}%
\pgfpathcurveto{\pgfqpoint{6.469506in}{5.555283in}}{\pgfqpoint{6.467192in}{5.549697in}}{\pgfqpoint{6.467192in}{5.543873in}}%
\pgfpathcurveto{\pgfqpoint{6.467192in}{5.538049in}}{\pgfqpoint{6.469506in}{5.532462in}}{\pgfqpoint{6.473624in}{5.528344in}}%
\pgfpathcurveto{\pgfqpoint{6.477742in}{5.524226in}}{\pgfqpoint{6.483328in}{5.521912in}}{\pgfqpoint{6.489152in}{5.521912in}}%
\pgfpathlineto{\pgfqpoint{6.489152in}{5.521912in}}%
\pgfpathclose%
\pgfusepath{stroke,fill}%
\end{pgfscope}%
\begin{pgfscope}%
\pgfpathrectangle{\pgfqpoint{1.000000in}{1.148311in}}{\pgfqpoint{6.200000in}{5.623377in}}%
\pgfusepath{clip}%
\pgfsetbuttcap%
\pgfsetroundjoin%
\definecolor{currentfill}{rgb}{0.200000,0.200000,0.800000}%
\pgfsetfillcolor{currentfill}%
\pgfsetlinewidth{1.003750pt}%
\definecolor{currentstroke}{rgb}{0.200000,0.200000,0.800000}%
\pgfsetstrokecolor{currentstroke}%
\pgfsetdash{}{0pt}%
\pgfpathmoveto{\pgfqpoint{6.494458in}{5.592803in}}%
\pgfpathcurveto{\pgfqpoint{6.500282in}{5.592803in}}{\pgfqpoint{6.505868in}{5.595117in}}{\pgfqpoint{6.509986in}{5.599235in}}%
\pgfpathcurveto{\pgfqpoint{6.514104in}{5.603353in}}{\pgfqpoint{6.516418in}{5.608940in}}{\pgfqpoint{6.516418in}{5.614763in}}%
\pgfpathcurveto{\pgfqpoint{6.516418in}{5.620587in}}{\pgfqpoint{6.514104in}{5.626174in}}{\pgfqpoint{6.509986in}{5.630292in}}%
\pgfpathcurveto{\pgfqpoint{6.505868in}{5.634410in}}{\pgfqpoint{6.500282in}{5.636724in}}{\pgfqpoint{6.494458in}{5.636724in}}%
\pgfpathcurveto{\pgfqpoint{6.488634in}{5.636724in}}{\pgfqpoint{6.483048in}{5.634410in}}{\pgfqpoint{6.478929in}{5.630292in}}%
\pgfpathcurveto{\pgfqpoint{6.474811in}{5.626174in}}{\pgfqpoint{6.472497in}{5.620587in}}{\pgfqpoint{6.472497in}{5.614763in}}%
\pgfpathcurveto{\pgfqpoint{6.472497in}{5.608940in}}{\pgfqpoint{6.474811in}{5.603353in}}{\pgfqpoint{6.478929in}{5.599235in}}%
\pgfpathcurveto{\pgfqpoint{6.483048in}{5.595117in}}{\pgfqpoint{6.488634in}{5.592803in}}{\pgfqpoint{6.494458in}{5.592803in}}%
\pgfpathlineto{\pgfqpoint{6.494458in}{5.592803in}}%
\pgfpathclose%
\pgfusepath{stroke,fill}%
\end{pgfscope}%
\begin{pgfscope}%
\pgfpathrectangle{\pgfqpoint{1.000000in}{1.148311in}}{\pgfqpoint{6.200000in}{5.623377in}}%
\pgfusepath{clip}%
\pgfsetbuttcap%
\pgfsetroundjoin%
\definecolor{currentfill}{rgb}{0.200000,0.200000,0.800000}%
\pgfsetfillcolor{currentfill}%
\pgfsetlinewidth{1.003750pt}%
\definecolor{currentstroke}{rgb}{0.200000,0.200000,0.800000}%
\pgfsetstrokecolor{currentstroke}%
\pgfsetdash{}{0pt}%
\pgfpathmoveto{\pgfqpoint{6.514501in}{5.671597in}}%
\pgfpathcurveto{\pgfqpoint{6.520325in}{5.671597in}}{\pgfqpoint{6.525911in}{5.673911in}}{\pgfqpoint{6.530030in}{5.678029in}}%
\pgfpathcurveto{\pgfqpoint{6.534148in}{5.682147in}}{\pgfqpoint{6.536462in}{5.687734in}}{\pgfqpoint{6.536462in}{5.693557in}}%
\pgfpathcurveto{\pgfqpoint{6.536462in}{5.699381in}}{\pgfqpoint{6.534148in}{5.704968in}}{\pgfqpoint{6.530030in}{5.709086in}}%
\pgfpathcurveto{\pgfqpoint{6.525911in}{5.713204in}}{\pgfqpoint{6.520325in}{5.715518in}}{\pgfqpoint{6.514501in}{5.715518in}}%
\pgfpathcurveto{\pgfqpoint{6.508677in}{5.715518in}}{\pgfqpoint{6.503091in}{5.713204in}}{\pgfqpoint{6.498973in}{5.709086in}}%
\pgfpathcurveto{\pgfqpoint{6.494855in}{5.704968in}}{\pgfqpoint{6.492541in}{5.699381in}}{\pgfqpoint{6.492541in}{5.693557in}}%
\pgfpathcurveto{\pgfqpoint{6.492541in}{5.687734in}}{\pgfqpoint{6.494855in}{5.682147in}}{\pgfqpoint{6.498973in}{5.678029in}}%
\pgfpathcurveto{\pgfqpoint{6.503091in}{5.673911in}}{\pgfqpoint{6.508677in}{5.671597in}}{\pgfqpoint{6.514501in}{5.671597in}}%
\pgfpathlineto{\pgfqpoint{6.514501in}{5.671597in}}%
\pgfpathclose%
\pgfusepath{stroke,fill}%
\end{pgfscope}%
\begin{pgfscope}%
\pgfpathrectangle{\pgfqpoint{1.000000in}{1.148311in}}{\pgfqpoint{6.200000in}{5.623377in}}%
\pgfusepath{clip}%
\pgfsetbuttcap%
\pgfsetroundjoin%
\definecolor{currentfill}{rgb}{0.200000,0.200000,0.800000}%
\pgfsetfillcolor{currentfill}%
\pgfsetlinewidth{1.003750pt}%
\definecolor{currentstroke}{rgb}{0.200000,0.200000,0.800000}%
\pgfsetstrokecolor{currentstroke}%
\pgfsetdash{}{0pt}%
\pgfpathmoveto{\pgfqpoint{6.433934in}{5.716074in}}%
\pgfpathcurveto{\pgfqpoint{6.439757in}{5.716074in}}{\pgfqpoint{6.445344in}{5.718388in}}{\pgfqpoint{6.449462in}{5.722506in}}%
\pgfpathcurveto{\pgfqpoint{6.453580in}{5.726624in}}{\pgfqpoint{6.455894in}{5.732210in}}{\pgfqpoint{6.455894in}{5.738034in}}%
\pgfpathcurveto{\pgfqpoint{6.455894in}{5.743858in}}{\pgfqpoint{6.453580in}{5.749444in}}{\pgfqpoint{6.449462in}{5.753562in}}%
\pgfpathcurveto{\pgfqpoint{6.445344in}{5.757680in}}{\pgfqpoint{6.439757in}{5.759994in}}{\pgfqpoint{6.433934in}{5.759994in}}%
\pgfpathcurveto{\pgfqpoint{6.428110in}{5.759994in}}{\pgfqpoint{6.422523in}{5.757680in}}{\pgfqpoint{6.418405in}{5.753562in}}%
\pgfpathcurveto{\pgfqpoint{6.414287in}{5.749444in}}{\pgfqpoint{6.411973in}{5.743858in}}{\pgfqpoint{6.411973in}{5.738034in}}%
\pgfpathcurveto{\pgfqpoint{6.411973in}{5.732210in}}{\pgfqpoint{6.414287in}{5.726624in}}{\pgfqpoint{6.418405in}{5.722506in}}%
\pgfpathcurveto{\pgfqpoint{6.422523in}{5.718388in}}{\pgfqpoint{6.428110in}{5.716074in}}{\pgfqpoint{6.433934in}{5.716074in}}%
\pgfpathlineto{\pgfqpoint{6.433934in}{5.716074in}}%
\pgfpathclose%
\pgfusepath{stroke,fill}%
\end{pgfscope}%
\begin{pgfscope}%
\pgfpathrectangle{\pgfqpoint{1.000000in}{1.148311in}}{\pgfqpoint{6.200000in}{5.623377in}}%
\pgfusepath{clip}%
\pgfsetbuttcap%
\pgfsetroundjoin%
\definecolor{currentfill}{rgb}{0.200000,0.200000,0.800000}%
\pgfsetfillcolor{currentfill}%
\pgfsetlinewidth{1.003750pt}%
\definecolor{currentstroke}{rgb}{0.200000,0.200000,0.800000}%
\pgfsetstrokecolor{currentstroke}%
\pgfsetdash{}{0pt}%
\pgfpathmoveto{\pgfqpoint{6.465813in}{5.805890in}}%
\pgfpathcurveto{\pgfqpoint{6.471637in}{5.805890in}}{\pgfqpoint{6.477223in}{5.808204in}}{\pgfqpoint{6.481341in}{5.812322in}}%
\pgfpathcurveto{\pgfqpoint{6.485459in}{5.816440in}}{\pgfqpoint{6.487773in}{5.822026in}}{\pgfqpoint{6.487773in}{5.827850in}}%
\pgfpathcurveto{\pgfqpoint{6.487773in}{5.833674in}}{\pgfqpoint{6.485459in}{5.839260in}}{\pgfqpoint{6.481341in}{5.843378in}}%
\pgfpathcurveto{\pgfqpoint{6.477223in}{5.847496in}}{\pgfqpoint{6.471637in}{5.849810in}}{\pgfqpoint{6.465813in}{5.849810in}}%
\pgfpathcurveto{\pgfqpoint{6.459989in}{5.849810in}}{\pgfqpoint{6.454403in}{5.847496in}}{\pgfqpoint{6.450285in}{5.843378in}}%
\pgfpathcurveto{\pgfqpoint{6.446166in}{5.839260in}}{\pgfqpoint{6.443853in}{5.833674in}}{\pgfqpoint{6.443853in}{5.827850in}}%
\pgfpathcurveto{\pgfqpoint{6.443853in}{5.822026in}}{\pgfqpoint{6.446166in}{5.816440in}}{\pgfqpoint{6.450285in}{5.812322in}}%
\pgfpathcurveto{\pgfqpoint{6.454403in}{5.808204in}}{\pgfqpoint{6.459989in}{5.805890in}}{\pgfqpoint{6.465813in}{5.805890in}}%
\pgfpathlineto{\pgfqpoint{6.465813in}{5.805890in}}%
\pgfpathclose%
\pgfusepath{stroke,fill}%
\end{pgfscope}%
\begin{pgfscope}%
\pgfpathrectangle{\pgfqpoint{1.000000in}{1.148311in}}{\pgfqpoint{6.200000in}{5.623377in}}%
\pgfusepath{clip}%
\pgfsetbuttcap%
\pgfsetroundjoin%
\definecolor{currentfill}{rgb}{0.200000,0.200000,0.800000}%
\pgfsetfillcolor{currentfill}%
\pgfsetlinewidth{1.003750pt}%
\definecolor{currentstroke}{rgb}{0.200000,0.200000,0.800000}%
\pgfsetstrokecolor{currentstroke}%
\pgfsetdash{}{0pt}%
\pgfpathmoveto{\pgfqpoint{6.391953in}{5.846656in}}%
\pgfpathcurveto{\pgfqpoint{6.397777in}{5.846656in}}{\pgfqpoint{6.403363in}{5.848970in}}{\pgfqpoint{6.407482in}{5.853088in}}%
\pgfpathcurveto{\pgfqpoint{6.411600in}{5.857206in}}{\pgfqpoint{6.413914in}{5.862793in}}{\pgfqpoint{6.413914in}{5.868617in}}%
\pgfpathcurveto{\pgfqpoint{6.413914in}{5.874441in}}{\pgfqpoint{6.411600in}{5.880027in}}{\pgfqpoint{6.407482in}{5.884145in}}%
\pgfpathcurveto{\pgfqpoint{6.403363in}{5.888263in}}{\pgfqpoint{6.397777in}{5.890577in}}{\pgfqpoint{6.391953in}{5.890577in}}%
\pgfpathcurveto{\pgfqpoint{6.386129in}{5.890577in}}{\pgfqpoint{6.380543in}{5.888263in}}{\pgfqpoint{6.376425in}{5.884145in}}%
\pgfpathcurveto{\pgfqpoint{6.372307in}{5.880027in}}{\pgfqpoint{6.369993in}{5.874441in}}{\pgfqpoint{6.369993in}{5.868617in}}%
\pgfpathcurveto{\pgfqpoint{6.369993in}{5.862793in}}{\pgfqpoint{6.372307in}{5.857206in}}{\pgfqpoint{6.376425in}{5.853088in}}%
\pgfpathcurveto{\pgfqpoint{6.380543in}{5.848970in}}{\pgfqpoint{6.386129in}{5.846656in}}{\pgfqpoint{6.391953in}{5.846656in}}%
\pgfpathlineto{\pgfqpoint{6.391953in}{5.846656in}}%
\pgfpathclose%
\pgfusepath{stroke,fill}%
\end{pgfscope}%
\begin{pgfscope}%
\pgfpathrectangle{\pgfqpoint{1.000000in}{1.148311in}}{\pgfqpoint{6.200000in}{5.623377in}}%
\pgfusepath{clip}%
\pgfsetbuttcap%
\pgfsetroundjoin%
\definecolor{currentfill}{rgb}{0.200000,0.200000,0.800000}%
\pgfsetfillcolor{currentfill}%
\pgfsetlinewidth{1.003750pt}%
\definecolor{currentstroke}{rgb}{0.200000,0.200000,0.800000}%
\pgfsetstrokecolor{currentstroke}%
\pgfsetdash{}{0pt}%
\pgfpathmoveto{\pgfqpoint{6.401348in}{5.934184in}}%
\pgfpathcurveto{\pgfqpoint{6.407172in}{5.934184in}}{\pgfqpoint{6.412758in}{5.936497in}}{\pgfqpoint{6.416876in}{5.940616in}}%
\pgfpathcurveto{\pgfqpoint{6.420994in}{5.944734in}}{\pgfqpoint{6.423308in}{5.950320in}}{\pgfqpoint{6.423308in}{5.956144in}}%
\pgfpathcurveto{\pgfqpoint{6.423308in}{5.961968in}}{\pgfqpoint{6.420994in}{5.967554in}}{\pgfqpoint{6.416876in}{5.971672in}}%
\pgfpathcurveto{\pgfqpoint{6.412758in}{5.975790in}}{\pgfqpoint{6.407172in}{5.978104in}}{\pgfqpoint{6.401348in}{5.978104in}}%
\pgfpathcurveto{\pgfqpoint{6.395524in}{5.978104in}}{\pgfqpoint{6.389938in}{5.975790in}}{\pgfqpoint{6.385819in}{5.971672in}}%
\pgfpathcurveto{\pgfqpoint{6.381701in}{5.967554in}}{\pgfqpoint{6.379387in}{5.961968in}}{\pgfqpoint{6.379387in}{5.956144in}}%
\pgfpathcurveto{\pgfqpoint{6.379387in}{5.950320in}}{\pgfqpoint{6.381701in}{5.944734in}}{\pgfqpoint{6.385819in}{5.940616in}}%
\pgfpathcurveto{\pgfqpoint{6.389938in}{5.936497in}}{\pgfqpoint{6.395524in}{5.934184in}}{\pgfqpoint{6.401348in}{5.934184in}}%
\pgfpathlineto{\pgfqpoint{6.401348in}{5.934184in}}%
\pgfpathclose%
\pgfusepath{stroke,fill}%
\end{pgfscope}%
\begin{pgfscope}%
\pgfpathrectangle{\pgfqpoint{1.000000in}{1.148311in}}{\pgfqpoint{6.200000in}{5.623377in}}%
\pgfusepath{clip}%
\pgfsetbuttcap%
\pgfsetroundjoin%
\definecolor{currentfill}{rgb}{0.200000,0.200000,0.800000}%
\pgfsetfillcolor{currentfill}%
\pgfsetlinewidth{1.003750pt}%
\definecolor{currentstroke}{rgb}{0.200000,0.200000,0.800000}%
\pgfsetstrokecolor{currentstroke}%
\pgfsetdash{}{0pt}%
\pgfpathmoveto{\pgfqpoint{6.369138in}{5.999777in}}%
\pgfpathcurveto{\pgfqpoint{6.374962in}{5.999777in}}{\pgfqpoint{6.380548in}{6.002091in}}{\pgfqpoint{6.384666in}{6.006209in}}%
\pgfpathcurveto{\pgfqpoint{6.388785in}{6.010328in}}{\pgfqpoint{6.391098in}{6.015914in}}{\pgfqpoint{6.391098in}{6.021738in}}%
\pgfpathcurveto{\pgfqpoint{6.391098in}{6.027562in}}{\pgfqpoint{6.388785in}{6.033148in}}{\pgfqpoint{6.384666in}{6.037266in}}%
\pgfpathcurveto{\pgfqpoint{6.380548in}{6.041384in}}{\pgfqpoint{6.374962in}{6.043698in}}{\pgfqpoint{6.369138in}{6.043698in}}%
\pgfpathcurveto{\pgfqpoint{6.363314in}{6.043698in}}{\pgfqpoint{6.357728in}{6.041384in}}{\pgfqpoint{6.353610in}{6.037266in}}%
\pgfpathcurveto{\pgfqpoint{6.349492in}{6.033148in}}{\pgfqpoint{6.347178in}{6.027562in}}{\pgfqpoint{6.347178in}{6.021738in}}%
\pgfpathcurveto{\pgfqpoint{6.347178in}{6.015914in}}{\pgfqpoint{6.349492in}{6.010328in}}{\pgfqpoint{6.353610in}{6.006209in}}%
\pgfpathcurveto{\pgfqpoint{6.357728in}{6.002091in}}{\pgfqpoint{6.363314in}{5.999777in}}{\pgfqpoint{6.369138in}{5.999777in}}%
\pgfpathlineto{\pgfqpoint{6.369138in}{5.999777in}}%
\pgfpathclose%
\pgfusepath{stroke,fill}%
\end{pgfscope}%
\begin{pgfscope}%
\pgfpathrectangle{\pgfqpoint{1.000000in}{1.148311in}}{\pgfqpoint{6.200000in}{5.623377in}}%
\pgfusepath{clip}%
\pgfsetbuttcap%
\pgfsetroundjoin%
\definecolor{currentfill}{rgb}{0.200000,0.200000,0.800000}%
\pgfsetfillcolor{currentfill}%
\pgfsetlinewidth{1.003750pt}%
\definecolor{currentstroke}{rgb}{0.200000,0.200000,0.800000}%
\pgfsetstrokecolor{currentstroke}%
\pgfsetdash{}{0pt}%
\pgfpathmoveto{\pgfqpoint{6.252624in}{5.996776in}}%
\pgfpathcurveto{\pgfqpoint{6.258448in}{5.996776in}}{\pgfqpoint{6.264034in}{5.999090in}}{\pgfqpoint{6.268152in}{6.003208in}}%
\pgfpathcurveto{\pgfqpoint{6.272270in}{6.007326in}}{\pgfqpoint{6.274584in}{6.012912in}}{\pgfqpoint{6.274584in}{6.018736in}}%
\pgfpathcurveto{\pgfqpoint{6.274584in}{6.024560in}}{\pgfqpoint{6.272270in}{6.030146in}}{\pgfqpoint{6.268152in}{6.034265in}}%
\pgfpathcurveto{\pgfqpoint{6.264034in}{6.038383in}}{\pgfqpoint{6.258448in}{6.040697in}}{\pgfqpoint{6.252624in}{6.040697in}}%
\pgfpathcurveto{\pgfqpoint{6.246800in}{6.040697in}}{\pgfqpoint{6.241214in}{6.038383in}}{\pgfqpoint{6.237095in}{6.034265in}}%
\pgfpathcurveto{\pgfqpoint{6.232977in}{6.030146in}}{\pgfqpoint{6.230663in}{6.024560in}}{\pgfqpoint{6.230663in}{6.018736in}}%
\pgfpathcurveto{\pgfqpoint{6.230663in}{6.012912in}}{\pgfqpoint{6.232977in}{6.007326in}}{\pgfqpoint{6.237095in}{6.003208in}}%
\pgfpathcurveto{\pgfqpoint{6.241214in}{5.999090in}}{\pgfqpoint{6.246800in}{5.996776in}}{\pgfqpoint{6.252624in}{5.996776in}}%
\pgfpathlineto{\pgfqpoint{6.252624in}{5.996776in}}%
\pgfpathclose%
\pgfusepath{stroke,fill}%
\end{pgfscope}%
\begin{pgfscope}%
\pgfpathrectangle{\pgfqpoint{1.000000in}{1.148311in}}{\pgfqpoint{6.200000in}{5.623377in}}%
\pgfusepath{clip}%
\pgfsetbuttcap%
\pgfsetroundjoin%
\definecolor{currentfill}{rgb}{0.200000,0.200000,0.800000}%
\pgfsetfillcolor{currentfill}%
\pgfsetlinewidth{1.003750pt}%
\definecolor{currentstroke}{rgb}{0.200000,0.200000,0.800000}%
\pgfsetstrokecolor{currentstroke}%
\pgfsetdash{}{0pt}%
\pgfpathmoveto{\pgfqpoint{6.176332in}{6.016145in}}%
\pgfpathcurveto{\pgfqpoint{6.182156in}{6.016145in}}{\pgfqpoint{6.187742in}{6.018459in}}{\pgfqpoint{6.191861in}{6.022577in}}%
\pgfpathcurveto{\pgfqpoint{6.195979in}{6.026696in}}{\pgfqpoint{6.198293in}{6.032282in}}{\pgfqpoint{6.198293in}{6.038106in}}%
\pgfpathcurveto{\pgfqpoint{6.198293in}{6.043930in}}{\pgfqpoint{6.195979in}{6.049516in}}{\pgfqpoint{6.191861in}{6.053634in}}%
\pgfpathcurveto{\pgfqpoint{6.187742in}{6.057752in}}{\pgfqpoint{6.182156in}{6.060066in}}{\pgfqpoint{6.176332in}{6.060066in}}%
\pgfpathcurveto{\pgfqpoint{6.170508in}{6.060066in}}{\pgfqpoint{6.164922in}{6.057752in}}{\pgfqpoint{6.160804in}{6.053634in}}%
\pgfpathcurveto{\pgfqpoint{6.156686in}{6.049516in}}{\pgfqpoint{6.154372in}{6.043930in}}{\pgfqpoint{6.154372in}{6.038106in}}%
\pgfpathcurveto{\pgfqpoint{6.154372in}{6.032282in}}{\pgfqpoint{6.156686in}{6.026696in}}{\pgfqpoint{6.160804in}{6.022577in}}%
\pgfpathcurveto{\pgfqpoint{6.164922in}{6.018459in}}{\pgfqpoint{6.170508in}{6.016145in}}{\pgfqpoint{6.176332in}{6.016145in}}%
\pgfpathlineto{\pgfqpoint{6.176332in}{6.016145in}}%
\pgfpathclose%
\pgfusepath{stroke,fill}%
\end{pgfscope}%
\begin{pgfscope}%
\pgfpathrectangle{\pgfqpoint{1.000000in}{1.148311in}}{\pgfqpoint{6.200000in}{5.623377in}}%
\pgfusepath{clip}%
\pgfsetbuttcap%
\pgfsetroundjoin%
\definecolor{currentfill}{rgb}{0.200000,0.200000,0.800000}%
\pgfsetfillcolor{currentfill}%
\pgfsetlinewidth{1.003750pt}%
\definecolor{currentstroke}{rgb}{0.200000,0.200000,0.800000}%
\pgfsetstrokecolor{currentstroke}%
\pgfsetdash{}{0pt}%
\pgfpathmoveto{\pgfqpoint{6.191892in}{6.127134in}}%
\pgfpathcurveto{\pgfqpoint{6.197716in}{6.127134in}}{\pgfqpoint{6.203302in}{6.129448in}}{\pgfqpoint{6.207420in}{6.133566in}}%
\pgfpathcurveto{\pgfqpoint{6.211538in}{6.137684in}}{\pgfqpoint{6.213852in}{6.143270in}}{\pgfqpoint{6.213852in}{6.149094in}}%
\pgfpathcurveto{\pgfqpoint{6.213852in}{6.154918in}}{\pgfqpoint{6.211538in}{6.160504in}}{\pgfqpoint{6.207420in}{6.164623in}}%
\pgfpathcurveto{\pgfqpoint{6.203302in}{6.168741in}}{\pgfqpoint{6.197716in}{6.171055in}}{\pgfqpoint{6.191892in}{6.171055in}}%
\pgfpathcurveto{\pgfqpoint{6.186068in}{6.171055in}}{\pgfqpoint{6.180482in}{6.168741in}}{\pgfqpoint{6.176363in}{6.164623in}}%
\pgfpathcurveto{\pgfqpoint{6.172245in}{6.160504in}}{\pgfqpoint{6.169931in}{6.154918in}}{\pgfqpoint{6.169931in}{6.149094in}}%
\pgfpathcurveto{\pgfqpoint{6.169931in}{6.143270in}}{\pgfqpoint{6.172245in}{6.137684in}}{\pgfqpoint{6.176363in}{6.133566in}}%
\pgfpathcurveto{\pgfqpoint{6.180482in}{6.129448in}}{\pgfqpoint{6.186068in}{6.127134in}}{\pgfqpoint{6.191892in}{6.127134in}}%
\pgfpathlineto{\pgfqpoint{6.191892in}{6.127134in}}%
\pgfpathclose%
\pgfusepath{stroke,fill}%
\end{pgfscope}%
\begin{pgfscope}%
\pgfpathrectangle{\pgfqpoint{1.000000in}{1.148311in}}{\pgfqpoint{6.200000in}{5.623377in}}%
\pgfusepath{clip}%
\pgfsetbuttcap%
\pgfsetroundjoin%
\definecolor{currentfill}{rgb}{0.200000,0.200000,0.800000}%
\pgfsetfillcolor{currentfill}%
\pgfsetlinewidth{1.003750pt}%
\definecolor{currentstroke}{rgb}{0.200000,0.200000,0.800000}%
\pgfsetstrokecolor{currentstroke}%
\pgfsetdash{}{0pt}%
\pgfpathmoveto{\pgfqpoint{6.141545in}{6.175525in}}%
\pgfpathcurveto{\pgfqpoint{6.147369in}{6.175525in}}{\pgfqpoint{6.152955in}{6.177839in}}{\pgfqpoint{6.157073in}{6.181957in}}%
\pgfpathcurveto{\pgfqpoint{6.161192in}{6.186075in}}{\pgfqpoint{6.163505in}{6.191661in}}{\pgfqpoint{6.163505in}{6.197485in}}%
\pgfpathcurveto{\pgfqpoint{6.163505in}{6.203309in}}{\pgfqpoint{6.161192in}{6.208895in}}{\pgfqpoint{6.157073in}{6.213013in}}%
\pgfpathcurveto{\pgfqpoint{6.152955in}{6.217131in}}{\pgfqpoint{6.147369in}{6.219445in}}{\pgfqpoint{6.141545in}{6.219445in}}%
\pgfpathcurveto{\pgfqpoint{6.135721in}{6.219445in}}{\pgfqpoint{6.130135in}{6.217131in}}{\pgfqpoint{6.126017in}{6.213013in}}%
\pgfpathcurveto{\pgfqpoint{6.121899in}{6.208895in}}{\pgfqpoint{6.119585in}{6.203309in}}{\pgfqpoint{6.119585in}{6.197485in}}%
\pgfpathcurveto{\pgfqpoint{6.119585in}{6.191661in}}{\pgfqpoint{6.121899in}{6.186075in}}{\pgfqpoint{6.126017in}{6.181957in}}%
\pgfpathcurveto{\pgfqpoint{6.130135in}{6.177839in}}{\pgfqpoint{6.135721in}{6.175525in}}{\pgfqpoint{6.141545in}{6.175525in}}%
\pgfpathlineto{\pgfqpoint{6.141545in}{6.175525in}}%
\pgfpathclose%
\pgfusepath{stroke,fill}%
\end{pgfscope}%
\begin{pgfscope}%
\pgfpathrectangle{\pgfqpoint{1.000000in}{1.148311in}}{\pgfqpoint{6.200000in}{5.623377in}}%
\pgfusepath{clip}%
\pgfsetbuttcap%
\pgfsetroundjoin%
\definecolor{currentfill}{rgb}{0.200000,0.200000,0.800000}%
\pgfsetfillcolor{currentfill}%
\pgfsetlinewidth{1.003750pt}%
\definecolor{currentstroke}{rgb}{0.200000,0.200000,0.800000}%
\pgfsetstrokecolor{currentstroke}%
\pgfsetdash{}{0pt}%
\pgfpathmoveto{\pgfqpoint{6.051987in}{6.170121in}}%
\pgfpathcurveto{\pgfqpoint{6.057811in}{6.170121in}}{\pgfqpoint{6.063397in}{6.172435in}}{\pgfqpoint{6.067515in}{6.176553in}}%
\pgfpathcurveto{\pgfqpoint{6.071634in}{6.180671in}}{\pgfqpoint{6.073947in}{6.186258in}}{\pgfqpoint{6.073947in}{6.192081in}}%
\pgfpathcurveto{\pgfqpoint{6.073947in}{6.197905in}}{\pgfqpoint{6.071634in}{6.203492in}}{\pgfqpoint{6.067515in}{6.207610in}}%
\pgfpathcurveto{\pgfqpoint{6.063397in}{6.211728in}}{\pgfqpoint{6.057811in}{6.214042in}}{\pgfqpoint{6.051987in}{6.214042in}}%
\pgfpathcurveto{\pgfqpoint{6.046163in}{6.214042in}}{\pgfqpoint{6.040577in}{6.211728in}}{\pgfqpoint{6.036459in}{6.207610in}}%
\pgfpathcurveto{\pgfqpoint{6.032341in}{6.203492in}}{\pgfqpoint{6.030027in}{6.197905in}}{\pgfqpoint{6.030027in}{6.192081in}}%
\pgfpathcurveto{\pgfqpoint{6.030027in}{6.186258in}}{\pgfqpoint{6.032341in}{6.180671in}}{\pgfqpoint{6.036459in}{6.176553in}}%
\pgfpathcurveto{\pgfqpoint{6.040577in}{6.172435in}}{\pgfqpoint{6.046163in}{6.170121in}}{\pgfqpoint{6.051987in}{6.170121in}}%
\pgfpathlineto{\pgfqpoint{6.051987in}{6.170121in}}%
\pgfpathclose%
\pgfusepath{stroke,fill}%
\end{pgfscope}%
\begin{pgfscope}%
\pgfpathrectangle{\pgfqpoint{1.000000in}{1.148311in}}{\pgfqpoint{6.200000in}{5.623377in}}%
\pgfusepath{clip}%
\pgfsetbuttcap%
\pgfsetroundjoin%
\definecolor{currentfill}{rgb}{0.200000,0.200000,0.800000}%
\pgfsetfillcolor{currentfill}%
\pgfsetlinewidth{1.003750pt}%
\definecolor{currentstroke}{rgb}{0.200000,0.200000,0.800000}%
\pgfsetstrokecolor{currentstroke}%
\pgfsetdash{}{0pt}%
\pgfpathmoveto{\pgfqpoint{6.057256in}{6.303991in}}%
\pgfpathcurveto{\pgfqpoint{6.063079in}{6.303991in}}{\pgfqpoint{6.068666in}{6.306305in}}{\pgfqpoint{6.072784in}{6.310423in}}%
\pgfpathcurveto{\pgfqpoint{6.076902in}{6.314542in}}{\pgfqpoint{6.079216in}{6.320128in}}{\pgfqpoint{6.079216in}{6.325952in}}%
\pgfpathcurveto{\pgfqpoint{6.079216in}{6.331776in}}{\pgfqpoint{6.076902in}{6.337362in}}{\pgfqpoint{6.072784in}{6.341480in}}%
\pgfpathcurveto{\pgfqpoint{6.068666in}{6.345598in}}{\pgfqpoint{6.063079in}{6.347912in}}{\pgfqpoint{6.057256in}{6.347912in}}%
\pgfpathcurveto{\pgfqpoint{6.051432in}{6.347912in}}{\pgfqpoint{6.045845in}{6.345598in}}{\pgfqpoint{6.041727in}{6.341480in}}%
\pgfpathcurveto{\pgfqpoint{6.037609in}{6.337362in}}{\pgfqpoint{6.035295in}{6.331776in}}{\pgfqpoint{6.035295in}{6.325952in}}%
\pgfpathcurveto{\pgfqpoint{6.035295in}{6.320128in}}{\pgfqpoint{6.037609in}{6.314542in}}{\pgfqpoint{6.041727in}{6.310423in}}%
\pgfpathcurveto{\pgfqpoint{6.045845in}{6.306305in}}{\pgfqpoint{6.051432in}{6.303991in}}{\pgfqpoint{6.057256in}{6.303991in}}%
\pgfpathlineto{\pgfqpoint{6.057256in}{6.303991in}}%
\pgfpathclose%
\pgfusepath{stroke,fill}%
\end{pgfscope}%
\begin{pgfscope}%
\pgfpathrectangle{\pgfqpoint{1.000000in}{1.148311in}}{\pgfqpoint{6.200000in}{5.623377in}}%
\pgfusepath{clip}%
\pgfsetbuttcap%
\pgfsetroundjoin%
\definecolor{currentfill}{rgb}{0.200000,0.200000,0.800000}%
\pgfsetfillcolor{currentfill}%
\pgfsetlinewidth{1.003750pt}%
\definecolor{currentstroke}{rgb}{0.200000,0.200000,0.800000}%
\pgfsetstrokecolor{currentstroke}%
\pgfsetdash{}{0pt}%
\pgfpathmoveto{\pgfqpoint{5.955744in}{6.269865in}}%
\pgfpathcurveto{\pgfqpoint{5.961568in}{6.269865in}}{\pgfqpoint{5.967154in}{6.272179in}}{\pgfqpoint{5.971272in}{6.276297in}}%
\pgfpathcurveto{\pgfqpoint{5.975390in}{6.280415in}}{\pgfqpoint{5.977704in}{6.286001in}}{\pgfqpoint{5.977704in}{6.291825in}}%
\pgfpathcurveto{\pgfqpoint{5.977704in}{6.297649in}}{\pgfqpoint{5.975390in}{6.303235in}}{\pgfqpoint{5.971272in}{6.307353in}}%
\pgfpathcurveto{\pgfqpoint{5.967154in}{6.311471in}}{\pgfqpoint{5.961568in}{6.313785in}}{\pgfqpoint{5.955744in}{6.313785in}}%
\pgfpathcurveto{\pgfqpoint{5.949920in}{6.313785in}}{\pgfqpoint{5.944334in}{6.311471in}}{\pgfqpoint{5.940216in}{6.307353in}}%
\pgfpathcurveto{\pgfqpoint{5.936097in}{6.303235in}}{\pgfqpoint{5.933784in}{6.297649in}}{\pgfqpoint{5.933784in}{6.291825in}}%
\pgfpathcurveto{\pgfqpoint{5.933784in}{6.286001in}}{\pgfqpoint{5.936097in}{6.280415in}}{\pgfqpoint{5.940216in}{6.276297in}}%
\pgfpathcurveto{\pgfqpoint{5.944334in}{6.272179in}}{\pgfqpoint{5.949920in}{6.269865in}}{\pgfqpoint{5.955744in}{6.269865in}}%
\pgfpathlineto{\pgfqpoint{5.955744in}{6.269865in}}%
\pgfpathclose%
\pgfusepath{stroke,fill}%
\end{pgfscope}%
\begin{pgfscope}%
\pgfpathrectangle{\pgfqpoint{1.000000in}{1.148311in}}{\pgfqpoint{6.200000in}{5.623377in}}%
\pgfusepath{clip}%
\pgfsetbuttcap%
\pgfsetroundjoin%
\definecolor{currentfill}{rgb}{0.200000,0.200000,0.800000}%
\pgfsetfillcolor{currentfill}%
\pgfsetlinewidth{1.003750pt}%
\definecolor{currentstroke}{rgb}{0.200000,0.200000,0.800000}%
\pgfsetstrokecolor{currentstroke}%
\pgfsetdash{}{0pt}%
\pgfpathmoveto{\pgfqpoint{5.911019in}{6.336260in}}%
\pgfpathcurveto{\pgfqpoint{5.916843in}{6.336260in}}{\pgfqpoint{5.922429in}{6.338574in}}{\pgfqpoint{5.926548in}{6.342692in}}%
\pgfpathcurveto{\pgfqpoint{5.930666in}{6.346810in}}{\pgfqpoint{5.932980in}{6.352397in}}{\pgfqpoint{5.932980in}{6.358221in}}%
\pgfpathcurveto{\pgfqpoint{5.932980in}{6.364044in}}{\pgfqpoint{5.930666in}{6.369631in}}{\pgfqpoint{5.926548in}{6.373749in}}%
\pgfpathcurveto{\pgfqpoint{5.922429in}{6.377867in}}{\pgfqpoint{5.916843in}{6.380181in}}{\pgfqpoint{5.911019in}{6.380181in}}%
\pgfpathcurveto{\pgfqpoint{5.905195in}{6.380181in}}{\pgfqpoint{5.899609in}{6.377867in}}{\pgfqpoint{5.895491in}{6.373749in}}%
\pgfpathcurveto{\pgfqpoint{5.891373in}{6.369631in}}{\pgfqpoint{5.889059in}{6.364044in}}{\pgfqpoint{5.889059in}{6.358221in}}%
\pgfpathcurveto{\pgfqpoint{5.889059in}{6.352397in}}{\pgfqpoint{5.891373in}{6.346810in}}{\pgfqpoint{5.895491in}{6.342692in}}%
\pgfpathcurveto{\pgfqpoint{5.899609in}{6.338574in}}{\pgfqpoint{5.905195in}{6.336260in}}{\pgfqpoint{5.911019in}{6.336260in}}%
\pgfpathlineto{\pgfqpoint{5.911019in}{6.336260in}}%
\pgfpathclose%
\pgfusepath{stroke,fill}%
\end{pgfscope}%
\begin{pgfscope}%
\pgfpathrectangle{\pgfqpoint{1.000000in}{1.148311in}}{\pgfqpoint{6.200000in}{5.623377in}}%
\pgfusepath{clip}%
\pgfsetbuttcap%
\pgfsetroundjoin%
\definecolor{currentfill}{rgb}{0.200000,0.200000,0.800000}%
\pgfsetfillcolor{currentfill}%
\pgfsetlinewidth{1.003750pt}%
\definecolor{currentstroke}{rgb}{0.200000,0.200000,0.800000}%
\pgfsetstrokecolor{currentstroke}%
\pgfsetdash{}{0pt}%
\pgfpathmoveto{\pgfqpoint{5.826316in}{6.312875in}}%
\pgfpathcurveto{\pgfqpoint{5.832140in}{6.312875in}}{\pgfqpoint{5.837726in}{6.315189in}}{\pgfqpoint{5.841845in}{6.319307in}}%
\pgfpathcurveto{\pgfqpoint{5.845963in}{6.323425in}}{\pgfqpoint{5.848277in}{6.329011in}}{\pgfqpoint{5.848277in}{6.334835in}}%
\pgfpathcurveto{\pgfqpoint{5.848277in}{6.340659in}}{\pgfqpoint{5.845963in}{6.346246in}}{\pgfqpoint{5.841845in}{6.350364in}}%
\pgfpathcurveto{\pgfqpoint{5.837726in}{6.354482in}}{\pgfqpoint{5.832140in}{6.356796in}}{\pgfqpoint{5.826316in}{6.356796in}}%
\pgfpathcurveto{\pgfqpoint{5.820492in}{6.356796in}}{\pgfqpoint{5.814906in}{6.354482in}}{\pgfqpoint{5.810788in}{6.350364in}}%
\pgfpathcurveto{\pgfqpoint{5.806670in}{6.346246in}}{\pgfqpoint{5.804356in}{6.340659in}}{\pgfqpoint{5.804356in}{6.334835in}}%
\pgfpathcurveto{\pgfqpoint{5.804356in}{6.329011in}}{\pgfqpoint{5.806670in}{6.323425in}}{\pgfqpoint{5.810788in}{6.319307in}}%
\pgfpathcurveto{\pgfqpoint{5.814906in}{6.315189in}}{\pgfqpoint{5.820492in}{6.312875in}}{\pgfqpoint{5.826316in}{6.312875in}}%
\pgfpathlineto{\pgfqpoint{5.826316in}{6.312875in}}%
\pgfpathclose%
\pgfusepath{stroke,fill}%
\end{pgfscope}%
\begin{pgfscope}%
\pgfpathrectangle{\pgfqpoint{1.000000in}{1.148311in}}{\pgfqpoint{6.200000in}{5.623377in}}%
\pgfusepath{clip}%
\pgfsetbuttcap%
\pgfsetroundjoin%
\definecolor{currentfill}{rgb}{0.200000,0.200000,0.800000}%
\pgfsetfillcolor{currentfill}%
\pgfsetlinewidth{1.003750pt}%
\definecolor{currentstroke}{rgb}{0.200000,0.200000,0.800000}%
\pgfsetstrokecolor{currentstroke}%
\pgfsetdash{}{0pt}%
\pgfpathmoveto{\pgfqpoint{5.753719in}{6.306177in}}%
\pgfpathcurveto{\pgfqpoint{5.759543in}{6.306177in}}{\pgfqpoint{5.765130in}{6.308491in}}{\pgfqpoint{5.769248in}{6.312609in}}%
\pgfpathcurveto{\pgfqpoint{5.773366in}{6.316727in}}{\pgfqpoint{5.775680in}{6.322313in}}{\pgfqpoint{5.775680in}{6.328137in}}%
\pgfpathcurveto{\pgfqpoint{5.775680in}{6.333961in}}{\pgfqpoint{5.773366in}{6.339547in}}{\pgfqpoint{5.769248in}{6.343666in}}%
\pgfpathcurveto{\pgfqpoint{5.765130in}{6.347784in}}{\pgfqpoint{5.759543in}{6.350098in}}{\pgfqpoint{5.753719in}{6.350098in}}%
\pgfpathcurveto{\pgfqpoint{5.747896in}{6.350098in}}{\pgfqpoint{5.742309in}{6.347784in}}{\pgfqpoint{5.738191in}{6.343666in}}%
\pgfpathcurveto{\pgfqpoint{5.734073in}{6.339547in}}{\pgfqpoint{5.731759in}{6.333961in}}{\pgfqpoint{5.731759in}{6.328137in}}%
\pgfpathcurveto{\pgfqpoint{5.731759in}{6.322313in}}{\pgfqpoint{5.734073in}{6.316727in}}{\pgfqpoint{5.738191in}{6.312609in}}%
\pgfpathcurveto{\pgfqpoint{5.742309in}{6.308491in}}{\pgfqpoint{5.747896in}{6.306177in}}{\pgfqpoint{5.753719in}{6.306177in}}%
\pgfpathlineto{\pgfqpoint{5.753719in}{6.306177in}}%
\pgfpathclose%
\pgfusepath{stroke,fill}%
\end{pgfscope}%
\begin{pgfscope}%
\pgfpathrectangle{\pgfqpoint{1.000000in}{1.148311in}}{\pgfqpoint{6.200000in}{5.623377in}}%
\pgfusepath{clip}%
\pgfsetbuttcap%
\pgfsetroundjoin%
\definecolor{currentfill}{rgb}{0.200000,0.200000,0.800000}%
\pgfsetfillcolor{currentfill}%
\pgfsetlinewidth{1.003750pt}%
\definecolor{currentstroke}{rgb}{0.200000,0.200000,0.800000}%
\pgfsetstrokecolor{currentstroke}%
\pgfsetdash{}{0pt}%
\pgfpathmoveto{\pgfqpoint{5.715456in}{6.425964in}}%
\pgfpathcurveto{\pgfqpoint{5.721280in}{6.425964in}}{\pgfqpoint{5.726866in}{6.428278in}}{\pgfqpoint{5.730984in}{6.432396in}}%
\pgfpathcurveto{\pgfqpoint{5.735103in}{6.436515in}}{\pgfqpoint{5.737416in}{6.442101in}}{\pgfqpoint{5.737416in}{6.447925in}}%
\pgfpathcurveto{\pgfqpoint{5.737416in}{6.453749in}}{\pgfqpoint{5.735103in}{6.459335in}}{\pgfqpoint{5.730984in}{6.463453in}}%
\pgfpathcurveto{\pgfqpoint{5.726866in}{6.467571in}}{\pgfqpoint{5.721280in}{6.469885in}}{\pgfqpoint{5.715456in}{6.469885in}}%
\pgfpathcurveto{\pgfqpoint{5.709632in}{6.469885in}}{\pgfqpoint{5.704046in}{6.467571in}}{\pgfqpoint{5.699928in}{6.463453in}}%
\pgfpathcurveto{\pgfqpoint{5.695810in}{6.459335in}}{\pgfqpoint{5.693496in}{6.453749in}}{\pgfqpoint{5.693496in}{6.447925in}}%
\pgfpathcurveto{\pgfqpoint{5.693496in}{6.442101in}}{\pgfqpoint{5.695810in}{6.436515in}}{\pgfqpoint{5.699928in}{6.432396in}}%
\pgfpathcurveto{\pgfqpoint{5.704046in}{6.428278in}}{\pgfqpoint{5.709632in}{6.425964in}}{\pgfqpoint{5.715456in}{6.425964in}}%
\pgfpathlineto{\pgfqpoint{5.715456in}{6.425964in}}%
\pgfpathclose%
\pgfusepath{stroke,fill}%
\end{pgfscope}%
\begin{pgfscope}%
\pgfpathrectangle{\pgfqpoint{1.000000in}{1.148311in}}{\pgfqpoint{6.200000in}{5.623377in}}%
\pgfusepath{clip}%
\pgfsetbuttcap%
\pgfsetroundjoin%
\definecolor{currentfill}{rgb}{0.200000,0.200000,0.800000}%
\pgfsetfillcolor{currentfill}%
\pgfsetlinewidth{1.003750pt}%
\definecolor{currentstroke}{rgb}{0.200000,0.200000,0.800000}%
\pgfsetstrokecolor{currentstroke}%
\pgfsetdash{}{0pt}%
\pgfpathmoveto{\pgfqpoint{5.630454in}{6.359130in}}%
\pgfpathcurveto{\pgfqpoint{5.636278in}{6.359130in}}{\pgfqpoint{5.641864in}{6.361443in}}{\pgfqpoint{5.645982in}{6.365562in}}%
\pgfpathcurveto{\pgfqpoint{5.650100in}{6.369680in}}{\pgfqpoint{5.652414in}{6.375266in}}{\pgfqpoint{5.652414in}{6.381090in}}%
\pgfpathcurveto{\pgfqpoint{5.652414in}{6.386914in}}{\pgfqpoint{5.650100in}{6.392500in}}{\pgfqpoint{5.645982in}{6.396618in}}%
\pgfpathcurveto{\pgfqpoint{5.641864in}{6.400736in}}{\pgfqpoint{5.636278in}{6.403050in}}{\pgfqpoint{5.630454in}{6.403050in}}%
\pgfpathcurveto{\pgfqpoint{5.624630in}{6.403050in}}{\pgfqpoint{5.619044in}{6.400736in}}{\pgfqpoint{5.614926in}{6.396618in}}%
\pgfpathcurveto{\pgfqpoint{5.610808in}{6.392500in}}{\pgfqpoint{5.608494in}{6.386914in}}{\pgfqpoint{5.608494in}{6.381090in}}%
\pgfpathcurveto{\pgfqpoint{5.608494in}{6.375266in}}{\pgfqpoint{5.610808in}{6.369680in}}{\pgfqpoint{5.614926in}{6.365562in}}%
\pgfpathcurveto{\pgfqpoint{5.619044in}{6.361443in}}{\pgfqpoint{5.624630in}{6.359130in}}{\pgfqpoint{5.630454in}{6.359130in}}%
\pgfpathlineto{\pgfqpoint{5.630454in}{6.359130in}}%
\pgfpathclose%
\pgfusepath{stroke,fill}%
\end{pgfscope}%
\begin{pgfscope}%
\pgfpathrectangle{\pgfqpoint{1.000000in}{1.148311in}}{\pgfqpoint{6.200000in}{5.623377in}}%
\pgfusepath{clip}%
\pgfsetbuttcap%
\pgfsetroundjoin%
\definecolor{currentfill}{rgb}{0.200000,0.200000,0.800000}%
\pgfsetfillcolor{currentfill}%
\pgfsetlinewidth{1.003750pt}%
\definecolor{currentstroke}{rgb}{0.200000,0.200000,0.800000}%
\pgfsetstrokecolor{currentstroke}%
\pgfsetdash{}{0pt}%
\pgfpathmoveto{\pgfqpoint{5.571761in}{6.435981in}}%
\pgfpathcurveto{\pgfqpoint{5.577585in}{6.435981in}}{\pgfqpoint{5.583171in}{6.438295in}}{\pgfqpoint{5.587290in}{6.442413in}}%
\pgfpathcurveto{\pgfqpoint{5.591408in}{6.446531in}}{\pgfqpoint{5.593722in}{6.452118in}}{\pgfqpoint{5.593722in}{6.457941in}}%
\pgfpathcurveto{\pgfqpoint{5.593722in}{6.463765in}}{\pgfqpoint{5.591408in}{6.469352in}}{\pgfqpoint{5.587290in}{6.473470in}}%
\pgfpathcurveto{\pgfqpoint{5.583171in}{6.477588in}}{\pgfqpoint{5.577585in}{6.479902in}}{\pgfqpoint{5.571761in}{6.479902in}}%
\pgfpathcurveto{\pgfqpoint{5.565937in}{6.479902in}}{\pgfqpoint{5.560351in}{6.477588in}}{\pgfqpoint{5.556233in}{6.473470in}}%
\pgfpathcurveto{\pgfqpoint{5.552115in}{6.469352in}}{\pgfqpoint{5.549801in}{6.463765in}}{\pgfqpoint{5.549801in}{6.457941in}}%
\pgfpathcurveto{\pgfqpoint{5.549801in}{6.452118in}}{\pgfqpoint{5.552115in}{6.446531in}}{\pgfqpoint{5.556233in}{6.442413in}}%
\pgfpathcurveto{\pgfqpoint{5.560351in}{6.438295in}}{\pgfqpoint{5.565937in}{6.435981in}}{\pgfqpoint{5.571761in}{6.435981in}}%
\pgfpathlineto{\pgfqpoint{5.571761in}{6.435981in}}%
\pgfpathclose%
\pgfusepath{stroke,fill}%
\end{pgfscope}%
\begin{pgfscope}%
\pgfpathrectangle{\pgfqpoint{1.000000in}{1.148311in}}{\pgfqpoint{6.200000in}{5.623377in}}%
\pgfusepath{clip}%
\pgfsetbuttcap%
\pgfsetroundjoin%
\definecolor{currentfill}{rgb}{0.200000,0.200000,0.800000}%
\pgfsetfillcolor{currentfill}%
\pgfsetlinewidth{1.003750pt}%
\definecolor{currentstroke}{rgb}{0.200000,0.200000,0.800000}%
\pgfsetstrokecolor{currentstroke}%
\pgfsetdash{}{0pt}%
\pgfpathmoveto{\pgfqpoint{5.497765in}{6.377839in}}%
\pgfpathcurveto{\pgfqpoint{5.503589in}{6.377839in}}{\pgfqpoint{5.509175in}{6.380153in}}{\pgfqpoint{5.513293in}{6.384271in}}%
\pgfpathcurveto{\pgfqpoint{5.517411in}{6.388390in}}{\pgfqpoint{5.519725in}{6.393976in}}{\pgfqpoint{5.519725in}{6.399800in}}%
\pgfpathcurveto{\pgfqpoint{5.519725in}{6.405624in}}{\pgfqpoint{5.517411in}{6.411210in}}{\pgfqpoint{5.513293in}{6.415328in}}%
\pgfpathcurveto{\pgfqpoint{5.509175in}{6.419446in}}{\pgfqpoint{5.503589in}{6.421760in}}{\pgfqpoint{5.497765in}{6.421760in}}%
\pgfpathcurveto{\pgfqpoint{5.491941in}{6.421760in}}{\pgfqpoint{5.486355in}{6.419446in}}{\pgfqpoint{5.482237in}{6.415328in}}%
\pgfpathcurveto{\pgfqpoint{5.478118in}{6.411210in}}{\pgfqpoint{5.475805in}{6.405624in}}{\pgfqpoint{5.475805in}{6.399800in}}%
\pgfpathcurveto{\pgfqpoint{5.475805in}{6.393976in}}{\pgfqpoint{5.478118in}{6.388390in}}{\pgfqpoint{5.482237in}{6.384271in}}%
\pgfpathcurveto{\pgfqpoint{5.486355in}{6.380153in}}{\pgfqpoint{5.491941in}{6.377839in}}{\pgfqpoint{5.497765in}{6.377839in}}%
\pgfpathlineto{\pgfqpoint{5.497765in}{6.377839in}}%
\pgfpathclose%
\pgfusepath{stroke,fill}%
\end{pgfscope}%
\begin{pgfscope}%
\pgfpathrectangle{\pgfqpoint{1.000000in}{1.148311in}}{\pgfqpoint{6.200000in}{5.623377in}}%
\pgfusepath{clip}%
\pgfsetbuttcap%
\pgfsetroundjoin%
\definecolor{currentfill}{rgb}{0.200000,0.200000,0.800000}%
\pgfsetfillcolor{currentfill}%
\pgfsetlinewidth{1.003750pt}%
\definecolor{currentstroke}{rgb}{0.200000,0.200000,0.800000}%
\pgfsetstrokecolor{currentstroke}%
\pgfsetdash{}{0pt}%
\pgfpathmoveto{\pgfqpoint{5.430262in}{6.402951in}}%
\pgfpathcurveto{\pgfqpoint{5.436086in}{6.402951in}}{\pgfqpoint{5.441672in}{6.405264in}}{\pgfqpoint{5.445790in}{6.409383in}}%
\pgfpathcurveto{\pgfqpoint{5.449908in}{6.413501in}}{\pgfqpoint{5.452222in}{6.419087in}}{\pgfqpoint{5.452222in}{6.424911in}}%
\pgfpathcurveto{\pgfqpoint{5.452222in}{6.430735in}}{\pgfqpoint{5.449908in}{6.436321in}}{\pgfqpoint{5.445790in}{6.440439in}}%
\pgfpathcurveto{\pgfqpoint{5.441672in}{6.444557in}}{\pgfqpoint{5.436086in}{6.446871in}}{\pgfqpoint{5.430262in}{6.446871in}}%
\pgfpathcurveto{\pgfqpoint{5.424438in}{6.446871in}}{\pgfqpoint{5.418852in}{6.444557in}}{\pgfqpoint{5.414733in}{6.440439in}}%
\pgfpathcurveto{\pgfqpoint{5.410615in}{6.436321in}}{\pgfqpoint{5.408301in}{6.430735in}}{\pgfqpoint{5.408301in}{6.424911in}}%
\pgfpathcurveto{\pgfqpoint{5.408301in}{6.419087in}}{\pgfqpoint{5.410615in}{6.413501in}}{\pgfqpoint{5.414733in}{6.409383in}}%
\pgfpathcurveto{\pgfqpoint{5.418852in}{6.405264in}}{\pgfqpoint{5.424438in}{6.402951in}}{\pgfqpoint{5.430262in}{6.402951in}}%
\pgfpathlineto{\pgfqpoint{5.430262in}{6.402951in}}%
\pgfpathclose%
\pgfusepath{stroke,fill}%
\end{pgfscope}%
\begin{pgfscope}%
\pgfpathrectangle{\pgfqpoint{1.000000in}{1.148311in}}{\pgfqpoint{6.200000in}{5.623377in}}%
\pgfusepath{clip}%
\pgfsetbuttcap%
\pgfsetroundjoin%
\definecolor{currentfill}{rgb}{0.200000,0.200000,0.800000}%
\pgfsetfillcolor{currentfill}%
\pgfsetlinewidth{1.003750pt}%
\definecolor{currentstroke}{rgb}{0.200000,0.200000,0.800000}%
\pgfsetstrokecolor{currentstroke}%
\pgfsetdash{}{0pt}%
\pgfpathmoveto{\pgfqpoint{5.361492in}{6.402086in}}%
\pgfpathcurveto{\pgfqpoint{5.367316in}{6.402086in}}{\pgfqpoint{5.372902in}{6.404399in}}{\pgfqpoint{5.377021in}{6.408518in}}%
\pgfpathcurveto{\pgfqpoint{5.381139in}{6.412636in}}{\pgfqpoint{5.383453in}{6.418222in}}{\pgfqpoint{5.383453in}{6.424046in}}%
\pgfpathcurveto{\pgfqpoint{5.383453in}{6.429870in}}{\pgfqpoint{5.381139in}{6.435456in}}{\pgfqpoint{5.377021in}{6.439574in}}%
\pgfpathcurveto{\pgfqpoint{5.372902in}{6.443692in}}{\pgfqpoint{5.367316in}{6.446006in}}{\pgfqpoint{5.361492in}{6.446006in}}%
\pgfpathcurveto{\pgfqpoint{5.355668in}{6.446006in}}{\pgfqpoint{5.350082in}{6.443692in}}{\pgfqpoint{5.345964in}{6.439574in}}%
\pgfpathcurveto{\pgfqpoint{5.341846in}{6.435456in}}{\pgfqpoint{5.339532in}{6.429870in}}{\pgfqpoint{5.339532in}{6.424046in}}%
\pgfpathcurveto{\pgfqpoint{5.339532in}{6.418222in}}{\pgfqpoint{5.341846in}{6.412636in}}{\pgfqpoint{5.345964in}{6.408518in}}%
\pgfpathcurveto{\pgfqpoint{5.350082in}{6.404399in}}{\pgfqpoint{5.355668in}{6.402086in}}{\pgfqpoint{5.361492in}{6.402086in}}%
\pgfpathlineto{\pgfqpoint{5.361492in}{6.402086in}}%
\pgfpathclose%
\pgfusepath{stroke,fill}%
\end{pgfscope}%
\begin{pgfscope}%
\pgfpathrectangle{\pgfqpoint{1.000000in}{1.148311in}}{\pgfqpoint{6.200000in}{5.623377in}}%
\pgfusepath{clip}%
\pgfsetbuttcap%
\pgfsetroundjoin%
\definecolor{currentfill}{rgb}{0.200000,0.200000,0.800000}%
\pgfsetfillcolor{currentfill}%
\pgfsetlinewidth{1.003750pt}%
\definecolor{currentstroke}{rgb}{0.200000,0.200000,0.800000}%
\pgfsetstrokecolor{currentstroke}%
\pgfsetdash{}{0pt}%
\pgfpathmoveto{\pgfqpoint{5.292358in}{6.399668in}}%
\pgfpathcurveto{\pgfqpoint{5.298182in}{6.399668in}}{\pgfqpoint{5.303769in}{6.401982in}}{\pgfqpoint{5.307887in}{6.406100in}}%
\pgfpathcurveto{\pgfqpoint{5.312005in}{6.410218in}}{\pgfqpoint{5.314319in}{6.415804in}}{\pgfqpoint{5.314319in}{6.421628in}}%
\pgfpathcurveto{\pgfqpoint{5.314319in}{6.427452in}}{\pgfqpoint{5.312005in}{6.433038in}}{\pgfqpoint{5.307887in}{6.437156in}}%
\pgfpathcurveto{\pgfqpoint{5.303769in}{6.441274in}}{\pgfqpoint{5.298182in}{6.443588in}}{\pgfqpoint{5.292358in}{6.443588in}}%
\pgfpathcurveto{\pgfqpoint{5.286535in}{6.443588in}}{\pgfqpoint{5.280948in}{6.441274in}}{\pgfqpoint{5.276830in}{6.437156in}}%
\pgfpathcurveto{\pgfqpoint{5.272712in}{6.433038in}}{\pgfqpoint{5.270398in}{6.427452in}}{\pgfqpoint{5.270398in}{6.421628in}}%
\pgfpathcurveto{\pgfqpoint{5.270398in}{6.415804in}}{\pgfqpoint{5.272712in}{6.410218in}}{\pgfqpoint{5.276830in}{6.406100in}}%
\pgfpathcurveto{\pgfqpoint{5.280948in}{6.401982in}}{\pgfqpoint{5.286535in}{6.399668in}}{\pgfqpoint{5.292358in}{6.399668in}}%
\pgfpathlineto{\pgfqpoint{5.292358in}{6.399668in}}%
\pgfpathclose%
\pgfusepath{stroke,fill}%
\end{pgfscope}%
\begin{pgfscope}%
\pgfpathrectangle{\pgfqpoint{1.000000in}{1.148311in}}{\pgfqpoint{6.200000in}{5.623377in}}%
\pgfusepath{clip}%
\pgfsetbuttcap%
\pgfsetroundjoin%
\definecolor{currentfill}{rgb}{0.200000,0.200000,0.800000}%
\pgfsetfillcolor{currentfill}%
\pgfsetlinewidth{1.003750pt}%
\definecolor{currentstroke}{rgb}{0.200000,0.200000,0.800000}%
\pgfsetstrokecolor{currentstroke}%
\pgfsetdash{}{0pt}%
\pgfpathmoveto{\pgfqpoint{5.229263in}{6.363781in}}%
\pgfpathcurveto{\pgfqpoint{5.235087in}{6.363781in}}{\pgfqpoint{5.240673in}{6.366095in}}{\pgfqpoint{5.244791in}{6.370213in}}%
\pgfpathcurveto{\pgfqpoint{5.248909in}{6.374332in}}{\pgfqpoint{5.251223in}{6.379918in}}{\pgfqpoint{5.251223in}{6.385742in}}%
\pgfpathcurveto{\pgfqpoint{5.251223in}{6.391566in}}{\pgfqpoint{5.248909in}{6.397152in}}{\pgfqpoint{5.244791in}{6.401270in}}%
\pgfpathcurveto{\pgfqpoint{5.240673in}{6.405388in}}{\pgfqpoint{5.235087in}{6.407702in}}{\pgfqpoint{5.229263in}{6.407702in}}%
\pgfpathcurveto{\pgfqpoint{5.223439in}{6.407702in}}{\pgfqpoint{5.217852in}{6.405388in}}{\pgfqpoint{5.213734in}{6.401270in}}%
\pgfpathcurveto{\pgfqpoint{5.209616in}{6.397152in}}{\pgfqpoint{5.207302in}{6.391566in}}{\pgfqpoint{5.207302in}{6.385742in}}%
\pgfpathcurveto{\pgfqpoint{5.207302in}{6.379918in}}{\pgfqpoint{5.209616in}{6.374332in}}{\pgfqpoint{5.213734in}{6.370213in}}%
\pgfpathcurveto{\pgfqpoint{5.217852in}{6.366095in}}{\pgfqpoint{5.223439in}{6.363781in}}{\pgfqpoint{5.229263in}{6.363781in}}%
\pgfpathlineto{\pgfqpoint{5.229263in}{6.363781in}}%
\pgfpathclose%
\pgfusepath{stroke,fill}%
\end{pgfscope}%
\begin{pgfscope}%
\pgfpathrectangle{\pgfqpoint{1.000000in}{1.148311in}}{\pgfqpoint{6.200000in}{5.623377in}}%
\pgfusepath{clip}%
\pgfsetbuttcap%
\pgfsetroundjoin%
\definecolor{currentfill}{rgb}{0.200000,0.200000,0.800000}%
\pgfsetfillcolor{currentfill}%
\pgfsetlinewidth{1.003750pt}%
\definecolor{currentstroke}{rgb}{0.200000,0.200000,0.800000}%
\pgfsetstrokecolor{currentstroke}%
\pgfsetdash{}{0pt}%
\pgfpathmoveto{\pgfqpoint{5.149205in}{6.399815in}}%
\pgfpathcurveto{\pgfqpoint{5.155029in}{6.399815in}}{\pgfqpoint{5.160615in}{6.402129in}}{\pgfqpoint{5.164733in}{6.406247in}}%
\pgfpathcurveto{\pgfqpoint{5.168851in}{6.410365in}}{\pgfqpoint{5.171165in}{6.415952in}}{\pgfqpoint{5.171165in}{6.421776in}}%
\pgfpathcurveto{\pgfqpoint{5.171165in}{6.427600in}}{\pgfqpoint{5.168851in}{6.433186in}}{\pgfqpoint{5.164733in}{6.437304in}}%
\pgfpathcurveto{\pgfqpoint{5.160615in}{6.441422in}}{\pgfqpoint{5.155029in}{6.443736in}}{\pgfqpoint{5.149205in}{6.443736in}}%
\pgfpathcurveto{\pgfqpoint{5.143381in}{6.443736in}}{\pgfqpoint{5.137795in}{6.441422in}}{\pgfqpoint{5.133677in}{6.437304in}}%
\pgfpathcurveto{\pgfqpoint{5.129559in}{6.433186in}}{\pgfqpoint{5.127245in}{6.427600in}}{\pgfqpoint{5.127245in}{6.421776in}}%
\pgfpathcurveto{\pgfqpoint{5.127245in}{6.415952in}}{\pgfqpoint{5.129559in}{6.410365in}}{\pgfqpoint{5.133677in}{6.406247in}}%
\pgfpathcurveto{\pgfqpoint{5.137795in}{6.402129in}}{\pgfqpoint{5.143381in}{6.399815in}}{\pgfqpoint{5.149205in}{6.399815in}}%
\pgfpathlineto{\pgfqpoint{5.149205in}{6.399815in}}%
\pgfpathclose%
\pgfusepath{stroke,fill}%
\end{pgfscope}%
\begin{pgfscope}%
\pgfpathrectangle{\pgfqpoint{1.000000in}{1.148311in}}{\pgfqpoint{6.200000in}{5.623377in}}%
\pgfusepath{clip}%
\pgfsetbuttcap%
\pgfsetroundjoin%
\definecolor{currentfill}{rgb}{0.200000,0.200000,0.800000}%
\pgfsetfillcolor{currentfill}%
\pgfsetlinewidth{1.003750pt}%
\definecolor{currentstroke}{rgb}{0.200000,0.200000,0.800000}%
\pgfsetstrokecolor{currentstroke}%
\pgfsetdash{}{0pt}%
\pgfpathmoveto{\pgfqpoint{5.078505in}{6.387068in}}%
\pgfpathcurveto{\pgfqpoint{5.084329in}{6.387068in}}{\pgfqpoint{5.089916in}{6.389382in}}{\pgfqpoint{5.094034in}{6.393500in}}%
\pgfpathcurveto{\pgfqpoint{5.098152in}{6.397618in}}{\pgfqpoint{5.100466in}{6.403204in}}{\pgfqpoint{5.100466in}{6.409028in}}%
\pgfpathcurveto{\pgfqpoint{5.100466in}{6.414852in}}{\pgfqpoint{5.098152in}{6.420438in}}{\pgfqpoint{5.094034in}{6.424556in}}%
\pgfpathcurveto{\pgfqpoint{5.089916in}{6.428674in}}{\pgfqpoint{5.084329in}{6.430988in}}{\pgfqpoint{5.078505in}{6.430988in}}%
\pgfpathcurveto{\pgfqpoint{5.072681in}{6.430988in}}{\pgfqpoint{5.067095in}{6.428674in}}{\pgfqpoint{5.062977in}{6.424556in}}%
\pgfpathcurveto{\pgfqpoint{5.058859in}{6.420438in}}{\pgfqpoint{5.056545in}{6.414852in}}{\pgfqpoint{5.056545in}{6.409028in}}%
\pgfpathcurveto{\pgfqpoint{5.056545in}{6.403204in}}{\pgfqpoint{5.058859in}{6.397618in}}{\pgfqpoint{5.062977in}{6.393500in}}%
\pgfpathcurveto{\pgfqpoint{5.067095in}{6.389382in}}{\pgfqpoint{5.072681in}{6.387068in}}{\pgfqpoint{5.078505in}{6.387068in}}%
\pgfpathlineto{\pgfqpoint{5.078505in}{6.387068in}}%
\pgfpathclose%
\pgfusepath{stroke,fill}%
\end{pgfscope}%
\begin{pgfscope}%
\pgfpathrectangle{\pgfqpoint{1.000000in}{1.148311in}}{\pgfqpoint{6.200000in}{5.623377in}}%
\pgfusepath{clip}%
\pgfsetbuttcap%
\pgfsetroundjoin%
\definecolor{currentfill}{rgb}{0.200000,0.200000,0.800000}%
\pgfsetfillcolor{currentfill}%
\pgfsetlinewidth{1.003750pt}%
\definecolor{currentstroke}{rgb}{0.200000,0.200000,0.800000}%
\pgfsetstrokecolor{currentstroke}%
\pgfsetdash{}{0pt}%
\pgfpathmoveto{\pgfqpoint{5.007381in}{6.371698in}}%
\pgfpathcurveto{\pgfqpoint{5.013205in}{6.371698in}}{\pgfqpoint{5.018791in}{6.374012in}}{\pgfqpoint{5.022909in}{6.378130in}}%
\pgfpathcurveto{\pgfqpoint{5.027027in}{6.382248in}}{\pgfqpoint{5.029341in}{6.387834in}}{\pgfqpoint{5.029341in}{6.393658in}}%
\pgfpathcurveto{\pgfqpoint{5.029341in}{6.399482in}}{\pgfqpoint{5.027027in}{6.405068in}}{\pgfqpoint{5.022909in}{6.409186in}}%
\pgfpathcurveto{\pgfqpoint{5.018791in}{6.413305in}}{\pgfqpoint{5.013205in}{6.415618in}}{\pgfqpoint{5.007381in}{6.415618in}}%
\pgfpathcurveto{\pgfqpoint{5.001557in}{6.415618in}}{\pgfqpoint{4.995971in}{6.413305in}}{\pgfqpoint{4.991853in}{6.409186in}}%
\pgfpathcurveto{\pgfqpoint{4.987734in}{6.405068in}}{\pgfqpoint{4.985420in}{6.399482in}}{\pgfqpoint{4.985420in}{6.393658in}}%
\pgfpathcurveto{\pgfqpoint{4.985420in}{6.387834in}}{\pgfqpoint{4.987734in}{6.382248in}}{\pgfqpoint{4.991853in}{6.378130in}}%
\pgfpathcurveto{\pgfqpoint{4.995971in}{6.374012in}}{\pgfqpoint{5.001557in}{6.371698in}}{\pgfqpoint{5.007381in}{6.371698in}}%
\pgfpathlineto{\pgfqpoint{5.007381in}{6.371698in}}%
\pgfpathclose%
\pgfusepath{stroke,fill}%
\end{pgfscope}%
\begin{pgfscope}%
\pgfpathrectangle{\pgfqpoint{1.000000in}{1.148311in}}{\pgfqpoint{6.200000in}{5.623377in}}%
\pgfusepath{clip}%
\pgfsetbuttcap%
\pgfsetroundjoin%
\definecolor{currentfill}{rgb}{0.800000,0.800000,0.200000}%
\pgfsetfillcolor{currentfill}%
\pgfsetlinewidth{1.003750pt}%
\definecolor{currentstroke}{rgb}{0.800000,0.800000,0.200000}%
\pgfsetstrokecolor{currentstroke}%
\pgfsetdash{}{0pt}%
\pgfpathmoveto{\pgfqpoint{4.970953in}{6.282521in}}%
\pgfpathcurveto{\pgfqpoint{4.976777in}{6.282521in}}{\pgfqpoint{4.982363in}{6.284835in}}{\pgfqpoint{4.986481in}{6.288953in}}%
\pgfpathcurveto{\pgfqpoint{4.990599in}{6.293071in}}{\pgfqpoint{4.992913in}{6.298657in}}{\pgfqpoint{4.992913in}{6.304481in}}%
\pgfpathcurveto{\pgfqpoint{4.992913in}{6.310305in}}{\pgfqpoint{4.990599in}{6.315891in}}{\pgfqpoint{4.986481in}{6.320009in}}%
\pgfpathcurveto{\pgfqpoint{4.982363in}{6.324127in}}{\pgfqpoint{4.976777in}{6.326441in}}{\pgfqpoint{4.970953in}{6.326441in}}%
\pgfpathcurveto{\pgfqpoint{4.965129in}{6.326441in}}{\pgfqpoint{4.959543in}{6.324127in}}{\pgfqpoint{4.955425in}{6.320009in}}%
\pgfpathcurveto{\pgfqpoint{4.951306in}{6.315891in}}{\pgfqpoint{4.948993in}{6.310305in}}{\pgfqpoint{4.948993in}{6.304481in}}%
\pgfpathcurveto{\pgfqpoint{4.948993in}{6.298657in}}{\pgfqpoint{4.951306in}{6.293071in}}{\pgfqpoint{4.955425in}{6.288953in}}%
\pgfpathcurveto{\pgfqpoint{4.959543in}{6.284835in}}{\pgfqpoint{4.965129in}{6.282521in}}{\pgfqpoint{4.970953in}{6.282521in}}%
\pgfpathlineto{\pgfqpoint{4.970953in}{6.282521in}}%
\pgfpathclose%
\pgfusepath{stroke,fill}%
\end{pgfscope}%
\begin{pgfscope}%
\pgfpathrectangle{\pgfqpoint{1.000000in}{1.148311in}}{\pgfqpoint{6.200000in}{5.623377in}}%
\pgfusepath{clip}%
\pgfsetbuttcap%
\pgfsetroundjoin%
\definecolor{currentfill}{rgb}{0.200000,0.200000,0.800000}%
\pgfsetfillcolor{currentfill}%
\pgfsetlinewidth{1.003750pt}%
\definecolor{currentstroke}{rgb}{0.200000,0.200000,0.800000}%
\pgfsetstrokecolor{currentstroke}%
\pgfsetdash{}{0pt}%
\pgfpathmoveto{\pgfqpoint{4.924243in}{6.227319in}}%
\pgfpathcurveto{\pgfqpoint{4.930067in}{6.227319in}}{\pgfqpoint{4.935653in}{6.229633in}}{\pgfqpoint{4.939771in}{6.233751in}}%
\pgfpathcurveto{\pgfqpoint{4.943889in}{6.237869in}}{\pgfqpoint{4.946203in}{6.243456in}}{\pgfqpoint{4.946203in}{6.249279in}}%
\pgfpathcurveto{\pgfqpoint{4.946203in}{6.255103in}}{\pgfqpoint{4.943889in}{6.260690in}}{\pgfqpoint{4.939771in}{6.264808in}}%
\pgfpathcurveto{\pgfqpoint{4.935653in}{6.268926in}}{\pgfqpoint{4.930067in}{6.271240in}}{\pgfqpoint{4.924243in}{6.271240in}}%
\pgfpathcurveto{\pgfqpoint{4.918419in}{6.271240in}}{\pgfqpoint{4.912833in}{6.268926in}}{\pgfqpoint{4.908715in}{6.264808in}}%
\pgfpathcurveto{\pgfqpoint{4.904597in}{6.260690in}}{\pgfqpoint{4.902283in}{6.255103in}}{\pgfqpoint{4.902283in}{6.249279in}}%
\pgfpathcurveto{\pgfqpoint{4.902283in}{6.243456in}}{\pgfqpoint{4.904597in}{6.237869in}}{\pgfqpoint{4.908715in}{6.233751in}}%
\pgfpathcurveto{\pgfqpoint{4.912833in}{6.229633in}}{\pgfqpoint{4.918419in}{6.227319in}}{\pgfqpoint{4.924243in}{6.227319in}}%
\pgfpathlineto{\pgfqpoint{4.924243in}{6.227319in}}%
\pgfpathclose%
\pgfusepath{stroke,fill}%
\end{pgfscope}%
\begin{pgfscope}%
\pgfpathrectangle{\pgfqpoint{1.000000in}{1.148311in}}{\pgfqpoint{6.200000in}{5.623377in}}%
\pgfusepath{clip}%
\pgfsetbuttcap%
\pgfsetroundjoin%
\definecolor{currentfill}{rgb}{0.200000,0.200000,0.800000}%
\pgfsetfillcolor{currentfill}%
\pgfsetlinewidth{1.003750pt}%
\definecolor{currentstroke}{rgb}{0.200000,0.200000,0.800000}%
\pgfsetstrokecolor{currentstroke}%
\pgfsetdash{}{0pt}%
\pgfpathmoveto{\pgfqpoint{4.838604in}{6.236230in}}%
\pgfpathcurveto{\pgfqpoint{4.844428in}{6.236230in}}{\pgfqpoint{4.850014in}{6.238543in}}{\pgfqpoint{4.854132in}{6.242662in}}%
\pgfpathcurveto{\pgfqpoint{4.858250in}{6.246780in}}{\pgfqpoint{4.860564in}{6.252366in}}{\pgfqpoint{4.860564in}{6.258190in}}%
\pgfpathcurveto{\pgfqpoint{4.860564in}{6.264014in}}{\pgfqpoint{4.858250in}{6.269600in}}{\pgfqpoint{4.854132in}{6.273718in}}%
\pgfpathcurveto{\pgfqpoint{4.850014in}{6.277836in}}{\pgfqpoint{4.844428in}{6.280150in}}{\pgfqpoint{4.838604in}{6.280150in}}%
\pgfpathcurveto{\pgfqpoint{4.832780in}{6.280150in}}{\pgfqpoint{4.827194in}{6.277836in}}{\pgfqpoint{4.823076in}{6.273718in}}%
\pgfpathcurveto{\pgfqpoint{4.818957in}{6.269600in}}{\pgfqpoint{4.816644in}{6.264014in}}{\pgfqpoint{4.816644in}{6.258190in}}%
\pgfpathcurveto{\pgfqpoint{4.816644in}{6.252366in}}{\pgfqpoint{4.818957in}{6.246780in}}{\pgfqpoint{4.823076in}{6.242662in}}%
\pgfpathcurveto{\pgfqpoint{4.827194in}{6.238543in}}{\pgfqpoint{4.832780in}{6.236230in}}{\pgfqpoint{4.838604in}{6.236230in}}%
\pgfpathlineto{\pgfqpoint{4.838604in}{6.236230in}}%
\pgfpathclose%
\pgfusepath{stroke,fill}%
\end{pgfscope}%
\begin{pgfscope}%
\pgfpathrectangle{\pgfqpoint{1.000000in}{1.148311in}}{\pgfqpoint{6.200000in}{5.623377in}}%
\pgfusepath{clip}%
\pgfsetbuttcap%
\pgfsetroundjoin%
\definecolor{currentfill}{rgb}{0.200000,0.200000,0.800000}%
\pgfsetfillcolor{currentfill}%
\pgfsetlinewidth{1.003750pt}%
\definecolor{currentstroke}{rgb}{0.200000,0.200000,0.800000}%
\pgfsetstrokecolor{currentstroke}%
\pgfsetdash{}{0pt}%
\pgfpathmoveto{\pgfqpoint{4.778034in}{6.200710in}}%
\pgfpathcurveto{\pgfqpoint{4.783858in}{6.200710in}}{\pgfqpoint{4.789444in}{6.203024in}}{\pgfqpoint{4.793562in}{6.207142in}}%
\pgfpathcurveto{\pgfqpoint{4.797680in}{6.211260in}}{\pgfqpoint{4.799994in}{6.216846in}}{\pgfqpoint{4.799994in}{6.222670in}}%
\pgfpathcurveto{\pgfqpoint{4.799994in}{6.228494in}}{\pgfqpoint{4.797680in}{6.234080in}}{\pgfqpoint{4.793562in}{6.238198in}}%
\pgfpathcurveto{\pgfqpoint{4.789444in}{6.242316in}}{\pgfqpoint{4.783858in}{6.244630in}}{\pgfqpoint{4.778034in}{6.244630in}}%
\pgfpathcurveto{\pgfqpoint{4.772210in}{6.244630in}}{\pgfqpoint{4.766624in}{6.242316in}}{\pgfqpoint{4.762506in}{6.238198in}}%
\pgfpathcurveto{\pgfqpoint{4.758388in}{6.234080in}}{\pgfqpoint{4.756074in}{6.228494in}}{\pgfqpoint{4.756074in}{6.222670in}}%
\pgfpathcurveto{\pgfqpoint{4.756074in}{6.216846in}}{\pgfqpoint{4.758388in}{6.211260in}}{\pgfqpoint{4.762506in}{6.207142in}}%
\pgfpathcurveto{\pgfqpoint{4.766624in}{6.203024in}}{\pgfqpoint{4.772210in}{6.200710in}}{\pgfqpoint{4.778034in}{6.200710in}}%
\pgfpathlineto{\pgfqpoint{4.778034in}{6.200710in}}%
\pgfpathclose%
\pgfusepath{stroke,fill}%
\end{pgfscope}%
\begin{pgfscope}%
\pgfpathrectangle{\pgfqpoint{1.000000in}{1.148311in}}{\pgfqpoint{6.200000in}{5.623377in}}%
\pgfusepath{clip}%
\pgfsetbuttcap%
\pgfsetroundjoin%
\definecolor{currentfill}{rgb}{0.200000,0.200000,0.800000}%
\pgfsetfillcolor{currentfill}%
\pgfsetlinewidth{1.003750pt}%
\definecolor{currentstroke}{rgb}{0.200000,0.200000,0.800000}%
\pgfsetstrokecolor{currentstroke}%
\pgfsetdash{}{0pt}%
\pgfpathmoveto{\pgfqpoint{4.723446in}{6.156656in}}%
\pgfpathcurveto{\pgfqpoint{4.729270in}{6.156656in}}{\pgfqpoint{4.734856in}{6.158970in}}{\pgfqpoint{4.738974in}{6.163088in}}%
\pgfpathcurveto{\pgfqpoint{4.743092in}{6.167206in}}{\pgfqpoint{4.745406in}{6.172792in}}{\pgfqpoint{4.745406in}{6.178616in}}%
\pgfpathcurveto{\pgfqpoint{4.745406in}{6.184440in}}{\pgfqpoint{4.743092in}{6.190026in}}{\pgfqpoint{4.738974in}{6.194145in}}%
\pgfpathcurveto{\pgfqpoint{4.734856in}{6.198263in}}{\pgfqpoint{4.729270in}{6.200577in}}{\pgfqpoint{4.723446in}{6.200577in}}%
\pgfpathcurveto{\pgfqpoint{4.717622in}{6.200577in}}{\pgfqpoint{4.712036in}{6.198263in}}{\pgfqpoint{4.707918in}{6.194145in}}%
\pgfpathcurveto{\pgfqpoint{4.703800in}{6.190026in}}{\pgfqpoint{4.701486in}{6.184440in}}{\pgfqpoint{4.701486in}{6.178616in}}%
\pgfpathcurveto{\pgfqpoint{4.701486in}{6.172792in}}{\pgfqpoint{4.703800in}{6.167206in}}{\pgfqpoint{4.707918in}{6.163088in}}%
\pgfpathcurveto{\pgfqpoint{4.712036in}{6.158970in}}{\pgfqpoint{4.717622in}{6.156656in}}{\pgfqpoint{4.723446in}{6.156656in}}%
\pgfpathlineto{\pgfqpoint{4.723446in}{6.156656in}}%
\pgfpathclose%
\pgfusepath{stroke,fill}%
\end{pgfscope}%
\begin{pgfscope}%
\pgfpathrectangle{\pgfqpoint{1.000000in}{1.148311in}}{\pgfqpoint{6.200000in}{5.623377in}}%
\pgfusepath{clip}%
\pgfsetbuttcap%
\pgfsetroundjoin%
\definecolor{currentfill}{rgb}{0.200000,0.200000,0.800000}%
\pgfsetfillcolor{currentfill}%
\pgfsetlinewidth{1.003750pt}%
\definecolor{currentstroke}{rgb}{0.200000,0.200000,0.800000}%
\pgfsetstrokecolor{currentstroke}%
\pgfsetdash{}{0pt}%
\pgfpathmoveto{\pgfqpoint{4.639781in}{6.141701in}}%
\pgfpathcurveto{\pgfqpoint{4.645605in}{6.141701in}}{\pgfqpoint{4.651191in}{6.144015in}}{\pgfqpoint{4.655309in}{6.148133in}}%
\pgfpathcurveto{\pgfqpoint{4.659427in}{6.152251in}}{\pgfqpoint{4.661741in}{6.157837in}}{\pgfqpoint{4.661741in}{6.163661in}}%
\pgfpathcurveto{\pgfqpoint{4.661741in}{6.169485in}}{\pgfqpoint{4.659427in}{6.175071in}}{\pgfqpoint{4.655309in}{6.179189in}}%
\pgfpathcurveto{\pgfqpoint{4.651191in}{6.183308in}}{\pgfqpoint{4.645605in}{6.185621in}}{\pgfqpoint{4.639781in}{6.185621in}}%
\pgfpathcurveto{\pgfqpoint{4.633957in}{6.185621in}}{\pgfqpoint{4.628371in}{6.183308in}}{\pgfqpoint{4.624253in}{6.179189in}}%
\pgfpathcurveto{\pgfqpoint{4.620134in}{6.175071in}}{\pgfqpoint{4.617821in}{6.169485in}}{\pgfqpoint{4.617821in}{6.163661in}}%
\pgfpathcurveto{\pgfqpoint{4.617821in}{6.157837in}}{\pgfqpoint{4.620134in}{6.152251in}}{\pgfqpoint{4.624253in}{6.148133in}}%
\pgfpathcurveto{\pgfqpoint{4.628371in}{6.144015in}}{\pgfqpoint{4.633957in}{6.141701in}}{\pgfqpoint{4.639781in}{6.141701in}}%
\pgfpathlineto{\pgfqpoint{4.639781in}{6.141701in}}%
\pgfpathclose%
\pgfusepath{stroke,fill}%
\end{pgfscope}%
\begin{pgfscope}%
\pgfpathrectangle{\pgfqpoint{1.000000in}{1.148311in}}{\pgfqpoint{6.200000in}{5.623377in}}%
\pgfusepath{clip}%
\pgfsetbuttcap%
\pgfsetroundjoin%
\definecolor{currentfill}{rgb}{0.200000,0.200000,0.800000}%
\pgfsetfillcolor{currentfill}%
\pgfsetlinewidth{1.003750pt}%
\definecolor{currentstroke}{rgb}{0.200000,0.200000,0.800000}%
\pgfsetstrokecolor{currentstroke}%
\pgfsetdash{}{0pt}%
\pgfpathmoveto{\pgfqpoint{4.631333in}{6.051276in}}%
\pgfpathcurveto{\pgfqpoint{4.637157in}{6.051276in}}{\pgfqpoint{4.642743in}{6.053589in}}{\pgfqpoint{4.646861in}{6.057708in}}%
\pgfpathcurveto{\pgfqpoint{4.650979in}{6.061826in}}{\pgfqpoint{4.653293in}{6.067412in}}{\pgfqpoint{4.653293in}{6.073236in}}%
\pgfpathcurveto{\pgfqpoint{4.653293in}{6.079060in}}{\pgfqpoint{4.650979in}{6.084646in}}{\pgfqpoint{4.646861in}{6.088764in}}%
\pgfpathcurveto{\pgfqpoint{4.642743in}{6.092882in}}{\pgfqpoint{4.637157in}{6.095196in}}{\pgfqpoint{4.631333in}{6.095196in}}%
\pgfpathcurveto{\pgfqpoint{4.625509in}{6.095196in}}{\pgfqpoint{4.619923in}{6.092882in}}{\pgfqpoint{4.615805in}{6.088764in}}%
\pgfpathcurveto{\pgfqpoint{4.611687in}{6.084646in}}{\pgfqpoint{4.609373in}{6.079060in}}{\pgfqpoint{4.609373in}{6.073236in}}%
\pgfpathcurveto{\pgfqpoint{4.609373in}{6.067412in}}{\pgfqpoint{4.611687in}{6.061826in}}{\pgfqpoint{4.615805in}{6.057708in}}%
\pgfpathcurveto{\pgfqpoint{4.619923in}{6.053589in}}{\pgfqpoint{4.625509in}{6.051276in}}{\pgfqpoint{4.631333in}{6.051276in}}%
\pgfpathlineto{\pgfqpoint{4.631333in}{6.051276in}}%
\pgfpathclose%
\pgfusepath{stroke,fill}%
\end{pgfscope}%
\begin{pgfscope}%
\pgfpathrectangle{\pgfqpoint{1.000000in}{1.148311in}}{\pgfqpoint{6.200000in}{5.623377in}}%
\pgfusepath{clip}%
\pgfsetbuttcap%
\pgfsetroundjoin%
\definecolor{currentfill}{rgb}{0.200000,0.200000,0.800000}%
\pgfsetfillcolor{currentfill}%
\pgfsetlinewidth{1.003750pt}%
\definecolor{currentstroke}{rgb}{0.200000,0.200000,0.800000}%
\pgfsetstrokecolor{currentstroke}%
\pgfsetdash{}{0pt}%
\pgfpathmoveto{\pgfqpoint{4.524824in}{6.046680in}}%
\pgfpathcurveto{\pgfqpoint{4.530648in}{6.046680in}}{\pgfqpoint{4.536234in}{6.048994in}}{\pgfqpoint{4.540352in}{6.053112in}}%
\pgfpathcurveto{\pgfqpoint{4.544470in}{6.057230in}}{\pgfqpoint{4.546784in}{6.062817in}}{\pgfqpoint{4.546784in}{6.068641in}}%
\pgfpathcurveto{\pgfqpoint{4.546784in}{6.074464in}}{\pgfqpoint{4.544470in}{6.080051in}}{\pgfqpoint{4.540352in}{6.084169in}}%
\pgfpathcurveto{\pgfqpoint{4.536234in}{6.088287in}}{\pgfqpoint{4.530648in}{6.090601in}}{\pgfqpoint{4.524824in}{6.090601in}}%
\pgfpathcurveto{\pgfqpoint{4.519000in}{6.090601in}}{\pgfqpoint{4.513414in}{6.088287in}}{\pgfqpoint{4.509296in}{6.084169in}}%
\pgfpathcurveto{\pgfqpoint{4.505177in}{6.080051in}}{\pgfqpoint{4.502864in}{6.074464in}}{\pgfqpoint{4.502864in}{6.068641in}}%
\pgfpathcurveto{\pgfqpoint{4.502864in}{6.062817in}}{\pgfqpoint{4.505177in}{6.057230in}}{\pgfqpoint{4.509296in}{6.053112in}}%
\pgfpathcurveto{\pgfqpoint{4.513414in}{6.048994in}}{\pgfqpoint{4.519000in}{6.046680in}}{\pgfqpoint{4.524824in}{6.046680in}}%
\pgfpathlineto{\pgfqpoint{4.524824in}{6.046680in}}%
\pgfpathclose%
\pgfusepath{stroke,fill}%
\end{pgfscope}%
\begin{pgfscope}%
\pgfpathrectangle{\pgfqpoint{1.000000in}{1.148311in}}{\pgfqpoint{6.200000in}{5.623377in}}%
\pgfusepath{clip}%
\pgfsetbuttcap%
\pgfsetroundjoin%
\definecolor{currentfill}{rgb}{0.200000,0.200000,0.800000}%
\pgfsetfillcolor{currentfill}%
\pgfsetlinewidth{1.003750pt}%
\definecolor{currentstroke}{rgb}{0.200000,0.200000,0.800000}%
\pgfsetstrokecolor{currentstroke}%
\pgfsetdash{}{0pt}%
\pgfpathmoveto{\pgfqpoint{4.520405in}{5.959345in}}%
\pgfpathcurveto{\pgfqpoint{4.526229in}{5.959345in}}{\pgfqpoint{4.531815in}{5.961659in}}{\pgfqpoint{4.535933in}{5.965777in}}%
\pgfpathcurveto{\pgfqpoint{4.540051in}{5.969895in}}{\pgfqpoint{4.542365in}{5.975482in}}{\pgfqpoint{4.542365in}{5.981306in}}%
\pgfpathcurveto{\pgfqpoint{4.542365in}{5.987129in}}{\pgfqpoint{4.540051in}{5.992716in}}{\pgfqpoint{4.535933in}{5.996834in}}%
\pgfpathcurveto{\pgfqpoint{4.531815in}{6.000952in}}{\pgfqpoint{4.526229in}{6.003266in}}{\pgfqpoint{4.520405in}{6.003266in}}%
\pgfpathcurveto{\pgfqpoint{4.514581in}{6.003266in}}{\pgfqpoint{4.508995in}{6.000952in}}{\pgfqpoint{4.504876in}{5.996834in}}%
\pgfpathcurveto{\pgfqpoint{4.500758in}{5.992716in}}{\pgfqpoint{4.498444in}{5.987129in}}{\pgfqpoint{4.498444in}{5.981306in}}%
\pgfpathcurveto{\pgfqpoint{4.498444in}{5.975482in}}{\pgfqpoint{4.500758in}{5.969895in}}{\pgfqpoint{4.504876in}{5.965777in}}%
\pgfpathcurveto{\pgfqpoint{4.508995in}{5.961659in}}{\pgfqpoint{4.514581in}{5.959345in}}{\pgfqpoint{4.520405in}{5.959345in}}%
\pgfpathlineto{\pgfqpoint{4.520405in}{5.959345in}}%
\pgfpathclose%
\pgfusepath{stroke,fill}%
\end{pgfscope}%
\begin{pgfscope}%
\pgfpathrectangle{\pgfqpoint{1.000000in}{1.148311in}}{\pgfqpoint{6.200000in}{5.623377in}}%
\pgfusepath{clip}%
\pgfsetbuttcap%
\pgfsetroundjoin%
\definecolor{currentfill}{rgb}{0.200000,0.200000,0.800000}%
\pgfsetfillcolor{currentfill}%
\pgfsetlinewidth{1.003750pt}%
\definecolor{currentstroke}{rgb}{0.200000,0.200000,0.800000}%
\pgfsetstrokecolor{currentstroke}%
\pgfsetdash{}{0pt}%
\pgfpathmoveto{\pgfqpoint{4.480745in}{5.899892in}}%
\pgfpathcurveto{\pgfqpoint{4.486569in}{5.899892in}}{\pgfqpoint{4.492155in}{5.902206in}}{\pgfqpoint{4.496273in}{5.906324in}}%
\pgfpathcurveto{\pgfqpoint{4.500392in}{5.910443in}}{\pgfqpoint{4.502705in}{5.916029in}}{\pgfqpoint{4.502705in}{5.921853in}}%
\pgfpathcurveto{\pgfqpoint{4.502705in}{5.927677in}}{\pgfqpoint{4.500392in}{5.933263in}}{\pgfqpoint{4.496273in}{5.937381in}}%
\pgfpathcurveto{\pgfqpoint{4.492155in}{5.941499in}}{\pgfqpoint{4.486569in}{5.943813in}}{\pgfqpoint{4.480745in}{5.943813in}}%
\pgfpathcurveto{\pgfqpoint{4.474921in}{5.943813in}}{\pgfqpoint{4.469335in}{5.941499in}}{\pgfqpoint{4.465217in}{5.937381in}}%
\pgfpathcurveto{\pgfqpoint{4.461099in}{5.933263in}}{\pgfqpoint{4.458785in}{5.927677in}}{\pgfqpoint{4.458785in}{5.921853in}}%
\pgfpathcurveto{\pgfqpoint{4.458785in}{5.916029in}}{\pgfqpoint{4.461099in}{5.910443in}}{\pgfqpoint{4.465217in}{5.906324in}}%
\pgfpathcurveto{\pgfqpoint{4.469335in}{5.902206in}}{\pgfqpoint{4.474921in}{5.899892in}}{\pgfqpoint{4.480745in}{5.899892in}}%
\pgfpathlineto{\pgfqpoint{4.480745in}{5.899892in}}%
\pgfpathclose%
\pgfusepath{stroke,fill}%
\end{pgfscope}%
\begin{pgfscope}%
\pgfpathrectangle{\pgfqpoint{1.000000in}{1.148311in}}{\pgfqpoint{6.200000in}{5.623377in}}%
\pgfusepath{clip}%
\pgfsetbuttcap%
\pgfsetroundjoin%
\definecolor{currentfill}{rgb}{0.200000,0.200000,0.800000}%
\pgfsetfillcolor{currentfill}%
\pgfsetlinewidth{1.003750pt}%
\definecolor{currentstroke}{rgb}{0.200000,0.200000,0.800000}%
\pgfsetstrokecolor{currentstroke}%
\pgfsetdash{}{0pt}%
\pgfpathmoveto{\pgfqpoint{4.474934in}{5.822492in}}%
\pgfpathcurveto{\pgfqpoint{4.480758in}{5.822492in}}{\pgfqpoint{4.486344in}{5.824806in}}{\pgfqpoint{4.490462in}{5.828924in}}%
\pgfpathcurveto{\pgfqpoint{4.494581in}{5.833042in}}{\pgfqpoint{4.496894in}{5.838628in}}{\pgfqpoint{4.496894in}{5.844452in}}%
\pgfpathcurveto{\pgfqpoint{4.496894in}{5.850276in}}{\pgfqpoint{4.494581in}{5.855862in}}{\pgfqpoint{4.490462in}{5.859980in}}%
\pgfpathcurveto{\pgfqpoint{4.486344in}{5.864098in}}{\pgfqpoint{4.480758in}{5.866412in}}{\pgfqpoint{4.474934in}{5.866412in}}%
\pgfpathcurveto{\pgfqpoint{4.469110in}{5.866412in}}{\pgfqpoint{4.463524in}{5.864098in}}{\pgfqpoint{4.459406in}{5.859980in}}%
\pgfpathcurveto{\pgfqpoint{4.455288in}{5.855862in}}{\pgfqpoint{4.452974in}{5.850276in}}{\pgfqpoint{4.452974in}{5.844452in}}%
\pgfpathcurveto{\pgfqpoint{4.452974in}{5.838628in}}{\pgfqpoint{4.455288in}{5.833042in}}{\pgfqpoint{4.459406in}{5.828924in}}%
\pgfpathcurveto{\pgfqpoint{4.463524in}{5.824806in}}{\pgfqpoint{4.469110in}{5.822492in}}{\pgfqpoint{4.474934in}{5.822492in}}%
\pgfpathlineto{\pgfqpoint{4.474934in}{5.822492in}}%
\pgfpathclose%
\pgfusepath{stroke,fill}%
\end{pgfscope}%
\begin{pgfscope}%
\pgfpathrectangle{\pgfqpoint{1.000000in}{1.148311in}}{\pgfqpoint{6.200000in}{5.623377in}}%
\pgfusepath{clip}%
\pgfsetbuttcap%
\pgfsetroundjoin%
\definecolor{currentfill}{rgb}{0.200000,0.200000,0.800000}%
\pgfsetfillcolor{currentfill}%
\pgfsetlinewidth{1.003750pt}%
\definecolor{currentstroke}{rgb}{0.200000,0.200000,0.800000}%
\pgfsetstrokecolor{currentstroke}%
\pgfsetdash{}{0pt}%
\pgfpathmoveto{\pgfqpoint{4.459565in}{5.753470in}}%
\pgfpathcurveto{\pgfqpoint{4.465389in}{5.753470in}}{\pgfqpoint{4.470975in}{5.755784in}}{\pgfqpoint{4.475093in}{5.759902in}}%
\pgfpathcurveto{\pgfqpoint{4.479211in}{5.764020in}}{\pgfqpoint{4.481525in}{5.769606in}}{\pgfqpoint{4.481525in}{5.775430in}}%
\pgfpathcurveto{\pgfqpoint{4.481525in}{5.781254in}}{\pgfqpoint{4.479211in}{5.786840in}}{\pgfqpoint{4.475093in}{5.790959in}}%
\pgfpathcurveto{\pgfqpoint{4.470975in}{5.795077in}}{\pgfqpoint{4.465389in}{5.797391in}}{\pgfqpoint{4.459565in}{5.797391in}}%
\pgfpathcurveto{\pgfqpoint{4.453741in}{5.797391in}}{\pgfqpoint{4.448155in}{5.795077in}}{\pgfqpoint{4.444037in}{5.790959in}}%
\pgfpathcurveto{\pgfqpoint{4.439919in}{5.786840in}}{\pgfqpoint{4.437605in}{5.781254in}}{\pgfqpoint{4.437605in}{5.775430in}}%
\pgfpathcurveto{\pgfqpoint{4.437605in}{5.769606in}}{\pgfqpoint{4.439919in}{5.764020in}}{\pgfqpoint{4.444037in}{5.759902in}}%
\pgfpathcurveto{\pgfqpoint{4.448155in}{5.755784in}}{\pgfqpoint{4.453741in}{5.753470in}}{\pgfqpoint{4.459565in}{5.753470in}}%
\pgfpathlineto{\pgfqpoint{4.459565in}{5.753470in}}%
\pgfpathclose%
\pgfusepath{stroke,fill}%
\end{pgfscope}%
\begin{pgfscope}%
\pgfpathrectangle{\pgfqpoint{1.000000in}{1.148311in}}{\pgfqpoint{6.200000in}{5.623377in}}%
\pgfusepath{clip}%
\pgfsetbuttcap%
\pgfsetroundjoin%
\definecolor{currentfill}{rgb}{0.200000,0.200000,0.800000}%
\pgfsetfillcolor{currentfill}%
\pgfsetlinewidth{1.003750pt}%
\definecolor{currentstroke}{rgb}{0.200000,0.200000,0.800000}%
\pgfsetstrokecolor{currentstroke}%
\pgfsetdash{}{0pt}%
\pgfpathmoveto{\pgfqpoint{4.445248in}{5.685900in}}%
\pgfpathcurveto{\pgfqpoint{4.451072in}{5.685900in}}{\pgfqpoint{4.456658in}{5.688214in}}{\pgfqpoint{4.460776in}{5.692332in}}%
\pgfpathcurveto{\pgfqpoint{4.464894in}{5.696450in}}{\pgfqpoint{4.467208in}{5.702036in}}{\pgfqpoint{4.467208in}{5.707860in}}%
\pgfpathcurveto{\pgfqpoint{4.467208in}{5.713684in}}{\pgfqpoint{4.464894in}{5.719271in}}{\pgfqpoint{4.460776in}{5.723389in}}%
\pgfpathcurveto{\pgfqpoint{4.456658in}{5.727507in}}{\pgfqpoint{4.451072in}{5.729821in}}{\pgfqpoint{4.445248in}{5.729821in}}%
\pgfpathcurveto{\pgfqpoint{4.439424in}{5.729821in}}{\pgfqpoint{4.433838in}{5.727507in}}{\pgfqpoint{4.429720in}{5.723389in}}%
\pgfpathcurveto{\pgfqpoint{4.425601in}{5.719271in}}{\pgfqpoint{4.423288in}{5.713684in}}{\pgfqpoint{4.423288in}{5.707860in}}%
\pgfpathcurveto{\pgfqpoint{4.423288in}{5.702036in}}{\pgfqpoint{4.425601in}{5.696450in}}{\pgfqpoint{4.429720in}{5.692332in}}%
\pgfpathcurveto{\pgfqpoint{4.433838in}{5.688214in}}{\pgfqpoint{4.439424in}{5.685900in}}{\pgfqpoint{4.445248in}{5.685900in}}%
\pgfpathlineto{\pgfqpoint{4.445248in}{5.685900in}}%
\pgfpathclose%
\pgfusepath{stroke,fill}%
\end{pgfscope}%
\begin{pgfscope}%
\pgfpathrectangle{\pgfqpoint{1.000000in}{1.148311in}}{\pgfqpoint{6.200000in}{5.623377in}}%
\pgfusepath{clip}%
\pgfsetbuttcap%
\pgfsetroundjoin%
\definecolor{currentfill}{rgb}{0.200000,0.200000,0.800000}%
\pgfsetfillcolor{currentfill}%
\pgfsetlinewidth{1.003750pt}%
\definecolor{currentstroke}{rgb}{0.200000,0.200000,0.800000}%
\pgfsetstrokecolor{currentstroke}%
\pgfsetdash{}{0pt}%
\pgfpathmoveto{\pgfqpoint{4.393997in}{5.630452in}}%
\pgfpathcurveto{\pgfqpoint{4.399821in}{5.630452in}}{\pgfqpoint{4.405407in}{5.632766in}}{\pgfqpoint{4.409525in}{5.636884in}}%
\pgfpathcurveto{\pgfqpoint{4.413643in}{5.641002in}}{\pgfqpoint{4.415957in}{5.646588in}}{\pgfqpoint{4.415957in}{5.652412in}}%
\pgfpathcurveto{\pgfqpoint{4.415957in}{5.658236in}}{\pgfqpoint{4.413643in}{5.663822in}}{\pgfqpoint{4.409525in}{5.667940in}}%
\pgfpathcurveto{\pgfqpoint{4.405407in}{5.672058in}}{\pgfqpoint{4.399821in}{5.674372in}}{\pgfqpoint{4.393997in}{5.674372in}}%
\pgfpathcurveto{\pgfqpoint{4.388173in}{5.674372in}}{\pgfqpoint{4.382587in}{5.672058in}}{\pgfqpoint{4.378468in}{5.667940in}}%
\pgfpathcurveto{\pgfqpoint{4.374350in}{5.663822in}}{\pgfqpoint{4.372036in}{5.658236in}}{\pgfqpoint{4.372036in}{5.652412in}}%
\pgfpathcurveto{\pgfqpoint{4.372036in}{5.646588in}}{\pgfqpoint{4.374350in}{5.641002in}}{\pgfqpoint{4.378468in}{5.636884in}}%
\pgfpathcurveto{\pgfqpoint{4.382587in}{5.632766in}}{\pgfqpoint{4.388173in}{5.630452in}}{\pgfqpoint{4.393997in}{5.630452in}}%
\pgfpathlineto{\pgfqpoint{4.393997in}{5.630452in}}%
\pgfpathclose%
\pgfusepath{stroke,fill}%
\end{pgfscope}%
\begin{pgfscope}%
\pgfpathrectangle{\pgfqpoint{1.000000in}{1.148311in}}{\pgfqpoint{6.200000in}{5.623377in}}%
\pgfusepath{clip}%
\pgfsetbuttcap%
\pgfsetroundjoin%
\definecolor{currentfill}{rgb}{0.200000,0.200000,0.800000}%
\pgfsetfillcolor{currentfill}%
\pgfsetlinewidth{1.003750pt}%
\definecolor{currentstroke}{rgb}{0.200000,0.200000,0.800000}%
\pgfsetstrokecolor{currentstroke}%
\pgfsetdash{}{0pt}%
\pgfpathmoveto{\pgfqpoint{4.349748in}{5.569058in}}%
\pgfpathcurveto{\pgfqpoint{4.355572in}{5.569058in}}{\pgfqpoint{4.361158in}{5.571372in}}{\pgfqpoint{4.365276in}{5.575490in}}%
\pgfpathcurveto{\pgfqpoint{4.369394in}{5.579609in}}{\pgfqpoint{4.371708in}{5.585195in}}{\pgfqpoint{4.371708in}{5.591019in}}%
\pgfpathcurveto{\pgfqpoint{4.371708in}{5.596843in}}{\pgfqpoint{4.369394in}{5.602429in}}{\pgfqpoint{4.365276in}{5.606547in}}%
\pgfpathcurveto{\pgfqpoint{4.361158in}{5.610665in}}{\pgfqpoint{4.355572in}{5.612979in}}{\pgfqpoint{4.349748in}{5.612979in}}%
\pgfpathcurveto{\pgfqpoint{4.343924in}{5.612979in}}{\pgfqpoint{4.338338in}{5.610665in}}{\pgfqpoint{4.334219in}{5.606547in}}%
\pgfpathcurveto{\pgfqpoint{4.330101in}{5.602429in}}{\pgfqpoint{4.327787in}{5.596843in}}{\pgfqpoint{4.327787in}{5.591019in}}%
\pgfpathcurveto{\pgfqpoint{4.327787in}{5.585195in}}{\pgfqpoint{4.330101in}{5.579609in}}{\pgfqpoint{4.334219in}{5.575490in}}%
\pgfpathcurveto{\pgfqpoint{4.338338in}{5.571372in}}{\pgfqpoint{4.343924in}{5.569058in}}{\pgfqpoint{4.349748in}{5.569058in}}%
\pgfpathlineto{\pgfqpoint{4.349748in}{5.569058in}}%
\pgfpathclose%
\pgfusepath{stroke,fill}%
\end{pgfscope}%
\begin{pgfscope}%
\pgfpathrectangle{\pgfqpoint{1.000000in}{1.148311in}}{\pgfqpoint{6.200000in}{5.623377in}}%
\pgfusepath{clip}%
\pgfsetbuttcap%
\pgfsetroundjoin%
\definecolor{currentfill}{rgb}{0.200000,0.200000,0.800000}%
\pgfsetfillcolor{currentfill}%
\pgfsetlinewidth{1.003750pt}%
\definecolor{currentstroke}{rgb}{0.200000,0.200000,0.800000}%
\pgfsetstrokecolor{currentstroke}%
\pgfsetdash{}{0pt}%
\pgfpathmoveto{\pgfqpoint{4.378166in}{5.492230in}}%
\pgfpathcurveto{\pgfqpoint{4.383990in}{5.492230in}}{\pgfqpoint{4.389576in}{5.494544in}}{\pgfqpoint{4.393694in}{5.498662in}}%
\pgfpathcurveto{\pgfqpoint{4.397812in}{5.502781in}}{\pgfqpoint{4.400126in}{5.508367in}}{\pgfqpoint{4.400126in}{5.514191in}}%
\pgfpathcurveto{\pgfqpoint{4.400126in}{5.520015in}}{\pgfqpoint{4.397812in}{5.525601in}}{\pgfqpoint{4.393694in}{5.529719in}}%
\pgfpathcurveto{\pgfqpoint{4.389576in}{5.533837in}}{\pgfqpoint{4.383990in}{5.536151in}}{\pgfqpoint{4.378166in}{5.536151in}}%
\pgfpathcurveto{\pgfqpoint{4.372342in}{5.536151in}}{\pgfqpoint{4.366756in}{5.533837in}}{\pgfqpoint{4.362638in}{5.529719in}}%
\pgfpathcurveto{\pgfqpoint{4.358520in}{5.525601in}}{\pgfqpoint{4.356206in}{5.520015in}}{\pgfqpoint{4.356206in}{5.514191in}}%
\pgfpathcurveto{\pgfqpoint{4.356206in}{5.508367in}}{\pgfqpoint{4.358520in}{5.502781in}}{\pgfqpoint{4.362638in}{5.498662in}}%
\pgfpathcurveto{\pgfqpoint{4.366756in}{5.494544in}}{\pgfqpoint{4.372342in}{5.492230in}}{\pgfqpoint{4.378166in}{5.492230in}}%
\pgfpathlineto{\pgfqpoint{4.378166in}{5.492230in}}%
\pgfpathclose%
\pgfusepath{stroke,fill}%
\end{pgfscope}%
\begin{pgfscope}%
\pgfpathrectangle{\pgfqpoint{1.000000in}{1.148311in}}{\pgfqpoint{6.200000in}{5.623377in}}%
\pgfusepath{clip}%
\pgfsetbuttcap%
\pgfsetroundjoin%
\definecolor{currentfill}{rgb}{0.200000,0.200000,0.800000}%
\pgfsetfillcolor{currentfill}%
\pgfsetlinewidth{1.003750pt}%
\definecolor{currentstroke}{rgb}{0.200000,0.200000,0.800000}%
\pgfsetstrokecolor{currentstroke}%
\pgfsetdash{}{0pt}%
\pgfpathmoveto{\pgfqpoint{4.376314in}{5.423415in}}%
\pgfpathcurveto{\pgfqpoint{4.382138in}{5.423415in}}{\pgfqpoint{4.387724in}{5.425729in}}{\pgfqpoint{4.391842in}{5.429847in}}%
\pgfpathcurveto{\pgfqpoint{4.395960in}{5.433966in}}{\pgfqpoint{4.398274in}{5.439552in}}{\pgfqpoint{4.398274in}{5.445376in}}%
\pgfpathcurveto{\pgfqpoint{4.398274in}{5.451200in}}{\pgfqpoint{4.395960in}{5.456786in}}{\pgfqpoint{4.391842in}{5.460904in}}%
\pgfpathcurveto{\pgfqpoint{4.387724in}{5.465022in}}{\pgfqpoint{4.382138in}{5.467336in}}{\pgfqpoint{4.376314in}{5.467336in}}%
\pgfpathcurveto{\pgfqpoint{4.370490in}{5.467336in}}{\pgfqpoint{4.364904in}{5.465022in}}{\pgfqpoint{4.360786in}{5.460904in}}%
\pgfpathcurveto{\pgfqpoint{4.356668in}{5.456786in}}{\pgfqpoint{4.354354in}{5.451200in}}{\pgfqpoint{4.354354in}{5.445376in}}%
\pgfpathcurveto{\pgfqpoint{4.354354in}{5.439552in}}{\pgfqpoint{4.356668in}{5.433966in}}{\pgfqpoint{4.360786in}{5.429847in}}%
\pgfpathcurveto{\pgfqpoint{4.364904in}{5.425729in}}{\pgfqpoint{4.370490in}{5.423415in}}{\pgfqpoint{4.376314in}{5.423415in}}%
\pgfpathlineto{\pgfqpoint{4.376314in}{5.423415in}}%
\pgfpathclose%
\pgfusepath{stroke,fill}%
\end{pgfscope}%
\begin{pgfscope}%
\pgfpathrectangle{\pgfqpoint{1.000000in}{1.148311in}}{\pgfqpoint{6.200000in}{5.623377in}}%
\pgfusepath{clip}%
\pgfsetbuttcap%
\pgfsetroundjoin%
\definecolor{currentfill}{rgb}{0.200000,0.200000,0.800000}%
\pgfsetfillcolor{currentfill}%
\pgfsetlinewidth{1.003750pt}%
\definecolor{currentstroke}{rgb}{0.200000,0.200000,0.800000}%
\pgfsetstrokecolor{currentstroke}%
\pgfsetdash{}{0pt}%
\pgfpathmoveto{\pgfqpoint{4.398112in}{5.354446in}}%
\pgfpathcurveto{\pgfqpoint{4.403936in}{5.354446in}}{\pgfqpoint{4.409522in}{5.356760in}}{\pgfqpoint{4.413640in}{5.360878in}}%
\pgfpathcurveto{\pgfqpoint{4.417759in}{5.364996in}}{\pgfqpoint{4.420072in}{5.370582in}}{\pgfqpoint{4.420072in}{5.376406in}}%
\pgfpathcurveto{\pgfqpoint{4.420072in}{5.382230in}}{\pgfqpoint{4.417759in}{5.387816in}}{\pgfqpoint{4.413640in}{5.391934in}}%
\pgfpathcurveto{\pgfqpoint{4.409522in}{5.396053in}}{\pgfqpoint{4.403936in}{5.398366in}}{\pgfqpoint{4.398112in}{5.398366in}}%
\pgfpathcurveto{\pgfqpoint{4.392288in}{5.398366in}}{\pgfqpoint{4.386702in}{5.396053in}}{\pgfqpoint{4.382584in}{5.391934in}}%
\pgfpathcurveto{\pgfqpoint{4.378466in}{5.387816in}}{\pgfqpoint{4.376152in}{5.382230in}}{\pgfqpoint{4.376152in}{5.376406in}}%
\pgfpathcurveto{\pgfqpoint{4.376152in}{5.370582in}}{\pgfqpoint{4.378466in}{5.364996in}}{\pgfqpoint{4.382584in}{5.360878in}}%
\pgfpathcurveto{\pgfqpoint{4.386702in}{5.356760in}}{\pgfqpoint{4.392288in}{5.354446in}}{\pgfqpoint{4.398112in}{5.354446in}}%
\pgfpathlineto{\pgfqpoint{4.398112in}{5.354446in}}%
\pgfpathclose%
\pgfusepath{stroke,fill}%
\end{pgfscope}%
\begin{pgfscope}%
\pgfpathrectangle{\pgfqpoint{1.000000in}{1.148311in}}{\pgfqpoint{6.200000in}{5.623377in}}%
\pgfusepath{clip}%
\pgfsetbuttcap%
\pgfsetroundjoin%
\definecolor{currentfill}{rgb}{0.200000,0.200000,0.800000}%
\pgfsetfillcolor{currentfill}%
\pgfsetlinewidth{1.003750pt}%
\definecolor{currentstroke}{rgb}{0.200000,0.200000,0.800000}%
\pgfsetstrokecolor{currentstroke}%
\pgfsetdash{}{0pt}%
\pgfpathmoveto{\pgfqpoint{4.337064in}{5.285889in}}%
\pgfpathcurveto{\pgfqpoint{4.342888in}{5.285889in}}{\pgfqpoint{4.348474in}{5.288203in}}{\pgfqpoint{4.352593in}{5.292321in}}%
\pgfpathcurveto{\pgfqpoint{4.356711in}{5.296439in}}{\pgfqpoint{4.359025in}{5.302026in}}{\pgfqpoint{4.359025in}{5.307849in}}%
\pgfpathcurveto{\pgfqpoint{4.359025in}{5.313673in}}{\pgfqpoint{4.356711in}{5.319260in}}{\pgfqpoint{4.352593in}{5.323378in}}%
\pgfpathcurveto{\pgfqpoint{4.348474in}{5.327496in}}{\pgfqpoint{4.342888in}{5.329810in}}{\pgfqpoint{4.337064in}{5.329810in}}%
\pgfpathcurveto{\pgfqpoint{4.331240in}{5.329810in}}{\pgfqpoint{4.325654in}{5.327496in}}{\pgfqpoint{4.321536in}{5.323378in}}%
\pgfpathcurveto{\pgfqpoint{4.317418in}{5.319260in}}{\pgfqpoint{4.315104in}{5.313673in}}{\pgfqpoint{4.315104in}{5.307849in}}%
\pgfpathcurveto{\pgfqpoint{4.315104in}{5.302026in}}{\pgfqpoint{4.317418in}{5.296439in}}{\pgfqpoint{4.321536in}{5.292321in}}%
\pgfpathcurveto{\pgfqpoint{4.325654in}{5.288203in}}{\pgfqpoint{4.331240in}{5.285889in}}{\pgfqpoint{4.337064in}{5.285889in}}%
\pgfpathlineto{\pgfqpoint{4.337064in}{5.285889in}}%
\pgfpathclose%
\pgfusepath{stroke,fill}%
\end{pgfscope}%
\begin{pgfscope}%
\pgfpathrectangle{\pgfqpoint{1.000000in}{1.148311in}}{\pgfqpoint{6.200000in}{5.623377in}}%
\pgfusepath{clip}%
\pgfsetbuttcap%
\pgfsetroundjoin%
\definecolor{currentfill}{rgb}{0.200000,0.200000,0.800000}%
\pgfsetfillcolor{currentfill}%
\pgfsetlinewidth{1.003750pt}%
\definecolor{currentstroke}{rgb}{0.200000,0.200000,0.800000}%
\pgfsetstrokecolor{currentstroke}%
\pgfsetdash{}{0pt}%
\pgfpathmoveto{\pgfqpoint{4.403808in}{5.221483in}}%
\pgfpathcurveto{\pgfqpoint{4.409632in}{5.221483in}}{\pgfqpoint{4.415218in}{5.223797in}}{\pgfqpoint{4.419336in}{5.227915in}}%
\pgfpathcurveto{\pgfqpoint{4.423454in}{5.232033in}}{\pgfqpoint{4.425768in}{5.237619in}}{\pgfqpoint{4.425768in}{5.243443in}}%
\pgfpathcurveto{\pgfqpoint{4.425768in}{5.249267in}}{\pgfqpoint{4.423454in}{5.254853in}}{\pgfqpoint{4.419336in}{5.258971in}}%
\pgfpathcurveto{\pgfqpoint{4.415218in}{5.263090in}}{\pgfqpoint{4.409632in}{5.265403in}}{\pgfqpoint{4.403808in}{5.265403in}}%
\pgfpathcurveto{\pgfqpoint{4.397984in}{5.265403in}}{\pgfqpoint{4.392398in}{5.263090in}}{\pgfqpoint{4.388280in}{5.258971in}}%
\pgfpathcurveto{\pgfqpoint{4.384161in}{5.254853in}}{\pgfqpoint{4.381848in}{5.249267in}}{\pgfqpoint{4.381848in}{5.243443in}}%
\pgfpathcurveto{\pgfqpoint{4.381848in}{5.237619in}}{\pgfqpoint{4.384161in}{5.232033in}}{\pgfqpoint{4.388280in}{5.227915in}}%
\pgfpathcurveto{\pgfqpoint{4.392398in}{5.223797in}}{\pgfqpoint{4.397984in}{5.221483in}}{\pgfqpoint{4.403808in}{5.221483in}}%
\pgfpathlineto{\pgfqpoint{4.403808in}{5.221483in}}%
\pgfpathclose%
\pgfusepath{stroke,fill}%
\end{pgfscope}%
\begin{pgfscope}%
\pgfpathrectangle{\pgfqpoint{1.000000in}{1.148311in}}{\pgfqpoint{6.200000in}{5.623377in}}%
\pgfusepath{clip}%
\pgfsetbuttcap%
\pgfsetroundjoin%
\definecolor{currentfill}{rgb}{0.200000,0.200000,0.800000}%
\pgfsetfillcolor{currentfill}%
\pgfsetlinewidth{1.003750pt}%
\definecolor{currentstroke}{rgb}{0.200000,0.200000,0.800000}%
\pgfsetstrokecolor{currentstroke}%
\pgfsetdash{}{0pt}%
\pgfpathmoveto{\pgfqpoint{4.354918in}{5.146322in}}%
\pgfpathcurveto{\pgfqpoint{4.360741in}{5.146322in}}{\pgfqpoint{4.366328in}{5.148636in}}{\pgfqpoint{4.370446in}{5.152754in}}%
\pgfpathcurveto{\pgfqpoint{4.374564in}{5.156873in}}{\pgfqpoint{4.376878in}{5.162459in}}{\pgfqpoint{4.376878in}{5.168283in}}%
\pgfpathcurveto{\pgfqpoint{4.376878in}{5.174107in}}{\pgfqpoint{4.374564in}{5.179693in}}{\pgfqpoint{4.370446in}{5.183811in}}%
\pgfpathcurveto{\pgfqpoint{4.366328in}{5.187929in}}{\pgfqpoint{4.360741in}{5.190243in}}{\pgfqpoint{4.354918in}{5.190243in}}%
\pgfpathcurveto{\pgfqpoint{4.349094in}{5.190243in}}{\pgfqpoint{4.343507in}{5.187929in}}{\pgfqpoint{4.339389in}{5.183811in}}%
\pgfpathcurveto{\pgfqpoint{4.335271in}{5.179693in}}{\pgfqpoint{4.332957in}{5.174107in}}{\pgfqpoint{4.332957in}{5.168283in}}%
\pgfpathcurveto{\pgfqpoint{4.332957in}{5.162459in}}{\pgfqpoint{4.335271in}{5.156873in}}{\pgfqpoint{4.339389in}{5.152754in}}%
\pgfpathcurveto{\pgfqpoint{4.343507in}{5.148636in}}{\pgfqpoint{4.349094in}{5.146322in}}{\pgfqpoint{4.354918in}{5.146322in}}%
\pgfpathlineto{\pgfqpoint{4.354918in}{5.146322in}}%
\pgfpathclose%
\pgfusepath{stroke,fill}%
\end{pgfscope}%
\begin{pgfscope}%
\pgfpathrectangle{\pgfqpoint{1.000000in}{1.148311in}}{\pgfqpoint{6.200000in}{5.623377in}}%
\pgfusepath{clip}%
\pgfsetbuttcap%
\pgfsetroundjoin%
\definecolor{currentfill}{rgb}{0.200000,0.200000,0.800000}%
\pgfsetfillcolor{currentfill}%
\pgfsetlinewidth{1.003750pt}%
\definecolor{currentstroke}{rgb}{0.200000,0.200000,0.800000}%
\pgfsetstrokecolor{currentstroke}%
\pgfsetdash{}{0pt}%
\pgfpathmoveto{\pgfqpoint{4.372634in}{5.078383in}}%
\pgfpathcurveto{\pgfqpoint{4.378457in}{5.078383in}}{\pgfqpoint{4.384044in}{5.080697in}}{\pgfqpoint{4.388162in}{5.084815in}}%
\pgfpathcurveto{\pgfqpoint{4.392280in}{5.088934in}}{\pgfqpoint{4.394594in}{5.094520in}}{\pgfqpoint{4.394594in}{5.100344in}}%
\pgfpathcurveto{\pgfqpoint{4.394594in}{5.106168in}}{\pgfqpoint{4.392280in}{5.111754in}}{\pgfqpoint{4.388162in}{5.115872in}}%
\pgfpathcurveto{\pgfqpoint{4.384044in}{5.119990in}}{\pgfqpoint{4.378457in}{5.122304in}}{\pgfqpoint{4.372634in}{5.122304in}}%
\pgfpathcurveto{\pgfqpoint{4.366810in}{5.122304in}}{\pgfqpoint{4.361223in}{5.119990in}}{\pgfqpoint{4.357105in}{5.115872in}}%
\pgfpathcurveto{\pgfqpoint{4.352987in}{5.111754in}}{\pgfqpoint{4.350673in}{5.106168in}}{\pgfqpoint{4.350673in}{5.100344in}}%
\pgfpathcurveto{\pgfqpoint{4.350673in}{5.094520in}}{\pgfqpoint{4.352987in}{5.088934in}}{\pgfqpoint{4.357105in}{5.084815in}}%
\pgfpathcurveto{\pgfqpoint{4.361223in}{5.080697in}}{\pgfqpoint{4.366810in}{5.078383in}}{\pgfqpoint{4.372634in}{5.078383in}}%
\pgfpathlineto{\pgfqpoint{4.372634in}{5.078383in}}%
\pgfpathclose%
\pgfusepath{stroke,fill}%
\end{pgfscope}%
\begin{pgfscope}%
\pgfpathrectangle{\pgfqpoint{1.000000in}{1.148311in}}{\pgfqpoint{6.200000in}{5.623377in}}%
\pgfusepath{clip}%
\pgfsetbuttcap%
\pgfsetroundjoin%
\definecolor{currentfill}{rgb}{0.200000,0.200000,0.800000}%
\pgfsetfillcolor{currentfill}%
\pgfsetlinewidth{1.003750pt}%
\definecolor{currentstroke}{rgb}{0.200000,0.200000,0.800000}%
\pgfsetstrokecolor{currentstroke}%
\pgfsetdash{}{0pt}%
\pgfpathmoveto{\pgfqpoint{4.406227in}{5.015412in}}%
\pgfpathcurveto{\pgfqpoint{4.412051in}{5.015412in}}{\pgfqpoint{4.417637in}{5.017726in}}{\pgfqpoint{4.421755in}{5.021844in}}%
\pgfpathcurveto{\pgfqpoint{4.425873in}{5.025962in}}{\pgfqpoint{4.428187in}{5.031549in}}{\pgfqpoint{4.428187in}{5.037372in}}%
\pgfpathcurveto{\pgfqpoint{4.428187in}{5.043196in}}{\pgfqpoint{4.425873in}{5.048783in}}{\pgfqpoint{4.421755in}{5.052901in}}%
\pgfpathcurveto{\pgfqpoint{4.417637in}{5.057019in}}{\pgfqpoint{4.412051in}{5.059333in}}{\pgfqpoint{4.406227in}{5.059333in}}%
\pgfpathcurveto{\pgfqpoint{4.400403in}{5.059333in}}{\pgfqpoint{4.394817in}{5.057019in}}{\pgfqpoint{4.390698in}{5.052901in}}%
\pgfpathcurveto{\pgfqpoint{4.386580in}{5.048783in}}{\pgfqpoint{4.384266in}{5.043196in}}{\pgfqpoint{4.384266in}{5.037372in}}%
\pgfpathcurveto{\pgfqpoint{4.384266in}{5.031549in}}{\pgfqpoint{4.386580in}{5.025962in}}{\pgfqpoint{4.390698in}{5.021844in}}%
\pgfpathcurveto{\pgfqpoint{4.394817in}{5.017726in}}{\pgfqpoint{4.400403in}{5.015412in}}{\pgfqpoint{4.406227in}{5.015412in}}%
\pgfpathlineto{\pgfqpoint{4.406227in}{5.015412in}}%
\pgfpathclose%
\pgfusepath{stroke,fill}%
\end{pgfscope}%
\begin{pgfscope}%
\pgfpathrectangle{\pgfqpoint{1.000000in}{1.148311in}}{\pgfqpoint{6.200000in}{5.623377in}}%
\pgfusepath{clip}%
\pgfsetbuttcap%
\pgfsetroundjoin%
\definecolor{currentfill}{rgb}{0.200000,0.200000,0.800000}%
\pgfsetfillcolor{currentfill}%
\pgfsetlinewidth{1.003750pt}%
\definecolor{currentstroke}{rgb}{0.200000,0.200000,0.800000}%
\pgfsetstrokecolor{currentstroke}%
\pgfsetdash{}{0pt}%
\pgfpathmoveto{\pgfqpoint{4.377717in}{4.931794in}}%
\pgfpathcurveto{\pgfqpoint{4.383541in}{4.931794in}}{\pgfqpoint{4.389128in}{4.934108in}}{\pgfqpoint{4.393246in}{4.938226in}}%
\pgfpathcurveto{\pgfqpoint{4.397364in}{4.942344in}}{\pgfqpoint{4.399678in}{4.947930in}}{\pgfqpoint{4.399678in}{4.953754in}}%
\pgfpathcurveto{\pgfqpoint{4.399678in}{4.959578in}}{\pgfqpoint{4.397364in}{4.965164in}}{\pgfqpoint{4.393246in}{4.969282in}}%
\pgfpathcurveto{\pgfqpoint{4.389128in}{4.973400in}}{\pgfqpoint{4.383541in}{4.975714in}}{\pgfqpoint{4.377717in}{4.975714in}}%
\pgfpathcurveto{\pgfqpoint{4.371894in}{4.975714in}}{\pgfqpoint{4.366307in}{4.973400in}}{\pgfqpoint{4.362189in}{4.969282in}}%
\pgfpathcurveto{\pgfqpoint{4.358071in}{4.965164in}}{\pgfqpoint{4.355757in}{4.959578in}}{\pgfqpoint{4.355757in}{4.953754in}}%
\pgfpathcurveto{\pgfqpoint{4.355757in}{4.947930in}}{\pgfqpoint{4.358071in}{4.942344in}}{\pgfqpoint{4.362189in}{4.938226in}}%
\pgfpathcurveto{\pgfqpoint{4.366307in}{4.934108in}}{\pgfqpoint{4.371894in}{4.931794in}}{\pgfqpoint{4.377717in}{4.931794in}}%
\pgfpathlineto{\pgfqpoint{4.377717in}{4.931794in}}%
\pgfpathclose%
\pgfusepath{stroke,fill}%
\end{pgfscope}%
\begin{pgfscope}%
\pgfpathrectangle{\pgfqpoint{1.000000in}{1.148311in}}{\pgfqpoint{6.200000in}{5.623377in}}%
\pgfusepath{clip}%
\pgfsetbuttcap%
\pgfsetroundjoin%
\definecolor{currentfill}{rgb}{0.200000,0.200000,0.800000}%
\pgfsetfillcolor{currentfill}%
\pgfsetlinewidth{1.003750pt}%
\definecolor{currentstroke}{rgb}{0.200000,0.200000,0.800000}%
\pgfsetstrokecolor{currentstroke}%
\pgfsetdash{}{0pt}%
\pgfpathmoveto{\pgfqpoint{4.390612in}{4.858626in}}%
\pgfpathcurveto{\pgfqpoint{4.396436in}{4.858626in}}{\pgfqpoint{4.402022in}{4.860940in}}{\pgfqpoint{4.406140in}{4.865058in}}%
\pgfpathcurveto{\pgfqpoint{4.410259in}{4.869176in}}{\pgfqpoint{4.412573in}{4.874762in}}{\pgfqpoint{4.412573in}{4.880586in}}%
\pgfpathcurveto{\pgfqpoint{4.412573in}{4.886410in}}{\pgfqpoint{4.410259in}{4.891997in}}{\pgfqpoint{4.406140in}{4.896115in}}%
\pgfpathcurveto{\pgfqpoint{4.402022in}{4.900233in}}{\pgfqpoint{4.396436in}{4.902547in}}{\pgfqpoint{4.390612in}{4.902547in}}%
\pgfpathcurveto{\pgfqpoint{4.384788in}{4.902547in}}{\pgfqpoint{4.379202in}{4.900233in}}{\pgfqpoint{4.375084in}{4.896115in}}%
\pgfpathcurveto{\pgfqpoint{4.370966in}{4.891997in}}{\pgfqpoint{4.368652in}{4.886410in}}{\pgfqpoint{4.368652in}{4.880586in}}%
\pgfpathcurveto{\pgfqpoint{4.368652in}{4.874762in}}{\pgfqpoint{4.370966in}{4.869176in}}{\pgfqpoint{4.375084in}{4.865058in}}%
\pgfpathcurveto{\pgfqpoint{4.379202in}{4.860940in}}{\pgfqpoint{4.384788in}{4.858626in}}{\pgfqpoint{4.390612in}{4.858626in}}%
\pgfpathlineto{\pgfqpoint{4.390612in}{4.858626in}}%
\pgfpathclose%
\pgfusepath{stroke,fill}%
\end{pgfscope}%
\begin{pgfscope}%
\pgfpathrectangle{\pgfqpoint{1.000000in}{1.148311in}}{\pgfqpoint{6.200000in}{5.623377in}}%
\pgfusepath{clip}%
\pgfsetbuttcap%
\pgfsetroundjoin%
\definecolor{currentfill}{rgb}{0.200000,0.200000,0.800000}%
\pgfsetfillcolor{currentfill}%
\pgfsetlinewidth{1.003750pt}%
\definecolor{currentstroke}{rgb}{0.200000,0.200000,0.800000}%
\pgfsetstrokecolor{currentstroke}%
\pgfsetdash{}{0pt}%
\pgfpathmoveto{\pgfqpoint{4.552122in}{4.859574in}}%
\pgfpathcurveto{\pgfqpoint{4.557946in}{4.859574in}}{\pgfqpoint{4.563532in}{4.861888in}}{\pgfqpoint{4.567651in}{4.866006in}}%
\pgfpathcurveto{\pgfqpoint{4.571769in}{4.870125in}}{\pgfqpoint{4.574083in}{4.875711in}}{\pgfqpoint{4.574083in}{4.881535in}}%
\pgfpathcurveto{\pgfqpoint{4.574083in}{4.887359in}}{\pgfqpoint{4.571769in}{4.892945in}}{\pgfqpoint{4.567651in}{4.897063in}}%
\pgfpathcurveto{\pgfqpoint{4.563532in}{4.901181in}}{\pgfqpoint{4.557946in}{4.903495in}}{\pgfqpoint{4.552122in}{4.903495in}}%
\pgfpathcurveto{\pgfqpoint{4.546298in}{4.903495in}}{\pgfqpoint{4.540712in}{4.901181in}}{\pgfqpoint{4.536594in}{4.897063in}}%
\pgfpathcurveto{\pgfqpoint{4.532476in}{4.892945in}}{\pgfqpoint{4.530162in}{4.887359in}}{\pgfqpoint{4.530162in}{4.881535in}}%
\pgfpathcurveto{\pgfqpoint{4.530162in}{4.875711in}}{\pgfqpoint{4.532476in}{4.870125in}}{\pgfqpoint{4.536594in}{4.866006in}}%
\pgfpathcurveto{\pgfqpoint{4.540712in}{4.861888in}}{\pgfqpoint{4.546298in}{4.859574in}}{\pgfqpoint{4.552122in}{4.859574in}}%
\pgfpathlineto{\pgfqpoint{4.552122in}{4.859574in}}%
\pgfpathclose%
\pgfusepath{stroke,fill}%
\end{pgfscope}%
\begin{pgfscope}%
\pgfpathrectangle{\pgfqpoint{1.000000in}{1.148311in}}{\pgfqpoint{6.200000in}{5.623377in}}%
\pgfusepath{clip}%
\pgfsetbuttcap%
\pgfsetroundjoin%
\definecolor{currentfill}{rgb}{0.200000,0.200000,0.800000}%
\pgfsetfillcolor{currentfill}%
\pgfsetlinewidth{1.003750pt}%
\definecolor{currentstroke}{rgb}{0.200000,0.200000,0.800000}%
\pgfsetstrokecolor{currentstroke}%
\pgfsetdash{}{0pt}%
\pgfpathmoveto{\pgfqpoint{4.562163in}{4.791126in}}%
\pgfpathcurveto{\pgfqpoint{4.567987in}{4.791126in}}{\pgfqpoint{4.573573in}{4.793440in}}{\pgfqpoint{4.577691in}{4.797558in}}%
\pgfpathcurveto{\pgfqpoint{4.581809in}{4.801676in}}{\pgfqpoint{4.584123in}{4.807262in}}{\pgfqpoint{4.584123in}{4.813086in}}%
\pgfpathcurveto{\pgfqpoint{4.584123in}{4.818910in}}{\pgfqpoint{4.581809in}{4.824496in}}{\pgfqpoint{4.577691in}{4.828614in}}%
\pgfpathcurveto{\pgfqpoint{4.573573in}{4.832732in}}{\pgfqpoint{4.567987in}{4.835046in}}{\pgfqpoint{4.562163in}{4.835046in}}%
\pgfpathcurveto{\pgfqpoint{4.556339in}{4.835046in}}{\pgfqpoint{4.550753in}{4.832732in}}{\pgfqpoint{4.546635in}{4.828614in}}%
\pgfpathcurveto{\pgfqpoint{4.542517in}{4.824496in}}{\pgfqpoint{4.540203in}{4.818910in}}{\pgfqpoint{4.540203in}{4.813086in}}%
\pgfpathcurveto{\pgfqpoint{4.540203in}{4.807262in}}{\pgfqpoint{4.542517in}{4.801676in}}{\pgfqpoint{4.546635in}{4.797558in}}%
\pgfpathcurveto{\pgfqpoint{4.550753in}{4.793440in}}{\pgfqpoint{4.556339in}{4.791126in}}{\pgfqpoint{4.562163in}{4.791126in}}%
\pgfpathlineto{\pgfqpoint{4.562163in}{4.791126in}}%
\pgfpathclose%
\pgfusepath{stroke,fill}%
\end{pgfscope}%
\begin{pgfscope}%
\pgfpathrectangle{\pgfqpoint{1.000000in}{1.148311in}}{\pgfqpoint{6.200000in}{5.623377in}}%
\pgfusepath{clip}%
\pgfsetbuttcap%
\pgfsetroundjoin%
\definecolor{currentfill}{rgb}{0.200000,0.200000,0.800000}%
\pgfsetfillcolor{currentfill}%
\pgfsetlinewidth{1.003750pt}%
\definecolor{currentstroke}{rgb}{0.200000,0.200000,0.800000}%
\pgfsetstrokecolor{currentstroke}%
\pgfsetdash{}{0pt}%
\pgfpathmoveto{\pgfqpoint{4.530019in}{4.689546in}}%
\pgfpathcurveto{\pgfqpoint{4.535843in}{4.689546in}}{\pgfqpoint{4.541429in}{4.691860in}}{\pgfqpoint{4.545547in}{4.695978in}}%
\pgfpathcurveto{\pgfqpoint{4.549665in}{4.700097in}}{\pgfqpoint{4.551979in}{4.705683in}}{\pgfqpoint{4.551979in}{4.711507in}}%
\pgfpathcurveto{\pgfqpoint{4.551979in}{4.717331in}}{\pgfqpoint{4.549665in}{4.722917in}}{\pgfqpoint{4.545547in}{4.727035in}}%
\pgfpathcurveto{\pgfqpoint{4.541429in}{4.731153in}}{\pgfqpoint{4.535843in}{4.733467in}}{\pgfqpoint{4.530019in}{4.733467in}}%
\pgfpathcurveto{\pgfqpoint{4.524195in}{4.733467in}}{\pgfqpoint{4.518609in}{4.731153in}}{\pgfqpoint{4.514490in}{4.727035in}}%
\pgfpathcurveto{\pgfqpoint{4.510372in}{4.722917in}}{\pgfqpoint{4.508058in}{4.717331in}}{\pgfqpoint{4.508058in}{4.711507in}}%
\pgfpathcurveto{\pgfqpoint{4.508058in}{4.705683in}}{\pgfqpoint{4.510372in}{4.700097in}}{\pgfqpoint{4.514490in}{4.695978in}}%
\pgfpathcurveto{\pgfqpoint{4.518609in}{4.691860in}}{\pgfqpoint{4.524195in}{4.689546in}}{\pgfqpoint{4.530019in}{4.689546in}}%
\pgfpathlineto{\pgfqpoint{4.530019in}{4.689546in}}%
\pgfpathclose%
\pgfusepath{stroke,fill}%
\end{pgfscope}%
\begin{pgfscope}%
\pgfpathrectangle{\pgfqpoint{1.000000in}{1.148311in}}{\pgfqpoint{6.200000in}{5.623377in}}%
\pgfusepath{clip}%
\pgfsetbuttcap%
\pgfsetroundjoin%
\definecolor{currentfill}{rgb}{0.200000,0.200000,0.800000}%
\pgfsetfillcolor{currentfill}%
\pgfsetlinewidth{1.003750pt}%
\definecolor{currentstroke}{rgb}{0.200000,0.200000,0.800000}%
\pgfsetstrokecolor{currentstroke}%
\pgfsetdash{}{0pt}%
\pgfpathmoveto{\pgfqpoint{4.537434in}{4.605510in}}%
\pgfpathcurveto{\pgfqpoint{4.543258in}{4.605510in}}{\pgfqpoint{4.548844in}{4.607823in}}{\pgfqpoint{4.552962in}{4.611942in}}%
\pgfpathcurveto{\pgfqpoint{4.557081in}{4.616060in}}{\pgfqpoint{4.559395in}{4.621646in}}{\pgfqpoint{4.559395in}{4.627470in}}%
\pgfpathcurveto{\pgfqpoint{4.559395in}{4.633294in}}{\pgfqpoint{4.557081in}{4.638880in}}{\pgfqpoint{4.552962in}{4.642998in}}%
\pgfpathcurveto{\pgfqpoint{4.548844in}{4.647116in}}{\pgfqpoint{4.543258in}{4.649430in}}{\pgfqpoint{4.537434in}{4.649430in}}%
\pgfpathcurveto{\pgfqpoint{4.531610in}{4.649430in}}{\pgfqpoint{4.526024in}{4.647116in}}{\pgfqpoint{4.521906in}{4.642998in}}%
\pgfpathcurveto{\pgfqpoint{4.517788in}{4.638880in}}{\pgfqpoint{4.515474in}{4.633294in}}{\pgfqpoint{4.515474in}{4.627470in}}%
\pgfpathcurveto{\pgfqpoint{4.515474in}{4.621646in}}{\pgfqpoint{4.517788in}{4.616060in}}{\pgfqpoint{4.521906in}{4.611942in}}%
\pgfpathcurveto{\pgfqpoint{4.526024in}{4.607823in}}{\pgfqpoint{4.531610in}{4.605510in}}{\pgfqpoint{4.537434in}{4.605510in}}%
\pgfpathlineto{\pgfqpoint{4.537434in}{4.605510in}}%
\pgfpathclose%
\pgfusepath{stroke,fill}%
\end{pgfscope}%
\begin{pgfscope}%
\pgfpathrectangle{\pgfqpoint{1.000000in}{1.148311in}}{\pgfqpoint{6.200000in}{5.623377in}}%
\pgfusepath{clip}%
\pgfsetbuttcap%
\pgfsetroundjoin%
\definecolor{currentfill}{rgb}{0.200000,0.200000,0.800000}%
\pgfsetfillcolor{currentfill}%
\pgfsetlinewidth{1.003750pt}%
\definecolor{currentstroke}{rgb}{0.200000,0.200000,0.800000}%
\pgfsetstrokecolor{currentstroke}%
\pgfsetdash{}{0pt}%
\pgfpathmoveto{\pgfqpoint{4.674768in}{4.629858in}}%
\pgfpathcurveto{\pgfqpoint{4.680592in}{4.629858in}}{\pgfqpoint{4.686178in}{4.632172in}}{\pgfqpoint{4.690296in}{4.636290in}}%
\pgfpathcurveto{\pgfqpoint{4.694415in}{4.640408in}}{\pgfqpoint{4.696728in}{4.645994in}}{\pgfqpoint{4.696728in}{4.651818in}}%
\pgfpathcurveto{\pgfqpoint{4.696728in}{4.657642in}}{\pgfqpoint{4.694415in}{4.663228in}}{\pgfqpoint{4.690296in}{4.667346in}}%
\pgfpathcurveto{\pgfqpoint{4.686178in}{4.671464in}}{\pgfqpoint{4.680592in}{4.673778in}}{\pgfqpoint{4.674768in}{4.673778in}}%
\pgfpathcurveto{\pgfqpoint{4.668944in}{4.673778in}}{\pgfqpoint{4.663358in}{4.671464in}}{\pgfqpoint{4.659240in}{4.667346in}}%
\pgfpathcurveto{\pgfqpoint{4.655122in}{4.663228in}}{\pgfqpoint{4.652808in}{4.657642in}}{\pgfqpoint{4.652808in}{4.651818in}}%
\pgfpathcurveto{\pgfqpoint{4.652808in}{4.645994in}}{\pgfqpoint{4.655122in}{4.640408in}}{\pgfqpoint{4.659240in}{4.636290in}}%
\pgfpathcurveto{\pgfqpoint{4.663358in}{4.632172in}}{\pgfqpoint{4.668944in}{4.629858in}}{\pgfqpoint{4.674768in}{4.629858in}}%
\pgfpathlineto{\pgfqpoint{4.674768in}{4.629858in}}%
\pgfpathclose%
\pgfusepath{stroke,fill}%
\end{pgfscope}%
\begin{pgfscope}%
\pgfpathrectangle{\pgfqpoint{1.000000in}{1.148311in}}{\pgfqpoint{6.200000in}{5.623377in}}%
\pgfusepath{clip}%
\pgfsetbuttcap%
\pgfsetroundjoin%
\definecolor{currentfill}{rgb}{0.200000,0.200000,0.800000}%
\pgfsetfillcolor{currentfill}%
\pgfsetlinewidth{1.003750pt}%
\definecolor{currentstroke}{rgb}{0.200000,0.200000,0.800000}%
\pgfsetstrokecolor{currentstroke}%
\pgfsetdash{}{0pt}%
\pgfpathmoveto{\pgfqpoint{4.670657in}{4.531942in}}%
\pgfpathcurveto{\pgfqpoint{4.676481in}{4.531942in}}{\pgfqpoint{4.682067in}{4.534256in}}{\pgfqpoint{4.686185in}{4.538374in}}%
\pgfpathcurveto{\pgfqpoint{4.690304in}{4.542492in}}{\pgfqpoint{4.692617in}{4.548079in}}{\pgfqpoint{4.692617in}{4.553903in}}%
\pgfpathcurveto{\pgfqpoint{4.692617in}{4.559726in}}{\pgfqpoint{4.690304in}{4.565313in}}{\pgfqpoint{4.686185in}{4.569431in}}%
\pgfpathcurveto{\pgfqpoint{4.682067in}{4.573549in}}{\pgfqpoint{4.676481in}{4.575863in}}{\pgfqpoint{4.670657in}{4.575863in}}%
\pgfpathcurveto{\pgfqpoint{4.664833in}{4.575863in}}{\pgfqpoint{4.659247in}{4.573549in}}{\pgfqpoint{4.655129in}{4.569431in}}%
\pgfpathcurveto{\pgfqpoint{4.651011in}{4.565313in}}{\pgfqpoint{4.648697in}{4.559726in}}{\pgfqpoint{4.648697in}{4.553903in}}%
\pgfpathcurveto{\pgfqpoint{4.648697in}{4.548079in}}{\pgfqpoint{4.651011in}{4.542492in}}{\pgfqpoint{4.655129in}{4.538374in}}%
\pgfpathcurveto{\pgfqpoint{4.659247in}{4.534256in}}{\pgfqpoint{4.664833in}{4.531942in}}{\pgfqpoint{4.670657in}{4.531942in}}%
\pgfpathlineto{\pgfqpoint{4.670657in}{4.531942in}}%
\pgfpathclose%
\pgfusepath{stroke,fill}%
\end{pgfscope}%
\begin{pgfscope}%
\pgfpathrectangle{\pgfqpoint{1.000000in}{1.148311in}}{\pgfqpoint{6.200000in}{5.623377in}}%
\pgfusepath{clip}%
\pgfsetbuttcap%
\pgfsetroundjoin%
\definecolor{currentfill}{rgb}{0.800000,0.800000,0.200000}%
\pgfsetfillcolor{currentfill}%
\pgfsetlinewidth{1.003750pt}%
\definecolor{currentstroke}{rgb}{0.800000,0.800000,0.200000}%
\pgfsetstrokecolor{currentstroke}%
\pgfsetdash{}{0pt}%
\pgfpathmoveto{\pgfqpoint{4.778443in}{4.549087in}}%
\pgfpathcurveto{\pgfqpoint{4.784267in}{4.549087in}}{\pgfqpoint{4.789853in}{4.551401in}}{\pgfqpoint{4.793971in}{4.555519in}}%
\pgfpathcurveto{\pgfqpoint{4.798090in}{4.559637in}}{\pgfqpoint{4.800403in}{4.565223in}}{\pgfqpoint{4.800403in}{4.571047in}}%
\pgfpathcurveto{\pgfqpoint{4.800403in}{4.576871in}}{\pgfqpoint{4.798090in}{4.582457in}}{\pgfqpoint{4.793971in}{4.586576in}}%
\pgfpathcurveto{\pgfqpoint{4.789853in}{4.590694in}}{\pgfqpoint{4.784267in}{4.593008in}}{\pgfqpoint{4.778443in}{4.593008in}}%
\pgfpathcurveto{\pgfqpoint{4.772619in}{4.593008in}}{\pgfqpoint{4.767033in}{4.590694in}}{\pgfqpoint{4.762915in}{4.586576in}}%
\pgfpathcurveto{\pgfqpoint{4.758797in}{4.582457in}}{\pgfqpoint{4.756483in}{4.576871in}}{\pgfqpoint{4.756483in}{4.571047in}}%
\pgfpathcurveto{\pgfqpoint{4.756483in}{4.565223in}}{\pgfqpoint{4.758797in}{4.559637in}}{\pgfqpoint{4.762915in}{4.555519in}}%
\pgfpathcurveto{\pgfqpoint{4.767033in}{4.551401in}}{\pgfqpoint{4.772619in}{4.549087in}}{\pgfqpoint{4.778443in}{4.549087in}}%
\pgfpathlineto{\pgfqpoint{4.778443in}{4.549087in}}%
\pgfpathclose%
\pgfusepath{stroke,fill}%
\end{pgfscope}%
\begin{pgfscope}%
\pgfpathrectangle{\pgfqpoint{1.000000in}{1.148311in}}{\pgfqpoint{6.200000in}{5.623377in}}%
\pgfusepath{clip}%
\pgfsetbuttcap%
\pgfsetroundjoin%
\definecolor{currentfill}{rgb}{0.800000,0.800000,0.200000}%
\pgfsetfillcolor{currentfill}%
\pgfsetlinewidth{1.003750pt}%
\definecolor{currentstroke}{rgb}{0.800000,0.800000,0.200000}%
\pgfsetstrokecolor{currentstroke}%
\pgfsetdash{}{0pt}%
\pgfpathmoveto{\pgfqpoint{4.831333in}{4.511597in}}%
\pgfpathcurveto{\pgfqpoint{4.837157in}{4.511597in}}{\pgfqpoint{4.842743in}{4.513911in}}{\pgfqpoint{4.846861in}{4.518029in}}%
\pgfpathcurveto{\pgfqpoint{4.850979in}{4.522147in}}{\pgfqpoint{4.853293in}{4.527733in}}{\pgfqpoint{4.853293in}{4.533557in}}%
\pgfpathcurveto{\pgfqpoint{4.853293in}{4.539381in}}{\pgfqpoint{4.850979in}{4.544968in}}{\pgfqpoint{4.846861in}{4.549086in}}%
\pgfpathcurveto{\pgfqpoint{4.842743in}{4.553204in}}{\pgfqpoint{4.837157in}{4.555518in}}{\pgfqpoint{4.831333in}{4.555518in}}%
\pgfpathcurveto{\pgfqpoint{4.825509in}{4.555518in}}{\pgfqpoint{4.819923in}{4.553204in}}{\pgfqpoint{4.815805in}{4.549086in}}%
\pgfpathcurveto{\pgfqpoint{4.811686in}{4.544968in}}{\pgfqpoint{4.809373in}{4.539381in}}{\pgfqpoint{4.809373in}{4.533557in}}%
\pgfpathcurveto{\pgfqpoint{4.809373in}{4.527733in}}{\pgfqpoint{4.811686in}{4.522147in}}{\pgfqpoint{4.815805in}{4.518029in}}%
\pgfpathcurveto{\pgfqpoint{4.819923in}{4.513911in}}{\pgfqpoint{4.825509in}{4.511597in}}{\pgfqpoint{4.831333in}{4.511597in}}%
\pgfpathlineto{\pgfqpoint{4.831333in}{4.511597in}}%
\pgfpathclose%
\pgfusepath{stroke,fill}%
\end{pgfscope}%
\begin{pgfscope}%
\pgfpathrectangle{\pgfqpoint{1.000000in}{1.148311in}}{\pgfqpoint{6.200000in}{5.623377in}}%
\pgfusepath{clip}%
\pgfsetbuttcap%
\pgfsetroundjoin%
\definecolor{currentfill}{rgb}{0.200000,0.200000,0.800000}%
\pgfsetfillcolor{currentfill}%
\pgfsetlinewidth{1.003750pt}%
\definecolor{currentstroke}{rgb}{0.200000,0.200000,0.800000}%
\pgfsetstrokecolor{currentstroke}%
\pgfsetdash{}{0pt}%
\pgfpathmoveto{\pgfqpoint{4.797939in}{4.344922in}}%
\pgfpathcurveto{\pgfqpoint{4.803763in}{4.344922in}}{\pgfqpoint{4.809349in}{4.347236in}}{\pgfqpoint{4.813467in}{4.351354in}}%
\pgfpathcurveto{\pgfqpoint{4.817585in}{4.355472in}}{\pgfqpoint{4.819899in}{4.361058in}}{\pgfqpoint{4.819899in}{4.366882in}}%
\pgfpathcurveto{\pgfqpoint{4.819899in}{4.372706in}}{\pgfqpoint{4.817585in}{4.378292in}}{\pgfqpoint{4.813467in}{4.382410in}}%
\pgfpathcurveto{\pgfqpoint{4.809349in}{4.386528in}}{\pgfqpoint{4.803763in}{4.388842in}}{\pgfqpoint{4.797939in}{4.388842in}}%
\pgfpathcurveto{\pgfqpoint{4.792115in}{4.388842in}}{\pgfqpoint{4.786529in}{4.386528in}}{\pgfqpoint{4.782411in}{4.382410in}}%
\pgfpathcurveto{\pgfqpoint{4.778293in}{4.378292in}}{\pgfqpoint{4.775979in}{4.372706in}}{\pgfqpoint{4.775979in}{4.366882in}}%
\pgfpathcurveto{\pgfqpoint{4.775979in}{4.361058in}}{\pgfqpoint{4.778293in}{4.355472in}}{\pgfqpoint{4.782411in}{4.351354in}}%
\pgfpathcurveto{\pgfqpoint{4.786529in}{4.347236in}}{\pgfqpoint{4.792115in}{4.344922in}}{\pgfqpoint{4.797939in}{4.344922in}}%
\pgfpathlineto{\pgfqpoint{4.797939in}{4.344922in}}%
\pgfpathclose%
\pgfusepath{stroke,fill}%
\end{pgfscope}%
\begin{pgfscope}%
\pgfpathrectangle{\pgfqpoint{1.000000in}{1.148311in}}{\pgfqpoint{6.200000in}{5.623377in}}%
\pgfusepath{clip}%
\pgfsetbuttcap%
\pgfsetroundjoin%
\definecolor{currentfill}{rgb}{0.200000,0.200000,0.800000}%
\pgfsetfillcolor{currentfill}%
\pgfsetlinewidth{1.003750pt}%
\definecolor{currentstroke}{rgb}{0.200000,0.200000,0.800000}%
\pgfsetstrokecolor{currentstroke}%
\pgfsetdash{}{0pt}%
\pgfpathmoveto{\pgfqpoint{4.894405in}{4.363273in}}%
\pgfpathcurveto{\pgfqpoint{4.900229in}{4.363273in}}{\pgfqpoint{4.905815in}{4.365587in}}{\pgfqpoint{4.909933in}{4.369705in}}%
\pgfpathcurveto{\pgfqpoint{4.914051in}{4.373823in}}{\pgfqpoint{4.916365in}{4.379409in}}{\pgfqpoint{4.916365in}{4.385233in}}%
\pgfpathcurveto{\pgfqpoint{4.916365in}{4.391057in}}{\pgfqpoint{4.914051in}{4.396643in}}{\pgfqpoint{4.909933in}{4.400762in}}%
\pgfpathcurveto{\pgfqpoint{4.905815in}{4.404880in}}{\pgfqpoint{4.900229in}{4.407194in}}{\pgfqpoint{4.894405in}{4.407194in}}%
\pgfpathcurveto{\pgfqpoint{4.888581in}{4.407194in}}{\pgfqpoint{4.882995in}{4.404880in}}{\pgfqpoint{4.878877in}{4.400762in}}%
\pgfpathcurveto{\pgfqpoint{4.874759in}{4.396643in}}{\pgfqpoint{4.872445in}{4.391057in}}{\pgfqpoint{4.872445in}{4.385233in}}%
\pgfpathcurveto{\pgfqpoint{4.872445in}{4.379409in}}{\pgfqpoint{4.874759in}{4.373823in}}{\pgfqpoint{4.878877in}{4.369705in}}%
\pgfpathcurveto{\pgfqpoint{4.882995in}{4.365587in}}{\pgfqpoint{4.888581in}{4.363273in}}{\pgfqpoint{4.894405in}{4.363273in}}%
\pgfpathlineto{\pgfqpoint{4.894405in}{4.363273in}}%
\pgfpathclose%
\pgfusepath{stroke,fill}%
\end{pgfscope}%
\begin{pgfscope}%
\pgfpathrectangle{\pgfqpoint{1.000000in}{1.148311in}}{\pgfqpoint{6.200000in}{5.623377in}}%
\pgfusepath{clip}%
\pgfsetbuttcap%
\pgfsetroundjoin%
\definecolor{currentfill}{rgb}{0.200000,0.200000,0.800000}%
\pgfsetfillcolor{currentfill}%
\pgfsetlinewidth{1.003750pt}%
\definecolor{currentstroke}{rgb}{0.200000,0.200000,0.800000}%
\pgfsetstrokecolor{currentstroke}%
\pgfsetdash{}{0pt}%
\pgfpathmoveto{\pgfqpoint{4.990131in}{4.398448in}}%
\pgfpathcurveto{\pgfqpoint{4.995955in}{4.398448in}}{\pgfqpoint{5.001541in}{4.400762in}}{\pgfqpoint{5.005659in}{4.404880in}}%
\pgfpathcurveto{\pgfqpoint{5.009777in}{4.408998in}}{\pgfqpoint{5.012091in}{4.414584in}}{\pgfqpoint{5.012091in}{4.420408in}}%
\pgfpathcurveto{\pgfqpoint{5.012091in}{4.426232in}}{\pgfqpoint{5.009777in}{4.431818in}}{\pgfqpoint{5.005659in}{4.435936in}}%
\pgfpathcurveto{\pgfqpoint{5.001541in}{4.440054in}}{\pgfqpoint{4.995955in}{4.442368in}}{\pgfqpoint{4.990131in}{4.442368in}}%
\pgfpathcurveto{\pgfqpoint{4.984307in}{4.442368in}}{\pgfqpoint{4.978721in}{4.440054in}}{\pgfqpoint{4.974603in}{4.435936in}}%
\pgfpathcurveto{\pgfqpoint{4.970484in}{4.431818in}}{\pgfqpoint{4.968171in}{4.426232in}}{\pgfqpoint{4.968171in}{4.420408in}}%
\pgfpathcurveto{\pgfqpoint{4.968171in}{4.414584in}}{\pgfqpoint{4.970484in}{4.408998in}}{\pgfqpoint{4.974603in}{4.404880in}}%
\pgfpathcurveto{\pgfqpoint{4.978721in}{4.400762in}}{\pgfqpoint{4.984307in}{4.398448in}}{\pgfqpoint{4.990131in}{4.398448in}}%
\pgfpathlineto{\pgfqpoint{4.990131in}{4.398448in}}%
\pgfpathclose%
\pgfusepath{stroke,fill}%
\end{pgfscope}%
\begin{pgfscope}%
\pgfpathrectangle{\pgfqpoint{1.000000in}{1.148311in}}{\pgfqpoint{6.200000in}{5.623377in}}%
\pgfusepath{clip}%
\pgfsetbuttcap%
\pgfsetroundjoin%
\definecolor{currentfill}{rgb}{0.200000,0.200000,0.800000}%
\pgfsetfillcolor{currentfill}%
\pgfsetlinewidth{1.003750pt}%
\definecolor{currentstroke}{rgb}{0.200000,0.200000,0.800000}%
\pgfsetstrokecolor{currentstroke}%
\pgfsetdash{}{0pt}%
\pgfpathmoveto{\pgfqpoint{5.029360in}{4.323047in}}%
\pgfpathcurveto{\pgfqpoint{5.035184in}{4.323047in}}{\pgfqpoint{5.040770in}{4.325361in}}{\pgfqpoint{5.044888in}{4.329479in}}%
\pgfpathcurveto{\pgfqpoint{5.049006in}{4.333597in}}{\pgfqpoint{5.051320in}{4.339183in}}{\pgfqpoint{5.051320in}{4.345007in}}%
\pgfpathcurveto{\pgfqpoint{5.051320in}{4.350831in}}{\pgfqpoint{5.049006in}{4.356418in}}{\pgfqpoint{5.044888in}{4.360536in}}%
\pgfpathcurveto{\pgfqpoint{5.040770in}{4.364654in}}{\pgfqpoint{5.035184in}{4.366968in}}{\pgfqpoint{5.029360in}{4.366968in}}%
\pgfpathcurveto{\pgfqpoint{5.023536in}{4.366968in}}{\pgfqpoint{5.017950in}{4.364654in}}{\pgfqpoint{5.013831in}{4.360536in}}%
\pgfpathcurveto{\pgfqpoint{5.009713in}{4.356418in}}{\pgfqpoint{5.007399in}{4.350831in}}{\pgfqpoint{5.007399in}{4.345007in}}%
\pgfpathcurveto{\pgfqpoint{5.007399in}{4.339183in}}{\pgfqpoint{5.009713in}{4.333597in}}{\pgfqpoint{5.013831in}{4.329479in}}%
\pgfpathcurveto{\pgfqpoint{5.017950in}{4.325361in}}{\pgfqpoint{5.023536in}{4.323047in}}{\pgfqpoint{5.029360in}{4.323047in}}%
\pgfpathlineto{\pgfqpoint{5.029360in}{4.323047in}}%
\pgfpathclose%
\pgfusepath{stroke,fill}%
\end{pgfscope}%
\begin{pgfscope}%
\pgfpathrectangle{\pgfqpoint{1.000000in}{1.148311in}}{\pgfqpoint{6.200000in}{5.623377in}}%
\pgfusepath{clip}%
\pgfsetbuttcap%
\pgfsetroundjoin%
\definecolor{currentfill}{rgb}{0.200000,0.200000,0.800000}%
\pgfsetfillcolor{currentfill}%
\pgfsetlinewidth{1.003750pt}%
\definecolor{currentstroke}{rgb}{0.200000,0.200000,0.800000}%
\pgfsetstrokecolor{currentstroke}%
\pgfsetdash{}{0pt}%
\pgfpathmoveto{\pgfqpoint{5.100818in}{4.319674in}}%
\pgfpathcurveto{\pgfqpoint{5.106642in}{4.319674in}}{\pgfqpoint{5.112228in}{4.321988in}}{\pgfqpoint{5.116347in}{4.326106in}}%
\pgfpathcurveto{\pgfqpoint{5.120465in}{4.330224in}}{\pgfqpoint{5.122779in}{4.335811in}}{\pgfqpoint{5.122779in}{4.341634in}}%
\pgfpathcurveto{\pgfqpoint{5.122779in}{4.347458in}}{\pgfqpoint{5.120465in}{4.353045in}}{\pgfqpoint{5.116347in}{4.357163in}}%
\pgfpathcurveto{\pgfqpoint{5.112228in}{4.361281in}}{\pgfqpoint{5.106642in}{4.363595in}}{\pgfqpoint{5.100818in}{4.363595in}}%
\pgfpathcurveto{\pgfqpoint{5.094994in}{4.363595in}}{\pgfqpoint{5.089408in}{4.361281in}}{\pgfqpoint{5.085290in}{4.357163in}}%
\pgfpathcurveto{\pgfqpoint{5.081172in}{4.353045in}}{\pgfqpoint{5.078858in}{4.347458in}}{\pgfqpoint{5.078858in}{4.341634in}}%
\pgfpathcurveto{\pgfqpoint{5.078858in}{4.335811in}}{\pgfqpoint{5.081172in}{4.330224in}}{\pgfqpoint{5.085290in}{4.326106in}}%
\pgfpathcurveto{\pgfqpoint{5.089408in}{4.321988in}}{\pgfqpoint{5.094994in}{4.319674in}}{\pgfqpoint{5.100818in}{4.319674in}}%
\pgfpathlineto{\pgfqpoint{5.100818in}{4.319674in}}%
\pgfpathclose%
\pgfusepath{stroke,fill}%
\end{pgfscope}%
\begin{pgfscope}%
\pgfpathrectangle{\pgfqpoint{1.000000in}{1.148311in}}{\pgfqpoint{6.200000in}{5.623377in}}%
\pgfusepath{clip}%
\pgfsetbuttcap%
\pgfsetroundjoin%
\definecolor{currentfill}{rgb}{0.200000,0.200000,0.800000}%
\pgfsetfillcolor{currentfill}%
\pgfsetlinewidth{1.003750pt}%
\definecolor{currentstroke}{rgb}{0.200000,0.200000,0.800000}%
\pgfsetstrokecolor{currentstroke}%
\pgfsetdash{}{0pt}%
\pgfpathmoveto{\pgfqpoint{5.161301in}{4.286208in}}%
\pgfpathcurveto{\pgfqpoint{5.167125in}{4.286208in}}{\pgfqpoint{5.172711in}{4.288522in}}{\pgfqpoint{5.176829in}{4.292640in}}%
\pgfpathcurveto{\pgfqpoint{5.180947in}{4.296759in}}{\pgfqpoint{5.183261in}{4.302345in}}{\pgfqpoint{5.183261in}{4.308169in}}%
\pgfpathcurveto{\pgfqpoint{5.183261in}{4.313993in}}{\pgfqpoint{5.180947in}{4.319579in}}{\pgfqpoint{5.176829in}{4.323697in}}%
\pgfpathcurveto{\pgfqpoint{5.172711in}{4.327815in}}{\pgfqpoint{5.167125in}{4.330129in}}{\pgfqpoint{5.161301in}{4.330129in}}%
\pgfpathcurveto{\pgfqpoint{5.155477in}{4.330129in}}{\pgfqpoint{5.149891in}{4.327815in}}{\pgfqpoint{5.145773in}{4.323697in}}%
\pgfpathcurveto{\pgfqpoint{5.141654in}{4.319579in}}{\pgfqpoint{5.139341in}{4.313993in}}{\pgfqpoint{5.139341in}{4.308169in}}%
\pgfpathcurveto{\pgfqpoint{5.139341in}{4.302345in}}{\pgfqpoint{5.141654in}{4.296759in}}{\pgfqpoint{5.145773in}{4.292640in}}%
\pgfpathcurveto{\pgfqpoint{5.149891in}{4.288522in}}{\pgfqpoint{5.155477in}{4.286208in}}{\pgfqpoint{5.161301in}{4.286208in}}%
\pgfpathlineto{\pgfqpoint{5.161301in}{4.286208in}}%
\pgfpathclose%
\pgfusepath{stroke,fill}%
\end{pgfscope}%
\begin{pgfscope}%
\pgfpathrectangle{\pgfqpoint{1.000000in}{1.148311in}}{\pgfqpoint{6.200000in}{5.623377in}}%
\pgfusepath{clip}%
\pgfsetbuttcap%
\pgfsetroundjoin%
\definecolor{currentfill}{rgb}{0.200000,0.200000,0.800000}%
\pgfsetfillcolor{currentfill}%
\pgfsetlinewidth{1.003750pt}%
\definecolor{currentstroke}{rgb}{0.200000,0.200000,0.800000}%
\pgfsetstrokecolor{currentstroke}%
\pgfsetdash{}{0pt}%
\pgfpathmoveto{\pgfqpoint{5.226594in}{4.265739in}}%
\pgfpathcurveto{\pgfqpoint{5.232418in}{4.265739in}}{\pgfqpoint{5.238004in}{4.268053in}}{\pgfqpoint{5.242123in}{4.272171in}}%
\pgfpathcurveto{\pgfqpoint{5.246241in}{4.276289in}}{\pgfqpoint{5.248555in}{4.281875in}}{\pgfqpoint{5.248555in}{4.287699in}}%
\pgfpathcurveto{\pgfqpoint{5.248555in}{4.293523in}}{\pgfqpoint{5.246241in}{4.299109in}}{\pgfqpoint{5.242123in}{4.303228in}}%
\pgfpathcurveto{\pgfqpoint{5.238004in}{4.307346in}}{\pgfqpoint{5.232418in}{4.309660in}}{\pgfqpoint{5.226594in}{4.309660in}}%
\pgfpathcurveto{\pgfqpoint{5.220770in}{4.309660in}}{\pgfqpoint{5.215184in}{4.307346in}}{\pgfqpoint{5.211066in}{4.303228in}}%
\pgfpathcurveto{\pgfqpoint{5.206948in}{4.299109in}}{\pgfqpoint{5.204634in}{4.293523in}}{\pgfqpoint{5.204634in}{4.287699in}}%
\pgfpathcurveto{\pgfqpoint{5.204634in}{4.281875in}}{\pgfqpoint{5.206948in}{4.276289in}}{\pgfqpoint{5.211066in}{4.272171in}}%
\pgfpathcurveto{\pgfqpoint{5.215184in}{4.268053in}}{\pgfqpoint{5.220770in}{4.265739in}}{\pgfqpoint{5.226594in}{4.265739in}}%
\pgfpathlineto{\pgfqpoint{5.226594in}{4.265739in}}%
\pgfpathclose%
\pgfusepath{stroke,fill}%
\end{pgfscope}%
\begin{pgfscope}%
\pgfpathrectangle{\pgfqpoint{1.000000in}{1.148311in}}{\pgfqpoint{6.200000in}{5.623377in}}%
\pgfusepath{clip}%
\pgfsetbuttcap%
\pgfsetroundjoin%
\definecolor{currentfill}{rgb}{0.200000,0.200000,0.800000}%
\pgfsetfillcolor{currentfill}%
\pgfsetlinewidth{1.003750pt}%
\definecolor{currentstroke}{rgb}{0.200000,0.200000,0.800000}%
\pgfsetstrokecolor{currentstroke}%
\pgfsetdash{}{0pt}%
\pgfpathmoveto{\pgfqpoint{5.291022in}{4.233309in}}%
\pgfpathcurveto{\pgfqpoint{5.296846in}{4.233309in}}{\pgfqpoint{5.302432in}{4.235623in}}{\pgfqpoint{5.306550in}{4.239741in}}%
\pgfpathcurveto{\pgfqpoint{5.310668in}{4.243860in}}{\pgfqpoint{5.312982in}{4.249446in}}{\pgfqpoint{5.312982in}{4.255270in}}%
\pgfpathcurveto{\pgfqpoint{5.312982in}{4.261094in}}{\pgfqpoint{5.310668in}{4.266680in}}{\pgfqpoint{5.306550in}{4.270798in}}%
\pgfpathcurveto{\pgfqpoint{5.302432in}{4.274916in}}{\pgfqpoint{5.296846in}{4.277230in}}{\pgfqpoint{5.291022in}{4.277230in}}%
\pgfpathcurveto{\pgfqpoint{5.285198in}{4.277230in}}{\pgfqpoint{5.279612in}{4.274916in}}{\pgfqpoint{5.275494in}{4.270798in}}%
\pgfpathcurveto{\pgfqpoint{5.271376in}{4.266680in}}{\pgfqpoint{5.269062in}{4.261094in}}{\pgfqpoint{5.269062in}{4.255270in}}%
\pgfpathcurveto{\pgfqpoint{5.269062in}{4.249446in}}{\pgfqpoint{5.271376in}{4.243860in}}{\pgfqpoint{5.275494in}{4.239741in}}%
\pgfpathcurveto{\pgfqpoint{5.279612in}{4.235623in}}{\pgfqpoint{5.285198in}{4.233309in}}{\pgfqpoint{5.291022in}{4.233309in}}%
\pgfpathlineto{\pgfqpoint{5.291022in}{4.233309in}}%
\pgfpathclose%
\pgfusepath{stroke,fill}%
\end{pgfscope}%
\begin{pgfscope}%
\pgfpathrectangle{\pgfqpoint{1.000000in}{1.148311in}}{\pgfqpoint{6.200000in}{5.623377in}}%
\pgfusepath{clip}%
\pgfsetbuttcap%
\pgfsetroundjoin%
\definecolor{currentfill}{rgb}{0.200000,0.200000,0.800000}%
\pgfsetfillcolor{currentfill}%
\pgfsetlinewidth{1.003750pt}%
\definecolor{currentstroke}{rgb}{0.200000,0.200000,0.800000}%
\pgfsetstrokecolor{currentstroke}%
\pgfsetdash{}{0pt}%
\pgfpathmoveto{\pgfqpoint{5.359584in}{4.216182in}}%
\pgfpathcurveto{\pgfqpoint{5.365408in}{4.216182in}}{\pgfqpoint{5.370994in}{4.218496in}}{\pgfqpoint{5.375112in}{4.222614in}}%
\pgfpathcurveto{\pgfqpoint{5.379230in}{4.226732in}}{\pgfqpoint{5.381544in}{4.232318in}}{\pgfqpoint{5.381544in}{4.238142in}}%
\pgfpathcurveto{\pgfqpoint{5.381544in}{4.243966in}}{\pgfqpoint{5.379230in}{4.249552in}}{\pgfqpoint{5.375112in}{4.253670in}}%
\pgfpathcurveto{\pgfqpoint{5.370994in}{4.257789in}}{\pgfqpoint{5.365408in}{4.260102in}}{\pgfqpoint{5.359584in}{4.260102in}}%
\pgfpathcurveto{\pgfqpoint{5.353760in}{4.260102in}}{\pgfqpoint{5.348174in}{4.257789in}}{\pgfqpoint{5.344056in}{4.253670in}}%
\pgfpathcurveto{\pgfqpoint{5.339938in}{4.249552in}}{\pgfqpoint{5.337624in}{4.243966in}}{\pgfqpoint{5.337624in}{4.238142in}}%
\pgfpathcurveto{\pgfqpoint{5.337624in}{4.232318in}}{\pgfqpoint{5.339938in}{4.226732in}}{\pgfqpoint{5.344056in}{4.222614in}}%
\pgfpathcurveto{\pgfqpoint{5.348174in}{4.218496in}}{\pgfqpoint{5.353760in}{4.216182in}}{\pgfqpoint{5.359584in}{4.216182in}}%
\pgfpathlineto{\pgfqpoint{5.359584in}{4.216182in}}%
\pgfpathclose%
\pgfusepath{stroke,fill}%
\end{pgfscope}%
\begin{pgfscope}%
\pgfpathrectangle{\pgfqpoint{1.000000in}{1.148311in}}{\pgfqpoint{6.200000in}{5.623377in}}%
\pgfusepath{clip}%
\pgfsetbuttcap%
\pgfsetroundjoin%
\definecolor{currentfill}{rgb}{0.200000,0.200000,0.800000}%
\pgfsetfillcolor{currentfill}%
\pgfsetlinewidth{1.003750pt}%
\definecolor{currentstroke}{rgb}{0.200000,0.200000,0.800000}%
\pgfsetstrokecolor{currentstroke}%
\pgfsetdash{}{0pt}%
\pgfpathmoveto{\pgfqpoint{5.430554in}{4.257774in}}%
\pgfpathcurveto{\pgfqpoint{5.436378in}{4.257774in}}{\pgfqpoint{5.441965in}{4.260088in}}{\pgfqpoint{5.446083in}{4.264206in}}%
\pgfpathcurveto{\pgfqpoint{5.450201in}{4.268325in}}{\pgfqpoint{5.452515in}{4.273911in}}{\pgfqpoint{5.452515in}{4.279735in}}%
\pgfpathcurveto{\pgfqpoint{5.452515in}{4.285559in}}{\pgfqpoint{5.450201in}{4.291145in}}{\pgfqpoint{5.446083in}{4.295263in}}%
\pgfpathcurveto{\pgfqpoint{5.441965in}{4.299381in}}{\pgfqpoint{5.436378in}{4.301695in}}{\pgfqpoint{5.430554in}{4.301695in}}%
\pgfpathcurveto{\pgfqpoint{5.424730in}{4.301695in}}{\pgfqpoint{5.419144in}{4.299381in}}{\pgfqpoint{5.415026in}{4.295263in}}%
\pgfpathcurveto{\pgfqpoint{5.410908in}{4.291145in}}{\pgfqpoint{5.408594in}{4.285559in}}{\pgfqpoint{5.408594in}{4.279735in}}%
\pgfpathcurveto{\pgfqpoint{5.408594in}{4.273911in}}{\pgfqpoint{5.410908in}{4.268325in}}{\pgfqpoint{5.415026in}{4.264206in}}%
\pgfpathcurveto{\pgfqpoint{5.419144in}{4.260088in}}{\pgfqpoint{5.424730in}{4.257774in}}{\pgfqpoint{5.430554in}{4.257774in}}%
\pgfpathlineto{\pgfqpoint{5.430554in}{4.257774in}}%
\pgfpathclose%
\pgfusepath{stroke,fill}%
\end{pgfscope}%
\begin{pgfscope}%
\pgfpathrectangle{\pgfqpoint{1.000000in}{1.148311in}}{\pgfqpoint{6.200000in}{5.623377in}}%
\pgfusepath{clip}%
\pgfsetbuttcap%
\pgfsetroundjoin%
\definecolor{currentfill}{rgb}{0.200000,0.200000,0.800000}%
\pgfsetfillcolor{currentfill}%
\pgfsetlinewidth{1.003750pt}%
\definecolor{currentstroke}{rgb}{0.200000,0.200000,0.800000}%
\pgfsetstrokecolor{currentstroke}%
\pgfsetdash{}{0pt}%
\pgfpathmoveto{\pgfqpoint{5.497935in}{4.260859in}}%
\pgfpathcurveto{\pgfqpoint{5.503759in}{4.260859in}}{\pgfqpoint{5.509345in}{4.263172in}}{\pgfqpoint{5.513463in}{4.267291in}}%
\pgfpathcurveto{\pgfqpoint{5.517581in}{4.271409in}}{\pgfqpoint{5.519895in}{4.276995in}}{\pgfqpoint{5.519895in}{4.282819in}}%
\pgfpathcurveto{\pgfqpoint{5.519895in}{4.288643in}}{\pgfqpoint{5.517581in}{4.294229in}}{\pgfqpoint{5.513463in}{4.298347in}}%
\pgfpathcurveto{\pgfqpoint{5.509345in}{4.302465in}}{\pgfqpoint{5.503759in}{4.304779in}}{\pgfqpoint{5.497935in}{4.304779in}}%
\pgfpathcurveto{\pgfqpoint{5.492111in}{4.304779in}}{\pgfqpoint{5.486525in}{4.302465in}}{\pgfqpoint{5.482407in}{4.298347in}}%
\pgfpathcurveto{\pgfqpoint{5.478289in}{4.294229in}}{\pgfqpoint{5.475975in}{4.288643in}}{\pgfqpoint{5.475975in}{4.282819in}}%
\pgfpathcurveto{\pgfqpoint{5.475975in}{4.276995in}}{\pgfqpoint{5.478289in}{4.271409in}}{\pgfqpoint{5.482407in}{4.267291in}}%
\pgfpathcurveto{\pgfqpoint{5.486525in}{4.263172in}}{\pgfqpoint{5.492111in}{4.260859in}}{\pgfqpoint{5.497935in}{4.260859in}}%
\pgfpathlineto{\pgfqpoint{5.497935in}{4.260859in}}%
\pgfpathclose%
\pgfusepath{stroke,fill}%
\end{pgfscope}%
\begin{pgfscope}%
\pgfpathrectangle{\pgfqpoint{1.000000in}{1.148311in}}{\pgfqpoint{6.200000in}{5.623377in}}%
\pgfusepath{clip}%
\pgfsetbuttcap%
\pgfsetroundjoin%
\definecolor{currentfill}{rgb}{0.200000,0.200000,0.800000}%
\pgfsetfillcolor{currentfill}%
\pgfsetlinewidth{1.003750pt}%
\definecolor{currentstroke}{rgb}{0.200000,0.200000,0.800000}%
\pgfsetstrokecolor{currentstroke}%
\pgfsetdash{}{0pt}%
\pgfpathmoveto{\pgfqpoint{5.570088in}{4.221295in}}%
\pgfpathcurveto{\pgfqpoint{5.575912in}{4.221295in}}{\pgfqpoint{5.581498in}{4.223608in}}{\pgfqpoint{5.585616in}{4.227727in}}%
\pgfpathcurveto{\pgfqpoint{5.589734in}{4.231845in}}{\pgfqpoint{5.592048in}{4.237431in}}{\pgfqpoint{5.592048in}{4.243255in}}%
\pgfpathcurveto{\pgfqpoint{5.592048in}{4.249079in}}{\pgfqpoint{5.589734in}{4.254665in}}{\pgfqpoint{5.585616in}{4.258783in}}%
\pgfpathcurveto{\pgfqpoint{5.581498in}{4.262901in}}{\pgfqpoint{5.575912in}{4.265215in}}{\pgfqpoint{5.570088in}{4.265215in}}%
\pgfpathcurveto{\pgfqpoint{5.564264in}{4.265215in}}{\pgfqpoint{5.558678in}{4.262901in}}{\pgfqpoint{5.554560in}{4.258783in}}%
\pgfpathcurveto{\pgfqpoint{5.550442in}{4.254665in}}{\pgfqpoint{5.548128in}{4.249079in}}{\pgfqpoint{5.548128in}{4.243255in}}%
\pgfpathcurveto{\pgfqpoint{5.548128in}{4.237431in}}{\pgfqpoint{5.550442in}{4.231845in}}{\pgfqpoint{5.554560in}{4.227727in}}%
\pgfpathcurveto{\pgfqpoint{5.558678in}{4.223608in}}{\pgfqpoint{5.564264in}{4.221295in}}{\pgfqpoint{5.570088in}{4.221295in}}%
\pgfpathlineto{\pgfqpoint{5.570088in}{4.221295in}}%
\pgfpathclose%
\pgfusepath{stroke,fill}%
\end{pgfscope}%
\begin{pgfscope}%
\pgfpathrectangle{\pgfqpoint{1.000000in}{1.148311in}}{\pgfqpoint{6.200000in}{5.623377in}}%
\pgfusepath{clip}%
\pgfsetbuttcap%
\pgfsetroundjoin%
\definecolor{currentfill}{rgb}{0.200000,0.200000,0.800000}%
\pgfsetfillcolor{currentfill}%
\pgfsetlinewidth{1.003750pt}%
\definecolor{currentstroke}{rgb}{0.200000,0.200000,0.800000}%
\pgfsetstrokecolor{currentstroke}%
\pgfsetdash{}{0pt}%
\pgfpathmoveto{\pgfqpoint{5.632345in}{4.272418in}}%
\pgfpathcurveto{\pgfqpoint{5.638169in}{4.272418in}}{\pgfqpoint{5.643755in}{4.274732in}}{\pgfqpoint{5.647873in}{4.278850in}}%
\pgfpathcurveto{\pgfqpoint{5.651992in}{4.282968in}}{\pgfqpoint{5.654305in}{4.288555in}}{\pgfqpoint{5.654305in}{4.294379in}}%
\pgfpathcurveto{\pgfqpoint{5.654305in}{4.300202in}}{\pgfqpoint{5.651992in}{4.305789in}}{\pgfqpoint{5.647873in}{4.309907in}}%
\pgfpathcurveto{\pgfqpoint{5.643755in}{4.314025in}}{\pgfqpoint{5.638169in}{4.316339in}}{\pgfqpoint{5.632345in}{4.316339in}}%
\pgfpathcurveto{\pgfqpoint{5.626521in}{4.316339in}}{\pgfqpoint{5.620935in}{4.314025in}}{\pgfqpoint{5.616817in}{4.309907in}}%
\pgfpathcurveto{\pgfqpoint{5.612699in}{4.305789in}}{\pgfqpoint{5.610385in}{4.300202in}}{\pgfqpoint{5.610385in}{4.294379in}}%
\pgfpathcurveto{\pgfqpoint{5.610385in}{4.288555in}}{\pgfqpoint{5.612699in}{4.282968in}}{\pgfqpoint{5.616817in}{4.278850in}}%
\pgfpathcurveto{\pgfqpoint{5.620935in}{4.274732in}}{\pgfqpoint{5.626521in}{4.272418in}}{\pgfqpoint{5.632345in}{4.272418in}}%
\pgfpathlineto{\pgfqpoint{5.632345in}{4.272418in}}%
\pgfpathclose%
\pgfusepath{stroke,fill}%
\end{pgfscope}%
\begin{pgfscope}%
\pgfpathrectangle{\pgfqpoint{1.000000in}{1.148311in}}{\pgfqpoint{6.200000in}{5.623377in}}%
\pgfusepath{clip}%
\pgfsetbuttcap%
\pgfsetroundjoin%
\definecolor{currentfill}{rgb}{0.200000,0.200000,0.800000}%
\pgfsetfillcolor{currentfill}%
\pgfsetlinewidth{1.003750pt}%
\definecolor{currentstroke}{rgb}{0.200000,0.200000,0.800000}%
\pgfsetstrokecolor{currentstroke}%
\pgfsetdash{}{0pt}%
\pgfpathmoveto{\pgfqpoint{5.698416in}{4.286550in}}%
\pgfpathcurveto{\pgfqpoint{5.704240in}{4.286550in}}{\pgfqpoint{5.709826in}{4.288863in}}{\pgfqpoint{5.713944in}{4.292982in}}%
\pgfpathcurveto{\pgfqpoint{5.718062in}{4.297100in}}{\pgfqpoint{5.720376in}{4.302686in}}{\pgfqpoint{5.720376in}{4.308510in}}%
\pgfpathcurveto{\pgfqpoint{5.720376in}{4.314334in}}{\pgfqpoint{5.718062in}{4.319920in}}{\pgfqpoint{5.713944in}{4.324038in}}%
\pgfpathcurveto{\pgfqpoint{5.709826in}{4.328156in}}{\pgfqpoint{5.704240in}{4.330470in}}{\pgfqpoint{5.698416in}{4.330470in}}%
\pgfpathcurveto{\pgfqpoint{5.692592in}{4.330470in}}{\pgfqpoint{5.687006in}{4.328156in}}{\pgfqpoint{5.682888in}{4.324038in}}%
\pgfpathcurveto{\pgfqpoint{5.678769in}{4.319920in}}{\pgfqpoint{5.676456in}{4.314334in}}{\pgfqpoint{5.676456in}{4.308510in}}%
\pgfpathcurveto{\pgfqpoint{5.676456in}{4.302686in}}{\pgfqpoint{5.678769in}{4.297100in}}{\pgfqpoint{5.682888in}{4.292982in}}%
\pgfpathcurveto{\pgfqpoint{5.687006in}{4.288863in}}{\pgfqpoint{5.692592in}{4.286550in}}{\pgfqpoint{5.698416in}{4.286550in}}%
\pgfpathlineto{\pgfqpoint{5.698416in}{4.286550in}}%
\pgfpathclose%
\pgfusepath{stroke,fill}%
\end{pgfscope}%
\begin{pgfscope}%
\pgfpathrectangle{\pgfqpoint{1.000000in}{1.148311in}}{\pgfqpoint{6.200000in}{5.623377in}}%
\pgfusepath{clip}%
\pgfsetbuttcap%
\pgfsetroundjoin%
\definecolor{currentfill}{rgb}{0.200000,0.200000,0.800000}%
\pgfsetfillcolor{currentfill}%
\pgfsetlinewidth{1.003750pt}%
\definecolor{currentstroke}{rgb}{0.200000,0.200000,0.800000}%
\pgfsetstrokecolor{currentstroke}%
\pgfsetdash{}{0pt}%
\pgfpathmoveto{\pgfqpoint{5.772507in}{4.275674in}}%
\pgfpathcurveto{\pgfqpoint{5.778331in}{4.275674in}}{\pgfqpoint{5.783917in}{4.277988in}}{\pgfqpoint{5.788035in}{4.282106in}}%
\pgfpathcurveto{\pgfqpoint{5.792154in}{4.286224in}}{\pgfqpoint{5.794467in}{4.291810in}}{\pgfqpoint{5.794467in}{4.297634in}}%
\pgfpathcurveto{\pgfqpoint{5.794467in}{4.303458in}}{\pgfqpoint{5.792154in}{4.309045in}}{\pgfqpoint{5.788035in}{4.313163in}}%
\pgfpathcurveto{\pgfqpoint{5.783917in}{4.317281in}}{\pgfqpoint{5.778331in}{4.319595in}}{\pgfqpoint{5.772507in}{4.319595in}}%
\pgfpathcurveto{\pgfqpoint{5.766683in}{4.319595in}}{\pgfqpoint{5.761097in}{4.317281in}}{\pgfqpoint{5.756979in}{4.313163in}}%
\pgfpathcurveto{\pgfqpoint{5.752861in}{4.309045in}}{\pgfqpoint{5.750547in}{4.303458in}}{\pgfqpoint{5.750547in}{4.297634in}}%
\pgfpathcurveto{\pgfqpoint{5.750547in}{4.291810in}}{\pgfqpoint{5.752861in}{4.286224in}}{\pgfqpoint{5.756979in}{4.282106in}}%
\pgfpathcurveto{\pgfqpoint{5.761097in}{4.277988in}}{\pgfqpoint{5.766683in}{4.275674in}}{\pgfqpoint{5.772507in}{4.275674in}}%
\pgfpathlineto{\pgfqpoint{5.772507in}{4.275674in}}%
\pgfpathclose%
\pgfusepath{stroke,fill}%
\end{pgfscope}%
\begin{pgfscope}%
\pgfpathrectangle{\pgfqpoint{1.000000in}{1.148311in}}{\pgfqpoint{6.200000in}{5.623377in}}%
\pgfusepath{clip}%
\pgfsetbuttcap%
\pgfsetroundjoin%
\definecolor{currentfill}{rgb}{0.200000,0.200000,0.800000}%
\pgfsetfillcolor{currentfill}%
\pgfsetlinewidth{1.003750pt}%
\definecolor{currentstroke}{rgb}{0.200000,0.200000,0.800000}%
\pgfsetstrokecolor{currentstroke}%
\pgfsetdash{}{0pt}%
\pgfpathmoveto{\pgfqpoint{5.837100in}{4.301173in}}%
\pgfpathcurveto{\pgfqpoint{5.842924in}{4.301173in}}{\pgfqpoint{5.848510in}{4.303486in}}{\pgfqpoint{5.852628in}{4.307605in}}%
\pgfpathcurveto{\pgfqpoint{5.856746in}{4.311723in}}{\pgfqpoint{5.859060in}{4.317309in}}{\pgfqpoint{5.859060in}{4.323133in}}%
\pgfpathcurveto{\pgfqpoint{5.859060in}{4.328957in}}{\pgfqpoint{5.856746in}{4.334543in}}{\pgfqpoint{5.852628in}{4.338661in}}%
\pgfpathcurveto{\pgfqpoint{5.848510in}{4.342779in}}{\pgfqpoint{5.842924in}{4.345093in}}{\pgfqpoint{5.837100in}{4.345093in}}%
\pgfpathcurveto{\pgfqpoint{5.831276in}{4.345093in}}{\pgfqpoint{5.825689in}{4.342779in}}{\pgfqpoint{5.821571in}{4.338661in}}%
\pgfpathcurveto{\pgfqpoint{5.817453in}{4.334543in}}{\pgfqpoint{5.815139in}{4.328957in}}{\pgfqpoint{5.815139in}{4.323133in}}%
\pgfpathcurveto{\pgfqpoint{5.815139in}{4.317309in}}{\pgfqpoint{5.817453in}{4.311723in}}{\pgfqpoint{5.821571in}{4.307605in}}%
\pgfpathcurveto{\pgfqpoint{5.825689in}{4.303486in}}{\pgfqpoint{5.831276in}{4.301173in}}{\pgfqpoint{5.837100in}{4.301173in}}%
\pgfpathlineto{\pgfqpoint{5.837100in}{4.301173in}}%
\pgfpathclose%
\pgfusepath{stroke,fill}%
\end{pgfscope}%
\begin{pgfscope}%
\pgfpathrectangle{\pgfqpoint{1.000000in}{1.148311in}}{\pgfqpoint{6.200000in}{5.623377in}}%
\pgfusepath{clip}%
\pgfsetbuttcap%
\pgfsetroundjoin%
\definecolor{currentfill}{rgb}{0.200000,0.200000,0.800000}%
\pgfsetfillcolor{currentfill}%
\pgfsetlinewidth{1.003750pt}%
\definecolor{currentstroke}{rgb}{0.200000,0.200000,0.800000}%
\pgfsetstrokecolor{currentstroke}%
\pgfsetdash{}{0pt}%
\pgfpathmoveto{\pgfqpoint{5.893199in}{4.345034in}}%
\pgfpathcurveto{\pgfqpoint{5.899023in}{4.345034in}}{\pgfqpoint{5.904609in}{4.347348in}}{\pgfqpoint{5.908727in}{4.351466in}}%
\pgfpathcurveto{\pgfqpoint{5.912845in}{4.355584in}}{\pgfqpoint{5.915159in}{4.361171in}}{\pgfqpoint{5.915159in}{4.366995in}}%
\pgfpathcurveto{\pgfqpoint{5.915159in}{4.372819in}}{\pgfqpoint{5.912845in}{4.378405in}}{\pgfqpoint{5.908727in}{4.382523in}}%
\pgfpathcurveto{\pgfqpoint{5.904609in}{4.386641in}}{\pgfqpoint{5.899023in}{4.388955in}}{\pgfqpoint{5.893199in}{4.388955in}}%
\pgfpathcurveto{\pgfqpoint{5.887375in}{4.388955in}}{\pgfqpoint{5.881789in}{4.386641in}}{\pgfqpoint{5.877670in}{4.382523in}}%
\pgfpathcurveto{\pgfqpoint{5.873552in}{4.378405in}}{\pgfqpoint{5.871238in}{4.372819in}}{\pgfqpoint{5.871238in}{4.366995in}}%
\pgfpathcurveto{\pgfqpoint{5.871238in}{4.361171in}}{\pgfqpoint{5.873552in}{4.355584in}}{\pgfqpoint{5.877670in}{4.351466in}}%
\pgfpathcurveto{\pgfqpoint{5.881789in}{4.347348in}}{\pgfqpoint{5.887375in}{4.345034in}}{\pgfqpoint{5.893199in}{4.345034in}}%
\pgfpathlineto{\pgfqpoint{5.893199in}{4.345034in}}%
\pgfpathclose%
\pgfusepath{stroke,fill}%
\end{pgfscope}%
\begin{pgfscope}%
\pgfpathrectangle{\pgfqpoint{1.000000in}{1.148311in}}{\pgfqpoint{6.200000in}{5.623377in}}%
\pgfusepath{clip}%
\pgfsetbuttcap%
\pgfsetroundjoin%
\definecolor{currentfill}{rgb}{0.200000,0.200000,0.800000}%
\pgfsetfillcolor{currentfill}%
\pgfsetlinewidth{1.003750pt}%
\definecolor{currentstroke}{rgb}{0.200000,0.200000,0.800000}%
\pgfsetstrokecolor{currentstroke}%
\pgfsetdash{}{0pt}%
\pgfpathmoveto{\pgfqpoint{5.982184in}{4.323060in}}%
\pgfpathcurveto{\pgfqpoint{5.988008in}{4.323060in}}{\pgfqpoint{5.993594in}{4.325374in}}{\pgfqpoint{5.997712in}{4.329492in}}%
\pgfpathcurveto{\pgfqpoint{6.001830in}{4.333610in}}{\pgfqpoint{6.004144in}{4.339197in}}{\pgfqpoint{6.004144in}{4.345021in}}%
\pgfpathcurveto{\pgfqpoint{6.004144in}{4.350845in}}{\pgfqpoint{6.001830in}{4.356431in}}{\pgfqpoint{5.997712in}{4.360549in}}%
\pgfpathcurveto{\pgfqpoint{5.993594in}{4.364667in}}{\pgfqpoint{5.988008in}{4.366981in}}{\pgfqpoint{5.982184in}{4.366981in}}%
\pgfpathcurveto{\pgfqpoint{5.976360in}{4.366981in}}{\pgfqpoint{5.970774in}{4.364667in}}{\pgfqpoint{5.966656in}{4.360549in}}%
\pgfpathcurveto{\pgfqpoint{5.962537in}{4.356431in}}{\pgfqpoint{5.960224in}{4.350845in}}{\pgfqpoint{5.960224in}{4.345021in}}%
\pgfpathcurveto{\pgfqpoint{5.960224in}{4.339197in}}{\pgfqpoint{5.962537in}{4.333610in}}{\pgfqpoint{5.966656in}{4.329492in}}%
\pgfpathcurveto{\pgfqpoint{5.970774in}{4.325374in}}{\pgfqpoint{5.976360in}{4.323060in}}{\pgfqpoint{5.982184in}{4.323060in}}%
\pgfpathlineto{\pgfqpoint{5.982184in}{4.323060in}}%
\pgfpathclose%
\pgfusepath{stroke,fill}%
\end{pgfscope}%
\begin{pgfscope}%
\pgfpathrectangle{\pgfqpoint{1.000000in}{1.148311in}}{\pgfqpoint{6.200000in}{5.623377in}}%
\pgfusepath{clip}%
\pgfsetbuttcap%
\pgfsetroundjoin%
\definecolor{currentfill}{rgb}{0.200000,0.200000,0.800000}%
\pgfsetfillcolor{currentfill}%
\pgfsetlinewidth{1.003750pt}%
\definecolor{currentstroke}{rgb}{0.200000,0.200000,0.800000}%
\pgfsetstrokecolor{currentstroke}%
\pgfsetdash{}{0pt}%
\pgfpathmoveto{\pgfqpoint{6.032756in}{4.377768in}}%
\pgfpathcurveto{\pgfqpoint{6.038579in}{4.377768in}}{\pgfqpoint{6.044166in}{4.380082in}}{\pgfqpoint{6.048284in}{4.384200in}}%
\pgfpathcurveto{\pgfqpoint{6.052402in}{4.388318in}}{\pgfqpoint{6.054716in}{4.393904in}}{\pgfqpoint{6.054716in}{4.399728in}}%
\pgfpathcurveto{\pgfqpoint{6.054716in}{4.405552in}}{\pgfqpoint{6.052402in}{4.411138in}}{\pgfqpoint{6.048284in}{4.415257in}}%
\pgfpathcurveto{\pgfqpoint{6.044166in}{4.419375in}}{\pgfqpoint{6.038579in}{4.421689in}}{\pgfqpoint{6.032756in}{4.421689in}}%
\pgfpathcurveto{\pgfqpoint{6.026932in}{4.421689in}}{\pgfqpoint{6.021345in}{4.419375in}}{\pgfqpoint{6.017227in}{4.415257in}}%
\pgfpathcurveto{\pgfqpoint{6.013109in}{4.411138in}}{\pgfqpoint{6.010795in}{4.405552in}}{\pgfqpoint{6.010795in}{4.399728in}}%
\pgfpathcurveto{\pgfqpoint{6.010795in}{4.393904in}}{\pgfqpoint{6.013109in}{4.388318in}}{\pgfqpoint{6.017227in}{4.384200in}}%
\pgfpathcurveto{\pgfqpoint{6.021345in}{4.380082in}}{\pgfqpoint{6.026932in}{4.377768in}}{\pgfqpoint{6.032756in}{4.377768in}}%
\pgfpathlineto{\pgfqpoint{6.032756in}{4.377768in}}%
\pgfpathclose%
\pgfusepath{stroke,fill}%
\end{pgfscope}%
\begin{pgfscope}%
\pgfpathrectangle{\pgfqpoint{1.000000in}{1.148311in}}{\pgfqpoint{6.200000in}{5.623377in}}%
\pgfusepath{clip}%
\pgfsetbuttcap%
\pgfsetroundjoin%
\definecolor{currentfill}{rgb}{0.200000,0.200000,0.800000}%
\pgfsetfillcolor{currentfill}%
\pgfsetlinewidth{1.003750pt}%
\definecolor{currentstroke}{rgb}{0.200000,0.200000,0.800000}%
\pgfsetstrokecolor{currentstroke}%
\pgfsetdash{}{0pt}%
\pgfpathmoveto{\pgfqpoint{6.126404in}{4.367648in}}%
\pgfpathcurveto{\pgfqpoint{6.132228in}{4.367648in}}{\pgfqpoint{6.137814in}{4.369961in}}{\pgfqpoint{6.141932in}{4.374080in}}%
\pgfpathcurveto{\pgfqpoint{6.146051in}{4.378198in}}{\pgfqpoint{6.148364in}{4.383784in}}{\pgfqpoint{6.148364in}{4.389608in}}%
\pgfpathcurveto{\pgfqpoint{6.148364in}{4.395432in}}{\pgfqpoint{6.146051in}{4.401018in}}{\pgfqpoint{6.141932in}{4.405136in}}%
\pgfpathcurveto{\pgfqpoint{6.137814in}{4.409254in}}{\pgfqpoint{6.132228in}{4.411568in}}{\pgfqpoint{6.126404in}{4.411568in}}%
\pgfpathcurveto{\pgfqpoint{6.120580in}{4.411568in}}{\pgfqpoint{6.114994in}{4.409254in}}{\pgfqpoint{6.110876in}{4.405136in}}%
\pgfpathcurveto{\pgfqpoint{6.106758in}{4.401018in}}{\pgfqpoint{6.104444in}{4.395432in}}{\pgfqpoint{6.104444in}{4.389608in}}%
\pgfpathcurveto{\pgfqpoint{6.104444in}{4.383784in}}{\pgfqpoint{6.106758in}{4.378198in}}{\pgfqpoint{6.110876in}{4.374080in}}%
\pgfpathcurveto{\pgfqpoint{6.114994in}{4.369961in}}{\pgfqpoint{6.120580in}{4.367648in}}{\pgfqpoint{6.126404in}{4.367648in}}%
\pgfpathlineto{\pgfqpoint{6.126404in}{4.367648in}}%
\pgfpathclose%
\pgfusepath{stroke,fill}%
\end{pgfscope}%
\begin{pgfscope}%
\pgfpathrectangle{\pgfqpoint{1.000000in}{1.148311in}}{\pgfqpoint{6.200000in}{5.623377in}}%
\pgfusepath{clip}%
\pgfsetbuttcap%
\pgfsetroundjoin%
\definecolor{currentfill}{rgb}{0.200000,0.200000,0.800000}%
\pgfsetfillcolor{currentfill}%
\pgfsetlinewidth{1.003750pt}%
\definecolor{currentstroke}{rgb}{0.200000,0.200000,0.800000}%
\pgfsetstrokecolor{currentstroke}%
\pgfsetdash{}{0pt}%
\pgfpathmoveto{\pgfqpoint{6.128945in}{4.482258in}}%
\pgfpathcurveto{\pgfqpoint{6.134769in}{4.482258in}}{\pgfqpoint{6.140355in}{4.484572in}}{\pgfqpoint{6.144473in}{4.488690in}}%
\pgfpathcurveto{\pgfqpoint{6.148591in}{4.492808in}}{\pgfqpoint{6.150905in}{4.498395in}}{\pgfqpoint{6.150905in}{4.504219in}}%
\pgfpathcurveto{\pgfqpoint{6.150905in}{4.510042in}}{\pgfqpoint{6.148591in}{4.515629in}}{\pgfqpoint{6.144473in}{4.519747in}}%
\pgfpathcurveto{\pgfqpoint{6.140355in}{4.523865in}}{\pgfqpoint{6.134769in}{4.526179in}}{\pgfqpoint{6.128945in}{4.526179in}}%
\pgfpathcurveto{\pgfqpoint{6.123121in}{4.526179in}}{\pgfqpoint{6.117535in}{4.523865in}}{\pgfqpoint{6.113416in}{4.519747in}}%
\pgfpathcurveto{\pgfqpoint{6.109298in}{4.515629in}}{\pgfqpoint{6.106984in}{4.510042in}}{\pgfqpoint{6.106984in}{4.504219in}}%
\pgfpathcurveto{\pgfqpoint{6.106984in}{4.498395in}}{\pgfqpoint{6.109298in}{4.492808in}}{\pgfqpoint{6.113416in}{4.488690in}}%
\pgfpathcurveto{\pgfqpoint{6.117535in}{4.484572in}}{\pgfqpoint{6.123121in}{4.482258in}}{\pgfqpoint{6.128945in}{4.482258in}}%
\pgfpathlineto{\pgfqpoint{6.128945in}{4.482258in}}%
\pgfpathclose%
\pgfusepath{stroke,fill}%
\end{pgfscope}%
\begin{pgfscope}%
\pgfpathrectangle{\pgfqpoint{1.000000in}{1.148311in}}{\pgfqpoint{6.200000in}{5.623377in}}%
\pgfusepath{clip}%
\pgfsetbuttcap%
\pgfsetroundjoin%
\definecolor{currentfill}{rgb}{0.200000,0.200000,0.800000}%
\pgfsetfillcolor{currentfill}%
\pgfsetlinewidth{1.003750pt}%
\definecolor{currentstroke}{rgb}{0.200000,0.200000,0.800000}%
\pgfsetstrokecolor{currentstroke}%
\pgfsetdash{}{0pt}%
\pgfpathmoveto{\pgfqpoint{6.206390in}{4.499442in}}%
\pgfpathcurveto{\pgfqpoint{6.212214in}{4.499442in}}{\pgfqpoint{6.217800in}{4.501756in}}{\pgfqpoint{6.221918in}{4.505874in}}%
\pgfpathcurveto{\pgfqpoint{6.226036in}{4.509993in}}{\pgfqpoint{6.228350in}{4.515579in}}{\pgfqpoint{6.228350in}{4.521403in}}%
\pgfpathcurveto{\pgfqpoint{6.228350in}{4.527227in}}{\pgfqpoint{6.226036in}{4.532813in}}{\pgfqpoint{6.221918in}{4.536931in}}%
\pgfpathcurveto{\pgfqpoint{6.217800in}{4.541049in}}{\pgfqpoint{6.212214in}{4.543363in}}{\pgfqpoint{6.206390in}{4.543363in}}%
\pgfpathcurveto{\pgfqpoint{6.200566in}{4.543363in}}{\pgfqpoint{6.194980in}{4.541049in}}{\pgfqpoint{6.190861in}{4.536931in}}%
\pgfpathcurveto{\pgfqpoint{6.186743in}{4.532813in}}{\pgfqpoint{6.184429in}{4.527227in}}{\pgfqpoint{6.184429in}{4.521403in}}%
\pgfpathcurveto{\pgfqpoint{6.184429in}{4.515579in}}{\pgfqpoint{6.186743in}{4.509993in}}{\pgfqpoint{6.190861in}{4.505874in}}%
\pgfpathcurveto{\pgfqpoint{6.194980in}{4.501756in}}{\pgfqpoint{6.200566in}{4.499442in}}{\pgfqpoint{6.206390in}{4.499442in}}%
\pgfpathlineto{\pgfqpoint{6.206390in}{4.499442in}}%
\pgfpathclose%
\pgfusepath{stroke,fill}%
\end{pgfscope}%
\begin{pgfscope}%
\pgfpathrectangle{\pgfqpoint{1.000000in}{1.148311in}}{\pgfqpoint{6.200000in}{5.623377in}}%
\pgfusepath{clip}%
\pgfsetbuttcap%
\pgfsetroundjoin%
\definecolor{currentfill}{rgb}{0.200000,0.200000,0.800000}%
\pgfsetfillcolor{currentfill}%
\pgfsetlinewidth{1.003750pt}%
\definecolor{currentstroke}{rgb}{0.200000,0.200000,0.800000}%
\pgfsetstrokecolor{currentstroke}%
\pgfsetdash{}{0pt}%
\pgfpathmoveto{\pgfqpoint{6.278530in}{4.528682in}}%
\pgfpathcurveto{\pgfqpoint{6.284354in}{4.528682in}}{\pgfqpoint{6.289940in}{4.530996in}}{\pgfqpoint{6.294058in}{4.535114in}}%
\pgfpathcurveto{\pgfqpoint{6.298177in}{4.539232in}}{\pgfqpoint{6.300490in}{4.544818in}}{\pgfqpoint{6.300490in}{4.550642in}}%
\pgfpathcurveto{\pgfqpoint{6.300490in}{4.556466in}}{\pgfqpoint{6.298177in}{4.562052in}}{\pgfqpoint{6.294058in}{4.566170in}}%
\pgfpathcurveto{\pgfqpoint{6.289940in}{4.570289in}}{\pgfqpoint{6.284354in}{4.572602in}}{\pgfqpoint{6.278530in}{4.572602in}}%
\pgfpathcurveto{\pgfqpoint{6.272706in}{4.572602in}}{\pgfqpoint{6.267120in}{4.570289in}}{\pgfqpoint{6.263002in}{4.566170in}}%
\pgfpathcurveto{\pgfqpoint{6.258884in}{4.562052in}}{\pgfqpoint{6.256570in}{4.556466in}}{\pgfqpoint{6.256570in}{4.550642in}}%
\pgfpathcurveto{\pgfqpoint{6.256570in}{4.544818in}}{\pgfqpoint{6.258884in}{4.539232in}}{\pgfqpoint{6.263002in}{4.535114in}}%
\pgfpathcurveto{\pgfqpoint{6.267120in}{4.530996in}}{\pgfqpoint{6.272706in}{4.528682in}}{\pgfqpoint{6.278530in}{4.528682in}}%
\pgfpathlineto{\pgfqpoint{6.278530in}{4.528682in}}%
\pgfpathclose%
\pgfusepath{stroke,fill}%
\end{pgfscope}%
\begin{pgfscope}%
\pgfpathrectangle{\pgfqpoint{1.000000in}{1.148311in}}{\pgfqpoint{6.200000in}{5.623377in}}%
\pgfusepath{clip}%
\pgfsetbuttcap%
\pgfsetroundjoin%
\definecolor{currentfill}{rgb}{0.200000,0.200000,0.800000}%
\pgfsetfillcolor{currentfill}%
\pgfsetlinewidth{1.003750pt}%
\definecolor{currentstroke}{rgb}{0.200000,0.200000,0.800000}%
\pgfsetstrokecolor{currentstroke}%
\pgfsetdash{}{0pt}%
\pgfpathmoveto{\pgfqpoint{6.255841in}{4.642798in}}%
\pgfpathcurveto{\pgfqpoint{6.261665in}{4.642798in}}{\pgfqpoint{6.267251in}{4.645111in}}{\pgfqpoint{6.271369in}{4.649230in}}%
\pgfpathcurveto{\pgfqpoint{6.275487in}{4.653348in}}{\pgfqpoint{6.277801in}{4.658934in}}{\pgfqpoint{6.277801in}{4.664758in}}%
\pgfpathcurveto{\pgfqpoint{6.277801in}{4.670582in}}{\pgfqpoint{6.275487in}{4.676168in}}{\pgfqpoint{6.271369in}{4.680286in}}%
\pgfpathcurveto{\pgfqpoint{6.267251in}{4.684404in}}{\pgfqpoint{6.261665in}{4.686718in}}{\pgfqpoint{6.255841in}{4.686718in}}%
\pgfpathcurveto{\pgfqpoint{6.250017in}{4.686718in}}{\pgfqpoint{6.244431in}{4.684404in}}{\pgfqpoint{6.240312in}{4.680286in}}%
\pgfpathcurveto{\pgfqpoint{6.236194in}{4.676168in}}{\pgfqpoint{6.233880in}{4.670582in}}{\pgfqpoint{6.233880in}{4.664758in}}%
\pgfpathcurveto{\pgfqpoint{6.233880in}{4.658934in}}{\pgfqpoint{6.236194in}{4.653348in}}{\pgfqpoint{6.240312in}{4.649230in}}%
\pgfpathcurveto{\pgfqpoint{6.244431in}{4.645111in}}{\pgfqpoint{6.250017in}{4.642798in}}{\pgfqpoint{6.255841in}{4.642798in}}%
\pgfpathlineto{\pgfqpoint{6.255841in}{4.642798in}}%
\pgfpathclose%
\pgfusepath{stroke,fill}%
\end{pgfscope}%
\begin{pgfscope}%
\pgfpathrectangle{\pgfqpoint{1.000000in}{1.148311in}}{\pgfqpoint{6.200000in}{5.623377in}}%
\pgfusepath{clip}%
\pgfsetbuttcap%
\pgfsetroundjoin%
\definecolor{currentfill}{rgb}{0.200000,0.200000,0.800000}%
\pgfsetfillcolor{currentfill}%
\pgfsetlinewidth{1.003750pt}%
\definecolor{currentstroke}{rgb}{0.200000,0.200000,0.800000}%
\pgfsetstrokecolor{currentstroke}%
\pgfsetdash{}{0pt}%
\pgfpathmoveto{\pgfqpoint{6.314273in}{4.682892in}}%
\pgfpathcurveto{\pgfqpoint{6.320097in}{4.682892in}}{\pgfqpoint{6.325683in}{4.685206in}}{\pgfqpoint{6.329801in}{4.689324in}}%
\pgfpathcurveto{\pgfqpoint{6.333919in}{4.693442in}}{\pgfqpoint{6.336233in}{4.699028in}}{\pgfqpoint{6.336233in}{4.704852in}}%
\pgfpathcurveto{\pgfqpoint{6.336233in}{4.710676in}}{\pgfqpoint{6.333919in}{4.716262in}}{\pgfqpoint{6.329801in}{4.720381in}}%
\pgfpathcurveto{\pgfqpoint{6.325683in}{4.724499in}}{\pgfqpoint{6.320097in}{4.726813in}}{\pgfqpoint{6.314273in}{4.726813in}}%
\pgfpathcurveto{\pgfqpoint{6.308449in}{4.726813in}}{\pgfqpoint{6.302863in}{4.724499in}}{\pgfqpoint{6.298745in}{4.720381in}}%
\pgfpathcurveto{\pgfqpoint{6.294626in}{4.716262in}}{\pgfqpoint{6.292313in}{4.710676in}}{\pgfqpoint{6.292313in}{4.704852in}}%
\pgfpathcurveto{\pgfqpoint{6.292313in}{4.699028in}}{\pgfqpoint{6.294626in}{4.693442in}}{\pgfqpoint{6.298745in}{4.689324in}}%
\pgfpathcurveto{\pgfqpoint{6.302863in}{4.685206in}}{\pgfqpoint{6.308449in}{4.682892in}}{\pgfqpoint{6.314273in}{4.682892in}}%
\pgfpathlineto{\pgfqpoint{6.314273in}{4.682892in}}%
\pgfpathclose%
\pgfusepath{stroke,fill}%
\end{pgfscope}%
\begin{pgfscope}%
\pgfpathrectangle{\pgfqpoint{1.000000in}{1.148311in}}{\pgfqpoint{6.200000in}{5.623377in}}%
\pgfusepath{clip}%
\pgfsetbuttcap%
\pgfsetroundjoin%
\definecolor{currentfill}{rgb}{0.200000,0.200000,0.800000}%
\pgfsetfillcolor{currentfill}%
\pgfsetlinewidth{1.003750pt}%
\definecolor{currentstroke}{rgb}{0.200000,0.200000,0.800000}%
\pgfsetstrokecolor{currentstroke}%
\pgfsetdash{}{0pt}%
\pgfpathmoveto{\pgfqpoint{6.405777in}{4.705243in}}%
\pgfpathcurveto{\pgfqpoint{6.411601in}{4.705243in}}{\pgfqpoint{6.417188in}{4.707556in}}{\pgfqpoint{6.421306in}{4.711675in}}%
\pgfpathcurveto{\pgfqpoint{6.425424in}{4.715793in}}{\pgfqpoint{6.427738in}{4.721379in}}{\pgfqpoint{6.427738in}{4.727203in}}%
\pgfpathcurveto{\pgfqpoint{6.427738in}{4.733027in}}{\pgfqpoint{6.425424in}{4.738613in}}{\pgfqpoint{6.421306in}{4.742731in}}%
\pgfpathcurveto{\pgfqpoint{6.417188in}{4.746849in}}{\pgfqpoint{6.411601in}{4.749163in}}{\pgfqpoint{6.405777in}{4.749163in}}%
\pgfpathcurveto{\pgfqpoint{6.399954in}{4.749163in}}{\pgfqpoint{6.394367in}{4.746849in}}{\pgfqpoint{6.390249in}{4.742731in}}%
\pgfpathcurveto{\pgfqpoint{6.386131in}{4.738613in}}{\pgfqpoint{6.383817in}{4.733027in}}{\pgfqpoint{6.383817in}{4.727203in}}%
\pgfpathcurveto{\pgfqpoint{6.383817in}{4.721379in}}{\pgfqpoint{6.386131in}{4.715793in}}{\pgfqpoint{6.390249in}{4.711675in}}%
\pgfpathcurveto{\pgfqpoint{6.394367in}{4.707556in}}{\pgfqpoint{6.399954in}{4.705243in}}{\pgfqpoint{6.405777in}{4.705243in}}%
\pgfpathlineto{\pgfqpoint{6.405777in}{4.705243in}}%
\pgfpathclose%
\pgfusepath{stroke,fill}%
\end{pgfscope}%
\begin{pgfscope}%
\pgfpathrectangle{\pgfqpoint{1.000000in}{1.148311in}}{\pgfqpoint{6.200000in}{5.623377in}}%
\pgfusepath{clip}%
\pgfsetbuttcap%
\pgfsetroundjoin%
\definecolor{currentfill}{rgb}{0.200000,0.200000,0.800000}%
\pgfsetfillcolor{currentfill}%
\pgfsetlinewidth{1.003750pt}%
\definecolor{currentstroke}{rgb}{0.200000,0.200000,0.800000}%
\pgfsetstrokecolor{currentstroke}%
\pgfsetdash{}{0pt}%
\pgfpathmoveto{\pgfqpoint{6.423786in}{4.777906in}}%
\pgfpathcurveto{\pgfqpoint{6.429610in}{4.777906in}}{\pgfqpoint{6.435196in}{4.780219in}}{\pgfqpoint{6.439314in}{4.784338in}}%
\pgfpathcurveto{\pgfqpoint{6.443432in}{4.788456in}}{\pgfqpoint{6.445746in}{4.794042in}}{\pgfqpoint{6.445746in}{4.799866in}}%
\pgfpathcurveto{\pgfqpoint{6.445746in}{4.805690in}}{\pgfqpoint{6.443432in}{4.811276in}}{\pgfqpoint{6.439314in}{4.815394in}}%
\pgfpathcurveto{\pgfqpoint{6.435196in}{4.819512in}}{\pgfqpoint{6.429610in}{4.821826in}}{\pgfqpoint{6.423786in}{4.821826in}}%
\pgfpathcurveto{\pgfqpoint{6.417962in}{4.821826in}}{\pgfqpoint{6.412376in}{4.819512in}}{\pgfqpoint{6.408257in}{4.815394in}}%
\pgfpathcurveto{\pgfqpoint{6.404139in}{4.811276in}}{\pgfqpoint{6.401825in}{4.805690in}}{\pgfqpoint{6.401825in}{4.799866in}}%
\pgfpathcurveto{\pgfqpoint{6.401825in}{4.794042in}}{\pgfqpoint{6.404139in}{4.788456in}}{\pgfqpoint{6.408257in}{4.784338in}}%
\pgfpathcurveto{\pgfqpoint{6.412376in}{4.780219in}}{\pgfqpoint{6.417962in}{4.777906in}}{\pgfqpoint{6.423786in}{4.777906in}}%
\pgfpathlineto{\pgfqpoint{6.423786in}{4.777906in}}%
\pgfpathclose%
\pgfusepath{stroke,fill}%
\end{pgfscope}%
\begin{pgfscope}%
\pgfpathrectangle{\pgfqpoint{1.000000in}{1.148311in}}{\pgfqpoint{6.200000in}{5.623377in}}%
\pgfusepath{clip}%
\pgfsetbuttcap%
\pgfsetroundjoin%
\definecolor{currentfill}{rgb}{0.800000,0.200000,0.200000}%
\pgfsetfillcolor{currentfill}%
\pgfsetlinewidth{1.003750pt}%
\definecolor{currentstroke}{rgb}{0.800000,0.200000,0.200000}%
\pgfsetstrokecolor{currentstroke}%
\pgfsetdash{}{0pt}%
\pgfpathmoveto{\pgfqpoint{6.431393in}{4.852767in}}%
\pgfpathcurveto{\pgfqpoint{6.437217in}{4.852767in}}{\pgfqpoint{6.442803in}{4.855081in}}{\pgfqpoint{6.446921in}{4.859199in}}%
\pgfpathcurveto{\pgfqpoint{6.451039in}{4.863317in}}{\pgfqpoint{6.453353in}{4.868903in}}{\pgfqpoint{6.453353in}{4.874727in}}%
\pgfpathcurveto{\pgfqpoint{6.453353in}{4.880551in}}{\pgfqpoint{6.451039in}{4.886137in}}{\pgfqpoint{6.446921in}{4.890256in}}%
\pgfpathcurveto{\pgfqpoint{6.442803in}{4.894374in}}{\pgfqpoint{6.437217in}{4.896688in}}{\pgfqpoint{6.431393in}{4.896688in}}%
\pgfpathcurveto{\pgfqpoint{6.425569in}{4.896688in}}{\pgfqpoint{6.419983in}{4.894374in}}{\pgfqpoint{6.415865in}{4.890256in}}%
\pgfpathcurveto{\pgfqpoint{6.411746in}{4.886137in}}{\pgfqpoint{6.409433in}{4.880551in}}{\pgfqpoint{6.409433in}{4.874727in}}%
\pgfpathcurveto{\pgfqpoint{6.409433in}{4.868903in}}{\pgfqpoint{6.411746in}{4.863317in}}{\pgfqpoint{6.415865in}{4.859199in}}%
\pgfpathcurveto{\pgfqpoint{6.419983in}{4.855081in}}{\pgfqpoint{6.425569in}{4.852767in}}{\pgfqpoint{6.431393in}{4.852767in}}%
\pgfpathlineto{\pgfqpoint{6.431393in}{4.852767in}}%
\pgfpathclose%
\pgfusepath{stroke,fill}%
\end{pgfscope}%
\begin{pgfscope}%
\pgfpathrectangle{\pgfqpoint{1.000000in}{1.148311in}}{\pgfqpoint{6.200000in}{5.623377in}}%
\pgfusepath{clip}%
\pgfsetbuttcap%
\pgfsetroundjoin%
\definecolor{currentfill}{rgb}{0.200000,0.200000,0.800000}%
\pgfsetfillcolor{currentfill}%
\pgfsetlinewidth{1.003750pt}%
\definecolor{currentstroke}{rgb}{0.200000,0.200000,0.800000}%
\pgfsetstrokecolor{currentstroke}%
\pgfsetdash{}{0pt}%
\pgfpathmoveto{\pgfqpoint{6.409190in}{4.936105in}}%
\pgfpathcurveto{\pgfqpoint{6.415013in}{4.936105in}}{\pgfqpoint{6.420600in}{4.938419in}}{\pgfqpoint{6.424718in}{4.942537in}}%
\pgfpathcurveto{\pgfqpoint{6.428836in}{4.946655in}}{\pgfqpoint{6.431150in}{4.952242in}}{\pgfqpoint{6.431150in}{4.958065in}}%
\pgfpathcurveto{\pgfqpoint{6.431150in}{4.963889in}}{\pgfqpoint{6.428836in}{4.969476in}}{\pgfqpoint{6.424718in}{4.973594in}}%
\pgfpathcurveto{\pgfqpoint{6.420600in}{4.977712in}}{\pgfqpoint{6.415013in}{4.980026in}}{\pgfqpoint{6.409190in}{4.980026in}}%
\pgfpathcurveto{\pgfqpoint{6.403366in}{4.980026in}}{\pgfqpoint{6.397779in}{4.977712in}}{\pgfqpoint{6.393661in}{4.973594in}}%
\pgfpathcurveto{\pgfqpoint{6.389543in}{4.969476in}}{\pgfqpoint{6.387229in}{4.963889in}}{\pgfqpoint{6.387229in}{4.958065in}}%
\pgfpathcurveto{\pgfqpoint{6.387229in}{4.952242in}}{\pgfqpoint{6.389543in}{4.946655in}}{\pgfqpoint{6.393661in}{4.942537in}}%
\pgfpathcurveto{\pgfqpoint{6.397779in}{4.938419in}}{\pgfqpoint{6.403366in}{4.936105in}}{\pgfqpoint{6.409190in}{4.936105in}}%
\pgfpathlineto{\pgfqpoint{6.409190in}{4.936105in}}%
\pgfpathclose%
\pgfusepath{stroke,fill}%
\end{pgfscope}%
\begin{pgfscope}%
\pgfpathrectangle{\pgfqpoint{1.000000in}{1.148311in}}{\pgfqpoint{6.200000in}{5.623377in}}%
\pgfusepath{clip}%
\pgfsetbuttcap%
\pgfsetroundjoin%
\definecolor{currentfill}{rgb}{0.200000,0.200000,0.800000}%
\pgfsetfillcolor{currentfill}%
\pgfsetlinewidth{1.003750pt}%
\definecolor{currentstroke}{rgb}{0.200000,0.200000,0.800000}%
\pgfsetstrokecolor{currentstroke}%
\pgfsetdash{}{0pt}%
\pgfpathmoveto{\pgfqpoint{6.434820in}{4.996846in}}%
\pgfpathcurveto{\pgfqpoint{6.440644in}{4.996846in}}{\pgfqpoint{6.446230in}{4.999160in}}{\pgfqpoint{6.450348in}{5.003278in}}%
\pgfpathcurveto{\pgfqpoint{6.454466in}{5.007396in}}{\pgfqpoint{6.456780in}{5.012982in}}{\pgfqpoint{6.456780in}{5.018806in}}%
\pgfpathcurveto{\pgfqpoint{6.456780in}{5.024630in}}{\pgfqpoint{6.454466in}{5.030216in}}{\pgfqpoint{6.450348in}{5.034334in}}%
\pgfpathcurveto{\pgfqpoint{6.446230in}{5.038452in}}{\pgfqpoint{6.440644in}{5.040766in}}{\pgfqpoint{6.434820in}{5.040766in}}%
\pgfpathcurveto{\pgfqpoint{6.428996in}{5.040766in}}{\pgfqpoint{6.423410in}{5.038452in}}{\pgfqpoint{6.419292in}{5.034334in}}%
\pgfpathcurveto{\pgfqpoint{6.415173in}{5.030216in}}{\pgfqpoint{6.412860in}{5.024630in}}{\pgfqpoint{6.412860in}{5.018806in}}%
\pgfpathcurveto{\pgfqpoint{6.412860in}{5.012982in}}{\pgfqpoint{6.415173in}{5.007396in}}{\pgfqpoint{6.419292in}{5.003278in}}%
\pgfpathcurveto{\pgfqpoint{6.423410in}{4.999160in}}{\pgfqpoint{6.428996in}{4.996846in}}{\pgfqpoint{6.434820in}{4.996846in}}%
\pgfpathlineto{\pgfqpoint{6.434820in}{4.996846in}}%
\pgfpathclose%
\pgfusepath{stroke,fill}%
\end{pgfscope}%
\begin{pgfscope}%
\pgfpathrectangle{\pgfqpoint{1.000000in}{1.148311in}}{\pgfqpoint{6.200000in}{5.623377in}}%
\pgfusepath{clip}%
\pgfsetbuttcap%
\pgfsetroundjoin%
\definecolor{currentfill}{rgb}{0.200000,0.200000,0.800000}%
\pgfsetfillcolor{currentfill}%
\pgfsetlinewidth{1.003750pt}%
\definecolor{currentstroke}{rgb}{0.200000,0.200000,0.800000}%
\pgfsetstrokecolor{currentstroke}%
\pgfsetdash{}{0pt}%
\pgfpathmoveto{\pgfqpoint{6.485764in}{5.051725in}}%
\pgfpathcurveto{\pgfqpoint{6.491588in}{5.051725in}}{\pgfqpoint{6.497175in}{5.054039in}}{\pgfqpoint{6.501293in}{5.058157in}}%
\pgfpathcurveto{\pgfqpoint{6.505411in}{5.062276in}}{\pgfqpoint{6.507725in}{5.067862in}}{\pgfqpoint{6.507725in}{5.073686in}}%
\pgfpathcurveto{\pgfqpoint{6.507725in}{5.079510in}}{\pgfqpoint{6.505411in}{5.085096in}}{\pgfqpoint{6.501293in}{5.089214in}}%
\pgfpathcurveto{\pgfqpoint{6.497175in}{5.093332in}}{\pgfqpoint{6.491588in}{5.095646in}}{\pgfqpoint{6.485764in}{5.095646in}}%
\pgfpathcurveto{\pgfqpoint{6.479940in}{5.095646in}}{\pgfqpoint{6.474354in}{5.093332in}}{\pgfqpoint{6.470236in}{5.089214in}}%
\pgfpathcurveto{\pgfqpoint{6.466118in}{5.085096in}}{\pgfqpoint{6.463804in}{5.079510in}}{\pgfqpoint{6.463804in}{5.073686in}}%
\pgfpathcurveto{\pgfqpoint{6.463804in}{5.067862in}}{\pgfqpoint{6.466118in}{5.062276in}}{\pgfqpoint{6.470236in}{5.058157in}}%
\pgfpathcurveto{\pgfqpoint{6.474354in}{5.054039in}}{\pgfqpoint{6.479940in}{5.051725in}}{\pgfqpoint{6.485764in}{5.051725in}}%
\pgfpathlineto{\pgfqpoint{6.485764in}{5.051725in}}%
\pgfpathclose%
\pgfusepath{stroke,fill}%
\end{pgfscope}%
\begin{pgfscope}%
\pgfpathrectangle{\pgfqpoint{1.000000in}{1.148311in}}{\pgfqpoint{6.200000in}{5.623377in}}%
\pgfusepath{clip}%
\pgfsetbuttcap%
\pgfsetroundjoin%
\definecolor{currentfill}{rgb}{0.200000,0.200000,0.800000}%
\pgfsetfillcolor{currentfill}%
\pgfsetlinewidth{1.003750pt}%
\definecolor{currentstroke}{rgb}{0.200000,0.200000,0.800000}%
\pgfsetstrokecolor{currentstroke}%
\pgfsetdash{}{0pt}%
\pgfpathmoveto{\pgfqpoint{6.487172in}{5.120742in}}%
\pgfpathcurveto{\pgfqpoint{6.492996in}{5.120742in}}{\pgfqpoint{6.498582in}{5.123056in}}{\pgfqpoint{6.502701in}{5.127174in}}%
\pgfpathcurveto{\pgfqpoint{6.506819in}{5.131292in}}{\pgfqpoint{6.509133in}{5.136878in}}{\pgfqpoint{6.509133in}{5.142702in}}%
\pgfpathcurveto{\pgfqpoint{6.509133in}{5.148526in}}{\pgfqpoint{6.506819in}{5.154113in}}{\pgfqpoint{6.502701in}{5.158231in}}%
\pgfpathcurveto{\pgfqpoint{6.498582in}{5.162349in}}{\pgfqpoint{6.492996in}{5.164663in}}{\pgfqpoint{6.487172in}{5.164663in}}%
\pgfpathcurveto{\pgfqpoint{6.481348in}{5.164663in}}{\pgfqpoint{6.475762in}{5.162349in}}{\pgfqpoint{6.471644in}{5.158231in}}%
\pgfpathcurveto{\pgfqpoint{6.467526in}{5.154113in}}{\pgfqpoint{6.465212in}{5.148526in}}{\pgfqpoint{6.465212in}{5.142702in}}%
\pgfpathcurveto{\pgfqpoint{6.465212in}{5.136878in}}{\pgfqpoint{6.467526in}{5.131292in}}{\pgfqpoint{6.471644in}{5.127174in}}%
\pgfpathcurveto{\pgfqpoint{6.475762in}{5.123056in}}{\pgfqpoint{6.481348in}{5.120742in}}{\pgfqpoint{6.487172in}{5.120742in}}%
\pgfpathlineto{\pgfqpoint{6.487172in}{5.120742in}}%
\pgfpathclose%
\pgfusepath{stroke,fill}%
\end{pgfscope}%
\begin{pgfscope}%
\pgfpathrectangle{\pgfqpoint{1.000000in}{1.148311in}}{\pgfqpoint{6.200000in}{5.623377in}}%
\pgfusepath{clip}%
\pgfsetbuttcap%
\pgfsetroundjoin%
\definecolor{currentfill}{rgb}{0.200000,0.200000,0.800000}%
\pgfsetfillcolor{currentfill}%
\pgfsetlinewidth{1.003750pt}%
\definecolor{currentstroke}{rgb}{0.200000,0.200000,0.800000}%
\pgfsetstrokecolor{currentstroke}%
\pgfsetdash{}{0pt}%
\pgfpathmoveto{\pgfqpoint{6.498707in}{5.186973in}}%
\pgfpathcurveto{\pgfqpoint{6.504531in}{5.186973in}}{\pgfqpoint{6.510117in}{5.189287in}}{\pgfqpoint{6.514235in}{5.193405in}}%
\pgfpathcurveto{\pgfqpoint{6.518353in}{5.197523in}}{\pgfqpoint{6.520667in}{5.203109in}}{\pgfqpoint{6.520667in}{5.208933in}}%
\pgfpathcurveto{\pgfqpoint{6.520667in}{5.214757in}}{\pgfqpoint{6.518353in}{5.220344in}}{\pgfqpoint{6.514235in}{5.224462in}}%
\pgfpathcurveto{\pgfqpoint{6.510117in}{5.228580in}}{\pgfqpoint{6.504531in}{5.230894in}}{\pgfqpoint{6.498707in}{5.230894in}}%
\pgfpathcurveto{\pgfqpoint{6.492883in}{5.230894in}}{\pgfqpoint{6.487297in}{5.228580in}}{\pgfqpoint{6.483178in}{5.224462in}}%
\pgfpathcurveto{\pgfqpoint{6.479060in}{5.220344in}}{\pgfqpoint{6.476746in}{5.214757in}}{\pgfqpoint{6.476746in}{5.208933in}}%
\pgfpathcurveto{\pgfqpoint{6.476746in}{5.203109in}}{\pgfqpoint{6.479060in}{5.197523in}}{\pgfqpoint{6.483178in}{5.193405in}}%
\pgfpathcurveto{\pgfqpoint{6.487297in}{5.189287in}}{\pgfqpoint{6.492883in}{5.186973in}}{\pgfqpoint{6.498707in}{5.186973in}}%
\pgfpathlineto{\pgfqpoint{6.498707in}{5.186973in}}%
\pgfpathclose%
\pgfusepath{stroke,fill}%
\end{pgfscope}%
\begin{pgfscope}%
\pgfpathrectangle{\pgfqpoint{1.000000in}{1.148311in}}{\pgfqpoint{6.200000in}{5.623377in}}%
\pgfusepath{clip}%
\pgfsetbuttcap%
\pgfsetroundjoin%
\definecolor{currentfill}{rgb}{0.200000,0.200000,0.800000}%
\pgfsetfillcolor{currentfill}%
\pgfsetlinewidth{1.003750pt}%
\definecolor{currentstroke}{rgb}{0.200000,0.200000,0.800000}%
\pgfsetstrokecolor{currentstroke}%
\pgfsetdash{}{0pt}%
\pgfpathmoveto{\pgfqpoint{6.463485in}{5.256564in}}%
\pgfpathcurveto{\pgfqpoint{6.469309in}{5.256564in}}{\pgfqpoint{6.474896in}{5.258878in}}{\pgfqpoint{6.479014in}{5.262996in}}%
\pgfpathcurveto{\pgfqpoint{6.483132in}{5.267114in}}{\pgfqpoint{6.485446in}{5.272700in}}{\pgfqpoint{6.485446in}{5.278524in}}%
\pgfpathcurveto{\pgfqpoint{6.485446in}{5.284348in}}{\pgfqpoint{6.483132in}{5.289934in}}{\pgfqpoint{6.479014in}{5.294053in}}%
\pgfpathcurveto{\pgfqpoint{6.474896in}{5.298171in}}{\pgfqpoint{6.469309in}{5.300485in}}{\pgfqpoint{6.463485in}{5.300485in}}%
\pgfpathcurveto{\pgfqpoint{6.457662in}{5.300485in}}{\pgfqpoint{6.452075in}{5.298171in}}{\pgfqpoint{6.447957in}{5.294053in}}%
\pgfpathcurveto{\pgfqpoint{6.443839in}{5.289934in}}{\pgfqpoint{6.441525in}{5.284348in}}{\pgfqpoint{6.441525in}{5.278524in}}%
\pgfpathcurveto{\pgfqpoint{6.441525in}{5.272700in}}{\pgfqpoint{6.443839in}{5.267114in}}{\pgfqpoint{6.447957in}{5.262996in}}%
\pgfpathcurveto{\pgfqpoint{6.452075in}{5.258878in}}{\pgfqpoint{6.457662in}{5.256564in}}{\pgfqpoint{6.463485in}{5.256564in}}%
\pgfpathlineto{\pgfqpoint{6.463485in}{5.256564in}}%
\pgfpathclose%
\pgfusepath{stroke,fill}%
\end{pgfscope}%
\begin{pgfscope}%
\pgfpathrectangle{\pgfqpoint{1.000000in}{1.148311in}}{\pgfqpoint{6.200000in}{5.623377in}}%
\pgfusepath{clip}%
\pgfsetbuttcap%
\pgfsetroundjoin%
\definecolor{currentfill}{rgb}{0.200000,0.200000,0.800000}%
\pgfsetfillcolor{currentfill}%
\pgfsetlinewidth{1.003750pt}%
\definecolor{currentstroke}{rgb}{0.200000,0.200000,0.800000}%
\pgfsetstrokecolor{currentstroke}%
\pgfsetdash{}{0pt}%
\pgfpathmoveto{\pgfqpoint{6.484391in}{5.321136in}}%
\pgfpathcurveto{\pgfqpoint{6.490215in}{5.321136in}}{\pgfqpoint{6.495801in}{5.323450in}}{\pgfqpoint{6.499919in}{5.327568in}}%
\pgfpathcurveto{\pgfqpoint{6.504037in}{5.331687in}}{\pgfqpoint{6.506351in}{5.337273in}}{\pgfqpoint{6.506351in}{5.343097in}}%
\pgfpathcurveto{\pgfqpoint{6.506351in}{5.348921in}}{\pgfqpoint{6.504037in}{5.354507in}}{\pgfqpoint{6.499919in}{5.358625in}}%
\pgfpathcurveto{\pgfqpoint{6.495801in}{5.362743in}}{\pgfqpoint{6.490215in}{5.365057in}}{\pgfqpoint{6.484391in}{5.365057in}}%
\pgfpathcurveto{\pgfqpoint{6.478567in}{5.365057in}}{\pgfqpoint{6.472981in}{5.362743in}}{\pgfqpoint{6.468862in}{5.358625in}}%
\pgfpathcurveto{\pgfqpoint{6.464744in}{5.354507in}}{\pgfqpoint{6.462430in}{5.348921in}}{\pgfqpoint{6.462430in}{5.343097in}}%
\pgfpathcurveto{\pgfqpoint{6.462430in}{5.337273in}}{\pgfqpoint{6.464744in}{5.331687in}}{\pgfqpoint{6.468862in}{5.327568in}}%
\pgfpathcurveto{\pgfqpoint{6.472981in}{5.323450in}}{\pgfqpoint{6.478567in}{5.321136in}}{\pgfqpoint{6.484391in}{5.321136in}}%
\pgfpathlineto{\pgfqpoint{6.484391in}{5.321136in}}%
\pgfpathclose%
\pgfusepath{stroke,fill}%
\end{pgfscope}%
\begin{pgfscope}%
\pgfpathrectangle{\pgfqpoint{1.000000in}{1.148311in}}{\pgfqpoint{6.200000in}{5.623377in}}%
\pgfusepath{clip}%
\pgfsetbuttcap%
\pgfsetroundjoin%
\definecolor{currentfill}{rgb}{0.800000,0.200000,0.200000}%
\pgfsetfillcolor{currentfill}%
\pgfsetlinewidth{1.003750pt}%
\definecolor{currentstroke}{rgb}{0.800000,0.200000,0.200000}%
\pgfsetstrokecolor{currentstroke}%
\pgfsetdash{}{0pt}%
\pgfpathmoveto{\pgfqpoint{6.855480in}{4.049087in}}%
\pgfpathcurveto{\pgfqpoint{6.861304in}{4.049087in}}{\pgfqpoint{6.866891in}{4.051400in}}{\pgfqpoint{6.871009in}{4.055519in}}%
\pgfpathcurveto{\pgfqpoint{6.875127in}{4.059637in}}{\pgfqpoint{6.877441in}{4.065223in}}{\pgfqpoint{6.877441in}{4.071047in}}%
\pgfpathcurveto{\pgfqpoint{6.877441in}{4.076871in}}{\pgfqpoint{6.875127in}{4.082457in}}{\pgfqpoint{6.871009in}{4.086575in}}%
\pgfpathcurveto{\pgfqpoint{6.866891in}{4.090693in}}{\pgfqpoint{6.861304in}{4.093007in}}{\pgfqpoint{6.855480in}{4.093007in}}%
\pgfpathcurveto{\pgfqpoint{6.849656in}{4.093007in}}{\pgfqpoint{6.844070in}{4.090693in}}{\pgfqpoint{6.839952in}{4.086575in}}%
\pgfpathcurveto{\pgfqpoint{6.835834in}{4.082457in}}{\pgfqpoint{6.833520in}{4.076871in}}{\pgfqpoint{6.833520in}{4.071047in}}%
\pgfpathcurveto{\pgfqpoint{6.833520in}{4.065223in}}{\pgfqpoint{6.835834in}{4.059637in}}{\pgfqpoint{6.839952in}{4.055519in}}%
\pgfpathcurveto{\pgfqpoint{6.844070in}{4.051400in}}{\pgfqpoint{6.849656in}{4.049087in}}{\pgfqpoint{6.855480in}{4.049087in}}%
\pgfpathlineto{\pgfqpoint{6.855480in}{4.049087in}}%
\pgfpathclose%
\pgfusepath{stroke,fill}%
\end{pgfscope}%
\begin{pgfscope}%
\pgfpathrectangle{\pgfqpoint{1.000000in}{1.148311in}}{\pgfqpoint{6.200000in}{5.623377in}}%
\pgfusepath{clip}%
\pgfsetbuttcap%
\pgfsetroundjoin%
\definecolor{currentfill}{rgb}{0.800000,0.200000,0.200000}%
\pgfsetfillcolor{currentfill}%
\pgfsetlinewidth{1.003750pt}%
\definecolor{currentstroke}{rgb}{0.800000,0.200000,0.200000}%
\pgfsetstrokecolor{currentstroke}%
\pgfsetdash{}{0pt}%
\pgfpathmoveto{\pgfqpoint{6.813178in}{4.113799in}}%
\pgfpathcurveto{\pgfqpoint{6.819002in}{4.113799in}}{\pgfqpoint{6.824589in}{4.116113in}}{\pgfqpoint{6.828707in}{4.120231in}}%
\pgfpathcurveto{\pgfqpoint{6.832825in}{4.124349in}}{\pgfqpoint{6.835139in}{4.129935in}}{\pgfqpoint{6.835139in}{4.135759in}}%
\pgfpathcurveto{\pgfqpoint{6.835139in}{4.141583in}}{\pgfqpoint{6.832825in}{4.147169in}}{\pgfqpoint{6.828707in}{4.151287in}}%
\pgfpathcurveto{\pgfqpoint{6.824589in}{4.155405in}}{\pgfqpoint{6.819002in}{4.157719in}}{\pgfqpoint{6.813178in}{4.157719in}}%
\pgfpathcurveto{\pgfqpoint{6.807354in}{4.157719in}}{\pgfqpoint{6.801768in}{4.155405in}}{\pgfqpoint{6.797650in}{4.151287in}}%
\pgfpathcurveto{\pgfqpoint{6.793532in}{4.147169in}}{\pgfqpoint{6.791218in}{4.141583in}}{\pgfqpoint{6.791218in}{4.135759in}}%
\pgfpathcurveto{\pgfqpoint{6.791218in}{4.129935in}}{\pgfqpoint{6.793532in}{4.124349in}}{\pgfqpoint{6.797650in}{4.120231in}}%
\pgfpathcurveto{\pgfqpoint{6.801768in}{4.116113in}}{\pgfqpoint{6.807354in}{4.113799in}}{\pgfqpoint{6.813178in}{4.113799in}}%
\pgfpathlineto{\pgfqpoint{6.813178in}{4.113799in}}%
\pgfpathclose%
\pgfusepath{stroke,fill}%
\end{pgfscope}%
\begin{pgfscope}%
\pgfpathrectangle{\pgfqpoint{1.000000in}{1.148311in}}{\pgfqpoint{6.200000in}{5.623377in}}%
\pgfusepath{clip}%
\pgfsetbuttcap%
\pgfsetroundjoin%
\definecolor{currentfill}{rgb}{0.800000,0.200000,0.200000}%
\pgfsetfillcolor{currentfill}%
\pgfsetlinewidth{1.003750pt}%
\definecolor{currentstroke}{rgb}{0.800000,0.200000,0.200000}%
\pgfsetstrokecolor{currentstroke}%
\pgfsetdash{}{0pt}%
\pgfpathmoveto{\pgfqpoint{6.918182in}{4.192436in}}%
\pgfpathcurveto{\pgfqpoint{6.924006in}{4.192436in}}{\pgfqpoint{6.929592in}{4.194750in}}{\pgfqpoint{6.933710in}{4.198868in}}%
\pgfpathcurveto{\pgfqpoint{6.937828in}{4.202986in}}{\pgfqpoint{6.940142in}{4.208573in}}{\pgfqpoint{6.940142in}{4.214397in}}%
\pgfpathcurveto{\pgfqpoint{6.940142in}{4.220221in}}{\pgfqpoint{6.937828in}{4.225807in}}{\pgfqpoint{6.933710in}{4.229925in}}%
\pgfpathcurveto{\pgfqpoint{6.929592in}{4.234043in}}{\pgfqpoint{6.924006in}{4.236357in}}{\pgfqpoint{6.918182in}{4.236357in}}%
\pgfpathcurveto{\pgfqpoint{6.912358in}{4.236357in}}{\pgfqpoint{6.906772in}{4.234043in}}{\pgfqpoint{6.902654in}{4.229925in}}%
\pgfpathcurveto{\pgfqpoint{6.898535in}{4.225807in}}{\pgfqpoint{6.896222in}{4.220221in}}{\pgfqpoint{6.896222in}{4.214397in}}%
\pgfpathcurveto{\pgfqpoint{6.896222in}{4.208573in}}{\pgfqpoint{6.898535in}{4.202986in}}{\pgfqpoint{6.902654in}{4.198868in}}%
\pgfpathcurveto{\pgfqpoint{6.906772in}{4.194750in}}{\pgfqpoint{6.912358in}{4.192436in}}{\pgfqpoint{6.918182in}{4.192436in}}%
\pgfpathlineto{\pgfqpoint{6.918182in}{4.192436in}}%
\pgfpathclose%
\pgfusepath{stroke,fill}%
\end{pgfscope}%
\begin{pgfscope}%
\pgfpathrectangle{\pgfqpoint{1.000000in}{1.148311in}}{\pgfqpoint{6.200000in}{5.623377in}}%
\pgfusepath{clip}%
\pgfsetbuttcap%
\pgfsetroundjoin%
\definecolor{currentfill}{rgb}{0.800000,0.200000,0.200000}%
\pgfsetfillcolor{currentfill}%
\pgfsetlinewidth{1.003750pt}%
\definecolor{currentstroke}{rgb}{0.800000,0.200000,0.200000}%
\pgfsetstrokecolor{currentstroke}%
\pgfsetdash{}{0pt}%
\pgfpathmoveto{\pgfqpoint{6.830617in}{4.248701in}}%
\pgfpathcurveto{\pgfqpoint{6.836441in}{4.248701in}}{\pgfqpoint{6.842027in}{4.251015in}}{\pgfqpoint{6.846145in}{4.255133in}}%
\pgfpathcurveto{\pgfqpoint{6.850263in}{4.259251in}}{\pgfqpoint{6.852577in}{4.264837in}}{\pgfqpoint{6.852577in}{4.270661in}}%
\pgfpathcurveto{\pgfqpoint{6.852577in}{4.276485in}}{\pgfqpoint{6.850263in}{4.282071in}}{\pgfqpoint{6.846145in}{4.286189in}}%
\pgfpathcurveto{\pgfqpoint{6.842027in}{4.290307in}}{\pgfqpoint{6.836441in}{4.292621in}}{\pgfqpoint{6.830617in}{4.292621in}}%
\pgfpathcurveto{\pgfqpoint{6.824793in}{4.292621in}}{\pgfqpoint{6.819207in}{4.290307in}}{\pgfqpoint{6.815089in}{4.286189in}}%
\pgfpathcurveto{\pgfqpoint{6.810971in}{4.282071in}}{\pgfqpoint{6.808657in}{4.276485in}}{\pgfqpoint{6.808657in}{4.270661in}}%
\pgfpathcurveto{\pgfqpoint{6.808657in}{4.264837in}}{\pgfqpoint{6.810971in}{4.259251in}}{\pgfqpoint{6.815089in}{4.255133in}}%
\pgfpathcurveto{\pgfqpoint{6.819207in}{4.251015in}}{\pgfqpoint{6.824793in}{4.248701in}}{\pgfqpoint{6.830617in}{4.248701in}}%
\pgfpathlineto{\pgfqpoint{6.830617in}{4.248701in}}%
\pgfpathclose%
\pgfusepath{stroke,fill}%
\end{pgfscope}%
\begin{pgfscope}%
\pgfpathrectangle{\pgfqpoint{1.000000in}{1.148311in}}{\pgfqpoint{6.200000in}{5.623377in}}%
\pgfusepath{clip}%
\pgfsetbuttcap%
\pgfsetroundjoin%
\definecolor{currentfill}{rgb}{0.800000,0.200000,0.200000}%
\pgfsetfillcolor{currentfill}%
\pgfsetlinewidth{1.003750pt}%
\definecolor{currentstroke}{rgb}{0.800000,0.200000,0.200000}%
\pgfsetstrokecolor{currentstroke}%
\pgfsetdash{}{0pt}%
\pgfpathmoveto{\pgfqpoint{6.733652in}{4.292654in}}%
\pgfpathcurveto{\pgfqpoint{6.739476in}{4.292654in}}{\pgfqpoint{6.745062in}{4.294968in}}{\pgfqpoint{6.749180in}{4.299086in}}%
\pgfpathcurveto{\pgfqpoint{6.753299in}{4.303204in}}{\pgfqpoint{6.755613in}{4.308790in}}{\pgfqpoint{6.755613in}{4.314614in}}%
\pgfpathcurveto{\pgfqpoint{6.755613in}{4.320438in}}{\pgfqpoint{6.753299in}{4.326024in}}{\pgfqpoint{6.749180in}{4.330142in}}%
\pgfpathcurveto{\pgfqpoint{6.745062in}{4.334261in}}{\pgfqpoint{6.739476in}{4.336574in}}{\pgfqpoint{6.733652in}{4.336574in}}%
\pgfpathcurveto{\pgfqpoint{6.727828in}{4.336574in}}{\pgfqpoint{6.722242in}{4.334261in}}{\pgfqpoint{6.718124in}{4.330142in}}%
\pgfpathcurveto{\pgfqpoint{6.714006in}{4.326024in}}{\pgfqpoint{6.711692in}{4.320438in}}{\pgfqpoint{6.711692in}{4.314614in}}%
\pgfpathcurveto{\pgfqpoint{6.711692in}{4.308790in}}{\pgfqpoint{6.714006in}{4.303204in}}{\pgfqpoint{6.718124in}{4.299086in}}%
\pgfpathcurveto{\pgfqpoint{6.722242in}{4.294968in}}{\pgfqpoint{6.727828in}{4.292654in}}{\pgfqpoint{6.733652in}{4.292654in}}%
\pgfpathlineto{\pgfqpoint{6.733652in}{4.292654in}}%
\pgfpathclose%
\pgfusepath{stroke,fill}%
\end{pgfscope}%
\begin{pgfscope}%
\pgfpathrectangle{\pgfqpoint{1.000000in}{1.148311in}}{\pgfqpoint{6.200000in}{5.623377in}}%
\pgfusepath{clip}%
\pgfsetbuttcap%
\pgfsetroundjoin%
\definecolor{currentfill}{rgb}{0.800000,0.200000,0.200000}%
\pgfsetfillcolor{currentfill}%
\pgfsetlinewidth{1.003750pt}%
\definecolor{currentstroke}{rgb}{0.800000,0.200000,0.200000}%
\pgfsetstrokecolor{currentstroke}%
\pgfsetdash{}{0pt}%
\pgfpathmoveto{\pgfqpoint{6.816725in}{4.384680in}}%
\pgfpathcurveto{\pgfqpoint{6.822549in}{4.384680in}}{\pgfqpoint{6.828135in}{4.386994in}}{\pgfqpoint{6.832253in}{4.391112in}}%
\pgfpathcurveto{\pgfqpoint{6.836371in}{4.395230in}}{\pgfqpoint{6.838685in}{4.400816in}}{\pgfqpoint{6.838685in}{4.406640in}}%
\pgfpathcurveto{\pgfqpoint{6.838685in}{4.412464in}}{\pgfqpoint{6.836371in}{4.418050in}}{\pgfqpoint{6.832253in}{4.422168in}}%
\pgfpathcurveto{\pgfqpoint{6.828135in}{4.426286in}}{\pgfqpoint{6.822549in}{4.428600in}}{\pgfqpoint{6.816725in}{4.428600in}}%
\pgfpathcurveto{\pgfqpoint{6.810901in}{4.428600in}}{\pgfqpoint{6.805315in}{4.426286in}}{\pgfqpoint{6.801197in}{4.422168in}}%
\pgfpathcurveto{\pgfqpoint{6.797079in}{4.418050in}}{\pgfqpoint{6.794765in}{4.412464in}}{\pgfqpoint{6.794765in}{4.406640in}}%
\pgfpathcurveto{\pgfqpoint{6.794765in}{4.400816in}}{\pgfqpoint{6.797079in}{4.395230in}}{\pgfqpoint{6.801197in}{4.391112in}}%
\pgfpathcurveto{\pgfqpoint{6.805315in}{4.386994in}}{\pgfqpoint{6.810901in}{4.384680in}}{\pgfqpoint{6.816725in}{4.384680in}}%
\pgfpathlineto{\pgfqpoint{6.816725in}{4.384680in}}%
\pgfpathclose%
\pgfusepath{stroke,fill}%
\end{pgfscope}%
\begin{pgfscope}%
\pgfpathrectangle{\pgfqpoint{1.000000in}{1.148311in}}{\pgfqpoint{6.200000in}{5.623377in}}%
\pgfusepath{clip}%
\pgfsetbuttcap%
\pgfsetroundjoin%
\definecolor{currentfill}{rgb}{0.800000,0.200000,0.200000}%
\pgfsetfillcolor{currentfill}%
\pgfsetlinewidth{1.003750pt}%
\definecolor{currentstroke}{rgb}{0.800000,0.200000,0.200000}%
\pgfsetstrokecolor{currentstroke}%
\pgfsetdash{}{0pt}%
\pgfpathmoveto{\pgfqpoint{6.733142in}{4.424694in}}%
\pgfpathcurveto{\pgfqpoint{6.738966in}{4.424694in}}{\pgfqpoint{6.744552in}{4.427008in}}{\pgfqpoint{6.748670in}{4.431126in}}%
\pgfpathcurveto{\pgfqpoint{6.752789in}{4.435244in}}{\pgfqpoint{6.755102in}{4.440830in}}{\pgfqpoint{6.755102in}{4.446654in}}%
\pgfpathcurveto{\pgfqpoint{6.755102in}{4.452478in}}{\pgfqpoint{6.752789in}{4.458064in}}{\pgfqpoint{6.748670in}{4.462183in}}%
\pgfpathcurveto{\pgfqpoint{6.744552in}{4.466301in}}{\pgfqpoint{6.738966in}{4.468615in}}{\pgfqpoint{6.733142in}{4.468615in}}%
\pgfpathcurveto{\pgfqpoint{6.727318in}{4.468615in}}{\pgfqpoint{6.721732in}{4.466301in}}{\pgfqpoint{6.717614in}{4.462183in}}%
\pgfpathcurveto{\pgfqpoint{6.713496in}{4.458064in}}{\pgfqpoint{6.711182in}{4.452478in}}{\pgfqpoint{6.711182in}{4.446654in}}%
\pgfpathcurveto{\pgfqpoint{6.711182in}{4.440830in}}{\pgfqpoint{6.713496in}{4.435244in}}{\pgfqpoint{6.717614in}{4.431126in}}%
\pgfpathcurveto{\pgfqpoint{6.721732in}{4.427008in}}{\pgfqpoint{6.727318in}{4.424694in}}{\pgfqpoint{6.733142in}{4.424694in}}%
\pgfpathlineto{\pgfqpoint{6.733142in}{4.424694in}}%
\pgfpathclose%
\pgfusepath{stroke,fill}%
\end{pgfscope}%
\begin{pgfscope}%
\pgfpathrectangle{\pgfqpoint{1.000000in}{1.148311in}}{\pgfqpoint{6.200000in}{5.623377in}}%
\pgfusepath{clip}%
\pgfsetbuttcap%
\pgfsetroundjoin%
\definecolor{currentfill}{rgb}{0.800000,0.200000,0.200000}%
\pgfsetfillcolor{currentfill}%
\pgfsetlinewidth{1.003750pt}%
\definecolor{currentstroke}{rgb}{0.800000,0.200000,0.200000}%
\pgfsetstrokecolor{currentstroke}%
\pgfsetdash{}{0pt}%
\pgfpathmoveto{\pgfqpoint{6.781677in}{4.518785in}}%
\pgfpathcurveto{\pgfqpoint{6.787501in}{4.518785in}}{\pgfqpoint{6.793087in}{4.521099in}}{\pgfqpoint{6.797205in}{4.525217in}}%
\pgfpathcurveto{\pgfqpoint{6.801323in}{4.529335in}}{\pgfqpoint{6.803637in}{4.534921in}}{\pgfqpoint{6.803637in}{4.540745in}}%
\pgfpathcurveto{\pgfqpoint{6.803637in}{4.546569in}}{\pgfqpoint{6.801323in}{4.552155in}}{\pgfqpoint{6.797205in}{4.556273in}}%
\pgfpathcurveto{\pgfqpoint{6.793087in}{4.560391in}}{\pgfqpoint{6.787501in}{4.562705in}}{\pgfqpoint{6.781677in}{4.562705in}}%
\pgfpathcurveto{\pgfqpoint{6.775853in}{4.562705in}}{\pgfqpoint{6.770267in}{4.560391in}}{\pgfqpoint{6.766149in}{4.556273in}}%
\pgfpathcurveto{\pgfqpoint{6.762030in}{4.552155in}}{\pgfqpoint{6.759717in}{4.546569in}}{\pgfqpoint{6.759717in}{4.540745in}}%
\pgfpathcurveto{\pgfqpoint{6.759717in}{4.534921in}}{\pgfqpoint{6.762030in}{4.529335in}}{\pgfqpoint{6.766149in}{4.525217in}}%
\pgfpathcurveto{\pgfqpoint{6.770267in}{4.521099in}}{\pgfqpoint{6.775853in}{4.518785in}}{\pgfqpoint{6.781677in}{4.518785in}}%
\pgfpathlineto{\pgfqpoint{6.781677in}{4.518785in}}%
\pgfpathclose%
\pgfusepath{stroke,fill}%
\end{pgfscope}%
\begin{pgfscope}%
\pgfpathrectangle{\pgfqpoint{1.000000in}{1.148311in}}{\pgfqpoint{6.200000in}{5.623377in}}%
\pgfusepath{clip}%
\pgfsetbuttcap%
\pgfsetroundjoin%
\definecolor{currentfill}{rgb}{0.800000,0.200000,0.200000}%
\pgfsetfillcolor{currentfill}%
\pgfsetlinewidth{1.003750pt}%
\definecolor{currentstroke}{rgb}{0.800000,0.200000,0.200000}%
\pgfsetstrokecolor{currentstroke}%
\pgfsetdash{}{0pt}%
\pgfpathmoveto{\pgfqpoint{6.737200in}{4.573357in}}%
\pgfpathcurveto{\pgfqpoint{6.743024in}{4.573357in}}{\pgfqpoint{6.748611in}{4.575671in}}{\pgfqpoint{6.752729in}{4.579789in}}%
\pgfpathcurveto{\pgfqpoint{6.756847in}{4.583907in}}{\pgfqpoint{6.759161in}{4.589494in}}{\pgfqpoint{6.759161in}{4.595318in}}%
\pgfpathcurveto{\pgfqpoint{6.759161in}{4.601142in}}{\pgfqpoint{6.756847in}{4.606728in}}{\pgfqpoint{6.752729in}{4.610846in}}%
\pgfpathcurveto{\pgfqpoint{6.748611in}{4.614964in}}{\pgfqpoint{6.743024in}{4.617278in}}{\pgfqpoint{6.737200in}{4.617278in}}%
\pgfpathcurveto{\pgfqpoint{6.731377in}{4.617278in}}{\pgfqpoint{6.725790in}{4.614964in}}{\pgfqpoint{6.721672in}{4.610846in}}%
\pgfpathcurveto{\pgfqpoint{6.717554in}{4.606728in}}{\pgfqpoint{6.715240in}{4.601142in}}{\pgfqpoint{6.715240in}{4.595318in}}%
\pgfpathcurveto{\pgfqpoint{6.715240in}{4.589494in}}{\pgfqpoint{6.717554in}{4.583907in}}{\pgfqpoint{6.721672in}{4.579789in}}%
\pgfpathcurveto{\pgfqpoint{6.725790in}{4.575671in}}{\pgfqpoint{6.731377in}{4.573357in}}{\pgfqpoint{6.737200in}{4.573357in}}%
\pgfpathlineto{\pgfqpoint{6.737200in}{4.573357in}}%
\pgfpathclose%
\pgfusepath{stroke,fill}%
\end{pgfscope}%
\begin{pgfscope}%
\pgfpathrectangle{\pgfqpoint{1.000000in}{1.148311in}}{\pgfqpoint{6.200000in}{5.623377in}}%
\pgfusepath{clip}%
\pgfsetbuttcap%
\pgfsetroundjoin%
\definecolor{currentfill}{rgb}{0.800000,0.200000,0.200000}%
\pgfsetfillcolor{currentfill}%
\pgfsetlinewidth{1.003750pt}%
\definecolor{currentstroke}{rgb}{0.800000,0.200000,0.200000}%
\pgfsetstrokecolor{currentstroke}%
\pgfsetdash{}{0pt}%
\pgfpathmoveto{\pgfqpoint{6.730517in}{4.650358in}}%
\pgfpathcurveto{\pgfqpoint{6.736341in}{4.650358in}}{\pgfqpoint{6.741927in}{4.652672in}}{\pgfqpoint{6.746046in}{4.656790in}}%
\pgfpathcurveto{\pgfqpoint{6.750164in}{4.660909in}}{\pgfqpoint{6.752478in}{4.666495in}}{\pgfqpoint{6.752478in}{4.672319in}}%
\pgfpathcurveto{\pgfqpoint{6.752478in}{4.678143in}}{\pgfqpoint{6.750164in}{4.683729in}}{\pgfqpoint{6.746046in}{4.687847in}}%
\pgfpathcurveto{\pgfqpoint{6.741927in}{4.691965in}}{\pgfqpoint{6.736341in}{4.694279in}}{\pgfqpoint{6.730517in}{4.694279in}}%
\pgfpathcurveto{\pgfqpoint{6.724693in}{4.694279in}}{\pgfqpoint{6.719107in}{4.691965in}}{\pgfqpoint{6.714989in}{4.687847in}}%
\pgfpathcurveto{\pgfqpoint{6.710871in}{4.683729in}}{\pgfqpoint{6.708557in}{4.678143in}}{\pgfqpoint{6.708557in}{4.672319in}}%
\pgfpathcurveto{\pgfqpoint{6.708557in}{4.666495in}}{\pgfqpoint{6.710871in}{4.660909in}}{\pgfqpoint{6.714989in}{4.656790in}}%
\pgfpathcurveto{\pgfqpoint{6.719107in}{4.652672in}}{\pgfqpoint{6.724693in}{4.650358in}}{\pgfqpoint{6.730517in}{4.650358in}}%
\pgfpathlineto{\pgfqpoint{6.730517in}{4.650358in}}%
\pgfpathclose%
\pgfusepath{stroke,fill}%
\end{pgfscope}%
\begin{pgfscope}%
\pgfpathrectangle{\pgfqpoint{1.000000in}{1.148311in}}{\pgfqpoint{6.200000in}{5.623377in}}%
\pgfusepath{clip}%
\pgfsetbuttcap%
\pgfsetroundjoin%
\definecolor{currentfill}{rgb}{0.800000,0.200000,0.200000}%
\pgfsetfillcolor{currentfill}%
\pgfsetlinewidth{1.003750pt}%
\definecolor{currentstroke}{rgb}{0.800000,0.200000,0.200000}%
\pgfsetstrokecolor{currentstroke}%
\pgfsetdash{}{0pt}%
\pgfpathmoveto{\pgfqpoint{6.649523in}{4.678318in}}%
\pgfpathcurveto{\pgfqpoint{6.655347in}{4.678318in}}{\pgfqpoint{6.660933in}{4.680632in}}{\pgfqpoint{6.665052in}{4.684750in}}%
\pgfpathcurveto{\pgfqpoint{6.669170in}{4.688868in}}{\pgfqpoint{6.671484in}{4.694454in}}{\pgfqpoint{6.671484in}{4.700278in}}%
\pgfpathcurveto{\pgfqpoint{6.671484in}{4.706102in}}{\pgfqpoint{6.669170in}{4.711688in}}{\pgfqpoint{6.665052in}{4.715806in}}%
\pgfpathcurveto{\pgfqpoint{6.660933in}{4.719924in}}{\pgfqpoint{6.655347in}{4.722238in}}{\pgfqpoint{6.649523in}{4.722238in}}%
\pgfpathcurveto{\pgfqpoint{6.643699in}{4.722238in}}{\pgfqpoint{6.638113in}{4.719924in}}{\pgfqpoint{6.633995in}{4.715806in}}%
\pgfpathcurveto{\pgfqpoint{6.629877in}{4.711688in}}{\pgfqpoint{6.627563in}{4.706102in}}{\pgfqpoint{6.627563in}{4.700278in}}%
\pgfpathcurveto{\pgfqpoint{6.627563in}{4.694454in}}{\pgfqpoint{6.629877in}{4.688868in}}{\pgfqpoint{6.633995in}{4.684750in}}%
\pgfpathcurveto{\pgfqpoint{6.638113in}{4.680632in}}{\pgfqpoint{6.643699in}{4.678318in}}{\pgfqpoint{6.649523in}{4.678318in}}%
\pgfpathlineto{\pgfqpoint{6.649523in}{4.678318in}}%
\pgfpathclose%
\pgfusepath{stroke,fill}%
\end{pgfscope}%
\begin{pgfscope}%
\pgfpathrectangle{\pgfqpoint{1.000000in}{1.148311in}}{\pgfqpoint{6.200000in}{5.623377in}}%
\pgfusepath{clip}%
\pgfsetbuttcap%
\pgfsetroundjoin%
\definecolor{currentfill}{rgb}{0.800000,0.200000,0.200000}%
\pgfsetfillcolor{currentfill}%
\pgfsetlinewidth{1.003750pt}%
\definecolor{currentstroke}{rgb}{0.800000,0.200000,0.200000}%
\pgfsetstrokecolor{currentstroke}%
\pgfsetdash{}{0pt}%
\pgfpathmoveto{\pgfqpoint{6.627061in}{4.747336in}}%
\pgfpathcurveto{\pgfqpoint{6.632885in}{4.747336in}}{\pgfqpoint{6.638471in}{4.749649in}}{\pgfqpoint{6.642589in}{4.753768in}}%
\pgfpathcurveto{\pgfqpoint{6.646707in}{4.757886in}}{\pgfqpoint{6.649021in}{4.763472in}}{\pgfqpoint{6.649021in}{4.769296in}}%
\pgfpathcurveto{\pgfqpoint{6.649021in}{4.775120in}}{\pgfqpoint{6.646707in}{4.780706in}}{\pgfqpoint{6.642589in}{4.784824in}}%
\pgfpathcurveto{\pgfqpoint{6.638471in}{4.788942in}}{\pgfqpoint{6.632885in}{4.791256in}}{\pgfqpoint{6.627061in}{4.791256in}}%
\pgfpathcurveto{\pgfqpoint{6.621237in}{4.791256in}}{\pgfqpoint{6.615651in}{4.788942in}}{\pgfqpoint{6.611533in}{4.784824in}}%
\pgfpathcurveto{\pgfqpoint{6.607414in}{4.780706in}}{\pgfqpoint{6.605101in}{4.775120in}}{\pgfqpoint{6.605101in}{4.769296in}}%
\pgfpathcurveto{\pgfqpoint{6.605101in}{4.763472in}}{\pgfqpoint{6.607414in}{4.757886in}}{\pgfqpoint{6.611533in}{4.753768in}}%
\pgfpathcurveto{\pgfqpoint{6.615651in}{4.749649in}}{\pgfqpoint{6.621237in}{4.747336in}}{\pgfqpoint{6.627061in}{4.747336in}}%
\pgfpathlineto{\pgfqpoint{6.627061in}{4.747336in}}%
\pgfpathclose%
\pgfusepath{stroke,fill}%
\end{pgfscope}%
\begin{pgfscope}%
\pgfpathrectangle{\pgfqpoint{1.000000in}{1.148311in}}{\pgfqpoint{6.200000in}{5.623377in}}%
\pgfusepath{clip}%
\pgfsetbuttcap%
\pgfsetroundjoin%
\definecolor{currentfill}{rgb}{0.800000,0.200000,0.200000}%
\pgfsetfillcolor{currentfill}%
\pgfsetlinewidth{1.003750pt}%
\definecolor{currentstroke}{rgb}{0.800000,0.200000,0.200000}%
\pgfsetstrokecolor{currentstroke}%
\pgfsetdash{}{0pt}%
\pgfpathmoveto{\pgfqpoint{6.558859in}{4.777501in}}%
\pgfpathcurveto{\pgfqpoint{6.564683in}{4.777501in}}{\pgfqpoint{6.570269in}{4.779815in}}{\pgfqpoint{6.574387in}{4.783933in}}%
\pgfpathcurveto{\pgfqpoint{6.578505in}{4.788051in}}{\pgfqpoint{6.580819in}{4.793637in}}{\pgfqpoint{6.580819in}{4.799461in}}%
\pgfpathcurveto{\pgfqpoint{6.580819in}{4.805285in}}{\pgfqpoint{6.578505in}{4.810871in}}{\pgfqpoint{6.574387in}{4.814989in}}%
\pgfpathcurveto{\pgfqpoint{6.570269in}{4.819108in}}{\pgfqpoint{6.564683in}{4.821421in}}{\pgfqpoint{6.558859in}{4.821421in}}%
\pgfpathcurveto{\pgfqpoint{6.553035in}{4.821421in}}{\pgfqpoint{6.547449in}{4.819108in}}{\pgfqpoint{6.543331in}{4.814989in}}%
\pgfpathcurveto{\pgfqpoint{6.539213in}{4.810871in}}{\pgfqpoint{6.536899in}{4.805285in}}{\pgfqpoint{6.536899in}{4.799461in}}%
\pgfpathcurveto{\pgfqpoint{6.536899in}{4.793637in}}{\pgfqpoint{6.539213in}{4.788051in}}{\pgfqpoint{6.543331in}{4.783933in}}%
\pgfpathcurveto{\pgfqpoint{6.547449in}{4.779815in}}{\pgfqpoint{6.553035in}{4.777501in}}{\pgfqpoint{6.558859in}{4.777501in}}%
\pgfpathlineto{\pgfqpoint{6.558859in}{4.777501in}}%
\pgfpathclose%
\pgfusepath{stroke,fill}%
\end{pgfscope}%
\begin{pgfscope}%
\pgfpathrectangle{\pgfqpoint{1.000000in}{1.148311in}}{\pgfqpoint{6.200000in}{5.623377in}}%
\pgfusepath{clip}%
\pgfsetbuttcap%
\pgfsetroundjoin%
\definecolor{currentfill}{rgb}{0.800000,0.200000,0.200000}%
\pgfsetfillcolor{currentfill}%
\pgfsetlinewidth{1.003750pt}%
\definecolor{currentstroke}{rgb}{0.800000,0.200000,0.200000}%
\pgfsetstrokecolor{currentstroke}%
\pgfsetdash{}{0pt}%
\pgfpathmoveto{\pgfqpoint{6.494138in}{4.806098in}}%
\pgfpathcurveto{\pgfqpoint{6.499962in}{4.806098in}}{\pgfqpoint{6.505548in}{4.808412in}}{\pgfqpoint{6.509666in}{4.812531in}}%
\pgfpathcurveto{\pgfqpoint{6.513784in}{4.816649in}}{\pgfqpoint{6.516098in}{4.822235in}}{\pgfqpoint{6.516098in}{4.828059in}}%
\pgfpathcurveto{\pgfqpoint{6.516098in}{4.833883in}}{\pgfqpoint{6.513784in}{4.839469in}}{\pgfqpoint{6.509666in}{4.843587in}}%
\pgfpathcurveto{\pgfqpoint{6.505548in}{4.847705in}}{\pgfqpoint{6.499962in}{4.850019in}}{\pgfqpoint{6.494138in}{4.850019in}}%
\pgfpathcurveto{\pgfqpoint{6.488314in}{4.850019in}}{\pgfqpoint{6.482728in}{4.847705in}}{\pgfqpoint{6.478610in}{4.843587in}}%
\pgfpathcurveto{\pgfqpoint{6.474492in}{4.839469in}}{\pgfqpoint{6.472178in}{4.833883in}}{\pgfqpoint{6.472178in}{4.828059in}}%
\pgfpathcurveto{\pgfqpoint{6.472178in}{4.822235in}}{\pgfqpoint{6.474492in}{4.816649in}}{\pgfqpoint{6.478610in}{4.812531in}}%
\pgfpathcurveto{\pgfqpoint{6.482728in}{4.808412in}}{\pgfqpoint{6.488314in}{4.806098in}}{\pgfqpoint{6.494138in}{4.806098in}}%
\pgfpathlineto{\pgfqpoint{6.494138in}{4.806098in}}%
\pgfpathclose%
\pgfusepath{stroke,fill}%
\end{pgfscope}%
\begin{pgfscope}%
\pgfpathrectangle{\pgfqpoint{1.000000in}{1.148311in}}{\pgfqpoint{6.200000in}{5.623377in}}%
\pgfusepath{clip}%
\pgfsetbuttcap%
\pgfsetroundjoin%
\definecolor{currentfill}{rgb}{0.200000,0.200000,0.800000}%
\pgfsetfillcolor{currentfill}%
\pgfsetlinewidth{1.003750pt}%
\definecolor{currentstroke}{rgb}{0.200000,0.200000,0.800000}%
\pgfsetstrokecolor{currentstroke}%
\pgfsetdash{}{0pt}%
\pgfpathmoveto{\pgfqpoint{6.486033in}{4.899777in}}%
\pgfpathcurveto{\pgfqpoint{6.491857in}{4.899777in}}{\pgfqpoint{6.497443in}{4.902091in}}{\pgfqpoint{6.501561in}{4.906209in}}%
\pgfpathcurveto{\pgfqpoint{6.505680in}{4.910327in}}{\pgfqpoint{6.507993in}{4.915913in}}{\pgfqpoint{6.507993in}{4.921737in}}%
\pgfpathcurveto{\pgfqpoint{6.507993in}{4.927561in}}{\pgfqpoint{6.505680in}{4.933147in}}{\pgfqpoint{6.501561in}{4.937265in}}%
\pgfpathcurveto{\pgfqpoint{6.497443in}{4.941383in}}{\pgfqpoint{6.491857in}{4.943697in}}{\pgfqpoint{6.486033in}{4.943697in}}%
\pgfpathcurveto{\pgfqpoint{6.480209in}{4.943697in}}{\pgfqpoint{6.474623in}{4.941383in}}{\pgfqpoint{6.470505in}{4.937265in}}%
\pgfpathcurveto{\pgfqpoint{6.466387in}{4.933147in}}{\pgfqpoint{6.464073in}{4.927561in}}{\pgfqpoint{6.464073in}{4.921737in}}%
\pgfpathcurveto{\pgfqpoint{6.464073in}{4.915913in}}{\pgfqpoint{6.466387in}{4.910327in}}{\pgfqpoint{6.470505in}{4.906209in}}%
\pgfpathcurveto{\pgfqpoint{6.474623in}{4.902091in}}{\pgfqpoint{6.480209in}{4.899777in}}{\pgfqpoint{6.486033in}{4.899777in}}%
\pgfpathlineto{\pgfqpoint{6.486033in}{4.899777in}}%
\pgfpathclose%
\pgfusepath{stroke,fill}%
\end{pgfscope}%
\begin{pgfscope}%
\pgfpathrectangle{\pgfqpoint{1.000000in}{1.148311in}}{\pgfqpoint{6.200000in}{5.623377in}}%
\pgfusepath{clip}%
\pgfsetbuttcap%
\pgfsetroundjoin%
\definecolor{currentfill}{rgb}{0.800000,0.200000,0.200000}%
\pgfsetfillcolor{currentfill}%
\pgfsetlinewidth{1.003750pt}%
\definecolor{currentstroke}{rgb}{0.800000,0.200000,0.200000}%
\pgfsetstrokecolor{currentstroke}%
\pgfsetdash{}{0pt}%
\pgfpathmoveto{\pgfqpoint{6.377968in}{4.867862in}}%
\pgfpathcurveto{\pgfqpoint{6.383792in}{4.867862in}}{\pgfqpoint{6.389378in}{4.870176in}}{\pgfqpoint{6.393496in}{4.874294in}}%
\pgfpathcurveto{\pgfqpoint{6.397614in}{4.878413in}}{\pgfqpoint{6.399928in}{4.883999in}}{\pgfqpoint{6.399928in}{4.889823in}}%
\pgfpathcurveto{\pgfqpoint{6.399928in}{4.895647in}}{\pgfqpoint{6.397614in}{4.901233in}}{\pgfqpoint{6.393496in}{4.905351in}}%
\pgfpathcurveto{\pgfqpoint{6.389378in}{4.909469in}}{\pgfqpoint{6.383792in}{4.911783in}}{\pgfqpoint{6.377968in}{4.911783in}}%
\pgfpathcurveto{\pgfqpoint{6.372144in}{4.911783in}}{\pgfqpoint{6.366558in}{4.909469in}}{\pgfqpoint{6.362439in}{4.905351in}}%
\pgfpathcurveto{\pgfqpoint{6.358321in}{4.901233in}}{\pgfqpoint{6.356007in}{4.895647in}}{\pgfqpoint{6.356007in}{4.889823in}}%
\pgfpathcurveto{\pgfqpoint{6.356007in}{4.883999in}}{\pgfqpoint{6.358321in}{4.878413in}}{\pgfqpoint{6.362439in}{4.874294in}}%
\pgfpathcurveto{\pgfqpoint{6.366558in}{4.870176in}}{\pgfqpoint{6.372144in}{4.867862in}}{\pgfqpoint{6.377968in}{4.867862in}}%
\pgfpathlineto{\pgfqpoint{6.377968in}{4.867862in}}%
\pgfpathclose%
\pgfusepath{stroke,fill}%
\end{pgfscope}%
\begin{pgfscope}%
\pgfpathrectangle{\pgfqpoint{1.000000in}{1.148311in}}{\pgfqpoint{6.200000in}{5.623377in}}%
\pgfusepath{clip}%
\pgfsetbuttcap%
\pgfsetroundjoin%
\definecolor{currentfill}{rgb}{0.800000,0.200000,0.200000}%
\pgfsetfillcolor{currentfill}%
\pgfsetlinewidth{1.003750pt}%
\definecolor{currentstroke}{rgb}{0.800000,0.200000,0.200000}%
\pgfsetstrokecolor{currentstroke}%
\pgfsetdash{}{0pt}%
\pgfpathmoveto{\pgfqpoint{6.349784in}{4.943357in}}%
\pgfpathcurveto{\pgfqpoint{6.355608in}{4.943357in}}{\pgfqpoint{6.361194in}{4.945671in}}{\pgfqpoint{6.365312in}{4.949789in}}%
\pgfpathcurveto{\pgfqpoint{6.369431in}{4.953907in}}{\pgfqpoint{6.371744in}{4.959493in}}{\pgfqpoint{6.371744in}{4.965317in}}%
\pgfpathcurveto{\pgfqpoint{6.371744in}{4.971141in}}{\pgfqpoint{6.369431in}{4.976727in}}{\pgfqpoint{6.365312in}{4.980845in}}%
\pgfpathcurveto{\pgfqpoint{6.361194in}{4.984964in}}{\pgfqpoint{6.355608in}{4.987277in}}{\pgfqpoint{6.349784in}{4.987277in}}%
\pgfpathcurveto{\pgfqpoint{6.343960in}{4.987277in}}{\pgfqpoint{6.338374in}{4.984964in}}{\pgfqpoint{6.334256in}{4.980845in}}%
\pgfpathcurveto{\pgfqpoint{6.330138in}{4.976727in}}{\pgfqpoint{6.327824in}{4.971141in}}{\pgfqpoint{6.327824in}{4.965317in}}%
\pgfpathcurveto{\pgfqpoint{6.327824in}{4.959493in}}{\pgfqpoint{6.330138in}{4.953907in}}{\pgfqpoint{6.334256in}{4.949789in}}%
\pgfpathcurveto{\pgfqpoint{6.338374in}{4.945671in}}{\pgfqpoint{6.343960in}{4.943357in}}{\pgfqpoint{6.349784in}{4.943357in}}%
\pgfpathlineto{\pgfqpoint{6.349784in}{4.943357in}}%
\pgfpathclose%
\pgfusepath{stroke,fill}%
\end{pgfscope}%
\begin{pgfscope}%
\pgfpathrectangle{\pgfqpoint{1.000000in}{1.148311in}}{\pgfqpoint{6.200000in}{5.623377in}}%
\pgfusepath{clip}%
\pgfsetbuttcap%
\pgfsetroundjoin%
\definecolor{currentfill}{rgb}{0.800000,0.200000,0.200000}%
\pgfsetfillcolor{currentfill}%
\pgfsetlinewidth{1.003750pt}%
\definecolor{currentstroke}{rgb}{0.800000,0.200000,0.200000}%
\pgfsetstrokecolor{currentstroke}%
\pgfsetdash{}{0pt}%
\pgfpathmoveto{\pgfqpoint{6.281999in}{4.958178in}}%
\pgfpathcurveto{\pgfqpoint{6.287823in}{4.958178in}}{\pgfqpoint{6.293410in}{4.960492in}}{\pgfqpoint{6.297528in}{4.964610in}}%
\pgfpathcurveto{\pgfqpoint{6.301646in}{4.968728in}}{\pgfqpoint{6.303960in}{4.974315in}}{\pgfqpoint{6.303960in}{4.980138in}}%
\pgfpathcurveto{\pgfqpoint{6.303960in}{4.985962in}}{\pgfqpoint{6.301646in}{4.991549in}}{\pgfqpoint{6.297528in}{4.995667in}}%
\pgfpathcurveto{\pgfqpoint{6.293410in}{4.999785in}}{\pgfqpoint{6.287823in}{5.002099in}}{\pgfqpoint{6.281999in}{5.002099in}}%
\pgfpathcurveto{\pgfqpoint{6.276175in}{5.002099in}}{\pgfqpoint{6.270589in}{4.999785in}}{\pgfqpoint{6.266471in}{4.995667in}}%
\pgfpathcurveto{\pgfqpoint{6.262353in}{4.991549in}}{\pgfqpoint{6.260039in}{4.985962in}}{\pgfqpoint{6.260039in}{4.980138in}}%
\pgfpathcurveto{\pgfqpoint{6.260039in}{4.974315in}}{\pgfqpoint{6.262353in}{4.968728in}}{\pgfqpoint{6.266471in}{4.964610in}}%
\pgfpathcurveto{\pgfqpoint{6.270589in}{4.960492in}}{\pgfqpoint{6.276175in}{4.958178in}}{\pgfqpoint{6.281999in}{4.958178in}}%
\pgfpathlineto{\pgfqpoint{6.281999in}{4.958178in}}%
\pgfpathclose%
\pgfusepath{stroke,fill}%
\end{pgfscope}%
\begin{pgfscope}%
\pgfpathrectangle{\pgfqpoint{1.000000in}{1.148311in}}{\pgfqpoint{6.200000in}{5.623377in}}%
\pgfusepath{clip}%
\pgfsetbuttcap%
\pgfsetroundjoin%
\definecolor{currentfill}{rgb}{0.800000,0.200000,0.200000}%
\pgfsetfillcolor{currentfill}%
\pgfsetlinewidth{1.003750pt}%
\definecolor{currentstroke}{rgb}{0.800000,0.200000,0.200000}%
\pgfsetstrokecolor{currentstroke}%
\pgfsetdash{}{0pt}%
\pgfpathmoveto{\pgfqpoint{6.246555in}{5.038030in}}%
\pgfpathcurveto{\pgfqpoint{6.252379in}{5.038030in}}{\pgfqpoint{6.257966in}{5.040344in}}{\pgfqpoint{6.262084in}{5.044462in}}%
\pgfpathcurveto{\pgfqpoint{6.266202in}{5.048580in}}{\pgfqpoint{6.268516in}{5.054166in}}{\pgfqpoint{6.268516in}{5.059990in}}%
\pgfpathcurveto{\pgfqpoint{6.268516in}{5.065814in}}{\pgfqpoint{6.266202in}{5.071400in}}{\pgfqpoint{6.262084in}{5.075519in}}%
\pgfpathcurveto{\pgfqpoint{6.257966in}{5.079637in}}{\pgfqpoint{6.252379in}{5.081951in}}{\pgfqpoint{6.246555in}{5.081951in}}%
\pgfpathcurveto{\pgfqpoint{6.240732in}{5.081951in}}{\pgfqpoint{6.235145in}{5.079637in}}{\pgfqpoint{6.231027in}{5.075519in}}%
\pgfpathcurveto{\pgfqpoint{6.226909in}{5.071400in}}{\pgfqpoint{6.224595in}{5.065814in}}{\pgfqpoint{6.224595in}{5.059990in}}%
\pgfpathcurveto{\pgfqpoint{6.224595in}{5.054166in}}{\pgfqpoint{6.226909in}{5.048580in}}{\pgfqpoint{6.231027in}{5.044462in}}%
\pgfpathcurveto{\pgfqpoint{6.235145in}{5.040344in}}{\pgfqpoint{6.240732in}{5.038030in}}{\pgfqpoint{6.246555in}{5.038030in}}%
\pgfpathlineto{\pgfqpoint{6.246555in}{5.038030in}}%
\pgfpathclose%
\pgfusepath{stroke,fill}%
\end{pgfscope}%
\begin{pgfscope}%
\pgfpathrectangle{\pgfqpoint{1.000000in}{1.148311in}}{\pgfqpoint{6.200000in}{5.623377in}}%
\pgfusepath{clip}%
\pgfsetbuttcap%
\pgfsetroundjoin%
\definecolor{currentfill}{rgb}{0.800000,0.200000,0.200000}%
\pgfsetfillcolor{currentfill}%
\pgfsetlinewidth{1.003750pt}%
\definecolor{currentstroke}{rgb}{0.800000,0.200000,0.200000}%
\pgfsetstrokecolor{currentstroke}%
\pgfsetdash{}{0pt}%
\pgfpathmoveto{\pgfqpoint{6.157655in}{4.998544in}}%
\pgfpathcurveto{\pgfqpoint{6.163479in}{4.998544in}}{\pgfqpoint{6.169066in}{5.000858in}}{\pgfqpoint{6.173184in}{5.004976in}}%
\pgfpathcurveto{\pgfqpoint{6.177302in}{5.009094in}}{\pgfqpoint{6.179616in}{5.014681in}}{\pgfqpoint{6.179616in}{5.020505in}}%
\pgfpathcurveto{\pgfqpoint{6.179616in}{5.026328in}}{\pgfqpoint{6.177302in}{5.031915in}}{\pgfqpoint{6.173184in}{5.036033in}}%
\pgfpathcurveto{\pgfqpoint{6.169066in}{5.040151in}}{\pgfqpoint{6.163479in}{5.042465in}}{\pgfqpoint{6.157655in}{5.042465in}}%
\pgfpathcurveto{\pgfqpoint{6.151831in}{5.042465in}}{\pgfqpoint{6.146245in}{5.040151in}}{\pgfqpoint{6.142127in}{5.036033in}}%
\pgfpathcurveto{\pgfqpoint{6.138009in}{5.031915in}}{\pgfqpoint{6.135695in}{5.026328in}}{\pgfqpoint{6.135695in}{5.020505in}}%
\pgfpathcurveto{\pgfqpoint{6.135695in}{5.014681in}}{\pgfqpoint{6.138009in}{5.009094in}}{\pgfqpoint{6.142127in}{5.004976in}}%
\pgfpathcurveto{\pgfqpoint{6.146245in}{5.000858in}}{\pgfqpoint{6.151831in}{4.998544in}}{\pgfqpoint{6.157655in}{4.998544in}}%
\pgfpathlineto{\pgfqpoint{6.157655in}{4.998544in}}%
\pgfpathclose%
\pgfusepath{stroke,fill}%
\end{pgfscope}%
\begin{pgfscope}%
\pgfpathrectangle{\pgfqpoint{1.000000in}{1.148311in}}{\pgfqpoint{6.200000in}{5.623377in}}%
\pgfusepath{clip}%
\pgfsetbuttcap%
\pgfsetroundjoin%
\definecolor{currentfill}{rgb}{0.800000,0.200000,0.200000}%
\pgfsetfillcolor{currentfill}%
\pgfsetlinewidth{1.003750pt}%
\definecolor{currentstroke}{rgb}{0.800000,0.200000,0.200000}%
\pgfsetstrokecolor{currentstroke}%
\pgfsetdash{}{0pt}%
\pgfpathmoveto{\pgfqpoint{6.103028in}{5.039967in}}%
\pgfpathcurveto{\pgfqpoint{6.108852in}{5.039967in}}{\pgfqpoint{6.114438in}{5.042281in}}{\pgfqpoint{6.118556in}{5.046399in}}%
\pgfpathcurveto{\pgfqpoint{6.122674in}{5.050518in}}{\pgfqpoint{6.124988in}{5.056104in}}{\pgfqpoint{6.124988in}{5.061928in}}%
\pgfpathcurveto{\pgfqpoint{6.124988in}{5.067752in}}{\pgfqpoint{6.122674in}{5.073338in}}{\pgfqpoint{6.118556in}{5.077456in}}%
\pgfpathcurveto{\pgfqpoint{6.114438in}{5.081574in}}{\pgfqpoint{6.108852in}{5.083888in}}{\pgfqpoint{6.103028in}{5.083888in}}%
\pgfpathcurveto{\pgfqpoint{6.097204in}{5.083888in}}{\pgfqpoint{6.091618in}{5.081574in}}{\pgfqpoint{6.087500in}{5.077456in}}%
\pgfpathcurveto{\pgfqpoint{6.083381in}{5.073338in}}{\pgfqpoint{6.081068in}{5.067752in}}{\pgfqpoint{6.081068in}{5.061928in}}%
\pgfpathcurveto{\pgfqpoint{6.081068in}{5.056104in}}{\pgfqpoint{6.083381in}{5.050518in}}{\pgfqpoint{6.087500in}{5.046399in}}%
\pgfpathcurveto{\pgfqpoint{6.091618in}{5.042281in}}{\pgfqpoint{6.097204in}{5.039967in}}{\pgfqpoint{6.103028in}{5.039967in}}%
\pgfpathlineto{\pgfqpoint{6.103028in}{5.039967in}}%
\pgfpathclose%
\pgfusepath{stroke,fill}%
\end{pgfscope}%
\begin{pgfscope}%
\pgfpathrectangle{\pgfqpoint{1.000000in}{1.148311in}}{\pgfqpoint{6.200000in}{5.623377in}}%
\pgfusepath{clip}%
\pgfsetbuttcap%
\pgfsetroundjoin%
\definecolor{currentfill}{rgb}{0.800000,0.200000,0.200000}%
\pgfsetfillcolor{currentfill}%
\pgfsetlinewidth{1.003750pt}%
\definecolor{currentstroke}{rgb}{0.800000,0.200000,0.200000}%
\pgfsetstrokecolor{currentstroke}%
\pgfsetdash{}{0pt}%
\pgfpathmoveto{\pgfqpoint{6.043787in}{5.074929in}}%
\pgfpathcurveto{\pgfqpoint{6.049611in}{5.074929in}}{\pgfqpoint{6.055197in}{5.077243in}}{\pgfqpoint{6.059315in}{5.081361in}}%
\pgfpathcurveto{\pgfqpoint{6.063433in}{5.085479in}}{\pgfqpoint{6.065747in}{5.091065in}}{\pgfqpoint{6.065747in}{5.096889in}}%
\pgfpathcurveto{\pgfqpoint{6.065747in}{5.102713in}}{\pgfqpoint{6.063433in}{5.108299in}}{\pgfqpoint{6.059315in}{5.112417in}}%
\pgfpathcurveto{\pgfqpoint{6.055197in}{5.116535in}}{\pgfqpoint{6.049611in}{5.118849in}}{\pgfqpoint{6.043787in}{5.118849in}}%
\pgfpathcurveto{\pgfqpoint{6.037963in}{5.118849in}}{\pgfqpoint{6.032377in}{5.116535in}}{\pgfqpoint{6.028259in}{5.112417in}}%
\pgfpathcurveto{\pgfqpoint{6.024140in}{5.108299in}}{\pgfqpoint{6.021827in}{5.102713in}}{\pgfqpoint{6.021827in}{5.096889in}}%
\pgfpathcurveto{\pgfqpoint{6.021827in}{5.091065in}}{\pgfqpoint{6.024140in}{5.085479in}}{\pgfqpoint{6.028259in}{5.081361in}}%
\pgfpathcurveto{\pgfqpoint{6.032377in}{5.077243in}}{\pgfqpoint{6.037963in}{5.074929in}}{\pgfqpoint{6.043787in}{5.074929in}}%
\pgfpathlineto{\pgfqpoint{6.043787in}{5.074929in}}%
\pgfpathclose%
\pgfusepath{stroke,fill}%
\end{pgfscope}%
\begin{pgfscope}%
\pgfpathrectangle{\pgfqpoint{1.000000in}{1.148311in}}{\pgfqpoint{6.200000in}{5.623377in}}%
\pgfusepath{clip}%
\pgfsetbuttcap%
\pgfsetroundjoin%
\definecolor{currentfill}{rgb}{0.800000,0.200000,0.200000}%
\pgfsetfillcolor{currentfill}%
\pgfsetlinewidth{1.003750pt}%
\definecolor{currentstroke}{rgb}{0.800000,0.200000,0.200000}%
\pgfsetstrokecolor{currentstroke}%
\pgfsetdash{}{0pt}%
\pgfpathmoveto{\pgfqpoint{5.988146in}{5.144927in}}%
\pgfpathcurveto{\pgfqpoint{5.993970in}{5.144927in}}{\pgfqpoint{5.999556in}{5.147241in}}{\pgfqpoint{6.003675in}{5.151359in}}%
\pgfpathcurveto{\pgfqpoint{6.007793in}{5.155477in}}{\pgfqpoint{6.010107in}{5.161063in}}{\pgfqpoint{6.010107in}{5.166887in}}%
\pgfpathcurveto{\pgfqpoint{6.010107in}{5.172711in}}{\pgfqpoint{6.007793in}{5.178297in}}{\pgfqpoint{6.003675in}{5.182415in}}%
\pgfpathcurveto{\pgfqpoint{5.999556in}{5.186533in}}{\pgfqpoint{5.993970in}{5.188847in}}{\pgfqpoint{5.988146in}{5.188847in}}%
\pgfpathcurveto{\pgfqpoint{5.982322in}{5.188847in}}{\pgfqpoint{5.976736in}{5.186533in}}{\pgfqpoint{5.972618in}{5.182415in}}%
\pgfpathcurveto{\pgfqpoint{5.968500in}{5.178297in}}{\pgfqpoint{5.966186in}{5.172711in}}{\pgfqpoint{5.966186in}{5.166887in}}%
\pgfpathcurveto{\pgfqpoint{5.966186in}{5.161063in}}{\pgfqpoint{5.968500in}{5.155477in}}{\pgfqpoint{5.972618in}{5.151359in}}%
\pgfpathcurveto{\pgfqpoint{5.976736in}{5.147241in}}{\pgfqpoint{5.982322in}{5.144927in}}{\pgfqpoint{5.988146in}{5.144927in}}%
\pgfpathlineto{\pgfqpoint{5.988146in}{5.144927in}}%
\pgfpathclose%
\pgfusepath{stroke,fill}%
\end{pgfscope}%
\begin{pgfscope}%
\pgfpathrectangle{\pgfqpoint{1.000000in}{1.148311in}}{\pgfqpoint{6.200000in}{5.623377in}}%
\pgfusepath{clip}%
\pgfsetbuttcap%
\pgfsetroundjoin%
\definecolor{currentfill}{rgb}{0.800000,0.200000,0.200000}%
\pgfsetfillcolor{currentfill}%
\pgfsetlinewidth{1.003750pt}%
\definecolor{currentstroke}{rgb}{0.800000,0.200000,0.200000}%
\pgfsetstrokecolor{currentstroke}%
\pgfsetdash{}{0pt}%
\pgfpathmoveto{\pgfqpoint{5.909392in}{5.075505in}}%
\pgfpathcurveto{\pgfqpoint{5.915216in}{5.075505in}}{\pgfqpoint{5.920802in}{5.077819in}}{\pgfqpoint{5.924920in}{5.081937in}}%
\pgfpathcurveto{\pgfqpoint{5.929038in}{5.086055in}}{\pgfqpoint{5.931352in}{5.091642in}}{\pgfqpoint{5.931352in}{5.097465in}}%
\pgfpathcurveto{\pgfqpoint{5.931352in}{5.103289in}}{\pgfqpoint{5.929038in}{5.108876in}}{\pgfqpoint{5.924920in}{5.112994in}}%
\pgfpathcurveto{\pgfqpoint{5.920802in}{5.117112in}}{\pgfqpoint{5.915216in}{5.119426in}}{\pgfqpoint{5.909392in}{5.119426in}}%
\pgfpathcurveto{\pgfqpoint{5.903568in}{5.119426in}}{\pgfqpoint{5.897982in}{5.117112in}}{\pgfqpoint{5.893864in}{5.112994in}}%
\pgfpathcurveto{\pgfqpoint{5.889745in}{5.108876in}}{\pgfqpoint{5.887432in}{5.103289in}}{\pgfqpoint{5.887432in}{5.097465in}}%
\pgfpathcurveto{\pgfqpoint{5.887432in}{5.091642in}}{\pgfqpoint{5.889745in}{5.086055in}}{\pgfqpoint{5.893864in}{5.081937in}}%
\pgfpathcurveto{\pgfqpoint{5.897982in}{5.077819in}}{\pgfqpoint{5.903568in}{5.075505in}}{\pgfqpoint{5.909392in}{5.075505in}}%
\pgfpathlineto{\pgfqpoint{5.909392in}{5.075505in}}%
\pgfpathclose%
\pgfusepath{stroke,fill}%
\end{pgfscope}%
\begin{pgfscope}%
\pgfpathrectangle{\pgfqpoint{1.000000in}{1.148311in}}{\pgfqpoint{6.200000in}{5.623377in}}%
\pgfusepath{clip}%
\pgfsetbuttcap%
\pgfsetroundjoin%
\definecolor{currentfill}{rgb}{0.800000,0.200000,0.200000}%
\pgfsetfillcolor{currentfill}%
\pgfsetlinewidth{1.003750pt}%
\definecolor{currentstroke}{rgb}{0.800000,0.200000,0.200000}%
\pgfsetstrokecolor{currentstroke}%
\pgfsetdash{}{0pt}%
\pgfpathmoveto{\pgfqpoint{5.844310in}{5.085903in}}%
\pgfpathcurveto{\pgfqpoint{5.850134in}{5.085903in}}{\pgfqpoint{5.855720in}{5.088217in}}{\pgfqpoint{5.859838in}{5.092335in}}%
\pgfpathcurveto{\pgfqpoint{5.863956in}{5.096453in}}{\pgfqpoint{5.866270in}{5.102039in}}{\pgfqpoint{5.866270in}{5.107863in}}%
\pgfpathcurveto{\pgfqpoint{5.866270in}{5.113687in}}{\pgfqpoint{5.863956in}{5.119273in}}{\pgfqpoint{5.859838in}{5.123391in}}%
\pgfpathcurveto{\pgfqpoint{5.855720in}{5.127510in}}{\pgfqpoint{5.850134in}{5.129823in}}{\pgfqpoint{5.844310in}{5.129823in}}%
\pgfpathcurveto{\pgfqpoint{5.838486in}{5.129823in}}{\pgfqpoint{5.832900in}{5.127510in}}{\pgfqpoint{5.828782in}{5.123391in}}%
\pgfpathcurveto{\pgfqpoint{5.824663in}{5.119273in}}{\pgfqpoint{5.822350in}{5.113687in}}{\pgfqpoint{5.822350in}{5.107863in}}%
\pgfpathcurveto{\pgfqpoint{5.822350in}{5.102039in}}{\pgfqpoint{5.824663in}{5.096453in}}{\pgfqpoint{5.828782in}{5.092335in}}%
\pgfpathcurveto{\pgfqpoint{5.832900in}{5.088217in}}{\pgfqpoint{5.838486in}{5.085903in}}{\pgfqpoint{5.844310in}{5.085903in}}%
\pgfpathlineto{\pgfqpoint{5.844310in}{5.085903in}}%
\pgfpathclose%
\pgfusepath{stroke,fill}%
\end{pgfscope}%
\begin{pgfscope}%
\pgfpathrectangle{\pgfqpoint{1.000000in}{1.148311in}}{\pgfqpoint{6.200000in}{5.623377in}}%
\pgfusepath{clip}%
\pgfsetbuttcap%
\pgfsetroundjoin%
\definecolor{currentfill}{rgb}{0.800000,0.200000,0.200000}%
\pgfsetfillcolor{currentfill}%
\pgfsetlinewidth{1.003750pt}%
\definecolor{currentstroke}{rgb}{0.800000,0.200000,0.200000}%
\pgfsetstrokecolor{currentstroke}%
\pgfsetdash{}{0pt}%
\pgfpathmoveto{\pgfqpoint{5.779046in}{5.049457in}}%
\pgfpathcurveto{\pgfqpoint{5.784870in}{5.049457in}}{\pgfqpoint{5.790456in}{5.051771in}}{\pgfqpoint{5.794575in}{5.055889in}}%
\pgfpathcurveto{\pgfqpoint{5.798693in}{5.060007in}}{\pgfqpoint{5.801007in}{5.065593in}}{\pgfqpoint{5.801007in}{5.071417in}}%
\pgfpathcurveto{\pgfqpoint{5.801007in}{5.077241in}}{\pgfqpoint{5.798693in}{5.082828in}}{\pgfqpoint{5.794575in}{5.086946in}}%
\pgfpathcurveto{\pgfqpoint{5.790456in}{5.091064in}}{\pgfqpoint{5.784870in}{5.093378in}}{\pgfqpoint{5.779046in}{5.093378in}}%
\pgfpathcurveto{\pgfqpoint{5.773222in}{5.093378in}}{\pgfqpoint{5.767636in}{5.091064in}}{\pgfqpoint{5.763518in}{5.086946in}}%
\pgfpathcurveto{\pgfqpoint{5.759400in}{5.082828in}}{\pgfqpoint{5.757086in}{5.077241in}}{\pgfqpoint{5.757086in}{5.071417in}}%
\pgfpathcurveto{\pgfqpoint{5.757086in}{5.065593in}}{\pgfqpoint{5.759400in}{5.060007in}}{\pgfqpoint{5.763518in}{5.055889in}}%
\pgfpathcurveto{\pgfqpoint{5.767636in}{5.051771in}}{\pgfqpoint{5.773222in}{5.049457in}}{\pgfqpoint{5.779046in}{5.049457in}}%
\pgfpathlineto{\pgfqpoint{5.779046in}{5.049457in}}%
\pgfpathclose%
\pgfusepath{stroke,fill}%
\end{pgfscope}%
\begin{pgfscope}%
\pgfpathrectangle{\pgfqpoint{1.000000in}{1.148311in}}{\pgfqpoint{6.200000in}{5.623377in}}%
\pgfusepath{clip}%
\pgfsetbuttcap%
\pgfsetroundjoin%
\definecolor{currentfill}{rgb}{0.800000,0.200000,0.200000}%
\pgfsetfillcolor{currentfill}%
\pgfsetlinewidth{1.003750pt}%
\definecolor{currentstroke}{rgb}{0.800000,0.200000,0.200000}%
\pgfsetstrokecolor{currentstroke}%
\pgfsetdash{}{0pt}%
\pgfpathmoveto{\pgfqpoint{5.714485in}{5.060846in}}%
\pgfpathcurveto{\pgfqpoint{5.720309in}{5.060846in}}{\pgfqpoint{5.725895in}{5.063160in}}{\pgfqpoint{5.730014in}{5.067278in}}%
\pgfpathcurveto{\pgfqpoint{5.734132in}{5.071396in}}{\pgfqpoint{5.736446in}{5.076982in}}{\pgfqpoint{5.736446in}{5.082806in}}%
\pgfpathcurveto{\pgfqpoint{5.736446in}{5.088630in}}{\pgfqpoint{5.734132in}{5.094216in}}{\pgfqpoint{5.730014in}{5.098335in}}%
\pgfpathcurveto{\pgfqpoint{5.725895in}{5.102453in}}{\pgfqpoint{5.720309in}{5.104767in}}{\pgfqpoint{5.714485in}{5.104767in}}%
\pgfpathcurveto{\pgfqpoint{5.708661in}{5.104767in}}{\pgfqpoint{5.703075in}{5.102453in}}{\pgfqpoint{5.698957in}{5.098335in}}%
\pgfpathcurveto{\pgfqpoint{5.694839in}{5.094216in}}{\pgfqpoint{5.692525in}{5.088630in}}{\pgfqpoint{5.692525in}{5.082806in}}%
\pgfpathcurveto{\pgfqpoint{5.692525in}{5.076982in}}{\pgfqpoint{5.694839in}{5.071396in}}{\pgfqpoint{5.698957in}{5.067278in}}%
\pgfpathcurveto{\pgfqpoint{5.703075in}{5.063160in}}{\pgfqpoint{5.708661in}{5.060846in}}{\pgfqpoint{5.714485in}{5.060846in}}%
\pgfpathlineto{\pgfqpoint{5.714485in}{5.060846in}}%
\pgfpathclose%
\pgfusepath{stroke,fill}%
\end{pgfscope}%
\begin{pgfscope}%
\pgfpathrectangle{\pgfqpoint{1.000000in}{1.148311in}}{\pgfqpoint{6.200000in}{5.623377in}}%
\pgfusepath{clip}%
\pgfsetbuttcap%
\pgfsetroundjoin%
\definecolor{currentfill}{rgb}{0.800000,0.200000,0.200000}%
\pgfsetfillcolor{currentfill}%
\pgfsetlinewidth{1.003750pt}%
\definecolor{currentstroke}{rgb}{0.800000,0.200000,0.200000}%
\pgfsetstrokecolor{currentstroke}%
\pgfsetdash{}{0pt}%
\pgfpathmoveto{\pgfqpoint{5.642614in}{5.108399in}}%
\pgfpathcurveto{\pgfqpoint{5.648438in}{5.108399in}}{\pgfqpoint{5.654024in}{5.110713in}}{\pgfqpoint{5.658142in}{5.114831in}}%
\pgfpathcurveto{\pgfqpoint{5.662260in}{5.118950in}}{\pgfqpoint{5.664574in}{5.124536in}}{\pgfqpoint{5.664574in}{5.130360in}}%
\pgfpathcurveto{\pgfqpoint{5.664574in}{5.136184in}}{\pgfqpoint{5.662260in}{5.141770in}}{\pgfqpoint{5.658142in}{5.145888in}}%
\pgfpathcurveto{\pgfqpoint{5.654024in}{5.150006in}}{\pgfqpoint{5.648438in}{5.152320in}}{\pgfqpoint{5.642614in}{5.152320in}}%
\pgfpathcurveto{\pgfqpoint{5.636790in}{5.152320in}}{\pgfqpoint{5.631204in}{5.150006in}}{\pgfqpoint{5.627086in}{5.145888in}}%
\pgfpathcurveto{\pgfqpoint{5.622968in}{5.141770in}}{\pgfqpoint{5.620654in}{5.136184in}}{\pgfqpoint{5.620654in}{5.130360in}}%
\pgfpathcurveto{\pgfqpoint{5.620654in}{5.124536in}}{\pgfqpoint{5.622968in}{5.118950in}}{\pgfqpoint{5.627086in}{5.114831in}}%
\pgfpathcurveto{\pgfqpoint{5.631204in}{5.110713in}}{\pgfqpoint{5.636790in}{5.108399in}}{\pgfqpoint{5.642614in}{5.108399in}}%
\pgfpathlineto{\pgfqpoint{5.642614in}{5.108399in}}%
\pgfpathclose%
\pgfusepath{stroke,fill}%
\end{pgfscope}%
\begin{pgfscope}%
\pgfpathrectangle{\pgfqpoint{1.000000in}{1.148311in}}{\pgfqpoint{6.200000in}{5.623377in}}%
\pgfusepath{clip}%
\pgfsetbuttcap%
\pgfsetroundjoin%
\definecolor{currentfill}{rgb}{0.800000,0.200000,0.200000}%
\pgfsetfillcolor{currentfill}%
\pgfsetlinewidth{1.003750pt}%
\definecolor{currentstroke}{rgb}{0.800000,0.200000,0.200000}%
\pgfsetstrokecolor{currentstroke}%
\pgfsetdash{}{0pt}%
\pgfpathmoveto{\pgfqpoint{5.578823in}{5.081849in}}%
\pgfpathcurveto{\pgfqpoint{5.584647in}{5.081849in}}{\pgfqpoint{5.590233in}{5.084163in}}{\pgfqpoint{5.594351in}{5.088281in}}%
\pgfpathcurveto{\pgfqpoint{5.598469in}{5.092400in}}{\pgfqpoint{5.600783in}{5.097986in}}{\pgfqpoint{5.600783in}{5.103810in}}%
\pgfpathcurveto{\pgfqpoint{5.600783in}{5.109634in}}{\pgfqpoint{5.598469in}{5.115220in}}{\pgfqpoint{5.594351in}{5.119338in}}%
\pgfpathcurveto{\pgfqpoint{5.590233in}{5.123456in}}{\pgfqpoint{5.584647in}{5.125770in}}{\pgfqpoint{5.578823in}{5.125770in}}%
\pgfpathcurveto{\pgfqpoint{5.572999in}{5.125770in}}{\pgfqpoint{5.567413in}{5.123456in}}{\pgfqpoint{5.563294in}{5.119338in}}%
\pgfpathcurveto{\pgfqpoint{5.559176in}{5.115220in}}{\pgfqpoint{5.556862in}{5.109634in}}{\pgfqpoint{5.556862in}{5.103810in}}%
\pgfpathcurveto{\pgfqpoint{5.556862in}{5.097986in}}{\pgfqpoint{5.559176in}{5.092400in}}{\pgfqpoint{5.563294in}{5.088281in}}%
\pgfpathcurveto{\pgfqpoint{5.567413in}{5.084163in}}{\pgfqpoint{5.572999in}{5.081849in}}{\pgfqpoint{5.578823in}{5.081849in}}%
\pgfpathlineto{\pgfqpoint{5.578823in}{5.081849in}}%
\pgfpathclose%
\pgfusepath{stroke,fill}%
\end{pgfscope}%
\begin{pgfscope}%
\pgfpathrectangle{\pgfqpoint{1.000000in}{1.148311in}}{\pgfqpoint{6.200000in}{5.623377in}}%
\pgfusepath{clip}%
\pgfsetbuttcap%
\pgfsetroundjoin%
\definecolor{currentfill}{rgb}{0.800000,0.200000,0.200000}%
\pgfsetfillcolor{currentfill}%
\pgfsetlinewidth{1.003750pt}%
\definecolor{currentstroke}{rgb}{0.800000,0.200000,0.200000}%
\pgfsetstrokecolor{currentstroke}%
\pgfsetdash{}{0pt}%
\pgfpathmoveto{\pgfqpoint{5.513587in}{5.066676in}}%
\pgfpathcurveto{\pgfqpoint{5.519411in}{5.066676in}}{\pgfqpoint{5.524997in}{5.068990in}}{\pgfqpoint{5.529115in}{5.073108in}}%
\pgfpathcurveto{\pgfqpoint{5.533233in}{5.077226in}}{\pgfqpoint{5.535547in}{5.082812in}}{\pgfqpoint{5.535547in}{5.088636in}}%
\pgfpathcurveto{\pgfqpoint{5.535547in}{5.094460in}}{\pgfqpoint{5.533233in}{5.100047in}}{\pgfqpoint{5.529115in}{5.104165in}}%
\pgfpathcurveto{\pgfqpoint{5.524997in}{5.108283in}}{\pgfqpoint{5.519411in}{5.110597in}}{\pgfqpoint{5.513587in}{5.110597in}}%
\pgfpathcurveto{\pgfqpoint{5.507763in}{5.110597in}}{\pgfqpoint{5.502177in}{5.108283in}}{\pgfqpoint{5.498059in}{5.104165in}}%
\pgfpathcurveto{\pgfqpoint{5.493940in}{5.100047in}}{\pgfqpoint{5.491627in}{5.094460in}}{\pgfqpoint{5.491627in}{5.088636in}}%
\pgfpathcurveto{\pgfqpoint{5.491627in}{5.082812in}}{\pgfqpoint{5.493940in}{5.077226in}}{\pgfqpoint{5.498059in}{5.073108in}}%
\pgfpathcurveto{\pgfqpoint{5.502177in}{5.068990in}}{\pgfqpoint{5.507763in}{5.066676in}}{\pgfqpoint{5.513587in}{5.066676in}}%
\pgfpathlineto{\pgfqpoint{5.513587in}{5.066676in}}%
\pgfpathclose%
\pgfusepath{stroke,fill}%
\end{pgfscope}%
\begin{pgfscope}%
\pgfpathrectangle{\pgfqpoint{1.000000in}{1.148311in}}{\pgfqpoint{6.200000in}{5.623377in}}%
\pgfusepath{clip}%
\pgfsetbuttcap%
\pgfsetroundjoin%
\definecolor{currentfill}{rgb}{0.800000,0.200000,0.200000}%
\pgfsetfillcolor{currentfill}%
\pgfsetlinewidth{1.003750pt}%
\definecolor{currentstroke}{rgb}{0.800000,0.200000,0.200000}%
\pgfsetstrokecolor{currentstroke}%
\pgfsetdash{}{0pt}%
\pgfpathmoveto{\pgfqpoint{5.427695in}{5.110114in}}%
\pgfpathcurveto{\pgfqpoint{5.433519in}{5.110114in}}{\pgfqpoint{5.439105in}{5.112427in}}{\pgfqpoint{5.443223in}{5.116546in}}%
\pgfpathcurveto{\pgfqpoint{5.447342in}{5.120664in}}{\pgfqpoint{5.449655in}{5.126250in}}{\pgfqpoint{5.449655in}{5.132074in}}%
\pgfpathcurveto{\pgfqpoint{5.449655in}{5.137898in}}{\pgfqpoint{5.447342in}{5.143484in}}{\pgfqpoint{5.443223in}{5.147602in}}%
\pgfpathcurveto{\pgfqpoint{5.439105in}{5.151720in}}{\pgfqpoint{5.433519in}{5.154034in}}{\pgfqpoint{5.427695in}{5.154034in}}%
\pgfpathcurveto{\pgfqpoint{5.421871in}{5.154034in}}{\pgfqpoint{5.416285in}{5.151720in}}{\pgfqpoint{5.412167in}{5.147602in}}%
\pgfpathcurveto{\pgfqpoint{5.408049in}{5.143484in}}{\pgfqpoint{5.405735in}{5.137898in}}{\pgfqpoint{5.405735in}{5.132074in}}%
\pgfpathcurveto{\pgfqpoint{5.405735in}{5.126250in}}{\pgfqpoint{5.408049in}{5.120664in}}{\pgfqpoint{5.412167in}{5.116546in}}%
\pgfpathcurveto{\pgfqpoint{5.416285in}{5.112427in}}{\pgfqpoint{5.421871in}{5.110114in}}{\pgfqpoint{5.427695in}{5.110114in}}%
\pgfpathlineto{\pgfqpoint{5.427695in}{5.110114in}}%
\pgfpathclose%
\pgfusepath{stroke,fill}%
\end{pgfscope}%
\begin{pgfscope}%
\pgfpathrectangle{\pgfqpoint{1.000000in}{1.148311in}}{\pgfqpoint{6.200000in}{5.623377in}}%
\pgfusepath{clip}%
\pgfsetbuttcap%
\pgfsetroundjoin%
\definecolor{currentfill}{rgb}{0.800000,0.200000,0.200000}%
\pgfsetfillcolor{currentfill}%
\pgfsetlinewidth{1.003750pt}%
\definecolor{currentstroke}{rgb}{0.800000,0.200000,0.200000}%
\pgfsetstrokecolor{currentstroke}%
\pgfsetdash{}{0pt}%
\pgfpathmoveto{\pgfqpoint{5.413339in}{4.960068in}}%
\pgfpathcurveto{\pgfqpoint{5.419163in}{4.960068in}}{\pgfqpoint{5.424749in}{4.962381in}}{\pgfqpoint{5.428867in}{4.966500in}}%
\pgfpathcurveto{\pgfqpoint{5.432985in}{4.970618in}}{\pgfqpoint{5.435299in}{4.976204in}}{\pgfqpoint{5.435299in}{4.982028in}}%
\pgfpathcurveto{\pgfqpoint{5.435299in}{4.987852in}}{\pgfqpoint{5.432985in}{4.993438in}}{\pgfqpoint{5.428867in}{4.997556in}}%
\pgfpathcurveto{\pgfqpoint{5.424749in}{5.001674in}}{\pgfqpoint{5.419163in}{5.003988in}}{\pgfqpoint{5.413339in}{5.003988in}}%
\pgfpathcurveto{\pgfqpoint{5.407515in}{5.003988in}}{\pgfqpoint{5.401929in}{5.001674in}}{\pgfqpoint{5.397811in}{4.997556in}}%
\pgfpathcurveto{\pgfqpoint{5.393693in}{4.993438in}}{\pgfqpoint{5.391379in}{4.987852in}}{\pgfqpoint{5.391379in}{4.982028in}}%
\pgfpathcurveto{\pgfqpoint{5.391379in}{4.976204in}}{\pgfqpoint{5.393693in}{4.970618in}}{\pgfqpoint{5.397811in}{4.966500in}}%
\pgfpathcurveto{\pgfqpoint{5.401929in}{4.962381in}}{\pgfqpoint{5.407515in}{4.960068in}}{\pgfqpoint{5.413339in}{4.960068in}}%
\pgfpathlineto{\pgfqpoint{5.413339in}{4.960068in}}%
\pgfpathclose%
\pgfusepath{stroke,fill}%
\end{pgfscope}%
\begin{pgfscope}%
\pgfpathrectangle{\pgfqpoint{1.000000in}{1.148311in}}{\pgfqpoint{6.200000in}{5.623377in}}%
\pgfusepath{clip}%
\pgfsetbuttcap%
\pgfsetroundjoin%
\definecolor{currentfill}{rgb}{0.800000,0.800000,0.200000}%
\pgfsetfillcolor{currentfill}%
\pgfsetlinewidth{1.003750pt}%
\definecolor{currentstroke}{rgb}{0.800000,0.800000,0.200000}%
\pgfsetstrokecolor{currentstroke}%
\pgfsetdash{}{0pt}%
\pgfpathmoveto{\pgfqpoint{5.286197in}{5.075538in}}%
\pgfpathcurveto{\pgfqpoint{5.292021in}{5.075538in}}{\pgfqpoint{5.297607in}{5.077852in}}{\pgfqpoint{5.301725in}{5.081970in}}%
\pgfpathcurveto{\pgfqpoint{5.305843in}{5.086089in}}{\pgfqpoint{5.308157in}{5.091675in}}{\pgfqpoint{5.308157in}{5.097499in}}%
\pgfpathcurveto{\pgfqpoint{5.308157in}{5.103323in}}{\pgfqpoint{5.305843in}{5.108909in}}{\pgfqpoint{5.301725in}{5.113027in}}%
\pgfpathcurveto{\pgfqpoint{5.297607in}{5.117145in}}{\pgfqpoint{5.292021in}{5.119459in}}{\pgfqpoint{5.286197in}{5.119459in}}%
\pgfpathcurveto{\pgfqpoint{5.280373in}{5.119459in}}{\pgfqpoint{5.274787in}{5.117145in}}{\pgfqpoint{5.270669in}{5.113027in}}%
\pgfpathcurveto{\pgfqpoint{5.266550in}{5.108909in}}{\pgfqpoint{5.264237in}{5.103323in}}{\pgfqpoint{5.264237in}{5.097499in}}%
\pgfpathcurveto{\pgfqpoint{5.264237in}{5.091675in}}{\pgfqpoint{5.266550in}{5.086089in}}{\pgfqpoint{5.270669in}{5.081970in}}%
\pgfpathcurveto{\pgfqpoint{5.274787in}{5.077852in}}{\pgfqpoint{5.280373in}{5.075538in}}{\pgfqpoint{5.286197in}{5.075538in}}%
\pgfpathlineto{\pgfqpoint{5.286197in}{5.075538in}}%
\pgfpathclose%
\pgfusepath{stroke,fill}%
\end{pgfscope}%
\begin{pgfscope}%
\pgfpathrectangle{\pgfqpoint{1.000000in}{1.148311in}}{\pgfqpoint{6.200000in}{5.623377in}}%
\pgfusepath{clip}%
\pgfsetbuttcap%
\pgfsetroundjoin%
\definecolor{currentfill}{rgb}{0.800000,0.800000,0.200000}%
\pgfsetfillcolor{currentfill}%
\pgfsetlinewidth{1.003750pt}%
\definecolor{currentstroke}{rgb}{0.800000,0.800000,0.200000}%
\pgfsetstrokecolor{currentstroke}%
\pgfsetdash{}{0pt}%
\pgfpathmoveto{\pgfqpoint{5.244672in}{5.002145in}}%
\pgfpathcurveto{\pgfqpoint{5.250496in}{5.002145in}}{\pgfqpoint{5.256082in}{5.004459in}}{\pgfqpoint{5.260200in}{5.008577in}}%
\pgfpathcurveto{\pgfqpoint{5.264318in}{5.012695in}}{\pgfqpoint{5.266632in}{5.018281in}}{\pgfqpoint{5.266632in}{5.024105in}}%
\pgfpathcurveto{\pgfqpoint{5.266632in}{5.029929in}}{\pgfqpoint{5.264318in}{5.035515in}}{\pgfqpoint{5.260200in}{5.039634in}}%
\pgfpathcurveto{\pgfqpoint{5.256082in}{5.043752in}}{\pgfqpoint{5.250496in}{5.046066in}}{\pgfqpoint{5.244672in}{5.046066in}}%
\pgfpathcurveto{\pgfqpoint{5.238848in}{5.046066in}}{\pgfqpoint{5.233262in}{5.043752in}}{\pgfqpoint{5.229143in}{5.039634in}}%
\pgfpathcurveto{\pgfqpoint{5.225025in}{5.035515in}}{\pgfqpoint{5.222711in}{5.029929in}}{\pgfqpoint{5.222711in}{5.024105in}}%
\pgfpathcurveto{\pgfqpoint{5.222711in}{5.018281in}}{\pgfqpoint{5.225025in}{5.012695in}}{\pgfqpoint{5.229143in}{5.008577in}}%
\pgfpathcurveto{\pgfqpoint{5.233262in}{5.004459in}}{\pgfqpoint{5.238848in}{5.002145in}}{\pgfqpoint{5.244672in}{5.002145in}}%
\pgfpathlineto{\pgfqpoint{5.244672in}{5.002145in}}%
\pgfpathclose%
\pgfusepath{stroke,fill}%
\end{pgfscope}%
\begin{pgfscope}%
\pgfpathrectangle{\pgfqpoint{1.000000in}{1.148311in}}{\pgfqpoint{6.200000in}{5.623377in}}%
\pgfusepath{clip}%
\pgfsetbuttcap%
\pgfsetroundjoin%
\definecolor{currentfill}{rgb}{0.800000,0.800000,0.200000}%
\pgfsetfillcolor{currentfill}%
\pgfsetlinewidth{1.003750pt}%
\definecolor{currentstroke}{rgb}{0.800000,0.800000,0.200000}%
\pgfsetstrokecolor{currentstroke}%
\pgfsetdash{}{0pt}%
\pgfpathmoveto{\pgfqpoint{5.218309in}{4.915762in}}%
\pgfpathcurveto{\pgfqpoint{5.224133in}{4.915762in}}{\pgfqpoint{5.229719in}{4.918076in}}{\pgfqpoint{5.233837in}{4.922194in}}%
\pgfpathcurveto{\pgfqpoint{5.237955in}{4.926312in}}{\pgfqpoint{5.240269in}{4.931899in}}{\pgfqpoint{5.240269in}{4.937723in}}%
\pgfpathcurveto{\pgfqpoint{5.240269in}{4.943546in}}{\pgfqpoint{5.237955in}{4.949133in}}{\pgfqpoint{5.233837in}{4.953251in}}%
\pgfpathcurveto{\pgfqpoint{5.229719in}{4.957369in}}{\pgfqpoint{5.224133in}{4.959683in}}{\pgfqpoint{5.218309in}{4.959683in}}%
\pgfpathcurveto{\pgfqpoint{5.212485in}{4.959683in}}{\pgfqpoint{5.206899in}{4.957369in}}{\pgfqpoint{5.202781in}{4.953251in}}%
\pgfpathcurveto{\pgfqpoint{5.198663in}{4.949133in}}{\pgfqpoint{5.196349in}{4.943546in}}{\pgfqpoint{5.196349in}{4.937723in}}%
\pgfpathcurveto{\pgfqpoint{5.196349in}{4.931899in}}{\pgfqpoint{5.198663in}{4.926312in}}{\pgfqpoint{5.202781in}{4.922194in}}%
\pgfpathcurveto{\pgfqpoint{5.206899in}{4.918076in}}{\pgfqpoint{5.212485in}{4.915762in}}{\pgfqpoint{5.218309in}{4.915762in}}%
\pgfpathlineto{\pgfqpoint{5.218309in}{4.915762in}}%
\pgfpathclose%
\pgfusepath{stroke,fill}%
\end{pgfscope}%
\begin{pgfscope}%
\pgfpathrectangle{\pgfqpoint{1.000000in}{1.148311in}}{\pgfqpoint{6.200000in}{5.623377in}}%
\pgfusepath{clip}%
\pgfsetbuttcap%
\pgfsetroundjoin%
\definecolor{currentfill}{rgb}{0.800000,0.800000,0.200000}%
\pgfsetfillcolor{currentfill}%
\pgfsetlinewidth{1.003750pt}%
\definecolor{currentstroke}{rgb}{0.800000,0.800000,0.200000}%
\pgfsetstrokecolor{currentstroke}%
\pgfsetdash{}{0pt}%
\pgfpathmoveto{\pgfqpoint{5.147314in}{4.900031in}}%
\pgfpathcurveto{\pgfqpoint{5.153138in}{4.900031in}}{\pgfqpoint{5.158725in}{4.902345in}}{\pgfqpoint{5.162843in}{4.906463in}}%
\pgfpathcurveto{\pgfqpoint{5.166961in}{4.910581in}}{\pgfqpoint{5.169275in}{4.916167in}}{\pgfqpoint{5.169275in}{4.921991in}}%
\pgfpathcurveto{\pgfqpoint{5.169275in}{4.927815in}}{\pgfqpoint{5.166961in}{4.933401in}}{\pgfqpoint{5.162843in}{4.937519in}}%
\pgfpathcurveto{\pgfqpoint{5.158725in}{4.941637in}}{\pgfqpoint{5.153138in}{4.943951in}}{\pgfqpoint{5.147314in}{4.943951in}}%
\pgfpathcurveto{\pgfqpoint{5.141490in}{4.943951in}}{\pgfqpoint{5.135904in}{4.941637in}}{\pgfqpoint{5.131786in}{4.937519in}}%
\pgfpathcurveto{\pgfqpoint{5.127668in}{4.933401in}}{\pgfqpoint{5.125354in}{4.927815in}}{\pgfqpoint{5.125354in}{4.921991in}}%
\pgfpathcurveto{\pgfqpoint{5.125354in}{4.916167in}}{\pgfqpoint{5.127668in}{4.910581in}}{\pgfqpoint{5.131786in}{4.906463in}}%
\pgfpathcurveto{\pgfqpoint{5.135904in}{4.902345in}}{\pgfqpoint{5.141490in}{4.900031in}}{\pgfqpoint{5.147314in}{4.900031in}}%
\pgfpathlineto{\pgfqpoint{5.147314in}{4.900031in}}%
\pgfpathclose%
\pgfusepath{stroke,fill}%
\end{pgfscope}%
\begin{pgfscope}%
\pgfpathrectangle{\pgfqpoint{1.000000in}{1.148311in}}{\pgfqpoint{6.200000in}{5.623377in}}%
\pgfusepath{clip}%
\pgfsetbuttcap%
\pgfsetroundjoin%
\definecolor{currentfill}{rgb}{0.800000,0.800000,0.200000}%
\pgfsetfillcolor{currentfill}%
\pgfsetlinewidth{1.003750pt}%
\definecolor{currentstroke}{rgb}{0.800000,0.800000,0.200000}%
\pgfsetstrokecolor{currentstroke}%
\pgfsetdash{}{0pt}%
\pgfpathmoveto{\pgfqpoint{5.122488in}{4.825114in}}%
\pgfpathcurveto{\pgfqpoint{5.128312in}{4.825114in}}{\pgfqpoint{5.133898in}{4.827428in}}{\pgfqpoint{5.138016in}{4.831546in}}%
\pgfpathcurveto{\pgfqpoint{5.142135in}{4.835665in}}{\pgfqpoint{5.144448in}{4.841251in}}{\pgfqpoint{5.144448in}{4.847075in}}%
\pgfpathcurveto{\pgfqpoint{5.144448in}{4.852899in}}{\pgfqpoint{5.142135in}{4.858485in}}{\pgfqpoint{5.138016in}{4.862603in}}%
\pgfpathcurveto{\pgfqpoint{5.133898in}{4.866721in}}{\pgfqpoint{5.128312in}{4.869035in}}{\pgfqpoint{5.122488in}{4.869035in}}%
\pgfpathcurveto{\pgfqpoint{5.116664in}{4.869035in}}{\pgfqpoint{5.111078in}{4.866721in}}{\pgfqpoint{5.106960in}{4.862603in}}%
\pgfpathcurveto{\pgfqpoint{5.102842in}{4.858485in}}{\pgfqpoint{5.100528in}{4.852899in}}{\pgfqpoint{5.100528in}{4.847075in}}%
\pgfpathcurveto{\pgfqpoint{5.100528in}{4.841251in}}{\pgfqpoint{5.102842in}{4.835665in}}{\pgfqpoint{5.106960in}{4.831546in}}%
\pgfpathcurveto{\pgfqpoint{5.111078in}{4.827428in}}{\pgfqpoint{5.116664in}{4.825114in}}{\pgfqpoint{5.122488in}{4.825114in}}%
\pgfpathlineto{\pgfqpoint{5.122488in}{4.825114in}}%
\pgfpathclose%
\pgfusepath{stroke,fill}%
\end{pgfscope}%
\begin{pgfscope}%
\pgfpathrectangle{\pgfqpoint{1.000000in}{1.148311in}}{\pgfqpoint{6.200000in}{5.623377in}}%
\pgfusepath{clip}%
\pgfsetbuttcap%
\pgfsetroundjoin%
\definecolor{currentfill}{rgb}{0.800000,0.200000,0.200000}%
\pgfsetfillcolor{currentfill}%
\pgfsetlinewidth{1.003750pt}%
\definecolor{currentstroke}{rgb}{0.800000,0.200000,0.200000}%
\pgfsetstrokecolor{currentstroke}%
\pgfsetdash{}{0pt}%
\pgfpathmoveto{\pgfqpoint{5.046800in}{4.809172in}}%
\pgfpathcurveto{\pgfqpoint{5.052624in}{4.809172in}}{\pgfqpoint{5.058211in}{4.811485in}}{\pgfqpoint{5.062329in}{4.815604in}}%
\pgfpathcurveto{\pgfqpoint{5.066447in}{4.819722in}}{\pgfqpoint{5.068761in}{4.825308in}}{\pgfqpoint{5.068761in}{4.831132in}}%
\pgfpathcurveto{\pgfqpoint{5.068761in}{4.836956in}}{\pgfqpoint{5.066447in}{4.842542in}}{\pgfqpoint{5.062329in}{4.846660in}}%
\pgfpathcurveto{\pgfqpoint{5.058211in}{4.850778in}}{\pgfqpoint{5.052624in}{4.853092in}}{\pgfqpoint{5.046800in}{4.853092in}}%
\pgfpathcurveto{\pgfqpoint{5.040977in}{4.853092in}}{\pgfqpoint{5.035390in}{4.850778in}}{\pgfqpoint{5.031272in}{4.846660in}}%
\pgfpathcurveto{\pgfqpoint{5.027154in}{4.842542in}}{\pgfqpoint{5.024840in}{4.836956in}}{\pgfqpoint{5.024840in}{4.831132in}}%
\pgfpathcurveto{\pgfqpoint{5.024840in}{4.825308in}}{\pgfqpoint{5.027154in}{4.819722in}}{\pgfqpoint{5.031272in}{4.815604in}}%
\pgfpathcurveto{\pgfqpoint{5.035390in}{4.811485in}}{\pgfqpoint{5.040977in}{4.809172in}}{\pgfqpoint{5.046800in}{4.809172in}}%
\pgfpathlineto{\pgfqpoint{5.046800in}{4.809172in}}%
\pgfpathclose%
\pgfusepath{stroke,fill}%
\end{pgfscope}%
\begin{pgfscope}%
\pgfpathrectangle{\pgfqpoint{1.000000in}{1.148311in}}{\pgfqpoint{6.200000in}{5.623377in}}%
\pgfusepath{clip}%
\pgfsetbuttcap%
\pgfsetroundjoin%
\definecolor{currentfill}{rgb}{0.800000,0.200000,0.200000}%
\pgfsetfillcolor{currentfill}%
\pgfsetlinewidth{1.003750pt}%
\definecolor{currentstroke}{rgb}{0.800000,0.200000,0.200000}%
\pgfsetstrokecolor{currentstroke}%
\pgfsetdash{}{0pt}%
\pgfpathmoveto{\pgfqpoint{4.946099in}{4.808505in}}%
\pgfpathcurveto{\pgfqpoint{4.951923in}{4.808505in}}{\pgfqpoint{4.957509in}{4.810819in}}{\pgfqpoint{4.961627in}{4.814937in}}%
\pgfpathcurveto{\pgfqpoint{4.965746in}{4.819055in}}{\pgfqpoint{4.968059in}{4.824641in}}{\pgfqpoint{4.968059in}{4.830465in}}%
\pgfpathcurveto{\pgfqpoint{4.968059in}{4.836289in}}{\pgfqpoint{4.965746in}{4.841875in}}{\pgfqpoint{4.961627in}{4.845994in}}%
\pgfpathcurveto{\pgfqpoint{4.957509in}{4.850112in}}{\pgfqpoint{4.951923in}{4.852426in}}{\pgfqpoint{4.946099in}{4.852426in}}%
\pgfpathcurveto{\pgfqpoint{4.940275in}{4.852426in}}{\pgfqpoint{4.934689in}{4.850112in}}{\pgfqpoint{4.930571in}{4.845994in}}%
\pgfpathcurveto{\pgfqpoint{4.926453in}{4.841875in}}{\pgfqpoint{4.924139in}{4.836289in}}{\pgfqpoint{4.924139in}{4.830465in}}%
\pgfpathcurveto{\pgfqpoint{4.924139in}{4.824641in}}{\pgfqpoint{4.926453in}{4.819055in}}{\pgfqpoint{4.930571in}{4.814937in}}%
\pgfpathcurveto{\pgfqpoint{4.934689in}{4.810819in}}{\pgfqpoint{4.940275in}{4.808505in}}{\pgfqpoint{4.946099in}{4.808505in}}%
\pgfpathlineto{\pgfqpoint{4.946099in}{4.808505in}}%
\pgfpathclose%
\pgfusepath{stroke,fill}%
\end{pgfscope}%
\begin{pgfscope}%
\pgfpathrectangle{\pgfqpoint{1.000000in}{1.148311in}}{\pgfqpoint{6.200000in}{5.623377in}}%
\pgfusepath{clip}%
\pgfsetbuttcap%
\pgfsetroundjoin%
\definecolor{currentfill}{rgb}{0.800000,0.800000,0.200000}%
\pgfsetfillcolor{currentfill}%
\pgfsetlinewidth{1.003750pt}%
\definecolor{currentstroke}{rgb}{0.800000,0.800000,0.200000}%
\pgfsetstrokecolor{currentstroke}%
\pgfsetdash{}{0pt}%
\pgfpathmoveto{\pgfqpoint{4.986638in}{4.684726in}}%
\pgfpathcurveto{\pgfqpoint{4.992462in}{4.684726in}}{\pgfqpoint{4.998048in}{4.687040in}}{\pgfqpoint{5.002167in}{4.691158in}}%
\pgfpathcurveto{\pgfqpoint{5.006285in}{4.695277in}}{\pgfqpoint{5.008599in}{4.700863in}}{\pgfqpoint{5.008599in}{4.706687in}}%
\pgfpathcurveto{\pgfqpoint{5.008599in}{4.712511in}}{\pgfqpoint{5.006285in}{4.718097in}}{\pgfqpoint{5.002167in}{4.722215in}}%
\pgfpathcurveto{\pgfqpoint{4.998048in}{4.726333in}}{\pgfqpoint{4.992462in}{4.728647in}}{\pgfqpoint{4.986638in}{4.728647in}}%
\pgfpathcurveto{\pgfqpoint{4.980814in}{4.728647in}}{\pgfqpoint{4.975228in}{4.726333in}}{\pgfqpoint{4.971110in}{4.722215in}}%
\pgfpathcurveto{\pgfqpoint{4.966992in}{4.718097in}}{\pgfqpoint{4.964678in}{4.712511in}}{\pgfqpoint{4.964678in}{4.706687in}}%
\pgfpathcurveto{\pgfqpoint{4.964678in}{4.700863in}}{\pgfqpoint{4.966992in}{4.695277in}}{\pgfqpoint{4.971110in}{4.691158in}}%
\pgfpathcurveto{\pgfqpoint{4.975228in}{4.687040in}}{\pgfqpoint{4.980814in}{4.684726in}}{\pgfqpoint{4.986638in}{4.684726in}}%
\pgfpathlineto{\pgfqpoint{4.986638in}{4.684726in}}%
\pgfpathclose%
\pgfusepath{stroke,fill}%
\end{pgfscope}%
\begin{pgfscope}%
\pgfpathrectangle{\pgfqpoint{1.000000in}{1.148311in}}{\pgfqpoint{6.200000in}{5.623377in}}%
\pgfusepath{clip}%
\pgfsetbuttcap%
\pgfsetroundjoin%
\definecolor{currentfill}{rgb}{0.800000,0.800000,0.200000}%
\pgfsetfillcolor{currentfill}%
\pgfsetlinewidth{1.003750pt}%
\definecolor{currentstroke}{rgb}{0.800000,0.800000,0.200000}%
\pgfsetstrokecolor{currentstroke}%
\pgfsetdash{}{0pt}%
\pgfpathmoveto{\pgfqpoint{4.934444in}{4.641481in}}%
\pgfpathcurveto{\pgfqpoint{4.940268in}{4.641481in}}{\pgfqpoint{4.945855in}{4.643795in}}{\pgfqpoint{4.949973in}{4.647913in}}%
\pgfpathcurveto{\pgfqpoint{4.954091in}{4.652031in}}{\pgfqpoint{4.956405in}{4.657617in}}{\pgfqpoint{4.956405in}{4.663441in}}%
\pgfpathcurveto{\pgfqpoint{4.956405in}{4.669265in}}{\pgfqpoint{4.954091in}{4.674851in}}{\pgfqpoint{4.949973in}{4.678969in}}%
\pgfpathcurveto{\pgfqpoint{4.945855in}{4.683087in}}{\pgfqpoint{4.940268in}{4.685401in}}{\pgfqpoint{4.934444in}{4.685401in}}%
\pgfpathcurveto{\pgfqpoint{4.928621in}{4.685401in}}{\pgfqpoint{4.923034in}{4.683087in}}{\pgfqpoint{4.918916in}{4.678969in}}%
\pgfpathcurveto{\pgfqpoint{4.914798in}{4.674851in}}{\pgfqpoint{4.912484in}{4.669265in}}{\pgfqpoint{4.912484in}{4.663441in}}%
\pgfpathcurveto{\pgfqpoint{4.912484in}{4.657617in}}{\pgfqpoint{4.914798in}{4.652031in}}{\pgfqpoint{4.918916in}{4.647913in}}%
\pgfpathcurveto{\pgfqpoint{4.923034in}{4.643795in}}{\pgfqpoint{4.928621in}{4.641481in}}{\pgfqpoint{4.934444in}{4.641481in}}%
\pgfpathlineto{\pgfqpoint{4.934444in}{4.641481in}}%
\pgfpathclose%
\pgfusepath{stroke,fill}%
\end{pgfscope}%
\begin{pgfscope}%
\pgfpathrectangle{\pgfqpoint{1.000000in}{1.148311in}}{\pgfqpoint{6.200000in}{5.623377in}}%
\pgfusepath{clip}%
\pgfsetbuttcap%
\pgfsetroundjoin%
\definecolor{currentfill}{rgb}{0.800000,0.800000,0.200000}%
\pgfsetfillcolor{currentfill}%
\pgfsetlinewidth{1.003750pt}%
\definecolor{currentstroke}{rgb}{0.800000,0.800000,0.200000}%
\pgfsetstrokecolor{currentstroke}%
\pgfsetdash{}{0pt}%
\pgfpathmoveto{\pgfqpoint{4.916426in}{4.575044in}}%
\pgfpathcurveto{\pgfqpoint{4.922250in}{4.575044in}}{\pgfqpoint{4.927836in}{4.577358in}}{\pgfqpoint{4.931954in}{4.581476in}}%
\pgfpathcurveto{\pgfqpoint{4.936072in}{4.585594in}}{\pgfqpoint{4.938386in}{4.591180in}}{\pgfqpoint{4.938386in}{4.597004in}}%
\pgfpathcurveto{\pgfqpoint{4.938386in}{4.602828in}}{\pgfqpoint{4.936072in}{4.608414in}}{\pgfqpoint{4.931954in}{4.612533in}}%
\pgfpathcurveto{\pgfqpoint{4.927836in}{4.616651in}}{\pgfqpoint{4.922250in}{4.618965in}}{\pgfqpoint{4.916426in}{4.618965in}}%
\pgfpathcurveto{\pgfqpoint{4.910602in}{4.618965in}}{\pgfqpoint{4.905016in}{4.616651in}}{\pgfqpoint{4.900897in}{4.612533in}}%
\pgfpathcurveto{\pgfqpoint{4.896779in}{4.608414in}}{\pgfqpoint{4.894465in}{4.602828in}}{\pgfqpoint{4.894465in}{4.597004in}}%
\pgfpathcurveto{\pgfqpoint{4.894465in}{4.591180in}}{\pgfqpoint{4.896779in}{4.585594in}}{\pgfqpoint{4.900897in}{4.581476in}}%
\pgfpathcurveto{\pgfqpoint{4.905016in}{4.577358in}}{\pgfqpoint{4.910602in}{4.575044in}}{\pgfqpoint{4.916426in}{4.575044in}}%
\pgfpathlineto{\pgfqpoint{4.916426in}{4.575044in}}%
\pgfpathclose%
\pgfusepath{stroke,fill}%
\end{pgfscope}%
\begin{pgfscope}%
\pgfpathrectangle{\pgfqpoint{1.000000in}{1.148311in}}{\pgfqpoint{6.200000in}{5.623377in}}%
\pgfusepath{clip}%
\pgfsetbuttcap%
\pgfsetroundjoin%
\definecolor{currentfill}{rgb}{0.800000,0.800000,0.200000}%
\pgfsetfillcolor{currentfill}%
\pgfsetlinewidth{1.003750pt}%
\definecolor{currentstroke}{rgb}{0.800000,0.800000,0.200000}%
\pgfsetstrokecolor{currentstroke}%
\pgfsetdash{}{0pt}%
\pgfpathmoveto{\pgfqpoint{4.879184in}{4.521181in}}%
\pgfpathcurveto{\pgfqpoint{4.885008in}{4.521181in}}{\pgfqpoint{4.890594in}{4.523495in}}{\pgfqpoint{4.894712in}{4.527613in}}%
\pgfpathcurveto{\pgfqpoint{4.898831in}{4.531731in}}{\pgfqpoint{4.901144in}{4.537318in}}{\pgfqpoint{4.901144in}{4.543142in}}%
\pgfpathcurveto{\pgfqpoint{4.901144in}{4.548966in}}{\pgfqpoint{4.898831in}{4.554552in}}{\pgfqpoint{4.894712in}{4.558670in}}%
\pgfpathcurveto{\pgfqpoint{4.890594in}{4.562788in}}{\pgfqpoint{4.885008in}{4.565102in}}{\pgfqpoint{4.879184in}{4.565102in}}%
\pgfpathcurveto{\pgfqpoint{4.873360in}{4.565102in}}{\pgfqpoint{4.867774in}{4.562788in}}{\pgfqpoint{4.863656in}{4.558670in}}%
\pgfpathcurveto{\pgfqpoint{4.859538in}{4.554552in}}{\pgfqpoint{4.857224in}{4.548966in}}{\pgfqpoint{4.857224in}{4.543142in}}%
\pgfpathcurveto{\pgfqpoint{4.857224in}{4.537318in}}{\pgfqpoint{4.859538in}{4.531731in}}{\pgfqpoint{4.863656in}{4.527613in}}%
\pgfpathcurveto{\pgfqpoint{4.867774in}{4.523495in}}{\pgfqpoint{4.873360in}{4.521181in}}{\pgfqpoint{4.879184in}{4.521181in}}%
\pgfpathlineto{\pgfqpoint{4.879184in}{4.521181in}}%
\pgfpathclose%
\pgfusepath{stroke,fill}%
\end{pgfscope}%
\begin{pgfscope}%
\pgfpathrectangle{\pgfqpoint{1.000000in}{1.148311in}}{\pgfqpoint{6.200000in}{5.623377in}}%
\pgfusepath{clip}%
\pgfsetbuttcap%
\pgfsetroundjoin%
\definecolor{currentfill}{rgb}{0.200000,0.200000,0.800000}%
\pgfsetfillcolor{currentfill}%
\pgfsetlinewidth{1.003750pt}%
\definecolor{currentstroke}{rgb}{0.200000,0.200000,0.800000}%
\pgfsetstrokecolor{currentstroke}%
\pgfsetdash{}{0pt}%
\pgfpathmoveto{\pgfqpoint{4.844072in}{4.465221in}}%
\pgfpathcurveto{\pgfqpoint{4.849896in}{4.465221in}}{\pgfqpoint{4.855482in}{4.467535in}}{\pgfqpoint{4.859601in}{4.471653in}}%
\pgfpathcurveto{\pgfqpoint{4.863719in}{4.475771in}}{\pgfqpoint{4.866033in}{4.481357in}}{\pgfqpoint{4.866033in}{4.487181in}}%
\pgfpathcurveto{\pgfqpoint{4.866033in}{4.493005in}}{\pgfqpoint{4.863719in}{4.498591in}}{\pgfqpoint{4.859601in}{4.502709in}}%
\pgfpathcurveto{\pgfqpoint{4.855482in}{4.506827in}}{\pgfqpoint{4.849896in}{4.509141in}}{\pgfqpoint{4.844072in}{4.509141in}}%
\pgfpathcurveto{\pgfqpoint{4.838248in}{4.509141in}}{\pgfqpoint{4.832662in}{4.506827in}}{\pgfqpoint{4.828544in}{4.502709in}}%
\pgfpathcurveto{\pgfqpoint{4.824426in}{4.498591in}}{\pgfqpoint{4.822112in}{4.493005in}}{\pgfqpoint{4.822112in}{4.487181in}}%
\pgfpathcurveto{\pgfqpoint{4.822112in}{4.481357in}}{\pgfqpoint{4.824426in}{4.475771in}}{\pgfqpoint{4.828544in}{4.471653in}}%
\pgfpathcurveto{\pgfqpoint{4.832662in}{4.467535in}}{\pgfqpoint{4.838248in}{4.465221in}}{\pgfqpoint{4.844072in}{4.465221in}}%
\pgfpathlineto{\pgfqpoint{4.844072in}{4.465221in}}%
\pgfpathclose%
\pgfusepath{stroke,fill}%
\end{pgfscope}%
\begin{pgfscope}%
\pgfpathrectangle{\pgfqpoint{1.000000in}{1.148311in}}{\pgfqpoint{6.200000in}{5.623377in}}%
\pgfusepath{clip}%
\pgfsetbuttcap%
\pgfsetroundjoin%
\definecolor{currentfill}{rgb}{0.200000,0.200000,0.800000}%
\pgfsetfillcolor{currentfill}%
\pgfsetlinewidth{1.003750pt}%
\definecolor{currentstroke}{rgb}{0.200000,0.200000,0.800000}%
\pgfsetstrokecolor{currentstroke}%
\pgfsetdash{}{0pt}%
\pgfpathmoveto{\pgfqpoint{4.845339in}{4.394706in}}%
\pgfpathcurveto{\pgfqpoint{4.851163in}{4.394706in}}{\pgfqpoint{4.856749in}{4.397020in}}{\pgfqpoint{4.860867in}{4.401138in}}%
\pgfpathcurveto{\pgfqpoint{4.864986in}{4.405256in}}{\pgfqpoint{4.867299in}{4.410842in}}{\pgfqpoint{4.867299in}{4.416666in}}%
\pgfpathcurveto{\pgfqpoint{4.867299in}{4.422490in}}{\pgfqpoint{4.864986in}{4.428076in}}{\pgfqpoint{4.860867in}{4.432194in}}%
\pgfpathcurveto{\pgfqpoint{4.856749in}{4.436312in}}{\pgfqpoint{4.851163in}{4.438626in}}{\pgfqpoint{4.845339in}{4.438626in}}%
\pgfpathcurveto{\pgfqpoint{4.839515in}{4.438626in}}{\pgfqpoint{4.833929in}{4.436312in}}{\pgfqpoint{4.829811in}{4.432194in}}%
\pgfpathcurveto{\pgfqpoint{4.825693in}{4.428076in}}{\pgfqpoint{4.823379in}{4.422490in}}{\pgfqpoint{4.823379in}{4.416666in}}%
\pgfpathcurveto{\pgfqpoint{4.823379in}{4.410842in}}{\pgfqpoint{4.825693in}{4.405256in}}{\pgfqpoint{4.829811in}{4.401138in}}%
\pgfpathcurveto{\pgfqpoint{4.833929in}{4.397020in}}{\pgfqpoint{4.839515in}{4.394706in}}{\pgfqpoint{4.845339in}{4.394706in}}%
\pgfpathlineto{\pgfqpoint{4.845339in}{4.394706in}}%
\pgfpathclose%
\pgfusepath{stroke,fill}%
\end{pgfscope}%
\begin{pgfscope}%
\pgfpathrectangle{\pgfqpoint{1.000000in}{1.148311in}}{\pgfqpoint{6.200000in}{5.623377in}}%
\pgfusepath{clip}%
\pgfsetbuttcap%
\pgfsetroundjoin%
\definecolor{currentfill}{rgb}{0.200000,0.200000,0.800000}%
\pgfsetfillcolor{currentfill}%
\pgfsetlinewidth{1.003750pt}%
\definecolor{currentstroke}{rgb}{0.200000,0.200000,0.800000}%
\pgfsetstrokecolor{currentstroke}%
\pgfsetdash{}{0pt}%
\pgfpathmoveto{\pgfqpoint{4.793761in}{4.343054in}}%
\pgfpathcurveto{\pgfqpoint{4.799584in}{4.343054in}}{\pgfqpoint{4.805171in}{4.345367in}}{\pgfqpoint{4.809289in}{4.349486in}}%
\pgfpathcurveto{\pgfqpoint{4.813407in}{4.353604in}}{\pgfqpoint{4.815721in}{4.359190in}}{\pgfqpoint{4.815721in}{4.365014in}}%
\pgfpathcurveto{\pgfqpoint{4.815721in}{4.370838in}}{\pgfqpoint{4.813407in}{4.376424in}}{\pgfqpoint{4.809289in}{4.380542in}}%
\pgfpathcurveto{\pgfqpoint{4.805171in}{4.384660in}}{\pgfqpoint{4.799584in}{4.386974in}}{\pgfqpoint{4.793761in}{4.386974in}}%
\pgfpathcurveto{\pgfqpoint{4.787937in}{4.386974in}}{\pgfqpoint{4.782350in}{4.384660in}}{\pgfqpoint{4.778232in}{4.380542in}}%
\pgfpathcurveto{\pgfqpoint{4.774114in}{4.376424in}}{\pgfqpoint{4.771800in}{4.370838in}}{\pgfqpoint{4.771800in}{4.365014in}}%
\pgfpathcurveto{\pgfqpoint{4.771800in}{4.359190in}}{\pgfqpoint{4.774114in}{4.353604in}}{\pgfqpoint{4.778232in}{4.349486in}}%
\pgfpathcurveto{\pgfqpoint{4.782350in}{4.345367in}}{\pgfqpoint{4.787937in}{4.343054in}}{\pgfqpoint{4.793761in}{4.343054in}}%
\pgfpathlineto{\pgfqpoint{4.793761in}{4.343054in}}%
\pgfpathclose%
\pgfusepath{stroke,fill}%
\end{pgfscope}%
\begin{pgfscope}%
\pgfpathrectangle{\pgfqpoint{1.000000in}{1.148311in}}{\pgfqpoint{6.200000in}{5.623377in}}%
\pgfusepath{clip}%
\pgfsetbuttcap%
\pgfsetroundjoin%
\definecolor{currentfill}{rgb}{0.800000,0.200000,0.200000}%
\pgfsetfillcolor{currentfill}%
\pgfsetlinewidth{1.003750pt}%
\definecolor{currentstroke}{rgb}{0.800000,0.200000,0.200000}%
\pgfsetstrokecolor{currentstroke}%
\pgfsetdash{}{0pt}%
\pgfpathmoveto{\pgfqpoint{4.764860in}{4.281736in}}%
\pgfpathcurveto{\pgfqpoint{4.770684in}{4.281736in}}{\pgfqpoint{4.776270in}{4.284050in}}{\pgfqpoint{4.780388in}{4.288168in}}%
\pgfpathcurveto{\pgfqpoint{4.784507in}{4.292286in}}{\pgfqpoint{4.786820in}{4.297872in}}{\pgfqpoint{4.786820in}{4.303696in}}%
\pgfpathcurveto{\pgfqpoint{4.786820in}{4.309520in}}{\pgfqpoint{4.784507in}{4.315106in}}{\pgfqpoint{4.780388in}{4.319224in}}%
\pgfpathcurveto{\pgfqpoint{4.776270in}{4.323343in}}{\pgfqpoint{4.770684in}{4.325656in}}{\pgfqpoint{4.764860in}{4.325656in}}%
\pgfpathcurveto{\pgfqpoint{4.759036in}{4.325656in}}{\pgfqpoint{4.753450in}{4.323343in}}{\pgfqpoint{4.749332in}{4.319224in}}%
\pgfpathcurveto{\pgfqpoint{4.745214in}{4.315106in}}{\pgfqpoint{4.742900in}{4.309520in}}{\pgfqpoint{4.742900in}{4.303696in}}%
\pgfpathcurveto{\pgfqpoint{4.742900in}{4.297872in}}{\pgfqpoint{4.745214in}{4.292286in}}{\pgfqpoint{4.749332in}{4.288168in}}%
\pgfpathcurveto{\pgfqpoint{4.753450in}{4.284050in}}{\pgfqpoint{4.759036in}{4.281736in}}{\pgfqpoint{4.764860in}{4.281736in}}%
\pgfpathlineto{\pgfqpoint{4.764860in}{4.281736in}}%
\pgfpathclose%
\pgfusepath{stroke,fill}%
\end{pgfscope}%
\begin{pgfscope}%
\pgfpathrectangle{\pgfqpoint{1.000000in}{1.148311in}}{\pgfqpoint{6.200000in}{5.623377in}}%
\pgfusepath{clip}%
\pgfsetbuttcap%
\pgfsetroundjoin%
\definecolor{currentfill}{rgb}{0.800000,0.200000,0.200000}%
\pgfsetfillcolor{currentfill}%
\pgfsetlinewidth{1.003750pt}%
\definecolor{currentstroke}{rgb}{0.800000,0.200000,0.200000}%
\pgfsetstrokecolor{currentstroke}%
\pgfsetdash{}{0pt}%
\pgfpathmoveto{\pgfqpoint{4.770055in}{4.213077in}}%
\pgfpathcurveto{\pgfqpoint{4.775879in}{4.213077in}}{\pgfqpoint{4.781465in}{4.215391in}}{\pgfqpoint{4.785583in}{4.219509in}}%
\pgfpathcurveto{\pgfqpoint{4.789701in}{4.223627in}}{\pgfqpoint{4.792015in}{4.229213in}}{\pgfqpoint{4.792015in}{4.235037in}}%
\pgfpathcurveto{\pgfqpoint{4.792015in}{4.240861in}}{\pgfqpoint{4.789701in}{4.246447in}}{\pgfqpoint{4.785583in}{4.250565in}}%
\pgfpathcurveto{\pgfqpoint{4.781465in}{4.254683in}}{\pgfqpoint{4.775879in}{4.256997in}}{\pgfqpoint{4.770055in}{4.256997in}}%
\pgfpathcurveto{\pgfqpoint{4.764231in}{4.256997in}}{\pgfqpoint{4.758645in}{4.254683in}}{\pgfqpoint{4.754527in}{4.250565in}}%
\pgfpathcurveto{\pgfqpoint{4.750408in}{4.246447in}}{\pgfqpoint{4.748094in}{4.240861in}}{\pgfqpoint{4.748094in}{4.235037in}}%
\pgfpathcurveto{\pgfqpoint{4.748094in}{4.229213in}}{\pgfqpoint{4.750408in}{4.223627in}}{\pgfqpoint{4.754527in}{4.219509in}}%
\pgfpathcurveto{\pgfqpoint{4.758645in}{4.215391in}}{\pgfqpoint{4.764231in}{4.213077in}}{\pgfqpoint{4.770055in}{4.213077in}}%
\pgfpathlineto{\pgfqpoint{4.770055in}{4.213077in}}%
\pgfpathclose%
\pgfusepath{stroke,fill}%
\end{pgfscope}%
\begin{pgfscope}%
\pgfpathrectangle{\pgfqpoint{1.000000in}{1.148311in}}{\pgfqpoint{6.200000in}{5.623377in}}%
\pgfusepath{clip}%
\pgfsetbuttcap%
\pgfsetroundjoin%
\definecolor{currentfill}{rgb}{0.800000,0.200000,0.200000}%
\pgfsetfillcolor{currentfill}%
\pgfsetlinewidth{1.003750pt}%
\definecolor{currentstroke}{rgb}{0.800000,0.200000,0.200000}%
\pgfsetstrokecolor{currentstroke}%
\pgfsetdash{}{0pt}%
\pgfpathmoveto{\pgfqpoint{4.836421in}{4.140612in}}%
\pgfpathcurveto{\pgfqpoint{4.842245in}{4.140612in}}{\pgfqpoint{4.847831in}{4.142926in}}{\pgfqpoint{4.851949in}{4.147044in}}%
\pgfpathcurveto{\pgfqpoint{4.856067in}{4.151162in}}{\pgfqpoint{4.858381in}{4.156749in}}{\pgfqpoint{4.858381in}{4.162572in}}%
\pgfpathcurveto{\pgfqpoint{4.858381in}{4.168396in}}{\pgfqpoint{4.856067in}{4.173983in}}{\pgfqpoint{4.851949in}{4.178101in}}%
\pgfpathcurveto{\pgfqpoint{4.847831in}{4.182219in}}{\pgfqpoint{4.842245in}{4.184533in}}{\pgfqpoint{4.836421in}{4.184533in}}%
\pgfpathcurveto{\pgfqpoint{4.830597in}{4.184533in}}{\pgfqpoint{4.825011in}{4.182219in}}{\pgfqpoint{4.820893in}{4.178101in}}%
\pgfpathcurveto{\pgfqpoint{4.816775in}{4.173983in}}{\pgfqpoint{4.814461in}{4.168396in}}{\pgfqpoint{4.814461in}{4.162572in}}%
\pgfpathcurveto{\pgfqpoint{4.814461in}{4.156749in}}{\pgfqpoint{4.816775in}{4.151162in}}{\pgfqpoint{4.820893in}{4.147044in}}%
\pgfpathcurveto{\pgfqpoint{4.825011in}{4.142926in}}{\pgfqpoint{4.830597in}{4.140612in}}{\pgfqpoint{4.836421in}{4.140612in}}%
\pgfpathlineto{\pgfqpoint{4.836421in}{4.140612in}}%
\pgfpathclose%
\pgfusepath{stroke,fill}%
\end{pgfscope}%
\begin{pgfscope}%
\pgfpathrectangle{\pgfqpoint{1.000000in}{1.148311in}}{\pgfqpoint{6.200000in}{5.623377in}}%
\pgfusepath{clip}%
\pgfsetbuttcap%
\pgfsetroundjoin%
\definecolor{currentfill}{rgb}{0.800000,0.200000,0.200000}%
\pgfsetfillcolor{currentfill}%
\pgfsetlinewidth{1.003750pt}%
\definecolor{currentstroke}{rgb}{0.800000,0.200000,0.200000}%
\pgfsetstrokecolor{currentstroke}%
\pgfsetdash{}{0pt}%
\pgfpathmoveto{\pgfqpoint{4.747696in}{4.082330in}}%
\pgfpathcurveto{\pgfqpoint{4.753520in}{4.082330in}}{\pgfqpoint{4.759106in}{4.084643in}}{\pgfqpoint{4.763224in}{4.088762in}}%
\pgfpathcurveto{\pgfqpoint{4.767342in}{4.092880in}}{\pgfqpoint{4.769656in}{4.098466in}}{\pgfqpoint{4.769656in}{4.104290in}}%
\pgfpathcurveto{\pgfqpoint{4.769656in}{4.110114in}}{\pgfqpoint{4.767342in}{4.115700in}}{\pgfqpoint{4.763224in}{4.119818in}}%
\pgfpathcurveto{\pgfqpoint{4.759106in}{4.123936in}}{\pgfqpoint{4.753520in}{4.126250in}}{\pgfqpoint{4.747696in}{4.126250in}}%
\pgfpathcurveto{\pgfqpoint{4.741872in}{4.126250in}}{\pgfqpoint{4.736286in}{4.123936in}}{\pgfqpoint{4.732168in}{4.119818in}}%
\pgfpathcurveto{\pgfqpoint{4.728050in}{4.115700in}}{\pgfqpoint{4.725736in}{4.110114in}}{\pgfqpoint{4.725736in}{4.104290in}}%
\pgfpathcurveto{\pgfqpoint{4.725736in}{4.098466in}}{\pgfqpoint{4.728050in}{4.092880in}}{\pgfqpoint{4.732168in}{4.088762in}}%
\pgfpathcurveto{\pgfqpoint{4.736286in}{4.084643in}}{\pgfqpoint{4.741872in}{4.082330in}}{\pgfqpoint{4.747696in}{4.082330in}}%
\pgfpathlineto{\pgfqpoint{4.747696in}{4.082330in}}%
\pgfpathclose%
\pgfusepath{stroke,fill}%
\end{pgfscope}%
\begin{pgfscope}%
\pgfpathrectangle{\pgfqpoint{1.000000in}{1.148311in}}{\pgfqpoint{6.200000in}{5.623377in}}%
\pgfusepath{clip}%
\pgfsetbuttcap%
\pgfsetroundjoin%
\definecolor{currentfill}{rgb}{0.800000,0.200000,0.200000}%
\pgfsetfillcolor{currentfill}%
\pgfsetlinewidth{1.003750pt}%
\definecolor{currentstroke}{rgb}{0.800000,0.200000,0.200000}%
\pgfsetstrokecolor{currentstroke}%
\pgfsetdash{}{0pt}%
\pgfpathmoveto{\pgfqpoint{4.743461in}{4.015709in}}%
\pgfpathcurveto{\pgfqpoint{4.749285in}{4.015709in}}{\pgfqpoint{4.754871in}{4.018023in}}{\pgfqpoint{4.758989in}{4.022141in}}%
\pgfpathcurveto{\pgfqpoint{4.763108in}{4.026259in}}{\pgfqpoint{4.765421in}{4.031846in}}{\pgfqpoint{4.765421in}{4.037669in}}%
\pgfpathcurveto{\pgfqpoint{4.765421in}{4.043493in}}{\pgfqpoint{4.763108in}{4.049080in}}{\pgfqpoint{4.758989in}{4.053198in}}%
\pgfpathcurveto{\pgfqpoint{4.754871in}{4.057316in}}{\pgfqpoint{4.749285in}{4.059630in}}{\pgfqpoint{4.743461in}{4.059630in}}%
\pgfpathcurveto{\pgfqpoint{4.737637in}{4.059630in}}{\pgfqpoint{4.732051in}{4.057316in}}{\pgfqpoint{4.727933in}{4.053198in}}%
\pgfpathcurveto{\pgfqpoint{4.723815in}{4.049080in}}{\pgfqpoint{4.721501in}{4.043493in}}{\pgfqpoint{4.721501in}{4.037669in}}%
\pgfpathcurveto{\pgfqpoint{4.721501in}{4.031846in}}{\pgfqpoint{4.723815in}{4.026259in}}{\pgfqpoint{4.727933in}{4.022141in}}%
\pgfpathcurveto{\pgfqpoint{4.732051in}{4.018023in}}{\pgfqpoint{4.737637in}{4.015709in}}{\pgfqpoint{4.743461in}{4.015709in}}%
\pgfpathlineto{\pgfqpoint{4.743461in}{4.015709in}}%
\pgfpathclose%
\pgfusepath{stroke,fill}%
\end{pgfscope}%
\begin{pgfscope}%
\pgfpathrectangle{\pgfqpoint{1.000000in}{1.148311in}}{\pgfqpoint{6.200000in}{5.623377in}}%
\pgfusepath{clip}%
\pgfsetbuttcap%
\pgfsetroundjoin%
\definecolor{currentfill}{rgb}{0.800000,0.200000,0.200000}%
\pgfsetfillcolor{currentfill}%
\pgfsetlinewidth{1.003750pt}%
\definecolor{currentstroke}{rgb}{0.800000,0.200000,0.200000}%
\pgfsetstrokecolor{currentstroke}%
\pgfsetdash{}{0pt}%
\pgfpathmoveto{\pgfqpoint{4.671066in}{3.941772in}}%
\pgfpathcurveto{\pgfqpoint{4.676890in}{3.941772in}}{\pgfqpoint{4.682476in}{3.944085in}}{\pgfqpoint{4.686594in}{3.948204in}}%
\pgfpathcurveto{\pgfqpoint{4.690712in}{3.952322in}}{\pgfqpoint{4.693026in}{3.957908in}}{\pgfqpoint{4.693026in}{3.963732in}}%
\pgfpathcurveto{\pgfqpoint{4.693026in}{3.969556in}}{\pgfqpoint{4.690712in}{3.975142in}}{\pgfqpoint{4.686594in}{3.979260in}}%
\pgfpathcurveto{\pgfqpoint{4.682476in}{3.983378in}}{\pgfqpoint{4.676890in}{3.985692in}}{\pgfqpoint{4.671066in}{3.985692in}}%
\pgfpathcurveto{\pgfqpoint{4.665242in}{3.985692in}}{\pgfqpoint{4.659656in}{3.983378in}}{\pgfqpoint{4.655538in}{3.979260in}}%
\pgfpathcurveto{\pgfqpoint{4.651419in}{3.975142in}}{\pgfqpoint{4.649106in}{3.969556in}}{\pgfqpoint{4.649106in}{3.963732in}}%
\pgfpathcurveto{\pgfqpoint{4.649106in}{3.957908in}}{\pgfqpoint{4.651419in}{3.952322in}}{\pgfqpoint{4.655538in}{3.948204in}}%
\pgfpathcurveto{\pgfqpoint{4.659656in}{3.944085in}}{\pgfqpoint{4.665242in}{3.941772in}}{\pgfqpoint{4.671066in}{3.941772in}}%
\pgfpathlineto{\pgfqpoint{4.671066in}{3.941772in}}%
\pgfpathclose%
\pgfusepath{stroke,fill}%
\end{pgfscope}%
\begin{pgfscope}%
\pgfpathrectangle{\pgfqpoint{1.000000in}{1.148311in}}{\pgfqpoint{6.200000in}{5.623377in}}%
\pgfusepath{clip}%
\pgfsetbuttcap%
\pgfsetroundjoin%
\definecolor{currentfill}{rgb}{0.800000,0.200000,0.200000}%
\pgfsetfillcolor{currentfill}%
\pgfsetlinewidth{1.003750pt}%
\definecolor{currentstroke}{rgb}{0.800000,0.200000,0.200000}%
\pgfsetstrokecolor{currentstroke}%
\pgfsetdash{}{0pt}%
\pgfpathmoveto{\pgfqpoint{4.753250in}{3.882408in}}%
\pgfpathcurveto{\pgfqpoint{4.759074in}{3.882408in}}{\pgfqpoint{4.764660in}{3.884721in}}{\pgfqpoint{4.768778in}{3.888840in}}%
\pgfpathcurveto{\pgfqpoint{4.772896in}{3.892958in}}{\pgfqpoint{4.775210in}{3.898544in}}{\pgfqpoint{4.775210in}{3.904368in}}%
\pgfpathcurveto{\pgfqpoint{4.775210in}{3.910192in}}{\pgfqpoint{4.772896in}{3.915778in}}{\pgfqpoint{4.768778in}{3.919896in}}%
\pgfpathcurveto{\pgfqpoint{4.764660in}{3.924014in}}{\pgfqpoint{4.759074in}{3.926328in}}{\pgfqpoint{4.753250in}{3.926328in}}%
\pgfpathcurveto{\pgfqpoint{4.747426in}{3.926328in}}{\pgfqpoint{4.741840in}{3.924014in}}{\pgfqpoint{4.737722in}{3.919896in}}%
\pgfpathcurveto{\pgfqpoint{4.733604in}{3.915778in}}{\pgfqpoint{4.731290in}{3.910192in}}{\pgfqpoint{4.731290in}{3.904368in}}%
\pgfpathcurveto{\pgfqpoint{4.731290in}{3.898544in}}{\pgfqpoint{4.733604in}{3.892958in}}{\pgfqpoint{4.737722in}{3.888840in}}%
\pgfpathcurveto{\pgfqpoint{4.741840in}{3.884721in}}{\pgfqpoint{4.747426in}{3.882408in}}{\pgfqpoint{4.753250in}{3.882408in}}%
\pgfpathlineto{\pgfqpoint{4.753250in}{3.882408in}}%
\pgfpathclose%
\pgfusepath{stroke,fill}%
\end{pgfscope}%
\begin{pgfscope}%
\pgfpathrectangle{\pgfqpoint{1.000000in}{1.148311in}}{\pgfqpoint{6.200000in}{5.623377in}}%
\pgfusepath{clip}%
\pgfsetbuttcap%
\pgfsetroundjoin%
\definecolor{currentfill}{rgb}{0.800000,0.200000,0.200000}%
\pgfsetfillcolor{currentfill}%
\pgfsetlinewidth{1.003750pt}%
\definecolor{currentstroke}{rgb}{0.800000,0.200000,0.200000}%
\pgfsetstrokecolor{currentstroke}%
\pgfsetdash{}{0pt}%
\pgfpathmoveto{\pgfqpoint{4.730832in}{3.808752in}}%
\pgfpathcurveto{\pgfqpoint{4.736656in}{3.808752in}}{\pgfqpoint{4.742243in}{3.811066in}}{\pgfqpoint{4.746361in}{3.815184in}}%
\pgfpathcurveto{\pgfqpoint{4.750479in}{3.819302in}}{\pgfqpoint{4.752793in}{3.824888in}}{\pgfqpoint{4.752793in}{3.830712in}}%
\pgfpathcurveto{\pgfqpoint{4.752793in}{3.836536in}}{\pgfqpoint{4.750479in}{3.842122in}}{\pgfqpoint{4.746361in}{3.846240in}}%
\pgfpathcurveto{\pgfqpoint{4.742243in}{3.850358in}}{\pgfqpoint{4.736656in}{3.852672in}}{\pgfqpoint{4.730832in}{3.852672in}}%
\pgfpathcurveto{\pgfqpoint{4.725008in}{3.852672in}}{\pgfqpoint{4.719422in}{3.850358in}}{\pgfqpoint{4.715304in}{3.846240in}}%
\pgfpathcurveto{\pgfqpoint{4.711186in}{3.842122in}}{\pgfqpoint{4.708872in}{3.836536in}}{\pgfqpoint{4.708872in}{3.830712in}}%
\pgfpathcurveto{\pgfqpoint{4.708872in}{3.824888in}}{\pgfqpoint{4.711186in}{3.819302in}}{\pgfqpoint{4.715304in}{3.815184in}}%
\pgfpathcurveto{\pgfqpoint{4.719422in}{3.811066in}}{\pgfqpoint{4.725008in}{3.808752in}}{\pgfqpoint{4.730832in}{3.808752in}}%
\pgfpathlineto{\pgfqpoint{4.730832in}{3.808752in}}%
\pgfpathclose%
\pgfusepath{stroke,fill}%
\end{pgfscope}%
\begin{pgfscope}%
\pgfpathrectangle{\pgfqpoint{1.000000in}{1.148311in}}{\pgfqpoint{6.200000in}{5.623377in}}%
\pgfusepath{clip}%
\pgfsetbuttcap%
\pgfsetroundjoin%
\definecolor{currentfill}{rgb}{0.800000,0.200000,0.200000}%
\pgfsetfillcolor{currentfill}%
\pgfsetlinewidth{1.003750pt}%
\definecolor{currentstroke}{rgb}{0.800000,0.200000,0.200000}%
\pgfsetstrokecolor{currentstroke}%
\pgfsetdash{}{0pt}%
\pgfpathmoveto{\pgfqpoint{4.799916in}{3.756927in}}%
\pgfpathcurveto{\pgfqpoint{4.805740in}{3.756927in}}{\pgfqpoint{4.811326in}{3.759241in}}{\pgfqpoint{4.815445in}{3.763359in}}%
\pgfpathcurveto{\pgfqpoint{4.819563in}{3.767477in}}{\pgfqpoint{4.821877in}{3.773063in}}{\pgfqpoint{4.821877in}{3.778887in}}%
\pgfpathcurveto{\pgfqpoint{4.821877in}{3.784711in}}{\pgfqpoint{4.819563in}{3.790297in}}{\pgfqpoint{4.815445in}{3.794416in}}%
\pgfpathcurveto{\pgfqpoint{4.811326in}{3.798534in}}{\pgfqpoint{4.805740in}{3.800848in}}{\pgfqpoint{4.799916in}{3.800848in}}%
\pgfpathcurveto{\pgfqpoint{4.794092in}{3.800848in}}{\pgfqpoint{4.788506in}{3.798534in}}{\pgfqpoint{4.784388in}{3.794416in}}%
\pgfpathcurveto{\pgfqpoint{4.780270in}{3.790297in}}{\pgfqpoint{4.777956in}{3.784711in}}{\pgfqpoint{4.777956in}{3.778887in}}%
\pgfpathcurveto{\pgfqpoint{4.777956in}{3.773063in}}{\pgfqpoint{4.780270in}{3.767477in}}{\pgfqpoint{4.784388in}{3.763359in}}%
\pgfpathcurveto{\pgfqpoint{4.788506in}{3.759241in}}{\pgfqpoint{4.794092in}{3.756927in}}{\pgfqpoint{4.799916in}{3.756927in}}%
\pgfpathlineto{\pgfqpoint{4.799916in}{3.756927in}}%
\pgfpathclose%
\pgfusepath{stroke,fill}%
\end{pgfscope}%
\begin{pgfscope}%
\pgfpathrectangle{\pgfqpoint{1.000000in}{1.148311in}}{\pgfqpoint{6.200000in}{5.623377in}}%
\pgfusepath{clip}%
\pgfsetbuttcap%
\pgfsetroundjoin%
\definecolor{currentfill}{rgb}{0.800000,0.200000,0.200000}%
\pgfsetfillcolor{currentfill}%
\pgfsetlinewidth{1.003750pt}%
\definecolor{currentstroke}{rgb}{0.800000,0.200000,0.200000}%
\pgfsetstrokecolor{currentstroke}%
\pgfsetdash{}{0pt}%
\pgfpathmoveto{\pgfqpoint{4.756332in}{3.671071in}}%
\pgfpathcurveto{\pgfqpoint{4.762156in}{3.671071in}}{\pgfqpoint{4.767742in}{3.673385in}}{\pgfqpoint{4.771860in}{3.677503in}}%
\pgfpathcurveto{\pgfqpoint{4.775978in}{3.681621in}}{\pgfqpoint{4.778292in}{3.687208in}}{\pgfqpoint{4.778292in}{3.693032in}}%
\pgfpathcurveto{\pgfqpoint{4.778292in}{3.698856in}}{\pgfqpoint{4.775978in}{3.704442in}}{\pgfqpoint{4.771860in}{3.708560in}}%
\pgfpathcurveto{\pgfqpoint{4.767742in}{3.712678in}}{\pgfqpoint{4.762156in}{3.714992in}}{\pgfqpoint{4.756332in}{3.714992in}}%
\pgfpathcurveto{\pgfqpoint{4.750508in}{3.714992in}}{\pgfqpoint{4.744922in}{3.712678in}}{\pgfqpoint{4.740803in}{3.708560in}}%
\pgfpathcurveto{\pgfqpoint{4.736685in}{3.704442in}}{\pgfqpoint{4.734371in}{3.698856in}}{\pgfqpoint{4.734371in}{3.693032in}}%
\pgfpathcurveto{\pgfqpoint{4.734371in}{3.687208in}}{\pgfqpoint{4.736685in}{3.681621in}}{\pgfqpoint{4.740803in}{3.677503in}}%
\pgfpathcurveto{\pgfqpoint{4.744922in}{3.673385in}}{\pgfqpoint{4.750508in}{3.671071in}}{\pgfqpoint{4.756332in}{3.671071in}}%
\pgfpathlineto{\pgfqpoint{4.756332in}{3.671071in}}%
\pgfpathclose%
\pgfusepath{stroke,fill}%
\end{pgfscope}%
\begin{pgfscope}%
\pgfpathrectangle{\pgfqpoint{1.000000in}{1.148311in}}{\pgfqpoint{6.200000in}{5.623377in}}%
\pgfusepath{clip}%
\pgfsetbuttcap%
\pgfsetroundjoin%
\definecolor{currentfill}{rgb}{0.800000,0.200000,0.200000}%
\pgfsetfillcolor{currentfill}%
\pgfsetlinewidth{1.003750pt}%
\definecolor{currentstroke}{rgb}{0.800000,0.200000,0.200000}%
\pgfsetstrokecolor{currentstroke}%
\pgfsetdash{}{0pt}%
\pgfpathmoveto{\pgfqpoint{4.863720in}{3.641551in}}%
\pgfpathcurveto{\pgfqpoint{4.869544in}{3.641551in}}{\pgfqpoint{4.875130in}{3.643865in}}{\pgfqpoint{4.879248in}{3.647983in}}%
\pgfpathcurveto{\pgfqpoint{4.883366in}{3.652101in}}{\pgfqpoint{4.885680in}{3.657688in}}{\pgfqpoint{4.885680in}{3.663512in}}%
\pgfpathcurveto{\pgfqpoint{4.885680in}{3.669335in}}{\pgfqpoint{4.883366in}{3.674922in}}{\pgfqpoint{4.879248in}{3.679040in}}%
\pgfpathcurveto{\pgfqpoint{4.875130in}{3.683158in}}{\pgfqpoint{4.869544in}{3.685472in}}{\pgfqpoint{4.863720in}{3.685472in}}%
\pgfpathcurveto{\pgfqpoint{4.857896in}{3.685472in}}{\pgfqpoint{4.852310in}{3.683158in}}{\pgfqpoint{4.848192in}{3.679040in}}%
\pgfpathcurveto{\pgfqpoint{4.844074in}{3.674922in}}{\pgfqpoint{4.841760in}{3.669335in}}{\pgfqpoint{4.841760in}{3.663512in}}%
\pgfpathcurveto{\pgfqpoint{4.841760in}{3.657688in}}{\pgfqpoint{4.844074in}{3.652101in}}{\pgfqpoint{4.848192in}{3.647983in}}%
\pgfpathcurveto{\pgfqpoint{4.852310in}{3.643865in}}{\pgfqpoint{4.857896in}{3.641551in}}{\pgfqpoint{4.863720in}{3.641551in}}%
\pgfpathlineto{\pgfqpoint{4.863720in}{3.641551in}}%
\pgfpathclose%
\pgfusepath{stroke,fill}%
\end{pgfscope}%
\begin{pgfscope}%
\pgfpathrectangle{\pgfqpoint{1.000000in}{1.148311in}}{\pgfqpoint{6.200000in}{5.623377in}}%
\pgfusepath{clip}%
\pgfsetbuttcap%
\pgfsetroundjoin%
\definecolor{currentfill}{rgb}{0.800000,0.200000,0.200000}%
\pgfsetfillcolor{currentfill}%
\pgfsetlinewidth{1.003750pt}%
\definecolor{currentstroke}{rgb}{0.800000,0.200000,0.200000}%
\pgfsetstrokecolor{currentstroke}%
\pgfsetdash{}{0pt}%
\pgfpathmoveto{\pgfqpoint{4.840446in}{3.557021in}}%
\pgfpathcurveto{\pgfqpoint{4.846270in}{3.557021in}}{\pgfqpoint{4.851856in}{3.559335in}}{\pgfqpoint{4.855974in}{3.563453in}}%
\pgfpathcurveto{\pgfqpoint{4.860093in}{3.567571in}}{\pgfqpoint{4.862407in}{3.573157in}}{\pgfqpoint{4.862407in}{3.578981in}}%
\pgfpathcurveto{\pgfqpoint{4.862407in}{3.584805in}}{\pgfqpoint{4.860093in}{3.590391in}}{\pgfqpoint{4.855974in}{3.594510in}}%
\pgfpathcurveto{\pgfqpoint{4.851856in}{3.598628in}}{\pgfqpoint{4.846270in}{3.600942in}}{\pgfqpoint{4.840446in}{3.600942in}}%
\pgfpathcurveto{\pgfqpoint{4.834622in}{3.600942in}}{\pgfqpoint{4.829036in}{3.598628in}}{\pgfqpoint{4.824918in}{3.594510in}}%
\pgfpathcurveto{\pgfqpoint{4.820800in}{3.590391in}}{\pgfqpoint{4.818486in}{3.584805in}}{\pgfqpoint{4.818486in}{3.578981in}}%
\pgfpathcurveto{\pgfqpoint{4.818486in}{3.573157in}}{\pgfqpoint{4.820800in}{3.567571in}}{\pgfqpoint{4.824918in}{3.563453in}}%
\pgfpathcurveto{\pgfqpoint{4.829036in}{3.559335in}}{\pgfqpoint{4.834622in}{3.557021in}}{\pgfqpoint{4.840446in}{3.557021in}}%
\pgfpathlineto{\pgfqpoint{4.840446in}{3.557021in}}%
\pgfpathclose%
\pgfusepath{stroke,fill}%
\end{pgfscope}%
\begin{pgfscope}%
\pgfpathrectangle{\pgfqpoint{1.000000in}{1.148311in}}{\pgfqpoint{6.200000in}{5.623377in}}%
\pgfusepath{clip}%
\pgfsetbuttcap%
\pgfsetroundjoin%
\definecolor{currentfill}{rgb}{0.800000,0.200000,0.200000}%
\pgfsetfillcolor{currentfill}%
\pgfsetlinewidth{1.003750pt}%
\definecolor{currentstroke}{rgb}{0.800000,0.200000,0.200000}%
\pgfsetstrokecolor{currentstroke}%
\pgfsetdash{}{0pt}%
\pgfpathmoveto{\pgfqpoint{4.862187in}{3.490656in}}%
\pgfpathcurveto{\pgfqpoint{4.868011in}{3.490656in}}{\pgfqpoint{4.873597in}{3.492970in}}{\pgfqpoint{4.877716in}{3.497089in}}%
\pgfpathcurveto{\pgfqpoint{4.881834in}{3.501207in}}{\pgfqpoint{4.884148in}{3.506793in}}{\pgfqpoint{4.884148in}{3.512617in}}%
\pgfpathcurveto{\pgfqpoint{4.884148in}{3.518441in}}{\pgfqpoint{4.881834in}{3.524027in}}{\pgfqpoint{4.877716in}{3.528145in}}%
\pgfpathcurveto{\pgfqpoint{4.873597in}{3.532263in}}{\pgfqpoint{4.868011in}{3.534577in}}{\pgfqpoint{4.862187in}{3.534577in}}%
\pgfpathcurveto{\pgfqpoint{4.856363in}{3.534577in}}{\pgfqpoint{4.850777in}{3.532263in}}{\pgfqpoint{4.846659in}{3.528145in}}%
\pgfpathcurveto{\pgfqpoint{4.842541in}{3.524027in}}{\pgfqpoint{4.840227in}{3.518441in}}{\pgfqpoint{4.840227in}{3.512617in}}%
\pgfpathcurveto{\pgfqpoint{4.840227in}{3.506793in}}{\pgfqpoint{4.842541in}{3.501207in}}{\pgfqpoint{4.846659in}{3.497089in}}%
\pgfpathcurveto{\pgfqpoint{4.850777in}{3.492970in}}{\pgfqpoint{4.856363in}{3.490656in}}{\pgfqpoint{4.862187in}{3.490656in}}%
\pgfpathlineto{\pgfqpoint{4.862187in}{3.490656in}}%
\pgfpathclose%
\pgfusepath{stroke,fill}%
\end{pgfscope}%
\begin{pgfscope}%
\pgfpathrectangle{\pgfqpoint{1.000000in}{1.148311in}}{\pgfqpoint{6.200000in}{5.623377in}}%
\pgfusepath{clip}%
\pgfsetbuttcap%
\pgfsetroundjoin%
\definecolor{currentfill}{rgb}{0.800000,0.200000,0.200000}%
\pgfsetfillcolor{currentfill}%
\pgfsetlinewidth{1.003750pt}%
\definecolor{currentstroke}{rgb}{0.800000,0.200000,0.200000}%
\pgfsetstrokecolor{currentstroke}%
\pgfsetdash{}{0pt}%
\pgfpathmoveto{\pgfqpoint{4.904268in}{3.435917in}}%
\pgfpathcurveto{\pgfqpoint{4.910092in}{3.435917in}}{\pgfqpoint{4.915678in}{3.438231in}}{\pgfqpoint{4.919796in}{3.442349in}}%
\pgfpathcurveto{\pgfqpoint{4.923914in}{3.446467in}}{\pgfqpoint{4.926228in}{3.452054in}}{\pgfqpoint{4.926228in}{3.457877in}}%
\pgfpathcurveto{\pgfqpoint{4.926228in}{3.463701in}}{\pgfqpoint{4.923914in}{3.469288in}}{\pgfqpoint{4.919796in}{3.473406in}}%
\pgfpathcurveto{\pgfqpoint{4.915678in}{3.477524in}}{\pgfqpoint{4.910092in}{3.479838in}}{\pgfqpoint{4.904268in}{3.479838in}}%
\pgfpathcurveto{\pgfqpoint{4.898444in}{3.479838in}}{\pgfqpoint{4.892858in}{3.477524in}}{\pgfqpoint{4.888739in}{3.473406in}}%
\pgfpathcurveto{\pgfqpoint{4.884621in}{3.469288in}}{\pgfqpoint{4.882307in}{3.463701in}}{\pgfqpoint{4.882307in}{3.457877in}}%
\pgfpathcurveto{\pgfqpoint{4.882307in}{3.452054in}}{\pgfqpoint{4.884621in}{3.446467in}}{\pgfqpoint{4.888739in}{3.442349in}}%
\pgfpathcurveto{\pgfqpoint{4.892858in}{3.438231in}}{\pgfqpoint{4.898444in}{3.435917in}}{\pgfqpoint{4.904268in}{3.435917in}}%
\pgfpathlineto{\pgfqpoint{4.904268in}{3.435917in}}%
\pgfpathclose%
\pgfusepath{stroke,fill}%
\end{pgfscope}%
\begin{pgfscope}%
\pgfpathrectangle{\pgfqpoint{1.000000in}{1.148311in}}{\pgfqpoint{6.200000in}{5.623377in}}%
\pgfusepath{clip}%
\pgfsetbuttcap%
\pgfsetroundjoin%
\definecolor{currentfill}{rgb}{0.800000,0.200000,0.200000}%
\pgfsetfillcolor{currentfill}%
\pgfsetlinewidth{1.003750pt}%
\definecolor{currentstroke}{rgb}{0.800000,0.200000,0.200000}%
\pgfsetstrokecolor{currentstroke}%
\pgfsetdash{}{0pt}%
\pgfpathmoveto{\pgfqpoint{4.899198in}{3.344683in}}%
\pgfpathcurveto{\pgfqpoint{4.905022in}{3.344683in}}{\pgfqpoint{4.910608in}{3.346997in}}{\pgfqpoint{4.914726in}{3.351115in}}%
\pgfpathcurveto{\pgfqpoint{4.918844in}{3.355233in}}{\pgfqpoint{4.921158in}{3.360819in}}{\pgfqpoint{4.921158in}{3.366643in}}%
\pgfpathcurveto{\pgfqpoint{4.921158in}{3.372467in}}{\pgfqpoint{4.918844in}{3.378053in}}{\pgfqpoint{4.914726in}{3.382171in}}%
\pgfpathcurveto{\pgfqpoint{4.910608in}{3.386289in}}{\pgfqpoint{4.905022in}{3.388603in}}{\pgfqpoint{4.899198in}{3.388603in}}%
\pgfpathcurveto{\pgfqpoint{4.893374in}{3.388603in}}{\pgfqpoint{4.887788in}{3.386289in}}{\pgfqpoint{4.883669in}{3.382171in}}%
\pgfpathcurveto{\pgfqpoint{4.879551in}{3.378053in}}{\pgfqpoint{4.877237in}{3.372467in}}{\pgfqpoint{4.877237in}{3.366643in}}%
\pgfpathcurveto{\pgfqpoint{4.877237in}{3.360819in}}{\pgfqpoint{4.879551in}{3.355233in}}{\pgfqpoint{4.883669in}{3.351115in}}%
\pgfpathcurveto{\pgfqpoint{4.887788in}{3.346997in}}{\pgfqpoint{4.893374in}{3.344683in}}{\pgfqpoint{4.899198in}{3.344683in}}%
\pgfpathlineto{\pgfqpoint{4.899198in}{3.344683in}}%
\pgfpathclose%
\pgfusepath{stroke,fill}%
\end{pgfscope}%
\begin{pgfscope}%
\pgfpathrectangle{\pgfqpoint{1.000000in}{1.148311in}}{\pgfqpoint{6.200000in}{5.623377in}}%
\pgfusepath{clip}%
\pgfsetbuttcap%
\pgfsetroundjoin%
\definecolor{currentfill}{rgb}{0.800000,0.200000,0.200000}%
\pgfsetfillcolor{currentfill}%
\pgfsetlinewidth{1.003750pt}%
\definecolor{currentstroke}{rgb}{0.800000,0.200000,0.200000}%
\pgfsetstrokecolor{currentstroke}%
\pgfsetdash{}{0pt}%
\pgfpathmoveto{\pgfqpoint{5.026754in}{3.361828in}}%
\pgfpathcurveto{\pgfqpoint{5.032578in}{3.361828in}}{\pgfqpoint{5.038164in}{3.364142in}}{\pgfqpoint{5.042282in}{3.368260in}}%
\pgfpathcurveto{\pgfqpoint{5.046400in}{3.372378in}}{\pgfqpoint{5.048714in}{3.377964in}}{\pgfqpoint{5.048714in}{3.383788in}}%
\pgfpathcurveto{\pgfqpoint{5.048714in}{3.389612in}}{\pgfqpoint{5.046400in}{3.395198in}}{\pgfqpoint{5.042282in}{3.399316in}}%
\pgfpathcurveto{\pgfqpoint{5.038164in}{3.403434in}}{\pgfqpoint{5.032578in}{3.405748in}}{\pgfqpoint{5.026754in}{3.405748in}}%
\pgfpathcurveto{\pgfqpoint{5.020930in}{3.405748in}}{\pgfqpoint{5.015344in}{3.403434in}}{\pgfqpoint{5.011226in}{3.399316in}}%
\pgfpathcurveto{\pgfqpoint{5.007107in}{3.395198in}}{\pgfqpoint{5.004794in}{3.389612in}}{\pgfqpoint{5.004794in}{3.383788in}}%
\pgfpathcurveto{\pgfqpoint{5.004794in}{3.377964in}}{\pgfqpoint{5.007107in}{3.372378in}}{\pgfqpoint{5.011226in}{3.368260in}}%
\pgfpathcurveto{\pgfqpoint{5.015344in}{3.364142in}}{\pgfqpoint{5.020930in}{3.361828in}}{\pgfqpoint{5.026754in}{3.361828in}}%
\pgfpathlineto{\pgfqpoint{5.026754in}{3.361828in}}%
\pgfpathclose%
\pgfusepath{stroke,fill}%
\end{pgfscope}%
\begin{pgfscope}%
\pgfpathrectangle{\pgfqpoint{1.000000in}{1.148311in}}{\pgfqpoint{6.200000in}{5.623377in}}%
\pgfusepath{clip}%
\pgfsetbuttcap%
\pgfsetroundjoin%
\definecolor{currentfill}{rgb}{0.800000,0.200000,0.200000}%
\pgfsetfillcolor{currentfill}%
\pgfsetlinewidth{1.003750pt}%
\definecolor{currentstroke}{rgb}{0.800000,0.200000,0.200000}%
\pgfsetstrokecolor{currentstroke}%
\pgfsetdash{}{0pt}%
\pgfpathmoveto{\pgfqpoint{5.016410in}{3.258125in}}%
\pgfpathcurveto{\pgfqpoint{5.022234in}{3.258125in}}{\pgfqpoint{5.027820in}{3.260439in}}{\pgfqpoint{5.031938in}{3.264557in}}%
\pgfpathcurveto{\pgfqpoint{5.036057in}{3.268675in}}{\pgfqpoint{5.038370in}{3.274262in}}{\pgfqpoint{5.038370in}{3.280086in}}%
\pgfpathcurveto{\pgfqpoint{5.038370in}{3.285909in}}{\pgfqpoint{5.036057in}{3.291496in}}{\pgfqpoint{5.031938in}{3.295614in}}%
\pgfpathcurveto{\pgfqpoint{5.027820in}{3.299732in}}{\pgfqpoint{5.022234in}{3.302046in}}{\pgfqpoint{5.016410in}{3.302046in}}%
\pgfpathcurveto{\pgfqpoint{5.010586in}{3.302046in}}{\pgfqpoint{5.005000in}{3.299732in}}{\pgfqpoint{5.000882in}{3.295614in}}%
\pgfpathcurveto{\pgfqpoint{4.996764in}{3.291496in}}{\pgfqpoint{4.994450in}{3.285909in}}{\pgfqpoint{4.994450in}{3.280086in}}%
\pgfpathcurveto{\pgfqpoint{4.994450in}{3.274262in}}{\pgfqpoint{4.996764in}{3.268675in}}{\pgfqpoint{5.000882in}{3.264557in}}%
\pgfpathcurveto{\pgfqpoint{5.005000in}{3.260439in}}{\pgfqpoint{5.010586in}{3.258125in}}{\pgfqpoint{5.016410in}{3.258125in}}%
\pgfpathlineto{\pgfqpoint{5.016410in}{3.258125in}}%
\pgfpathclose%
\pgfusepath{stroke,fill}%
\end{pgfscope}%
\begin{pgfscope}%
\pgfpathrectangle{\pgfqpoint{1.000000in}{1.148311in}}{\pgfqpoint{6.200000in}{5.623377in}}%
\pgfusepath{clip}%
\pgfsetbuttcap%
\pgfsetroundjoin%
\definecolor{currentfill}{rgb}{0.800000,0.200000,0.200000}%
\pgfsetfillcolor{currentfill}%
\pgfsetlinewidth{1.003750pt}%
\definecolor{currentstroke}{rgb}{0.800000,0.200000,0.200000}%
\pgfsetstrokecolor{currentstroke}%
\pgfsetdash{}{0pt}%
\pgfpathmoveto{\pgfqpoint{5.074834in}{3.218063in}}%
\pgfpathcurveto{\pgfqpoint{5.080658in}{3.218063in}}{\pgfqpoint{5.086245in}{3.220377in}}{\pgfqpoint{5.090363in}{3.224495in}}%
\pgfpathcurveto{\pgfqpoint{5.094481in}{3.228614in}}{\pgfqpoint{5.096795in}{3.234200in}}{\pgfqpoint{5.096795in}{3.240024in}}%
\pgfpathcurveto{\pgfqpoint{5.096795in}{3.245848in}}{\pgfqpoint{5.094481in}{3.251434in}}{\pgfqpoint{5.090363in}{3.255552in}}%
\pgfpathcurveto{\pgfqpoint{5.086245in}{3.259670in}}{\pgfqpoint{5.080658in}{3.261984in}}{\pgfqpoint{5.074834in}{3.261984in}}%
\pgfpathcurveto{\pgfqpoint{5.069010in}{3.261984in}}{\pgfqpoint{5.063424in}{3.259670in}}{\pgfqpoint{5.059306in}{3.255552in}}%
\pgfpathcurveto{\pgfqpoint{5.055188in}{3.251434in}}{\pgfqpoint{5.052874in}{3.245848in}}{\pgfqpoint{5.052874in}{3.240024in}}%
\pgfpathcurveto{\pgfqpoint{5.052874in}{3.234200in}}{\pgfqpoint{5.055188in}{3.228614in}}{\pgfqpoint{5.059306in}{3.224495in}}%
\pgfpathcurveto{\pgfqpoint{5.063424in}{3.220377in}}{\pgfqpoint{5.069010in}{3.218063in}}{\pgfqpoint{5.074834in}{3.218063in}}%
\pgfpathlineto{\pgfqpoint{5.074834in}{3.218063in}}%
\pgfpathclose%
\pgfusepath{stroke,fill}%
\end{pgfscope}%
\begin{pgfscope}%
\pgfpathrectangle{\pgfqpoint{1.000000in}{1.148311in}}{\pgfqpoint{6.200000in}{5.623377in}}%
\pgfusepath{clip}%
\pgfsetbuttcap%
\pgfsetroundjoin%
\definecolor{currentfill}{rgb}{0.800000,0.200000,0.200000}%
\pgfsetfillcolor{currentfill}%
\pgfsetlinewidth{1.003750pt}%
\definecolor{currentstroke}{rgb}{0.800000,0.200000,0.200000}%
\pgfsetstrokecolor{currentstroke}%
\pgfsetdash{}{0pt}%
\pgfpathmoveto{\pgfqpoint{5.155507in}{3.208907in}}%
\pgfpathcurveto{\pgfqpoint{5.161331in}{3.208907in}}{\pgfqpoint{5.166917in}{3.211221in}}{\pgfqpoint{5.171035in}{3.215339in}}%
\pgfpathcurveto{\pgfqpoint{5.175153in}{3.219457in}}{\pgfqpoint{5.177467in}{3.225044in}}{\pgfqpoint{5.177467in}{3.230867in}}%
\pgfpathcurveto{\pgfqpoint{5.177467in}{3.236691in}}{\pgfqpoint{5.175153in}{3.242278in}}{\pgfqpoint{5.171035in}{3.246396in}}%
\pgfpathcurveto{\pgfqpoint{5.166917in}{3.250514in}}{\pgfqpoint{5.161331in}{3.252828in}}{\pgfqpoint{5.155507in}{3.252828in}}%
\pgfpathcurveto{\pgfqpoint{5.149683in}{3.252828in}}{\pgfqpoint{5.144097in}{3.250514in}}{\pgfqpoint{5.139979in}{3.246396in}}%
\pgfpathcurveto{\pgfqpoint{5.135860in}{3.242278in}}{\pgfqpoint{5.133546in}{3.236691in}}{\pgfqpoint{5.133546in}{3.230867in}}%
\pgfpathcurveto{\pgfqpoint{5.133546in}{3.225044in}}{\pgfqpoint{5.135860in}{3.219457in}}{\pgfqpoint{5.139979in}{3.215339in}}%
\pgfpathcurveto{\pgfqpoint{5.144097in}{3.211221in}}{\pgfqpoint{5.149683in}{3.208907in}}{\pgfqpoint{5.155507in}{3.208907in}}%
\pgfpathlineto{\pgfqpoint{5.155507in}{3.208907in}}%
\pgfpathclose%
\pgfusepath{stroke,fill}%
\end{pgfscope}%
\begin{pgfscope}%
\pgfpathrectangle{\pgfqpoint{1.000000in}{1.148311in}}{\pgfqpoint{6.200000in}{5.623377in}}%
\pgfusepath{clip}%
\pgfsetbuttcap%
\pgfsetroundjoin%
\definecolor{currentfill}{rgb}{0.800000,0.200000,0.200000}%
\pgfsetfillcolor{currentfill}%
\pgfsetlinewidth{1.003750pt}%
\definecolor{currentstroke}{rgb}{0.800000,0.200000,0.200000}%
\pgfsetstrokecolor{currentstroke}%
\pgfsetdash{}{0pt}%
\pgfpathmoveto{\pgfqpoint{5.170841in}{3.111064in}}%
\pgfpathcurveto{\pgfqpoint{5.176665in}{3.111064in}}{\pgfqpoint{5.182251in}{3.113378in}}{\pgfqpoint{5.186369in}{3.117496in}}%
\pgfpathcurveto{\pgfqpoint{5.190487in}{3.121614in}}{\pgfqpoint{5.192801in}{3.127200in}}{\pgfqpoint{5.192801in}{3.133024in}}%
\pgfpathcurveto{\pgfqpoint{5.192801in}{3.138848in}}{\pgfqpoint{5.190487in}{3.144434in}}{\pgfqpoint{5.186369in}{3.148552in}}%
\pgfpathcurveto{\pgfqpoint{5.182251in}{3.152670in}}{\pgfqpoint{5.176665in}{3.154984in}}{\pgfqpoint{5.170841in}{3.154984in}}%
\pgfpathcurveto{\pgfqpoint{5.165017in}{3.154984in}}{\pgfqpoint{5.159431in}{3.152670in}}{\pgfqpoint{5.155313in}{3.148552in}}%
\pgfpathcurveto{\pgfqpoint{5.151194in}{3.144434in}}{\pgfqpoint{5.148881in}{3.138848in}}{\pgfqpoint{5.148881in}{3.133024in}}%
\pgfpathcurveto{\pgfqpoint{5.148881in}{3.127200in}}{\pgfqpoint{5.151194in}{3.121614in}}{\pgfqpoint{5.155313in}{3.117496in}}%
\pgfpathcurveto{\pgfqpoint{5.159431in}{3.113378in}}{\pgfqpoint{5.165017in}{3.111064in}}{\pgfqpoint{5.170841in}{3.111064in}}%
\pgfpathlineto{\pgfqpoint{5.170841in}{3.111064in}}%
\pgfpathclose%
\pgfusepath{stroke,fill}%
\end{pgfscope}%
\begin{pgfscope}%
\pgfpathrectangle{\pgfqpoint{1.000000in}{1.148311in}}{\pgfqpoint{6.200000in}{5.623377in}}%
\pgfusepath{clip}%
\pgfsetbuttcap%
\pgfsetroundjoin%
\definecolor{currentfill}{rgb}{0.800000,0.200000,0.200000}%
\pgfsetfillcolor{currentfill}%
\pgfsetlinewidth{1.003750pt}%
\definecolor{currentstroke}{rgb}{0.800000,0.200000,0.200000}%
\pgfsetstrokecolor{currentstroke}%
\pgfsetdash{}{0pt}%
\pgfpathmoveto{\pgfqpoint{5.270891in}{3.141441in}}%
\pgfpathcurveto{\pgfqpoint{5.276715in}{3.141441in}}{\pgfqpoint{5.282301in}{3.143755in}}{\pgfqpoint{5.286419in}{3.147873in}}%
\pgfpathcurveto{\pgfqpoint{5.290537in}{3.151991in}}{\pgfqpoint{5.292851in}{3.157577in}}{\pgfqpoint{5.292851in}{3.163401in}}%
\pgfpathcurveto{\pgfqpoint{5.292851in}{3.169225in}}{\pgfqpoint{5.290537in}{3.174811in}}{\pgfqpoint{5.286419in}{3.178930in}}%
\pgfpathcurveto{\pgfqpoint{5.282301in}{3.183048in}}{\pgfqpoint{5.276715in}{3.185362in}}{\pgfqpoint{5.270891in}{3.185362in}}%
\pgfpathcurveto{\pgfqpoint{5.265067in}{3.185362in}}{\pgfqpoint{5.259481in}{3.183048in}}{\pgfqpoint{5.255363in}{3.178930in}}%
\pgfpathcurveto{\pgfqpoint{5.251244in}{3.174811in}}{\pgfqpoint{5.248930in}{3.169225in}}{\pgfqpoint{5.248930in}{3.163401in}}%
\pgfpathcurveto{\pgfqpoint{5.248930in}{3.157577in}}{\pgfqpoint{5.251244in}{3.151991in}}{\pgfqpoint{5.255363in}{3.147873in}}%
\pgfpathcurveto{\pgfqpoint{5.259481in}{3.143755in}}{\pgfqpoint{5.265067in}{3.141441in}}{\pgfqpoint{5.270891in}{3.141441in}}%
\pgfpathlineto{\pgfqpoint{5.270891in}{3.141441in}}%
\pgfpathclose%
\pgfusepath{stroke,fill}%
\end{pgfscope}%
\begin{pgfscope}%
\pgfpathrectangle{\pgfqpoint{1.000000in}{1.148311in}}{\pgfqpoint{6.200000in}{5.623377in}}%
\pgfusepath{clip}%
\pgfsetbuttcap%
\pgfsetroundjoin%
\definecolor{currentfill}{rgb}{0.800000,0.200000,0.200000}%
\pgfsetfillcolor{currentfill}%
\pgfsetlinewidth{1.003750pt}%
\definecolor{currentstroke}{rgb}{0.800000,0.200000,0.200000}%
\pgfsetstrokecolor{currentstroke}%
\pgfsetdash{}{0pt}%
\pgfpathmoveto{\pgfqpoint{5.326544in}{3.104043in}}%
\pgfpathcurveto{\pgfqpoint{5.332368in}{3.104043in}}{\pgfqpoint{5.337954in}{3.106357in}}{\pgfqpoint{5.342072in}{3.110475in}}%
\pgfpathcurveto{\pgfqpoint{5.346190in}{3.114594in}}{\pgfqpoint{5.348504in}{3.120180in}}{\pgfqpoint{5.348504in}{3.126004in}}%
\pgfpathcurveto{\pgfqpoint{5.348504in}{3.131828in}}{\pgfqpoint{5.346190in}{3.137414in}}{\pgfqpoint{5.342072in}{3.141532in}}%
\pgfpathcurveto{\pgfqpoint{5.337954in}{3.145650in}}{\pgfqpoint{5.332368in}{3.147964in}}{\pgfqpoint{5.326544in}{3.147964in}}%
\pgfpathcurveto{\pgfqpoint{5.320720in}{3.147964in}}{\pgfqpoint{5.315134in}{3.145650in}}{\pgfqpoint{5.311016in}{3.141532in}}%
\pgfpathcurveto{\pgfqpoint{5.306898in}{3.137414in}}{\pgfqpoint{5.304584in}{3.131828in}}{\pgfqpoint{5.304584in}{3.126004in}}%
\pgfpathcurveto{\pgfqpoint{5.304584in}{3.120180in}}{\pgfqpoint{5.306898in}{3.114594in}}{\pgfqpoint{5.311016in}{3.110475in}}%
\pgfpathcurveto{\pgfqpoint{5.315134in}{3.106357in}}{\pgfqpoint{5.320720in}{3.104043in}}{\pgfqpoint{5.326544in}{3.104043in}}%
\pgfpathlineto{\pgfqpoint{5.326544in}{3.104043in}}%
\pgfpathclose%
\pgfusepath{stroke,fill}%
\end{pgfscope}%
\begin{pgfscope}%
\pgfpathrectangle{\pgfqpoint{1.000000in}{1.148311in}}{\pgfqpoint{6.200000in}{5.623377in}}%
\pgfusepath{clip}%
\pgfsetbuttcap%
\pgfsetroundjoin%
\definecolor{currentfill}{rgb}{0.800000,0.200000,0.200000}%
\pgfsetfillcolor{currentfill}%
\pgfsetlinewidth{1.003750pt}%
\definecolor{currentstroke}{rgb}{0.800000,0.200000,0.200000}%
\pgfsetstrokecolor{currentstroke}%
\pgfsetdash{}{0pt}%
\pgfpathmoveto{\pgfqpoint{5.387850in}{3.077255in}}%
\pgfpathcurveto{\pgfqpoint{5.393674in}{3.077255in}}{\pgfqpoint{5.399261in}{3.079569in}}{\pgfqpoint{5.403379in}{3.083687in}}%
\pgfpathcurveto{\pgfqpoint{5.407497in}{3.087805in}}{\pgfqpoint{5.409811in}{3.093391in}}{\pgfqpoint{5.409811in}{3.099215in}}%
\pgfpathcurveto{\pgfqpoint{5.409811in}{3.105039in}}{\pgfqpoint{5.407497in}{3.110625in}}{\pgfqpoint{5.403379in}{3.114743in}}%
\pgfpathcurveto{\pgfqpoint{5.399261in}{3.118861in}}{\pgfqpoint{5.393674in}{3.121175in}}{\pgfqpoint{5.387850in}{3.121175in}}%
\pgfpathcurveto{\pgfqpoint{5.382027in}{3.121175in}}{\pgfqpoint{5.376440in}{3.118861in}}{\pgfqpoint{5.372322in}{3.114743in}}%
\pgfpathcurveto{\pgfqpoint{5.368204in}{3.110625in}}{\pgfqpoint{5.365890in}{3.105039in}}{\pgfqpoint{5.365890in}{3.099215in}}%
\pgfpathcurveto{\pgfqpoint{5.365890in}{3.093391in}}{\pgfqpoint{5.368204in}{3.087805in}}{\pgfqpoint{5.372322in}{3.083687in}}%
\pgfpathcurveto{\pgfqpoint{5.376440in}{3.079569in}}{\pgfqpoint{5.382027in}{3.077255in}}{\pgfqpoint{5.387850in}{3.077255in}}%
\pgfpathlineto{\pgfqpoint{5.387850in}{3.077255in}}%
\pgfpathclose%
\pgfusepath{stroke,fill}%
\end{pgfscope}%
\begin{pgfscope}%
\pgfpathrectangle{\pgfqpoint{1.000000in}{1.148311in}}{\pgfqpoint{6.200000in}{5.623377in}}%
\pgfusepath{clip}%
\pgfsetbuttcap%
\pgfsetroundjoin%
\definecolor{currentfill}{rgb}{0.800000,0.200000,0.200000}%
\pgfsetfillcolor{currentfill}%
\pgfsetlinewidth{1.003750pt}%
\definecolor{currentstroke}{rgb}{0.800000,0.200000,0.200000}%
\pgfsetstrokecolor{currentstroke}%
\pgfsetdash{}{0pt}%
\pgfpathmoveto{\pgfqpoint{5.466719in}{3.100811in}}%
\pgfpathcurveto{\pgfqpoint{5.472543in}{3.100811in}}{\pgfqpoint{5.478129in}{3.103125in}}{\pgfqpoint{5.482247in}{3.107243in}}%
\pgfpathcurveto{\pgfqpoint{5.486365in}{3.111361in}}{\pgfqpoint{5.488679in}{3.116947in}}{\pgfqpoint{5.488679in}{3.122771in}}%
\pgfpathcurveto{\pgfqpoint{5.488679in}{3.128595in}}{\pgfqpoint{5.486365in}{3.134181in}}{\pgfqpoint{5.482247in}{3.138299in}}%
\pgfpathcurveto{\pgfqpoint{5.478129in}{3.142417in}}{\pgfqpoint{5.472543in}{3.144731in}}{\pgfqpoint{5.466719in}{3.144731in}}%
\pgfpathcurveto{\pgfqpoint{5.460895in}{3.144731in}}{\pgfqpoint{5.455309in}{3.142417in}}{\pgfqpoint{5.451190in}{3.138299in}}%
\pgfpathcurveto{\pgfqpoint{5.447072in}{3.134181in}}{\pgfqpoint{5.444758in}{3.128595in}}{\pgfqpoint{5.444758in}{3.122771in}}%
\pgfpathcurveto{\pgfqpoint{5.444758in}{3.116947in}}{\pgfqpoint{5.447072in}{3.111361in}}{\pgfqpoint{5.451190in}{3.107243in}}%
\pgfpathcurveto{\pgfqpoint{5.455309in}{3.103125in}}{\pgfqpoint{5.460895in}{3.100811in}}{\pgfqpoint{5.466719in}{3.100811in}}%
\pgfpathlineto{\pgfqpoint{5.466719in}{3.100811in}}%
\pgfpathclose%
\pgfusepath{stroke,fill}%
\end{pgfscope}%
\begin{pgfscope}%
\pgfpathrectangle{\pgfqpoint{1.000000in}{1.148311in}}{\pgfqpoint{6.200000in}{5.623377in}}%
\pgfusepath{clip}%
\pgfsetbuttcap%
\pgfsetroundjoin%
\definecolor{currentfill}{rgb}{0.800000,0.200000,0.200000}%
\pgfsetfillcolor{currentfill}%
\pgfsetlinewidth{1.003750pt}%
\definecolor{currentstroke}{rgb}{0.800000,0.200000,0.200000}%
\pgfsetstrokecolor{currentstroke}%
\pgfsetdash{}{0pt}%
\pgfpathmoveto{\pgfqpoint{5.506062in}{3.004281in}}%
\pgfpathcurveto{\pgfqpoint{5.511886in}{3.004281in}}{\pgfqpoint{5.517472in}{3.006594in}}{\pgfqpoint{5.521590in}{3.010713in}}%
\pgfpathcurveto{\pgfqpoint{5.525709in}{3.014831in}}{\pgfqpoint{5.528022in}{3.020417in}}{\pgfqpoint{5.528022in}{3.026241in}}%
\pgfpathcurveto{\pgfqpoint{5.528022in}{3.032065in}}{\pgfqpoint{5.525709in}{3.037651in}}{\pgfqpoint{5.521590in}{3.041769in}}%
\pgfpathcurveto{\pgfqpoint{5.517472in}{3.045887in}}{\pgfqpoint{5.511886in}{3.048201in}}{\pgfqpoint{5.506062in}{3.048201in}}%
\pgfpathcurveto{\pgfqpoint{5.500238in}{3.048201in}}{\pgfqpoint{5.494652in}{3.045887in}}{\pgfqpoint{5.490534in}{3.041769in}}%
\pgfpathcurveto{\pgfqpoint{5.486416in}{3.037651in}}{\pgfqpoint{5.484102in}{3.032065in}}{\pgfqpoint{5.484102in}{3.026241in}}%
\pgfpathcurveto{\pgfqpoint{5.484102in}{3.020417in}}{\pgfqpoint{5.486416in}{3.014831in}}{\pgfqpoint{5.490534in}{3.010713in}}%
\pgfpathcurveto{\pgfqpoint{5.494652in}{3.006594in}}{\pgfqpoint{5.500238in}{3.004281in}}{\pgfqpoint{5.506062in}{3.004281in}}%
\pgfpathlineto{\pgfqpoint{5.506062in}{3.004281in}}%
\pgfpathclose%
\pgfusepath{stroke,fill}%
\end{pgfscope}%
\begin{pgfscope}%
\pgfpathrectangle{\pgfqpoint{1.000000in}{1.148311in}}{\pgfqpoint{6.200000in}{5.623377in}}%
\pgfusepath{clip}%
\pgfsetbuttcap%
\pgfsetroundjoin%
\definecolor{currentfill}{rgb}{0.800000,0.200000,0.200000}%
\pgfsetfillcolor{currentfill}%
\pgfsetlinewidth{1.003750pt}%
\definecolor{currentstroke}{rgb}{0.800000,0.200000,0.200000}%
\pgfsetstrokecolor{currentstroke}%
\pgfsetdash{}{0pt}%
\pgfpathmoveto{\pgfqpoint{5.574175in}{2.994112in}}%
\pgfpathcurveto{\pgfqpoint{5.579999in}{2.994112in}}{\pgfqpoint{5.585585in}{2.996426in}}{\pgfqpoint{5.589703in}{3.000544in}}%
\pgfpathcurveto{\pgfqpoint{5.593821in}{3.004662in}}{\pgfqpoint{5.596135in}{3.010248in}}{\pgfqpoint{5.596135in}{3.016072in}}%
\pgfpathcurveto{\pgfqpoint{5.596135in}{3.021896in}}{\pgfqpoint{5.593821in}{3.027482in}}{\pgfqpoint{5.589703in}{3.031600in}}%
\pgfpathcurveto{\pgfqpoint{5.585585in}{3.035718in}}{\pgfqpoint{5.579999in}{3.038032in}}{\pgfqpoint{5.574175in}{3.038032in}}%
\pgfpathcurveto{\pgfqpoint{5.568351in}{3.038032in}}{\pgfqpoint{5.562765in}{3.035718in}}{\pgfqpoint{5.558647in}{3.031600in}}%
\pgfpathcurveto{\pgfqpoint{5.554529in}{3.027482in}}{\pgfqpoint{5.552215in}{3.021896in}}{\pgfqpoint{5.552215in}{3.016072in}}%
\pgfpathcurveto{\pgfqpoint{5.552215in}{3.010248in}}{\pgfqpoint{5.554529in}{3.004662in}}{\pgfqpoint{5.558647in}{3.000544in}}%
\pgfpathcurveto{\pgfqpoint{5.562765in}{2.996426in}}{\pgfqpoint{5.568351in}{2.994112in}}{\pgfqpoint{5.574175in}{2.994112in}}%
\pgfpathlineto{\pgfqpoint{5.574175in}{2.994112in}}%
\pgfpathclose%
\pgfusepath{stroke,fill}%
\end{pgfscope}%
\begin{pgfscope}%
\pgfpathrectangle{\pgfqpoint{1.000000in}{1.148311in}}{\pgfqpoint{6.200000in}{5.623377in}}%
\pgfusepath{clip}%
\pgfsetbuttcap%
\pgfsetroundjoin%
\definecolor{currentfill}{rgb}{0.800000,0.200000,0.200000}%
\pgfsetfillcolor{currentfill}%
\pgfsetlinewidth{1.003750pt}%
\definecolor{currentstroke}{rgb}{0.800000,0.200000,0.200000}%
\pgfsetstrokecolor{currentstroke}%
\pgfsetdash{}{0pt}%
\pgfpathmoveto{\pgfqpoint{5.645530in}{3.010055in}}%
\pgfpathcurveto{\pgfqpoint{5.651354in}{3.010055in}}{\pgfqpoint{5.656940in}{3.012369in}}{\pgfqpoint{5.661058in}{3.016487in}}%
\pgfpathcurveto{\pgfqpoint{5.665176in}{3.020605in}}{\pgfqpoint{5.667490in}{3.026191in}}{\pgfqpoint{5.667490in}{3.032015in}}%
\pgfpathcurveto{\pgfqpoint{5.667490in}{3.037839in}}{\pgfqpoint{5.665176in}{3.043425in}}{\pgfqpoint{5.661058in}{3.047543in}}%
\pgfpathcurveto{\pgfqpoint{5.656940in}{3.051661in}}{\pgfqpoint{5.651354in}{3.053975in}}{\pgfqpoint{5.645530in}{3.053975in}}%
\pgfpathcurveto{\pgfqpoint{5.639706in}{3.053975in}}{\pgfqpoint{5.634120in}{3.051661in}}{\pgfqpoint{5.630002in}{3.047543in}}%
\pgfpathcurveto{\pgfqpoint{5.625884in}{3.043425in}}{\pgfqpoint{5.623570in}{3.037839in}}{\pgfqpoint{5.623570in}{3.032015in}}%
\pgfpathcurveto{\pgfqpoint{5.623570in}{3.026191in}}{\pgfqpoint{5.625884in}{3.020605in}}{\pgfqpoint{5.630002in}{3.016487in}}%
\pgfpathcurveto{\pgfqpoint{5.634120in}{3.012369in}}{\pgfqpoint{5.639706in}{3.010055in}}{\pgfqpoint{5.645530in}{3.010055in}}%
\pgfpathlineto{\pgfqpoint{5.645530in}{3.010055in}}%
\pgfpathclose%
\pgfusepath{stroke,fill}%
\end{pgfscope}%
\begin{pgfscope}%
\pgfpathrectangle{\pgfqpoint{1.000000in}{1.148311in}}{\pgfqpoint{6.200000in}{5.623377in}}%
\pgfusepath{clip}%
\pgfsetbuttcap%
\pgfsetroundjoin%
\definecolor{currentfill}{rgb}{0.800000,0.200000,0.200000}%
\pgfsetfillcolor{currentfill}%
\pgfsetlinewidth{1.003750pt}%
\definecolor{currentstroke}{rgb}{0.800000,0.200000,0.200000}%
\pgfsetstrokecolor{currentstroke}%
\pgfsetdash{}{0pt}%
\pgfpathmoveto{\pgfqpoint{5.705873in}{2.928998in}}%
\pgfpathcurveto{\pgfqpoint{5.711697in}{2.928998in}}{\pgfqpoint{5.717283in}{2.931312in}}{\pgfqpoint{5.721401in}{2.935430in}}%
\pgfpathcurveto{\pgfqpoint{5.725519in}{2.939548in}}{\pgfqpoint{5.727833in}{2.945134in}}{\pgfqpoint{5.727833in}{2.950958in}}%
\pgfpathcurveto{\pgfqpoint{5.727833in}{2.956782in}}{\pgfqpoint{5.725519in}{2.962368in}}{\pgfqpoint{5.721401in}{2.966487in}}%
\pgfpathcurveto{\pgfqpoint{5.717283in}{2.970605in}}{\pgfqpoint{5.711697in}{2.972919in}}{\pgfqpoint{5.705873in}{2.972919in}}%
\pgfpathcurveto{\pgfqpoint{5.700049in}{2.972919in}}{\pgfqpoint{5.694463in}{2.970605in}}{\pgfqpoint{5.690345in}{2.966487in}}%
\pgfpathcurveto{\pgfqpoint{5.686227in}{2.962368in}}{\pgfqpoint{5.683913in}{2.956782in}}{\pgfqpoint{5.683913in}{2.950958in}}%
\pgfpathcurveto{\pgfqpoint{5.683913in}{2.945134in}}{\pgfqpoint{5.686227in}{2.939548in}}{\pgfqpoint{5.690345in}{2.935430in}}%
\pgfpathcurveto{\pgfqpoint{5.694463in}{2.931312in}}{\pgfqpoint{5.700049in}{2.928998in}}{\pgfqpoint{5.705873in}{2.928998in}}%
\pgfpathlineto{\pgfqpoint{5.705873in}{2.928998in}}%
\pgfpathclose%
\pgfusepath{stroke,fill}%
\end{pgfscope}%
\begin{pgfscope}%
\pgfpathrectangle{\pgfqpoint{1.000000in}{1.148311in}}{\pgfqpoint{6.200000in}{5.623377in}}%
\pgfusepath{clip}%
\pgfsetbuttcap%
\pgfsetroundjoin%
\definecolor{currentfill}{rgb}{0.800000,0.200000,0.200000}%
\pgfsetfillcolor{currentfill}%
\pgfsetlinewidth{1.003750pt}%
\definecolor{currentstroke}{rgb}{0.800000,0.200000,0.200000}%
\pgfsetstrokecolor{currentstroke}%
\pgfsetdash{}{0pt}%
\pgfpathmoveto{\pgfqpoint{5.778474in}{3.012663in}}%
\pgfpathcurveto{\pgfqpoint{5.784298in}{3.012663in}}{\pgfqpoint{5.789884in}{3.014977in}}{\pgfqpoint{5.794002in}{3.019095in}}%
\pgfpathcurveto{\pgfqpoint{5.798121in}{3.023213in}}{\pgfqpoint{5.800434in}{3.028799in}}{\pgfqpoint{5.800434in}{3.034623in}}%
\pgfpathcurveto{\pgfqpoint{5.800434in}{3.040447in}}{\pgfqpoint{5.798121in}{3.046033in}}{\pgfqpoint{5.794002in}{3.050152in}}%
\pgfpathcurveto{\pgfqpoint{5.789884in}{3.054270in}}{\pgfqpoint{5.784298in}{3.056584in}}{\pgfqpoint{5.778474in}{3.056584in}}%
\pgfpathcurveto{\pgfqpoint{5.772650in}{3.056584in}}{\pgfqpoint{5.767064in}{3.054270in}}{\pgfqpoint{5.762946in}{3.050152in}}%
\pgfpathcurveto{\pgfqpoint{5.758828in}{3.046033in}}{\pgfqpoint{5.756514in}{3.040447in}}{\pgfqpoint{5.756514in}{3.034623in}}%
\pgfpathcurveto{\pgfqpoint{5.756514in}{3.028799in}}{\pgfqpoint{5.758828in}{3.023213in}}{\pgfqpoint{5.762946in}{3.019095in}}%
\pgfpathcurveto{\pgfqpoint{5.767064in}{3.014977in}}{\pgfqpoint{5.772650in}{3.012663in}}{\pgfqpoint{5.778474in}{3.012663in}}%
\pgfpathlineto{\pgfqpoint{5.778474in}{3.012663in}}%
\pgfpathclose%
\pgfusepath{stroke,fill}%
\end{pgfscope}%
\begin{pgfscope}%
\pgfpathrectangle{\pgfqpoint{1.000000in}{1.148311in}}{\pgfqpoint{6.200000in}{5.623377in}}%
\pgfusepath{clip}%
\pgfsetbuttcap%
\pgfsetroundjoin%
\definecolor{currentfill}{rgb}{0.800000,0.200000,0.200000}%
\pgfsetfillcolor{currentfill}%
\pgfsetlinewidth{1.003750pt}%
\definecolor{currentstroke}{rgb}{0.800000,0.200000,0.200000}%
\pgfsetstrokecolor{currentstroke}%
\pgfsetdash{}{0pt}%
\pgfpathmoveto{\pgfqpoint{5.841807in}{3.064810in}}%
\pgfpathcurveto{\pgfqpoint{5.847631in}{3.064810in}}{\pgfqpoint{5.853217in}{3.067124in}}{\pgfqpoint{5.857335in}{3.071242in}}%
\pgfpathcurveto{\pgfqpoint{5.861453in}{3.075360in}}{\pgfqpoint{5.863767in}{3.080947in}}{\pgfqpoint{5.863767in}{3.086770in}}%
\pgfpathcurveto{\pgfqpoint{5.863767in}{3.092594in}}{\pgfqpoint{5.861453in}{3.098181in}}{\pgfqpoint{5.857335in}{3.102299in}}%
\pgfpathcurveto{\pgfqpoint{5.853217in}{3.106417in}}{\pgfqpoint{5.847631in}{3.108731in}}{\pgfqpoint{5.841807in}{3.108731in}}%
\pgfpathcurveto{\pgfqpoint{5.835983in}{3.108731in}}{\pgfqpoint{5.830397in}{3.106417in}}{\pgfqpoint{5.826279in}{3.102299in}}%
\pgfpathcurveto{\pgfqpoint{5.822161in}{3.098181in}}{\pgfqpoint{5.819847in}{3.092594in}}{\pgfqpoint{5.819847in}{3.086770in}}%
\pgfpathcurveto{\pgfqpoint{5.819847in}{3.080947in}}{\pgfqpoint{5.822161in}{3.075360in}}{\pgfqpoint{5.826279in}{3.071242in}}%
\pgfpathcurveto{\pgfqpoint{5.830397in}{3.067124in}}{\pgfqpoint{5.835983in}{3.064810in}}{\pgfqpoint{5.841807in}{3.064810in}}%
\pgfpathlineto{\pgfqpoint{5.841807in}{3.064810in}}%
\pgfpathclose%
\pgfusepath{stroke,fill}%
\end{pgfscope}%
\begin{pgfscope}%
\pgfpathrectangle{\pgfqpoint{1.000000in}{1.148311in}}{\pgfqpoint{6.200000in}{5.623377in}}%
\pgfusepath{clip}%
\pgfsetbuttcap%
\pgfsetroundjoin%
\definecolor{currentfill}{rgb}{0.800000,0.200000,0.200000}%
\pgfsetfillcolor{currentfill}%
\pgfsetlinewidth{1.003750pt}%
\definecolor{currentstroke}{rgb}{0.800000,0.200000,0.200000}%
\pgfsetstrokecolor{currentstroke}%
\pgfsetdash{}{0pt}%
\pgfpathmoveto{\pgfqpoint{5.920073in}{2.926893in}}%
\pgfpathcurveto{\pgfqpoint{5.925897in}{2.926893in}}{\pgfqpoint{5.931483in}{2.929207in}}{\pgfqpoint{5.935601in}{2.933325in}}%
\pgfpathcurveto{\pgfqpoint{5.939720in}{2.937443in}}{\pgfqpoint{5.942033in}{2.943029in}}{\pgfqpoint{5.942033in}{2.948853in}}%
\pgfpathcurveto{\pgfqpoint{5.942033in}{2.954677in}}{\pgfqpoint{5.939720in}{2.960263in}}{\pgfqpoint{5.935601in}{2.964381in}}%
\pgfpathcurveto{\pgfqpoint{5.931483in}{2.968500in}}{\pgfqpoint{5.925897in}{2.970813in}}{\pgfqpoint{5.920073in}{2.970813in}}%
\pgfpathcurveto{\pgfqpoint{5.914249in}{2.970813in}}{\pgfqpoint{5.908663in}{2.968500in}}{\pgfqpoint{5.904545in}{2.964381in}}%
\pgfpathcurveto{\pgfqpoint{5.900427in}{2.960263in}}{\pgfqpoint{5.898113in}{2.954677in}}{\pgfqpoint{5.898113in}{2.948853in}}%
\pgfpathcurveto{\pgfqpoint{5.898113in}{2.943029in}}{\pgfqpoint{5.900427in}{2.937443in}}{\pgfqpoint{5.904545in}{2.933325in}}%
\pgfpathcurveto{\pgfqpoint{5.908663in}{2.929207in}}{\pgfqpoint{5.914249in}{2.926893in}}{\pgfqpoint{5.920073in}{2.926893in}}%
\pgfpathlineto{\pgfqpoint{5.920073in}{2.926893in}}%
\pgfpathclose%
\pgfusepath{stroke,fill}%
\end{pgfscope}%
\begin{pgfscope}%
\pgfpathrectangle{\pgfqpoint{1.000000in}{1.148311in}}{\pgfqpoint{6.200000in}{5.623377in}}%
\pgfusepath{clip}%
\pgfsetbuttcap%
\pgfsetroundjoin%
\definecolor{currentfill}{rgb}{0.800000,0.200000,0.200000}%
\pgfsetfillcolor{currentfill}%
\pgfsetlinewidth{1.003750pt}%
\definecolor{currentstroke}{rgb}{0.800000,0.200000,0.200000}%
\pgfsetstrokecolor{currentstroke}%
\pgfsetdash{}{0pt}%
\pgfpathmoveto{\pgfqpoint{5.984870in}{2.971825in}}%
\pgfpathcurveto{\pgfqpoint{5.990694in}{2.971825in}}{\pgfqpoint{5.996281in}{2.974139in}}{\pgfqpoint{6.000399in}{2.978257in}}%
\pgfpathcurveto{\pgfqpoint{6.004517in}{2.982375in}}{\pgfqpoint{6.006831in}{2.987961in}}{\pgfqpoint{6.006831in}{2.993785in}}%
\pgfpathcurveto{\pgfqpoint{6.006831in}{2.999609in}}{\pgfqpoint{6.004517in}{3.005195in}}{\pgfqpoint{6.000399in}{3.009313in}}%
\pgfpathcurveto{\pgfqpoint{5.996281in}{3.013431in}}{\pgfqpoint{5.990694in}{3.015745in}}{\pgfqpoint{5.984870in}{3.015745in}}%
\pgfpathcurveto{\pgfqpoint{5.979047in}{3.015745in}}{\pgfqpoint{5.973460in}{3.013431in}}{\pgfqpoint{5.969342in}{3.009313in}}%
\pgfpathcurveto{\pgfqpoint{5.965224in}{3.005195in}}{\pgfqpoint{5.962910in}{2.999609in}}{\pgfqpoint{5.962910in}{2.993785in}}%
\pgfpathcurveto{\pgfqpoint{5.962910in}{2.987961in}}{\pgfqpoint{5.965224in}{2.982375in}}{\pgfqpoint{5.969342in}{2.978257in}}%
\pgfpathcurveto{\pgfqpoint{5.973460in}{2.974139in}}{\pgfqpoint{5.979047in}{2.971825in}}{\pgfqpoint{5.984870in}{2.971825in}}%
\pgfpathlineto{\pgfqpoint{5.984870in}{2.971825in}}%
\pgfpathclose%
\pgfusepath{stroke,fill}%
\end{pgfscope}%
\begin{pgfscope}%
\pgfpathrectangle{\pgfqpoint{1.000000in}{1.148311in}}{\pgfqpoint{6.200000in}{5.623377in}}%
\pgfusepath{clip}%
\pgfsetbuttcap%
\pgfsetroundjoin%
\definecolor{currentfill}{rgb}{0.800000,0.200000,0.200000}%
\pgfsetfillcolor{currentfill}%
\pgfsetlinewidth{1.003750pt}%
\definecolor{currentstroke}{rgb}{0.800000,0.200000,0.200000}%
\pgfsetstrokecolor{currentstroke}%
\pgfsetdash{}{0pt}%
\pgfpathmoveto{\pgfqpoint{6.049219in}{3.000853in}}%
\pgfpathcurveto{\pgfqpoint{6.055043in}{3.000853in}}{\pgfqpoint{6.060629in}{3.003167in}}{\pgfqpoint{6.064747in}{3.007285in}}%
\pgfpathcurveto{\pgfqpoint{6.068865in}{3.011403in}}{\pgfqpoint{6.071179in}{3.016989in}}{\pgfqpoint{6.071179in}{3.022813in}}%
\pgfpathcurveto{\pgfqpoint{6.071179in}{3.028637in}}{\pgfqpoint{6.068865in}{3.034223in}}{\pgfqpoint{6.064747in}{3.038341in}}%
\pgfpathcurveto{\pgfqpoint{6.060629in}{3.042459in}}{\pgfqpoint{6.055043in}{3.044773in}}{\pgfqpoint{6.049219in}{3.044773in}}%
\pgfpathcurveto{\pgfqpoint{6.043395in}{3.044773in}}{\pgfqpoint{6.037809in}{3.042459in}}{\pgfqpoint{6.033691in}{3.038341in}}%
\pgfpathcurveto{\pgfqpoint{6.029573in}{3.034223in}}{\pgfqpoint{6.027259in}{3.028637in}}{\pgfqpoint{6.027259in}{3.022813in}}%
\pgfpathcurveto{\pgfqpoint{6.027259in}{3.016989in}}{\pgfqpoint{6.029573in}{3.011403in}}{\pgfqpoint{6.033691in}{3.007285in}}%
\pgfpathcurveto{\pgfqpoint{6.037809in}{3.003167in}}{\pgfqpoint{6.043395in}{3.000853in}}{\pgfqpoint{6.049219in}{3.000853in}}%
\pgfpathlineto{\pgfqpoint{6.049219in}{3.000853in}}%
\pgfpathclose%
\pgfusepath{stroke,fill}%
\end{pgfscope}%
\begin{pgfscope}%
\pgfpathrectangle{\pgfqpoint{1.000000in}{1.148311in}}{\pgfqpoint{6.200000in}{5.623377in}}%
\pgfusepath{clip}%
\pgfsetbuttcap%
\pgfsetroundjoin%
\definecolor{currentfill}{rgb}{0.800000,0.200000,0.200000}%
\pgfsetfillcolor{currentfill}%
\pgfsetlinewidth{1.003750pt}%
\definecolor{currentstroke}{rgb}{0.800000,0.200000,0.200000}%
\pgfsetstrokecolor{currentstroke}%
\pgfsetdash{}{0pt}%
\pgfpathmoveto{\pgfqpoint{6.103843in}{3.055583in}}%
\pgfpathcurveto{\pgfqpoint{6.109667in}{3.055583in}}{\pgfqpoint{6.115253in}{3.057897in}}{\pgfqpoint{6.119372in}{3.062015in}}%
\pgfpathcurveto{\pgfqpoint{6.123490in}{3.066133in}}{\pgfqpoint{6.125804in}{3.071720in}}{\pgfqpoint{6.125804in}{3.077544in}}%
\pgfpathcurveto{\pgfqpoint{6.125804in}{3.083368in}}{\pgfqpoint{6.123490in}{3.088954in}}{\pgfqpoint{6.119372in}{3.093072in}}%
\pgfpathcurveto{\pgfqpoint{6.115253in}{3.097190in}}{\pgfqpoint{6.109667in}{3.099504in}}{\pgfqpoint{6.103843in}{3.099504in}}%
\pgfpathcurveto{\pgfqpoint{6.098019in}{3.099504in}}{\pgfqpoint{6.092433in}{3.097190in}}{\pgfqpoint{6.088315in}{3.093072in}}%
\pgfpathcurveto{\pgfqpoint{6.084197in}{3.088954in}}{\pgfqpoint{6.081883in}{3.083368in}}{\pgfqpoint{6.081883in}{3.077544in}}%
\pgfpathcurveto{\pgfqpoint{6.081883in}{3.071720in}}{\pgfqpoint{6.084197in}{3.066133in}}{\pgfqpoint{6.088315in}{3.062015in}}%
\pgfpathcurveto{\pgfqpoint{6.092433in}{3.057897in}}{\pgfqpoint{6.098019in}{3.055583in}}{\pgfqpoint{6.103843in}{3.055583in}}%
\pgfpathlineto{\pgfqpoint{6.103843in}{3.055583in}}%
\pgfpathclose%
\pgfusepath{stroke,fill}%
\end{pgfscope}%
\begin{pgfscope}%
\pgfpathrectangle{\pgfqpoint{1.000000in}{1.148311in}}{\pgfqpoint{6.200000in}{5.623377in}}%
\pgfusepath{clip}%
\pgfsetbuttcap%
\pgfsetroundjoin%
\definecolor{currentfill}{rgb}{0.800000,0.200000,0.200000}%
\pgfsetfillcolor{currentfill}%
\pgfsetlinewidth{1.003750pt}%
\definecolor{currentstroke}{rgb}{0.800000,0.200000,0.200000}%
\pgfsetstrokecolor{currentstroke}%
\pgfsetdash{}{0pt}%
\pgfpathmoveto{\pgfqpoint{6.175861in}{3.051977in}}%
\pgfpathcurveto{\pgfqpoint{6.181684in}{3.051977in}}{\pgfqpoint{6.187271in}{3.054291in}}{\pgfqpoint{6.191389in}{3.058409in}}%
\pgfpathcurveto{\pgfqpoint{6.195507in}{3.062527in}}{\pgfqpoint{6.197821in}{3.068113in}}{\pgfqpoint{6.197821in}{3.073937in}}%
\pgfpathcurveto{\pgfqpoint{6.197821in}{3.079761in}}{\pgfqpoint{6.195507in}{3.085348in}}{\pgfqpoint{6.191389in}{3.089466in}}%
\pgfpathcurveto{\pgfqpoint{6.187271in}{3.093584in}}{\pgfqpoint{6.181684in}{3.095898in}}{\pgfqpoint{6.175861in}{3.095898in}}%
\pgfpathcurveto{\pgfqpoint{6.170037in}{3.095898in}}{\pgfqpoint{6.164450in}{3.093584in}}{\pgfqpoint{6.160332in}{3.089466in}}%
\pgfpathcurveto{\pgfqpoint{6.156214in}{3.085348in}}{\pgfqpoint{6.153900in}{3.079761in}}{\pgfqpoint{6.153900in}{3.073937in}}%
\pgfpathcurveto{\pgfqpoint{6.153900in}{3.068113in}}{\pgfqpoint{6.156214in}{3.062527in}}{\pgfqpoint{6.160332in}{3.058409in}}%
\pgfpathcurveto{\pgfqpoint{6.164450in}{3.054291in}}{\pgfqpoint{6.170037in}{3.051977in}}{\pgfqpoint{6.175861in}{3.051977in}}%
\pgfpathlineto{\pgfqpoint{6.175861in}{3.051977in}}%
\pgfpathclose%
\pgfusepath{stroke,fill}%
\end{pgfscope}%
\begin{pgfscope}%
\pgfpathrectangle{\pgfqpoint{1.000000in}{1.148311in}}{\pgfqpoint{6.200000in}{5.623377in}}%
\pgfusepath{clip}%
\pgfsetbuttcap%
\pgfsetroundjoin%
\definecolor{currentfill}{rgb}{0.800000,0.200000,0.200000}%
\pgfsetfillcolor{currentfill}%
\pgfsetlinewidth{1.003750pt}%
\definecolor{currentstroke}{rgb}{0.800000,0.200000,0.200000}%
\pgfsetstrokecolor{currentstroke}%
\pgfsetdash{}{0pt}%
\pgfpathmoveto{\pgfqpoint{6.211985in}{3.135841in}}%
\pgfpathcurveto{\pgfqpoint{6.217809in}{3.135841in}}{\pgfqpoint{6.223396in}{3.138155in}}{\pgfqpoint{6.227514in}{3.142273in}}%
\pgfpathcurveto{\pgfqpoint{6.231632in}{3.146391in}}{\pgfqpoint{6.233946in}{3.151977in}}{\pgfqpoint{6.233946in}{3.157801in}}%
\pgfpathcurveto{\pgfqpoint{6.233946in}{3.163625in}}{\pgfqpoint{6.231632in}{3.169211in}}{\pgfqpoint{6.227514in}{3.173330in}}%
\pgfpathcurveto{\pgfqpoint{6.223396in}{3.177448in}}{\pgfqpoint{6.217809in}{3.179762in}}{\pgfqpoint{6.211985in}{3.179762in}}%
\pgfpathcurveto{\pgfqpoint{6.206162in}{3.179762in}}{\pgfqpoint{6.200575in}{3.177448in}}{\pgfqpoint{6.196457in}{3.173330in}}%
\pgfpathcurveto{\pgfqpoint{6.192339in}{3.169211in}}{\pgfqpoint{6.190025in}{3.163625in}}{\pgfqpoint{6.190025in}{3.157801in}}%
\pgfpathcurveto{\pgfqpoint{6.190025in}{3.151977in}}{\pgfqpoint{6.192339in}{3.146391in}}{\pgfqpoint{6.196457in}{3.142273in}}%
\pgfpathcurveto{\pgfqpoint{6.200575in}{3.138155in}}{\pgfqpoint{6.206162in}{3.135841in}}{\pgfqpoint{6.211985in}{3.135841in}}%
\pgfpathlineto{\pgfqpoint{6.211985in}{3.135841in}}%
\pgfpathclose%
\pgfusepath{stroke,fill}%
\end{pgfscope}%
\begin{pgfscope}%
\pgfpathrectangle{\pgfqpoint{1.000000in}{1.148311in}}{\pgfqpoint{6.200000in}{5.623377in}}%
\pgfusepath{clip}%
\pgfsetbuttcap%
\pgfsetroundjoin%
\definecolor{currentfill}{rgb}{0.800000,0.200000,0.200000}%
\pgfsetfillcolor{currentfill}%
\pgfsetlinewidth{1.003750pt}%
\definecolor{currentstroke}{rgb}{0.800000,0.200000,0.200000}%
\pgfsetstrokecolor{currentstroke}%
\pgfsetdash{}{0pt}%
\pgfpathmoveto{\pgfqpoint{6.288419in}{3.128013in}}%
\pgfpathcurveto{\pgfqpoint{6.294243in}{3.128013in}}{\pgfqpoint{6.299829in}{3.130327in}}{\pgfqpoint{6.303947in}{3.134445in}}%
\pgfpathcurveto{\pgfqpoint{6.308065in}{3.138564in}}{\pgfqpoint{6.310379in}{3.144150in}}{\pgfqpoint{6.310379in}{3.149974in}}%
\pgfpathcurveto{\pgfqpoint{6.310379in}{3.155798in}}{\pgfqpoint{6.308065in}{3.161384in}}{\pgfqpoint{6.303947in}{3.165502in}}%
\pgfpathcurveto{\pgfqpoint{6.299829in}{3.169620in}}{\pgfqpoint{6.294243in}{3.171934in}}{\pgfqpoint{6.288419in}{3.171934in}}%
\pgfpathcurveto{\pgfqpoint{6.282595in}{3.171934in}}{\pgfqpoint{6.277009in}{3.169620in}}{\pgfqpoint{6.272891in}{3.165502in}}%
\pgfpathcurveto{\pgfqpoint{6.268773in}{3.161384in}}{\pgfqpoint{6.266459in}{3.155798in}}{\pgfqpoint{6.266459in}{3.149974in}}%
\pgfpathcurveto{\pgfqpoint{6.266459in}{3.144150in}}{\pgfqpoint{6.268773in}{3.138564in}}{\pgfqpoint{6.272891in}{3.134445in}}%
\pgfpathcurveto{\pgfqpoint{6.277009in}{3.130327in}}{\pgfqpoint{6.282595in}{3.128013in}}{\pgfqpoint{6.288419in}{3.128013in}}%
\pgfpathlineto{\pgfqpoint{6.288419in}{3.128013in}}%
\pgfpathclose%
\pgfusepath{stroke,fill}%
\end{pgfscope}%
\begin{pgfscope}%
\pgfpathrectangle{\pgfqpoint{1.000000in}{1.148311in}}{\pgfqpoint{6.200000in}{5.623377in}}%
\pgfusepath{clip}%
\pgfsetbuttcap%
\pgfsetroundjoin%
\definecolor{currentfill}{rgb}{0.800000,0.200000,0.200000}%
\pgfsetfillcolor{currentfill}%
\pgfsetlinewidth{1.003750pt}%
\definecolor{currentstroke}{rgb}{0.800000,0.200000,0.200000}%
\pgfsetstrokecolor{currentstroke}%
\pgfsetdash{}{0pt}%
\pgfpathmoveto{\pgfqpoint{6.331683in}{3.183989in}}%
\pgfpathcurveto{\pgfqpoint{6.337507in}{3.183989in}}{\pgfqpoint{6.343093in}{3.186303in}}{\pgfqpoint{6.347212in}{3.190421in}}%
\pgfpathcurveto{\pgfqpoint{6.351330in}{3.194539in}}{\pgfqpoint{6.353644in}{3.200126in}}{\pgfqpoint{6.353644in}{3.205950in}}%
\pgfpathcurveto{\pgfqpoint{6.353644in}{3.211774in}}{\pgfqpoint{6.351330in}{3.217360in}}{\pgfqpoint{6.347212in}{3.221478in}}%
\pgfpathcurveto{\pgfqpoint{6.343093in}{3.225596in}}{\pgfqpoint{6.337507in}{3.227910in}}{\pgfqpoint{6.331683in}{3.227910in}}%
\pgfpathcurveto{\pgfqpoint{6.325859in}{3.227910in}}{\pgfqpoint{6.320273in}{3.225596in}}{\pgfqpoint{6.316155in}{3.221478in}}%
\pgfpathcurveto{\pgfqpoint{6.312037in}{3.217360in}}{\pgfqpoint{6.309723in}{3.211774in}}{\pgfqpoint{6.309723in}{3.205950in}}%
\pgfpathcurveto{\pgfqpoint{6.309723in}{3.200126in}}{\pgfqpoint{6.312037in}{3.194539in}}{\pgfqpoint{6.316155in}{3.190421in}}%
\pgfpathcurveto{\pgfqpoint{6.320273in}{3.186303in}}{\pgfqpoint{6.325859in}{3.183989in}}{\pgfqpoint{6.331683in}{3.183989in}}%
\pgfpathlineto{\pgfqpoint{6.331683in}{3.183989in}}%
\pgfpathclose%
\pgfusepath{stroke,fill}%
\end{pgfscope}%
\begin{pgfscope}%
\pgfpathrectangle{\pgfqpoint{1.000000in}{1.148311in}}{\pgfqpoint{6.200000in}{5.623377in}}%
\pgfusepath{clip}%
\pgfsetbuttcap%
\pgfsetroundjoin%
\definecolor{currentfill}{rgb}{0.800000,0.200000,0.200000}%
\pgfsetfillcolor{currentfill}%
\pgfsetlinewidth{1.003750pt}%
\definecolor{currentstroke}{rgb}{0.800000,0.200000,0.200000}%
\pgfsetstrokecolor{currentstroke}%
\pgfsetdash{}{0pt}%
\pgfpathmoveto{\pgfqpoint{6.412231in}{3.182194in}}%
\pgfpathcurveto{\pgfqpoint{6.418055in}{3.182194in}}{\pgfqpoint{6.423641in}{3.184508in}}{\pgfqpoint{6.427760in}{3.188626in}}%
\pgfpathcurveto{\pgfqpoint{6.431878in}{3.192744in}}{\pgfqpoint{6.434192in}{3.198331in}}{\pgfqpoint{6.434192in}{3.204155in}}%
\pgfpathcurveto{\pgfqpoint{6.434192in}{3.209979in}}{\pgfqpoint{6.431878in}{3.215565in}}{\pgfqpoint{6.427760in}{3.219683in}}%
\pgfpathcurveto{\pgfqpoint{6.423641in}{3.223801in}}{\pgfqpoint{6.418055in}{3.226115in}}{\pgfqpoint{6.412231in}{3.226115in}}%
\pgfpathcurveto{\pgfqpoint{6.406407in}{3.226115in}}{\pgfqpoint{6.400821in}{3.223801in}}{\pgfqpoint{6.396703in}{3.219683in}}%
\pgfpathcurveto{\pgfqpoint{6.392585in}{3.215565in}}{\pgfqpoint{6.390271in}{3.209979in}}{\pgfqpoint{6.390271in}{3.204155in}}%
\pgfpathcurveto{\pgfqpoint{6.390271in}{3.198331in}}{\pgfqpoint{6.392585in}{3.192744in}}{\pgfqpoint{6.396703in}{3.188626in}}%
\pgfpathcurveto{\pgfqpoint{6.400821in}{3.184508in}}{\pgfqpoint{6.406407in}{3.182194in}}{\pgfqpoint{6.412231in}{3.182194in}}%
\pgfpathlineto{\pgfqpoint{6.412231in}{3.182194in}}%
\pgfpathclose%
\pgfusepath{stroke,fill}%
\end{pgfscope}%
\begin{pgfscope}%
\pgfpathrectangle{\pgfqpoint{1.000000in}{1.148311in}}{\pgfqpoint{6.200000in}{5.623377in}}%
\pgfusepath{clip}%
\pgfsetbuttcap%
\pgfsetroundjoin%
\definecolor{currentfill}{rgb}{0.800000,0.200000,0.200000}%
\pgfsetfillcolor{currentfill}%
\pgfsetlinewidth{1.003750pt}%
\definecolor{currentstroke}{rgb}{0.800000,0.200000,0.200000}%
\pgfsetstrokecolor{currentstroke}%
\pgfsetdash{}{0pt}%
\pgfpathmoveto{\pgfqpoint{6.482482in}{3.202768in}}%
\pgfpathcurveto{\pgfqpoint{6.488306in}{3.202768in}}{\pgfqpoint{6.493892in}{3.205081in}}{\pgfqpoint{6.498010in}{3.209200in}}%
\pgfpathcurveto{\pgfqpoint{6.502128in}{3.213318in}}{\pgfqpoint{6.504442in}{3.218904in}}{\pgfqpoint{6.504442in}{3.224728in}}%
\pgfpathcurveto{\pgfqpoint{6.504442in}{3.230552in}}{\pgfqpoint{6.502128in}{3.236138in}}{\pgfqpoint{6.498010in}{3.240256in}}%
\pgfpathcurveto{\pgfqpoint{6.493892in}{3.244374in}}{\pgfqpoint{6.488306in}{3.246688in}}{\pgfqpoint{6.482482in}{3.246688in}}%
\pgfpathcurveto{\pgfqpoint{6.476658in}{3.246688in}}{\pgfqpoint{6.471072in}{3.244374in}}{\pgfqpoint{6.466954in}{3.240256in}}%
\pgfpathcurveto{\pgfqpoint{6.462836in}{3.236138in}}{\pgfqpoint{6.460522in}{3.230552in}}{\pgfqpoint{6.460522in}{3.224728in}}%
\pgfpathcurveto{\pgfqpoint{6.460522in}{3.218904in}}{\pgfqpoint{6.462836in}{3.213318in}}{\pgfqpoint{6.466954in}{3.209200in}}%
\pgfpathcurveto{\pgfqpoint{6.471072in}{3.205081in}}{\pgfqpoint{6.476658in}{3.202768in}}{\pgfqpoint{6.482482in}{3.202768in}}%
\pgfpathlineto{\pgfqpoint{6.482482in}{3.202768in}}%
\pgfpathclose%
\pgfusepath{stroke,fill}%
\end{pgfscope}%
\begin{pgfscope}%
\pgfpathrectangle{\pgfqpoint{1.000000in}{1.148311in}}{\pgfqpoint{6.200000in}{5.623377in}}%
\pgfusepath{clip}%
\pgfsetbuttcap%
\pgfsetroundjoin%
\definecolor{currentfill}{rgb}{0.800000,0.200000,0.200000}%
\pgfsetfillcolor{currentfill}%
\pgfsetlinewidth{1.003750pt}%
\definecolor{currentstroke}{rgb}{0.800000,0.200000,0.200000}%
\pgfsetstrokecolor{currentstroke}%
\pgfsetdash{}{0pt}%
\pgfpathmoveto{\pgfqpoint{6.481839in}{3.305391in}}%
\pgfpathcurveto{\pgfqpoint{6.487663in}{3.305391in}}{\pgfqpoint{6.493249in}{3.307705in}}{\pgfqpoint{6.497367in}{3.311823in}}%
\pgfpathcurveto{\pgfqpoint{6.501485in}{3.315941in}}{\pgfqpoint{6.503799in}{3.321527in}}{\pgfqpoint{6.503799in}{3.327351in}}%
\pgfpathcurveto{\pgfqpoint{6.503799in}{3.333175in}}{\pgfqpoint{6.501485in}{3.338761in}}{\pgfqpoint{6.497367in}{3.342879in}}%
\pgfpathcurveto{\pgfqpoint{6.493249in}{3.346997in}}{\pgfqpoint{6.487663in}{3.349311in}}{\pgfqpoint{6.481839in}{3.349311in}}%
\pgfpathcurveto{\pgfqpoint{6.476015in}{3.349311in}}{\pgfqpoint{6.470429in}{3.346997in}}{\pgfqpoint{6.466310in}{3.342879in}}%
\pgfpathcurveto{\pgfqpoint{6.462192in}{3.338761in}}{\pgfqpoint{6.459878in}{3.333175in}}{\pgfqpoint{6.459878in}{3.327351in}}%
\pgfpathcurveto{\pgfqpoint{6.459878in}{3.321527in}}{\pgfqpoint{6.462192in}{3.315941in}}{\pgfqpoint{6.466310in}{3.311823in}}%
\pgfpathcurveto{\pgfqpoint{6.470429in}{3.307705in}}{\pgfqpoint{6.476015in}{3.305391in}}{\pgfqpoint{6.481839in}{3.305391in}}%
\pgfpathlineto{\pgfqpoint{6.481839in}{3.305391in}}%
\pgfpathclose%
\pgfusepath{stroke,fill}%
\end{pgfscope}%
\begin{pgfscope}%
\pgfpathrectangle{\pgfqpoint{1.000000in}{1.148311in}}{\pgfqpoint{6.200000in}{5.623377in}}%
\pgfusepath{clip}%
\pgfsetbuttcap%
\pgfsetroundjoin%
\definecolor{currentfill}{rgb}{0.800000,0.200000,0.200000}%
\pgfsetfillcolor{currentfill}%
\pgfsetlinewidth{1.003750pt}%
\definecolor{currentstroke}{rgb}{0.800000,0.200000,0.200000}%
\pgfsetstrokecolor{currentstroke}%
\pgfsetdash{}{0pt}%
\pgfpathmoveto{\pgfqpoint{6.578400in}{3.302040in}}%
\pgfpathcurveto{\pgfqpoint{6.584224in}{3.302040in}}{\pgfqpoint{6.589810in}{3.304354in}}{\pgfqpoint{6.593928in}{3.308472in}}%
\pgfpathcurveto{\pgfqpoint{6.598047in}{3.312590in}}{\pgfqpoint{6.600360in}{3.318176in}}{\pgfqpoint{6.600360in}{3.324000in}}%
\pgfpathcurveto{\pgfqpoint{6.600360in}{3.329824in}}{\pgfqpoint{6.598047in}{3.335410in}}{\pgfqpoint{6.593928in}{3.339528in}}%
\pgfpathcurveto{\pgfqpoint{6.589810in}{3.343647in}}{\pgfqpoint{6.584224in}{3.345960in}}{\pgfqpoint{6.578400in}{3.345960in}}%
\pgfpathcurveto{\pgfqpoint{6.572576in}{3.345960in}}{\pgfqpoint{6.566990in}{3.343647in}}{\pgfqpoint{6.562872in}{3.339528in}}%
\pgfpathcurveto{\pgfqpoint{6.558754in}{3.335410in}}{\pgfqpoint{6.556440in}{3.329824in}}{\pgfqpoint{6.556440in}{3.324000in}}%
\pgfpathcurveto{\pgfqpoint{6.556440in}{3.318176in}}{\pgfqpoint{6.558754in}{3.312590in}}{\pgfqpoint{6.562872in}{3.308472in}}%
\pgfpathcurveto{\pgfqpoint{6.566990in}{3.304354in}}{\pgfqpoint{6.572576in}{3.302040in}}{\pgfqpoint{6.578400in}{3.302040in}}%
\pgfpathlineto{\pgfqpoint{6.578400in}{3.302040in}}%
\pgfpathclose%
\pgfusepath{stroke,fill}%
\end{pgfscope}%
\begin{pgfscope}%
\pgfpathrectangle{\pgfqpoint{1.000000in}{1.148311in}}{\pgfqpoint{6.200000in}{5.623377in}}%
\pgfusepath{clip}%
\pgfsetbuttcap%
\pgfsetroundjoin%
\definecolor{currentfill}{rgb}{0.800000,0.200000,0.200000}%
\pgfsetfillcolor{currentfill}%
\pgfsetlinewidth{1.003750pt}%
\definecolor{currentstroke}{rgb}{0.800000,0.200000,0.200000}%
\pgfsetstrokecolor{currentstroke}%
\pgfsetdash{}{0pt}%
\pgfpathmoveto{\pgfqpoint{6.608280in}{3.366597in}}%
\pgfpathcurveto{\pgfqpoint{6.614104in}{3.366597in}}{\pgfqpoint{6.619690in}{3.368910in}}{\pgfqpoint{6.623808in}{3.373029in}}%
\pgfpathcurveto{\pgfqpoint{6.627926in}{3.377147in}}{\pgfqpoint{6.630240in}{3.382733in}}{\pgfqpoint{6.630240in}{3.388557in}}%
\pgfpathcurveto{\pgfqpoint{6.630240in}{3.394381in}}{\pgfqpoint{6.627926in}{3.399967in}}{\pgfqpoint{6.623808in}{3.404085in}}%
\pgfpathcurveto{\pgfqpoint{6.619690in}{3.408203in}}{\pgfqpoint{6.614104in}{3.410517in}}{\pgfqpoint{6.608280in}{3.410517in}}%
\pgfpathcurveto{\pgfqpoint{6.602456in}{3.410517in}}{\pgfqpoint{6.596870in}{3.408203in}}{\pgfqpoint{6.592752in}{3.404085in}}%
\pgfpathcurveto{\pgfqpoint{6.588634in}{3.399967in}}{\pgfqpoint{6.586320in}{3.394381in}}{\pgfqpoint{6.586320in}{3.388557in}}%
\pgfpathcurveto{\pgfqpoint{6.586320in}{3.382733in}}{\pgfqpoint{6.588634in}{3.377147in}}{\pgfqpoint{6.592752in}{3.373029in}}%
\pgfpathcurveto{\pgfqpoint{6.596870in}{3.368910in}}{\pgfqpoint{6.602456in}{3.366597in}}{\pgfqpoint{6.608280in}{3.366597in}}%
\pgfpathlineto{\pgfqpoint{6.608280in}{3.366597in}}%
\pgfpathclose%
\pgfusepath{stroke,fill}%
\end{pgfscope}%
\begin{pgfscope}%
\pgfpathrectangle{\pgfqpoint{1.000000in}{1.148311in}}{\pgfqpoint{6.200000in}{5.623377in}}%
\pgfusepath{clip}%
\pgfsetbuttcap%
\pgfsetroundjoin%
\definecolor{currentfill}{rgb}{0.800000,0.200000,0.200000}%
\pgfsetfillcolor{currentfill}%
\pgfsetlinewidth{1.003750pt}%
\definecolor{currentstroke}{rgb}{0.800000,0.200000,0.200000}%
\pgfsetstrokecolor{currentstroke}%
\pgfsetdash{}{0pt}%
\pgfpathmoveto{\pgfqpoint{6.665359in}{3.408196in}}%
\pgfpathcurveto{\pgfqpoint{6.671183in}{3.408196in}}{\pgfqpoint{6.676769in}{3.410510in}}{\pgfqpoint{6.680887in}{3.414628in}}%
\pgfpathcurveto{\pgfqpoint{6.685005in}{3.418746in}}{\pgfqpoint{6.687319in}{3.424333in}}{\pgfqpoint{6.687319in}{3.430157in}}%
\pgfpathcurveto{\pgfqpoint{6.687319in}{3.435980in}}{\pgfqpoint{6.685005in}{3.441567in}}{\pgfqpoint{6.680887in}{3.445685in}}%
\pgfpathcurveto{\pgfqpoint{6.676769in}{3.449803in}}{\pgfqpoint{6.671183in}{3.452117in}}{\pgfqpoint{6.665359in}{3.452117in}}%
\pgfpathcurveto{\pgfqpoint{6.659535in}{3.452117in}}{\pgfqpoint{6.653949in}{3.449803in}}{\pgfqpoint{6.649830in}{3.445685in}}%
\pgfpathcurveto{\pgfqpoint{6.645712in}{3.441567in}}{\pgfqpoint{6.643398in}{3.435980in}}{\pgfqpoint{6.643398in}{3.430157in}}%
\pgfpathcurveto{\pgfqpoint{6.643398in}{3.424333in}}{\pgfqpoint{6.645712in}{3.418746in}}{\pgfqpoint{6.649830in}{3.414628in}}%
\pgfpathcurveto{\pgfqpoint{6.653949in}{3.410510in}}{\pgfqpoint{6.659535in}{3.408196in}}{\pgfqpoint{6.665359in}{3.408196in}}%
\pgfpathlineto{\pgfqpoint{6.665359in}{3.408196in}}%
\pgfpathclose%
\pgfusepath{stroke,fill}%
\end{pgfscope}%
\begin{pgfscope}%
\pgfpathrectangle{\pgfqpoint{1.000000in}{1.148311in}}{\pgfqpoint{6.200000in}{5.623377in}}%
\pgfusepath{clip}%
\pgfsetbuttcap%
\pgfsetroundjoin%
\definecolor{currentfill}{rgb}{0.800000,0.200000,0.200000}%
\pgfsetfillcolor{currentfill}%
\pgfsetlinewidth{1.003750pt}%
\definecolor{currentstroke}{rgb}{0.800000,0.200000,0.200000}%
\pgfsetstrokecolor{currentstroke}%
\pgfsetdash{}{0pt}%
\pgfpathmoveto{\pgfqpoint{6.673418in}{3.484510in}}%
\pgfpathcurveto{\pgfqpoint{6.679242in}{3.484510in}}{\pgfqpoint{6.684828in}{3.486824in}}{\pgfqpoint{6.688946in}{3.490942in}}%
\pgfpathcurveto{\pgfqpoint{6.693064in}{3.495061in}}{\pgfqpoint{6.695378in}{3.500647in}}{\pgfqpoint{6.695378in}{3.506471in}}%
\pgfpathcurveto{\pgfqpoint{6.695378in}{3.512295in}}{\pgfqpoint{6.693064in}{3.517881in}}{\pgfqpoint{6.688946in}{3.521999in}}%
\pgfpathcurveto{\pgfqpoint{6.684828in}{3.526117in}}{\pgfqpoint{6.679242in}{3.528431in}}{\pgfqpoint{6.673418in}{3.528431in}}%
\pgfpathcurveto{\pgfqpoint{6.667594in}{3.528431in}}{\pgfqpoint{6.662008in}{3.526117in}}{\pgfqpoint{6.657890in}{3.521999in}}%
\pgfpathcurveto{\pgfqpoint{6.653771in}{3.517881in}}{\pgfqpoint{6.651458in}{3.512295in}}{\pgfqpoint{6.651458in}{3.506471in}}%
\pgfpathcurveto{\pgfqpoint{6.651458in}{3.500647in}}{\pgfqpoint{6.653771in}{3.495061in}}{\pgfqpoint{6.657890in}{3.490942in}}%
\pgfpathcurveto{\pgfqpoint{6.662008in}{3.486824in}}{\pgfqpoint{6.667594in}{3.484510in}}{\pgfqpoint{6.673418in}{3.484510in}}%
\pgfpathlineto{\pgfqpoint{6.673418in}{3.484510in}}%
\pgfpathclose%
\pgfusepath{stroke,fill}%
\end{pgfscope}%
\begin{pgfscope}%
\pgfpathrectangle{\pgfqpoint{1.000000in}{1.148311in}}{\pgfqpoint{6.200000in}{5.623377in}}%
\pgfusepath{clip}%
\pgfsetbuttcap%
\pgfsetroundjoin%
\definecolor{currentfill}{rgb}{0.800000,0.200000,0.200000}%
\pgfsetfillcolor{currentfill}%
\pgfsetlinewidth{1.003750pt}%
\definecolor{currentstroke}{rgb}{0.800000,0.200000,0.200000}%
\pgfsetstrokecolor{currentstroke}%
\pgfsetdash{}{0pt}%
\pgfpathmoveto{\pgfqpoint{6.740185in}{3.523155in}}%
\pgfpathcurveto{\pgfqpoint{6.746009in}{3.523155in}}{\pgfqpoint{6.751595in}{3.525469in}}{\pgfqpoint{6.755713in}{3.529587in}}%
\pgfpathcurveto{\pgfqpoint{6.759832in}{3.533705in}}{\pgfqpoint{6.762145in}{3.539292in}}{\pgfqpoint{6.762145in}{3.545116in}}%
\pgfpathcurveto{\pgfqpoint{6.762145in}{3.550939in}}{\pgfqpoint{6.759832in}{3.556526in}}{\pgfqpoint{6.755713in}{3.560644in}}%
\pgfpathcurveto{\pgfqpoint{6.751595in}{3.564762in}}{\pgfqpoint{6.746009in}{3.567076in}}{\pgfqpoint{6.740185in}{3.567076in}}%
\pgfpathcurveto{\pgfqpoint{6.734361in}{3.567076in}}{\pgfqpoint{6.728775in}{3.564762in}}{\pgfqpoint{6.724657in}{3.560644in}}%
\pgfpathcurveto{\pgfqpoint{6.720539in}{3.556526in}}{\pgfqpoint{6.718225in}{3.550939in}}{\pgfqpoint{6.718225in}{3.545116in}}%
\pgfpathcurveto{\pgfqpoint{6.718225in}{3.539292in}}{\pgfqpoint{6.720539in}{3.533705in}}{\pgfqpoint{6.724657in}{3.529587in}}%
\pgfpathcurveto{\pgfqpoint{6.728775in}{3.525469in}}{\pgfqpoint{6.734361in}{3.523155in}}{\pgfqpoint{6.740185in}{3.523155in}}%
\pgfpathlineto{\pgfqpoint{6.740185in}{3.523155in}}%
\pgfpathclose%
\pgfusepath{stroke,fill}%
\end{pgfscope}%
\begin{pgfscope}%
\pgfpathrectangle{\pgfqpoint{1.000000in}{1.148311in}}{\pgfqpoint{6.200000in}{5.623377in}}%
\pgfusepath{clip}%
\pgfsetbuttcap%
\pgfsetroundjoin%
\definecolor{currentfill}{rgb}{0.800000,0.200000,0.200000}%
\pgfsetfillcolor{currentfill}%
\pgfsetlinewidth{1.003750pt}%
\definecolor{currentstroke}{rgb}{0.800000,0.200000,0.200000}%
\pgfsetstrokecolor{currentstroke}%
\pgfsetdash{}{0pt}%
\pgfpathmoveto{\pgfqpoint{6.721611in}{3.607980in}}%
\pgfpathcurveto{\pgfqpoint{6.727435in}{3.607980in}}{\pgfqpoint{6.733021in}{3.610294in}}{\pgfqpoint{6.737139in}{3.614412in}}%
\pgfpathcurveto{\pgfqpoint{6.741257in}{3.618530in}}{\pgfqpoint{6.743571in}{3.624116in}}{\pgfqpoint{6.743571in}{3.629940in}}%
\pgfpathcurveto{\pgfqpoint{6.743571in}{3.635764in}}{\pgfqpoint{6.741257in}{3.641350in}}{\pgfqpoint{6.737139in}{3.645468in}}%
\pgfpathcurveto{\pgfqpoint{6.733021in}{3.649587in}}{\pgfqpoint{6.727435in}{3.651900in}}{\pgfqpoint{6.721611in}{3.651900in}}%
\pgfpathcurveto{\pgfqpoint{6.715787in}{3.651900in}}{\pgfqpoint{6.710201in}{3.649587in}}{\pgfqpoint{6.706083in}{3.645468in}}%
\pgfpathcurveto{\pgfqpoint{6.701964in}{3.641350in}}{\pgfqpoint{6.699650in}{3.635764in}}{\pgfqpoint{6.699650in}{3.629940in}}%
\pgfpathcurveto{\pgfqpoint{6.699650in}{3.624116in}}{\pgfqpoint{6.701964in}{3.618530in}}{\pgfqpoint{6.706083in}{3.614412in}}%
\pgfpathcurveto{\pgfqpoint{6.710201in}{3.610294in}}{\pgfqpoint{6.715787in}{3.607980in}}{\pgfqpoint{6.721611in}{3.607980in}}%
\pgfpathlineto{\pgfqpoint{6.721611in}{3.607980in}}%
\pgfpathclose%
\pgfusepath{stroke,fill}%
\end{pgfscope}%
\begin{pgfscope}%
\pgfpathrectangle{\pgfqpoint{1.000000in}{1.148311in}}{\pgfqpoint{6.200000in}{5.623377in}}%
\pgfusepath{clip}%
\pgfsetbuttcap%
\pgfsetroundjoin%
\definecolor{currentfill}{rgb}{0.800000,0.200000,0.200000}%
\pgfsetfillcolor{currentfill}%
\pgfsetlinewidth{1.003750pt}%
\definecolor{currentstroke}{rgb}{0.800000,0.200000,0.200000}%
\pgfsetstrokecolor{currentstroke}%
\pgfsetdash{}{0pt}%
\pgfpathmoveto{\pgfqpoint{6.711474in}{3.682154in}}%
\pgfpathcurveto{\pgfqpoint{6.717298in}{3.682154in}}{\pgfqpoint{6.722884in}{3.684468in}}{\pgfqpoint{6.727002in}{3.688586in}}%
\pgfpathcurveto{\pgfqpoint{6.731120in}{3.692704in}}{\pgfqpoint{6.733434in}{3.698290in}}{\pgfqpoint{6.733434in}{3.704114in}}%
\pgfpathcurveto{\pgfqpoint{6.733434in}{3.709938in}}{\pgfqpoint{6.731120in}{3.715524in}}{\pgfqpoint{6.727002in}{3.719642in}}%
\pgfpathcurveto{\pgfqpoint{6.722884in}{3.723760in}}{\pgfqpoint{6.717298in}{3.726074in}}{\pgfqpoint{6.711474in}{3.726074in}}%
\pgfpathcurveto{\pgfqpoint{6.705650in}{3.726074in}}{\pgfqpoint{6.700064in}{3.723760in}}{\pgfqpoint{6.695946in}{3.719642in}}%
\pgfpathcurveto{\pgfqpoint{6.691828in}{3.715524in}}{\pgfqpoint{6.689514in}{3.709938in}}{\pgfqpoint{6.689514in}{3.704114in}}%
\pgfpathcurveto{\pgfqpoint{6.689514in}{3.698290in}}{\pgfqpoint{6.691828in}{3.692704in}}{\pgfqpoint{6.695946in}{3.688586in}}%
\pgfpathcurveto{\pgfqpoint{6.700064in}{3.684468in}}{\pgfqpoint{6.705650in}{3.682154in}}{\pgfqpoint{6.711474in}{3.682154in}}%
\pgfpathlineto{\pgfqpoint{6.711474in}{3.682154in}}%
\pgfpathclose%
\pgfusepath{stroke,fill}%
\end{pgfscope}%
\begin{pgfscope}%
\pgfpathrectangle{\pgfqpoint{1.000000in}{1.148311in}}{\pgfqpoint{6.200000in}{5.623377in}}%
\pgfusepath{clip}%
\pgfsetbuttcap%
\pgfsetroundjoin%
\definecolor{currentfill}{rgb}{0.800000,0.200000,0.200000}%
\pgfsetfillcolor{currentfill}%
\pgfsetlinewidth{1.003750pt}%
\definecolor{currentstroke}{rgb}{0.800000,0.200000,0.200000}%
\pgfsetstrokecolor{currentstroke}%
\pgfsetdash{}{0pt}%
\pgfpathmoveto{\pgfqpoint{6.741487in}{3.738204in}}%
\pgfpathcurveto{\pgfqpoint{6.747311in}{3.738204in}}{\pgfqpoint{6.752897in}{3.740518in}}{\pgfqpoint{6.757015in}{3.744636in}}%
\pgfpathcurveto{\pgfqpoint{6.761133in}{3.748754in}}{\pgfqpoint{6.763447in}{3.754340in}}{\pgfqpoint{6.763447in}{3.760164in}}%
\pgfpathcurveto{\pgfqpoint{6.763447in}{3.765988in}}{\pgfqpoint{6.761133in}{3.771574in}}{\pgfqpoint{6.757015in}{3.775693in}}%
\pgfpathcurveto{\pgfqpoint{6.752897in}{3.779811in}}{\pgfqpoint{6.747311in}{3.782125in}}{\pgfqpoint{6.741487in}{3.782125in}}%
\pgfpathcurveto{\pgfqpoint{6.735663in}{3.782125in}}{\pgfqpoint{6.730077in}{3.779811in}}{\pgfqpoint{6.725959in}{3.775693in}}%
\pgfpathcurveto{\pgfqpoint{6.721841in}{3.771574in}}{\pgfqpoint{6.719527in}{3.765988in}}{\pgfqpoint{6.719527in}{3.760164in}}%
\pgfpathcurveto{\pgfqpoint{6.719527in}{3.754340in}}{\pgfqpoint{6.721841in}{3.748754in}}{\pgfqpoint{6.725959in}{3.744636in}}%
\pgfpathcurveto{\pgfqpoint{6.730077in}{3.740518in}}{\pgfqpoint{6.735663in}{3.738204in}}{\pgfqpoint{6.741487in}{3.738204in}}%
\pgfpathlineto{\pgfqpoint{6.741487in}{3.738204in}}%
\pgfpathclose%
\pgfusepath{stroke,fill}%
\end{pgfscope}%
\begin{pgfscope}%
\pgfpathrectangle{\pgfqpoint{1.000000in}{1.148311in}}{\pgfqpoint{6.200000in}{5.623377in}}%
\pgfusepath{clip}%
\pgfsetbuttcap%
\pgfsetroundjoin%
\definecolor{currentfill}{rgb}{0.800000,0.200000,0.200000}%
\pgfsetfillcolor{currentfill}%
\pgfsetlinewidth{1.003750pt}%
\definecolor{currentstroke}{rgb}{0.800000,0.200000,0.200000}%
\pgfsetstrokecolor{currentstroke}%
\pgfsetdash{}{0pt}%
\pgfpathmoveto{\pgfqpoint{6.795740in}{3.789410in}}%
\pgfpathcurveto{\pgfqpoint{6.801564in}{3.789410in}}{\pgfqpoint{6.807150in}{3.791724in}}{\pgfqpoint{6.811268in}{3.795842in}}%
\pgfpathcurveto{\pgfqpoint{6.815386in}{3.799960in}}{\pgfqpoint{6.817700in}{3.805546in}}{\pgfqpoint{6.817700in}{3.811370in}}%
\pgfpathcurveto{\pgfqpoint{6.817700in}{3.817194in}}{\pgfqpoint{6.815386in}{3.822780in}}{\pgfqpoint{6.811268in}{3.826898in}}%
\pgfpathcurveto{\pgfqpoint{6.807150in}{3.831016in}}{\pgfqpoint{6.801564in}{3.833330in}}{\pgfqpoint{6.795740in}{3.833330in}}%
\pgfpathcurveto{\pgfqpoint{6.789916in}{3.833330in}}{\pgfqpoint{6.784330in}{3.831016in}}{\pgfqpoint{6.780212in}{3.826898in}}%
\pgfpathcurveto{\pgfqpoint{6.776094in}{3.822780in}}{\pgfqpoint{6.773780in}{3.817194in}}{\pgfqpoint{6.773780in}{3.811370in}}%
\pgfpathcurveto{\pgfqpoint{6.773780in}{3.805546in}}{\pgfqpoint{6.776094in}{3.799960in}}{\pgfqpoint{6.780212in}{3.795842in}}%
\pgfpathcurveto{\pgfqpoint{6.784330in}{3.791724in}}{\pgfqpoint{6.789916in}{3.789410in}}{\pgfqpoint{6.795740in}{3.789410in}}%
\pgfpathlineto{\pgfqpoint{6.795740in}{3.789410in}}%
\pgfpathclose%
\pgfusepath{stroke,fill}%
\end{pgfscope}%
\begin{pgfscope}%
\pgfpathrectangle{\pgfqpoint{1.000000in}{1.148311in}}{\pgfqpoint{6.200000in}{5.623377in}}%
\pgfusepath{clip}%
\pgfsetbuttcap%
\pgfsetroundjoin%
\definecolor{currentfill}{rgb}{0.800000,0.200000,0.200000}%
\pgfsetfillcolor{currentfill}%
\pgfsetlinewidth{1.003750pt}%
\definecolor{currentstroke}{rgb}{0.800000,0.200000,0.200000}%
\pgfsetstrokecolor{currentstroke}%
\pgfsetdash{}{0pt}%
\pgfpathmoveto{\pgfqpoint{6.767063in}{3.861721in}}%
\pgfpathcurveto{\pgfqpoint{6.772887in}{3.861721in}}{\pgfqpoint{6.778473in}{3.864035in}}{\pgfqpoint{6.782592in}{3.868153in}}%
\pgfpathcurveto{\pgfqpoint{6.786710in}{3.872272in}}{\pgfqpoint{6.789024in}{3.877858in}}{\pgfqpoint{6.789024in}{3.883682in}}%
\pgfpathcurveto{\pgfqpoint{6.789024in}{3.889506in}}{\pgfqpoint{6.786710in}{3.895092in}}{\pgfqpoint{6.782592in}{3.899210in}}%
\pgfpathcurveto{\pgfqpoint{6.778473in}{3.903328in}}{\pgfqpoint{6.772887in}{3.905642in}}{\pgfqpoint{6.767063in}{3.905642in}}%
\pgfpathcurveto{\pgfqpoint{6.761239in}{3.905642in}}{\pgfqpoint{6.755653in}{3.903328in}}{\pgfqpoint{6.751535in}{3.899210in}}%
\pgfpathcurveto{\pgfqpoint{6.747417in}{3.895092in}}{\pgfqpoint{6.745103in}{3.889506in}}{\pgfqpoint{6.745103in}{3.883682in}}%
\pgfpathcurveto{\pgfqpoint{6.745103in}{3.877858in}}{\pgfqpoint{6.747417in}{3.872272in}}{\pgfqpoint{6.751535in}{3.868153in}}%
\pgfpathcurveto{\pgfqpoint{6.755653in}{3.864035in}}{\pgfqpoint{6.761239in}{3.861721in}}{\pgfqpoint{6.767063in}{3.861721in}}%
\pgfpathlineto{\pgfqpoint{6.767063in}{3.861721in}}%
\pgfpathclose%
\pgfusepath{stroke,fill}%
\end{pgfscope}%
\begin{pgfscope}%
\pgfpathrectangle{\pgfqpoint{1.000000in}{1.148311in}}{\pgfqpoint{6.200000in}{5.623377in}}%
\pgfusepath{clip}%
\pgfsetbuttcap%
\pgfsetroundjoin%
\definecolor{currentfill}{rgb}{0.800000,0.200000,0.200000}%
\pgfsetfillcolor{currentfill}%
\pgfsetlinewidth{1.003750pt}%
\definecolor{currentstroke}{rgb}{0.800000,0.200000,0.200000}%
\pgfsetstrokecolor{currentstroke}%
\pgfsetdash{}{0pt}%
\pgfpathmoveto{\pgfqpoint{6.901173in}{3.907908in}}%
\pgfpathcurveto{\pgfqpoint{6.906997in}{3.907908in}}{\pgfqpoint{6.912583in}{3.910221in}}{\pgfqpoint{6.916701in}{3.914340in}}%
\pgfpathcurveto{\pgfqpoint{6.920819in}{3.918458in}}{\pgfqpoint{6.923133in}{3.924044in}}{\pgfqpoint{6.923133in}{3.929868in}}%
\pgfpathcurveto{\pgfqpoint{6.923133in}{3.935692in}}{\pgfqpoint{6.920819in}{3.941278in}}{\pgfqpoint{6.916701in}{3.945396in}}%
\pgfpathcurveto{\pgfqpoint{6.912583in}{3.949514in}}{\pgfqpoint{6.906997in}{3.951828in}}{\pgfqpoint{6.901173in}{3.951828in}}%
\pgfpathcurveto{\pgfqpoint{6.895349in}{3.951828in}}{\pgfqpoint{6.889763in}{3.949514in}}{\pgfqpoint{6.885645in}{3.945396in}}%
\pgfpathcurveto{\pgfqpoint{6.881527in}{3.941278in}}{\pgfqpoint{6.879213in}{3.935692in}}{\pgfqpoint{6.879213in}{3.929868in}}%
\pgfpathcurveto{\pgfqpoint{6.879213in}{3.924044in}}{\pgfqpoint{6.881527in}{3.918458in}}{\pgfqpoint{6.885645in}{3.914340in}}%
\pgfpathcurveto{\pgfqpoint{6.889763in}{3.910221in}}{\pgfqpoint{6.895349in}{3.907908in}}{\pgfqpoint{6.901173in}{3.907908in}}%
\pgfpathlineto{\pgfqpoint{6.901173in}{3.907908in}}%
\pgfpathclose%
\pgfusepath{stroke,fill}%
\end{pgfscope}%
\begin{pgfscope}%
\pgfpathrectangle{\pgfqpoint{1.000000in}{1.148311in}}{\pgfqpoint{6.200000in}{5.623377in}}%
\pgfusepath{clip}%
\pgfsetbuttcap%
\pgfsetroundjoin%
\definecolor{currentfill}{rgb}{0.800000,0.200000,0.200000}%
\pgfsetfillcolor{currentfill}%
\pgfsetlinewidth{1.003750pt}%
\definecolor{currentstroke}{rgb}{0.800000,0.200000,0.200000}%
\pgfsetstrokecolor{currentstroke}%
\pgfsetdash{}{0pt}%
\pgfpathmoveto{\pgfqpoint{6.843386in}{3.982455in}}%
\pgfpathcurveto{\pgfqpoint{6.849210in}{3.982455in}}{\pgfqpoint{6.854796in}{3.984769in}}{\pgfqpoint{6.858914in}{3.988887in}}%
\pgfpathcurveto{\pgfqpoint{6.863033in}{3.993005in}}{\pgfqpoint{6.865346in}{3.998591in}}{\pgfqpoint{6.865346in}{4.004415in}}%
\pgfpathcurveto{\pgfqpoint{6.865346in}{4.010239in}}{\pgfqpoint{6.863033in}{4.015825in}}{\pgfqpoint{6.858914in}{4.019943in}}%
\pgfpathcurveto{\pgfqpoint{6.854796in}{4.024061in}}{\pgfqpoint{6.849210in}{4.026375in}}{\pgfqpoint{6.843386in}{4.026375in}}%
\pgfpathcurveto{\pgfqpoint{6.837562in}{4.026375in}}{\pgfqpoint{6.831976in}{4.024061in}}{\pgfqpoint{6.827858in}{4.019943in}}%
\pgfpathcurveto{\pgfqpoint{6.823740in}{4.015825in}}{\pgfqpoint{6.821426in}{4.010239in}}{\pgfqpoint{6.821426in}{4.004415in}}%
\pgfpathcurveto{\pgfqpoint{6.821426in}{3.998591in}}{\pgfqpoint{6.823740in}{3.993005in}}{\pgfqpoint{6.827858in}{3.988887in}}%
\pgfpathcurveto{\pgfqpoint{6.831976in}{3.984769in}}{\pgfqpoint{6.837562in}{3.982455in}}{\pgfqpoint{6.843386in}{3.982455in}}%
\pgfpathlineto{\pgfqpoint{6.843386in}{3.982455in}}%
\pgfpathclose%
\pgfusepath{stroke,fill}%
\end{pgfscope}%
\begin{pgfscope}%
\pgfpathrectangle{\pgfqpoint{1.000000in}{1.148311in}}{\pgfqpoint{6.200000in}{5.623377in}}%
\pgfusepath{clip}%
\pgfsetbuttcap%
\pgfsetroundjoin%
\definecolor{currentfill}{rgb}{0.800000,0.200000,0.200000}%
\pgfsetfillcolor{currentfill}%
\pgfsetlinewidth{1.003750pt}%
\definecolor{currentstroke}{rgb}{0.800000,0.200000,0.200000}%
\pgfsetstrokecolor{currentstroke}%
\pgfsetdash{}{0pt}%
\pgfpathmoveto{\pgfqpoint{6.885166in}{4.049087in}}%
\pgfpathcurveto{\pgfqpoint{6.890990in}{4.049087in}}{\pgfqpoint{6.896576in}{4.051400in}}{\pgfqpoint{6.900694in}{4.055519in}}%
\pgfpathcurveto{\pgfqpoint{6.904812in}{4.059637in}}{\pgfqpoint{6.907126in}{4.065223in}}{\pgfqpoint{6.907126in}{4.071047in}}%
\pgfpathcurveto{\pgfqpoint{6.907126in}{4.076871in}}{\pgfqpoint{6.904812in}{4.082457in}}{\pgfqpoint{6.900694in}{4.086575in}}%
\pgfpathcurveto{\pgfqpoint{6.896576in}{4.090693in}}{\pgfqpoint{6.890990in}{4.093007in}}{\pgfqpoint{6.885166in}{4.093007in}}%
\pgfpathcurveto{\pgfqpoint{6.879342in}{4.093007in}}{\pgfqpoint{6.873756in}{4.090693in}}{\pgfqpoint{6.869637in}{4.086575in}}%
\pgfpathcurveto{\pgfqpoint{6.865519in}{4.082457in}}{\pgfqpoint{6.863205in}{4.076871in}}{\pgfqpoint{6.863205in}{4.071047in}}%
\pgfpathcurveto{\pgfqpoint{6.863205in}{4.065223in}}{\pgfqpoint{6.865519in}{4.059637in}}{\pgfqpoint{6.869637in}{4.055519in}}%
\pgfpathcurveto{\pgfqpoint{6.873756in}{4.051400in}}{\pgfqpoint{6.879342in}{4.049087in}}{\pgfqpoint{6.885166in}{4.049087in}}%
\pgfpathlineto{\pgfqpoint{6.885166in}{4.049087in}}%
\pgfpathclose%
\pgfusepath{stroke,fill}%
\end{pgfscope}%
\begin{pgfscope}%
\pgfpathrectangle{\pgfqpoint{1.000000in}{1.148311in}}{\pgfqpoint{6.200000in}{5.623377in}}%
\pgfusepath{clip}%
\pgfsetbuttcap%
\pgfsetroundjoin%
\definecolor{currentfill}{rgb}{0.800000,0.800000,0.200000}%
\pgfsetfillcolor{currentfill}%
\pgfsetlinewidth{1.003750pt}%
\definecolor{currentstroke}{rgb}{0.800000,0.800000,0.200000}%
\pgfsetstrokecolor{currentstroke}%
\pgfsetdash{}{0pt}%
\pgfpathmoveto{\pgfqpoint{5.419977in}{5.448217in}}%
\pgfpathcurveto{\pgfqpoint{5.425801in}{5.448217in}}{\pgfqpoint{5.431388in}{5.450531in}}{\pgfqpoint{5.435506in}{5.454649in}}%
\pgfpathcurveto{\pgfqpoint{5.439624in}{5.458767in}}{\pgfqpoint{5.441938in}{5.464353in}}{\pgfqpoint{5.441938in}{5.470177in}}%
\pgfpathcurveto{\pgfqpoint{5.441938in}{5.476001in}}{\pgfqpoint{5.439624in}{5.481587in}}{\pgfqpoint{5.435506in}{5.485705in}}%
\pgfpathcurveto{\pgfqpoint{5.431388in}{5.489823in}}{\pgfqpoint{5.425801in}{5.492137in}}{\pgfqpoint{5.419977in}{5.492137in}}%
\pgfpathcurveto{\pgfqpoint{5.414154in}{5.492137in}}{\pgfqpoint{5.408567in}{5.489823in}}{\pgfqpoint{5.404449in}{5.485705in}}%
\pgfpathcurveto{\pgfqpoint{5.400331in}{5.481587in}}{\pgfqpoint{5.398017in}{5.476001in}}{\pgfqpoint{5.398017in}{5.470177in}}%
\pgfpathcurveto{\pgfqpoint{5.398017in}{5.464353in}}{\pgfqpoint{5.400331in}{5.458767in}}{\pgfqpoint{5.404449in}{5.454649in}}%
\pgfpathcurveto{\pgfqpoint{5.408567in}{5.450531in}}{\pgfqpoint{5.414154in}{5.448217in}}{\pgfqpoint{5.419977in}{5.448217in}}%
\pgfpathlineto{\pgfqpoint{5.419977in}{5.448217in}}%
\pgfpathclose%
\pgfusepath{stroke,fill}%
\end{pgfscope}%
\begin{pgfscope}%
\pgfpathrectangle{\pgfqpoint{1.000000in}{1.148311in}}{\pgfqpoint{6.200000in}{5.623377in}}%
\pgfusepath{clip}%
\pgfsetbuttcap%
\pgfsetroundjoin%
\definecolor{currentfill}{rgb}{0.800000,0.800000,0.200000}%
\pgfsetfillcolor{currentfill}%
\pgfsetlinewidth{1.003750pt}%
\definecolor{currentstroke}{rgb}{0.800000,0.800000,0.200000}%
\pgfsetstrokecolor{currentstroke}%
\pgfsetdash{}{0pt}%
\pgfpathmoveto{\pgfqpoint{5.448951in}{5.513215in}}%
\pgfpathcurveto{\pgfqpoint{5.454775in}{5.513215in}}{\pgfqpoint{5.460361in}{5.515529in}}{\pgfqpoint{5.464479in}{5.519647in}}%
\pgfpathcurveto{\pgfqpoint{5.468598in}{5.523766in}}{\pgfqpoint{5.470911in}{5.529352in}}{\pgfqpoint{5.470911in}{5.535176in}}%
\pgfpathcurveto{\pgfqpoint{5.470911in}{5.541000in}}{\pgfqpoint{5.468598in}{5.546586in}}{\pgfqpoint{5.464479in}{5.550704in}}%
\pgfpathcurveto{\pgfqpoint{5.460361in}{5.554822in}}{\pgfqpoint{5.454775in}{5.557136in}}{\pgfqpoint{5.448951in}{5.557136in}}%
\pgfpathcurveto{\pgfqpoint{5.443127in}{5.557136in}}{\pgfqpoint{5.437541in}{5.554822in}}{\pgfqpoint{5.433423in}{5.550704in}}%
\pgfpathcurveto{\pgfqpoint{5.429305in}{5.546586in}}{\pgfqpoint{5.426991in}{5.541000in}}{\pgfqpoint{5.426991in}{5.535176in}}%
\pgfpathcurveto{\pgfqpoint{5.426991in}{5.529352in}}{\pgfqpoint{5.429305in}{5.523766in}}{\pgfqpoint{5.433423in}{5.519647in}}%
\pgfpathcurveto{\pgfqpoint{5.437541in}{5.515529in}}{\pgfqpoint{5.443127in}{5.513215in}}{\pgfqpoint{5.448951in}{5.513215in}}%
\pgfpathlineto{\pgfqpoint{5.448951in}{5.513215in}}%
\pgfpathclose%
\pgfusepath{stroke,fill}%
\end{pgfscope}%
\begin{pgfscope}%
\pgfpathrectangle{\pgfqpoint{1.000000in}{1.148311in}}{\pgfqpoint{6.200000in}{5.623377in}}%
\pgfusepath{clip}%
\pgfsetbuttcap%
\pgfsetroundjoin%
\definecolor{currentfill}{rgb}{0.800000,0.800000,0.200000}%
\pgfsetfillcolor{currentfill}%
\pgfsetlinewidth{1.003750pt}%
\definecolor{currentstroke}{rgb}{0.800000,0.800000,0.200000}%
\pgfsetstrokecolor{currentstroke}%
\pgfsetdash{}{0pt}%
\pgfpathmoveto{\pgfqpoint{5.322488in}{5.562602in}}%
\pgfpathcurveto{\pgfqpoint{5.328312in}{5.562602in}}{\pgfqpoint{5.333899in}{5.564916in}}{\pgfqpoint{5.338017in}{5.569034in}}%
\pgfpathcurveto{\pgfqpoint{5.342135in}{5.573152in}}{\pgfqpoint{5.344449in}{5.578738in}}{\pgfqpoint{5.344449in}{5.584562in}}%
\pgfpathcurveto{\pgfqpoint{5.344449in}{5.590386in}}{\pgfqpoint{5.342135in}{5.595972in}}{\pgfqpoint{5.338017in}{5.600091in}}%
\pgfpathcurveto{\pgfqpoint{5.333899in}{5.604209in}}{\pgfqpoint{5.328312in}{5.606523in}}{\pgfqpoint{5.322488in}{5.606523in}}%
\pgfpathcurveto{\pgfqpoint{5.316664in}{5.606523in}}{\pgfqpoint{5.311078in}{5.604209in}}{\pgfqpoint{5.306960in}{5.600091in}}%
\pgfpathcurveto{\pgfqpoint{5.302842in}{5.595972in}}{\pgfqpoint{5.300528in}{5.590386in}}{\pgfqpoint{5.300528in}{5.584562in}}%
\pgfpathcurveto{\pgfqpoint{5.300528in}{5.578738in}}{\pgfqpoint{5.302842in}{5.573152in}}{\pgfqpoint{5.306960in}{5.569034in}}%
\pgfpathcurveto{\pgfqpoint{5.311078in}{5.564916in}}{\pgfqpoint{5.316664in}{5.562602in}}{\pgfqpoint{5.322488in}{5.562602in}}%
\pgfpathlineto{\pgfqpoint{5.322488in}{5.562602in}}%
\pgfpathclose%
\pgfusepath{stroke,fill}%
\end{pgfscope}%
\begin{pgfscope}%
\pgfpathrectangle{\pgfqpoint{1.000000in}{1.148311in}}{\pgfqpoint{6.200000in}{5.623377in}}%
\pgfusepath{clip}%
\pgfsetbuttcap%
\pgfsetroundjoin%
\definecolor{currentfill}{rgb}{0.800000,0.800000,0.200000}%
\pgfsetfillcolor{currentfill}%
\pgfsetlinewidth{1.003750pt}%
\definecolor{currentstroke}{rgb}{0.800000,0.800000,0.200000}%
\pgfsetstrokecolor{currentstroke}%
\pgfsetdash{}{0pt}%
\pgfpathmoveto{\pgfqpoint{5.403183in}{5.636517in}}%
\pgfpathcurveto{\pgfqpoint{5.409006in}{5.636517in}}{\pgfqpoint{5.414593in}{5.638831in}}{\pgfqpoint{5.418711in}{5.642949in}}%
\pgfpathcurveto{\pgfqpoint{5.422829in}{5.647068in}}{\pgfqpoint{5.425143in}{5.652654in}}{\pgfqpoint{5.425143in}{5.658478in}}%
\pgfpathcurveto{\pgfqpoint{5.425143in}{5.664302in}}{\pgfqpoint{5.422829in}{5.669888in}}{\pgfqpoint{5.418711in}{5.674006in}}%
\pgfpathcurveto{\pgfqpoint{5.414593in}{5.678124in}}{\pgfqpoint{5.409006in}{5.680438in}}{\pgfqpoint{5.403183in}{5.680438in}}%
\pgfpathcurveto{\pgfqpoint{5.397359in}{5.680438in}}{\pgfqpoint{5.391772in}{5.678124in}}{\pgfqpoint{5.387654in}{5.674006in}}%
\pgfpathcurveto{\pgfqpoint{5.383536in}{5.669888in}}{\pgfqpoint{5.381222in}{5.664302in}}{\pgfqpoint{5.381222in}{5.658478in}}%
\pgfpathcurveto{\pgfqpoint{5.381222in}{5.652654in}}{\pgfqpoint{5.383536in}{5.647068in}}{\pgfqpoint{5.387654in}{5.642949in}}%
\pgfpathcurveto{\pgfqpoint{5.391772in}{5.638831in}}{\pgfqpoint{5.397359in}{5.636517in}}{\pgfqpoint{5.403183in}{5.636517in}}%
\pgfpathlineto{\pgfqpoint{5.403183in}{5.636517in}}%
\pgfpathclose%
\pgfusepath{stroke,fill}%
\end{pgfscope}%
\begin{pgfscope}%
\pgfpathrectangle{\pgfqpoint{1.000000in}{1.148311in}}{\pgfqpoint{6.200000in}{5.623377in}}%
\pgfusepath{clip}%
\pgfsetbuttcap%
\pgfsetroundjoin%
\definecolor{currentfill}{rgb}{0.800000,0.800000,0.200000}%
\pgfsetfillcolor{currentfill}%
\pgfsetlinewidth{1.003750pt}%
\definecolor{currentstroke}{rgb}{0.800000,0.800000,0.200000}%
\pgfsetstrokecolor{currentstroke}%
\pgfsetdash{}{0pt}%
\pgfpathmoveto{\pgfqpoint{5.357777in}{5.689931in}}%
\pgfpathcurveto{\pgfqpoint{5.363601in}{5.689931in}}{\pgfqpoint{5.369187in}{5.692245in}}{\pgfqpoint{5.373305in}{5.696363in}}%
\pgfpathcurveto{\pgfqpoint{5.377423in}{5.700481in}}{\pgfqpoint{5.379737in}{5.706068in}}{\pgfqpoint{5.379737in}{5.711891in}}%
\pgfpathcurveto{\pgfqpoint{5.379737in}{5.717715in}}{\pgfqpoint{5.377423in}{5.723302in}}{\pgfqpoint{5.373305in}{5.727420in}}%
\pgfpathcurveto{\pgfqpoint{5.369187in}{5.731538in}}{\pgfqpoint{5.363601in}{5.733852in}}{\pgfqpoint{5.357777in}{5.733852in}}%
\pgfpathcurveto{\pgfqpoint{5.351953in}{5.733852in}}{\pgfqpoint{5.346367in}{5.731538in}}{\pgfqpoint{5.342249in}{5.727420in}}%
\pgfpathcurveto{\pgfqpoint{5.338131in}{5.723302in}}{\pgfqpoint{5.335817in}{5.717715in}}{\pgfqpoint{5.335817in}{5.711891in}}%
\pgfpathcurveto{\pgfqpoint{5.335817in}{5.706068in}}{\pgfqpoint{5.338131in}{5.700481in}}{\pgfqpoint{5.342249in}{5.696363in}}%
\pgfpathcurveto{\pgfqpoint{5.346367in}{5.692245in}}{\pgfqpoint{5.351953in}{5.689931in}}{\pgfqpoint{5.357777in}{5.689931in}}%
\pgfpathlineto{\pgfqpoint{5.357777in}{5.689931in}}%
\pgfpathclose%
\pgfusepath{stroke,fill}%
\end{pgfscope}%
\begin{pgfscope}%
\pgfpathrectangle{\pgfqpoint{1.000000in}{1.148311in}}{\pgfqpoint{6.200000in}{5.623377in}}%
\pgfusepath{clip}%
\pgfsetbuttcap%
\pgfsetroundjoin%
\definecolor{currentfill}{rgb}{0.800000,0.800000,0.200000}%
\pgfsetfillcolor{currentfill}%
\pgfsetlinewidth{1.003750pt}%
\definecolor{currentstroke}{rgb}{0.800000,0.800000,0.200000}%
\pgfsetstrokecolor{currentstroke}%
\pgfsetdash{}{0pt}%
\pgfpathmoveto{\pgfqpoint{5.428180in}{5.777303in}}%
\pgfpathcurveto{\pgfqpoint{5.434004in}{5.777303in}}{\pgfqpoint{5.439590in}{5.779617in}}{\pgfqpoint{5.443708in}{5.783735in}}%
\pgfpathcurveto{\pgfqpoint{5.447827in}{5.787854in}}{\pgfqpoint{5.450140in}{5.793440in}}{\pgfqpoint{5.450140in}{5.799264in}}%
\pgfpathcurveto{\pgfqpoint{5.450140in}{5.805088in}}{\pgfqpoint{5.447827in}{5.810674in}}{\pgfqpoint{5.443708in}{5.814792in}}%
\pgfpathcurveto{\pgfqpoint{5.439590in}{5.818910in}}{\pgfqpoint{5.434004in}{5.821224in}}{\pgfqpoint{5.428180in}{5.821224in}}%
\pgfpathcurveto{\pgfqpoint{5.422356in}{5.821224in}}{\pgfqpoint{5.416770in}{5.818910in}}{\pgfqpoint{5.412652in}{5.814792in}}%
\pgfpathcurveto{\pgfqpoint{5.408534in}{5.810674in}}{\pgfqpoint{5.406220in}{5.805088in}}{\pgfqpoint{5.406220in}{5.799264in}}%
\pgfpathcurveto{\pgfqpoint{5.406220in}{5.793440in}}{\pgfqpoint{5.408534in}{5.787854in}}{\pgfqpoint{5.412652in}{5.783735in}}%
\pgfpathcurveto{\pgfqpoint{5.416770in}{5.779617in}}{\pgfqpoint{5.422356in}{5.777303in}}{\pgfqpoint{5.428180in}{5.777303in}}%
\pgfpathlineto{\pgfqpoint{5.428180in}{5.777303in}}%
\pgfpathclose%
\pgfusepath{stroke,fill}%
\end{pgfscope}%
\begin{pgfscope}%
\pgfpathrectangle{\pgfqpoint{1.000000in}{1.148311in}}{\pgfqpoint{6.200000in}{5.623377in}}%
\pgfusepath{clip}%
\pgfsetbuttcap%
\pgfsetroundjoin%
\definecolor{currentfill}{rgb}{0.800000,0.800000,0.200000}%
\pgfsetfillcolor{currentfill}%
\pgfsetlinewidth{1.003750pt}%
\definecolor{currentstroke}{rgb}{0.800000,0.800000,0.200000}%
\pgfsetstrokecolor{currentstroke}%
\pgfsetdash{}{0pt}%
\pgfpathmoveto{\pgfqpoint{5.301381in}{5.798592in}}%
\pgfpathcurveto{\pgfqpoint{5.307205in}{5.798592in}}{\pgfqpoint{5.312792in}{5.800906in}}{\pgfqpoint{5.316910in}{5.805024in}}%
\pgfpathcurveto{\pgfqpoint{5.321028in}{5.809142in}}{\pgfqpoint{5.323342in}{5.814728in}}{\pgfqpoint{5.323342in}{5.820552in}}%
\pgfpathcurveto{\pgfqpoint{5.323342in}{5.826376in}}{\pgfqpoint{5.321028in}{5.831963in}}{\pgfqpoint{5.316910in}{5.836081in}}%
\pgfpathcurveto{\pgfqpoint{5.312792in}{5.840199in}}{\pgfqpoint{5.307205in}{5.842513in}}{\pgfqpoint{5.301381in}{5.842513in}}%
\pgfpathcurveto{\pgfqpoint{5.295557in}{5.842513in}}{\pgfqpoint{5.289971in}{5.840199in}}{\pgfqpoint{5.285853in}{5.836081in}}%
\pgfpathcurveto{\pgfqpoint{5.281735in}{5.831963in}}{\pgfqpoint{5.279421in}{5.826376in}}{\pgfqpoint{5.279421in}{5.820552in}}%
\pgfpathcurveto{\pgfqpoint{5.279421in}{5.814728in}}{\pgfqpoint{5.281735in}{5.809142in}}{\pgfqpoint{5.285853in}{5.805024in}}%
\pgfpathcurveto{\pgfqpoint{5.289971in}{5.800906in}}{\pgfqpoint{5.295557in}{5.798592in}}{\pgfqpoint{5.301381in}{5.798592in}}%
\pgfpathlineto{\pgfqpoint{5.301381in}{5.798592in}}%
\pgfpathclose%
\pgfusepath{stroke,fill}%
\end{pgfscope}%
\begin{pgfscope}%
\pgfpathrectangle{\pgfqpoint{1.000000in}{1.148311in}}{\pgfqpoint{6.200000in}{5.623377in}}%
\pgfusepath{clip}%
\pgfsetbuttcap%
\pgfsetroundjoin%
\definecolor{currentfill}{rgb}{0.800000,0.800000,0.200000}%
\pgfsetfillcolor{currentfill}%
\pgfsetlinewidth{1.003750pt}%
\definecolor{currentstroke}{rgb}{0.800000,0.800000,0.200000}%
\pgfsetstrokecolor{currentstroke}%
\pgfsetdash{}{0pt}%
\pgfpathmoveto{\pgfqpoint{5.400635in}{5.912056in}}%
\pgfpathcurveto{\pgfqpoint{5.406459in}{5.912056in}}{\pgfqpoint{5.412045in}{5.914370in}}{\pgfqpoint{5.416163in}{5.918488in}}%
\pgfpathcurveto{\pgfqpoint{5.420281in}{5.922607in}}{\pgfqpoint{5.422595in}{5.928193in}}{\pgfqpoint{5.422595in}{5.934017in}}%
\pgfpathcurveto{\pgfqpoint{5.422595in}{5.939841in}}{\pgfqpoint{5.420281in}{5.945427in}}{\pgfqpoint{5.416163in}{5.949545in}}%
\pgfpathcurveto{\pgfqpoint{5.412045in}{5.953663in}}{\pgfqpoint{5.406459in}{5.955977in}}{\pgfqpoint{5.400635in}{5.955977in}}%
\pgfpathcurveto{\pgfqpoint{5.394811in}{5.955977in}}{\pgfqpoint{5.389225in}{5.953663in}}{\pgfqpoint{5.385107in}{5.949545in}}%
\pgfpathcurveto{\pgfqpoint{5.380988in}{5.945427in}}{\pgfqpoint{5.378675in}{5.939841in}}{\pgfqpoint{5.378675in}{5.934017in}}%
\pgfpathcurveto{\pgfqpoint{5.378675in}{5.928193in}}{\pgfqpoint{5.380988in}{5.922607in}}{\pgfqpoint{5.385107in}{5.918488in}}%
\pgfpathcurveto{\pgfqpoint{5.389225in}{5.914370in}}{\pgfqpoint{5.394811in}{5.912056in}}{\pgfqpoint{5.400635in}{5.912056in}}%
\pgfpathlineto{\pgfqpoint{5.400635in}{5.912056in}}%
\pgfpathclose%
\pgfusepath{stroke,fill}%
\end{pgfscope}%
\begin{pgfscope}%
\pgfpathrectangle{\pgfqpoint{1.000000in}{1.148311in}}{\pgfqpoint{6.200000in}{5.623377in}}%
\pgfusepath{clip}%
\pgfsetbuttcap%
\pgfsetroundjoin%
\definecolor{currentfill}{rgb}{0.800000,0.800000,0.200000}%
\pgfsetfillcolor{currentfill}%
\pgfsetlinewidth{1.003750pt}%
\definecolor{currentstroke}{rgb}{0.800000,0.800000,0.200000}%
\pgfsetstrokecolor{currentstroke}%
\pgfsetdash{}{0pt}%
\pgfpathmoveto{\pgfqpoint{5.339203in}{5.956206in}}%
\pgfpathcurveto{\pgfqpoint{5.345027in}{5.956206in}}{\pgfqpoint{5.350613in}{5.958519in}}{\pgfqpoint{5.354731in}{5.962638in}}%
\pgfpathcurveto{\pgfqpoint{5.358849in}{5.966756in}}{\pgfqpoint{5.361163in}{5.972342in}}{\pgfqpoint{5.361163in}{5.978166in}}%
\pgfpathcurveto{\pgfqpoint{5.361163in}{5.983990in}}{\pgfqpoint{5.358849in}{5.989576in}}{\pgfqpoint{5.354731in}{5.993694in}}%
\pgfpathcurveto{\pgfqpoint{5.350613in}{5.997812in}}{\pgfqpoint{5.345027in}{6.000126in}}{\pgfqpoint{5.339203in}{6.000126in}}%
\pgfpathcurveto{\pgfqpoint{5.333379in}{6.000126in}}{\pgfqpoint{5.327792in}{5.997812in}}{\pgfqpoint{5.323674in}{5.993694in}}%
\pgfpathcurveto{\pgfqpoint{5.319556in}{5.989576in}}{\pgfqpoint{5.317242in}{5.983990in}}{\pgfqpoint{5.317242in}{5.978166in}}%
\pgfpathcurveto{\pgfqpoint{5.317242in}{5.972342in}}{\pgfqpoint{5.319556in}{5.966756in}}{\pgfqpoint{5.323674in}{5.962638in}}%
\pgfpathcurveto{\pgfqpoint{5.327792in}{5.958519in}}{\pgfqpoint{5.333379in}{5.956206in}}{\pgfqpoint{5.339203in}{5.956206in}}%
\pgfpathlineto{\pgfqpoint{5.339203in}{5.956206in}}%
\pgfpathclose%
\pgfusepath{stroke,fill}%
\end{pgfscope}%
\begin{pgfscope}%
\pgfpathrectangle{\pgfqpoint{1.000000in}{1.148311in}}{\pgfqpoint{6.200000in}{5.623377in}}%
\pgfusepath{clip}%
\pgfsetbuttcap%
\pgfsetroundjoin%
\definecolor{currentfill}{rgb}{0.800000,0.800000,0.200000}%
\pgfsetfillcolor{currentfill}%
\pgfsetlinewidth{1.003750pt}%
\definecolor{currentstroke}{rgb}{0.800000,0.800000,0.200000}%
\pgfsetstrokecolor{currentstroke}%
\pgfsetdash{}{0pt}%
\pgfpathmoveto{\pgfqpoint{5.265898in}{5.987867in}}%
\pgfpathcurveto{\pgfqpoint{5.271722in}{5.987867in}}{\pgfqpoint{5.277308in}{5.990181in}}{\pgfqpoint{5.281427in}{5.994299in}}%
\pgfpathcurveto{\pgfqpoint{5.285545in}{5.998417in}}{\pgfqpoint{5.287859in}{6.004003in}}{\pgfqpoint{5.287859in}{6.009827in}}%
\pgfpathcurveto{\pgfqpoint{5.287859in}{6.015651in}}{\pgfqpoint{5.285545in}{6.021237in}}{\pgfqpoint{5.281427in}{6.025356in}}%
\pgfpathcurveto{\pgfqpoint{5.277308in}{6.029474in}}{\pgfqpoint{5.271722in}{6.031788in}}{\pgfqpoint{5.265898in}{6.031788in}}%
\pgfpathcurveto{\pgfqpoint{5.260074in}{6.031788in}}{\pgfqpoint{5.254488in}{6.029474in}}{\pgfqpoint{5.250370in}{6.025356in}}%
\pgfpathcurveto{\pgfqpoint{5.246252in}{6.021237in}}{\pgfqpoint{5.243938in}{6.015651in}}{\pgfqpoint{5.243938in}{6.009827in}}%
\pgfpathcurveto{\pgfqpoint{5.243938in}{6.004003in}}{\pgfqpoint{5.246252in}{5.998417in}}{\pgfqpoint{5.250370in}{5.994299in}}%
\pgfpathcurveto{\pgfqpoint{5.254488in}{5.990181in}}{\pgfqpoint{5.260074in}{5.987867in}}{\pgfqpoint{5.265898in}{5.987867in}}%
\pgfpathlineto{\pgfqpoint{5.265898in}{5.987867in}}%
\pgfpathclose%
\pgfusepath{stroke,fill}%
\end{pgfscope}%
\begin{pgfscope}%
\pgfpathrectangle{\pgfqpoint{1.000000in}{1.148311in}}{\pgfqpoint{6.200000in}{5.623377in}}%
\pgfusepath{clip}%
\pgfsetbuttcap%
\pgfsetroundjoin%
\definecolor{currentfill}{rgb}{0.800000,0.800000,0.200000}%
\pgfsetfillcolor{currentfill}%
\pgfsetlinewidth{1.003750pt}%
\definecolor{currentstroke}{rgb}{0.800000,0.800000,0.200000}%
\pgfsetstrokecolor{currentstroke}%
\pgfsetdash{}{0pt}%
\pgfpathmoveto{\pgfqpoint{5.180604in}{6.003683in}}%
\pgfpathcurveto{\pgfqpoint{5.186428in}{6.003683in}}{\pgfqpoint{5.192014in}{6.005997in}}{\pgfqpoint{5.196132in}{6.010115in}}%
\pgfpathcurveto{\pgfqpoint{5.200250in}{6.014233in}}{\pgfqpoint{5.202564in}{6.019819in}}{\pgfqpoint{5.202564in}{6.025643in}}%
\pgfpathcurveto{\pgfqpoint{5.202564in}{6.031467in}}{\pgfqpoint{5.200250in}{6.037053in}}{\pgfqpoint{5.196132in}{6.041171in}}%
\pgfpathcurveto{\pgfqpoint{5.192014in}{6.045290in}}{\pgfqpoint{5.186428in}{6.047603in}}{\pgfqpoint{5.180604in}{6.047603in}}%
\pgfpathcurveto{\pgfqpoint{5.174780in}{6.047603in}}{\pgfqpoint{5.169194in}{6.045290in}}{\pgfqpoint{5.165076in}{6.041171in}}%
\pgfpathcurveto{\pgfqpoint{5.160957in}{6.037053in}}{\pgfqpoint{5.158644in}{6.031467in}}{\pgfqpoint{5.158644in}{6.025643in}}%
\pgfpathcurveto{\pgfqpoint{5.158644in}{6.019819in}}{\pgfqpoint{5.160957in}{6.014233in}}{\pgfqpoint{5.165076in}{6.010115in}}%
\pgfpathcurveto{\pgfqpoint{5.169194in}{6.005997in}}{\pgfqpoint{5.174780in}{6.003683in}}{\pgfqpoint{5.180604in}{6.003683in}}%
\pgfpathlineto{\pgfqpoint{5.180604in}{6.003683in}}%
\pgfpathclose%
\pgfusepath{stroke,fill}%
\end{pgfscope}%
\begin{pgfscope}%
\pgfpathrectangle{\pgfqpoint{1.000000in}{1.148311in}}{\pgfqpoint{6.200000in}{5.623377in}}%
\pgfusepath{clip}%
\pgfsetbuttcap%
\pgfsetroundjoin%
\definecolor{currentfill}{rgb}{0.800000,0.800000,0.200000}%
\pgfsetfillcolor{currentfill}%
\pgfsetlinewidth{1.003750pt}%
\definecolor{currentstroke}{rgb}{0.800000,0.800000,0.200000}%
\pgfsetstrokecolor{currentstroke}%
\pgfsetdash{}{0pt}%
\pgfpathmoveto{\pgfqpoint{5.173471in}{6.075264in}}%
\pgfpathcurveto{\pgfqpoint{5.179295in}{6.075264in}}{\pgfqpoint{5.184882in}{6.077578in}}{\pgfqpoint{5.189000in}{6.081696in}}%
\pgfpathcurveto{\pgfqpoint{5.193118in}{6.085814in}}{\pgfqpoint{5.195432in}{6.091400in}}{\pgfqpoint{5.195432in}{6.097224in}}%
\pgfpathcurveto{\pgfqpoint{5.195432in}{6.103048in}}{\pgfqpoint{5.193118in}{6.108634in}}{\pgfqpoint{5.189000in}{6.112752in}}%
\pgfpathcurveto{\pgfqpoint{5.184882in}{6.116870in}}{\pgfqpoint{5.179295in}{6.119184in}}{\pgfqpoint{5.173471in}{6.119184in}}%
\pgfpathcurveto{\pgfqpoint{5.167648in}{6.119184in}}{\pgfqpoint{5.162061in}{6.116870in}}{\pgfqpoint{5.157943in}{6.112752in}}%
\pgfpathcurveto{\pgfqpoint{5.153825in}{6.108634in}}{\pgfqpoint{5.151511in}{6.103048in}}{\pgfqpoint{5.151511in}{6.097224in}}%
\pgfpathcurveto{\pgfqpoint{5.151511in}{6.091400in}}{\pgfqpoint{5.153825in}{6.085814in}}{\pgfqpoint{5.157943in}{6.081696in}}%
\pgfpathcurveto{\pgfqpoint{5.162061in}{6.077578in}}{\pgfqpoint{5.167648in}{6.075264in}}{\pgfqpoint{5.173471in}{6.075264in}}%
\pgfpathlineto{\pgfqpoint{5.173471in}{6.075264in}}%
\pgfpathclose%
\pgfusepath{stroke,fill}%
\end{pgfscope}%
\begin{pgfscope}%
\pgfpathrectangle{\pgfqpoint{1.000000in}{1.148311in}}{\pgfqpoint{6.200000in}{5.623377in}}%
\pgfusepath{clip}%
\pgfsetbuttcap%
\pgfsetroundjoin%
\definecolor{currentfill}{rgb}{0.800000,0.800000,0.200000}%
\pgfsetfillcolor{currentfill}%
\pgfsetlinewidth{1.003750pt}%
\definecolor{currentstroke}{rgb}{0.800000,0.800000,0.200000}%
\pgfsetstrokecolor{currentstroke}%
\pgfsetdash{}{0pt}%
\pgfpathmoveto{\pgfqpoint{5.164997in}{6.152672in}}%
\pgfpathcurveto{\pgfqpoint{5.170821in}{6.152672in}}{\pgfqpoint{5.176407in}{6.154986in}}{\pgfqpoint{5.180525in}{6.159104in}}%
\pgfpathcurveto{\pgfqpoint{5.184644in}{6.163222in}}{\pgfqpoint{5.186957in}{6.168808in}}{\pgfqpoint{5.186957in}{6.174632in}}%
\pgfpathcurveto{\pgfqpoint{5.186957in}{6.180456in}}{\pgfqpoint{5.184644in}{6.186042in}}{\pgfqpoint{5.180525in}{6.190160in}}%
\pgfpathcurveto{\pgfqpoint{5.176407in}{6.194279in}}{\pgfqpoint{5.170821in}{6.196592in}}{\pgfqpoint{5.164997in}{6.196592in}}%
\pgfpathcurveto{\pgfqpoint{5.159173in}{6.196592in}}{\pgfqpoint{5.153587in}{6.194279in}}{\pgfqpoint{5.149469in}{6.190160in}}%
\pgfpathcurveto{\pgfqpoint{5.145351in}{6.186042in}}{\pgfqpoint{5.143037in}{6.180456in}}{\pgfqpoint{5.143037in}{6.174632in}}%
\pgfpathcurveto{\pgfqpoint{5.143037in}{6.168808in}}{\pgfqpoint{5.145351in}{6.163222in}}{\pgfqpoint{5.149469in}{6.159104in}}%
\pgfpathcurveto{\pgfqpoint{5.153587in}{6.154986in}}{\pgfqpoint{5.159173in}{6.152672in}}{\pgfqpoint{5.164997in}{6.152672in}}%
\pgfpathlineto{\pgfqpoint{5.164997in}{6.152672in}}%
\pgfpathclose%
\pgfusepath{stroke,fill}%
\end{pgfscope}%
\begin{pgfscope}%
\pgfpathrectangle{\pgfqpoint{1.000000in}{1.148311in}}{\pgfqpoint{6.200000in}{5.623377in}}%
\pgfusepath{clip}%
\pgfsetbuttcap%
\pgfsetroundjoin%
\definecolor{currentfill}{rgb}{0.800000,0.800000,0.200000}%
\pgfsetfillcolor{currentfill}%
\pgfsetlinewidth{1.003750pt}%
\definecolor{currentstroke}{rgb}{0.800000,0.800000,0.200000}%
\pgfsetstrokecolor{currentstroke}%
\pgfsetdash{}{0pt}%
\pgfpathmoveto{\pgfqpoint{5.103364in}{6.181367in}}%
\pgfpathcurveto{\pgfqpoint{5.109188in}{6.181367in}}{\pgfqpoint{5.114774in}{6.183681in}}{\pgfqpoint{5.118892in}{6.187799in}}%
\pgfpathcurveto{\pgfqpoint{5.123010in}{6.191917in}}{\pgfqpoint{5.125324in}{6.197503in}}{\pgfqpoint{5.125324in}{6.203327in}}%
\pgfpathcurveto{\pgfqpoint{5.125324in}{6.209151in}}{\pgfqpoint{5.123010in}{6.214737in}}{\pgfqpoint{5.118892in}{6.218855in}}%
\pgfpathcurveto{\pgfqpoint{5.114774in}{6.222974in}}{\pgfqpoint{5.109188in}{6.225287in}}{\pgfqpoint{5.103364in}{6.225287in}}%
\pgfpathcurveto{\pgfqpoint{5.097540in}{6.225287in}}{\pgfqpoint{5.091954in}{6.222974in}}{\pgfqpoint{5.087836in}{6.218855in}}%
\pgfpathcurveto{\pgfqpoint{5.083717in}{6.214737in}}{\pgfqpoint{5.081404in}{6.209151in}}{\pgfqpoint{5.081404in}{6.203327in}}%
\pgfpathcurveto{\pgfqpoint{5.081404in}{6.197503in}}{\pgfqpoint{5.083717in}{6.191917in}}{\pgfqpoint{5.087836in}{6.187799in}}%
\pgfpathcurveto{\pgfqpoint{5.091954in}{6.183681in}}{\pgfqpoint{5.097540in}{6.181367in}}{\pgfqpoint{5.103364in}{6.181367in}}%
\pgfpathlineto{\pgfqpoint{5.103364in}{6.181367in}}%
\pgfpathclose%
\pgfusepath{stroke,fill}%
\end{pgfscope}%
\begin{pgfscope}%
\pgfpathrectangle{\pgfqpoint{1.000000in}{1.148311in}}{\pgfqpoint{6.200000in}{5.623377in}}%
\pgfusepath{clip}%
\pgfsetbuttcap%
\pgfsetroundjoin%
\definecolor{currentfill}{rgb}{0.800000,0.800000,0.200000}%
\pgfsetfillcolor{currentfill}%
\pgfsetlinewidth{1.003750pt}%
\definecolor{currentstroke}{rgb}{0.800000,0.800000,0.200000}%
\pgfsetstrokecolor{currentstroke}%
\pgfsetdash{}{0pt}%
\pgfpathmoveto{\pgfqpoint{5.025332in}{6.185705in}}%
\pgfpathcurveto{\pgfqpoint{5.031156in}{6.185705in}}{\pgfqpoint{5.036743in}{6.188019in}}{\pgfqpoint{5.040861in}{6.192137in}}%
\pgfpathcurveto{\pgfqpoint{5.044979in}{6.196256in}}{\pgfqpoint{5.047293in}{6.201842in}}{\pgfqpoint{5.047293in}{6.207666in}}%
\pgfpathcurveto{\pgfqpoint{5.047293in}{6.213490in}}{\pgfqpoint{5.044979in}{6.219076in}}{\pgfqpoint{5.040861in}{6.223194in}}%
\pgfpathcurveto{\pgfqpoint{5.036743in}{6.227312in}}{\pgfqpoint{5.031156in}{6.229626in}}{\pgfqpoint{5.025332in}{6.229626in}}%
\pgfpathcurveto{\pgfqpoint{5.019509in}{6.229626in}}{\pgfqpoint{5.013922in}{6.227312in}}{\pgfqpoint{5.009804in}{6.223194in}}%
\pgfpathcurveto{\pgfqpoint{5.005686in}{6.219076in}}{\pgfqpoint{5.003372in}{6.213490in}}{\pgfqpoint{5.003372in}{6.207666in}}%
\pgfpathcurveto{\pgfqpoint{5.003372in}{6.201842in}}{\pgfqpoint{5.005686in}{6.196256in}}{\pgfqpoint{5.009804in}{6.192137in}}%
\pgfpathcurveto{\pgfqpoint{5.013922in}{6.188019in}}{\pgfqpoint{5.019509in}{6.185705in}}{\pgfqpoint{5.025332in}{6.185705in}}%
\pgfpathlineto{\pgfqpoint{5.025332in}{6.185705in}}%
\pgfpathclose%
\pgfusepath{stroke,fill}%
\end{pgfscope}%
\begin{pgfscope}%
\pgfpathrectangle{\pgfqpoint{1.000000in}{1.148311in}}{\pgfqpoint{6.200000in}{5.623377in}}%
\pgfusepath{clip}%
\pgfsetbuttcap%
\pgfsetroundjoin%
\definecolor{currentfill}{rgb}{0.800000,0.800000,0.200000}%
\pgfsetfillcolor{currentfill}%
\pgfsetlinewidth{1.003750pt}%
\definecolor{currentstroke}{rgb}{0.800000,0.800000,0.200000}%
\pgfsetstrokecolor{currentstroke}%
\pgfsetdash{}{0pt}%
\pgfpathmoveto{\pgfqpoint{5.028000in}{6.293347in}}%
\pgfpathcurveto{\pgfqpoint{5.033824in}{6.293347in}}{\pgfqpoint{5.039410in}{6.295661in}}{\pgfqpoint{5.043528in}{6.299779in}}%
\pgfpathcurveto{\pgfqpoint{5.047647in}{6.303897in}}{\pgfqpoint{5.049960in}{6.309483in}}{\pgfqpoint{5.049960in}{6.315307in}}%
\pgfpathcurveto{\pgfqpoint{5.049960in}{6.321131in}}{\pgfqpoint{5.047647in}{6.326717in}}{\pgfqpoint{5.043528in}{6.330835in}}%
\pgfpathcurveto{\pgfqpoint{5.039410in}{6.334953in}}{\pgfqpoint{5.033824in}{6.337267in}}{\pgfqpoint{5.028000in}{6.337267in}}%
\pgfpathcurveto{\pgfqpoint{5.022176in}{6.337267in}}{\pgfqpoint{5.016590in}{6.334953in}}{\pgfqpoint{5.012472in}{6.330835in}}%
\pgfpathcurveto{\pgfqpoint{5.008354in}{6.326717in}}{\pgfqpoint{5.006040in}{6.321131in}}{\pgfqpoint{5.006040in}{6.315307in}}%
\pgfpathcurveto{\pgfqpoint{5.006040in}{6.309483in}}{\pgfqpoint{5.008354in}{6.303897in}}{\pgfqpoint{5.012472in}{6.299779in}}%
\pgfpathcurveto{\pgfqpoint{5.016590in}{6.295661in}}{\pgfqpoint{5.022176in}{6.293347in}}{\pgfqpoint{5.028000in}{6.293347in}}%
\pgfpathlineto{\pgfqpoint{5.028000in}{6.293347in}}%
\pgfpathclose%
\pgfusepath{stroke,fill}%
\end{pgfscope}%
\begin{pgfscope}%
\pgfpathrectangle{\pgfqpoint{1.000000in}{1.148311in}}{\pgfqpoint{6.200000in}{5.623377in}}%
\pgfusepath{clip}%
\pgfsetbuttcap%
\pgfsetroundjoin%
\definecolor{currentfill}{rgb}{0.200000,0.200000,0.800000}%
\pgfsetfillcolor{currentfill}%
\pgfsetlinewidth{1.003750pt}%
\definecolor{currentstroke}{rgb}{0.200000,0.200000,0.800000}%
\pgfsetstrokecolor{currentstroke}%
\pgfsetdash{}{0pt}%
\pgfpathmoveto{\pgfqpoint{4.948071in}{6.289335in}}%
\pgfpathcurveto{\pgfqpoint{4.953895in}{6.289335in}}{\pgfqpoint{4.959481in}{6.291649in}}{\pgfqpoint{4.963599in}{6.295767in}}%
\pgfpathcurveto{\pgfqpoint{4.967717in}{6.299885in}}{\pgfqpoint{4.970031in}{6.305471in}}{\pgfqpoint{4.970031in}{6.311295in}}%
\pgfpathcurveto{\pgfqpoint{4.970031in}{6.317119in}}{\pgfqpoint{4.967717in}{6.322706in}}{\pgfqpoint{4.963599in}{6.326824in}}%
\pgfpathcurveto{\pgfqpoint{4.959481in}{6.330942in}}{\pgfqpoint{4.953895in}{6.333256in}}{\pgfqpoint{4.948071in}{6.333256in}}%
\pgfpathcurveto{\pgfqpoint{4.942247in}{6.333256in}}{\pgfqpoint{4.936661in}{6.330942in}}{\pgfqpoint{4.932543in}{6.326824in}}%
\pgfpathcurveto{\pgfqpoint{4.928425in}{6.322706in}}{\pgfqpoint{4.926111in}{6.317119in}}{\pgfqpoint{4.926111in}{6.311295in}}%
\pgfpathcurveto{\pgfqpoint{4.926111in}{6.305471in}}{\pgfqpoint{4.928425in}{6.299885in}}{\pgfqpoint{4.932543in}{6.295767in}}%
\pgfpathcurveto{\pgfqpoint{4.936661in}{6.291649in}}{\pgfqpoint{4.942247in}{6.289335in}}{\pgfqpoint{4.948071in}{6.289335in}}%
\pgfpathlineto{\pgfqpoint{4.948071in}{6.289335in}}%
\pgfpathclose%
\pgfusepath{stroke,fill}%
\end{pgfscope}%
\begin{pgfscope}%
\pgfpathrectangle{\pgfqpoint{1.000000in}{1.148311in}}{\pgfqpoint{6.200000in}{5.623377in}}%
\pgfusepath{clip}%
\pgfsetbuttcap%
\pgfsetroundjoin%
\definecolor{currentfill}{rgb}{0.800000,0.800000,0.200000}%
\pgfsetfillcolor{currentfill}%
\pgfsetlinewidth{1.003750pt}%
\definecolor{currentstroke}{rgb}{0.800000,0.800000,0.200000}%
\pgfsetstrokecolor{currentstroke}%
\pgfsetdash{}{0pt}%
\pgfpathmoveto{\pgfqpoint{4.900753in}{6.333956in}}%
\pgfpathcurveto{\pgfqpoint{4.906577in}{6.333956in}}{\pgfqpoint{4.912163in}{6.336270in}}{\pgfqpoint{4.916281in}{6.340388in}}%
\pgfpathcurveto{\pgfqpoint{4.920400in}{6.344506in}}{\pgfqpoint{4.922713in}{6.350092in}}{\pgfqpoint{4.922713in}{6.355916in}}%
\pgfpathcurveto{\pgfqpoint{4.922713in}{6.361740in}}{\pgfqpoint{4.920400in}{6.367326in}}{\pgfqpoint{4.916281in}{6.371444in}}%
\pgfpathcurveto{\pgfqpoint{4.912163in}{6.375562in}}{\pgfqpoint{4.906577in}{6.377876in}}{\pgfqpoint{4.900753in}{6.377876in}}%
\pgfpathcurveto{\pgfqpoint{4.894929in}{6.377876in}}{\pgfqpoint{4.889343in}{6.375562in}}{\pgfqpoint{4.885225in}{6.371444in}}%
\pgfpathcurveto{\pgfqpoint{4.881107in}{6.367326in}}{\pgfqpoint{4.878793in}{6.361740in}}{\pgfqpoint{4.878793in}{6.355916in}}%
\pgfpathcurveto{\pgfqpoint{4.878793in}{6.350092in}}{\pgfqpoint{4.881107in}{6.344506in}}{\pgfqpoint{4.885225in}{6.340388in}}%
\pgfpathcurveto{\pgfqpoint{4.889343in}{6.336270in}}{\pgfqpoint{4.894929in}{6.333956in}}{\pgfqpoint{4.900753in}{6.333956in}}%
\pgfpathlineto{\pgfqpoint{4.900753in}{6.333956in}}%
\pgfpathclose%
\pgfusepath{stroke,fill}%
\end{pgfscope}%
\begin{pgfscope}%
\pgfpathrectangle{\pgfqpoint{1.000000in}{1.148311in}}{\pgfqpoint{6.200000in}{5.623377in}}%
\pgfusepath{clip}%
\pgfsetbuttcap%
\pgfsetroundjoin%
\definecolor{currentfill}{rgb}{0.800000,0.800000,0.200000}%
\pgfsetfillcolor{currentfill}%
\pgfsetlinewidth{1.003750pt}%
\definecolor{currentstroke}{rgb}{0.800000,0.800000,0.200000}%
\pgfsetstrokecolor{currentstroke}%
\pgfsetdash{}{0pt}%
\pgfpathmoveto{\pgfqpoint{4.855329in}{6.387909in}}%
\pgfpathcurveto{\pgfqpoint{4.861153in}{6.387909in}}{\pgfqpoint{4.866739in}{6.390223in}}{\pgfqpoint{4.870857in}{6.394341in}}%
\pgfpathcurveto{\pgfqpoint{4.874976in}{6.398459in}}{\pgfqpoint{4.877289in}{6.404046in}}{\pgfqpoint{4.877289in}{6.409870in}}%
\pgfpathcurveto{\pgfqpoint{4.877289in}{6.415694in}}{\pgfqpoint{4.874976in}{6.421280in}}{\pgfqpoint{4.870857in}{6.425398in}}%
\pgfpathcurveto{\pgfqpoint{4.866739in}{6.429516in}}{\pgfqpoint{4.861153in}{6.431830in}}{\pgfqpoint{4.855329in}{6.431830in}}%
\pgfpathcurveto{\pgfqpoint{4.849505in}{6.431830in}}{\pgfqpoint{4.843919in}{6.429516in}}{\pgfqpoint{4.839801in}{6.425398in}}%
\pgfpathcurveto{\pgfqpoint{4.835683in}{6.421280in}}{\pgfqpoint{4.833369in}{6.415694in}}{\pgfqpoint{4.833369in}{6.409870in}}%
\pgfpathcurveto{\pgfqpoint{4.833369in}{6.404046in}}{\pgfqpoint{4.835683in}{6.398459in}}{\pgfqpoint{4.839801in}{6.394341in}}%
\pgfpathcurveto{\pgfqpoint{4.843919in}{6.390223in}}{\pgfqpoint{4.849505in}{6.387909in}}{\pgfqpoint{4.855329in}{6.387909in}}%
\pgfpathlineto{\pgfqpoint{4.855329in}{6.387909in}}%
\pgfpathclose%
\pgfusepath{stroke,fill}%
\end{pgfscope}%
\begin{pgfscope}%
\pgfpathrectangle{\pgfqpoint{1.000000in}{1.148311in}}{\pgfqpoint{6.200000in}{5.623377in}}%
\pgfusepath{clip}%
\pgfsetbuttcap%
\pgfsetroundjoin%
\definecolor{currentfill}{rgb}{0.800000,0.800000,0.200000}%
\pgfsetfillcolor{currentfill}%
\pgfsetlinewidth{1.003750pt}%
\definecolor{currentstroke}{rgb}{0.800000,0.800000,0.200000}%
\pgfsetstrokecolor{currentstroke}%
\pgfsetdash{}{0pt}%
\pgfpathmoveto{\pgfqpoint{4.793276in}{6.409074in}}%
\pgfpathcurveto{\pgfqpoint{4.799100in}{6.409074in}}{\pgfqpoint{4.804686in}{6.411388in}}{\pgfqpoint{4.808805in}{6.415506in}}%
\pgfpathcurveto{\pgfqpoint{4.812923in}{6.419624in}}{\pgfqpoint{4.815237in}{6.425210in}}{\pgfqpoint{4.815237in}{6.431034in}}%
\pgfpathcurveto{\pgfqpoint{4.815237in}{6.436858in}}{\pgfqpoint{4.812923in}{6.442445in}}{\pgfqpoint{4.808805in}{6.446563in}}%
\pgfpathcurveto{\pgfqpoint{4.804686in}{6.450681in}}{\pgfqpoint{4.799100in}{6.452995in}}{\pgfqpoint{4.793276in}{6.452995in}}%
\pgfpathcurveto{\pgfqpoint{4.787452in}{6.452995in}}{\pgfqpoint{4.781866in}{6.450681in}}{\pgfqpoint{4.777748in}{6.446563in}}%
\pgfpathcurveto{\pgfqpoint{4.773630in}{6.442445in}}{\pgfqpoint{4.771316in}{6.436858in}}{\pgfqpoint{4.771316in}{6.431034in}}%
\pgfpathcurveto{\pgfqpoint{4.771316in}{6.425210in}}{\pgfqpoint{4.773630in}{6.419624in}}{\pgfqpoint{4.777748in}{6.415506in}}%
\pgfpathcurveto{\pgfqpoint{4.781866in}{6.411388in}}{\pgfqpoint{4.787452in}{6.409074in}}{\pgfqpoint{4.793276in}{6.409074in}}%
\pgfpathlineto{\pgfqpoint{4.793276in}{6.409074in}}%
\pgfpathclose%
\pgfusepath{stroke,fill}%
\end{pgfscope}%
\begin{pgfscope}%
\pgfpathrectangle{\pgfqpoint{1.000000in}{1.148311in}}{\pgfqpoint{6.200000in}{5.623377in}}%
\pgfusepath{clip}%
\pgfsetbuttcap%
\pgfsetroundjoin%
\definecolor{currentfill}{rgb}{0.800000,0.800000,0.200000}%
\pgfsetfillcolor{currentfill}%
\pgfsetlinewidth{1.003750pt}%
\definecolor{currentstroke}{rgb}{0.800000,0.800000,0.200000}%
\pgfsetstrokecolor{currentstroke}%
\pgfsetdash{}{0pt}%
\pgfpathmoveto{\pgfqpoint{4.733920in}{6.437897in}}%
\pgfpathcurveto{\pgfqpoint{4.739744in}{6.437897in}}{\pgfqpoint{4.745330in}{6.440210in}}{\pgfqpoint{4.749448in}{6.444329in}}%
\pgfpathcurveto{\pgfqpoint{4.753567in}{6.448447in}}{\pgfqpoint{4.755880in}{6.454033in}}{\pgfqpoint{4.755880in}{6.459857in}}%
\pgfpathcurveto{\pgfqpoint{4.755880in}{6.465681in}}{\pgfqpoint{4.753567in}{6.471267in}}{\pgfqpoint{4.749448in}{6.475385in}}%
\pgfpathcurveto{\pgfqpoint{4.745330in}{6.479503in}}{\pgfqpoint{4.739744in}{6.481817in}}{\pgfqpoint{4.733920in}{6.481817in}}%
\pgfpathcurveto{\pgfqpoint{4.728096in}{6.481817in}}{\pgfqpoint{4.722510in}{6.479503in}}{\pgfqpoint{4.718392in}{6.475385in}}%
\pgfpathcurveto{\pgfqpoint{4.714274in}{6.471267in}}{\pgfqpoint{4.711960in}{6.465681in}}{\pgfqpoint{4.711960in}{6.459857in}}%
\pgfpathcurveto{\pgfqpoint{4.711960in}{6.454033in}}{\pgfqpoint{4.714274in}{6.448447in}}{\pgfqpoint{4.718392in}{6.444329in}}%
\pgfpathcurveto{\pgfqpoint{4.722510in}{6.440210in}}{\pgfqpoint{4.728096in}{6.437897in}}{\pgfqpoint{4.733920in}{6.437897in}}%
\pgfpathlineto{\pgfqpoint{4.733920in}{6.437897in}}%
\pgfpathclose%
\pgfusepath{stroke,fill}%
\end{pgfscope}%
\begin{pgfscope}%
\pgfpathrectangle{\pgfqpoint{1.000000in}{1.148311in}}{\pgfqpoint{6.200000in}{5.623377in}}%
\pgfusepath{clip}%
\pgfsetbuttcap%
\pgfsetroundjoin%
\definecolor{currentfill}{rgb}{0.800000,0.800000,0.200000}%
\pgfsetfillcolor{currentfill}%
\pgfsetlinewidth{1.003750pt}%
\definecolor{currentstroke}{rgb}{0.800000,0.800000,0.200000}%
\pgfsetstrokecolor{currentstroke}%
\pgfsetdash{}{0pt}%
\pgfpathmoveto{\pgfqpoint{4.666972in}{6.440752in}}%
\pgfpathcurveto{\pgfqpoint{4.672796in}{6.440752in}}{\pgfqpoint{4.678382in}{6.443065in}}{\pgfqpoint{4.682501in}{6.447184in}}%
\pgfpathcurveto{\pgfqpoint{4.686619in}{6.451302in}}{\pgfqpoint{4.688933in}{6.456888in}}{\pgfqpoint{4.688933in}{6.462712in}}%
\pgfpathcurveto{\pgfqpoint{4.688933in}{6.468536in}}{\pgfqpoint{4.686619in}{6.474122in}}{\pgfqpoint{4.682501in}{6.478240in}}%
\pgfpathcurveto{\pgfqpoint{4.678382in}{6.482358in}}{\pgfqpoint{4.672796in}{6.484672in}}{\pgfqpoint{4.666972in}{6.484672in}}%
\pgfpathcurveto{\pgfqpoint{4.661148in}{6.484672in}}{\pgfqpoint{4.655562in}{6.482358in}}{\pgfqpoint{4.651444in}{6.478240in}}%
\pgfpathcurveto{\pgfqpoint{4.647326in}{6.474122in}}{\pgfqpoint{4.645012in}{6.468536in}}{\pgfqpoint{4.645012in}{6.462712in}}%
\pgfpathcurveto{\pgfqpoint{4.645012in}{6.456888in}}{\pgfqpoint{4.647326in}{6.451302in}}{\pgfqpoint{4.651444in}{6.447184in}}%
\pgfpathcurveto{\pgfqpoint{4.655562in}{6.443065in}}{\pgfqpoint{4.661148in}{6.440752in}}{\pgfqpoint{4.666972in}{6.440752in}}%
\pgfpathlineto{\pgfqpoint{4.666972in}{6.440752in}}%
\pgfpathclose%
\pgfusepath{stroke,fill}%
\end{pgfscope}%
\begin{pgfscope}%
\pgfpathrectangle{\pgfqpoint{1.000000in}{1.148311in}}{\pgfqpoint{6.200000in}{5.623377in}}%
\pgfusepath{clip}%
\pgfsetbuttcap%
\pgfsetroundjoin%
\definecolor{currentfill}{rgb}{0.800000,0.800000,0.200000}%
\pgfsetfillcolor{currentfill}%
\pgfsetlinewidth{1.003750pt}%
\definecolor{currentstroke}{rgb}{0.800000,0.800000,0.200000}%
\pgfsetstrokecolor{currentstroke}%
\pgfsetdash{}{0pt}%
\pgfpathmoveto{\pgfqpoint{4.594133in}{6.400690in}}%
\pgfpathcurveto{\pgfqpoint{4.599957in}{6.400690in}}{\pgfqpoint{4.605543in}{6.403004in}}{\pgfqpoint{4.609661in}{6.407122in}}%
\pgfpathcurveto{\pgfqpoint{4.613779in}{6.411240in}}{\pgfqpoint{4.616093in}{6.416827in}}{\pgfqpoint{4.616093in}{6.422650in}}%
\pgfpathcurveto{\pgfqpoint{4.616093in}{6.428474in}}{\pgfqpoint{4.613779in}{6.434061in}}{\pgfqpoint{4.609661in}{6.438179in}}%
\pgfpathcurveto{\pgfqpoint{4.605543in}{6.442297in}}{\pgfqpoint{4.599957in}{6.444611in}}{\pgfqpoint{4.594133in}{6.444611in}}%
\pgfpathcurveto{\pgfqpoint{4.588309in}{6.444611in}}{\pgfqpoint{4.582723in}{6.442297in}}{\pgfqpoint{4.578604in}{6.438179in}}%
\pgfpathcurveto{\pgfqpoint{4.574486in}{6.434061in}}{\pgfqpoint{4.572172in}{6.428474in}}{\pgfqpoint{4.572172in}{6.422650in}}%
\pgfpathcurveto{\pgfqpoint{4.572172in}{6.416827in}}{\pgfqpoint{4.574486in}{6.411240in}}{\pgfqpoint{4.578604in}{6.407122in}}%
\pgfpathcurveto{\pgfqpoint{4.582723in}{6.403004in}}{\pgfqpoint{4.588309in}{6.400690in}}{\pgfqpoint{4.594133in}{6.400690in}}%
\pgfpathlineto{\pgfqpoint{4.594133in}{6.400690in}}%
\pgfpathclose%
\pgfusepath{stroke,fill}%
\end{pgfscope}%
\begin{pgfscope}%
\pgfpathrectangle{\pgfqpoint{1.000000in}{1.148311in}}{\pgfqpoint{6.200000in}{5.623377in}}%
\pgfusepath{clip}%
\pgfsetbuttcap%
\pgfsetroundjoin%
\definecolor{currentfill}{rgb}{0.800000,0.800000,0.200000}%
\pgfsetfillcolor{currentfill}%
\pgfsetlinewidth{1.003750pt}%
\definecolor{currentstroke}{rgb}{0.800000,0.800000,0.200000}%
\pgfsetstrokecolor{currentstroke}%
\pgfsetdash{}{0pt}%
\pgfpathmoveto{\pgfqpoint{4.536137in}{6.434102in}}%
\pgfpathcurveto{\pgfqpoint{4.541961in}{6.434102in}}{\pgfqpoint{4.547547in}{6.436416in}}{\pgfqpoint{4.551665in}{6.440534in}}%
\pgfpathcurveto{\pgfqpoint{4.555784in}{6.444652in}}{\pgfqpoint{4.558097in}{6.450239in}}{\pgfqpoint{4.558097in}{6.456063in}}%
\pgfpathcurveto{\pgfqpoint{4.558097in}{6.461886in}}{\pgfqpoint{4.555784in}{6.467473in}}{\pgfqpoint{4.551665in}{6.471591in}}%
\pgfpathcurveto{\pgfqpoint{4.547547in}{6.475709in}}{\pgfqpoint{4.541961in}{6.478023in}}{\pgfqpoint{4.536137in}{6.478023in}}%
\pgfpathcurveto{\pgfqpoint{4.530313in}{6.478023in}}{\pgfqpoint{4.524727in}{6.475709in}}{\pgfqpoint{4.520609in}{6.471591in}}%
\pgfpathcurveto{\pgfqpoint{4.516491in}{6.467473in}}{\pgfqpoint{4.514177in}{6.461886in}}{\pgfqpoint{4.514177in}{6.456063in}}%
\pgfpathcurveto{\pgfqpoint{4.514177in}{6.450239in}}{\pgfqpoint{4.516491in}{6.444652in}}{\pgfqpoint{4.520609in}{6.440534in}}%
\pgfpathcurveto{\pgfqpoint{4.524727in}{6.436416in}}{\pgfqpoint{4.530313in}{6.434102in}}{\pgfqpoint{4.536137in}{6.434102in}}%
\pgfpathlineto{\pgfqpoint{4.536137in}{6.434102in}}%
\pgfpathclose%
\pgfusepath{stroke,fill}%
\end{pgfscope}%
\begin{pgfscope}%
\pgfpathrectangle{\pgfqpoint{1.000000in}{1.148311in}}{\pgfqpoint{6.200000in}{5.623377in}}%
\pgfusepath{clip}%
\pgfsetbuttcap%
\pgfsetroundjoin%
\definecolor{currentfill}{rgb}{0.800000,0.800000,0.200000}%
\pgfsetfillcolor{currentfill}%
\pgfsetlinewidth{1.003750pt}%
\definecolor{currentstroke}{rgb}{0.800000,0.800000,0.200000}%
\pgfsetstrokecolor{currentstroke}%
\pgfsetdash{}{0pt}%
\pgfpathmoveto{\pgfqpoint{4.476008in}{6.494120in}}%
\pgfpathcurveto{\pgfqpoint{4.481832in}{6.494120in}}{\pgfqpoint{4.487419in}{6.496434in}}{\pgfqpoint{4.491537in}{6.500552in}}%
\pgfpathcurveto{\pgfqpoint{4.495655in}{6.504670in}}{\pgfqpoint{4.497969in}{6.510257in}}{\pgfqpoint{4.497969in}{6.516080in}}%
\pgfpathcurveto{\pgfqpoint{4.497969in}{6.521904in}}{\pgfqpoint{4.495655in}{6.527491in}}{\pgfqpoint{4.491537in}{6.531609in}}%
\pgfpathcurveto{\pgfqpoint{4.487419in}{6.535727in}}{\pgfqpoint{4.481832in}{6.538041in}}{\pgfqpoint{4.476008in}{6.538041in}}%
\pgfpathcurveto{\pgfqpoint{4.470184in}{6.538041in}}{\pgfqpoint{4.464598in}{6.535727in}}{\pgfqpoint{4.460480in}{6.531609in}}%
\pgfpathcurveto{\pgfqpoint{4.456362in}{6.527491in}}{\pgfqpoint{4.454048in}{6.521904in}}{\pgfqpoint{4.454048in}{6.516080in}}%
\pgfpathcurveto{\pgfqpoint{4.454048in}{6.510257in}}{\pgfqpoint{4.456362in}{6.504670in}}{\pgfqpoint{4.460480in}{6.500552in}}%
\pgfpathcurveto{\pgfqpoint{4.464598in}{6.496434in}}{\pgfqpoint{4.470184in}{6.494120in}}{\pgfqpoint{4.476008in}{6.494120in}}%
\pgfpathlineto{\pgfqpoint{4.476008in}{6.494120in}}%
\pgfpathclose%
\pgfusepath{stroke,fill}%
\end{pgfscope}%
\begin{pgfscope}%
\pgfpathrectangle{\pgfqpoint{1.000000in}{1.148311in}}{\pgfqpoint{6.200000in}{5.623377in}}%
\pgfusepath{clip}%
\pgfsetbuttcap%
\pgfsetroundjoin%
\definecolor{currentfill}{rgb}{0.800000,0.800000,0.200000}%
\pgfsetfillcolor{currentfill}%
\pgfsetlinewidth{1.003750pt}%
\definecolor{currentstroke}{rgb}{0.800000,0.800000,0.200000}%
\pgfsetstrokecolor{currentstroke}%
\pgfsetdash{}{0pt}%
\pgfpathmoveto{\pgfqpoint{4.411267in}{6.388379in}}%
\pgfpathcurveto{\pgfqpoint{4.417091in}{6.388379in}}{\pgfqpoint{4.422678in}{6.390693in}}{\pgfqpoint{4.426796in}{6.394811in}}%
\pgfpathcurveto{\pgfqpoint{4.430914in}{6.398929in}}{\pgfqpoint{4.433228in}{6.404515in}}{\pgfqpoint{4.433228in}{6.410339in}}%
\pgfpathcurveto{\pgfqpoint{4.433228in}{6.416163in}}{\pgfqpoint{4.430914in}{6.421749in}}{\pgfqpoint{4.426796in}{6.425867in}}%
\pgfpathcurveto{\pgfqpoint{4.422678in}{6.429985in}}{\pgfqpoint{4.417091in}{6.432299in}}{\pgfqpoint{4.411267in}{6.432299in}}%
\pgfpathcurveto{\pgfqpoint{4.405444in}{6.432299in}}{\pgfqpoint{4.399857in}{6.429985in}}{\pgfqpoint{4.395739in}{6.425867in}}%
\pgfpathcurveto{\pgfqpoint{4.391621in}{6.421749in}}{\pgfqpoint{4.389307in}{6.416163in}}{\pgfqpoint{4.389307in}{6.410339in}}%
\pgfpathcurveto{\pgfqpoint{4.389307in}{6.404515in}}{\pgfqpoint{4.391621in}{6.398929in}}{\pgfqpoint{4.395739in}{6.394811in}}%
\pgfpathcurveto{\pgfqpoint{4.399857in}{6.390693in}}{\pgfqpoint{4.405444in}{6.388379in}}{\pgfqpoint{4.411267in}{6.388379in}}%
\pgfpathlineto{\pgfqpoint{4.411267in}{6.388379in}}%
\pgfpathclose%
\pgfusepath{stroke,fill}%
\end{pgfscope}%
\begin{pgfscope}%
\pgfpathrectangle{\pgfqpoint{1.000000in}{1.148311in}}{\pgfqpoint{6.200000in}{5.623377in}}%
\pgfusepath{clip}%
\pgfsetbuttcap%
\pgfsetroundjoin%
\definecolor{currentfill}{rgb}{0.800000,0.800000,0.200000}%
\pgfsetfillcolor{currentfill}%
\pgfsetlinewidth{1.003750pt}%
\definecolor{currentstroke}{rgb}{0.800000,0.800000,0.200000}%
\pgfsetstrokecolor{currentstroke}%
\pgfsetdash{}{0pt}%
\pgfpathmoveto{\pgfqpoint{4.350976in}{6.394256in}}%
\pgfpathcurveto{\pgfqpoint{4.356800in}{6.394256in}}{\pgfqpoint{4.362386in}{6.396570in}}{\pgfqpoint{4.366504in}{6.400688in}}%
\pgfpathcurveto{\pgfqpoint{4.370622in}{6.404806in}}{\pgfqpoint{4.372936in}{6.410392in}}{\pgfqpoint{4.372936in}{6.416216in}}%
\pgfpathcurveto{\pgfqpoint{4.372936in}{6.422040in}}{\pgfqpoint{4.370622in}{6.427626in}}{\pgfqpoint{4.366504in}{6.431744in}}%
\pgfpathcurveto{\pgfqpoint{4.362386in}{6.435862in}}{\pgfqpoint{4.356800in}{6.438176in}}{\pgfqpoint{4.350976in}{6.438176in}}%
\pgfpathcurveto{\pgfqpoint{4.345152in}{6.438176in}}{\pgfqpoint{4.339566in}{6.435862in}}{\pgfqpoint{4.335448in}{6.431744in}}%
\pgfpathcurveto{\pgfqpoint{4.331329in}{6.427626in}}{\pgfqpoint{4.329016in}{6.422040in}}{\pgfqpoint{4.329016in}{6.416216in}}%
\pgfpathcurveto{\pgfqpoint{4.329016in}{6.410392in}}{\pgfqpoint{4.331329in}{6.404806in}}{\pgfqpoint{4.335448in}{6.400688in}}%
\pgfpathcurveto{\pgfqpoint{4.339566in}{6.396570in}}{\pgfqpoint{4.345152in}{6.394256in}}{\pgfqpoint{4.350976in}{6.394256in}}%
\pgfpathlineto{\pgfqpoint{4.350976in}{6.394256in}}%
\pgfpathclose%
\pgfusepath{stroke,fill}%
\end{pgfscope}%
\begin{pgfscope}%
\pgfpathrectangle{\pgfqpoint{1.000000in}{1.148311in}}{\pgfqpoint{6.200000in}{5.623377in}}%
\pgfusepath{clip}%
\pgfsetbuttcap%
\pgfsetroundjoin%
\definecolor{currentfill}{rgb}{0.800000,0.800000,0.200000}%
\pgfsetfillcolor{currentfill}%
\pgfsetlinewidth{1.003750pt}%
\definecolor{currentstroke}{rgb}{0.800000,0.800000,0.200000}%
\pgfsetstrokecolor{currentstroke}%
\pgfsetdash{}{0pt}%
\pgfpathmoveto{\pgfqpoint{4.281723in}{6.452979in}}%
\pgfpathcurveto{\pgfqpoint{4.287547in}{6.452979in}}{\pgfqpoint{4.293133in}{6.455293in}}{\pgfqpoint{4.297251in}{6.459411in}}%
\pgfpathcurveto{\pgfqpoint{4.301369in}{6.463529in}}{\pgfqpoint{4.303683in}{6.469115in}}{\pgfqpoint{4.303683in}{6.474939in}}%
\pgfpathcurveto{\pgfqpoint{4.303683in}{6.480763in}}{\pgfqpoint{4.301369in}{6.486349in}}{\pgfqpoint{4.297251in}{6.490468in}}%
\pgfpathcurveto{\pgfqpoint{4.293133in}{6.494586in}}{\pgfqpoint{4.287547in}{6.496900in}}{\pgfqpoint{4.281723in}{6.496900in}}%
\pgfpathcurveto{\pgfqpoint{4.275899in}{6.496900in}}{\pgfqpoint{4.270313in}{6.494586in}}{\pgfqpoint{4.266195in}{6.490468in}}%
\pgfpathcurveto{\pgfqpoint{4.262076in}{6.486349in}}{\pgfqpoint{4.259763in}{6.480763in}}{\pgfqpoint{4.259763in}{6.474939in}}%
\pgfpathcurveto{\pgfqpoint{4.259763in}{6.469115in}}{\pgfqpoint{4.262076in}{6.463529in}}{\pgfqpoint{4.266195in}{6.459411in}}%
\pgfpathcurveto{\pgfqpoint{4.270313in}{6.455293in}}{\pgfqpoint{4.275899in}{6.452979in}}{\pgfqpoint{4.281723in}{6.452979in}}%
\pgfpathlineto{\pgfqpoint{4.281723in}{6.452979in}}%
\pgfpathclose%
\pgfusepath{stroke,fill}%
\end{pgfscope}%
\begin{pgfscope}%
\pgfpathrectangle{\pgfqpoint{1.000000in}{1.148311in}}{\pgfqpoint{6.200000in}{5.623377in}}%
\pgfusepath{clip}%
\pgfsetbuttcap%
\pgfsetroundjoin%
\definecolor{currentfill}{rgb}{0.800000,0.800000,0.200000}%
\pgfsetfillcolor{currentfill}%
\pgfsetlinewidth{1.003750pt}%
\definecolor{currentstroke}{rgb}{0.800000,0.800000,0.200000}%
\pgfsetstrokecolor{currentstroke}%
\pgfsetdash{}{0pt}%
\pgfpathmoveto{\pgfqpoint{4.223407in}{6.417330in}}%
\pgfpathcurveto{\pgfqpoint{4.229231in}{6.417330in}}{\pgfqpoint{4.234817in}{6.419644in}}{\pgfqpoint{4.238935in}{6.423762in}}%
\pgfpathcurveto{\pgfqpoint{4.243053in}{6.427880in}}{\pgfqpoint{4.245367in}{6.433466in}}{\pgfqpoint{4.245367in}{6.439290in}}%
\pgfpathcurveto{\pgfqpoint{4.245367in}{6.445114in}}{\pgfqpoint{4.243053in}{6.450700in}}{\pgfqpoint{4.238935in}{6.454818in}}%
\pgfpathcurveto{\pgfqpoint{4.234817in}{6.458936in}}{\pgfqpoint{4.229231in}{6.461250in}}{\pgfqpoint{4.223407in}{6.461250in}}%
\pgfpathcurveto{\pgfqpoint{4.217583in}{6.461250in}}{\pgfqpoint{4.211997in}{6.458936in}}{\pgfqpoint{4.207878in}{6.454818in}}%
\pgfpathcurveto{\pgfqpoint{4.203760in}{6.450700in}}{\pgfqpoint{4.201446in}{6.445114in}}{\pgfqpoint{4.201446in}{6.439290in}}%
\pgfpathcurveto{\pgfqpoint{4.201446in}{6.433466in}}{\pgfqpoint{4.203760in}{6.427880in}}{\pgfqpoint{4.207878in}{6.423762in}}%
\pgfpathcurveto{\pgfqpoint{4.211997in}{6.419644in}}{\pgfqpoint{4.217583in}{6.417330in}}{\pgfqpoint{4.223407in}{6.417330in}}%
\pgfpathlineto{\pgfqpoint{4.223407in}{6.417330in}}%
\pgfpathclose%
\pgfusepath{stroke,fill}%
\end{pgfscope}%
\begin{pgfscope}%
\pgfpathrectangle{\pgfqpoint{1.000000in}{1.148311in}}{\pgfqpoint{6.200000in}{5.623377in}}%
\pgfusepath{clip}%
\pgfsetbuttcap%
\pgfsetroundjoin%
\definecolor{currentfill}{rgb}{0.800000,0.800000,0.200000}%
\pgfsetfillcolor{currentfill}%
\pgfsetlinewidth{1.003750pt}%
\definecolor{currentstroke}{rgb}{0.800000,0.800000,0.200000}%
\pgfsetstrokecolor{currentstroke}%
\pgfsetdash{}{0pt}%
\pgfpathmoveto{\pgfqpoint{4.179069in}{6.342043in}}%
\pgfpathcurveto{\pgfqpoint{4.184893in}{6.342043in}}{\pgfqpoint{4.190480in}{6.344357in}}{\pgfqpoint{4.194598in}{6.348475in}}%
\pgfpathcurveto{\pgfqpoint{4.198716in}{6.352593in}}{\pgfqpoint{4.201030in}{6.358179in}}{\pgfqpoint{4.201030in}{6.364003in}}%
\pgfpathcurveto{\pgfqpoint{4.201030in}{6.369827in}}{\pgfqpoint{4.198716in}{6.375413in}}{\pgfqpoint{4.194598in}{6.379531in}}%
\pgfpathcurveto{\pgfqpoint{4.190480in}{6.383649in}}{\pgfqpoint{4.184893in}{6.385963in}}{\pgfqpoint{4.179069in}{6.385963in}}%
\pgfpathcurveto{\pgfqpoint{4.173246in}{6.385963in}}{\pgfqpoint{4.167659in}{6.383649in}}{\pgfqpoint{4.163541in}{6.379531in}}%
\pgfpathcurveto{\pgfqpoint{4.159423in}{6.375413in}}{\pgfqpoint{4.157109in}{6.369827in}}{\pgfqpoint{4.157109in}{6.364003in}}%
\pgfpathcurveto{\pgfqpoint{4.157109in}{6.358179in}}{\pgfqpoint{4.159423in}{6.352593in}}{\pgfqpoint{4.163541in}{6.348475in}}%
\pgfpathcurveto{\pgfqpoint{4.167659in}{6.344357in}}{\pgfqpoint{4.173246in}{6.342043in}}{\pgfqpoint{4.179069in}{6.342043in}}%
\pgfpathlineto{\pgfqpoint{4.179069in}{6.342043in}}%
\pgfpathclose%
\pgfusepath{stroke,fill}%
\end{pgfscope}%
\begin{pgfscope}%
\pgfpathrectangle{\pgfqpoint{1.000000in}{1.148311in}}{\pgfqpoint{6.200000in}{5.623377in}}%
\pgfusepath{clip}%
\pgfsetbuttcap%
\pgfsetroundjoin%
\definecolor{currentfill}{rgb}{0.800000,0.800000,0.200000}%
\pgfsetfillcolor{currentfill}%
\pgfsetlinewidth{1.003750pt}%
\definecolor{currentstroke}{rgb}{0.800000,0.800000,0.200000}%
\pgfsetstrokecolor{currentstroke}%
\pgfsetdash{}{0pt}%
\pgfpathmoveto{\pgfqpoint{4.092139in}{6.413382in}}%
\pgfpathcurveto{\pgfqpoint{4.097963in}{6.413382in}}{\pgfqpoint{4.103549in}{6.415696in}}{\pgfqpoint{4.107667in}{6.419814in}}%
\pgfpathcurveto{\pgfqpoint{4.111785in}{6.423932in}}{\pgfqpoint{4.114099in}{6.429518in}}{\pgfqpoint{4.114099in}{6.435342in}}%
\pgfpathcurveto{\pgfqpoint{4.114099in}{6.441166in}}{\pgfqpoint{4.111785in}{6.446752in}}{\pgfqpoint{4.107667in}{6.450870in}}%
\pgfpathcurveto{\pgfqpoint{4.103549in}{6.454988in}}{\pgfqpoint{4.097963in}{6.457302in}}{\pgfqpoint{4.092139in}{6.457302in}}%
\pgfpathcurveto{\pgfqpoint{4.086315in}{6.457302in}}{\pgfqpoint{4.080729in}{6.454988in}}{\pgfqpoint{4.076611in}{6.450870in}}%
\pgfpathcurveto{\pgfqpoint{4.072493in}{6.446752in}}{\pgfqpoint{4.070179in}{6.441166in}}{\pgfqpoint{4.070179in}{6.435342in}}%
\pgfpathcurveto{\pgfqpoint{4.070179in}{6.429518in}}{\pgfqpoint{4.072493in}{6.423932in}}{\pgfqpoint{4.076611in}{6.419814in}}%
\pgfpathcurveto{\pgfqpoint{4.080729in}{6.415696in}}{\pgfqpoint{4.086315in}{6.413382in}}{\pgfqpoint{4.092139in}{6.413382in}}%
\pgfpathlineto{\pgfqpoint{4.092139in}{6.413382in}}%
\pgfpathclose%
\pgfusepath{stroke,fill}%
\end{pgfscope}%
\begin{pgfscope}%
\pgfpathrectangle{\pgfqpoint{1.000000in}{1.148311in}}{\pgfqpoint{6.200000in}{5.623377in}}%
\pgfusepath{clip}%
\pgfsetbuttcap%
\pgfsetroundjoin%
\definecolor{currentfill}{rgb}{0.800000,0.800000,0.200000}%
\pgfsetfillcolor{currentfill}%
\pgfsetlinewidth{1.003750pt}%
\definecolor{currentstroke}{rgb}{0.800000,0.800000,0.200000}%
\pgfsetstrokecolor{currentstroke}%
\pgfsetdash{}{0pt}%
\pgfpathmoveto{\pgfqpoint{4.035359in}{6.381270in}}%
\pgfpathcurveto{\pgfqpoint{4.041183in}{6.381270in}}{\pgfqpoint{4.046770in}{6.383584in}}{\pgfqpoint{4.050888in}{6.387702in}}%
\pgfpathcurveto{\pgfqpoint{4.055006in}{6.391820in}}{\pgfqpoint{4.057320in}{6.397406in}}{\pgfqpoint{4.057320in}{6.403230in}}%
\pgfpathcurveto{\pgfqpoint{4.057320in}{6.409054in}}{\pgfqpoint{4.055006in}{6.414640in}}{\pgfqpoint{4.050888in}{6.418758in}}%
\pgfpathcurveto{\pgfqpoint{4.046770in}{6.422876in}}{\pgfqpoint{4.041183in}{6.425190in}}{\pgfqpoint{4.035359in}{6.425190in}}%
\pgfpathcurveto{\pgfqpoint{4.029536in}{6.425190in}}{\pgfqpoint{4.023949in}{6.422876in}}{\pgfqpoint{4.019831in}{6.418758in}}%
\pgfpathcurveto{\pgfqpoint{4.015713in}{6.414640in}}{\pgfqpoint{4.013399in}{6.409054in}}{\pgfqpoint{4.013399in}{6.403230in}}%
\pgfpathcurveto{\pgfqpoint{4.013399in}{6.397406in}}{\pgfqpoint{4.015713in}{6.391820in}}{\pgfqpoint{4.019831in}{6.387702in}}%
\pgfpathcurveto{\pgfqpoint{4.023949in}{6.383584in}}{\pgfqpoint{4.029536in}{6.381270in}}{\pgfqpoint{4.035359in}{6.381270in}}%
\pgfpathlineto{\pgfqpoint{4.035359in}{6.381270in}}%
\pgfpathclose%
\pgfusepath{stroke,fill}%
\end{pgfscope}%
\begin{pgfscope}%
\pgfpathrectangle{\pgfqpoint{1.000000in}{1.148311in}}{\pgfqpoint{6.200000in}{5.623377in}}%
\pgfusepath{clip}%
\pgfsetbuttcap%
\pgfsetroundjoin%
\definecolor{currentfill}{rgb}{0.800000,0.800000,0.200000}%
\pgfsetfillcolor{currentfill}%
\pgfsetlinewidth{1.003750pt}%
\definecolor{currentstroke}{rgb}{0.800000,0.800000,0.200000}%
\pgfsetstrokecolor{currentstroke}%
\pgfsetdash{}{0pt}%
\pgfpathmoveto{\pgfqpoint{3.938626in}{6.431967in}}%
\pgfpathcurveto{\pgfqpoint{3.944450in}{6.431967in}}{\pgfqpoint{3.950036in}{6.434280in}}{\pgfqpoint{3.954154in}{6.438399in}}%
\pgfpathcurveto{\pgfqpoint{3.958273in}{6.442517in}}{\pgfqpoint{3.960586in}{6.448103in}}{\pgfqpoint{3.960586in}{6.453927in}}%
\pgfpathcurveto{\pgfqpoint{3.960586in}{6.459751in}}{\pgfqpoint{3.958273in}{6.465337in}}{\pgfqpoint{3.954154in}{6.469455in}}%
\pgfpathcurveto{\pgfqpoint{3.950036in}{6.473573in}}{\pgfqpoint{3.944450in}{6.475887in}}{\pgfqpoint{3.938626in}{6.475887in}}%
\pgfpathcurveto{\pgfqpoint{3.932802in}{6.475887in}}{\pgfqpoint{3.927216in}{6.473573in}}{\pgfqpoint{3.923098in}{6.469455in}}%
\pgfpathcurveto{\pgfqpoint{3.918980in}{6.465337in}}{\pgfqpoint{3.916666in}{6.459751in}}{\pgfqpoint{3.916666in}{6.453927in}}%
\pgfpathcurveto{\pgfqpoint{3.916666in}{6.448103in}}{\pgfqpoint{3.918980in}{6.442517in}}{\pgfqpoint{3.923098in}{6.438399in}}%
\pgfpathcurveto{\pgfqpoint{3.927216in}{6.434280in}}{\pgfqpoint{3.932802in}{6.431967in}}{\pgfqpoint{3.938626in}{6.431967in}}%
\pgfpathlineto{\pgfqpoint{3.938626in}{6.431967in}}%
\pgfpathclose%
\pgfusepath{stroke,fill}%
\end{pgfscope}%
\begin{pgfscope}%
\pgfpathrectangle{\pgfqpoint{1.000000in}{1.148311in}}{\pgfqpoint{6.200000in}{5.623377in}}%
\pgfusepath{clip}%
\pgfsetbuttcap%
\pgfsetroundjoin%
\definecolor{currentfill}{rgb}{0.800000,0.800000,0.200000}%
\pgfsetfillcolor{currentfill}%
\pgfsetlinewidth{1.003750pt}%
\definecolor{currentstroke}{rgb}{0.800000,0.800000,0.200000}%
\pgfsetstrokecolor{currentstroke}%
\pgfsetdash{}{0pt}%
\pgfpathmoveto{\pgfqpoint{3.925515in}{6.315404in}}%
\pgfpathcurveto{\pgfqpoint{3.931339in}{6.315404in}}{\pgfqpoint{3.936925in}{6.317718in}}{\pgfqpoint{3.941043in}{6.321836in}}%
\pgfpathcurveto{\pgfqpoint{3.945162in}{6.325954in}}{\pgfqpoint{3.947475in}{6.331540in}}{\pgfqpoint{3.947475in}{6.337364in}}%
\pgfpathcurveto{\pgfqpoint{3.947475in}{6.343188in}}{\pgfqpoint{3.945162in}{6.348774in}}{\pgfqpoint{3.941043in}{6.352892in}}%
\pgfpathcurveto{\pgfqpoint{3.936925in}{6.357010in}}{\pgfqpoint{3.931339in}{6.359324in}}{\pgfqpoint{3.925515in}{6.359324in}}%
\pgfpathcurveto{\pgfqpoint{3.919691in}{6.359324in}}{\pgfqpoint{3.914105in}{6.357010in}}{\pgfqpoint{3.909987in}{6.352892in}}%
\pgfpathcurveto{\pgfqpoint{3.905869in}{6.348774in}}{\pgfqpoint{3.903555in}{6.343188in}}{\pgfqpoint{3.903555in}{6.337364in}}%
\pgfpathcurveto{\pgfqpoint{3.903555in}{6.331540in}}{\pgfqpoint{3.905869in}{6.325954in}}{\pgfqpoint{3.909987in}{6.321836in}}%
\pgfpathcurveto{\pgfqpoint{3.914105in}{6.317718in}}{\pgfqpoint{3.919691in}{6.315404in}}{\pgfqpoint{3.925515in}{6.315404in}}%
\pgfpathlineto{\pgfqpoint{3.925515in}{6.315404in}}%
\pgfpathclose%
\pgfusepath{stroke,fill}%
\end{pgfscope}%
\begin{pgfscope}%
\pgfpathrectangle{\pgfqpoint{1.000000in}{1.148311in}}{\pgfqpoint{6.200000in}{5.623377in}}%
\pgfusepath{clip}%
\pgfsetbuttcap%
\pgfsetroundjoin%
\definecolor{currentfill}{rgb}{0.800000,0.800000,0.200000}%
\pgfsetfillcolor{currentfill}%
\pgfsetlinewidth{1.003750pt}%
\definecolor{currentstroke}{rgb}{0.800000,0.800000,0.200000}%
\pgfsetstrokecolor{currentstroke}%
\pgfsetdash{}{0pt}%
\pgfpathmoveto{\pgfqpoint{3.829290in}{6.345382in}}%
\pgfpathcurveto{\pgfqpoint{3.835114in}{6.345382in}}{\pgfqpoint{3.840700in}{6.347696in}}{\pgfqpoint{3.844818in}{6.351814in}}%
\pgfpathcurveto{\pgfqpoint{3.848936in}{6.355932in}}{\pgfqpoint{3.851250in}{6.361518in}}{\pgfqpoint{3.851250in}{6.367342in}}%
\pgfpathcurveto{\pgfqpoint{3.851250in}{6.373166in}}{\pgfqpoint{3.848936in}{6.378752in}}{\pgfqpoint{3.844818in}{6.382870in}}%
\pgfpathcurveto{\pgfqpoint{3.840700in}{6.386988in}}{\pgfqpoint{3.835114in}{6.389302in}}{\pgfqpoint{3.829290in}{6.389302in}}%
\pgfpathcurveto{\pgfqpoint{3.823466in}{6.389302in}}{\pgfqpoint{3.817880in}{6.386988in}}{\pgfqpoint{3.813761in}{6.382870in}}%
\pgfpathcurveto{\pgfqpoint{3.809643in}{6.378752in}}{\pgfqpoint{3.807329in}{6.373166in}}{\pgfqpoint{3.807329in}{6.367342in}}%
\pgfpathcurveto{\pgfqpoint{3.807329in}{6.361518in}}{\pgfqpoint{3.809643in}{6.355932in}}{\pgfqpoint{3.813761in}{6.351814in}}%
\pgfpathcurveto{\pgfqpoint{3.817880in}{6.347696in}}{\pgfqpoint{3.823466in}{6.345382in}}{\pgfqpoint{3.829290in}{6.345382in}}%
\pgfpathlineto{\pgfqpoint{3.829290in}{6.345382in}}%
\pgfpathclose%
\pgfusepath{stroke,fill}%
\end{pgfscope}%
\begin{pgfscope}%
\pgfpathrectangle{\pgfqpoint{1.000000in}{1.148311in}}{\pgfqpoint{6.200000in}{5.623377in}}%
\pgfusepath{clip}%
\pgfsetbuttcap%
\pgfsetroundjoin%
\definecolor{currentfill}{rgb}{0.800000,0.800000,0.200000}%
\pgfsetfillcolor{currentfill}%
\pgfsetlinewidth{1.003750pt}%
\definecolor{currentstroke}{rgb}{0.800000,0.800000,0.200000}%
\pgfsetstrokecolor{currentstroke}%
\pgfsetdash{}{0pt}%
\pgfpathmoveto{\pgfqpoint{3.800831in}{6.269924in}}%
\pgfpathcurveto{\pgfqpoint{3.806655in}{6.269924in}}{\pgfqpoint{3.812241in}{6.272238in}}{\pgfqpoint{3.816359in}{6.276356in}}%
\pgfpathcurveto{\pgfqpoint{3.820477in}{6.280474in}}{\pgfqpoint{3.822791in}{6.286060in}}{\pgfqpoint{3.822791in}{6.291884in}}%
\pgfpathcurveto{\pgfqpoint{3.822791in}{6.297708in}}{\pgfqpoint{3.820477in}{6.303294in}}{\pgfqpoint{3.816359in}{6.307413in}}%
\pgfpathcurveto{\pgfqpoint{3.812241in}{6.311531in}}{\pgfqpoint{3.806655in}{6.313845in}}{\pgfqpoint{3.800831in}{6.313845in}}%
\pgfpathcurveto{\pgfqpoint{3.795007in}{6.313845in}}{\pgfqpoint{3.789420in}{6.311531in}}{\pgfqpoint{3.785302in}{6.307413in}}%
\pgfpathcurveto{\pgfqpoint{3.781184in}{6.303294in}}{\pgfqpoint{3.778870in}{6.297708in}}{\pgfqpoint{3.778870in}{6.291884in}}%
\pgfpathcurveto{\pgfqpoint{3.778870in}{6.286060in}}{\pgfqpoint{3.781184in}{6.280474in}}{\pgfqpoint{3.785302in}{6.276356in}}%
\pgfpathcurveto{\pgfqpoint{3.789420in}{6.272238in}}{\pgfqpoint{3.795007in}{6.269924in}}{\pgfqpoint{3.800831in}{6.269924in}}%
\pgfpathlineto{\pgfqpoint{3.800831in}{6.269924in}}%
\pgfpathclose%
\pgfusepath{stroke,fill}%
\end{pgfscope}%
\begin{pgfscope}%
\pgfpathrectangle{\pgfqpoint{1.000000in}{1.148311in}}{\pgfqpoint{6.200000in}{5.623377in}}%
\pgfusepath{clip}%
\pgfsetbuttcap%
\pgfsetroundjoin%
\definecolor{currentfill}{rgb}{0.800000,0.800000,0.200000}%
\pgfsetfillcolor{currentfill}%
\pgfsetlinewidth{1.003750pt}%
\definecolor{currentstroke}{rgb}{0.800000,0.800000,0.200000}%
\pgfsetstrokecolor{currentstroke}%
\pgfsetdash{}{0pt}%
\pgfpathmoveto{\pgfqpoint{3.824249in}{6.142889in}}%
\pgfpathcurveto{\pgfqpoint{3.830073in}{6.142889in}}{\pgfqpoint{3.835659in}{6.145203in}}{\pgfqpoint{3.839777in}{6.149321in}}%
\pgfpathcurveto{\pgfqpoint{3.843896in}{6.153439in}}{\pgfqpoint{3.846209in}{6.159025in}}{\pgfqpoint{3.846209in}{6.164849in}}%
\pgfpathcurveto{\pgfqpoint{3.846209in}{6.170673in}}{\pgfqpoint{3.843896in}{6.176259in}}{\pgfqpoint{3.839777in}{6.180377in}}%
\pgfpathcurveto{\pgfqpoint{3.835659in}{6.184495in}}{\pgfqpoint{3.830073in}{6.186809in}}{\pgfqpoint{3.824249in}{6.186809in}}%
\pgfpathcurveto{\pgfqpoint{3.818425in}{6.186809in}}{\pgfqpoint{3.812839in}{6.184495in}}{\pgfqpoint{3.808721in}{6.180377in}}%
\pgfpathcurveto{\pgfqpoint{3.804603in}{6.176259in}}{\pgfqpoint{3.802289in}{6.170673in}}{\pgfqpoint{3.802289in}{6.164849in}}%
\pgfpathcurveto{\pgfqpoint{3.802289in}{6.159025in}}{\pgfqpoint{3.804603in}{6.153439in}}{\pgfqpoint{3.808721in}{6.149321in}}%
\pgfpathcurveto{\pgfqpoint{3.812839in}{6.145203in}}{\pgfqpoint{3.818425in}{6.142889in}}{\pgfqpoint{3.824249in}{6.142889in}}%
\pgfpathlineto{\pgfqpoint{3.824249in}{6.142889in}}%
\pgfpathclose%
\pgfusepath{stroke,fill}%
\end{pgfscope}%
\begin{pgfscope}%
\pgfpathrectangle{\pgfqpoint{1.000000in}{1.148311in}}{\pgfqpoint{6.200000in}{5.623377in}}%
\pgfusepath{clip}%
\pgfsetbuttcap%
\pgfsetroundjoin%
\definecolor{currentfill}{rgb}{0.800000,0.800000,0.200000}%
\pgfsetfillcolor{currentfill}%
\pgfsetlinewidth{1.003750pt}%
\definecolor{currentstroke}{rgb}{0.800000,0.800000,0.200000}%
\pgfsetstrokecolor{currentstroke}%
\pgfsetdash{}{0pt}%
\pgfpathmoveto{\pgfqpoint{3.728301in}{6.157264in}}%
\pgfpathcurveto{\pgfqpoint{3.734124in}{6.157264in}}{\pgfqpoint{3.739711in}{6.159578in}}{\pgfqpoint{3.743829in}{6.163696in}}%
\pgfpathcurveto{\pgfqpoint{3.747947in}{6.167814in}}{\pgfqpoint{3.750261in}{6.173400in}}{\pgfqpoint{3.750261in}{6.179224in}}%
\pgfpathcurveto{\pgfqpoint{3.750261in}{6.185048in}}{\pgfqpoint{3.747947in}{6.190634in}}{\pgfqpoint{3.743829in}{6.194752in}}%
\pgfpathcurveto{\pgfqpoint{3.739711in}{6.198871in}}{\pgfqpoint{3.734124in}{6.201184in}}{\pgfqpoint{3.728301in}{6.201184in}}%
\pgfpathcurveto{\pgfqpoint{3.722477in}{6.201184in}}{\pgfqpoint{3.716890in}{6.198871in}}{\pgfqpoint{3.712772in}{6.194752in}}%
\pgfpathcurveto{\pgfqpoint{3.708654in}{6.190634in}}{\pgfqpoint{3.706340in}{6.185048in}}{\pgfqpoint{3.706340in}{6.179224in}}%
\pgfpathcurveto{\pgfqpoint{3.706340in}{6.173400in}}{\pgfqpoint{3.708654in}{6.167814in}}{\pgfqpoint{3.712772in}{6.163696in}}%
\pgfpathcurveto{\pgfqpoint{3.716890in}{6.159578in}}{\pgfqpoint{3.722477in}{6.157264in}}{\pgfqpoint{3.728301in}{6.157264in}}%
\pgfpathlineto{\pgfqpoint{3.728301in}{6.157264in}}%
\pgfpathclose%
\pgfusepath{stroke,fill}%
\end{pgfscope}%
\begin{pgfscope}%
\pgfpathrectangle{\pgfqpoint{1.000000in}{1.148311in}}{\pgfqpoint{6.200000in}{5.623377in}}%
\pgfusepath{clip}%
\pgfsetbuttcap%
\pgfsetroundjoin%
\definecolor{currentfill}{rgb}{0.800000,0.800000,0.200000}%
\pgfsetfillcolor{currentfill}%
\pgfsetlinewidth{1.003750pt}%
\definecolor{currentstroke}{rgb}{0.800000,0.800000,0.200000}%
\pgfsetstrokecolor{currentstroke}%
\pgfsetdash{}{0pt}%
\pgfpathmoveto{\pgfqpoint{3.687532in}{6.109071in}}%
\pgfpathcurveto{\pgfqpoint{3.693356in}{6.109071in}}{\pgfqpoint{3.698942in}{6.111385in}}{\pgfqpoint{3.703060in}{6.115503in}}%
\pgfpathcurveto{\pgfqpoint{3.707179in}{6.119621in}}{\pgfqpoint{3.709492in}{6.125208in}}{\pgfqpoint{3.709492in}{6.131032in}}%
\pgfpathcurveto{\pgfqpoint{3.709492in}{6.136855in}}{\pgfqpoint{3.707179in}{6.142442in}}{\pgfqpoint{3.703060in}{6.146560in}}%
\pgfpathcurveto{\pgfqpoint{3.698942in}{6.150678in}}{\pgfqpoint{3.693356in}{6.152992in}}{\pgfqpoint{3.687532in}{6.152992in}}%
\pgfpathcurveto{\pgfqpoint{3.681708in}{6.152992in}}{\pgfqpoint{3.676122in}{6.150678in}}{\pgfqpoint{3.672004in}{6.146560in}}%
\pgfpathcurveto{\pgfqpoint{3.667886in}{6.142442in}}{\pgfqpoint{3.665572in}{6.136855in}}{\pgfqpoint{3.665572in}{6.131032in}}%
\pgfpathcurveto{\pgfqpoint{3.665572in}{6.125208in}}{\pgfqpoint{3.667886in}{6.119621in}}{\pgfqpoint{3.672004in}{6.115503in}}%
\pgfpathcurveto{\pgfqpoint{3.676122in}{6.111385in}}{\pgfqpoint{3.681708in}{6.109071in}}{\pgfqpoint{3.687532in}{6.109071in}}%
\pgfpathlineto{\pgfqpoint{3.687532in}{6.109071in}}%
\pgfpathclose%
\pgfusepath{stroke,fill}%
\end{pgfscope}%
\begin{pgfscope}%
\pgfpathrectangle{\pgfqpoint{1.000000in}{1.148311in}}{\pgfqpoint{6.200000in}{5.623377in}}%
\pgfusepath{clip}%
\pgfsetbuttcap%
\pgfsetroundjoin%
\definecolor{currentfill}{rgb}{0.800000,0.800000,0.200000}%
\pgfsetfillcolor{currentfill}%
\pgfsetlinewidth{1.003750pt}%
\definecolor{currentstroke}{rgb}{0.800000,0.800000,0.200000}%
\pgfsetstrokecolor{currentstroke}%
\pgfsetdash{}{0pt}%
\pgfpathmoveto{\pgfqpoint{3.647877in}{6.060285in}}%
\pgfpathcurveto{\pgfqpoint{3.653701in}{6.060285in}}{\pgfqpoint{3.659288in}{6.062599in}}{\pgfqpoint{3.663406in}{6.066717in}}%
\pgfpathcurveto{\pgfqpoint{3.667524in}{6.070835in}}{\pgfqpoint{3.669838in}{6.076422in}}{\pgfqpoint{3.669838in}{6.082245in}}%
\pgfpathcurveto{\pgfqpoint{3.669838in}{6.088069in}}{\pgfqpoint{3.667524in}{6.093656in}}{\pgfqpoint{3.663406in}{6.097774in}}%
\pgfpathcurveto{\pgfqpoint{3.659288in}{6.101892in}}{\pgfqpoint{3.653701in}{6.104206in}}{\pgfqpoint{3.647877in}{6.104206in}}%
\pgfpathcurveto{\pgfqpoint{3.642054in}{6.104206in}}{\pgfqpoint{3.636467in}{6.101892in}}{\pgfqpoint{3.632349in}{6.097774in}}%
\pgfpathcurveto{\pgfqpoint{3.628231in}{6.093656in}}{\pgfqpoint{3.625917in}{6.088069in}}{\pgfqpoint{3.625917in}{6.082245in}}%
\pgfpathcurveto{\pgfqpoint{3.625917in}{6.076422in}}{\pgfqpoint{3.628231in}{6.070835in}}{\pgfqpoint{3.632349in}{6.066717in}}%
\pgfpathcurveto{\pgfqpoint{3.636467in}{6.062599in}}{\pgfqpoint{3.642054in}{6.060285in}}{\pgfqpoint{3.647877in}{6.060285in}}%
\pgfpathlineto{\pgfqpoint{3.647877in}{6.060285in}}%
\pgfpathclose%
\pgfusepath{stroke,fill}%
\end{pgfscope}%
\begin{pgfscope}%
\pgfpathrectangle{\pgfqpoint{1.000000in}{1.148311in}}{\pgfqpoint{6.200000in}{5.623377in}}%
\pgfusepath{clip}%
\pgfsetbuttcap%
\pgfsetroundjoin%
\definecolor{currentfill}{rgb}{0.800000,0.800000,0.200000}%
\pgfsetfillcolor{currentfill}%
\pgfsetlinewidth{1.003750pt}%
\definecolor{currentstroke}{rgb}{0.800000,0.800000,0.200000}%
\pgfsetstrokecolor{currentstroke}%
\pgfsetdash{}{0pt}%
\pgfpathmoveto{\pgfqpoint{3.635898in}{5.992290in}}%
\pgfpathcurveto{\pgfqpoint{3.641722in}{5.992290in}}{\pgfqpoint{3.647308in}{5.994604in}}{\pgfqpoint{3.651426in}{5.998722in}}%
\pgfpathcurveto{\pgfqpoint{3.655544in}{6.002841in}}{\pgfqpoint{3.657858in}{6.008427in}}{\pgfqpoint{3.657858in}{6.014251in}}%
\pgfpathcurveto{\pgfqpoint{3.657858in}{6.020075in}}{\pgfqpoint{3.655544in}{6.025661in}}{\pgfqpoint{3.651426in}{6.029779in}}%
\pgfpathcurveto{\pgfqpoint{3.647308in}{6.033897in}}{\pgfqpoint{3.641722in}{6.036211in}}{\pgfqpoint{3.635898in}{6.036211in}}%
\pgfpathcurveto{\pgfqpoint{3.630074in}{6.036211in}}{\pgfqpoint{3.624488in}{6.033897in}}{\pgfqpoint{3.620369in}{6.029779in}}%
\pgfpathcurveto{\pgfqpoint{3.616251in}{6.025661in}}{\pgfqpoint{3.613937in}{6.020075in}}{\pgfqpoint{3.613937in}{6.014251in}}%
\pgfpathcurveto{\pgfqpoint{3.613937in}{6.008427in}}{\pgfqpoint{3.616251in}{6.002841in}}{\pgfqpoint{3.620369in}{5.998722in}}%
\pgfpathcurveto{\pgfqpoint{3.624488in}{5.994604in}}{\pgfqpoint{3.630074in}{5.992290in}}{\pgfqpoint{3.635898in}{5.992290in}}%
\pgfpathlineto{\pgfqpoint{3.635898in}{5.992290in}}%
\pgfpathclose%
\pgfusepath{stroke,fill}%
\end{pgfscope}%
\begin{pgfscope}%
\pgfpathrectangle{\pgfqpoint{1.000000in}{1.148311in}}{\pgfqpoint{6.200000in}{5.623377in}}%
\pgfusepath{clip}%
\pgfsetbuttcap%
\pgfsetroundjoin%
\definecolor{currentfill}{rgb}{0.800000,0.800000,0.200000}%
\pgfsetfillcolor{currentfill}%
\pgfsetlinewidth{1.003750pt}%
\definecolor{currentstroke}{rgb}{0.800000,0.800000,0.200000}%
\pgfsetstrokecolor{currentstroke}%
\pgfsetdash{}{0pt}%
\pgfpathmoveto{\pgfqpoint{3.546399in}{5.974948in}}%
\pgfpathcurveto{\pgfqpoint{3.552223in}{5.974948in}}{\pgfqpoint{3.557809in}{5.977262in}}{\pgfqpoint{3.561928in}{5.981380in}}%
\pgfpathcurveto{\pgfqpoint{3.566046in}{5.985498in}}{\pgfqpoint{3.568360in}{5.991084in}}{\pgfqpoint{3.568360in}{5.996908in}}%
\pgfpathcurveto{\pgfqpoint{3.568360in}{6.002732in}}{\pgfqpoint{3.566046in}{6.008318in}}{\pgfqpoint{3.561928in}{6.012436in}}%
\pgfpathcurveto{\pgfqpoint{3.557809in}{6.016554in}}{\pgfqpoint{3.552223in}{6.018868in}}{\pgfqpoint{3.546399in}{6.018868in}}%
\pgfpathcurveto{\pgfqpoint{3.540575in}{6.018868in}}{\pgfqpoint{3.534989in}{6.016554in}}{\pgfqpoint{3.530871in}{6.012436in}}%
\pgfpathcurveto{\pgfqpoint{3.526753in}{6.008318in}}{\pgfqpoint{3.524439in}{6.002732in}}{\pgfqpoint{3.524439in}{5.996908in}}%
\pgfpathcurveto{\pgfqpoint{3.524439in}{5.991084in}}{\pgfqpoint{3.526753in}{5.985498in}}{\pgfqpoint{3.530871in}{5.981380in}}%
\pgfpathcurveto{\pgfqpoint{3.534989in}{5.977262in}}{\pgfqpoint{3.540575in}{5.974948in}}{\pgfqpoint{3.546399in}{5.974948in}}%
\pgfpathlineto{\pgfqpoint{3.546399in}{5.974948in}}%
\pgfpathclose%
\pgfusepath{stroke,fill}%
\end{pgfscope}%
\begin{pgfscope}%
\pgfpathrectangle{\pgfqpoint{1.000000in}{1.148311in}}{\pgfqpoint{6.200000in}{5.623377in}}%
\pgfusepath{clip}%
\pgfsetbuttcap%
\pgfsetroundjoin%
\definecolor{currentfill}{rgb}{0.800000,0.800000,0.200000}%
\pgfsetfillcolor{currentfill}%
\pgfsetlinewidth{1.003750pt}%
\definecolor{currentstroke}{rgb}{0.800000,0.800000,0.200000}%
\pgfsetstrokecolor{currentstroke}%
\pgfsetdash{}{0pt}%
\pgfpathmoveto{\pgfqpoint{3.555084in}{5.897301in}}%
\pgfpathcurveto{\pgfqpoint{3.560908in}{5.897301in}}{\pgfqpoint{3.566494in}{5.899615in}}{\pgfqpoint{3.570612in}{5.903733in}}%
\pgfpathcurveto{\pgfqpoint{3.574730in}{5.907851in}}{\pgfqpoint{3.577044in}{5.913437in}}{\pgfqpoint{3.577044in}{5.919261in}}%
\pgfpathcurveto{\pgfqpoint{3.577044in}{5.925085in}}{\pgfqpoint{3.574730in}{5.930671in}}{\pgfqpoint{3.570612in}{5.934789in}}%
\pgfpathcurveto{\pgfqpoint{3.566494in}{5.938908in}}{\pgfqpoint{3.560908in}{5.941221in}}{\pgfqpoint{3.555084in}{5.941221in}}%
\pgfpathcurveto{\pgfqpoint{3.549260in}{5.941221in}}{\pgfqpoint{3.543674in}{5.938908in}}{\pgfqpoint{3.539556in}{5.934789in}}%
\pgfpathcurveto{\pgfqpoint{3.535438in}{5.930671in}}{\pgfqpoint{3.533124in}{5.925085in}}{\pgfqpoint{3.533124in}{5.919261in}}%
\pgfpathcurveto{\pgfqpoint{3.533124in}{5.913437in}}{\pgfqpoint{3.535438in}{5.907851in}}{\pgfqpoint{3.539556in}{5.903733in}}%
\pgfpathcurveto{\pgfqpoint{3.543674in}{5.899615in}}{\pgfqpoint{3.549260in}{5.897301in}}{\pgfqpoint{3.555084in}{5.897301in}}%
\pgfpathlineto{\pgfqpoint{3.555084in}{5.897301in}}%
\pgfpathclose%
\pgfusepath{stroke,fill}%
\end{pgfscope}%
\begin{pgfscope}%
\pgfpathrectangle{\pgfqpoint{1.000000in}{1.148311in}}{\pgfqpoint{6.200000in}{5.623377in}}%
\pgfusepath{clip}%
\pgfsetbuttcap%
\pgfsetroundjoin%
\definecolor{currentfill}{rgb}{0.800000,0.800000,0.200000}%
\pgfsetfillcolor{currentfill}%
\pgfsetlinewidth{1.003750pt}%
\definecolor{currentstroke}{rgb}{0.800000,0.800000,0.200000}%
\pgfsetstrokecolor{currentstroke}%
\pgfsetdash{}{0pt}%
\pgfpathmoveto{\pgfqpoint{3.473237in}{5.865270in}}%
\pgfpathcurveto{\pgfqpoint{3.479060in}{5.865270in}}{\pgfqpoint{3.484647in}{5.867584in}}{\pgfqpoint{3.488765in}{5.871703in}}%
\pgfpathcurveto{\pgfqpoint{3.492883in}{5.875821in}}{\pgfqpoint{3.495197in}{5.881407in}}{\pgfqpoint{3.495197in}{5.887231in}}%
\pgfpathcurveto{\pgfqpoint{3.495197in}{5.893055in}}{\pgfqpoint{3.492883in}{5.898641in}}{\pgfqpoint{3.488765in}{5.902759in}}%
\pgfpathcurveto{\pgfqpoint{3.484647in}{5.906877in}}{\pgfqpoint{3.479060in}{5.909191in}}{\pgfqpoint{3.473237in}{5.909191in}}%
\pgfpathcurveto{\pgfqpoint{3.467413in}{5.909191in}}{\pgfqpoint{3.461826in}{5.906877in}}{\pgfqpoint{3.457708in}{5.902759in}}%
\pgfpathcurveto{\pgfqpoint{3.453590in}{5.898641in}}{\pgfqpoint{3.451276in}{5.893055in}}{\pgfqpoint{3.451276in}{5.887231in}}%
\pgfpathcurveto{\pgfqpoint{3.451276in}{5.881407in}}{\pgfqpoint{3.453590in}{5.875821in}}{\pgfqpoint{3.457708in}{5.871703in}}%
\pgfpathcurveto{\pgfqpoint{3.461826in}{5.867584in}}{\pgfqpoint{3.467413in}{5.865270in}}{\pgfqpoint{3.473237in}{5.865270in}}%
\pgfpathlineto{\pgfqpoint{3.473237in}{5.865270in}}%
\pgfpathclose%
\pgfusepath{stroke,fill}%
\end{pgfscope}%
\begin{pgfscope}%
\pgfpathrectangle{\pgfqpoint{1.000000in}{1.148311in}}{\pgfqpoint{6.200000in}{5.623377in}}%
\pgfusepath{clip}%
\pgfsetbuttcap%
\pgfsetroundjoin%
\definecolor{currentfill}{rgb}{0.800000,0.800000,0.200000}%
\pgfsetfillcolor{currentfill}%
\pgfsetlinewidth{1.003750pt}%
\definecolor{currentstroke}{rgb}{0.800000,0.800000,0.200000}%
\pgfsetstrokecolor{currentstroke}%
\pgfsetdash{}{0pt}%
\pgfpathmoveto{\pgfqpoint{3.454169in}{5.802002in}}%
\pgfpathcurveto{\pgfqpoint{3.459993in}{5.802002in}}{\pgfqpoint{3.465579in}{5.804316in}}{\pgfqpoint{3.469697in}{5.808434in}}%
\pgfpathcurveto{\pgfqpoint{3.473815in}{5.812552in}}{\pgfqpoint{3.476129in}{5.818138in}}{\pgfqpoint{3.476129in}{5.823962in}}%
\pgfpathcurveto{\pgfqpoint{3.476129in}{5.829786in}}{\pgfqpoint{3.473815in}{5.835372in}}{\pgfqpoint{3.469697in}{5.839490in}}%
\pgfpathcurveto{\pgfqpoint{3.465579in}{5.843609in}}{\pgfqpoint{3.459993in}{5.845922in}}{\pgfqpoint{3.454169in}{5.845922in}}%
\pgfpathcurveto{\pgfqpoint{3.448345in}{5.845922in}}{\pgfqpoint{3.442759in}{5.843609in}}{\pgfqpoint{3.438641in}{5.839490in}}%
\pgfpathcurveto{\pgfqpoint{3.434523in}{5.835372in}}{\pgfqpoint{3.432209in}{5.829786in}}{\pgfqpoint{3.432209in}{5.823962in}}%
\pgfpathcurveto{\pgfqpoint{3.432209in}{5.818138in}}{\pgfqpoint{3.434523in}{5.812552in}}{\pgfqpoint{3.438641in}{5.808434in}}%
\pgfpathcurveto{\pgfqpoint{3.442759in}{5.804316in}}{\pgfqpoint{3.448345in}{5.802002in}}{\pgfqpoint{3.454169in}{5.802002in}}%
\pgfpathlineto{\pgfqpoint{3.454169in}{5.802002in}}%
\pgfpathclose%
\pgfusepath{stroke,fill}%
\end{pgfscope}%
\begin{pgfscope}%
\pgfpathrectangle{\pgfqpoint{1.000000in}{1.148311in}}{\pgfqpoint{6.200000in}{5.623377in}}%
\pgfusepath{clip}%
\pgfsetbuttcap%
\pgfsetroundjoin%
\definecolor{currentfill}{rgb}{0.800000,0.800000,0.200000}%
\pgfsetfillcolor{currentfill}%
\pgfsetlinewidth{1.003750pt}%
\definecolor{currentstroke}{rgb}{0.800000,0.800000,0.200000}%
\pgfsetstrokecolor{currentstroke}%
\pgfsetdash{}{0pt}%
\pgfpathmoveto{\pgfqpoint{3.481834in}{5.725504in}}%
\pgfpathcurveto{\pgfqpoint{3.487658in}{5.725504in}}{\pgfqpoint{3.493244in}{5.727817in}}{\pgfqpoint{3.497362in}{5.731936in}}%
\pgfpathcurveto{\pgfqpoint{3.501480in}{5.736054in}}{\pgfqpoint{3.503794in}{5.741640in}}{\pgfqpoint{3.503794in}{5.747464in}}%
\pgfpathcurveto{\pgfqpoint{3.503794in}{5.753288in}}{\pgfqpoint{3.501480in}{5.758874in}}{\pgfqpoint{3.497362in}{5.762992in}}%
\pgfpathcurveto{\pgfqpoint{3.493244in}{5.767110in}}{\pgfqpoint{3.487658in}{5.769424in}}{\pgfqpoint{3.481834in}{5.769424in}}%
\pgfpathcurveto{\pgfqpoint{3.476010in}{5.769424in}}{\pgfqpoint{3.470424in}{5.767110in}}{\pgfqpoint{3.466305in}{5.762992in}}%
\pgfpathcurveto{\pgfqpoint{3.462187in}{5.758874in}}{\pgfqpoint{3.459873in}{5.753288in}}{\pgfqpoint{3.459873in}{5.747464in}}%
\pgfpathcurveto{\pgfqpoint{3.459873in}{5.741640in}}{\pgfqpoint{3.462187in}{5.736054in}}{\pgfqpoint{3.466305in}{5.731936in}}%
\pgfpathcurveto{\pgfqpoint{3.470424in}{5.727817in}}{\pgfqpoint{3.476010in}{5.725504in}}{\pgfqpoint{3.481834in}{5.725504in}}%
\pgfpathlineto{\pgfqpoint{3.481834in}{5.725504in}}%
\pgfpathclose%
\pgfusepath{stroke,fill}%
\end{pgfscope}%
\begin{pgfscope}%
\pgfpathrectangle{\pgfqpoint{1.000000in}{1.148311in}}{\pgfqpoint{6.200000in}{5.623377in}}%
\pgfusepath{clip}%
\pgfsetbuttcap%
\pgfsetroundjoin%
\definecolor{currentfill}{rgb}{0.800000,0.800000,0.200000}%
\pgfsetfillcolor{currentfill}%
\pgfsetlinewidth{1.003750pt}%
\definecolor{currentstroke}{rgb}{0.800000,0.800000,0.200000}%
\pgfsetstrokecolor{currentstroke}%
\pgfsetdash{}{0pt}%
\pgfpathmoveto{\pgfqpoint{3.480348in}{5.661844in}}%
\pgfpathcurveto{\pgfqpoint{3.486172in}{5.661844in}}{\pgfqpoint{3.491758in}{5.664158in}}{\pgfqpoint{3.495876in}{5.668276in}}%
\pgfpathcurveto{\pgfqpoint{3.499994in}{5.672394in}}{\pgfqpoint{3.502308in}{5.677980in}}{\pgfqpoint{3.502308in}{5.683804in}}%
\pgfpathcurveto{\pgfqpoint{3.502308in}{5.689628in}}{\pgfqpoint{3.499994in}{5.695214in}}{\pgfqpoint{3.495876in}{5.699332in}}%
\pgfpathcurveto{\pgfqpoint{3.491758in}{5.703450in}}{\pgfqpoint{3.486172in}{5.705764in}}{\pgfqpoint{3.480348in}{5.705764in}}%
\pgfpathcurveto{\pgfqpoint{3.474524in}{5.705764in}}{\pgfqpoint{3.468938in}{5.703450in}}{\pgfqpoint{3.464820in}{5.699332in}}%
\pgfpathcurveto{\pgfqpoint{3.460701in}{5.695214in}}{\pgfqpoint{3.458388in}{5.689628in}}{\pgfqpoint{3.458388in}{5.683804in}}%
\pgfpathcurveto{\pgfqpoint{3.458388in}{5.677980in}}{\pgfqpoint{3.460701in}{5.672394in}}{\pgfqpoint{3.464820in}{5.668276in}}%
\pgfpathcurveto{\pgfqpoint{3.468938in}{5.664158in}}{\pgfqpoint{3.474524in}{5.661844in}}{\pgfqpoint{3.480348in}{5.661844in}}%
\pgfpathlineto{\pgfqpoint{3.480348in}{5.661844in}}%
\pgfpathclose%
\pgfusepath{stroke,fill}%
\end{pgfscope}%
\begin{pgfscope}%
\pgfpathrectangle{\pgfqpoint{1.000000in}{1.148311in}}{\pgfqpoint{6.200000in}{5.623377in}}%
\pgfusepath{clip}%
\pgfsetbuttcap%
\pgfsetroundjoin%
\definecolor{currentfill}{rgb}{0.800000,0.800000,0.200000}%
\pgfsetfillcolor{currentfill}%
\pgfsetlinewidth{1.003750pt}%
\definecolor{currentstroke}{rgb}{0.800000,0.800000,0.200000}%
\pgfsetstrokecolor{currentstroke}%
\pgfsetdash{}{0pt}%
\pgfpathmoveto{\pgfqpoint{3.466819in}{5.601726in}}%
\pgfpathcurveto{\pgfqpoint{3.472643in}{5.601726in}}{\pgfqpoint{3.478229in}{5.604040in}}{\pgfqpoint{3.482347in}{5.608158in}}%
\pgfpathcurveto{\pgfqpoint{3.486465in}{5.612276in}}{\pgfqpoint{3.488779in}{5.617863in}}{\pgfqpoint{3.488779in}{5.623687in}}%
\pgfpathcurveto{\pgfqpoint{3.488779in}{5.629511in}}{\pgfqpoint{3.486465in}{5.635097in}}{\pgfqpoint{3.482347in}{5.639215in}}%
\pgfpathcurveto{\pgfqpoint{3.478229in}{5.643333in}}{\pgfqpoint{3.472643in}{5.645647in}}{\pgfqpoint{3.466819in}{5.645647in}}%
\pgfpathcurveto{\pgfqpoint{3.460995in}{5.645647in}}{\pgfqpoint{3.455409in}{5.643333in}}{\pgfqpoint{3.451290in}{5.639215in}}%
\pgfpathcurveto{\pgfqpoint{3.447172in}{5.635097in}}{\pgfqpoint{3.444858in}{5.629511in}}{\pgfqpoint{3.444858in}{5.623687in}}%
\pgfpathcurveto{\pgfqpoint{3.444858in}{5.617863in}}{\pgfqpoint{3.447172in}{5.612276in}}{\pgfqpoint{3.451290in}{5.608158in}}%
\pgfpathcurveto{\pgfqpoint{3.455409in}{5.604040in}}{\pgfqpoint{3.460995in}{5.601726in}}{\pgfqpoint{3.466819in}{5.601726in}}%
\pgfpathlineto{\pgfqpoint{3.466819in}{5.601726in}}%
\pgfpathclose%
\pgfusepath{stroke,fill}%
\end{pgfscope}%
\begin{pgfscope}%
\pgfpathrectangle{\pgfqpoint{1.000000in}{1.148311in}}{\pgfqpoint{6.200000in}{5.623377in}}%
\pgfusepath{clip}%
\pgfsetbuttcap%
\pgfsetroundjoin%
\definecolor{currentfill}{rgb}{0.800000,0.800000,0.200000}%
\pgfsetfillcolor{currentfill}%
\pgfsetlinewidth{1.003750pt}%
\definecolor{currentstroke}{rgb}{0.800000,0.800000,0.200000}%
\pgfsetstrokecolor{currentstroke}%
\pgfsetdash{}{0pt}%
\pgfpathmoveto{\pgfqpoint{3.435951in}{5.542773in}}%
\pgfpathcurveto{\pgfqpoint{3.441775in}{5.542773in}}{\pgfqpoint{3.447361in}{5.545087in}}{\pgfqpoint{3.451479in}{5.549205in}}%
\pgfpathcurveto{\pgfqpoint{3.455598in}{5.553323in}}{\pgfqpoint{3.457911in}{5.558909in}}{\pgfqpoint{3.457911in}{5.564733in}}%
\pgfpathcurveto{\pgfqpoint{3.457911in}{5.570557in}}{\pgfqpoint{3.455598in}{5.576143in}}{\pgfqpoint{3.451479in}{5.580261in}}%
\pgfpathcurveto{\pgfqpoint{3.447361in}{5.584379in}}{\pgfqpoint{3.441775in}{5.586693in}}{\pgfqpoint{3.435951in}{5.586693in}}%
\pgfpathcurveto{\pgfqpoint{3.430127in}{5.586693in}}{\pgfqpoint{3.424541in}{5.584379in}}{\pgfqpoint{3.420423in}{5.580261in}}%
\pgfpathcurveto{\pgfqpoint{3.416305in}{5.576143in}}{\pgfqpoint{3.413991in}{5.570557in}}{\pgfqpoint{3.413991in}{5.564733in}}%
\pgfpathcurveto{\pgfqpoint{3.413991in}{5.558909in}}{\pgfqpoint{3.416305in}{5.553323in}}{\pgfqpoint{3.420423in}{5.549205in}}%
\pgfpathcurveto{\pgfqpoint{3.424541in}{5.545087in}}{\pgfqpoint{3.430127in}{5.542773in}}{\pgfqpoint{3.435951in}{5.542773in}}%
\pgfpathlineto{\pgfqpoint{3.435951in}{5.542773in}}%
\pgfpathclose%
\pgfusepath{stroke,fill}%
\end{pgfscope}%
\begin{pgfscope}%
\pgfpathrectangle{\pgfqpoint{1.000000in}{1.148311in}}{\pgfqpoint{6.200000in}{5.623377in}}%
\pgfusepath{clip}%
\pgfsetbuttcap%
\pgfsetroundjoin%
\definecolor{currentfill}{rgb}{0.800000,0.800000,0.200000}%
\pgfsetfillcolor{currentfill}%
\pgfsetlinewidth{1.003750pt}%
\definecolor{currentstroke}{rgb}{0.800000,0.800000,0.200000}%
\pgfsetstrokecolor{currentstroke}%
\pgfsetdash{}{0pt}%
\pgfpathmoveto{\pgfqpoint{3.420267in}{5.480148in}}%
\pgfpathcurveto{\pgfqpoint{3.426091in}{5.480148in}}{\pgfqpoint{3.431677in}{5.482462in}}{\pgfqpoint{3.435796in}{5.486581in}}%
\pgfpathcurveto{\pgfqpoint{3.439914in}{5.490699in}}{\pgfqpoint{3.442228in}{5.496285in}}{\pgfqpoint{3.442228in}{5.502109in}}%
\pgfpathcurveto{\pgfqpoint{3.442228in}{5.507933in}}{\pgfqpoint{3.439914in}{5.513519in}}{\pgfqpoint{3.435796in}{5.517637in}}%
\pgfpathcurveto{\pgfqpoint{3.431677in}{5.521755in}}{\pgfqpoint{3.426091in}{5.524069in}}{\pgfqpoint{3.420267in}{5.524069in}}%
\pgfpathcurveto{\pgfqpoint{3.414443in}{5.524069in}}{\pgfqpoint{3.408857in}{5.521755in}}{\pgfqpoint{3.404739in}{5.517637in}}%
\pgfpathcurveto{\pgfqpoint{3.400621in}{5.513519in}}{\pgfqpoint{3.398307in}{5.507933in}}{\pgfqpoint{3.398307in}{5.502109in}}%
\pgfpathcurveto{\pgfqpoint{3.398307in}{5.496285in}}{\pgfqpoint{3.400621in}{5.490699in}}{\pgfqpoint{3.404739in}{5.486581in}}%
\pgfpathcurveto{\pgfqpoint{3.408857in}{5.482462in}}{\pgfqpoint{3.414443in}{5.480148in}}{\pgfqpoint{3.420267in}{5.480148in}}%
\pgfpathlineto{\pgfqpoint{3.420267in}{5.480148in}}%
\pgfpathclose%
\pgfusepath{stroke,fill}%
\end{pgfscope}%
\begin{pgfscope}%
\pgfpathrectangle{\pgfqpoint{1.000000in}{1.148311in}}{\pgfqpoint{6.200000in}{5.623377in}}%
\pgfusepath{clip}%
\pgfsetbuttcap%
\pgfsetroundjoin%
\definecolor{currentfill}{rgb}{0.800000,0.800000,0.200000}%
\pgfsetfillcolor{currentfill}%
\pgfsetlinewidth{1.003750pt}%
\definecolor{currentstroke}{rgb}{0.800000,0.800000,0.200000}%
\pgfsetstrokecolor{currentstroke}%
\pgfsetdash{}{0pt}%
\pgfpathmoveto{\pgfqpoint{3.450742in}{5.417252in}}%
\pgfpathcurveto{\pgfqpoint{3.456566in}{5.417252in}}{\pgfqpoint{3.462152in}{5.419566in}}{\pgfqpoint{3.466270in}{5.423684in}}%
\pgfpathcurveto{\pgfqpoint{3.470388in}{5.427802in}}{\pgfqpoint{3.472702in}{5.433389in}}{\pgfqpoint{3.472702in}{5.439213in}}%
\pgfpathcurveto{\pgfqpoint{3.472702in}{5.445036in}}{\pgfqpoint{3.470388in}{5.450623in}}{\pgfqpoint{3.466270in}{5.454741in}}%
\pgfpathcurveto{\pgfqpoint{3.462152in}{5.458859in}}{\pgfqpoint{3.456566in}{5.461173in}}{\pgfqpoint{3.450742in}{5.461173in}}%
\pgfpathcurveto{\pgfqpoint{3.444918in}{5.461173in}}{\pgfqpoint{3.439332in}{5.458859in}}{\pgfqpoint{3.435214in}{5.454741in}}%
\pgfpathcurveto{\pgfqpoint{3.431095in}{5.450623in}}{\pgfqpoint{3.428782in}{5.445036in}}{\pgfqpoint{3.428782in}{5.439213in}}%
\pgfpathcurveto{\pgfqpoint{3.428782in}{5.433389in}}{\pgfqpoint{3.431095in}{5.427802in}}{\pgfqpoint{3.435214in}{5.423684in}}%
\pgfpathcurveto{\pgfqpoint{3.439332in}{5.419566in}}{\pgfqpoint{3.444918in}{5.417252in}}{\pgfqpoint{3.450742in}{5.417252in}}%
\pgfpathlineto{\pgfqpoint{3.450742in}{5.417252in}}%
\pgfpathclose%
\pgfusepath{stroke,fill}%
\end{pgfscope}%
\begin{pgfscope}%
\pgfpathrectangle{\pgfqpoint{1.000000in}{1.148311in}}{\pgfqpoint{6.200000in}{5.623377in}}%
\pgfusepath{clip}%
\pgfsetbuttcap%
\pgfsetroundjoin%
\definecolor{currentfill}{rgb}{0.800000,0.800000,0.200000}%
\pgfsetfillcolor{currentfill}%
\pgfsetlinewidth{1.003750pt}%
\definecolor{currentstroke}{rgb}{0.800000,0.800000,0.200000}%
\pgfsetstrokecolor{currentstroke}%
\pgfsetdash{}{0pt}%
\pgfpathmoveto{\pgfqpoint{3.453364in}{5.355323in}}%
\pgfpathcurveto{\pgfqpoint{3.459188in}{5.355323in}}{\pgfqpoint{3.464774in}{5.357637in}}{\pgfqpoint{3.468892in}{5.361755in}}%
\pgfpathcurveto{\pgfqpoint{3.473010in}{5.365874in}}{\pgfqpoint{3.475324in}{5.371460in}}{\pgfqpoint{3.475324in}{5.377284in}}%
\pgfpathcurveto{\pgfqpoint{3.475324in}{5.383108in}}{\pgfqpoint{3.473010in}{5.388694in}}{\pgfqpoint{3.468892in}{5.392812in}}%
\pgfpathcurveto{\pgfqpoint{3.464774in}{5.396930in}}{\pgfqpoint{3.459188in}{5.399244in}}{\pgfqpoint{3.453364in}{5.399244in}}%
\pgfpathcurveto{\pgfqpoint{3.447540in}{5.399244in}}{\pgfqpoint{3.441954in}{5.396930in}}{\pgfqpoint{3.437836in}{5.392812in}}%
\pgfpathcurveto{\pgfqpoint{3.433717in}{5.388694in}}{\pgfqpoint{3.431404in}{5.383108in}}{\pgfqpoint{3.431404in}{5.377284in}}%
\pgfpathcurveto{\pgfqpoint{3.431404in}{5.371460in}}{\pgfqpoint{3.433717in}{5.365874in}}{\pgfqpoint{3.437836in}{5.361755in}}%
\pgfpathcurveto{\pgfqpoint{3.441954in}{5.357637in}}{\pgfqpoint{3.447540in}{5.355323in}}{\pgfqpoint{3.453364in}{5.355323in}}%
\pgfpathlineto{\pgfqpoint{3.453364in}{5.355323in}}%
\pgfpathclose%
\pgfusepath{stroke,fill}%
\end{pgfscope}%
\begin{pgfscope}%
\pgfpathrectangle{\pgfqpoint{1.000000in}{1.148311in}}{\pgfqpoint{6.200000in}{5.623377in}}%
\pgfusepath{clip}%
\pgfsetbuttcap%
\pgfsetroundjoin%
\definecolor{currentfill}{rgb}{0.800000,0.800000,0.200000}%
\pgfsetfillcolor{currentfill}%
\pgfsetlinewidth{1.003750pt}%
\definecolor{currentstroke}{rgb}{0.800000,0.800000,0.200000}%
\pgfsetstrokecolor{currentstroke}%
\pgfsetdash{}{0pt}%
\pgfpathmoveto{\pgfqpoint{3.433848in}{5.289431in}}%
\pgfpathcurveto{\pgfqpoint{3.439672in}{5.289431in}}{\pgfqpoint{3.445258in}{5.291745in}}{\pgfqpoint{3.449376in}{5.295863in}}%
\pgfpathcurveto{\pgfqpoint{3.453495in}{5.299982in}}{\pgfqpoint{3.455808in}{5.305568in}}{\pgfqpoint{3.455808in}{5.311392in}}%
\pgfpathcurveto{\pgfqpoint{3.455808in}{5.317216in}}{\pgfqpoint{3.453495in}{5.322802in}}{\pgfqpoint{3.449376in}{5.326920in}}%
\pgfpathcurveto{\pgfqpoint{3.445258in}{5.331038in}}{\pgfqpoint{3.439672in}{5.333352in}}{\pgfqpoint{3.433848in}{5.333352in}}%
\pgfpathcurveto{\pgfqpoint{3.428024in}{5.333352in}}{\pgfqpoint{3.422438in}{5.331038in}}{\pgfqpoint{3.418320in}{5.326920in}}%
\pgfpathcurveto{\pgfqpoint{3.414202in}{5.322802in}}{\pgfqpoint{3.411888in}{5.317216in}}{\pgfqpoint{3.411888in}{5.311392in}}%
\pgfpathcurveto{\pgfqpoint{3.411888in}{5.305568in}}{\pgfqpoint{3.414202in}{5.299982in}}{\pgfqpoint{3.418320in}{5.295863in}}%
\pgfpathcurveto{\pgfqpoint{3.422438in}{5.291745in}}{\pgfqpoint{3.428024in}{5.289431in}}{\pgfqpoint{3.433848in}{5.289431in}}%
\pgfpathlineto{\pgfqpoint{3.433848in}{5.289431in}}%
\pgfpathclose%
\pgfusepath{stroke,fill}%
\end{pgfscope}%
\begin{pgfscope}%
\pgfpathrectangle{\pgfqpoint{1.000000in}{1.148311in}}{\pgfqpoint{6.200000in}{5.623377in}}%
\pgfusepath{clip}%
\pgfsetbuttcap%
\pgfsetroundjoin%
\definecolor{currentfill}{rgb}{0.800000,0.800000,0.200000}%
\pgfsetfillcolor{currentfill}%
\pgfsetlinewidth{1.003750pt}%
\definecolor{currentstroke}{rgb}{0.800000,0.800000,0.200000}%
\pgfsetstrokecolor{currentstroke}%
\pgfsetdash{}{0pt}%
\pgfpathmoveto{\pgfqpoint{3.417042in}{5.220291in}}%
\pgfpathcurveto{\pgfqpoint{3.422866in}{5.220291in}}{\pgfqpoint{3.428452in}{5.222605in}}{\pgfqpoint{3.432570in}{5.226723in}}%
\pgfpathcurveto{\pgfqpoint{3.436688in}{5.230842in}}{\pgfqpoint{3.439002in}{5.236428in}}{\pgfqpoint{3.439002in}{5.242252in}}%
\pgfpathcurveto{\pgfqpoint{3.439002in}{5.248076in}}{\pgfqpoint{3.436688in}{5.253662in}}{\pgfqpoint{3.432570in}{5.257780in}}%
\pgfpathcurveto{\pgfqpoint{3.428452in}{5.261898in}}{\pgfqpoint{3.422866in}{5.264212in}}{\pgfqpoint{3.417042in}{5.264212in}}%
\pgfpathcurveto{\pgfqpoint{3.411218in}{5.264212in}}{\pgfqpoint{3.405632in}{5.261898in}}{\pgfqpoint{3.401513in}{5.257780in}}%
\pgfpathcurveto{\pgfqpoint{3.397395in}{5.253662in}}{\pgfqpoint{3.395081in}{5.248076in}}{\pgfqpoint{3.395081in}{5.242252in}}%
\pgfpathcurveto{\pgfqpoint{3.395081in}{5.236428in}}{\pgfqpoint{3.397395in}{5.230842in}}{\pgfqpoint{3.401513in}{5.226723in}}%
\pgfpathcurveto{\pgfqpoint{3.405632in}{5.222605in}}{\pgfqpoint{3.411218in}{5.220291in}}{\pgfqpoint{3.417042in}{5.220291in}}%
\pgfpathlineto{\pgfqpoint{3.417042in}{5.220291in}}%
\pgfpathclose%
\pgfusepath{stroke,fill}%
\end{pgfscope}%
\begin{pgfscope}%
\pgfpathrectangle{\pgfqpoint{1.000000in}{1.148311in}}{\pgfqpoint{6.200000in}{5.623377in}}%
\pgfusepath{clip}%
\pgfsetbuttcap%
\pgfsetroundjoin%
\definecolor{currentfill}{rgb}{0.800000,0.800000,0.200000}%
\pgfsetfillcolor{currentfill}%
\pgfsetlinewidth{1.003750pt}%
\definecolor{currentstroke}{rgb}{0.800000,0.800000,0.200000}%
\pgfsetstrokecolor{currentstroke}%
\pgfsetdash{}{0pt}%
\pgfpathmoveto{\pgfqpoint{3.463526in}{5.165554in}}%
\pgfpathcurveto{\pgfqpoint{3.469350in}{5.165554in}}{\pgfqpoint{3.474936in}{5.167868in}}{\pgfqpoint{3.479054in}{5.171986in}}%
\pgfpathcurveto{\pgfqpoint{3.483172in}{5.176104in}}{\pgfqpoint{3.485486in}{5.181691in}}{\pgfqpoint{3.485486in}{5.187515in}}%
\pgfpathcurveto{\pgfqpoint{3.485486in}{5.193338in}}{\pgfqpoint{3.483172in}{5.198925in}}{\pgfqpoint{3.479054in}{5.203043in}}%
\pgfpathcurveto{\pgfqpoint{3.474936in}{5.207161in}}{\pgfqpoint{3.469350in}{5.209475in}}{\pgfqpoint{3.463526in}{5.209475in}}%
\pgfpathcurveto{\pgfqpoint{3.457702in}{5.209475in}}{\pgfqpoint{3.452116in}{5.207161in}}{\pgfqpoint{3.447998in}{5.203043in}}%
\pgfpathcurveto{\pgfqpoint{3.443880in}{5.198925in}}{\pgfqpoint{3.441566in}{5.193338in}}{\pgfqpoint{3.441566in}{5.187515in}}%
\pgfpathcurveto{\pgfqpoint{3.441566in}{5.181691in}}{\pgfqpoint{3.443880in}{5.176104in}}{\pgfqpoint{3.447998in}{5.171986in}}%
\pgfpathcurveto{\pgfqpoint{3.452116in}{5.167868in}}{\pgfqpoint{3.457702in}{5.165554in}}{\pgfqpoint{3.463526in}{5.165554in}}%
\pgfpathlineto{\pgfqpoint{3.463526in}{5.165554in}}%
\pgfpathclose%
\pgfusepath{stroke,fill}%
\end{pgfscope}%
\begin{pgfscope}%
\pgfpathrectangle{\pgfqpoint{1.000000in}{1.148311in}}{\pgfqpoint{6.200000in}{5.623377in}}%
\pgfusepath{clip}%
\pgfsetbuttcap%
\pgfsetroundjoin%
\definecolor{currentfill}{rgb}{0.800000,0.800000,0.200000}%
\pgfsetfillcolor{currentfill}%
\pgfsetlinewidth{1.003750pt}%
\definecolor{currentstroke}{rgb}{0.800000,0.800000,0.200000}%
\pgfsetstrokecolor{currentstroke}%
\pgfsetdash{}{0pt}%
\pgfpathmoveto{\pgfqpoint{3.493477in}{5.108738in}}%
\pgfpathcurveto{\pgfqpoint{3.499301in}{5.108738in}}{\pgfqpoint{3.504887in}{5.111052in}}{\pgfqpoint{3.509005in}{5.115170in}}%
\pgfpathcurveto{\pgfqpoint{3.513123in}{5.119288in}}{\pgfqpoint{3.515437in}{5.124875in}}{\pgfqpoint{3.515437in}{5.130699in}}%
\pgfpathcurveto{\pgfqpoint{3.515437in}{5.136522in}}{\pgfqpoint{3.513123in}{5.142109in}}{\pgfqpoint{3.509005in}{5.146227in}}%
\pgfpathcurveto{\pgfqpoint{3.504887in}{5.150345in}}{\pgfqpoint{3.499301in}{5.152659in}}{\pgfqpoint{3.493477in}{5.152659in}}%
\pgfpathcurveto{\pgfqpoint{3.487653in}{5.152659in}}{\pgfqpoint{3.482067in}{5.150345in}}{\pgfqpoint{3.477948in}{5.146227in}}%
\pgfpathcurveto{\pgfqpoint{3.473830in}{5.142109in}}{\pgfqpoint{3.471516in}{5.136522in}}{\pgfqpoint{3.471516in}{5.130699in}}%
\pgfpathcurveto{\pgfqpoint{3.471516in}{5.124875in}}{\pgfqpoint{3.473830in}{5.119288in}}{\pgfqpoint{3.477948in}{5.115170in}}%
\pgfpathcurveto{\pgfqpoint{3.482067in}{5.111052in}}{\pgfqpoint{3.487653in}{5.108738in}}{\pgfqpoint{3.493477in}{5.108738in}}%
\pgfpathlineto{\pgfqpoint{3.493477in}{5.108738in}}%
\pgfpathclose%
\pgfusepath{stroke,fill}%
\end{pgfscope}%
\begin{pgfscope}%
\pgfpathrectangle{\pgfqpoint{1.000000in}{1.148311in}}{\pgfqpoint{6.200000in}{5.623377in}}%
\pgfusepath{clip}%
\pgfsetbuttcap%
\pgfsetroundjoin%
\definecolor{currentfill}{rgb}{0.800000,0.800000,0.200000}%
\pgfsetfillcolor{currentfill}%
\pgfsetlinewidth{1.003750pt}%
\definecolor{currentstroke}{rgb}{0.800000,0.800000,0.200000}%
\pgfsetstrokecolor{currentstroke}%
\pgfsetdash{}{0pt}%
\pgfpathmoveto{\pgfqpoint{3.507276in}{5.046060in}}%
\pgfpathcurveto{\pgfqpoint{3.513100in}{5.046060in}}{\pgfqpoint{3.518686in}{5.048374in}}{\pgfqpoint{3.522804in}{5.052492in}}%
\pgfpathcurveto{\pgfqpoint{3.526922in}{5.056610in}}{\pgfqpoint{3.529236in}{5.062196in}}{\pgfqpoint{3.529236in}{5.068020in}}%
\pgfpathcurveto{\pgfqpoint{3.529236in}{5.073844in}}{\pgfqpoint{3.526922in}{5.079430in}}{\pgfqpoint{3.522804in}{5.083549in}}%
\pgfpathcurveto{\pgfqpoint{3.518686in}{5.087667in}}{\pgfqpoint{3.513100in}{5.089981in}}{\pgfqpoint{3.507276in}{5.089981in}}%
\pgfpathcurveto{\pgfqpoint{3.501452in}{5.089981in}}{\pgfqpoint{3.495866in}{5.087667in}}{\pgfqpoint{3.491747in}{5.083549in}}%
\pgfpathcurveto{\pgfqpoint{3.487629in}{5.079430in}}{\pgfqpoint{3.485315in}{5.073844in}}{\pgfqpoint{3.485315in}{5.068020in}}%
\pgfpathcurveto{\pgfqpoint{3.485315in}{5.062196in}}{\pgfqpoint{3.487629in}{5.056610in}}{\pgfqpoint{3.491747in}{5.052492in}}%
\pgfpathcurveto{\pgfqpoint{3.495866in}{5.048374in}}{\pgfqpoint{3.501452in}{5.046060in}}{\pgfqpoint{3.507276in}{5.046060in}}%
\pgfpathlineto{\pgfqpoint{3.507276in}{5.046060in}}%
\pgfpathclose%
\pgfusepath{stroke,fill}%
\end{pgfscope}%
\begin{pgfscope}%
\pgfpathrectangle{\pgfqpoint{1.000000in}{1.148311in}}{\pgfqpoint{6.200000in}{5.623377in}}%
\pgfusepath{clip}%
\pgfsetbuttcap%
\pgfsetroundjoin%
\definecolor{currentfill}{rgb}{0.800000,0.800000,0.200000}%
\pgfsetfillcolor{currentfill}%
\pgfsetlinewidth{1.003750pt}%
\definecolor{currentstroke}{rgb}{0.800000,0.800000,0.200000}%
\pgfsetstrokecolor{currentstroke}%
\pgfsetdash{}{0pt}%
\pgfpathmoveto{\pgfqpoint{3.606156in}{5.025462in}}%
\pgfpathcurveto{\pgfqpoint{3.611980in}{5.025462in}}{\pgfqpoint{3.617567in}{5.027776in}}{\pgfqpoint{3.621685in}{5.031894in}}%
\pgfpathcurveto{\pgfqpoint{3.625803in}{5.036012in}}{\pgfqpoint{3.628117in}{5.041598in}}{\pgfqpoint{3.628117in}{5.047422in}}%
\pgfpathcurveto{\pgfqpoint{3.628117in}{5.053246in}}{\pgfqpoint{3.625803in}{5.058832in}}{\pgfqpoint{3.621685in}{5.062951in}}%
\pgfpathcurveto{\pgfqpoint{3.617567in}{5.067069in}}{\pgfqpoint{3.611980in}{5.069383in}}{\pgfqpoint{3.606156in}{5.069383in}}%
\pgfpathcurveto{\pgfqpoint{3.600332in}{5.069383in}}{\pgfqpoint{3.594746in}{5.067069in}}{\pgfqpoint{3.590628in}{5.062951in}}%
\pgfpathcurveto{\pgfqpoint{3.586510in}{5.058832in}}{\pgfqpoint{3.584196in}{5.053246in}}{\pgfqpoint{3.584196in}{5.047422in}}%
\pgfpathcurveto{\pgfqpoint{3.584196in}{5.041598in}}{\pgfqpoint{3.586510in}{5.036012in}}{\pgfqpoint{3.590628in}{5.031894in}}%
\pgfpathcurveto{\pgfqpoint{3.594746in}{5.027776in}}{\pgfqpoint{3.600332in}{5.025462in}}{\pgfqpoint{3.606156in}{5.025462in}}%
\pgfpathlineto{\pgfqpoint{3.606156in}{5.025462in}}%
\pgfpathclose%
\pgfusepath{stroke,fill}%
\end{pgfscope}%
\begin{pgfscope}%
\pgfpathrectangle{\pgfqpoint{1.000000in}{1.148311in}}{\pgfqpoint{6.200000in}{5.623377in}}%
\pgfusepath{clip}%
\pgfsetbuttcap%
\pgfsetroundjoin%
\definecolor{currentfill}{rgb}{0.800000,0.800000,0.200000}%
\pgfsetfillcolor{currentfill}%
\pgfsetlinewidth{1.003750pt}%
\definecolor{currentstroke}{rgb}{0.800000,0.800000,0.200000}%
\pgfsetstrokecolor{currentstroke}%
\pgfsetdash{}{0pt}%
\pgfpathmoveto{\pgfqpoint{3.581503in}{4.942502in}}%
\pgfpathcurveto{\pgfqpoint{3.587327in}{4.942502in}}{\pgfqpoint{3.592913in}{4.944816in}}{\pgfqpoint{3.597031in}{4.948934in}}%
\pgfpathcurveto{\pgfqpoint{3.601149in}{4.953052in}}{\pgfqpoint{3.603463in}{4.958639in}}{\pgfqpoint{3.603463in}{4.964463in}}%
\pgfpathcurveto{\pgfqpoint{3.603463in}{4.970287in}}{\pgfqpoint{3.601149in}{4.975873in}}{\pgfqpoint{3.597031in}{4.979991in}}%
\pgfpathcurveto{\pgfqpoint{3.592913in}{4.984109in}}{\pgfqpoint{3.587327in}{4.986423in}}{\pgfqpoint{3.581503in}{4.986423in}}%
\pgfpathcurveto{\pgfqpoint{3.575679in}{4.986423in}}{\pgfqpoint{3.570093in}{4.984109in}}{\pgfqpoint{3.565975in}{4.979991in}}%
\pgfpathcurveto{\pgfqpoint{3.561857in}{4.975873in}}{\pgfqpoint{3.559543in}{4.970287in}}{\pgfqpoint{3.559543in}{4.964463in}}%
\pgfpathcurveto{\pgfqpoint{3.559543in}{4.958639in}}{\pgfqpoint{3.561857in}{4.953052in}}{\pgfqpoint{3.565975in}{4.948934in}}%
\pgfpathcurveto{\pgfqpoint{3.570093in}{4.944816in}}{\pgfqpoint{3.575679in}{4.942502in}}{\pgfqpoint{3.581503in}{4.942502in}}%
\pgfpathlineto{\pgfqpoint{3.581503in}{4.942502in}}%
\pgfpathclose%
\pgfusepath{stroke,fill}%
\end{pgfscope}%
\begin{pgfscope}%
\pgfpathrectangle{\pgfqpoint{1.000000in}{1.148311in}}{\pgfqpoint{6.200000in}{5.623377in}}%
\pgfusepath{clip}%
\pgfsetbuttcap%
\pgfsetroundjoin%
\definecolor{currentfill}{rgb}{0.800000,0.800000,0.200000}%
\pgfsetfillcolor{currentfill}%
\pgfsetlinewidth{1.003750pt}%
\definecolor{currentstroke}{rgb}{0.800000,0.800000,0.200000}%
\pgfsetstrokecolor{currentstroke}%
\pgfsetdash{}{0pt}%
\pgfpathmoveto{\pgfqpoint{3.648691in}{4.912951in}}%
\pgfpathcurveto{\pgfqpoint{3.654515in}{4.912951in}}{\pgfqpoint{3.660101in}{4.915264in}}{\pgfqpoint{3.664219in}{4.919383in}}%
\pgfpathcurveto{\pgfqpoint{3.668337in}{4.923501in}}{\pgfqpoint{3.670651in}{4.929087in}}{\pgfqpoint{3.670651in}{4.934911in}}%
\pgfpathcurveto{\pgfqpoint{3.670651in}{4.940735in}}{\pgfqpoint{3.668337in}{4.946321in}}{\pgfqpoint{3.664219in}{4.950439in}}%
\pgfpathcurveto{\pgfqpoint{3.660101in}{4.954557in}}{\pgfqpoint{3.654515in}{4.956871in}}{\pgfqpoint{3.648691in}{4.956871in}}%
\pgfpathcurveto{\pgfqpoint{3.642867in}{4.956871in}}{\pgfqpoint{3.637281in}{4.954557in}}{\pgfqpoint{3.633163in}{4.950439in}}%
\pgfpathcurveto{\pgfqpoint{3.629045in}{4.946321in}}{\pgfqpoint{3.626731in}{4.940735in}}{\pgfqpoint{3.626731in}{4.934911in}}%
\pgfpathcurveto{\pgfqpoint{3.626731in}{4.929087in}}{\pgfqpoint{3.629045in}{4.923501in}}{\pgfqpoint{3.633163in}{4.919383in}}%
\pgfpathcurveto{\pgfqpoint{3.637281in}{4.915264in}}{\pgfqpoint{3.642867in}{4.912951in}}{\pgfqpoint{3.648691in}{4.912951in}}%
\pgfpathlineto{\pgfqpoint{3.648691in}{4.912951in}}%
\pgfpathclose%
\pgfusepath{stroke,fill}%
\end{pgfscope}%
\begin{pgfscope}%
\pgfpathrectangle{\pgfqpoint{1.000000in}{1.148311in}}{\pgfqpoint{6.200000in}{5.623377in}}%
\pgfusepath{clip}%
\pgfsetbuttcap%
\pgfsetroundjoin%
\definecolor{currentfill}{rgb}{0.800000,0.800000,0.200000}%
\pgfsetfillcolor{currentfill}%
\pgfsetlinewidth{1.003750pt}%
\definecolor{currentstroke}{rgb}{0.800000,0.800000,0.200000}%
\pgfsetstrokecolor{currentstroke}%
\pgfsetdash{}{0pt}%
\pgfpathmoveto{\pgfqpoint{3.662070in}{4.847309in}}%
\pgfpathcurveto{\pgfqpoint{3.667894in}{4.847309in}}{\pgfqpoint{3.673480in}{4.849623in}}{\pgfqpoint{3.677598in}{4.853741in}}%
\pgfpathcurveto{\pgfqpoint{3.681717in}{4.857860in}}{\pgfqpoint{3.684030in}{4.863446in}}{\pgfqpoint{3.684030in}{4.869270in}}%
\pgfpathcurveto{\pgfqpoint{3.684030in}{4.875094in}}{\pgfqpoint{3.681717in}{4.880680in}}{\pgfqpoint{3.677598in}{4.884798in}}%
\pgfpathcurveto{\pgfqpoint{3.673480in}{4.888916in}}{\pgfqpoint{3.667894in}{4.891230in}}{\pgfqpoint{3.662070in}{4.891230in}}%
\pgfpathcurveto{\pgfqpoint{3.656246in}{4.891230in}}{\pgfqpoint{3.650660in}{4.888916in}}{\pgfqpoint{3.646542in}{4.884798in}}%
\pgfpathcurveto{\pgfqpoint{3.642424in}{4.880680in}}{\pgfqpoint{3.640110in}{4.875094in}}{\pgfqpoint{3.640110in}{4.869270in}}%
\pgfpathcurveto{\pgfqpoint{3.640110in}{4.863446in}}{\pgfqpoint{3.642424in}{4.857860in}}{\pgfqpoint{3.646542in}{4.853741in}}%
\pgfpathcurveto{\pgfqpoint{3.650660in}{4.849623in}}{\pgfqpoint{3.656246in}{4.847309in}}{\pgfqpoint{3.662070in}{4.847309in}}%
\pgfpathlineto{\pgfqpoint{3.662070in}{4.847309in}}%
\pgfpathclose%
\pgfusepath{stroke,fill}%
\end{pgfscope}%
\begin{pgfscope}%
\pgfpathrectangle{\pgfqpoint{1.000000in}{1.148311in}}{\pgfqpoint{6.200000in}{5.623377in}}%
\pgfusepath{clip}%
\pgfsetbuttcap%
\pgfsetroundjoin%
\definecolor{currentfill}{rgb}{0.800000,0.800000,0.200000}%
\pgfsetfillcolor{currentfill}%
\pgfsetlinewidth{1.003750pt}%
\definecolor{currentstroke}{rgb}{0.800000,0.800000,0.200000}%
\pgfsetstrokecolor{currentstroke}%
\pgfsetdash{}{0pt}%
\pgfpathmoveto{\pgfqpoint{3.712625in}{4.809812in}}%
\pgfpathcurveto{\pgfqpoint{3.718448in}{4.809812in}}{\pgfqpoint{3.724035in}{4.812126in}}{\pgfqpoint{3.728153in}{4.816244in}}%
\pgfpathcurveto{\pgfqpoint{3.732271in}{4.820362in}}{\pgfqpoint{3.734585in}{4.825948in}}{\pgfqpoint{3.734585in}{4.831772in}}%
\pgfpathcurveto{\pgfqpoint{3.734585in}{4.837596in}}{\pgfqpoint{3.732271in}{4.843182in}}{\pgfqpoint{3.728153in}{4.847300in}}%
\pgfpathcurveto{\pgfqpoint{3.724035in}{4.851418in}}{\pgfqpoint{3.718448in}{4.853732in}}{\pgfqpoint{3.712625in}{4.853732in}}%
\pgfpathcurveto{\pgfqpoint{3.706801in}{4.853732in}}{\pgfqpoint{3.701214in}{4.851418in}}{\pgfqpoint{3.697096in}{4.847300in}}%
\pgfpathcurveto{\pgfqpoint{3.692978in}{4.843182in}}{\pgfqpoint{3.690664in}{4.837596in}}{\pgfqpoint{3.690664in}{4.831772in}}%
\pgfpathcurveto{\pgfqpoint{3.690664in}{4.825948in}}{\pgfqpoint{3.692978in}{4.820362in}}{\pgfqpoint{3.697096in}{4.816244in}}%
\pgfpathcurveto{\pgfqpoint{3.701214in}{4.812126in}}{\pgfqpoint{3.706801in}{4.809812in}}{\pgfqpoint{3.712625in}{4.809812in}}%
\pgfpathlineto{\pgfqpoint{3.712625in}{4.809812in}}%
\pgfpathclose%
\pgfusepath{stroke,fill}%
\end{pgfscope}%
\begin{pgfscope}%
\pgfpathrectangle{\pgfqpoint{1.000000in}{1.148311in}}{\pgfqpoint{6.200000in}{5.623377in}}%
\pgfusepath{clip}%
\pgfsetbuttcap%
\pgfsetroundjoin%
\definecolor{currentfill}{rgb}{0.800000,0.800000,0.200000}%
\pgfsetfillcolor{currentfill}%
\pgfsetlinewidth{1.003750pt}%
\definecolor{currentstroke}{rgb}{0.800000,0.800000,0.200000}%
\pgfsetstrokecolor{currentstroke}%
\pgfsetdash{}{0pt}%
\pgfpathmoveto{\pgfqpoint{3.754928in}{4.766223in}}%
\pgfpathcurveto{\pgfqpoint{3.760752in}{4.766223in}}{\pgfqpoint{3.766338in}{4.768537in}}{\pgfqpoint{3.770457in}{4.772655in}}%
\pgfpathcurveto{\pgfqpoint{3.774575in}{4.776773in}}{\pgfqpoint{3.776889in}{4.782359in}}{\pgfqpoint{3.776889in}{4.788183in}}%
\pgfpathcurveto{\pgfqpoint{3.776889in}{4.794007in}}{\pgfqpoint{3.774575in}{4.799593in}}{\pgfqpoint{3.770457in}{4.803712in}}%
\pgfpathcurveto{\pgfqpoint{3.766338in}{4.807830in}}{\pgfqpoint{3.760752in}{4.810144in}}{\pgfqpoint{3.754928in}{4.810144in}}%
\pgfpathcurveto{\pgfqpoint{3.749104in}{4.810144in}}{\pgfqpoint{3.743518in}{4.807830in}}{\pgfqpoint{3.739400in}{4.803712in}}%
\pgfpathcurveto{\pgfqpoint{3.735282in}{4.799593in}}{\pgfqpoint{3.732968in}{4.794007in}}{\pgfqpoint{3.732968in}{4.788183in}}%
\pgfpathcurveto{\pgfqpoint{3.732968in}{4.782359in}}{\pgfqpoint{3.735282in}{4.776773in}}{\pgfqpoint{3.739400in}{4.772655in}}%
\pgfpathcurveto{\pgfqpoint{3.743518in}{4.768537in}}{\pgfqpoint{3.749104in}{4.766223in}}{\pgfqpoint{3.754928in}{4.766223in}}%
\pgfpathlineto{\pgfqpoint{3.754928in}{4.766223in}}%
\pgfpathclose%
\pgfusepath{stroke,fill}%
\end{pgfscope}%
\begin{pgfscope}%
\pgfpathrectangle{\pgfqpoint{1.000000in}{1.148311in}}{\pgfqpoint{6.200000in}{5.623377in}}%
\pgfusepath{clip}%
\pgfsetbuttcap%
\pgfsetroundjoin%
\definecolor{currentfill}{rgb}{0.800000,0.800000,0.200000}%
\pgfsetfillcolor{currentfill}%
\pgfsetlinewidth{1.003750pt}%
\definecolor{currentstroke}{rgb}{0.800000,0.800000,0.200000}%
\pgfsetstrokecolor{currentstroke}%
\pgfsetdash{}{0pt}%
\pgfpathmoveto{\pgfqpoint{3.801002in}{4.726716in}}%
\pgfpathcurveto{\pgfqpoint{3.806826in}{4.726716in}}{\pgfqpoint{3.812412in}{4.729030in}}{\pgfqpoint{3.816530in}{4.733148in}}%
\pgfpathcurveto{\pgfqpoint{3.820649in}{4.737266in}}{\pgfqpoint{3.822962in}{4.742852in}}{\pgfqpoint{3.822962in}{4.748676in}}%
\pgfpathcurveto{\pgfqpoint{3.822962in}{4.754500in}}{\pgfqpoint{3.820649in}{4.760087in}}{\pgfqpoint{3.816530in}{4.764205in}}%
\pgfpathcurveto{\pgfqpoint{3.812412in}{4.768323in}}{\pgfqpoint{3.806826in}{4.770637in}}{\pgfqpoint{3.801002in}{4.770637in}}%
\pgfpathcurveto{\pgfqpoint{3.795178in}{4.770637in}}{\pgfqpoint{3.789592in}{4.768323in}}{\pgfqpoint{3.785474in}{4.764205in}}%
\pgfpathcurveto{\pgfqpoint{3.781356in}{4.760087in}}{\pgfqpoint{3.779042in}{4.754500in}}{\pgfqpoint{3.779042in}{4.748676in}}%
\pgfpathcurveto{\pgfqpoint{3.779042in}{4.742852in}}{\pgfqpoint{3.781356in}{4.737266in}}{\pgfqpoint{3.785474in}{4.733148in}}%
\pgfpathcurveto{\pgfqpoint{3.789592in}{4.729030in}}{\pgfqpoint{3.795178in}{4.726716in}}{\pgfqpoint{3.801002in}{4.726716in}}%
\pgfpathlineto{\pgfqpoint{3.801002in}{4.726716in}}%
\pgfpathclose%
\pgfusepath{stroke,fill}%
\end{pgfscope}%
\begin{pgfscope}%
\pgfpathrectangle{\pgfqpoint{1.000000in}{1.148311in}}{\pgfqpoint{6.200000in}{5.623377in}}%
\pgfusepath{clip}%
\pgfsetbuttcap%
\pgfsetroundjoin%
\definecolor{currentfill}{rgb}{0.800000,0.800000,0.200000}%
\pgfsetfillcolor{currentfill}%
\pgfsetlinewidth{1.003750pt}%
\definecolor{currentstroke}{rgb}{0.800000,0.800000,0.200000}%
\pgfsetstrokecolor{currentstroke}%
\pgfsetdash{}{0pt}%
\pgfpathmoveto{\pgfqpoint{3.768668in}{4.584249in}}%
\pgfpathcurveto{\pgfqpoint{3.774492in}{4.584249in}}{\pgfqpoint{3.780079in}{4.586563in}}{\pgfqpoint{3.784197in}{4.590681in}}%
\pgfpathcurveto{\pgfqpoint{3.788315in}{4.594799in}}{\pgfqpoint{3.790629in}{4.600385in}}{\pgfqpoint{3.790629in}{4.606209in}}%
\pgfpathcurveto{\pgfqpoint{3.790629in}{4.612033in}}{\pgfqpoint{3.788315in}{4.617619in}}{\pgfqpoint{3.784197in}{4.621737in}}%
\pgfpathcurveto{\pgfqpoint{3.780079in}{4.625856in}}{\pgfqpoint{3.774492in}{4.628169in}}{\pgfqpoint{3.768668in}{4.628169in}}%
\pgfpathcurveto{\pgfqpoint{3.762845in}{4.628169in}}{\pgfqpoint{3.757258in}{4.625856in}}{\pgfqpoint{3.753140in}{4.621737in}}%
\pgfpathcurveto{\pgfqpoint{3.749022in}{4.617619in}}{\pgfqpoint{3.746708in}{4.612033in}}{\pgfqpoint{3.746708in}{4.606209in}}%
\pgfpathcurveto{\pgfqpoint{3.746708in}{4.600385in}}{\pgfqpoint{3.749022in}{4.594799in}}{\pgfqpoint{3.753140in}{4.590681in}}%
\pgfpathcurveto{\pgfqpoint{3.757258in}{4.586563in}}{\pgfqpoint{3.762845in}{4.584249in}}{\pgfqpoint{3.768668in}{4.584249in}}%
\pgfpathlineto{\pgfqpoint{3.768668in}{4.584249in}}%
\pgfpathclose%
\pgfusepath{stroke,fill}%
\end{pgfscope}%
\begin{pgfscope}%
\pgfpathrectangle{\pgfqpoint{1.000000in}{1.148311in}}{\pgfqpoint{6.200000in}{5.623377in}}%
\pgfusepath{clip}%
\pgfsetbuttcap%
\pgfsetroundjoin%
\definecolor{currentfill}{rgb}{0.800000,0.800000,0.200000}%
\pgfsetfillcolor{currentfill}%
\pgfsetlinewidth{1.003750pt}%
\definecolor{currentstroke}{rgb}{0.800000,0.800000,0.200000}%
\pgfsetstrokecolor{currentstroke}%
\pgfsetdash{}{0pt}%
\pgfpathmoveto{\pgfqpoint{3.877009in}{4.622776in}}%
\pgfpathcurveto{\pgfqpoint{3.882833in}{4.622776in}}{\pgfqpoint{3.888419in}{4.625090in}}{\pgfqpoint{3.892537in}{4.629208in}}%
\pgfpathcurveto{\pgfqpoint{3.896655in}{4.633326in}}{\pgfqpoint{3.898969in}{4.638912in}}{\pgfqpoint{3.898969in}{4.644736in}}%
\pgfpathcurveto{\pgfqpoint{3.898969in}{4.650560in}}{\pgfqpoint{3.896655in}{4.656146in}}{\pgfqpoint{3.892537in}{4.660265in}}%
\pgfpathcurveto{\pgfqpoint{3.888419in}{4.664383in}}{\pgfqpoint{3.882833in}{4.666697in}}{\pgfqpoint{3.877009in}{4.666697in}}%
\pgfpathcurveto{\pgfqpoint{3.871185in}{4.666697in}}{\pgfqpoint{3.865599in}{4.664383in}}{\pgfqpoint{3.861481in}{4.660265in}}%
\pgfpathcurveto{\pgfqpoint{3.857363in}{4.656146in}}{\pgfqpoint{3.855049in}{4.650560in}}{\pgfqpoint{3.855049in}{4.644736in}}%
\pgfpathcurveto{\pgfqpoint{3.855049in}{4.638912in}}{\pgfqpoint{3.857363in}{4.633326in}}{\pgfqpoint{3.861481in}{4.629208in}}%
\pgfpathcurveto{\pgfqpoint{3.865599in}{4.625090in}}{\pgfqpoint{3.871185in}{4.622776in}}{\pgfqpoint{3.877009in}{4.622776in}}%
\pgfpathlineto{\pgfqpoint{3.877009in}{4.622776in}}%
\pgfpathclose%
\pgfusepath{stroke,fill}%
\end{pgfscope}%
\begin{pgfscope}%
\pgfpathrectangle{\pgfqpoint{1.000000in}{1.148311in}}{\pgfqpoint{6.200000in}{5.623377in}}%
\pgfusepath{clip}%
\pgfsetbuttcap%
\pgfsetroundjoin%
\definecolor{currentfill}{rgb}{0.800000,0.800000,0.200000}%
\pgfsetfillcolor{currentfill}%
\pgfsetlinewidth{1.003750pt}%
\definecolor{currentstroke}{rgb}{0.800000,0.800000,0.200000}%
\pgfsetstrokecolor{currentstroke}%
\pgfsetdash{}{0pt}%
\pgfpathmoveto{\pgfqpoint{3.905514in}{4.546387in}}%
\pgfpathcurveto{\pgfqpoint{3.911338in}{4.546387in}}{\pgfqpoint{3.916924in}{4.548700in}}{\pgfqpoint{3.921042in}{4.552819in}}%
\pgfpathcurveto{\pgfqpoint{3.925160in}{4.556937in}}{\pgfqpoint{3.927474in}{4.562523in}}{\pgfqpoint{3.927474in}{4.568347in}}%
\pgfpathcurveto{\pgfqpoint{3.927474in}{4.574171in}}{\pgfqpoint{3.925160in}{4.579757in}}{\pgfqpoint{3.921042in}{4.583875in}}%
\pgfpathcurveto{\pgfqpoint{3.916924in}{4.587993in}}{\pgfqpoint{3.911338in}{4.590307in}}{\pgfqpoint{3.905514in}{4.590307in}}%
\pgfpathcurveto{\pgfqpoint{3.899690in}{4.590307in}}{\pgfqpoint{3.894104in}{4.587993in}}{\pgfqpoint{3.889986in}{4.583875in}}%
\pgfpathcurveto{\pgfqpoint{3.885868in}{4.579757in}}{\pgfqpoint{3.883554in}{4.574171in}}{\pgfqpoint{3.883554in}{4.568347in}}%
\pgfpathcurveto{\pgfqpoint{3.883554in}{4.562523in}}{\pgfqpoint{3.885868in}{4.556937in}}{\pgfqpoint{3.889986in}{4.552819in}}%
\pgfpathcurveto{\pgfqpoint{3.894104in}{4.548700in}}{\pgfqpoint{3.899690in}{4.546387in}}{\pgfqpoint{3.905514in}{4.546387in}}%
\pgfpathlineto{\pgfqpoint{3.905514in}{4.546387in}}%
\pgfpathclose%
\pgfusepath{stroke,fill}%
\end{pgfscope}%
\begin{pgfscope}%
\pgfpathrectangle{\pgfqpoint{1.000000in}{1.148311in}}{\pgfqpoint{6.200000in}{5.623377in}}%
\pgfusepath{clip}%
\pgfsetbuttcap%
\pgfsetroundjoin%
\definecolor{currentfill}{rgb}{0.800000,0.800000,0.200000}%
\pgfsetfillcolor{currentfill}%
\pgfsetlinewidth{1.003750pt}%
\definecolor{currentstroke}{rgb}{0.800000,0.800000,0.200000}%
\pgfsetstrokecolor{currentstroke}%
\pgfsetdash{}{0pt}%
\pgfpathmoveto{\pgfqpoint{3.971560in}{4.530916in}}%
\pgfpathcurveto{\pgfqpoint{3.977383in}{4.530916in}}{\pgfqpoint{3.982970in}{4.533230in}}{\pgfqpoint{3.987088in}{4.537348in}}%
\pgfpathcurveto{\pgfqpoint{3.991206in}{4.541467in}}{\pgfqpoint{3.993520in}{4.547053in}}{\pgfqpoint{3.993520in}{4.552877in}}%
\pgfpathcurveto{\pgfqpoint{3.993520in}{4.558701in}}{\pgfqpoint{3.991206in}{4.564287in}}{\pgfqpoint{3.987088in}{4.568405in}}%
\pgfpathcurveto{\pgfqpoint{3.982970in}{4.572523in}}{\pgfqpoint{3.977383in}{4.574837in}}{\pgfqpoint{3.971560in}{4.574837in}}%
\pgfpathcurveto{\pgfqpoint{3.965736in}{4.574837in}}{\pgfqpoint{3.960149in}{4.572523in}}{\pgfqpoint{3.956031in}{4.568405in}}%
\pgfpathcurveto{\pgfqpoint{3.951913in}{4.564287in}}{\pgfqpoint{3.949599in}{4.558701in}}{\pgfqpoint{3.949599in}{4.552877in}}%
\pgfpathcurveto{\pgfqpoint{3.949599in}{4.547053in}}{\pgfqpoint{3.951913in}{4.541467in}}{\pgfqpoint{3.956031in}{4.537348in}}%
\pgfpathcurveto{\pgfqpoint{3.960149in}{4.533230in}}{\pgfqpoint{3.965736in}{4.530916in}}{\pgfqpoint{3.971560in}{4.530916in}}%
\pgfpathlineto{\pgfqpoint{3.971560in}{4.530916in}}%
\pgfpathclose%
\pgfusepath{stroke,fill}%
\end{pgfscope}%
\begin{pgfscope}%
\pgfpathrectangle{\pgfqpoint{1.000000in}{1.148311in}}{\pgfqpoint{6.200000in}{5.623377in}}%
\pgfusepath{clip}%
\pgfsetbuttcap%
\pgfsetroundjoin%
\definecolor{currentfill}{rgb}{0.800000,0.800000,0.200000}%
\pgfsetfillcolor{currentfill}%
\pgfsetlinewidth{1.003750pt}%
\definecolor{currentstroke}{rgb}{0.800000,0.800000,0.200000}%
\pgfsetstrokecolor{currentstroke}%
\pgfsetdash{}{0pt}%
\pgfpathmoveto{\pgfqpoint{4.031758in}{4.506565in}}%
\pgfpathcurveto{\pgfqpoint{4.037582in}{4.506565in}}{\pgfqpoint{4.043168in}{4.508879in}}{\pgfqpoint{4.047286in}{4.512997in}}%
\pgfpathcurveto{\pgfqpoint{4.051404in}{4.517115in}}{\pgfqpoint{4.053718in}{4.522701in}}{\pgfqpoint{4.053718in}{4.528525in}}%
\pgfpathcurveto{\pgfqpoint{4.053718in}{4.534349in}}{\pgfqpoint{4.051404in}{4.539935in}}{\pgfqpoint{4.047286in}{4.544053in}}%
\pgfpathcurveto{\pgfqpoint{4.043168in}{4.548172in}}{\pgfqpoint{4.037582in}{4.550485in}}{\pgfqpoint{4.031758in}{4.550485in}}%
\pgfpathcurveto{\pgfqpoint{4.025934in}{4.550485in}}{\pgfqpoint{4.020348in}{4.548172in}}{\pgfqpoint{4.016229in}{4.544053in}}%
\pgfpathcurveto{\pgfqpoint{4.012111in}{4.539935in}}{\pgfqpoint{4.009797in}{4.534349in}}{\pgfqpoint{4.009797in}{4.528525in}}%
\pgfpathcurveto{\pgfqpoint{4.009797in}{4.522701in}}{\pgfqpoint{4.012111in}{4.517115in}}{\pgfqpoint{4.016229in}{4.512997in}}%
\pgfpathcurveto{\pgfqpoint{4.020348in}{4.508879in}}{\pgfqpoint{4.025934in}{4.506565in}}{\pgfqpoint{4.031758in}{4.506565in}}%
\pgfpathlineto{\pgfqpoint{4.031758in}{4.506565in}}%
\pgfpathclose%
\pgfusepath{stroke,fill}%
\end{pgfscope}%
\begin{pgfscope}%
\pgfpathrectangle{\pgfqpoint{1.000000in}{1.148311in}}{\pgfqpoint{6.200000in}{5.623377in}}%
\pgfusepath{clip}%
\pgfsetbuttcap%
\pgfsetroundjoin%
\definecolor{currentfill}{rgb}{0.800000,0.800000,0.200000}%
\pgfsetfillcolor{currentfill}%
\pgfsetlinewidth{1.003750pt}%
\definecolor{currentstroke}{rgb}{0.800000,0.800000,0.200000}%
\pgfsetstrokecolor{currentstroke}%
\pgfsetdash{}{0pt}%
\pgfpathmoveto{\pgfqpoint{4.094148in}{4.488856in}}%
\pgfpathcurveto{\pgfqpoint{4.099972in}{4.488856in}}{\pgfqpoint{4.105558in}{4.491170in}}{\pgfqpoint{4.109676in}{4.495288in}}%
\pgfpathcurveto{\pgfqpoint{4.113794in}{4.499406in}}{\pgfqpoint{4.116108in}{4.504992in}}{\pgfqpoint{4.116108in}{4.510816in}}%
\pgfpathcurveto{\pgfqpoint{4.116108in}{4.516640in}}{\pgfqpoint{4.113794in}{4.522226in}}{\pgfqpoint{4.109676in}{4.526345in}}%
\pgfpathcurveto{\pgfqpoint{4.105558in}{4.530463in}}{\pgfqpoint{4.099972in}{4.532777in}}{\pgfqpoint{4.094148in}{4.532777in}}%
\pgfpathcurveto{\pgfqpoint{4.088324in}{4.532777in}}{\pgfqpoint{4.082738in}{4.530463in}}{\pgfqpoint{4.078620in}{4.526345in}}%
\pgfpathcurveto{\pgfqpoint{4.074502in}{4.522226in}}{\pgfqpoint{4.072188in}{4.516640in}}{\pgfqpoint{4.072188in}{4.510816in}}%
\pgfpathcurveto{\pgfqpoint{4.072188in}{4.504992in}}{\pgfqpoint{4.074502in}{4.499406in}}{\pgfqpoint{4.078620in}{4.495288in}}%
\pgfpathcurveto{\pgfqpoint{4.082738in}{4.491170in}}{\pgfqpoint{4.088324in}{4.488856in}}{\pgfqpoint{4.094148in}{4.488856in}}%
\pgfpathlineto{\pgfqpoint{4.094148in}{4.488856in}}%
\pgfpathclose%
\pgfusepath{stroke,fill}%
\end{pgfscope}%
\begin{pgfscope}%
\pgfpathrectangle{\pgfqpoint{1.000000in}{1.148311in}}{\pgfqpoint{6.200000in}{5.623377in}}%
\pgfusepath{clip}%
\pgfsetbuttcap%
\pgfsetroundjoin%
\definecolor{currentfill}{rgb}{0.800000,0.800000,0.200000}%
\pgfsetfillcolor{currentfill}%
\pgfsetlinewidth{1.003750pt}%
\definecolor{currentstroke}{rgb}{0.800000,0.800000,0.200000}%
\pgfsetstrokecolor{currentstroke}%
\pgfsetdash{}{0pt}%
\pgfpathmoveto{\pgfqpoint{4.139397in}{4.410894in}}%
\pgfpathcurveto{\pgfqpoint{4.145221in}{4.410894in}}{\pgfqpoint{4.150807in}{4.413208in}}{\pgfqpoint{4.154925in}{4.417326in}}%
\pgfpathcurveto{\pgfqpoint{4.159043in}{4.421444in}}{\pgfqpoint{4.161357in}{4.427030in}}{\pgfqpoint{4.161357in}{4.432854in}}%
\pgfpathcurveto{\pgfqpoint{4.161357in}{4.438678in}}{\pgfqpoint{4.159043in}{4.444265in}}{\pgfqpoint{4.154925in}{4.448383in}}%
\pgfpathcurveto{\pgfqpoint{4.150807in}{4.452501in}}{\pgfqpoint{4.145221in}{4.454815in}}{\pgfqpoint{4.139397in}{4.454815in}}%
\pgfpathcurveto{\pgfqpoint{4.133573in}{4.454815in}}{\pgfqpoint{4.127987in}{4.452501in}}{\pgfqpoint{4.123869in}{4.448383in}}%
\pgfpathcurveto{\pgfqpoint{4.119751in}{4.444265in}}{\pgfqpoint{4.117437in}{4.438678in}}{\pgfqpoint{4.117437in}{4.432854in}}%
\pgfpathcurveto{\pgfqpoint{4.117437in}{4.427030in}}{\pgfqpoint{4.119751in}{4.421444in}}{\pgfqpoint{4.123869in}{4.417326in}}%
\pgfpathcurveto{\pgfqpoint{4.127987in}{4.413208in}}{\pgfqpoint{4.133573in}{4.410894in}}{\pgfqpoint{4.139397in}{4.410894in}}%
\pgfpathlineto{\pgfqpoint{4.139397in}{4.410894in}}%
\pgfpathclose%
\pgfusepath{stroke,fill}%
\end{pgfscope}%
\begin{pgfscope}%
\pgfpathrectangle{\pgfqpoint{1.000000in}{1.148311in}}{\pgfqpoint{6.200000in}{5.623377in}}%
\pgfusepath{clip}%
\pgfsetbuttcap%
\pgfsetroundjoin%
\definecolor{currentfill}{rgb}{0.800000,0.800000,0.200000}%
\pgfsetfillcolor{currentfill}%
\pgfsetlinewidth{1.003750pt}%
\definecolor{currentstroke}{rgb}{0.800000,0.800000,0.200000}%
\pgfsetstrokecolor{currentstroke}%
\pgfsetdash{}{0pt}%
\pgfpathmoveto{\pgfqpoint{4.219210in}{4.459045in}}%
\pgfpathcurveto{\pgfqpoint{4.225033in}{4.459045in}}{\pgfqpoint{4.230620in}{4.461359in}}{\pgfqpoint{4.234738in}{4.465477in}}%
\pgfpathcurveto{\pgfqpoint{4.238856in}{4.469595in}}{\pgfqpoint{4.241170in}{4.475181in}}{\pgfqpoint{4.241170in}{4.481005in}}%
\pgfpathcurveto{\pgfqpoint{4.241170in}{4.486829in}}{\pgfqpoint{4.238856in}{4.492415in}}{\pgfqpoint{4.234738in}{4.496533in}}%
\pgfpathcurveto{\pgfqpoint{4.230620in}{4.500652in}}{\pgfqpoint{4.225033in}{4.502965in}}{\pgfqpoint{4.219210in}{4.502965in}}%
\pgfpathcurveto{\pgfqpoint{4.213386in}{4.502965in}}{\pgfqpoint{4.207799in}{4.500652in}}{\pgfqpoint{4.203681in}{4.496533in}}%
\pgfpathcurveto{\pgfqpoint{4.199563in}{4.492415in}}{\pgfqpoint{4.197249in}{4.486829in}}{\pgfqpoint{4.197249in}{4.481005in}}%
\pgfpathcurveto{\pgfqpoint{4.197249in}{4.475181in}}{\pgfqpoint{4.199563in}{4.469595in}}{\pgfqpoint{4.203681in}{4.465477in}}%
\pgfpathcurveto{\pgfqpoint{4.207799in}{4.461359in}}{\pgfqpoint{4.213386in}{4.459045in}}{\pgfqpoint{4.219210in}{4.459045in}}%
\pgfpathlineto{\pgfqpoint{4.219210in}{4.459045in}}%
\pgfpathclose%
\pgfusepath{stroke,fill}%
\end{pgfscope}%
\begin{pgfscope}%
\pgfpathrectangle{\pgfqpoint{1.000000in}{1.148311in}}{\pgfqpoint{6.200000in}{5.623377in}}%
\pgfusepath{clip}%
\pgfsetbuttcap%
\pgfsetroundjoin%
\definecolor{currentfill}{rgb}{0.800000,0.800000,0.200000}%
\pgfsetfillcolor{currentfill}%
\pgfsetlinewidth{1.003750pt}%
\definecolor{currentstroke}{rgb}{0.800000,0.800000,0.200000}%
\pgfsetstrokecolor{currentstroke}%
\pgfsetdash{}{0pt}%
\pgfpathmoveto{\pgfqpoint{4.294801in}{4.534412in}}%
\pgfpathcurveto{\pgfqpoint{4.300625in}{4.534412in}}{\pgfqpoint{4.306211in}{4.536726in}}{\pgfqpoint{4.310329in}{4.540844in}}%
\pgfpathcurveto{\pgfqpoint{4.314447in}{4.544963in}}{\pgfqpoint{4.316761in}{4.550549in}}{\pgfqpoint{4.316761in}{4.556373in}}%
\pgfpathcurveto{\pgfqpoint{4.316761in}{4.562197in}}{\pgfqpoint{4.314447in}{4.567783in}}{\pgfqpoint{4.310329in}{4.571901in}}%
\pgfpathcurveto{\pgfqpoint{4.306211in}{4.576019in}}{\pgfqpoint{4.300625in}{4.578333in}}{\pgfqpoint{4.294801in}{4.578333in}}%
\pgfpathcurveto{\pgfqpoint{4.288977in}{4.578333in}}{\pgfqpoint{4.283391in}{4.576019in}}{\pgfqpoint{4.279272in}{4.571901in}}%
\pgfpathcurveto{\pgfqpoint{4.275154in}{4.567783in}}{\pgfqpoint{4.272840in}{4.562197in}}{\pgfqpoint{4.272840in}{4.556373in}}%
\pgfpathcurveto{\pgfqpoint{4.272840in}{4.550549in}}{\pgfqpoint{4.275154in}{4.544963in}}{\pgfqpoint{4.279272in}{4.540844in}}%
\pgfpathcurveto{\pgfqpoint{4.283391in}{4.536726in}}{\pgfqpoint{4.288977in}{4.534412in}}{\pgfqpoint{4.294801in}{4.534412in}}%
\pgfpathlineto{\pgfqpoint{4.294801in}{4.534412in}}%
\pgfpathclose%
\pgfusepath{stroke,fill}%
\end{pgfscope}%
\begin{pgfscope}%
\pgfpathrectangle{\pgfqpoint{1.000000in}{1.148311in}}{\pgfqpoint{6.200000in}{5.623377in}}%
\pgfusepath{clip}%
\pgfsetbuttcap%
\pgfsetroundjoin%
\definecolor{currentfill}{rgb}{0.800000,0.800000,0.200000}%
\pgfsetfillcolor{currentfill}%
\pgfsetlinewidth{1.003750pt}%
\definecolor{currentstroke}{rgb}{0.800000,0.800000,0.200000}%
\pgfsetstrokecolor{currentstroke}%
\pgfsetdash{}{0pt}%
\pgfpathmoveto{\pgfqpoint{4.347984in}{4.464543in}}%
\pgfpathcurveto{\pgfqpoint{4.353808in}{4.464543in}}{\pgfqpoint{4.359394in}{4.466857in}}{\pgfqpoint{4.363512in}{4.470975in}}%
\pgfpathcurveto{\pgfqpoint{4.367630in}{4.475093in}}{\pgfqpoint{4.369944in}{4.480679in}}{\pgfqpoint{4.369944in}{4.486503in}}%
\pgfpathcurveto{\pgfqpoint{4.369944in}{4.492327in}}{\pgfqpoint{4.367630in}{4.497914in}}{\pgfqpoint{4.363512in}{4.502032in}}%
\pgfpathcurveto{\pgfqpoint{4.359394in}{4.506150in}}{\pgfqpoint{4.353808in}{4.508464in}}{\pgfqpoint{4.347984in}{4.508464in}}%
\pgfpathcurveto{\pgfqpoint{4.342160in}{4.508464in}}{\pgfqpoint{4.336574in}{4.506150in}}{\pgfqpoint{4.332456in}{4.502032in}}%
\pgfpathcurveto{\pgfqpoint{4.328337in}{4.497914in}}{\pgfqpoint{4.326024in}{4.492327in}}{\pgfqpoint{4.326024in}{4.486503in}}%
\pgfpathcurveto{\pgfqpoint{4.326024in}{4.480679in}}{\pgfqpoint{4.328337in}{4.475093in}}{\pgfqpoint{4.332456in}{4.470975in}}%
\pgfpathcurveto{\pgfqpoint{4.336574in}{4.466857in}}{\pgfqpoint{4.342160in}{4.464543in}}{\pgfqpoint{4.347984in}{4.464543in}}%
\pgfpathlineto{\pgfqpoint{4.347984in}{4.464543in}}%
\pgfpathclose%
\pgfusepath{stroke,fill}%
\end{pgfscope}%
\begin{pgfscope}%
\pgfpathrectangle{\pgfqpoint{1.000000in}{1.148311in}}{\pgfqpoint{6.200000in}{5.623377in}}%
\pgfusepath{clip}%
\pgfsetbuttcap%
\pgfsetroundjoin%
\definecolor{currentfill}{rgb}{0.800000,0.800000,0.200000}%
\pgfsetfillcolor{currentfill}%
\pgfsetlinewidth{1.003750pt}%
\definecolor{currentstroke}{rgb}{0.800000,0.800000,0.200000}%
\pgfsetstrokecolor{currentstroke}%
\pgfsetdash{}{0pt}%
\pgfpathmoveto{\pgfqpoint{4.411077in}{4.496077in}}%
\pgfpathcurveto{\pgfqpoint{4.416901in}{4.496077in}}{\pgfqpoint{4.422487in}{4.498391in}}{\pgfqpoint{4.426606in}{4.502509in}}%
\pgfpathcurveto{\pgfqpoint{4.430724in}{4.506628in}}{\pgfqpoint{4.433038in}{4.512214in}}{\pgfqpoint{4.433038in}{4.518038in}}%
\pgfpathcurveto{\pgfqpoint{4.433038in}{4.523862in}}{\pgfqpoint{4.430724in}{4.529448in}}{\pgfqpoint{4.426606in}{4.533566in}}%
\pgfpathcurveto{\pgfqpoint{4.422487in}{4.537684in}}{\pgfqpoint{4.416901in}{4.539998in}}{\pgfqpoint{4.411077in}{4.539998in}}%
\pgfpathcurveto{\pgfqpoint{4.405253in}{4.539998in}}{\pgfqpoint{4.399667in}{4.537684in}}{\pgfqpoint{4.395549in}{4.533566in}}%
\pgfpathcurveto{\pgfqpoint{4.391431in}{4.529448in}}{\pgfqpoint{4.389117in}{4.523862in}}{\pgfqpoint{4.389117in}{4.518038in}}%
\pgfpathcurveto{\pgfqpoint{4.389117in}{4.512214in}}{\pgfqpoint{4.391431in}{4.506628in}}{\pgfqpoint{4.395549in}{4.502509in}}%
\pgfpathcurveto{\pgfqpoint{4.399667in}{4.498391in}}{\pgfqpoint{4.405253in}{4.496077in}}{\pgfqpoint{4.411077in}{4.496077in}}%
\pgfpathlineto{\pgfqpoint{4.411077in}{4.496077in}}%
\pgfpathclose%
\pgfusepath{stroke,fill}%
\end{pgfscope}%
\begin{pgfscope}%
\pgfpathrectangle{\pgfqpoint{1.000000in}{1.148311in}}{\pgfqpoint{6.200000in}{5.623377in}}%
\pgfusepath{clip}%
\pgfsetbuttcap%
\pgfsetroundjoin%
\definecolor{currentfill}{rgb}{0.800000,0.800000,0.200000}%
\pgfsetfillcolor{currentfill}%
\pgfsetlinewidth{1.003750pt}%
\definecolor{currentstroke}{rgb}{0.800000,0.800000,0.200000}%
\pgfsetstrokecolor{currentstroke}%
\pgfsetdash{}{0pt}%
\pgfpathmoveto{\pgfqpoint{4.470830in}{4.511021in}}%
\pgfpathcurveto{\pgfqpoint{4.476654in}{4.511021in}}{\pgfqpoint{4.482240in}{4.513335in}}{\pgfqpoint{4.486358in}{4.517453in}}%
\pgfpathcurveto{\pgfqpoint{4.490476in}{4.521571in}}{\pgfqpoint{4.492790in}{4.527157in}}{\pgfqpoint{4.492790in}{4.532981in}}%
\pgfpathcurveto{\pgfqpoint{4.492790in}{4.538805in}}{\pgfqpoint{4.490476in}{4.544391in}}{\pgfqpoint{4.486358in}{4.548509in}}%
\pgfpathcurveto{\pgfqpoint{4.482240in}{4.552627in}}{\pgfqpoint{4.476654in}{4.554941in}}{\pgfqpoint{4.470830in}{4.554941in}}%
\pgfpathcurveto{\pgfqpoint{4.465006in}{4.554941in}}{\pgfqpoint{4.459420in}{4.552627in}}{\pgfqpoint{4.455302in}{4.548509in}}%
\pgfpathcurveto{\pgfqpoint{4.451184in}{4.544391in}}{\pgfqpoint{4.448870in}{4.538805in}}{\pgfqpoint{4.448870in}{4.532981in}}%
\pgfpathcurveto{\pgfqpoint{4.448870in}{4.527157in}}{\pgfqpoint{4.451184in}{4.521571in}}{\pgfqpoint{4.455302in}{4.517453in}}%
\pgfpathcurveto{\pgfqpoint{4.459420in}{4.513335in}}{\pgfqpoint{4.465006in}{4.511021in}}{\pgfqpoint{4.470830in}{4.511021in}}%
\pgfpathlineto{\pgfqpoint{4.470830in}{4.511021in}}%
\pgfpathclose%
\pgfusepath{stroke,fill}%
\end{pgfscope}%
\begin{pgfscope}%
\pgfpathrectangle{\pgfqpoint{1.000000in}{1.148311in}}{\pgfqpoint{6.200000in}{5.623377in}}%
\pgfusepath{clip}%
\pgfsetbuttcap%
\pgfsetroundjoin%
\definecolor{currentfill}{rgb}{0.800000,0.800000,0.200000}%
\pgfsetfillcolor{currentfill}%
\pgfsetlinewidth{1.003750pt}%
\definecolor{currentstroke}{rgb}{0.800000,0.800000,0.200000}%
\pgfsetstrokecolor{currentstroke}%
\pgfsetdash{}{0pt}%
\pgfpathmoveto{\pgfqpoint{4.538216in}{4.443687in}}%
\pgfpathcurveto{\pgfqpoint{4.544040in}{4.443687in}}{\pgfqpoint{4.549627in}{4.446001in}}{\pgfqpoint{4.553745in}{4.450119in}}%
\pgfpathcurveto{\pgfqpoint{4.557863in}{4.454237in}}{\pgfqpoint{4.560177in}{4.459824in}}{\pgfqpoint{4.560177in}{4.465647in}}%
\pgfpathcurveto{\pgfqpoint{4.560177in}{4.471471in}}{\pgfqpoint{4.557863in}{4.477058in}}{\pgfqpoint{4.553745in}{4.481176in}}%
\pgfpathcurveto{\pgfqpoint{4.549627in}{4.485294in}}{\pgfqpoint{4.544040in}{4.487608in}}{\pgfqpoint{4.538216in}{4.487608in}}%
\pgfpathcurveto{\pgfqpoint{4.532392in}{4.487608in}}{\pgfqpoint{4.526806in}{4.485294in}}{\pgfqpoint{4.522688in}{4.481176in}}%
\pgfpathcurveto{\pgfqpoint{4.518570in}{4.477058in}}{\pgfqpoint{4.516256in}{4.471471in}}{\pgfqpoint{4.516256in}{4.465647in}}%
\pgfpathcurveto{\pgfqpoint{4.516256in}{4.459824in}}{\pgfqpoint{4.518570in}{4.454237in}}{\pgfqpoint{4.522688in}{4.450119in}}%
\pgfpathcurveto{\pgfqpoint{4.526806in}{4.446001in}}{\pgfqpoint{4.532392in}{4.443687in}}{\pgfqpoint{4.538216in}{4.443687in}}%
\pgfpathlineto{\pgfqpoint{4.538216in}{4.443687in}}%
\pgfpathclose%
\pgfusepath{stroke,fill}%
\end{pgfscope}%
\begin{pgfscope}%
\pgfpathrectangle{\pgfqpoint{1.000000in}{1.148311in}}{\pgfqpoint{6.200000in}{5.623377in}}%
\pgfusepath{clip}%
\pgfsetbuttcap%
\pgfsetroundjoin%
\definecolor{currentfill}{rgb}{0.200000,0.200000,0.800000}%
\pgfsetfillcolor{currentfill}%
\pgfsetlinewidth{1.003750pt}%
\definecolor{currentstroke}{rgb}{0.200000,0.200000,0.800000}%
\pgfsetstrokecolor{currentstroke}%
\pgfsetdash{}{0pt}%
\pgfpathmoveto{\pgfqpoint{4.584478in}{4.550497in}}%
\pgfpathcurveto{\pgfqpoint{4.590302in}{4.550497in}}{\pgfqpoint{4.595888in}{4.552810in}}{\pgfqpoint{4.600006in}{4.556929in}}%
\pgfpathcurveto{\pgfqpoint{4.604125in}{4.561047in}}{\pgfqpoint{4.606438in}{4.566633in}}{\pgfqpoint{4.606438in}{4.572457in}}%
\pgfpathcurveto{\pgfqpoint{4.606438in}{4.578281in}}{\pgfqpoint{4.604125in}{4.583867in}}{\pgfqpoint{4.600006in}{4.587985in}}%
\pgfpathcurveto{\pgfqpoint{4.595888in}{4.592103in}}{\pgfqpoint{4.590302in}{4.594417in}}{\pgfqpoint{4.584478in}{4.594417in}}%
\pgfpathcurveto{\pgfqpoint{4.578654in}{4.594417in}}{\pgfqpoint{4.573068in}{4.592103in}}{\pgfqpoint{4.568950in}{4.587985in}}%
\pgfpathcurveto{\pgfqpoint{4.564832in}{4.583867in}}{\pgfqpoint{4.562518in}{4.578281in}}{\pgfqpoint{4.562518in}{4.572457in}}%
\pgfpathcurveto{\pgfqpoint{4.562518in}{4.566633in}}{\pgfqpoint{4.564832in}{4.561047in}}{\pgfqpoint{4.568950in}{4.556929in}}%
\pgfpathcurveto{\pgfqpoint{4.573068in}{4.552810in}}{\pgfqpoint{4.578654in}{4.550497in}}{\pgfqpoint{4.584478in}{4.550497in}}%
\pgfpathlineto{\pgfqpoint{4.584478in}{4.550497in}}%
\pgfpathclose%
\pgfusepath{stroke,fill}%
\end{pgfscope}%
\begin{pgfscope}%
\pgfpathrectangle{\pgfqpoint{1.000000in}{1.148311in}}{\pgfqpoint{6.200000in}{5.623377in}}%
\pgfusepath{clip}%
\pgfsetbuttcap%
\pgfsetroundjoin%
\definecolor{currentfill}{rgb}{0.800000,0.800000,0.200000}%
\pgfsetfillcolor{currentfill}%
\pgfsetlinewidth{1.003750pt}%
\definecolor{currentstroke}{rgb}{0.800000,0.800000,0.200000}%
\pgfsetstrokecolor{currentstroke}%
\pgfsetdash{}{0pt}%
\pgfpathmoveto{\pgfqpoint{4.661467in}{4.478376in}}%
\pgfpathcurveto{\pgfqpoint{4.667291in}{4.478376in}}{\pgfqpoint{4.672877in}{4.480689in}}{\pgfqpoint{4.676995in}{4.484808in}}%
\pgfpathcurveto{\pgfqpoint{4.681113in}{4.488926in}}{\pgfqpoint{4.683427in}{4.494512in}}{\pgfqpoint{4.683427in}{4.500336in}}%
\pgfpathcurveto{\pgfqpoint{4.683427in}{4.506160in}}{\pgfqpoint{4.681113in}{4.511746in}}{\pgfqpoint{4.676995in}{4.515864in}}%
\pgfpathcurveto{\pgfqpoint{4.672877in}{4.519982in}}{\pgfqpoint{4.667291in}{4.522296in}}{\pgfqpoint{4.661467in}{4.522296in}}%
\pgfpathcurveto{\pgfqpoint{4.655643in}{4.522296in}}{\pgfqpoint{4.650057in}{4.519982in}}{\pgfqpoint{4.645939in}{4.515864in}}%
\pgfpathcurveto{\pgfqpoint{4.641820in}{4.511746in}}{\pgfqpoint{4.639507in}{4.506160in}}{\pgfqpoint{4.639507in}{4.500336in}}%
\pgfpathcurveto{\pgfqpoint{4.639507in}{4.494512in}}{\pgfqpoint{4.641820in}{4.488926in}}{\pgfqpoint{4.645939in}{4.484808in}}%
\pgfpathcurveto{\pgfqpoint{4.650057in}{4.480689in}}{\pgfqpoint{4.655643in}{4.478376in}}{\pgfqpoint{4.661467in}{4.478376in}}%
\pgfpathlineto{\pgfqpoint{4.661467in}{4.478376in}}%
\pgfpathclose%
\pgfusepath{stroke,fill}%
\end{pgfscope}%
\begin{pgfscope}%
\pgfpathrectangle{\pgfqpoint{1.000000in}{1.148311in}}{\pgfqpoint{6.200000in}{5.623377in}}%
\pgfusepath{clip}%
\pgfsetbuttcap%
\pgfsetroundjoin%
\definecolor{currentfill}{rgb}{0.200000,0.200000,0.800000}%
\pgfsetfillcolor{currentfill}%
\pgfsetlinewidth{1.003750pt}%
\definecolor{currentstroke}{rgb}{0.200000,0.200000,0.800000}%
\pgfsetstrokecolor{currentstroke}%
\pgfsetdash{}{0pt}%
\pgfpathmoveto{\pgfqpoint{4.701865in}{4.561627in}}%
\pgfpathcurveto{\pgfqpoint{4.707689in}{4.561627in}}{\pgfqpoint{4.713275in}{4.563941in}}{\pgfqpoint{4.717393in}{4.568059in}}%
\pgfpathcurveto{\pgfqpoint{4.721511in}{4.572178in}}{\pgfqpoint{4.723825in}{4.577764in}}{\pgfqpoint{4.723825in}{4.583588in}}%
\pgfpathcurveto{\pgfqpoint{4.723825in}{4.589412in}}{\pgfqpoint{4.721511in}{4.594998in}}{\pgfqpoint{4.717393in}{4.599116in}}%
\pgfpathcurveto{\pgfqpoint{4.713275in}{4.603234in}}{\pgfqpoint{4.707689in}{4.605548in}}{\pgfqpoint{4.701865in}{4.605548in}}%
\pgfpathcurveto{\pgfqpoint{4.696041in}{4.605548in}}{\pgfqpoint{4.690455in}{4.603234in}}{\pgfqpoint{4.686337in}{4.599116in}}%
\pgfpathcurveto{\pgfqpoint{4.682218in}{4.594998in}}{\pgfqpoint{4.679905in}{4.589412in}}{\pgfqpoint{4.679905in}{4.583588in}}%
\pgfpathcurveto{\pgfqpoint{4.679905in}{4.577764in}}{\pgfqpoint{4.682218in}{4.572178in}}{\pgfqpoint{4.686337in}{4.568059in}}%
\pgfpathcurveto{\pgfqpoint{4.690455in}{4.563941in}}{\pgfqpoint{4.696041in}{4.561627in}}{\pgfqpoint{4.701865in}{4.561627in}}%
\pgfpathlineto{\pgfqpoint{4.701865in}{4.561627in}}%
\pgfpathclose%
\pgfusepath{stroke,fill}%
\end{pgfscope}%
\begin{pgfscope}%
\pgfpathrectangle{\pgfqpoint{1.000000in}{1.148311in}}{\pgfqpoint{6.200000in}{5.623377in}}%
\pgfusepath{clip}%
\pgfsetbuttcap%
\pgfsetroundjoin%
\definecolor{currentfill}{rgb}{0.800000,0.200000,0.200000}%
\pgfsetfillcolor{currentfill}%
\pgfsetlinewidth{1.003750pt}%
\definecolor{currentstroke}{rgb}{0.800000,0.200000,0.200000}%
\pgfsetstrokecolor{currentstroke}%
\pgfsetdash{}{0pt}%
\pgfpathmoveto{\pgfqpoint{4.778336in}{4.526467in}}%
\pgfpathcurveto{\pgfqpoint{4.784160in}{4.526467in}}{\pgfqpoint{4.789746in}{4.528781in}}{\pgfqpoint{4.793864in}{4.532899in}}%
\pgfpathcurveto{\pgfqpoint{4.797982in}{4.537017in}}{\pgfqpoint{4.800296in}{4.542603in}}{\pgfqpoint{4.800296in}{4.548427in}}%
\pgfpathcurveto{\pgfqpoint{4.800296in}{4.554251in}}{\pgfqpoint{4.797982in}{4.559837in}}{\pgfqpoint{4.793864in}{4.563955in}}%
\pgfpathcurveto{\pgfqpoint{4.789746in}{4.568073in}}{\pgfqpoint{4.784160in}{4.570387in}}{\pgfqpoint{4.778336in}{4.570387in}}%
\pgfpathcurveto{\pgfqpoint{4.772512in}{4.570387in}}{\pgfqpoint{4.766925in}{4.568073in}}{\pgfqpoint{4.762807in}{4.563955in}}%
\pgfpathcurveto{\pgfqpoint{4.758689in}{4.559837in}}{\pgfqpoint{4.756375in}{4.554251in}}{\pgfqpoint{4.756375in}{4.548427in}}%
\pgfpathcurveto{\pgfqpoint{4.756375in}{4.542603in}}{\pgfqpoint{4.758689in}{4.537017in}}{\pgfqpoint{4.762807in}{4.532899in}}%
\pgfpathcurveto{\pgfqpoint{4.766925in}{4.528781in}}{\pgfqpoint{4.772512in}{4.526467in}}{\pgfqpoint{4.778336in}{4.526467in}}%
\pgfpathlineto{\pgfqpoint{4.778336in}{4.526467in}}%
\pgfpathclose%
\pgfusepath{stroke,fill}%
\end{pgfscope}%
\begin{pgfscope}%
\pgfpathrectangle{\pgfqpoint{1.000000in}{1.148311in}}{\pgfqpoint{6.200000in}{5.623377in}}%
\pgfusepath{clip}%
\pgfsetbuttcap%
\pgfsetroundjoin%
\definecolor{currentfill}{rgb}{0.800000,0.800000,0.200000}%
\pgfsetfillcolor{currentfill}%
\pgfsetlinewidth{1.003750pt}%
\definecolor{currentstroke}{rgb}{0.800000,0.800000,0.200000}%
\pgfsetstrokecolor{currentstroke}%
\pgfsetdash{}{0pt}%
\pgfpathmoveto{\pgfqpoint{4.851741in}{4.516382in}}%
\pgfpathcurveto{\pgfqpoint{4.857565in}{4.516382in}}{\pgfqpoint{4.863151in}{4.518695in}}{\pgfqpoint{4.867269in}{4.522814in}}%
\pgfpathcurveto{\pgfqpoint{4.871387in}{4.526932in}}{\pgfqpoint{4.873701in}{4.532518in}}{\pgfqpoint{4.873701in}{4.538342in}}%
\pgfpathcurveto{\pgfqpoint{4.873701in}{4.544166in}}{\pgfqpoint{4.871387in}{4.549752in}}{\pgfqpoint{4.867269in}{4.553870in}}%
\pgfpathcurveto{\pgfqpoint{4.863151in}{4.557988in}}{\pgfqpoint{4.857565in}{4.560302in}}{\pgfqpoint{4.851741in}{4.560302in}}%
\pgfpathcurveto{\pgfqpoint{4.845917in}{4.560302in}}{\pgfqpoint{4.840331in}{4.557988in}}{\pgfqpoint{4.836213in}{4.553870in}}%
\pgfpathcurveto{\pgfqpoint{4.832094in}{4.549752in}}{\pgfqpoint{4.829781in}{4.544166in}}{\pgfqpoint{4.829781in}{4.538342in}}%
\pgfpathcurveto{\pgfqpoint{4.829781in}{4.532518in}}{\pgfqpoint{4.832094in}{4.526932in}}{\pgfqpoint{4.836213in}{4.522814in}}%
\pgfpathcurveto{\pgfqpoint{4.840331in}{4.518695in}}{\pgfqpoint{4.845917in}{4.516382in}}{\pgfqpoint{4.851741in}{4.516382in}}%
\pgfpathlineto{\pgfqpoint{4.851741in}{4.516382in}}%
\pgfpathclose%
\pgfusepath{stroke,fill}%
\end{pgfscope}%
\begin{pgfscope}%
\pgfpathrectangle{\pgfqpoint{1.000000in}{1.148311in}}{\pgfqpoint{6.200000in}{5.623377in}}%
\pgfusepath{clip}%
\pgfsetbuttcap%
\pgfsetroundjoin%
\definecolor{currentfill}{rgb}{0.800000,0.800000,0.200000}%
\pgfsetfillcolor{currentfill}%
\pgfsetlinewidth{1.003750pt}%
\definecolor{currentstroke}{rgb}{0.800000,0.800000,0.200000}%
\pgfsetstrokecolor{currentstroke}%
\pgfsetdash{}{0pt}%
\pgfpathmoveto{\pgfqpoint{4.883262in}{4.595124in}}%
\pgfpathcurveto{\pgfqpoint{4.889086in}{4.595124in}}{\pgfqpoint{4.894672in}{4.597437in}}{\pgfqpoint{4.898790in}{4.601556in}}%
\pgfpathcurveto{\pgfqpoint{4.902908in}{4.605674in}}{\pgfqpoint{4.905222in}{4.611260in}}{\pgfqpoint{4.905222in}{4.617084in}}%
\pgfpathcurveto{\pgfqpoint{4.905222in}{4.622908in}}{\pgfqpoint{4.902908in}{4.628494in}}{\pgfqpoint{4.898790in}{4.632612in}}%
\pgfpathcurveto{\pgfqpoint{4.894672in}{4.636730in}}{\pgfqpoint{4.889086in}{4.639044in}}{\pgfqpoint{4.883262in}{4.639044in}}%
\pgfpathcurveto{\pgfqpoint{4.877438in}{4.639044in}}{\pgfqpoint{4.871852in}{4.636730in}}{\pgfqpoint{4.867734in}{4.632612in}}%
\pgfpathcurveto{\pgfqpoint{4.863616in}{4.628494in}}{\pgfqpoint{4.861302in}{4.622908in}}{\pgfqpoint{4.861302in}{4.617084in}}%
\pgfpathcurveto{\pgfqpoint{4.861302in}{4.611260in}}{\pgfqpoint{4.863616in}{4.605674in}}{\pgfqpoint{4.867734in}{4.601556in}}%
\pgfpathcurveto{\pgfqpoint{4.871852in}{4.597437in}}{\pgfqpoint{4.877438in}{4.595124in}}{\pgfqpoint{4.883262in}{4.595124in}}%
\pgfpathlineto{\pgfqpoint{4.883262in}{4.595124in}}%
\pgfpathclose%
\pgfusepath{stroke,fill}%
\end{pgfscope}%
\begin{pgfscope}%
\pgfpathrectangle{\pgfqpoint{1.000000in}{1.148311in}}{\pgfqpoint{6.200000in}{5.623377in}}%
\pgfusepath{clip}%
\pgfsetbuttcap%
\pgfsetroundjoin%
\definecolor{currentfill}{rgb}{0.800000,0.800000,0.200000}%
\pgfsetfillcolor{currentfill}%
\pgfsetlinewidth{1.003750pt}%
\definecolor{currentstroke}{rgb}{0.800000,0.800000,0.200000}%
\pgfsetstrokecolor{currentstroke}%
\pgfsetdash{}{0pt}%
\pgfpathmoveto{\pgfqpoint{4.925396in}{4.643644in}}%
\pgfpathcurveto{\pgfqpoint{4.931220in}{4.643644in}}{\pgfqpoint{4.936806in}{4.645958in}}{\pgfqpoint{4.940924in}{4.650076in}}%
\pgfpathcurveto{\pgfqpoint{4.945042in}{4.654194in}}{\pgfqpoint{4.947356in}{4.659780in}}{\pgfqpoint{4.947356in}{4.665604in}}%
\pgfpathcurveto{\pgfqpoint{4.947356in}{4.671428in}}{\pgfqpoint{4.945042in}{4.677014in}}{\pgfqpoint{4.940924in}{4.681132in}}%
\pgfpathcurveto{\pgfqpoint{4.936806in}{4.685250in}}{\pgfqpoint{4.931220in}{4.687564in}}{\pgfqpoint{4.925396in}{4.687564in}}%
\pgfpathcurveto{\pgfqpoint{4.919572in}{4.687564in}}{\pgfqpoint{4.913986in}{4.685250in}}{\pgfqpoint{4.909868in}{4.681132in}}%
\pgfpathcurveto{\pgfqpoint{4.905749in}{4.677014in}}{\pgfqpoint{4.903436in}{4.671428in}}{\pgfqpoint{4.903436in}{4.665604in}}%
\pgfpathcurveto{\pgfqpoint{4.903436in}{4.659780in}}{\pgfqpoint{4.905749in}{4.654194in}}{\pgfqpoint{4.909868in}{4.650076in}}%
\pgfpathcurveto{\pgfqpoint{4.913986in}{4.645958in}}{\pgfqpoint{4.919572in}{4.643644in}}{\pgfqpoint{4.925396in}{4.643644in}}%
\pgfpathlineto{\pgfqpoint{4.925396in}{4.643644in}}%
\pgfpathclose%
\pgfusepath{stroke,fill}%
\end{pgfscope}%
\begin{pgfscope}%
\pgfpathrectangle{\pgfqpoint{1.000000in}{1.148311in}}{\pgfqpoint{6.200000in}{5.623377in}}%
\pgfusepath{clip}%
\pgfsetbuttcap%
\pgfsetroundjoin%
\definecolor{currentfill}{rgb}{0.800000,0.800000,0.200000}%
\pgfsetfillcolor{currentfill}%
\pgfsetlinewidth{1.003750pt}%
\definecolor{currentstroke}{rgb}{0.800000,0.800000,0.200000}%
\pgfsetstrokecolor{currentstroke}%
\pgfsetdash{}{0pt}%
\pgfpathmoveto{\pgfqpoint{4.994753in}{4.649776in}}%
\pgfpathcurveto{\pgfqpoint{5.000577in}{4.649776in}}{\pgfqpoint{5.006163in}{4.652090in}}{\pgfqpoint{5.010281in}{4.656208in}}%
\pgfpathcurveto{\pgfqpoint{5.014399in}{4.660326in}}{\pgfqpoint{5.016713in}{4.665912in}}{\pgfqpoint{5.016713in}{4.671736in}}%
\pgfpathcurveto{\pgfqpoint{5.016713in}{4.677560in}}{\pgfqpoint{5.014399in}{4.683146in}}{\pgfqpoint{5.010281in}{4.687264in}}%
\pgfpathcurveto{\pgfqpoint{5.006163in}{4.691382in}}{\pgfqpoint{5.000577in}{4.693696in}}{\pgfqpoint{4.994753in}{4.693696in}}%
\pgfpathcurveto{\pgfqpoint{4.988929in}{4.693696in}}{\pgfqpoint{4.983343in}{4.691382in}}{\pgfqpoint{4.979225in}{4.687264in}}%
\pgfpathcurveto{\pgfqpoint{4.975107in}{4.683146in}}{\pgfqpoint{4.972793in}{4.677560in}}{\pgfqpoint{4.972793in}{4.671736in}}%
\pgfpathcurveto{\pgfqpoint{4.972793in}{4.665912in}}{\pgfqpoint{4.975107in}{4.660326in}}{\pgfqpoint{4.979225in}{4.656208in}}%
\pgfpathcurveto{\pgfqpoint{4.983343in}{4.652090in}}{\pgfqpoint{4.988929in}{4.649776in}}{\pgfqpoint{4.994753in}{4.649776in}}%
\pgfpathlineto{\pgfqpoint{4.994753in}{4.649776in}}%
\pgfpathclose%
\pgfusepath{stroke,fill}%
\end{pgfscope}%
\begin{pgfscope}%
\pgfpathrectangle{\pgfqpoint{1.000000in}{1.148311in}}{\pgfqpoint{6.200000in}{5.623377in}}%
\pgfusepath{clip}%
\pgfsetbuttcap%
\pgfsetroundjoin%
\definecolor{currentfill}{rgb}{0.800000,0.800000,0.200000}%
\pgfsetfillcolor{currentfill}%
\pgfsetlinewidth{1.003750pt}%
\definecolor{currentstroke}{rgb}{0.800000,0.800000,0.200000}%
\pgfsetstrokecolor{currentstroke}%
\pgfsetdash{}{0pt}%
\pgfpathmoveto{\pgfqpoint{5.068205in}{4.657956in}}%
\pgfpathcurveto{\pgfqpoint{5.074029in}{4.657956in}}{\pgfqpoint{5.079615in}{4.660270in}}{\pgfqpoint{5.083733in}{4.664388in}}%
\pgfpathcurveto{\pgfqpoint{5.087851in}{4.668507in}}{\pgfqpoint{5.090165in}{4.674093in}}{\pgfqpoint{5.090165in}{4.679917in}}%
\pgfpathcurveto{\pgfqpoint{5.090165in}{4.685741in}}{\pgfqpoint{5.087851in}{4.691327in}}{\pgfqpoint{5.083733in}{4.695445in}}%
\pgfpathcurveto{\pgfqpoint{5.079615in}{4.699563in}}{\pgfqpoint{5.074029in}{4.701877in}}{\pgfqpoint{5.068205in}{4.701877in}}%
\pgfpathcurveto{\pgfqpoint{5.062381in}{4.701877in}}{\pgfqpoint{5.056795in}{4.699563in}}{\pgfqpoint{5.052677in}{4.695445in}}%
\pgfpathcurveto{\pgfqpoint{5.048558in}{4.691327in}}{\pgfqpoint{5.046245in}{4.685741in}}{\pgfqpoint{5.046245in}{4.679917in}}%
\pgfpathcurveto{\pgfqpoint{5.046245in}{4.674093in}}{\pgfqpoint{5.048558in}{4.668507in}}{\pgfqpoint{5.052677in}{4.664388in}}%
\pgfpathcurveto{\pgfqpoint{5.056795in}{4.660270in}}{\pgfqpoint{5.062381in}{4.657956in}}{\pgfqpoint{5.068205in}{4.657956in}}%
\pgfpathlineto{\pgfqpoint{5.068205in}{4.657956in}}%
\pgfpathclose%
\pgfusepath{stroke,fill}%
\end{pgfscope}%
\begin{pgfscope}%
\pgfpathrectangle{\pgfqpoint{1.000000in}{1.148311in}}{\pgfqpoint{6.200000in}{5.623377in}}%
\pgfusepath{clip}%
\pgfsetbuttcap%
\pgfsetroundjoin%
\definecolor{currentfill}{rgb}{0.800000,0.800000,0.200000}%
\pgfsetfillcolor{currentfill}%
\pgfsetlinewidth{1.003750pt}%
\definecolor{currentstroke}{rgb}{0.800000,0.800000,0.200000}%
\pgfsetstrokecolor{currentstroke}%
\pgfsetdash{}{0pt}%
\pgfpathmoveto{\pgfqpoint{5.108551in}{4.709451in}}%
\pgfpathcurveto{\pgfqpoint{5.114374in}{4.709451in}}{\pgfqpoint{5.119961in}{4.711765in}}{\pgfqpoint{5.124079in}{4.715883in}}%
\pgfpathcurveto{\pgfqpoint{5.128197in}{4.720001in}}{\pgfqpoint{5.130511in}{4.725587in}}{\pgfqpoint{5.130511in}{4.731411in}}%
\pgfpathcurveto{\pgfqpoint{5.130511in}{4.737235in}}{\pgfqpoint{5.128197in}{4.742822in}}{\pgfqpoint{5.124079in}{4.746940in}}%
\pgfpathcurveto{\pgfqpoint{5.119961in}{4.751058in}}{\pgfqpoint{5.114374in}{4.753372in}}{\pgfqpoint{5.108551in}{4.753372in}}%
\pgfpathcurveto{\pgfqpoint{5.102727in}{4.753372in}}{\pgfqpoint{5.097140in}{4.751058in}}{\pgfqpoint{5.093022in}{4.746940in}}%
\pgfpathcurveto{\pgfqpoint{5.088904in}{4.742822in}}{\pgfqpoint{5.086590in}{4.737235in}}{\pgfqpoint{5.086590in}{4.731411in}}%
\pgfpathcurveto{\pgfqpoint{5.086590in}{4.725587in}}{\pgfqpoint{5.088904in}{4.720001in}}{\pgfqpoint{5.093022in}{4.715883in}}%
\pgfpathcurveto{\pgfqpoint{5.097140in}{4.711765in}}{\pgfqpoint{5.102727in}{4.709451in}}{\pgfqpoint{5.108551in}{4.709451in}}%
\pgfpathlineto{\pgfqpoint{5.108551in}{4.709451in}}%
\pgfpathclose%
\pgfusepath{stroke,fill}%
\end{pgfscope}%
\begin{pgfscope}%
\pgfpathrectangle{\pgfqpoint{1.000000in}{1.148311in}}{\pgfqpoint{6.200000in}{5.623377in}}%
\pgfusepath{clip}%
\pgfsetbuttcap%
\pgfsetroundjoin%
\definecolor{currentfill}{rgb}{0.800000,0.800000,0.200000}%
\pgfsetfillcolor{currentfill}%
\pgfsetlinewidth{1.003750pt}%
\definecolor{currentstroke}{rgb}{0.800000,0.800000,0.200000}%
\pgfsetstrokecolor{currentstroke}%
\pgfsetdash{}{0pt}%
\pgfpathmoveto{\pgfqpoint{5.143288in}{4.764461in}}%
\pgfpathcurveto{\pgfqpoint{5.149112in}{4.764461in}}{\pgfqpoint{5.154698in}{4.766775in}}{\pgfqpoint{5.158816in}{4.770894in}}%
\pgfpathcurveto{\pgfqpoint{5.162934in}{4.775012in}}{\pgfqpoint{5.165248in}{4.780598in}}{\pgfqpoint{5.165248in}{4.786422in}}%
\pgfpathcurveto{\pgfqpoint{5.165248in}{4.792246in}}{\pgfqpoint{5.162934in}{4.797832in}}{\pgfqpoint{5.158816in}{4.801950in}}%
\pgfpathcurveto{\pgfqpoint{5.154698in}{4.806068in}}{\pgfqpoint{5.149112in}{4.808382in}}{\pgfqpoint{5.143288in}{4.808382in}}%
\pgfpathcurveto{\pgfqpoint{5.137464in}{4.808382in}}{\pgfqpoint{5.131878in}{4.806068in}}{\pgfqpoint{5.127759in}{4.801950in}}%
\pgfpathcurveto{\pgfqpoint{5.123641in}{4.797832in}}{\pgfqpoint{5.121327in}{4.792246in}}{\pgfqpoint{5.121327in}{4.786422in}}%
\pgfpathcurveto{\pgfqpoint{5.121327in}{4.780598in}}{\pgfqpoint{5.123641in}{4.775012in}}{\pgfqpoint{5.127759in}{4.770894in}}%
\pgfpathcurveto{\pgfqpoint{5.131878in}{4.766775in}}{\pgfqpoint{5.137464in}{4.764461in}}{\pgfqpoint{5.143288in}{4.764461in}}%
\pgfpathlineto{\pgfqpoint{5.143288in}{4.764461in}}%
\pgfpathclose%
\pgfusepath{stroke,fill}%
\end{pgfscope}%
\begin{pgfscope}%
\pgfpathrectangle{\pgfqpoint{1.000000in}{1.148311in}}{\pgfqpoint{6.200000in}{5.623377in}}%
\pgfusepath{clip}%
\pgfsetbuttcap%
\pgfsetroundjoin%
\definecolor{currentfill}{rgb}{0.800000,0.800000,0.200000}%
\pgfsetfillcolor{currentfill}%
\pgfsetlinewidth{1.003750pt}%
\definecolor{currentstroke}{rgb}{0.800000,0.800000,0.200000}%
\pgfsetstrokecolor{currentstroke}%
\pgfsetdash{}{0pt}%
\pgfpathmoveto{\pgfqpoint{5.162015in}{4.830783in}}%
\pgfpathcurveto{\pgfqpoint{5.167839in}{4.830783in}}{\pgfqpoint{5.173425in}{4.833097in}}{\pgfqpoint{5.177543in}{4.837215in}}%
\pgfpathcurveto{\pgfqpoint{5.181661in}{4.841333in}}{\pgfqpoint{5.183975in}{4.846919in}}{\pgfqpoint{5.183975in}{4.852743in}}%
\pgfpathcurveto{\pgfqpoint{5.183975in}{4.858567in}}{\pgfqpoint{5.181661in}{4.864153in}}{\pgfqpoint{5.177543in}{4.868271in}}%
\pgfpathcurveto{\pgfqpoint{5.173425in}{4.872389in}}{\pgfqpoint{5.167839in}{4.874703in}}{\pgfqpoint{5.162015in}{4.874703in}}%
\pgfpathcurveto{\pgfqpoint{5.156191in}{4.874703in}}{\pgfqpoint{5.150605in}{4.872389in}}{\pgfqpoint{5.146487in}{4.868271in}}%
\pgfpathcurveto{\pgfqpoint{5.142369in}{4.864153in}}{\pgfqpoint{5.140055in}{4.858567in}}{\pgfqpoint{5.140055in}{4.852743in}}%
\pgfpathcurveto{\pgfqpoint{5.140055in}{4.846919in}}{\pgfqpoint{5.142369in}{4.841333in}}{\pgfqpoint{5.146487in}{4.837215in}}%
\pgfpathcurveto{\pgfqpoint{5.150605in}{4.833097in}}{\pgfqpoint{5.156191in}{4.830783in}}{\pgfqpoint{5.162015in}{4.830783in}}%
\pgfpathlineto{\pgfqpoint{5.162015in}{4.830783in}}%
\pgfpathclose%
\pgfusepath{stroke,fill}%
\end{pgfscope}%
\begin{pgfscope}%
\pgfpathrectangle{\pgfqpoint{1.000000in}{1.148311in}}{\pgfqpoint{6.200000in}{5.623377in}}%
\pgfusepath{clip}%
\pgfsetbuttcap%
\pgfsetroundjoin%
\definecolor{currentfill}{rgb}{0.800000,0.800000,0.200000}%
\pgfsetfillcolor{currentfill}%
\pgfsetlinewidth{1.003750pt}%
\definecolor{currentstroke}{rgb}{0.800000,0.800000,0.200000}%
\pgfsetstrokecolor{currentstroke}%
\pgfsetdash{}{0pt}%
\pgfpathmoveto{\pgfqpoint{5.213390in}{4.868611in}}%
\pgfpathcurveto{\pgfqpoint{5.219214in}{4.868611in}}{\pgfqpoint{5.224800in}{4.870925in}}{\pgfqpoint{5.228918in}{4.875043in}}%
\pgfpathcurveto{\pgfqpoint{5.233036in}{4.879161in}}{\pgfqpoint{5.235350in}{4.884747in}}{\pgfqpoint{5.235350in}{4.890571in}}%
\pgfpathcurveto{\pgfqpoint{5.235350in}{4.896395in}}{\pgfqpoint{5.233036in}{4.901981in}}{\pgfqpoint{5.228918in}{4.906099in}}%
\pgfpathcurveto{\pgfqpoint{5.224800in}{4.910218in}}{\pgfqpoint{5.219214in}{4.912531in}}{\pgfqpoint{5.213390in}{4.912531in}}%
\pgfpathcurveto{\pgfqpoint{5.207566in}{4.912531in}}{\pgfqpoint{5.201979in}{4.910218in}}{\pgfqpoint{5.197861in}{4.906099in}}%
\pgfpathcurveto{\pgfqpoint{5.193743in}{4.901981in}}{\pgfqpoint{5.191429in}{4.896395in}}{\pgfqpoint{5.191429in}{4.890571in}}%
\pgfpathcurveto{\pgfqpoint{5.191429in}{4.884747in}}{\pgfqpoint{5.193743in}{4.879161in}}{\pgfqpoint{5.197861in}{4.875043in}}%
\pgfpathcurveto{\pgfqpoint{5.201979in}{4.870925in}}{\pgfqpoint{5.207566in}{4.868611in}}{\pgfqpoint{5.213390in}{4.868611in}}%
\pgfpathlineto{\pgfqpoint{5.213390in}{4.868611in}}%
\pgfpathclose%
\pgfusepath{stroke,fill}%
\end{pgfscope}%
\begin{pgfscope}%
\pgfpathrectangle{\pgfqpoint{1.000000in}{1.148311in}}{\pgfqpoint{6.200000in}{5.623377in}}%
\pgfusepath{clip}%
\pgfsetbuttcap%
\pgfsetroundjoin%
\definecolor{currentfill}{rgb}{0.800000,0.800000,0.200000}%
\pgfsetfillcolor{currentfill}%
\pgfsetlinewidth{1.003750pt}%
\definecolor{currentstroke}{rgb}{0.800000,0.800000,0.200000}%
\pgfsetstrokecolor{currentstroke}%
\pgfsetdash{}{0pt}%
\pgfpathmoveto{\pgfqpoint{5.285081in}{4.896238in}}%
\pgfpathcurveto{\pgfqpoint{5.290905in}{4.896238in}}{\pgfqpoint{5.296491in}{4.898552in}}{\pgfqpoint{5.300609in}{4.902670in}}%
\pgfpathcurveto{\pgfqpoint{5.304727in}{4.906789in}}{\pgfqpoint{5.307041in}{4.912375in}}{\pgfqpoint{5.307041in}{4.918199in}}%
\pgfpathcurveto{\pgfqpoint{5.307041in}{4.924023in}}{\pgfqpoint{5.304727in}{4.929609in}}{\pgfqpoint{5.300609in}{4.933727in}}%
\pgfpathcurveto{\pgfqpoint{5.296491in}{4.937845in}}{\pgfqpoint{5.290905in}{4.940159in}}{\pgfqpoint{5.285081in}{4.940159in}}%
\pgfpathcurveto{\pgfqpoint{5.279257in}{4.940159in}}{\pgfqpoint{5.273671in}{4.937845in}}{\pgfqpoint{5.269553in}{4.933727in}}%
\pgfpathcurveto{\pgfqpoint{5.265434in}{4.929609in}}{\pgfqpoint{5.263121in}{4.924023in}}{\pgfqpoint{5.263121in}{4.918199in}}%
\pgfpathcurveto{\pgfqpoint{5.263121in}{4.912375in}}{\pgfqpoint{5.265434in}{4.906789in}}{\pgfqpoint{5.269553in}{4.902670in}}%
\pgfpathcurveto{\pgfqpoint{5.273671in}{4.898552in}}{\pgfqpoint{5.279257in}{4.896238in}}{\pgfqpoint{5.285081in}{4.896238in}}%
\pgfpathlineto{\pgfqpoint{5.285081in}{4.896238in}}%
\pgfpathclose%
\pgfusepath{stroke,fill}%
\end{pgfscope}%
\begin{pgfscope}%
\pgfpathrectangle{\pgfqpoint{1.000000in}{1.148311in}}{\pgfqpoint{6.200000in}{5.623377in}}%
\pgfusepath{clip}%
\pgfsetbuttcap%
\pgfsetroundjoin%
\definecolor{currentfill}{rgb}{0.800000,0.800000,0.200000}%
\pgfsetfillcolor{currentfill}%
\pgfsetlinewidth{1.003750pt}%
\definecolor{currentstroke}{rgb}{0.800000,0.800000,0.200000}%
\pgfsetstrokecolor{currentstroke}%
\pgfsetdash{}{0pt}%
\pgfpathmoveto{\pgfqpoint{5.289881in}{4.967670in}}%
\pgfpathcurveto{\pgfqpoint{5.295705in}{4.967670in}}{\pgfqpoint{5.301291in}{4.969984in}}{\pgfqpoint{5.305409in}{4.974102in}}%
\pgfpathcurveto{\pgfqpoint{5.309527in}{4.978220in}}{\pgfqpoint{5.311841in}{4.983806in}}{\pgfqpoint{5.311841in}{4.989630in}}%
\pgfpathcurveto{\pgfqpoint{5.311841in}{4.995454in}}{\pgfqpoint{5.309527in}{5.001040in}}{\pgfqpoint{5.305409in}{5.005158in}}%
\pgfpathcurveto{\pgfqpoint{5.301291in}{5.009277in}}{\pgfqpoint{5.295705in}{5.011590in}}{\pgfqpoint{5.289881in}{5.011590in}}%
\pgfpathcurveto{\pgfqpoint{5.284057in}{5.011590in}}{\pgfqpoint{5.278470in}{5.009277in}}{\pgfqpoint{5.274352in}{5.005158in}}%
\pgfpathcurveto{\pgfqpoint{5.270234in}{5.001040in}}{\pgfqpoint{5.267920in}{4.995454in}}{\pgfqpoint{5.267920in}{4.989630in}}%
\pgfpathcurveto{\pgfqpoint{5.267920in}{4.983806in}}{\pgfqpoint{5.270234in}{4.978220in}}{\pgfqpoint{5.274352in}{4.974102in}}%
\pgfpathcurveto{\pgfqpoint{5.278470in}{4.969984in}}{\pgfqpoint{5.284057in}{4.967670in}}{\pgfqpoint{5.289881in}{4.967670in}}%
\pgfpathlineto{\pgfqpoint{5.289881in}{4.967670in}}%
\pgfpathclose%
\pgfusepath{stroke,fill}%
\end{pgfscope}%
\begin{pgfscope}%
\pgfpathrectangle{\pgfqpoint{1.000000in}{1.148311in}}{\pgfqpoint{6.200000in}{5.623377in}}%
\pgfusepath{clip}%
\pgfsetbuttcap%
\pgfsetroundjoin%
\definecolor{currentfill}{rgb}{0.800000,0.200000,0.200000}%
\pgfsetfillcolor{currentfill}%
\pgfsetlinewidth{1.003750pt}%
\definecolor{currentstroke}{rgb}{0.800000,0.200000,0.200000}%
\pgfsetstrokecolor{currentstroke}%
\pgfsetdash{}{0pt}%
\pgfpathmoveto{\pgfqpoint{5.332049in}{5.017024in}}%
\pgfpathcurveto{\pgfqpoint{5.337873in}{5.017024in}}{\pgfqpoint{5.343459in}{5.019338in}}{\pgfqpoint{5.347577in}{5.023456in}}%
\pgfpathcurveto{\pgfqpoint{5.351695in}{5.027574in}}{\pgfqpoint{5.354009in}{5.033160in}}{\pgfqpoint{5.354009in}{5.038984in}}%
\pgfpathcurveto{\pgfqpoint{5.354009in}{5.044808in}}{\pgfqpoint{5.351695in}{5.050394in}}{\pgfqpoint{5.347577in}{5.054513in}}%
\pgfpathcurveto{\pgfqpoint{5.343459in}{5.058631in}}{\pgfqpoint{5.337873in}{5.060945in}}{\pgfqpoint{5.332049in}{5.060945in}}%
\pgfpathcurveto{\pgfqpoint{5.326225in}{5.060945in}}{\pgfqpoint{5.320639in}{5.058631in}}{\pgfqpoint{5.316521in}{5.054513in}}%
\pgfpathcurveto{\pgfqpoint{5.312402in}{5.050394in}}{\pgfqpoint{5.310089in}{5.044808in}}{\pgfqpoint{5.310089in}{5.038984in}}%
\pgfpathcurveto{\pgfqpoint{5.310089in}{5.033160in}}{\pgfqpoint{5.312402in}{5.027574in}}{\pgfqpoint{5.316521in}{5.023456in}}%
\pgfpathcurveto{\pgfqpoint{5.320639in}{5.019338in}}{\pgfqpoint{5.326225in}{5.017024in}}{\pgfqpoint{5.332049in}{5.017024in}}%
\pgfpathlineto{\pgfqpoint{5.332049in}{5.017024in}}%
\pgfpathclose%
\pgfusepath{stroke,fill}%
\end{pgfscope}%
\begin{pgfscope}%
\pgfpathrectangle{\pgfqpoint{1.000000in}{1.148311in}}{\pgfqpoint{6.200000in}{5.623377in}}%
\pgfusepath{clip}%
\pgfsetbuttcap%
\pgfsetroundjoin%
\definecolor{currentfill}{rgb}{0.800000,0.200000,0.200000}%
\pgfsetfillcolor{currentfill}%
\pgfsetlinewidth{1.003750pt}%
\definecolor{currentstroke}{rgb}{0.800000,0.200000,0.200000}%
\pgfsetstrokecolor{currentstroke}%
\pgfsetdash{}{0pt}%
\pgfpathmoveto{\pgfqpoint{5.335445in}{5.084204in}}%
\pgfpathcurveto{\pgfqpoint{5.341269in}{5.084204in}}{\pgfqpoint{5.346855in}{5.086518in}}{\pgfqpoint{5.350973in}{5.090636in}}%
\pgfpathcurveto{\pgfqpoint{5.355091in}{5.094755in}}{\pgfqpoint{5.357405in}{5.100341in}}{\pgfqpoint{5.357405in}{5.106165in}}%
\pgfpathcurveto{\pgfqpoint{5.357405in}{5.111989in}}{\pgfqpoint{5.355091in}{5.117575in}}{\pgfqpoint{5.350973in}{5.121693in}}%
\pgfpathcurveto{\pgfqpoint{5.346855in}{5.125811in}}{\pgfqpoint{5.341269in}{5.128125in}}{\pgfqpoint{5.335445in}{5.128125in}}%
\pgfpathcurveto{\pgfqpoint{5.329621in}{5.128125in}}{\pgfqpoint{5.324035in}{5.125811in}}{\pgfqpoint{5.319916in}{5.121693in}}%
\pgfpathcurveto{\pgfqpoint{5.315798in}{5.117575in}}{\pgfqpoint{5.313484in}{5.111989in}}{\pgfqpoint{5.313484in}{5.106165in}}%
\pgfpathcurveto{\pgfqpoint{5.313484in}{5.100341in}}{\pgfqpoint{5.315798in}{5.094755in}}{\pgfqpoint{5.319916in}{5.090636in}}%
\pgfpathcurveto{\pgfqpoint{5.324035in}{5.086518in}}{\pgfqpoint{5.329621in}{5.084204in}}{\pgfqpoint{5.335445in}{5.084204in}}%
\pgfpathlineto{\pgfqpoint{5.335445in}{5.084204in}}%
\pgfpathclose%
\pgfusepath{stroke,fill}%
\end{pgfscope}%
\begin{pgfscope}%
\pgfpathrectangle{\pgfqpoint{1.000000in}{1.148311in}}{\pgfqpoint{6.200000in}{5.623377in}}%
\pgfusepath{clip}%
\pgfsetbuttcap%
\pgfsetroundjoin%
\definecolor{currentfill}{rgb}{0.800000,0.800000,0.200000}%
\pgfsetfillcolor{currentfill}%
\pgfsetlinewidth{1.003750pt}%
\definecolor{currentstroke}{rgb}{0.800000,0.800000,0.200000}%
\pgfsetstrokecolor{currentstroke}%
\pgfsetdash{}{0pt}%
\pgfpathmoveto{\pgfqpoint{5.327746in}{5.152116in}}%
\pgfpathcurveto{\pgfqpoint{5.333570in}{5.152116in}}{\pgfqpoint{5.339156in}{5.154430in}}{\pgfqpoint{5.343274in}{5.158548in}}%
\pgfpathcurveto{\pgfqpoint{5.347392in}{5.162666in}}{\pgfqpoint{5.349706in}{5.168252in}}{\pgfqpoint{5.349706in}{5.174076in}}%
\pgfpathcurveto{\pgfqpoint{5.349706in}{5.179900in}}{\pgfqpoint{5.347392in}{5.185486in}}{\pgfqpoint{5.343274in}{5.189604in}}%
\pgfpathcurveto{\pgfqpoint{5.339156in}{5.193722in}}{\pgfqpoint{5.333570in}{5.196036in}}{\pgfqpoint{5.327746in}{5.196036in}}%
\pgfpathcurveto{\pgfqpoint{5.321922in}{5.196036in}}{\pgfqpoint{5.316336in}{5.193722in}}{\pgfqpoint{5.312218in}{5.189604in}}%
\pgfpathcurveto{\pgfqpoint{5.308100in}{5.185486in}}{\pgfqpoint{5.305786in}{5.179900in}}{\pgfqpoint{5.305786in}{5.174076in}}%
\pgfpathcurveto{\pgfqpoint{5.305786in}{5.168252in}}{\pgfqpoint{5.308100in}{5.162666in}}{\pgfqpoint{5.312218in}{5.158548in}}%
\pgfpathcurveto{\pgfqpoint{5.316336in}{5.154430in}}{\pgfqpoint{5.321922in}{5.152116in}}{\pgfqpoint{5.327746in}{5.152116in}}%
\pgfpathlineto{\pgfqpoint{5.327746in}{5.152116in}}%
\pgfpathclose%
\pgfusepath{stroke,fill}%
\end{pgfscope}%
\begin{pgfscope}%
\pgfpathrectangle{\pgfqpoint{1.000000in}{1.148311in}}{\pgfqpoint{6.200000in}{5.623377in}}%
\pgfusepath{clip}%
\pgfsetbuttcap%
\pgfsetroundjoin%
\definecolor{currentfill}{rgb}{0.800000,0.200000,0.200000}%
\pgfsetfillcolor{currentfill}%
\pgfsetlinewidth{1.003750pt}%
\definecolor{currentstroke}{rgb}{0.800000,0.200000,0.200000}%
\pgfsetstrokecolor{currentstroke}%
\pgfsetdash{}{0pt}%
\pgfpathmoveto{\pgfqpoint{5.464733in}{5.178751in}}%
\pgfpathcurveto{\pgfqpoint{5.470557in}{5.178751in}}{\pgfqpoint{5.476144in}{5.181065in}}{\pgfqpoint{5.480262in}{5.185183in}}%
\pgfpathcurveto{\pgfqpoint{5.484380in}{5.189301in}}{\pgfqpoint{5.486694in}{5.194887in}}{\pgfqpoint{5.486694in}{5.200711in}}%
\pgfpathcurveto{\pgfqpoint{5.486694in}{5.206535in}}{\pgfqpoint{5.484380in}{5.212121in}}{\pgfqpoint{5.480262in}{5.216239in}}%
\pgfpathcurveto{\pgfqpoint{5.476144in}{5.220358in}}{\pgfqpoint{5.470557in}{5.222671in}}{\pgfqpoint{5.464733in}{5.222671in}}%
\pgfpathcurveto{\pgfqpoint{5.458909in}{5.222671in}}{\pgfqpoint{5.453323in}{5.220358in}}{\pgfqpoint{5.449205in}{5.216239in}}%
\pgfpathcurveto{\pgfqpoint{5.445087in}{5.212121in}}{\pgfqpoint{5.442773in}{5.206535in}}{\pgfqpoint{5.442773in}{5.200711in}}%
\pgfpathcurveto{\pgfqpoint{5.442773in}{5.194887in}}{\pgfqpoint{5.445087in}{5.189301in}}{\pgfqpoint{5.449205in}{5.185183in}}%
\pgfpathcurveto{\pgfqpoint{5.453323in}{5.181065in}}{\pgfqpoint{5.458909in}{5.178751in}}{\pgfqpoint{5.464733in}{5.178751in}}%
\pgfpathlineto{\pgfqpoint{5.464733in}{5.178751in}}%
\pgfpathclose%
\pgfusepath{stroke,fill}%
\end{pgfscope}%
\begin{pgfscope}%
\pgfpathrectangle{\pgfqpoint{1.000000in}{1.148311in}}{\pgfqpoint{6.200000in}{5.623377in}}%
\pgfusepath{clip}%
\pgfsetbuttcap%
\pgfsetroundjoin%
\definecolor{currentfill}{rgb}{0.800000,0.800000,0.200000}%
\pgfsetfillcolor{currentfill}%
\pgfsetlinewidth{1.003750pt}%
\definecolor{currentstroke}{rgb}{0.800000,0.800000,0.200000}%
\pgfsetstrokecolor{currentstroke}%
\pgfsetdash{}{0pt}%
\pgfpathmoveto{\pgfqpoint{5.432378in}{5.254289in}}%
\pgfpathcurveto{\pgfqpoint{5.438202in}{5.254289in}}{\pgfqpoint{5.443788in}{5.256603in}}{\pgfqpoint{5.447906in}{5.260721in}}%
\pgfpathcurveto{\pgfqpoint{5.452024in}{5.264839in}}{\pgfqpoint{5.454338in}{5.270425in}}{\pgfqpoint{5.454338in}{5.276249in}}%
\pgfpathcurveto{\pgfqpoint{5.454338in}{5.282073in}}{\pgfqpoint{5.452024in}{5.287659in}}{\pgfqpoint{5.447906in}{5.291778in}}%
\pgfpathcurveto{\pgfqpoint{5.443788in}{5.295896in}}{\pgfqpoint{5.438202in}{5.298210in}}{\pgfqpoint{5.432378in}{5.298210in}}%
\pgfpathcurveto{\pgfqpoint{5.426554in}{5.298210in}}{\pgfqpoint{5.420968in}{5.295896in}}{\pgfqpoint{5.416850in}{5.291778in}}%
\pgfpathcurveto{\pgfqpoint{5.412732in}{5.287659in}}{\pgfqpoint{5.410418in}{5.282073in}}{\pgfqpoint{5.410418in}{5.276249in}}%
\pgfpathcurveto{\pgfqpoint{5.410418in}{5.270425in}}{\pgfqpoint{5.412732in}{5.264839in}}{\pgfqpoint{5.416850in}{5.260721in}}%
\pgfpathcurveto{\pgfqpoint{5.420968in}{5.256603in}}{\pgfqpoint{5.426554in}{5.254289in}}{\pgfqpoint{5.432378in}{5.254289in}}%
\pgfpathlineto{\pgfqpoint{5.432378in}{5.254289in}}%
\pgfpathclose%
\pgfusepath{stroke,fill}%
\end{pgfscope}%
\begin{pgfscope}%
\pgfpathrectangle{\pgfqpoint{1.000000in}{1.148311in}}{\pgfqpoint{6.200000in}{5.623377in}}%
\pgfusepath{clip}%
\pgfsetbuttcap%
\pgfsetroundjoin%
\definecolor{currentfill}{rgb}{0.800000,0.800000,0.200000}%
\pgfsetfillcolor{currentfill}%
\pgfsetlinewidth{1.003750pt}%
\definecolor{currentstroke}{rgb}{0.800000,0.800000,0.200000}%
\pgfsetstrokecolor{currentstroke}%
\pgfsetdash{}{0pt}%
\pgfpathmoveto{\pgfqpoint{5.422751in}{5.321036in}}%
\pgfpathcurveto{\pgfqpoint{5.428575in}{5.321036in}}{\pgfqpoint{5.434161in}{5.323350in}}{\pgfqpoint{5.438280in}{5.327468in}}%
\pgfpathcurveto{\pgfqpoint{5.442398in}{5.331586in}}{\pgfqpoint{5.444712in}{5.337172in}}{\pgfqpoint{5.444712in}{5.342996in}}%
\pgfpathcurveto{\pgfqpoint{5.444712in}{5.348820in}}{\pgfqpoint{5.442398in}{5.354406in}}{\pgfqpoint{5.438280in}{5.358524in}}%
\pgfpathcurveto{\pgfqpoint{5.434161in}{5.362642in}}{\pgfqpoint{5.428575in}{5.364956in}}{\pgfqpoint{5.422751in}{5.364956in}}%
\pgfpathcurveto{\pgfqpoint{5.416927in}{5.364956in}}{\pgfqpoint{5.411341in}{5.362642in}}{\pgfqpoint{5.407223in}{5.358524in}}%
\pgfpathcurveto{\pgfqpoint{5.403105in}{5.354406in}}{\pgfqpoint{5.400791in}{5.348820in}}{\pgfqpoint{5.400791in}{5.342996in}}%
\pgfpathcurveto{\pgfqpoint{5.400791in}{5.337172in}}{\pgfqpoint{5.403105in}{5.331586in}}{\pgfqpoint{5.407223in}{5.327468in}}%
\pgfpathcurveto{\pgfqpoint{5.411341in}{5.323350in}}{\pgfqpoint{5.416927in}{5.321036in}}{\pgfqpoint{5.422751in}{5.321036in}}%
\pgfpathlineto{\pgfqpoint{5.422751in}{5.321036in}}%
\pgfpathclose%
\pgfusepath{stroke,fill}%
\end{pgfscope}%
\begin{pgfscope}%
\pgfpathrectangle{\pgfqpoint{1.000000in}{1.148311in}}{\pgfqpoint{6.200000in}{5.623377in}}%
\pgfusepath{clip}%
\pgfsetbuttcap%
\pgfsetroundjoin%
\definecolor{currentfill}{rgb}{0.800000,0.800000,0.200000}%
\pgfsetfillcolor{currentfill}%
\pgfsetlinewidth{1.003750pt}%
\definecolor{currentstroke}{rgb}{0.800000,0.800000,0.200000}%
\pgfsetstrokecolor{currentstroke}%
\pgfsetdash{}{0pt}%
\pgfpathmoveto{\pgfqpoint{5.411364in}{5.385607in}}%
\pgfpathcurveto{\pgfqpoint{5.417188in}{5.385607in}}{\pgfqpoint{5.422774in}{5.387921in}}{\pgfqpoint{5.426892in}{5.392039in}}%
\pgfpathcurveto{\pgfqpoint{5.431010in}{5.396157in}}{\pgfqpoint{5.433324in}{5.401743in}}{\pgfqpoint{5.433324in}{5.407567in}}%
\pgfpathcurveto{\pgfqpoint{5.433324in}{5.413391in}}{\pgfqpoint{5.431010in}{5.418977in}}{\pgfqpoint{5.426892in}{5.423095in}}%
\pgfpathcurveto{\pgfqpoint{5.422774in}{5.427213in}}{\pgfqpoint{5.417188in}{5.429527in}}{\pgfqpoint{5.411364in}{5.429527in}}%
\pgfpathcurveto{\pgfqpoint{5.405540in}{5.429527in}}{\pgfqpoint{5.399954in}{5.427213in}}{\pgfqpoint{5.395836in}{5.423095in}}%
\pgfpathcurveto{\pgfqpoint{5.391717in}{5.418977in}}{\pgfqpoint{5.389404in}{5.413391in}}{\pgfqpoint{5.389404in}{5.407567in}}%
\pgfpathcurveto{\pgfqpoint{5.389404in}{5.401743in}}{\pgfqpoint{5.391717in}{5.396157in}}{\pgfqpoint{5.395836in}{5.392039in}}%
\pgfpathcurveto{\pgfqpoint{5.399954in}{5.387921in}}{\pgfqpoint{5.405540in}{5.385607in}}{\pgfqpoint{5.411364in}{5.385607in}}%
\pgfpathlineto{\pgfqpoint{5.411364in}{5.385607in}}%
\pgfpathclose%
\pgfusepath{stroke,fill}%
\end{pgfscope}%
\begin{pgfscope}%
\pgfpathrectangle{\pgfqpoint{1.000000in}{1.148311in}}{\pgfqpoint{6.200000in}{5.623377in}}%
\pgfusepath{clip}%
\pgfsetbuttcap%
\pgfsetroundjoin%
\definecolor{currentfill}{rgb}{0.800000,0.800000,0.200000}%
\pgfsetfillcolor{currentfill}%
\pgfsetlinewidth{1.003750pt}%
\definecolor{currentstroke}{rgb}{0.800000,0.800000,0.200000}%
\pgfsetstrokecolor{currentstroke}%
\pgfsetdash{}{0pt}%
\pgfpathmoveto{\pgfqpoint{5.476130in}{5.448217in}}%
\pgfpathcurveto{\pgfqpoint{5.481954in}{5.448217in}}{\pgfqpoint{5.487540in}{5.450531in}}{\pgfqpoint{5.491658in}{5.454649in}}%
\pgfpathcurveto{\pgfqpoint{5.495776in}{5.458767in}}{\pgfqpoint{5.498090in}{5.464353in}}{\pgfqpoint{5.498090in}{5.470177in}}%
\pgfpathcurveto{\pgfqpoint{5.498090in}{5.476001in}}{\pgfqpoint{5.495776in}{5.481587in}}{\pgfqpoint{5.491658in}{5.485705in}}%
\pgfpathcurveto{\pgfqpoint{5.487540in}{5.489823in}}{\pgfqpoint{5.481954in}{5.492137in}}{\pgfqpoint{5.476130in}{5.492137in}}%
\pgfpathcurveto{\pgfqpoint{5.470306in}{5.492137in}}{\pgfqpoint{5.464720in}{5.489823in}}{\pgfqpoint{5.460602in}{5.485705in}}%
\pgfpathcurveto{\pgfqpoint{5.456484in}{5.481587in}}{\pgfqpoint{5.454170in}{5.476001in}}{\pgfqpoint{5.454170in}{5.470177in}}%
\pgfpathcurveto{\pgfqpoint{5.454170in}{5.464353in}}{\pgfqpoint{5.456484in}{5.458767in}}{\pgfqpoint{5.460602in}{5.454649in}}%
\pgfpathcurveto{\pgfqpoint{5.464720in}{5.450531in}}{\pgfqpoint{5.470306in}{5.448217in}}{\pgfqpoint{5.476130in}{5.448217in}}%
\pgfpathlineto{\pgfqpoint{5.476130in}{5.448217in}}%
\pgfpathclose%
\pgfusepath{stroke,fill}%
\end{pgfscope}%
\begin{pgfscope}%
\pgfpathrectangle{\pgfqpoint{1.000000in}{1.148311in}}{\pgfqpoint{6.200000in}{5.623377in}}%
\pgfusepath{clip}%
\pgfsetbuttcap%
\pgfsetroundjoin%
\definecolor{currentfill}{rgb}{0.800000,0.200000,0.200000}%
\pgfsetfillcolor{currentfill}%
\pgfsetlinewidth{1.003750pt}%
\definecolor{currentstroke}{rgb}{0.800000,0.200000,0.200000}%
\pgfsetstrokecolor{currentstroke}%
\pgfsetdash{}{0pt}%
\pgfpathmoveto{\pgfqpoint{4.757402in}{3.617487in}}%
\pgfpathcurveto{\pgfqpoint{4.763226in}{3.617487in}}{\pgfqpoint{4.768812in}{3.619800in}}{\pgfqpoint{4.772930in}{3.623919in}}%
\pgfpathcurveto{\pgfqpoint{4.777048in}{3.628037in}}{\pgfqpoint{4.779362in}{3.633623in}}{\pgfqpoint{4.779362in}{3.639447in}}%
\pgfpathcurveto{\pgfqpoint{4.779362in}{3.645271in}}{\pgfqpoint{4.777048in}{3.650857in}}{\pgfqpoint{4.772930in}{3.654975in}}%
\pgfpathcurveto{\pgfqpoint{4.768812in}{3.659093in}}{\pgfqpoint{4.763226in}{3.661407in}}{\pgfqpoint{4.757402in}{3.661407in}}%
\pgfpathcurveto{\pgfqpoint{4.751578in}{3.661407in}}{\pgfqpoint{4.745992in}{3.659093in}}{\pgfqpoint{4.741874in}{3.654975in}}%
\pgfpathcurveto{\pgfqpoint{4.737755in}{3.650857in}}{\pgfqpoint{4.735442in}{3.645271in}}{\pgfqpoint{4.735442in}{3.639447in}}%
\pgfpathcurveto{\pgfqpoint{4.735442in}{3.633623in}}{\pgfqpoint{4.737755in}{3.628037in}}{\pgfqpoint{4.741874in}{3.623919in}}%
\pgfpathcurveto{\pgfqpoint{4.745992in}{3.619800in}}{\pgfqpoint{4.751578in}{3.617487in}}{\pgfqpoint{4.757402in}{3.617487in}}%
\pgfpathlineto{\pgfqpoint{4.757402in}{3.617487in}}%
\pgfpathclose%
\pgfusepath{stroke,fill}%
\end{pgfscope}%
\begin{pgfscope}%
\pgfpathrectangle{\pgfqpoint{1.000000in}{1.148311in}}{\pgfqpoint{6.200000in}{5.623377in}}%
\pgfusepath{clip}%
\pgfsetbuttcap%
\pgfsetroundjoin%
\definecolor{currentfill}{rgb}{0.200000,0.800000,0.200000}%
\pgfsetfillcolor{currentfill}%
\pgfsetlinewidth{1.003750pt}%
\definecolor{currentstroke}{rgb}{0.200000,0.800000,0.200000}%
\pgfsetstrokecolor{currentstroke}%
\pgfsetdash{}{0pt}%
\pgfpathmoveto{\pgfqpoint{2.205008in}{1.381959in}}%
\pgfpathcurveto{\pgfqpoint{2.210832in}{1.381959in}}{\pgfqpoint{2.216418in}{1.384273in}}{\pgfqpoint{2.220536in}{1.388391in}}%
\pgfpathcurveto{\pgfqpoint{2.224654in}{1.392509in}}{\pgfqpoint{2.226968in}{1.398096in}}{\pgfqpoint{2.226968in}{1.403920in}}%
\pgfpathcurveto{\pgfqpoint{2.226968in}{1.409743in}}{\pgfqpoint{2.224654in}{1.415330in}}{\pgfqpoint{2.220536in}{1.419448in}}%
\pgfpathcurveto{\pgfqpoint{2.216418in}{1.423566in}}{\pgfqpoint{2.210832in}{1.425880in}}{\pgfqpoint{2.205008in}{1.425880in}}%
\pgfpathcurveto{\pgfqpoint{2.199184in}{1.425880in}}{\pgfqpoint{2.193598in}{1.423566in}}{\pgfqpoint{2.189480in}{1.419448in}}%
\pgfpathcurveto{\pgfqpoint{2.185362in}{1.415330in}}{\pgfqpoint{2.183048in}{1.409743in}}{\pgfqpoint{2.183048in}{1.403920in}}%
\pgfpathcurveto{\pgfqpoint{2.183048in}{1.398096in}}{\pgfqpoint{2.185362in}{1.392509in}}{\pgfqpoint{2.189480in}{1.388391in}}%
\pgfpathcurveto{\pgfqpoint{2.193598in}{1.384273in}}{\pgfqpoint{2.199184in}{1.381959in}}{\pgfqpoint{2.205008in}{1.381959in}}%
\pgfpathlineto{\pgfqpoint{2.205008in}{1.381959in}}%
\pgfpathclose%
\pgfusepath{stroke,fill}%
\end{pgfscope}%
\begin{pgfscope}%
\pgfpathrectangle{\pgfqpoint{1.000000in}{1.148311in}}{\pgfqpoint{6.200000in}{5.623377in}}%
\pgfusepath{clip}%
\pgfsetbuttcap%
\pgfsetroundjoin%
\definecolor{currentfill}{rgb}{0.800000,0.800000,0.200000}%
\pgfsetfillcolor{currentfill}%
\pgfsetlinewidth{1.003750pt}%
\definecolor{currentstroke}{rgb}{0.800000,0.800000,0.200000}%
\pgfsetstrokecolor{currentstroke}%
\pgfsetdash{}{0pt}%
\pgfpathmoveto{\pgfqpoint{1.281818in}{5.145514in}}%
\pgfpathcurveto{\pgfqpoint{1.287642in}{5.145514in}}{\pgfqpoint{1.293228in}{5.147828in}}{\pgfqpoint{1.297346in}{5.151946in}}%
\pgfpathcurveto{\pgfqpoint{1.301465in}{5.156064in}}{\pgfqpoint{1.303778in}{5.161650in}}{\pgfqpoint{1.303778in}{5.167474in}}%
\pgfpathcurveto{\pgfqpoint{1.303778in}{5.173298in}}{\pgfqpoint{1.301465in}{5.178884in}}{\pgfqpoint{1.297346in}{5.183002in}}%
\pgfpathcurveto{\pgfqpoint{1.293228in}{5.187121in}}{\pgfqpoint{1.287642in}{5.189434in}}{\pgfqpoint{1.281818in}{5.189434in}}%
\pgfpathcurveto{\pgfqpoint{1.275994in}{5.189434in}}{\pgfqpoint{1.270408in}{5.187121in}}{\pgfqpoint{1.266290in}{5.183002in}}%
\pgfpathcurveto{\pgfqpoint{1.262172in}{5.178884in}}{\pgfqpoint{1.259858in}{5.173298in}}{\pgfqpoint{1.259858in}{5.167474in}}%
\pgfpathcurveto{\pgfqpoint{1.259858in}{5.161650in}}{\pgfqpoint{1.262172in}{5.156064in}}{\pgfqpoint{1.266290in}{5.151946in}}%
\pgfpathcurveto{\pgfqpoint{1.270408in}{5.147828in}}{\pgfqpoint{1.275994in}{5.145514in}}{\pgfqpoint{1.281818in}{5.145514in}}%
\pgfpathlineto{\pgfqpoint{1.281818in}{5.145514in}}%
\pgfpathclose%
\pgfusepath{stroke,fill}%
\end{pgfscope}%
\begin{pgfscope}%
\pgfpathrectangle{\pgfqpoint{1.000000in}{1.148311in}}{\pgfqpoint{6.200000in}{5.623377in}}%
\pgfusepath{clip}%
\pgfsetbuttcap%
\pgfsetroundjoin%
\definecolor{currentfill}{rgb}{0.800000,0.800000,0.200000}%
\pgfsetfillcolor{currentfill}%
\pgfsetlinewidth{1.003750pt}%
\definecolor{currentstroke}{rgb}{0.800000,0.800000,0.200000}%
\pgfsetstrokecolor{currentstroke}%
\pgfsetdash{}{0pt}%
\pgfpathmoveto{\pgfqpoint{3.924024in}{4.280440in}}%
\pgfpathcurveto{\pgfqpoint{3.929848in}{4.280440in}}{\pgfqpoint{3.935434in}{4.282754in}}{\pgfqpoint{3.939553in}{4.286872in}}%
\pgfpathcurveto{\pgfqpoint{3.943671in}{4.290990in}}{\pgfqpoint{3.945985in}{4.296576in}}{\pgfqpoint{3.945985in}{4.302400in}}%
\pgfpathcurveto{\pgfqpoint{3.945985in}{4.308224in}}{\pgfqpoint{3.943671in}{4.313811in}}{\pgfqpoint{3.939553in}{4.317929in}}%
\pgfpathcurveto{\pgfqpoint{3.935434in}{4.322047in}}{\pgfqpoint{3.929848in}{4.324361in}}{\pgfqpoint{3.924024in}{4.324361in}}%
\pgfpathcurveto{\pgfqpoint{3.918200in}{4.324361in}}{\pgfqpoint{3.912614in}{4.322047in}}{\pgfqpoint{3.908496in}{4.317929in}}%
\pgfpathcurveto{\pgfqpoint{3.904378in}{4.313811in}}{\pgfqpoint{3.902064in}{4.308224in}}{\pgfqpoint{3.902064in}{4.302400in}}%
\pgfpathcurveto{\pgfqpoint{3.902064in}{4.296576in}}{\pgfqpoint{3.904378in}{4.290990in}}{\pgfqpoint{3.908496in}{4.286872in}}%
\pgfpathcurveto{\pgfqpoint{3.912614in}{4.282754in}}{\pgfqpoint{3.918200in}{4.280440in}}{\pgfqpoint{3.924024in}{4.280440in}}%
\pgfpathlineto{\pgfqpoint{3.924024in}{4.280440in}}%
\pgfpathclose%
\pgfusepath{stroke,fill}%
\end{pgfscope}%
\begin{pgfscope}%
\pgfpathrectangle{\pgfqpoint{1.000000in}{1.148311in}}{\pgfqpoint{6.200000in}{5.623377in}}%
\pgfusepath{clip}%
\pgfsetbuttcap%
\pgfsetroundjoin%
\definecolor{currentfill}{rgb}{0.800000,0.800000,0.200000}%
\pgfsetfillcolor{currentfill}%
\pgfsetlinewidth{1.003750pt}%
\definecolor{currentstroke}{rgb}{0.800000,0.800000,0.200000}%
\pgfsetstrokecolor{currentstroke}%
\pgfsetdash{}{0pt}%
\pgfpathmoveto{\pgfqpoint{5.079291in}{4.815620in}}%
\pgfpathcurveto{\pgfqpoint{5.085115in}{4.815620in}}{\pgfqpoint{5.090701in}{4.817934in}}{\pgfqpoint{5.094819in}{4.822052in}}%
\pgfpathcurveto{\pgfqpoint{5.098937in}{4.826170in}}{\pgfqpoint{5.101251in}{4.831757in}}{\pgfqpoint{5.101251in}{4.837580in}}%
\pgfpathcurveto{\pgfqpoint{5.101251in}{4.843404in}}{\pgfqpoint{5.098937in}{4.848991in}}{\pgfqpoint{5.094819in}{4.853109in}}%
\pgfpathcurveto{\pgfqpoint{5.090701in}{4.857227in}}{\pgfqpoint{5.085115in}{4.859541in}}{\pgfqpoint{5.079291in}{4.859541in}}%
\pgfpathcurveto{\pgfqpoint{5.073467in}{4.859541in}}{\pgfqpoint{5.067881in}{4.857227in}}{\pgfqpoint{5.063762in}{4.853109in}}%
\pgfpathcurveto{\pgfqpoint{5.059644in}{4.848991in}}{\pgfqpoint{5.057330in}{4.843404in}}{\pgfqpoint{5.057330in}{4.837580in}}%
\pgfpathcurveto{\pgfqpoint{5.057330in}{4.831757in}}{\pgfqpoint{5.059644in}{4.826170in}}{\pgfqpoint{5.063762in}{4.822052in}}%
\pgfpathcurveto{\pgfqpoint{5.067881in}{4.817934in}}{\pgfqpoint{5.073467in}{4.815620in}}{\pgfqpoint{5.079291in}{4.815620in}}%
\pgfpathlineto{\pgfqpoint{5.079291in}{4.815620in}}%
\pgfpathclose%
\pgfusepath{stroke,fill}%
\end{pgfscope}%
\begin{pgfscope}%
\pgfpathrectangle{\pgfqpoint{1.000000in}{1.148311in}}{\pgfqpoint{6.200000in}{5.623377in}}%
\pgfusepath{clip}%
\pgfsetbuttcap%
\pgfsetroundjoin%
\definecolor{currentfill}{rgb}{0.800000,0.800000,0.200000}%
\pgfsetfillcolor{currentfill}%
\pgfsetlinewidth{1.003750pt}%
\definecolor{currentstroke}{rgb}{0.800000,0.800000,0.200000}%
\pgfsetstrokecolor{currentstroke}%
\pgfsetdash{}{0pt}%
\pgfpathmoveto{\pgfqpoint{3.275340in}{5.626369in}}%
\pgfpathcurveto{\pgfqpoint{3.281163in}{5.626369in}}{\pgfqpoint{3.286750in}{5.628683in}}{\pgfqpoint{3.290868in}{5.632801in}}%
\pgfpathcurveto{\pgfqpoint{3.294986in}{5.636919in}}{\pgfqpoint{3.297300in}{5.642506in}}{\pgfqpoint{3.297300in}{5.648330in}}%
\pgfpathcurveto{\pgfqpoint{3.297300in}{5.654153in}}{\pgfqpoint{3.294986in}{5.659740in}}{\pgfqpoint{3.290868in}{5.663858in}}%
\pgfpathcurveto{\pgfqpoint{3.286750in}{5.667976in}}{\pgfqpoint{3.281163in}{5.670290in}}{\pgfqpoint{3.275340in}{5.670290in}}%
\pgfpathcurveto{\pgfqpoint{3.269516in}{5.670290in}}{\pgfqpoint{3.263929in}{5.667976in}}{\pgfqpoint{3.259811in}{5.663858in}}%
\pgfpathcurveto{\pgfqpoint{3.255693in}{5.659740in}}{\pgfqpoint{3.253379in}{5.654153in}}{\pgfqpoint{3.253379in}{5.648330in}}%
\pgfpathcurveto{\pgfqpoint{3.253379in}{5.642506in}}{\pgfqpoint{3.255693in}{5.636919in}}{\pgfqpoint{3.259811in}{5.632801in}}%
\pgfpathcurveto{\pgfqpoint{3.263929in}{5.628683in}}{\pgfqpoint{3.269516in}{5.626369in}}{\pgfqpoint{3.275340in}{5.626369in}}%
\pgfpathlineto{\pgfqpoint{3.275340in}{5.626369in}}%
\pgfpathclose%
\pgfusepath{stroke,fill}%
\end{pgfscope}%
\begin{pgfscope}%
\pgfpathrectangle{\pgfqpoint{1.000000in}{1.148311in}}{\pgfqpoint{6.200000in}{5.623377in}}%
\pgfusepath{clip}%
\pgfsetbuttcap%
\pgfsetroundjoin%
\definecolor{currentfill}{rgb}{0.200000,0.800000,0.200000}%
\pgfsetfillcolor{currentfill}%
\pgfsetlinewidth{1.003750pt}%
\definecolor{currentstroke}{rgb}{0.200000,0.800000,0.200000}%
\pgfsetstrokecolor{currentstroke}%
\pgfsetdash{}{0pt}%
\pgfpathmoveto{\pgfqpoint{3.920377in}{2.705891in}}%
\pgfpathcurveto{\pgfqpoint{3.926201in}{2.705891in}}{\pgfqpoint{3.931787in}{2.708205in}}{\pgfqpoint{3.935905in}{2.712323in}}%
\pgfpathcurveto{\pgfqpoint{3.940023in}{2.716442in}}{\pgfqpoint{3.942337in}{2.722028in}}{\pgfqpoint{3.942337in}{2.727852in}}%
\pgfpathcurveto{\pgfqpoint{3.942337in}{2.733676in}}{\pgfqpoint{3.940023in}{2.739262in}}{\pgfqpoint{3.935905in}{2.743380in}}%
\pgfpathcurveto{\pgfqpoint{3.931787in}{2.747498in}}{\pgfqpoint{3.926201in}{2.749812in}}{\pgfqpoint{3.920377in}{2.749812in}}%
\pgfpathcurveto{\pgfqpoint{3.914553in}{2.749812in}}{\pgfqpoint{3.908967in}{2.747498in}}{\pgfqpoint{3.904849in}{2.743380in}}%
\pgfpathcurveto{\pgfqpoint{3.900731in}{2.739262in}}{\pgfqpoint{3.898417in}{2.733676in}}{\pgfqpoint{3.898417in}{2.727852in}}%
\pgfpathcurveto{\pgfqpoint{3.898417in}{2.722028in}}{\pgfqpoint{3.900731in}{2.716442in}}{\pgfqpoint{3.904849in}{2.712323in}}%
\pgfpathcurveto{\pgfqpoint{3.908967in}{2.708205in}}{\pgfqpoint{3.914553in}{2.705891in}}{\pgfqpoint{3.920377in}{2.705891in}}%
\pgfpathlineto{\pgfqpoint{3.920377in}{2.705891in}}%
\pgfpathclose%
\pgfusepath{stroke,fill}%
\end{pgfscope}%
\begin{pgfscope}%
\pgfpathrectangle{\pgfqpoint{1.000000in}{1.148311in}}{\pgfqpoint{6.200000in}{5.623377in}}%
\pgfusepath{clip}%
\pgfsetbuttcap%
\pgfsetroundjoin%
\definecolor{currentfill}{rgb}{0.800000,0.800000,0.200000}%
\pgfsetfillcolor{currentfill}%
\pgfsetlinewidth{1.003750pt}%
\definecolor{currentstroke}{rgb}{0.800000,0.800000,0.200000}%
\pgfsetstrokecolor{currentstroke}%
\pgfsetdash{}{0pt}%
\pgfpathmoveto{\pgfqpoint{3.221846in}{3.735896in}}%
\pgfpathcurveto{\pgfqpoint{3.227670in}{3.735896in}}{\pgfqpoint{3.233257in}{3.738210in}}{\pgfqpoint{3.237375in}{3.742328in}}%
\pgfpathcurveto{\pgfqpoint{3.241493in}{3.746447in}}{\pgfqpoint{3.243807in}{3.752033in}}{\pgfqpoint{3.243807in}{3.757857in}}%
\pgfpathcurveto{\pgfqpoint{3.243807in}{3.763681in}}{\pgfqpoint{3.241493in}{3.769267in}}{\pgfqpoint{3.237375in}{3.773385in}}%
\pgfpathcurveto{\pgfqpoint{3.233257in}{3.777503in}}{\pgfqpoint{3.227670in}{3.779817in}}{\pgfqpoint{3.221846in}{3.779817in}}%
\pgfpathcurveto{\pgfqpoint{3.216023in}{3.779817in}}{\pgfqpoint{3.210436in}{3.777503in}}{\pgfqpoint{3.206318in}{3.773385in}}%
\pgfpathcurveto{\pgfqpoint{3.202200in}{3.769267in}}{\pgfqpoint{3.199886in}{3.763681in}}{\pgfqpoint{3.199886in}{3.757857in}}%
\pgfpathcurveto{\pgfqpoint{3.199886in}{3.752033in}}{\pgfqpoint{3.202200in}{3.746447in}}{\pgfqpoint{3.206318in}{3.742328in}}%
\pgfpathcurveto{\pgfqpoint{3.210436in}{3.738210in}}{\pgfqpoint{3.216023in}{3.735896in}}{\pgfqpoint{3.221846in}{3.735896in}}%
\pgfpathlineto{\pgfqpoint{3.221846in}{3.735896in}}%
\pgfpathclose%
\pgfusepath{stroke,fill}%
\end{pgfscope}%
\begin{pgfscope}%
\pgfpathrectangle{\pgfqpoint{1.000000in}{1.148311in}}{\pgfqpoint{6.200000in}{5.623377in}}%
\pgfusepath{clip}%
\pgfsetbuttcap%
\pgfsetroundjoin%
\definecolor{currentfill}{rgb}{0.800000,0.800000,0.200000}%
\pgfsetfillcolor{currentfill}%
\pgfsetlinewidth{1.003750pt}%
\definecolor{currentstroke}{rgb}{0.800000,0.800000,0.200000}%
\pgfsetstrokecolor{currentstroke}%
\pgfsetdash{}{0pt}%
\pgfpathmoveto{\pgfqpoint{3.658141in}{4.208084in}}%
\pgfpathcurveto{\pgfqpoint{3.663965in}{4.208084in}}{\pgfqpoint{3.669551in}{4.210398in}}{\pgfqpoint{3.673669in}{4.214516in}}%
\pgfpathcurveto{\pgfqpoint{3.677787in}{4.218634in}}{\pgfqpoint{3.680101in}{4.224220in}}{\pgfqpoint{3.680101in}{4.230044in}}%
\pgfpathcurveto{\pgfqpoint{3.680101in}{4.235868in}}{\pgfqpoint{3.677787in}{4.241454in}}{\pgfqpoint{3.673669in}{4.245572in}}%
\pgfpathcurveto{\pgfqpoint{3.669551in}{4.249690in}}{\pgfqpoint{3.663965in}{4.252004in}}{\pgfqpoint{3.658141in}{4.252004in}}%
\pgfpathcurveto{\pgfqpoint{3.652317in}{4.252004in}}{\pgfqpoint{3.646731in}{4.249690in}}{\pgfqpoint{3.642612in}{4.245572in}}%
\pgfpathcurveto{\pgfqpoint{3.638494in}{4.241454in}}{\pgfqpoint{3.636180in}{4.235868in}}{\pgfqpoint{3.636180in}{4.230044in}}%
\pgfpathcurveto{\pgfqpoint{3.636180in}{4.224220in}}{\pgfqpoint{3.638494in}{4.218634in}}{\pgfqpoint{3.642612in}{4.214516in}}%
\pgfpathcurveto{\pgfqpoint{3.646731in}{4.210398in}}{\pgfqpoint{3.652317in}{4.208084in}}{\pgfqpoint{3.658141in}{4.208084in}}%
\pgfpathlineto{\pgfqpoint{3.658141in}{4.208084in}}%
\pgfpathclose%
\pgfusepath{stroke,fill}%
\end{pgfscope}%
\begin{pgfscope}%
\pgfpathrectangle{\pgfqpoint{1.000000in}{1.148311in}}{\pgfqpoint{6.200000in}{5.623377in}}%
\pgfusepath{clip}%
\pgfsetbuttcap%
\pgfsetroundjoin%
\definecolor{currentfill}{rgb}{0.800000,0.200000,0.200000}%
\pgfsetfillcolor{currentfill}%
\pgfsetlinewidth{1.003750pt}%
\definecolor{currentstroke}{rgb}{0.800000,0.200000,0.200000}%
\pgfsetstrokecolor{currentstroke}%
\pgfsetdash{}{0pt}%
\pgfpathmoveto{\pgfqpoint{5.417004in}{2.986966in}}%
\pgfpathcurveto{\pgfqpoint{5.422828in}{2.986966in}}{\pgfqpoint{5.428414in}{2.989279in}}{\pgfqpoint{5.432532in}{2.993398in}}%
\pgfpathcurveto{\pgfqpoint{5.436650in}{2.997516in}}{\pgfqpoint{5.438964in}{3.003102in}}{\pgfqpoint{5.438964in}{3.008926in}}%
\pgfpathcurveto{\pgfqpoint{5.438964in}{3.014750in}}{\pgfqpoint{5.436650in}{3.020336in}}{\pgfqpoint{5.432532in}{3.024454in}}%
\pgfpathcurveto{\pgfqpoint{5.428414in}{3.028572in}}{\pgfqpoint{5.422828in}{3.030886in}}{\pgfqpoint{5.417004in}{3.030886in}}%
\pgfpathcurveto{\pgfqpoint{5.411180in}{3.030886in}}{\pgfqpoint{5.405594in}{3.028572in}}{\pgfqpoint{5.401475in}{3.024454in}}%
\pgfpathcurveto{\pgfqpoint{5.397357in}{3.020336in}}{\pgfqpoint{5.395043in}{3.014750in}}{\pgfqpoint{5.395043in}{3.008926in}}%
\pgfpathcurveto{\pgfqpoint{5.395043in}{3.003102in}}{\pgfqpoint{5.397357in}{2.997516in}}{\pgfqpoint{5.401475in}{2.993398in}}%
\pgfpathcurveto{\pgfqpoint{5.405594in}{2.989279in}}{\pgfqpoint{5.411180in}{2.986966in}}{\pgfqpoint{5.417004in}{2.986966in}}%
\pgfpathlineto{\pgfqpoint{5.417004in}{2.986966in}}%
\pgfpathclose%
\pgfusepath{stroke,fill}%
\end{pgfscope}%
\begin{pgfscope}%
\pgfpathrectangle{\pgfqpoint{1.000000in}{1.148311in}}{\pgfqpoint{6.200000in}{5.623377in}}%
\pgfusepath{clip}%
\pgfsetbuttcap%
\pgfsetmiterjoin%
\pgfsetlinewidth{1.003750pt}%
\definecolor{currentstroke}{rgb}{0.800000,0.200000,0.200000}%
\pgfsetstrokecolor{currentstroke}%
\pgfsetdash{}{0pt}%
\pgfpathmoveto{\pgfqpoint{5.855499in}{2.948820in}}%
\pgfpathcurveto{\pgfqpoint{6.138763in}{2.948820in}}{\pgfqpoint{6.410463in}{3.061362in}}{\pgfqpoint{6.610761in}{3.261659in}}%
\pgfpathcurveto{\pgfqpoint{6.811058in}{3.461957in}}{\pgfqpoint{6.923600in}{3.733657in}}{\pgfqpoint{6.923600in}{4.016921in}}%
\pgfpathcurveto{\pgfqpoint{6.923600in}{4.300184in}}{\pgfqpoint{6.811058in}{4.571884in}}{\pgfqpoint{6.610761in}{4.772182in}}%
\pgfpathcurveto{\pgfqpoint{6.410463in}{4.972479in}}{\pgfqpoint{6.138763in}{5.085021in}}{\pgfqpoint{5.855499in}{5.085021in}}%
\pgfpathcurveto{\pgfqpoint{5.572236in}{5.085021in}}{\pgfqpoint{5.300536in}{4.972479in}}{\pgfqpoint{5.100238in}{4.772182in}}%
\pgfpathcurveto{\pgfqpoint{4.899941in}{4.571884in}}{\pgfqpoint{4.787399in}{4.300184in}}{\pgfqpoint{4.787399in}{4.016921in}}%
\pgfpathcurveto{\pgfqpoint{4.787399in}{3.733657in}}{\pgfqpoint{4.899941in}{3.461957in}}{\pgfqpoint{5.100238in}{3.261659in}}%
\pgfpathcurveto{\pgfqpoint{5.300536in}{3.061362in}}{\pgfqpoint{5.572236in}{2.948820in}}{\pgfqpoint{5.855499in}{2.948820in}}%
\pgfpathlineto{\pgfqpoint{5.855499in}{2.948820in}}%
\pgfpathclose%
\pgfusepath{stroke}%
\end{pgfscope}%
\begin{pgfscope}%
\pgfpathrectangle{\pgfqpoint{1.000000in}{1.148311in}}{\pgfqpoint{6.200000in}{5.623377in}}%
\pgfusepath{clip}%
\pgfsetbuttcap%
\pgfsetroundjoin%
\definecolor{currentfill}{rgb}{0.000000,0.000000,0.000000}%
\pgfsetfillcolor{currentfill}%
\pgfsetlinewidth{1.003750pt}%
\definecolor{currentstroke}{rgb}{0.000000,0.000000,0.000000}%
\pgfsetstrokecolor{currentstroke}%
\pgfsetdash{}{0pt}%
\pgfsys@defobject{currentmarker}{\pgfqpoint{-0.021960in}{-0.021960in}}{\pgfqpoint{0.021960in}{0.021960in}}{%
\pgfpathmoveto{\pgfqpoint{0.000000in}{-0.021960in}}%
\pgfpathcurveto{\pgfqpoint{0.005824in}{-0.021960in}}{\pgfqpoint{0.011410in}{-0.019646in}}{\pgfqpoint{0.015528in}{-0.015528in}}%
\pgfpathcurveto{\pgfqpoint{0.019646in}{-0.011410in}}{\pgfqpoint{0.021960in}{-0.005824in}}{\pgfqpoint{0.021960in}{0.000000in}}%
\pgfpathcurveto{\pgfqpoint{0.021960in}{0.005824in}}{\pgfqpoint{0.019646in}{0.011410in}}{\pgfqpoint{0.015528in}{0.015528in}}%
\pgfpathcurveto{\pgfqpoint{0.011410in}{0.019646in}}{\pgfqpoint{0.005824in}{0.021960in}}{\pgfqpoint{0.000000in}{0.021960in}}%
\pgfpathcurveto{\pgfqpoint{-0.005824in}{0.021960in}}{\pgfqpoint{-0.011410in}{0.019646in}}{\pgfqpoint{-0.015528in}{0.015528in}}%
\pgfpathcurveto{\pgfqpoint{-0.019646in}{0.011410in}}{\pgfqpoint{-0.021960in}{0.005824in}}{\pgfqpoint{-0.021960in}{0.000000in}}%
\pgfpathcurveto{\pgfqpoint{-0.021960in}{-0.005824in}}{\pgfqpoint{-0.019646in}{-0.011410in}}{\pgfqpoint{-0.015528in}{-0.015528in}}%
\pgfpathcurveto{\pgfqpoint{-0.011410in}{-0.019646in}}{\pgfqpoint{-0.005824in}{-0.021960in}}{\pgfqpoint{0.000000in}{-0.021960in}}%
\pgfpathlineto{\pgfqpoint{0.000000in}{-0.021960in}}%
\pgfpathclose%
\pgfusepath{stroke,fill}%
}%
\begin{pgfscope}%
\pgfsys@transformshift{5.855499in}{4.016921in}%
\pgfsys@useobject{currentmarker}{}%
\end{pgfscope}%
\end{pgfscope}%
\begin{pgfscope}%
\pgfpathrectangle{\pgfqpoint{1.000000in}{1.148311in}}{\pgfqpoint{6.200000in}{5.623377in}}%
\pgfusepath{clip}%
\pgfsetbuttcap%
\pgfsetmiterjoin%
\pgfsetlinewidth{1.003750pt}%
\definecolor{currentstroke}{rgb}{0.200000,0.800000,0.200000}%
\pgfsetstrokecolor{currentstroke}%
\pgfsetdash{}{0pt}%
\pgfpathmoveto{\pgfqpoint{3.982176in}{1.818348in}}%
\pgfpathcurveto{\pgfqpoint{4.109184in}{1.818348in}}{\pgfqpoint{4.231008in}{1.868809in}}{\pgfqpoint{4.320816in}{1.958618in}}%
\pgfpathcurveto{\pgfqpoint{4.410625in}{2.048426in}}{\pgfqpoint{4.461086in}{2.170250in}}{\pgfqpoint{4.461086in}{2.297258in}}%
\pgfpathcurveto{\pgfqpoint{4.461086in}{2.424267in}}{\pgfqpoint{4.410625in}{2.546090in}}{\pgfqpoint{4.320816in}{2.635899in}}%
\pgfpathcurveto{\pgfqpoint{4.231008in}{2.725707in}}{\pgfqpoint{4.109184in}{2.776168in}}{\pgfqpoint{3.982176in}{2.776168in}}%
\pgfpathcurveto{\pgfqpoint{3.855167in}{2.776168in}}{\pgfqpoint{3.733344in}{2.725707in}}{\pgfqpoint{3.643535in}{2.635899in}}%
\pgfpathcurveto{\pgfqpoint{3.553726in}{2.546090in}}{\pgfqpoint{3.503265in}{2.424267in}}{\pgfqpoint{3.503265in}{2.297258in}}%
\pgfpathcurveto{\pgfqpoint{3.503265in}{2.170250in}}{\pgfqpoint{3.553726in}{2.048426in}}{\pgfqpoint{3.643535in}{1.958618in}}%
\pgfpathcurveto{\pgfqpoint{3.733344in}{1.868809in}}{\pgfqpoint{3.855167in}{1.818348in}}{\pgfqpoint{3.982176in}{1.818348in}}%
\pgfpathlineto{\pgfqpoint{3.982176in}{1.818348in}}%
\pgfpathclose%
\pgfusepath{stroke}%
\end{pgfscope}%
\begin{pgfscope}%
\pgfpathrectangle{\pgfqpoint{1.000000in}{1.148311in}}{\pgfqpoint{6.200000in}{5.623377in}}%
\pgfusepath{clip}%
\pgfsetbuttcap%
\pgfsetroundjoin%
\definecolor{currentfill}{rgb}{0.000000,0.000000,0.000000}%
\pgfsetfillcolor{currentfill}%
\pgfsetlinewidth{1.003750pt}%
\definecolor{currentstroke}{rgb}{0.000000,0.000000,0.000000}%
\pgfsetstrokecolor{currentstroke}%
\pgfsetdash{}{0pt}%
\pgfsys@defobject{currentmarker}{\pgfqpoint{-0.021960in}{-0.021960in}}{\pgfqpoint{0.021960in}{0.021960in}}{%
\pgfpathmoveto{\pgfqpoint{0.000000in}{-0.021960in}}%
\pgfpathcurveto{\pgfqpoint{0.005824in}{-0.021960in}}{\pgfqpoint{0.011410in}{-0.019646in}}{\pgfqpoint{0.015528in}{-0.015528in}}%
\pgfpathcurveto{\pgfqpoint{0.019646in}{-0.011410in}}{\pgfqpoint{0.021960in}{-0.005824in}}{\pgfqpoint{0.021960in}{0.000000in}}%
\pgfpathcurveto{\pgfqpoint{0.021960in}{0.005824in}}{\pgfqpoint{0.019646in}{0.011410in}}{\pgfqpoint{0.015528in}{0.015528in}}%
\pgfpathcurveto{\pgfqpoint{0.011410in}{0.019646in}}{\pgfqpoint{0.005824in}{0.021960in}}{\pgfqpoint{0.000000in}{0.021960in}}%
\pgfpathcurveto{\pgfqpoint{-0.005824in}{0.021960in}}{\pgfqpoint{-0.011410in}{0.019646in}}{\pgfqpoint{-0.015528in}{0.015528in}}%
\pgfpathcurveto{\pgfqpoint{-0.019646in}{0.011410in}}{\pgfqpoint{-0.021960in}{0.005824in}}{\pgfqpoint{-0.021960in}{0.000000in}}%
\pgfpathcurveto{\pgfqpoint{-0.021960in}{-0.005824in}}{\pgfqpoint{-0.019646in}{-0.011410in}}{\pgfqpoint{-0.015528in}{-0.015528in}}%
\pgfpathcurveto{\pgfqpoint{-0.011410in}{-0.019646in}}{\pgfqpoint{-0.005824in}{-0.021960in}}{\pgfqpoint{0.000000in}{-0.021960in}}%
\pgfpathlineto{\pgfqpoint{0.000000in}{-0.021960in}}%
\pgfpathclose%
\pgfusepath{stroke,fill}%
}%
\begin{pgfscope}%
\pgfsys@transformshift{3.982176in}{2.297258in}%
\pgfsys@useobject{currentmarker}{}%
\end{pgfscope}%
\end{pgfscope}%
\begin{pgfscope}%
\pgfpathrectangle{\pgfqpoint{1.000000in}{1.148311in}}{\pgfqpoint{6.200000in}{5.623377in}}%
\pgfusepath{clip}%
\pgfsetbuttcap%
\pgfsetmiterjoin%
\pgfsetlinewidth{1.003750pt}%
\definecolor{currentstroke}{rgb}{0.200000,0.200000,0.800000}%
\pgfsetstrokecolor{currentstroke}%
\pgfsetdash{}{0pt}%
\pgfpathmoveto{\pgfqpoint{5.433044in}{4.238134in}}%
\pgfpathcurveto{\pgfqpoint{5.722590in}{4.238134in}}{\pgfqpoint{6.000315in}{4.353172in}}{\pgfqpoint{6.205055in}{4.557911in}}%
\pgfpathcurveto{\pgfqpoint{6.409794in}{4.762651in}}{\pgfqpoint{6.524832in}{5.040376in}}{\pgfqpoint{6.524832in}{5.329922in}}%
\pgfpathcurveto{\pgfqpoint{6.524832in}{5.619467in}}{\pgfqpoint{6.409794in}{5.897192in}}{\pgfqpoint{6.205055in}{6.101932in}}%
\pgfpathcurveto{\pgfqpoint{6.000315in}{6.306671in}}{\pgfqpoint{5.722590in}{6.421709in}}{\pgfqpoint{5.433044in}{6.421709in}}%
\pgfpathcurveto{\pgfqpoint{5.143499in}{6.421709in}}{\pgfqpoint{4.865774in}{6.306671in}}{\pgfqpoint{4.661034in}{6.101932in}}%
\pgfpathcurveto{\pgfqpoint{4.456294in}{5.897192in}}{\pgfqpoint{4.341257in}{5.619467in}}{\pgfqpoint{4.341257in}{5.329922in}}%
\pgfpathcurveto{\pgfqpoint{4.341257in}{5.040376in}}{\pgfqpoint{4.456294in}{4.762651in}}{\pgfqpoint{4.661034in}{4.557911in}}%
\pgfpathcurveto{\pgfqpoint{4.865774in}{4.353172in}}{\pgfqpoint{5.143499in}{4.238134in}}{\pgfqpoint{5.433044in}{4.238134in}}%
\pgfpathlineto{\pgfqpoint{5.433044in}{4.238134in}}%
\pgfpathclose%
\pgfusepath{stroke}%
\end{pgfscope}%
\begin{pgfscope}%
\pgfpathrectangle{\pgfqpoint{1.000000in}{1.148311in}}{\pgfqpoint{6.200000in}{5.623377in}}%
\pgfusepath{clip}%
\pgfsetbuttcap%
\pgfsetroundjoin%
\definecolor{currentfill}{rgb}{0.000000,0.000000,0.000000}%
\pgfsetfillcolor{currentfill}%
\pgfsetlinewidth{1.003750pt}%
\definecolor{currentstroke}{rgb}{0.000000,0.000000,0.000000}%
\pgfsetstrokecolor{currentstroke}%
\pgfsetdash{}{0pt}%
\pgfsys@defobject{currentmarker}{\pgfqpoint{-0.021960in}{-0.021960in}}{\pgfqpoint{0.021960in}{0.021960in}}{%
\pgfpathmoveto{\pgfqpoint{0.000000in}{-0.021960in}}%
\pgfpathcurveto{\pgfqpoint{0.005824in}{-0.021960in}}{\pgfqpoint{0.011410in}{-0.019646in}}{\pgfqpoint{0.015528in}{-0.015528in}}%
\pgfpathcurveto{\pgfqpoint{0.019646in}{-0.011410in}}{\pgfqpoint{0.021960in}{-0.005824in}}{\pgfqpoint{0.021960in}{0.000000in}}%
\pgfpathcurveto{\pgfqpoint{0.021960in}{0.005824in}}{\pgfqpoint{0.019646in}{0.011410in}}{\pgfqpoint{0.015528in}{0.015528in}}%
\pgfpathcurveto{\pgfqpoint{0.011410in}{0.019646in}}{\pgfqpoint{0.005824in}{0.021960in}}{\pgfqpoint{0.000000in}{0.021960in}}%
\pgfpathcurveto{\pgfqpoint{-0.005824in}{0.021960in}}{\pgfqpoint{-0.011410in}{0.019646in}}{\pgfqpoint{-0.015528in}{0.015528in}}%
\pgfpathcurveto{\pgfqpoint{-0.019646in}{0.011410in}}{\pgfqpoint{-0.021960in}{0.005824in}}{\pgfqpoint{-0.021960in}{0.000000in}}%
\pgfpathcurveto{\pgfqpoint{-0.021960in}{-0.005824in}}{\pgfqpoint{-0.019646in}{-0.011410in}}{\pgfqpoint{-0.015528in}{-0.015528in}}%
\pgfpathcurveto{\pgfqpoint{-0.011410in}{-0.019646in}}{\pgfqpoint{-0.005824in}{-0.021960in}}{\pgfqpoint{0.000000in}{-0.021960in}}%
\pgfpathlineto{\pgfqpoint{0.000000in}{-0.021960in}}%
\pgfpathclose%
\pgfusepath{stroke,fill}%
}%
\begin{pgfscope}%
\pgfsys@transformshift{5.433044in}{5.329922in}%
\pgfsys@useobject{currentmarker}{}%
\end{pgfscope}%
\end{pgfscope}%
\begin{pgfscope}%
\pgfpathrectangle{\pgfqpoint{1.000000in}{1.148311in}}{\pgfqpoint{6.200000in}{5.623377in}}%
\pgfusepath{clip}%
\pgfsetbuttcap%
\pgfsetmiterjoin%
\pgfsetlinewidth{1.003750pt}%
\definecolor{currentstroke}{rgb}{0.800000,0.800000,0.200000}%
\pgfsetstrokecolor{currentstroke}%
\pgfsetdash{}{0pt}%
\pgfpathmoveto{\pgfqpoint{4.405384in}{4.367448in}}%
\pgfpathcurveto{\pgfqpoint{4.679124in}{4.367448in}}{\pgfqpoint{4.941688in}{4.476206in}}{\pgfqpoint{5.135251in}{4.669769in}}%
\pgfpathcurveto{\pgfqpoint{5.328814in}{4.863332in}}{\pgfqpoint{5.437572in}{5.125897in}}{\pgfqpoint{5.437572in}{5.399636in}}%
\pgfpathcurveto{\pgfqpoint{5.437572in}{5.673376in}}{\pgfqpoint{5.328814in}{5.935940in}}{\pgfqpoint{5.135251in}{6.129503in}}%
\pgfpathcurveto{\pgfqpoint{4.941688in}{6.323066in}}{\pgfqpoint{4.679124in}{6.431824in}}{\pgfqpoint{4.405384in}{6.431824in}}%
\pgfpathcurveto{\pgfqpoint{4.131645in}{6.431824in}}{\pgfqpoint{3.869080in}{6.323066in}}{\pgfqpoint{3.675517in}{6.129503in}}%
\pgfpathcurveto{\pgfqpoint{3.481954in}{5.935940in}}{\pgfqpoint{3.373196in}{5.673376in}}{\pgfqpoint{3.373196in}{5.399636in}}%
\pgfpathcurveto{\pgfqpoint{3.373196in}{5.125897in}}{\pgfqpoint{3.481954in}{4.863332in}}{\pgfqpoint{3.675517in}{4.669769in}}%
\pgfpathcurveto{\pgfqpoint{3.869080in}{4.476206in}}{\pgfqpoint{4.131645in}{4.367448in}}{\pgfqpoint{4.405384in}{4.367448in}}%
\pgfpathlineto{\pgfqpoint{4.405384in}{4.367448in}}%
\pgfpathclose%
\pgfusepath{stroke}%
\end{pgfscope}%
\begin{pgfscope}%
\pgfpathrectangle{\pgfqpoint{1.000000in}{1.148311in}}{\pgfqpoint{6.200000in}{5.623377in}}%
\pgfusepath{clip}%
\pgfsetbuttcap%
\pgfsetroundjoin%
\definecolor{currentfill}{rgb}{0.000000,0.000000,0.000000}%
\pgfsetfillcolor{currentfill}%
\pgfsetlinewidth{1.003750pt}%
\definecolor{currentstroke}{rgb}{0.000000,0.000000,0.000000}%
\pgfsetstrokecolor{currentstroke}%
\pgfsetdash{}{0pt}%
\pgfsys@defobject{currentmarker}{\pgfqpoint{-0.021960in}{-0.021960in}}{\pgfqpoint{0.021960in}{0.021960in}}{%
\pgfpathmoveto{\pgfqpoint{0.000000in}{-0.021960in}}%
\pgfpathcurveto{\pgfqpoint{0.005824in}{-0.021960in}}{\pgfqpoint{0.011410in}{-0.019646in}}{\pgfqpoint{0.015528in}{-0.015528in}}%
\pgfpathcurveto{\pgfqpoint{0.019646in}{-0.011410in}}{\pgfqpoint{0.021960in}{-0.005824in}}{\pgfqpoint{0.021960in}{0.000000in}}%
\pgfpathcurveto{\pgfqpoint{0.021960in}{0.005824in}}{\pgfqpoint{0.019646in}{0.011410in}}{\pgfqpoint{0.015528in}{0.015528in}}%
\pgfpathcurveto{\pgfqpoint{0.011410in}{0.019646in}}{\pgfqpoint{0.005824in}{0.021960in}}{\pgfqpoint{0.000000in}{0.021960in}}%
\pgfpathcurveto{\pgfqpoint{-0.005824in}{0.021960in}}{\pgfqpoint{-0.011410in}{0.019646in}}{\pgfqpoint{-0.015528in}{0.015528in}}%
\pgfpathcurveto{\pgfqpoint{-0.019646in}{0.011410in}}{\pgfqpoint{-0.021960in}{0.005824in}}{\pgfqpoint{-0.021960in}{0.000000in}}%
\pgfpathcurveto{\pgfqpoint{-0.021960in}{-0.005824in}}{\pgfqpoint{-0.019646in}{-0.011410in}}{\pgfqpoint{-0.015528in}{-0.015528in}}%
\pgfpathcurveto{\pgfqpoint{-0.011410in}{-0.019646in}}{\pgfqpoint{-0.005824in}{-0.021960in}}{\pgfqpoint{0.000000in}{-0.021960in}}%
\pgfpathlineto{\pgfqpoint{0.000000in}{-0.021960in}}%
\pgfpathclose%
\pgfusepath{stroke,fill}%
}%
\begin{pgfscope}%
\pgfsys@transformshift{4.405384in}{5.399636in}%
\pgfsys@useobject{currentmarker}{}%
\end{pgfscope}%
\end{pgfscope}%
\begin{pgfscope}%
\pgfsetbuttcap%
\pgfsetroundjoin%
\definecolor{currentfill}{rgb}{0.000000,0.000000,0.000000}%
\pgfsetfillcolor{currentfill}%
\pgfsetlinewidth{0.803000pt}%
\definecolor{currentstroke}{rgb}{0.000000,0.000000,0.000000}%
\pgfsetstrokecolor{currentstroke}%
\pgfsetdash{}{0pt}%
\pgfsys@defobject{currentmarker}{\pgfqpoint{0.000000in}{-0.048611in}}{\pgfqpoint{0.000000in}{0.000000in}}{%
\pgfpathmoveto{\pgfqpoint{0.000000in}{0.000000in}}%
\pgfpathlineto{\pgfqpoint{0.000000in}{-0.048611in}}%
\pgfusepath{stroke,fill}%
}%
\begin{pgfscope}%
\pgfsys@transformshift{1.216602in}{1.148311in}%
\pgfsys@useobject{currentmarker}{}%
\end{pgfscope}%
\end{pgfscope}%
\begin{pgfscope}%
\definecolor{textcolor}{rgb}{0.000000,0.000000,0.000000}%
\pgfsetstrokecolor{textcolor}%
\pgfsetfillcolor{textcolor}%
\pgftext[x=1.216602in,y=1.051089in,,top]{\color{textcolor}{\sffamily\fontsize{10.000000}{12.000000}\selectfont\catcode`\^=\active\def^{\ifmmode\sp\else\^{}\fi}\catcode`\%=\active\def%{\%}\ensuremath{-}600}}%
\end{pgfscope}%
\begin{pgfscope}%
\pgfsetbuttcap%
\pgfsetroundjoin%
\definecolor{currentfill}{rgb}{0.000000,0.000000,0.000000}%
\pgfsetfillcolor{currentfill}%
\pgfsetlinewidth{0.803000pt}%
\definecolor{currentstroke}{rgb}{0.000000,0.000000,0.000000}%
\pgfsetstrokecolor{currentstroke}%
\pgfsetdash{}{0pt}%
\pgfsys@defobject{currentmarker}{\pgfqpoint{0.000000in}{-0.048611in}}{\pgfqpoint{0.000000in}{0.000000in}}{%
\pgfpathmoveto{\pgfqpoint{0.000000in}{0.000000in}}%
\pgfpathlineto{\pgfqpoint{0.000000in}{-0.048611in}}%
\pgfusepath{stroke,fill}%
}%
\begin{pgfscope}%
\pgfsys@transformshift{2.008223in}{1.148311in}%
\pgfsys@useobject{currentmarker}{}%
\end{pgfscope}%
\end{pgfscope}%
\begin{pgfscope}%
\definecolor{textcolor}{rgb}{0.000000,0.000000,0.000000}%
\pgfsetstrokecolor{textcolor}%
\pgfsetfillcolor{textcolor}%
\pgftext[x=2.008223in,y=1.051089in,,top]{\color{textcolor}{\sffamily\fontsize{10.000000}{12.000000}\selectfont\catcode`\^=\active\def^{\ifmmode\sp\else\^{}\fi}\catcode`\%=\active\def%{\%}\ensuremath{-}400}}%
\end{pgfscope}%
\begin{pgfscope}%
\pgfsetbuttcap%
\pgfsetroundjoin%
\definecolor{currentfill}{rgb}{0.000000,0.000000,0.000000}%
\pgfsetfillcolor{currentfill}%
\pgfsetlinewidth{0.803000pt}%
\definecolor{currentstroke}{rgb}{0.000000,0.000000,0.000000}%
\pgfsetstrokecolor{currentstroke}%
\pgfsetdash{}{0pt}%
\pgfsys@defobject{currentmarker}{\pgfqpoint{0.000000in}{-0.048611in}}{\pgfqpoint{0.000000in}{0.000000in}}{%
\pgfpathmoveto{\pgfqpoint{0.000000in}{0.000000in}}%
\pgfpathlineto{\pgfqpoint{0.000000in}{-0.048611in}}%
\pgfusepath{stroke,fill}%
}%
\begin{pgfscope}%
\pgfsys@transformshift{2.799844in}{1.148311in}%
\pgfsys@useobject{currentmarker}{}%
\end{pgfscope}%
\end{pgfscope}%
\begin{pgfscope}%
\definecolor{textcolor}{rgb}{0.000000,0.000000,0.000000}%
\pgfsetstrokecolor{textcolor}%
\pgfsetfillcolor{textcolor}%
\pgftext[x=2.799844in,y=1.051089in,,top]{\color{textcolor}{\sffamily\fontsize{10.000000}{12.000000}\selectfont\catcode`\^=\active\def^{\ifmmode\sp\else\^{}\fi}\catcode`\%=\active\def%{\%}\ensuremath{-}200}}%
\end{pgfscope}%
\begin{pgfscope}%
\pgfsetbuttcap%
\pgfsetroundjoin%
\definecolor{currentfill}{rgb}{0.000000,0.000000,0.000000}%
\pgfsetfillcolor{currentfill}%
\pgfsetlinewidth{0.803000pt}%
\definecolor{currentstroke}{rgb}{0.000000,0.000000,0.000000}%
\pgfsetstrokecolor{currentstroke}%
\pgfsetdash{}{0pt}%
\pgfsys@defobject{currentmarker}{\pgfqpoint{0.000000in}{-0.048611in}}{\pgfqpoint{0.000000in}{0.000000in}}{%
\pgfpathmoveto{\pgfqpoint{0.000000in}{0.000000in}}%
\pgfpathlineto{\pgfqpoint{0.000000in}{-0.048611in}}%
\pgfusepath{stroke,fill}%
}%
\begin{pgfscope}%
\pgfsys@transformshift{3.591464in}{1.148311in}%
\pgfsys@useobject{currentmarker}{}%
\end{pgfscope}%
\end{pgfscope}%
\begin{pgfscope}%
\definecolor{textcolor}{rgb}{0.000000,0.000000,0.000000}%
\pgfsetstrokecolor{textcolor}%
\pgfsetfillcolor{textcolor}%
\pgftext[x=3.591464in,y=1.051089in,,top]{\color{textcolor}{\sffamily\fontsize{10.000000}{12.000000}\selectfont\catcode`\^=\active\def^{\ifmmode\sp\else\^{}\fi}\catcode`\%=\active\def%{\%}0}}%
\end{pgfscope}%
\begin{pgfscope}%
\pgfsetbuttcap%
\pgfsetroundjoin%
\definecolor{currentfill}{rgb}{0.000000,0.000000,0.000000}%
\pgfsetfillcolor{currentfill}%
\pgfsetlinewidth{0.803000pt}%
\definecolor{currentstroke}{rgb}{0.000000,0.000000,0.000000}%
\pgfsetstrokecolor{currentstroke}%
\pgfsetdash{}{0pt}%
\pgfsys@defobject{currentmarker}{\pgfqpoint{0.000000in}{-0.048611in}}{\pgfqpoint{0.000000in}{0.000000in}}{%
\pgfpathmoveto{\pgfqpoint{0.000000in}{0.000000in}}%
\pgfpathlineto{\pgfqpoint{0.000000in}{-0.048611in}}%
\pgfusepath{stroke,fill}%
}%
\begin{pgfscope}%
\pgfsys@transformshift{4.383085in}{1.148311in}%
\pgfsys@useobject{currentmarker}{}%
\end{pgfscope}%
\end{pgfscope}%
\begin{pgfscope}%
\definecolor{textcolor}{rgb}{0.000000,0.000000,0.000000}%
\pgfsetstrokecolor{textcolor}%
\pgfsetfillcolor{textcolor}%
\pgftext[x=4.383085in,y=1.051089in,,top]{\color{textcolor}{\sffamily\fontsize{10.000000}{12.000000}\selectfont\catcode`\^=\active\def^{\ifmmode\sp\else\^{}\fi}\catcode`\%=\active\def%{\%}200}}%
\end{pgfscope}%
\begin{pgfscope}%
\pgfsetbuttcap%
\pgfsetroundjoin%
\definecolor{currentfill}{rgb}{0.000000,0.000000,0.000000}%
\pgfsetfillcolor{currentfill}%
\pgfsetlinewidth{0.803000pt}%
\definecolor{currentstroke}{rgb}{0.000000,0.000000,0.000000}%
\pgfsetstrokecolor{currentstroke}%
\pgfsetdash{}{0pt}%
\pgfsys@defobject{currentmarker}{\pgfqpoint{0.000000in}{-0.048611in}}{\pgfqpoint{0.000000in}{0.000000in}}{%
\pgfpathmoveto{\pgfqpoint{0.000000in}{0.000000in}}%
\pgfpathlineto{\pgfqpoint{0.000000in}{-0.048611in}}%
\pgfusepath{stroke,fill}%
}%
\begin{pgfscope}%
\pgfsys@transformshift{5.174706in}{1.148311in}%
\pgfsys@useobject{currentmarker}{}%
\end{pgfscope}%
\end{pgfscope}%
\begin{pgfscope}%
\definecolor{textcolor}{rgb}{0.000000,0.000000,0.000000}%
\pgfsetstrokecolor{textcolor}%
\pgfsetfillcolor{textcolor}%
\pgftext[x=5.174706in,y=1.051089in,,top]{\color{textcolor}{\sffamily\fontsize{10.000000}{12.000000}\selectfont\catcode`\^=\active\def^{\ifmmode\sp\else\^{}\fi}\catcode`\%=\active\def%{\%}400}}%
\end{pgfscope}%
\begin{pgfscope}%
\pgfsetbuttcap%
\pgfsetroundjoin%
\definecolor{currentfill}{rgb}{0.000000,0.000000,0.000000}%
\pgfsetfillcolor{currentfill}%
\pgfsetlinewidth{0.803000pt}%
\definecolor{currentstroke}{rgb}{0.000000,0.000000,0.000000}%
\pgfsetstrokecolor{currentstroke}%
\pgfsetdash{}{0pt}%
\pgfsys@defobject{currentmarker}{\pgfqpoint{0.000000in}{-0.048611in}}{\pgfqpoint{0.000000in}{0.000000in}}{%
\pgfpathmoveto{\pgfqpoint{0.000000in}{0.000000in}}%
\pgfpathlineto{\pgfqpoint{0.000000in}{-0.048611in}}%
\pgfusepath{stroke,fill}%
}%
\begin{pgfscope}%
\pgfsys@transformshift{5.966327in}{1.148311in}%
\pgfsys@useobject{currentmarker}{}%
\end{pgfscope}%
\end{pgfscope}%
\begin{pgfscope}%
\definecolor{textcolor}{rgb}{0.000000,0.000000,0.000000}%
\pgfsetstrokecolor{textcolor}%
\pgfsetfillcolor{textcolor}%
\pgftext[x=5.966327in,y=1.051089in,,top]{\color{textcolor}{\sffamily\fontsize{10.000000}{12.000000}\selectfont\catcode`\^=\active\def^{\ifmmode\sp\else\^{}\fi}\catcode`\%=\active\def%{\%}600}}%
\end{pgfscope}%
\begin{pgfscope}%
\pgfsetbuttcap%
\pgfsetroundjoin%
\definecolor{currentfill}{rgb}{0.000000,0.000000,0.000000}%
\pgfsetfillcolor{currentfill}%
\pgfsetlinewidth{0.803000pt}%
\definecolor{currentstroke}{rgb}{0.000000,0.000000,0.000000}%
\pgfsetstrokecolor{currentstroke}%
\pgfsetdash{}{0pt}%
\pgfsys@defobject{currentmarker}{\pgfqpoint{0.000000in}{-0.048611in}}{\pgfqpoint{0.000000in}{0.000000in}}{%
\pgfpathmoveto{\pgfqpoint{0.000000in}{0.000000in}}%
\pgfpathlineto{\pgfqpoint{0.000000in}{-0.048611in}}%
\pgfusepath{stroke,fill}%
}%
\begin{pgfscope}%
\pgfsys@transformshift{6.757948in}{1.148311in}%
\pgfsys@useobject{currentmarker}{}%
\end{pgfscope}%
\end{pgfscope}%
\begin{pgfscope}%
\definecolor{textcolor}{rgb}{0.000000,0.000000,0.000000}%
\pgfsetstrokecolor{textcolor}%
\pgfsetfillcolor{textcolor}%
\pgftext[x=6.757948in,y=1.051089in,,top]{\color{textcolor}{\sffamily\fontsize{10.000000}{12.000000}\selectfont\catcode`\^=\active\def^{\ifmmode\sp\else\^{}\fi}\catcode`\%=\active\def%{\%}800}}%
\end{pgfscope}%
\begin{pgfscope}%
\pgfsetbuttcap%
\pgfsetroundjoin%
\definecolor{currentfill}{rgb}{0.000000,0.000000,0.000000}%
\pgfsetfillcolor{currentfill}%
\pgfsetlinewidth{0.803000pt}%
\definecolor{currentstroke}{rgb}{0.000000,0.000000,0.000000}%
\pgfsetstrokecolor{currentstroke}%
\pgfsetdash{}{0pt}%
\pgfsys@defobject{currentmarker}{\pgfqpoint{-0.048611in}{0.000000in}}{\pgfqpoint{-0.000000in}{0.000000in}}{%
\pgfpathmoveto{\pgfqpoint{-0.000000in}{0.000000in}}%
\pgfpathlineto{\pgfqpoint{-0.048611in}{0.000000in}}%
\pgfusepath{stroke,fill}%
}%
\begin{pgfscope}%
\pgfsys@transformshift{1.000000in}{1.368413in}%
\pgfsys@useobject{currentmarker}{}%
\end{pgfscope}%
\end{pgfscope}%
\begin{pgfscope}%
\definecolor{textcolor}{rgb}{0.000000,0.000000,0.000000}%
\pgfsetstrokecolor{textcolor}%
\pgfsetfillcolor{textcolor}%
\pgftext[x=0.529657in, y=1.315651in, left, base]{\color{textcolor}{\sffamily\fontsize{10.000000}{12.000000}\selectfont\catcode`\^=\active\def^{\ifmmode\sp\else\^{}\fi}\catcode`\%=\active\def%{\%}\ensuremath{-}600}}%
\end{pgfscope}%
\begin{pgfscope}%
\pgfsetbuttcap%
\pgfsetroundjoin%
\definecolor{currentfill}{rgb}{0.000000,0.000000,0.000000}%
\pgfsetfillcolor{currentfill}%
\pgfsetlinewidth{0.803000pt}%
\definecolor{currentstroke}{rgb}{0.000000,0.000000,0.000000}%
\pgfsetstrokecolor{currentstroke}%
\pgfsetdash{}{0pt}%
\pgfsys@defobject{currentmarker}{\pgfqpoint{-0.048611in}{0.000000in}}{\pgfqpoint{-0.000000in}{0.000000in}}{%
\pgfpathmoveto{\pgfqpoint{-0.000000in}{0.000000in}}%
\pgfpathlineto{\pgfqpoint{-0.048611in}{0.000000in}}%
\pgfusepath{stroke,fill}%
}%
\begin{pgfscope}%
\pgfsys@transformshift{1.000000in}{2.160033in}%
\pgfsys@useobject{currentmarker}{}%
\end{pgfscope}%
\end{pgfscope}%
\begin{pgfscope}%
\definecolor{textcolor}{rgb}{0.000000,0.000000,0.000000}%
\pgfsetstrokecolor{textcolor}%
\pgfsetfillcolor{textcolor}%
\pgftext[x=0.529657in, y=2.107272in, left, base]{\color{textcolor}{\sffamily\fontsize{10.000000}{12.000000}\selectfont\catcode`\^=\active\def^{\ifmmode\sp\else\^{}\fi}\catcode`\%=\active\def%{\%}\ensuremath{-}400}}%
\end{pgfscope}%
\begin{pgfscope}%
\pgfsetbuttcap%
\pgfsetroundjoin%
\definecolor{currentfill}{rgb}{0.000000,0.000000,0.000000}%
\pgfsetfillcolor{currentfill}%
\pgfsetlinewidth{0.803000pt}%
\definecolor{currentstroke}{rgb}{0.000000,0.000000,0.000000}%
\pgfsetstrokecolor{currentstroke}%
\pgfsetdash{}{0pt}%
\pgfsys@defobject{currentmarker}{\pgfqpoint{-0.048611in}{0.000000in}}{\pgfqpoint{-0.000000in}{0.000000in}}{%
\pgfpathmoveto{\pgfqpoint{-0.000000in}{0.000000in}}%
\pgfpathlineto{\pgfqpoint{-0.048611in}{0.000000in}}%
\pgfusepath{stroke,fill}%
}%
\begin{pgfscope}%
\pgfsys@transformshift{1.000000in}{2.951654in}%
\pgfsys@useobject{currentmarker}{}%
\end{pgfscope}%
\end{pgfscope}%
\begin{pgfscope}%
\definecolor{textcolor}{rgb}{0.000000,0.000000,0.000000}%
\pgfsetstrokecolor{textcolor}%
\pgfsetfillcolor{textcolor}%
\pgftext[x=0.529657in, y=2.898893in, left, base]{\color{textcolor}{\sffamily\fontsize{10.000000}{12.000000}\selectfont\catcode`\^=\active\def^{\ifmmode\sp\else\^{}\fi}\catcode`\%=\active\def%{\%}\ensuremath{-}200}}%
\end{pgfscope}%
\begin{pgfscope}%
\pgfsetbuttcap%
\pgfsetroundjoin%
\definecolor{currentfill}{rgb}{0.000000,0.000000,0.000000}%
\pgfsetfillcolor{currentfill}%
\pgfsetlinewidth{0.803000pt}%
\definecolor{currentstroke}{rgb}{0.000000,0.000000,0.000000}%
\pgfsetstrokecolor{currentstroke}%
\pgfsetdash{}{0pt}%
\pgfsys@defobject{currentmarker}{\pgfqpoint{-0.048611in}{0.000000in}}{\pgfqpoint{-0.000000in}{0.000000in}}{%
\pgfpathmoveto{\pgfqpoint{-0.000000in}{0.000000in}}%
\pgfpathlineto{\pgfqpoint{-0.048611in}{0.000000in}}%
\pgfusepath{stroke,fill}%
}%
\begin{pgfscope}%
\pgfsys@transformshift{1.000000in}{3.743275in}%
\pgfsys@useobject{currentmarker}{}%
\end{pgfscope}%
\end{pgfscope}%
\begin{pgfscope}%
\definecolor{textcolor}{rgb}{0.000000,0.000000,0.000000}%
\pgfsetstrokecolor{textcolor}%
\pgfsetfillcolor{textcolor}%
\pgftext[x=0.814412in, y=3.690514in, left, base]{\color{textcolor}{\sffamily\fontsize{10.000000}{12.000000}\selectfont\catcode`\^=\active\def^{\ifmmode\sp\else\^{}\fi}\catcode`\%=\active\def%{\%}0}}%
\end{pgfscope}%
\begin{pgfscope}%
\pgfsetbuttcap%
\pgfsetroundjoin%
\definecolor{currentfill}{rgb}{0.000000,0.000000,0.000000}%
\pgfsetfillcolor{currentfill}%
\pgfsetlinewidth{0.803000pt}%
\definecolor{currentstroke}{rgb}{0.000000,0.000000,0.000000}%
\pgfsetstrokecolor{currentstroke}%
\pgfsetdash{}{0pt}%
\pgfsys@defobject{currentmarker}{\pgfqpoint{-0.048611in}{0.000000in}}{\pgfqpoint{-0.000000in}{0.000000in}}{%
\pgfpathmoveto{\pgfqpoint{-0.000000in}{0.000000in}}%
\pgfpathlineto{\pgfqpoint{-0.048611in}{0.000000in}}%
\pgfusepath{stroke,fill}%
}%
\begin{pgfscope}%
\pgfsys@transformshift{1.000000in}{4.534896in}%
\pgfsys@useobject{currentmarker}{}%
\end{pgfscope}%
\end{pgfscope}%
\begin{pgfscope}%
\definecolor{textcolor}{rgb}{0.000000,0.000000,0.000000}%
\pgfsetstrokecolor{textcolor}%
\pgfsetfillcolor{textcolor}%
\pgftext[x=0.637682in, y=4.482135in, left, base]{\color{textcolor}{\sffamily\fontsize{10.000000}{12.000000}\selectfont\catcode`\^=\active\def^{\ifmmode\sp\else\^{}\fi}\catcode`\%=\active\def%{\%}200}}%
\end{pgfscope}%
\begin{pgfscope}%
\pgfsetbuttcap%
\pgfsetroundjoin%
\definecolor{currentfill}{rgb}{0.000000,0.000000,0.000000}%
\pgfsetfillcolor{currentfill}%
\pgfsetlinewidth{0.803000pt}%
\definecolor{currentstroke}{rgb}{0.000000,0.000000,0.000000}%
\pgfsetstrokecolor{currentstroke}%
\pgfsetdash{}{0pt}%
\pgfsys@defobject{currentmarker}{\pgfqpoint{-0.048611in}{0.000000in}}{\pgfqpoint{-0.000000in}{0.000000in}}{%
\pgfpathmoveto{\pgfqpoint{-0.000000in}{0.000000in}}%
\pgfpathlineto{\pgfqpoint{-0.048611in}{0.000000in}}%
\pgfusepath{stroke,fill}%
}%
\begin{pgfscope}%
\pgfsys@transformshift{1.000000in}{5.326517in}%
\pgfsys@useobject{currentmarker}{}%
\end{pgfscope}%
\end{pgfscope}%
\begin{pgfscope}%
\definecolor{textcolor}{rgb}{0.000000,0.000000,0.000000}%
\pgfsetstrokecolor{textcolor}%
\pgfsetfillcolor{textcolor}%
\pgftext[x=0.637682in, y=5.273756in, left, base]{\color{textcolor}{\sffamily\fontsize{10.000000}{12.000000}\selectfont\catcode`\^=\active\def^{\ifmmode\sp\else\^{}\fi}\catcode`\%=\active\def%{\%}400}}%
\end{pgfscope}%
\begin{pgfscope}%
\pgfsetbuttcap%
\pgfsetroundjoin%
\definecolor{currentfill}{rgb}{0.000000,0.000000,0.000000}%
\pgfsetfillcolor{currentfill}%
\pgfsetlinewidth{0.803000pt}%
\definecolor{currentstroke}{rgb}{0.000000,0.000000,0.000000}%
\pgfsetstrokecolor{currentstroke}%
\pgfsetdash{}{0pt}%
\pgfsys@defobject{currentmarker}{\pgfqpoint{-0.048611in}{0.000000in}}{\pgfqpoint{-0.000000in}{0.000000in}}{%
\pgfpathmoveto{\pgfqpoint{-0.000000in}{0.000000in}}%
\pgfpathlineto{\pgfqpoint{-0.048611in}{0.000000in}}%
\pgfusepath{stroke,fill}%
}%
\begin{pgfscope}%
\pgfsys@transformshift{1.000000in}{6.118138in}%
\pgfsys@useobject{currentmarker}{}%
\end{pgfscope}%
\end{pgfscope}%
\begin{pgfscope}%
\definecolor{textcolor}{rgb}{0.000000,0.000000,0.000000}%
\pgfsetstrokecolor{textcolor}%
\pgfsetfillcolor{textcolor}%
\pgftext[x=0.637682in, y=6.065377in, left, base]{\color{textcolor}{\sffamily\fontsize{10.000000}{12.000000}\selectfont\catcode`\^=\active\def^{\ifmmode\sp\else\^{}\fi}\catcode`\%=\active\def%{\%}600}}%
\end{pgfscope}%
\begin{pgfscope}%
\pgfsetrectcap%
\pgfsetmiterjoin%
\pgfsetlinewidth{0.803000pt}%
\definecolor{currentstroke}{rgb}{0.000000,0.000000,0.000000}%
\pgfsetstrokecolor{currentstroke}%
\pgfsetdash{}{0pt}%
\pgfpathmoveto{\pgfqpoint{1.000000in}{1.148311in}}%
\pgfpathlineto{\pgfqpoint{1.000000in}{6.771689in}}%
\pgfusepath{stroke}%
\end{pgfscope}%
\begin{pgfscope}%
\pgfsetrectcap%
\pgfsetmiterjoin%
\pgfsetlinewidth{0.803000pt}%
\definecolor{currentstroke}{rgb}{0.000000,0.000000,0.000000}%
\pgfsetstrokecolor{currentstroke}%
\pgfsetdash{}{0pt}%
\pgfpathmoveto{\pgfqpoint{7.200000in}{1.148311in}}%
\pgfpathlineto{\pgfqpoint{7.200000in}{6.771689in}}%
\pgfusepath{stroke}%
\end{pgfscope}%
\begin{pgfscope}%
\pgfsetrectcap%
\pgfsetmiterjoin%
\pgfsetlinewidth{0.803000pt}%
\definecolor{currentstroke}{rgb}{0.000000,0.000000,0.000000}%
\pgfsetstrokecolor{currentstroke}%
\pgfsetdash{}{0pt}%
\pgfpathmoveto{\pgfqpoint{1.000000in}{1.148311in}}%
\pgfpathlineto{\pgfqpoint{7.200000in}{1.148311in}}%
\pgfusepath{stroke}%
\end{pgfscope}%
\begin{pgfscope}%
\pgfsetrectcap%
\pgfsetmiterjoin%
\pgfsetlinewidth{0.803000pt}%
\definecolor{currentstroke}{rgb}{0.000000,0.000000,0.000000}%
\pgfsetstrokecolor{currentstroke}%
\pgfsetdash{}{0pt}%
\pgfpathmoveto{\pgfqpoint{1.000000in}{6.771689in}}%
\pgfpathlineto{\pgfqpoint{7.200000in}{6.771689in}}%
\pgfusepath{stroke}%
\end{pgfscope}%
\end{pgfpicture}%
\makeatother%
\endgroup%
}
    \label{fig:excentric_rings}
    \caption{Example of a dataset with 4 noisy rings, and background noise, and a good classification.}
\end{figure}

\begin{figure*}[!ht]
\centering
\begin{tabular}{rrrrrrrr}
    \hline
       Number of rings &   Ring noise &   Background noise &   Avg. Error &   Avg. Runtime &   Iterations &   Experiments &   Avg. Detected Noise \\
    \hline
                     1 &            0 &                  0 &      1.83853 &    0.000812167 &         2    &             3 &               0       \\
                     1 &           20 &                  0 &     16.2363  &    0.000727667 &         2    &             3 &               0       \\
                     1 &           20 &                 20 &     16.1469  &    0.00178713  &         6    &             3 &              18       \\
                     2 &            0 &                  0 &      1.6219  &    0.499586    &      2505.25 &             4 &               0       \\
                     2 &           10 &                  0 &     13.9375  &    0.0042801   &        14    &             3 &               0       \\
                     2 &           20 &                  0 &     16.0684  &    0.692698    &      3359    &             3 &               0       \\
                     2 &           10 &                 10 &     13.4885  &    0.731724    &      3346.33 &             3 &               3.33333 \\
                     3 &            0 &                  0 &     74.1423  &    2.47752     &      8005    &             5 &              15.4     \\
                     3 &           10 &                  0 &     10.2179  &    1.57101     &      5011.5  &             4 &               0       \\
                     3 &           10 &                 15 &     21.5381  &    2.1628      &      6671.33 &             3 &               3.33333 \\
                     4 &            0 &                  0 &     28.8584  &    2.72986     &      6010.8  &             5 &               0       \\
                     4 &           10 &                  0 &     15.5972  &    2.31535     &      5007.75 &             4 &               6.5     \\
                     4 &           10 &                 10 &     24.6138  &    1.61891     &      3397    &             3 &              64       \\
                     5 &            0 &                  0 &     13.6484  &    4.82393     &      7523    &             4 &               7       \\
                     5 &           10 &                  0 &     11.4402  &    6.47775     &     10000    &             2 &               0       \\
                     5 &           10 &                 10 &     30.3023  &    6.46981     &     10000    &             3 &              34       \\
    \hline
    \end{tabular}
\caption{Results of the general test with excentric rings.}
\end{figure*}

\subsubsection{Concentric Rings}
A concentric ring dataset is a dataset in which all the rings share the same center, and may vary in radius.

The hyperparameters were set as follows:

\begin{itemize}
    \item q: 1.1
    \item convergence\_eps: $10^{-5}$
    \item max\_iters: 10000
    \item noise\_distance\_threshold: 100
    \item max\_noise\_checks: 20
    \item apply\_noise\_removal: True
    \item init\_method: "concentric"
\end{itemize}

For the data generation:
\begin{itemize}
    \item For each ring, we randomly select a radius between 50 and 1000.
    \item For each ring, we randomly select 100 samples in the circunference. For each sample, we add a noise with the following equation:
    \begin{equation}
        X_{\text{noise}} = X_{\text{ring}} + \text{randn}(0, 1) \cdot \text{noise\_level}
    \end{equation}
    Where $X_{\text{ring}}$ is the point in the circunference, and $\text{noise\_level}$ is the noise level, and $\text{randn}(0, 1)$ is a random number from a normal distribution with mean 0 and variance 1.
    \item To add the background noise, we select N points in a rect centered at $(0, 0)$ with sides of length 1200, sampled from an uniform distribution.
\end{itemize}

\begin{figure}[H]
    \centering
    \resizebox{0.9\linewidth}{!}{%% Creator: Matplotlib, PGF backend
%%
%% To include the figure in your LaTeX document, write
%%   \input{<filename>.pgf}
%%
%% Make sure the required packages are loaded in your preamble
%%   \usepackage{pgf}
%%
%% Also ensure that all the required font packages are loaded; for instance,
%% the lmodern package is sometimes necessary when using math font.
%%   \usepackage{lmodern}
%%
%% Figures using additional raster images can only be included by \input if
%% they are in the same directory as the main LaTeX file. For loading figures
%% from other directories you can use the `import` package
%%   \usepackage{import}
%%
%% and then include the figures with
%%   \import{<path to file>}{<filename>.pgf}
%%
%% Matplotlib used the following preamble
%%   \def\mathdefault#1{#1}
%%   \everymath=\expandafter{\the\everymath\displaystyle}
%%   
%%   \usepackage{fontspec}
%%   \setmainfont{DejaVuSerif.ttf}[Path=\detokenize{C:/Users/dagom/anaconda3/envs/pytorch/lib/site-packages/matplotlib/mpl-data/fonts/ttf/}]
%%   \setsansfont{DejaVuSans.ttf}[Path=\detokenize{C:/Users/dagom/anaconda3/envs/pytorch/lib/site-packages/matplotlib/mpl-data/fonts/ttf/}]
%%   \setmonofont{DejaVuSansMono.ttf}[Path=\detokenize{C:/Users/dagom/anaconda3/envs/pytorch/lib/site-packages/matplotlib/mpl-data/fonts/ttf/}]
%%   \makeatletter\@ifpackageloaded{underscore}{}{\usepackage[strings]{underscore}}\makeatother
%%
\begingroup%
\makeatletter%
\begin{pgfpicture}%
\pgfpathrectangle{\pgfpointorigin}{\pgfqpoint{8.000000in}{8.000000in}}%
\pgfusepath{use as bounding box, clip}%
\begin{pgfscope}%
\pgfsetbuttcap%
\pgfsetmiterjoin%
\definecolor{currentfill}{rgb}{1.000000,1.000000,1.000000}%
\pgfsetfillcolor{currentfill}%
\pgfsetlinewidth{0.000000pt}%
\definecolor{currentstroke}{rgb}{1.000000,1.000000,1.000000}%
\pgfsetstrokecolor{currentstroke}%
\pgfsetdash{}{0pt}%
\pgfpathmoveto{\pgfqpoint{0.000000in}{0.000000in}}%
\pgfpathlineto{\pgfqpoint{8.000000in}{0.000000in}}%
\pgfpathlineto{\pgfqpoint{8.000000in}{8.000000in}}%
\pgfpathlineto{\pgfqpoint{0.000000in}{8.000000in}}%
\pgfpathlineto{\pgfqpoint{0.000000in}{0.000000in}}%
\pgfpathclose%
\pgfusepath{fill}%
\end{pgfscope}%
\begin{pgfscope}%
\pgfsetbuttcap%
\pgfsetmiterjoin%
\definecolor{currentfill}{rgb}{1.000000,1.000000,1.000000}%
\pgfsetfillcolor{currentfill}%
\pgfsetlinewidth{0.000000pt}%
\definecolor{currentstroke}{rgb}{0.000000,0.000000,0.000000}%
\pgfsetstrokecolor{currentstroke}%
\pgfsetstrokeopacity{0.000000}%
\pgfsetdash{}{0pt}%
\pgfpathmoveto{\pgfqpoint{1.073501in}{0.880000in}}%
\pgfpathlineto{\pgfqpoint{7.126499in}{0.880000in}}%
\pgfpathlineto{\pgfqpoint{7.126499in}{7.040000in}}%
\pgfpathlineto{\pgfqpoint{1.073501in}{7.040000in}}%
\pgfpathlineto{\pgfqpoint{1.073501in}{0.880000in}}%
\pgfpathclose%
\pgfusepath{fill}%
\end{pgfscope}%
\begin{pgfscope}%
\pgfpathrectangle{\pgfqpoint{1.073501in}{0.880000in}}{\pgfqpoint{6.052998in}{6.160000in}}%
\pgfusepath{clip}%
\pgfsetbuttcap%
\pgfsetroundjoin%
\definecolor{currentfill}{rgb}{0.800000,0.200000,0.200000}%
\pgfsetfillcolor{currentfill}%
\pgfsetlinewidth{1.003750pt}%
\definecolor{currentstroke}{rgb}{0.800000,0.200000,0.200000}%
\pgfsetstrokecolor{currentstroke}%
\pgfsetdash{}{0pt}%
\pgfpathmoveto{\pgfqpoint{6.276293in}{3.925790in}}%
\pgfpathcurveto{\pgfqpoint{6.282117in}{3.925790in}}{\pgfqpoint{6.287703in}{3.928103in}}{\pgfqpoint{6.291822in}{3.932222in}}%
\pgfpathcurveto{\pgfqpoint{6.295940in}{3.936340in}}{\pgfqpoint{6.298254in}{3.941926in}}{\pgfqpoint{6.298254in}{3.947750in}}%
\pgfpathcurveto{\pgfqpoint{6.298254in}{3.953574in}}{\pgfqpoint{6.295940in}{3.959160in}}{\pgfqpoint{6.291822in}{3.963278in}}%
\pgfpathcurveto{\pgfqpoint{6.287703in}{3.967396in}}{\pgfqpoint{6.282117in}{3.969710in}}{\pgfqpoint{6.276293in}{3.969710in}}%
\pgfpathcurveto{\pgfqpoint{6.270469in}{3.969710in}}{\pgfqpoint{6.264883in}{3.967396in}}{\pgfqpoint{6.260765in}{3.963278in}}%
\pgfpathcurveto{\pgfqpoint{6.256647in}{3.959160in}}{\pgfqpoint{6.254333in}{3.953574in}}{\pgfqpoint{6.254333in}{3.947750in}}%
\pgfpathcurveto{\pgfqpoint{6.254333in}{3.941926in}}{\pgfqpoint{6.256647in}{3.936340in}}{\pgfqpoint{6.260765in}{3.932222in}}%
\pgfpathcurveto{\pgfqpoint{6.264883in}{3.928103in}}{\pgfqpoint{6.270469in}{3.925790in}}{\pgfqpoint{6.276293in}{3.925790in}}%
\pgfpathlineto{\pgfqpoint{6.276293in}{3.925790in}}%
\pgfpathclose%
\pgfusepath{stroke,fill}%
\end{pgfscope}%
\begin{pgfscope}%
\pgfpathrectangle{\pgfqpoint{1.073501in}{0.880000in}}{\pgfqpoint{6.052998in}{6.160000in}}%
\pgfusepath{clip}%
\pgfsetbuttcap%
\pgfsetroundjoin%
\definecolor{currentfill}{rgb}{0.800000,0.200000,0.200000}%
\pgfsetfillcolor{currentfill}%
\pgfsetlinewidth{1.003750pt}%
\definecolor{currentstroke}{rgb}{0.800000,0.200000,0.200000}%
\pgfsetstrokecolor{currentstroke}%
\pgfsetdash{}{0pt}%
\pgfpathmoveto{\pgfqpoint{6.362277in}{4.067934in}}%
\pgfpathcurveto{\pgfqpoint{6.368101in}{4.067934in}}{\pgfqpoint{6.373687in}{4.070248in}}{\pgfqpoint{6.377805in}{4.074366in}}%
\pgfpathcurveto{\pgfqpoint{6.381923in}{4.078484in}}{\pgfqpoint{6.384237in}{4.084070in}}{\pgfqpoint{6.384237in}{4.089894in}}%
\pgfpathcurveto{\pgfqpoint{6.384237in}{4.095718in}}{\pgfqpoint{6.381923in}{4.101304in}}{\pgfqpoint{6.377805in}{4.105422in}}%
\pgfpathcurveto{\pgfqpoint{6.373687in}{4.109541in}}{\pgfqpoint{6.368101in}{4.111855in}}{\pgfqpoint{6.362277in}{4.111855in}}%
\pgfpathcurveto{\pgfqpoint{6.356453in}{4.111855in}}{\pgfqpoint{6.350867in}{4.109541in}}{\pgfqpoint{6.346749in}{4.105422in}}%
\pgfpathcurveto{\pgfqpoint{6.342631in}{4.101304in}}{\pgfqpoint{6.340317in}{4.095718in}}{\pgfqpoint{6.340317in}{4.089894in}}%
\pgfpathcurveto{\pgfqpoint{6.340317in}{4.084070in}}{\pgfqpoint{6.342631in}{4.078484in}}{\pgfqpoint{6.346749in}{4.074366in}}%
\pgfpathcurveto{\pgfqpoint{6.350867in}{4.070248in}}{\pgfqpoint{6.356453in}{4.067934in}}{\pgfqpoint{6.362277in}{4.067934in}}%
\pgfpathlineto{\pgfqpoint{6.362277in}{4.067934in}}%
\pgfpathclose%
\pgfusepath{stroke,fill}%
\end{pgfscope}%
\begin{pgfscope}%
\pgfpathrectangle{\pgfqpoint{1.073501in}{0.880000in}}{\pgfqpoint{6.052998in}{6.160000in}}%
\pgfusepath{clip}%
\pgfsetbuttcap%
\pgfsetroundjoin%
\definecolor{currentfill}{rgb}{0.800000,0.200000,0.200000}%
\pgfsetfillcolor{currentfill}%
\pgfsetlinewidth{1.003750pt}%
\definecolor{currentstroke}{rgb}{0.800000,0.200000,0.200000}%
\pgfsetstrokecolor{currentstroke}%
\pgfsetdash{}{0pt}%
\pgfpathmoveto{\pgfqpoint{6.435097in}{4.220524in}}%
\pgfpathcurveto{\pgfqpoint{6.440921in}{4.220524in}}{\pgfqpoint{6.446507in}{4.222838in}}{\pgfqpoint{6.450625in}{4.226957in}}%
\pgfpathcurveto{\pgfqpoint{6.454743in}{4.231075in}}{\pgfqpoint{6.457057in}{4.236661in}}{\pgfqpoint{6.457057in}{4.242485in}}%
\pgfpathcurveto{\pgfqpoint{6.457057in}{4.248309in}}{\pgfqpoint{6.454743in}{4.253895in}}{\pgfqpoint{6.450625in}{4.258013in}}%
\pgfpathcurveto{\pgfqpoint{6.446507in}{4.262131in}}{\pgfqpoint{6.440921in}{4.264445in}}{\pgfqpoint{6.435097in}{4.264445in}}%
\pgfpathcurveto{\pgfqpoint{6.429273in}{4.264445in}}{\pgfqpoint{6.423687in}{4.262131in}}{\pgfqpoint{6.419569in}{4.258013in}}%
\pgfpathcurveto{\pgfqpoint{6.415450in}{4.253895in}}{\pgfqpoint{6.413137in}{4.248309in}}{\pgfqpoint{6.413137in}{4.242485in}}%
\pgfpathcurveto{\pgfqpoint{6.413137in}{4.236661in}}{\pgfqpoint{6.415450in}{4.231075in}}{\pgfqpoint{6.419569in}{4.226957in}}%
\pgfpathcurveto{\pgfqpoint{6.423687in}{4.222838in}}{\pgfqpoint{6.429273in}{4.220524in}}{\pgfqpoint{6.435097in}{4.220524in}}%
\pgfpathlineto{\pgfqpoint{6.435097in}{4.220524in}}%
\pgfpathclose%
\pgfusepath{stroke,fill}%
\end{pgfscope}%
\begin{pgfscope}%
\pgfpathrectangle{\pgfqpoint{1.073501in}{0.880000in}}{\pgfqpoint{6.052998in}{6.160000in}}%
\pgfusepath{clip}%
\pgfsetbuttcap%
\pgfsetroundjoin%
\definecolor{currentfill}{rgb}{0.800000,0.200000,0.200000}%
\pgfsetfillcolor{currentfill}%
\pgfsetlinewidth{1.003750pt}%
\definecolor{currentstroke}{rgb}{0.800000,0.200000,0.200000}%
\pgfsetstrokecolor{currentstroke}%
\pgfsetdash{}{0pt}%
\pgfpathmoveto{\pgfqpoint{6.257562in}{4.336690in}}%
\pgfpathcurveto{\pgfqpoint{6.263386in}{4.336690in}}{\pgfqpoint{6.268973in}{4.339004in}}{\pgfqpoint{6.273091in}{4.343122in}}%
\pgfpathcurveto{\pgfqpoint{6.277209in}{4.347240in}}{\pgfqpoint{6.279523in}{4.352826in}}{\pgfqpoint{6.279523in}{4.358650in}}%
\pgfpathcurveto{\pgfqpoint{6.279523in}{4.364474in}}{\pgfqpoint{6.277209in}{4.370060in}}{\pgfqpoint{6.273091in}{4.374178in}}%
\pgfpathcurveto{\pgfqpoint{6.268973in}{4.378297in}}{\pgfqpoint{6.263386in}{4.380610in}}{\pgfqpoint{6.257562in}{4.380610in}}%
\pgfpathcurveto{\pgfqpoint{6.251739in}{4.380610in}}{\pgfqpoint{6.246152in}{4.378297in}}{\pgfqpoint{6.242034in}{4.374178in}}%
\pgfpathcurveto{\pgfqpoint{6.237916in}{4.370060in}}{\pgfqpoint{6.235602in}{4.364474in}}{\pgfqpoint{6.235602in}{4.358650in}}%
\pgfpathcurveto{\pgfqpoint{6.235602in}{4.352826in}}{\pgfqpoint{6.237916in}{4.347240in}}{\pgfqpoint{6.242034in}{4.343122in}}%
\pgfpathcurveto{\pgfqpoint{6.246152in}{4.339004in}}{\pgfqpoint{6.251739in}{4.336690in}}{\pgfqpoint{6.257562in}{4.336690in}}%
\pgfpathlineto{\pgfqpoint{6.257562in}{4.336690in}}%
\pgfpathclose%
\pgfusepath{stroke,fill}%
\end{pgfscope}%
\begin{pgfscope}%
\pgfpathrectangle{\pgfqpoint{1.073501in}{0.880000in}}{\pgfqpoint{6.052998in}{6.160000in}}%
\pgfusepath{clip}%
\pgfsetbuttcap%
\pgfsetroundjoin%
\definecolor{currentfill}{rgb}{0.800000,0.200000,0.200000}%
\pgfsetfillcolor{currentfill}%
\pgfsetlinewidth{1.003750pt}%
\definecolor{currentstroke}{rgb}{0.800000,0.200000,0.200000}%
\pgfsetstrokecolor{currentstroke}%
\pgfsetdash{}{0pt}%
\pgfpathmoveto{\pgfqpoint{6.316931in}{4.494359in}}%
\pgfpathcurveto{\pgfqpoint{6.322755in}{4.494359in}}{\pgfqpoint{6.328341in}{4.496673in}}{\pgfqpoint{6.332459in}{4.500791in}}%
\pgfpathcurveto{\pgfqpoint{6.336577in}{4.504909in}}{\pgfqpoint{6.338891in}{4.510495in}}{\pgfqpoint{6.338891in}{4.516319in}}%
\pgfpathcurveto{\pgfqpoint{6.338891in}{4.522143in}}{\pgfqpoint{6.336577in}{4.527729in}}{\pgfqpoint{6.332459in}{4.531848in}}%
\pgfpathcurveto{\pgfqpoint{6.328341in}{4.535966in}}{\pgfqpoint{6.322755in}{4.538280in}}{\pgfqpoint{6.316931in}{4.538280in}}%
\pgfpathcurveto{\pgfqpoint{6.311107in}{4.538280in}}{\pgfqpoint{6.305521in}{4.535966in}}{\pgfqpoint{6.301403in}{4.531848in}}%
\pgfpathcurveto{\pgfqpoint{6.297284in}{4.527729in}}{\pgfqpoint{6.294970in}{4.522143in}}{\pgfqpoint{6.294970in}{4.516319in}}%
\pgfpathcurveto{\pgfqpoint{6.294970in}{4.510495in}}{\pgfqpoint{6.297284in}{4.504909in}}{\pgfqpoint{6.301403in}{4.500791in}}%
\pgfpathcurveto{\pgfqpoint{6.305521in}{4.496673in}}{\pgfqpoint{6.311107in}{4.494359in}}{\pgfqpoint{6.316931in}{4.494359in}}%
\pgfpathlineto{\pgfqpoint{6.316931in}{4.494359in}}%
\pgfpathclose%
\pgfusepath{stroke,fill}%
\end{pgfscope}%
\begin{pgfscope}%
\pgfpathrectangle{\pgfqpoint{1.073501in}{0.880000in}}{\pgfqpoint{6.052998in}{6.160000in}}%
\pgfusepath{clip}%
\pgfsetbuttcap%
\pgfsetroundjoin%
\definecolor{currentfill}{rgb}{0.800000,0.200000,0.200000}%
\pgfsetfillcolor{currentfill}%
\pgfsetlinewidth{1.003750pt}%
\definecolor{currentstroke}{rgb}{0.800000,0.200000,0.200000}%
\pgfsetstrokecolor{currentstroke}%
\pgfsetdash{}{0pt}%
\pgfpathmoveto{\pgfqpoint{6.185784in}{4.602416in}}%
\pgfpathcurveto{\pgfqpoint{6.191608in}{4.602416in}}{\pgfqpoint{6.197194in}{4.604730in}}{\pgfqpoint{6.201312in}{4.608848in}}%
\pgfpathcurveto{\pgfqpoint{6.205430in}{4.612966in}}{\pgfqpoint{6.207744in}{4.618553in}}{\pgfqpoint{6.207744in}{4.624377in}}%
\pgfpathcurveto{\pgfqpoint{6.207744in}{4.630200in}}{\pgfqpoint{6.205430in}{4.635787in}}{\pgfqpoint{6.201312in}{4.639905in}}%
\pgfpathcurveto{\pgfqpoint{6.197194in}{4.644023in}}{\pgfqpoint{6.191608in}{4.646337in}}{\pgfqpoint{6.185784in}{4.646337in}}%
\pgfpathcurveto{\pgfqpoint{6.179960in}{4.646337in}}{\pgfqpoint{6.174374in}{4.644023in}}{\pgfqpoint{6.170256in}{4.639905in}}%
\pgfpathcurveto{\pgfqpoint{6.166138in}{4.635787in}}{\pgfqpoint{6.163824in}{4.630200in}}{\pgfqpoint{6.163824in}{4.624377in}}%
\pgfpathcurveto{\pgfqpoint{6.163824in}{4.618553in}}{\pgfqpoint{6.166138in}{4.612966in}}{\pgfqpoint{6.170256in}{4.608848in}}%
\pgfpathcurveto{\pgfqpoint{6.174374in}{4.604730in}}{\pgfqpoint{6.179960in}{4.602416in}}{\pgfqpoint{6.185784in}{4.602416in}}%
\pgfpathlineto{\pgfqpoint{6.185784in}{4.602416in}}%
\pgfpathclose%
\pgfusepath{stroke,fill}%
\end{pgfscope}%
\begin{pgfscope}%
\pgfpathrectangle{\pgfqpoint{1.073501in}{0.880000in}}{\pgfqpoint{6.052998in}{6.160000in}}%
\pgfusepath{clip}%
\pgfsetbuttcap%
\pgfsetroundjoin%
\definecolor{currentfill}{rgb}{0.800000,0.200000,0.200000}%
\pgfsetfillcolor{currentfill}%
\pgfsetlinewidth{1.003750pt}%
\definecolor{currentstroke}{rgb}{0.800000,0.200000,0.200000}%
\pgfsetstrokecolor{currentstroke}%
\pgfsetdash{}{0pt}%
\pgfpathmoveto{\pgfqpoint{6.142225in}{4.733121in}}%
\pgfpathcurveto{\pgfqpoint{6.148049in}{4.733121in}}{\pgfqpoint{6.153635in}{4.735435in}}{\pgfqpoint{6.157753in}{4.739553in}}%
\pgfpathcurveto{\pgfqpoint{6.161871in}{4.743671in}}{\pgfqpoint{6.164185in}{4.749257in}}{\pgfqpoint{6.164185in}{4.755081in}}%
\pgfpathcurveto{\pgfqpoint{6.164185in}{4.760905in}}{\pgfqpoint{6.161871in}{4.766491in}}{\pgfqpoint{6.157753in}{4.770609in}}%
\pgfpathcurveto{\pgfqpoint{6.153635in}{4.774727in}}{\pgfqpoint{6.148049in}{4.777041in}}{\pgfqpoint{6.142225in}{4.777041in}}%
\pgfpathcurveto{\pgfqpoint{6.136401in}{4.777041in}}{\pgfqpoint{6.130815in}{4.774727in}}{\pgfqpoint{6.126696in}{4.770609in}}%
\pgfpathcurveto{\pgfqpoint{6.122578in}{4.766491in}}{\pgfqpoint{6.120264in}{4.760905in}}{\pgfqpoint{6.120264in}{4.755081in}}%
\pgfpathcurveto{\pgfqpoint{6.120264in}{4.749257in}}{\pgfqpoint{6.122578in}{4.743671in}}{\pgfqpoint{6.126696in}{4.739553in}}%
\pgfpathcurveto{\pgfqpoint{6.130815in}{4.735435in}}{\pgfqpoint{6.136401in}{4.733121in}}{\pgfqpoint{6.142225in}{4.733121in}}%
\pgfpathlineto{\pgfqpoint{6.142225in}{4.733121in}}%
\pgfpathclose%
\pgfusepath{stroke,fill}%
\end{pgfscope}%
\begin{pgfscope}%
\pgfpathrectangle{\pgfqpoint{1.073501in}{0.880000in}}{\pgfqpoint{6.052998in}{6.160000in}}%
\pgfusepath{clip}%
\pgfsetbuttcap%
\pgfsetroundjoin%
\definecolor{currentfill}{rgb}{0.800000,0.200000,0.200000}%
\pgfsetfillcolor{currentfill}%
\pgfsetlinewidth{1.003750pt}%
\definecolor{currentstroke}{rgb}{0.800000,0.200000,0.200000}%
\pgfsetstrokecolor{currentstroke}%
\pgfsetdash{}{0pt}%
\pgfpathmoveto{\pgfqpoint{6.128358in}{4.879102in}}%
\pgfpathcurveto{\pgfqpoint{6.134182in}{4.879102in}}{\pgfqpoint{6.139768in}{4.881416in}}{\pgfqpoint{6.143887in}{4.885534in}}%
\pgfpathcurveto{\pgfqpoint{6.148005in}{4.889653in}}{\pgfqpoint{6.150319in}{4.895239in}}{\pgfqpoint{6.150319in}{4.901063in}}%
\pgfpathcurveto{\pgfqpoint{6.150319in}{4.906887in}}{\pgfqpoint{6.148005in}{4.912473in}}{\pgfqpoint{6.143887in}{4.916591in}}%
\pgfpathcurveto{\pgfqpoint{6.139768in}{4.920709in}}{\pgfqpoint{6.134182in}{4.923023in}}{\pgfqpoint{6.128358in}{4.923023in}}%
\pgfpathcurveto{\pgfqpoint{6.122534in}{4.923023in}}{\pgfqpoint{6.116948in}{4.920709in}}{\pgfqpoint{6.112830in}{4.916591in}}%
\pgfpathcurveto{\pgfqpoint{6.108712in}{4.912473in}}{\pgfqpoint{6.106398in}{4.906887in}}{\pgfqpoint{6.106398in}{4.901063in}}%
\pgfpathcurveto{\pgfqpoint{6.106398in}{4.895239in}}{\pgfqpoint{6.108712in}{4.889653in}}{\pgfqpoint{6.112830in}{4.885534in}}%
\pgfpathcurveto{\pgfqpoint{6.116948in}{4.881416in}}{\pgfqpoint{6.122534in}{4.879102in}}{\pgfqpoint{6.128358in}{4.879102in}}%
\pgfpathlineto{\pgfqpoint{6.128358in}{4.879102in}}%
\pgfpathclose%
\pgfusepath{stroke,fill}%
\end{pgfscope}%
\begin{pgfscope}%
\pgfpathrectangle{\pgfqpoint{1.073501in}{0.880000in}}{\pgfqpoint{6.052998in}{6.160000in}}%
\pgfusepath{clip}%
\pgfsetbuttcap%
\pgfsetroundjoin%
\definecolor{currentfill}{rgb}{0.800000,0.200000,0.200000}%
\pgfsetfillcolor{currentfill}%
\pgfsetlinewidth{1.003750pt}%
\definecolor{currentstroke}{rgb}{0.800000,0.200000,0.200000}%
\pgfsetstrokecolor{currentstroke}%
\pgfsetdash{}{0pt}%
\pgfpathmoveto{\pgfqpoint{6.194080in}{5.076656in}}%
\pgfpathcurveto{\pgfqpoint{6.199904in}{5.076656in}}{\pgfqpoint{6.205490in}{5.078970in}}{\pgfqpoint{6.209608in}{5.083088in}}%
\pgfpathcurveto{\pgfqpoint{6.213726in}{5.087206in}}{\pgfqpoint{6.216040in}{5.092792in}}{\pgfqpoint{6.216040in}{5.098616in}}%
\pgfpathcurveto{\pgfqpoint{6.216040in}{5.104440in}}{\pgfqpoint{6.213726in}{5.110026in}}{\pgfqpoint{6.209608in}{5.114145in}}%
\pgfpathcurveto{\pgfqpoint{6.205490in}{5.118263in}}{\pgfqpoint{6.199904in}{5.120577in}}{\pgfqpoint{6.194080in}{5.120577in}}%
\pgfpathcurveto{\pgfqpoint{6.188256in}{5.120577in}}{\pgfqpoint{6.182670in}{5.118263in}}{\pgfqpoint{6.178552in}{5.114145in}}%
\pgfpathcurveto{\pgfqpoint{6.174434in}{5.110026in}}{\pgfqpoint{6.172120in}{5.104440in}}{\pgfqpoint{6.172120in}{5.098616in}}%
\pgfpathcurveto{\pgfqpoint{6.172120in}{5.092792in}}{\pgfqpoint{6.174434in}{5.087206in}}{\pgfqpoint{6.178552in}{5.083088in}}%
\pgfpathcurveto{\pgfqpoint{6.182670in}{5.078970in}}{\pgfqpoint{6.188256in}{5.076656in}}{\pgfqpoint{6.194080in}{5.076656in}}%
\pgfpathlineto{\pgfqpoint{6.194080in}{5.076656in}}%
\pgfpathclose%
\pgfusepath{stroke,fill}%
\end{pgfscope}%
\begin{pgfscope}%
\pgfpathrectangle{\pgfqpoint{1.073501in}{0.880000in}}{\pgfqpoint{6.052998in}{6.160000in}}%
\pgfusepath{clip}%
\pgfsetbuttcap%
\pgfsetroundjoin%
\definecolor{currentfill}{rgb}{0.800000,0.200000,0.200000}%
\pgfsetfillcolor{currentfill}%
\pgfsetlinewidth{1.003750pt}%
\definecolor{currentstroke}{rgb}{0.800000,0.200000,0.200000}%
\pgfsetstrokecolor{currentstroke}%
\pgfsetdash{}{0pt}%
\pgfpathmoveto{\pgfqpoint{5.985723in}{5.121213in}}%
\pgfpathcurveto{\pgfqpoint{5.991547in}{5.121213in}}{\pgfqpoint{5.997134in}{5.123526in}}{\pgfqpoint{6.001252in}{5.127645in}}%
\pgfpathcurveto{\pgfqpoint{6.005370in}{5.131763in}}{\pgfqpoint{6.007684in}{5.137349in}}{\pgfqpoint{6.007684in}{5.143173in}}%
\pgfpathcurveto{\pgfqpoint{6.007684in}{5.148997in}}{\pgfqpoint{6.005370in}{5.154583in}}{\pgfqpoint{6.001252in}{5.158701in}}%
\pgfpathcurveto{\pgfqpoint{5.997134in}{5.162819in}}{\pgfqpoint{5.991547in}{5.165133in}}{\pgfqpoint{5.985723in}{5.165133in}}%
\pgfpathcurveto{\pgfqpoint{5.979899in}{5.165133in}}{\pgfqpoint{5.974313in}{5.162819in}}{\pgfqpoint{5.970195in}{5.158701in}}%
\pgfpathcurveto{\pgfqpoint{5.966077in}{5.154583in}}{\pgfqpoint{5.963763in}{5.148997in}}{\pgfqpoint{5.963763in}{5.143173in}}%
\pgfpathcurveto{\pgfqpoint{5.963763in}{5.137349in}}{\pgfqpoint{5.966077in}{5.131763in}}{\pgfqpoint{5.970195in}{5.127645in}}%
\pgfpathcurveto{\pgfqpoint{5.974313in}{5.123526in}}{\pgfqpoint{5.979899in}{5.121213in}}{\pgfqpoint{5.985723in}{5.121213in}}%
\pgfpathlineto{\pgfqpoint{5.985723in}{5.121213in}}%
\pgfpathclose%
\pgfusepath{stroke,fill}%
\end{pgfscope}%
\begin{pgfscope}%
\pgfpathrectangle{\pgfqpoint{1.073501in}{0.880000in}}{\pgfqpoint{6.052998in}{6.160000in}}%
\pgfusepath{clip}%
\pgfsetbuttcap%
\pgfsetroundjoin%
\definecolor{currentfill}{rgb}{0.800000,0.200000,0.200000}%
\pgfsetfillcolor{currentfill}%
\pgfsetlinewidth{1.003750pt}%
\definecolor{currentstroke}{rgb}{0.800000,0.200000,0.200000}%
\pgfsetstrokecolor{currentstroke}%
\pgfsetdash{}{0pt}%
\pgfpathmoveto{\pgfqpoint{5.877255in}{5.215499in}}%
\pgfpathcurveto{\pgfqpoint{5.883079in}{5.215499in}}{\pgfqpoint{5.888665in}{5.217813in}}{\pgfqpoint{5.892783in}{5.221931in}}%
\pgfpathcurveto{\pgfqpoint{5.896901in}{5.226049in}}{\pgfqpoint{5.899215in}{5.231636in}}{\pgfqpoint{5.899215in}{5.237459in}}%
\pgfpathcurveto{\pgfqpoint{5.899215in}{5.243283in}}{\pgfqpoint{5.896901in}{5.248870in}}{\pgfqpoint{5.892783in}{5.252988in}}%
\pgfpathcurveto{\pgfqpoint{5.888665in}{5.257106in}}{\pgfqpoint{5.883079in}{5.259420in}}{\pgfqpoint{5.877255in}{5.259420in}}%
\pgfpathcurveto{\pgfqpoint{5.871431in}{5.259420in}}{\pgfqpoint{5.865845in}{5.257106in}}{\pgfqpoint{5.861727in}{5.252988in}}%
\pgfpathcurveto{\pgfqpoint{5.857609in}{5.248870in}}{\pgfqpoint{5.855295in}{5.243283in}}{\pgfqpoint{5.855295in}{5.237459in}}%
\pgfpathcurveto{\pgfqpoint{5.855295in}{5.231636in}}{\pgfqpoint{5.857609in}{5.226049in}}{\pgfqpoint{5.861727in}{5.221931in}}%
\pgfpathcurveto{\pgfqpoint{5.865845in}{5.217813in}}{\pgfqpoint{5.871431in}{5.215499in}}{\pgfqpoint{5.877255in}{5.215499in}}%
\pgfpathlineto{\pgfqpoint{5.877255in}{5.215499in}}%
\pgfpathclose%
\pgfusepath{stroke,fill}%
\end{pgfscope}%
\begin{pgfscope}%
\pgfpathrectangle{\pgfqpoint{1.073501in}{0.880000in}}{\pgfqpoint{6.052998in}{6.160000in}}%
\pgfusepath{clip}%
\pgfsetbuttcap%
\pgfsetroundjoin%
\definecolor{currentfill}{rgb}{0.800000,0.200000,0.200000}%
\pgfsetfillcolor{currentfill}%
\pgfsetlinewidth{1.003750pt}%
\definecolor{currentstroke}{rgb}{0.800000,0.200000,0.200000}%
\pgfsetstrokecolor{currentstroke}%
\pgfsetdash{}{0pt}%
\pgfpathmoveto{\pgfqpoint{5.831195in}{5.356947in}}%
\pgfpathcurveto{\pgfqpoint{5.837019in}{5.356947in}}{\pgfqpoint{5.842606in}{5.359260in}}{\pgfqpoint{5.846724in}{5.363379in}}%
\pgfpathcurveto{\pgfqpoint{5.850842in}{5.367497in}}{\pgfqpoint{5.853156in}{5.373083in}}{\pgfqpoint{5.853156in}{5.378907in}}%
\pgfpathcurveto{\pgfqpoint{5.853156in}{5.384731in}}{\pgfqpoint{5.850842in}{5.390317in}}{\pgfqpoint{5.846724in}{5.394435in}}%
\pgfpathcurveto{\pgfqpoint{5.842606in}{5.398553in}}{\pgfqpoint{5.837019in}{5.400867in}}{\pgfqpoint{5.831195in}{5.400867in}}%
\pgfpathcurveto{\pgfqpoint{5.825372in}{5.400867in}}{\pgfqpoint{5.819785in}{5.398553in}}{\pgfqpoint{5.815667in}{5.394435in}}%
\pgfpathcurveto{\pgfqpoint{5.811549in}{5.390317in}}{\pgfqpoint{5.809235in}{5.384731in}}{\pgfqpoint{5.809235in}{5.378907in}}%
\pgfpathcurveto{\pgfqpoint{5.809235in}{5.373083in}}{\pgfqpoint{5.811549in}{5.367497in}}{\pgfqpoint{5.815667in}{5.363379in}}%
\pgfpathcurveto{\pgfqpoint{5.819785in}{5.359260in}}{\pgfqpoint{5.825372in}{5.356947in}}{\pgfqpoint{5.831195in}{5.356947in}}%
\pgfpathlineto{\pgfqpoint{5.831195in}{5.356947in}}%
\pgfpathclose%
\pgfusepath{stroke,fill}%
\end{pgfscope}%
\begin{pgfscope}%
\pgfpathrectangle{\pgfqpoint{1.073501in}{0.880000in}}{\pgfqpoint{6.052998in}{6.160000in}}%
\pgfusepath{clip}%
\pgfsetbuttcap%
\pgfsetroundjoin%
\definecolor{currentfill}{rgb}{0.800000,0.200000,0.200000}%
\pgfsetfillcolor{currentfill}%
\pgfsetlinewidth{1.003750pt}%
\definecolor{currentstroke}{rgb}{0.800000,0.200000,0.200000}%
\pgfsetstrokecolor{currentstroke}%
\pgfsetdash{}{0pt}%
\pgfpathmoveto{\pgfqpoint{5.737664in}{5.462881in}}%
\pgfpathcurveto{\pgfqpoint{5.743488in}{5.462881in}}{\pgfqpoint{5.749074in}{5.465195in}}{\pgfqpoint{5.753192in}{5.469313in}}%
\pgfpathcurveto{\pgfqpoint{5.757310in}{5.473431in}}{\pgfqpoint{5.759624in}{5.479017in}}{\pgfqpoint{5.759624in}{5.484841in}}%
\pgfpathcurveto{\pgfqpoint{5.759624in}{5.490665in}}{\pgfqpoint{5.757310in}{5.496251in}}{\pgfqpoint{5.753192in}{5.500369in}}%
\pgfpathcurveto{\pgfqpoint{5.749074in}{5.504487in}}{\pgfqpoint{5.743488in}{5.506801in}}{\pgfqpoint{5.737664in}{5.506801in}}%
\pgfpathcurveto{\pgfqpoint{5.731840in}{5.506801in}}{\pgfqpoint{5.726254in}{5.504487in}}{\pgfqpoint{5.722135in}{5.500369in}}%
\pgfpathcurveto{\pgfqpoint{5.718017in}{5.496251in}}{\pgfqpoint{5.715703in}{5.490665in}}{\pgfqpoint{5.715703in}{5.484841in}}%
\pgfpathcurveto{\pgfqpoint{5.715703in}{5.479017in}}{\pgfqpoint{5.718017in}{5.473431in}}{\pgfqpoint{5.722135in}{5.469313in}}%
\pgfpathcurveto{\pgfqpoint{5.726254in}{5.465195in}}{\pgfqpoint{5.731840in}{5.462881in}}{\pgfqpoint{5.737664in}{5.462881in}}%
\pgfpathlineto{\pgfqpoint{5.737664in}{5.462881in}}%
\pgfpathclose%
\pgfusepath{stroke,fill}%
\end{pgfscope}%
\begin{pgfscope}%
\pgfpathrectangle{\pgfqpoint{1.073501in}{0.880000in}}{\pgfqpoint{6.052998in}{6.160000in}}%
\pgfusepath{clip}%
\pgfsetbuttcap%
\pgfsetroundjoin%
\definecolor{currentfill}{rgb}{0.800000,0.200000,0.200000}%
\pgfsetfillcolor{currentfill}%
\pgfsetlinewidth{1.003750pt}%
\definecolor{currentstroke}{rgb}{0.800000,0.200000,0.200000}%
\pgfsetstrokecolor{currentstroke}%
\pgfsetdash{}{0pt}%
\pgfpathmoveto{\pgfqpoint{5.648270in}{5.574306in}}%
\pgfpathcurveto{\pgfqpoint{5.654094in}{5.574306in}}{\pgfqpoint{5.659680in}{5.576620in}}{\pgfqpoint{5.663798in}{5.580738in}}%
\pgfpathcurveto{\pgfqpoint{5.667916in}{5.584856in}}{\pgfqpoint{5.670230in}{5.590443in}}{\pgfqpoint{5.670230in}{5.596266in}}%
\pgfpathcurveto{\pgfqpoint{5.670230in}{5.602090in}}{\pgfqpoint{5.667916in}{5.607677in}}{\pgfqpoint{5.663798in}{5.611795in}}%
\pgfpathcurveto{\pgfqpoint{5.659680in}{5.615913in}}{\pgfqpoint{5.654094in}{5.618227in}}{\pgfqpoint{5.648270in}{5.618227in}}%
\pgfpathcurveto{\pgfqpoint{5.642446in}{5.618227in}}{\pgfqpoint{5.636860in}{5.615913in}}{\pgfqpoint{5.632742in}{5.611795in}}%
\pgfpathcurveto{\pgfqpoint{5.628623in}{5.607677in}}{\pgfqpoint{5.626310in}{5.602090in}}{\pgfqpoint{5.626310in}{5.596266in}}%
\pgfpathcurveto{\pgfqpoint{5.626310in}{5.590443in}}{\pgfqpoint{5.628623in}{5.584856in}}{\pgfqpoint{5.632742in}{5.580738in}}%
\pgfpathcurveto{\pgfqpoint{5.636860in}{5.576620in}}{\pgfqpoint{5.642446in}{5.574306in}}{\pgfqpoint{5.648270in}{5.574306in}}%
\pgfpathlineto{\pgfqpoint{5.648270in}{5.574306in}}%
\pgfpathclose%
\pgfusepath{stroke,fill}%
\end{pgfscope}%
\begin{pgfscope}%
\pgfpathrectangle{\pgfqpoint{1.073501in}{0.880000in}}{\pgfqpoint{6.052998in}{6.160000in}}%
\pgfusepath{clip}%
\pgfsetbuttcap%
\pgfsetroundjoin%
\definecolor{currentfill}{rgb}{0.800000,0.200000,0.200000}%
\pgfsetfillcolor{currentfill}%
\pgfsetlinewidth{1.003750pt}%
\definecolor{currentstroke}{rgb}{0.800000,0.200000,0.200000}%
\pgfsetstrokecolor{currentstroke}%
\pgfsetdash{}{0pt}%
\pgfpathmoveto{\pgfqpoint{5.524878in}{5.648148in}}%
\pgfpathcurveto{\pgfqpoint{5.530702in}{5.648148in}}{\pgfqpoint{5.536288in}{5.650462in}}{\pgfqpoint{5.540406in}{5.654580in}}%
\pgfpathcurveto{\pgfqpoint{5.544524in}{5.658698in}}{\pgfqpoint{5.546838in}{5.664284in}}{\pgfqpoint{5.546838in}{5.670108in}}%
\pgfpathcurveto{\pgfqpoint{5.546838in}{5.675932in}}{\pgfqpoint{5.544524in}{5.681518in}}{\pgfqpoint{5.540406in}{5.685636in}}%
\pgfpathcurveto{\pgfqpoint{5.536288in}{5.689754in}}{\pgfqpoint{5.530702in}{5.692068in}}{\pgfqpoint{5.524878in}{5.692068in}}%
\pgfpathcurveto{\pgfqpoint{5.519054in}{5.692068in}}{\pgfqpoint{5.513468in}{5.689754in}}{\pgfqpoint{5.509349in}{5.685636in}}%
\pgfpathcurveto{\pgfqpoint{5.505231in}{5.681518in}}{\pgfqpoint{5.502917in}{5.675932in}}{\pgfqpoint{5.502917in}{5.670108in}}%
\pgfpathcurveto{\pgfqpoint{5.502917in}{5.664284in}}{\pgfqpoint{5.505231in}{5.658698in}}{\pgfqpoint{5.509349in}{5.654580in}}%
\pgfpathcurveto{\pgfqpoint{5.513468in}{5.650462in}}{\pgfqpoint{5.519054in}{5.648148in}}{\pgfqpoint{5.524878in}{5.648148in}}%
\pgfpathlineto{\pgfqpoint{5.524878in}{5.648148in}}%
\pgfpathclose%
\pgfusepath{stroke,fill}%
\end{pgfscope}%
\begin{pgfscope}%
\pgfpathrectangle{\pgfqpoint{1.073501in}{0.880000in}}{\pgfqpoint{6.052998in}{6.160000in}}%
\pgfusepath{clip}%
\pgfsetbuttcap%
\pgfsetroundjoin%
\definecolor{currentfill}{rgb}{0.800000,0.200000,0.200000}%
\pgfsetfillcolor{currentfill}%
\pgfsetlinewidth{1.003750pt}%
\definecolor{currentstroke}{rgb}{0.800000,0.200000,0.200000}%
\pgfsetstrokecolor{currentstroke}%
\pgfsetdash{}{0pt}%
\pgfpathmoveto{\pgfqpoint{5.416366in}{5.738404in}}%
\pgfpathcurveto{\pgfqpoint{5.422189in}{5.738404in}}{\pgfqpoint{5.427776in}{5.740717in}}{\pgfqpoint{5.431894in}{5.744836in}}%
\pgfpathcurveto{\pgfqpoint{5.436012in}{5.748954in}}{\pgfqpoint{5.438326in}{5.754540in}}{\pgfqpoint{5.438326in}{5.760364in}}%
\pgfpathcurveto{\pgfqpoint{5.438326in}{5.766188in}}{\pgfqpoint{5.436012in}{5.771774in}}{\pgfqpoint{5.431894in}{5.775892in}}%
\pgfpathcurveto{\pgfqpoint{5.427776in}{5.780010in}}{\pgfqpoint{5.422189in}{5.782324in}}{\pgfqpoint{5.416366in}{5.782324in}}%
\pgfpathcurveto{\pgfqpoint{5.410542in}{5.782324in}}{\pgfqpoint{5.404955in}{5.780010in}}{\pgfqpoint{5.400837in}{5.775892in}}%
\pgfpathcurveto{\pgfqpoint{5.396719in}{5.771774in}}{\pgfqpoint{5.394405in}{5.766188in}}{\pgfqpoint{5.394405in}{5.760364in}}%
\pgfpathcurveto{\pgfqpoint{5.394405in}{5.754540in}}{\pgfqpoint{5.396719in}{5.748954in}}{\pgfqpoint{5.400837in}{5.744836in}}%
\pgfpathcurveto{\pgfqpoint{5.404955in}{5.740717in}}{\pgfqpoint{5.410542in}{5.738404in}}{\pgfqpoint{5.416366in}{5.738404in}}%
\pgfpathlineto{\pgfqpoint{5.416366in}{5.738404in}}%
\pgfpathclose%
\pgfusepath{stroke,fill}%
\end{pgfscope}%
\begin{pgfscope}%
\pgfpathrectangle{\pgfqpoint{1.073501in}{0.880000in}}{\pgfqpoint{6.052998in}{6.160000in}}%
\pgfusepath{clip}%
\pgfsetbuttcap%
\pgfsetroundjoin%
\definecolor{currentfill}{rgb}{0.800000,0.200000,0.200000}%
\pgfsetfillcolor{currentfill}%
\pgfsetlinewidth{1.003750pt}%
\definecolor{currentstroke}{rgb}{0.800000,0.200000,0.200000}%
\pgfsetstrokecolor{currentstroke}%
\pgfsetdash{}{0pt}%
\pgfpathmoveto{\pgfqpoint{5.329227in}{5.865653in}}%
\pgfpathcurveto{\pgfqpoint{5.335051in}{5.865653in}}{\pgfqpoint{5.340637in}{5.867966in}}{\pgfqpoint{5.344756in}{5.872085in}}%
\pgfpathcurveto{\pgfqpoint{5.348874in}{5.876203in}}{\pgfqpoint{5.351188in}{5.881789in}}{\pgfqpoint{5.351188in}{5.887613in}}%
\pgfpathcurveto{\pgfqpoint{5.351188in}{5.893437in}}{\pgfqpoint{5.348874in}{5.899023in}}{\pgfqpoint{5.344756in}{5.903141in}}%
\pgfpathcurveto{\pgfqpoint{5.340637in}{5.907259in}}{\pgfqpoint{5.335051in}{5.909573in}}{\pgfqpoint{5.329227in}{5.909573in}}%
\pgfpathcurveto{\pgfqpoint{5.323403in}{5.909573in}}{\pgfqpoint{5.317817in}{5.907259in}}{\pgfqpoint{5.313699in}{5.903141in}}%
\pgfpathcurveto{\pgfqpoint{5.309581in}{5.899023in}}{\pgfqpoint{5.307267in}{5.893437in}}{\pgfqpoint{5.307267in}{5.887613in}}%
\pgfpathcurveto{\pgfqpoint{5.307267in}{5.881789in}}{\pgfqpoint{5.309581in}{5.876203in}}{\pgfqpoint{5.313699in}{5.872085in}}%
\pgfpathcurveto{\pgfqpoint{5.317817in}{5.867966in}}{\pgfqpoint{5.323403in}{5.865653in}}{\pgfqpoint{5.329227in}{5.865653in}}%
\pgfpathlineto{\pgfqpoint{5.329227in}{5.865653in}}%
\pgfpathclose%
\pgfusepath{stroke,fill}%
\end{pgfscope}%
\begin{pgfscope}%
\pgfpathrectangle{\pgfqpoint{1.073501in}{0.880000in}}{\pgfqpoint{6.052998in}{6.160000in}}%
\pgfusepath{clip}%
\pgfsetbuttcap%
\pgfsetroundjoin%
\definecolor{currentfill}{rgb}{0.800000,0.200000,0.200000}%
\pgfsetfillcolor{currentfill}%
\pgfsetlinewidth{1.003750pt}%
\definecolor{currentstroke}{rgb}{0.800000,0.200000,0.200000}%
\pgfsetstrokecolor{currentstroke}%
\pgfsetdash{}{0pt}%
\pgfpathmoveto{\pgfqpoint{5.191590in}{5.915352in}}%
\pgfpathcurveto{\pgfqpoint{5.197414in}{5.915352in}}{\pgfqpoint{5.203001in}{5.917666in}}{\pgfqpoint{5.207119in}{5.921784in}}%
\pgfpathcurveto{\pgfqpoint{5.211237in}{5.925902in}}{\pgfqpoint{5.213551in}{5.931489in}}{\pgfqpoint{5.213551in}{5.937313in}}%
\pgfpathcurveto{\pgfqpoint{5.213551in}{5.943136in}}{\pgfqpoint{5.211237in}{5.948723in}}{\pgfqpoint{5.207119in}{5.952841in}}%
\pgfpathcurveto{\pgfqpoint{5.203001in}{5.956959in}}{\pgfqpoint{5.197414in}{5.959273in}}{\pgfqpoint{5.191590in}{5.959273in}}%
\pgfpathcurveto{\pgfqpoint{5.185766in}{5.959273in}}{\pgfqpoint{5.180180in}{5.956959in}}{\pgfqpoint{5.176062in}{5.952841in}}%
\pgfpathcurveto{\pgfqpoint{5.171944in}{5.948723in}}{\pgfqpoint{5.169630in}{5.943136in}}{\pgfqpoint{5.169630in}{5.937313in}}%
\pgfpathcurveto{\pgfqpoint{5.169630in}{5.931489in}}{\pgfqpoint{5.171944in}{5.925902in}}{\pgfqpoint{5.176062in}{5.921784in}}%
\pgfpathcurveto{\pgfqpoint{5.180180in}{5.917666in}}{\pgfqpoint{5.185766in}{5.915352in}}{\pgfqpoint{5.191590in}{5.915352in}}%
\pgfpathlineto{\pgfqpoint{5.191590in}{5.915352in}}%
\pgfpathclose%
\pgfusepath{stroke,fill}%
\end{pgfscope}%
\begin{pgfscope}%
\pgfpathrectangle{\pgfqpoint{1.073501in}{0.880000in}}{\pgfqpoint{6.052998in}{6.160000in}}%
\pgfusepath{clip}%
\pgfsetbuttcap%
\pgfsetroundjoin%
\definecolor{currentfill}{rgb}{0.800000,0.200000,0.200000}%
\pgfsetfillcolor{currentfill}%
\pgfsetlinewidth{1.003750pt}%
\definecolor{currentstroke}{rgb}{0.800000,0.200000,0.200000}%
\pgfsetstrokecolor{currentstroke}%
\pgfsetdash{}{0pt}%
\pgfpathmoveto{\pgfqpoint{5.039194in}{5.926261in}}%
\pgfpathcurveto{\pgfqpoint{5.045018in}{5.926261in}}{\pgfqpoint{5.050604in}{5.928575in}}{\pgfqpoint{5.054722in}{5.932693in}}%
\pgfpathcurveto{\pgfqpoint{5.058840in}{5.936812in}}{\pgfqpoint{5.061154in}{5.942398in}}{\pgfqpoint{5.061154in}{5.948222in}}%
\pgfpathcurveto{\pgfqpoint{5.061154in}{5.954046in}}{\pgfqpoint{5.058840in}{5.959632in}}{\pgfqpoint{5.054722in}{5.963750in}}%
\pgfpathcurveto{\pgfqpoint{5.050604in}{5.967868in}}{\pgfqpoint{5.045018in}{5.970182in}}{\pgfqpoint{5.039194in}{5.970182in}}%
\pgfpathcurveto{\pgfqpoint{5.033370in}{5.970182in}}{\pgfqpoint{5.027784in}{5.967868in}}{\pgfqpoint{5.023666in}{5.963750in}}%
\pgfpathcurveto{\pgfqpoint{5.019547in}{5.959632in}}{\pgfqpoint{5.017234in}{5.954046in}}{\pgfqpoint{5.017234in}{5.948222in}}%
\pgfpathcurveto{\pgfqpoint{5.017234in}{5.942398in}}{\pgfqpoint{5.019547in}{5.936812in}}{\pgfqpoint{5.023666in}{5.932693in}}%
\pgfpathcurveto{\pgfqpoint{5.027784in}{5.928575in}}{\pgfqpoint{5.033370in}{5.926261in}}{\pgfqpoint{5.039194in}{5.926261in}}%
\pgfpathlineto{\pgfqpoint{5.039194in}{5.926261in}}%
\pgfpathclose%
\pgfusepath{stroke,fill}%
\end{pgfscope}%
\begin{pgfscope}%
\pgfpathrectangle{\pgfqpoint{1.073501in}{0.880000in}}{\pgfqpoint{6.052998in}{6.160000in}}%
\pgfusepath{clip}%
\pgfsetbuttcap%
\pgfsetroundjoin%
\definecolor{currentfill}{rgb}{0.800000,0.200000,0.200000}%
\pgfsetfillcolor{currentfill}%
\pgfsetlinewidth{1.003750pt}%
\definecolor{currentstroke}{rgb}{0.800000,0.200000,0.200000}%
\pgfsetstrokecolor{currentstroke}%
\pgfsetdash{}{0pt}%
\pgfpathmoveto{\pgfqpoint{4.944108in}{6.068207in}}%
\pgfpathcurveto{\pgfqpoint{4.949932in}{6.068207in}}{\pgfqpoint{4.955518in}{6.070521in}}{\pgfqpoint{4.959636in}{6.074639in}}%
\pgfpathcurveto{\pgfqpoint{4.963755in}{6.078757in}}{\pgfqpoint{4.966068in}{6.084343in}}{\pgfqpoint{4.966068in}{6.090167in}}%
\pgfpathcurveto{\pgfqpoint{4.966068in}{6.095991in}}{\pgfqpoint{4.963755in}{6.101577in}}{\pgfqpoint{4.959636in}{6.105695in}}%
\pgfpathcurveto{\pgfqpoint{4.955518in}{6.109814in}}{\pgfqpoint{4.949932in}{6.112127in}}{\pgfqpoint{4.944108in}{6.112127in}}%
\pgfpathcurveto{\pgfqpoint{4.938284in}{6.112127in}}{\pgfqpoint{4.932698in}{6.109814in}}{\pgfqpoint{4.928580in}{6.105695in}}%
\pgfpathcurveto{\pgfqpoint{4.924462in}{6.101577in}}{\pgfqpoint{4.922148in}{6.095991in}}{\pgfqpoint{4.922148in}{6.090167in}}%
\pgfpathcurveto{\pgfqpoint{4.922148in}{6.084343in}}{\pgfqpoint{4.924462in}{6.078757in}}{\pgfqpoint{4.928580in}{6.074639in}}%
\pgfpathcurveto{\pgfqpoint{4.932698in}{6.070521in}}{\pgfqpoint{4.938284in}{6.068207in}}{\pgfqpoint{4.944108in}{6.068207in}}%
\pgfpathlineto{\pgfqpoint{4.944108in}{6.068207in}}%
\pgfpathclose%
\pgfusepath{stroke,fill}%
\end{pgfscope}%
\begin{pgfscope}%
\pgfpathrectangle{\pgfqpoint{1.073501in}{0.880000in}}{\pgfqpoint{6.052998in}{6.160000in}}%
\pgfusepath{clip}%
\pgfsetbuttcap%
\pgfsetroundjoin%
\definecolor{currentfill}{rgb}{0.800000,0.200000,0.200000}%
\pgfsetfillcolor{currentfill}%
\pgfsetlinewidth{1.003750pt}%
\definecolor{currentstroke}{rgb}{0.800000,0.200000,0.200000}%
\pgfsetstrokecolor{currentstroke}%
\pgfsetdash{}{0pt}%
\pgfpathmoveto{\pgfqpoint{4.835578in}{6.209065in}}%
\pgfpathcurveto{\pgfqpoint{4.841402in}{6.209065in}}{\pgfqpoint{4.846988in}{6.211379in}}{\pgfqpoint{4.851106in}{6.215498in}}%
\pgfpathcurveto{\pgfqpoint{4.855225in}{6.219616in}}{\pgfqpoint{4.857538in}{6.225202in}}{\pgfqpoint{4.857538in}{6.231026in}}%
\pgfpathcurveto{\pgfqpoint{4.857538in}{6.236850in}}{\pgfqpoint{4.855225in}{6.242436in}}{\pgfqpoint{4.851106in}{6.246554in}}%
\pgfpathcurveto{\pgfqpoint{4.846988in}{6.250672in}}{\pgfqpoint{4.841402in}{6.252986in}}{\pgfqpoint{4.835578in}{6.252986in}}%
\pgfpathcurveto{\pgfqpoint{4.829754in}{6.252986in}}{\pgfqpoint{4.824168in}{6.250672in}}{\pgfqpoint{4.820050in}{6.246554in}}%
\pgfpathcurveto{\pgfqpoint{4.815932in}{6.242436in}}{\pgfqpoint{4.813618in}{6.236850in}}{\pgfqpoint{4.813618in}{6.231026in}}%
\pgfpathcurveto{\pgfqpoint{4.813618in}{6.225202in}}{\pgfqpoint{4.815932in}{6.219616in}}{\pgfqpoint{4.820050in}{6.215498in}}%
\pgfpathcurveto{\pgfqpoint{4.824168in}{6.211379in}}{\pgfqpoint{4.829754in}{6.209065in}}{\pgfqpoint{4.835578in}{6.209065in}}%
\pgfpathlineto{\pgfqpoint{4.835578in}{6.209065in}}%
\pgfpathclose%
\pgfusepath{stroke,fill}%
\end{pgfscope}%
\begin{pgfscope}%
\pgfpathrectangle{\pgfqpoint{1.073501in}{0.880000in}}{\pgfqpoint{6.052998in}{6.160000in}}%
\pgfusepath{clip}%
\pgfsetbuttcap%
\pgfsetroundjoin%
\definecolor{currentfill}{rgb}{0.800000,0.200000,0.200000}%
\pgfsetfillcolor{currentfill}%
\pgfsetlinewidth{1.003750pt}%
\definecolor{currentstroke}{rgb}{0.800000,0.200000,0.200000}%
\pgfsetstrokecolor{currentstroke}%
\pgfsetdash{}{0pt}%
\pgfpathmoveto{\pgfqpoint{4.664006in}{6.145091in}}%
\pgfpathcurveto{\pgfqpoint{4.669830in}{6.145091in}}{\pgfqpoint{4.675416in}{6.147405in}}{\pgfqpoint{4.679534in}{6.151523in}}%
\pgfpathcurveto{\pgfqpoint{4.683652in}{6.155641in}}{\pgfqpoint{4.685966in}{6.161227in}}{\pgfqpoint{4.685966in}{6.167051in}}%
\pgfpathcurveto{\pgfqpoint{4.685966in}{6.172875in}}{\pgfqpoint{4.683652in}{6.178461in}}{\pgfqpoint{4.679534in}{6.182579in}}%
\pgfpathcurveto{\pgfqpoint{4.675416in}{6.186698in}}{\pgfqpoint{4.669830in}{6.189011in}}{\pgfqpoint{4.664006in}{6.189011in}}%
\pgfpathcurveto{\pgfqpoint{4.658182in}{6.189011in}}{\pgfqpoint{4.652596in}{6.186698in}}{\pgfqpoint{4.648477in}{6.182579in}}%
\pgfpathcurveto{\pgfqpoint{4.644359in}{6.178461in}}{\pgfqpoint{4.642045in}{6.172875in}}{\pgfqpoint{4.642045in}{6.167051in}}%
\pgfpathcurveto{\pgfqpoint{4.642045in}{6.161227in}}{\pgfqpoint{4.644359in}{6.155641in}}{\pgfqpoint{4.648477in}{6.151523in}}%
\pgfpathcurveto{\pgfqpoint{4.652596in}{6.147405in}}{\pgfqpoint{4.658182in}{6.145091in}}{\pgfqpoint{4.664006in}{6.145091in}}%
\pgfpathlineto{\pgfqpoint{4.664006in}{6.145091in}}%
\pgfpathclose%
\pgfusepath{stroke,fill}%
\end{pgfscope}%
\begin{pgfscope}%
\pgfpathrectangle{\pgfqpoint{1.073501in}{0.880000in}}{\pgfqpoint{6.052998in}{6.160000in}}%
\pgfusepath{clip}%
\pgfsetbuttcap%
\pgfsetroundjoin%
\definecolor{currentfill}{rgb}{0.800000,0.200000,0.200000}%
\pgfsetfillcolor{currentfill}%
\pgfsetlinewidth{1.003750pt}%
\definecolor{currentstroke}{rgb}{0.800000,0.200000,0.200000}%
\pgfsetstrokecolor{currentstroke}%
\pgfsetdash{}{0pt}%
\pgfpathmoveto{\pgfqpoint{4.530110in}{6.219827in}}%
\pgfpathcurveto{\pgfqpoint{4.535934in}{6.219827in}}{\pgfqpoint{4.541520in}{6.222140in}}{\pgfqpoint{4.545638in}{6.226259in}}%
\pgfpathcurveto{\pgfqpoint{4.549756in}{6.230377in}}{\pgfqpoint{4.552070in}{6.235963in}}{\pgfqpoint{4.552070in}{6.241787in}}%
\pgfpathcurveto{\pgfqpoint{4.552070in}{6.247611in}}{\pgfqpoint{4.549756in}{6.253197in}}{\pgfqpoint{4.545638in}{6.257315in}}%
\pgfpathcurveto{\pgfqpoint{4.541520in}{6.261433in}}{\pgfqpoint{4.535934in}{6.263747in}}{\pgfqpoint{4.530110in}{6.263747in}}%
\pgfpathcurveto{\pgfqpoint{4.524286in}{6.263747in}}{\pgfqpoint{4.518699in}{6.261433in}}{\pgfqpoint{4.514581in}{6.257315in}}%
\pgfpathcurveto{\pgfqpoint{4.510463in}{6.253197in}}{\pgfqpoint{4.508149in}{6.247611in}}{\pgfqpoint{4.508149in}{6.241787in}}%
\pgfpathcurveto{\pgfqpoint{4.508149in}{6.235963in}}{\pgfqpoint{4.510463in}{6.230377in}}{\pgfqpoint{4.514581in}{6.226259in}}%
\pgfpathcurveto{\pgfqpoint{4.518699in}{6.222140in}}{\pgfqpoint{4.524286in}{6.219827in}}{\pgfqpoint{4.530110in}{6.219827in}}%
\pgfpathlineto{\pgfqpoint{4.530110in}{6.219827in}}%
\pgfpathclose%
\pgfusepath{stroke,fill}%
\end{pgfscope}%
\begin{pgfscope}%
\pgfpathrectangle{\pgfqpoint{1.073501in}{0.880000in}}{\pgfqpoint{6.052998in}{6.160000in}}%
\pgfusepath{clip}%
\pgfsetbuttcap%
\pgfsetroundjoin%
\definecolor{currentfill}{rgb}{0.800000,0.200000,0.200000}%
\pgfsetfillcolor{currentfill}%
\pgfsetlinewidth{1.003750pt}%
\definecolor{currentstroke}{rgb}{0.800000,0.200000,0.200000}%
\pgfsetstrokecolor{currentstroke}%
\pgfsetdash{}{0pt}%
\pgfpathmoveto{\pgfqpoint{4.372369in}{6.138379in}}%
\pgfpathcurveto{\pgfqpoint{4.378193in}{6.138379in}}{\pgfqpoint{4.383779in}{6.140693in}}{\pgfqpoint{4.387897in}{6.144811in}}%
\pgfpathcurveto{\pgfqpoint{4.392015in}{6.148929in}}{\pgfqpoint{4.394329in}{6.154515in}}{\pgfqpoint{4.394329in}{6.160339in}}%
\pgfpathcurveto{\pgfqpoint{4.394329in}{6.166163in}}{\pgfqpoint{4.392015in}{6.171749in}}{\pgfqpoint{4.387897in}{6.175867in}}%
\pgfpathcurveto{\pgfqpoint{4.383779in}{6.179985in}}{\pgfqpoint{4.378193in}{6.182299in}}{\pgfqpoint{4.372369in}{6.182299in}}%
\pgfpathcurveto{\pgfqpoint{4.366545in}{6.182299in}}{\pgfqpoint{4.360959in}{6.179985in}}{\pgfqpoint{4.356841in}{6.175867in}}%
\pgfpathcurveto{\pgfqpoint{4.352722in}{6.171749in}}{\pgfqpoint{4.350409in}{6.166163in}}{\pgfqpoint{4.350409in}{6.160339in}}%
\pgfpathcurveto{\pgfqpoint{4.350409in}{6.154515in}}{\pgfqpoint{4.352722in}{6.148929in}}{\pgfqpoint{4.356841in}{6.144811in}}%
\pgfpathcurveto{\pgfqpoint{4.360959in}{6.140693in}}{\pgfqpoint{4.366545in}{6.138379in}}{\pgfqpoint{4.372369in}{6.138379in}}%
\pgfpathlineto{\pgfqpoint{4.372369in}{6.138379in}}%
\pgfpathclose%
\pgfusepath{stroke,fill}%
\end{pgfscope}%
\begin{pgfscope}%
\pgfpathrectangle{\pgfqpoint{1.073501in}{0.880000in}}{\pgfqpoint{6.052998in}{6.160000in}}%
\pgfusepath{clip}%
\pgfsetbuttcap%
\pgfsetroundjoin%
\definecolor{currentfill}{rgb}{0.800000,0.200000,0.200000}%
\pgfsetfillcolor{currentfill}%
\pgfsetlinewidth{1.003750pt}%
\definecolor{currentstroke}{rgb}{0.800000,0.200000,0.200000}%
\pgfsetstrokecolor{currentstroke}%
\pgfsetdash{}{0pt}%
\pgfpathmoveto{\pgfqpoint{4.233325in}{6.187034in}}%
\pgfpathcurveto{\pgfqpoint{4.239149in}{6.187034in}}{\pgfqpoint{4.244735in}{6.189348in}}{\pgfqpoint{4.248854in}{6.193466in}}%
\pgfpathcurveto{\pgfqpoint{4.252972in}{6.197584in}}{\pgfqpoint{4.255286in}{6.203171in}}{\pgfqpoint{4.255286in}{6.208994in}}%
\pgfpathcurveto{\pgfqpoint{4.255286in}{6.214818in}}{\pgfqpoint{4.252972in}{6.220405in}}{\pgfqpoint{4.248854in}{6.224523in}}%
\pgfpathcurveto{\pgfqpoint{4.244735in}{6.228641in}}{\pgfqpoint{4.239149in}{6.230955in}}{\pgfqpoint{4.233325in}{6.230955in}}%
\pgfpathcurveto{\pgfqpoint{4.227501in}{6.230955in}}{\pgfqpoint{4.221915in}{6.228641in}}{\pgfqpoint{4.217797in}{6.224523in}}%
\pgfpathcurveto{\pgfqpoint{4.213679in}{6.220405in}}{\pgfqpoint{4.211365in}{6.214818in}}{\pgfqpoint{4.211365in}{6.208994in}}%
\pgfpathcurveto{\pgfqpoint{4.211365in}{6.203171in}}{\pgfqpoint{4.213679in}{6.197584in}}{\pgfqpoint{4.217797in}{6.193466in}}%
\pgfpathcurveto{\pgfqpoint{4.221915in}{6.189348in}}{\pgfqpoint{4.227501in}{6.187034in}}{\pgfqpoint{4.233325in}{6.187034in}}%
\pgfpathlineto{\pgfqpoint{4.233325in}{6.187034in}}%
\pgfpathclose%
\pgfusepath{stroke,fill}%
\end{pgfscope}%
\begin{pgfscope}%
\pgfpathrectangle{\pgfqpoint{1.073501in}{0.880000in}}{\pgfqpoint{6.052998in}{6.160000in}}%
\pgfusepath{clip}%
\pgfsetbuttcap%
\pgfsetroundjoin%
\definecolor{currentfill}{rgb}{0.800000,0.200000,0.200000}%
\pgfsetfillcolor{currentfill}%
\pgfsetlinewidth{1.003750pt}%
\definecolor{currentstroke}{rgb}{0.800000,0.200000,0.200000}%
\pgfsetstrokecolor{currentstroke}%
\pgfsetdash{}{0pt}%
\pgfpathmoveto{\pgfqpoint{4.090821in}{6.118117in}}%
\pgfpathcurveto{\pgfqpoint{4.096645in}{6.118117in}}{\pgfqpoint{4.102231in}{6.120430in}}{\pgfqpoint{4.106349in}{6.124549in}}%
\pgfpathcurveto{\pgfqpoint{4.110468in}{6.128667in}}{\pgfqpoint{4.112781in}{6.134253in}}{\pgfqpoint{4.112781in}{6.140077in}}%
\pgfpathcurveto{\pgfqpoint{4.112781in}{6.145901in}}{\pgfqpoint{4.110468in}{6.151487in}}{\pgfqpoint{4.106349in}{6.155605in}}%
\pgfpathcurveto{\pgfqpoint{4.102231in}{6.159723in}}{\pgfqpoint{4.096645in}{6.162037in}}{\pgfqpoint{4.090821in}{6.162037in}}%
\pgfpathcurveto{\pgfqpoint{4.084997in}{6.162037in}}{\pgfqpoint{4.079411in}{6.159723in}}{\pgfqpoint{4.075293in}{6.155605in}}%
\pgfpathcurveto{\pgfqpoint{4.071175in}{6.151487in}}{\pgfqpoint{4.068861in}{6.145901in}}{\pgfqpoint{4.068861in}{6.140077in}}%
\pgfpathcurveto{\pgfqpoint{4.068861in}{6.134253in}}{\pgfqpoint{4.071175in}{6.128667in}}{\pgfqpoint{4.075293in}{6.124549in}}%
\pgfpathcurveto{\pgfqpoint{4.079411in}{6.120430in}}{\pgfqpoint{4.084997in}{6.118117in}}{\pgfqpoint{4.090821in}{6.118117in}}%
\pgfpathlineto{\pgfqpoint{4.090821in}{6.118117in}}%
\pgfpathclose%
\pgfusepath{stroke,fill}%
\end{pgfscope}%
\begin{pgfscope}%
\pgfpathrectangle{\pgfqpoint{1.073501in}{0.880000in}}{\pgfqpoint{6.052998in}{6.160000in}}%
\pgfusepath{clip}%
\pgfsetbuttcap%
\pgfsetroundjoin%
\definecolor{currentfill}{rgb}{0.800000,0.200000,0.200000}%
\pgfsetfillcolor{currentfill}%
\pgfsetlinewidth{1.003750pt}%
\definecolor{currentstroke}{rgb}{0.800000,0.200000,0.200000}%
\pgfsetstrokecolor{currentstroke}%
\pgfsetdash{}{0pt}%
\pgfpathmoveto{\pgfqpoint{3.957905in}{6.035270in}}%
\pgfpathcurveto{\pgfqpoint{3.963729in}{6.035270in}}{\pgfqpoint{3.969315in}{6.037584in}}{\pgfqpoint{3.973434in}{6.041702in}}%
\pgfpathcurveto{\pgfqpoint{3.977552in}{6.045820in}}{\pgfqpoint{3.979866in}{6.051406in}}{\pgfqpoint{3.979866in}{6.057230in}}%
\pgfpathcurveto{\pgfqpoint{3.979866in}{6.063054in}}{\pgfqpoint{3.977552in}{6.068640in}}{\pgfqpoint{3.973434in}{6.072758in}}%
\pgfpathcurveto{\pgfqpoint{3.969315in}{6.076876in}}{\pgfqpoint{3.963729in}{6.079190in}}{\pgfqpoint{3.957905in}{6.079190in}}%
\pgfpathcurveto{\pgfqpoint{3.952081in}{6.079190in}}{\pgfqpoint{3.946495in}{6.076876in}}{\pgfqpoint{3.942377in}{6.072758in}}%
\pgfpathcurveto{\pgfqpoint{3.938259in}{6.068640in}}{\pgfqpoint{3.935945in}{6.063054in}}{\pgfqpoint{3.935945in}{6.057230in}}%
\pgfpathcurveto{\pgfqpoint{3.935945in}{6.051406in}}{\pgfqpoint{3.938259in}{6.045820in}}{\pgfqpoint{3.942377in}{6.041702in}}%
\pgfpathcurveto{\pgfqpoint{3.946495in}{6.037584in}}{\pgfqpoint{3.952081in}{6.035270in}}{\pgfqpoint{3.957905in}{6.035270in}}%
\pgfpathlineto{\pgfqpoint{3.957905in}{6.035270in}}%
\pgfpathclose%
\pgfusepath{stroke,fill}%
\end{pgfscope}%
\begin{pgfscope}%
\pgfpathrectangle{\pgfqpoint{1.073501in}{0.880000in}}{\pgfqpoint{6.052998in}{6.160000in}}%
\pgfusepath{clip}%
\pgfsetbuttcap%
\pgfsetroundjoin%
\definecolor{currentfill}{rgb}{0.800000,0.200000,0.200000}%
\pgfsetfillcolor{currentfill}%
\pgfsetlinewidth{1.003750pt}%
\definecolor{currentstroke}{rgb}{0.800000,0.200000,0.200000}%
\pgfsetstrokecolor{currentstroke}%
\pgfsetdash{}{0pt}%
\pgfpathmoveto{\pgfqpoint{3.810322in}{6.118659in}}%
\pgfpathcurveto{\pgfqpoint{3.816146in}{6.118659in}}{\pgfqpoint{3.821732in}{6.120972in}}{\pgfqpoint{3.825850in}{6.125091in}}%
\pgfpathcurveto{\pgfqpoint{3.829968in}{6.129209in}}{\pgfqpoint{3.832282in}{6.134795in}}{\pgfqpoint{3.832282in}{6.140619in}}%
\pgfpathcurveto{\pgfqpoint{3.832282in}{6.146443in}}{\pgfqpoint{3.829968in}{6.152029in}}{\pgfqpoint{3.825850in}{6.156147in}}%
\pgfpathcurveto{\pgfqpoint{3.821732in}{6.160265in}}{\pgfqpoint{3.816146in}{6.162579in}}{\pgfqpoint{3.810322in}{6.162579in}}%
\pgfpathcurveto{\pgfqpoint{3.804498in}{6.162579in}}{\pgfqpoint{3.798912in}{6.160265in}}{\pgfqpoint{3.794794in}{6.156147in}}%
\pgfpathcurveto{\pgfqpoint{3.790676in}{6.152029in}}{\pgfqpoint{3.788362in}{6.146443in}}{\pgfqpoint{3.788362in}{6.140619in}}%
\pgfpathcurveto{\pgfqpoint{3.788362in}{6.134795in}}{\pgfqpoint{3.790676in}{6.129209in}}{\pgfqpoint{3.794794in}{6.125091in}}%
\pgfpathcurveto{\pgfqpoint{3.798912in}{6.120972in}}{\pgfqpoint{3.804498in}{6.118659in}}{\pgfqpoint{3.810322in}{6.118659in}}%
\pgfpathlineto{\pgfqpoint{3.810322in}{6.118659in}}%
\pgfpathclose%
\pgfusepath{stroke,fill}%
\end{pgfscope}%
\begin{pgfscope}%
\pgfpathrectangle{\pgfqpoint{1.073501in}{0.880000in}}{\pgfqpoint{6.052998in}{6.160000in}}%
\pgfusepath{clip}%
\pgfsetbuttcap%
\pgfsetroundjoin%
\definecolor{currentfill}{rgb}{0.800000,0.200000,0.200000}%
\pgfsetfillcolor{currentfill}%
\pgfsetlinewidth{1.003750pt}%
\definecolor{currentstroke}{rgb}{0.800000,0.200000,0.200000}%
\pgfsetstrokecolor{currentstroke}%
\pgfsetdash{}{0pt}%
\pgfpathmoveto{\pgfqpoint{3.668246in}{6.111596in}}%
\pgfpathcurveto{\pgfqpoint{3.674070in}{6.111596in}}{\pgfqpoint{3.679656in}{6.113910in}}{\pgfqpoint{3.683775in}{6.118028in}}%
\pgfpathcurveto{\pgfqpoint{3.687893in}{6.122146in}}{\pgfqpoint{3.690207in}{6.127732in}}{\pgfqpoint{3.690207in}{6.133556in}}%
\pgfpathcurveto{\pgfqpoint{3.690207in}{6.139380in}}{\pgfqpoint{3.687893in}{6.144966in}}{\pgfqpoint{3.683775in}{6.149084in}}%
\pgfpathcurveto{\pgfqpoint{3.679656in}{6.153203in}}{\pgfqpoint{3.674070in}{6.155516in}}{\pgfqpoint{3.668246in}{6.155516in}}%
\pgfpathcurveto{\pgfqpoint{3.662422in}{6.155516in}}{\pgfqpoint{3.656836in}{6.153203in}}{\pgfqpoint{3.652718in}{6.149084in}}%
\pgfpathcurveto{\pgfqpoint{3.648600in}{6.144966in}}{\pgfqpoint{3.646286in}{6.139380in}}{\pgfqpoint{3.646286in}{6.133556in}}%
\pgfpathcurveto{\pgfqpoint{3.646286in}{6.127732in}}{\pgfqpoint{3.648600in}{6.122146in}}{\pgfqpoint{3.652718in}{6.118028in}}%
\pgfpathcurveto{\pgfqpoint{3.656836in}{6.113910in}}{\pgfqpoint{3.662422in}{6.111596in}}{\pgfqpoint{3.668246in}{6.111596in}}%
\pgfpathlineto{\pgfqpoint{3.668246in}{6.111596in}}%
\pgfpathclose%
\pgfusepath{stroke,fill}%
\end{pgfscope}%
\begin{pgfscope}%
\pgfpathrectangle{\pgfqpoint{1.073501in}{0.880000in}}{\pgfqpoint{6.052998in}{6.160000in}}%
\pgfusepath{clip}%
\pgfsetbuttcap%
\pgfsetroundjoin%
\definecolor{currentfill}{rgb}{0.800000,0.200000,0.200000}%
\pgfsetfillcolor{currentfill}%
\pgfsetlinewidth{1.003750pt}%
\definecolor{currentstroke}{rgb}{0.800000,0.200000,0.200000}%
\pgfsetstrokecolor{currentstroke}%
\pgfsetdash{}{0pt}%
\pgfpathmoveto{\pgfqpoint{3.542028in}{6.036614in}}%
\pgfpathcurveto{\pgfqpoint{3.547852in}{6.036614in}}{\pgfqpoint{3.553439in}{6.038928in}}{\pgfqpoint{3.557557in}{6.043046in}}%
\pgfpathcurveto{\pgfqpoint{3.561675in}{6.047164in}}{\pgfqpoint{3.563989in}{6.052750in}}{\pgfqpoint{3.563989in}{6.058574in}}%
\pgfpathcurveto{\pgfqpoint{3.563989in}{6.064398in}}{\pgfqpoint{3.561675in}{6.069985in}}{\pgfqpoint{3.557557in}{6.074103in}}%
\pgfpathcurveto{\pgfqpoint{3.553439in}{6.078221in}}{\pgfqpoint{3.547852in}{6.080535in}}{\pgfqpoint{3.542028in}{6.080535in}}%
\pgfpathcurveto{\pgfqpoint{3.536205in}{6.080535in}}{\pgfqpoint{3.530618in}{6.078221in}}{\pgfqpoint{3.526500in}{6.074103in}}%
\pgfpathcurveto{\pgfqpoint{3.522382in}{6.069985in}}{\pgfqpoint{3.520068in}{6.064398in}}{\pgfqpoint{3.520068in}{6.058574in}}%
\pgfpathcurveto{\pgfqpoint{3.520068in}{6.052750in}}{\pgfqpoint{3.522382in}{6.047164in}}{\pgfqpoint{3.526500in}{6.043046in}}%
\pgfpathcurveto{\pgfqpoint{3.530618in}{6.038928in}}{\pgfqpoint{3.536205in}{6.036614in}}{\pgfqpoint{3.542028in}{6.036614in}}%
\pgfpathlineto{\pgfqpoint{3.542028in}{6.036614in}}%
\pgfpathclose%
\pgfusepath{stroke,fill}%
\end{pgfscope}%
\begin{pgfscope}%
\pgfpathrectangle{\pgfqpoint{1.073501in}{0.880000in}}{\pgfqpoint{6.052998in}{6.160000in}}%
\pgfusepath{clip}%
\pgfsetbuttcap%
\pgfsetroundjoin%
\definecolor{currentfill}{rgb}{0.800000,0.200000,0.200000}%
\pgfsetfillcolor{currentfill}%
\pgfsetlinewidth{1.003750pt}%
\definecolor{currentstroke}{rgb}{0.800000,0.200000,0.200000}%
\pgfsetstrokecolor{currentstroke}%
\pgfsetdash{}{0pt}%
\pgfpathmoveto{\pgfqpoint{3.391039in}{6.048191in}}%
\pgfpathcurveto{\pgfqpoint{3.396863in}{6.048191in}}{\pgfqpoint{3.402449in}{6.050505in}}{\pgfqpoint{3.406567in}{6.054623in}}%
\pgfpathcurveto{\pgfqpoint{3.410685in}{6.058741in}}{\pgfqpoint{3.412999in}{6.064328in}}{\pgfqpoint{3.412999in}{6.070151in}}%
\pgfpathcurveto{\pgfqpoint{3.412999in}{6.075975in}}{\pgfqpoint{3.410685in}{6.081562in}}{\pgfqpoint{3.406567in}{6.085680in}}%
\pgfpathcurveto{\pgfqpoint{3.402449in}{6.089798in}}{\pgfqpoint{3.396863in}{6.092112in}}{\pgfqpoint{3.391039in}{6.092112in}}%
\pgfpathcurveto{\pgfqpoint{3.385215in}{6.092112in}}{\pgfqpoint{3.379628in}{6.089798in}}{\pgfqpoint{3.375510in}{6.085680in}}%
\pgfpathcurveto{\pgfqpoint{3.371392in}{6.081562in}}{\pgfqpoint{3.369078in}{6.075975in}}{\pgfqpoint{3.369078in}{6.070151in}}%
\pgfpathcurveto{\pgfqpoint{3.369078in}{6.064328in}}{\pgfqpoint{3.371392in}{6.058741in}}{\pgfqpoint{3.375510in}{6.054623in}}%
\pgfpathcurveto{\pgfqpoint{3.379628in}{6.050505in}}{\pgfqpoint{3.385215in}{6.048191in}}{\pgfqpoint{3.391039in}{6.048191in}}%
\pgfpathlineto{\pgfqpoint{3.391039in}{6.048191in}}%
\pgfpathclose%
\pgfusepath{stroke,fill}%
\end{pgfscope}%
\begin{pgfscope}%
\pgfpathrectangle{\pgfqpoint{1.073501in}{0.880000in}}{\pgfqpoint{6.052998in}{6.160000in}}%
\pgfusepath{clip}%
\pgfsetbuttcap%
\pgfsetroundjoin%
\definecolor{currentfill}{rgb}{0.800000,0.200000,0.200000}%
\pgfsetfillcolor{currentfill}%
\pgfsetlinewidth{1.003750pt}%
\definecolor{currentstroke}{rgb}{0.800000,0.200000,0.200000}%
\pgfsetstrokecolor{currentstroke}%
\pgfsetdash{}{0pt}%
\pgfpathmoveto{\pgfqpoint{3.281351in}{5.941357in}}%
\pgfpathcurveto{\pgfqpoint{3.287175in}{5.941357in}}{\pgfqpoint{3.292762in}{5.943670in}}{\pgfqpoint{3.296880in}{5.947789in}}%
\pgfpathcurveto{\pgfqpoint{3.300998in}{5.951907in}}{\pgfqpoint{3.303312in}{5.957493in}}{\pgfqpoint{3.303312in}{5.963317in}}%
\pgfpathcurveto{\pgfqpoint{3.303312in}{5.969141in}}{\pgfqpoint{3.300998in}{5.974727in}}{\pgfqpoint{3.296880in}{5.978845in}}%
\pgfpathcurveto{\pgfqpoint{3.292762in}{5.982963in}}{\pgfqpoint{3.287175in}{5.985277in}}{\pgfqpoint{3.281351in}{5.985277in}}%
\pgfpathcurveto{\pgfqpoint{3.275528in}{5.985277in}}{\pgfqpoint{3.269941in}{5.982963in}}{\pgfqpoint{3.265823in}{5.978845in}}%
\pgfpathcurveto{\pgfqpoint{3.261705in}{5.974727in}}{\pgfqpoint{3.259391in}{5.969141in}}{\pgfqpoint{3.259391in}{5.963317in}}%
\pgfpathcurveto{\pgfqpoint{3.259391in}{5.957493in}}{\pgfqpoint{3.261705in}{5.951907in}}{\pgfqpoint{3.265823in}{5.947789in}}%
\pgfpathcurveto{\pgfqpoint{3.269941in}{5.943670in}}{\pgfqpoint{3.275528in}{5.941357in}}{\pgfqpoint{3.281351in}{5.941357in}}%
\pgfpathlineto{\pgfqpoint{3.281351in}{5.941357in}}%
\pgfpathclose%
\pgfusepath{stroke,fill}%
\end{pgfscope}%
\begin{pgfscope}%
\pgfpathrectangle{\pgfqpoint{1.073501in}{0.880000in}}{\pgfqpoint{6.052998in}{6.160000in}}%
\pgfusepath{clip}%
\pgfsetbuttcap%
\pgfsetroundjoin%
\definecolor{currentfill}{rgb}{0.800000,0.200000,0.200000}%
\pgfsetfillcolor{currentfill}%
\pgfsetlinewidth{1.003750pt}%
\definecolor{currentstroke}{rgb}{0.800000,0.200000,0.200000}%
\pgfsetstrokecolor{currentstroke}%
\pgfsetdash{}{0pt}%
\pgfpathmoveto{\pgfqpoint{3.072516in}{6.050616in}}%
\pgfpathcurveto{\pgfqpoint{3.078340in}{6.050616in}}{\pgfqpoint{3.083926in}{6.052930in}}{\pgfqpoint{3.088045in}{6.057048in}}%
\pgfpathcurveto{\pgfqpoint{3.092163in}{6.061166in}}{\pgfqpoint{3.094477in}{6.066752in}}{\pgfqpoint{3.094477in}{6.072576in}}%
\pgfpathcurveto{\pgfqpoint{3.094477in}{6.078400in}}{\pgfqpoint{3.092163in}{6.083986in}}{\pgfqpoint{3.088045in}{6.088104in}}%
\pgfpathcurveto{\pgfqpoint{3.083926in}{6.092223in}}{\pgfqpoint{3.078340in}{6.094536in}}{\pgfqpoint{3.072516in}{6.094536in}}%
\pgfpathcurveto{\pgfqpoint{3.066692in}{6.094536in}}{\pgfqpoint{3.061106in}{6.092223in}}{\pgfqpoint{3.056988in}{6.088104in}}%
\pgfpathcurveto{\pgfqpoint{3.052870in}{6.083986in}}{\pgfqpoint{3.050556in}{6.078400in}}{\pgfqpoint{3.050556in}{6.072576in}}%
\pgfpathcurveto{\pgfqpoint{3.050556in}{6.066752in}}{\pgfqpoint{3.052870in}{6.061166in}}{\pgfqpoint{3.056988in}{6.057048in}}%
\pgfpathcurveto{\pgfqpoint{3.061106in}{6.052930in}}{\pgfqpoint{3.066692in}{6.050616in}}{\pgfqpoint{3.072516in}{6.050616in}}%
\pgfpathlineto{\pgfqpoint{3.072516in}{6.050616in}}%
\pgfpathclose%
\pgfusepath{stroke,fill}%
\end{pgfscope}%
\begin{pgfscope}%
\pgfpathrectangle{\pgfqpoint{1.073501in}{0.880000in}}{\pgfqpoint{6.052998in}{6.160000in}}%
\pgfusepath{clip}%
\pgfsetbuttcap%
\pgfsetroundjoin%
\definecolor{currentfill}{rgb}{0.800000,0.200000,0.200000}%
\pgfsetfillcolor{currentfill}%
\pgfsetlinewidth{1.003750pt}%
\definecolor{currentstroke}{rgb}{0.800000,0.200000,0.200000}%
\pgfsetstrokecolor{currentstroke}%
\pgfsetdash{}{0pt}%
\pgfpathmoveto{\pgfqpoint{3.034116in}{5.816311in}}%
\pgfpathcurveto{\pgfqpoint{3.039940in}{5.816311in}}{\pgfqpoint{3.045526in}{5.818625in}}{\pgfqpoint{3.049644in}{5.822743in}}%
\pgfpathcurveto{\pgfqpoint{3.053762in}{5.826862in}}{\pgfqpoint{3.056076in}{5.832448in}}{\pgfqpoint{3.056076in}{5.838272in}}%
\pgfpathcurveto{\pgfqpoint{3.056076in}{5.844096in}}{\pgfqpoint{3.053762in}{5.849682in}}{\pgfqpoint{3.049644in}{5.853800in}}%
\pgfpathcurveto{\pgfqpoint{3.045526in}{5.857918in}}{\pgfqpoint{3.039940in}{5.860232in}}{\pgfqpoint{3.034116in}{5.860232in}}%
\pgfpathcurveto{\pgfqpoint{3.028292in}{5.860232in}}{\pgfqpoint{3.022706in}{5.857918in}}{\pgfqpoint{3.018587in}{5.853800in}}%
\pgfpathcurveto{\pgfqpoint{3.014469in}{5.849682in}}{\pgfqpoint{3.012155in}{5.844096in}}{\pgfqpoint{3.012155in}{5.838272in}}%
\pgfpathcurveto{\pgfqpoint{3.012155in}{5.832448in}}{\pgfqpoint{3.014469in}{5.826862in}}{\pgfqpoint{3.018587in}{5.822743in}}%
\pgfpathcurveto{\pgfqpoint{3.022706in}{5.818625in}}{\pgfqpoint{3.028292in}{5.816311in}}{\pgfqpoint{3.034116in}{5.816311in}}%
\pgfpathlineto{\pgfqpoint{3.034116in}{5.816311in}}%
\pgfpathclose%
\pgfusepath{stroke,fill}%
\end{pgfscope}%
\begin{pgfscope}%
\pgfpathrectangle{\pgfqpoint{1.073501in}{0.880000in}}{\pgfqpoint{6.052998in}{6.160000in}}%
\pgfusepath{clip}%
\pgfsetbuttcap%
\pgfsetroundjoin%
\definecolor{currentfill}{rgb}{0.800000,0.200000,0.200000}%
\pgfsetfillcolor{currentfill}%
\pgfsetlinewidth{1.003750pt}%
\definecolor{currentstroke}{rgb}{0.800000,0.200000,0.200000}%
\pgfsetstrokecolor{currentstroke}%
\pgfsetdash{}{0pt}%
\pgfpathmoveto{\pgfqpoint{2.902681in}{5.763912in}}%
\pgfpathcurveto{\pgfqpoint{2.908505in}{5.763912in}}{\pgfqpoint{2.914091in}{5.766226in}}{\pgfqpoint{2.918209in}{5.770344in}}%
\pgfpathcurveto{\pgfqpoint{2.922328in}{5.774462in}}{\pgfqpoint{2.924641in}{5.780048in}}{\pgfqpoint{2.924641in}{5.785872in}}%
\pgfpathcurveto{\pgfqpoint{2.924641in}{5.791696in}}{\pgfqpoint{2.922328in}{5.797282in}}{\pgfqpoint{2.918209in}{5.801400in}}%
\pgfpathcurveto{\pgfqpoint{2.914091in}{5.805519in}}{\pgfqpoint{2.908505in}{5.807832in}}{\pgfqpoint{2.902681in}{5.807832in}}%
\pgfpathcurveto{\pgfqpoint{2.896857in}{5.807832in}}{\pgfqpoint{2.891271in}{5.805519in}}{\pgfqpoint{2.887153in}{5.801400in}}%
\pgfpathcurveto{\pgfqpoint{2.883035in}{5.797282in}}{\pgfqpoint{2.880721in}{5.791696in}}{\pgfqpoint{2.880721in}{5.785872in}}%
\pgfpathcurveto{\pgfqpoint{2.880721in}{5.780048in}}{\pgfqpoint{2.883035in}{5.774462in}}{\pgfqpoint{2.887153in}{5.770344in}}%
\pgfpathcurveto{\pgfqpoint{2.891271in}{5.766226in}}{\pgfqpoint{2.896857in}{5.763912in}}{\pgfqpoint{2.902681in}{5.763912in}}%
\pgfpathlineto{\pgfqpoint{2.902681in}{5.763912in}}%
\pgfpathclose%
\pgfusepath{stroke,fill}%
\end{pgfscope}%
\begin{pgfscope}%
\pgfpathrectangle{\pgfqpoint{1.073501in}{0.880000in}}{\pgfqpoint{6.052998in}{6.160000in}}%
\pgfusepath{clip}%
\pgfsetbuttcap%
\pgfsetroundjoin%
\definecolor{currentfill}{rgb}{0.800000,0.200000,0.200000}%
\pgfsetfillcolor{currentfill}%
\pgfsetlinewidth{1.003750pt}%
\definecolor{currentstroke}{rgb}{0.800000,0.200000,0.200000}%
\pgfsetstrokecolor{currentstroke}%
\pgfsetdash{}{0pt}%
\pgfpathmoveto{\pgfqpoint{2.780574in}{5.693145in}}%
\pgfpathcurveto{\pgfqpoint{2.786398in}{5.693145in}}{\pgfqpoint{2.791984in}{5.695459in}}{\pgfqpoint{2.796102in}{5.699577in}}%
\pgfpathcurveto{\pgfqpoint{2.800220in}{5.703696in}}{\pgfqpoint{2.802534in}{5.709282in}}{\pgfqpoint{2.802534in}{5.715106in}}%
\pgfpathcurveto{\pgfqpoint{2.802534in}{5.720930in}}{\pgfqpoint{2.800220in}{5.726516in}}{\pgfqpoint{2.796102in}{5.730634in}}%
\pgfpathcurveto{\pgfqpoint{2.791984in}{5.734752in}}{\pgfqpoint{2.786398in}{5.737066in}}{\pgfqpoint{2.780574in}{5.737066in}}%
\pgfpathcurveto{\pgfqpoint{2.774750in}{5.737066in}}{\pgfqpoint{2.769163in}{5.734752in}}{\pgfqpoint{2.765045in}{5.730634in}}%
\pgfpathcurveto{\pgfqpoint{2.760927in}{5.726516in}}{\pgfqpoint{2.758613in}{5.720930in}}{\pgfqpoint{2.758613in}{5.715106in}}%
\pgfpathcurveto{\pgfqpoint{2.758613in}{5.709282in}}{\pgfqpoint{2.760927in}{5.703696in}}{\pgfqpoint{2.765045in}{5.699577in}}%
\pgfpathcurveto{\pgfqpoint{2.769163in}{5.695459in}}{\pgfqpoint{2.774750in}{5.693145in}}{\pgfqpoint{2.780574in}{5.693145in}}%
\pgfpathlineto{\pgfqpoint{2.780574in}{5.693145in}}%
\pgfpathclose%
\pgfusepath{stroke,fill}%
\end{pgfscope}%
\begin{pgfscope}%
\pgfpathrectangle{\pgfqpoint{1.073501in}{0.880000in}}{\pgfqpoint{6.052998in}{6.160000in}}%
\pgfusepath{clip}%
\pgfsetbuttcap%
\pgfsetroundjoin%
\definecolor{currentfill}{rgb}{0.800000,0.200000,0.200000}%
\pgfsetfillcolor{currentfill}%
\pgfsetlinewidth{1.003750pt}%
\definecolor{currentstroke}{rgb}{0.800000,0.200000,0.200000}%
\pgfsetstrokecolor{currentstroke}%
\pgfsetdash{}{0pt}%
\pgfpathmoveto{\pgfqpoint{2.610012in}{5.674882in}}%
\pgfpathcurveto{\pgfqpoint{2.615835in}{5.674882in}}{\pgfqpoint{2.621422in}{5.677196in}}{\pgfqpoint{2.625540in}{5.681314in}}%
\pgfpathcurveto{\pgfqpoint{2.629658in}{5.685432in}}{\pgfqpoint{2.631972in}{5.691019in}}{\pgfqpoint{2.631972in}{5.696843in}}%
\pgfpathcurveto{\pgfqpoint{2.631972in}{5.702666in}}{\pgfqpoint{2.629658in}{5.708253in}}{\pgfqpoint{2.625540in}{5.712371in}}%
\pgfpathcurveto{\pgfqpoint{2.621422in}{5.716489in}}{\pgfqpoint{2.615835in}{5.718803in}}{\pgfqpoint{2.610012in}{5.718803in}}%
\pgfpathcurveto{\pgfqpoint{2.604188in}{5.718803in}}{\pgfqpoint{2.598601in}{5.716489in}}{\pgfqpoint{2.594483in}{5.712371in}}%
\pgfpathcurveto{\pgfqpoint{2.590365in}{5.708253in}}{\pgfqpoint{2.588051in}{5.702666in}}{\pgfqpoint{2.588051in}{5.696843in}}%
\pgfpathcurveto{\pgfqpoint{2.588051in}{5.691019in}}{\pgfqpoint{2.590365in}{5.685432in}}{\pgfqpoint{2.594483in}{5.681314in}}%
\pgfpathcurveto{\pgfqpoint{2.598601in}{5.677196in}}{\pgfqpoint{2.604188in}{5.674882in}}{\pgfqpoint{2.610012in}{5.674882in}}%
\pgfpathlineto{\pgfqpoint{2.610012in}{5.674882in}}%
\pgfpathclose%
\pgfusepath{stroke,fill}%
\end{pgfscope}%
\begin{pgfscope}%
\pgfpathrectangle{\pgfqpoint{1.073501in}{0.880000in}}{\pgfqpoint{6.052998in}{6.160000in}}%
\pgfusepath{clip}%
\pgfsetbuttcap%
\pgfsetroundjoin%
\definecolor{currentfill}{rgb}{0.800000,0.200000,0.200000}%
\pgfsetfillcolor{currentfill}%
\pgfsetlinewidth{1.003750pt}%
\definecolor{currentstroke}{rgb}{0.800000,0.200000,0.200000}%
\pgfsetstrokecolor{currentstroke}%
\pgfsetdash{}{0pt}%
\pgfpathmoveto{\pgfqpoint{2.511818in}{5.565391in}}%
\pgfpathcurveto{\pgfqpoint{2.517642in}{5.565391in}}{\pgfqpoint{2.523229in}{5.567705in}}{\pgfqpoint{2.527347in}{5.571823in}}%
\pgfpathcurveto{\pgfqpoint{2.531465in}{5.575941in}}{\pgfqpoint{2.533779in}{5.581527in}}{\pgfqpoint{2.533779in}{5.587351in}}%
\pgfpathcurveto{\pgfqpoint{2.533779in}{5.593175in}}{\pgfqpoint{2.531465in}{5.598761in}}{\pgfqpoint{2.527347in}{5.602879in}}%
\pgfpathcurveto{\pgfqpoint{2.523229in}{5.606997in}}{\pgfqpoint{2.517642in}{5.609311in}}{\pgfqpoint{2.511818in}{5.609311in}}%
\pgfpathcurveto{\pgfqpoint{2.505994in}{5.609311in}}{\pgfqpoint{2.500408in}{5.606997in}}{\pgfqpoint{2.496290in}{5.602879in}}%
\pgfpathcurveto{\pgfqpoint{2.492172in}{5.598761in}}{\pgfqpoint{2.489858in}{5.593175in}}{\pgfqpoint{2.489858in}{5.587351in}}%
\pgfpathcurveto{\pgfqpoint{2.489858in}{5.581527in}}{\pgfqpoint{2.492172in}{5.575941in}}{\pgfqpoint{2.496290in}{5.571823in}}%
\pgfpathcurveto{\pgfqpoint{2.500408in}{5.567705in}}{\pgfqpoint{2.505994in}{5.565391in}}{\pgfqpoint{2.511818in}{5.565391in}}%
\pgfpathlineto{\pgfqpoint{2.511818in}{5.565391in}}%
\pgfpathclose%
\pgfusepath{stroke,fill}%
\end{pgfscope}%
\begin{pgfscope}%
\pgfpathrectangle{\pgfqpoint{1.073501in}{0.880000in}}{\pgfqpoint{6.052998in}{6.160000in}}%
\pgfusepath{clip}%
\pgfsetbuttcap%
\pgfsetroundjoin%
\definecolor{currentfill}{rgb}{0.800000,0.200000,0.200000}%
\pgfsetfillcolor{currentfill}%
\pgfsetlinewidth{1.003750pt}%
\definecolor{currentstroke}{rgb}{0.800000,0.200000,0.200000}%
\pgfsetstrokecolor{currentstroke}%
\pgfsetdash{}{0pt}%
\pgfpathmoveto{\pgfqpoint{2.440580in}{5.433342in}}%
\pgfpathcurveto{\pgfqpoint{2.446404in}{5.433342in}}{\pgfqpoint{2.451990in}{5.435656in}}{\pgfqpoint{2.456108in}{5.439774in}}%
\pgfpathcurveto{\pgfqpoint{2.460226in}{5.443892in}}{\pgfqpoint{2.462540in}{5.449478in}}{\pgfqpoint{2.462540in}{5.455302in}}%
\pgfpathcurveto{\pgfqpoint{2.462540in}{5.461126in}}{\pgfqpoint{2.460226in}{5.466712in}}{\pgfqpoint{2.456108in}{5.470831in}}%
\pgfpathcurveto{\pgfqpoint{2.451990in}{5.474949in}}{\pgfqpoint{2.446404in}{5.477263in}}{\pgfqpoint{2.440580in}{5.477263in}}%
\pgfpathcurveto{\pgfqpoint{2.434756in}{5.477263in}}{\pgfqpoint{2.429170in}{5.474949in}}{\pgfqpoint{2.425051in}{5.470831in}}%
\pgfpathcurveto{\pgfqpoint{2.420933in}{5.466712in}}{\pgfqpoint{2.418619in}{5.461126in}}{\pgfqpoint{2.418619in}{5.455302in}}%
\pgfpathcurveto{\pgfqpoint{2.418619in}{5.449478in}}{\pgfqpoint{2.420933in}{5.443892in}}{\pgfqpoint{2.425051in}{5.439774in}}%
\pgfpathcurveto{\pgfqpoint{2.429170in}{5.435656in}}{\pgfqpoint{2.434756in}{5.433342in}}{\pgfqpoint{2.440580in}{5.433342in}}%
\pgfpathlineto{\pgfqpoint{2.440580in}{5.433342in}}%
\pgfpathclose%
\pgfusepath{stroke,fill}%
\end{pgfscope}%
\begin{pgfscope}%
\pgfpathrectangle{\pgfqpoint{1.073501in}{0.880000in}}{\pgfqpoint{6.052998in}{6.160000in}}%
\pgfusepath{clip}%
\pgfsetbuttcap%
\pgfsetroundjoin%
\definecolor{currentfill}{rgb}{0.800000,0.200000,0.200000}%
\pgfsetfillcolor{currentfill}%
\pgfsetlinewidth{1.003750pt}%
\definecolor{currentstroke}{rgb}{0.800000,0.200000,0.200000}%
\pgfsetstrokecolor{currentstroke}%
\pgfsetdash{}{0pt}%
\pgfpathmoveto{\pgfqpoint{2.380583in}{5.298093in}}%
\pgfpathcurveto{\pgfqpoint{2.386407in}{5.298093in}}{\pgfqpoint{2.391994in}{5.300407in}}{\pgfqpoint{2.396112in}{5.304525in}}%
\pgfpathcurveto{\pgfqpoint{2.400230in}{5.308643in}}{\pgfqpoint{2.402544in}{5.314229in}}{\pgfqpoint{2.402544in}{5.320053in}}%
\pgfpathcurveto{\pgfqpoint{2.402544in}{5.325877in}}{\pgfqpoint{2.400230in}{5.331463in}}{\pgfqpoint{2.396112in}{5.335581in}}%
\pgfpathcurveto{\pgfqpoint{2.391994in}{5.339699in}}{\pgfqpoint{2.386407in}{5.342013in}}{\pgfqpoint{2.380583in}{5.342013in}}%
\pgfpathcurveto{\pgfqpoint{2.374760in}{5.342013in}}{\pgfqpoint{2.369173in}{5.339699in}}{\pgfqpoint{2.365055in}{5.335581in}}%
\pgfpathcurveto{\pgfqpoint{2.360937in}{5.331463in}}{\pgfqpoint{2.358623in}{5.325877in}}{\pgfqpoint{2.358623in}{5.320053in}}%
\pgfpathcurveto{\pgfqpoint{2.358623in}{5.314229in}}{\pgfqpoint{2.360937in}{5.308643in}}{\pgfqpoint{2.365055in}{5.304525in}}%
\pgfpathcurveto{\pgfqpoint{2.369173in}{5.300407in}}{\pgfqpoint{2.374760in}{5.298093in}}{\pgfqpoint{2.380583in}{5.298093in}}%
\pgfpathlineto{\pgfqpoint{2.380583in}{5.298093in}}%
\pgfpathclose%
\pgfusepath{stroke,fill}%
\end{pgfscope}%
\begin{pgfscope}%
\pgfpathrectangle{\pgfqpoint{1.073501in}{0.880000in}}{\pgfqpoint{6.052998in}{6.160000in}}%
\pgfusepath{clip}%
\pgfsetbuttcap%
\pgfsetroundjoin%
\definecolor{currentfill}{rgb}{0.800000,0.200000,0.200000}%
\pgfsetfillcolor{currentfill}%
\pgfsetlinewidth{1.003750pt}%
\definecolor{currentstroke}{rgb}{0.800000,0.200000,0.200000}%
\pgfsetstrokecolor{currentstroke}%
\pgfsetdash{}{0pt}%
\pgfpathmoveto{\pgfqpoint{2.398636in}{5.114723in}}%
\pgfpathcurveto{\pgfqpoint{2.404460in}{5.114723in}}{\pgfqpoint{2.410046in}{5.117037in}}{\pgfqpoint{2.414164in}{5.121155in}}%
\pgfpathcurveto{\pgfqpoint{2.418283in}{5.125274in}}{\pgfqpoint{2.420596in}{5.130860in}}{\pgfqpoint{2.420596in}{5.136684in}}%
\pgfpathcurveto{\pgfqpoint{2.420596in}{5.142508in}}{\pgfqpoint{2.418283in}{5.148094in}}{\pgfqpoint{2.414164in}{5.152212in}}%
\pgfpathcurveto{\pgfqpoint{2.410046in}{5.156330in}}{\pgfqpoint{2.404460in}{5.158644in}}{\pgfqpoint{2.398636in}{5.158644in}}%
\pgfpathcurveto{\pgfqpoint{2.392812in}{5.158644in}}{\pgfqpoint{2.387226in}{5.156330in}}{\pgfqpoint{2.383108in}{5.152212in}}%
\pgfpathcurveto{\pgfqpoint{2.378990in}{5.148094in}}{\pgfqpoint{2.376676in}{5.142508in}}{\pgfqpoint{2.376676in}{5.136684in}}%
\pgfpathcurveto{\pgfqpoint{2.376676in}{5.130860in}}{\pgfqpoint{2.378990in}{5.125274in}}{\pgfqpoint{2.383108in}{5.121155in}}%
\pgfpathcurveto{\pgfqpoint{2.387226in}{5.117037in}}{\pgfqpoint{2.392812in}{5.114723in}}{\pgfqpoint{2.398636in}{5.114723in}}%
\pgfpathlineto{\pgfqpoint{2.398636in}{5.114723in}}%
\pgfpathclose%
\pgfusepath{stroke,fill}%
\end{pgfscope}%
\begin{pgfscope}%
\pgfpathrectangle{\pgfqpoint{1.073501in}{0.880000in}}{\pgfqpoint{6.052998in}{6.160000in}}%
\pgfusepath{clip}%
\pgfsetbuttcap%
\pgfsetroundjoin%
\definecolor{currentfill}{rgb}{0.800000,0.200000,0.200000}%
\pgfsetfillcolor{currentfill}%
\pgfsetlinewidth{1.003750pt}%
\definecolor{currentstroke}{rgb}{0.800000,0.200000,0.200000}%
\pgfsetstrokecolor{currentstroke}%
\pgfsetdash{}{0pt}%
\pgfpathmoveto{\pgfqpoint{2.232331in}{5.059301in}}%
\pgfpathcurveto{\pgfqpoint{2.238155in}{5.059301in}}{\pgfqpoint{2.243741in}{5.061615in}}{\pgfqpoint{2.247859in}{5.065733in}}%
\pgfpathcurveto{\pgfqpoint{2.251977in}{5.069851in}}{\pgfqpoint{2.254291in}{5.075438in}}{\pgfqpoint{2.254291in}{5.081262in}}%
\pgfpathcurveto{\pgfqpoint{2.254291in}{5.087085in}}{\pgfqpoint{2.251977in}{5.092672in}}{\pgfqpoint{2.247859in}{5.096790in}}%
\pgfpathcurveto{\pgfqpoint{2.243741in}{5.100908in}}{\pgfqpoint{2.238155in}{5.103222in}}{\pgfqpoint{2.232331in}{5.103222in}}%
\pgfpathcurveto{\pgfqpoint{2.226507in}{5.103222in}}{\pgfqpoint{2.220921in}{5.100908in}}{\pgfqpoint{2.216802in}{5.096790in}}%
\pgfpathcurveto{\pgfqpoint{2.212684in}{5.092672in}}{\pgfqpoint{2.210370in}{5.087085in}}{\pgfqpoint{2.210370in}{5.081262in}}%
\pgfpathcurveto{\pgfqpoint{2.210370in}{5.075438in}}{\pgfqpoint{2.212684in}{5.069851in}}{\pgfqpoint{2.216802in}{5.065733in}}%
\pgfpathcurveto{\pgfqpoint{2.220921in}{5.061615in}}{\pgfqpoint{2.226507in}{5.059301in}}{\pgfqpoint{2.232331in}{5.059301in}}%
\pgfpathlineto{\pgfqpoint{2.232331in}{5.059301in}}%
\pgfpathclose%
\pgfusepath{stroke,fill}%
\end{pgfscope}%
\begin{pgfscope}%
\pgfpathrectangle{\pgfqpoint{1.073501in}{0.880000in}}{\pgfqpoint{6.052998in}{6.160000in}}%
\pgfusepath{clip}%
\pgfsetbuttcap%
\pgfsetroundjoin%
\definecolor{currentfill}{rgb}{0.800000,0.200000,0.200000}%
\pgfsetfillcolor{currentfill}%
\pgfsetlinewidth{1.003750pt}%
\definecolor{currentstroke}{rgb}{0.800000,0.200000,0.200000}%
\pgfsetstrokecolor{currentstroke}%
\pgfsetdash{}{0pt}%
\pgfpathmoveto{\pgfqpoint{2.115757in}{4.961940in}}%
\pgfpathcurveto{\pgfqpoint{2.121581in}{4.961940in}}{\pgfqpoint{2.127167in}{4.964254in}}{\pgfqpoint{2.131285in}{4.968372in}}%
\pgfpathcurveto{\pgfqpoint{2.135403in}{4.972491in}}{\pgfqpoint{2.137717in}{4.978077in}}{\pgfqpoint{2.137717in}{4.983901in}}%
\pgfpathcurveto{\pgfqpoint{2.137717in}{4.989725in}}{\pgfqpoint{2.135403in}{4.995311in}}{\pgfqpoint{2.131285in}{4.999429in}}%
\pgfpathcurveto{\pgfqpoint{2.127167in}{5.003547in}}{\pgfqpoint{2.121581in}{5.005861in}}{\pgfqpoint{2.115757in}{5.005861in}}%
\pgfpathcurveto{\pgfqpoint{2.109933in}{5.005861in}}{\pgfqpoint{2.104347in}{5.003547in}}{\pgfqpoint{2.100228in}{4.999429in}}%
\pgfpathcurveto{\pgfqpoint{2.096110in}{4.995311in}}{\pgfqpoint{2.093796in}{4.989725in}}{\pgfqpoint{2.093796in}{4.983901in}}%
\pgfpathcurveto{\pgfqpoint{2.093796in}{4.978077in}}{\pgfqpoint{2.096110in}{4.972491in}}{\pgfqpoint{2.100228in}{4.968372in}}%
\pgfpathcurveto{\pgfqpoint{2.104347in}{4.964254in}}{\pgfqpoint{2.109933in}{4.961940in}}{\pgfqpoint{2.115757in}{4.961940in}}%
\pgfpathlineto{\pgfqpoint{2.115757in}{4.961940in}}%
\pgfpathclose%
\pgfusepath{stroke,fill}%
\end{pgfscope}%
\begin{pgfscope}%
\pgfpathrectangle{\pgfqpoint{1.073501in}{0.880000in}}{\pgfqpoint{6.052998in}{6.160000in}}%
\pgfusepath{clip}%
\pgfsetbuttcap%
\pgfsetroundjoin%
\definecolor{currentfill}{rgb}{0.800000,0.200000,0.200000}%
\pgfsetfillcolor{currentfill}%
\pgfsetlinewidth{1.003750pt}%
\definecolor{currentstroke}{rgb}{0.800000,0.200000,0.200000}%
\pgfsetstrokecolor{currentstroke}%
\pgfsetdash{}{0pt}%
\pgfpathmoveto{\pgfqpoint{2.078615in}{4.821647in}}%
\pgfpathcurveto{\pgfqpoint{2.084439in}{4.821647in}}{\pgfqpoint{2.090025in}{4.823961in}}{\pgfqpoint{2.094143in}{4.828079in}}%
\pgfpathcurveto{\pgfqpoint{2.098261in}{4.832197in}}{\pgfqpoint{2.100575in}{4.837783in}}{\pgfqpoint{2.100575in}{4.843607in}}%
\pgfpathcurveto{\pgfqpoint{2.100575in}{4.849431in}}{\pgfqpoint{2.098261in}{4.855017in}}{\pgfqpoint{2.094143in}{4.859135in}}%
\pgfpathcurveto{\pgfqpoint{2.090025in}{4.863253in}}{\pgfqpoint{2.084439in}{4.865567in}}{\pgfqpoint{2.078615in}{4.865567in}}%
\pgfpathcurveto{\pgfqpoint{2.072791in}{4.865567in}}{\pgfqpoint{2.067205in}{4.863253in}}{\pgfqpoint{2.063087in}{4.859135in}}%
\pgfpathcurveto{\pgfqpoint{2.058969in}{4.855017in}}{\pgfqpoint{2.056655in}{4.849431in}}{\pgfqpoint{2.056655in}{4.843607in}}%
\pgfpathcurveto{\pgfqpoint{2.056655in}{4.837783in}}{\pgfqpoint{2.058969in}{4.832197in}}{\pgfqpoint{2.063087in}{4.828079in}}%
\pgfpathcurveto{\pgfqpoint{2.067205in}{4.823961in}}{\pgfqpoint{2.072791in}{4.821647in}}{\pgfqpoint{2.078615in}{4.821647in}}%
\pgfpathlineto{\pgfqpoint{2.078615in}{4.821647in}}%
\pgfpathclose%
\pgfusepath{stroke,fill}%
\end{pgfscope}%
\begin{pgfscope}%
\pgfpathrectangle{\pgfqpoint{1.073501in}{0.880000in}}{\pgfqpoint{6.052998in}{6.160000in}}%
\pgfusepath{clip}%
\pgfsetbuttcap%
\pgfsetroundjoin%
\definecolor{currentfill}{rgb}{0.800000,0.200000,0.200000}%
\pgfsetfillcolor{currentfill}%
\pgfsetlinewidth{1.003750pt}%
\definecolor{currentstroke}{rgb}{0.800000,0.200000,0.200000}%
\pgfsetstrokecolor{currentstroke}%
\pgfsetdash{}{0pt}%
\pgfpathmoveto{\pgfqpoint{1.941123in}{4.720878in}}%
\pgfpathcurveto{\pgfqpoint{1.946947in}{4.720878in}}{\pgfqpoint{1.952533in}{4.723191in}}{\pgfqpoint{1.956651in}{4.727310in}}%
\pgfpathcurveto{\pgfqpoint{1.960769in}{4.731428in}}{\pgfqpoint{1.963083in}{4.737014in}}{\pgfqpoint{1.963083in}{4.742838in}}%
\pgfpathcurveto{\pgfqpoint{1.963083in}{4.748662in}}{\pgfqpoint{1.960769in}{4.754248in}}{\pgfqpoint{1.956651in}{4.758366in}}%
\pgfpathcurveto{\pgfqpoint{1.952533in}{4.762484in}}{\pgfqpoint{1.946947in}{4.764798in}}{\pgfqpoint{1.941123in}{4.764798in}}%
\pgfpathcurveto{\pgfqpoint{1.935299in}{4.764798in}}{\pgfqpoint{1.929712in}{4.762484in}}{\pgfqpoint{1.925594in}{4.758366in}}%
\pgfpathcurveto{\pgfqpoint{1.921476in}{4.754248in}}{\pgfqpoint{1.919162in}{4.748662in}}{\pgfqpoint{1.919162in}{4.742838in}}%
\pgfpathcurveto{\pgfqpoint{1.919162in}{4.737014in}}{\pgfqpoint{1.921476in}{4.731428in}}{\pgfqpoint{1.925594in}{4.727310in}}%
\pgfpathcurveto{\pgfqpoint{1.929712in}{4.723191in}}{\pgfqpoint{1.935299in}{4.720878in}}{\pgfqpoint{1.941123in}{4.720878in}}%
\pgfpathlineto{\pgfqpoint{1.941123in}{4.720878in}}%
\pgfpathclose%
\pgfusepath{stroke,fill}%
\end{pgfscope}%
\begin{pgfscope}%
\pgfpathrectangle{\pgfqpoint{1.073501in}{0.880000in}}{\pgfqpoint{6.052998in}{6.160000in}}%
\pgfusepath{clip}%
\pgfsetbuttcap%
\pgfsetroundjoin%
\definecolor{currentfill}{rgb}{0.800000,0.200000,0.200000}%
\pgfsetfillcolor{currentfill}%
\pgfsetlinewidth{1.003750pt}%
\definecolor{currentstroke}{rgb}{0.800000,0.200000,0.200000}%
\pgfsetstrokecolor{currentstroke}%
\pgfsetdash{}{0pt}%
\pgfpathmoveto{\pgfqpoint{1.930895in}{4.570216in}}%
\pgfpathcurveto{\pgfqpoint{1.936719in}{4.570216in}}{\pgfqpoint{1.942305in}{4.572530in}}{\pgfqpoint{1.946423in}{4.576648in}}%
\pgfpathcurveto{\pgfqpoint{1.950542in}{4.580766in}}{\pgfqpoint{1.952855in}{4.586352in}}{\pgfqpoint{1.952855in}{4.592176in}}%
\pgfpathcurveto{\pgfqpoint{1.952855in}{4.598000in}}{\pgfqpoint{1.950542in}{4.603586in}}{\pgfqpoint{1.946423in}{4.607704in}}%
\pgfpathcurveto{\pgfqpoint{1.942305in}{4.611822in}}{\pgfqpoint{1.936719in}{4.614136in}}{\pgfqpoint{1.930895in}{4.614136in}}%
\pgfpathcurveto{\pgfqpoint{1.925071in}{4.614136in}}{\pgfqpoint{1.919485in}{4.611822in}}{\pgfqpoint{1.915367in}{4.607704in}}%
\pgfpathcurveto{\pgfqpoint{1.911249in}{4.603586in}}{\pgfqpoint{1.908935in}{4.598000in}}{\pgfqpoint{1.908935in}{4.592176in}}%
\pgfpathcurveto{\pgfqpoint{1.908935in}{4.586352in}}{\pgfqpoint{1.911249in}{4.580766in}}{\pgfqpoint{1.915367in}{4.576648in}}%
\pgfpathcurveto{\pgfqpoint{1.919485in}{4.572530in}}{\pgfqpoint{1.925071in}{4.570216in}}{\pgfqpoint{1.930895in}{4.570216in}}%
\pgfpathlineto{\pgfqpoint{1.930895in}{4.570216in}}%
\pgfpathclose%
\pgfusepath{stroke,fill}%
\end{pgfscope}%
\begin{pgfscope}%
\pgfpathrectangle{\pgfqpoint{1.073501in}{0.880000in}}{\pgfqpoint{6.052998in}{6.160000in}}%
\pgfusepath{clip}%
\pgfsetbuttcap%
\pgfsetroundjoin%
\definecolor{currentfill}{rgb}{0.800000,0.200000,0.200000}%
\pgfsetfillcolor{currentfill}%
\pgfsetlinewidth{1.003750pt}%
\definecolor{currentstroke}{rgb}{0.800000,0.200000,0.200000}%
\pgfsetstrokecolor{currentstroke}%
\pgfsetdash{}{0pt}%
\pgfpathmoveto{\pgfqpoint{1.962302in}{4.414394in}}%
\pgfpathcurveto{\pgfqpoint{1.968126in}{4.414394in}}{\pgfqpoint{1.973712in}{4.416708in}}{\pgfqpoint{1.977831in}{4.420826in}}%
\pgfpathcurveto{\pgfqpoint{1.981949in}{4.424944in}}{\pgfqpoint{1.984263in}{4.430530in}}{\pgfqpoint{1.984263in}{4.436354in}}%
\pgfpathcurveto{\pgfqpoint{1.984263in}{4.442178in}}{\pgfqpoint{1.981949in}{4.447764in}}{\pgfqpoint{1.977831in}{4.451882in}}%
\pgfpathcurveto{\pgfqpoint{1.973712in}{4.456001in}}{\pgfqpoint{1.968126in}{4.458314in}}{\pgfqpoint{1.962302in}{4.458314in}}%
\pgfpathcurveto{\pgfqpoint{1.956478in}{4.458314in}}{\pgfqpoint{1.950892in}{4.456001in}}{\pgfqpoint{1.946774in}{4.451882in}}%
\pgfpathcurveto{\pgfqpoint{1.942656in}{4.447764in}}{\pgfqpoint{1.940342in}{4.442178in}}{\pgfqpoint{1.940342in}{4.436354in}}%
\pgfpathcurveto{\pgfqpoint{1.940342in}{4.430530in}}{\pgfqpoint{1.942656in}{4.424944in}}{\pgfqpoint{1.946774in}{4.420826in}}%
\pgfpathcurveto{\pgfqpoint{1.950892in}{4.416708in}}{\pgfqpoint{1.956478in}{4.414394in}}{\pgfqpoint{1.962302in}{4.414394in}}%
\pgfpathlineto{\pgfqpoint{1.962302in}{4.414394in}}%
\pgfpathclose%
\pgfusepath{stroke,fill}%
\end{pgfscope}%
\begin{pgfscope}%
\pgfpathrectangle{\pgfqpoint{1.073501in}{0.880000in}}{\pgfqpoint{6.052998in}{6.160000in}}%
\pgfusepath{clip}%
\pgfsetbuttcap%
\pgfsetroundjoin%
\definecolor{currentfill}{rgb}{0.800000,0.200000,0.200000}%
\pgfsetfillcolor{currentfill}%
\pgfsetlinewidth{1.003750pt}%
\definecolor{currentstroke}{rgb}{0.800000,0.200000,0.200000}%
\pgfsetstrokecolor{currentstroke}%
\pgfsetdash{}{0pt}%
\pgfpathmoveto{\pgfqpoint{1.855358in}{4.289055in}}%
\pgfpathcurveto{\pgfqpoint{1.861182in}{4.289055in}}{\pgfqpoint{1.866768in}{4.291369in}}{\pgfqpoint{1.870886in}{4.295487in}}%
\pgfpathcurveto{\pgfqpoint{1.875004in}{4.299606in}}{\pgfqpoint{1.877318in}{4.305192in}}{\pgfqpoint{1.877318in}{4.311016in}}%
\pgfpathcurveto{\pgfqpoint{1.877318in}{4.316840in}}{\pgfqpoint{1.875004in}{4.322426in}}{\pgfqpoint{1.870886in}{4.326544in}}%
\pgfpathcurveto{\pgfqpoint{1.866768in}{4.330662in}}{\pgfqpoint{1.861182in}{4.332976in}}{\pgfqpoint{1.855358in}{4.332976in}}%
\pgfpathcurveto{\pgfqpoint{1.849534in}{4.332976in}}{\pgfqpoint{1.843948in}{4.330662in}}{\pgfqpoint{1.839830in}{4.326544in}}%
\pgfpathcurveto{\pgfqpoint{1.835711in}{4.322426in}}{\pgfqpoint{1.833398in}{4.316840in}}{\pgfqpoint{1.833398in}{4.311016in}}%
\pgfpathcurveto{\pgfqpoint{1.833398in}{4.305192in}}{\pgfqpoint{1.835711in}{4.299606in}}{\pgfqpoint{1.839830in}{4.295487in}}%
\pgfpathcurveto{\pgfqpoint{1.843948in}{4.291369in}}{\pgfqpoint{1.849534in}{4.289055in}}{\pgfqpoint{1.855358in}{4.289055in}}%
\pgfpathlineto{\pgfqpoint{1.855358in}{4.289055in}}%
\pgfpathclose%
\pgfusepath{stroke,fill}%
\end{pgfscope}%
\begin{pgfscope}%
\pgfpathrectangle{\pgfqpoint{1.073501in}{0.880000in}}{\pgfqpoint{6.052998in}{6.160000in}}%
\pgfusepath{clip}%
\pgfsetbuttcap%
\pgfsetroundjoin%
\definecolor{currentfill}{rgb}{0.800000,0.200000,0.200000}%
\pgfsetfillcolor{currentfill}%
\pgfsetlinewidth{1.003750pt}%
\definecolor{currentstroke}{rgb}{0.800000,0.200000,0.200000}%
\pgfsetstrokecolor{currentstroke}%
\pgfsetdash{}{0pt}%
\pgfpathmoveto{\pgfqpoint{1.965364in}{4.132068in}}%
\pgfpathcurveto{\pgfqpoint{1.971188in}{4.132068in}}{\pgfqpoint{1.976774in}{4.134382in}}{\pgfqpoint{1.980892in}{4.138500in}}%
\pgfpathcurveto{\pgfqpoint{1.985010in}{4.142618in}}{\pgfqpoint{1.987324in}{4.148204in}}{\pgfqpoint{1.987324in}{4.154028in}}%
\pgfpathcurveto{\pgfqpoint{1.987324in}{4.159852in}}{\pgfqpoint{1.985010in}{4.165438in}}{\pgfqpoint{1.980892in}{4.169556in}}%
\pgfpathcurveto{\pgfqpoint{1.976774in}{4.173675in}}{\pgfqpoint{1.971188in}{4.175988in}}{\pgfqpoint{1.965364in}{4.175988in}}%
\pgfpathcurveto{\pgfqpoint{1.959540in}{4.175988in}}{\pgfqpoint{1.953954in}{4.173675in}}{\pgfqpoint{1.949836in}{4.169556in}}%
\pgfpathcurveto{\pgfqpoint{1.945717in}{4.165438in}}{\pgfqpoint{1.943404in}{4.159852in}}{\pgfqpoint{1.943404in}{4.154028in}}%
\pgfpathcurveto{\pgfqpoint{1.943404in}{4.148204in}}{\pgfqpoint{1.945717in}{4.142618in}}{\pgfqpoint{1.949836in}{4.138500in}}%
\pgfpathcurveto{\pgfqpoint{1.953954in}{4.134382in}}{\pgfqpoint{1.959540in}{4.132068in}}{\pgfqpoint{1.965364in}{4.132068in}}%
\pgfpathlineto{\pgfqpoint{1.965364in}{4.132068in}}%
\pgfpathclose%
\pgfusepath{stroke,fill}%
\end{pgfscope}%
\begin{pgfscope}%
\pgfpathrectangle{\pgfqpoint{1.073501in}{0.880000in}}{\pgfqpoint{6.052998in}{6.160000in}}%
\pgfusepath{clip}%
\pgfsetbuttcap%
\pgfsetroundjoin%
\definecolor{currentfill}{rgb}{0.800000,0.200000,0.200000}%
\pgfsetfillcolor{currentfill}%
\pgfsetlinewidth{1.003750pt}%
\definecolor{currentstroke}{rgb}{0.800000,0.200000,0.200000}%
\pgfsetstrokecolor{currentstroke}%
\pgfsetdash{}{0pt}%
\pgfpathmoveto{\pgfqpoint{1.851488in}{3.997979in}}%
\pgfpathcurveto{\pgfqpoint{1.857311in}{3.997979in}}{\pgfqpoint{1.862898in}{4.000293in}}{\pgfqpoint{1.867016in}{4.004411in}}%
\pgfpathcurveto{\pgfqpoint{1.871134in}{4.008529in}}{\pgfqpoint{1.873448in}{4.014115in}}{\pgfqpoint{1.873448in}{4.019939in}}%
\pgfpathcurveto{\pgfqpoint{1.873448in}{4.025763in}}{\pgfqpoint{1.871134in}{4.031349in}}{\pgfqpoint{1.867016in}{4.035468in}}%
\pgfpathcurveto{\pgfqpoint{1.862898in}{4.039586in}}{\pgfqpoint{1.857311in}{4.041900in}}{\pgfqpoint{1.851488in}{4.041900in}}%
\pgfpathcurveto{\pgfqpoint{1.845664in}{4.041900in}}{\pgfqpoint{1.840077in}{4.039586in}}{\pgfqpoint{1.835959in}{4.035468in}}%
\pgfpathcurveto{\pgfqpoint{1.831841in}{4.031349in}}{\pgfqpoint{1.829527in}{4.025763in}}{\pgfqpoint{1.829527in}{4.019939in}}%
\pgfpathcurveto{\pgfqpoint{1.829527in}{4.014115in}}{\pgfqpoint{1.831841in}{4.008529in}}{\pgfqpoint{1.835959in}{4.004411in}}%
\pgfpathcurveto{\pgfqpoint{1.840077in}{4.000293in}}{\pgfqpoint{1.845664in}{3.997979in}}{\pgfqpoint{1.851488in}{3.997979in}}%
\pgfpathlineto{\pgfqpoint{1.851488in}{3.997979in}}%
\pgfpathclose%
\pgfusepath{stroke,fill}%
\end{pgfscope}%
\begin{pgfscope}%
\pgfpathrectangle{\pgfqpoint{1.073501in}{0.880000in}}{\pgfqpoint{6.052998in}{6.160000in}}%
\pgfusepath{clip}%
\pgfsetbuttcap%
\pgfsetroundjoin%
\definecolor{currentfill}{rgb}{0.800000,0.200000,0.200000}%
\pgfsetfillcolor{currentfill}%
\pgfsetlinewidth{1.003750pt}%
\definecolor{currentstroke}{rgb}{0.800000,0.200000,0.200000}%
\pgfsetstrokecolor{currentstroke}%
\pgfsetdash{}{0pt}%
\pgfpathmoveto{\pgfqpoint{1.809129in}{3.852255in}}%
\pgfpathcurveto{\pgfqpoint{1.814953in}{3.852255in}}{\pgfqpoint{1.820540in}{3.854569in}}{\pgfqpoint{1.824658in}{3.858687in}}%
\pgfpathcurveto{\pgfqpoint{1.828776in}{3.862806in}}{\pgfqpoint{1.831090in}{3.868392in}}{\pgfqpoint{1.831090in}{3.874216in}}%
\pgfpathcurveto{\pgfqpoint{1.831090in}{3.880040in}}{\pgfqpoint{1.828776in}{3.885626in}}{\pgfqpoint{1.824658in}{3.889744in}}%
\pgfpathcurveto{\pgfqpoint{1.820540in}{3.893862in}}{\pgfqpoint{1.814953in}{3.896176in}}{\pgfqpoint{1.809129in}{3.896176in}}%
\pgfpathcurveto{\pgfqpoint{1.803305in}{3.896176in}}{\pgfqpoint{1.797719in}{3.893862in}}{\pgfqpoint{1.793601in}{3.889744in}}%
\pgfpathcurveto{\pgfqpoint{1.789483in}{3.885626in}}{\pgfqpoint{1.787169in}{3.880040in}}{\pgfqpoint{1.787169in}{3.874216in}}%
\pgfpathcurveto{\pgfqpoint{1.787169in}{3.868392in}}{\pgfqpoint{1.789483in}{3.862806in}}{\pgfqpoint{1.793601in}{3.858687in}}%
\pgfpathcurveto{\pgfqpoint{1.797719in}{3.854569in}}{\pgfqpoint{1.803305in}{3.852255in}}{\pgfqpoint{1.809129in}{3.852255in}}%
\pgfpathlineto{\pgfqpoint{1.809129in}{3.852255in}}%
\pgfpathclose%
\pgfusepath{stroke,fill}%
\end{pgfscope}%
\begin{pgfscope}%
\pgfpathrectangle{\pgfqpoint{1.073501in}{0.880000in}}{\pgfqpoint{6.052998in}{6.160000in}}%
\pgfusepath{clip}%
\pgfsetbuttcap%
\pgfsetroundjoin%
\definecolor{currentfill}{rgb}{0.800000,0.200000,0.200000}%
\pgfsetfillcolor{currentfill}%
\pgfsetlinewidth{1.003750pt}%
\definecolor{currentstroke}{rgb}{0.800000,0.200000,0.200000}%
\pgfsetstrokecolor{currentstroke}%
\pgfsetdash{}{0pt}%
\pgfpathmoveto{\pgfqpoint{1.842922in}{3.707819in}}%
\pgfpathcurveto{\pgfqpoint{1.848746in}{3.707819in}}{\pgfqpoint{1.854332in}{3.710133in}}{\pgfqpoint{1.858450in}{3.714251in}}%
\pgfpathcurveto{\pgfqpoint{1.862569in}{3.718370in}}{\pgfqpoint{1.864882in}{3.723956in}}{\pgfqpoint{1.864882in}{3.729780in}}%
\pgfpathcurveto{\pgfqpoint{1.864882in}{3.735604in}}{\pgfqpoint{1.862569in}{3.741190in}}{\pgfqpoint{1.858450in}{3.745308in}}%
\pgfpathcurveto{\pgfqpoint{1.854332in}{3.749426in}}{\pgfqpoint{1.848746in}{3.751740in}}{\pgfqpoint{1.842922in}{3.751740in}}%
\pgfpathcurveto{\pgfqpoint{1.837098in}{3.751740in}}{\pgfqpoint{1.831512in}{3.749426in}}{\pgfqpoint{1.827394in}{3.745308in}}%
\pgfpathcurveto{\pgfqpoint{1.823276in}{3.741190in}}{\pgfqpoint{1.820962in}{3.735604in}}{\pgfqpoint{1.820962in}{3.729780in}}%
\pgfpathcurveto{\pgfqpoint{1.820962in}{3.723956in}}{\pgfqpoint{1.823276in}{3.718370in}}{\pgfqpoint{1.827394in}{3.714251in}}%
\pgfpathcurveto{\pgfqpoint{1.831512in}{3.710133in}}{\pgfqpoint{1.837098in}{3.707819in}}{\pgfqpoint{1.842922in}{3.707819in}}%
\pgfpathlineto{\pgfqpoint{1.842922in}{3.707819in}}%
\pgfpathclose%
\pgfusepath{stroke,fill}%
\end{pgfscope}%
\begin{pgfscope}%
\pgfpathrectangle{\pgfqpoint{1.073501in}{0.880000in}}{\pgfqpoint{6.052998in}{6.160000in}}%
\pgfusepath{clip}%
\pgfsetbuttcap%
\pgfsetroundjoin%
\definecolor{currentfill}{rgb}{0.800000,0.200000,0.200000}%
\pgfsetfillcolor{currentfill}%
\pgfsetlinewidth{1.003750pt}%
\definecolor{currentstroke}{rgb}{0.800000,0.200000,0.200000}%
\pgfsetstrokecolor{currentstroke}%
\pgfsetdash{}{0pt}%
\pgfpathmoveto{\pgfqpoint{1.929900in}{3.574451in}}%
\pgfpathcurveto{\pgfqpoint{1.935724in}{3.574451in}}{\pgfqpoint{1.941310in}{3.576765in}}{\pgfqpoint{1.945428in}{3.580883in}}%
\pgfpathcurveto{\pgfqpoint{1.949547in}{3.585001in}}{\pgfqpoint{1.951860in}{3.590588in}}{\pgfqpoint{1.951860in}{3.596412in}}%
\pgfpathcurveto{\pgfqpoint{1.951860in}{3.602235in}}{\pgfqpoint{1.949547in}{3.607822in}}{\pgfqpoint{1.945428in}{3.611940in}}%
\pgfpathcurveto{\pgfqpoint{1.941310in}{3.616058in}}{\pgfqpoint{1.935724in}{3.618372in}}{\pgfqpoint{1.929900in}{3.618372in}}%
\pgfpathcurveto{\pgfqpoint{1.924076in}{3.618372in}}{\pgfqpoint{1.918490in}{3.616058in}}{\pgfqpoint{1.914372in}{3.611940in}}%
\pgfpathcurveto{\pgfqpoint{1.910254in}{3.607822in}}{\pgfqpoint{1.907940in}{3.602235in}}{\pgfqpoint{1.907940in}{3.596412in}}%
\pgfpathcurveto{\pgfqpoint{1.907940in}{3.590588in}}{\pgfqpoint{1.910254in}{3.585001in}}{\pgfqpoint{1.914372in}{3.580883in}}%
\pgfpathcurveto{\pgfqpoint{1.918490in}{3.576765in}}{\pgfqpoint{1.924076in}{3.574451in}}{\pgfqpoint{1.929900in}{3.574451in}}%
\pgfpathlineto{\pgfqpoint{1.929900in}{3.574451in}}%
\pgfpathclose%
\pgfusepath{stroke,fill}%
\end{pgfscope}%
\begin{pgfscope}%
\pgfpathrectangle{\pgfqpoint{1.073501in}{0.880000in}}{\pgfqpoint{6.052998in}{6.160000in}}%
\pgfusepath{clip}%
\pgfsetbuttcap%
\pgfsetroundjoin%
\definecolor{currentfill}{rgb}{0.800000,0.200000,0.200000}%
\pgfsetfillcolor{currentfill}%
\pgfsetlinewidth{1.003750pt}%
\definecolor{currentstroke}{rgb}{0.800000,0.200000,0.200000}%
\pgfsetstrokecolor{currentstroke}%
\pgfsetdash{}{0pt}%
\pgfpathmoveto{\pgfqpoint{1.875607in}{3.417604in}}%
\pgfpathcurveto{\pgfqpoint{1.881431in}{3.417604in}}{\pgfqpoint{1.887017in}{3.419918in}}{\pgfqpoint{1.891135in}{3.424036in}}%
\pgfpathcurveto{\pgfqpoint{1.895253in}{3.428154in}}{\pgfqpoint{1.897567in}{3.433741in}}{\pgfqpoint{1.897567in}{3.439564in}}%
\pgfpathcurveto{\pgfqpoint{1.897567in}{3.445388in}}{\pgfqpoint{1.895253in}{3.450975in}}{\pgfqpoint{1.891135in}{3.455093in}}%
\pgfpathcurveto{\pgfqpoint{1.887017in}{3.459211in}}{\pgfqpoint{1.881431in}{3.461525in}}{\pgfqpoint{1.875607in}{3.461525in}}%
\pgfpathcurveto{\pgfqpoint{1.869783in}{3.461525in}}{\pgfqpoint{1.864197in}{3.459211in}}{\pgfqpoint{1.860079in}{3.455093in}}%
\pgfpathcurveto{\pgfqpoint{1.855961in}{3.450975in}}{\pgfqpoint{1.853647in}{3.445388in}}{\pgfqpoint{1.853647in}{3.439564in}}%
\pgfpathcurveto{\pgfqpoint{1.853647in}{3.433741in}}{\pgfqpoint{1.855961in}{3.428154in}}{\pgfqpoint{1.860079in}{3.424036in}}%
\pgfpathcurveto{\pgfqpoint{1.864197in}{3.419918in}}{\pgfqpoint{1.869783in}{3.417604in}}{\pgfqpoint{1.875607in}{3.417604in}}%
\pgfpathlineto{\pgfqpoint{1.875607in}{3.417604in}}%
\pgfpathclose%
\pgfusepath{stroke,fill}%
\end{pgfscope}%
\begin{pgfscope}%
\pgfpathrectangle{\pgfqpoint{1.073501in}{0.880000in}}{\pgfqpoint{6.052998in}{6.160000in}}%
\pgfusepath{clip}%
\pgfsetbuttcap%
\pgfsetroundjoin%
\definecolor{currentfill}{rgb}{0.800000,0.200000,0.200000}%
\pgfsetfillcolor{currentfill}%
\pgfsetlinewidth{1.003750pt}%
\definecolor{currentstroke}{rgb}{0.800000,0.200000,0.200000}%
\pgfsetstrokecolor{currentstroke}%
\pgfsetdash{}{0pt}%
\pgfpathmoveto{\pgfqpoint{2.058545in}{3.318845in}}%
\pgfpathcurveto{\pgfqpoint{2.064369in}{3.318845in}}{\pgfqpoint{2.069955in}{3.321159in}}{\pgfqpoint{2.074073in}{3.325277in}}%
\pgfpathcurveto{\pgfqpoint{2.078191in}{3.329395in}}{\pgfqpoint{2.080505in}{3.334981in}}{\pgfqpoint{2.080505in}{3.340805in}}%
\pgfpathcurveto{\pgfqpoint{2.080505in}{3.346629in}}{\pgfqpoint{2.078191in}{3.352215in}}{\pgfqpoint{2.074073in}{3.356333in}}%
\pgfpathcurveto{\pgfqpoint{2.069955in}{3.360451in}}{\pgfqpoint{2.064369in}{3.362765in}}{\pgfqpoint{2.058545in}{3.362765in}}%
\pgfpathcurveto{\pgfqpoint{2.052721in}{3.362765in}}{\pgfqpoint{2.047135in}{3.360451in}}{\pgfqpoint{2.043017in}{3.356333in}}%
\pgfpathcurveto{\pgfqpoint{2.038898in}{3.352215in}}{\pgfqpoint{2.036585in}{3.346629in}}{\pgfqpoint{2.036585in}{3.340805in}}%
\pgfpathcurveto{\pgfqpoint{2.036585in}{3.334981in}}{\pgfqpoint{2.038898in}{3.329395in}}{\pgfqpoint{2.043017in}{3.325277in}}%
\pgfpathcurveto{\pgfqpoint{2.047135in}{3.321159in}}{\pgfqpoint{2.052721in}{3.318845in}}{\pgfqpoint{2.058545in}{3.318845in}}%
\pgfpathlineto{\pgfqpoint{2.058545in}{3.318845in}}%
\pgfpathclose%
\pgfusepath{stroke,fill}%
\end{pgfscope}%
\begin{pgfscope}%
\pgfpathrectangle{\pgfqpoint{1.073501in}{0.880000in}}{\pgfqpoint{6.052998in}{6.160000in}}%
\pgfusepath{clip}%
\pgfsetbuttcap%
\pgfsetroundjoin%
\definecolor{currentfill}{rgb}{0.800000,0.200000,0.200000}%
\pgfsetfillcolor{currentfill}%
\pgfsetlinewidth{1.003750pt}%
\definecolor{currentstroke}{rgb}{0.800000,0.200000,0.200000}%
\pgfsetstrokecolor{currentstroke}%
\pgfsetdash{}{0pt}%
\pgfpathmoveto{\pgfqpoint{2.067844in}{3.176824in}}%
\pgfpathcurveto{\pgfqpoint{2.073668in}{3.176824in}}{\pgfqpoint{2.079254in}{3.179138in}}{\pgfqpoint{2.083373in}{3.183256in}}%
\pgfpathcurveto{\pgfqpoint{2.087491in}{3.187375in}}{\pgfqpoint{2.089805in}{3.192961in}}{\pgfqpoint{2.089805in}{3.198785in}}%
\pgfpathcurveto{\pgfqpoint{2.089805in}{3.204609in}}{\pgfqpoint{2.087491in}{3.210195in}}{\pgfqpoint{2.083373in}{3.214313in}}%
\pgfpathcurveto{\pgfqpoint{2.079254in}{3.218431in}}{\pgfqpoint{2.073668in}{3.220745in}}{\pgfqpoint{2.067844in}{3.220745in}}%
\pgfpathcurveto{\pgfqpoint{2.062020in}{3.220745in}}{\pgfqpoint{2.056434in}{3.218431in}}{\pgfqpoint{2.052316in}{3.214313in}}%
\pgfpathcurveto{\pgfqpoint{2.048198in}{3.210195in}}{\pgfqpoint{2.045884in}{3.204609in}}{\pgfqpoint{2.045884in}{3.198785in}}%
\pgfpathcurveto{\pgfqpoint{2.045884in}{3.192961in}}{\pgfqpoint{2.048198in}{3.187375in}}{\pgfqpoint{2.052316in}{3.183256in}}%
\pgfpathcurveto{\pgfqpoint{2.056434in}{3.179138in}}{\pgfqpoint{2.062020in}{3.176824in}}{\pgfqpoint{2.067844in}{3.176824in}}%
\pgfpathlineto{\pgfqpoint{2.067844in}{3.176824in}}%
\pgfpathclose%
\pgfusepath{stroke,fill}%
\end{pgfscope}%
\begin{pgfscope}%
\pgfpathrectangle{\pgfqpoint{1.073501in}{0.880000in}}{\pgfqpoint{6.052998in}{6.160000in}}%
\pgfusepath{clip}%
\pgfsetbuttcap%
\pgfsetroundjoin%
\definecolor{currentfill}{rgb}{0.800000,0.200000,0.200000}%
\pgfsetfillcolor{currentfill}%
\pgfsetlinewidth{1.003750pt}%
\definecolor{currentstroke}{rgb}{0.800000,0.200000,0.200000}%
\pgfsetstrokecolor{currentstroke}%
\pgfsetdash{}{0pt}%
\pgfpathmoveto{\pgfqpoint{2.108374in}{3.042956in}}%
\pgfpathcurveto{\pgfqpoint{2.114198in}{3.042956in}}{\pgfqpoint{2.119784in}{3.045270in}}{\pgfqpoint{2.123902in}{3.049388in}}%
\pgfpathcurveto{\pgfqpoint{2.128020in}{3.053506in}}{\pgfqpoint{2.130334in}{3.059093in}}{\pgfqpoint{2.130334in}{3.064916in}}%
\pgfpathcurveto{\pgfqpoint{2.130334in}{3.070740in}}{\pgfqpoint{2.128020in}{3.076327in}}{\pgfqpoint{2.123902in}{3.080445in}}%
\pgfpathcurveto{\pgfqpoint{2.119784in}{3.084563in}}{\pgfqpoint{2.114198in}{3.086877in}}{\pgfqpoint{2.108374in}{3.086877in}}%
\pgfpathcurveto{\pgfqpoint{2.102550in}{3.086877in}}{\pgfqpoint{2.096964in}{3.084563in}}{\pgfqpoint{2.092846in}{3.080445in}}%
\pgfpathcurveto{\pgfqpoint{2.088728in}{3.076327in}}{\pgfqpoint{2.086414in}{3.070740in}}{\pgfqpoint{2.086414in}{3.064916in}}%
\pgfpathcurveto{\pgfqpoint{2.086414in}{3.059093in}}{\pgfqpoint{2.088728in}{3.053506in}}{\pgfqpoint{2.092846in}{3.049388in}}%
\pgfpathcurveto{\pgfqpoint{2.096964in}{3.045270in}}{\pgfqpoint{2.102550in}{3.042956in}}{\pgfqpoint{2.108374in}{3.042956in}}%
\pgfpathlineto{\pgfqpoint{2.108374in}{3.042956in}}%
\pgfpathclose%
\pgfusepath{stroke,fill}%
\end{pgfscope}%
\begin{pgfscope}%
\pgfpathrectangle{\pgfqpoint{1.073501in}{0.880000in}}{\pgfqpoint{6.052998in}{6.160000in}}%
\pgfusepath{clip}%
\pgfsetbuttcap%
\pgfsetroundjoin%
\definecolor{currentfill}{rgb}{0.800000,0.200000,0.200000}%
\pgfsetfillcolor{currentfill}%
\pgfsetlinewidth{1.003750pt}%
\definecolor{currentstroke}{rgb}{0.800000,0.200000,0.200000}%
\pgfsetstrokecolor{currentstroke}%
\pgfsetdash{}{0pt}%
\pgfpathmoveto{\pgfqpoint{2.214606in}{2.940599in}}%
\pgfpathcurveto{\pgfqpoint{2.220430in}{2.940599in}}{\pgfqpoint{2.226016in}{2.942913in}}{\pgfqpoint{2.230134in}{2.947031in}}%
\pgfpathcurveto{\pgfqpoint{2.234252in}{2.951149in}}{\pgfqpoint{2.236566in}{2.956735in}}{\pgfqpoint{2.236566in}{2.962559in}}%
\pgfpathcurveto{\pgfqpoint{2.236566in}{2.968383in}}{\pgfqpoint{2.234252in}{2.973969in}}{\pgfqpoint{2.230134in}{2.978087in}}%
\pgfpathcurveto{\pgfqpoint{2.226016in}{2.982206in}}{\pgfqpoint{2.220430in}{2.984519in}}{\pgfqpoint{2.214606in}{2.984519in}}%
\pgfpathcurveto{\pgfqpoint{2.208782in}{2.984519in}}{\pgfqpoint{2.203196in}{2.982206in}}{\pgfqpoint{2.199077in}{2.978087in}}%
\pgfpathcurveto{\pgfqpoint{2.194959in}{2.973969in}}{\pgfqpoint{2.192645in}{2.968383in}}{\pgfqpoint{2.192645in}{2.962559in}}%
\pgfpathcurveto{\pgfqpoint{2.192645in}{2.956735in}}{\pgfqpoint{2.194959in}{2.951149in}}{\pgfqpoint{2.199077in}{2.947031in}}%
\pgfpathcurveto{\pgfqpoint{2.203196in}{2.942913in}}{\pgfqpoint{2.208782in}{2.940599in}}{\pgfqpoint{2.214606in}{2.940599in}}%
\pgfpathlineto{\pgfqpoint{2.214606in}{2.940599in}}%
\pgfpathclose%
\pgfusepath{stroke,fill}%
\end{pgfscope}%
\begin{pgfscope}%
\pgfpathrectangle{\pgfqpoint{1.073501in}{0.880000in}}{\pgfqpoint{6.052998in}{6.160000in}}%
\pgfusepath{clip}%
\pgfsetbuttcap%
\pgfsetroundjoin%
\definecolor{currentfill}{rgb}{0.800000,0.200000,0.200000}%
\pgfsetfillcolor{currentfill}%
\pgfsetlinewidth{1.003750pt}%
\definecolor{currentstroke}{rgb}{0.800000,0.200000,0.200000}%
\pgfsetstrokecolor{currentstroke}%
\pgfsetdash{}{0pt}%
\pgfpathmoveto{\pgfqpoint{2.133798in}{2.733286in}}%
\pgfpathcurveto{\pgfqpoint{2.139622in}{2.733286in}}{\pgfqpoint{2.145208in}{2.735600in}}{\pgfqpoint{2.149326in}{2.739718in}}%
\pgfpathcurveto{\pgfqpoint{2.153444in}{2.743836in}}{\pgfqpoint{2.155758in}{2.749422in}}{\pgfqpoint{2.155758in}{2.755246in}}%
\pgfpathcurveto{\pgfqpoint{2.155758in}{2.761070in}}{\pgfqpoint{2.153444in}{2.766656in}}{\pgfqpoint{2.149326in}{2.770774in}}%
\pgfpathcurveto{\pgfqpoint{2.145208in}{2.774892in}}{\pgfqpoint{2.139622in}{2.777206in}}{\pgfqpoint{2.133798in}{2.777206in}}%
\pgfpathcurveto{\pgfqpoint{2.127974in}{2.777206in}}{\pgfqpoint{2.122387in}{2.774892in}}{\pgfqpoint{2.118269in}{2.770774in}}%
\pgfpathcurveto{\pgfqpoint{2.114151in}{2.766656in}}{\pgfqpoint{2.111837in}{2.761070in}}{\pgfqpoint{2.111837in}{2.755246in}}%
\pgfpathcurveto{\pgfqpoint{2.111837in}{2.749422in}}{\pgfqpoint{2.114151in}{2.743836in}}{\pgfqpoint{2.118269in}{2.739718in}}%
\pgfpathcurveto{\pgfqpoint{2.122387in}{2.735600in}}{\pgfqpoint{2.127974in}{2.733286in}}{\pgfqpoint{2.133798in}{2.733286in}}%
\pgfpathlineto{\pgfqpoint{2.133798in}{2.733286in}}%
\pgfpathclose%
\pgfusepath{stroke,fill}%
\end{pgfscope}%
\begin{pgfscope}%
\pgfpathrectangle{\pgfqpoint{1.073501in}{0.880000in}}{\pgfqpoint{6.052998in}{6.160000in}}%
\pgfusepath{clip}%
\pgfsetbuttcap%
\pgfsetroundjoin%
\definecolor{currentfill}{rgb}{0.800000,0.200000,0.200000}%
\pgfsetfillcolor{currentfill}%
\pgfsetlinewidth{1.003750pt}%
\definecolor{currentstroke}{rgb}{0.800000,0.200000,0.200000}%
\pgfsetstrokecolor{currentstroke}%
\pgfsetdash{}{0pt}%
\pgfpathmoveto{\pgfqpoint{2.192140in}{2.594694in}}%
\pgfpathcurveto{\pgfqpoint{2.197964in}{2.594694in}}{\pgfqpoint{2.203550in}{2.597007in}}{\pgfqpoint{2.207668in}{2.601126in}}%
\pgfpathcurveto{\pgfqpoint{2.211787in}{2.605244in}}{\pgfqpoint{2.214100in}{2.610830in}}{\pgfqpoint{2.214100in}{2.616654in}}%
\pgfpathcurveto{\pgfqpoint{2.214100in}{2.622478in}}{\pgfqpoint{2.211787in}{2.628064in}}{\pgfqpoint{2.207668in}{2.632182in}}%
\pgfpathcurveto{\pgfqpoint{2.203550in}{2.636300in}}{\pgfqpoint{2.197964in}{2.638614in}}{\pgfqpoint{2.192140in}{2.638614in}}%
\pgfpathcurveto{\pgfqpoint{2.186316in}{2.638614in}}{\pgfqpoint{2.180730in}{2.636300in}}{\pgfqpoint{2.176612in}{2.632182in}}%
\pgfpathcurveto{\pgfqpoint{2.172494in}{2.628064in}}{\pgfqpoint{2.170180in}{2.622478in}}{\pgfqpoint{2.170180in}{2.616654in}}%
\pgfpathcurveto{\pgfqpoint{2.170180in}{2.610830in}}{\pgfqpoint{2.172494in}{2.605244in}}{\pgfqpoint{2.176612in}{2.601126in}}%
\pgfpathcurveto{\pgfqpoint{2.180730in}{2.597007in}}{\pgfqpoint{2.186316in}{2.594694in}}{\pgfqpoint{2.192140in}{2.594694in}}%
\pgfpathlineto{\pgfqpoint{2.192140in}{2.594694in}}%
\pgfpathclose%
\pgfusepath{stroke,fill}%
\end{pgfscope}%
\begin{pgfscope}%
\pgfpathrectangle{\pgfqpoint{1.073501in}{0.880000in}}{\pgfqpoint{6.052998in}{6.160000in}}%
\pgfusepath{clip}%
\pgfsetbuttcap%
\pgfsetroundjoin%
\definecolor{currentfill}{rgb}{0.800000,0.200000,0.200000}%
\pgfsetfillcolor{currentfill}%
\pgfsetlinewidth{1.003750pt}%
\definecolor{currentstroke}{rgb}{0.800000,0.200000,0.200000}%
\pgfsetstrokecolor{currentstroke}%
\pgfsetdash{}{0pt}%
\pgfpathmoveto{\pgfqpoint{2.359980in}{2.537284in}}%
\pgfpathcurveto{\pgfqpoint{2.365804in}{2.537284in}}{\pgfqpoint{2.371390in}{2.539598in}}{\pgfqpoint{2.375508in}{2.543716in}}%
\pgfpathcurveto{\pgfqpoint{2.379627in}{2.547834in}}{\pgfqpoint{2.381941in}{2.553420in}}{\pgfqpoint{2.381941in}{2.559244in}}%
\pgfpathcurveto{\pgfqpoint{2.381941in}{2.565068in}}{\pgfqpoint{2.379627in}{2.570654in}}{\pgfqpoint{2.375508in}{2.574772in}}%
\pgfpathcurveto{\pgfqpoint{2.371390in}{2.578890in}}{\pgfqpoint{2.365804in}{2.581204in}}{\pgfqpoint{2.359980in}{2.581204in}}%
\pgfpathcurveto{\pgfqpoint{2.354156in}{2.581204in}}{\pgfqpoint{2.348570in}{2.578890in}}{\pgfqpoint{2.344452in}{2.574772in}}%
\pgfpathcurveto{\pgfqpoint{2.340334in}{2.570654in}}{\pgfqpoint{2.338020in}{2.565068in}}{\pgfqpoint{2.338020in}{2.559244in}}%
\pgfpathcurveto{\pgfqpoint{2.338020in}{2.553420in}}{\pgfqpoint{2.340334in}{2.547834in}}{\pgfqpoint{2.344452in}{2.543716in}}%
\pgfpathcurveto{\pgfqpoint{2.348570in}{2.539598in}}{\pgfqpoint{2.354156in}{2.537284in}}{\pgfqpoint{2.359980in}{2.537284in}}%
\pgfpathlineto{\pgfqpoint{2.359980in}{2.537284in}}%
\pgfpathclose%
\pgfusepath{stroke,fill}%
\end{pgfscope}%
\begin{pgfscope}%
\pgfpathrectangle{\pgfqpoint{1.073501in}{0.880000in}}{\pgfqpoint{6.052998in}{6.160000in}}%
\pgfusepath{clip}%
\pgfsetbuttcap%
\pgfsetroundjoin%
\definecolor{currentfill}{rgb}{0.800000,0.200000,0.200000}%
\pgfsetfillcolor{currentfill}%
\pgfsetlinewidth{1.003750pt}%
\definecolor{currentstroke}{rgb}{0.800000,0.200000,0.200000}%
\pgfsetstrokecolor{currentstroke}%
\pgfsetdash{}{0pt}%
\pgfpathmoveto{\pgfqpoint{2.387983in}{2.371181in}}%
\pgfpathcurveto{\pgfqpoint{2.393807in}{2.371181in}}{\pgfqpoint{2.399394in}{2.373494in}}{\pgfqpoint{2.403512in}{2.377613in}}%
\pgfpathcurveto{\pgfqpoint{2.407630in}{2.381731in}}{\pgfqpoint{2.409944in}{2.387317in}}{\pgfqpoint{2.409944in}{2.393141in}}%
\pgfpathcurveto{\pgfqpoint{2.409944in}{2.398965in}}{\pgfqpoint{2.407630in}{2.404551in}}{\pgfqpoint{2.403512in}{2.408669in}}%
\pgfpathcurveto{\pgfqpoint{2.399394in}{2.412787in}}{\pgfqpoint{2.393807in}{2.415101in}}{\pgfqpoint{2.387983in}{2.415101in}}%
\pgfpathcurveto{\pgfqpoint{2.382160in}{2.415101in}}{\pgfqpoint{2.376573in}{2.412787in}}{\pgfqpoint{2.372455in}{2.408669in}}%
\pgfpathcurveto{\pgfqpoint{2.368337in}{2.404551in}}{\pgfqpoint{2.366023in}{2.398965in}}{\pgfqpoint{2.366023in}{2.393141in}}%
\pgfpathcurveto{\pgfqpoint{2.366023in}{2.387317in}}{\pgfqpoint{2.368337in}{2.381731in}}{\pgfqpoint{2.372455in}{2.377613in}}%
\pgfpathcurveto{\pgfqpoint{2.376573in}{2.373494in}}{\pgfqpoint{2.382160in}{2.371181in}}{\pgfqpoint{2.387983in}{2.371181in}}%
\pgfpathlineto{\pgfqpoint{2.387983in}{2.371181in}}%
\pgfpathclose%
\pgfusepath{stroke,fill}%
\end{pgfscope}%
\begin{pgfscope}%
\pgfpathrectangle{\pgfqpoint{1.073501in}{0.880000in}}{\pgfqpoint{6.052998in}{6.160000in}}%
\pgfusepath{clip}%
\pgfsetbuttcap%
\pgfsetroundjoin%
\definecolor{currentfill}{rgb}{0.800000,0.200000,0.200000}%
\pgfsetfillcolor{currentfill}%
\pgfsetlinewidth{1.003750pt}%
\definecolor{currentstroke}{rgb}{0.800000,0.200000,0.200000}%
\pgfsetstrokecolor{currentstroke}%
\pgfsetdash{}{0pt}%
\pgfpathmoveto{\pgfqpoint{2.583632in}{2.359151in}}%
\pgfpathcurveto{\pgfqpoint{2.589456in}{2.359151in}}{\pgfqpoint{2.595042in}{2.361465in}}{\pgfqpoint{2.599160in}{2.365583in}}%
\pgfpathcurveto{\pgfqpoint{2.603279in}{2.369701in}}{\pgfqpoint{2.605592in}{2.375287in}}{\pgfqpoint{2.605592in}{2.381111in}}%
\pgfpathcurveto{\pgfqpoint{2.605592in}{2.386935in}}{\pgfqpoint{2.603279in}{2.392521in}}{\pgfqpoint{2.599160in}{2.396639in}}%
\pgfpathcurveto{\pgfqpoint{2.595042in}{2.400757in}}{\pgfqpoint{2.589456in}{2.403071in}}{\pgfqpoint{2.583632in}{2.403071in}}%
\pgfpathcurveto{\pgfqpoint{2.577808in}{2.403071in}}{\pgfqpoint{2.572222in}{2.400757in}}{\pgfqpoint{2.568104in}{2.396639in}}%
\pgfpathcurveto{\pgfqpoint{2.563986in}{2.392521in}}{\pgfqpoint{2.561672in}{2.386935in}}{\pgfqpoint{2.561672in}{2.381111in}}%
\pgfpathcurveto{\pgfqpoint{2.561672in}{2.375287in}}{\pgfqpoint{2.563986in}{2.369701in}}{\pgfqpoint{2.568104in}{2.365583in}}%
\pgfpathcurveto{\pgfqpoint{2.572222in}{2.361465in}}{\pgfqpoint{2.577808in}{2.359151in}}{\pgfqpoint{2.583632in}{2.359151in}}%
\pgfpathlineto{\pgfqpoint{2.583632in}{2.359151in}}%
\pgfpathclose%
\pgfusepath{stroke,fill}%
\end{pgfscope}%
\begin{pgfscope}%
\pgfpathrectangle{\pgfqpoint{1.073501in}{0.880000in}}{\pgfqpoint{6.052998in}{6.160000in}}%
\pgfusepath{clip}%
\pgfsetbuttcap%
\pgfsetroundjoin%
\definecolor{currentfill}{rgb}{0.800000,0.200000,0.200000}%
\pgfsetfillcolor{currentfill}%
\pgfsetlinewidth{1.003750pt}%
\definecolor{currentstroke}{rgb}{0.800000,0.200000,0.200000}%
\pgfsetstrokecolor{currentstroke}%
\pgfsetdash{}{0pt}%
\pgfpathmoveto{\pgfqpoint{2.663717in}{2.238676in}}%
\pgfpathcurveto{\pgfqpoint{2.669541in}{2.238676in}}{\pgfqpoint{2.675127in}{2.240990in}}{\pgfqpoint{2.679245in}{2.245108in}}%
\pgfpathcurveto{\pgfqpoint{2.683363in}{2.249226in}}{\pgfqpoint{2.685677in}{2.254812in}}{\pgfqpoint{2.685677in}{2.260636in}}%
\pgfpathcurveto{\pgfqpoint{2.685677in}{2.266460in}}{\pgfqpoint{2.683363in}{2.272047in}}{\pgfqpoint{2.679245in}{2.276165in}}%
\pgfpathcurveto{\pgfqpoint{2.675127in}{2.280283in}}{\pgfqpoint{2.669541in}{2.282597in}}{\pgfqpoint{2.663717in}{2.282597in}}%
\pgfpathcurveto{\pgfqpoint{2.657893in}{2.282597in}}{\pgfqpoint{2.652307in}{2.280283in}}{\pgfqpoint{2.648189in}{2.276165in}}%
\pgfpathcurveto{\pgfqpoint{2.644070in}{2.272047in}}{\pgfqpoint{2.641757in}{2.266460in}}{\pgfqpoint{2.641757in}{2.260636in}}%
\pgfpathcurveto{\pgfqpoint{2.641757in}{2.254812in}}{\pgfqpoint{2.644070in}{2.249226in}}{\pgfqpoint{2.648189in}{2.245108in}}%
\pgfpathcurveto{\pgfqpoint{2.652307in}{2.240990in}}{\pgfqpoint{2.657893in}{2.238676in}}{\pgfqpoint{2.663717in}{2.238676in}}%
\pgfpathlineto{\pgfqpoint{2.663717in}{2.238676in}}%
\pgfpathclose%
\pgfusepath{stroke,fill}%
\end{pgfscope}%
\begin{pgfscope}%
\pgfpathrectangle{\pgfqpoint{1.073501in}{0.880000in}}{\pgfqpoint{6.052998in}{6.160000in}}%
\pgfusepath{clip}%
\pgfsetbuttcap%
\pgfsetroundjoin%
\definecolor{currentfill}{rgb}{0.800000,0.200000,0.200000}%
\pgfsetfillcolor{currentfill}%
\pgfsetlinewidth{1.003750pt}%
\definecolor{currentstroke}{rgb}{0.800000,0.200000,0.200000}%
\pgfsetstrokecolor{currentstroke}%
\pgfsetdash{}{0pt}%
\pgfpathmoveto{\pgfqpoint{2.771394in}{2.146372in}}%
\pgfpathcurveto{\pgfqpoint{2.777218in}{2.146372in}}{\pgfqpoint{2.782804in}{2.148686in}}{\pgfqpoint{2.786922in}{2.152804in}}%
\pgfpathcurveto{\pgfqpoint{2.791040in}{2.156922in}}{\pgfqpoint{2.793354in}{2.162508in}}{\pgfqpoint{2.793354in}{2.168332in}}%
\pgfpathcurveto{\pgfqpoint{2.793354in}{2.174156in}}{\pgfqpoint{2.791040in}{2.179742in}}{\pgfqpoint{2.786922in}{2.183860in}}%
\pgfpathcurveto{\pgfqpoint{2.782804in}{2.187978in}}{\pgfqpoint{2.777218in}{2.190292in}}{\pgfqpoint{2.771394in}{2.190292in}}%
\pgfpathcurveto{\pgfqpoint{2.765570in}{2.190292in}}{\pgfqpoint{2.759984in}{2.187978in}}{\pgfqpoint{2.755866in}{2.183860in}}%
\pgfpathcurveto{\pgfqpoint{2.751748in}{2.179742in}}{\pgfqpoint{2.749434in}{2.174156in}}{\pgfqpoint{2.749434in}{2.168332in}}%
\pgfpathcurveto{\pgfqpoint{2.749434in}{2.162508in}}{\pgfqpoint{2.751748in}{2.156922in}}{\pgfqpoint{2.755866in}{2.152804in}}%
\pgfpathcurveto{\pgfqpoint{2.759984in}{2.148686in}}{\pgfqpoint{2.765570in}{2.146372in}}{\pgfqpoint{2.771394in}{2.146372in}}%
\pgfpathlineto{\pgfqpoint{2.771394in}{2.146372in}}%
\pgfpathclose%
\pgfusepath{stroke,fill}%
\end{pgfscope}%
\begin{pgfscope}%
\pgfpathrectangle{\pgfqpoint{1.073501in}{0.880000in}}{\pgfqpoint{6.052998in}{6.160000in}}%
\pgfusepath{clip}%
\pgfsetbuttcap%
\pgfsetroundjoin%
\definecolor{currentfill}{rgb}{0.800000,0.200000,0.200000}%
\pgfsetfillcolor{currentfill}%
\pgfsetlinewidth{1.003750pt}%
\definecolor{currentstroke}{rgb}{0.800000,0.200000,0.200000}%
\pgfsetstrokecolor{currentstroke}%
\pgfsetdash{}{0pt}%
\pgfpathmoveto{\pgfqpoint{2.911554in}{2.101004in}}%
\pgfpathcurveto{\pgfqpoint{2.917378in}{2.101004in}}{\pgfqpoint{2.922964in}{2.103317in}}{\pgfqpoint{2.927082in}{2.107436in}}%
\pgfpathcurveto{\pgfqpoint{2.931200in}{2.111554in}}{\pgfqpoint{2.933514in}{2.117140in}}{\pgfqpoint{2.933514in}{2.122964in}}%
\pgfpathcurveto{\pgfqpoint{2.933514in}{2.128788in}}{\pgfqpoint{2.931200in}{2.134374in}}{\pgfqpoint{2.927082in}{2.138492in}}%
\pgfpathcurveto{\pgfqpoint{2.922964in}{2.142610in}}{\pgfqpoint{2.917378in}{2.144924in}}{\pgfqpoint{2.911554in}{2.144924in}}%
\pgfpathcurveto{\pgfqpoint{2.905730in}{2.144924in}}{\pgfqpoint{2.900144in}{2.142610in}}{\pgfqpoint{2.896026in}{2.138492in}}%
\pgfpathcurveto{\pgfqpoint{2.891908in}{2.134374in}}{\pgfqpoint{2.889594in}{2.128788in}}{\pgfqpoint{2.889594in}{2.122964in}}%
\pgfpathcurveto{\pgfqpoint{2.889594in}{2.117140in}}{\pgfqpoint{2.891908in}{2.111554in}}{\pgfqpoint{2.896026in}{2.107436in}}%
\pgfpathcurveto{\pgfqpoint{2.900144in}{2.103317in}}{\pgfqpoint{2.905730in}{2.101004in}}{\pgfqpoint{2.911554in}{2.101004in}}%
\pgfpathlineto{\pgfqpoint{2.911554in}{2.101004in}}%
\pgfpathclose%
\pgfusepath{stroke,fill}%
\end{pgfscope}%
\begin{pgfscope}%
\pgfpathrectangle{\pgfqpoint{1.073501in}{0.880000in}}{\pgfqpoint{6.052998in}{6.160000in}}%
\pgfusepath{clip}%
\pgfsetbuttcap%
\pgfsetroundjoin%
\definecolor{currentfill}{rgb}{0.800000,0.200000,0.200000}%
\pgfsetfillcolor{currentfill}%
\pgfsetlinewidth{1.003750pt}%
\definecolor{currentstroke}{rgb}{0.800000,0.200000,0.200000}%
\pgfsetstrokecolor{currentstroke}%
\pgfsetdash{}{0pt}%
\pgfpathmoveto{\pgfqpoint{2.981556in}{1.944232in}}%
\pgfpathcurveto{\pgfqpoint{2.987380in}{1.944232in}}{\pgfqpoint{2.992966in}{1.946545in}}{\pgfqpoint{2.997084in}{1.950664in}}%
\pgfpathcurveto{\pgfqpoint{3.001202in}{1.954782in}}{\pgfqpoint{3.003516in}{1.960368in}}{\pgfqpoint{3.003516in}{1.966192in}}%
\pgfpathcurveto{\pgfqpoint{3.003516in}{1.972016in}}{\pgfqpoint{3.001202in}{1.977602in}}{\pgfqpoint{2.997084in}{1.981720in}}%
\pgfpathcurveto{\pgfqpoint{2.992966in}{1.985838in}}{\pgfqpoint{2.987380in}{1.988152in}}{\pgfqpoint{2.981556in}{1.988152in}}%
\pgfpathcurveto{\pgfqpoint{2.975732in}{1.988152in}}{\pgfqpoint{2.970146in}{1.985838in}}{\pgfqpoint{2.966028in}{1.981720in}}%
\pgfpathcurveto{\pgfqpoint{2.961910in}{1.977602in}}{\pgfqpoint{2.959596in}{1.972016in}}{\pgfqpoint{2.959596in}{1.966192in}}%
\pgfpathcurveto{\pgfqpoint{2.959596in}{1.960368in}}{\pgfqpoint{2.961910in}{1.954782in}}{\pgfqpoint{2.966028in}{1.950664in}}%
\pgfpathcurveto{\pgfqpoint{2.970146in}{1.946545in}}{\pgfqpoint{2.975732in}{1.944232in}}{\pgfqpoint{2.981556in}{1.944232in}}%
\pgfpathlineto{\pgfqpoint{2.981556in}{1.944232in}}%
\pgfpathclose%
\pgfusepath{stroke,fill}%
\end{pgfscope}%
\begin{pgfscope}%
\pgfpathrectangle{\pgfqpoint{1.073501in}{0.880000in}}{\pgfqpoint{6.052998in}{6.160000in}}%
\pgfusepath{clip}%
\pgfsetbuttcap%
\pgfsetroundjoin%
\definecolor{currentfill}{rgb}{0.800000,0.200000,0.200000}%
\pgfsetfillcolor{currentfill}%
\pgfsetlinewidth{1.003750pt}%
\definecolor{currentstroke}{rgb}{0.800000,0.200000,0.200000}%
\pgfsetstrokecolor{currentstroke}%
\pgfsetdash{}{0pt}%
\pgfpathmoveto{\pgfqpoint{3.124871in}{1.906600in}}%
\pgfpathcurveto{\pgfqpoint{3.130695in}{1.906600in}}{\pgfqpoint{3.136281in}{1.908914in}}{\pgfqpoint{3.140400in}{1.913032in}}%
\pgfpathcurveto{\pgfqpoint{3.144518in}{1.917150in}}{\pgfqpoint{3.146832in}{1.922736in}}{\pgfqpoint{3.146832in}{1.928560in}}%
\pgfpathcurveto{\pgfqpoint{3.146832in}{1.934384in}}{\pgfqpoint{3.144518in}{1.939970in}}{\pgfqpoint{3.140400in}{1.944088in}}%
\pgfpathcurveto{\pgfqpoint{3.136281in}{1.948207in}}{\pgfqpoint{3.130695in}{1.950520in}}{\pgfqpoint{3.124871in}{1.950520in}}%
\pgfpathcurveto{\pgfqpoint{3.119047in}{1.950520in}}{\pgfqpoint{3.113461in}{1.948207in}}{\pgfqpoint{3.109343in}{1.944088in}}%
\pgfpathcurveto{\pgfqpoint{3.105225in}{1.939970in}}{\pgfqpoint{3.102911in}{1.934384in}}{\pgfqpoint{3.102911in}{1.928560in}}%
\pgfpathcurveto{\pgfqpoint{3.102911in}{1.922736in}}{\pgfqpoint{3.105225in}{1.917150in}}{\pgfqpoint{3.109343in}{1.913032in}}%
\pgfpathcurveto{\pgfqpoint{3.113461in}{1.908914in}}{\pgfqpoint{3.119047in}{1.906600in}}{\pgfqpoint{3.124871in}{1.906600in}}%
\pgfpathlineto{\pgfqpoint{3.124871in}{1.906600in}}%
\pgfpathclose%
\pgfusepath{stroke,fill}%
\end{pgfscope}%
\begin{pgfscope}%
\pgfpathrectangle{\pgfqpoint{1.073501in}{0.880000in}}{\pgfqpoint{6.052998in}{6.160000in}}%
\pgfusepath{clip}%
\pgfsetbuttcap%
\pgfsetroundjoin%
\definecolor{currentfill}{rgb}{0.800000,0.200000,0.200000}%
\pgfsetfillcolor{currentfill}%
\pgfsetlinewidth{1.003750pt}%
\definecolor{currentstroke}{rgb}{0.800000,0.200000,0.200000}%
\pgfsetstrokecolor{currentstroke}%
\pgfsetdash{}{0pt}%
\pgfpathmoveto{\pgfqpoint{3.264264in}{1.869428in}}%
\pgfpathcurveto{\pgfqpoint{3.270088in}{1.869428in}}{\pgfqpoint{3.275674in}{1.871742in}}{\pgfqpoint{3.279792in}{1.875860in}}%
\pgfpathcurveto{\pgfqpoint{3.283911in}{1.879979in}}{\pgfqpoint{3.286224in}{1.885565in}}{\pgfqpoint{3.286224in}{1.891389in}}%
\pgfpathcurveto{\pgfqpoint{3.286224in}{1.897213in}}{\pgfqpoint{3.283911in}{1.902799in}}{\pgfqpoint{3.279792in}{1.906917in}}%
\pgfpathcurveto{\pgfqpoint{3.275674in}{1.911035in}}{\pgfqpoint{3.270088in}{1.913349in}}{\pgfqpoint{3.264264in}{1.913349in}}%
\pgfpathcurveto{\pgfqpoint{3.258440in}{1.913349in}}{\pgfqpoint{3.252854in}{1.911035in}}{\pgfqpoint{3.248736in}{1.906917in}}%
\pgfpathcurveto{\pgfqpoint{3.244618in}{1.902799in}}{\pgfqpoint{3.242304in}{1.897213in}}{\pgfqpoint{3.242304in}{1.891389in}}%
\pgfpathcurveto{\pgfqpoint{3.242304in}{1.885565in}}{\pgfqpoint{3.244618in}{1.879979in}}{\pgfqpoint{3.248736in}{1.875860in}}%
\pgfpathcurveto{\pgfqpoint{3.252854in}{1.871742in}}{\pgfqpoint{3.258440in}{1.869428in}}{\pgfqpoint{3.264264in}{1.869428in}}%
\pgfpathlineto{\pgfqpoint{3.264264in}{1.869428in}}%
\pgfpathclose%
\pgfusepath{stroke,fill}%
\end{pgfscope}%
\begin{pgfscope}%
\pgfpathrectangle{\pgfqpoint{1.073501in}{0.880000in}}{\pgfqpoint{6.052998in}{6.160000in}}%
\pgfusepath{clip}%
\pgfsetbuttcap%
\pgfsetroundjoin%
\definecolor{currentfill}{rgb}{0.800000,0.200000,0.200000}%
\pgfsetfillcolor{currentfill}%
\pgfsetlinewidth{1.003750pt}%
\definecolor{currentstroke}{rgb}{0.800000,0.200000,0.200000}%
\pgfsetstrokecolor{currentstroke}%
\pgfsetdash{}{0pt}%
\pgfpathmoveto{\pgfqpoint{3.385163in}{1.786412in}}%
\pgfpathcurveto{\pgfqpoint{3.390987in}{1.786412in}}{\pgfqpoint{3.396573in}{1.788725in}}{\pgfqpoint{3.400691in}{1.792844in}}%
\pgfpathcurveto{\pgfqpoint{3.404809in}{1.796962in}}{\pgfqpoint{3.407123in}{1.802548in}}{\pgfqpoint{3.407123in}{1.808372in}}%
\pgfpathcurveto{\pgfqpoint{3.407123in}{1.814196in}}{\pgfqpoint{3.404809in}{1.819782in}}{\pgfqpoint{3.400691in}{1.823900in}}%
\pgfpathcurveto{\pgfqpoint{3.396573in}{1.828018in}}{\pgfqpoint{3.390987in}{1.830332in}}{\pgfqpoint{3.385163in}{1.830332in}}%
\pgfpathcurveto{\pgfqpoint{3.379339in}{1.830332in}}{\pgfqpoint{3.373753in}{1.828018in}}{\pgfqpoint{3.369635in}{1.823900in}}%
\pgfpathcurveto{\pgfqpoint{3.365517in}{1.819782in}}{\pgfqpoint{3.363203in}{1.814196in}}{\pgfqpoint{3.363203in}{1.808372in}}%
\pgfpathcurveto{\pgfqpoint{3.363203in}{1.802548in}}{\pgfqpoint{3.365517in}{1.796962in}}{\pgfqpoint{3.369635in}{1.792844in}}%
\pgfpathcurveto{\pgfqpoint{3.373753in}{1.788725in}}{\pgfqpoint{3.379339in}{1.786412in}}{\pgfqpoint{3.385163in}{1.786412in}}%
\pgfpathlineto{\pgfqpoint{3.385163in}{1.786412in}}%
\pgfpathclose%
\pgfusepath{stroke,fill}%
\end{pgfscope}%
\begin{pgfscope}%
\pgfpathrectangle{\pgfqpoint{1.073501in}{0.880000in}}{\pgfqpoint{6.052998in}{6.160000in}}%
\pgfusepath{clip}%
\pgfsetbuttcap%
\pgfsetroundjoin%
\definecolor{currentfill}{rgb}{0.800000,0.200000,0.200000}%
\pgfsetfillcolor{currentfill}%
\pgfsetlinewidth{1.003750pt}%
\definecolor{currentstroke}{rgb}{0.800000,0.200000,0.200000}%
\pgfsetstrokecolor{currentstroke}%
\pgfsetdash{}{0pt}%
\pgfpathmoveto{\pgfqpoint{3.509596in}{1.697657in}}%
\pgfpathcurveto{\pgfqpoint{3.515420in}{1.697657in}}{\pgfqpoint{3.521006in}{1.699971in}}{\pgfqpoint{3.525125in}{1.704089in}}%
\pgfpathcurveto{\pgfqpoint{3.529243in}{1.708207in}}{\pgfqpoint{3.531557in}{1.713794in}}{\pgfqpoint{3.531557in}{1.719618in}}%
\pgfpathcurveto{\pgfqpoint{3.531557in}{1.725441in}}{\pgfqpoint{3.529243in}{1.731028in}}{\pgfqpoint{3.525125in}{1.735146in}}%
\pgfpathcurveto{\pgfqpoint{3.521006in}{1.739264in}}{\pgfqpoint{3.515420in}{1.741578in}}{\pgfqpoint{3.509596in}{1.741578in}}%
\pgfpathcurveto{\pgfqpoint{3.503772in}{1.741578in}}{\pgfqpoint{3.498186in}{1.739264in}}{\pgfqpoint{3.494068in}{1.735146in}}%
\pgfpathcurveto{\pgfqpoint{3.489950in}{1.731028in}}{\pgfqpoint{3.487636in}{1.725441in}}{\pgfqpoint{3.487636in}{1.719618in}}%
\pgfpathcurveto{\pgfqpoint{3.487636in}{1.713794in}}{\pgfqpoint{3.489950in}{1.708207in}}{\pgfqpoint{3.494068in}{1.704089in}}%
\pgfpathcurveto{\pgfqpoint{3.498186in}{1.699971in}}{\pgfqpoint{3.503772in}{1.697657in}}{\pgfqpoint{3.509596in}{1.697657in}}%
\pgfpathlineto{\pgfqpoint{3.509596in}{1.697657in}}%
\pgfpathclose%
\pgfusepath{stroke,fill}%
\end{pgfscope}%
\begin{pgfscope}%
\pgfpathrectangle{\pgfqpoint{1.073501in}{0.880000in}}{\pgfqpoint{6.052998in}{6.160000in}}%
\pgfusepath{clip}%
\pgfsetbuttcap%
\pgfsetroundjoin%
\definecolor{currentfill}{rgb}{0.800000,0.200000,0.200000}%
\pgfsetfillcolor{currentfill}%
\pgfsetlinewidth{1.003750pt}%
\definecolor{currentstroke}{rgb}{0.800000,0.200000,0.200000}%
\pgfsetstrokecolor{currentstroke}%
\pgfsetdash{}{0pt}%
\pgfpathmoveto{\pgfqpoint{3.667823in}{1.737959in}}%
\pgfpathcurveto{\pgfqpoint{3.673647in}{1.737959in}}{\pgfqpoint{3.679233in}{1.740273in}}{\pgfqpoint{3.683351in}{1.744391in}}%
\pgfpathcurveto{\pgfqpoint{3.687469in}{1.748509in}}{\pgfqpoint{3.689783in}{1.754095in}}{\pgfqpoint{3.689783in}{1.759919in}}%
\pgfpathcurveto{\pgfqpoint{3.689783in}{1.765743in}}{\pgfqpoint{3.687469in}{1.771329in}}{\pgfqpoint{3.683351in}{1.775447in}}%
\pgfpathcurveto{\pgfqpoint{3.679233in}{1.779566in}}{\pgfqpoint{3.673647in}{1.781879in}}{\pgfqpoint{3.667823in}{1.781879in}}%
\pgfpathcurveto{\pgfqpoint{3.661999in}{1.781879in}}{\pgfqpoint{3.656413in}{1.779566in}}{\pgfqpoint{3.652294in}{1.775447in}}%
\pgfpathcurveto{\pgfqpoint{3.648176in}{1.771329in}}{\pgfqpoint{3.645862in}{1.765743in}}{\pgfqpoint{3.645862in}{1.759919in}}%
\pgfpathcurveto{\pgfqpoint{3.645862in}{1.754095in}}{\pgfqpoint{3.648176in}{1.748509in}}{\pgfqpoint{3.652294in}{1.744391in}}%
\pgfpathcurveto{\pgfqpoint{3.656413in}{1.740273in}}{\pgfqpoint{3.661999in}{1.737959in}}{\pgfqpoint{3.667823in}{1.737959in}}%
\pgfpathlineto{\pgfqpoint{3.667823in}{1.737959in}}%
\pgfpathclose%
\pgfusepath{stroke,fill}%
\end{pgfscope}%
\begin{pgfscope}%
\pgfpathrectangle{\pgfqpoint{1.073501in}{0.880000in}}{\pgfqpoint{6.052998in}{6.160000in}}%
\pgfusepath{clip}%
\pgfsetbuttcap%
\pgfsetroundjoin%
\definecolor{currentfill}{rgb}{0.800000,0.200000,0.200000}%
\pgfsetfillcolor{currentfill}%
\pgfsetlinewidth{1.003750pt}%
\definecolor{currentstroke}{rgb}{0.800000,0.200000,0.200000}%
\pgfsetstrokecolor{currentstroke}%
\pgfsetdash{}{0pt}%
\pgfpathmoveto{\pgfqpoint{3.785051in}{1.557157in}}%
\pgfpathcurveto{\pgfqpoint{3.790875in}{1.557157in}}{\pgfqpoint{3.796461in}{1.559471in}}{\pgfqpoint{3.800579in}{1.563589in}}%
\pgfpathcurveto{\pgfqpoint{3.804697in}{1.567707in}}{\pgfqpoint{3.807011in}{1.573293in}}{\pgfqpoint{3.807011in}{1.579117in}}%
\pgfpathcurveto{\pgfqpoint{3.807011in}{1.584941in}}{\pgfqpoint{3.804697in}{1.590527in}}{\pgfqpoint{3.800579in}{1.594646in}}%
\pgfpathcurveto{\pgfqpoint{3.796461in}{1.598764in}}{\pgfqpoint{3.790875in}{1.601078in}}{\pgfqpoint{3.785051in}{1.601078in}}%
\pgfpathcurveto{\pgfqpoint{3.779227in}{1.601078in}}{\pgfqpoint{3.773641in}{1.598764in}}{\pgfqpoint{3.769523in}{1.594646in}}%
\pgfpathcurveto{\pgfqpoint{3.765405in}{1.590527in}}{\pgfqpoint{3.763091in}{1.584941in}}{\pgfqpoint{3.763091in}{1.579117in}}%
\pgfpathcurveto{\pgfqpoint{3.763091in}{1.573293in}}{\pgfqpoint{3.765405in}{1.567707in}}{\pgfqpoint{3.769523in}{1.563589in}}%
\pgfpathcurveto{\pgfqpoint{3.773641in}{1.559471in}}{\pgfqpoint{3.779227in}{1.557157in}}{\pgfqpoint{3.785051in}{1.557157in}}%
\pgfpathlineto{\pgfqpoint{3.785051in}{1.557157in}}%
\pgfpathclose%
\pgfusepath{stroke,fill}%
\end{pgfscope}%
\begin{pgfscope}%
\pgfpathrectangle{\pgfqpoint{1.073501in}{0.880000in}}{\pgfqpoint{6.052998in}{6.160000in}}%
\pgfusepath{clip}%
\pgfsetbuttcap%
\pgfsetroundjoin%
\definecolor{currentfill}{rgb}{0.800000,0.200000,0.200000}%
\pgfsetfillcolor{currentfill}%
\pgfsetlinewidth{1.003750pt}%
\definecolor{currentstroke}{rgb}{0.800000,0.200000,0.200000}%
\pgfsetstrokecolor{currentstroke}%
\pgfsetdash{}{0pt}%
\pgfpathmoveto{\pgfqpoint{3.943284in}{1.632390in}}%
\pgfpathcurveto{\pgfqpoint{3.949108in}{1.632390in}}{\pgfqpoint{3.954694in}{1.634704in}}{\pgfqpoint{3.958812in}{1.638822in}}%
\pgfpathcurveto{\pgfqpoint{3.962930in}{1.642940in}}{\pgfqpoint{3.965244in}{1.648526in}}{\pgfqpoint{3.965244in}{1.654350in}}%
\pgfpathcurveto{\pgfqpoint{3.965244in}{1.660174in}}{\pgfqpoint{3.962930in}{1.665760in}}{\pgfqpoint{3.958812in}{1.669878in}}%
\pgfpathcurveto{\pgfqpoint{3.954694in}{1.673996in}}{\pgfqpoint{3.949108in}{1.676310in}}{\pgfqpoint{3.943284in}{1.676310in}}%
\pgfpathcurveto{\pgfqpoint{3.937460in}{1.676310in}}{\pgfqpoint{3.931874in}{1.673996in}}{\pgfqpoint{3.927755in}{1.669878in}}%
\pgfpathcurveto{\pgfqpoint{3.923637in}{1.665760in}}{\pgfqpoint{3.921323in}{1.660174in}}{\pgfqpoint{3.921323in}{1.654350in}}%
\pgfpathcurveto{\pgfqpoint{3.921323in}{1.648526in}}{\pgfqpoint{3.923637in}{1.642940in}}{\pgfqpoint{3.927755in}{1.638822in}}%
\pgfpathcurveto{\pgfqpoint{3.931874in}{1.634704in}}{\pgfqpoint{3.937460in}{1.632390in}}{\pgfqpoint{3.943284in}{1.632390in}}%
\pgfpathlineto{\pgfqpoint{3.943284in}{1.632390in}}%
\pgfpathclose%
\pgfusepath{stroke,fill}%
\end{pgfscope}%
\begin{pgfscope}%
\pgfpathrectangle{\pgfqpoint{1.073501in}{0.880000in}}{\pgfqpoint{6.052998in}{6.160000in}}%
\pgfusepath{clip}%
\pgfsetbuttcap%
\pgfsetroundjoin%
\definecolor{currentfill}{rgb}{0.800000,0.200000,0.200000}%
\pgfsetfillcolor{currentfill}%
\pgfsetlinewidth{1.003750pt}%
\definecolor{currentstroke}{rgb}{0.800000,0.200000,0.200000}%
\pgfsetstrokecolor{currentstroke}%
\pgfsetdash{}{0pt}%
\pgfpathmoveto{\pgfqpoint{4.089480in}{1.648943in}}%
\pgfpathcurveto{\pgfqpoint{4.095304in}{1.648943in}}{\pgfqpoint{4.100890in}{1.651257in}}{\pgfqpoint{4.105008in}{1.655375in}}%
\pgfpathcurveto{\pgfqpoint{4.109126in}{1.659493in}}{\pgfqpoint{4.111440in}{1.665080in}}{\pgfqpoint{4.111440in}{1.670904in}}%
\pgfpathcurveto{\pgfqpoint{4.111440in}{1.676727in}}{\pgfqpoint{4.109126in}{1.682314in}}{\pgfqpoint{4.105008in}{1.686432in}}%
\pgfpathcurveto{\pgfqpoint{4.100890in}{1.690550in}}{\pgfqpoint{4.095304in}{1.692864in}}{\pgfqpoint{4.089480in}{1.692864in}}%
\pgfpathcurveto{\pgfqpoint{4.083656in}{1.692864in}}{\pgfqpoint{4.078070in}{1.690550in}}{\pgfqpoint{4.073952in}{1.686432in}}%
\pgfpathcurveto{\pgfqpoint{4.069834in}{1.682314in}}{\pgfqpoint{4.067520in}{1.676727in}}{\pgfqpoint{4.067520in}{1.670904in}}%
\pgfpathcurveto{\pgfqpoint{4.067520in}{1.665080in}}{\pgfqpoint{4.069834in}{1.659493in}}{\pgfqpoint{4.073952in}{1.655375in}}%
\pgfpathcurveto{\pgfqpoint{4.078070in}{1.651257in}}{\pgfqpoint{4.083656in}{1.648943in}}{\pgfqpoint{4.089480in}{1.648943in}}%
\pgfpathlineto{\pgfqpoint{4.089480in}{1.648943in}}%
\pgfpathclose%
\pgfusepath{stroke,fill}%
\end{pgfscope}%
\begin{pgfscope}%
\pgfpathrectangle{\pgfqpoint{1.073501in}{0.880000in}}{\pgfqpoint{6.052998in}{6.160000in}}%
\pgfusepath{clip}%
\pgfsetbuttcap%
\pgfsetroundjoin%
\definecolor{currentfill}{rgb}{0.800000,0.200000,0.200000}%
\pgfsetfillcolor{currentfill}%
\pgfsetlinewidth{1.003750pt}%
\definecolor{currentstroke}{rgb}{0.800000,0.200000,0.200000}%
\pgfsetstrokecolor{currentstroke}%
\pgfsetdash{}{0pt}%
\pgfpathmoveto{\pgfqpoint{4.233310in}{1.664862in}}%
\pgfpathcurveto{\pgfqpoint{4.239134in}{1.664862in}}{\pgfqpoint{4.244720in}{1.667176in}}{\pgfqpoint{4.248838in}{1.671294in}}%
\pgfpathcurveto{\pgfqpoint{4.252957in}{1.675412in}}{\pgfqpoint{4.255270in}{1.680998in}}{\pgfqpoint{4.255270in}{1.686822in}}%
\pgfpathcurveto{\pgfqpoint{4.255270in}{1.692646in}}{\pgfqpoint{4.252957in}{1.698232in}}{\pgfqpoint{4.248838in}{1.702351in}}%
\pgfpathcurveto{\pgfqpoint{4.244720in}{1.706469in}}{\pgfqpoint{4.239134in}{1.708783in}}{\pgfqpoint{4.233310in}{1.708783in}}%
\pgfpathcurveto{\pgfqpoint{4.227486in}{1.708783in}}{\pgfqpoint{4.221900in}{1.706469in}}{\pgfqpoint{4.217782in}{1.702351in}}%
\pgfpathcurveto{\pgfqpoint{4.213664in}{1.698232in}}{\pgfqpoint{4.211350in}{1.692646in}}{\pgfqpoint{4.211350in}{1.686822in}}%
\pgfpathcurveto{\pgfqpoint{4.211350in}{1.680998in}}{\pgfqpoint{4.213664in}{1.675412in}}{\pgfqpoint{4.217782in}{1.671294in}}%
\pgfpathcurveto{\pgfqpoint{4.221900in}{1.667176in}}{\pgfqpoint{4.227486in}{1.664862in}}{\pgfqpoint{4.233310in}{1.664862in}}%
\pgfpathlineto{\pgfqpoint{4.233310in}{1.664862in}}%
\pgfpathclose%
\pgfusepath{stroke,fill}%
\end{pgfscope}%
\begin{pgfscope}%
\pgfpathrectangle{\pgfqpoint{1.073501in}{0.880000in}}{\pgfqpoint{6.052998in}{6.160000in}}%
\pgfusepath{clip}%
\pgfsetbuttcap%
\pgfsetroundjoin%
\definecolor{currentfill}{rgb}{0.800000,0.200000,0.200000}%
\pgfsetfillcolor{currentfill}%
\pgfsetlinewidth{1.003750pt}%
\definecolor{currentstroke}{rgb}{0.800000,0.200000,0.200000}%
\pgfsetstrokecolor{currentstroke}%
\pgfsetdash{}{0pt}%
\pgfpathmoveto{\pgfqpoint{4.388384in}{1.569598in}}%
\pgfpathcurveto{\pgfqpoint{4.394208in}{1.569598in}}{\pgfqpoint{4.399794in}{1.571912in}}{\pgfqpoint{4.403912in}{1.576031in}}%
\pgfpathcurveto{\pgfqpoint{4.408030in}{1.580149in}}{\pgfqpoint{4.410344in}{1.585735in}}{\pgfqpoint{4.410344in}{1.591559in}}%
\pgfpathcurveto{\pgfqpoint{4.410344in}{1.597383in}}{\pgfqpoint{4.408030in}{1.602969in}}{\pgfqpoint{4.403912in}{1.607087in}}%
\pgfpathcurveto{\pgfqpoint{4.399794in}{1.611205in}}{\pgfqpoint{4.394208in}{1.613519in}}{\pgfqpoint{4.388384in}{1.613519in}}%
\pgfpathcurveto{\pgfqpoint{4.382560in}{1.613519in}}{\pgfqpoint{4.376974in}{1.611205in}}{\pgfqpoint{4.372856in}{1.607087in}}%
\pgfpathcurveto{\pgfqpoint{4.368738in}{1.602969in}}{\pgfqpoint{4.366424in}{1.597383in}}{\pgfqpoint{4.366424in}{1.591559in}}%
\pgfpathcurveto{\pgfqpoint{4.366424in}{1.585735in}}{\pgfqpoint{4.368738in}{1.580149in}}{\pgfqpoint{4.372856in}{1.576031in}}%
\pgfpathcurveto{\pgfqpoint{4.376974in}{1.571912in}}{\pgfqpoint{4.382560in}{1.569598in}}{\pgfqpoint{4.388384in}{1.569598in}}%
\pgfpathlineto{\pgfqpoint{4.388384in}{1.569598in}}%
\pgfpathclose%
\pgfusepath{stroke,fill}%
\end{pgfscope}%
\begin{pgfscope}%
\pgfpathrectangle{\pgfqpoint{1.073501in}{0.880000in}}{\pgfqpoint{6.052998in}{6.160000in}}%
\pgfusepath{clip}%
\pgfsetbuttcap%
\pgfsetroundjoin%
\definecolor{currentfill}{rgb}{0.800000,0.200000,0.200000}%
\pgfsetfillcolor{currentfill}%
\pgfsetlinewidth{1.003750pt}%
\definecolor{currentstroke}{rgb}{0.800000,0.200000,0.200000}%
\pgfsetstrokecolor{currentstroke}%
\pgfsetdash{}{0pt}%
\pgfpathmoveto{\pgfqpoint{4.521217in}{1.682185in}}%
\pgfpathcurveto{\pgfqpoint{4.527041in}{1.682185in}}{\pgfqpoint{4.532627in}{1.684498in}}{\pgfqpoint{4.536745in}{1.688617in}}%
\pgfpathcurveto{\pgfqpoint{4.540863in}{1.692735in}}{\pgfqpoint{4.543177in}{1.698321in}}{\pgfqpoint{4.543177in}{1.704145in}}%
\pgfpathcurveto{\pgfqpoint{4.543177in}{1.709969in}}{\pgfqpoint{4.540863in}{1.715555in}}{\pgfqpoint{4.536745in}{1.719673in}}%
\pgfpathcurveto{\pgfqpoint{4.532627in}{1.723791in}}{\pgfqpoint{4.527041in}{1.726105in}}{\pgfqpoint{4.521217in}{1.726105in}}%
\pgfpathcurveto{\pgfqpoint{4.515393in}{1.726105in}}{\pgfqpoint{4.509807in}{1.723791in}}{\pgfqpoint{4.505689in}{1.719673in}}%
\pgfpathcurveto{\pgfqpoint{4.501571in}{1.715555in}}{\pgfqpoint{4.499257in}{1.709969in}}{\pgfqpoint{4.499257in}{1.704145in}}%
\pgfpathcurveto{\pgfqpoint{4.499257in}{1.698321in}}{\pgfqpoint{4.501571in}{1.692735in}}{\pgfqpoint{4.505689in}{1.688617in}}%
\pgfpathcurveto{\pgfqpoint{4.509807in}{1.684498in}}{\pgfqpoint{4.515393in}{1.682185in}}{\pgfqpoint{4.521217in}{1.682185in}}%
\pgfpathlineto{\pgfqpoint{4.521217in}{1.682185in}}%
\pgfpathclose%
\pgfusepath{stroke,fill}%
\end{pgfscope}%
\begin{pgfscope}%
\pgfpathrectangle{\pgfqpoint{1.073501in}{0.880000in}}{\pgfqpoint{6.052998in}{6.160000in}}%
\pgfusepath{clip}%
\pgfsetbuttcap%
\pgfsetroundjoin%
\definecolor{currentfill}{rgb}{0.800000,0.200000,0.200000}%
\pgfsetfillcolor{currentfill}%
\pgfsetlinewidth{1.003750pt}%
\definecolor{currentstroke}{rgb}{0.800000,0.200000,0.200000}%
\pgfsetstrokecolor{currentstroke}%
\pgfsetdash{}{0pt}%
\pgfpathmoveto{\pgfqpoint{4.656505in}{1.737407in}}%
\pgfpathcurveto{\pgfqpoint{4.662329in}{1.737407in}}{\pgfqpoint{4.667915in}{1.739721in}}{\pgfqpoint{4.672033in}{1.743839in}}%
\pgfpathcurveto{\pgfqpoint{4.676151in}{1.747958in}}{\pgfqpoint{4.678465in}{1.753544in}}{\pgfqpoint{4.678465in}{1.759368in}}%
\pgfpathcurveto{\pgfqpoint{4.678465in}{1.765192in}}{\pgfqpoint{4.676151in}{1.770778in}}{\pgfqpoint{4.672033in}{1.774896in}}%
\pgfpathcurveto{\pgfqpoint{4.667915in}{1.779014in}}{\pgfqpoint{4.662329in}{1.781328in}}{\pgfqpoint{4.656505in}{1.781328in}}%
\pgfpathcurveto{\pgfqpoint{4.650681in}{1.781328in}}{\pgfqpoint{4.645095in}{1.779014in}}{\pgfqpoint{4.640976in}{1.774896in}}%
\pgfpathcurveto{\pgfqpoint{4.636858in}{1.770778in}}{\pgfqpoint{4.634544in}{1.765192in}}{\pgfqpoint{4.634544in}{1.759368in}}%
\pgfpathcurveto{\pgfqpoint{4.634544in}{1.753544in}}{\pgfqpoint{4.636858in}{1.747958in}}{\pgfqpoint{4.640976in}{1.743839in}}%
\pgfpathcurveto{\pgfqpoint{4.645095in}{1.739721in}}{\pgfqpoint{4.650681in}{1.737407in}}{\pgfqpoint{4.656505in}{1.737407in}}%
\pgfpathlineto{\pgfqpoint{4.656505in}{1.737407in}}%
\pgfpathclose%
\pgfusepath{stroke,fill}%
\end{pgfscope}%
\begin{pgfscope}%
\pgfpathrectangle{\pgfqpoint{1.073501in}{0.880000in}}{\pgfqpoint{6.052998in}{6.160000in}}%
\pgfusepath{clip}%
\pgfsetbuttcap%
\pgfsetroundjoin%
\definecolor{currentfill}{rgb}{0.800000,0.200000,0.200000}%
\pgfsetfillcolor{currentfill}%
\pgfsetlinewidth{1.003750pt}%
\definecolor{currentstroke}{rgb}{0.800000,0.200000,0.200000}%
\pgfsetstrokecolor{currentstroke}%
\pgfsetdash{}{0pt}%
\pgfpathmoveto{\pgfqpoint{4.780342in}{1.820155in}}%
\pgfpathcurveto{\pgfqpoint{4.786166in}{1.820155in}}{\pgfqpoint{4.791752in}{1.822469in}}{\pgfqpoint{4.795870in}{1.826587in}}%
\pgfpathcurveto{\pgfqpoint{4.799988in}{1.830706in}}{\pgfqpoint{4.802302in}{1.836292in}}{\pgfqpoint{4.802302in}{1.842116in}}%
\pgfpathcurveto{\pgfqpoint{4.802302in}{1.847940in}}{\pgfqpoint{4.799988in}{1.853526in}}{\pgfqpoint{4.795870in}{1.857644in}}%
\pgfpathcurveto{\pgfqpoint{4.791752in}{1.861762in}}{\pgfqpoint{4.786166in}{1.864076in}}{\pgfqpoint{4.780342in}{1.864076in}}%
\pgfpathcurveto{\pgfqpoint{4.774518in}{1.864076in}}{\pgfqpoint{4.768932in}{1.861762in}}{\pgfqpoint{4.764813in}{1.857644in}}%
\pgfpathcurveto{\pgfqpoint{4.760695in}{1.853526in}}{\pgfqpoint{4.758381in}{1.847940in}}{\pgfqpoint{4.758381in}{1.842116in}}%
\pgfpathcurveto{\pgfqpoint{4.758381in}{1.836292in}}{\pgfqpoint{4.760695in}{1.830706in}}{\pgfqpoint{4.764813in}{1.826587in}}%
\pgfpathcurveto{\pgfqpoint{4.768932in}{1.822469in}}{\pgfqpoint{4.774518in}{1.820155in}}{\pgfqpoint{4.780342in}{1.820155in}}%
\pgfpathlineto{\pgfqpoint{4.780342in}{1.820155in}}%
\pgfpathclose%
\pgfusepath{stroke,fill}%
\end{pgfscope}%
\begin{pgfscope}%
\pgfpathrectangle{\pgfqpoint{1.073501in}{0.880000in}}{\pgfqpoint{6.052998in}{6.160000in}}%
\pgfusepath{clip}%
\pgfsetbuttcap%
\pgfsetroundjoin%
\definecolor{currentfill}{rgb}{0.800000,0.200000,0.200000}%
\pgfsetfillcolor{currentfill}%
\pgfsetlinewidth{1.003750pt}%
\definecolor{currentstroke}{rgb}{0.800000,0.200000,0.200000}%
\pgfsetstrokecolor{currentstroke}%
\pgfsetdash{}{0pt}%
\pgfpathmoveto{\pgfqpoint{4.938712in}{1.797496in}}%
\pgfpathcurveto{\pgfqpoint{4.944536in}{1.797496in}}{\pgfqpoint{4.950122in}{1.799810in}}{\pgfqpoint{4.954241in}{1.803928in}}%
\pgfpathcurveto{\pgfqpoint{4.958359in}{1.808046in}}{\pgfqpoint{4.960673in}{1.813632in}}{\pgfqpoint{4.960673in}{1.819456in}}%
\pgfpathcurveto{\pgfqpoint{4.960673in}{1.825280in}}{\pgfqpoint{4.958359in}{1.830866in}}{\pgfqpoint{4.954241in}{1.834984in}}%
\pgfpathcurveto{\pgfqpoint{4.950122in}{1.839102in}}{\pgfqpoint{4.944536in}{1.841416in}}{\pgfqpoint{4.938712in}{1.841416in}}%
\pgfpathcurveto{\pgfqpoint{4.932888in}{1.841416in}}{\pgfqpoint{4.927302in}{1.839102in}}{\pgfqpoint{4.923184in}{1.834984in}}%
\pgfpathcurveto{\pgfqpoint{4.919066in}{1.830866in}}{\pgfqpoint{4.916752in}{1.825280in}}{\pgfqpoint{4.916752in}{1.819456in}}%
\pgfpathcurveto{\pgfqpoint{4.916752in}{1.813632in}}{\pgfqpoint{4.919066in}{1.808046in}}{\pgfqpoint{4.923184in}{1.803928in}}%
\pgfpathcurveto{\pgfqpoint{4.927302in}{1.799810in}}{\pgfqpoint{4.932888in}{1.797496in}}{\pgfqpoint{4.938712in}{1.797496in}}%
\pgfpathlineto{\pgfqpoint{4.938712in}{1.797496in}}%
\pgfpathclose%
\pgfusepath{stroke,fill}%
\end{pgfscope}%
\begin{pgfscope}%
\pgfpathrectangle{\pgfqpoint{1.073501in}{0.880000in}}{\pgfqpoint{6.052998in}{6.160000in}}%
\pgfusepath{clip}%
\pgfsetbuttcap%
\pgfsetroundjoin%
\definecolor{currentfill}{rgb}{0.800000,0.200000,0.200000}%
\pgfsetfillcolor{currentfill}%
\pgfsetlinewidth{1.003750pt}%
\definecolor{currentstroke}{rgb}{0.800000,0.200000,0.200000}%
\pgfsetstrokecolor{currentstroke}%
\pgfsetdash{}{0pt}%
\pgfpathmoveto{\pgfqpoint{5.078265in}{1.839764in}}%
\pgfpathcurveto{\pgfqpoint{5.084089in}{1.839764in}}{\pgfqpoint{5.089675in}{1.842077in}}{\pgfqpoint{5.093793in}{1.846196in}}%
\pgfpathcurveto{\pgfqpoint{5.097912in}{1.850314in}}{\pgfqpoint{5.100225in}{1.855900in}}{\pgfqpoint{5.100225in}{1.861724in}}%
\pgfpathcurveto{\pgfqpoint{5.100225in}{1.867548in}}{\pgfqpoint{5.097912in}{1.873134in}}{\pgfqpoint{5.093793in}{1.877252in}}%
\pgfpathcurveto{\pgfqpoint{5.089675in}{1.881370in}}{\pgfqpoint{5.084089in}{1.883684in}}{\pgfqpoint{5.078265in}{1.883684in}}%
\pgfpathcurveto{\pgfqpoint{5.072441in}{1.883684in}}{\pgfqpoint{5.066855in}{1.881370in}}{\pgfqpoint{5.062737in}{1.877252in}}%
\pgfpathcurveto{\pgfqpoint{5.058619in}{1.873134in}}{\pgfqpoint{5.056305in}{1.867548in}}{\pgfqpoint{5.056305in}{1.861724in}}%
\pgfpathcurveto{\pgfqpoint{5.056305in}{1.855900in}}{\pgfqpoint{5.058619in}{1.850314in}}{\pgfqpoint{5.062737in}{1.846196in}}%
\pgfpathcurveto{\pgfqpoint{5.066855in}{1.842077in}}{\pgfqpoint{5.072441in}{1.839764in}}{\pgfqpoint{5.078265in}{1.839764in}}%
\pgfpathlineto{\pgfqpoint{5.078265in}{1.839764in}}%
\pgfpathclose%
\pgfusepath{stroke,fill}%
\end{pgfscope}%
\begin{pgfscope}%
\pgfpathrectangle{\pgfqpoint{1.073501in}{0.880000in}}{\pgfqpoint{6.052998in}{6.160000in}}%
\pgfusepath{clip}%
\pgfsetbuttcap%
\pgfsetroundjoin%
\definecolor{currentfill}{rgb}{0.800000,0.200000,0.200000}%
\pgfsetfillcolor{currentfill}%
\pgfsetlinewidth{1.003750pt}%
\definecolor{currentstroke}{rgb}{0.800000,0.200000,0.200000}%
\pgfsetstrokecolor{currentstroke}%
\pgfsetdash{}{0pt}%
\pgfpathmoveto{\pgfqpoint{5.240925in}{1.844149in}}%
\pgfpathcurveto{\pgfqpoint{5.246749in}{1.844149in}}{\pgfqpoint{5.252335in}{1.846463in}}{\pgfqpoint{5.256453in}{1.850581in}}%
\pgfpathcurveto{\pgfqpoint{5.260571in}{1.854699in}}{\pgfqpoint{5.262885in}{1.860285in}}{\pgfqpoint{5.262885in}{1.866109in}}%
\pgfpathcurveto{\pgfqpoint{5.262885in}{1.871933in}}{\pgfqpoint{5.260571in}{1.877519in}}{\pgfqpoint{5.256453in}{1.881637in}}%
\pgfpathcurveto{\pgfqpoint{5.252335in}{1.885755in}}{\pgfqpoint{5.246749in}{1.888069in}}{\pgfqpoint{5.240925in}{1.888069in}}%
\pgfpathcurveto{\pgfqpoint{5.235101in}{1.888069in}}{\pgfqpoint{5.229515in}{1.885755in}}{\pgfqpoint{5.225396in}{1.881637in}}%
\pgfpathcurveto{\pgfqpoint{5.221278in}{1.877519in}}{\pgfqpoint{5.218964in}{1.871933in}}{\pgfqpoint{5.218964in}{1.866109in}}%
\pgfpathcurveto{\pgfqpoint{5.218964in}{1.860285in}}{\pgfqpoint{5.221278in}{1.854699in}}{\pgfqpoint{5.225396in}{1.850581in}}%
\pgfpathcurveto{\pgfqpoint{5.229515in}{1.846463in}}{\pgfqpoint{5.235101in}{1.844149in}}{\pgfqpoint{5.240925in}{1.844149in}}%
\pgfpathlineto{\pgfqpoint{5.240925in}{1.844149in}}%
\pgfpathclose%
\pgfusepath{stroke,fill}%
\end{pgfscope}%
\begin{pgfscope}%
\pgfpathrectangle{\pgfqpoint{1.073501in}{0.880000in}}{\pgfqpoint{6.052998in}{6.160000in}}%
\pgfusepath{clip}%
\pgfsetbuttcap%
\pgfsetroundjoin%
\definecolor{currentfill}{rgb}{0.800000,0.200000,0.200000}%
\pgfsetfillcolor{currentfill}%
\pgfsetlinewidth{1.003750pt}%
\definecolor{currentstroke}{rgb}{0.800000,0.200000,0.200000}%
\pgfsetstrokecolor{currentstroke}%
\pgfsetdash{}{0pt}%
\pgfpathmoveto{\pgfqpoint{5.310156in}{2.016664in}}%
\pgfpathcurveto{\pgfqpoint{5.315980in}{2.016664in}}{\pgfqpoint{5.321566in}{2.018978in}}{\pgfqpoint{5.325684in}{2.023096in}}%
\pgfpathcurveto{\pgfqpoint{5.329802in}{2.027214in}}{\pgfqpoint{5.332116in}{2.032800in}}{\pgfqpoint{5.332116in}{2.038624in}}%
\pgfpathcurveto{\pgfqpoint{5.332116in}{2.044448in}}{\pgfqpoint{5.329802in}{2.050034in}}{\pgfqpoint{5.325684in}{2.054152in}}%
\pgfpathcurveto{\pgfqpoint{5.321566in}{2.058270in}}{\pgfqpoint{5.315980in}{2.060584in}}{\pgfqpoint{5.310156in}{2.060584in}}%
\pgfpathcurveto{\pgfqpoint{5.304332in}{2.060584in}}{\pgfqpoint{5.298746in}{2.058270in}}{\pgfqpoint{5.294628in}{2.054152in}}%
\pgfpathcurveto{\pgfqpoint{5.290510in}{2.050034in}}{\pgfqpoint{5.288196in}{2.044448in}}{\pgfqpoint{5.288196in}{2.038624in}}%
\pgfpathcurveto{\pgfqpoint{5.288196in}{2.032800in}}{\pgfqpoint{5.290510in}{2.027214in}}{\pgfqpoint{5.294628in}{2.023096in}}%
\pgfpathcurveto{\pgfqpoint{5.298746in}{2.018978in}}{\pgfqpoint{5.304332in}{2.016664in}}{\pgfqpoint{5.310156in}{2.016664in}}%
\pgfpathlineto{\pgfqpoint{5.310156in}{2.016664in}}%
\pgfpathclose%
\pgfusepath{stroke,fill}%
\end{pgfscope}%
\begin{pgfscope}%
\pgfpathrectangle{\pgfqpoint{1.073501in}{0.880000in}}{\pgfqpoint{6.052998in}{6.160000in}}%
\pgfusepath{clip}%
\pgfsetbuttcap%
\pgfsetroundjoin%
\definecolor{currentfill}{rgb}{0.800000,0.200000,0.200000}%
\pgfsetfillcolor{currentfill}%
\pgfsetlinewidth{1.003750pt}%
\definecolor{currentstroke}{rgb}{0.800000,0.200000,0.200000}%
\pgfsetstrokecolor{currentstroke}%
\pgfsetdash{}{0pt}%
\pgfpathmoveto{\pgfqpoint{5.444049in}{2.074300in}}%
\pgfpathcurveto{\pgfqpoint{5.449873in}{2.074300in}}{\pgfqpoint{5.455459in}{2.076614in}}{\pgfqpoint{5.459577in}{2.080732in}}%
\pgfpathcurveto{\pgfqpoint{5.463695in}{2.084850in}}{\pgfqpoint{5.466009in}{2.090436in}}{\pgfqpoint{5.466009in}{2.096260in}}%
\pgfpathcurveto{\pgfqpoint{5.466009in}{2.102084in}}{\pgfqpoint{5.463695in}{2.107670in}}{\pgfqpoint{5.459577in}{2.111788in}}%
\pgfpathcurveto{\pgfqpoint{5.455459in}{2.115906in}}{\pgfqpoint{5.449873in}{2.118220in}}{\pgfqpoint{5.444049in}{2.118220in}}%
\pgfpathcurveto{\pgfqpoint{5.438225in}{2.118220in}}{\pgfqpoint{5.432639in}{2.115906in}}{\pgfqpoint{5.428521in}{2.111788in}}%
\pgfpathcurveto{\pgfqpoint{5.424403in}{2.107670in}}{\pgfqpoint{5.422089in}{2.102084in}}{\pgfqpoint{5.422089in}{2.096260in}}%
\pgfpathcurveto{\pgfqpoint{5.422089in}{2.090436in}}{\pgfqpoint{5.424403in}{2.084850in}}{\pgfqpoint{5.428521in}{2.080732in}}%
\pgfpathcurveto{\pgfqpoint{5.432639in}{2.076614in}}{\pgfqpoint{5.438225in}{2.074300in}}{\pgfqpoint{5.444049in}{2.074300in}}%
\pgfpathlineto{\pgfqpoint{5.444049in}{2.074300in}}%
\pgfpathclose%
\pgfusepath{stroke,fill}%
\end{pgfscope}%
\begin{pgfscope}%
\pgfpathrectangle{\pgfqpoint{1.073501in}{0.880000in}}{\pgfqpoint{6.052998in}{6.160000in}}%
\pgfusepath{clip}%
\pgfsetbuttcap%
\pgfsetroundjoin%
\definecolor{currentfill}{rgb}{0.800000,0.200000,0.200000}%
\pgfsetfillcolor{currentfill}%
\pgfsetlinewidth{1.003750pt}%
\definecolor{currentstroke}{rgb}{0.800000,0.200000,0.200000}%
\pgfsetstrokecolor{currentstroke}%
\pgfsetdash{}{0pt}%
\pgfpathmoveto{\pgfqpoint{5.539242in}{2.185750in}}%
\pgfpathcurveto{\pgfqpoint{5.545066in}{2.185750in}}{\pgfqpoint{5.550652in}{2.188064in}}{\pgfqpoint{5.554771in}{2.192182in}}%
\pgfpathcurveto{\pgfqpoint{5.558889in}{2.196300in}}{\pgfqpoint{5.561203in}{2.201886in}}{\pgfqpoint{5.561203in}{2.207710in}}%
\pgfpathcurveto{\pgfqpoint{5.561203in}{2.213534in}}{\pgfqpoint{5.558889in}{2.219121in}}{\pgfqpoint{5.554771in}{2.223239in}}%
\pgfpathcurveto{\pgfqpoint{5.550652in}{2.227357in}}{\pgfqpoint{5.545066in}{2.229671in}}{\pgfqpoint{5.539242in}{2.229671in}}%
\pgfpathcurveto{\pgfqpoint{5.533418in}{2.229671in}}{\pgfqpoint{5.527832in}{2.227357in}}{\pgfqpoint{5.523714in}{2.223239in}}%
\pgfpathcurveto{\pgfqpoint{5.519596in}{2.219121in}}{\pgfqpoint{5.517282in}{2.213534in}}{\pgfqpoint{5.517282in}{2.207710in}}%
\pgfpathcurveto{\pgfqpoint{5.517282in}{2.201886in}}{\pgfqpoint{5.519596in}{2.196300in}}{\pgfqpoint{5.523714in}{2.192182in}}%
\pgfpathcurveto{\pgfqpoint{5.527832in}{2.188064in}}{\pgfqpoint{5.533418in}{2.185750in}}{\pgfqpoint{5.539242in}{2.185750in}}%
\pgfpathlineto{\pgfqpoint{5.539242in}{2.185750in}}%
\pgfpathclose%
\pgfusepath{stroke,fill}%
\end{pgfscope}%
\begin{pgfscope}%
\pgfpathrectangle{\pgfqpoint{1.073501in}{0.880000in}}{\pgfqpoint{6.052998in}{6.160000in}}%
\pgfusepath{clip}%
\pgfsetbuttcap%
\pgfsetroundjoin%
\definecolor{currentfill}{rgb}{0.800000,0.200000,0.200000}%
\pgfsetfillcolor{currentfill}%
\pgfsetlinewidth{1.003750pt}%
\definecolor{currentstroke}{rgb}{0.800000,0.200000,0.200000}%
\pgfsetstrokecolor{currentstroke}%
\pgfsetdash{}{0pt}%
\pgfpathmoveto{\pgfqpoint{5.682121in}{2.240624in}}%
\pgfpathcurveto{\pgfqpoint{5.687945in}{2.240624in}}{\pgfqpoint{5.693531in}{2.242938in}}{\pgfqpoint{5.697649in}{2.247056in}}%
\pgfpathcurveto{\pgfqpoint{5.701767in}{2.251174in}}{\pgfqpoint{5.704081in}{2.256760in}}{\pgfqpoint{5.704081in}{2.262584in}}%
\pgfpathcurveto{\pgfqpoint{5.704081in}{2.268408in}}{\pgfqpoint{5.701767in}{2.273994in}}{\pgfqpoint{5.697649in}{2.278112in}}%
\pgfpathcurveto{\pgfqpoint{5.693531in}{2.282231in}}{\pgfqpoint{5.687945in}{2.284544in}}{\pgfqpoint{5.682121in}{2.284544in}}%
\pgfpathcurveto{\pgfqpoint{5.676297in}{2.284544in}}{\pgfqpoint{5.670711in}{2.282231in}}{\pgfqpoint{5.666593in}{2.278112in}}%
\pgfpathcurveto{\pgfqpoint{5.662475in}{2.273994in}}{\pgfqpoint{5.660161in}{2.268408in}}{\pgfqpoint{5.660161in}{2.262584in}}%
\pgfpathcurveto{\pgfqpoint{5.660161in}{2.256760in}}{\pgfqpoint{5.662475in}{2.251174in}}{\pgfqpoint{5.666593in}{2.247056in}}%
\pgfpathcurveto{\pgfqpoint{5.670711in}{2.242938in}}{\pgfqpoint{5.676297in}{2.240624in}}{\pgfqpoint{5.682121in}{2.240624in}}%
\pgfpathlineto{\pgfqpoint{5.682121in}{2.240624in}}%
\pgfpathclose%
\pgfusepath{stroke,fill}%
\end{pgfscope}%
\begin{pgfscope}%
\pgfpathrectangle{\pgfqpoint{1.073501in}{0.880000in}}{\pgfqpoint{6.052998in}{6.160000in}}%
\pgfusepath{clip}%
\pgfsetbuttcap%
\pgfsetroundjoin%
\definecolor{currentfill}{rgb}{0.800000,0.200000,0.200000}%
\pgfsetfillcolor{currentfill}%
\pgfsetlinewidth{1.003750pt}%
\definecolor{currentstroke}{rgb}{0.800000,0.200000,0.200000}%
\pgfsetstrokecolor{currentstroke}%
\pgfsetdash{}{0pt}%
\pgfpathmoveto{\pgfqpoint{5.788132in}{2.340577in}}%
\pgfpathcurveto{\pgfqpoint{5.793956in}{2.340577in}}{\pgfqpoint{5.799542in}{2.342891in}}{\pgfqpoint{5.803661in}{2.347009in}}%
\pgfpathcurveto{\pgfqpoint{5.807779in}{2.351127in}}{\pgfqpoint{5.810093in}{2.356713in}}{\pgfqpoint{5.810093in}{2.362537in}}%
\pgfpathcurveto{\pgfqpoint{5.810093in}{2.368361in}}{\pgfqpoint{5.807779in}{2.373947in}}{\pgfqpoint{5.803661in}{2.378065in}}%
\pgfpathcurveto{\pgfqpoint{5.799542in}{2.382183in}}{\pgfqpoint{5.793956in}{2.384497in}}{\pgfqpoint{5.788132in}{2.384497in}}%
\pgfpathcurveto{\pgfqpoint{5.782308in}{2.384497in}}{\pgfqpoint{5.776722in}{2.382183in}}{\pgfqpoint{5.772604in}{2.378065in}}%
\pgfpathcurveto{\pgfqpoint{5.768486in}{2.373947in}}{\pgfqpoint{5.766172in}{2.368361in}}{\pgfqpoint{5.766172in}{2.362537in}}%
\pgfpathcurveto{\pgfqpoint{5.766172in}{2.356713in}}{\pgfqpoint{5.768486in}{2.351127in}}{\pgfqpoint{5.772604in}{2.347009in}}%
\pgfpathcurveto{\pgfqpoint{5.776722in}{2.342891in}}{\pgfqpoint{5.782308in}{2.340577in}}{\pgfqpoint{5.788132in}{2.340577in}}%
\pgfpathlineto{\pgfqpoint{5.788132in}{2.340577in}}%
\pgfpathclose%
\pgfusepath{stroke,fill}%
\end{pgfscope}%
\begin{pgfscope}%
\pgfpathrectangle{\pgfqpoint{1.073501in}{0.880000in}}{\pgfqpoint{6.052998in}{6.160000in}}%
\pgfusepath{clip}%
\pgfsetbuttcap%
\pgfsetroundjoin%
\definecolor{currentfill}{rgb}{0.800000,0.200000,0.200000}%
\pgfsetfillcolor{currentfill}%
\pgfsetlinewidth{1.003750pt}%
\definecolor{currentstroke}{rgb}{0.800000,0.200000,0.200000}%
\pgfsetstrokecolor{currentstroke}%
\pgfsetdash{}{0pt}%
\pgfpathmoveto{\pgfqpoint{5.831910in}{2.494033in}}%
\pgfpathcurveto{\pgfqpoint{5.837734in}{2.494033in}}{\pgfqpoint{5.843321in}{2.496347in}}{\pgfqpoint{5.847439in}{2.500465in}}%
\pgfpathcurveto{\pgfqpoint{5.851557in}{2.504583in}}{\pgfqpoint{5.853871in}{2.510169in}}{\pgfqpoint{5.853871in}{2.515993in}}%
\pgfpathcurveto{\pgfqpoint{5.853871in}{2.521817in}}{\pgfqpoint{5.851557in}{2.527403in}}{\pgfqpoint{5.847439in}{2.531521in}}%
\pgfpathcurveto{\pgfqpoint{5.843321in}{2.535639in}}{\pgfqpoint{5.837734in}{2.537953in}}{\pgfqpoint{5.831910in}{2.537953in}}%
\pgfpathcurveto{\pgfqpoint{5.826086in}{2.537953in}}{\pgfqpoint{5.820500in}{2.535639in}}{\pgfqpoint{5.816382in}{2.531521in}}%
\pgfpathcurveto{\pgfqpoint{5.812264in}{2.527403in}}{\pgfqpoint{5.809950in}{2.521817in}}{\pgfqpoint{5.809950in}{2.515993in}}%
\pgfpathcurveto{\pgfqpoint{5.809950in}{2.510169in}}{\pgfqpoint{5.812264in}{2.504583in}}{\pgfqpoint{5.816382in}{2.500465in}}%
\pgfpathcurveto{\pgfqpoint{5.820500in}{2.496347in}}{\pgfqpoint{5.826086in}{2.494033in}}{\pgfqpoint{5.831910in}{2.494033in}}%
\pgfpathlineto{\pgfqpoint{5.831910in}{2.494033in}}%
\pgfpathclose%
\pgfusepath{stroke,fill}%
\end{pgfscope}%
\begin{pgfscope}%
\pgfpathrectangle{\pgfqpoint{1.073501in}{0.880000in}}{\pgfqpoint{6.052998in}{6.160000in}}%
\pgfusepath{clip}%
\pgfsetbuttcap%
\pgfsetroundjoin%
\definecolor{currentfill}{rgb}{0.800000,0.200000,0.200000}%
\pgfsetfillcolor{currentfill}%
\pgfsetlinewidth{1.003750pt}%
\definecolor{currentstroke}{rgb}{0.800000,0.200000,0.200000}%
\pgfsetstrokecolor{currentstroke}%
\pgfsetdash{}{0pt}%
\pgfpathmoveto{\pgfqpoint{5.947376in}{2.584451in}}%
\pgfpathcurveto{\pgfqpoint{5.953200in}{2.584451in}}{\pgfqpoint{5.958786in}{2.586765in}}{\pgfqpoint{5.962904in}{2.590883in}}%
\pgfpathcurveto{\pgfqpoint{5.967022in}{2.595001in}}{\pgfqpoint{5.969336in}{2.600587in}}{\pgfqpoint{5.969336in}{2.606411in}}%
\pgfpathcurveto{\pgfqpoint{5.969336in}{2.612235in}}{\pgfqpoint{5.967022in}{2.617821in}}{\pgfqpoint{5.962904in}{2.621939in}}%
\pgfpathcurveto{\pgfqpoint{5.958786in}{2.626058in}}{\pgfqpoint{5.953200in}{2.628371in}}{\pgfqpoint{5.947376in}{2.628371in}}%
\pgfpathcurveto{\pgfqpoint{5.941552in}{2.628371in}}{\pgfqpoint{5.935966in}{2.626058in}}{\pgfqpoint{5.931848in}{2.621939in}}%
\pgfpathcurveto{\pgfqpoint{5.927730in}{2.617821in}}{\pgfqpoint{5.925416in}{2.612235in}}{\pgfqpoint{5.925416in}{2.606411in}}%
\pgfpathcurveto{\pgfqpoint{5.925416in}{2.600587in}}{\pgfqpoint{5.927730in}{2.595001in}}{\pgfqpoint{5.931848in}{2.590883in}}%
\pgfpathcurveto{\pgfqpoint{5.935966in}{2.586765in}}{\pgfqpoint{5.941552in}{2.584451in}}{\pgfqpoint{5.947376in}{2.584451in}}%
\pgfpathlineto{\pgfqpoint{5.947376in}{2.584451in}}%
\pgfpathclose%
\pgfusepath{stroke,fill}%
\end{pgfscope}%
\begin{pgfscope}%
\pgfpathrectangle{\pgfqpoint{1.073501in}{0.880000in}}{\pgfqpoint{6.052998in}{6.160000in}}%
\pgfusepath{clip}%
\pgfsetbuttcap%
\pgfsetroundjoin%
\definecolor{currentfill}{rgb}{0.800000,0.200000,0.200000}%
\pgfsetfillcolor{currentfill}%
\pgfsetlinewidth{1.003750pt}%
\definecolor{currentstroke}{rgb}{0.800000,0.200000,0.200000}%
\pgfsetstrokecolor{currentstroke}%
\pgfsetdash{}{0pt}%
\pgfpathmoveto{\pgfqpoint{6.031056in}{2.701233in}}%
\pgfpathcurveto{\pgfqpoint{6.036880in}{2.701233in}}{\pgfqpoint{6.042466in}{2.703547in}}{\pgfqpoint{6.046584in}{2.707665in}}%
\pgfpathcurveto{\pgfqpoint{6.050702in}{2.711783in}}{\pgfqpoint{6.053016in}{2.717369in}}{\pgfqpoint{6.053016in}{2.723193in}}%
\pgfpathcurveto{\pgfqpoint{6.053016in}{2.729017in}}{\pgfqpoint{6.050702in}{2.734604in}}{\pgfqpoint{6.046584in}{2.738722in}}%
\pgfpathcurveto{\pgfqpoint{6.042466in}{2.742840in}}{\pgfqpoint{6.036880in}{2.745154in}}{\pgfqpoint{6.031056in}{2.745154in}}%
\pgfpathcurveto{\pgfqpoint{6.025232in}{2.745154in}}{\pgfqpoint{6.019646in}{2.742840in}}{\pgfqpoint{6.015528in}{2.738722in}}%
\pgfpathcurveto{\pgfqpoint{6.011410in}{2.734604in}}{\pgfqpoint{6.009096in}{2.729017in}}{\pgfqpoint{6.009096in}{2.723193in}}%
\pgfpathcurveto{\pgfqpoint{6.009096in}{2.717369in}}{\pgfqpoint{6.011410in}{2.711783in}}{\pgfqpoint{6.015528in}{2.707665in}}%
\pgfpathcurveto{\pgfqpoint{6.019646in}{2.703547in}}{\pgfqpoint{6.025232in}{2.701233in}}{\pgfqpoint{6.031056in}{2.701233in}}%
\pgfpathlineto{\pgfqpoint{6.031056in}{2.701233in}}%
\pgfpathclose%
\pgfusepath{stroke,fill}%
\end{pgfscope}%
\begin{pgfscope}%
\pgfpathrectangle{\pgfqpoint{1.073501in}{0.880000in}}{\pgfqpoint{6.052998in}{6.160000in}}%
\pgfusepath{clip}%
\pgfsetbuttcap%
\pgfsetroundjoin%
\definecolor{currentfill}{rgb}{0.800000,0.200000,0.200000}%
\pgfsetfillcolor{currentfill}%
\pgfsetlinewidth{1.003750pt}%
\definecolor{currentstroke}{rgb}{0.800000,0.200000,0.200000}%
\pgfsetstrokecolor{currentstroke}%
\pgfsetdash{}{0pt}%
\pgfpathmoveto{\pgfqpoint{6.116632in}{2.818014in}}%
\pgfpathcurveto{\pgfqpoint{6.122456in}{2.818014in}}{\pgfqpoint{6.128042in}{2.820328in}}{\pgfqpoint{6.132160in}{2.824446in}}%
\pgfpathcurveto{\pgfqpoint{6.136278in}{2.828564in}}{\pgfqpoint{6.138592in}{2.834150in}}{\pgfqpoint{6.138592in}{2.839974in}}%
\pgfpathcurveto{\pgfqpoint{6.138592in}{2.845798in}}{\pgfqpoint{6.136278in}{2.851384in}}{\pgfqpoint{6.132160in}{2.855503in}}%
\pgfpathcurveto{\pgfqpoint{6.128042in}{2.859621in}}{\pgfqpoint{6.122456in}{2.861935in}}{\pgfqpoint{6.116632in}{2.861935in}}%
\pgfpathcurveto{\pgfqpoint{6.110808in}{2.861935in}}{\pgfqpoint{6.105222in}{2.859621in}}{\pgfqpoint{6.101104in}{2.855503in}}%
\pgfpathcurveto{\pgfqpoint{6.096986in}{2.851384in}}{\pgfqpoint{6.094672in}{2.845798in}}{\pgfqpoint{6.094672in}{2.839974in}}%
\pgfpathcurveto{\pgfqpoint{6.094672in}{2.834150in}}{\pgfqpoint{6.096986in}{2.828564in}}{\pgfqpoint{6.101104in}{2.824446in}}%
\pgfpathcurveto{\pgfqpoint{6.105222in}{2.820328in}}{\pgfqpoint{6.110808in}{2.818014in}}{\pgfqpoint{6.116632in}{2.818014in}}%
\pgfpathlineto{\pgfqpoint{6.116632in}{2.818014in}}%
\pgfpathclose%
\pgfusepath{stroke,fill}%
\end{pgfscope}%
\begin{pgfscope}%
\pgfpathrectangle{\pgfqpoint{1.073501in}{0.880000in}}{\pgfqpoint{6.052998in}{6.160000in}}%
\pgfusepath{clip}%
\pgfsetbuttcap%
\pgfsetroundjoin%
\definecolor{currentfill}{rgb}{0.800000,0.200000,0.200000}%
\pgfsetfillcolor{currentfill}%
\pgfsetlinewidth{1.003750pt}%
\definecolor{currentstroke}{rgb}{0.800000,0.200000,0.200000}%
\pgfsetstrokecolor{currentstroke}%
\pgfsetdash{}{0pt}%
\pgfpathmoveto{\pgfqpoint{6.221510in}{2.928136in}}%
\pgfpathcurveto{\pgfqpoint{6.227334in}{2.928136in}}{\pgfqpoint{6.232920in}{2.930450in}}{\pgfqpoint{6.237038in}{2.934568in}}%
\pgfpathcurveto{\pgfqpoint{6.241156in}{2.938687in}}{\pgfqpoint{6.243470in}{2.944273in}}{\pgfqpoint{6.243470in}{2.950097in}}%
\pgfpathcurveto{\pgfqpoint{6.243470in}{2.955921in}}{\pgfqpoint{6.241156in}{2.961507in}}{\pgfqpoint{6.237038in}{2.965625in}}%
\pgfpathcurveto{\pgfqpoint{6.232920in}{2.969743in}}{\pgfqpoint{6.227334in}{2.972057in}}{\pgfqpoint{6.221510in}{2.972057in}}%
\pgfpathcurveto{\pgfqpoint{6.215686in}{2.972057in}}{\pgfqpoint{6.210100in}{2.969743in}}{\pgfqpoint{6.205981in}{2.965625in}}%
\pgfpathcurveto{\pgfqpoint{6.201863in}{2.961507in}}{\pgfqpoint{6.199549in}{2.955921in}}{\pgfqpoint{6.199549in}{2.950097in}}%
\pgfpathcurveto{\pgfqpoint{6.199549in}{2.944273in}}{\pgfqpoint{6.201863in}{2.938687in}}{\pgfqpoint{6.205981in}{2.934568in}}%
\pgfpathcurveto{\pgfqpoint{6.210100in}{2.930450in}}{\pgfqpoint{6.215686in}{2.928136in}}{\pgfqpoint{6.221510in}{2.928136in}}%
\pgfpathlineto{\pgfqpoint{6.221510in}{2.928136in}}%
\pgfpathclose%
\pgfusepath{stroke,fill}%
\end{pgfscope}%
\begin{pgfscope}%
\pgfpathrectangle{\pgfqpoint{1.073501in}{0.880000in}}{\pgfqpoint{6.052998in}{6.160000in}}%
\pgfusepath{clip}%
\pgfsetbuttcap%
\pgfsetroundjoin%
\definecolor{currentfill}{rgb}{0.800000,0.200000,0.200000}%
\pgfsetfillcolor{currentfill}%
\pgfsetlinewidth{1.003750pt}%
\definecolor{currentstroke}{rgb}{0.800000,0.200000,0.200000}%
\pgfsetstrokecolor{currentstroke}%
\pgfsetdash{}{0pt}%
\pgfpathmoveto{\pgfqpoint{6.307045in}{3.052474in}}%
\pgfpathcurveto{\pgfqpoint{6.312869in}{3.052474in}}{\pgfqpoint{6.318455in}{3.054788in}}{\pgfqpoint{6.322573in}{3.058906in}}%
\pgfpathcurveto{\pgfqpoint{6.326691in}{3.063024in}}{\pgfqpoint{6.329005in}{3.068611in}}{\pgfqpoint{6.329005in}{3.074435in}}%
\pgfpathcurveto{\pgfqpoint{6.329005in}{3.080258in}}{\pgfqpoint{6.326691in}{3.085845in}}{\pgfqpoint{6.322573in}{3.089963in}}%
\pgfpathcurveto{\pgfqpoint{6.318455in}{3.094081in}}{\pgfqpoint{6.312869in}{3.096395in}}{\pgfqpoint{6.307045in}{3.096395in}}%
\pgfpathcurveto{\pgfqpoint{6.301221in}{3.096395in}}{\pgfqpoint{6.295635in}{3.094081in}}{\pgfqpoint{6.291517in}{3.089963in}}%
\pgfpathcurveto{\pgfqpoint{6.287399in}{3.085845in}}{\pgfqpoint{6.285085in}{3.080258in}}{\pgfqpoint{6.285085in}{3.074435in}}%
\pgfpathcurveto{\pgfqpoint{6.285085in}{3.068611in}}{\pgfqpoint{6.287399in}{3.063024in}}{\pgfqpoint{6.291517in}{3.058906in}}%
\pgfpathcurveto{\pgfqpoint{6.295635in}{3.054788in}}{\pgfqpoint{6.301221in}{3.052474in}}{\pgfqpoint{6.307045in}{3.052474in}}%
\pgfpathlineto{\pgfqpoint{6.307045in}{3.052474in}}%
\pgfpathclose%
\pgfusepath{stroke,fill}%
\end{pgfscope}%
\begin{pgfscope}%
\pgfpathrectangle{\pgfqpoint{1.073501in}{0.880000in}}{\pgfqpoint{6.052998in}{6.160000in}}%
\pgfusepath{clip}%
\pgfsetbuttcap%
\pgfsetroundjoin%
\definecolor{currentfill}{rgb}{0.800000,0.200000,0.200000}%
\pgfsetfillcolor{currentfill}%
\pgfsetlinewidth{1.003750pt}%
\definecolor{currentstroke}{rgb}{0.800000,0.200000,0.200000}%
\pgfsetstrokecolor{currentstroke}%
\pgfsetdash{}{0pt}%
\pgfpathmoveto{\pgfqpoint{6.190510in}{3.247611in}}%
\pgfpathcurveto{\pgfqpoint{6.196334in}{3.247611in}}{\pgfqpoint{6.201920in}{3.249924in}}{\pgfqpoint{6.206039in}{3.254043in}}%
\pgfpathcurveto{\pgfqpoint{6.210157in}{3.258161in}}{\pgfqpoint{6.212471in}{3.263747in}}{\pgfqpoint{6.212471in}{3.269571in}}%
\pgfpathcurveto{\pgfqpoint{6.212471in}{3.275395in}}{\pgfqpoint{6.210157in}{3.280981in}}{\pgfqpoint{6.206039in}{3.285099in}}%
\pgfpathcurveto{\pgfqpoint{6.201920in}{3.289217in}}{\pgfqpoint{6.196334in}{3.291531in}}{\pgfqpoint{6.190510in}{3.291531in}}%
\pgfpathcurveto{\pgfqpoint{6.184686in}{3.291531in}}{\pgfqpoint{6.179100in}{3.289217in}}{\pgfqpoint{6.174982in}{3.285099in}}%
\pgfpathcurveto{\pgfqpoint{6.170864in}{3.280981in}}{\pgfqpoint{6.168550in}{3.275395in}}{\pgfqpoint{6.168550in}{3.269571in}}%
\pgfpathcurveto{\pgfqpoint{6.168550in}{3.263747in}}{\pgfqpoint{6.170864in}{3.258161in}}{\pgfqpoint{6.174982in}{3.254043in}}%
\pgfpathcurveto{\pgfqpoint{6.179100in}{3.249924in}}{\pgfqpoint{6.184686in}{3.247611in}}{\pgfqpoint{6.190510in}{3.247611in}}%
\pgfpathlineto{\pgfqpoint{6.190510in}{3.247611in}}%
\pgfpathclose%
\pgfusepath{stroke,fill}%
\end{pgfscope}%
\begin{pgfscope}%
\pgfpathrectangle{\pgfqpoint{1.073501in}{0.880000in}}{\pgfqpoint{6.052998in}{6.160000in}}%
\pgfusepath{clip}%
\pgfsetbuttcap%
\pgfsetroundjoin%
\definecolor{currentfill}{rgb}{0.800000,0.200000,0.200000}%
\pgfsetfillcolor{currentfill}%
\pgfsetlinewidth{1.003750pt}%
\definecolor{currentstroke}{rgb}{0.800000,0.200000,0.200000}%
\pgfsetstrokecolor{currentstroke}%
\pgfsetdash{}{0pt}%
\pgfpathmoveto{\pgfqpoint{6.366764in}{3.344290in}}%
\pgfpathcurveto{\pgfqpoint{6.372587in}{3.344290in}}{\pgfqpoint{6.378174in}{3.346604in}}{\pgfqpoint{6.382292in}{3.350722in}}%
\pgfpathcurveto{\pgfqpoint{6.386410in}{3.354840in}}{\pgfqpoint{6.388724in}{3.360427in}}{\pgfqpoint{6.388724in}{3.366251in}}%
\pgfpathcurveto{\pgfqpoint{6.388724in}{3.372074in}}{\pgfqpoint{6.386410in}{3.377661in}}{\pgfqpoint{6.382292in}{3.381779in}}%
\pgfpathcurveto{\pgfqpoint{6.378174in}{3.385897in}}{\pgfqpoint{6.372587in}{3.388211in}}{\pgfqpoint{6.366764in}{3.388211in}}%
\pgfpathcurveto{\pgfqpoint{6.360940in}{3.388211in}}{\pgfqpoint{6.355353in}{3.385897in}}{\pgfqpoint{6.351235in}{3.381779in}}%
\pgfpathcurveto{\pgfqpoint{6.347117in}{3.377661in}}{\pgfqpoint{6.344803in}{3.372074in}}{\pgfqpoint{6.344803in}{3.366251in}}%
\pgfpathcurveto{\pgfqpoint{6.344803in}{3.360427in}}{\pgfqpoint{6.347117in}{3.354840in}}{\pgfqpoint{6.351235in}{3.350722in}}%
\pgfpathcurveto{\pgfqpoint{6.355353in}{3.346604in}}{\pgfqpoint{6.360940in}{3.344290in}}{\pgfqpoint{6.366764in}{3.344290in}}%
\pgfpathlineto{\pgfqpoint{6.366764in}{3.344290in}}%
\pgfpathclose%
\pgfusepath{stroke,fill}%
\end{pgfscope}%
\begin{pgfscope}%
\pgfpathrectangle{\pgfqpoint{1.073501in}{0.880000in}}{\pgfqpoint{6.052998in}{6.160000in}}%
\pgfusepath{clip}%
\pgfsetbuttcap%
\pgfsetroundjoin%
\definecolor{currentfill}{rgb}{0.800000,0.200000,0.200000}%
\pgfsetfillcolor{currentfill}%
\pgfsetlinewidth{1.003750pt}%
\definecolor{currentstroke}{rgb}{0.800000,0.200000,0.200000}%
\pgfsetstrokecolor{currentstroke}%
\pgfsetdash{}{0pt}%
\pgfpathmoveto{\pgfqpoint{6.288491in}{3.508928in}}%
\pgfpathcurveto{\pgfqpoint{6.294315in}{3.508928in}}{\pgfqpoint{6.299901in}{3.511242in}}{\pgfqpoint{6.304019in}{3.515360in}}%
\pgfpathcurveto{\pgfqpoint{6.308137in}{3.519478in}}{\pgfqpoint{6.310451in}{3.525065in}}{\pgfqpoint{6.310451in}{3.530888in}}%
\pgfpathcurveto{\pgfqpoint{6.310451in}{3.536712in}}{\pgfqpoint{6.308137in}{3.542299in}}{\pgfqpoint{6.304019in}{3.546417in}}%
\pgfpathcurveto{\pgfqpoint{6.299901in}{3.550535in}}{\pgfqpoint{6.294315in}{3.552849in}}{\pgfqpoint{6.288491in}{3.552849in}}%
\pgfpathcurveto{\pgfqpoint{6.282667in}{3.552849in}}{\pgfqpoint{6.277081in}{3.550535in}}{\pgfqpoint{6.272963in}{3.546417in}}%
\pgfpathcurveto{\pgfqpoint{6.268844in}{3.542299in}}{\pgfqpoint{6.266531in}{3.536712in}}{\pgfqpoint{6.266531in}{3.530888in}}%
\pgfpathcurveto{\pgfqpoint{6.266531in}{3.525065in}}{\pgfqpoint{6.268844in}{3.519478in}}{\pgfqpoint{6.272963in}{3.515360in}}%
\pgfpathcurveto{\pgfqpoint{6.277081in}{3.511242in}}{\pgfqpoint{6.282667in}{3.508928in}}{\pgfqpoint{6.288491in}{3.508928in}}%
\pgfpathlineto{\pgfqpoint{6.288491in}{3.508928in}}%
\pgfpathclose%
\pgfusepath{stroke,fill}%
\end{pgfscope}%
\begin{pgfscope}%
\pgfpathrectangle{\pgfqpoint{1.073501in}{0.880000in}}{\pgfqpoint{6.052998in}{6.160000in}}%
\pgfusepath{clip}%
\pgfsetbuttcap%
\pgfsetroundjoin%
\definecolor{currentfill}{rgb}{0.800000,0.200000,0.200000}%
\pgfsetfillcolor{currentfill}%
\pgfsetlinewidth{1.003750pt}%
\definecolor{currentstroke}{rgb}{0.800000,0.200000,0.200000}%
\pgfsetstrokecolor{currentstroke}%
\pgfsetdash{}{0pt}%
\pgfpathmoveto{\pgfqpoint{6.400270in}{3.635499in}}%
\pgfpathcurveto{\pgfqpoint{6.406094in}{3.635499in}}{\pgfqpoint{6.411680in}{3.637813in}}{\pgfqpoint{6.415799in}{3.641931in}}%
\pgfpathcurveto{\pgfqpoint{6.419917in}{3.646049in}}{\pgfqpoint{6.422231in}{3.651635in}}{\pgfqpoint{6.422231in}{3.657459in}}%
\pgfpathcurveto{\pgfqpoint{6.422231in}{3.663283in}}{\pgfqpoint{6.419917in}{3.668870in}}{\pgfqpoint{6.415799in}{3.672988in}}%
\pgfpathcurveto{\pgfqpoint{6.411680in}{3.677106in}}{\pgfqpoint{6.406094in}{3.679420in}}{\pgfqpoint{6.400270in}{3.679420in}}%
\pgfpathcurveto{\pgfqpoint{6.394446in}{3.679420in}}{\pgfqpoint{6.388860in}{3.677106in}}{\pgfqpoint{6.384742in}{3.672988in}}%
\pgfpathcurveto{\pgfqpoint{6.380624in}{3.668870in}}{\pgfqpoint{6.378310in}{3.663283in}}{\pgfqpoint{6.378310in}{3.657459in}}%
\pgfpathcurveto{\pgfqpoint{6.378310in}{3.651635in}}{\pgfqpoint{6.380624in}{3.646049in}}{\pgfqpoint{6.384742in}{3.641931in}}%
\pgfpathcurveto{\pgfqpoint{6.388860in}{3.637813in}}{\pgfqpoint{6.394446in}{3.635499in}}{\pgfqpoint{6.400270in}{3.635499in}}%
\pgfpathlineto{\pgfqpoint{6.400270in}{3.635499in}}%
\pgfpathclose%
\pgfusepath{stroke,fill}%
\end{pgfscope}%
\begin{pgfscope}%
\pgfpathrectangle{\pgfqpoint{1.073501in}{0.880000in}}{\pgfqpoint{6.052998in}{6.160000in}}%
\pgfusepath{clip}%
\pgfsetbuttcap%
\pgfsetroundjoin%
\definecolor{currentfill}{rgb}{0.800000,0.200000,0.200000}%
\pgfsetfillcolor{currentfill}%
\pgfsetlinewidth{1.003750pt}%
\definecolor{currentstroke}{rgb}{0.800000,0.200000,0.200000}%
\pgfsetstrokecolor{currentstroke}%
\pgfsetdash{}{0pt}%
\pgfpathmoveto{\pgfqpoint{6.398375in}{3.781351in}}%
\pgfpathcurveto{\pgfqpoint{6.404198in}{3.781351in}}{\pgfqpoint{6.409785in}{3.783665in}}{\pgfqpoint{6.413903in}{3.787783in}}%
\pgfpathcurveto{\pgfqpoint{6.418021in}{3.791901in}}{\pgfqpoint{6.420335in}{3.797487in}}{\pgfqpoint{6.420335in}{3.803311in}}%
\pgfpathcurveto{\pgfqpoint{6.420335in}{3.809135in}}{\pgfqpoint{6.418021in}{3.814721in}}{\pgfqpoint{6.413903in}{3.818840in}}%
\pgfpathcurveto{\pgfqpoint{6.409785in}{3.822958in}}{\pgfqpoint{6.404198in}{3.825272in}}{\pgfqpoint{6.398375in}{3.825272in}}%
\pgfpathcurveto{\pgfqpoint{6.392551in}{3.825272in}}{\pgfqpoint{6.386964in}{3.822958in}}{\pgfqpoint{6.382846in}{3.818840in}}%
\pgfpathcurveto{\pgfqpoint{6.378728in}{3.814721in}}{\pgfqpoint{6.376414in}{3.809135in}}{\pgfqpoint{6.376414in}{3.803311in}}%
\pgfpathcurveto{\pgfqpoint{6.376414in}{3.797487in}}{\pgfqpoint{6.378728in}{3.791901in}}{\pgfqpoint{6.382846in}{3.787783in}}%
\pgfpathcurveto{\pgfqpoint{6.386964in}{3.783665in}}{\pgfqpoint{6.392551in}{3.781351in}}{\pgfqpoint{6.398375in}{3.781351in}}%
\pgfpathlineto{\pgfqpoint{6.398375in}{3.781351in}}%
\pgfpathclose%
\pgfusepath{stroke,fill}%
\end{pgfscope}%
\begin{pgfscope}%
\pgfpathrectangle{\pgfqpoint{1.073501in}{0.880000in}}{\pgfqpoint{6.052998in}{6.160000in}}%
\pgfusepath{clip}%
\pgfsetbuttcap%
\pgfsetroundjoin%
\definecolor{currentfill}{rgb}{0.800000,0.200000,0.200000}%
\pgfsetfillcolor{currentfill}%
\pgfsetlinewidth{1.003750pt}%
\definecolor{currentstroke}{rgb}{0.800000,0.200000,0.200000}%
\pgfsetstrokecolor{currentstroke}%
\pgfsetdash{}{0pt}%
\pgfpathmoveto{\pgfqpoint{6.347133in}{3.925790in}}%
\pgfpathcurveto{\pgfqpoint{6.352956in}{3.925790in}}{\pgfqpoint{6.358543in}{3.928103in}}{\pgfqpoint{6.362661in}{3.932222in}}%
\pgfpathcurveto{\pgfqpoint{6.366779in}{3.936340in}}{\pgfqpoint{6.369093in}{3.941926in}}{\pgfqpoint{6.369093in}{3.947750in}}%
\pgfpathcurveto{\pgfqpoint{6.369093in}{3.953574in}}{\pgfqpoint{6.366779in}{3.959160in}}{\pgfqpoint{6.362661in}{3.963278in}}%
\pgfpathcurveto{\pgfqpoint{6.358543in}{3.967396in}}{\pgfqpoint{6.352956in}{3.969710in}}{\pgfqpoint{6.347133in}{3.969710in}}%
\pgfpathcurveto{\pgfqpoint{6.341309in}{3.969710in}}{\pgfqpoint{6.335722in}{3.967396in}}{\pgfqpoint{6.331604in}{3.963278in}}%
\pgfpathcurveto{\pgfqpoint{6.327486in}{3.959160in}}{\pgfqpoint{6.325172in}{3.953574in}}{\pgfqpoint{6.325172in}{3.947750in}}%
\pgfpathcurveto{\pgfqpoint{6.325172in}{3.941926in}}{\pgfqpoint{6.327486in}{3.936340in}}{\pgfqpoint{6.331604in}{3.932222in}}%
\pgfpathcurveto{\pgfqpoint{6.335722in}{3.928103in}}{\pgfqpoint{6.341309in}{3.925790in}}{\pgfqpoint{6.347133in}{3.925790in}}%
\pgfpathlineto{\pgfqpoint{6.347133in}{3.925790in}}%
\pgfpathclose%
\pgfusepath{stroke,fill}%
\end{pgfscope}%
\begin{pgfscope}%
\pgfpathrectangle{\pgfqpoint{1.073501in}{0.880000in}}{\pgfqpoint{6.052998in}{6.160000in}}%
\pgfusepath{clip}%
\pgfsetbuttcap%
\pgfsetroundjoin%
\definecolor{currentfill}{rgb}{0.200000,0.200000,0.800000}%
\pgfsetfillcolor{currentfill}%
\pgfsetlinewidth{1.003750pt}%
\definecolor{currentstroke}{rgb}{0.200000,0.200000,0.800000}%
\pgfsetstrokecolor{currentstroke}%
\pgfsetdash{}{0pt}%
\pgfpathmoveto{\pgfqpoint{6.819033in}{3.925790in}}%
\pgfpathcurveto{\pgfqpoint{6.824857in}{3.925790in}}{\pgfqpoint{6.830443in}{3.928103in}}{\pgfqpoint{6.834562in}{3.932222in}}%
\pgfpathcurveto{\pgfqpoint{6.838680in}{3.936340in}}{\pgfqpoint{6.840994in}{3.941926in}}{\pgfqpoint{6.840994in}{3.947750in}}%
\pgfpathcurveto{\pgfqpoint{6.840994in}{3.953574in}}{\pgfqpoint{6.838680in}{3.959160in}}{\pgfqpoint{6.834562in}{3.963278in}}%
\pgfpathcurveto{\pgfqpoint{6.830443in}{3.967396in}}{\pgfqpoint{6.824857in}{3.969710in}}{\pgfqpoint{6.819033in}{3.969710in}}%
\pgfpathcurveto{\pgfqpoint{6.813209in}{3.969710in}}{\pgfqpoint{6.807623in}{3.967396in}}{\pgfqpoint{6.803505in}{3.963278in}}%
\pgfpathcurveto{\pgfqpoint{6.799387in}{3.959160in}}{\pgfqpoint{6.797073in}{3.953574in}}{\pgfqpoint{6.797073in}{3.947750in}}%
\pgfpathcurveto{\pgfqpoint{6.797073in}{3.941926in}}{\pgfqpoint{6.799387in}{3.936340in}}{\pgfqpoint{6.803505in}{3.932222in}}%
\pgfpathcurveto{\pgfqpoint{6.807623in}{3.928103in}}{\pgfqpoint{6.813209in}{3.925790in}}{\pgfqpoint{6.819033in}{3.925790in}}%
\pgfpathlineto{\pgfqpoint{6.819033in}{3.925790in}}%
\pgfpathclose%
\pgfusepath{stroke,fill}%
\end{pgfscope}%
\begin{pgfscope}%
\pgfpathrectangle{\pgfqpoint{1.073501in}{0.880000in}}{\pgfqpoint{6.052998in}{6.160000in}}%
\pgfusepath{clip}%
\pgfsetbuttcap%
\pgfsetroundjoin%
\definecolor{currentfill}{rgb}{0.200000,0.200000,0.800000}%
\pgfsetfillcolor{currentfill}%
\pgfsetlinewidth{1.003750pt}%
\definecolor{currentstroke}{rgb}{0.200000,0.200000,0.800000}%
\pgfsetstrokecolor{currentstroke}%
\pgfsetdash{}{0pt}%
\pgfpathmoveto{\pgfqpoint{6.834731in}{4.097959in}}%
\pgfpathcurveto{\pgfqpoint{6.840555in}{4.097959in}}{\pgfqpoint{6.846141in}{4.100273in}}{\pgfqpoint{6.850259in}{4.104391in}}%
\pgfpathcurveto{\pgfqpoint{6.854377in}{4.108509in}}{\pgfqpoint{6.856691in}{4.114096in}}{\pgfqpoint{6.856691in}{4.119920in}}%
\pgfpathcurveto{\pgfqpoint{6.856691in}{4.125744in}}{\pgfqpoint{6.854377in}{4.131330in}}{\pgfqpoint{6.850259in}{4.135448in}}%
\pgfpathcurveto{\pgfqpoint{6.846141in}{4.139566in}}{\pgfqpoint{6.840555in}{4.141880in}}{\pgfqpoint{6.834731in}{4.141880in}}%
\pgfpathcurveto{\pgfqpoint{6.828907in}{4.141880in}}{\pgfqpoint{6.823321in}{4.139566in}}{\pgfqpoint{6.819203in}{4.135448in}}%
\pgfpathcurveto{\pgfqpoint{6.815084in}{4.131330in}}{\pgfqpoint{6.812771in}{4.125744in}}{\pgfqpoint{6.812771in}{4.119920in}}%
\pgfpathcurveto{\pgfqpoint{6.812771in}{4.114096in}}{\pgfqpoint{6.815084in}{4.108509in}}{\pgfqpoint{6.819203in}{4.104391in}}%
\pgfpathcurveto{\pgfqpoint{6.823321in}{4.100273in}}{\pgfqpoint{6.828907in}{4.097959in}}{\pgfqpoint{6.834731in}{4.097959in}}%
\pgfpathlineto{\pgfqpoint{6.834731in}{4.097959in}}%
\pgfpathclose%
\pgfusepath{stroke,fill}%
\end{pgfscope}%
\begin{pgfscope}%
\pgfpathrectangle{\pgfqpoint{1.073501in}{0.880000in}}{\pgfqpoint{6.052998in}{6.160000in}}%
\pgfusepath{clip}%
\pgfsetbuttcap%
\pgfsetroundjoin%
\definecolor{currentfill}{rgb}{0.200000,0.200000,0.800000}%
\pgfsetfillcolor{currentfill}%
\pgfsetlinewidth{1.003750pt}%
\definecolor{currentstroke}{rgb}{0.200000,0.200000,0.800000}%
\pgfsetstrokecolor{currentstroke}%
\pgfsetdash{}{0pt}%
\pgfpathmoveto{\pgfqpoint{6.810935in}{4.268489in}}%
\pgfpathcurveto{\pgfqpoint{6.816759in}{4.268489in}}{\pgfqpoint{6.822345in}{4.270802in}}{\pgfqpoint{6.826463in}{4.274921in}}%
\pgfpathcurveto{\pgfqpoint{6.830581in}{4.279039in}}{\pgfqpoint{6.832895in}{4.284625in}}{\pgfqpoint{6.832895in}{4.290449in}}%
\pgfpathcurveto{\pgfqpoint{6.832895in}{4.296273in}}{\pgfqpoint{6.830581in}{4.301859in}}{\pgfqpoint{6.826463in}{4.305977in}}%
\pgfpathcurveto{\pgfqpoint{6.822345in}{4.310095in}}{\pgfqpoint{6.816759in}{4.312409in}}{\pgfqpoint{6.810935in}{4.312409in}}%
\pgfpathcurveto{\pgfqpoint{6.805111in}{4.312409in}}{\pgfqpoint{6.799525in}{4.310095in}}{\pgfqpoint{6.795406in}{4.305977in}}%
\pgfpathcurveto{\pgfqpoint{6.791288in}{4.301859in}}{\pgfqpoint{6.788974in}{4.296273in}}{\pgfqpoint{6.788974in}{4.290449in}}%
\pgfpathcurveto{\pgfqpoint{6.788974in}{4.284625in}}{\pgfqpoint{6.791288in}{4.279039in}}{\pgfqpoint{6.795406in}{4.274921in}}%
\pgfpathcurveto{\pgfqpoint{6.799525in}{4.270802in}}{\pgfqpoint{6.805111in}{4.268489in}}{\pgfqpoint{6.810935in}{4.268489in}}%
\pgfpathlineto{\pgfqpoint{6.810935in}{4.268489in}}%
\pgfpathclose%
\pgfusepath{stroke,fill}%
\end{pgfscope}%
\begin{pgfscope}%
\pgfpathrectangle{\pgfqpoint{1.073501in}{0.880000in}}{\pgfqpoint{6.052998in}{6.160000in}}%
\pgfusepath{clip}%
\pgfsetbuttcap%
\pgfsetroundjoin%
\definecolor{currentfill}{rgb}{0.200000,0.200000,0.800000}%
\pgfsetfillcolor{currentfill}%
\pgfsetlinewidth{1.003750pt}%
\definecolor{currentstroke}{rgb}{0.200000,0.200000,0.800000}%
\pgfsetstrokecolor{currentstroke}%
\pgfsetdash{}{0pt}%
\pgfpathmoveto{\pgfqpoint{6.798333in}{4.440915in}}%
\pgfpathcurveto{\pgfqpoint{6.804157in}{4.440915in}}{\pgfqpoint{6.809743in}{4.443229in}}{\pgfqpoint{6.813861in}{4.447347in}}%
\pgfpathcurveto{\pgfqpoint{6.817980in}{4.451465in}}{\pgfqpoint{6.820293in}{4.457051in}}{\pgfqpoint{6.820293in}{4.462875in}}%
\pgfpathcurveto{\pgfqpoint{6.820293in}{4.468699in}}{\pgfqpoint{6.817980in}{4.474285in}}{\pgfqpoint{6.813861in}{4.478403in}}%
\pgfpathcurveto{\pgfqpoint{6.809743in}{4.482522in}}{\pgfqpoint{6.804157in}{4.484835in}}{\pgfqpoint{6.798333in}{4.484835in}}%
\pgfpathcurveto{\pgfqpoint{6.792509in}{4.484835in}}{\pgfqpoint{6.786923in}{4.482522in}}{\pgfqpoint{6.782805in}{4.478403in}}%
\pgfpathcurveto{\pgfqpoint{6.778687in}{4.474285in}}{\pgfqpoint{6.776373in}{4.468699in}}{\pgfqpoint{6.776373in}{4.462875in}}%
\pgfpathcurveto{\pgfqpoint{6.776373in}{4.457051in}}{\pgfqpoint{6.778687in}{4.451465in}}{\pgfqpoint{6.782805in}{4.447347in}}%
\pgfpathcurveto{\pgfqpoint{6.786923in}{4.443229in}}{\pgfqpoint{6.792509in}{4.440915in}}{\pgfqpoint{6.798333in}{4.440915in}}%
\pgfpathlineto{\pgfqpoint{6.798333in}{4.440915in}}%
\pgfpathclose%
\pgfusepath{stroke,fill}%
\end{pgfscope}%
\begin{pgfscope}%
\pgfpathrectangle{\pgfqpoint{1.073501in}{0.880000in}}{\pgfqpoint{6.052998in}{6.160000in}}%
\pgfusepath{clip}%
\pgfsetbuttcap%
\pgfsetroundjoin%
\definecolor{currentfill}{rgb}{0.200000,0.200000,0.800000}%
\pgfsetfillcolor{currentfill}%
\pgfsetlinewidth{1.003750pt}%
\definecolor{currentstroke}{rgb}{0.200000,0.200000,0.800000}%
\pgfsetstrokecolor{currentstroke}%
\pgfsetdash{}{0pt}%
\pgfpathmoveto{\pgfqpoint{6.742766in}{4.604848in}}%
\pgfpathcurveto{\pgfqpoint{6.748590in}{4.604848in}}{\pgfqpoint{6.754176in}{4.607162in}}{\pgfqpoint{6.758294in}{4.611280in}}%
\pgfpathcurveto{\pgfqpoint{6.762413in}{4.615398in}}{\pgfqpoint{6.764726in}{4.620984in}}{\pgfqpoint{6.764726in}{4.626808in}}%
\pgfpathcurveto{\pgfqpoint{6.764726in}{4.632632in}}{\pgfqpoint{6.762413in}{4.638218in}}{\pgfqpoint{6.758294in}{4.642337in}}%
\pgfpathcurveto{\pgfqpoint{6.754176in}{4.646455in}}{\pgfqpoint{6.748590in}{4.648769in}}{\pgfqpoint{6.742766in}{4.648769in}}%
\pgfpathcurveto{\pgfqpoint{6.736942in}{4.648769in}}{\pgfqpoint{6.731356in}{4.646455in}}{\pgfqpoint{6.727238in}{4.642337in}}%
\pgfpathcurveto{\pgfqpoint{6.723120in}{4.638218in}}{\pgfqpoint{6.720806in}{4.632632in}}{\pgfqpoint{6.720806in}{4.626808in}}%
\pgfpathcurveto{\pgfqpoint{6.720806in}{4.620984in}}{\pgfqpoint{6.723120in}{4.615398in}}{\pgfqpoint{6.727238in}{4.611280in}}%
\pgfpathcurveto{\pgfqpoint{6.731356in}{4.607162in}}{\pgfqpoint{6.736942in}{4.604848in}}{\pgfqpoint{6.742766in}{4.604848in}}%
\pgfpathlineto{\pgfqpoint{6.742766in}{4.604848in}}%
\pgfpathclose%
\pgfusepath{stroke,fill}%
\end{pgfscope}%
\begin{pgfscope}%
\pgfpathrectangle{\pgfqpoint{1.073501in}{0.880000in}}{\pgfqpoint{6.052998in}{6.160000in}}%
\pgfusepath{clip}%
\pgfsetbuttcap%
\pgfsetroundjoin%
\definecolor{currentfill}{rgb}{0.200000,0.200000,0.800000}%
\pgfsetfillcolor{currentfill}%
\pgfsetlinewidth{1.003750pt}%
\definecolor{currentstroke}{rgb}{0.200000,0.200000,0.800000}%
\pgfsetstrokecolor{currentstroke}%
\pgfsetdash{}{0pt}%
\pgfpathmoveto{\pgfqpoint{6.752824in}{4.788650in}}%
\pgfpathcurveto{\pgfqpoint{6.758648in}{4.788650in}}{\pgfqpoint{6.764235in}{4.790964in}}{\pgfqpoint{6.768353in}{4.795082in}}%
\pgfpathcurveto{\pgfqpoint{6.772471in}{4.799200in}}{\pgfqpoint{6.774785in}{4.804787in}}{\pgfqpoint{6.774785in}{4.810611in}}%
\pgfpathcurveto{\pgfqpoint{6.774785in}{4.816435in}}{\pgfqpoint{6.772471in}{4.822021in}}{\pgfqpoint{6.768353in}{4.826139in}}%
\pgfpathcurveto{\pgfqpoint{6.764235in}{4.830257in}}{\pgfqpoint{6.758648in}{4.832571in}}{\pgfqpoint{6.752824in}{4.832571in}}%
\pgfpathcurveto{\pgfqpoint{6.747001in}{4.832571in}}{\pgfqpoint{6.741414in}{4.830257in}}{\pgfqpoint{6.737296in}{4.826139in}}%
\pgfpathcurveto{\pgfqpoint{6.733178in}{4.822021in}}{\pgfqpoint{6.730864in}{4.816435in}}{\pgfqpoint{6.730864in}{4.810611in}}%
\pgfpathcurveto{\pgfqpoint{6.730864in}{4.804787in}}{\pgfqpoint{6.733178in}{4.799200in}}{\pgfqpoint{6.737296in}{4.795082in}}%
\pgfpathcurveto{\pgfqpoint{6.741414in}{4.790964in}}{\pgfqpoint{6.747001in}{4.788650in}}{\pgfqpoint{6.752824in}{4.788650in}}%
\pgfpathlineto{\pgfqpoint{6.752824in}{4.788650in}}%
\pgfpathclose%
\pgfusepath{stroke,fill}%
\end{pgfscope}%
\begin{pgfscope}%
\pgfpathrectangle{\pgfqpoint{1.073501in}{0.880000in}}{\pgfqpoint{6.052998in}{6.160000in}}%
\pgfusepath{clip}%
\pgfsetbuttcap%
\pgfsetroundjoin%
\definecolor{currentfill}{rgb}{0.200000,0.200000,0.800000}%
\pgfsetfillcolor{currentfill}%
\pgfsetlinewidth{1.003750pt}%
\definecolor{currentstroke}{rgb}{0.200000,0.200000,0.800000}%
\pgfsetstrokecolor{currentstroke}%
\pgfsetdash{}{0pt}%
\pgfpathmoveto{\pgfqpoint{6.594419in}{4.914152in}}%
\pgfpathcurveto{\pgfqpoint{6.600243in}{4.914152in}}{\pgfqpoint{6.605829in}{4.916466in}}{\pgfqpoint{6.609947in}{4.920584in}}%
\pgfpathcurveto{\pgfqpoint{6.614065in}{4.924702in}}{\pgfqpoint{6.616379in}{4.930288in}}{\pgfqpoint{6.616379in}{4.936112in}}%
\pgfpathcurveto{\pgfqpoint{6.616379in}{4.941936in}}{\pgfqpoint{6.614065in}{4.947522in}}{\pgfqpoint{6.609947in}{4.951640in}}%
\pgfpathcurveto{\pgfqpoint{6.605829in}{4.955758in}}{\pgfqpoint{6.600243in}{4.958072in}}{\pgfqpoint{6.594419in}{4.958072in}}%
\pgfpathcurveto{\pgfqpoint{6.588595in}{4.958072in}}{\pgfqpoint{6.583008in}{4.955758in}}{\pgfqpoint{6.578890in}{4.951640in}}%
\pgfpathcurveto{\pgfqpoint{6.574772in}{4.947522in}}{\pgfqpoint{6.572458in}{4.941936in}}{\pgfqpoint{6.572458in}{4.936112in}}%
\pgfpathcurveto{\pgfqpoint{6.572458in}{4.930288in}}{\pgfqpoint{6.574772in}{4.924702in}}{\pgfqpoint{6.578890in}{4.920584in}}%
\pgfpathcurveto{\pgfqpoint{6.583008in}{4.916466in}}{\pgfqpoint{6.588595in}{4.914152in}}{\pgfqpoint{6.594419in}{4.914152in}}%
\pgfpathlineto{\pgfqpoint{6.594419in}{4.914152in}}%
\pgfpathclose%
\pgfusepath{stroke,fill}%
\end{pgfscope}%
\begin{pgfscope}%
\pgfpathrectangle{\pgfqpoint{1.073501in}{0.880000in}}{\pgfqpoint{6.052998in}{6.160000in}}%
\pgfusepath{clip}%
\pgfsetbuttcap%
\pgfsetroundjoin%
\definecolor{currentfill}{rgb}{0.200000,0.200000,0.800000}%
\pgfsetfillcolor{currentfill}%
\pgfsetlinewidth{1.003750pt}%
\definecolor{currentstroke}{rgb}{0.200000,0.200000,0.800000}%
\pgfsetstrokecolor{currentstroke}%
\pgfsetdash{}{0pt}%
\pgfpathmoveto{\pgfqpoint{6.508543in}{5.060071in}}%
\pgfpathcurveto{\pgfqpoint{6.514367in}{5.060071in}}{\pgfqpoint{6.519953in}{5.062385in}}{\pgfqpoint{6.524071in}{5.066503in}}%
\pgfpathcurveto{\pgfqpoint{6.528189in}{5.070621in}}{\pgfqpoint{6.530503in}{5.076207in}}{\pgfqpoint{6.530503in}{5.082031in}}%
\pgfpathcurveto{\pgfqpoint{6.530503in}{5.087855in}}{\pgfqpoint{6.528189in}{5.093442in}}{\pgfqpoint{6.524071in}{5.097560in}}%
\pgfpathcurveto{\pgfqpoint{6.519953in}{5.101678in}}{\pgfqpoint{6.514367in}{5.103992in}}{\pgfqpoint{6.508543in}{5.103992in}}%
\pgfpathcurveto{\pgfqpoint{6.502719in}{5.103992in}}{\pgfqpoint{6.497133in}{5.101678in}}{\pgfqpoint{6.493015in}{5.097560in}}%
\pgfpathcurveto{\pgfqpoint{6.488897in}{5.093442in}}{\pgfqpoint{6.486583in}{5.087855in}}{\pgfqpoint{6.486583in}{5.082031in}}%
\pgfpathcurveto{\pgfqpoint{6.486583in}{5.076207in}}{\pgfqpoint{6.488897in}{5.070621in}}{\pgfqpoint{6.493015in}{5.066503in}}%
\pgfpathcurveto{\pgfqpoint{6.497133in}{5.062385in}}{\pgfqpoint{6.502719in}{5.060071in}}{\pgfqpoint{6.508543in}{5.060071in}}%
\pgfpathlineto{\pgfqpoint{6.508543in}{5.060071in}}%
\pgfpathclose%
\pgfusepath{stroke,fill}%
\end{pgfscope}%
\begin{pgfscope}%
\pgfpathrectangle{\pgfqpoint{1.073501in}{0.880000in}}{\pgfqpoint{6.052998in}{6.160000in}}%
\pgfusepath{clip}%
\pgfsetbuttcap%
\pgfsetroundjoin%
\definecolor{currentfill}{rgb}{0.200000,0.200000,0.800000}%
\pgfsetfillcolor{currentfill}%
\pgfsetlinewidth{1.003750pt}%
\definecolor{currentstroke}{rgb}{0.200000,0.200000,0.800000}%
\pgfsetstrokecolor{currentstroke}%
\pgfsetdash{}{0pt}%
\pgfpathmoveto{\pgfqpoint{6.506774in}{5.250634in}}%
\pgfpathcurveto{\pgfqpoint{6.512598in}{5.250634in}}{\pgfqpoint{6.518184in}{5.252948in}}{\pgfqpoint{6.522302in}{5.257066in}}%
\pgfpathcurveto{\pgfqpoint{6.526420in}{5.261184in}}{\pgfqpoint{6.528734in}{5.266771in}}{\pgfqpoint{6.528734in}{5.272594in}}%
\pgfpathcurveto{\pgfqpoint{6.528734in}{5.278418in}}{\pgfqpoint{6.526420in}{5.284005in}}{\pgfqpoint{6.522302in}{5.288123in}}%
\pgfpathcurveto{\pgfqpoint{6.518184in}{5.292241in}}{\pgfqpoint{6.512598in}{5.294555in}}{\pgfqpoint{6.506774in}{5.294555in}}%
\pgfpathcurveto{\pgfqpoint{6.500950in}{5.294555in}}{\pgfqpoint{6.495364in}{5.292241in}}{\pgfqpoint{6.491246in}{5.288123in}}%
\pgfpathcurveto{\pgfqpoint{6.487128in}{5.284005in}}{\pgfqpoint{6.484814in}{5.278418in}}{\pgfqpoint{6.484814in}{5.272594in}}%
\pgfpathcurveto{\pgfqpoint{6.484814in}{5.266771in}}{\pgfqpoint{6.487128in}{5.261184in}}{\pgfqpoint{6.491246in}{5.257066in}}%
\pgfpathcurveto{\pgfqpoint{6.495364in}{5.252948in}}{\pgfqpoint{6.500950in}{5.250634in}}{\pgfqpoint{6.506774in}{5.250634in}}%
\pgfpathlineto{\pgfqpoint{6.506774in}{5.250634in}}%
\pgfpathclose%
\pgfusepath{stroke,fill}%
\end{pgfscope}%
\begin{pgfscope}%
\pgfpathrectangle{\pgfqpoint{1.073501in}{0.880000in}}{\pgfqpoint{6.052998in}{6.160000in}}%
\pgfusepath{clip}%
\pgfsetbuttcap%
\pgfsetroundjoin%
\definecolor{currentfill}{rgb}{0.200000,0.200000,0.800000}%
\pgfsetfillcolor{currentfill}%
\pgfsetlinewidth{1.003750pt}%
\definecolor{currentstroke}{rgb}{0.200000,0.200000,0.800000}%
\pgfsetstrokecolor{currentstroke}%
\pgfsetdash{}{0pt}%
\pgfpathmoveto{\pgfqpoint{6.433047in}{5.408690in}}%
\pgfpathcurveto{\pgfqpoint{6.438871in}{5.408690in}}{\pgfqpoint{6.444457in}{5.411004in}}{\pgfqpoint{6.448575in}{5.415122in}}%
\pgfpathcurveto{\pgfqpoint{6.452693in}{5.419240in}}{\pgfqpoint{6.455007in}{5.424826in}}{\pgfqpoint{6.455007in}{5.430650in}}%
\pgfpathcurveto{\pgfqpoint{6.455007in}{5.436474in}}{\pgfqpoint{6.452693in}{5.442060in}}{\pgfqpoint{6.448575in}{5.446178in}}%
\pgfpathcurveto{\pgfqpoint{6.444457in}{5.450296in}}{\pgfqpoint{6.438871in}{5.452610in}}{\pgfqpoint{6.433047in}{5.452610in}}%
\pgfpathcurveto{\pgfqpoint{6.427223in}{5.452610in}}{\pgfqpoint{6.421637in}{5.450296in}}{\pgfqpoint{6.417518in}{5.446178in}}%
\pgfpathcurveto{\pgfqpoint{6.413400in}{5.442060in}}{\pgfqpoint{6.411086in}{5.436474in}}{\pgfqpoint{6.411086in}{5.430650in}}%
\pgfpathcurveto{\pgfqpoint{6.411086in}{5.424826in}}{\pgfqpoint{6.413400in}{5.419240in}}{\pgfqpoint{6.417518in}{5.415122in}}%
\pgfpathcurveto{\pgfqpoint{6.421637in}{5.411004in}}{\pgfqpoint{6.427223in}{5.408690in}}{\pgfqpoint{6.433047in}{5.408690in}}%
\pgfpathlineto{\pgfqpoint{6.433047in}{5.408690in}}%
\pgfpathclose%
\pgfusepath{stroke,fill}%
\end{pgfscope}%
\begin{pgfscope}%
\pgfpathrectangle{\pgfqpoint{1.073501in}{0.880000in}}{\pgfqpoint{6.052998in}{6.160000in}}%
\pgfusepath{clip}%
\pgfsetbuttcap%
\pgfsetroundjoin%
\definecolor{currentfill}{rgb}{0.200000,0.200000,0.800000}%
\pgfsetfillcolor{currentfill}%
\pgfsetlinewidth{1.003750pt}%
\definecolor{currentstroke}{rgb}{0.200000,0.200000,0.800000}%
\pgfsetstrokecolor{currentstroke}%
\pgfsetdash{}{0pt}%
\pgfpathmoveto{\pgfqpoint{6.279473in}{5.511646in}}%
\pgfpathcurveto{\pgfqpoint{6.285297in}{5.511646in}}{\pgfqpoint{6.290884in}{5.513960in}}{\pgfqpoint{6.295002in}{5.518078in}}%
\pgfpathcurveto{\pgfqpoint{6.299120in}{5.522197in}}{\pgfqpoint{6.301434in}{5.527783in}}{\pgfqpoint{6.301434in}{5.533607in}}%
\pgfpathcurveto{\pgfqpoint{6.301434in}{5.539431in}}{\pgfqpoint{6.299120in}{5.545017in}}{\pgfqpoint{6.295002in}{5.549135in}}%
\pgfpathcurveto{\pgfqpoint{6.290884in}{5.553253in}}{\pgfqpoint{6.285297in}{5.555567in}}{\pgfqpoint{6.279473in}{5.555567in}}%
\pgfpathcurveto{\pgfqpoint{6.273650in}{5.555567in}}{\pgfqpoint{6.268063in}{5.553253in}}{\pgfqpoint{6.263945in}{5.549135in}}%
\pgfpathcurveto{\pgfqpoint{6.259827in}{5.545017in}}{\pgfqpoint{6.257513in}{5.539431in}}{\pgfqpoint{6.257513in}{5.533607in}}%
\pgfpathcurveto{\pgfqpoint{6.257513in}{5.527783in}}{\pgfqpoint{6.259827in}{5.522197in}}{\pgfqpoint{6.263945in}{5.518078in}}%
\pgfpathcurveto{\pgfqpoint{6.268063in}{5.513960in}}{\pgfqpoint{6.273650in}{5.511646in}}{\pgfqpoint{6.279473in}{5.511646in}}%
\pgfpathlineto{\pgfqpoint{6.279473in}{5.511646in}}%
\pgfpathclose%
\pgfusepath{stroke,fill}%
\end{pgfscope}%
\begin{pgfscope}%
\pgfpathrectangle{\pgfqpoint{1.073501in}{0.880000in}}{\pgfqpoint{6.052998in}{6.160000in}}%
\pgfusepath{clip}%
\pgfsetbuttcap%
\pgfsetroundjoin%
\definecolor{currentfill}{rgb}{0.200000,0.200000,0.800000}%
\pgfsetfillcolor{currentfill}%
\pgfsetlinewidth{1.003750pt}%
\definecolor{currentstroke}{rgb}{0.200000,0.200000,0.800000}%
\pgfsetstrokecolor{currentstroke}%
\pgfsetdash{}{0pt}%
\pgfpathmoveto{\pgfqpoint{6.148367in}{5.623085in}}%
\pgfpathcurveto{\pgfqpoint{6.154191in}{5.623085in}}{\pgfqpoint{6.159777in}{5.625399in}}{\pgfqpoint{6.163895in}{5.629517in}}%
\pgfpathcurveto{\pgfqpoint{6.168013in}{5.633635in}}{\pgfqpoint{6.170327in}{5.639221in}}{\pgfqpoint{6.170327in}{5.645045in}}%
\pgfpathcurveto{\pgfqpoint{6.170327in}{5.650869in}}{\pgfqpoint{6.168013in}{5.656455in}}{\pgfqpoint{6.163895in}{5.660573in}}%
\pgfpathcurveto{\pgfqpoint{6.159777in}{5.664691in}}{\pgfqpoint{6.154191in}{5.667005in}}{\pgfqpoint{6.148367in}{5.667005in}}%
\pgfpathcurveto{\pgfqpoint{6.142543in}{5.667005in}}{\pgfqpoint{6.136957in}{5.664691in}}{\pgfqpoint{6.132838in}{5.660573in}}%
\pgfpathcurveto{\pgfqpoint{6.128720in}{5.656455in}}{\pgfqpoint{6.126406in}{5.650869in}}{\pgfqpoint{6.126406in}{5.645045in}}%
\pgfpathcurveto{\pgfqpoint{6.126406in}{5.639221in}}{\pgfqpoint{6.128720in}{5.633635in}}{\pgfqpoint{6.132838in}{5.629517in}}%
\pgfpathcurveto{\pgfqpoint{6.136957in}{5.625399in}}{\pgfqpoint{6.142543in}{5.623085in}}{\pgfqpoint{6.148367in}{5.623085in}}%
\pgfpathlineto{\pgfqpoint{6.148367in}{5.623085in}}%
\pgfpathclose%
\pgfusepath{stroke,fill}%
\end{pgfscope}%
\begin{pgfscope}%
\pgfpathrectangle{\pgfqpoint{1.073501in}{0.880000in}}{\pgfqpoint{6.052998in}{6.160000in}}%
\pgfusepath{clip}%
\pgfsetbuttcap%
\pgfsetroundjoin%
\definecolor{currentfill}{rgb}{0.200000,0.200000,0.800000}%
\pgfsetfillcolor{currentfill}%
\pgfsetlinewidth{1.003750pt}%
\definecolor{currentstroke}{rgb}{0.200000,0.200000,0.800000}%
\pgfsetstrokecolor{currentstroke}%
\pgfsetdash{}{0pt}%
\pgfpathmoveto{\pgfqpoint{6.088395in}{5.797303in}}%
\pgfpathcurveto{\pgfqpoint{6.094219in}{5.797303in}}{\pgfqpoint{6.099805in}{5.799616in}}{\pgfqpoint{6.103924in}{5.803735in}}%
\pgfpathcurveto{\pgfqpoint{6.108042in}{5.807853in}}{\pgfqpoint{6.110356in}{5.813439in}}{\pgfqpoint{6.110356in}{5.819263in}}%
\pgfpathcurveto{\pgfqpoint{6.110356in}{5.825087in}}{\pgfqpoint{6.108042in}{5.830673in}}{\pgfqpoint{6.103924in}{5.834791in}}%
\pgfpathcurveto{\pgfqpoint{6.099805in}{5.838909in}}{\pgfqpoint{6.094219in}{5.841223in}}{\pgfqpoint{6.088395in}{5.841223in}}%
\pgfpathcurveto{\pgfqpoint{6.082571in}{5.841223in}}{\pgfqpoint{6.076985in}{5.838909in}}{\pgfqpoint{6.072867in}{5.834791in}}%
\pgfpathcurveto{\pgfqpoint{6.068749in}{5.830673in}}{\pgfqpoint{6.066435in}{5.825087in}}{\pgfqpoint{6.066435in}{5.819263in}}%
\pgfpathcurveto{\pgfqpoint{6.066435in}{5.813439in}}{\pgfqpoint{6.068749in}{5.807853in}}{\pgfqpoint{6.072867in}{5.803735in}}%
\pgfpathcurveto{\pgfqpoint{6.076985in}{5.799616in}}{\pgfqpoint{6.082571in}{5.797303in}}{\pgfqpoint{6.088395in}{5.797303in}}%
\pgfpathlineto{\pgfqpoint{6.088395in}{5.797303in}}%
\pgfpathclose%
\pgfusepath{stroke,fill}%
\end{pgfscope}%
\begin{pgfscope}%
\pgfpathrectangle{\pgfqpoint{1.073501in}{0.880000in}}{\pgfqpoint{6.052998in}{6.160000in}}%
\pgfusepath{clip}%
\pgfsetbuttcap%
\pgfsetroundjoin%
\definecolor{currentfill}{rgb}{0.200000,0.200000,0.800000}%
\pgfsetfillcolor{currentfill}%
\pgfsetlinewidth{1.003750pt}%
\definecolor{currentstroke}{rgb}{0.200000,0.200000,0.800000}%
\pgfsetstrokecolor{currentstroke}%
\pgfsetdash{}{0pt}%
\pgfpathmoveto{\pgfqpoint{5.950079in}{5.901062in}}%
\pgfpathcurveto{\pgfqpoint{5.955903in}{5.901062in}}{\pgfqpoint{5.961489in}{5.903376in}}{\pgfqpoint{5.965608in}{5.907494in}}%
\pgfpathcurveto{\pgfqpoint{5.969726in}{5.911612in}}{\pgfqpoint{5.972040in}{5.917198in}}{\pgfqpoint{5.972040in}{5.923022in}}%
\pgfpathcurveto{\pgfqpoint{5.972040in}{5.928846in}}{\pgfqpoint{5.969726in}{5.934432in}}{\pgfqpoint{5.965608in}{5.938550in}}%
\pgfpathcurveto{\pgfqpoint{5.961489in}{5.942668in}}{\pgfqpoint{5.955903in}{5.944982in}}{\pgfqpoint{5.950079in}{5.944982in}}%
\pgfpathcurveto{\pgfqpoint{5.944255in}{5.944982in}}{\pgfqpoint{5.938669in}{5.942668in}}{\pgfqpoint{5.934551in}{5.938550in}}%
\pgfpathcurveto{\pgfqpoint{5.930433in}{5.934432in}}{\pgfqpoint{5.928119in}{5.928846in}}{\pgfqpoint{5.928119in}{5.923022in}}%
\pgfpathcurveto{\pgfqpoint{5.928119in}{5.917198in}}{\pgfqpoint{5.930433in}{5.911612in}}{\pgfqpoint{5.934551in}{5.907494in}}%
\pgfpathcurveto{\pgfqpoint{5.938669in}{5.903376in}}{\pgfqpoint{5.944255in}{5.901062in}}{\pgfqpoint{5.950079in}{5.901062in}}%
\pgfpathlineto{\pgfqpoint{5.950079in}{5.901062in}}%
\pgfpathclose%
\pgfusepath{stroke,fill}%
\end{pgfscope}%
\begin{pgfscope}%
\pgfpathrectangle{\pgfqpoint{1.073501in}{0.880000in}}{\pgfqpoint{6.052998in}{6.160000in}}%
\pgfusepath{clip}%
\pgfsetbuttcap%
\pgfsetroundjoin%
\definecolor{currentfill}{rgb}{0.200000,0.200000,0.800000}%
\pgfsetfillcolor{currentfill}%
\pgfsetlinewidth{1.003750pt}%
\definecolor{currentstroke}{rgb}{0.200000,0.200000,0.800000}%
\pgfsetstrokecolor{currentstroke}%
\pgfsetdash{}{0pt}%
\pgfpathmoveto{\pgfqpoint{5.812261in}{6.001887in}}%
\pgfpathcurveto{\pgfqpoint{5.818085in}{6.001887in}}{\pgfqpoint{5.823671in}{6.004201in}}{\pgfqpoint{5.827789in}{6.008319in}}%
\pgfpathcurveto{\pgfqpoint{5.831907in}{6.012437in}}{\pgfqpoint{5.834221in}{6.018023in}}{\pgfqpoint{5.834221in}{6.023847in}}%
\pgfpathcurveto{\pgfqpoint{5.834221in}{6.029671in}}{\pgfqpoint{5.831907in}{6.035257in}}{\pgfqpoint{5.827789in}{6.039375in}}%
\pgfpathcurveto{\pgfqpoint{5.823671in}{6.043493in}}{\pgfqpoint{5.818085in}{6.045807in}}{\pgfqpoint{5.812261in}{6.045807in}}%
\pgfpathcurveto{\pgfqpoint{5.806437in}{6.045807in}}{\pgfqpoint{5.800850in}{6.043493in}}{\pgfqpoint{5.796732in}{6.039375in}}%
\pgfpathcurveto{\pgfqpoint{5.792614in}{6.035257in}}{\pgfqpoint{5.790300in}{6.029671in}}{\pgfqpoint{5.790300in}{6.023847in}}%
\pgfpathcurveto{\pgfqpoint{5.790300in}{6.018023in}}{\pgfqpoint{5.792614in}{6.012437in}}{\pgfqpoint{5.796732in}{6.008319in}}%
\pgfpathcurveto{\pgfqpoint{5.800850in}{6.004201in}}{\pgfqpoint{5.806437in}{6.001887in}}{\pgfqpoint{5.812261in}{6.001887in}}%
\pgfpathlineto{\pgfqpoint{5.812261in}{6.001887in}}%
\pgfpathclose%
\pgfusepath{stroke,fill}%
\end{pgfscope}%
\begin{pgfscope}%
\pgfpathrectangle{\pgfqpoint{1.073501in}{0.880000in}}{\pgfqpoint{6.052998in}{6.160000in}}%
\pgfusepath{clip}%
\pgfsetbuttcap%
\pgfsetroundjoin%
\definecolor{currentfill}{rgb}{0.200000,0.200000,0.800000}%
\pgfsetfillcolor{currentfill}%
\pgfsetlinewidth{1.003750pt}%
\definecolor{currentstroke}{rgb}{0.200000,0.200000,0.800000}%
\pgfsetstrokecolor{currentstroke}%
\pgfsetdash{}{0pt}%
\pgfpathmoveto{\pgfqpoint{5.690350in}{6.123160in}}%
\pgfpathcurveto{\pgfqpoint{5.696173in}{6.123160in}}{\pgfqpoint{5.701760in}{6.125474in}}{\pgfqpoint{5.705878in}{6.129592in}}%
\pgfpathcurveto{\pgfqpoint{5.709996in}{6.133710in}}{\pgfqpoint{5.712310in}{6.139297in}}{\pgfqpoint{5.712310in}{6.145120in}}%
\pgfpathcurveto{\pgfqpoint{5.712310in}{6.150944in}}{\pgfqpoint{5.709996in}{6.156531in}}{\pgfqpoint{5.705878in}{6.160649in}}%
\pgfpathcurveto{\pgfqpoint{5.701760in}{6.164767in}}{\pgfqpoint{5.696173in}{6.167081in}}{\pgfqpoint{5.690350in}{6.167081in}}%
\pgfpathcurveto{\pgfqpoint{5.684526in}{6.167081in}}{\pgfqpoint{5.678939in}{6.164767in}}{\pgfqpoint{5.674821in}{6.160649in}}%
\pgfpathcurveto{\pgfqpoint{5.670703in}{6.156531in}}{\pgfqpoint{5.668389in}{6.150944in}}{\pgfqpoint{5.668389in}{6.145120in}}%
\pgfpathcurveto{\pgfqpoint{5.668389in}{6.139297in}}{\pgfqpoint{5.670703in}{6.133710in}}{\pgfqpoint{5.674821in}{6.129592in}}%
\pgfpathcurveto{\pgfqpoint{5.678939in}{6.125474in}}{\pgfqpoint{5.684526in}{6.123160in}}{\pgfqpoint{5.690350in}{6.123160in}}%
\pgfpathlineto{\pgfqpoint{5.690350in}{6.123160in}}%
\pgfpathclose%
\pgfusepath{stroke,fill}%
\end{pgfscope}%
\begin{pgfscope}%
\pgfpathrectangle{\pgfqpoint{1.073501in}{0.880000in}}{\pgfqpoint{6.052998in}{6.160000in}}%
\pgfusepath{clip}%
\pgfsetbuttcap%
\pgfsetroundjoin%
\definecolor{currentfill}{rgb}{0.200000,0.200000,0.800000}%
\pgfsetfillcolor{currentfill}%
\pgfsetlinewidth{1.003750pt}%
\definecolor{currentstroke}{rgb}{0.200000,0.200000,0.800000}%
\pgfsetstrokecolor{currentstroke}%
\pgfsetdash{}{0pt}%
\pgfpathmoveto{\pgfqpoint{5.544380in}{6.212412in}}%
\pgfpathcurveto{\pgfqpoint{5.550204in}{6.212412in}}{\pgfqpoint{5.555790in}{6.214726in}}{\pgfqpoint{5.559908in}{6.218844in}}%
\pgfpathcurveto{\pgfqpoint{5.564026in}{6.222962in}}{\pgfqpoint{5.566340in}{6.228548in}}{\pgfqpoint{5.566340in}{6.234372in}}%
\pgfpathcurveto{\pgfqpoint{5.566340in}{6.240196in}}{\pgfqpoint{5.564026in}{6.245782in}}{\pgfqpoint{5.559908in}{6.249901in}}%
\pgfpathcurveto{\pgfqpoint{5.555790in}{6.254019in}}{\pgfqpoint{5.550204in}{6.256333in}}{\pgfqpoint{5.544380in}{6.256333in}}%
\pgfpathcurveto{\pgfqpoint{5.538556in}{6.256333in}}{\pgfqpoint{5.532970in}{6.254019in}}{\pgfqpoint{5.528851in}{6.249901in}}%
\pgfpathcurveto{\pgfqpoint{5.524733in}{6.245782in}}{\pgfqpoint{5.522419in}{6.240196in}}{\pgfqpoint{5.522419in}{6.234372in}}%
\pgfpathcurveto{\pgfqpoint{5.522419in}{6.228548in}}{\pgfqpoint{5.524733in}{6.222962in}}{\pgfqpoint{5.528851in}{6.218844in}}%
\pgfpathcurveto{\pgfqpoint{5.532970in}{6.214726in}}{\pgfqpoint{5.538556in}{6.212412in}}{\pgfqpoint{5.544380in}{6.212412in}}%
\pgfpathlineto{\pgfqpoint{5.544380in}{6.212412in}}%
\pgfpathclose%
\pgfusepath{stroke,fill}%
\end{pgfscope}%
\begin{pgfscope}%
\pgfpathrectangle{\pgfqpoint{1.073501in}{0.880000in}}{\pgfqpoint{6.052998in}{6.160000in}}%
\pgfusepath{clip}%
\pgfsetbuttcap%
\pgfsetroundjoin%
\definecolor{currentfill}{rgb}{0.200000,0.200000,0.800000}%
\pgfsetfillcolor{currentfill}%
\pgfsetlinewidth{1.003750pt}%
\definecolor{currentstroke}{rgb}{0.200000,0.200000,0.800000}%
\pgfsetstrokecolor{currentstroke}%
\pgfsetdash{}{0pt}%
\pgfpathmoveto{\pgfqpoint{5.439283in}{6.377648in}}%
\pgfpathcurveto{\pgfqpoint{5.445106in}{6.377648in}}{\pgfqpoint{5.450693in}{6.379962in}}{\pgfqpoint{5.454811in}{6.384080in}}%
\pgfpathcurveto{\pgfqpoint{5.458929in}{6.388198in}}{\pgfqpoint{5.461243in}{6.393785in}}{\pgfqpoint{5.461243in}{6.399609in}}%
\pgfpathcurveto{\pgfqpoint{5.461243in}{6.405432in}}{\pgfqpoint{5.458929in}{6.411019in}}{\pgfqpoint{5.454811in}{6.415137in}}%
\pgfpathcurveto{\pgfqpoint{5.450693in}{6.419255in}}{\pgfqpoint{5.445106in}{6.421569in}}{\pgfqpoint{5.439283in}{6.421569in}}%
\pgfpathcurveto{\pgfqpoint{5.433459in}{6.421569in}}{\pgfqpoint{5.427872in}{6.419255in}}{\pgfqpoint{5.423754in}{6.415137in}}%
\pgfpathcurveto{\pgfqpoint{5.419636in}{6.411019in}}{\pgfqpoint{5.417322in}{6.405432in}}{\pgfqpoint{5.417322in}{6.399609in}}%
\pgfpathcurveto{\pgfqpoint{5.417322in}{6.393785in}}{\pgfqpoint{5.419636in}{6.388198in}}{\pgfqpoint{5.423754in}{6.384080in}}%
\pgfpathcurveto{\pgfqpoint{5.427872in}{6.379962in}}{\pgfqpoint{5.433459in}{6.377648in}}{\pgfqpoint{5.439283in}{6.377648in}}%
\pgfpathlineto{\pgfqpoint{5.439283in}{6.377648in}}%
\pgfpathclose%
\pgfusepath{stroke,fill}%
\end{pgfscope}%
\begin{pgfscope}%
\pgfpathrectangle{\pgfqpoint{1.073501in}{0.880000in}}{\pgfqpoint{6.052998in}{6.160000in}}%
\pgfusepath{clip}%
\pgfsetbuttcap%
\pgfsetroundjoin%
\definecolor{currentfill}{rgb}{0.200000,0.200000,0.800000}%
\pgfsetfillcolor{currentfill}%
\pgfsetlinewidth{1.003750pt}%
\definecolor{currentstroke}{rgb}{0.200000,0.200000,0.800000}%
\pgfsetstrokecolor{currentstroke}%
\pgfsetdash{}{0pt}%
\pgfpathmoveto{\pgfqpoint{5.264806in}{6.420283in}}%
\pgfpathcurveto{\pgfqpoint{5.270630in}{6.420283in}}{\pgfqpoint{5.276216in}{6.422597in}}{\pgfqpoint{5.280334in}{6.426715in}}%
\pgfpathcurveto{\pgfqpoint{5.284453in}{6.430834in}}{\pgfqpoint{5.286766in}{6.436420in}}{\pgfqpoint{5.286766in}{6.442244in}}%
\pgfpathcurveto{\pgfqpoint{5.286766in}{6.448068in}}{\pgfqpoint{5.284453in}{6.453654in}}{\pgfqpoint{5.280334in}{6.457772in}}%
\pgfpathcurveto{\pgfqpoint{5.276216in}{6.461890in}}{\pgfqpoint{5.270630in}{6.464204in}}{\pgfqpoint{5.264806in}{6.464204in}}%
\pgfpathcurveto{\pgfqpoint{5.258982in}{6.464204in}}{\pgfqpoint{5.253396in}{6.461890in}}{\pgfqpoint{5.249278in}{6.457772in}}%
\pgfpathcurveto{\pgfqpoint{5.245160in}{6.453654in}}{\pgfqpoint{5.242846in}{6.448068in}}{\pgfqpoint{5.242846in}{6.442244in}}%
\pgfpathcurveto{\pgfqpoint{5.242846in}{6.436420in}}{\pgfqpoint{5.245160in}{6.430834in}}{\pgfqpoint{5.249278in}{6.426715in}}%
\pgfpathcurveto{\pgfqpoint{5.253396in}{6.422597in}}{\pgfqpoint{5.258982in}{6.420283in}}{\pgfqpoint{5.264806in}{6.420283in}}%
\pgfpathlineto{\pgfqpoint{5.264806in}{6.420283in}}%
\pgfpathclose%
\pgfusepath{stroke,fill}%
\end{pgfscope}%
\begin{pgfscope}%
\pgfpathrectangle{\pgfqpoint{1.073501in}{0.880000in}}{\pgfqpoint{6.052998in}{6.160000in}}%
\pgfusepath{clip}%
\pgfsetbuttcap%
\pgfsetroundjoin%
\definecolor{currentfill}{rgb}{0.200000,0.200000,0.800000}%
\pgfsetfillcolor{currentfill}%
\pgfsetlinewidth{1.003750pt}%
\definecolor{currentstroke}{rgb}{0.200000,0.200000,0.800000}%
\pgfsetstrokecolor{currentstroke}%
\pgfsetdash{}{0pt}%
\pgfpathmoveto{\pgfqpoint{5.122860in}{6.536089in}}%
\pgfpathcurveto{\pgfqpoint{5.128684in}{6.536089in}}{\pgfqpoint{5.134270in}{6.538403in}}{\pgfqpoint{5.138388in}{6.542521in}}%
\pgfpathcurveto{\pgfqpoint{5.142506in}{6.546639in}}{\pgfqpoint{5.144820in}{6.552225in}}{\pgfqpoint{5.144820in}{6.558049in}}%
\pgfpathcurveto{\pgfqpoint{5.144820in}{6.563873in}}{\pgfqpoint{5.142506in}{6.569459in}}{\pgfqpoint{5.138388in}{6.573577in}}%
\pgfpathcurveto{\pgfqpoint{5.134270in}{6.577695in}}{\pgfqpoint{5.128684in}{6.580009in}}{\pgfqpoint{5.122860in}{6.580009in}}%
\pgfpathcurveto{\pgfqpoint{5.117036in}{6.580009in}}{\pgfqpoint{5.111450in}{6.577695in}}{\pgfqpoint{5.107332in}{6.573577in}}%
\pgfpathcurveto{\pgfqpoint{5.103214in}{6.569459in}}{\pgfqpoint{5.100900in}{6.563873in}}{\pgfqpoint{5.100900in}{6.558049in}}%
\pgfpathcurveto{\pgfqpoint{5.100900in}{6.552225in}}{\pgfqpoint{5.103214in}{6.546639in}}{\pgfqpoint{5.107332in}{6.542521in}}%
\pgfpathcurveto{\pgfqpoint{5.111450in}{6.538403in}}{\pgfqpoint{5.117036in}{6.536089in}}{\pgfqpoint{5.122860in}{6.536089in}}%
\pgfpathlineto{\pgfqpoint{5.122860in}{6.536089in}}%
\pgfpathclose%
\pgfusepath{stroke,fill}%
\end{pgfscope}%
\begin{pgfscope}%
\pgfpathrectangle{\pgfqpoint{1.073501in}{0.880000in}}{\pgfqpoint{6.052998in}{6.160000in}}%
\pgfusepath{clip}%
\pgfsetbuttcap%
\pgfsetroundjoin%
\definecolor{currentfill}{rgb}{0.200000,0.200000,0.800000}%
\pgfsetfillcolor{currentfill}%
\pgfsetlinewidth{1.003750pt}%
\definecolor{currentstroke}{rgb}{0.200000,0.200000,0.800000}%
\pgfsetstrokecolor{currentstroke}%
\pgfsetdash{}{0pt}%
\pgfpathmoveto{\pgfqpoint{4.912241in}{6.455614in}}%
\pgfpathcurveto{\pgfqpoint{4.918065in}{6.455614in}}{\pgfqpoint{4.923651in}{6.457928in}}{\pgfqpoint{4.927769in}{6.462046in}}%
\pgfpathcurveto{\pgfqpoint{4.931887in}{6.466164in}}{\pgfqpoint{4.934201in}{6.471750in}}{\pgfqpoint{4.934201in}{6.477574in}}%
\pgfpathcurveto{\pgfqpoint{4.934201in}{6.483398in}}{\pgfqpoint{4.931887in}{6.488985in}}{\pgfqpoint{4.927769in}{6.493103in}}%
\pgfpathcurveto{\pgfqpoint{4.923651in}{6.497221in}}{\pgfqpoint{4.918065in}{6.499535in}}{\pgfqpoint{4.912241in}{6.499535in}}%
\pgfpathcurveto{\pgfqpoint{4.906417in}{6.499535in}}{\pgfqpoint{4.900831in}{6.497221in}}{\pgfqpoint{4.896713in}{6.493103in}}%
\pgfpathcurveto{\pgfqpoint{4.892594in}{6.488985in}}{\pgfqpoint{4.890281in}{6.483398in}}{\pgfqpoint{4.890281in}{6.477574in}}%
\pgfpathcurveto{\pgfqpoint{4.890281in}{6.471750in}}{\pgfqpoint{4.892594in}{6.466164in}}{\pgfqpoint{4.896713in}{6.462046in}}%
\pgfpathcurveto{\pgfqpoint{4.900831in}{6.457928in}}{\pgfqpoint{4.906417in}{6.455614in}}{\pgfqpoint{4.912241in}{6.455614in}}%
\pgfpathlineto{\pgfqpoint{4.912241in}{6.455614in}}%
\pgfpathclose%
\pgfusepath{stroke,fill}%
\end{pgfscope}%
\begin{pgfscope}%
\pgfpathrectangle{\pgfqpoint{1.073501in}{0.880000in}}{\pgfqpoint{6.052998in}{6.160000in}}%
\pgfusepath{clip}%
\pgfsetbuttcap%
\pgfsetroundjoin%
\definecolor{currentfill}{rgb}{0.200000,0.200000,0.800000}%
\pgfsetfillcolor{currentfill}%
\pgfsetlinewidth{1.003750pt}%
\definecolor{currentstroke}{rgb}{0.200000,0.200000,0.800000}%
\pgfsetstrokecolor{currentstroke}%
\pgfsetdash{}{0pt}%
\pgfpathmoveto{\pgfqpoint{4.762709in}{6.551954in}}%
\pgfpathcurveto{\pgfqpoint{4.768533in}{6.551954in}}{\pgfqpoint{4.774120in}{6.554268in}}{\pgfqpoint{4.778238in}{6.558386in}}%
\pgfpathcurveto{\pgfqpoint{4.782356in}{6.562504in}}{\pgfqpoint{4.784670in}{6.568090in}}{\pgfqpoint{4.784670in}{6.573914in}}%
\pgfpathcurveto{\pgfqpoint{4.784670in}{6.579738in}}{\pgfqpoint{4.782356in}{6.585324in}}{\pgfqpoint{4.778238in}{6.589442in}}%
\pgfpathcurveto{\pgfqpoint{4.774120in}{6.593560in}}{\pgfqpoint{4.768533in}{6.595874in}}{\pgfqpoint{4.762709in}{6.595874in}}%
\pgfpathcurveto{\pgfqpoint{4.756886in}{6.595874in}}{\pgfqpoint{4.751299in}{6.593560in}}{\pgfqpoint{4.747181in}{6.589442in}}%
\pgfpathcurveto{\pgfqpoint{4.743063in}{6.585324in}}{\pgfqpoint{4.740749in}{6.579738in}}{\pgfqpoint{4.740749in}{6.573914in}}%
\pgfpathcurveto{\pgfqpoint{4.740749in}{6.568090in}}{\pgfqpoint{4.743063in}{6.562504in}}{\pgfqpoint{4.747181in}{6.558386in}}%
\pgfpathcurveto{\pgfqpoint{4.751299in}{6.554268in}}{\pgfqpoint{4.756886in}{6.551954in}}{\pgfqpoint{4.762709in}{6.551954in}}%
\pgfpathlineto{\pgfqpoint{4.762709in}{6.551954in}}%
\pgfpathclose%
\pgfusepath{stroke,fill}%
\end{pgfscope}%
\begin{pgfscope}%
\pgfpathrectangle{\pgfqpoint{1.073501in}{0.880000in}}{\pgfqpoint{6.052998in}{6.160000in}}%
\pgfusepath{clip}%
\pgfsetbuttcap%
\pgfsetroundjoin%
\definecolor{currentfill}{rgb}{0.200000,0.200000,0.800000}%
\pgfsetfillcolor{currentfill}%
\pgfsetlinewidth{1.003750pt}%
\definecolor{currentstroke}{rgb}{0.200000,0.200000,0.800000}%
\pgfsetstrokecolor{currentstroke}%
\pgfsetdash{}{0pt}%
\pgfpathmoveto{\pgfqpoint{4.604967in}{6.644366in}}%
\pgfpathcurveto{\pgfqpoint{4.610791in}{6.644366in}}{\pgfqpoint{4.616377in}{6.646679in}}{\pgfqpoint{4.620496in}{6.650798in}}%
\pgfpathcurveto{\pgfqpoint{4.624614in}{6.654916in}}{\pgfqpoint{4.626928in}{6.660502in}}{\pgfqpoint{4.626928in}{6.666326in}}%
\pgfpathcurveto{\pgfqpoint{4.626928in}{6.672150in}}{\pgfqpoint{4.624614in}{6.677736in}}{\pgfqpoint{4.620496in}{6.681854in}}%
\pgfpathcurveto{\pgfqpoint{4.616377in}{6.685972in}}{\pgfqpoint{4.610791in}{6.688286in}}{\pgfqpoint{4.604967in}{6.688286in}}%
\pgfpathcurveto{\pgfqpoint{4.599143in}{6.688286in}}{\pgfqpoint{4.593557in}{6.685972in}}{\pgfqpoint{4.589439in}{6.681854in}}%
\pgfpathcurveto{\pgfqpoint{4.585321in}{6.677736in}}{\pgfqpoint{4.583007in}{6.672150in}}{\pgfqpoint{4.583007in}{6.666326in}}%
\pgfpathcurveto{\pgfqpoint{4.583007in}{6.660502in}}{\pgfqpoint{4.585321in}{6.654916in}}{\pgfqpoint{4.589439in}{6.650798in}}%
\pgfpathcurveto{\pgfqpoint{4.593557in}{6.646679in}}{\pgfqpoint{4.599143in}{6.644366in}}{\pgfqpoint{4.604967in}{6.644366in}}%
\pgfpathlineto{\pgfqpoint{4.604967in}{6.644366in}}%
\pgfpathclose%
\pgfusepath{stroke,fill}%
\end{pgfscope}%
\begin{pgfscope}%
\pgfpathrectangle{\pgfqpoint{1.073501in}{0.880000in}}{\pgfqpoint{6.052998in}{6.160000in}}%
\pgfusepath{clip}%
\pgfsetbuttcap%
\pgfsetroundjoin%
\definecolor{currentfill}{rgb}{0.200000,0.200000,0.800000}%
\pgfsetfillcolor{currentfill}%
\pgfsetlinewidth{1.003750pt}%
\definecolor{currentstroke}{rgb}{0.200000,0.200000,0.800000}%
\pgfsetstrokecolor{currentstroke}%
\pgfsetdash{}{0pt}%
\pgfpathmoveto{\pgfqpoint{4.439246in}{6.738040in}}%
\pgfpathcurveto{\pgfqpoint{4.445070in}{6.738040in}}{\pgfqpoint{4.450656in}{6.740354in}}{\pgfqpoint{4.454774in}{6.744472in}}%
\pgfpathcurveto{\pgfqpoint{4.458893in}{6.748590in}}{\pgfqpoint{4.461206in}{6.754176in}}{\pgfqpoint{4.461206in}{6.760000in}}%
\pgfpathcurveto{\pgfqpoint{4.461206in}{6.765824in}}{\pgfqpoint{4.458893in}{6.771410in}}{\pgfqpoint{4.454774in}{6.775528in}}%
\pgfpathcurveto{\pgfqpoint{4.450656in}{6.779646in}}{\pgfqpoint{4.445070in}{6.781960in}}{\pgfqpoint{4.439246in}{6.781960in}}%
\pgfpathcurveto{\pgfqpoint{4.433422in}{6.781960in}}{\pgfqpoint{4.427836in}{6.779646in}}{\pgfqpoint{4.423718in}{6.775528in}}%
\pgfpathcurveto{\pgfqpoint{4.419600in}{6.771410in}}{\pgfqpoint{4.417286in}{6.765824in}}{\pgfqpoint{4.417286in}{6.760000in}}%
\pgfpathcurveto{\pgfqpoint{4.417286in}{6.754176in}}{\pgfqpoint{4.419600in}{6.748590in}}{\pgfqpoint{4.423718in}{6.744472in}}%
\pgfpathcurveto{\pgfqpoint{4.427836in}{6.740354in}}{\pgfqpoint{4.433422in}{6.738040in}}{\pgfqpoint{4.439246in}{6.738040in}}%
\pgfpathlineto{\pgfqpoint{4.439246in}{6.738040in}}%
\pgfpathclose%
\pgfusepath{stroke,fill}%
\end{pgfscope}%
\begin{pgfscope}%
\pgfpathrectangle{\pgfqpoint{1.073501in}{0.880000in}}{\pgfqpoint{6.052998in}{6.160000in}}%
\pgfusepath{clip}%
\pgfsetbuttcap%
\pgfsetroundjoin%
\definecolor{currentfill}{rgb}{0.200000,0.200000,0.800000}%
\pgfsetfillcolor{currentfill}%
\pgfsetlinewidth{1.003750pt}%
\definecolor{currentstroke}{rgb}{0.200000,0.200000,0.800000}%
\pgfsetstrokecolor{currentstroke}%
\pgfsetdash{}{0pt}%
\pgfpathmoveto{\pgfqpoint{4.254490in}{6.631341in}}%
\pgfpathcurveto{\pgfqpoint{4.260314in}{6.631341in}}{\pgfqpoint{4.265900in}{6.633655in}}{\pgfqpoint{4.270018in}{6.637773in}}%
\pgfpathcurveto{\pgfqpoint{4.274137in}{6.641891in}}{\pgfqpoint{4.276451in}{6.647477in}}{\pgfqpoint{4.276451in}{6.653301in}}%
\pgfpathcurveto{\pgfqpoint{4.276451in}{6.659125in}}{\pgfqpoint{4.274137in}{6.664711in}}{\pgfqpoint{4.270018in}{6.668829in}}%
\pgfpathcurveto{\pgfqpoint{4.265900in}{6.672947in}}{\pgfqpoint{4.260314in}{6.675261in}}{\pgfqpoint{4.254490in}{6.675261in}}%
\pgfpathcurveto{\pgfqpoint{4.248666in}{6.675261in}}{\pgfqpoint{4.243080in}{6.672947in}}{\pgfqpoint{4.238962in}{6.668829in}}%
\pgfpathcurveto{\pgfqpoint{4.234844in}{6.664711in}}{\pgfqpoint{4.232530in}{6.659125in}}{\pgfqpoint{4.232530in}{6.653301in}}%
\pgfpathcurveto{\pgfqpoint{4.232530in}{6.647477in}}{\pgfqpoint{4.234844in}{6.641891in}}{\pgfqpoint{4.238962in}{6.637773in}}%
\pgfpathcurveto{\pgfqpoint{4.243080in}{6.633655in}}{\pgfqpoint{4.248666in}{6.631341in}}{\pgfqpoint{4.254490in}{6.631341in}}%
\pgfpathlineto{\pgfqpoint{4.254490in}{6.631341in}}%
\pgfpathclose%
\pgfusepath{stroke,fill}%
\end{pgfscope}%
\begin{pgfscope}%
\pgfpathrectangle{\pgfqpoint{1.073501in}{0.880000in}}{\pgfqpoint{6.052998in}{6.160000in}}%
\pgfusepath{clip}%
\pgfsetbuttcap%
\pgfsetroundjoin%
\definecolor{currentfill}{rgb}{0.200000,0.200000,0.800000}%
\pgfsetfillcolor{currentfill}%
\pgfsetlinewidth{1.003750pt}%
\definecolor{currentstroke}{rgb}{0.200000,0.200000,0.800000}%
\pgfsetstrokecolor{currentstroke}%
\pgfsetdash{}{0pt}%
\pgfpathmoveto{\pgfqpoint{4.082482in}{6.643683in}}%
\pgfpathcurveto{\pgfqpoint{4.088305in}{6.643683in}}{\pgfqpoint{4.093892in}{6.645997in}}{\pgfqpoint{4.098010in}{6.650115in}}%
\pgfpathcurveto{\pgfqpoint{4.102128in}{6.654233in}}{\pgfqpoint{4.104442in}{6.659819in}}{\pgfqpoint{4.104442in}{6.665643in}}%
\pgfpathcurveto{\pgfqpoint{4.104442in}{6.671467in}}{\pgfqpoint{4.102128in}{6.677053in}}{\pgfqpoint{4.098010in}{6.681172in}}%
\pgfpathcurveto{\pgfqpoint{4.093892in}{6.685290in}}{\pgfqpoint{4.088305in}{6.687604in}}{\pgfqpoint{4.082482in}{6.687604in}}%
\pgfpathcurveto{\pgfqpoint{4.076658in}{6.687604in}}{\pgfqpoint{4.071071in}{6.685290in}}{\pgfqpoint{4.066953in}{6.681172in}}%
\pgfpathcurveto{\pgfqpoint{4.062835in}{6.677053in}}{\pgfqpoint{4.060521in}{6.671467in}}{\pgfqpoint{4.060521in}{6.665643in}}%
\pgfpathcurveto{\pgfqpoint{4.060521in}{6.659819in}}{\pgfqpoint{4.062835in}{6.654233in}}{\pgfqpoint{4.066953in}{6.650115in}}%
\pgfpathcurveto{\pgfqpoint{4.071071in}{6.645997in}}{\pgfqpoint{4.076658in}{6.643683in}}{\pgfqpoint{4.082482in}{6.643683in}}%
\pgfpathlineto{\pgfqpoint{4.082482in}{6.643683in}}%
\pgfpathclose%
\pgfusepath{stroke,fill}%
\end{pgfscope}%
\begin{pgfscope}%
\pgfpathrectangle{\pgfqpoint{1.073501in}{0.880000in}}{\pgfqpoint{6.052998in}{6.160000in}}%
\pgfusepath{clip}%
\pgfsetbuttcap%
\pgfsetroundjoin%
\definecolor{currentfill}{rgb}{0.200000,0.200000,0.800000}%
\pgfsetfillcolor{currentfill}%
\pgfsetlinewidth{1.003750pt}%
\definecolor{currentstroke}{rgb}{0.200000,0.200000,0.800000}%
\pgfsetstrokecolor{currentstroke}%
\pgfsetdash{}{0pt}%
\pgfpathmoveto{\pgfqpoint{3.907825in}{6.665207in}}%
\pgfpathcurveto{\pgfqpoint{3.913649in}{6.665207in}}{\pgfqpoint{3.919235in}{6.667521in}}{\pgfqpoint{3.923354in}{6.671639in}}%
\pgfpathcurveto{\pgfqpoint{3.927472in}{6.675757in}}{\pgfqpoint{3.929786in}{6.681343in}}{\pgfqpoint{3.929786in}{6.687167in}}%
\pgfpathcurveto{\pgfqpoint{3.929786in}{6.692991in}}{\pgfqpoint{3.927472in}{6.698577in}}{\pgfqpoint{3.923354in}{6.702696in}}%
\pgfpathcurveto{\pgfqpoint{3.919235in}{6.706814in}}{\pgfqpoint{3.913649in}{6.709128in}}{\pgfqpoint{3.907825in}{6.709128in}}%
\pgfpathcurveto{\pgfqpoint{3.902001in}{6.709128in}}{\pgfqpoint{3.896415in}{6.706814in}}{\pgfqpoint{3.892297in}{6.702696in}}%
\pgfpathcurveto{\pgfqpoint{3.888179in}{6.698577in}}{\pgfqpoint{3.885865in}{6.692991in}}{\pgfqpoint{3.885865in}{6.687167in}}%
\pgfpathcurveto{\pgfqpoint{3.885865in}{6.681343in}}{\pgfqpoint{3.888179in}{6.675757in}}{\pgfqpoint{3.892297in}{6.671639in}}%
\pgfpathcurveto{\pgfqpoint{3.896415in}{6.667521in}}{\pgfqpoint{3.902001in}{6.665207in}}{\pgfqpoint{3.907825in}{6.665207in}}%
\pgfpathlineto{\pgfqpoint{3.907825in}{6.665207in}}%
\pgfpathclose%
\pgfusepath{stroke,fill}%
\end{pgfscope}%
\begin{pgfscope}%
\pgfpathrectangle{\pgfqpoint{1.073501in}{0.880000in}}{\pgfqpoint{6.052998in}{6.160000in}}%
\pgfusepath{clip}%
\pgfsetbuttcap%
\pgfsetroundjoin%
\definecolor{currentfill}{rgb}{0.200000,0.200000,0.800000}%
\pgfsetfillcolor{currentfill}%
\pgfsetlinewidth{1.003750pt}%
\definecolor{currentstroke}{rgb}{0.200000,0.200000,0.800000}%
\pgfsetstrokecolor{currentstroke}%
\pgfsetdash{}{0pt}%
\pgfpathmoveto{\pgfqpoint{3.739845in}{6.608834in}}%
\pgfpathcurveto{\pgfqpoint{3.745669in}{6.608834in}}{\pgfqpoint{3.751256in}{6.611148in}}{\pgfqpoint{3.755374in}{6.615266in}}%
\pgfpathcurveto{\pgfqpoint{3.759492in}{6.619384in}}{\pgfqpoint{3.761806in}{6.624970in}}{\pgfqpoint{3.761806in}{6.630794in}}%
\pgfpathcurveto{\pgfqpoint{3.761806in}{6.636618in}}{\pgfqpoint{3.759492in}{6.642204in}}{\pgfqpoint{3.755374in}{6.646322in}}%
\pgfpathcurveto{\pgfqpoint{3.751256in}{6.650440in}}{\pgfqpoint{3.745669in}{6.652754in}}{\pgfqpoint{3.739845in}{6.652754in}}%
\pgfpathcurveto{\pgfqpoint{3.734022in}{6.652754in}}{\pgfqpoint{3.728435in}{6.650440in}}{\pgfqpoint{3.724317in}{6.646322in}}%
\pgfpathcurveto{\pgfqpoint{3.720199in}{6.642204in}}{\pgfqpoint{3.717885in}{6.636618in}}{\pgfqpoint{3.717885in}{6.630794in}}%
\pgfpathcurveto{\pgfqpoint{3.717885in}{6.624970in}}{\pgfqpoint{3.720199in}{6.619384in}}{\pgfqpoint{3.724317in}{6.615266in}}%
\pgfpathcurveto{\pgfqpoint{3.728435in}{6.611148in}}{\pgfqpoint{3.734022in}{6.608834in}}{\pgfqpoint{3.739845in}{6.608834in}}%
\pgfpathlineto{\pgfqpoint{3.739845in}{6.608834in}}%
\pgfpathclose%
\pgfusepath{stroke,fill}%
\end{pgfscope}%
\begin{pgfscope}%
\pgfpathrectangle{\pgfqpoint{1.073501in}{0.880000in}}{\pgfqpoint{6.052998in}{6.160000in}}%
\pgfusepath{clip}%
\pgfsetbuttcap%
\pgfsetroundjoin%
\definecolor{currentfill}{rgb}{0.200000,0.200000,0.800000}%
\pgfsetfillcolor{currentfill}%
\pgfsetlinewidth{1.003750pt}%
\definecolor{currentstroke}{rgb}{0.200000,0.200000,0.800000}%
\pgfsetstrokecolor{currentstroke}%
\pgfsetdash{}{0pt}%
\pgfpathmoveto{\pgfqpoint{3.588573in}{6.492366in}}%
\pgfpathcurveto{\pgfqpoint{3.594397in}{6.492366in}}{\pgfqpoint{3.599983in}{6.494680in}}{\pgfqpoint{3.604101in}{6.498798in}}%
\pgfpathcurveto{\pgfqpoint{3.608220in}{6.502916in}}{\pgfqpoint{3.610533in}{6.508502in}}{\pgfqpoint{3.610533in}{6.514326in}}%
\pgfpathcurveto{\pgfqpoint{3.610533in}{6.520150in}}{\pgfqpoint{3.608220in}{6.525736in}}{\pgfqpoint{3.604101in}{6.529854in}}%
\pgfpathcurveto{\pgfqpoint{3.599983in}{6.533973in}}{\pgfqpoint{3.594397in}{6.536286in}}{\pgfqpoint{3.588573in}{6.536286in}}%
\pgfpathcurveto{\pgfqpoint{3.582749in}{6.536286in}}{\pgfqpoint{3.577163in}{6.533973in}}{\pgfqpoint{3.573045in}{6.529854in}}%
\pgfpathcurveto{\pgfqpoint{3.568927in}{6.525736in}}{\pgfqpoint{3.566613in}{6.520150in}}{\pgfqpoint{3.566613in}{6.514326in}}%
\pgfpathcurveto{\pgfqpoint{3.566613in}{6.508502in}}{\pgfqpoint{3.568927in}{6.502916in}}{\pgfqpoint{3.573045in}{6.498798in}}%
\pgfpathcurveto{\pgfqpoint{3.577163in}{6.494680in}}{\pgfqpoint{3.582749in}{6.492366in}}{\pgfqpoint{3.588573in}{6.492366in}}%
\pgfpathlineto{\pgfqpoint{3.588573in}{6.492366in}}%
\pgfpathclose%
\pgfusepath{stroke,fill}%
\end{pgfscope}%
\begin{pgfscope}%
\pgfpathrectangle{\pgfqpoint{1.073501in}{0.880000in}}{\pgfqpoint{6.052998in}{6.160000in}}%
\pgfusepath{clip}%
\pgfsetbuttcap%
\pgfsetroundjoin%
\definecolor{currentfill}{rgb}{0.200000,0.200000,0.800000}%
\pgfsetfillcolor{currentfill}%
\pgfsetlinewidth{1.003750pt}%
\definecolor{currentstroke}{rgb}{0.200000,0.200000,0.800000}%
\pgfsetstrokecolor{currentstroke}%
\pgfsetdash{}{0pt}%
\pgfpathmoveto{\pgfqpoint{3.405887in}{6.529041in}}%
\pgfpathcurveto{\pgfqpoint{3.411711in}{6.529041in}}{\pgfqpoint{3.417297in}{6.531354in}}{\pgfqpoint{3.421415in}{6.535473in}}%
\pgfpathcurveto{\pgfqpoint{3.425534in}{6.539591in}}{\pgfqpoint{3.427847in}{6.545177in}}{\pgfqpoint{3.427847in}{6.551001in}}%
\pgfpathcurveto{\pgfqpoint{3.427847in}{6.556825in}}{\pgfqpoint{3.425534in}{6.562411in}}{\pgfqpoint{3.421415in}{6.566529in}}%
\pgfpathcurveto{\pgfqpoint{3.417297in}{6.570647in}}{\pgfqpoint{3.411711in}{6.572961in}}{\pgfqpoint{3.405887in}{6.572961in}}%
\pgfpathcurveto{\pgfqpoint{3.400063in}{6.572961in}}{\pgfqpoint{3.394477in}{6.570647in}}{\pgfqpoint{3.390359in}{6.566529in}}%
\pgfpathcurveto{\pgfqpoint{3.386241in}{6.562411in}}{\pgfqpoint{3.383927in}{6.556825in}}{\pgfqpoint{3.383927in}{6.551001in}}%
\pgfpathcurveto{\pgfqpoint{3.383927in}{6.545177in}}{\pgfqpoint{3.386241in}{6.539591in}}{\pgfqpoint{3.390359in}{6.535473in}}%
\pgfpathcurveto{\pgfqpoint{3.394477in}{6.531354in}}{\pgfqpoint{3.400063in}{6.529041in}}{\pgfqpoint{3.405887in}{6.529041in}}%
\pgfpathlineto{\pgfqpoint{3.405887in}{6.529041in}}%
\pgfpathclose%
\pgfusepath{stroke,fill}%
\end{pgfscope}%
\begin{pgfscope}%
\pgfpathrectangle{\pgfqpoint{1.073501in}{0.880000in}}{\pgfqpoint{6.052998in}{6.160000in}}%
\pgfusepath{clip}%
\pgfsetbuttcap%
\pgfsetroundjoin%
\definecolor{currentfill}{rgb}{0.200000,0.200000,0.800000}%
\pgfsetfillcolor{currentfill}%
\pgfsetlinewidth{1.003750pt}%
\definecolor{currentstroke}{rgb}{0.200000,0.200000,0.800000}%
\pgfsetstrokecolor{currentstroke}%
\pgfsetdash{}{0pt}%
\pgfpathmoveto{\pgfqpoint{3.270581in}{6.396231in}}%
\pgfpathcurveto{\pgfqpoint{3.276405in}{6.396231in}}{\pgfqpoint{3.281991in}{6.398545in}}{\pgfqpoint{3.286109in}{6.402663in}}%
\pgfpathcurveto{\pgfqpoint{3.290227in}{6.406781in}}{\pgfqpoint{3.292541in}{6.412368in}}{\pgfqpoint{3.292541in}{6.418192in}}%
\pgfpathcurveto{\pgfqpoint{3.292541in}{6.424016in}}{\pgfqpoint{3.290227in}{6.429602in}}{\pgfqpoint{3.286109in}{6.433720in}}%
\pgfpathcurveto{\pgfqpoint{3.281991in}{6.437838in}}{\pgfqpoint{3.276405in}{6.440152in}}{\pgfqpoint{3.270581in}{6.440152in}}%
\pgfpathcurveto{\pgfqpoint{3.264757in}{6.440152in}}{\pgfqpoint{3.259171in}{6.437838in}}{\pgfqpoint{3.255052in}{6.433720in}}%
\pgfpathcurveto{\pgfqpoint{3.250934in}{6.429602in}}{\pgfqpoint{3.248620in}{6.424016in}}{\pgfqpoint{3.248620in}{6.418192in}}%
\pgfpathcurveto{\pgfqpoint{3.248620in}{6.412368in}}{\pgfqpoint{3.250934in}{6.406781in}}{\pgfqpoint{3.255052in}{6.402663in}}%
\pgfpathcurveto{\pgfqpoint{3.259171in}{6.398545in}}{\pgfqpoint{3.264757in}{6.396231in}}{\pgfqpoint{3.270581in}{6.396231in}}%
\pgfpathlineto{\pgfqpoint{3.270581in}{6.396231in}}%
\pgfpathclose%
\pgfusepath{stroke,fill}%
\end{pgfscope}%
\begin{pgfscope}%
\pgfpathrectangle{\pgfqpoint{1.073501in}{0.880000in}}{\pgfqpoint{6.052998in}{6.160000in}}%
\pgfusepath{clip}%
\pgfsetbuttcap%
\pgfsetroundjoin%
\definecolor{currentfill}{rgb}{0.200000,0.200000,0.800000}%
\pgfsetfillcolor{currentfill}%
\pgfsetlinewidth{1.003750pt}%
\definecolor{currentstroke}{rgb}{0.200000,0.200000,0.800000}%
\pgfsetstrokecolor{currentstroke}%
\pgfsetdash{}{0pt}%
\pgfpathmoveto{\pgfqpoint{3.098172in}{6.378676in}}%
\pgfpathcurveto{\pgfqpoint{3.103996in}{6.378676in}}{\pgfqpoint{3.109582in}{6.380990in}}{\pgfqpoint{3.113700in}{6.385108in}}%
\pgfpathcurveto{\pgfqpoint{3.117818in}{6.389227in}}{\pgfqpoint{3.120132in}{6.394813in}}{\pgfqpoint{3.120132in}{6.400637in}}%
\pgfpathcurveto{\pgfqpoint{3.120132in}{6.406461in}}{\pgfqpoint{3.117818in}{6.412047in}}{\pgfqpoint{3.113700in}{6.416165in}}%
\pgfpathcurveto{\pgfqpoint{3.109582in}{6.420283in}}{\pgfqpoint{3.103996in}{6.422597in}}{\pgfqpoint{3.098172in}{6.422597in}}%
\pgfpathcurveto{\pgfqpoint{3.092348in}{6.422597in}}{\pgfqpoint{3.086762in}{6.420283in}}{\pgfqpoint{3.082644in}{6.416165in}}%
\pgfpathcurveto{\pgfqpoint{3.078526in}{6.412047in}}{\pgfqpoint{3.076212in}{6.406461in}}{\pgfqpoint{3.076212in}{6.400637in}}%
\pgfpathcurveto{\pgfqpoint{3.076212in}{6.394813in}}{\pgfqpoint{3.078526in}{6.389227in}}{\pgfqpoint{3.082644in}{6.385108in}}%
\pgfpathcurveto{\pgfqpoint{3.086762in}{6.380990in}}{\pgfqpoint{3.092348in}{6.378676in}}{\pgfqpoint{3.098172in}{6.378676in}}%
\pgfpathlineto{\pgfqpoint{3.098172in}{6.378676in}}%
\pgfpathclose%
\pgfusepath{stroke,fill}%
\end{pgfscope}%
\begin{pgfscope}%
\pgfpathrectangle{\pgfqpoint{1.073501in}{0.880000in}}{\pgfqpoint{6.052998in}{6.160000in}}%
\pgfusepath{clip}%
\pgfsetbuttcap%
\pgfsetroundjoin%
\definecolor{currentfill}{rgb}{0.200000,0.200000,0.800000}%
\pgfsetfillcolor{currentfill}%
\pgfsetlinewidth{1.003750pt}%
\definecolor{currentstroke}{rgb}{0.200000,0.200000,0.800000}%
\pgfsetstrokecolor{currentstroke}%
\pgfsetdash{}{0pt}%
\pgfpathmoveto{\pgfqpoint{2.896276in}{6.406216in}}%
\pgfpathcurveto{\pgfqpoint{2.902100in}{6.406216in}}{\pgfqpoint{2.907686in}{6.408530in}}{\pgfqpoint{2.911804in}{6.412648in}}%
\pgfpathcurveto{\pgfqpoint{2.915922in}{6.416766in}}{\pgfqpoint{2.918236in}{6.422352in}}{\pgfqpoint{2.918236in}{6.428176in}}%
\pgfpathcurveto{\pgfqpoint{2.918236in}{6.434000in}}{\pgfqpoint{2.915922in}{6.439587in}}{\pgfqpoint{2.911804in}{6.443705in}}%
\pgfpathcurveto{\pgfqpoint{2.907686in}{6.447823in}}{\pgfqpoint{2.902100in}{6.450137in}}{\pgfqpoint{2.896276in}{6.450137in}}%
\pgfpathcurveto{\pgfqpoint{2.890452in}{6.450137in}}{\pgfqpoint{2.884866in}{6.447823in}}{\pgfqpoint{2.880748in}{6.443705in}}%
\pgfpathcurveto{\pgfqpoint{2.876630in}{6.439587in}}{\pgfqpoint{2.874316in}{6.434000in}}{\pgfqpoint{2.874316in}{6.428176in}}%
\pgfpathcurveto{\pgfqpoint{2.874316in}{6.422352in}}{\pgfqpoint{2.876630in}{6.416766in}}{\pgfqpoint{2.880748in}{6.412648in}}%
\pgfpathcurveto{\pgfqpoint{2.884866in}{6.408530in}}{\pgfqpoint{2.890452in}{6.406216in}}{\pgfqpoint{2.896276in}{6.406216in}}%
\pgfpathlineto{\pgfqpoint{2.896276in}{6.406216in}}%
\pgfpathclose%
\pgfusepath{stroke,fill}%
\end{pgfscope}%
\begin{pgfscope}%
\pgfpathrectangle{\pgfqpoint{1.073501in}{0.880000in}}{\pgfqpoint{6.052998in}{6.160000in}}%
\pgfusepath{clip}%
\pgfsetbuttcap%
\pgfsetroundjoin%
\definecolor{currentfill}{rgb}{0.200000,0.200000,0.800000}%
\pgfsetfillcolor{currentfill}%
\pgfsetlinewidth{1.003750pt}%
\definecolor{currentstroke}{rgb}{0.200000,0.200000,0.800000}%
\pgfsetstrokecolor{currentstroke}%
\pgfsetdash{}{0pt}%
\pgfpathmoveto{\pgfqpoint{2.771689in}{6.270848in}}%
\pgfpathcurveto{\pgfqpoint{2.777513in}{6.270848in}}{\pgfqpoint{2.783099in}{6.273162in}}{\pgfqpoint{2.787217in}{6.277280in}}%
\pgfpathcurveto{\pgfqpoint{2.791335in}{6.281398in}}{\pgfqpoint{2.793649in}{6.286985in}}{\pgfqpoint{2.793649in}{6.292809in}}%
\pgfpathcurveto{\pgfqpoint{2.793649in}{6.298632in}}{\pgfqpoint{2.791335in}{6.304219in}}{\pgfqpoint{2.787217in}{6.308337in}}%
\pgfpathcurveto{\pgfqpoint{2.783099in}{6.312455in}}{\pgfqpoint{2.777513in}{6.314769in}}{\pgfqpoint{2.771689in}{6.314769in}}%
\pgfpathcurveto{\pgfqpoint{2.765865in}{6.314769in}}{\pgfqpoint{2.760279in}{6.312455in}}{\pgfqpoint{2.756160in}{6.308337in}}%
\pgfpathcurveto{\pgfqpoint{2.752042in}{6.304219in}}{\pgfqpoint{2.749728in}{6.298632in}}{\pgfqpoint{2.749728in}{6.292809in}}%
\pgfpathcurveto{\pgfqpoint{2.749728in}{6.286985in}}{\pgfqpoint{2.752042in}{6.281398in}}{\pgfqpoint{2.756160in}{6.277280in}}%
\pgfpathcurveto{\pgfqpoint{2.760279in}{6.273162in}}{\pgfqpoint{2.765865in}{6.270848in}}{\pgfqpoint{2.771689in}{6.270848in}}%
\pgfpathlineto{\pgfqpoint{2.771689in}{6.270848in}}%
\pgfpathclose%
\pgfusepath{stroke,fill}%
\end{pgfscope}%
\begin{pgfscope}%
\pgfpathrectangle{\pgfqpoint{1.073501in}{0.880000in}}{\pgfqpoint{6.052998in}{6.160000in}}%
\pgfusepath{clip}%
\pgfsetbuttcap%
\pgfsetroundjoin%
\definecolor{currentfill}{rgb}{0.200000,0.200000,0.800000}%
\pgfsetfillcolor{currentfill}%
\pgfsetlinewidth{1.003750pt}%
\definecolor{currentstroke}{rgb}{0.200000,0.200000,0.800000}%
\pgfsetstrokecolor{currentstroke}%
\pgfsetdash{}{0pt}%
\pgfpathmoveto{\pgfqpoint{2.625870in}{6.179974in}}%
\pgfpathcurveto{\pgfqpoint{2.631694in}{6.179974in}}{\pgfqpoint{2.637280in}{6.182287in}}{\pgfqpoint{2.641398in}{6.186406in}}%
\pgfpathcurveto{\pgfqpoint{2.645516in}{6.190524in}}{\pgfqpoint{2.647830in}{6.196110in}}{\pgfqpoint{2.647830in}{6.201934in}}%
\pgfpathcurveto{\pgfqpoint{2.647830in}{6.207758in}}{\pgfqpoint{2.645516in}{6.213344in}}{\pgfqpoint{2.641398in}{6.217462in}}%
\pgfpathcurveto{\pgfqpoint{2.637280in}{6.221580in}}{\pgfqpoint{2.631694in}{6.223894in}}{\pgfqpoint{2.625870in}{6.223894in}}%
\pgfpathcurveto{\pgfqpoint{2.620046in}{6.223894in}}{\pgfqpoint{2.614460in}{6.221580in}}{\pgfqpoint{2.610341in}{6.217462in}}%
\pgfpathcurveto{\pgfqpoint{2.606223in}{6.213344in}}{\pgfqpoint{2.603909in}{6.207758in}}{\pgfqpoint{2.603909in}{6.201934in}}%
\pgfpathcurveto{\pgfqpoint{2.603909in}{6.196110in}}{\pgfqpoint{2.606223in}{6.190524in}}{\pgfqpoint{2.610341in}{6.186406in}}%
\pgfpathcurveto{\pgfqpoint{2.614460in}{6.182287in}}{\pgfqpoint{2.620046in}{6.179974in}}{\pgfqpoint{2.625870in}{6.179974in}}%
\pgfpathlineto{\pgfqpoint{2.625870in}{6.179974in}}%
\pgfpathclose%
\pgfusepath{stroke,fill}%
\end{pgfscope}%
\begin{pgfscope}%
\pgfpathrectangle{\pgfqpoint{1.073501in}{0.880000in}}{\pgfqpoint{6.052998in}{6.160000in}}%
\pgfusepath{clip}%
\pgfsetbuttcap%
\pgfsetroundjoin%
\definecolor{currentfill}{rgb}{0.200000,0.200000,0.800000}%
\pgfsetfillcolor{currentfill}%
\pgfsetlinewidth{1.003750pt}%
\definecolor{currentstroke}{rgb}{0.200000,0.200000,0.800000}%
\pgfsetstrokecolor{currentstroke}%
\pgfsetdash{}{0pt}%
\pgfpathmoveto{\pgfqpoint{2.495173in}{6.068157in}}%
\pgfpathcurveto{\pgfqpoint{2.500997in}{6.068157in}}{\pgfqpoint{2.506583in}{6.070471in}}{\pgfqpoint{2.510701in}{6.074589in}}%
\pgfpathcurveto{\pgfqpoint{2.514819in}{6.078708in}}{\pgfqpoint{2.517133in}{6.084294in}}{\pgfqpoint{2.517133in}{6.090118in}}%
\pgfpathcurveto{\pgfqpoint{2.517133in}{6.095942in}}{\pgfqpoint{2.514819in}{6.101528in}}{\pgfqpoint{2.510701in}{6.105646in}}%
\pgfpathcurveto{\pgfqpoint{2.506583in}{6.109764in}}{\pgfqpoint{2.500997in}{6.112078in}}{\pgfqpoint{2.495173in}{6.112078in}}%
\pgfpathcurveto{\pgfqpoint{2.489349in}{6.112078in}}{\pgfqpoint{2.483763in}{6.109764in}}{\pgfqpoint{2.479645in}{6.105646in}}%
\pgfpathcurveto{\pgfqpoint{2.475527in}{6.101528in}}{\pgfqpoint{2.473213in}{6.095942in}}{\pgfqpoint{2.473213in}{6.090118in}}%
\pgfpathcurveto{\pgfqpoint{2.473213in}{6.084294in}}{\pgfqpoint{2.475527in}{6.078708in}}{\pgfqpoint{2.479645in}{6.074589in}}%
\pgfpathcurveto{\pgfqpoint{2.483763in}{6.070471in}}{\pgfqpoint{2.489349in}{6.068157in}}{\pgfqpoint{2.495173in}{6.068157in}}%
\pgfpathlineto{\pgfqpoint{2.495173in}{6.068157in}}%
\pgfpathclose%
\pgfusepath{stroke,fill}%
\end{pgfscope}%
\begin{pgfscope}%
\pgfpathrectangle{\pgfqpoint{1.073501in}{0.880000in}}{\pgfqpoint{6.052998in}{6.160000in}}%
\pgfusepath{clip}%
\pgfsetbuttcap%
\pgfsetroundjoin%
\definecolor{currentfill}{rgb}{0.200000,0.200000,0.800000}%
\pgfsetfillcolor{currentfill}%
\pgfsetlinewidth{1.003750pt}%
\definecolor{currentstroke}{rgb}{0.200000,0.200000,0.800000}%
\pgfsetstrokecolor{currentstroke}%
\pgfsetdash{}{0pt}%
\pgfpathmoveto{\pgfqpoint{2.418911in}{5.895424in}}%
\pgfpathcurveto{\pgfqpoint{2.424735in}{5.895424in}}{\pgfqpoint{2.430321in}{5.897738in}}{\pgfqpoint{2.434439in}{5.901856in}}%
\pgfpathcurveto{\pgfqpoint{2.438557in}{5.905975in}}{\pgfqpoint{2.440871in}{5.911561in}}{\pgfqpoint{2.440871in}{5.917385in}}%
\pgfpathcurveto{\pgfqpoint{2.440871in}{5.923209in}}{\pgfqpoint{2.438557in}{5.928795in}}{\pgfqpoint{2.434439in}{5.932913in}}%
\pgfpathcurveto{\pgfqpoint{2.430321in}{5.937031in}}{\pgfqpoint{2.424735in}{5.939345in}}{\pgfqpoint{2.418911in}{5.939345in}}%
\pgfpathcurveto{\pgfqpoint{2.413087in}{5.939345in}}{\pgfqpoint{2.407500in}{5.937031in}}{\pgfqpoint{2.403382in}{5.932913in}}%
\pgfpathcurveto{\pgfqpoint{2.399264in}{5.928795in}}{\pgfqpoint{2.396950in}{5.923209in}}{\pgfqpoint{2.396950in}{5.917385in}}%
\pgfpathcurveto{\pgfqpoint{2.396950in}{5.911561in}}{\pgfqpoint{2.399264in}{5.905975in}}{\pgfqpoint{2.403382in}{5.901856in}}%
\pgfpathcurveto{\pgfqpoint{2.407500in}{5.897738in}}{\pgfqpoint{2.413087in}{5.895424in}}{\pgfqpoint{2.418911in}{5.895424in}}%
\pgfpathlineto{\pgfqpoint{2.418911in}{5.895424in}}%
\pgfpathclose%
\pgfusepath{stroke,fill}%
\end{pgfscope}%
\begin{pgfscope}%
\pgfpathrectangle{\pgfqpoint{1.073501in}{0.880000in}}{\pgfqpoint{6.052998in}{6.160000in}}%
\pgfusepath{clip}%
\pgfsetbuttcap%
\pgfsetroundjoin%
\definecolor{currentfill}{rgb}{0.200000,0.200000,0.800000}%
\pgfsetfillcolor{currentfill}%
\pgfsetlinewidth{1.003750pt}%
\definecolor{currentstroke}{rgb}{0.200000,0.200000,0.800000}%
\pgfsetstrokecolor{currentstroke}%
\pgfsetdash{}{0pt}%
\pgfpathmoveto{\pgfqpoint{2.204384in}{5.877742in}}%
\pgfpathcurveto{\pgfqpoint{2.210208in}{5.877742in}}{\pgfqpoint{2.215794in}{5.880056in}}{\pgfqpoint{2.219912in}{5.884174in}}%
\pgfpathcurveto{\pgfqpoint{2.224031in}{5.888292in}}{\pgfqpoint{2.226344in}{5.893878in}}{\pgfqpoint{2.226344in}{5.899702in}}%
\pgfpathcurveto{\pgfqpoint{2.226344in}{5.905526in}}{\pgfqpoint{2.224031in}{5.911112in}}{\pgfqpoint{2.219912in}{5.915231in}}%
\pgfpathcurveto{\pgfqpoint{2.215794in}{5.919349in}}{\pgfqpoint{2.210208in}{5.921663in}}{\pgfqpoint{2.204384in}{5.921663in}}%
\pgfpathcurveto{\pgfqpoint{2.198560in}{5.921663in}}{\pgfqpoint{2.192974in}{5.919349in}}{\pgfqpoint{2.188856in}{5.915231in}}%
\pgfpathcurveto{\pgfqpoint{2.184738in}{5.911112in}}{\pgfqpoint{2.182424in}{5.905526in}}{\pgfqpoint{2.182424in}{5.899702in}}%
\pgfpathcurveto{\pgfqpoint{2.182424in}{5.893878in}}{\pgfqpoint{2.184738in}{5.888292in}}{\pgfqpoint{2.188856in}{5.884174in}}%
\pgfpathcurveto{\pgfqpoint{2.192974in}{5.880056in}}{\pgfqpoint{2.198560in}{5.877742in}}{\pgfqpoint{2.204384in}{5.877742in}}%
\pgfpathlineto{\pgfqpoint{2.204384in}{5.877742in}}%
\pgfpathclose%
\pgfusepath{stroke,fill}%
\end{pgfscope}%
\begin{pgfscope}%
\pgfpathrectangle{\pgfqpoint{1.073501in}{0.880000in}}{\pgfqpoint{6.052998in}{6.160000in}}%
\pgfusepath{clip}%
\pgfsetbuttcap%
\pgfsetroundjoin%
\definecolor{currentfill}{rgb}{0.200000,0.200000,0.800000}%
\pgfsetfillcolor{currentfill}%
\pgfsetlinewidth{1.003750pt}%
\definecolor{currentstroke}{rgb}{0.200000,0.200000,0.800000}%
\pgfsetstrokecolor{currentstroke}%
\pgfsetdash{}{0pt}%
\pgfpathmoveto{\pgfqpoint{2.065772in}{5.768673in}}%
\pgfpathcurveto{\pgfqpoint{2.071596in}{5.768673in}}{\pgfqpoint{2.077182in}{5.770987in}}{\pgfqpoint{2.081300in}{5.775105in}}%
\pgfpathcurveto{\pgfqpoint{2.085418in}{5.779223in}}{\pgfqpoint{2.087732in}{5.784809in}}{\pgfqpoint{2.087732in}{5.790633in}}%
\pgfpathcurveto{\pgfqpoint{2.087732in}{5.796457in}}{\pgfqpoint{2.085418in}{5.802043in}}{\pgfqpoint{2.081300in}{5.806162in}}%
\pgfpathcurveto{\pgfqpoint{2.077182in}{5.810280in}}{\pgfqpoint{2.071596in}{5.812594in}}{\pgfqpoint{2.065772in}{5.812594in}}%
\pgfpathcurveto{\pgfqpoint{2.059948in}{5.812594in}}{\pgfqpoint{2.054362in}{5.810280in}}{\pgfqpoint{2.050244in}{5.806162in}}%
\pgfpathcurveto{\pgfqpoint{2.046125in}{5.802043in}}{\pgfqpoint{2.043812in}{5.796457in}}{\pgfqpoint{2.043812in}{5.790633in}}%
\pgfpathcurveto{\pgfqpoint{2.043812in}{5.784809in}}{\pgfqpoint{2.046125in}{5.779223in}}{\pgfqpoint{2.050244in}{5.775105in}}%
\pgfpathcurveto{\pgfqpoint{2.054362in}{5.770987in}}{\pgfqpoint{2.059948in}{5.768673in}}{\pgfqpoint{2.065772in}{5.768673in}}%
\pgfpathlineto{\pgfqpoint{2.065772in}{5.768673in}}%
\pgfpathclose%
\pgfusepath{stroke,fill}%
\end{pgfscope}%
\begin{pgfscope}%
\pgfpathrectangle{\pgfqpoint{1.073501in}{0.880000in}}{\pgfqpoint{6.052998in}{6.160000in}}%
\pgfusepath{clip}%
\pgfsetbuttcap%
\pgfsetroundjoin%
\definecolor{currentfill}{rgb}{0.200000,0.200000,0.800000}%
\pgfsetfillcolor{currentfill}%
\pgfsetlinewidth{1.003750pt}%
\definecolor{currentstroke}{rgb}{0.200000,0.200000,0.800000}%
\pgfsetstrokecolor{currentstroke}%
\pgfsetdash{}{0pt}%
\pgfpathmoveto{\pgfqpoint{1.993551in}{5.602458in}}%
\pgfpathcurveto{\pgfqpoint{1.999375in}{5.602458in}}{\pgfqpoint{2.004961in}{5.604772in}}{\pgfqpoint{2.009080in}{5.608890in}}%
\pgfpathcurveto{\pgfqpoint{2.013198in}{5.613008in}}{\pgfqpoint{2.015512in}{5.618595in}}{\pgfqpoint{2.015512in}{5.624418in}}%
\pgfpathcurveto{\pgfqpoint{2.015512in}{5.630242in}}{\pgfqpoint{2.013198in}{5.635829in}}{\pgfqpoint{2.009080in}{5.639947in}}%
\pgfpathcurveto{\pgfqpoint{2.004961in}{5.644065in}}{\pgfqpoint{1.999375in}{5.646379in}}{\pgfqpoint{1.993551in}{5.646379in}}%
\pgfpathcurveto{\pgfqpoint{1.987727in}{5.646379in}}{\pgfqpoint{1.982141in}{5.644065in}}{\pgfqpoint{1.978023in}{5.639947in}}%
\pgfpathcurveto{\pgfqpoint{1.973905in}{5.635829in}}{\pgfqpoint{1.971591in}{5.630242in}}{\pgfqpoint{1.971591in}{5.624418in}}%
\pgfpathcurveto{\pgfqpoint{1.971591in}{5.618595in}}{\pgfqpoint{1.973905in}{5.613008in}}{\pgfqpoint{1.978023in}{5.608890in}}%
\pgfpathcurveto{\pgfqpoint{1.982141in}{5.604772in}}{\pgfqpoint{1.987727in}{5.602458in}}{\pgfqpoint{1.993551in}{5.602458in}}%
\pgfpathlineto{\pgfqpoint{1.993551in}{5.602458in}}%
\pgfpathclose%
\pgfusepath{stroke,fill}%
\end{pgfscope}%
\begin{pgfscope}%
\pgfpathrectangle{\pgfqpoint{1.073501in}{0.880000in}}{\pgfqpoint{6.052998in}{6.160000in}}%
\pgfusepath{clip}%
\pgfsetbuttcap%
\pgfsetroundjoin%
\definecolor{currentfill}{rgb}{0.200000,0.200000,0.800000}%
\pgfsetfillcolor{currentfill}%
\pgfsetlinewidth{1.003750pt}%
\definecolor{currentstroke}{rgb}{0.200000,0.200000,0.800000}%
\pgfsetstrokecolor{currentstroke}%
\pgfsetdash{}{0pt}%
\pgfpathmoveto{\pgfqpoint{1.911135in}{5.450343in}}%
\pgfpathcurveto{\pgfqpoint{1.916959in}{5.450343in}}{\pgfqpoint{1.922545in}{5.452657in}}{\pgfqpoint{1.926664in}{5.456775in}}%
\pgfpathcurveto{\pgfqpoint{1.930782in}{5.460893in}}{\pgfqpoint{1.933096in}{5.466479in}}{\pgfqpoint{1.933096in}{5.472303in}}%
\pgfpathcurveto{\pgfqpoint{1.933096in}{5.478127in}}{\pgfqpoint{1.930782in}{5.483714in}}{\pgfqpoint{1.926664in}{5.487832in}}%
\pgfpathcurveto{\pgfqpoint{1.922545in}{5.491950in}}{\pgfqpoint{1.916959in}{5.494264in}}{\pgfqpoint{1.911135in}{5.494264in}}%
\pgfpathcurveto{\pgfqpoint{1.905311in}{5.494264in}}{\pgfqpoint{1.899725in}{5.491950in}}{\pgfqpoint{1.895607in}{5.487832in}}%
\pgfpathcurveto{\pgfqpoint{1.891489in}{5.483714in}}{\pgfqpoint{1.889175in}{5.478127in}}{\pgfqpoint{1.889175in}{5.472303in}}%
\pgfpathcurveto{\pgfqpoint{1.889175in}{5.466479in}}{\pgfqpoint{1.891489in}{5.460893in}}{\pgfqpoint{1.895607in}{5.456775in}}%
\pgfpathcurveto{\pgfqpoint{1.899725in}{5.452657in}}{\pgfqpoint{1.905311in}{5.450343in}}{\pgfqpoint{1.911135in}{5.450343in}}%
\pgfpathlineto{\pgfqpoint{1.911135in}{5.450343in}}%
\pgfpathclose%
\pgfusepath{stroke,fill}%
\end{pgfscope}%
\begin{pgfscope}%
\pgfpathrectangle{\pgfqpoint{1.073501in}{0.880000in}}{\pgfqpoint{6.052998in}{6.160000in}}%
\pgfusepath{clip}%
\pgfsetbuttcap%
\pgfsetroundjoin%
\definecolor{currentfill}{rgb}{0.200000,0.200000,0.800000}%
\pgfsetfillcolor{currentfill}%
\pgfsetlinewidth{1.003750pt}%
\definecolor{currentstroke}{rgb}{0.200000,0.200000,0.800000}%
\pgfsetstrokecolor{currentstroke}%
\pgfsetdash{}{0pt}%
\pgfpathmoveto{\pgfqpoint{1.784561in}{5.327383in}}%
\pgfpathcurveto{\pgfqpoint{1.790385in}{5.327383in}}{\pgfqpoint{1.795971in}{5.329697in}}{\pgfqpoint{1.800089in}{5.333815in}}%
\pgfpathcurveto{\pgfqpoint{1.804207in}{5.337933in}}{\pgfqpoint{1.806521in}{5.343519in}}{\pgfqpoint{1.806521in}{5.349343in}}%
\pgfpathcurveto{\pgfqpoint{1.806521in}{5.355167in}}{\pgfqpoint{1.804207in}{5.360753in}}{\pgfqpoint{1.800089in}{5.364871in}}%
\pgfpathcurveto{\pgfqpoint{1.795971in}{5.368989in}}{\pgfqpoint{1.790385in}{5.371303in}}{\pgfqpoint{1.784561in}{5.371303in}}%
\pgfpathcurveto{\pgfqpoint{1.778737in}{5.371303in}}{\pgfqpoint{1.773150in}{5.368989in}}{\pgfqpoint{1.769032in}{5.364871in}}%
\pgfpathcurveto{\pgfqpoint{1.764914in}{5.360753in}}{\pgfqpoint{1.762600in}{5.355167in}}{\pgfqpoint{1.762600in}{5.349343in}}%
\pgfpathcurveto{\pgfqpoint{1.762600in}{5.343519in}}{\pgfqpoint{1.764914in}{5.337933in}}{\pgfqpoint{1.769032in}{5.333815in}}%
\pgfpathcurveto{\pgfqpoint{1.773150in}{5.329697in}}{\pgfqpoint{1.778737in}{5.327383in}}{\pgfqpoint{1.784561in}{5.327383in}}%
\pgfpathlineto{\pgfqpoint{1.784561in}{5.327383in}}%
\pgfpathclose%
\pgfusepath{stroke,fill}%
\end{pgfscope}%
\begin{pgfscope}%
\pgfpathrectangle{\pgfqpoint{1.073501in}{0.880000in}}{\pgfqpoint{6.052998in}{6.160000in}}%
\pgfusepath{clip}%
\pgfsetbuttcap%
\pgfsetroundjoin%
\definecolor{currentfill}{rgb}{0.200000,0.200000,0.800000}%
\pgfsetfillcolor{currentfill}%
\pgfsetlinewidth{1.003750pt}%
\definecolor{currentstroke}{rgb}{0.200000,0.200000,0.800000}%
\pgfsetstrokecolor{currentstroke}%
\pgfsetdash{}{0pt}%
\pgfpathmoveto{\pgfqpoint{1.627355in}{5.213729in}}%
\pgfpathcurveto{\pgfqpoint{1.633179in}{5.213729in}}{\pgfqpoint{1.638765in}{5.216043in}}{\pgfqpoint{1.642883in}{5.220161in}}%
\pgfpathcurveto{\pgfqpoint{1.647002in}{5.224279in}}{\pgfqpoint{1.649315in}{5.229865in}}{\pgfqpoint{1.649315in}{5.235689in}}%
\pgfpathcurveto{\pgfqpoint{1.649315in}{5.241513in}}{\pgfqpoint{1.647002in}{5.247099in}}{\pgfqpoint{1.642883in}{5.251217in}}%
\pgfpathcurveto{\pgfqpoint{1.638765in}{5.255336in}}{\pgfqpoint{1.633179in}{5.257649in}}{\pgfqpoint{1.627355in}{5.257649in}}%
\pgfpathcurveto{\pgfqpoint{1.621531in}{5.257649in}}{\pgfqpoint{1.615945in}{5.255336in}}{\pgfqpoint{1.611827in}{5.251217in}}%
\pgfpathcurveto{\pgfqpoint{1.607709in}{5.247099in}}{\pgfqpoint{1.605395in}{5.241513in}}{\pgfqpoint{1.605395in}{5.235689in}}%
\pgfpathcurveto{\pgfqpoint{1.605395in}{5.229865in}}{\pgfqpoint{1.607709in}{5.224279in}}{\pgfqpoint{1.611827in}{5.220161in}}%
\pgfpathcurveto{\pgfqpoint{1.615945in}{5.216043in}}{\pgfqpoint{1.621531in}{5.213729in}}{\pgfqpoint{1.627355in}{5.213729in}}%
\pgfpathlineto{\pgfqpoint{1.627355in}{5.213729in}}%
\pgfpathclose%
\pgfusepath{stroke,fill}%
\end{pgfscope}%
\begin{pgfscope}%
\pgfpathrectangle{\pgfqpoint{1.073501in}{0.880000in}}{\pgfqpoint{6.052998in}{6.160000in}}%
\pgfusepath{clip}%
\pgfsetbuttcap%
\pgfsetroundjoin%
\definecolor{currentfill}{rgb}{0.200000,0.200000,0.800000}%
\pgfsetfillcolor{currentfill}%
\pgfsetlinewidth{1.003750pt}%
\definecolor{currentstroke}{rgb}{0.200000,0.200000,0.800000}%
\pgfsetstrokecolor{currentstroke}%
\pgfsetdash{}{0pt}%
\pgfpathmoveto{\pgfqpoint{1.636887in}{5.014967in}}%
\pgfpathcurveto{\pgfqpoint{1.642711in}{5.014967in}}{\pgfqpoint{1.648297in}{5.017281in}}{\pgfqpoint{1.652415in}{5.021399in}}%
\pgfpathcurveto{\pgfqpoint{1.656533in}{5.025517in}}{\pgfqpoint{1.658847in}{5.031103in}}{\pgfqpoint{1.658847in}{5.036927in}}%
\pgfpathcurveto{\pgfqpoint{1.658847in}{5.042751in}}{\pgfqpoint{1.656533in}{5.048337in}}{\pgfqpoint{1.652415in}{5.052455in}}%
\pgfpathcurveto{\pgfqpoint{1.648297in}{5.056574in}}{\pgfqpoint{1.642711in}{5.058887in}}{\pgfqpoint{1.636887in}{5.058887in}}%
\pgfpathcurveto{\pgfqpoint{1.631063in}{5.058887in}}{\pgfqpoint{1.625477in}{5.056574in}}{\pgfqpoint{1.621359in}{5.052455in}}%
\pgfpathcurveto{\pgfqpoint{1.617241in}{5.048337in}}{\pgfqpoint{1.614927in}{5.042751in}}{\pgfqpoint{1.614927in}{5.036927in}}%
\pgfpathcurveto{\pgfqpoint{1.614927in}{5.031103in}}{\pgfqpoint{1.617241in}{5.025517in}}{\pgfqpoint{1.621359in}{5.021399in}}%
\pgfpathcurveto{\pgfqpoint{1.625477in}{5.017281in}}{\pgfqpoint{1.631063in}{5.014967in}}{\pgfqpoint{1.636887in}{5.014967in}}%
\pgfpathlineto{\pgfqpoint{1.636887in}{5.014967in}}%
\pgfpathclose%
\pgfusepath{stroke,fill}%
\end{pgfscope}%
\begin{pgfscope}%
\pgfpathrectangle{\pgfqpoint{1.073501in}{0.880000in}}{\pgfqpoint{6.052998in}{6.160000in}}%
\pgfusepath{clip}%
\pgfsetbuttcap%
\pgfsetroundjoin%
\definecolor{currentfill}{rgb}{0.200000,0.200000,0.800000}%
\pgfsetfillcolor{currentfill}%
\pgfsetlinewidth{1.003750pt}%
\definecolor{currentstroke}{rgb}{0.200000,0.200000,0.800000}%
\pgfsetstrokecolor{currentstroke}%
\pgfsetdash{}{0pt}%
\pgfpathmoveto{\pgfqpoint{1.649592in}{4.826986in}}%
\pgfpathcurveto{\pgfqpoint{1.655416in}{4.826986in}}{\pgfqpoint{1.661002in}{4.829300in}}{\pgfqpoint{1.665120in}{4.833418in}}%
\pgfpathcurveto{\pgfqpoint{1.669238in}{4.837536in}}{\pgfqpoint{1.671552in}{4.843123in}}{\pgfqpoint{1.671552in}{4.848946in}}%
\pgfpathcurveto{\pgfqpoint{1.671552in}{4.854770in}}{\pgfqpoint{1.669238in}{4.860357in}}{\pgfqpoint{1.665120in}{4.864475in}}%
\pgfpathcurveto{\pgfqpoint{1.661002in}{4.868593in}}{\pgfqpoint{1.655416in}{4.870907in}}{\pgfqpoint{1.649592in}{4.870907in}}%
\pgfpathcurveto{\pgfqpoint{1.643768in}{4.870907in}}{\pgfqpoint{1.638181in}{4.868593in}}{\pgfqpoint{1.634063in}{4.864475in}}%
\pgfpathcurveto{\pgfqpoint{1.629945in}{4.860357in}}{\pgfqpoint{1.627631in}{4.854770in}}{\pgfqpoint{1.627631in}{4.848946in}}%
\pgfpathcurveto{\pgfqpoint{1.627631in}{4.843123in}}{\pgfqpoint{1.629945in}{4.837536in}}{\pgfqpoint{1.634063in}{4.833418in}}%
\pgfpathcurveto{\pgfqpoint{1.638181in}{4.829300in}}{\pgfqpoint{1.643768in}{4.826986in}}{\pgfqpoint{1.649592in}{4.826986in}}%
\pgfpathlineto{\pgfqpoint{1.649592in}{4.826986in}}%
\pgfpathclose%
\pgfusepath{stroke,fill}%
\end{pgfscope}%
\begin{pgfscope}%
\pgfpathrectangle{\pgfqpoint{1.073501in}{0.880000in}}{\pgfqpoint{6.052998in}{6.160000in}}%
\pgfusepath{clip}%
\pgfsetbuttcap%
\pgfsetroundjoin%
\definecolor{currentfill}{rgb}{0.200000,0.200000,0.800000}%
\pgfsetfillcolor{currentfill}%
\pgfsetlinewidth{1.003750pt}%
\definecolor{currentstroke}{rgb}{0.200000,0.200000,0.800000}%
\pgfsetstrokecolor{currentstroke}%
\pgfsetdash{}{0pt}%
\pgfpathmoveto{\pgfqpoint{1.485227in}{4.701076in}}%
\pgfpathcurveto{\pgfqpoint{1.491051in}{4.701076in}}{\pgfqpoint{1.496637in}{4.703389in}}{\pgfqpoint{1.500755in}{4.707508in}}%
\pgfpathcurveto{\pgfqpoint{1.504873in}{4.711626in}}{\pgfqpoint{1.507187in}{4.717212in}}{\pgfqpoint{1.507187in}{4.723036in}}%
\pgfpathcurveto{\pgfqpoint{1.507187in}{4.728860in}}{\pgfqpoint{1.504873in}{4.734446in}}{\pgfqpoint{1.500755in}{4.738564in}}%
\pgfpathcurveto{\pgfqpoint{1.496637in}{4.742682in}}{\pgfqpoint{1.491051in}{4.744996in}}{\pgfqpoint{1.485227in}{4.744996in}}%
\pgfpathcurveto{\pgfqpoint{1.479403in}{4.744996in}}{\pgfqpoint{1.473817in}{4.742682in}}{\pgfqpoint{1.469699in}{4.738564in}}%
\pgfpathcurveto{\pgfqpoint{1.465581in}{4.734446in}}{\pgfqpoint{1.463267in}{4.728860in}}{\pgfqpoint{1.463267in}{4.723036in}}%
\pgfpathcurveto{\pgfqpoint{1.463267in}{4.717212in}}{\pgfqpoint{1.465581in}{4.711626in}}{\pgfqpoint{1.469699in}{4.707508in}}%
\pgfpathcurveto{\pgfqpoint{1.473817in}{4.703389in}}{\pgfqpoint{1.479403in}{4.701076in}}{\pgfqpoint{1.485227in}{4.701076in}}%
\pgfpathlineto{\pgfqpoint{1.485227in}{4.701076in}}%
\pgfpathclose%
\pgfusepath{stroke,fill}%
\end{pgfscope}%
\begin{pgfscope}%
\pgfpathrectangle{\pgfqpoint{1.073501in}{0.880000in}}{\pgfqpoint{6.052998in}{6.160000in}}%
\pgfusepath{clip}%
\pgfsetbuttcap%
\pgfsetroundjoin%
\definecolor{currentfill}{rgb}{0.200000,0.200000,0.800000}%
\pgfsetfillcolor{currentfill}%
\pgfsetlinewidth{1.003750pt}%
\definecolor{currentstroke}{rgb}{0.200000,0.200000,0.800000}%
\pgfsetstrokecolor{currentstroke}%
\pgfsetdash{}{0pt}%
\pgfpathmoveto{\pgfqpoint{1.631564in}{4.489094in}}%
\pgfpathcurveto{\pgfqpoint{1.637388in}{4.489094in}}{\pgfqpoint{1.642974in}{4.491408in}}{\pgfqpoint{1.647092in}{4.495527in}}%
\pgfpathcurveto{\pgfqpoint{1.651210in}{4.499645in}}{\pgfqpoint{1.653524in}{4.505231in}}{\pgfqpoint{1.653524in}{4.511055in}}%
\pgfpathcurveto{\pgfqpoint{1.653524in}{4.516879in}}{\pgfqpoint{1.651210in}{4.522465in}}{\pgfqpoint{1.647092in}{4.526583in}}%
\pgfpathcurveto{\pgfqpoint{1.642974in}{4.530701in}}{\pgfqpoint{1.637388in}{4.533015in}}{\pgfqpoint{1.631564in}{4.533015in}}%
\pgfpathcurveto{\pgfqpoint{1.625740in}{4.533015in}}{\pgfqpoint{1.620154in}{4.530701in}}{\pgfqpoint{1.616036in}{4.526583in}}%
\pgfpathcurveto{\pgfqpoint{1.611917in}{4.522465in}}{\pgfqpoint{1.609604in}{4.516879in}}{\pgfqpoint{1.609604in}{4.511055in}}%
\pgfpathcurveto{\pgfqpoint{1.609604in}{4.505231in}}{\pgfqpoint{1.611917in}{4.499645in}}{\pgfqpoint{1.616036in}{4.495527in}}%
\pgfpathcurveto{\pgfqpoint{1.620154in}{4.491408in}}{\pgfqpoint{1.625740in}{4.489094in}}{\pgfqpoint{1.631564in}{4.489094in}}%
\pgfpathlineto{\pgfqpoint{1.631564in}{4.489094in}}%
\pgfpathclose%
\pgfusepath{stroke,fill}%
\end{pgfscope}%
\begin{pgfscope}%
\pgfpathrectangle{\pgfqpoint{1.073501in}{0.880000in}}{\pgfqpoint{6.052998in}{6.160000in}}%
\pgfusepath{clip}%
\pgfsetbuttcap%
\pgfsetroundjoin%
\definecolor{currentfill}{rgb}{0.200000,0.200000,0.800000}%
\pgfsetfillcolor{currentfill}%
\pgfsetlinewidth{1.003750pt}%
\definecolor{currentstroke}{rgb}{0.200000,0.200000,0.800000}%
\pgfsetstrokecolor{currentstroke}%
\pgfsetdash{}{0pt}%
\pgfpathmoveto{\pgfqpoint{1.442957in}{4.355044in}}%
\pgfpathcurveto{\pgfqpoint{1.448781in}{4.355044in}}{\pgfqpoint{1.454367in}{4.357358in}}{\pgfqpoint{1.458485in}{4.361476in}}%
\pgfpathcurveto{\pgfqpoint{1.462603in}{4.365594in}}{\pgfqpoint{1.464917in}{4.371181in}}{\pgfqpoint{1.464917in}{4.377005in}}%
\pgfpathcurveto{\pgfqpoint{1.464917in}{4.382828in}}{\pgfqpoint{1.462603in}{4.388415in}}{\pgfqpoint{1.458485in}{4.392533in}}%
\pgfpathcurveto{\pgfqpoint{1.454367in}{4.396651in}}{\pgfqpoint{1.448781in}{4.398965in}}{\pgfqpoint{1.442957in}{4.398965in}}%
\pgfpathcurveto{\pgfqpoint{1.437133in}{4.398965in}}{\pgfqpoint{1.431547in}{4.396651in}}{\pgfqpoint{1.427429in}{4.392533in}}%
\pgfpathcurveto{\pgfqpoint{1.423310in}{4.388415in}}{\pgfqpoint{1.420997in}{4.382828in}}{\pgfqpoint{1.420997in}{4.377005in}}%
\pgfpathcurveto{\pgfqpoint{1.420997in}{4.371181in}}{\pgfqpoint{1.423310in}{4.365594in}}{\pgfqpoint{1.427429in}{4.361476in}}%
\pgfpathcurveto{\pgfqpoint{1.431547in}{4.357358in}}{\pgfqpoint{1.437133in}{4.355044in}}{\pgfqpoint{1.442957in}{4.355044in}}%
\pgfpathlineto{\pgfqpoint{1.442957in}{4.355044in}}%
\pgfpathclose%
\pgfusepath{stroke,fill}%
\end{pgfscope}%
\begin{pgfscope}%
\pgfpathrectangle{\pgfqpoint{1.073501in}{0.880000in}}{\pgfqpoint{6.052998in}{6.160000in}}%
\pgfusepath{clip}%
\pgfsetbuttcap%
\pgfsetroundjoin%
\definecolor{currentfill}{rgb}{0.200000,0.200000,0.800000}%
\pgfsetfillcolor{currentfill}%
\pgfsetlinewidth{1.003750pt}%
\definecolor{currentstroke}{rgb}{0.200000,0.200000,0.800000}%
\pgfsetstrokecolor{currentstroke}%
\pgfsetdash{}{0pt}%
\pgfpathmoveto{\pgfqpoint{1.519306in}{4.174661in}}%
\pgfpathcurveto{\pgfqpoint{1.525130in}{4.174661in}}{\pgfqpoint{1.530716in}{4.176975in}}{\pgfqpoint{1.534834in}{4.181093in}}%
\pgfpathcurveto{\pgfqpoint{1.538952in}{4.185211in}}{\pgfqpoint{1.541266in}{4.190798in}}{\pgfqpoint{1.541266in}{4.196622in}}%
\pgfpathcurveto{\pgfqpoint{1.541266in}{4.202446in}}{\pgfqpoint{1.538952in}{4.208032in}}{\pgfqpoint{1.534834in}{4.212150in}}%
\pgfpathcurveto{\pgfqpoint{1.530716in}{4.216268in}}{\pgfqpoint{1.525130in}{4.218582in}}{\pgfqpoint{1.519306in}{4.218582in}}%
\pgfpathcurveto{\pgfqpoint{1.513482in}{4.218582in}}{\pgfqpoint{1.507896in}{4.216268in}}{\pgfqpoint{1.503778in}{4.212150in}}%
\pgfpathcurveto{\pgfqpoint{1.499660in}{4.208032in}}{\pgfqpoint{1.497346in}{4.202446in}}{\pgfqpoint{1.497346in}{4.196622in}}%
\pgfpathcurveto{\pgfqpoint{1.497346in}{4.190798in}}{\pgfqpoint{1.499660in}{4.185211in}}{\pgfqpoint{1.503778in}{4.181093in}}%
\pgfpathcurveto{\pgfqpoint{1.507896in}{4.176975in}}{\pgfqpoint{1.513482in}{4.174661in}}{\pgfqpoint{1.519306in}{4.174661in}}%
\pgfpathlineto{\pgfqpoint{1.519306in}{4.174661in}}%
\pgfpathclose%
\pgfusepath{stroke,fill}%
\end{pgfscope}%
\begin{pgfscope}%
\pgfpathrectangle{\pgfqpoint{1.073501in}{0.880000in}}{\pgfqpoint{6.052998in}{6.160000in}}%
\pgfusepath{clip}%
\pgfsetbuttcap%
\pgfsetroundjoin%
\definecolor{currentfill}{rgb}{0.200000,0.200000,0.800000}%
\pgfsetfillcolor{currentfill}%
\pgfsetlinewidth{1.003750pt}%
\definecolor{currentstroke}{rgb}{0.200000,0.200000,0.800000}%
\pgfsetstrokecolor{currentstroke}%
\pgfsetdash{}{0pt}%
\pgfpathmoveto{\pgfqpoint{1.454652in}{4.010576in}}%
\pgfpathcurveto{\pgfqpoint{1.460476in}{4.010576in}}{\pgfqpoint{1.466062in}{4.012890in}}{\pgfqpoint{1.470180in}{4.017008in}}%
\pgfpathcurveto{\pgfqpoint{1.474298in}{4.021126in}}{\pgfqpoint{1.476612in}{4.026713in}}{\pgfqpoint{1.476612in}{4.032536in}}%
\pgfpathcurveto{\pgfqpoint{1.476612in}{4.038360in}}{\pgfqpoint{1.474298in}{4.043947in}}{\pgfqpoint{1.470180in}{4.048065in}}%
\pgfpathcurveto{\pgfqpoint{1.466062in}{4.052183in}}{\pgfqpoint{1.460476in}{4.054497in}}{\pgfqpoint{1.454652in}{4.054497in}}%
\pgfpathcurveto{\pgfqpoint{1.448828in}{4.054497in}}{\pgfqpoint{1.443242in}{4.052183in}}{\pgfqpoint{1.439124in}{4.048065in}}%
\pgfpathcurveto{\pgfqpoint{1.435006in}{4.043947in}}{\pgfqpoint{1.432692in}{4.038360in}}{\pgfqpoint{1.432692in}{4.032536in}}%
\pgfpathcurveto{\pgfqpoint{1.432692in}{4.026713in}}{\pgfqpoint{1.435006in}{4.021126in}}{\pgfqpoint{1.439124in}{4.017008in}}%
\pgfpathcurveto{\pgfqpoint{1.443242in}{4.012890in}}{\pgfqpoint{1.448828in}{4.010576in}}{\pgfqpoint{1.454652in}{4.010576in}}%
\pgfpathlineto{\pgfqpoint{1.454652in}{4.010576in}}%
\pgfpathclose%
\pgfusepath{stroke,fill}%
\end{pgfscope}%
\begin{pgfscope}%
\pgfpathrectangle{\pgfqpoint{1.073501in}{0.880000in}}{\pgfqpoint{6.052998in}{6.160000in}}%
\pgfusepath{clip}%
\pgfsetbuttcap%
\pgfsetroundjoin%
\definecolor{currentfill}{rgb}{0.200000,0.200000,0.800000}%
\pgfsetfillcolor{currentfill}%
\pgfsetlinewidth{1.003750pt}%
\definecolor{currentstroke}{rgb}{0.200000,0.200000,0.800000}%
\pgfsetstrokecolor{currentstroke}%
\pgfsetdash{}{0pt}%
\pgfpathmoveto{\pgfqpoint{1.348637in}{3.837638in}}%
\pgfpathcurveto{\pgfqpoint{1.354461in}{3.837638in}}{\pgfqpoint{1.360047in}{3.839951in}}{\pgfqpoint{1.364165in}{3.844070in}}%
\pgfpathcurveto{\pgfqpoint{1.368283in}{3.848188in}}{\pgfqpoint{1.370597in}{3.853774in}}{\pgfqpoint{1.370597in}{3.859598in}}%
\pgfpathcurveto{\pgfqpoint{1.370597in}{3.865422in}}{\pgfqpoint{1.368283in}{3.871008in}}{\pgfqpoint{1.364165in}{3.875126in}}%
\pgfpathcurveto{\pgfqpoint{1.360047in}{3.879244in}}{\pgfqpoint{1.354461in}{3.881558in}}{\pgfqpoint{1.348637in}{3.881558in}}%
\pgfpathcurveto{\pgfqpoint{1.342813in}{3.881558in}}{\pgfqpoint{1.337227in}{3.879244in}}{\pgfqpoint{1.333109in}{3.875126in}}%
\pgfpathcurveto{\pgfqpoint{1.328991in}{3.871008in}}{\pgfqpoint{1.326677in}{3.865422in}}{\pgfqpoint{1.326677in}{3.859598in}}%
\pgfpathcurveto{\pgfqpoint{1.326677in}{3.853774in}}{\pgfqpoint{1.328991in}{3.848188in}}{\pgfqpoint{1.333109in}{3.844070in}}%
\pgfpathcurveto{\pgfqpoint{1.337227in}{3.839951in}}{\pgfqpoint{1.342813in}{3.837638in}}{\pgfqpoint{1.348637in}{3.837638in}}%
\pgfpathlineto{\pgfqpoint{1.348637in}{3.837638in}}%
\pgfpathclose%
\pgfusepath{stroke,fill}%
\end{pgfscope}%
\begin{pgfscope}%
\pgfpathrectangle{\pgfqpoint{1.073501in}{0.880000in}}{\pgfqpoint{6.052998in}{6.160000in}}%
\pgfusepath{clip}%
\pgfsetbuttcap%
\pgfsetroundjoin%
\definecolor{currentfill}{rgb}{0.200000,0.200000,0.800000}%
\pgfsetfillcolor{currentfill}%
\pgfsetlinewidth{1.003750pt}%
\definecolor{currentstroke}{rgb}{0.200000,0.200000,0.800000}%
\pgfsetstrokecolor{currentstroke}%
\pgfsetdash{}{0pt}%
\pgfpathmoveto{\pgfqpoint{1.479733in}{3.673139in}}%
\pgfpathcurveto{\pgfqpoint{1.485557in}{3.673139in}}{\pgfqpoint{1.491143in}{3.675453in}}{\pgfqpoint{1.495262in}{3.679571in}}%
\pgfpathcurveto{\pgfqpoint{1.499380in}{3.683689in}}{\pgfqpoint{1.501694in}{3.689275in}}{\pgfqpoint{1.501694in}{3.695099in}}%
\pgfpathcurveto{\pgfqpoint{1.501694in}{3.700923in}}{\pgfqpoint{1.499380in}{3.706509in}}{\pgfqpoint{1.495262in}{3.710628in}}%
\pgfpathcurveto{\pgfqpoint{1.491143in}{3.714746in}}{\pgfqpoint{1.485557in}{3.717060in}}{\pgfqpoint{1.479733in}{3.717060in}}%
\pgfpathcurveto{\pgfqpoint{1.473909in}{3.717060in}}{\pgfqpoint{1.468323in}{3.714746in}}{\pgfqpoint{1.464205in}{3.710628in}}%
\pgfpathcurveto{\pgfqpoint{1.460087in}{3.706509in}}{\pgfqpoint{1.457773in}{3.700923in}}{\pgfqpoint{1.457773in}{3.695099in}}%
\pgfpathcurveto{\pgfqpoint{1.457773in}{3.689275in}}{\pgfqpoint{1.460087in}{3.683689in}}{\pgfqpoint{1.464205in}{3.679571in}}%
\pgfpathcurveto{\pgfqpoint{1.468323in}{3.675453in}}{\pgfqpoint{1.473909in}{3.673139in}}{\pgfqpoint{1.479733in}{3.673139in}}%
\pgfpathlineto{\pgfqpoint{1.479733in}{3.673139in}}%
\pgfpathclose%
\pgfusepath{stroke,fill}%
\end{pgfscope}%
\begin{pgfscope}%
\pgfpathrectangle{\pgfqpoint{1.073501in}{0.880000in}}{\pgfqpoint{6.052998in}{6.160000in}}%
\pgfusepath{clip}%
\pgfsetbuttcap%
\pgfsetroundjoin%
\definecolor{currentfill}{rgb}{0.200000,0.200000,0.800000}%
\pgfsetfillcolor{currentfill}%
\pgfsetlinewidth{1.003750pt}%
\definecolor{currentstroke}{rgb}{0.200000,0.200000,0.800000}%
\pgfsetstrokecolor{currentstroke}%
\pgfsetdash{}{0pt}%
\pgfpathmoveto{\pgfqpoint{1.490792in}{3.504189in}}%
\pgfpathcurveto{\pgfqpoint{1.496616in}{3.504189in}}{\pgfqpoint{1.502202in}{3.506503in}}{\pgfqpoint{1.506320in}{3.510621in}}%
\pgfpathcurveto{\pgfqpoint{1.510438in}{3.514739in}}{\pgfqpoint{1.512752in}{3.520325in}}{\pgfqpoint{1.512752in}{3.526149in}}%
\pgfpathcurveto{\pgfqpoint{1.512752in}{3.531973in}}{\pgfqpoint{1.510438in}{3.537559in}}{\pgfqpoint{1.506320in}{3.541678in}}%
\pgfpathcurveto{\pgfqpoint{1.502202in}{3.545796in}}{\pgfqpoint{1.496616in}{3.548110in}}{\pgfqpoint{1.490792in}{3.548110in}}%
\pgfpathcurveto{\pgfqpoint{1.484968in}{3.548110in}}{\pgfqpoint{1.479382in}{3.545796in}}{\pgfqpoint{1.475264in}{3.541678in}}%
\pgfpathcurveto{\pgfqpoint{1.471146in}{3.537559in}}{\pgfqpoint{1.468832in}{3.531973in}}{\pgfqpoint{1.468832in}{3.526149in}}%
\pgfpathcurveto{\pgfqpoint{1.468832in}{3.520325in}}{\pgfqpoint{1.471146in}{3.514739in}}{\pgfqpoint{1.475264in}{3.510621in}}%
\pgfpathcurveto{\pgfqpoint{1.479382in}{3.506503in}}{\pgfqpoint{1.484968in}{3.504189in}}{\pgfqpoint{1.490792in}{3.504189in}}%
\pgfpathlineto{\pgfqpoint{1.490792in}{3.504189in}}%
\pgfpathclose%
\pgfusepath{stroke,fill}%
\end{pgfscope}%
\begin{pgfscope}%
\pgfpathrectangle{\pgfqpoint{1.073501in}{0.880000in}}{\pgfqpoint{6.052998in}{6.160000in}}%
\pgfusepath{clip}%
\pgfsetbuttcap%
\pgfsetroundjoin%
\definecolor{currentfill}{rgb}{0.200000,0.200000,0.800000}%
\pgfsetfillcolor{currentfill}%
\pgfsetlinewidth{1.003750pt}%
\definecolor{currentstroke}{rgb}{0.200000,0.200000,0.800000}%
\pgfsetstrokecolor{currentstroke}%
\pgfsetdash{}{0pt}%
\pgfpathmoveto{\pgfqpoint{1.494163in}{3.331451in}}%
\pgfpathcurveto{\pgfqpoint{1.499987in}{3.331451in}}{\pgfqpoint{1.505573in}{3.333765in}}{\pgfqpoint{1.509691in}{3.337883in}}%
\pgfpathcurveto{\pgfqpoint{1.513810in}{3.342001in}}{\pgfqpoint{1.516123in}{3.347588in}}{\pgfqpoint{1.516123in}{3.353412in}}%
\pgfpathcurveto{\pgfqpoint{1.516123in}{3.359236in}}{\pgfqpoint{1.513810in}{3.364822in}}{\pgfqpoint{1.509691in}{3.368940in}}%
\pgfpathcurveto{\pgfqpoint{1.505573in}{3.373058in}}{\pgfqpoint{1.499987in}{3.375372in}}{\pgfqpoint{1.494163in}{3.375372in}}%
\pgfpathcurveto{\pgfqpoint{1.488339in}{3.375372in}}{\pgfqpoint{1.482753in}{3.373058in}}{\pgfqpoint{1.478635in}{3.368940in}}%
\pgfpathcurveto{\pgfqpoint{1.474517in}{3.364822in}}{\pgfqpoint{1.472203in}{3.359236in}}{\pgfqpoint{1.472203in}{3.353412in}}%
\pgfpathcurveto{\pgfqpoint{1.472203in}{3.347588in}}{\pgfqpoint{1.474517in}{3.342001in}}{\pgfqpoint{1.478635in}{3.337883in}}%
\pgfpathcurveto{\pgfqpoint{1.482753in}{3.333765in}}{\pgfqpoint{1.488339in}{3.331451in}}{\pgfqpoint{1.494163in}{3.331451in}}%
\pgfpathlineto{\pgfqpoint{1.494163in}{3.331451in}}%
\pgfpathclose%
\pgfusepath{stroke,fill}%
\end{pgfscope}%
\begin{pgfscope}%
\pgfpathrectangle{\pgfqpoint{1.073501in}{0.880000in}}{\pgfqpoint{6.052998in}{6.160000in}}%
\pgfusepath{clip}%
\pgfsetbuttcap%
\pgfsetroundjoin%
\definecolor{currentfill}{rgb}{0.200000,0.200000,0.800000}%
\pgfsetfillcolor{currentfill}%
\pgfsetlinewidth{1.003750pt}%
\definecolor{currentstroke}{rgb}{0.200000,0.200000,0.800000}%
\pgfsetstrokecolor{currentstroke}%
\pgfsetdash{}{0pt}%
\pgfpathmoveto{\pgfqpoint{1.452024in}{3.140754in}}%
\pgfpathcurveto{\pgfqpoint{1.457848in}{3.140754in}}{\pgfqpoint{1.463434in}{3.143068in}}{\pgfqpoint{1.467552in}{3.147186in}}%
\pgfpathcurveto{\pgfqpoint{1.471670in}{3.151304in}}{\pgfqpoint{1.473984in}{3.156890in}}{\pgfqpoint{1.473984in}{3.162714in}}%
\pgfpathcurveto{\pgfqpoint{1.473984in}{3.168538in}}{\pgfqpoint{1.471670in}{3.174124in}}{\pgfqpoint{1.467552in}{3.178243in}}%
\pgfpathcurveto{\pgfqpoint{1.463434in}{3.182361in}}{\pgfqpoint{1.457848in}{3.184675in}}{\pgfqpoint{1.452024in}{3.184675in}}%
\pgfpathcurveto{\pgfqpoint{1.446200in}{3.184675in}}{\pgfqpoint{1.440614in}{3.182361in}}{\pgfqpoint{1.436495in}{3.178243in}}%
\pgfpathcurveto{\pgfqpoint{1.432377in}{3.174124in}}{\pgfqpoint{1.430063in}{3.168538in}}{\pgfqpoint{1.430063in}{3.162714in}}%
\pgfpathcurveto{\pgfqpoint{1.430063in}{3.156890in}}{\pgfqpoint{1.432377in}{3.151304in}}{\pgfqpoint{1.436495in}{3.147186in}}%
\pgfpathcurveto{\pgfqpoint{1.440614in}{3.143068in}}{\pgfqpoint{1.446200in}{3.140754in}}{\pgfqpoint{1.452024in}{3.140754in}}%
\pgfpathlineto{\pgfqpoint{1.452024in}{3.140754in}}%
\pgfpathclose%
\pgfusepath{stroke,fill}%
\end{pgfscope}%
\begin{pgfscope}%
\pgfpathrectangle{\pgfqpoint{1.073501in}{0.880000in}}{\pgfqpoint{6.052998in}{6.160000in}}%
\pgfusepath{clip}%
\pgfsetbuttcap%
\pgfsetroundjoin%
\definecolor{currentfill}{rgb}{0.200000,0.200000,0.800000}%
\pgfsetfillcolor{currentfill}%
\pgfsetlinewidth{1.003750pt}%
\definecolor{currentstroke}{rgb}{0.200000,0.200000,0.800000}%
\pgfsetstrokecolor{currentstroke}%
\pgfsetdash{}{0pt}%
\pgfpathmoveto{\pgfqpoint{1.562159in}{2.992770in}}%
\pgfpathcurveto{\pgfqpoint{1.567983in}{2.992770in}}{\pgfqpoint{1.573569in}{2.995084in}}{\pgfqpoint{1.577687in}{2.999202in}}%
\pgfpathcurveto{\pgfqpoint{1.581805in}{3.003320in}}{\pgfqpoint{1.584119in}{3.008906in}}{\pgfqpoint{1.584119in}{3.014730in}}%
\pgfpathcurveto{\pgfqpoint{1.584119in}{3.020554in}}{\pgfqpoint{1.581805in}{3.026140in}}{\pgfqpoint{1.577687in}{3.030258in}}%
\pgfpathcurveto{\pgfqpoint{1.573569in}{3.034377in}}{\pgfqpoint{1.567983in}{3.036690in}}{\pgfqpoint{1.562159in}{3.036690in}}%
\pgfpathcurveto{\pgfqpoint{1.556335in}{3.036690in}}{\pgfqpoint{1.550749in}{3.034377in}}{\pgfqpoint{1.546630in}{3.030258in}}%
\pgfpathcurveto{\pgfqpoint{1.542512in}{3.026140in}}{\pgfqpoint{1.540198in}{3.020554in}}{\pgfqpoint{1.540198in}{3.014730in}}%
\pgfpathcurveto{\pgfqpoint{1.540198in}{3.008906in}}{\pgfqpoint{1.542512in}{3.003320in}}{\pgfqpoint{1.546630in}{2.999202in}}%
\pgfpathcurveto{\pgfqpoint{1.550749in}{2.995084in}}{\pgfqpoint{1.556335in}{2.992770in}}{\pgfqpoint{1.562159in}{2.992770in}}%
\pgfpathlineto{\pgfqpoint{1.562159in}{2.992770in}}%
\pgfpathclose%
\pgfusepath{stroke,fill}%
\end{pgfscope}%
\begin{pgfscope}%
\pgfpathrectangle{\pgfqpoint{1.073501in}{0.880000in}}{\pgfqpoint{6.052998in}{6.160000in}}%
\pgfusepath{clip}%
\pgfsetbuttcap%
\pgfsetroundjoin%
\definecolor{currentfill}{rgb}{0.200000,0.200000,0.800000}%
\pgfsetfillcolor{currentfill}%
\pgfsetlinewidth{1.003750pt}%
\definecolor{currentstroke}{rgb}{0.200000,0.200000,0.800000}%
\pgfsetstrokecolor{currentstroke}%
\pgfsetdash{}{0pt}%
\pgfpathmoveto{\pgfqpoint{1.635118in}{2.835838in}}%
\pgfpathcurveto{\pgfqpoint{1.640942in}{2.835838in}}{\pgfqpoint{1.646528in}{2.838152in}}{\pgfqpoint{1.650647in}{2.842270in}}%
\pgfpathcurveto{\pgfqpoint{1.654765in}{2.846388in}}{\pgfqpoint{1.657079in}{2.851974in}}{\pgfqpoint{1.657079in}{2.857798in}}%
\pgfpathcurveto{\pgfqpoint{1.657079in}{2.863622in}}{\pgfqpoint{1.654765in}{2.869208in}}{\pgfqpoint{1.650647in}{2.873327in}}%
\pgfpathcurveto{\pgfqpoint{1.646528in}{2.877445in}}{\pgfqpoint{1.640942in}{2.879759in}}{\pgfqpoint{1.635118in}{2.879759in}}%
\pgfpathcurveto{\pgfqpoint{1.629294in}{2.879759in}}{\pgfqpoint{1.623708in}{2.877445in}}{\pgfqpoint{1.619590in}{2.873327in}}%
\pgfpathcurveto{\pgfqpoint{1.615472in}{2.869208in}}{\pgfqpoint{1.613158in}{2.863622in}}{\pgfqpoint{1.613158in}{2.857798in}}%
\pgfpathcurveto{\pgfqpoint{1.613158in}{2.851974in}}{\pgfqpoint{1.615472in}{2.846388in}}{\pgfqpoint{1.619590in}{2.842270in}}%
\pgfpathcurveto{\pgfqpoint{1.623708in}{2.838152in}}{\pgfqpoint{1.629294in}{2.835838in}}{\pgfqpoint{1.635118in}{2.835838in}}%
\pgfpathlineto{\pgfqpoint{1.635118in}{2.835838in}}%
\pgfpathclose%
\pgfusepath{stroke,fill}%
\end{pgfscope}%
\begin{pgfscope}%
\pgfpathrectangle{\pgfqpoint{1.073501in}{0.880000in}}{\pgfqpoint{6.052998in}{6.160000in}}%
\pgfusepath{clip}%
\pgfsetbuttcap%
\pgfsetroundjoin%
\definecolor{currentfill}{rgb}{0.200000,0.200000,0.800000}%
\pgfsetfillcolor{currentfill}%
\pgfsetlinewidth{1.003750pt}%
\definecolor{currentstroke}{rgb}{0.200000,0.200000,0.800000}%
\pgfsetstrokecolor{currentstroke}%
\pgfsetdash{}{0pt}%
\pgfpathmoveto{\pgfqpoint{1.753836in}{2.703055in}}%
\pgfpathcurveto{\pgfqpoint{1.759660in}{2.703055in}}{\pgfqpoint{1.765246in}{2.705369in}}{\pgfqpoint{1.769364in}{2.709487in}}%
\pgfpathcurveto{\pgfqpoint{1.773482in}{2.713606in}}{\pgfqpoint{1.775796in}{2.719192in}}{\pgfqpoint{1.775796in}{2.725016in}}%
\pgfpathcurveto{\pgfqpoint{1.775796in}{2.730840in}}{\pgfqpoint{1.773482in}{2.736426in}}{\pgfqpoint{1.769364in}{2.740544in}}%
\pgfpathcurveto{\pgfqpoint{1.765246in}{2.744662in}}{\pgfqpoint{1.759660in}{2.746976in}}{\pgfqpoint{1.753836in}{2.746976in}}%
\pgfpathcurveto{\pgfqpoint{1.748012in}{2.746976in}}{\pgfqpoint{1.742426in}{2.744662in}}{\pgfqpoint{1.738307in}{2.740544in}}%
\pgfpathcurveto{\pgfqpoint{1.734189in}{2.736426in}}{\pgfqpoint{1.731875in}{2.730840in}}{\pgfqpoint{1.731875in}{2.725016in}}%
\pgfpathcurveto{\pgfqpoint{1.731875in}{2.719192in}}{\pgfqpoint{1.734189in}{2.713606in}}{\pgfqpoint{1.738307in}{2.709487in}}%
\pgfpathcurveto{\pgfqpoint{1.742426in}{2.705369in}}{\pgfqpoint{1.748012in}{2.703055in}}{\pgfqpoint{1.753836in}{2.703055in}}%
\pgfpathlineto{\pgfqpoint{1.753836in}{2.703055in}}%
\pgfpathclose%
\pgfusepath{stroke,fill}%
\end{pgfscope}%
\begin{pgfscope}%
\pgfpathrectangle{\pgfqpoint{1.073501in}{0.880000in}}{\pgfqpoint{6.052998in}{6.160000in}}%
\pgfusepath{clip}%
\pgfsetbuttcap%
\pgfsetroundjoin%
\definecolor{currentfill}{rgb}{0.200000,0.200000,0.800000}%
\pgfsetfillcolor{currentfill}%
\pgfsetlinewidth{1.003750pt}%
\definecolor{currentstroke}{rgb}{0.200000,0.200000,0.800000}%
\pgfsetstrokecolor{currentstroke}%
\pgfsetdash{}{0pt}%
\pgfpathmoveto{\pgfqpoint{1.898900in}{2.592652in}}%
\pgfpathcurveto{\pgfqpoint{1.904723in}{2.592652in}}{\pgfqpoint{1.910310in}{2.594965in}}{\pgfqpoint{1.914428in}{2.599084in}}%
\pgfpathcurveto{\pgfqpoint{1.918546in}{2.603202in}}{\pgfqpoint{1.920860in}{2.608788in}}{\pgfqpoint{1.920860in}{2.614612in}}%
\pgfpathcurveto{\pgfqpoint{1.920860in}{2.620436in}}{\pgfqpoint{1.918546in}{2.626022in}}{\pgfqpoint{1.914428in}{2.630140in}}%
\pgfpathcurveto{\pgfqpoint{1.910310in}{2.634258in}}{\pgfqpoint{1.904723in}{2.636572in}}{\pgfqpoint{1.898900in}{2.636572in}}%
\pgfpathcurveto{\pgfqpoint{1.893076in}{2.636572in}}{\pgfqpoint{1.887489in}{2.634258in}}{\pgfqpoint{1.883371in}{2.630140in}}%
\pgfpathcurveto{\pgfqpoint{1.879253in}{2.626022in}}{\pgfqpoint{1.876939in}{2.620436in}}{\pgfqpoint{1.876939in}{2.614612in}}%
\pgfpathcurveto{\pgfqpoint{1.876939in}{2.608788in}}{\pgfqpoint{1.879253in}{2.603202in}}{\pgfqpoint{1.883371in}{2.599084in}}%
\pgfpathcurveto{\pgfqpoint{1.887489in}{2.594965in}}{\pgfqpoint{1.893076in}{2.592652in}}{\pgfqpoint{1.898900in}{2.592652in}}%
\pgfpathlineto{\pgfqpoint{1.898900in}{2.592652in}}%
\pgfpathclose%
\pgfusepath{stroke,fill}%
\end{pgfscope}%
\begin{pgfscope}%
\pgfpathrectangle{\pgfqpoint{1.073501in}{0.880000in}}{\pgfqpoint{6.052998in}{6.160000in}}%
\pgfusepath{clip}%
\pgfsetbuttcap%
\pgfsetroundjoin%
\definecolor{currentfill}{rgb}{0.200000,0.200000,0.800000}%
\pgfsetfillcolor{currentfill}%
\pgfsetlinewidth{1.003750pt}%
\definecolor{currentstroke}{rgb}{0.200000,0.200000,0.800000}%
\pgfsetstrokecolor{currentstroke}%
\pgfsetdash{}{0pt}%
\pgfpathmoveto{\pgfqpoint{1.833420in}{2.347733in}}%
\pgfpathcurveto{\pgfqpoint{1.839244in}{2.347733in}}{\pgfqpoint{1.844830in}{2.350047in}}{\pgfqpoint{1.848948in}{2.354165in}}%
\pgfpathcurveto{\pgfqpoint{1.853067in}{2.358283in}}{\pgfqpoint{1.855380in}{2.363869in}}{\pgfqpoint{1.855380in}{2.369693in}}%
\pgfpathcurveto{\pgfqpoint{1.855380in}{2.375517in}}{\pgfqpoint{1.853067in}{2.381103in}}{\pgfqpoint{1.848948in}{2.385221in}}%
\pgfpathcurveto{\pgfqpoint{1.844830in}{2.389340in}}{\pgfqpoint{1.839244in}{2.391653in}}{\pgfqpoint{1.833420in}{2.391653in}}%
\pgfpathcurveto{\pgfqpoint{1.827596in}{2.391653in}}{\pgfqpoint{1.822010in}{2.389340in}}{\pgfqpoint{1.817892in}{2.385221in}}%
\pgfpathcurveto{\pgfqpoint{1.813774in}{2.381103in}}{\pgfqpoint{1.811460in}{2.375517in}}{\pgfqpoint{1.811460in}{2.369693in}}%
\pgfpathcurveto{\pgfqpoint{1.811460in}{2.363869in}}{\pgfqpoint{1.813774in}{2.358283in}}{\pgfqpoint{1.817892in}{2.354165in}}%
\pgfpathcurveto{\pgfqpoint{1.822010in}{2.350047in}}{\pgfqpoint{1.827596in}{2.347733in}}{\pgfqpoint{1.833420in}{2.347733in}}%
\pgfpathlineto{\pgfqpoint{1.833420in}{2.347733in}}%
\pgfpathclose%
\pgfusepath{stroke,fill}%
\end{pgfscope}%
\begin{pgfscope}%
\pgfpathrectangle{\pgfqpoint{1.073501in}{0.880000in}}{\pgfqpoint{6.052998in}{6.160000in}}%
\pgfusepath{clip}%
\pgfsetbuttcap%
\pgfsetroundjoin%
\definecolor{currentfill}{rgb}{0.200000,0.200000,0.800000}%
\pgfsetfillcolor{currentfill}%
\pgfsetlinewidth{1.003750pt}%
\definecolor{currentstroke}{rgb}{0.200000,0.200000,0.800000}%
\pgfsetstrokecolor{currentstroke}%
\pgfsetdash{}{0pt}%
\pgfpathmoveto{\pgfqpoint{2.003715in}{2.257114in}}%
\pgfpathcurveto{\pgfqpoint{2.009539in}{2.257114in}}{\pgfqpoint{2.015126in}{2.259428in}}{\pgfqpoint{2.019244in}{2.263546in}}%
\pgfpathcurveto{\pgfqpoint{2.023362in}{2.267664in}}{\pgfqpoint{2.025676in}{2.273250in}}{\pgfqpoint{2.025676in}{2.279074in}}%
\pgfpathcurveto{\pgfqpoint{2.025676in}{2.284898in}}{\pgfqpoint{2.023362in}{2.290484in}}{\pgfqpoint{2.019244in}{2.294603in}}%
\pgfpathcurveto{\pgfqpoint{2.015126in}{2.298721in}}{\pgfqpoint{2.009539in}{2.301035in}}{\pgfqpoint{2.003715in}{2.301035in}}%
\pgfpathcurveto{\pgfqpoint{1.997892in}{2.301035in}}{\pgfqpoint{1.992305in}{2.298721in}}{\pgfqpoint{1.988187in}{2.294603in}}%
\pgfpathcurveto{\pgfqpoint{1.984069in}{2.290484in}}{\pgfqpoint{1.981755in}{2.284898in}}{\pgfqpoint{1.981755in}{2.279074in}}%
\pgfpathcurveto{\pgfqpoint{1.981755in}{2.273250in}}{\pgfqpoint{1.984069in}{2.267664in}}{\pgfqpoint{1.988187in}{2.263546in}}%
\pgfpathcurveto{\pgfqpoint{1.992305in}{2.259428in}}{\pgfqpoint{1.997892in}{2.257114in}}{\pgfqpoint{2.003715in}{2.257114in}}%
\pgfpathlineto{\pgfqpoint{2.003715in}{2.257114in}}%
\pgfpathclose%
\pgfusepath{stroke,fill}%
\end{pgfscope}%
\begin{pgfscope}%
\pgfpathrectangle{\pgfqpoint{1.073501in}{0.880000in}}{\pgfqpoint{6.052998in}{6.160000in}}%
\pgfusepath{clip}%
\pgfsetbuttcap%
\pgfsetroundjoin%
\definecolor{currentfill}{rgb}{0.200000,0.200000,0.800000}%
\pgfsetfillcolor{currentfill}%
\pgfsetlinewidth{1.003750pt}%
\definecolor{currentstroke}{rgb}{0.200000,0.200000,0.800000}%
\pgfsetstrokecolor{currentstroke}%
\pgfsetdash{}{0pt}%
\pgfpathmoveto{\pgfqpoint{2.143188in}{2.152168in}}%
\pgfpathcurveto{\pgfqpoint{2.149012in}{2.152168in}}{\pgfqpoint{2.154598in}{2.154482in}}{\pgfqpoint{2.158716in}{2.158600in}}%
\pgfpathcurveto{\pgfqpoint{2.162834in}{2.162718in}}{\pgfqpoint{2.165148in}{2.168304in}}{\pgfqpoint{2.165148in}{2.174128in}}%
\pgfpathcurveto{\pgfqpoint{2.165148in}{2.179952in}}{\pgfqpoint{2.162834in}{2.185538in}}{\pgfqpoint{2.158716in}{2.189657in}}%
\pgfpathcurveto{\pgfqpoint{2.154598in}{2.193775in}}{\pgfqpoint{2.149012in}{2.196089in}}{\pgfqpoint{2.143188in}{2.196089in}}%
\pgfpathcurveto{\pgfqpoint{2.137364in}{2.196089in}}{\pgfqpoint{2.131778in}{2.193775in}}{\pgfqpoint{2.127659in}{2.189657in}}%
\pgfpathcurveto{\pgfqpoint{2.123541in}{2.185538in}}{\pgfqpoint{2.121227in}{2.179952in}}{\pgfqpoint{2.121227in}{2.174128in}}%
\pgfpathcurveto{\pgfqpoint{2.121227in}{2.168304in}}{\pgfqpoint{2.123541in}{2.162718in}}{\pgfqpoint{2.127659in}{2.158600in}}%
\pgfpathcurveto{\pgfqpoint{2.131778in}{2.154482in}}{\pgfqpoint{2.137364in}{2.152168in}}{\pgfqpoint{2.143188in}{2.152168in}}%
\pgfpathlineto{\pgfqpoint{2.143188in}{2.152168in}}%
\pgfpathclose%
\pgfusepath{stroke,fill}%
\end{pgfscope}%
\begin{pgfscope}%
\pgfpathrectangle{\pgfqpoint{1.073501in}{0.880000in}}{\pgfqpoint{6.052998in}{6.160000in}}%
\pgfusepath{clip}%
\pgfsetbuttcap%
\pgfsetroundjoin%
\definecolor{currentfill}{rgb}{0.200000,0.200000,0.800000}%
\pgfsetfillcolor{currentfill}%
\pgfsetlinewidth{1.003750pt}%
\definecolor{currentstroke}{rgb}{0.200000,0.200000,0.800000}%
\pgfsetstrokecolor{currentstroke}%
\pgfsetdash{}{0pt}%
\pgfpathmoveto{\pgfqpoint{2.261728in}{2.032098in}}%
\pgfpathcurveto{\pgfqpoint{2.267552in}{2.032098in}}{\pgfqpoint{2.273138in}{2.034412in}}{\pgfqpoint{2.277256in}{2.038530in}}%
\pgfpathcurveto{\pgfqpoint{2.281374in}{2.042648in}}{\pgfqpoint{2.283688in}{2.048234in}}{\pgfqpoint{2.283688in}{2.054058in}}%
\pgfpathcurveto{\pgfqpoint{2.283688in}{2.059882in}}{\pgfqpoint{2.281374in}{2.065468in}}{\pgfqpoint{2.277256in}{2.069586in}}%
\pgfpathcurveto{\pgfqpoint{2.273138in}{2.073705in}}{\pgfqpoint{2.267552in}{2.076018in}}{\pgfqpoint{2.261728in}{2.076018in}}%
\pgfpathcurveto{\pgfqpoint{2.255904in}{2.076018in}}{\pgfqpoint{2.250318in}{2.073705in}}{\pgfqpoint{2.246200in}{2.069586in}}%
\pgfpathcurveto{\pgfqpoint{2.242082in}{2.065468in}}{\pgfqpoint{2.239768in}{2.059882in}}{\pgfqpoint{2.239768in}{2.054058in}}%
\pgfpathcurveto{\pgfqpoint{2.239768in}{2.048234in}}{\pgfqpoint{2.242082in}{2.042648in}}{\pgfqpoint{2.246200in}{2.038530in}}%
\pgfpathcurveto{\pgfqpoint{2.250318in}{2.034412in}}{\pgfqpoint{2.255904in}{2.032098in}}{\pgfqpoint{2.261728in}{2.032098in}}%
\pgfpathlineto{\pgfqpoint{2.261728in}{2.032098in}}%
\pgfpathclose%
\pgfusepath{stroke,fill}%
\end{pgfscope}%
\begin{pgfscope}%
\pgfpathrectangle{\pgfqpoint{1.073501in}{0.880000in}}{\pgfqpoint{6.052998in}{6.160000in}}%
\pgfusepath{clip}%
\pgfsetbuttcap%
\pgfsetroundjoin%
\definecolor{currentfill}{rgb}{0.200000,0.200000,0.800000}%
\pgfsetfillcolor{currentfill}%
\pgfsetlinewidth{1.003750pt}%
\definecolor{currentstroke}{rgb}{0.200000,0.200000,0.800000}%
\pgfsetstrokecolor{currentstroke}%
\pgfsetdash{}{0pt}%
\pgfpathmoveto{\pgfqpoint{2.362046in}{1.890530in}}%
\pgfpathcurveto{\pgfqpoint{2.367870in}{1.890530in}}{\pgfqpoint{2.373457in}{1.892844in}}{\pgfqpoint{2.377575in}{1.896962in}}%
\pgfpathcurveto{\pgfqpoint{2.381693in}{1.901080in}}{\pgfqpoint{2.384007in}{1.906666in}}{\pgfqpoint{2.384007in}{1.912490in}}%
\pgfpathcurveto{\pgfqpoint{2.384007in}{1.918314in}}{\pgfqpoint{2.381693in}{1.923900in}}{\pgfqpoint{2.377575in}{1.928018in}}%
\pgfpathcurveto{\pgfqpoint{2.373457in}{1.932137in}}{\pgfqpoint{2.367870in}{1.934450in}}{\pgfqpoint{2.362046in}{1.934450in}}%
\pgfpathcurveto{\pgfqpoint{2.356223in}{1.934450in}}{\pgfqpoint{2.350636in}{1.932137in}}{\pgfqpoint{2.346518in}{1.928018in}}%
\pgfpathcurveto{\pgfqpoint{2.342400in}{1.923900in}}{\pgfqpoint{2.340086in}{1.918314in}}{\pgfqpoint{2.340086in}{1.912490in}}%
\pgfpathcurveto{\pgfqpoint{2.340086in}{1.906666in}}{\pgfqpoint{2.342400in}{1.901080in}}{\pgfqpoint{2.346518in}{1.896962in}}%
\pgfpathcurveto{\pgfqpoint{2.350636in}{1.892844in}}{\pgfqpoint{2.356223in}{1.890530in}}{\pgfqpoint{2.362046in}{1.890530in}}%
\pgfpathlineto{\pgfqpoint{2.362046in}{1.890530in}}%
\pgfpathclose%
\pgfusepath{stroke,fill}%
\end{pgfscope}%
\begin{pgfscope}%
\pgfpathrectangle{\pgfqpoint{1.073501in}{0.880000in}}{\pgfqpoint{6.052998in}{6.160000in}}%
\pgfusepath{clip}%
\pgfsetbuttcap%
\pgfsetroundjoin%
\definecolor{currentfill}{rgb}{0.200000,0.200000,0.800000}%
\pgfsetfillcolor{currentfill}%
\pgfsetlinewidth{1.003750pt}%
\definecolor{currentstroke}{rgb}{0.200000,0.200000,0.800000}%
\pgfsetstrokecolor{currentstroke}%
\pgfsetdash{}{0pt}%
\pgfpathmoveto{\pgfqpoint{2.449804in}{1.723807in}}%
\pgfpathcurveto{\pgfqpoint{2.455628in}{1.723807in}}{\pgfqpoint{2.461214in}{1.726121in}}{\pgfqpoint{2.465332in}{1.730239in}}%
\pgfpathcurveto{\pgfqpoint{2.469450in}{1.734357in}}{\pgfqpoint{2.471764in}{1.739943in}}{\pgfqpoint{2.471764in}{1.745767in}}%
\pgfpathcurveto{\pgfqpoint{2.471764in}{1.751591in}}{\pgfqpoint{2.469450in}{1.757177in}}{\pgfqpoint{2.465332in}{1.761296in}}%
\pgfpathcurveto{\pgfqpoint{2.461214in}{1.765414in}}{\pgfqpoint{2.455628in}{1.767728in}}{\pgfqpoint{2.449804in}{1.767728in}}%
\pgfpathcurveto{\pgfqpoint{2.443980in}{1.767728in}}{\pgfqpoint{2.438394in}{1.765414in}}{\pgfqpoint{2.434275in}{1.761296in}}%
\pgfpathcurveto{\pgfqpoint{2.430157in}{1.757177in}}{\pgfqpoint{2.427843in}{1.751591in}}{\pgfqpoint{2.427843in}{1.745767in}}%
\pgfpathcurveto{\pgfqpoint{2.427843in}{1.739943in}}{\pgfqpoint{2.430157in}{1.734357in}}{\pgfqpoint{2.434275in}{1.730239in}}%
\pgfpathcurveto{\pgfqpoint{2.438394in}{1.726121in}}{\pgfqpoint{2.443980in}{1.723807in}}{\pgfqpoint{2.449804in}{1.723807in}}%
\pgfpathlineto{\pgfqpoint{2.449804in}{1.723807in}}%
\pgfpathclose%
\pgfusepath{stroke,fill}%
\end{pgfscope}%
\begin{pgfscope}%
\pgfpathrectangle{\pgfqpoint{1.073501in}{0.880000in}}{\pgfqpoint{6.052998in}{6.160000in}}%
\pgfusepath{clip}%
\pgfsetbuttcap%
\pgfsetroundjoin%
\definecolor{currentfill}{rgb}{0.200000,0.200000,0.800000}%
\pgfsetfillcolor{currentfill}%
\pgfsetlinewidth{1.003750pt}%
\definecolor{currentstroke}{rgb}{0.200000,0.200000,0.800000}%
\pgfsetstrokecolor{currentstroke}%
\pgfsetdash{}{0pt}%
\pgfpathmoveto{\pgfqpoint{2.641020in}{1.694378in}}%
\pgfpathcurveto{\pgfqpoint{2.646844in}{1.694378in}}{\pgfqpoint{2.652430in}{1.696692in}}{\pgfqpoint{2.656548in}{1.700810in}}%
\pgfpathcurveto{\pgfqpoint{2.660667in}{1.704928in}}{\pgfqpoint{2.662980in}{1.710514in}}{\pgfqpoint{2.662980in}{1.716338in}}%
\pgfpathcurveto{\pgfqpoint{2.662980in}{1.722162in}}{\pgfqpoint{2.660667in}{1.727748in}}{\pgfqpoint{2.656548in}{1.731866in}}%
\pgfpathcurveto{\pgfqpoint{2.652430in}{1.735984in}}{\pgfqpoint{2.646844in}{1.738298in}}{\pgfqpoint{2.641020in}{1.738298in}}%
\pgfpathcurveto{\pgfqpoint{2.635196in}{1.738298in}}{\pgfqpoint{2.629610in}{1.735984in}}{\pgfqpoint{2.625492in}{1.731866in}}%
\pgfpathcurveto{\pgfqpoint{2.621374in}{1.727748in}}{\pgfqpoint{2.619060in}{1.722162in}}{\pgfqpoint{2.619060in}{1.716338in}}%
\pgfpathcurveto{\pgfqpoint{2.619060in}{1.710514in}}{\pgfqpoint{2.621374in}{1.704928in}}{\pgfqpoint{2.625492in}{1.700810in}}%
\pgfpathcurveto{\pgfqpoint{2.629610in}{1.696692in}}{\pgfqpoint{2.635196in}{1.694378in}}{\pgfqpoint{2.641020in}{1.694378in}}%
\pgfpathlineto{\pgfqpoint{2.641020in}{1.694378in}}%
\pgfpathclose%
\pgfusepath{stroke,fill}%
\end{pgfscope}%
\begin{pgfscope}%
\pgfpathrectangle{\pgfqpoint{1.073501in}{0.880000in}}{\pgfqpoint{6.052998in}{6.160000in}}%
\pgfusepath{clip}%
\pgfsetbuttcap%
\pgfsetroundjoin%
\definecolor{currentfill}{rgb}{0.200000,0.200000,0.800000}%
\pgfsetfillcolor{currentfill}%
\pgfsetlinewidth{1.003750pt}%
\definecolor{currentstroke}{rgb}{0.200000,0.200000,0.800000}%
\pgfsetstrokecolor{currentstroke}%
\pgfsetdash{}{0pt}%
\pgfpathmoveto{\pgfqpoint{2.780464in}{1.595931in}}%
\pgfpathcurveto{\pgfqpoint{2.786288in}{1.595931in}}{\pgfqpoint{2.791875in}{1.598245in}}{\pgfqpoint{2.795993in}{1.602363in}}%
\pgfpathcurveto{\pgfqpoint{2.800111in}{1.606481in}}{\pgfqpoint{2.802425in}{1.612067in}}{\pgfqpoint{2.802425in}{1.617891in}}%
\pgfpathcurveto{\pgfqpoint{2.802425in}{1.623715in}}{\pgfqpoint{2.800111in}{1.629301in}}{\pgfqpoint{2.795993in}{1.633419in}}%
\pgfpathcurveto{\pgfqpoint{2.791875in}{1.637537in}}{\pgfqpoint{2.786288in}{1.639851in}}{\pgfqpoint{2.780464in}{1.639851in}}%
\pgfpathcurveto{\pgfqpoint{2.774640in}{1.639851in}}{\pgfqpoint{2.769054in}{1.637537in}}{\pgfqpoint{2.764936in}{1.633419in}}%
\pgfpathcurveto{\pgfqpoint{2.760818in}{1.629301in}}{\pgfqpoint{2.758504in}{1.623715in}}{\pgfqpoint{2.758504in}{1.617891in}}%
\pgfpathcurveto{\pgfqpoint{2.758504in}{1.612067in}}{\pgfqpoint{2.760818in}{1.606481in}}{\pgfqpoint{2.764936in}{1.602363in}}%
\pgfpathcurveto{\pgfqpoint{2.769054in}{1.598245in}}{\pgfqpoint{2.774640in}{1.595931in}}{\pgfqpoint{2.780464in}{1.595931in}}%
\pgfpathlineto{\pgfqpoint{2.780464in}{1.595931in}}%
\pgfpathclose%
\pgfusepath{stroke,fill}%
\end{pgfscope}%
\begin{pgfscope}%
\pgfpathrectangle{\pgfqpoint{1.073501in}{0.880000in}}{\pgfqpoint{6.052998in}{6.160000in}}%
\pgfusepath{clip}%
\pgfsetbuttcap%
\pgfsetroundjoin%
\definecolor{currentfill}{rgb}{0.200000,0.200000,0.800000}%
\pgfsetfillcolor{currentfill}%
\pgfsetlinewidth{1.003750pt}%
\definecolor{currentstroke}{rgb}{0.200000,0.200000,0.800000}%
\pgfsetstrokecolor{currentstroke}%
\pgfsetdash{}{0pt}%
\pgfpathmoveto{\pgfqpoint{2.912183in}{1.477458in}}%
\pgfpathcurveto{\pgfqpoint{2.918007in}{1.477458in}}{\pgfqpoint{2.923593in}{1.479772in}}{\pgfqpoint{2.927711in}{1.483890in}}%
\pgfpathcurveto{\pgfqpoint{2.931829in}{1.488008in}}{\pgfqpoint{2.934143in}{1.493594in}}{\pgfqpoint{2.934143in}{1.499418in}}%
\pgfpathcurveto{\pgfqpoint{2.934143in}{1.505242in}}{\pgfqpoint{2.931829in}{1.510829in}}{\pgfqpoint{2.927711in}{1.514947in}}%
\pgfpathcurveto{\pgfqpoint{2.923593in}{1.519065in}}{\pgfqpoint{2.918007in}{1.521379in}}{\pgfqpoint{2.912183in}{1.521379in}}%
\pgfpathcurveto{\pgfqpoint{2.906359in}{1.521379in}}{\pgfqpoint{2.900773in}{1.519065in}}{\pgfqpoint{2.896655in}{1.514947in}}%
\pgfpathcurveto{\pgfqpoint{2.892536in}{1.510829in}}{\pgfqpoint{2.890223in}{1.505242in}}{\pgfqpoint{2.890223in}{1.499418in}}%
\pgfpathcurveto{\pgfqpoint{2.890223in}{1.493594in}}{\pgfqpoint{2.892536in}{1.488008in}}{\pgfqpoint{2.896655in}{1.483890in}}%
\pgfpathcurveto{\pgfqpoint{2.900773in}{1.479772in}}{\pgfqpoint{2.906359in}{1.477458in}}{\pgfqpoint{2.912183in}{1.477458in}}%
\pgfpathlineto{\pgfqpoint{2.912183in}{1.477458in}}%
\pgfpathclose%
\pgfusepath{stroke,fill}%
\end{pgfscope}%
\begin{pgfscope}%
\pgfpathrectangle{\pgfqpoint{1.073501in}{0.880000in}}{\pgfqpoint{6.052998in}{6.160000in}}%
\pgfusepath{clip}%
\pgfsetbuttcap%
\pgfsetroundjoin%
\definecolor{currentfill}{rgb}{0.200000,0.200000,0.800000}%
\pgfsetfillcolor{currentfill}%
\pgfsetlinewidth{1.003750pt}%
\definecolor{currentstroke}{rgb}{0.200000,0.200000,0.800000}%
\pgfsetstrokecolor{currentstroke}%
\pgfsetdash{}{0pt}%
\pgfpathmoveto{\pgfqpoint{3.102901in}{1.484193in}}%
\pgfpathcurveto{\pgfqpoint{3.108725in}{1.484193in}}{\pgfqpoint{3.114311in}{1.486507in}}{\pgfqpoint{3.118430in}{1.490625in}}%
\pgfpathcurveto{\pgfqpoint{3.122548in}{1.494743in}}{\pgfqpoint{3.124862in}{1.500330in}}{\pgfqpoint{3.124862in}{1.506154in}}%
\pgfpathcurveto{\pgfqpoint{3.124862in}{1.511977in}}{\pgfqpoint{3.122548in}{1.517564in}}{\pgfqpoint{3.118430in}{1.521682in}}%
\pgfpathcurveto{\pgfqpoint{3.114311in}{1.525800in}}{\pgfqpoint{3.108725in}{1.528114in}}{\pgfqpoint{3.102901in}{1.528114in}}%
\pgfpathcurveto{\pgfqpoint{3.097077in}{1.528114in}}{\pgfqpoint{3.091491in}{1.525800in}}{\pgfqpoint{3.087373in}{1.521682in}}%
\pgfpathcurveto{\pgfqpoint{3.083255in}{1.517564in}}{\pgfqpoint{3.080941in}{1.511977in}}{\pgfqpoint{3.080941in}{1.506154in}}%
\pgfpathcurveto{\pgfqpoint{3.080941in}{1.500330in}}{\pgfqpoint{3.083255in}{1.494743in}}{\pgfqpoint{3.087373in}{1.490625in}}%
\pgfpathcurveto{\pgfqpoint{3.091491in}{1.486507in}}{\pgfqpoint{3.097077in}{1.484193in}}{\pgfqpoint{3.102901in}{1.484193in}}%
\pgfpathlineto{\pgfqpoint{3.102901in}{1.484193in}}%
\pgfpathclose%
\pgfusepath{stroke,fill}%
\end{pgfscope}%
\begin{pgfscope}%
\pgfpathrectangle{\pgfqpoint{1.073501in}{0.880000in}}{\pgfqpoint{6.052998in}{6.160000in}}%
\pgfusepath{clip}%
\pgfsetbuttcap%
\pgfsetroundjoin%
\definecolor{currentfill}{rgb}{0.200000,0.200000,0.800000}%
\pgfsetfillcolor{currentfill}%
\pgfsetlinewidth{1.003750pt}%
\definecolor{currentstroke}{rgb}{0.200000,0.200000,0.800000}%
\pgfsetstrokecolor{currentstroke}%
\pgfsetdash{}{0pt}%
\pgfpathmoveto{\pgfqpoint{3.259869in}{1.424398in}}%
\pgfpathcurveto{\pgfqpoint{3.265693in}{1.424398in}}{\pgfqpoint{3.271279in}{1.426711in}}{\pgfqpoint{3.275397in}{1.430830in}}%
\pgfpathcurveto{\pgfqpoint{3.279515in}{1.434948in}}{\pgfqpoint{3.281829in}{1.440534in}}{\pgfqpoint{3.281829in}{1.446358in}}%
\pgfpathcurveto{\pgfqpoint{3.281829in}{1.452182in}}{\pgfqpoint{3.279515in}{1.457768in}}{\pgfqpoint{3.275397in}{1.461886in}}%
\pgfpathcurveto{\pgfqpoint{3.271279in}{1.466004in}}{\pgfqpoint{3.265693in}{1.468318in}}{\pgfqpoint{3.259869in}{1.468318in}}%
\pgfpathcurveto{\pgfqpoint{3.254045in}{1.468318in}}{\pgfqpoint{3.248459in}{1.466004in}}{\pgfqpoint{3.244341in}{1.461886in}}%
\pgfpathcurveto{\pgfqpoint{3.240222in}{1.457768in}}{\pgfqpoint{3.237909in}{1.452182in}}{\pgfqpoint{3.237909in}{1.446358in}}%
\pgfpathcurveto{\pgfqpoint{3.237909in}{1.440534in}}{\pgfqpoint{3.240222in}{1.434948in}}{\pgfqpoint{3.244341in}{1.430830in}}%
\pgfpathcurveto{\pgfqpoint{3.248459in}{1.426711in}}{\pgfqpoint{3.254045in}{1.424398in}}{\pgfqpoint{3.259869in}{1.424398in}}%
\pgfpathlineto{\pgfqpoint{3.259869in}{1.424398in}}%
\pgfpathclose%
\pgfusepath{stroke,fill}%
\end{pgfscope}%
\begin{pgfscope}%
\pgfpathrectangle{\pgfqpoint{1.073501in}{0.880000in}}{\pgfqpoint{6.052998in}{6.160000in}}%
\pgfusepath{clip}%
\pgfsetbuttcap%
\pgfsetroundjoin%
\definecolor{currentfill}{rgb}{0.200000,0.200000,0.800000}%
\pgfsetfillcolor{currentfill}%
\pgfsetlinewidth{1.003750pt}%
\definecolor{currentstroke}{rgb}{0.200000,0.200000,0.800000}%
\pgfsetstrokecolor{currentstroke}%
\pgfsetdash{}{0pt}%
\pgfpathmoveto{\pgfqpoint{3.427166in}{1.399505in}}%
\pgfpathcurveto{\pgfqpoint{3.432990in}{1.399505in}}{\pgfqpoint{3.438576in}{1.401819in}}{\pgfqpoint{3.442694in}{1.405937in}}%
\pgfpathcurveto{\pgfqpoint{3.446813in}{1.410055in}}{\pgfqpoint{3.449126in}{1.415641in}}{\pgfqpoint{3.449126in}{1.421465in}}%
\pgfpathcurveto{\pgfqpoint{3.449126in}{1.427289in}}{\pgfqpoint{3.446813in}{1.432875in}}{\pgfqpoint{3.442694in}{1.436994in}}%
\pgfpathcurveto{\pgfqpoint{3.438576in}{1.441112in}}{\pgfqpoint{3.432990in}{1.443426in}}{\pgfqpoint{3.427166in}{1.443426in}}%
\pgfpathcurveto{\pgfqpoint{3.421342in}{1.443426in}}{\pgfqpoint{3.415756in}{1.441112in}}{\pgfqpoint{3.411638in}{1.436994in}}%
\pgfpathcurveto{\pgfqpoint{3.407520in}{1.432875in}}{\pgfqpoint{3.405206in}{1.427289in}}{\pgfqpoint{3.405206in}{1.421465in}}%
\pgfpathcurveto{\pgfqpoint{3.405206in}{1.415641in}}{\pgfqpoint{3.407520in}{1.410055in}}{\pgfqpoint{3.411638in}{1.405937in}}%
\pgfpathcurveto{\pgfqpoint{3.415756in}{1.401819in}}{\pgfqpoint{3.421342in}{1.399505in}}{\pgfqpoint{3.427166in}{1.399505in}}%
\pgfpathlineto{\pgfqpoint{3.427166in}{1.399505in}}%
\pgfpathclose%
\pgfusepath{stroke,fill}%
\end{pgfscope}%
\begin{pgfscope}%
\pgfpathrectangle{\pgfqpoint{1.073501in}{0.880000in}}{\pgfqpoint{6.052998in}{6.160000in}}%
\pgfusepath{clip}%
\pgfsetbuttcap%
\pgfsetroundjoin%
\definecolor{currentfill}{rgb}{0.200000,0.200000,0.800000}%
\pgfsetfillcolor{currentfill}%
\pgfsetlinewidth{1.003750pt}%
\definecolor{currentstroke}{rgb}{0.200000,0.200000,0.800000}%
\pgfsetstrokecolor{currentstroke}%
\pgfsetdash{}{0pt}%
\pgfpathmoveto{\pgfqpoint{3.553659in}{1.192355in}}%
\pgfpathcurveto{\pgfqpoint{3.559483in}{1.192355in}}{\pgfqpoint{3.565070in}{1.194669in}}{\pgfqpoint{3.569188in}{1.198787in}}%
\pgfpathcurveto{\pgfqpoint{3.573306in}{1.202905in}}{\pgfqpoint{3.575620in}{1.208492in}}{\pgfqpoint{3.575620in}{1.214316in}}%
\pgfpathcurveto{\pgfqpoint{3.575620in}{1.220139in}}{\pgfqpoint{3.573306in}{1.225726in}}{\pgfqpoint{3.569188in}{1.229844in}}%
\pgfpathcurveto{\pgfqpoint{3.565070in}{1.233962in}}{\pgfqpoint{3.559483in}{1.236276in}}{\pgfqpoint{3.553659in}{1.236276in}}%
\pgfpathcurveto{\pgfqpoint{3.547836in}{1.236276in}}{\pgfqpoint{3.542249in}{1.233962in}}{\pgfqpoint{3.538131in}{1.229844in}}%
\pgfpathcurveto{\pgfqpoint{3.534013in}{1.225726in}}{\pgfqpoint{3.531699in}{1.220139in}}{\pgfqpoint{3.531699in}{1.214316in}}%
\pgfpathcurveto{\pgfqpoint{3.531699in}{1.208492in}}{\pgfqpoint{3.534013in}{1.202905in}}{\pgfqpoint{3.538131in}{1.198787in}}%
\pgfpathcurveto{\pgfqpoint{3.542249in}{1.194669in}}{\pgfqpoint{3.547836in}{1.192355in}}{\pgfqpoint{3.553659in}{1.192355in}}%
\pgfpathlineto{\pgfqpoint{3.553659in}{1.192355in}}%
\pgfpathclose%
\pgfusepath{stroke,fill}%
\end{pgfscope}%
\begin{pgfscope}%
\pgfpathrectangle{\pgfqpoint{1.073501in}{0.880000in}}{\pgfqpoint{6.052998in}{6.160000in}}%
\pgfusepath{clip}%
\pgfsetbuttcap%
\pgfsetroundjoin%
\definecolor{currentfill}{rgb}{0.200000,0.200000,0.800000}%
\pgfsetfillcolor{currentfill}%
\pgfsetlinewidth{1.003750pt}%
\definecolor{currentstroke}{rgb}{0.200000,0.200000,0.800000}%
\pgfsetstrokecolor{currentstroke}%
\pgfsetdash{}{0pt}%
\pgfpathmoveto{\pgfqpoint{3.753024in}{1.334407in}}%
\pgfpathcurveto{\pgfqpoint{3.758848in}{1.334407in}}{\pgfqpoint{3.764435in}{1.336721in}}{\pgfqpoint{3.768553in}{1.340839in}}%
\pgfpathcurveto{\pgfqpoint{3.772671in}{1.344957in}}{\pgfqpoint{3.774985in}{1.350544in}}{\pgfqpoint{3.774985in}{1.356367in}}%
\pgfpathcurveto{\pgfqpoint{3.774985in}{1.362191in}}{\pgfqpoint{3.772671in}{1.367778in}}{\pgfqpoint{3.768553in}{1.371896in}}%
\pgfpathcurveto{\pgfqpoint{3.764435in}{1.376014in}}{\pgfqpoint{3.758848in}{1.378328in}}{\pgfqpoint{3.753024in}{1.378328in}}%
\pgfpathcurveto{\pgfqpoint{3.747201in}{1.378328in}}{\pgfqpoint{3.741614in}{1.376014in}}{\pgfqpoint{3.737496in}{1.371896in}}%
\pgfpathcurveto{\pgfqpoint{3.733378in}{1.367778in}}{\pgfqpoint{3.731064in}{1.362191in}}{\pgfqpoint{3.731064in}{1.356367in}}%
\pgfpathcurveto{\pgfqpoint{3.731064in}{1.350544in}}{\pgfqpoint{3.733378in}{1.344957in}}{\pgfqpoint{3.737496in}{1.340839in}}%
\pgfpathcurveto{\pgfqpoint{3.741614in}{1.336721in}}{\pgfqpoint{3.747201in}{1.334407in}}{\pgfqpoint{3.753024in}{1.334407in}}%
\pgfpathlineto{\pgfqpoint{3.753024in}{1.334407in}}%
\pgfpathclose%
\pgfusepath{stroke,fill}%
\end{pgfscope}%
\begin{pgfscope}%
\pgfpathrectangle{\pgfqpoint{1.073501in}{0.880000in}}{\pgfqpoint{6.052998in}{6.160000in}}%
\pgfusepath{clip}%
\pgfsetbuttcap%
\pgfsetroundjoin%
\definecolor{currentfill}{rgb}{0.200000,0.200000,0.800000}%
\pgfsetfillcolor{currentfill}%
\pgfsetlinewidth{1.003750pt}%
\definecolor{currentstroke}{rgb}{0.200000,0.200000,0.800000}%
\pgfsetstrokecolor{currentstroke}%
\pgfsetdash{}{0pt}%
\pgfpathmoveto{\pgfqpoint{3.903983in}{1.138040in}}%
\pgfpathcurveto{\pgfqpoint{3.909807in}{1.138040in}}{\pgfqpoint{3.915393in}{1.140354in}}{\pgfqpoint{3.919511in}{1.144472in}}%
\pgfpathcurveto{\pgfqpoint{3.923629in}{1.148590in}}{\pgfqpoint{3.925943in}{1.154176in}}{\pgfqpoint{3.925943in}{1.160000in}}%
\pgfpathcurveto{\pgfqpoint{3.925943in}{1.165824in}}{\pgfqpoint{3.923629in}{1.171410in}}{\pgfqpoint{3.919511in}{1.175528in}}%
\pgfpathcurveto{\pgfqpoint{3.915393in}{1.179646in}}{\pgfqpoint{3.909807in}{1.181960in}}{\pgfqpoint{3.903983in}{1.181960in}}%
\pgfpathcurveto{\pgfqpoint{3.898159in}{1.181960in}}{\pgfqpoint{3.892573in}{1.179646in}}{\pgfqpoint{3.888455in}{1.175528in}}%
\pgfpathcurveto{\pgfqpoint{3.884336in}{1.171410in}}{\pgfqpoint{3.882023in}{1.165824in}}{\pgfqpoint{3.882023in}{1.160000in}}%
\pgfpathcurveto{\pgfqpoint{3.882023in}{1.154176in}}{\pgfqpoint{3.884336in}{1.148590in}}{\pgfqpoint{3.888455in}{1.144472in}}%
\pgfpathcurveto{\pgfqpoint{3.892573in}{1.140354in}}{\pgfqpoint{3.898159in}{1.138040in}}{\pgfqpoint{3.903983in}{1.138040in}}%
\pgfpathlineto{\pgfqpoint{3.903983in}{1.138040in}}%
\pgfpathclose%
\pgfusepath{stroke,fill}%
\end{pgfscope}%
\begin{pgfscope}%
\pgfpathrectangle{\pgfqpoint{1.073501in}{0.880000in}}{\pgfqpoint{6.052998in}{6.160000in}}%
\pgfusepath{clip}%
\pgfsetbuttcap%
\pgfsetroundjoin%
\definecolor{currentfill}{rgb}{0.200000,0.200000,0.800000}%
\pgfsetfillcolor{currentfill}%
\pgfsetlinewidth{1.003750pt}%
\definecolor{currentstroke}{rgb}{0.200000,0.200000,0.800000}%
\pgfsetstrokecolor{currentstroke}%
\pgfsetdash{}{0pt}%
\pgfpathmoveto{\pgfqpoint{4.081989in}{1.176830in}}%
\pgfpathcurveto{\pgfqpoint{4.087813in}{1.176830in}}{\pgfqpoint{4.093399in}{1.179144in}}{\pgfqpoint{4.097517in}{1.183262in}}%
\pgfpathcurveto{\pgfqpoint{4.101635in}{1.187380in}}{\pgfqpoint{4.103949in}{1.192967in}}{\pgfqpoint{4.103949in}{1.198790in}}%
\pgfpathcurveto{\pgfqpoint{4.103949in}{1.204614in}}{\pgfqpoint{4.101635in}{1.210201in}}{\pgfqpoint{4.097517in}{1.214319in}}%
\pgfpathcurveto{\pgfqpoint{4.093399in}{1.218437in}}{\pgfqpoint{4.087813in}{1.220751in}}{\pgfqpoint{4.081989in}{1.220751in}}%
\pgfpathcurveto{\pgfqpoint{4.076165in}{1.220751in}}{\pgfqpoint{4.070578in}{1.218437in}}{\pgfqpoint{4.066460in}{1.214319in}}%
\pgfpathcurveto{\pgfqpoint{4.062342in}{1.210201in}}{\pgfqpoint{4.060028in}{1.204614in}}{\pgfqpoint{4.060028in}{1.198790in}}%
\pgfpathcurveto{\pgfqpoint{4.060028in}{1.192967in}}{\pgfqpoint{4.062342in}{1.187380in}}{\pgfqpoint{4.066460in}{1.183262in}}%
\pgfpathcurveto{\pgfqpoint{4.070578in}{1.179144in}}{\pgfqpoint{4.076165in}{1.176830in}}{\pgfqpoint{4.081989in}{1.176830in}}%
\pgfpathlineto{\pgfqpoint{4.081989in}{1.176830in}}%
\pgfpathclose%
\pgfusepath{stroke,fill}%
\end{pgfscope}%
\begin{pgfscope}%
\pgfpathrectangle{\pgfqpoint{1.073501in}{0.880000in}}{\pgfqpoint{6.052998in}{6.160000in}}%
\pgfusepath{clip}%
\pgfsetbuttcap%
\pgfsetroundjoin%
\definecolor{currentfill}{rgb}{0.200000,0.200000,0.800000}%
\pgfsetfillcolor{currentfill}%
\pgfsetlinewidth{1.003750pt}%
\definecolor{currentstroke}{rgb}{0.200000,0.200000,0.800000}%
\pgfsetstrokecolor{currentstroke}%
\pgfsetdash{}{0pt}%
\pgfpathmoveto{\pgfqpoint{4.255018in}{1.209150in}}%
\pgfpathcurveto{\pgfqpoint{4.260842in}{1.209150in}}{\pgfqpoint{4.266429in}{1.211464in}}{\pgfqpoint{4.270547in}{1.215582in}}%
\pgfpathcurveto{\pgfqpoint{4.274665in}{1.219700in}}{\pgfqpoint{4.276979in}{1.225287in}}{\pgfqpoint{4.276979in}{1.231110in}}%
\pgfpathcurveto{\pgfqpoint{4.276979in}{1.236934in}}{\pgfqpoint{4.274665in}{1.242521in}}{\pgfqpoint{4.270547in}{1.246639in}}%
\pgfpathcurveto{\pgfqpoint{4.266429in}{1.250757in}}{\pgfqpoint{4.260842in}{1.253071in}}{\pgfqpoint{4.255018in}{1.253071in}}%
\pgfpathcurveto{\pgfqpoint{4.249195in}{1.253071in}}{\pgfqpoint{4.243608in}{1.250757in}}{\pgfqpoint{4.239490in}{1.246639in}}%
\pgfpathcurveto{\pgfqpoint{4.235372in}{1.242521in}}{\pgfqpoint{4.233058in}{1.236934in}}{\pgfqpoint{4.233058in}{1.231110in}}%
\pgfpathcurveto{\pgfqpoint{4.233058in}{1.225287in}}{\pgfqpoint{4.235372in}{1.219700in}}{\pgfqpoint{4.239490in}{1.215582in}}%
\pgfpathcurveto{\pgfqpoint{4.243608in}{1.211464in}}{\pgfqpoint{4.249195in}{1.209150in}}{\pgfqpoint{4.255018in}{1.209150in}}%
\pgfpathlineto{\pgfqpoint{4.255018in}{1.209150in}}%
\pgfpathclose%
\pgfusepath{stroke,fill}%
\end{pgfscope}%
\begin{pgfscope}%
\pgfpathrectangle{\pgfqpoint{1.073501in}{0.880000in}}{\pgfqpoint{6.052998in}{6.160000in}}%
\pgfusepath{clip}%
\pgfsetbuttcap%
\pgfsetroundjoin%
\definecolor{currentfill}{rgb}{0.200000,0.200000,0.800000}%
\pgfsetfillcolor{currentfill}%
\pgfsetlinewidth{1.003750pt}%
\definecolor{currentstroke}{rgb}{0.200000,0.200000,0.800000}%
\pgfsetstrokecolor{currentstroke}%
\pgfsetdash{}{0pt}%
\pgfpathmoveto{\pgfqpoint{4.425999in}{1.232323in}}%
\pgfpathcurveto{\pgfqpoint{4.431823in}{1.232323in}}{\pgfqpoint{4.437409in}{1.234637in}}{\pgfqpoint{4.441527in}{1.238755in}}%
\pgfpathcurveto{\pgfqpoint{4.445645in}{1.242873in}}{\pgfqpoint{4.447959in}{1.248459in}}{\pgfqpoint{4.447959in}{1.254283in}}%
\pgfpathcurveto{\pgfqpoint{4.447959in}{1.260107in}}{\pgfqpoint{4.445645in}{1.265693in}}{\pgfqpoint{4.441527in}{1.269811in}}%
\pgfpathcurveto{\pgfqpoint{4.437409in}{1.273929in}}{\pgfqpoint{4.431823in}{1.276243in}}{\pgfqpoint{4.425999in}{1.276243in}}%
\pgfpathcurveto{\pgfqpoint{4.420175in}{1.276243in}}{\pgfqpoint{4.414589in}{1.273929in}}{\pgfqpoint{4.410471in}{1.269811in}}%
\pgfpathcurveto{\pgfqpoint{4.406352in}{1.265693in}}{\pgfqpoint{4.404039in}{1.260107in}}{\pgfqpoint{4.404039in}{1.254283in}}%
\pgfpathcurveto{\pgfqpoint{4.404039in}{1.248459in}}{\pgfqpoint{4.406352in}{1.242873in}}{\pgfqpoint{4.410471in}{1.238755in}}%
\pgfpathcurveto{\pgfqpoint{4.414589in}{1.234637in}}{\pgfqpoint{4.420175in}{1.232323in}}{\pgfqpoint{4.425999in}{1.232323in}}%
\pgfpathlineto{\pgfqpoint{4.425999in}{1.232323in}}%
\pgfpathclose%
\pgfusepath{stroke,fill}%
\end{pgfscope}%
\begin{pgfscope}%
\pgfpathrectangle{\pgfqpoint{1.073501in}{0.880000in}}{\pgfqpoint{6.052998in}{6.160000in}}%
\pgfusepath{clip}%
\pgfsetbuttcap%
\pgfsetroundjoin%
\definecolor{currentfill}{rgb}{0.200000,0.200000,0.800000}%
\pgfsetfillcolor{currentfill}%
\pgfsetlinewidth{1.003750pt}%
\definecolor{currentstroke}{rgb}{0.200000,0.200000,0.800000}%
\pgfsetstrokecolor{currentstroke}%
\pgfsetdash{}{0pt}%
\pgfpathmoveto{\pgfqpoint{4.601155in}{1.228834in}}%
\pgfpathcurveto{\pgfqpoint{4.606979in}{1.228834in}}{\pgfqpoint{4.612565in}{1.231148in}}{\pgfqpoint{4.616683in}{1.235266in}}%
\pgfpathcurveto{\pgfqpoint{4.620801in}{1.239384in}}{\pgfqpoint{4.623115in}{1.244971in}}{\pgfqpoint{4.623115in}{1.250795in}}%
\pgfpathcurveto{\pgfqpoint{4.623115in}{1.256618in}}{\pgfqpoint{4.620801in}{1.262205in}}{\pgfqpoint{4.616683in}{1.266323in}}%
\pgfpathcurveto{\pgfqpoint{4.612565in}{1.270441in}}{\pgfqpoint{4.606979in}{1.272755in}}{\pgfqpoint{4.601155in}{1.272755in}}%
\pgfpathcurveto{\pgfqpoint{4.595331in}{1.272755in}}{\pgfqpoint{4.589745in}{1.270441in}}{\pgfqpoint{4.585627in}{1.266323in}}%
\pgfpathcurveto{\pgfqpoint{4.581509in}{1.262205in}}{\pgfqpoint{4.579195in}{1.256618in}}{\pgfqpoint{4.579195in}{1.250795in}}%
\pgfpathcurveto{\pgfqpoint{4.579195in}{1.244971in}}{\pgfqpoint{4.581509in}{1.239384in}}{\pgfqpoint{4.585627in}{1.235266in}}%
\pgfpathcurveto{\pgfqpoint{4.589745in}{1.231148in}}{\pgfqpoint{4.595331in}{1.228834in}}{\pgfqpoint{4.601155in}{1.228834in}}%
\pgfpathlineto{\pgfqpoint{4.601155in}{1.228834in}}%
\pgfpathclose%
\pgfusepath{stroke,fill}%
\end{pgfscope}%
\begin{pgfscope}%
\pgfpathrectangle{\pgfqpoint{1.073501in}{0.880000in}}{\pgfqpoint{6.052998in}{6.160000in}}%
\pgfusepath{clip}%
\pgfsetbuttcap%
\pgfsetroundjoin%
\definecolor{currentfill}{rgb}{0.200000,0.200000,0.800000}%
\pgfsetfillcolor{currentfill}%
\pgfsetlinewidth{1.003750pt}%
\definecolor{currentstroke}{rgb}{0.200000,0.200000,0.800000}%
\pgfsetstrokecolor{currentstroke}%
\pgfsetdash{}{0pt}%
\pgfpathmoveto{\pgfqpoint{4.732277in}{1.425069in}}%
\pgfpathcurveto{\pgfqpoint{4.738101in}{1.425069in}}{\pgfqpoint{4.743687in}{1.427383in}}{\pgfqpoint{4.747805in}{1.431501in}}%
\pgfpathcurveto{\pgfqpoint{4.751924in}{1.435619in}}{\pgfqpoint{4.754237in}{1.441205in}}{\pgfqpoint{4.754237in}{1.447029in}}%
\pgfpathcurveto{\pgfqpoint{4.754237in}{1.452853in}}{\pgfqpoint{4.751924in}{1.458439in}}{\pgfqpoint{4.747805in}{1.462558in}}%
\pgfpathcurveto{\pgfqpoint{4.743687in}{1.466676in}}{\pgfqpoint{4.738101in}{1.468990in}}{\pgfqpoint{4.732277in}{1.468990in}}%
\pgfpathcurveto{\pgfqpoint{4.726453in}{1.468990in}}{\pgfqpoint{4.720867in}{1.466676in}}{\pgfqpoint{4.716749in}{1.462558in}}%
\pgfpathcurveto{\pgfqpoint{4.712631in}{1.458439in}}{\pgfqpoint{4.710317in}{1.452853in}}{\pgfqpoint{4.710317in}{1.447029in}}%
\pgfpathcurveto{\pgfqpoint{4.710317in}{1.441205in}}{\pgfqpoint{4.712631in}{1.435619in}}{\pgfqpoint{4.716749in}{1.431501in}}%
\pgfpathcurveto{\pgfqpoint{4.720867in}{1.427383in}}{\pgfqpoint{4.726453in}{1.425069in}}{\pgfqpoint{4.732277in}{1.425069in}}%
\pgfpathlineto{\pgfqpoint{4.732277in}{1.425069in}}%
\pgfpathclose%
\pgfusepath{stroke,fill}%
\end{pgfscope}%
\begin{pgfscope}%
\pgfpathrectangle{\pgfqpoint{1.073501in}{0.880000in}}{\pgfqpoint{6.052998in}{6.160000in}}%
\pgfusepath{clip}%
\pgfsetbuttcap%
\pgfsetroundjoin%
\definecolor{currentfill}{rgb}{0.200000,0.200000,0.800000}%
\pgfsetfillcolor{currentfill}%
\pgfsetlinewidth{1.003750pt}%
\definecolor{currentstroke}{rgb}{0.200000,0.200000,0.800000}%
\pgfsetstrokecolor{currentstroke}%
\pgfsetdash{}{0pt}%
\pgfpathmoveto{\pgfqpoint{4.923681in}{1.359172in}}%
\pgfpathcurveto{\pgfqpoint{4.929505in}{1.359172in}}{\pgfqpoint{4.935091in}{1.361486in}}{\pgfqpoint{4.939209in}{1.365604in}}%
\pgfpathcurveto{\pgfqpoint{4.943328in}{1.369723in}}{\pgfqpoint{4.945641in}{1.375309in}}{\pgfqpoint{4.945641in}{1.381133in}}%
\pgfpathcurveto{\pgfqpoint{4.945641in}{1.386957in}}{\pgfqpoint{4.943328in}{1.392543in}}{\pgfqpoint{4.939209in}{1.396661in}}%
\pgfpathcurveto{\pgfqpoint{4.935091in}{1.400779in}}{\pgfqpoint{4.929505in}{1.403093in}}{\pgfqpoint{4.923681in}{1.403093in}}%
\pgfpathcurveto{\pgfqpoint{4.917857in}{1.403093in}}{\pgfqpoint{4.912271in}{1.400779in}}{\pgfqpoint{4.908153in}{1.396661in}}%
\pgfpathcurveto{\pgfqpoint{4.904035in}{1.392543in}}{\pgfqpoint{4.901721in}{1.386957in}}{\pgfqpoint{4.901721in}{1.381133in}}%
\pgfpathcurveto{\pgfqpoint{4.901721in}{1.375309in}}{\pgfqpoint{4.904035in}{1.369723in}}{\pgfqpoint{4.908153in}{1.365604in}}%
\pgfpathcurveto{\pgfqpoint{4.912271in}{1.361486in}}{\pgfqpoint{4.917857in}{1.359172in}}{\pgfqpoint{4.923681in}{1.359172in}}%
\pgfpathlineto{\pgfqpoint{4.923681in}{1.359172in}}%
\pgfpathclose%
\pgfusepath{stroke,fill}%
\end{pgfscope}%
\begin{pgfscope}%
\pgfpathrectangle{\pgfqpoint{1.073501in}{0.880000in}}{\pgfqpoint{6.052998in}{6.160000in}}%
\pgfusepath{clip}%
\pgfsetbuttcap%
\pgfsetroundjoin%
\definecolor{currentfill}{rgb}{0.200000,0.200000,0.800000}%
\pgfsetfillcolor{currentfill}%
\pgfsetlinewidth{1.003750pt}%
\definecolor{currentstroke}{rgb}{0.200000,0.200000,0.800000}%
\pgfsetstrokecolor{currentstroke}%
\pgfsetdash{}{0pt}%
\pgfpathmoveto{\pgfqpoint{5.116521in}{1.332083in}}%
\pgfpathcurveto{\pgfqpoint{5.122345in}{1.332083in}}{\pgfqpoint{5.127931in}{1.334397in}}{\pgfqpoint{5.132049in}{1.338515in}}%
\pgfpathcurveto{\pgfqpoint{5.136167in}{1.342633in}}{\pgfqpoint{5.138481in}{1.348219in}}{\pgfqpoint{5.138481in}{1.354043in}}%
\pgfpathcurveto{\pgfqpoint{5.138481in}{1.359867in}}{\pgfqpoint{5.136167in}{1.365453in}}{\pgfqpoint{5.132049in}{1.369571in}}%
\pgfpathcurveto{\pgfqpoint{5.127931in}{1.373689in}}{\pgfqpoint{5.122345in}{1.376003in}}{\pgfqpoint{5.116521in}{1.376003in}}%
\pgfpathcurveto{\pgfqpoint{5.110697in}{1.376003in}}{\pgfqpoint{5.105111in}{1.373689in}}{\pgfqpoint{5.100993in}{1.369571in}}%
\pgfpathcurveto{\pgfqpoint{5.096874in}{1.365453in}}{\pgfqpoint{5.094561in}{1.359867in}}{\pgfqpoint{5.094561in}{1.354043in}}%
\pgfpathcurveto{\pgfqpoint{5.094561in}{1.348219in}}{\pgfqpoint{5.096874in}{1.342633in}}{\pgfqpoint{5.100993in}{1.338515in}}%
\pgfpathcurveto{\pgfqpoint{5.105111in}{1.334397in}}{\pgfqpoint{5.110697in}{1.332083in}}{\pgfqpoint{5.116521in}{1.332083in}}%
\pgfpathlineto{\pgfqpoint{5.116521in}{1.332083in}}%
\pgfpathclose%
\pgfusepath{stroke,fill}%
\end{pgfscope}%
\begin{pgfscope}%
\pgfpathrectangle{\pgfqpoint{1.073501in}{0.880000in}}{\pgfqpoint{6.052998in}{6.160000in}}%
\pgfusepath{clip}%
\pgfsetbuttcap%
\pgfsetroundjoin%
\definecolor{currentfill}{rgb}{0.200000,0.200000,0.800000}%
\pgfsetfillcolor{currentfill}%
\pgfsetlinewidth{1.003750pt}%
\definecolor{currentstroke}{rgb}{0.200000,0.200000,0.800000}%
\pgfsetstrokecolor{currentstroke}%
\pgfsetdash{}{0pt}%
\pgfpathmoveto{\pgfqpoint{5.247009in}{1.470265in}}%
\pgfpathcurveto{\pgfqpoint{5.252833in}{1.470265in}}{\pgfqpoint{5.258419in}{1.472579in}}{\pgfqpoint{5.262538in}{1.476697in}}%
\pgfpathcurveto{\pgfqpoint{5.266656in}{1.480815in}}{\pgfqpoint{5.268970in}{1.486402in}}{\pgfqpoint{5.268970in}{1.492226in}}%
\pgfpathcurveto{\pgfqpoint{5.268970in}{1.498049in}}{\pgfqpoint{5.266656in}{1.503636in}}{\pgfqpoint{5.262538in}{1.507754in}}%
\pgfpathcurveto{\pgfqpoint{5.258419in}{1.511872in}}{\pgfqpoint{5.252833in}{1.514186in}}{\pgfqpoint{5.247009in}{1.514186in}}%
\pgfpathcurveto{\pgfqpoint{5.241185in}{1.514186in}}{\pgfqpoint{5.235599in}{1.511872in}}{\pgfqpoint{5.231481in}{1.507754in}}%
\pgfpathcurveto{\pgfqpoint{5.227363in}{1.503636in}}{\pgfqpoint{5.225049in}{1.498049in}}{\pgfqpoint{5.225049in}{1.492226in}}%
\pgfpathcurveto{\pgfqpoint{5.225049in}{1.486402in}}{\pgfqpoint{5.227363in}{1.480815in}}{\pgfqpoint{5.231481in}{1.476697in}}%
\pgfpathcurveto{\pgfqpoint{5.235599in}{1.472579in}}{\pgfqpoint{5.241185in}{1.470265in}}{\pgfqpoint{5.247009in}{1.470265in}}%
\pgfpathlineto{\pgfqpoint{5.247009in}{1.470265in}}%
\pgfpathclose%
\pgfusepath{stroke,fill}%
\end{pgfscope}%
\begin{pgfscope}%
\pgfpathrectangle{\pgfqpoint{1.073501in}{0.880000in}}{\pgfqpoint{6.052998in}{6.160000in}}%
\pgfusepath{clip}%
\pgfsetbuttcap%
\pgfsetroundjoin%
\definecolor{currentfill}{rgb}{0.200000,0.200000,0.800000}%
\pgfsetfillcolor{currentfill}%
\pgfsetlinewidth{1.003750pt}%
\definecolor{currentstroke}{rgb}{0.200000,0.200000,0.800000}%
\pgfsetstrokecolor{currentstroke}%
\pgfsetdash{}{0pt}%
\pgfpathmoveto{\pgfqpoint{5.395627in}{1.555411in}}%
\pgfpathcurveto{\pgfqpoint{5.401450in}{1.555411in}}{\pgfqpoint{5.407037in}{1.557725in}}{\pgfqpoint{5.411155in}{1.561843in}}%
\pgfpathcurveto{\pgfqpoint{5.415273in}{1.565961in}}{\pgfqpoint{5.417587in}{1.571547in}}{\pgfqpoint{5.417587in}{1.577371in}}%
\pgfpathcurveto{\pgfqpoint{5.417587in}{1.583195in}}{\pgfqpoint{5.415273in}{1.588781in}}{\pgfqpoint{5.411155in}{1.592899in}}%
\pgfpathcurveto{\pgfqpoint{5.407037in}{1.597018in}}{\pgfqpoint{5.401450in}{1.599331in}}{\pgfqpoint{5.395627in}{1.599331in}}%
\pgfpathcurveto{\pgfqpoint{5.389803in}{1.599331in}}{\pgfqpoint{5.384216in}{1.597018in}}{\pgfqpoint{5.380098in}{1.592899in}}%
\pgfpathcurveto{\pgfqpoint{5.375980in}{1.588781in}}{\pgfqpoint{5.373666in}{1.583195in}}{\pgfqpoint{5.373666in}{1.577371in}}%
\pgfpathcurveto{\pgfqpoint{5.373666in}{1.571547in}}{\pgfqpoint{5.375980in}{1.565961in}}{\pgfqpoint{5.380098in}{1.561843in}}%
\pgfpathcurveto{\pgfqpoint{5.384216in}{1.557725in}}{\pgfqpoint{5.389803in}{1.555411in}}{\pgfqpoint{5.395627in}{1.555411in}}%
\pgfpathlineto{\pgfqpoint{5.395627in}{1.555411in}}%
\pgfpathclose%
\pgfusepath{stroke,fill}%
\end{pgfscope}%
\begin{pgfscope}%
\pgfpathrectangle{\pgfqpoint{1.073501in}{0.880000in}}{\pgfqpoint{6.052998in}{6.160000in}}%
\pgfusepath{clip}%
\pgfsetbuttcap%
\pgfsetroundjoin%
\definecolor{currentfill}{rgb}{0.200000,0.200000,0.800000}%
\pgfsetfillcolor{currentfill}%
\pgfsetlinewidth{1.003750pt}%
\definecolor{currentstroke}{rgb}{0.200000,0.200000,0.800000}%
\pgfsetstrokecolor{currentstroke}%
\pgfsetdash{}{0pt}%
\pgfpathmoveto{\pgfqpoint{5.611368in}{1.531203in}}%
\pgfpathcurveto{\pgfqpoint{5.617192in}{1.531203in}}{\pgfqpoint{5.622778in}{1.533517in}}{\pgfqpoint{5.626896in}{1.537635in}}%
\pgfpathcurveto{\pgfqpoint{5.631014in}{1.541753in}}{\pgfqpoint{5.633328in}{1.547339in}}{\pgfqpoint{5.633328in}{1.553163in}}%
\pgfpathcurveto{\pgfqpoint{5.633328in}{1.558987in}}{\pgfqpoint{5.631014in}{1.564573in}}{\pgfqpoint{5.626896in}{1.568692in}}%
\pgfpathcurveto{\pgfqpoint{5.622778in}{1.572810in}}{\pgfqpoint{5.617192in}{1.575124in}}{\pgfqpoint{5.611368in}{1.575124in}}%
\pgfpathcurveto{\pgfqpoint{5.605544in}{1.575124in}}{\pgfqpoint{5.599958in}{1.572810in}}{\pgfqpoint{5.595839in}{1.568692in}}%
\pgfpathcurveto{\pgfqpoint{5.591721in}{1.564573in}}{\pgfqpoint{5.589407in}{1.558987in}}{\pgfqpoint{5.589407in}{1.553163in}}%
\pgfpathcurveto{\pgfqpoint{5.589407in}{1.547339in}}{\pgfqpoint{5.591721in}{1.541753in}}{\pgfqpoint{5.595839in}{1.537635in}}%
\pgfpathcurveto{\pgfqpoint{5.599958in}{1.533517in}}{\pgfqpoint{5.605544in}{1.531203in}}{\pgfqpoint{5.611368in}{1.531203in}}%
\pgfpathlineto{\pgfqpoint{5.611368in}{1.531203in}}%
\pgfpathclose%
\pgfusepath{stroke,fill}%
\end{pgfscope}%
\begin{pgfscope}%
\pgfpathrectangle{\pgfqpoint{1.073501in}{0.880000in}}{\pgfqpoint{6.052998in}{6.160000in}}%
\pgfusepath{clip}%
\pgfsetbuttcap%
\pgfsetroundjoin%
\definecolor{currentfill}{rgb}{0.200000,0.200000,0.800000}%
\pgfsetfillcolor{currentfill}%
\pgfsetlinewidth{1.003750pt}%
\definecolor{currentstroke}{rgb}{0.200000,0.200000,0.800000}%
\pgfsetstrokecolor{currentstroke}%
\pgfsetdash{}{0pt}%
\pgfpathmoveto{\pgfqpoint{5.691571in}{1.726703in}}%
\pgfpathcurveto{\pgfqpoint{5.697395in}{1.726703in}}{\pgfqpoint{5.702981in}{1.729017in}}{\pgfqpoint{5.707099in}{1.733135in}}%
\pgfpathcurveto{\pgfqpoint{5.711217in}{1.737254in}}{\pgfqpoint{5.713531in}{1.742840in}}{\pgfqpoint{5.713531in}{1.748664in}}%
\pgfpathcurveto{\pgfqpoint{5.713531in}{1.754488in}}{\pgfqpoint{5.711217in}{1.760074in}}{\pgfqpoint{5.707099in}{1.764192in}}%
\pgfpathcurveto{\pgfqpoint{5.702981in}{1.768310in}}{\pgfqpoint{5.697395in}{1.770624in}}{\pgfqpoint{5.691571in}{1.770624in}}%
\pgfpathcurveto{\pgfqpoint{5.685747in}{1.770624in}}{\pgfqpoint{5.680161in}{1.768310in}}{\pgfqpoint{5.676043in}{1.764192in}}%
\pgfpathcurveto{\pgfqpoint{5.671925in}{1.760074in}}{\pgfqpoint{5.669611in}{1.754488in}}{\pgfqpoint{5.669611in}{1.748664in}}%
\pgfpathcurveto{\pgfqpoint{5.669611in}{1.742840in}}{\pgfqpoint{5.671925in}{1.737254in}}{\pgfqpoint{5.676043in}{1.733135in}}%
\pgfpathcurveto{\pgfqpoint{5.680161in}{1.729017in}}{\pgfqpoint{5.685747in}{1.726703in}}{\pgfqpoint{5.691571in}{1.726703in}}%
\pgfpathlineto{\pgfqpoint{5.691571in}{1.726703in}}%
\pgfpathclose%
\pgfusepath{stroke,fill}%
\end{pgfscope}%
\begin{pgfscope}%
\pgfpathrectangle{\pgfqpoint{1.073501in}{0.880000in}}{\pgfqpoint{6.052998in}{6.160000in}}%
\pgfusepath{clip}%
\pgfsetbuttcap%
\pgfsetroundjoin%
\definecolor{currentfill}{rgb}{0.200000,0.200000,0.800000}%
\pgfsetfillcolor{currentfill}%
\pgfsetlinewidth{1.003750pt}%
\definecolor{currentstroke}{rgb}{0.200000,0.200000,0.800000}%
\pgfsetstrokecolor{currentstroke}%
\pgfsetdash{}{0pt}%
\pgfpathmoveto{\pgfqpoint{5.880475in}{1.765727in}}%
\pgfpathcurveto{\pgfqpoint{5.886299in}{1.765727in}}{\pgfqpoint{5.891885in}{1.768041in}}{\pgfqpoint{5.896003in}{1.772160in}}%
\pgfpathcurveto{\pgfqpoint{5.900121in}{1.776278in}}{\pgfqpoint{5.902435in}{1.781864in}}{\pgfqpoint{5.902435in}{1.787688in}}%
\pgfpathcurveto{\pgfqpoint{5.902435in}{1.793512in}}{\pgfqpoint{5.900121in}{1.799098in}}{\pgfqpoint{5.896003in}{1.803216in}}%
\pgfpathcurveto{\pgfqpoint{5.891885in}{1.807334in}}{\pgfqpoint{5.886299in}{1.809648in}}{\pgfqpoint{5.880475in}{1.809648in}}%
\pgfpathcurveto{\pgfqpoint{5.874651in}{1.809648in}}{\pgfqpoint{5.869065in}{1.807334in}}{\pgfqpoint{5.864947in}{1.803216in}}%
\pgfpathcurveto{\pgfqpoint{5.860828in}{1.799098in}}{\pgfqpoint{5.858515in}{1.793512in}}{\pgfqpoint{5.858515in}{1.787688in}}%
\pgfpathcurveto{\pgfqpoint{5.858515in}{1.781864in}}{\pgfqpoint{5.860828in}{1.776278in}}{\pgfqpoint{5.864947in}{1.772160in}}%
\pgfpathcurveto{\pgfqpoint{5.869065in}{1.768041in}}{\pgfqpoint{5.874651in}{1.765727in}}{\pgfqpoint{5.880475in}{1.765727in}}%
\pgfpathlineto{\pgfqpoint{5.880475in}{1.765727in}}%
\pgfpathclose%
\pgfusepath{stroke,fill}%
\end{pgfscope}%
\begin{pgfscope}%
\pgfpathrectangle{\pgfqpoint{1.073501in}{0.880000in}}{\pgfqpoint{6.052998in}{6.160000in}}%
\pgfusepath{clip}%
\pgfsetbuttcap%
\pgfsetroundjoin%
\definecolor{currentfill}{rgb}{0.200000,0.200000,0.800000}%
\pgfsetfillcolor{currentfill}%
\pgfsetlinewidth{1.003750pt}%
\definecolor{currentstroke}{rgb}{0.200000,0.200000,0.800000}%
\pgfsetstrokecolor{currentstroke}%
\pgfsetdash{}{0pt}%
\pgfpathmoveto{\pgfqpoint{5.982754in}{1.915142in}}%
\pgfpathcurveto{\pgfqpoint{5.988578in}{1.915142in}}{\pgfqpoint{5.994164in}{1.917456in}}{\pgfqpoint{5.998282in}{1.921574in}}%
\pgfpathcurveto{\pgfqpoint{6.002400in}{1.925692in}}{\pgfqpoint{6.004714in}{1.931278in}}{\pgfqpoint{6.004714in}{1.937102in}}%
\pgfpathcurveto{\pgfqpoint{6.004714in}{1.942926in}}{\pgfqpoint{6.002400in}{1.948512in}}{\pgfqpoint{5.998282in}{1.952630in}}%
\pgfpathcurveto{\pgfqpoint{5.994164in}{1.956749in}}{\pgfqpoint{5.988578in}{1.959062in}}{\pgfqpoint{5.982754in}{1.959062in}}%
\pgfpathcurveto{\pgfqpoint{5.976930in}{1.959062in}}{\pgfqpoint{5.971344in}{1.956749in}}{\pgfqpoint{5.967226in}{1.952630in}}%
\pgfpathcurveto{\pgfqpoint{5.963108in}{1.948512in}}{\pgfqpoint{5.960794in}{1.942926in}}{\pgfqpoint{5.960794in}{1.937102in}}%
\pgfpathcurveto{\pgfqpoint{5.960794in}{1.931278in}}{\pgfqpoint{5.963108in}{1.925692in}}{\pgfqpoint{5.967226in}{1.921574in}}%
\pgfpathcurveto{\pgfqpoint{5.971344in}{1.917456in}}{\pgfqpoint{5.976930in}{1.915142in}}{\pgfqpoint{5.982754in}{1.915142in}}%
\pgfpathlineto{\pgfqpoint{5.982754in}{1.915142in}}%
\pgfpathclose%
\pgfusepath{stroke,fill}%
\end{pgfscope}%
\begin{pgfscope}%
\pgfpathrectangle{\pgfqpoint{1.073501in}{0.880000in}}{\pgfqpoint{6.052998in}{6.160000in}}%
\pgfusepath{clip}%
\pgfsetbuttcap%
\pgfsetroundjoin%
\definecolor{currentfill}{rgb}{0.200000,0.200000,0.800000}%
\pgfsetfillcolor{currentfill}%
\pgfsetlinewidth{1.003750pt}%
\definecolor{currentstroke}{rgb}{0.200000,0.200000,0.800000}%
\pgfsetstrokecolor{currentstroke}%
\pgfsetdash{}{0pt}%
\pgfpathmoveto{\pgfqpoint{6.092742in}{2.050132in}}%
\pgfpathcurveto{\pgfqpoint{6.098566in}{2.050132in}}{\pgfqpoint{6.104153in}{2.052445in}}{\pgfqpoint{6.108271in}{2.056564in}}%
\pgfpathcurveto{\pgfqpoint{6.112389in}{2.060682in}}{\pgfqpoint{6.114703in}{2.066268in}}{\pgfqpoint{6.114703in}{2.072092in}}%
\pgfpathcurveto{\pgfqpoint{6.114703in}{2.077916in}}{\pgfqpoint{6.112389in}{2.083502in}}{\pgfqpoint{6.108271in}{2.087620in}}%
\pgfpathcurveto{\pgfqpoint{6.104153in}{2.091738in}}{\pgfqpoint{6.098566in}{2.094052in}}{\pgfqpoint{6.092742in}{2.094052in}}%
\pgfpathcurveto{\pgfqpoint{6.086919in}{2.094052in}}{\pgfqpoint{6.081332in}{2.091738in}}{\pgfqpoint{6.077214in}{2.087620in}}%
\pgfpathcurveto{\pgfqpoint{6.073096in}{2.083502in}}{\pgfqpoint{6.070782in}{2.077916in}}{\pgfqpoint{6.070782in}{2.072092in}}%
\pgfpathcurveto{\pgfqpoint{6.070782in}{2.066268in}}{\pgfqpoint{6.073096in}{2.060682in}}{\pgfqpoint{6.077214in}{2.056564in}}%
\pgfpathcurveto{\pgfqpoint{6.081332in}{2.052445in}}{\pgfqpoint{6.086919in}{2.050132in}}{\pgfqpoint{6.092742in}{2.050132in}}%
\pgfpathlineto{\pgfqpoint{6.092742in}{2.050132in}}%
\pgfpathclose%
\pgfusepath{stroke,fill}%
\end{pgfscope}%
\begin{pgfscope}%
\pgfpathrectangle{\pgfqpoint{1.073501in}{0.880000in}}{\pgfqpoint{6.052998in}{6.160000in}}%
\pgfusepath{clip}%
\pgfsetbuttcap%
\pgfsetroundjoin%
\definecolor{currentfill}{rgb}{0.200000,0.200000,0.800000}%
\pgfsetfillcolor{currentfill}%
\pgfsetlinewidth{1.003750pt}%
\definecolor{currentstroke}{rgb}{0.200000,0.200000,0.800000}%
\pgfsetstrokecolor{currentstroke}%
\pgfsetdash{}{0pt}%
\pgfpathmoveto{\pgfqpoint{6.128651in}{2.245037in}}%
\pgfpathcurveto{\pgfqpoint{6.134475in}{2.245037in}}{\pgfqpoint{6.140061in}{2.247351in}}{\pgfqpoint{6.144180in}{2.251469in}}%
\pgfpathcurveto{\pgfqpoint{6.148298in}{2.255588in}}{\pgfqpoint{6.150612in}{2.261174in}}{\pgfqpoint{6.150612in}{2.266998in}}%
\pgfpathcurveto{\pgfqpoint{6.150612in}{2.272822in}}{\pgfqpoint{6.148298in}{2.278408in}}{\pgfqpoint{6.144180in}{2.282526in}}%
\pgfpathcurveto{\pgfqpoint{6.140061in}{2.286644in}}{\pgfqpoint{6.134475in}{2.288958in}}{\pgfqpoint{6.128651in}{2.288958in}}%
\pgfpathcurveto{\pgfqpoint{6.122827in}{2.288958in}}{\pgfqpoint{6.117241in}{2.286644in}}{\pgfqpoint{6.113123in}{2.282526in}}%
\pgfpathcurveto{\pgfqpoint{6.109005in}{2.278408in}}{\pgfqpoint{6.106691in}{2.272822in}}{\pgfqpoint{6.106691in}{2.266998in}}%
\pgfpathcurveto{\pgfqpoint{6.106691in}{2.261174in}}{\pgfqpoint{6.109005in}{2.255588in}}{\pgfqpoint{6.113123in}{2.251469in}}%
\pgfpathcurveto{\pgfqpoint{6.117241in}{2.247351in}}{\pgfqpoint{6.122827in}{2.245037in}}{\pgfqpoint{6.128651in}{2.245037in}}%
\pgfpathlineto{\pgfqpoint{6.128651in}{2.245037in}}%
\pgfpathclose%
\pgfusepath{stroke,fill}%
\end{pgfscope}%
\begin{pgfscope}%
\pgfpathrectangle{\pgfqpoint{1.073501in}{0.880000in}}{\pgfqpoint{6.052998in}{6.160000in}}%
\pgfusepath{clip}%
\pgfsetbuttcap%
\pgfsetroundjoin%
\definecolor{currentfill}{rgb}{0.200000,0.200000,0.800000}%
\pgfsetfillcolor{currentfill}%
\pgfsetlinewidth{1.003750pt}%
\definecolor{currentstroke}{rgb}{0.200000,0.200000,0.800000}%
\pgfsetstrokecolor{currentstroke}%
\pgfsetdash{}{0pt}%
\pgfpathmoveto{\pgfqpoint{6.314467in}{2.314167in}}%
\pgfpathcurveto{\pgfqpoint{6.320291in}{2.314167in}}{\pgfqpoint{6.325877in}{2.316481in}}{\pgfqpoint{6.329995in}{2.320599in}}%
\pgfpathcurveto{\pgfqpoint{6.334113in}{2.324718in}}{\pgfqpoint{6.336427in}{2.330304in}}{\pgfqpoint{6.336427in}{2.336128in}}%
\pgfpathcurveto{\pgfqpoint{6.336427in}{2.341952in}}{\pgfqpoint{6.334113in}{2.347538in}}{\pgfqpoint{6.329995in}{2.351656in}}%
\pgfpathcurveto{\pgfqpoint{6.325877in}{2.355774in}}{\pgfqpoint{6.320291in}{2.358088in}}{\pgfqpoint{6.314467in}{2.358088in}}%
\pgfpathcurveto{\pgfqpoint{6.308643in}{2.358088in}}{\pgfqpoint{6.303057in}{2.355774in}}{\pgfqpoint{6.298939in}{2.351656in}}%
\pgfpathcurveto{\pgfqpoint{6.294821in}{2.347538in}}{\pgfqpoint{6.292507in}{2.341952in}}{\pgfqpoint{6.292507in}{2.336128in}}%
\pgfpathcurveto{\pgfqpoint{6.292507in}{2.330304in}}{\pgfqpoint{6.294821in}{2.324718in}}{\pgfqpoint{6.298939in}{2.320599in}}%
\pgfpathcurveto{\pgfqpoint{6.303057in}{2.316481in}}{\pgfqpoint{6.308643in}{2.314167in}}{\pgfqpoint{6.314467in}{2.314167in}}%
\pgfpathlineto{\pgfqpoint{6.314467in}{2.314167in}}%
\pgfpathclose%
\pgfusepath{stroke,fill}%
\end{pgfscope}%
\begin{pgfscope}%
\pgfpathrectangle{\pgfqpoint{1.073501in}{0.880000in}}{\pgfqpoint{6.052998in}{6.160000in}}%
\pgfusepath{clip}%
\pgfsetbuttcap%
\pgfsetroundjoin%
\definecolor{currentfill}{rgb}{0.200000,0.200000,0.800000}%
\pgfsetfillcolor{currentfill}%
\pgfsetlinewidth{1.003750pt}%
\definecolor{currentstroke}{rgb}{0.200000,0.200000,0.800000}%
\pgfsetstrokecolor{currentstroke}%
\pgfsetdash{}{0pt}%
\pgfpathmoveto{\pgfqpoint{6.482319in}{2.411224in}}%
\pgfpathcurveto{\pgfqpoint{6.488143in}{2.411224in}}{\pgfqpoint{6.493729in}{2.413538in}}{\pgfqpoint{6.497847in}{2.417656in}}%
\pgfpathcurveto{\pgfqpoint{6.501965in}{2.421774in}}{\pgfqpoint{6.504279in}{2.427361in}}{\pgfqpoint{6.504279in}{2.433184in}}%
\pgfpathcurveto{\pgfqpoint{6.504279in}{2.439008in}}{\pgfqpoint{6.501965in}{2.444595in}}{\pgfqpoint{6.497847in}{2.448713in}}%
\pgfpathcurveto{\pgfqpoint{6.493729in}{2.452831in}}{\pgfqpoint{6.488143in}{2.455145in}}{\pgfqpoint{6.482319in}{2.455145in}}%
\pgfpathcurveto{\pgfqpoint{6.476495in}{2.455145in}}{\pgfqpoint{6.470909in}{2.452831in}}{\pgfqpoint{6.466790in}{2.448713in}}%
\pgfpathcurveto{\pgfqpoint{6.462672in}{2.444595in}}{\pgfqpoint{6.460358in}{2.439008in}}{\pgfqpoint{6.460358in}{2.433184in}}%
\pgfpathcurveto{\pgfqpoint{6.460358in}{2.427361in}}{\pgfqpoint{6.462672in}{2.421774in}}{\pgfqpoint{6.466790in}{2.417656in}}%
\pgfpathcurveto{\pgfqpoint{6.470909in}{2.413538in}}{\pgfqpoint{6.476495in}{2.411224in}}{\pgfqpoint{6.482319in}{2.411224in}}%
\pgfpathlineto{\pgfqpoint{6.482319in}{2.411224in}}%
\pgfpathclose%
\pgfusepath{stroke,fill}%
\end{pgfscope}%
\begin{pgfscope}%
\pgfpathrectangle{\pgfqpoint{1.073501in}{0.880000in}}{\pgfqpoint{6.052998in}{6.160000in}}%
\pgfusepath{clip}%
\pgfsetbuttcap%
\pgfsetroundjoin%
\definecolor{currentfill}{rgb}{0.200000,0.200000,0.800000}%
\pgfsetfillcolor{currentfill}%
\pgfsetlinewidth{1.003750pt}%
\definecolor{currentstroke}{rgb}{0.200000,0.200000,0.800000}%
\pgfsetstrokecolor{currentstroke}%
\pgfsetdash{}{0pt}%
\pgfpathmoveto{\pgfqpoint{6.605627in}{2.545945in}}%
\pgfpathcurveto{\pgfqpoint{6.611451in}{2.545945in}}{\pgfqpoint{6.617037in}{2.548258in}}{\pgfqpoint{6.621155in}{2.552377in}}%
\pgfpathcurveto{\pgfqpoint{6.625273in}{2.556495in}}{\pgfqpoint{6.627587in}{2.562081in}}{\pgfqpoint{6.627587in}{2.567905in}}%
\pgfpathcurveto{\pgfqpoint{6.627587in}{2.573729in}}{\pgfqpoint{6.625273in}{2.579315in}}{\pgfqpoint{6.621155in}{2.583433in}}%
\pgfpathcurveto{\pgfqpoint{6.617037in}{2.587551in}}{\pgfqpoint{6.611451in}{2.589865in}}{\pgfqpoint{6.605627in}{2.589865in}}%
\pgfpathcurveto{\pgfqpoint{6.599803in}{2.589865in}}{\pgfqpoint{6.594217in}{2.587551in}}{\pgfqpoint{6.590099in}{2.583433in}}%
\pgfpathcurveto{\pgfqpoint{6.585981in}{2.579315in}}{\pgfqpoint{6.583667in}{2.573729in}}{\pgfqpoint{6.583667in}{2.567905in}}%
\pgfpathcurveto{\pgfqpoint{6.583667in}{2.562081in}}{\pgfqpoint{6.585981in}{2.556495in}}{\pgfqpoint{6.590099in}{2.552377in}}%
\pgfpathcurveto{\pgfqpoint{6.594217in}{2.548258in}}{\pgfqpoint{6.599803in}{2.545945in}}{\pgfqpoint{6.605627in}{2.545945in}}%
\pgfpathlineto{\pgfqpoint{6.605627in}{2.545945in}}%
\pgfpathclose%
\pgfusepath{stroke,fill}%
\end{pgfscope}%
\begin{pgfscope}%
\pgfpathrectangle{\pgfqpoint{1.073501in}{0.880000in}}{\pgfqpoint{6.052998in}{6.160000in}}%
\pgfusepath{clip}%
\pgfsetbuttcap%
\pgfsetroundjoin%
\definecolor{currentfill}{rgb}{0.200000,0.200000,0.800000}%
\pgfsetfillcolor{currentfill}%
\pgfsetlinewidth{1.003750pt}%
\definecolor{currentstroke}{rgb}{0.200000,0.200000,0.800000}%
\pgfsetstrokecolor{currentstroke}%
\pgfsetdash{}{0pt}%
\pgfpathmoveto{\pgfqpoint{6.616838in}{2.739959in}}%
\pgfpathcurveto{\pgfqpoint{6.622662in}{2.739959in}}{\pgfqpoint{6.628248in}{2.742273in}}{\pgfqpoint{6.632366in}{2.746391in}}%
\pgfpathcurveto{\pgfqpoint{6.636484in}{2.750509in}}{\pgfqpoint{6.638798in}{2.756096in}}{\pgfqpoint{6.638798in}{2.761920in}}%
\pgfpathcurveto{\pgfqpoint{6.638798in}{2.767743in}}{\pgfqpoint{6.636484in}{2.773330in}}{\pgfqpoint{6.632366in}{2.777448in}}%
\pgfpathcurveto{\pgfqpoint{6.628248in}{2.781566in}}{\pgfqpoint{6.622662in}{2.783880in}}{\pgfqpoint{6.616838in}{2.783880in}}%
\pgfpathcurveto{\pgfqpoint{6.611014in}{2.783880in}}{\pgfqpoint{6.605428in}{2.781566in}}{\pgfqpoint{6.601310in}{2.777448in}}%
\pgfpathcurveto{\pgfqpoint{6.597192in}{2.773330in}}{\pgfqpoint{6.594878in}{2.767743in}}{\pgfqpoint{6.594878in}{2.761920in}}%
\pgfpathcurveto{\pgfqpoint{6.594878in}{2.756096in}}{\pgfqpoint{6.597192in}{2.750509in}}{\pgfqpoint{6.601310in}{2.746391in}}%
\pgfpathcurveto{\pgfqpoint{6.605428in}{2.742273in}}{\pgfqpoint{6.611014in}{2.739959in}}{\pgfqpoint{6.616838in}{2.739959in}}%
\pgfpathlineto{\pgfqpoint{6.616838in}{2.739959in}}%
\pgfpathclose%
\pgfusepath{stroke,fill}%
\end{pgfscope}%
\begin{pgfscope}%
\pgfpathrectangle{\pgfqpoint{1.073501in}{0.880000in}}{\pgfqpoint{6.052998in}{6.160000in}}%
\pgfusepath{clip}%
\pgfsetbuttcap%
\pgfsetroundjoin%
\definecolor{currentfill}{rgb}{0.200000,0.200000,0.800000}%
\pgfsetfillcolor{currentfill}%
\pgfsetlinewidth{1.003750pt}%
\definecolor{currentstroke}{rgb}{0.200000,0.200000,0.800000}%
\pgfsetstrokecolor{currentstroke}%
\pgfsetdash{}{0pt}%
\pgfpathmoveto{\pgfqpoint{6.576288in}{2.944686in}}%
\pgfpathcurveto{\pgfqpoint{6.582111in}{2.944686in}}{\pgfqpoint{6.587698in}{2.947000in}}{\pgfqpoint{6.591816in}{2.951118in}}%
\pgfpathcurveto{\pgfqpoint{6.595934in}{2.955236in}}{\pgfqpoint{6.598248in}{2.960822in}}{\pgfqpoint{6.598248in}{2.966646in}}%
\pgfpathcurveto{\pgfqpoint{6.598248in}{2.972470in}}{\pgfqpoint{6.595934in}{2.978056in}}{\pgfqpoint{6.591816in}{2.982174in}}%
\pgfpathcurveto{\pgfqpoint{6.587698in}{2.986292in}}{\pgfqpoint{6.582111in}{2.988606in}}{\pgfqpoint{6.576288in}{2.988606in}}%
\pgfpathcurveto{\pgfqpoint{6.570464in}{2.988606in}}{\pgfqpoint{6.564877in}{2.986292in}}{\pgfqpoint{6.560759in}{2.982174in}}%
\pgfpathcurveto{\pgfqpoint{6.556641in}{2.978056in}}{\pgfqpoint{6.554327in}{2.972470in}}{\pgfqpoint{6.554327in}{2.966646in}}%
\pgfpathcurveto{\pgfqpoint{6.554327in}{2.960822in}}{\pgfqpoint{6.556641in}{2.955236in}}{\pgfqpoint{6.560759in}{2.951118in}}%
\pgfpathcurveto{\pgfqpoint{6.564877in}{2.947000in}}{\pgfqpoint{6.570464in}{2.944686in}}{\pgfqpoint{6.576288in}{2.944686in}}%
\pgfpathlineto{\pgfqpoint{6.576288in}{2.944686in}}%
\pgfpathclose%
\pgfusepath{stroke,fill}%
\end{pgfscope}%
\begin{pgfscope}%
\pgfpathrectangle{\pgfqpoint{1.073501in}{0.880000in}}{\pgfqpoint{6.052998in}{6.160000in}}%
\pgfusepath{clip}%
\pgfsetbuttcap%
\pgfsetroundjoin%
\definecolor{currentfill}{rgb}{0.200000,0.200000,0.800000}%
\pgfsetfillcolor{currentfill}%
\pgfsetlinewidth{1.003750pt}%
\definecolor{currentstroke}{rgb}{0.200000,0.200000,0.800000}%
\pgfsetstrokecolor{currentstroke}%
\pgfsetdash{}{0pt}%
\pgfpathmoveto{\pgfqpoint{6.694394in}{3.082119in}}%
\pgfpathcurveto{\pgfqpoint{6.700218in}{3.082119in}}{\pgfqpoint{6.705804in}{3.084433in}}{\pgfqpoint{6.709922in}{3.088551in}}%
\pgfpathcurveto{\pgfqpoint{6.714040in}{3.092669in}}{\pgfqpoint{6.716354in}{3.098256in}}{\pgfqpoint{6.716354in}{3.104080in}}%
\pgfpathcurveto{\pgfqpoint{6.716354in}{3.109903in}}{\pgfqpoint{6.714040in}{3.115490in}}{\pgfqpoint{6.709922in}{3.119608in}}%
\pgfpathcurveto{\pgfqpoint{6.705804in}{3.123726in}}{\pgfqpoint{6.700218in}{3.126040in}}{\pgfqpoint{6.694394in}{3.126040in}}%
\pgfpathcurveto{\pgfqpoint{6.688570in}{3.126040in}}{\pgfqpoint{6.682984in}{3.123726in}}{\pgfqpoint{6.678866in}{3.119608in}}%
\pgfpathcurveto{\pgfqpoint{6.674747in}{3.115490in}}{\pgfqpoint{6.672434in}{3.109903in}}{\pgfqpoint{6.672434in}{3.104080in}}%
\pgfpathcurveto{\pgfqpoint{6.672434in}{3.098256in}}{\pgfqpoint{6.674747in}{3.092669in}}{\pgfqpoint{6.678866in}{3.088551in}}%
\pgfpathcurveto{\pgfqpoint{6.682984in}{3.084433in}}{\pgfqpoint{6.688570in}{3.082119in}}{\pgfqpoint{6.694394in}{3.082119in}}%
\pgfpathlineto{\pgfqpoint{6.694394in}{3.082119in}}%
\pgfpathclose%
\pgfusepath{stroke,fill}%
\end{pgfscope}%
\begin{pgfscope}%
\pgfpathrectangle{\pgfqpoint{1.073501in}{0.880000in}}{\pgfqpoint{6.052998in}{6.160000in}}%
\pgfusepath{clip}%
\pgfsetbuttcap%
\pgfsetroundjoin%
\definecolor{currentfill}{rgb}{0.200000,0.200000,0.800000}%
\pgfsetfillcolor{currentfill}%
\pgfsetlinewidth{1.003750pt}%
\definecolor{currentstroke}{rgb}{0.200000,0.200000,0.800000}%
\pgfsetstrokecolor{currentstroke}%
\pgfsetdash{}{0pt}%
\pgfpathmoveto{\pgfqpoint{6.675115in}{3.264284in}}%
\pgfpathcurveto{\pgfqpoint{6.680939in}{3.264284in}}{\pgfqpoint{6.686525in}{3.266598in}}{\pgfqpoint{6.690643in}{3.270716in}}%
\pgfpathcurveto{\pgfqpoint{6.694761in}{3.274834in}}{\pgfqpoint{6.697075in}{3.280421in}}{\pgfqpoint{6.697075in}{3.286244in}}%
\pgfpathcurveto{\pgfqpoint{6.697075in}{3.292068in}}{\pgfqpoint{6.694761in}{3.297655in}}{\pgfqpoint{6.690643in}{3.301773in}}%
\pgfpathcurveto{\pgfqpoint{6.686525in}{3.305891in}}{\pgfqpoint{6.680939in}{3.308205in}}{\pgfqpoint{6.675115in}{3.308205in}}%
\pgfpathcurveto{\pgfqpoint{6.669291in}{3.308205in}}{\pgfqpoint{6.663705in}{3.305891in}}{\pgfqpoint{6.659586in}{3.301773in}}%
\pgfpathcurveto{\pgfqpoint{6.655468in}{3.297655in}}{\pgfqpoint{6.653154in}{3.292068in}}{\pgfqpoint{6.653154in}{3.286244in}}%
\pgfpathcurveto{\pgfqpoint{6.653154in}{3.280421in}}{\pgfqpoint{6.655468in}{3.274834in}}{\pgfqpoint{6.659586in}{3.270716in}}%
\pgfpathcurveto{\pgfqpoint{6.663705in}{3.266598in}}{\pgfqpoint{6.669291in}{3.264284in}}{\pgfqpoint{6.675115in}{3.264284in}}%
\pgfpathlineto{\pgfqpoint{6.675115in}{3.264284in}}%
\pgfpathclose%
\pgfusepath{stroke,fill}%
\end{pgfscope}%
\begin{pgfscope}%
\pgfpathrectangle{\pgfqpoint{1.073501in}{0.880000in}}{\pgfqpoint{6.052998in}{6.160000in}}%
\pgfusepath{clip}%
\pgfsetbuttcap%
\pgfsetroundjoin%
\definecolor{currentfill}{rgb}{0.200000,0.200000,0.800000}%
\pgfsetfillcolor{currentfill}%
\pgfsetlinewidth{1.003750pt}%
\definecolor{currentstroke}{rgb}{0.200000,0.200000,0.800000}%
\pgfsetstrokecolor{currentstroke}%
\pgfsetdash{}{0pt}%
\pgfpathmoveto{\pgfqpoint{6.771614in}{3.415814in}}%
\pgfpathcurveto{\pgfqpoint{6.777438in}{3.415814in}}{\pgfqpoint{6.783024in}{3.418128in}}{\pgfqpoint{6.787143in}{3.422246in}}%
\pgfpathcurveto{\pgfqpoint{6.791261in}{3.426364in}}{\pgfqpoint{6.793575in}{3.431950in}}{\pgfqpoint{6.793575in}{3.437774in}}%
\pgfpathcurveto{\pgfqpoint{6.793575in}{3.443598in}}{\pgfqpoint{6.791261in}{3.449184in}}{\pgfqpoint{6.787143in}{3.453302in}}%
\pgfpathcurveto{\pgfqpoint{6.783024in}{3.457420in}}{\pgfqpoint{6.777438in}{3.459734in}}{\pgfqpoint{6.771614in}{3.459734in}}%
\pgfpathcurveto{\pgfqpoint{6.765790in}{3.459734in}}{\pgfqpoint{6.760204in}{3.457420in}}{\pgfqpoint{6.756086in}{3.453302in}}%
\pgfpathcurveto{\pgfqpoint{6.751968in}{3.449184in}}{\pgfqpoint{6.749654in}{3.443598in}}{\pgfqpoint{6.749654in}{3.437774in}}%
\pgfpathcurveto{\pgfqpoint{6.749654in}{3.431950in}}{\pgfqpoint{6.751968in}{3.426364in}}{\pgfqpoint{6.756086in}{3.422246in}}%
\pgfpathcurveto{\pgfqpoint{6.760204in}{3.418128in}}{\pgfqpoint{6.765790in}{3.415814in}}{\pgfqpoint{6.771614in}{3.415814in}}%
\pgfpathlineto{\pgfqpoint{6.771614in}{3.415814in}}%
\pgfpathclose%
\pgfusepath{stroke,fill}%
\end{pgfscope}%
\begin{pgfscope}%
\pgfpathrectangle{\pgfqpoint{1.073501in}{0.880000in}}{\pgfqpoint{6.052998in}{6.160000in}}%
\pgfusepath{clip}%
\pgfsetbuttcap%
\pgfsetroundjoin%
\definecolor{currentfill}{rgb}{0.200000,0.200000,0.800000}%
\pgfsetfillcolor{currentfill}%
\pgfsetlinewidth{1.003750pt}%
\definecolor{currentstroke}{rgb}{0.200000,0.200000,0.800000}%
\pgfsetstrokecolor{currentstroke}%
\pgfsetdash{}{0pt}%
\pgfpathmoveto{\pgfqpoint{6.845892in}{3.578629in}}%
\pgfpathcurveto{\pgfqpoint{6.851716in}{3.578629in}}{\pgfqpoint{6.857302in}{3.580943in}}{\pgfqpoint{6.861420in}{3.585061in}}%
\pgfpathcurveto{\pgfqpoint{6.865538in}{3.589179in}}{\pgfqpoint{6.867852in}{3.594766in}}{\pgfqpoint{6.867852in}{3.600590in}}%
\pgfpathcurveto{\pgfqpoint{6.867852in}{3.606413in}}{\pgfqpoint{6.865538in}{3.612000in}}{\pgfqpoint{6.861420in}{3.616118in}}%
\pgfpathcurveto{\pgfqpoint{6.857302in}{3.620236in}}{\pgfqpoint{6.851716in}{3.622550in}}{\pgfqpoint{6.845892in}{3.622550in}}%
\pgfpathcurveto{\pgfqpoint{6.840068in}{3.622550in}}{\pgfqpoint{6.834482in}{3.620236in}}{\pgfqpoint{6.830364in}{3.616118in}}%
\pgfpathcurveto{\pgfqpoint{6.826245in}{3.612000in}}{\pgfqpoint{6.823932in}{3.606413in}}{\pgfqpoint{6.823932in}{3.600590in}}%
\pgfpathcurveto{\pgfqpoint{6.823932in}{3.594766in}}{\pgfqpoint{6.826245in}{3.589179in}}{\pgfqpoint{6.830364in}{3.585061in}}%
\pgfpathcurveto{\pgfqpoint{6.834482in}{3.580943in}}{\pgfqpoint{6.840068in}{3.578629in}}{\pgfqpoint{6.845892in}{3.578629in}}%
\pgfpathlineto{\pgfqpoint{6.845892in}{3.578629in}}%
\pgfpathclose%
\pgfusepath{stroke,fill}%
\end{pgfscope}%
\begin{pgfscope}%
\pgfpathrectangle{\pgfqpoint{1.073501in}{0.880000in}}{\pgfqpoint{6.052998in}{6.160000in}}%
\pgfusepath{clip}%
\pgfsetbuttcap%
\pgfsetroundjoin%
\definecolor{currentfill}{rgb}{0.200000,0.200000,0.800000}%
\pgfsetfillcolor{currentfill}%
\pgfsetlinewidth{1.003750pt}%
\definecolor{currentstroke}{rgb}{0.200000,0.200000,0.800000}%
\pgfsetstrokecolor{currentstroke}%
\pgfsetdash{}{0pt}%
\pgfpathmoveto{\pgfqpoint{6.851363in}{3.752563in}}%
\pgfpathcurveto{\pgfqpoint{6.857187in}{3.752563in}}{\pgfqpoint{6.862773in}{3.754877in}}{\pgfqpoint{6.866891in}{3.758995in}}%
\pgfpathcurveto{\pgfqpoint{6.871009in}{3.763113in}}{\pgfqpoint{6.873323in}{3.768699in}}{\pgfqpoint{6.873323in}{3.774523in}}%
\pgfpathcurveto{\pgfqpoint{6.873323in}{3.780347in}}{\pgfqpoint{6.871009in}{3.785933in}}{\pgfqpoint{6.866891in}{3.790051in}}%
\pgfpathcurveto{\pgfqpoint{6.862773in}{3.794169in}}{\pgfqpoint{6.857187in}{3.796483in}}{\pgfqpoint{6.851363in}{3.796483in}}%
\pgfpathcurveto{\pgfqpoint{6.845539in}{3.796483in}}{\pgfqpoint{6.839953in}{3.794169in}}{\pgfqpoint{6.835835in}{3.790051in}}%
\pgfpathcurveto{\pgfqpoint{6.831717in}{3.785933in}}{\pgfqpoint{6.829403in}{3.780347in}}{\pgfqpoint{6.829403in}{3.774523in}}%
\pgfpathcurveto{\pgfqpoint{6.829403in}{3.768699in}}{\pgfqpoint{6.831717in}{3.763113in}}{\pgfqpoint{6.835835in}{3.758995in}}%
\pgfpathcurveto{\pgfqpoint{6.839953in}{3.754877in}}{\pgfqpoint{6.845539in}{3.752563in}}{\pgfqpoint{6.851363in}{3.752563in}}%
\pgfpathlineto{\pgfqpoint{6.851363in}{3.752563in}}%
\pgfpathclose%
\pgfusepath{stroke,fill}%
\end{pgfscope}%
\begin{pgfscope}%
\pgfpathrectangle{\pgfqpoint{1.073501in}{0.880000in}}{\pgfqpoint{6.052998in}{6.160000in}}%
\pgfusepath{clip}%
\pgfsetbuttcap%
\pgfsetroundjoin%
\definecolor{currentfill}{rgb}{0.200000,0.200000,0.800000}%
\pgfsetfillcolor{currentfill}%
\pgfsetlinewidth{1.003750pt}%
\definecolor{currentstroke}{rgb}{0.200000,0.200000,0.800000}%
\pgfsetstrokecolor{currentstroke}%
\pgfsetdash{}{0pt}%
\pgfpathmoveto{\pgfqpoint{6.771317in}{3.925790in}}%
\pgfpathcurveto{\pgfqpoint{6.777141in}{3.925790in}}{\pgfqpoint{6.782728in}{3.928103in}}{\pgfqpoint{6.786846in}{3.932222in}}%
\pgfpathcurveto{\pgfqpoint{6.790964in}{3.936340in}}{\pgfqpoint{6.793278in}{3.941926in}}{\pgfqpoint{6.793278in}{3.947750in}}%
\pgfpathcurveto{\pgfqpoint{6.793278in}{3.953574in}}{\pgfqpoint{6.790964in}{3.959160in}}{\pgfqpoint{6.786846in}{3.963278in}}%
\pgfpathcurveto{\pgfqpoint{6.782728in}{3.967396in}}{\pgfqpoint{6.777141in}{3.969710in}}{\pgfqpoint{6.771317in}{3.969710in}}%
\pgfpathcurveto{\pgfqpoint{6.765494in}{3.969710in}}{\pgfqpoint{6.759907in}{3.967396in}}{\pgfqpoint{6.755789in}{3.963278in}}%
\pgfpathcurveto{\pgfqpoint{6.751671in}{3.959160in}}{\pgfqpoint{6.749357in}{3.953574in}}{\pgfqpoint{6.749357in}{3.947750in}}%
\pgfpathcurveto{\pgfqpoint{6.749357in}{3.941926in}}{\pgfqpoint{6.751671in}{3.936340in}}{\pgfqpoint{6.755789in}{3.932222in}}%
\pgfpathcurveto{\pgfqpoint{6.759907in}{3.928103in}}{\pgfqpoint{6.765494in}{3.925790in}}{\pgfqpoint{6.771317in}{3.925790in}}%
\pgfpathlineto{\pgfqpoint{6.771317in}{3.925790in}}%
\pgfpathclose%
\pgfusepath{stroke,fill}%
\end{pgfscope}%
\begin{pgfscope}%
\pgfpathrectangle{\pgfqpoint{1.073501in}{0.880000in}}{\pgfqpoint{6.052998in}{6.160000in}}%
\pgfusepath{clip}%
\pgfsetbuttcap%
\pgfsetroundjoin%
\definecolor{currentfill}{rgb}{0.200000,0.800000,0.200000}%
\pgfsetfillcolor{currentfill}%
\pgfsetlinewidth{1.003750pt}%
\definecolor{currentstroke}{rgb}{0.200000,0.800000,0.200000}%
\pgfsetstrokecolor{currentstroke}%
\pgfsetdash{}{0pt}%
\pgfpathmoveto{\pgfqpoint{5.617958in}{3.925790in}}%
\pgfpathcurveto{\pgfqpoint{5.623781in}{3.925790in}}{\pgfqpoint{5.629368in}{3.928103in}}{\pgfqpoint{5.633486in}{3.932222in}}%
\pgfpathcurveto{\pgfqpoint{5.637604in}{3.936340in}}{\pgfqpoint{5.639918in}{3.941926in}}{\pgfqpoint{5.639918in}{3.947750in}}%
\pgfpathcurveto{\pgfqpoint{5.639918in}{3.953574in}}{\pgfqpoint{5.637604in}{3.959160in}}{\pgfqpoint{5.633486in}{3.963278in}}%
\pgfpathcurveto{\pgfqpoint{5.629368in}{3.967396in}}{\pgfqpoint{5.623781in}{3.969710in}}{\pgfqpoint{5.617958in}{3.969710in}}%
\pgfpathcurveto{\pgfqpoint{5.612134in}{3.969710in}}{\pgfqpoint{5.606547in}{3.967396in}}{\pgfqpoint{5.602429in}{3.963278in}}%
\pgfpathcurveto{\pgfqpoint{5.598311in}{3.959160in}}{\pgfqpoint{5.595997in}{3.953574in}}{\pgfqpoint{5.595997in}{3.947750in}}%
\pgfpathcurveto{\pgfqpoint{5.595997in}{3.941926in}}{\pgfqpoint{5.598311in}{3.936340in}}{\pgfqpoint{5.602429in}{3.932222in}}%
\pgfpathcurveto{\pgfqpoint{5.606547in}{3.928103in}}{\pgfqpoint{5.612134in}{3.925790in}}{\pgfqpoint{5.617958in}{3.925790in}}%
\pgfpathlineto{\pgfqpoint{5.617958in}{3.925790in}}%
\pgfpathclose%
\pgfusepath{stroke,fill}%
\end{pgfscope}%
\begin{pgfscope}%
\pgfpathrectangle{\pgfqpoint{1.073501in}{0.880000in}}{\pgfqpoint{6.052998in}{6.160000in}}%
\pgfusepath{clip}%
\pgfsetbuttcap%
\pgfsetroundjoin%
\definecolor{currentfill}{rgb}{0.200000,0.800000,0.200000}%
\pgfsetfillcolor{currentfill}%
\pgfsetlinewidth{1.003750pt}%
\definecolor{currentstroke}{rgb}{0.200000,0.800000,0.200000}%
\pgfsetstrokecolor{currentstroke}%
\pgfsetdash{}{0pt}%
\pgfpathmoveto{\pgfqpoint{5.578538in}{4.018126in}}%
\pgfpathcurveto{\pgfqpoint{5.584362in}{4.018126in}}{\pgfqpoint{5.589949in}{4.020440in}}{\pgfqpoint{5.594067in}{4.024558in}}%
\pgfpathcurveto{\pgfqpoint{5.598185in}{4.028676in}}{\pgfqpoint{5.600499in}{4.034262in}}{\pgfqpoint{5.600499in}{4.040086in}}%
\pgfpathcurveto{\pgfqpoint{5.600499in}{4.045910in}}{\pgfqpoint{5.598185in}{4.051496in}}{\pgfqpoint{5.594067in}{4.055614in}}%
\pgfpathcurveto{\pgfqpoint{5.589949in}{4.059733in}}{\pgfqpoint{5.584362in}{4.062046in}}{\pgfqpoint{5.578538in}{4.062046in}}%
\pgfpathcurveto{\pgfqpoint{5.572714in}{4.062046in}}{\pgfqpoint{5.567128in}{4.059733in}}{\pgfqpoint{5.563010in}{4.055614in}}%
\pgfpathcurveto{\pgfqpoint{5.558892in}{4.051496in}}{\pgfqpoint{5.556578in}{4.045910in}}{\pgfqpoint{5.556578in}{4.040086in}}%
\pgfpathcurveto{\pgfqpoint{5.556578in}{4.034262in}}{\pgfqpoint{5.558892in}{4.028676in}}{\pgfqpoint{5.563010in}{4.024558in}}%
\pgfpathcurveto{\pgfqpoint{5.567128in}{4.020440in}}{\pgfqpoint{5.572714in}{4.018126in}}{\pgfqpoint{5.578538in}{4.018126in}}%
\pgfpathlineto{\pgfqpoint{5.578538in}{4.018126in}}%
\pgfpathclose%
\pgfusepath{stroke,fill}%
\end{pgfscope}%
\begin{pgfscope}%
\pgfpathrectangle{\pgfqpoint{1.073501in}{0.880000in}}{\pgfqpoint{6.052998in}{6.160000in}}%
\pgfusepath{clip}%
\pgfsetbuttcap%
\pgfsetroundjoin%
\definecolor{currentfill}{rgb}{0.200000,0.800000,0.200000}%
\pgfsetfillcolor{currentfill}%
\pgfsetlinewidth{1.003750pt}%
\definecolor{currentstroke}{rgb}{0.200000,0.800000,0.200000}%
\pgfsetstrokecolor{currentstroke}%
\pgfsetdash{}{0pt}%
\pgfpathmoveto{\pgfqpoint{5.561088in}{4.108984in}}%
\pgfpathcurveto{\pgfqpoint{5.566912in}{4.108984in}}{\pgfqpoint{5.572498in}{4.111298in}}{\pgfqpoint{5.576616in}{4.115416in}}%
\pgfpathcurveto{\pgfqpoint{5.580734in}{4.119534in}}{\pgfqpoint{5.583048in}{4.125121in}}{\pgfqpoint{5.583048in}{4.130944in}}%
\pgfpathcurveto{\pgfqpoint{5.583048in}{4.136768in}}{\pgfqpoint{5.580734in}{4.142355in}}{\pgfqpoint{5.576616in}{4.146473in}}%
\pgfpathcurveto{\pgfqpoint{5.572498in}{4.150591in}}{\pgfqpoint{5.566912in}{4.152905in}}{\pgfqpoint{5.561088in}{4.152905in}}%
\pgfpathcurveto{\pgfqpoint{5.555264in}{4.152905in}}{\pgfqpoint{5.549678in}{4.150591in}}{\pgfqpoint{5.545559in}{4.146473in}}%
\pgfpathcurveto{\pgfqpoint{5.541441in}{4.142355in}}{\pgfqpoint{5.539127in}{4.136768in}}{\pgfqpoint{5.539127in}{4.130944in}}%
\pgfpathcurveto{\pgfqpoint{5.539127in}{4.125121in}}{\pgfqpoint{5.541441in}{4.119534in}}{\pgfqpoint{5.545559in}{4.115416in}}%
\pgfpathcurveto{\pgfqpoint{5.549678in}{4.111298in}}{\pgfqpoint{5.555264in}{4.108984in}}{\pgfqpoint{5.561088in}{4.108984in}}%
\pgfpathlineto{\pgfqpoint{5.561088in}{4.108984in}}%
\pgfpathclose%
\pgfusepath{stroke,fill}%
\end{pgfscope}%
\begin{pgfscope}%
\pgfpathrectangle{\pgfqpoint{1.073501in}{0.880000in}}{\pgfqpoint{6.052998in}{6.160000in}}%
\pgfusepath{clip}%
\pgfsetbuttcap%
\pgfsetroundjoin%
\definecolor{currentfill}{rgb}{0.200000,0.800000,0.200000}%
\pgfsetfillcolor{currentfill}%
\pgfsetlinewidth{1.003750pt}%
\definecolor{currentstroke}{rgb}{0.200000,0.800000,0.200000}%
\pgfsetstrokecolor{currentstroke}%
\pgfsetdash{}{0pt}%
\pgfpathmoveto{\pgfqpoint{5.584600in}{4.206987in}}%
\pgfpathcurveto{\pgfqpoint{5.590424in}{4.206987in}}{\pgfqpoint{5.596011in}{4.209301in}}{\pgfqpoint{5.600129in}{4.213419in}}%
\pgfpathcurveto{\pgfqpoint{5.604247in}{4.217537in}}{\pgfqpoint{5.606561in}{4.223123in}}{\pgfqpoint{5.606561in}{4.228947in}}%
\pgfpathcurveto{\pgfqpoint{5.606561in}{4.234771in}}{\pgfqpoint{5.604247in}{4.240357in}}{\pgfqpoint{5.600129in}{4.244476in}}%
\pgfpathcurveto{\pgfqpoint{5.596011in}{4.248594in}}{\pgfqpoint{5.590424in}{4.250908in}}{\pgfqpoint{5.584600in}{4.250908in}}%
\pgfpathcurveto{\pgfqpoint{5.578776in}{4.250908in}}{\pgfqpoint{5.573190in}{4.248594in}}{\pgfqpoint{5.569072in}{4.244476in}}%
\pgfpathcurveto{\pgfqpoint{5.564954in}{4.240357in}}{\pgfqpoint{5.562640in}{4.234771in}}{\pgfqpoint{5.562640in}{4.228947in}}%
\pgfpathcurveto{\pgfqpoint{5.562640in}{4.223123in}}{\pgfqpoint{5.564954in}{4.217537in}}{\pgfqpoint{5.569072in}{4.213419in}}%
\pgfpathcurveto{\pgfqpoint{5.573190in}{4.209301in}}{\pgfqpoint{5.578776in}{4.206987in}}{\pgfqpoint{5.584600in}{4.206987in}}%
\pgfpathlineto{\pgfqpoint{5.584600in}{4.206987in}}%
\pgfpathclose%
\pgfusepath{stroke,fill}%
\end{pgfscope}%
\begin{pgfscope}%
\pgfpathrectangle{\pgfqpoint{1.073501in}{0.880000in}}{\pgfqpoint{6.052998in}{6.160000in}}%
\pgfusepath{clip}%
\pgfsetbuttcap%
\pgfsetroundjoin%
\definecolor{currentfill}{rgb}{0.200000,0.800000,0.200000}%
\pgfsetfillcolor{currentfill}%
\pgfsetlinewidth{1.003750pt}%
\definecolor{currentstroke}{rgb}{0.200000,0.800000,0.200000}%
\pgfsetstrokecolor{currentstroke}%
\pgfsetdash{}{0pt}%
\pgfpathmoveto{\pgfqpoint{5.556462in}{4.297045in}}%
\pgfpathcurveto{\pgfqpoint{5.562286in}{4.297045in}}{\pgfqpoint{5.567872in}{4.299359in}}{\pgfqpoint{5.571990in}{4.303477in}}%
\pgfpathcurveto{\pgfqpoint{5.576109in}{4.307595in}}{\pgfqpoint{5.578422in}{4.313181in}}{\pgfqpoint{5.578422in}{4.319005in}}%
\pgfpathcurveto{\pgfqpoint{5.578422in}{4.324829in}}{\pgfqpoint{5.576109in}{4.330415in}}{\pgfqpoint{5.571990in}{4.334533in}}%
\pgfpathcurveto{\pgfqpoint{5.567872in}{4.338651in}}{\pgfqpoint{5.562286in}{4.340965in}}{\pgfqpoint{5.556462in}{4.340965in}}%
\pgfpathcurveto{\pgfqpoint{5.550638in}{4.340965in}}{\pgfqpoint{5.545052in}{4.338651in}}{\pgfqpoint{5.540934in}{4.334533in}}%
\pgfpathcurveto{\pgfqpoint{5.536816in}{4.330415in}}{\pgfqpoint{5.534502in}{4.324829in}}{\pgfqpoint{5.534502in}{4.319005in}}%
\pgfpathcurveto{\pgfqpoint{5.534502in}{4.313181in}}{\pgfqpoint{5.536816in}{4.307595in}}{\pgfqpoint{5.540934in}{4.303477in}}%
\pgfpathcurveto{\pgfqpoint{5.545052in}{4.299359in}}{\pgfqpoint{5.550638in}{4.297045in}}{\pgfqpoint{5.556462in}{4.297045in}}%
\pgfpathlineto{\pgfqpoint{5.556462in}{4.297045in}}%
\pgfpathclose%
\pgfusepath{stroke,fill}%
\end{pgfscope}%
\begin{pgfscope}%
\pgfpathrectangle{\pgfqpoint{1.073501in}{0.880000in}}{\pgfqpoint{6.052998in}{6.160000in}}%
\pgfusepath{clip}%
\pgfsetbuttcap%
\pgfsetroundjoin%
\definecolor{currentfill}{rgb}{0.200000,0.800000,0.200000}%
\pgfsetfillcolor{currentfill}%
\pgfsetlinewidth{1.003750pt}%
\definecolor{currentstroke}{rgb}{0.200000,0.800000,0.200000}%
\pgfsetstrokecolor{currentstroke}%
\pgfsetdash{}{0pt}%
\pgfpathmoveto{\pgfqpoint{5.377332in}{4.336895in}}%
\pgfpathcurveto{\pgfqpoint{5.383156in}{4.336895in}}{\pgfqpoint{5.388742in}{4.339209in}}{\pgfqpoint{5.392860in}{4.343327in}}%
\pgfpathcurveto{\pgfqpoint{5.396978in}{4.347445in}}{\pgfqpoint{5.399292in}{4.353031in}}{\pgfqpoint{5.399292in}{4.358855in}}%
\pgfpathcurveto{\pgfqpoint{5.399292in}{4.364679in}}{\pgfqpoint{5.396978in}{4.370265in}}{\pgfqpoint{5.392860in}{4.374383in}}%
\pgfpathcurveto{\pgfqpoint{5.388742in}{4.378502in}}{\pgfqpoint{5.383156in}{4.380815in}}{\pgfqpoint{5.377332in}{4.380815in}}%
\pgfpathcurveto{\pgfqpoint{5.371508in}{4.380815in}}{\pgfqpoint{5.365922in}{4.378502in}}{\pgfqpoint{5.361804in}{4.374383in}}%
\pgfpathcurveto{\pgfqpoint{5.357685in}{4.370265in}}{\pgfqpoint{5.355372in}{4.364679in}}{\pgfqpoint{5.355372in}{4.358855in}}%
\pgfpathcurveto{\pgfqpoint{5.355372in}{4.353031in}}{\pgfqpoint{5.357685in}{4.347445in}}{\pgfqpoint{5.361804in}{4.343327in}}%
\pgfpathcurveto{\pgfqpoint{5.365922in}{4.339209in}}{\pgfqpoint{5.371508in}{4.336895in}}{\pgfqpoint{5.377332in}{4.336895in}}%
\pgfpathlineto{\pgfqpoint{5.377332in}{4.336895in}}%
\pgfpathclose%
\pgfusepath{stroke,fill}%
\end{pgfscope}%
\begin{pgfscope}%
\pgfpathrectangle{\pgfqpoint{1.073501in}{0.880000in}}{\pgfqpoint{6.052998in}{6.160000in}}%
\pgfusepath{clip}%
\pgfsetbuttcap%
\pgfsetroundjoin%
\definecolor{currentfill}{rgb}{0.200000,0.800000,0.200000}%
\pgfsetfillcolor{currentfill}%
\pgfsetlinewidth{1.003750pt}%
\definecolor{currentstroke}{rgb}{0.200000,0.800000,0.200000}%
\pgfsetstrokecolor{currentstroke}%
\pgfsetdash{}{0pt}%
\pgfpathmoveto{\pgfqpoint{5.511712in}{4.480702in}}%
\pgfpathcurveto{\pgfqpoint{5.517536in}{4.480702in}}{\pgfqpoint{5.523122in}{4.483016in}}{\pgfqpoint{5.527241in}{4.487134in}}%
\pgfpathcurveto{\pgfqpoint{5.531359in}{4.491252in}}{\pgfqpoint{5.533673in}{4.496838in}}{\pgfqpoint{5.533673in}{4.502662in}}%
\pgfpathcurveto{\pgfqpoint{5.533673in}{4.508486in}}{\pgfqpoint{5.531359in}{4.514072in}}{\pgfqpoint{5.527241in}{4.518190in}}%
\pgfpathcurveto{\pgfqpoint{5.523122in}{4.522308in}}{\pgfqpoint{5.517536in}{4.524622in}}{\pgfqpoint{5.511712in}{4.524622in}}%
\pgfpathcurveto{\pgfqpoint{5.505888in}{4.524622in}}{\pgfqpoint{5.500302in}{4.522308in}}{\pgfqpoint{5.496184in}{4.518190in}}%
\pgfpathcurveto{\pgfqpoint{5.492066in}{4.514072in}}{\pgfqpoint{5.489752in}{4.508486in}}{\pgfqpoint{5.489752in}{4.502662in}}%
\pgfpathcurveto{\pgfqpoint{5.489752in}{4.496838in}}{\pgfqpoint{5.492066in}{4.491252in}}{\pgfqpoint{5.496184in}{4.487134in}}%
\pgfpathcurveto{\pgfqpoint{5.500302in}{4.483016in}}{\pgfqpoint{5.505888in}{4.480702in}}{\pgfqpoint{5.511712in}{4.480702in}}%
\pgfpathlineto{\pgfqpoint{5.511712in}{4.480702in}}%
\pgfpathclose%
\pgfusepath{stroke,fill}%
\end{pgfscope}%
\begin{pgfscope}%
\pgfpathrectangle{\pgfqpoint{1.073501in}{0.880000in}}{\pgfqpoint{6.052998in}{6.160000in}}%
\pgfusepath{clip}%
\pgfsetbuttcap%
\pgfsetroundjoin%
\definecolor{currentfill}{rgb}{0.200000,0.800000,0.200000}%
\pgfsetfillcolor{currentfill}%
\pgfsetlinewidth{1.003750pt}%
\definecolor{currentstroke}{rgb}{0.200000,0.800000,0.200000}%
\pgfsetstrokecolor{currentstroke}%
\pgfsetdash{}{0pt}%
\pgfpathmoveto{\pgfqpoint{5.498415in}{4.579248in}}%
\pgfpathcurveto{\pgfqpoint{5.504239in}{4.579248in}}{\pgfqpoint{5.509825in}{4.581562in}}{\pgfqpoint{5.513943in}{4.585680in}}%
\pgfpathcurveto{\pgfqpoint{5.518062in}{4.589798in}}{\pgfqpoint{5.520375in}{4.595385in}}{\pgfqpoint{5.520375in}{4.601208in}}%
\pgfpathcurveto{\pgfqpoint{5.520375in}{4.607032in}}{\pgfqpoint{5.518062in}{4.612619in}}{\pgfqpoint{5.513943in}{4.616737in}}%
\pgfpathcurveto{\pgfqpoint{5.509825in}{4.620855in}}{\pgfqpoint{5.504239in}{4.623169in}}{\pgfqpoint{5.498415in}{4.623169in}}%
\pgfpathcurveto{\pgfqpoint{5.492591in}{4.623169in}}{\pgfqpoint{5.487005in}{4.620855in}}{\pgfqpoint{5.482887in}{4.616737in}}%
\pgfpathcurveto{\pgfqpoint{5.478769in}{4.612619in}}{\pgfqpoint{5.476455in}{4.607032in}}{\pgfqpoint{5.476455in}{4.601208in}}%
\pgfpathcurveto{\pgfqpoint{5.476455in}{4.595385in}}{\pgfqpoint{5.478769in}{4.589798in}}{\pgfqpoint{5.482887in}{4.585680in}}%
\pgfpathcurveto{\pgfqpoint{5.487005in}{4.581562in}}{\pgfqpoint{5.492591in}{4.579248in}}{\pgfqpoint{5.498415in}{4.579248in}}%
\pgfpathlineto{\pgfqpoint{5.498415in}{4.579248in}}%
\pgfpathclose%
\pgfusepath{stroke,fill}%
\end{pgfscope}%
\begin{pgfscope}%
\pgfpathrectangle{\pgfqpoint{1.073501in}{0.880000in}}{\pgfqpoint{6.052998in}{6.160000in}}%
\pgfusepath{clip}%
\pgfsetbuttcap%
\pgfsetroundjoin%
\definecolor{currentfill}{rgb}{0.200000,0.800000,0.200000}%
\pgfsetfillcolor{currentfill}%
\pgfsetlinewidth{1.003750pt}%
\definecolor{currentstroke}{rgb}{0.200000,0.800000,0.200000}%
\pgfsetstrokecolor{currentstroke}%
\pgfsetdash{}{0pt}%
\pgfpathmoveto{\pgfqpoint{5.404437in}{4.637310in}}%
\pgfpathcurveto{\pgfqpoint{5.410261in}{4.637310in}}{\pgfqpoint{5.415847in}{4.639624in}}{\pgfqpoint{5.419965in}{4.643742in}}%
\pgfpathcurveto{\pgfqpoint{5.424083in}{4.647860in}}{\pgfqpoint{5.426397in}{4.653446in}}{\pgfqpoint{5.426397in}{4.659270in}}%
\pgfpathcurveto{\pgfqpoint{5.426397in}{4.665094in}}{\pgfqpoint{5.424083in}{4.670681in}}{\pgfqpoint{5.419965in}{4.674799in}}%
\pgfpathcurveto{\pgfqpoint{5.415847in}{4.678917in}}{\pgfqpoint{5.410261in}{4.681231in}}{\pgfqpoint{5.404437in}{4.681231in}}%
\pgfpathcurveto{\pgfqpoint{5.398613in}{4.681231in}}{\pgfqpoint{5.393027in}{4.678917in}}{\pgfqpoint{5.388908in}{4.674799in}}%
\pgfpathcurveto{\pgfqpoint{5.384790in}{4.670681in}}{\pgfqpoint{5.382476in}{4.665094in}}{\pgfqpoint{5.382476in}{4.659270in}}%
\pgfpathcurveto{\pgfqpoint{5.382476in}{4.653446in}}{\pgfqpoint{5.384790in}{4.647860in}}{\pgfqpoint{5.388908in}{4.643742in}}%
\pgfpathcurveto{\pgfqpoint{5.393027in}{4.639624in}}{\pgfqpoint{5.398613in}{4.637310in}}{\pgfqpoint{5.404437in}{4.637310in}}%
\pgfpathlineto{\pgfqpoint{5.404437in}{4.637310in}}%
\pgfpathclose%
\pgfusepath{stroke,fill}%
\end{pgfscope}%
\begin{pgfscope}%
\pgfpathrectangle{\pgfqpoint{1.073501in}{0.880000in}}{\pgfqpoint{6.052998in}{6.160000in}}%
\pgfusepath{clip}%
\pgfsetbuttcap%
\pgfsetroundjoin%
\definecolor{currentfill}{rgb}{0.200000,0.800000,0.200000}%
\pgfsetfillcolor{currentfill}%
\pgfsetlinewidth{1.003750pt}%
\definecolor{currentstroke}{rgb}{0.200000,0.800000,0.200000}%
\pgfsetstrokecolor{currentstroke}%
\pgfsetdash{}{0pt}%
\pgfpathmoveto{\pgfqpoint{5.258212in}{4.653669in}}%
\pgfpathcurveto{\pgfqpoint{5.264036in}{4.653669in}}{\pgfqpoint{5.269622in}{4.655983in}}{\pgfqpoint{5.273740in}{4.660101in}}%
\pgfpathcurveto{\pgfqpoint{5.277858in}{4.664219in}}{\pgfqpoint{5.280172in}{4.669805in}}{\pgfqpoint{5.280172in}{4.675629in}}%
\pgfpathcurveto{\pgfqpoint{5.280172in}{4.681453in}}{\pgfqpoint{5.277858in}{4.687040in}}{\pgfqpoint{5.273740in}{4.691158in}}%
\pgfpathcurveto{\pgfqpoint{5.269622in}{4.695276in}}{\pgfqpoint{5.264036in}{4.697590in}}{\pgfqpoint{5.258212in}{4.697590in}}%
\pgfpathcurveto{\pgfqpoint{5.252388in}{4.697590in}}{\pgfqpoint{5.246802in}{4.695276in}}{\pgfqpoint{5.242683in}{4.691158in}}%
\pgfpathcurveto{\pgfqpoint{5.238565in}{4.687040in}}{\pgfqpoint{5.236251in}{4.681453in}}{\pgfqpoint{5.236251in}{4.675629in}}%
\pgfpathcurveto{\pgfqpoint{5.236251in}{4.669805in}}{\pgfqpoint{5.238565in}{4.664219in}}{\pgfqpoint{5.242683in}{4.660101in}}%
\pgfpathcurveto{\pgfqpoint{5.246802in}{4.655983in}}{\pgfqpoint{5.252388in}{4.653669in}}{\pgfqpoint{5.258212in}{4.653669in}}%
\pgfpathlineto{\pgfqpoint{5.258212in}{4.653669in}}%
\pgfpathclose%
\pgfusepath{stroke,fill}%
\end{pgfscope}%
\begin{pgfscope}%
\pgfpathrectangle{\pgfqpoint{1.073501in}{0.880000in}}{\pgfqpoint{6.052998in}{6.160000in}}%
\pgfusepath{clip}%
\pgfsetbuttcap%
\pgfsetroundjoin%
\definecolor{currentfill}{rgb}{0.200000,0.800000,0.200000}%
\pgfsetfillcolor{currentfill}%
\pgfsetlinewidth{1.003750pt}%
\definecolor{currentstroke}{rgb}{0.200000,0.800000,0.200000}%
\pgfsetstrokecolor{currentstroke}%
\pgfsetdash{}{0pt}%
\pgfpathmoveto{\pgfqpoint{5.216588in}{4.729061in}}%
\pgfpathcurveto{\pgfqpoint{5.222412in}{4.729061in}}{\pgfqpoint{5.227999in}{4.731375in}}{\pgfqpoint{5.232117in}{4.735493in}}%
\pgfpathcurveto{\pgfqpoint{5.236235in}{4.739611in}}{\pgfqpoint{5.238549in}{4.745197in}}{\pgfqpoint{5.238549in}{4.751021in}}%
\pgfpathcurveto{\pgfqpoint{5.238549in}{4.756845in}}{\pgfqpoint{5.236235in}{4.762431in}}{\pgfqpoint{5.232117in}{4.766549in}}%
\pgfpathcurveto{\pgfqpoint{5.227999in}{4.770667in}}{\pgfqpoint{5.222412in}{4.772981in}}{\pgfqpoint{5.216588in}{4.772981in}}%
\pgfpathcurveto{\pgfqpoint{5.210765in}{4.772981in}}{\pgfqpoint{5.205178in}{4.770667in}}{\pgfqpoint{5.201060in}{4.766549in}}%
\pgfpathcurveto{\pgfqpoint{5.196942in}{4.762431in}}{\pgfqpoint{5.194628in}{4.756845in}}{\pgfqpoint{5.194628in}{4.751021in}}%
\pgfpathcurveto{\pgfqpoint{5.194628in}{4.745197in}}{\pgfqpoint{5.196942in}{4.739611in}}{\pgfqpoint{5.201060in}{4.735493in}}%
\pgfpathcurveto{\pgfqpoint{5.205178in}{4.731375in}}{\pgfqpoint{5.210765in}{4.729061in}}{\pgfqpoint{5.216588in}{4.729061in}}%
\pgfpathlineto{\pgfqpoint{5.216588in}{4.729061in}}%
\pgfpathclose%
\pgfusepath{stroke,fill}%
\end{pgfscope}%
\begin{pgfscope}%
\pgfpathrectangle{\pgfqpoint{1.073501in}{0.880000in}}{\pgfqpoint{6.052998in}{6.160000in}}%
\pgfusepath{clip}%
\pgfsetbuttcap%
\pgfsetroundjoin%
\definecolor{currentfill}{rgb}{0.200000,0.800000,0.200000}%
\pgfsetfillcolor{currentfill}%
\pgfsetlinewidth{1.003750pt}%
\definecolor{currentstroke}{rgb}{0.200000,0.800000,0.200000}%
\pgfsetstrokecolor{currentstroke}%
\pgfsetdash{}{0pt}%
\pgfpathmoveto{\pgfqpoint{5.117835in}{4.758366in}}%
\pgfpathcurveto{\pgfqpoint{5.123659in}{4.758366in}}{\pgfqpoint{5.129245in}{4.760680in}}{\pgfqpoint{5.133363in}{4.764798in}}%
\pgfpathcurveto{\pgfqpoint{5.137481in}{4.768916in}}{\pgfqpoint{5.139795in}{4.774502in}}{\pgfqpoint{5.139795in}{4.780326in}}%
\pgfpathcurveto{\pgfqpoint{5.139795in}{4.786150in}}{\pgfqpoint{5.137481in}{4.791736in}}{\pgfqpoint{5.133363in}{4.795855in}}%
\pgfpathcurveto{\pgfqpoint{5.129245in}{4.799973in}}{\pgfqpoint{5.123659in}{4.802287in}}{\pgfqpoint{5.117835in}{4.802287in}}%
\pgfpathcurveto{\pgfqpoint{5.112011in}{4.802287in}}{\pgfqpoint{5.106425in}{4.799973in}}{\pgfqpoint{5.102307in}{4.795855in}}%
\pgfpathcurveto{\pgfqpoint{5.098189in}{4.791736in}}{\pgfqpoint{5.095875in}{4.786150in}}{\pgfqpoint{5.095875in}{4.780326in}}%
\pgfpathcurveto{\pgfqpoint{5.095875in}{4.774502in}}{\pgfqpoint{5.098189in}{4.768916in}}{\pgfqpoint{5.102307in}{4.764798in}}%
\pgfpathcurveto{\pgfqpoint{5.106425in}{4.760680in}}{\pgfqpoint{5.112011in}{4.758366in}}{\pgfqpoint{5.117835in}{4.758366in}}%
\pgfpathlineto{\pgfqpoint{5.117835in}{4.758366in}}%
\pgfpathclose%
\pgfusepath{stroke,fill}%
\end{pgfscope}%
\begin{pgfscope}%
\pgfpathrectangle{\pgfqpoint{1.073501in}{0.880000in}}{\pgfqpoint{6.052998in}{6.160000in}}%
\pgfusepath{clip}%
\pgfsetbuttcap%
\pgfsetroundjoin%
\definecolor{currentfill}{rgb}{0.200000,0.800000,0.200000}%
\pgfsetfillcolor{currentfill}%
\pgfsetlinewidth{1.003750pt}%
\definecolor{currentstroke}{rgb}{0.200000,0.800000,0.200000}%
\pgfsetstrokecolor{currentstroke}%
\pgfsetdash{}{0pt}%
\pgfpathmoveto{\pgfqpoint{5.204761in}{4.954759in}}%
\pgfpathcurveto{\pgfqpoint{5.210585in}{4.954759in}}{\pgfqpoint{5.216171in}{4.957073in}}{\pgfqpoint{5.220290in}{4.961191in}}%
\pgfpathcurveto{\pgfqpoint{5.224408in}{4.965309in}}{\pgfqpoint{5.226722in}{4.970896in}}{\pgfqpoint{5.226722in}{4.976719in}}%
\pgfpathcurveto{\pgfqpoint{5.226722in}{4.982543in}}{\pgfqpoint{5.224408in}{4.988130in}}{\pgfqpoint{5.220290in}{4.992248in}}%
\pgfpathcurveto{\pgfqpoint{5.216171in}{4.996366in}}{\pgfqpoint{5.210585in}{4.998680in}}{\pgfqpoint{5.204761in}{4.998680in}}%
\pgfpathcurveto{\pgfqpoint{5.198937in}{4.998680in}}{\pgfqpoint{5.193351in}{4.996366in}}{\pgfqpoint{5.189233in}{4.992248in}}%
\pgfpathcurveto{\pgfqpoint{5.185115in}{4.988130in}}{\pgfqpoint{5.182801in}{4.982543in}}{\pgfqpoint{5.182801in}{4.976719in}}%
\pgfpathcurveto{\pgfqpoint{5.182801in}{4.970896in}}{\pgfqpoint{5.185115in}{4.965309in}}{\pgfqpoint{5.189233in}{4.961191in}}%
\pgfpathcurveto{\pgfqpoint{5.193351in}{4.957073in}}{\pgfqpoint{5.198937in}{4.954759in}}{\pgfqpoint{5.204761in}{4.954759in}}%
\pgfpathlineto{\pgfqpoint{5.204761in}{4.954759in}}%
\pgfpathclose%
\pgfusepath{stroke,fill}%
\end{pgfscope}%
\begin{pgfscope}%
\pgfpathrectangle{\pgfqpoint{1.073501in}{0.880000in}}{\pgfqpoint{6.052998in}{6.160000in}}%
\pgfusepath{clip}%
\pgfsetbuttcap%
\pgfsetroundjoin%
\definecolor{currentfill}{rgb}{0.200000,0.800000,0.200000}%
\pgfsetfillcolor{currentfill}%
\pgfsetlinewidth{1.003750pt}%
\definecolor{currentstroke}{rgb}{0.200000,0.800000,0.200000}%
\pgfsetstrokecolor{currentstroke}%
\pgfsetdash{}{0pt}%
\pgfpathmoveto{\pgfqpoint{5.147604in}{5.032258in}}%
\pgfpathcurveto{\pgfqpoint{5.153428in}{5.032258in}}{\pgfqpoint{5.159014in}{5.034572in}}{\pgfqpoint{5.163132in}{5.038690in}}%
\pgfpathcurveto{\pgfqpoint{5.167250in}{5.042808in}}{\pgfqpoint{5.169564in}{5.048394in}}{\pgfqpoint{5.169564in}{5.054218in}}%
\pgfpathcurveto{\pgfqpoint{5.169564in}{5.060042in}}{\pgfqpoint{5.167250in}{5.065628in}}{\pgfqpoint{5.163132in}{5.069746in}}%
\pgfpathcurveto{\pgfqpoint{5.159014in}{5.073864in}}{\pgfqpoint{5.153428in}{5.076178in}}{\pgfqpoint{5.147604in}{5.076178in}}%
\pgfpathcurveto{\pgfqpoint{5.141780in}{5.076178in}}{\pgfqpoint{5.136194in}{5.073864in}}{\pgfqpoint{5.132076in}{5.069746in}}%
\pgfpathcurveto{\pgfqpoint{5.127958in}{5.065628in}}{\pgfqpoint{5.125644in}{5.060042in}}{\pgfqpoint{5.125644in}{5.054218in}}%
\pgfpathcurveto{\pgfqpoint{5.125644in}{5.048394in}}{\pgfqpoint{5.127958in}{5.042808in}}{\pgfqpoint{5.132076in}{5.038690in}}%
\pgfpathcurveto{\pgfqpoint{5.136194in}{5.034572in}}{\pgfqpoint{5.141780in}{5.032258in}}{\pgfqpoint{5.147604in}{5.032258in}}%
\pgfpathlineto{\pgfqpoint{5.147604in}{5.032258in}}%
\pgfpathclose%
\pgfusepath{stroke,fill}%
\end{pgfscope}%
\begin{pgfscope}%
\pgfpathrectangle{\pgfqpoint{1.073501in}{0.880000in}}{\pgfqpoint{6.052998in}{6.160000in}}%
\pgfusepath{clip}%
\pgfsetbuttcap%
\pgfsetroundjoin%
\definecolor{currentfill}{rgb}{0.200000,0.800000,0.200000}%
\pgfsetfillcolor{currentfill}%
\pgfsetlinewidth{1.003750pt}%
\definecolor{currentstroke}{rgb}{0.200000,0.800000,0.200000}%
\pgfsetstrokecolor{currentstroke}%
\pgfsetdash{}{0pt}%
\pgfpathmoveto{\pgfqpoint{5.043200in}{5.055251in}}%
\pgfpathcurveto{\pgfqpoint{5.049024in}{5.055251in}}{\pgfqpoint{5.054610in}{5.057565in}}{\pgfqpoint{5.058728in}{5.061683in}}%
\pgfpathcurveto{\pgfqpoint{5.062846in}{5.065801in}}{\pgfqpoint{5.065160in}{5.071388in}}{\pgfqpoint{5.065160in}{5.077212in}}%
\pgfpathcurveto{\pgfqpoint{5.065160in}{5.083035in}}{\pgfqpoint{5.062846in}{5.088622in}}{\pgfqpoint{5.058728in}{5.092740in}}%
\pgfpathcurveto{\pgfqpoint{5.054610in}{5.096858in}}{\pgfqpoint{5.049024in}{5.099172in}}{\pgfqpoint{5.043200in}{5.099172in}}%
\pgfpathcurveto{\pgfqpoint{5.037376in}{5.099172in}}{\pgfqpoint{5.031790in}{5.096858in}}{\pgfqpoint{5.027672in}{5.092740in}}%
\pgfpathcurveto{\pgfqpoint{5.023554in}{5.088622in}}{\pgfqpoint{5.021240in}{5.083035in}}{\pgfqpoint{5.021240in}{5.077212in}}%
\pgfpathcurveto{\pgfqpoint{5.021240in}{5.071388in}}{\pgfqpoint{5.023554in}{5.065801in}}{\pgfqpoint{5.027672in}{5.061683in}}%
\pgfpathcurveto{\pgfqpoint{5.031790in}{5.057565in}}{\pgfqpoint{5.037376in}{5.055251in}}{\pgfqpoint{5.043200in}{5.055251in}}%
\pgfpathlineto{\pgfqpoint{5.043200in}{5.055251in}}%
\pgfpathclose%
\pgfusepath{stroke,fill}%
\end{pgfscope}%
\begin{pgfscope}%
\pgfpathrectangle{\pgfqpoint{1.073501in}{0.880000in}}{\pgfqpoint{6.052998in}{6.160000in}}%
\pgfusepath{clip}%
\pgfsetbuttcap%
\pgfsetroundjoin%
\definecolor{currentfill}{rgb}{0.200000,0.800000,0.200000}%
\pgfsetfillcolor{currentfill}%
\pgfsetlinewidth{1.003750pt}%
\definecolor{currentstroke}{rgb}{0.200000,0.800000,0.200000}%
\pgfsetstrokecolor{currentstroke}%
\pgfsetdash{}{0pt}%
\pgfpathmoveto{\pgfqpoint{4.981717in}{5.128024in}}%
\pgfpathcurveto{\pgfqpoint{4.987541in}{5.128024in}}{\pgfqpoint{4.993127in}{5.130338in}}{\pgfqpoint{4.997245in}{5.134457in}}%
\pgfpathcurveto{\pgfqpoint{5.001363in}{5.138575in}}{\pgfqpoint{5.003677in}{5.144161in}}{\pgfqpoint{5.003677in}{5.149985in}}%
\pgfpathcurveto{\pgfqpoint{5.003677in}{5.155809in}}{\pgfqpoint{5.001363in}{5.161395in}}{\pgfqpoint{4.997245in}{5.165513in}}%
\pgfpathcurveto{\pgfqpoint{4.993127in}{5.169631in}}{\pgfqpoint{4.987541in}{5.171945in}}{\pgfqpoint{4.981717in}{5.171945in}}%
\pgfpathcurveto{\pgfqpoint{4.975893in}{5.171945in}}{\pgfqpoint{4.970306in}{5.169631in}}{\pgfqpoint{4.966188in}{5.165513in}}%
\pgfpathcurveto{\pgfqpoint{4.962070in}{5.161395in}}{\pgfqpoint{4.959756in}{5.155809in}}{\pgfqpoint{4.959756in}{5.149985in}}%
\pgfpathcurveto{\pgfqpoint{4.959756in}{5.144161in}}{\pgfqpoint{4.962070in}{5.138575in}}{\pgfqpoint{4.966188in}{5.134457in}}%
\pgfpathcurveto{\pgfqpoint{4.970306in}{5.130338in}}{\pgfqpoint{4.975893in}{5.128024in}}{\pgfqpoint{4.981717in}{5.128024in}}%
\pgfpathlineto{\pgfqpoint{4.981717in}{5.128024in}}%
\pgfpathclose%
\pgfusepath{stroke,fill}%
\end{pgfscope}%
\begin{pgfscope}%
\pgfpathrectangle{\pgfqpoint{1.073501in}{0.880000in}}{\pgfqpoint{6.052998in}{6.160000in}}%
\pgfusepath{clip}%
\pgfsetbuttcap%
\pgfsetroundjoin%
\definecolor{currentfill}{rgb}{0.200000,0.800000,0.200000}%
\pgfsetfillcolor{currentfill}%
\pgfsetlinewidth{1.003750pt}%
\definecolor{currentstroke}{rgb}{0.200000,0.800000,0.200000}%
\pgfsetstrokecolor{currentstroke}%
\pgfsetdash{}{0pt}%
\pgfpathmoveto{\pgfqpoint{4.869050in}{5.123988in}}%
\pgfpathcurveto{\pgfqpoint{4.874874in}{5.123988in}}{\pgfqpoint{4.880460in}{5.126302in}}{\pgfqpoint{4.884578in}{5.130420in}}%
\pgfpathcurveto{\pgfqpoint{4.888696in}{5.134538in}}{\pgfqpoint{4.891010in}{5.140124in}}{\pgfqpoint{4.891010in}{5.145948in}}%
\pgfpathcurveto{\pgfqpoint{4.891010in}{5.151772in}}{\pgfqpoint{4.888696in}{5.157358in}}{\pgfqpoint{4.884578in}{5.161477in}}%
\pgfpathcurveto{\pgfqpoint{4.880460in}{5.165595in}}{\pgfqpoint{4.874874in}{5.167909in}}{\pgfqpoint{4.869050in}{5.167909in}}%
\pgfpathcurveto{\pgfqpoint{4.863226in}{5.167909in}}{\pgfqpoint{4.857640in}{5.165595in}}{\pgfqpoint{4.853522in}{5.161477in}}%
\pgfpathcurveto{\pgfqpoint{4.849404in}{5.157358in}}{\pgfqpoint{4.847090in}{5.151772in}}{\pgfqpoint{4.847090in}{5.145948in}}%
\pgfpathcurveto{\pgfqpoint{4.847090in}{5.140124in}}{\pgfqpoint{4.849404in}{5.134538in}}{\pgfqpoint{4.853522in}{5.130420in}}%
\pgfpathcurveto{\pgfqpoint{4.857640in}{5.126302in}}{\pgfqpoint{4.863226in}{5.123988in}}{\pgfqpoint{4.869050in}{5.123988in}}%
\pgfpathlineto{\pgfqpoint{4.869050in}{5.123988in}}%
\pgfpathclose%
\pgfusepath{stroke,fill}%
\end{pgfscope}%
\begin{pgfscope}%
\pgfpathrectangle{\pgfqpoint{1.073501in}{0.880000in}}{\pgfqpoint{6.052998in}{6.160000in}}%
\pgfusepath{clip}%
\pgfsetbuttcap%
\pgfsetroundjoin%
\definecolor{currentfill}{rgb}{0.200000,0.800000,0.200000}%
\pgfsetfillcolor{currentfill}%
\pgfsetlinewidth{1.003750pt}%
\definecolor{currentstroke}{rgb}{0.200000,0.800000,0.200000}%
\pgfsetstrokecolor{currentstroke}%
\pgfsetdash{}{0pt}%
\pgfpathmoveto{\pgfqpoint{4.815265in}{5.212972in}}%
\pgfpathcurveto{\pgfqpoint{4.821088in}{5.212972in}}{\pgfqpoint{4.826675in}{5.215286in}}{\pgfqpoint{4.830793in}{5.219404in}}%
\pgfpathcurveto{\pgfqpoint{4.834911in}{5.223522in}}{\pgfqpoint{4.837225in}{5.229109in}}{\pgfqpoint{4.837225in}{5.234933in}}%
\pgfpathcurveto{\pgfqpoint{4.837225in}{5.240757in}}{\pgfqpoint{4.834911in}{5.246343in}}{\pgfqpoint{4.830793in}{5.250461in}}%
\pgfpathcurveto{\pgfqpoint{4.826675in}{5.254579in}}{\pgfqpoint{4.821088in}{5.256893in}}{\pgfqpoint{4.815265in}{5.256893in}}%
\pgfpathcurveto{\pgfqpoint{4.809441in}{5.256893in}}{\pgfqpoint{4.803854in}{5.254579in}}{\pgfqpoint{4.799736in}{5.250461in}}%
\pgfpathcurveto{\pgfqpoint{4.795618in}{5.246343in}}{\pgfqpoint{4.793304in}{5.240757in}}{\pgfqpoint{4.793304in}{5.234933in}}%
\pgfpathcurveto{\pgfqpoint{4.793304in}{5.229109in}}{\pgfqpoint{4.795618in}{5.223522in}}{\pgfqpoint{4.799736in}{5.219404in}}%
\pgfpathcurveto{\pgfqpoint{4.803854in}{5.215286in}}{\pgfqpoint{4.809441in}{5.212972in}}{\pgfqpoint{4.815265in}{5.212972in}}%
\pgfpathlineto{\pgfqpoint{4.815265in}{5.212972in}}%
\pgfpathclose%
\pgfusepath{stroke,fill}%
\end{pgfscope}%
\begin{pgfscope}%
\pgfpathrectangle{\pgfqpoint{1.073501in}{0.880000in}}{\pgfqpoint{6.052998in}{6.160000in}}%
\pgfusepath{clip}%
\pgfsetbuttcap%
\pgfsetroundjoin%
\definecolor{currentfill}{rgb}{0.200000,0.800000,0.200000}%
\pgfsetfillcolor{currentfill}%
\pgfsetlinewidth{1.003750pt}%
\definecolor{currentstroke}{rgb}{0.200000,0.800000,0.200000}%
\pgfsetstrokecolor{currentstroke}%
\pgfsetdash{}{0pt}%
\pgfpathmoveto{\pgfqpoint{4.718208in}{5.223401in}}%
\pgfpathcurveto{\pgfqpoint{4.724032in}{5.223401in}}{\pgfqpoint{4.729618in}{5.225715in}}{\pgfqpoint{4.733736in}{5.229833in}}%
\pgfpathcurveto{\pgfqpoint{4.737855in}{5.233951in}}{\pgfqpoint{4.740168in}{5.239537in}}{\pgfqpoint{4.740168in}{5.245361in}}%
\pgfpathcurveto{\pgfqpoint{4.740168in}{5.251185in}}{\pgfqpoint{4.737855in}{5.256771in}}{\pgfqpoint{4.733736in}{5.260889in}}%
\pgfpathcurveto{\pgfqpoint{4.729618in}{5.265008in}}{\pgfqpoint{4.724032in}{5.267321in}}{\pgfqpoint{4.718208in}{5.267321in}}%
\pgfpathcurveto{\pgfqpoint{4.712384in}{5.267321in}}{\pgfqpoint{4.706798in}{5.265008in}}{\pgfqpoint{4.702680in}{5.260889in}}%
\pgfpathcurveto{\pgfqpoint{4.698562in}{5.256771in}}{\pgfqpoint{4.696248in}{5.251185in}}{\pgfqpoint{4.696248in}{5.245361in}}%
\pgfpathcurveto{\pgfqpoint{4.696248in}{5.239537in}}{\pgfqpoint{4.698562in}{5.233951in}}{\pgfqpoint{4.702680in}{5.229833in}}%
\pgfpathcurveto{\pgfqpoint{4.706798in}{5.225715in}}{\pgfqpoint{4.712384in}{5.223401in}}{\pgfqpoint{4.718208in}{5.223401in}}%
\pgfpathlineto{\pgfqpoint{4.718208in}{5.223401in}}%
\pgfpathclose%
\pgfusepath{stroke,fill}%
\end{pgfscope}%
\begin{pgfscope}%
\pgfpathrectangle{\pgfqpoint{1.073501in}{0.880000in}}{\pgfqpoint{6.052998in}{6.160000in}}%
\pgfusepath{clip}%
\pgfsetbuttcap%
\pgfsetroundjoin%
\definecolor{currentfill}{rgb}{0.200000,0.800000,0.200000}%
\pgfsetfillcolor{currentfill}%
\pgfsetlinewidth{1.003750pt}%
\definecolor{currentstroke}{rgb}{0.200000,0.800000,0.200000}%
\pgfsetstrokecolor{currentstroke}%
\pgfsetdash{}{0pt}%
\pgfpathmoveto{\pgfqpoint{4.641081in}{5.275034in}}%
\pgfpathcurveto{\pgfqpoint{4.646905in}{5.275034in}}{\pgfqpoint{4.652491in}{5.277348in}}{\pgfqpoint{4.656609in}{5.281466in}}%
\pgfpathcurveto{\pgfqpoint{4.660727in}{5.285584in}}{\pgfqpoint{4.663041in}{5.291170in}}{\pgfqpoint{4.663041in}{5.296994in}}%
\pgfpathcurveto{\pgfqpoint{4.663041in}{5.302818in}}{\pgfqpoint{4.660727in}{5.308404in}}{\pgfqpoint{4.656609in}{5.312523in}}%
\pgfpathcurveto{\pgfqpoint{4.652491in}{5.316641in}}{\pgfqpoint{4.646905in}{5.318955in}}{\pgfqpoint{4.641081in}{5.318955in}}%
\pgfpathcurveto{\pgfqpoint{4.635257in}{5.318955in}}{\pgfqpoint{4.629671in}{5.316641in}}{\pgfqpoint{4.625552in}{5.312523in}}%
\pgfpathcurveto{\pgfqpoint{4.621434in}{5.308404in}}{\pgfqpoint{4.619120in}{5.302818in}}{\pgfqpoint{4.619120in}{5.296994in}}%
\pgfpathcurveto{\pgfqpoint{4.619120in}{5.291170in}}{\pgfqpoint{4.621434in}{5.285584in}}{\pgfqpoint{4.625552in}{5.281466in}}%
\pgfpathcurveto{\pgfqpoint{4.629671in}{5.277348in}}{\pgfqpoint{4.635257in}{5.275034in}}{\pgfqpoint{4.641081in}{5.275034in}}%
\pgfpathlineto{\pgfqpoint{4.641081in}{5.275034in}}%
\pgfpathclose%
\pgfusepath{stroke,fill}%
\end{pgfscope}%
\begin{pgfscope}%
\pgfpathrectangle{\pgfqpoint{1.073501in}{0.880000in}}{\pgfqpoint{6.052998in}{6.160000in}}%
\pgfusepath{clip}%
\pgfsetbuttcap%
\pgfsetroundjoin%
\definecolor{currentfill}{rgb}{0.200000,0.800000,0.200000}%
\pgfsetfillcolor{currentfill}%
\pgfsetlinewidth{1.003750pt}%
\definecolor{currentstroke}{rgb}{0.200000,0.800000,0.200000}%
\pgfsetstrokecolor{currentstroke}%
\pgfsetdash{}{0pt}%
\pgfpathmoveto{\pgfqpoint{4.553918in}{5.303240in}}%
\pgfpathcurveto{\pgfqpoint{4.559742in}{5.303240in}}{\pgfqpoint{4.565328in}{5.305554in}}{\pgfqpoint{4.569446in}{5.309672in}}%
\pgfpathcurveto{\pgfqpoint{4.573564in}{5.313790in}}{\pgfqpoint{4.575878in}{5.319376in}}{\pgfqpoint{4.575878in}{5.325200in}}%
\pgfpathcurveto{\pgfqpoint{4.575878in}{5.331024in}}{\pgfqpoint{4.573564in}{5.336610in}}{\pgfqpoint{4.569446in}{5.340728in}}%
\pgfpathcurveto{\pgfqpoint{4.565328in}{5.344846in}}{\pgfqpoint{4.559742in}{5.347160in}}{\pgfqpoint{4.553918in}{5.347160in}}%
\pgfpathcurveto{\pgfqpoint{4.548094in}{5.347160in}}{\pgfqpoint{4.542508in}{5.344846in}}{\pgfqpoint{4.538390in}{5.340728in}}%
\pgfpathcurveto{\pgfqpoint{4.534271in}{5.336610in}}{\pgfqpoint{4.531957in}{5.331024in}}{\pgfqpoint{4.531957in}{5.325200in}}%
\pgfpathcurveto{\pgfqpoint{4.531957in}{5.319376in}}{\pgfqpoint{4.534271in}{5.313790in}}{\pgfqpoint{4.538390in}{5.309672in}}%
\pgfpathcurveto{\pgfqpoint{4.542508in}{5.305554in}}{\pgfqpoint{4.548094in}{5.303240in}}{\pgfqpoint{4.553918in}{5.303240in}}%
\pgfpathlineto{\pgfqpoint{4.553918in}{5.303240in}}%
\pgfpathclose%
\pgfusepath{stroke,fill}%
\end{pgfscope}%
\begin{pgfscope}%
\pgfpathrectangle{\pgfqpoint{1.073501in}{0.880000in}}{\pgfqpoint{6.052998in}{6.160000in}}%
\pgfusepath{clip}%
\pgfsetbuttcap%
\pgfsetroundjoin%
\definecolor{currentfill}{rgb}{0.200000,0.800000,0.200000}%
\pgfsetfillcolor{currentfill}%
\pgfsetlinewidth{1.003750pt}%
\definecolor{currentstroke}{rgb}{0.200000,0.800000,0.200000}%
\pgfsetstrokecolor{currentstroke}%
\pgfsetdash{}{0pt}%
\pgfpathmoveto{\pgfqpoint{4.471959in}{5.353465in}}%
\pgfpathcurveto{\pgfqpoint{4.477783in}{5.353465in}}{\pgfqpoint{4.483369in}{5.355778in}}{\pgfqpoint{4.487487in}{5.359897in}}%
\pgfpathcurveto{\pgfqpoint{4.491606in}{5.364015in}}{\pgfqpoint{4.493919in}{5.369601in}}{\pgfqpoint{4.493919in}{5.375425in}}%
\pgfpathcurveto{\pgfqpoint{4.493919in}{5.381249in}}{\pgfqpoint{4.491606in}{5.386835in}}{\pgfqpoint{4.487487in}{5.390953in}}%
\pgfpathcurveto{\pgfqpoint{4.483369in}{5.395071in}}{\pgfqpoint{4.477783in}{5.397385in}}{\pgfqpoint{4.471959in}{5.397385in}}%
\pgfpathcurveto{\pgfqpoint{4.466135in}{5.397385in}}{\pgfqpoint{4.460549in}{5.395071in}}{\pgfqpoint{4.456431in}{5.390953in}}%
\pgfpathcurveto{\pgfqpoint{4.452313in}{5.386835in}}{\pgfqpoint{4.449999in}{5.381249in}}{\pgfqpoint{4.449999in}{5.375425in}}%
\pgfpathcurveto{\pgfqpoint{4.449999in}{5.369601in}}{\pgfqpoint{4.452313in}{5.364015in}}{\pgfqpoint{4.456431in}{5.359897in}}%
\pgfpathcurveto{\pgfqpoint{4.460549in}{5.355778in}}{\pgfqpoint{4.466135in}{5.353465in}}{\pgfqpoint{4.471959in}{5.353465in}}%
\pgfpathlineto{\pgfqpoint{4.471959in}{5.353465in}}%
\pgfpathclose%
\pgfusepath{stroke,fill}%
\end{pgfscope}%
\begin{pgfscope}%
\pgfpathrectangle{\pgfqpoint{1.073501in}{0.880000in}}{\pgfqpoint{6.052998in}{6.160000in}}%
\pgfusepath{clip}%
\pgfsetbuttcap%
\pgfsetroundjoin%
\definecolor{currentfill}{rgb}{0.200000,0.800000,0.200000}%
\pgfsetfillcolor{currentfill}%
\pgfsetlinewidth{1.003750pt}%
\definecolor{currentstroke}{rgb}{0.200000,0.800000,0.200000}%
\pgfsetstrokecolor{currentstroke}%
\pgfsetdash{}{0pt}%
\pgfpathmoveto{\pgfqpoint{4.364267in}{5.279285in}}%
\pgfpathcurveto{\pgfqpoint{4.370091in}{5.279285in}}{\pgfqpoint{4.375677in}{5.281599in}}{\pgfqpoint{4.379795in}{5.285717in}}%
\pgfpathcurveto{\pgfqpoint{4.383913in}{5.289835in}}{\pgfqpoint{4.386227in}{5.295421in}}{\pgfqpoint{4.386227in}{5.301245in}}%
\pgfpathcurveto{\pgfqpoint{4.386227in}{5.307069in}}{\pgfqpoint{4.383913in}{5.312655in}}{\pgfqpoint{4.379795in}{5.316773in}}%
\pgfpathcurveto{\pgfqpoint{4.375677in}{5.320891in}}{\pgfqpoint{4.370091in}{5.323205in}}{\pgfqpoint{4.364267in}{5.323205in}}%
\pgfpathcurveto{\pgfqpoint{4.358443in}{5.323205in}}{\pgfqpoint{4.352857in}{5.320891in}}{\pgfqpoint{4.348738in}{5.316773in}}%
\pgfpathcurveto{\pgfqpoint{4.344620in}{5.312655in}}{\pgfqpoint{4.342306in}{5.307069in}}{\pgfqpoint{4.342306in}{5.301245in}}%
\pgfpathcurveto{\pgfqpoint{4.342306in}{5.295421in}}{\pgfqpoint{4.344620in}{5.289835in}}{\pgfqpoint{4.348738in}{5.285717in}}%
\pgfpathcurveto{\pgfqpoint{4.352857in}{5.281599in}}{\pgfqpoint{4.358443in}{5.279285in}}{\pgfqpoint{4.364267in}{5.279285in}}%
\pgfpathlineto{\pgfqpoint{4.364267in}{5.279285in}}%
\pgfpathclose%
\pgfusepath{stroke,fill}%
\end{pgfscope}%
\begin{pgfscope}%
\pgfpathrectangle{\pgfqpoint{1.073501in}{0.880000in}}{\pgfqpoint{6.052998in}{6.160000in}}%
\pgfusepath{clip}%
\pgfsetbuttcap%
\pgfsetroundjoin%
\definecolor{currentfill}{rgb}{0.200000,0.800000,0.200000}%
\pgfsetfillcolor{currentfill}%
\pgfsetlinewidth{1.003750pt}%
\definecolor{currentstroke}{rgb}{0.200000,0.800000,0.200000}%
\pgfsetstrokecolor{currentstroke}%
\pgfsetdash{}{0pt}%
\pgfpathmoveto{\pgfqpoint{4.275815in}{5.272619in}}%
\pgfpathcurveto{\pgfqpoint{4.281639in}{5.272619in}}{\pgfqpoint{4.287225in}{5.274933in}}{\pgfqpoint{4.291343in}{5.279051in}}%
\pgfpathcurveto{\pgfqpoint{4.295461in}{5.283169in}}{\pgfqpoint{4.297775in}{5.288755in}}{\pgfqpoint{4.297775in}{5.294579in}}%
\pgfpathcurveto{\pgfqpoint{4.297775in}{5.300403in}}{\pgfqpoint{4.295461in}{5.305989in}}{\pgfqpoint{4.291343in}{5.310107in}}%
\pgfpathcurveto{\pgfqpoint{4.287225in}{5.314226in}}{\pgfqpoint{4.281639in}{5.316539in}}{\pgfqpoint{4.275815in}{5.316539in}}%
\pgfpathcurveto{\pgfqpoint{4.269991in}{5.316539in}}{\pgfqpoint{4.264404in}{5.314226in}}{\pgfqpoint{4.260286in}{5.310107in}}%
\pgfpathcurveto{\pgfqpoint{4.256168in}{5.305989in}}{\pgfqpoint{4.253854in}{5.300403in}}{\pgfqpoint{4.253854in}{5.294579in}}%
\pgfpathcurveto{\pgfqpoint{4.253854in}{5.288755in}}{\pgfqpoint{4.256168in}{5.283169in}}{\pgfqpoint{4.260286in}{5.279051in}}%
\pgfpathcurveto{\pgfqpoint{4.264404in}{5.274933in}}{\pgfqpoint{4.269991in}{5.272619in}}{\pgfqpoint{4.275815in}{5.272619in}}%
\pgfpathlineto{\pgfqpoint{4.275815in}{5.272619in}}%
\pgfpathclose%
\pgfusepath{stroke,fill}%
\end{pgfscope}%
\begin{pgfscope}%
\pgfpathrectangle{\pgfqpoint{1.073501in}{0.880000in}}{\pgfqpoint{6.052998in}{6.160000in}}%
\pgfusepath{clip}%
\pgfsetbuttcap%
\pgfsetroundjoin%
\definecolor{currentfill}{rgb}{0.200000,0.800000,0.200000}%
\pgfsetfillcolor{currentfill}%
\pgfsetlinewidth{1.003750pt}%
\definecolor{currentstroke}{rgb}{0.200000,0.800000,0.200000}%
\pgfsetstrokecolor{currentstroke}%
\pgfsetdash{}{0pt}%
\pgfpathmoveto{\pgfqpoint{4.197675in}{5.438645in}}%
\pgfpathcurveto{\pgfqpoint{4.203499in}{5.438645in}}{\pgfqpoint{4.209085in}{5.440959in}}{\pgfqpoint{4.213203in}{5.445077in}}%
\pgfpathcurveto{\pgfqpoint{4.217322in}{5.449195in}}{\pgfqpoint{4.219635in}{5.454781in}}{\pgfqpoint{4.219635in}{5.460605in}}%
\pgfpathcurveto{\pgfqpoint{4.219635in}{5.466429in}}{\pgfqpoint{4.217322in}{5.472015in}}{\pgfqpoint{4.213203in}{5.476133in}}%
\pgfpathcurveto{\pgfqpoint{4.209085in}{5.480251in}}{\pgfqpoint{4.203499in}{5.482565in}}{\pgfqpoint{4.197675in}{5.482565in}}%
\pgfpathcurveto{\pgfqpoint{4.191851in}{5.482565in}}{\pgfqpoint{4.186265in}{5.480251in}}{\pgfqpoint{4.182147in}{5.476133in}}%
\pgfpathcurveto{\pgfqpoint{4.178029in}{5.472015in}}{\pgfqpoint{4.175715in}{5.466429in}}{\pgfqpoint{4.175715in}{5.460605in}}%
\pgfpathcurveto{\pgfqpoint{4.175715in}{5.454781in}}{\pgfqpoint{4.178029in}{5.449195in}}{\pgfqpoint{4.182147in}{5.445077in}}%
\pgfpathcurveto{\pgfqpoint{4.186265in}{5.440959in}}{\pgfqpoint{4.191851in}{5.438645in}}{\pgfqpoint{4.197675in}{5.438645in}}%
\pgfpathlineto{\pgfqpoint{4.197675in}{5.438645in}}%
\pgfpathclose%
\pgfusepath{stroke,fill}%
\end{pgfscope}%
\begin{pgfscope}%
\pgfpathrectangle{\pgfqpoint{1.073501in}{0.880000in}}{\pgfqpoint{6.052998in}{6.160000in}}%
\pgfusepath{clip}%
\pgfsetbuttcap%
\pgfsetroundjoin%
\definecolor{currentfill}{rgb}{0.200000,0.800000,0.200000}%
\pgfsetfillcolor{currentfill}%
\pgfsetlinewidth{1.003750pt}%
\definecolor{currentstroke}{rgb}{0.200000,0.800000,0.200000}%
\pgfsetstrokecolor{currentstroke}%
\pgfsetdash{}{0pt}%
\pgfpathmoveto{\pgfqpoint{4.101927in}{5.418250in}}%
\pgfpathcurveto{\pgfqpoint{4.107751in}{5.418250in}}{\pgfqpoint{4.113337in}{5.420564in}}{\pgfqpoint{4.117455in}{5.424682in}}%
\pgfpathcurveto{\pgfqpoint{4.121573in}{5.428800in}}{\pgfqpoint{4.123887in}{5.434386in}}{\pgfqpoint{4.123887in}{5.440210in}}%
\pgfpathcurveto{\pgfqpoint{4.123887in}{5.446034in}}{\pgfqpoint{4.121573in}{5.451620in}}{\pgfqpoint{4.117455in}{5.455739in}}%
\pgfpathcurveto{\pgfqpoint{4.113337in}{5.459857in}}{\pgfqpoint{4.107751in}{5.462171in}}{\pgfqpoint{4.101927in}{5.462171in}}%
\pgfpathcurveto{\pgfqpoint{4.096103in}{5.462171in}}{\pgfqpoint{4.090517in}{5.459857in}}{\pgfqpoint{4.086398in}{5.455739in}}%
\pgfpathcurveto{\pgfqpoint{4.082280in}{5.451620in}}{\pgfqpoint{4.079966in}{5.446034in}}{\pgfqpoint{4.079966in}{5.440210in}}%
\pgfpathcurveto{\pgfqpoint{4.079966in}{5.434386in}}{\pgfqpoint{4.082280in}{5.428800in}}{\pgfqpoint{4.086398in}{5.424682in}}%
\pgfpathcurveto{\pgfqpoint{4.090517in}{5.420564in}}{\pgfqpoint{4.096103in}{5.418250in}}{\pgfqpoint{4.101927in}{5.418250in}}%
\pgfpathlineto{\pgfqpoint{4.101927in}{5.418250in}}%
\pgfpathclose%
\pgfusepath{stroke,fill}%
\end{pgfscope}%
\begin{pgfscope}%
\pgfpathrectangle{\pgfqpoint{1.073501in}{0.880000in}}{\pgfqpoint{6.052998in}{6.160000in}}%
\pgfusepath{clip}%
\pgfsetbuttcap%
\pgfsetroundjoin%
\definecolor{currentfill}{rgb}{0.200000,0.800000,0.200000}%
\pgfsetfillcolor{currentfill}%
\pgfsetlinewidth{1.003750pt}%
\definecolor{currentstroke}{rgb}{0.200000,0.800000,0.200000}%
\pgfsetstrokecolor{currentstroke}%
\pgfsetdash{}{0pt}%
\pgfpathmoveto{\pgfqpoint{4.009360in}{5.388044in}}%
\pgfpathcurveto{\pgfqpoint{4.015184in}{5.388044in}}{\pgfqpoint{4.020770in}{5.390357in}}{\pgfqpoint{4.024888in}{5.394476in}}%
\pgfpathcurveto{\pgfqpoint{4.029006in}{5.398594in}}{\pgfqpoint{4.031320in}{5.404180in}}{\pgfqpoint{4.031320in}{5.410004in}}%
\pgfpathcurveto{\pgfqpoint{4.031320in}{5.415828in}}{\pgfqpoint{4.029006in}{5.421414in}}{\pgfqpoint{4.024888in}{5.425532in}}%
\pgfpathcurveto{\pgfqpoint{4.020770in}{5.429650in}}{\pgfqpoint{4.015184in}{5.431964in}}{\pgfqpoint{4.009360in}{5.431964in}}%
\pgfpathcurveto{\pgfqpoint{4.003536in}{5.431964in}}{\pgfqpoint{3.997950in}{5.429650in}}{\pgfqpoint{3.993832in}{5.425532in}}%
\pgfpathcurveto{\pgfqpoint{3.989713in}{5.421414in}}{\pgfqpoint{3.987400in}{5.415828in}}{\pgfqpoint{3.987400in}{5.410004in}}%
\pgfpathcurveto{\pgfqpoint{3.987400in}{5.404180in}}{\pgfqpoint{3.989713in}{5.398594in}}{\pgfqpoint{3.993832in}{5.394476in}}%
\pgfpathcurveto{\pgfqpoint{3.997950in}{5.390357in}}{\pgfqpoint{4.003536in}{5.388044in}}{\pgfqpoint{4.009360in}{5.388044in}}%
\pgfpathlineto{\pgfqpoint{4.009360in}{5.388044in}}%
\pgfpathclose%
\pgfusepath{stroke,fill}%
\end{pgfscope}%
\begin{pgfscope}%
\pgfpathrectangle{\pgfqpoint{1.073501in}{0.880000in}}{\pgfqpoint{6.052998in}{6.160000in}}%
\pgfusepath{clip}%
\pgfsetbuttcap%
\pgfsetroundjoin%
\definecolor{currentfill}{rgb}{0.200000,0.800000,0.200000}%
\pgfsetfillcolor{currentfill}%
\pgfsetlinewidth{1.003750pt}%
\definecolor{currentstroke}{rgb}{0.200000,0.800000,0.200000}%
\pgfsetstrokecolor{currentstroke}%
\pgfsetdash{}{0pt}%
\pgfpathmoveto{\pgfqpoint{3.921671in}{5.344207in}}%
\pgfpathcurveto{\pgfqpoint{3.927495in}{5.344207in}}{\pgfqpoint{3.933082in}{5.346520in}}{\pgfqpoint{3.937200in}{5.350639in}}%
\pgfpathcurveto{\pgfqpoint{3.941318in}{5.354757in}}{\pgfqpoint{3.943632in}{5.360343in}}{\pgfqpoint{3.943632in}{5.366167in}}%
\pgfpathcurveto{\pgfqpoint{3.943632in}{5.371991in}}{\pgfqpoint{3.941318in}{5.377577in}}{\pgfqpoint{3.937200in}{5.381695in}}%
\pgfpathcurveto{\pgfqpoint{3.933082in}{5.385813in}}{\pgfqpoint{3.927495in}{5.388127in}}{\pgfqpoint{3.921671in}{5.388127in}}%
\pgfpathcurveto{\pgfqpoint{3.915847in}{5.388127in}}{\pgfqpoint{3.910261in}{5.385813in}}{\pgfqpoint{3.906143in}{5.381695in}}%
\pgfpathcurveto{\pgfqpoint{3.902025in}{5.377577in}}{\pgfqpoint{3.899711in}{5.371991in}}{\pgfqpoint{3.899711in}{5.366167in}}%
\pgfpathcurveto{\pgfqpoint{3.899711in}{5.360343in}}{\pgfqpoint{3.902025in}{5.354757in}}{\pgfqpoint{3.906143in}{5.350639in}}%
\pgfpathcurveto{\pgfqpoint{3.910261in}{5.346520in}}{\pgfqpoint{3.915847in}{5.344207in}}{\pgfqpoint{3.921671in}{5.344207in}}%
\pgfpathlineto{\pgfqpoint{3.921671in}{5.344207in}}%
\pgfpathclose%
\pgfusepath{stroke,fill}%
\end{pgfscope}%
\begin{pgfscope}%
\pgfpathrectangle{\pgfqpoint{1.073501in}{0.880000in}}{\pgfqpoint{6.052998in}{6.160000in}}%
\pgfusepath{clip}%
\pgfsetbuttcap%
\pgfsetroundjoin%
\definecolor{currentfill}{rgb}{0.200000,0.800000,0.200000}%
\pgfsetfillcolor{currentfill}%
\pgfsetlinewidth{1.003750pt}%
\definecolor{currentstroke}{rgb}{0.200000,0.800000,0.200000}%
\pgfsetstrokecolor{currentstroke}%
\pgfsetdash{}{0pt}%
\pgfpathmoveto{\pgfqpoint{3.845281in}{5.265518in}}%
\pgfpathcurveto{\pgfqpoint{3.851105in}{5.265518in}}{\pgfqpoint{3.856692in}{5.267831in}}{\pgfqpoint{3.860810in}{5.271950in}}%
\pgfpathcurveto{\pgfqpoint{3.864928in}{5.276068in}}{\pgfqpoint{3.867242in}{5.281654in}}{\pgfqpoint{3.867242in}{5.287478in}}%
\pgfpathcurveto{\pgfqpoint{3.867242in}{5.293302in}}{\pgfqpoint{3.864928in}{5.298888in}}{\pgfqpoint{3.860810in}{5.303006in}}%
\pgfpathcurveto{\pgfqpoint{3.856692in}{5.307124in}}{\pgfqpoint{3.851105in}{5.309438in}}{\pgfqpoint{3.845281in}{5.309438in}}%
\pgfpathcurveto{\pgfqpoint{3.839458in}{5.309438in}}{\pgfqpoint{3.833871in}{5.307124in}}{\pgfqpoint{3.829753in}{5.303006in}}%
\pgfpathcurveto{\pgfqpoint{3.825635in}{5.298888in}}{\pgfqpoint{3.823321in}{5.293302in}}{\pgfqpoint{3.823321in}{5.287478in}}%
\pgfpathcurveto{\pgfqpoint{3.823321in}{5.281654in}}{\pgfqpoint{3.825635in}{5.276068in}}{\pgfqpoint{3.829753in}{5.271950in}}%
\pgfpathcurveto{\pgfqpoint{3.833871in}{5.267831in}}{\pgfqpoint{3.839458in}{5.265518in}}{\pgfqpoint{3.845281in}{5.265518in}}%
\pgfpathlineto{\pgfqpoint{3.845281in}{5.265518in}}%
\pgfpathclose%
\pgfusepath{stroke,fill}%
\end{pgfscope}%
\begin{pgfscope}%
\pgfpathrectangle{\pgfqpoint{1.073501in}{0.880000in}}{\pgfqpoint{6.052998in}{6.160000in}}%
\pgfusepath{clip}%
\pgfsetbuttcap%
\pgfsetroundjoin%
\definecolor{currentfill}{rgb}{0.200000,0.800000,0.200000}%
\pgfsetfillcolor{currentfill}%
\pgfsetlinewidth{1.003750pt}%
\definecolor{currentstroke}{rgb}{0.200000,0.800000,0.200000}%
\pgfsetstrokecolor{currentstroke}%
\pgfsetdash{}{0pt}%
\pgfpathmoveto{\pgfqpoint{3.737557in}{5.329381in}}%
\pgfpathcurveto{\pgfqpoint{3.743381in}{5.329381in}}{\pgfqpoint{3.748968in}{5.331695in}}{\pgfqpoint{3.753086in}{5.335813in}}%
\pgfpathcurveto{\pgfqpoint{3.757204in}{5.339931in}}{\pgfqpoint{3.759518in}{5.345517in}}{\pgfqpoint{3.759518in}{5.351341in}}%
\pgfpathcurveto{\pgfqpoint{3.759518in}{5.357165in}}{\pgfqpoint{3.757204in}{5.362752in}}{\pgfqpoint{3.753086in}{5.366870in}}%
\pgfpathcurveto{\pgfqpoint{3.748968in}{5.370988in}}{\pgfqpoint{3.743381in}{5.373302in}}{\pgfqpoint{3.737557in}{5.373302in}}%
\pgfpathcurveto{\pgfqpoint{3.731734in}{5.373302in}}{\pgfqpoint{3.726147in}{5.370988in}}{\pgfqpoint{3.722029in}{5.366870in}}%
\pgfpathcurveto{\pgfqpoint{3.717911in}{5.362752in}}{\pgfqpoint{3.715597in}{5.357165in}}{\pgfqpoint{3.715597in}{5.351341in}}%
\pgfpathcurveto{\pgfqpoint{3.715597in}{5.345517in}}{\pgfqpoint{3.717911in}{5.339931in}}{\pgfqpoint{3.722029in}{5.335813in}}%
\pgfpathcurveto{\pgfqpoint{3.726147in}{5.331695in}}{\pgfqpoint{3.731734in}{5.329381in}}{\pgfqpoint{3.737557in}{5.329381in}}%
\pgfpathlineto{\pgfqpoint{3.737557in}{5.329381in}}%
\pgfpathclose%
\pgfusepath{stroke,fill}%
\end{pgfscope}%
\begin{pgfscope}%
\pgfpathrectangle{\pgfqpoint{1.073501in}{0.880000in}}{\pgfqpoint{6.052998in}{6.160000in}}%
\pgfusepath{clip}%
\pgfsetbuttcap%
\pgfsetroundjoin%
\definecolor{currentfill}{rgb}{0.200000,0.800000,0.200000}%
\pgfsetfillcolor{currentfill}%
\pgfsetlinewidth{1.003750pt}%
\definecolor{currentstroke}{rgb}{0.200000,0.800000,0.200000}%
\pgfsetstrokecolor{currentstroke}%
\pgfsetdash{}{0pt}%
\pgfpathmoveto{\pgfqpoint{3.676156in}{5.224400in}}%
\pgfpathcurveto{\pgfqpoint{3.681980in}{5.224400in}}{\pgfqpoint{3.687566in}{5.226714in}}{\pgfqpoint{3.691684in}{5.230832in}}%
\pgfpathcurveto{\pgfqpoint{3.695802in}{5.234950in}}{\pgfqpoint{3.698116in}{5.240536in}}{\pgfqpoint{3.698116in}{5.246360in}}%
\pgfpathcurveto{\pgfqpoint{3.698116in}{5.252184in}}{\pgfqpoint{3.695802in}{5.257770in}}{\pgfqpoint{3.691684in}{5.261888in}}%
\pgfpathcurveto{\pgfqpoint{3.687566in}{5.266006in}}{\pgfqpoint{3.681980in}{5.268320in}}{\pgfqpoint{3.676156in}{5.268320in}}%
\pgfpathcurveto{\pgfqpoint{3.670332in}{5.268320in}}{\pgfqpoint{3.664745in}{5.266006in}}{\pgfqpoint{3.660627in}{5.261888in}}%
\pgfpathcurveto{\pgfqpoint{3.656509in}{5.257770in}}{\pgfqpoint{3.654195in}{5.252184in}}{\pgfqpoint{3.654195in}{5.246360in}}%
\pgfpathcurveto{\pgfqpoint{3.654195in}{5.240536in}}{\pgfqpoint{3.656509in}{5.234950in}}{\pgfqpoint{3.660627in}{5.230832in}}%
\pgfpathcurveto{\pgfqpoint{3.664745in}{5.226714in}}{\pgfqpoint{3.670332in}{5.224400in}}{\pgfqpoint{3.676156in}{5.224400in}}%
\pgfpathlineto{\pgfqpoint{3.676156in}{5.224400in}}%
\pgfpathclose%
\pgfusepath{stroke,fill}%
\end{pgfscope}%
\begin{pgfscope}%
\pgfpathrectangle{\pgfqpoint{1.073501in}{0.880000in}}{\pgfqpoint{6.052998in}{6.160000in}}%
\pgfusepath{clip}%
\pgfsetbuttcap%
\pgfsetroundjoin%
\definecolor{currentfill}{rgb}{0.200000,0.800000,0.200000}%
\pgfsetfillcolor{currentfill}%
\pgfsetlinewidth{1.003750pt}%
\definecolor{currentstroke}{rgb}{0.200000,0.800000,0.200000}%
\pgfsetstrokecolor{currentstroke}%
\pgfsetdash{}{0pt}%
\pgfpathmoveto{\pgfqpoint{3.549760in}{5.300562in}}%
\pgfpathcurveto{\pgfqpoint{3.555584in}{5.300562in}}{\pgfqpoint{3.561170in}{5.302876in}}{\pgfqpoint{3.565288in}{5.306994in}}%
\pgfpathcurveto{\pgfqpoint{3.569407in}{5.311112in}}{\pgfqpoint{3.571720in}{5.316698in}}{\pgfqpoint{3.571720in}{5.322522in}}%
\pgfpathcurveto{\pgfqpoint{3.571720in}{5.328346in}}{\pgfqpoint{3.569407in}{5.333932in}}{\pgfqpoint{3.565288in}{5.338050in}}%
\pgfpathcurveto{\pgfqpoint{3.561170in}{5.342169in}}{\pgfqpoint{3.555584in}{5.344482in}}{\pgfqpoint{3.549760in}{5.344482in}}%
\pgfpathcurveto{\pgfqpoint{3.543936in}{5.344482in}}{\pgfqpoint{3.538350in}{5.342169in}}{\pgfqpoint{3.534232in}{5.338050in}}%
\pgfpathcurveto{\pgfqpoint{3.530114in}{5.333932in}}{\pgfqpoint{3.527800in}{5.328346in}}{\pgfqpoint{3.527800in}{5.322522in}}%
\pgfpathcurveto{\pgfqpoint{3.527800in}{5.316698in}}{\pgfqpoint{3.530114in}{5.311112in}}{\pgfqpoint{3.534232in}{5.306994in}}%
\pgfpathcurveto{\pgfqpoint{3.538350in}{5.302876in}}{\pgfqpoint{3.543936in}{5.300562in}}{\pgfqpoint{3.549760in}{5.300562in}}%
\pgfpathlineto{\pgfqpoint{3.549760in}{5.300562in}}%
\pgfpathclose%
\pgfusepath{stroke,fill}%
\end{pgfscope}%
\begin{pgfscope}%
\pgfpathrectangle{\pgfqpoint{1.073501in}{0.880000in}}{\pgfqpoint{6.052998in}{6.160000in}}%
\pgfusepath{clip}%
\pgfsetbuttcap%
\pgfsetroundjoin%
\definecolor{currentfill}{rgb}{0.200000,0.800000,0.200000}%
\pgfsetfillcolor{currentfill}%
\pgfsetlinewidth{1.003750pt}%
\definecolor{currentstroke}{rgb}{0.200000,0.800000,0.200000}%
\pgfsetstrokecolor{currentstroke}%
\pgfsetdash{}{0pt}%
\pgfpathmoveto{\pgfqpoint{3.524525in}{5.138598in}}%
\pgfpathcurveto{\pgfqpoint{3.530348in}{5.138598in}}{\pgfqpoint{3.535935in}{5.140912in}}{\pgfqpoint{3.540053in}{5.145030in}}%
\pgfpathcurveto{\pgfqpoint{3.544171in}{5.149149in}}{\pgfqpoint{3.546485in}{5.154735in}}{\pgfqpoint{3.546485in}{5.160559in}}%
\pgfpathcurveto{\pgfqpoint{3.546485in}{5.166383in}}{\pgfqpoint{3.544171in}{5.171969in}}{\pgfqpoint{3.540053in}{5.176087in}}%
\pgfpathcurveto{\pgfqpoint{3.535935in}{5.180205in}}{\pgfqpoint{3.530348in}{5.182519in}}{\pgfqpoint{3.524525in}{5.182519in}}%
\pgfpathcurveto{\pgfqpoint{3.518701in}{5.182519in}}{\pgfqpoint{3.513114in}{5.180205in}}{\pgfqpoint{3.508996in}{5.176087in}}%
\pgfpathcurveto{\pgfqpoint{3.504878in}{5.171969in}}{\pgfqpoint{3.502564in}{5.166383in}}{\pgfqpoint{3.502564in}{5.160559in}}%
\pgfpathcurveto{\pgfqpoint{3.502564in}{5.154735in}}{\pgfqpoint{3.504878in}{5.149149in}}{\pgfqpoint{3.508996in}{5.145030in}}%
\pgfpathcurveto{\pgfqpoint{3.513114in}{5.140912in}}{\pgfqpoint{3.518701in}{5.138598in}}{\pgfqpoint{3.524525in}{5.138598in}}%
\pgfpathlineto{\pgfqpoint{3.524525in}{5.138598in}}%
\pgfpathclose%
\pgfusepath{stroke,fill}%
\end{pgfscope}%
\begin{pgfscope}%
\pgfpathrectangle{\pgfqpoint{1.073501in}{0.880000in}}{\pgfqpoint{6.052998in}{6.160000in}}%
\pgfusepath{clip}%
\pgfsetbuttcap%
\pgfsetroundjoin%
\definecolor{currentfill}{rgb}{0.200000,0.800000,0.200000}%
\pgfsetfillcolor{currentfill}%
\pgfsetlinewidth{1.003750pt}%
\definecolor{currentstroke}{rgb}{0.200000,0.800000,0.200000}%
\pgfsetstrokecolor{currentstroke}%
\pgfsetdash{}{0pt}%
\pgfpathmoveto{\pgfqpoint{3.411255in}{5.163088in}}%
\pgfpathcurveto{\pgfqpoint{3.417078in}{5.163088in}}{\pgfqpoint{3.422665in}{5.165402in}}{\pgfqpoint{3.426783in}{5.169520in}}%
\pgfpathcurveto{\pgfqpoint{3.430901in}{5.173638in}}{\pgfqpoint{3.433215in}{5.179224in}}{\pgfqpoint{3.433215in}{5.185048in}}%
\pgfpathcurveto{\pgfqpoint{3.433215in}{5.190872in}}{\pgfqpoint{3.430901in}{5.196458in}}{\pgfqpoint{3.426783in}{5.200576in}}%
\pgfpathcurveto{\pgfqpoint{3.422665in}{5.204694in}}{\pgfqpoint{3.417078in}{5.207008in}}{\pgfqpoint{3.411255in}{5.207008in}}%
\pgfpathcurveto{\pgfqpoint{3.405431in}{5.207008in}}{\pgfqpoint{3.399844in}{5.204694in}}{\pgfqpoint{3.395726in}{5.200576in}}%
\pgfpathcurveto{\pgfqpoint{3.391608in}{5.196458in}}{\pgfqpoint{3.389294in}{5.190872in}}{\pgfqpoint{3.389294in}{5.185048in}}%
\pgfpathcurveto{\pgfqpoint{3.389294in}{5.179224in}}{\pgfqpoint{3.391608in}{5.173638in}}{\pgfqpoint{3.395726in}{5.169520in}}%
\pgfpathcurveto{\pgfqpoint{3.399844in}{5.165402in}}{\pgfqpoint{3.405431in}{5.163088in}}{\pgfqpoint{3.411255in}{5.163088in}}%
\pgfpathlineto{\pgfqpoint{3.411255in}{5.163088in}}%
\pgfpathclose%
\pgfusepath{stroke,fill}%
\end{pgfscope}%
\begin{pgfscope}%
\pgfpathrectangle{\pgfqpoint{1.073501in}{0.880000in}}{\pgfqpoint{6.052998in}{6.160000in}}%
\pgfusepath{clip}%
\pgfsetbuttcap%
\pgfsetroundjoin%
\definecolor{currentfill}{rgb}{0.200000,0.800000,0.200000}%
\pgfsetfillcolor{currentfill}%
\pgfsetlinewidth{1.003750pt}%
\definecolor{currentstroke}{rgb}{0.200000,0.800000,0.200000}%
\pgfsetstrokecolor{currentstroke}%
\pgfsetdash{}{0pt}%
\pgfpathmoveto{\pgfqpoint{3.312837in}{5.147427in}}%
\pgfpathcurveto{\pgfqpoint{3.318661in}{5.147427in}}{\pgfqpoint{3.324247in}{5.149741in}}{\pgfqpoint{3.328365in}{5.153859in}}%
\pgfpathcurveto{\pgfqpoint{3.332483in}{5.157977in}}{\pgfqpoint{3.334797in}{5.163564in}}{\pgfqpoint{3.334797in}{5.169388in}}%
\pgfpathcurveto{\pgfqpoint{3.334797in}{5.175211in}}{\pgfqpoint{3.332483in}{5.180798in}}{\pgfqpoint{3.328365in}{5.184916in}}%
\pgfpathcurveto{\pgfqpoint{3.324247in}{5.189034in}}{\pgfqpoint{3.318661in}{5.191348in}}{\pgfqpoint{3.312837in}{5.191348in}}%
\pgfpathcurveto{\pgfqpoint{3.307013in}{5.191348in}}{\pgfqpoint{3.301427in}{5.189034in}}{\pgfqpoint{3.297309in}{5.184916in}}%
\pgfpathcurveto{\pgfqpoint{3.293190in}{5.180798in}}{\pgfqpoint{3.290876in}{5.175211in}}{\pgfqpoint{3.290876in}{5.169388in}}%
\pgfpathcurveto{\pgfqpoint{3.290876in}{5.163564in}}{\pgfqpoint{3.293190in}{5.157977in}}{\pgfqpoint{3.297309in}{5.153859in}}%
\pgfpathcurveto{\pgfqpoint{3.301427in}{5.149741in}}{\pgfqpoint{3.307013in}{5.147427in}}{\pgfqpoint{3.312837in}{5.147427in}}%
\pgfpathlineto{\pgfqpoint{3.312837in}{5.147427in}}%
\pgfpathclose%
\pgfusepath{stroke,fill}%
\end{pgfscope}%
\begin{pgfscope}%
\pgfpathrectangle{\pgfqpoint{1.073501in}{0.880000in}}{\pgfqpoint{6.052998in}{6.160000in}}%
\pgfusepath{clip}%
\pgfsetbuttcap%
\pgfsetroundjoin%
\definecolor{currentfill}{rgb}{0.200000,0.800000,0.200000}%
\pgfsetfillcolor{currentfill}%
\pgfsetlinewidth{1.003750pt}%
\definecolor{currentstroke}{rgb}{0.200000,0.800000,0.200000}%
\pgfsetstrokecolor{currentstroke}%
\pgfsetdash{}{0pt}%
\pgfpathmoveto{\pgfqpoint{3.245289in}{5.082517in}}%
\pgfpathcurveto{\pgfqpoint{3.251113in}{5.082517in}}{\pgfqpoint{3.256699in}{5.084830in}}{\pgfqpoint{3.260817in}{5.088949in}}%
\pgfpathcurveto{\pgfqpoint{3.264935in}{5.093067in}}{\pgfqpoint{3.267249in}{5.098653in}}{\pgfqpoint{3.267249in}{5.104477in}}%
\pgfpathcurveto{\pgfqpoint{3.267249in}{5.110301in}}{\pgfqpoint{3.264935in}{5.115887in}}{\pgfqpoint{3.260817in}{5.120005in}}%
\pgfpathcurveto{\pgfqpoint{3.256699in}{5.124123in}}{\pgfqpoint{3.251113in}{5.126437in}}{\pgfqpoint{3.245289in}{5.126437in}}%
\pgfpathcurveto{\pgfqpoint{3.239465in}{5.126437in}}{\pgfqpoint{3.233879in}{5.124123in}}{\pgfqpoint{3.229761in}{5.120005in}}%
\pgfpathcurveto{\pgfqpoint{3.225643in}{5.115887in}}{\pgfqpoint{3.223329in}{5.110301in}}{\pgfqpoint{3.223329in}{5.104477in}}%
\pgfpathcurveto{\pgfqpoint{3.223329in}{5.098653in}}{\pgfqpoint{3.225643in}{5.093067in}}{\pgfqpoint{3.229761in}{5.088949in}}%
\pgfpathcurveto{\pgfqpoint{3.233879in}{5.084830in}}{\pgfqpoint{3.239465in}{5.082517in}}{\pgfqpoint{3.245289in}{5.082517in}}%
\pgfpathlineto{\pgfqpoint{3.245289in}{5.082517in}}%
\pgfpathclose%
\pgfusepath{stroke,fill}%
\end{pgfscope}%
\begin{pgfscope}%
\pgfpathrectangle{\pgfqpoint{1.073501in}{0.880000in}}{\pgfqpoint{6.052998in}{6.160000in}}%
\pgfusepath{clip}%
\pgfsetbuttcap%
\pgfsetroundjoin%
\definecolor{currentfill}{rgb}{0.200000,0.800000,0.200000}%
\pgfsetfillcolor{currentfill}%
\pgfsetlinewidth{1.003750pt}%
\definecolor{currentstroke}{rgb}{0.200000,0.800000,0.200000}%
\pgfsetstrokecolor{currentstroke}%
\pgfsetdash{}{0pt}%
\pgfpathmoveto{\pgfqpoint{3.152425in}{5.048903in}}%
\pgfpathcurveto{\pgfqpoint{3.158249in}{5.048903in}}{\pgfqpoint{3.163836in}{5.051217in}}{\pgfqpoint{3.167954in}{5.055335in}}%
\pgfpathcurveto{\pgfqpoint{3.172072in}{5.059453in}}{\pgfqpoint{3.174386in}{5.065040in}}{\pgfqpoint{3.174386in}{5.070863in}}%
\pgfpathcurveto{\pgfqpoint{3.174386in}{5.076687in}}{\pgfqpoint{3.172072in}{5.082274in}}{\pgfqpoint{3.167954in}{5.086392in}}%
\pgfpathcurveto{\pgfqpoint{3.163836in}{5.090510in}}{\pgfqpoint{3.158249in}{5.092824in}}{\pgfqpoint{3.152425in}{5.092824in}}%
\pgfpathcurveto{\pgfqpoint{3.146602in}{5.092824in}}{\pgfqpoint{3.141015in}{5.090510in}}{\pgfqpoint{3.136897in}{5.086392in}}%
\pgfpathcurveto{\pgfqpoint{3.132779in}{5.082274in}}{\pgfqpoint{3.130465in}{5.076687in}}{\pgfqpoint{3.130465in}{5.070863in}}%
\pgfpathcurveto{\pgfqpoint{3.130465in}{5.065040in}}{\pgfqpoint{3.132779in}{5.059453in}}{\pgfqpoint{3.136897in}{5.055335in}}%
\pgfpathcurveto{\pgfqpoint{3.141015in}{5.051217in}}{\pgfqpoint{3.146602in}{5.048903in}}{\pgfqpoint{3.152425in}{5.048903in}}%
\pgfpathlineto{\pgfqpoint{3.152425in}{5.048903in}}%
\pgfpathclose%
\pgfusepath{stroke,fill}%
\end{pgfscope}%
\begin{pgfscope}%
\pgfpathrectangle{\pgfqpoint{1.073501in}{0.880000in}}{\pgfqpoint{6.052998in}{6.160000in}}%
\pgfusepath{clip}%
\pgfsetbuttcap%
\pgfsetroundjoin%
\definecolor{currentfill}{rgb}{0.200000,0.800000,0.200000}%
\pgfsetfillcolor{currentfill}%
\pgfsetlinewidth{1.003750pt}%
\definecolor{currentstroke}{rgb}{0.200000,0.800000,0.200000}%
\pgfsetstrokecolor{currentstroke}%
\pgfsetdash{}{0pt}%
\pgfpathmoveto{\pgfqpoint{3.162504in}{4.904298in}}%
\pgfpathcurveto{\pgfqpoint{3.168328in}{4.904298in}}{\pgfqpoint{3.173914in}{4.906612in}}{\pgfqpoint{3.178032in}{4.910730in}}%
\pgfpathcurveto{\pgfqpoint{3.182150in}{4.914849in}}{\pgfqpoint{3.184464in}{4.920435in}}{\pgfqpoint{3.184464in}{4.926259in}}%
\pgfpathcurveto{\pgfqpoint{3.184464in}{4.932083in}}{\pgfqpoint{3.182150in}{4.937669in}}{\pgfqpoint{3.178032in}{4.941787in}}%
\pgfpathcurveto{\pgfqpoint{3.173914in}{4.945905in}}{\pgfqpoint{3.168328in}{4.948219in}}{\pgfqpoint{3.162504in}{4.948219in}}%
\pgfpathcurveto{\pgfqpoint{3.156680in}{4.948219in}}{\pgfqpoint{3.151094in}{4.945905in}}{\pgfqpoint{3.146976in}{4.941787in}}%
\pgfpathcurveto{\pgfqpoint{3.142857in}{4.937669in}}{\pgfqpoint{3.140544in}{4.932083in}}{\pgfqpoint{3.140544in}{4.926259in}}%
\pgfpathcurveto{\pgfqpoint{3.140544in}{4.920435in}}{\pgfqpoint{3.142857in}{4.914849in}}{\pgfqpoint{3.146976in}{4.910730in}}%
\pgfpathcurveto{\pgfqpoint{3.151094in}{4.906612in}}{\pgfqpoint{3.156680in}{4.904298in}}{\pgfqpoint{3.162504in}{4.904298in}}%
\pgfpathlineto{\pgfqpoint{3.162504in}{4.904298in}}%
\pgfpathclose%
\pgfusepath{stroke,fill}%
\end{pgfscope}%
\begin{pgfscope}%
\pgfpathrectangle{\pgfqpoint{1.073501in}{0.880000in}}{\pgfqpoint{6.052998in}{6.160000in}}%
\pgfusepath{clip}%
\pgfsetbuttcap%
\pgfsetroundjoin%
\definecolor{currentfill}{rgb}{0.200000,0.800000,0.200000}%
\pgfsetfillcolor{currentfill}%
\pgfsetlinewidth{1.003750pt}%
\definecolor{currentstroke}{rgb}{0.200000,0.800000,0.200000}%
\pgfsetstrokecolor{currentstroke}%
\pgfsetdash{}{0pt}%
\pgfpathmoveto{\pgfqpoint{3.034039in}{4.902389in}}%
\pgfpathcurveto{\pgfqpoint{3.039863in}{4.902389in}}{\pgfqpoint{3.045450in}{4.904703in}}{\pgfqpoint{3.049568in}{4.908821in}}%
\pgfpathcurveto{\pgfqpoint{3.053686in}{4.912939in}}{\pgfqpoint{3.056000in}{4.918525in}}{\pgfqpoint{3.056000in}{4.924349in}}%
\pgfpathcurveto{\pgfqpoint{3.056000in}{4.930173in}}{\pgfqpoint{3.053686in}{4.935759in}}{\pgfqpoint{3.049568in}{4.939877in}}%
\pgfpathcurveto{\pgfqpoint{3.045450in}{4.943995in}}{\pgfqpoint{3.039863in}{4.946309in}}{\pgfqpoint{3.034039in}{4.946309in}}%
\pgfpathcurveto{\pgfqpoint{3.028216in}{4.946309in}}{\pgfqpoint{3.022629in}{4.943995in}}{\pgfqpoint{3.018511in}{4.939877in}}%
\pgfpathcurveto{\pgfqpoint{3.014393in}{4.935759in}}{\pgfqpoint{3.012079in}{4.930173in}}{\pgfqpoint{3.012079in}{4.924349in}}%
\pgfpathcurveto{\pgfqpoint{3.012079in}{4.918525in}}{\pgfqpoint{3.014393in}{4.912939in}}{\pgfqpoint{3.018511in}{4.908821in}}%
\pgfpathcurveto{\pgfqpoint{3.022629in}{4.904703in}}{\pgfqpoint{3.028216in}{4.902389in}}{\pgfqpoint{3.034039in}{4.902389in}}%
\pgfpathlineto{\pgfqpoint{3.034039in}{4.902389in}}%
\pgfpathclose%
\pgfusepath{stroke,fill}%
\end{pgfscope}%
\begin{pgfscope}%
\pgfpathrectangle{\pgfqpoint{1.073501in}{0.880000in}}{\pgfqpoint{6.052998in}{6.160000in}}%
\pgfusepath{clip}%
\pgfsetbuttcap%
\pgfsetroundjoin%
\definecolor{currentfill}{rgb}{0.200000,0.800000,0.200000}%
\pgfsetfillcolor{currentfill}%
\pgfsetlinewidth{1.003750pt}%
\definecolor{currentstroke}{rgb}{0.200000,0.800000,0.200000}%
\pgfsetstrokecolor{currentstroke}%
\pgfsetdash{}{0pt}%
\pgfpathmoveto{\pgfqpoint{3.017579in}{4.797154in}}%
\pgfpathcurveto{\pgfqpoint{3.023403in}{4.797154in}}{\pgfqpoint{3.028989in}{4.799468in}}{\pgfqpoint{3.033107in}{4.803586in}}%
\pgfpathcurveto{\pgfqpoint{3.037225in}{4.807704in}}{\pgfqpoint{3.039539in}{4.813290in}}{\pgfqpoint{3.039539in}{4.819114in}}%
\pgfpathcurveto{\pgfqpoint{3.039539in}{4.824938in}}{\pgfqpoint{3.037225in}{4.830524in}}{\pgfqpoint{3.033107in}{4.834642in}}%
\pgfpathcurveto{\pgfqpoint{3.028989in}{4.838761in}}{\pgfqpoint{3.023403in}{4.841074in}}{\pgfqpoint{3.017579in}{4.841074in}}%
\pgfpathcurveto{\pgfqpoint{3.011755in}{4.841074in}}{\pgfqpoint{3.006169in}{4.838761in}}{\pgfqpoint{3.002051in}{4.834642in}}%
\pgfpathcurveto{\pgfqpoint{2.997933in}{4.830524in}}{\pgfqpoint{2.995619in}{4.824938in}}{\pgfqpoint{2.995619in}{4.819114in}}%
\pgfpathcurveto{\pgfqpoint{2.995619in}{4.813290in}}{\pgfqpoint{2.997933in}{4.807704in}}{\pgfqpoint{3.002051in}{4.803586in}}%
\pgfpathcurveto{\pgfqpoint{3.006169in}{4.799468in}}{\pgfqpoint{3.011755in}{4.797154in}}{\pgfqpoint{3.017579in}{4.797154in}}%
\pgfpathlineto{\pgfqpoint{3.017579in}{4.797154in}}%
\pgfpathclose%
\pgfusepath{stroke,fill}%
\end{pgfscope}%
\begin{pgfscope}%
\pgfpathrectangle{\pgfqpoint{1.073501in}{0.880000in}}{\pgfqpoint{6.052998in}{6.160000in}}%
\pgfusepath{clip}%
\pgfsetbuttcap%
\pgfsetroundjoin%
\definecolor{currentfill}{rgb}{0.200000,0.800000,0.200000}%
\pgfsetfillcolor{currentfill}%
\pgfsetlinewidth{1.003750pt}%
\definecolor{currentstroke}{rgb}{0.200000,0.800000,0.200000}%
\pgfsetstrokecolor{currentstroke}%
\pgfsetdash{}{0pt}%
\pgfpathmoveto{\pgfqpoint{2.941667in}{4.740874in}}%
\pgfpathcurveto{\pgfqpoint{2.947491in}{4.740874in}}{\pgfqpoint{2.953077in}{4.743188in}}{\pgfqpoint{2.957195in}{4.747306in}}%
\pgfpathcurveto{\pgfqpoint{2.961313in}{4.751424in}}{\pgfqpoint{2.963627in}{4.757011in}}{\pgfqpoint{2.963627in}{4.762835in}}%
\pgfpathcurveto{\pgfqpoint{2.963627in}{4.768658in}}{\pgfqpoint{2.961313in}{4.774245in}}{\pgfqpoint{2.957195in}{4.778363in}}%
\pgfpathcurveto{\pgfqpoint{2.953077in}{4.782481in}}{\pgfqpoint{2.947491in}{4.784795in}}{\pgfqpoint{2.941667in}{4.784795in}}%
\pgfpathcurveto{\pgfqpoint{2.935843in}{4.784795in}}{\pgfqpoint{2.930257in}{4.782481in}}{\pgfqpoint{2.926138in}{4.778363in}}%
\pgfpathcurveto{\pgfqpoint{2.922020in}{4.774245in}}{\pgfqpoint{2.919706in}{4.768658in}}{\pgfqpoint{2.919706in}{4.762835in}}%
\pgfpathcurveto{\pgfqpoint{2.919706in}{4.757011in}}{\pgfqpoint{2.922020in}{4.751424in}}{\pgfqpoint{2.926138in}{4.747306in}}%
\pgfpathcurveto{\pgfqpoint{2.930257in}{4.743188in}}{\pgfqpoint{2.935843in}{4.740874in}}{\pgfqpoint{2.941667in}{4.740874in}}%
\pgfpathlineto{\pgfqpoint{2.941667in}{4.740874in}}%
\pgfpathclose%
\pgfusepath{stroke,fill}%
\end{pgfscope}%
\begin{pgfscope}%
\pgfpathrectangle{\pgfqpoint{1.073501in}{0.880000in}}{\pgfqpoint{6.052998in}{6.160000in}}%
\pgfusepath{clip}%
\pgfsetbuttcap%
\pgfsetroundjoin%
\definecolor{currentfill}{rgb}{0.200000,0.800000,0.200000}%
\pgfsetfillcolor{currentfill}%
\pgfsetlinewidth{1.003750pt}%
\definecolor{currentstroke}{rgb}{0.200000,0.800000,0.200000}%
\pgfsetstrokecolor{currentstroke}%
\pgfsetdash{}{0pt}%
\pgfpathmoveto{\pgfqpoint{2.995527in}{4.602373in}}%
\pgfpathcurveto{\pgfqpoint{3.001351in}{4.602373in}}{\pgfqpoint{3.006937in}{4.604687in}}{\pgfqpoint{3.011056in}{4.608805in}}%
\pgfpathcurveto{\pgfqpoint{3.015174in}{4.612923in}}{\pgfqpoint{3.017488in}{4.618509in}}{\pgfqpoint{3.017488in}{4.624333in}}%
\pgfpathcurveto{\pgfqpoint{3.017488in}{4.630157in}}{\pgfqpoint{3.015174in}{4.635743in}}{\pgfqpoint{3.011056in}{4.639862in}}%
\pgfpathcurveto{\pgfqpoint{3.006937in}{4.643980in}}{\pgfqpoint{3.001351in}{4.646294in}}{\pgfqpoint{2.995527in}{4.646294in}}%
\pgfpathcurveto{\pgfqpoint{2.989703in}{4.646294in}}{\pgfqpoint{2.984117in}{4.643980in}}{\pgfqpoint{2.979999in}{4.639862in}}%
\pgfpathcurveto{\pgfqpoint{2.975881in}{4.635743in}}{\pgfqpoint{2.973567in}{4.630157in}}{\pgfqpoint{2.973567in}{4.624333in}}%
\pgfpathcurveto{\pgfqpoint{2.973567in}{4.618509in}}{\pgfqpoint{2.975881in}{4.612923in}}{\pgfqpoint{2.979999in}{4.608805in}}%
\pgfpathcurveto{\pgfqpoint{2.984117in}{4.604687in}}{\pgfqpoint{2.989703in}{4.602373in}}{\pgfqpoint{2.995527in}{4.602373in}}%
\pgfpathlineto{\pgfqpoint{2.995527in}{4.602373in}}%
\pgfpathclose%
\pgfusepath{stroke,fill}%
\end{pgfscope}%
\begin{pgfscope}%
\pgfpathrectangle{\pgfqpoint{1.073501in}{0.880000in}}{\pgfqpoint{6.052998in}{6.160000in}}%
\pgfusepath{clip}%
\pgfsetbuttcap%
\pgfsetroundjoin%
\definecolor{currentfill}{rgb}{0.200000,0.800000,0.200000}%
\pgfsetfillcolor{currentfill}%
\pgfsetlinewidth{1.003750pt}%
\definecolor{currentstroke}{rgb}{0.200000,0.800000,0.200000}%
\pgfsetstrokecolor{currentstroke}%
\pgfsetdash{}{0pt}%
\pgfpathmoveto{\pgfqpoint{2.799734in}{4.609326in}}%
\pgfpathcurveto{\pgfqpoint{2.805557in}{4.609326in}}{\pgfqpoint{2.811144in}{4.611640in}}{\pgfqpoint{2.815262in}{4.615758in}}%
\pgfpathcurveto{\pgfqpoint{2.819380in}{4.619876in}}{\pgfqpoint{2.821694in}{4.625462in}}{\pgfqpoint{2.821694in}{4.631286in}}%
\pgfpathcurveto{\pgfqpoint{2.821694in}{4.637110in}}{\pgfqpoint{2.819380in}{4.642696in}}{\pgfqpoint{2.815262in}{4.646814in}}%
\pgfpathcurveto{\pgfqpoint{2.811144in}{4.650932in}}{\pgfqpoint{2.805557in}{4.653246in}}{\pgfqpoint{2.799734in}{4.653246in}}%
\pgfpathcurveto{\pgfqpoint{2.793910in}{4.653246in}}{\pgfqpoint{2.788323in}{4.650932in}}{\pgfqpoint{2.784205in}{4.646814in}}%
\pgfpathcurveto{\pgfqpoint{2.780087in}{4.642696in}}{\pgfqpoint{2.777773in}{4.637110in}}{\pgfqpoint{2.777773in}{4.631286in}}%
\pgfpathcurveto{\pgfqpoint{2.777773in}{4.625462in}}{\pgfqpoint{2.780087in}{4.619876in}}{\pgfqpoint{2.784205in}{4.615758in}}%
\pgfpathcurveto{\pgfqpoint{2.788323in}{4.611640in}}{\pgfqpoint{2.793910in}{4.609326in}}{\pgfqpoint{2.799734in}{4.609326in}}%
\pgfpathlineto{\pgfqpoint{2.799734in}{4.609326in}}%
\pgfpathclose%
\pgfusepath{stroke,fill}%
\end{pgfscope}%
\begin{pgfscope}%
\pgfpathrectangle{\pgfqpoint{1.073501in}{0.880000in}}{\pgfqpoint{6.052998in}{6.160000in}}%
\pgfusepath{clip}%
\pgfsetbuttcap%
\pgfsetroundjoin%
\definecolor{currentfill}{rgb}{0.200000,0.800000,0.200000}%
\pgfsetfillcolor{currentfill}%
\pgfsetlinewidth{1.003750pt}%
\definecolor{currentstroke}{rgb}{0.200000,0.800000,0.200000}%
\pgfsetstrokecolor{currentstroke}%
\pgfsetdash{}{0pt}%
\pgfpathmoveto{\pgfqpoint{2.871498in}{4.474645in}}%
\pgfpathcurveto{\pgfqpoint{2.877322in}{4.474645in}}{\pgfqpoint{2.882908in}{4.476959in}}{\pgfqpoint{2.887026in}{4.481077in}}%
\pgfpathcurveto{\pgfqpoint{2.891144in}{4.485195in}}{\pgfqpoint{2.893458in}{4.490782in}}{\pgfqpoint{2.893458in}{4.496606in}}%
\pgfpathcurveto{\pgfqpoint{2.893458in}{4.502430in}}{\pgfqpoint{2.891144in}{4.508016in}}{\pgfqpoint{2.887026in}{4.512134in}}%
\pgfpathcurveto{\pgfqpoint{2.882908in}{4.516252in}}{\pgfqpoint{2.877322in}{4.518566in}}{\pgfqpoint{2.871498in}{4.518566in}}%
\pgfpathcurveto{\pgfqpoint{2.865674in}{4.518566in}}{\pgfqpoint{2.860088in}{4.516252in}}{\pgfqpoint{2.855970in}{4.512134in}}%
\pgfpathcurveto{\pgfqpoint{2.851852in}{4.508016in}}{\pgfqpoint{2.849538in}{4.502430in}}{\pgfqpoint{2.849538in}{4.496606in}}%
\pgfpathcurveto{\pgfqpoint{2.849538in}{4.490782in}}{\pgfqpoint{2.851852in}{4.485195in}}{\pgfqpoint{2.855970in}{4.481077in}}%
\pgfpathcurveto{\pgfqpoint{2.860088in}{4.476959in}}{\pgfqpoint{2.865674in}{4.474645in}}{\pgfqpoint{2.871498in}{4.474645in}}%
\pgfpathlineto{\pgfqpoint{2.871498in}{4.474645in}}%
\pgfpathclose%
\pgfusepath{stroke,fill}%
\end{pgfscope}%
\begin{pgfscope}%
\pgfpathrectangle{\pgfqpoint{1.073501in}{0.880000in}}{\pgfqpoint{6.052998in}{6.160000in}}%
\pgfusepath{clip}%
\pgfsetbuttcap%
\pgfsetroundjoin%
\definecolor{currentfill}{rgb}{0.200000,0.800000,0.200000}%
\pgfsetfillcolor{currentfill}%
\pgfsetlinewidth{1.003750pt}%
\definecolor{currentstroke}{rgb}{0.200000,0.800000,0.200000}%
\pgfsetstrokecolor{currentstroke}%
\pgfsetdash{}{0pt}%
\pgfpathmoveto{\pgfqpoint{2.765305in}{4.420900in}}%
\pgfpathcurveto{\pgfqpoint{2.771129in}{4.420900in}}{\pgfqpoint{2.776715in}{4.423214in}}{\pgfqpoint{2.780833in}{4.427332in}}%
\pgfpathcurveto{\pgfqpoint{2.784951in}{4.431450in}}{\pgfqpoint{2.787265in}{4.437036in}}{\pgfqpoint{2.787265in}{4.442860in}}%
\pgfpathcurveto{\pgfqpoint{2.787265in}{4.448684in}}{\pgfqpoint{2.784951in}{4.454270in}}{\pgfqpoint{2.780833in}{4.458388in}}%
\pgfpathcurveto{\pgfqpoint{2.776715in}{4.462506in}}{\pgfqpoint{2.771129in}{4.464820in}}{\pgfqpoint{2.765305in}{4.464820in}}%
\pgfpathcurveto{\pgfqpoint{2.759481in}{4.464820in}}{\pgfqpoint{2.753895in}{4.462506in}}{\pgfqpoint{2.749776in}{4.458388in}}%
\pgfpathcurveto{\pgfqpoint{2.745658in}{4.454270in}}{\pgfqpoint{2.743344in}{4.448684in}}{\pgfqpoint{2.743344in}{4.442860in}}%
\pgfpathcurveto{\pgfqpoint{2.743344in}{4.437036in}}{\pgfqpoint{2.745658in}{4.431450in}}{\pgfqpoint{2.749776in}{4.427332in}}%
\pgfpathcurveto{\pgfqpoint{2.753895in}{4.423214in}}{\pgfqpoint{2.759481in}{4.420900in}}{\pgfqpoint{2.765305in}{4.420900in}}%
\pgfpathlineto{\pgfqpoint{2.765305in}{4.420900in}}%
\pgfpathclose%
\pgfusepath{stroke,fill}%
\end{pgfscope}%
\begin{pgfscope}%
\pgfpathrectangle{\pgfqpoint{1.073501in}{0.880000in}}{\pgfqpoint{6.052998in}{6.160000in}}%
\pgfusepath{clip}%
\pgfsetbuttcap%
\pgfsetroundjoin%
\definecolor{currentfill}{rgb}{0.200000,0.800000,0.200000}%
\pgfsetfillcolor{currentfill}%
\pgfsetlinewidth{1.003750pt}%
\definecolor{currentstroke}{rgb}{0.200000,0.800000,0.200000}%
\pgfsetstrokecolor{currentstroke}%
\pgfsetdash{}{0pt}%
\pgfpathmoveto{\pgfqpoint{2.677447in}{4.351008in}}%
\pgfpathcurveto{\pgfqpoint{2.683270in}{4.351008in}}{\pgfqpoint{2.688857in}{4.353322in}}{\pgfqpoint{2.692975in}{4.357440in}}%
\pgfpathcurveto{\pgfqpoint{2.697093in}{4.361559in}}{\pgfqpoint{2.699407in}{4.367145in}}{\pgfqpoint{2.699407in}{4.372969in}}%
\pgfpathcurveto{\pgfqpoint{2.699407in}{4.378793in}}{\pgfqpoint{2.697093in}{4.384379in}}{\pgfqpoint{2.692975in}{4.388497in}}%
\pgfpathcurveto{\pgfqpoint{2.688857in}{4.392615in}}{\pgfqpoint{2.683270in}{4.394929in}}{\pgfqpoint{2.677447in}{4.394929in}}%
\pgfpathcurveto{\pgfqpoint{2.671623in}{4.394929in}}{\pgfqpoint{2.666036in}{4.392615in}}{\pgfqpoint{2.661918in}{4.388497in}}%
\pgfpathcurveto{\pgfqpoint{2.657800in}{4.384379in}}{\pgfqpoint{2.655486in}{4.378793in}}{\pgfqpoint{2.655486in}{4.372969in}}%
\pgfpathcurveto{\pgfqpoint{2.655486in}{4.367145in}}{\pgfqpoint{2.657800in}{4.361559in}}{\pgfqpoint{2.661918in}{4.357440in}}%
\pgfpathcurveto{\pgfqpoint{2.666036in}{4.353322in}}{\pgfqpoint{2.671623in}{4.351008in}}{\pgfqpoint{2.677447in}{4.351008in}}%
\pgfpathlineto{\pgfqpoint{2.677447in}{4.351008in}}%
\pgfpathclose%
\pgfusepath{stroke,fill}%
\end{pgfscope}%
\begin{pgfscope}%
\pgfpathrectangle{\pgfqpoint{1.073501in}{0.880000in}}{\pgfqpoint{6.052998in}{6.160000in}}%
\pgfusepath{clip}%
\pgfsetbuttcap%
\pgfsetroundjoin%
\definecolor{currentfill}{rgb}{0.200000,0.800000,0.200000}%
\pgfsetfillcolor{currentfill}%
\pgfsetlinewidth{1.003750pt}%
\definecolor{currentstroke}{rgb}{0.200000,0.800000,0.200000}%
\pgfsetstrokecolor{currentstroke}%
\pgfsetdash{}{0pt}%
\pgfpathmoveto{\pgfqpoint{2.603264in}{4.269626in}}%
\pgfpathcurveto{\pgfqpoint{2.609088in}{4.269626in}}{\pgfqpoint{2.614675in}{4.271940in}}{\pgfqpoint{2.618793in}{4.276058in}}%
\pgfpathcurveto{\pgfqpoint{2.622911in}{4.280176in}}{\pgfqpoint{2.625225in}{4.285763in}}{\pgfqpoint{2.625225in}{4.291586in}}%
\pgfpathcurveto{\pgfqpoint{2.625225in}{4.297410in}}{\pgfqpoint{2.622911in}{4.302997in}}{\pgfqpoint{2.618793in}{4.307115in}}%
\pgfpathcurveto{\pgfqpoint{2.614675in}{4.311233in}}{\pgfqpoint{2.609088in}{4.313547in}}{\pgfqpoint{2.603264in}{4.313547in}}%
\pgfpathcurveto{\pgfqpoint{2.597440in}{4.313547in}}{\pgfqpoint{2.591854in}{4.311233in}}{\pgfqpoint{2.587736in}{4.307115in}}%
\pgfpathcurveto{\pgfqpoint{2.583618in}{4.302997in}}{\pgfqpoint{2.581304in}{4.297410in}}{\pgfqpoint{2.581304in}{4.291586in}}%
\pgfpathcurveto{\pgfqpoint{2.581304in}{4.285763in}}{\pgfqpoint{2.583618in}{4.280176in}}{\pgfqpoint{2.587736in}{4.276058in}}%
\pgfpathcurveto{\pgfqpoint{2.591854in}{4.271940in}}{\pgfqpoint{2.597440in}{4.269626in}}{\pgfqpoint{2.603264in}{4.269626in}}%
\pgfpathlineto{\pgfqpoint{2.603264in}{4.269626in}}%
\pgfpathclose%
\pgfusepath{stroke,fill}%
\end{pgfscope}%
\begin{pgfscope}%
\pgfpathrectangle{\pgfqpoint{1.073501in}{0.880000in}}{\pgfqpoint{6.052998in}{6.160000in}}%
\pgfusepath{clip}%
\pgfsetbuttcap%
\pgfsetroundjoin%
\definecolor{currentfill}{rgb}{0.200000,0.800000,0.200000}%
\pgfsetfillcolor{currentfill}%
\pgfsetlinewidth{1.003750pt}%
\definecolor{currentstroke}{rgb}{0.200000,0.800000,0.200000}%
\pgfsetstrokecolor{currentstroke}%
\pgfsetdash{}{0pt}%
\pgfpathmoveto{\pgfqpoint{2.669314in}{4.158813in}}%
\pgfpathcurveto{\pgfqpoint{2.675138in}{4.158813in}}{\pgfqpoint{2.680724in}{4.161127in}}{\pgfqpoint{2.684842in}{4.165245in}}%
\pgfpathcurveto{\pgfqpoint{2.688960in}{4.169363in}}{\pgfqpoint{2.691274in}{4.174950in}}{\pgfqpoint{2.691274in}{4.180774in}}%
\pgfpathcurveto{\pgfqpoint{2.691274in}{4.186597in}}{\pgfqpoint{2.688960in}{4.192184in}}{\pgfqpoint{2.684842in}{4.196302in}}%
\pgfpathcurveto{\pgfqpoint{2.680724in}{4.200420in}}{\pgfqpoint{2.675138in}{4.202734in}}{\pgfqpoint{2.669314in}{4.202734in}}%
\pgfpathcurveto{\pgfqpoint{2.663490in}{4.202734in}}{\pgfqpoint{2.657904in}{4.200420in}}{\pgfqpoint{2.653786in}{4.196302in}}%
\pgfpathcurveto{\pgfqpoint{2.649667in}{4.192184in}}{\pgfqpoint{2.647354in}{4.186597in}}{\pgfqpoint{2.647354in}{4.180774in}}%
\pgfpathcurveto{\pgfqpoint{2.647354in}{4.174950in}}{\pgfqpoint{2.649667in}{4.169363in}}{\pgfqpoint{2.653786in}{4.165245in}}%
\pgfpathcurveto{\pgfqpoint{2.657904in}{4.161127in}}{\pgfqpoint{2.663490in}{4.158813in}}{\pgfqpoint{2.669314in}{4.158813in}}%
\pgfpathlineto{\pgfqpoint{2.669314in}{4.158813in}}%
\pgfpathclose%
\pgfusepath{stroke,fill}%
\end{pgfscope}%
\begin{pgfscope}%
\pgfpathrectangle{\pgfqpoint{1.073501in}{0.880000in}}{\pgfqpoint{6.052998in}{6.160000in}}%
\pgfusepath{clip}%
\pgfsetbuttcap%
\pgfsetroundjoin%
\definecolor{currentfill}{rgb}{0.200000,0.800000,0.200000}%
\pgfsetfillcolor{currentfill}%
\pgfsetlinewidth{1.003750pt}%
\definecolor{currentstroke}{rgb}{0.200000,0.800000,0.200000}%
\pgfsetstrokecolor{currentstroke}%
\pgfsetdash{}{0pt}%
\pgfpathmoveto{\pgfqpoint{2.670084in}{4.064775in}}%
\pgfpathcurveto{\pgfqpoint{2.675908in}{4.064775in}}{\pgfqpoint{2.681494in}{4.067089in}}{\pgfqpoint{2.685612in}{4.071207in}}%
\pgfpathcurveto{\pgfqpoint{2.689730in}{4.075326in}}{\pgfqpoint{2.692044in}{4.080912in}}{\pgfqpoint{2.692044in}{4.086736in}}%
\pgfpathcurveto{\pgfqpoint{2.692044in}{4.092560in}}{\pgfqpoint{2.689730in}{4.098146in}}{\pgfqpoint{2.685612in}{4.102264in}}%
\pgfpathcurveto{\pgfqpoint{2.681494in}{4.106382in}}{\pgfqpoint{2.675908in}{4.108696in}}{\pgfqpoint{2.670084in}{4.108696in}}%
\pgfpathcurveto{\pgfqpoint{2.664260in}{4.108696in}}{\pgfqpoint{2.658673in}{4.106382in}}{\pgfqpoint{2.654555in}{4.102264in}}%
\pgfpathcurveto{\pgfqpoint{2.650437in}{4.098146in}}{\pgfqpoint{2.648123in}{4.092560in}}{\pgfqpoint{2.648123in}{4.086736in}}%
\pgfpathcurveto{\pgfqpoint{2.648123in}{4.080912in}}{\pgfqpoint{2.650437in}{4.075326in}}{\pgfqpoint{2.654555in}{4.071207in}}%
\pgfpathcurveto{\pgfqpoint{2.658673in}{4.067089in}}{\pgfqpoint{2.664260in}{4.064775in}}{\pgfqpoint{2.670084in}{4.064775in}}%
\pgfpathlineto{\pgfqpoint{2.670084in}{4.064775in}}%
\pgfpathclose%
\pgfusepath{stroke,fill}%
\end{pgfscope}%
\begin{pgfscope}%
\pgfpathrectangle{\pgfqpoint{1.073501in}{0.880000in}}{\pgfqpoint{6.052998in}{6.160000in}}%
\pgfusepath{clip}%
\pgfsetbuttcap%
\pgfsetroundjoin%
\definecolor{currentfill}{rgb}{0.200000,0.800000,0.200000}%
\pgfsetfillcolor{currentfill}%
\pgfsetlinewidth{1.003750pt}%
\definecolor{currentstroke}{rgb}{0.200000,0.800000,0.200000}%
\pgfsetstrokecolor{currentstroke}%
\pgfsetdash{}{0pt}%
\pgfpathmoveto{\pgfqpoint{2.645894in}{3.972761in}}%
\pgfpathcurveto{\pgfqpoint{2.651718in}{3.972761in}}{\pgfqpoint{2.657304in}{3.975075in}}{\pgfqpoint{2.661422in}{3.979194in}}%
\pgfpathcurveto{\pgfqpoint{2.665540in}{3.983312in}}{\pgfqpoint{2.667854in}{3.988898in}}{\pgfqpoint{2.667854in}{3.994722in}}%
\pgfpathcurveto{\pgfqpoint{2.667854in}{4.000546in}}{\pgfqpoint{2.665540in}{4.006132in}}{\pgfqpoint{2.661422in}{4.010250in}}%
\pgfpathcurveto{\pgfqpoint{2.657304in}{4.014368in}}{\pgfqpoint{2.651718in}{4.016682in}}{\pgfqpoint{2.645894in}{4.016682in}}%
\pgfpathcurveto{\pgfqpoint{2.640070in}{4.016682in}}{\pgfqpoint{2.634484in}{4.014368in}}{\pgfqpoint{2.630366in}{4.010250in}}%
\pgfpathcurveto{\pgfqpoint{2.626248in}{4.006132in}}{\pgfqpoint{2.623934in}{4.000546in}}{\pgfqpoint{2.623934in}{3.994722in}}%
\pgfpathcurveto{\pgfqpoint{2.623934in}{3.988898in}}{\pgfqpoint{2.626248in}{3.983312in}}{\pgfqpoint{2.630366in}{3.979194in}}%
\pgfpathcurveto{\pgfqpoint{2.634484in}{3.975075in}}{\pgfqpoint{2.640070in}{3.972761in}}{\pgfqpoint{2.645894in}{3.972761in}}%
\pgfpathlineto{\pgfqpoint{2.645894in}{3.972761in}}%
\pgfpathclose%
\pgfusepath{stroke,fill}%
\end{pgfscope}%
\begin{pgfscope}%
\pgfpathrectangle{\pgfqpoint{1.073501in}{0.880000in}}{\pgfqpoint{6.052998in}{6.160000in}}%
\pgfusepath{clip}%
\pgfsetbuttcap%
\pgfsetroundjoin%
\definecolor{currentfill}{rgb}{0.200000,0.800000,0.200000}%
\pgfsetfillcolor{currentfill}%
\pgfsetlinewidth{1.003750pt}%
\definecolor{currentstroke}{rgb}{0.200000,0.800000,0.200000}%
\pgfsetstrokecolor{currentstroke}%
\pgfsetdash{}{0pt}%
\pgfpathmoveto{\pgfqpoint{2.726804in}{3.881386in}}%
\pgfpathcurveto{\pgfqpoint{2.732628in}{3.881386in}}{\pgfqpoint{2.738214in}{3.883700in}}{\pgfqpoint{2.742333in}{3.887818in}}%
\pgfpathcurveto{\pgfqpoint{2.746451in}{3.891936in}}{\pgfqpoint{2.748765in}{3.897522in}}{\pgfqpoint{2.748765in}{3.903346in}}%
\pgfpathcurveto{\pgfqpoint{2.748765in}{3.909170in}}{\pgfqpoint{2.746451in}{3.914756in}}{\pgfqpoint{2.742333in}{3.918875in}}%
\pgfpathcurveto{\pgfqpoint{2.738214in}{3.922993in}}{\pgfqpoint{2.732628in}{3.925307in}}{\pgfqpoint{2.726804in}{3.925307in}}%
\pgfpathcurveto{\pgfqpoint{2.720980in}{3.925307in}}{\pgfqpoint{2.715394in}{3.922993in}}{\pgfqpoint{2.711276in}{3.918875in}}%
\pgfpathcurveto{\pgfqpoint{2.707158in}{3.914756in}}{\pgfqpoint{2.704844in}{3.909170in}}{\pgfqpoint{2.704844in}{3.903346in}}%
\pgfpathcurveto{\pgfqpoint{2.704844in}{3.897522in}}{\pgfqpoint{2.707158in}{3.891936in}}{\pgfqpoint{2.711276in}{3.887818in}}%
\pgfpathcurveto{\pgfqpoint{2.715394in}{3.883700in}}{\pgfqpoint{2.720980in}{3.881386in}}{\pgfqpoint{2.726804in}{3.881386in}}%
\pgfpathlineto{\pgfqpoint{2.726804in}{3.881386in}}%
\pgfpathclose%
\pgfusepath{stroke,fill}%
\end{pgfscope}%
\begin{pgfscope}%
\pgfpathrectangle{\pgfqpoint{1.073501in}{0.880000in}}{\pgfqpoint{6.052998in}{6.160000in}}%
\pgfusepath{clip}%
\pgfsetbuttcap%
\pgfsetroundjoin%
\definecolor{currentfill}{rgb}{0.200000,0.800000,0.200000}%
\pgfsetfillcolor{currentfill}%
\pgfsetlinewidth{1.003750pt}%
\definecolor{currentstroke}{rgb}{0.200000,0.800000,0.200000}%
\pgfsetstrokecolor{currentstroke}%
\pgfsetdash{}{0pt}%
\pgfpathmoveto{\pgfqpoint{2.703162in}{3.789962in}}%
\pgfpathcurveto{\pgfqpoint{2.708986in}{3.789962in}}{\pgfqpoint{2.714572in}{3.792276in}}{\pgfqpoint{2.718690in}{3.796394in}}%
\pgfpathcurveto{\pgfqpoint{2.722808in}{3.800512in}}{\pgfqpoint{2.725122in}{3.806099in}}{\pgfqpoint{2.725122in}{3.811923in}}%
\pgfpathcurveto{\pgfqpoint{2.725122in}{3.817747in}}{\pgfqpoint{2.722808in}{3.823333in}}{\pgfqpoint{2.718690in}{3.827451in}}%
\pgfpathcurveto{\pgfqpoint{2.714572in}{3.831569in}}{\pgfqpoint{2.708986in}{3.833883in}}{\pgfqpoint{2.703162in}{3.833883in}}%
\pgfpathcurveto{\pgfqpoint{2.697338in}{3.833883in}}{\pgfqpoint{2.691752in}{3.831569in}}{\pgfqpoint{2.687633in}{3.827451in}}%
\pgfpathcurveto{\pgfqpoint{2.683515in}{3.823333in}}{\pgfqpoint{2.681201in}{3.817747in}}{\pgfqpoint{2.681201in}{3.811923in}}%
\pgfpathcurveto{\pgfqpoint{2.681201in}{3.806099in}}{\pgfqpoint{2.683515in}{3.800512in}}{\pgfqpoint{2.687633in}{3.796394in}}%
\pgfpathcurveto{\pgfqpoint{2.691752in}{3.792276in}}{\pgfqpoint{2.697338in}{3.789962in}}{\pgfqpoint{2.703162in}{3.789962in}}%
\pgfpathlineto{\pgfqpoint{2.703162in}{3.789962in}}%
\pgfpathclose%
\pgfusepath{stroke,fill}%
\end{pgfscope}%
\begin{pgfscope}%
\pgfpathrectangle{\pgfqpoint{1.073501in}{0.880000in}}{\pgfqpoint{6.052998in}{6.160000in}}%
\pgfusepath{clip}%
\pgfsetbuttcap%
\pgfsetroundjoin%
\definecolor{currentfill}{rgb}{0.200000,0.800000,0.200000}%
\pgfsetfillcolor{currentfill}%
\pgfsetlinewidth{1.003750pt}%
\definecolor{currentstroke}{rgb}{0.200000,0.800000,0.200000}%
\pgfsetstrokecolor{currentstroke}%
\pgfsetdash{}{0pt}%
\pgfpathmoveto{\pgfqpoint{2.663983in}{3.691913in}}%
\pgfpathcurveto{\pgfqpoint{2.669807in}{3.691913in}}{\pgfqpoint{2.675393in}{3.694227in}}{\pgfqpoint{2.679511in}{3.698345in}}%
\pgfpathcurveto{\pgfqpoint{2.683629in}{3.702463in}}{\pgfqpoint{2.685943in}{3.708049in}}{\pgfqpoint{2.685943in}{3.713873in}}%
\pgfpathcurveto{\pgfqpoint{2.685943in}{3.719697in}}{\pgfqpoint{2.683629in}{3.725283in}}{\pgfqpoint{2.679511in}{3.729401in}}%
\pgfpathcurveto{\pgfqpoint{2.675393in}{3.733519in}}{\pgfqpoint{2.669807in}{3.735833in}}{\pgfqpoint{2.663983in}{3.735833in}}%
\pgfpathcurveto{\pgfqpoint{2.658159in}{3.735833in}}{\pgfqpoint{2.652573in}{3.733519in}}{\pgfqpoint{2.648455in}{3.729401in}}%
\pgfpathcurveto{\pgfqpoint{2.644336in}{3.725283in}}{\pgfqpoint{2.642023in}{3.719697in}}{\pgfqpoint{2.642023in}{3.713873in}}%
\pgfpathcurveto{\pgfqpoint{2.642023in}{3.708049in}}{\pgfqpoint{2.644336in}{3.702463in}}{\pgfqpoint{2.648455in}{3.698345in}}%
\pgfpathcurveto{\pgfqpoint{2.652573in}{3.694227in}}{\pgfqpoint{2.658159in}{3.691913in}}{\pgfqpoint{2.663983in}{3.691913in}}%
\pgfpathlineto{\pgfqpoint{2.663983in}{3.691913in}}%
\pgfpathclose%
\pgfusepath{stroke,fill}%
\end{pgfscope}%
\begin{pgfscope}%
\pgfpathrectangle{\pgfqpoint{1.073501in}{0.880000in}}{\pgfqpoint{6.052998in}{6.160000in}}%
\pgfusepath{clip}%
\pgfsetbuttcap%
\pgfsetroundjoin%
\definecolor{currentfill}{rgb}{0.200000,0.800000,0.200000}%
\pgfsetfillcolor{currentfill}%
\pgfsetlinewidth{1.003750pt}%
\definecolor{currentstroke}{rgb}{0.200000,0.800000,0.200000}%
\pgfsetstrokecolor{currentstroke}%
\pgfsetdash{}{0pt}%
\pgfpathmoveto{\pgfqpoint{2.744415in}{3.613833in}}%
\pgfpathcurveto{\pgfqpoint{2.750238in}{3.613833in}}{\pgfqpoint{2.755825in}{3.616147in}}{\pgfqpoint{2.759943in}{3.620265in}}%
\pgfpathcurveto{\pgfqpoint{2.764061in}{3.624383in}}{\pgfqpoint{2.766375in}{3.629969in}}{\pgfqpoint{2.766375in}{3.635793in}}%
\pgfpathcurveto{\pgfqpoint{2.766375in}{3.641617in}}{\pgfqpoint{2.764061in}{3.647203in}}{\pgfqpoint{2.759943in}{3.651322in}}%
\pgfpathcurveto{\pgfqpoint{2.755825in}{3.655440in}}{\pgfqpoint{2.750238in}{3.657754in}}{\pgfqpoint{2.744415in}{3.657754in}}%
\pgfpathcurveto{\pgfqpoint{2.738591in}{3.657754in}}{\pgfqpoint{2.733004in}{3.655440in}}{\pgfqpoint{2.728886in}{3.651322in}}%
\pgfpathcurveto{\pgfqpoint{2.724768in}{3.647203in}}{\pgfqpoint{2.722454in}{3.641617in}}{\pgfqpoint{2.722454in}{3.635793in}}%
\pgfpathcurveto{\pgfqpoint{2.722454in}{3.629969in}}{\pgfqpoint{2.724768in}{3.624383in}}{\pgfqpoint{2.728886in}{3.620265in}}%
\pgfpathcurveto{\pgfqpoint{2.733004in}{3.616147in}}{\pgfqpoint{2.738591in}{3.613833in}}{\pgfqpoint{2.744415in}{3.613833in}}%
\pgfpathlineto{\pgfqpoint{2.744415in}{3.613833in}}%
\pgfpathclose%
\pgfusepath{stroke,fill}%
\end{pgfscope}%
\begin{pgfscope}%
\pgfpathrectangle{\pgfqpoint{1.073501in}{0.880000in}}{\pgfqpoint{6.052998in}{6.160000in}}%
\pgfusepath{clip}%
\pgfsetbuttcap%
\pgfsetroundjoin%
\definecolor{currentfill}{rgb}{0.200000,0.800000,0.200000}%
\pgfsetfillcolor{currentfill}%
\pgfsetlinewidth{1.003750pt}%
\definecolor{currentstroke}{rgb}{0.200000,0.800000,0.200000}%
\pgfsetstrokecolor{currentstroke}%
\pgfsetdash{}{0pt}%
\pgfpathmoveto{\pgfqpoint{2.862629in}{3.554945in}}%
\pgfpathcurveto{\pgfqpoint{2.868452in}{3.554945in}}{\pgfqpoint{2.874039in}{3.557259in}}{\pgfqpoint{2.878157in}{3.561377in}}%
\pgfpathcurveto{\pgfqpoint{2.882275in}{3.565495in}}{\pgfqpoint{2.884589in}{3.571081in}}{\pgfqpoint{2.884589in}{3.576905in}}%
\pgfpathcurveto{\pgfqpoint{2.884589in}{3.582729in}}{\pgfqpoint{2.882275in}{3.588315in}}{\pgfqpoint{2.878157in}{3.592434in}}%
\pgfpathcurveto{\pgfqpoint{2.874039in}{3.596552in}}{\pgfqpoint{2.868452in}{3.598866in}}{\pgfqpoint{2.862629in}{3.598866in}}%
\pgfpathcurveto{\pgfqpoint{2.856805in}{3.598866in}}{\pgfqpoint{2.851218in}{3.596552in}}{\pgfqpoint{2.847100in}{3.592434in}}%
\pgfpathcurveto{\pgfqpoint{2.842982in}{3.588315in}}{\pgfqpoint{2.840668in}{3.582729in}}{\pgfqpoint{2.840668in}{3.576905in}}%
\pgfpathcurveto{\pgfqpoint{2.840668in}{3.571081in}}{\pgfqpoint{2.842982in}{3.565495in}}{\pgfqpoint{2.847100in}{3.561377in}}%
\pgfpathcurveto{\pgfqpoint{2.851218in}{3.557259in}}{\pgfqpoint{2.856805in}{3.554945in}}{\pgfqpoint{2.862629in}{3.554945in}}%
\pgfpathlineto{\pgfqpoint{2.862629in}{3.554945in}}%
\pgfpathclose%
\pgfusepath{stroke,fill}%
\end{pgfscope}%
\begin{pgfscope}%
\pgfpathrectangle{\pgfqpoint{1.073501in}{0.880000in}}{\pgfqpoint{6.052998in}{6.160000in}}%
\pgfusepath{clip}%
\pgfsetbuttcap%
\pgfsetroundjoin%
\definecolor{currentfill}{rgb}{0.200000,0.800000,0.200000}%
\pgfsetfillcolor{currentfill}%
\pgfsetlinewidth{1.003750pt}%
\definecolor{currentstroke}{rgb}{0.200000,0.800000,0.200000}%
\pgfsetstrokecolor{currentstroke}%
\pgfsetdash{}{0pt}%
\pgfpathmoveto{\pgfqpoint{2.713520in}{3.411831in}}%
\pgfpathcurveto{\pgfqpoint{2.719344in}{3.411831in}}{\pgfqpoint{2.724930in}{3.414145in}}{\pgfqpoint{2.729048in}{3.418263in}}%
\pgfpathcurveto{\pgfqpoint{2.733166in}{3.422381in}}{\pgfqpoint{2.735480in}{3.427968in}}{\pgfqpoint{2.735480in}{3.433791in}}%
\pgfpathcurveto{\pgfqpoint{2.735480in}{3.439615in}}{\pgfqpoint{2.733166in}{3.445202in}}{\pgfqpoint{2.729048in}{3.449320in}}%
\pgfpathcurveto{\pgfqpoint{2.724930in}{3.453438in}}{\pgfqpoint{2.719344in}{3.455752in}}{\pgfqpoint{2.713520in}{3.455752in}}%
\pgfpathcurveto{\pgfqpoint{2.707696in}{3.455752in}}{\pgfqpoint{2.702110in}{3.453438in}}{\pgfqpoint{2.697992in}{3.449320in}}%
\pgfpathcurveto{\pgfqpoint{2.693874in}{3.445202in}}{\pgfqpoint{2.691560in}{3.439615in}}{\pgfqpoint{2.691560in}{3.433791in}}%
\pgfpathcurveto{\pgfqpoint{2.691560in}{3.427968in}}{\pgfqpoint{2.693874in}{3.422381in}}{\pgfqpoint{2.697992in}{3.418263in}}%
\pgfpathcurveto{\pgfqpoint{2.702110in}{3.414145in}}{\pgfqpoint{2.707696in}{3.411831in}}{\pgfqpoint{2.713520in}{3.411831in}}%
\pgfpathlineto{\pgfqpoint{2.713520in}{3.411831in}}%
\pgfpathclose%
\pgfusepath{stroke,fill}%
\end{pgfscope}%
\begin{pgfscope}%
\pgfpathrectangle{\pgfqpoint{1.073501in}{0.880000in}}{\pgfqpoint{6.052998in}{6.160000in}}%
\pgfusepath{clip}%
\pgfsetbuttcap%
\pgfsetroundjoin%
\definecolor{currentfill}{rgb}{0.200000,0.800000,0.200000}%
\pgfsetfillcolor{currentfill}%
\pgfsetlinewidth{1.003750pt}%
\definecolor{currentstroke}{rgb}{0.200000,0.800000,0.200000}%
\pgfsetstrokecolor{currentstroke}%
\pgfsetdash{}{0pt}%
\pgfpathmoveto{\pgfqpoint{2.777862in}{3.335954in}}%
\pgfpathcurveto{\pgfqpoint{2.783686in}{3.335954in}}{\pgfqpoint{2.789272in}{3.338268in}}{\pgfqpoint{2.793390in}{3.342386in}}%
\pgfpathcurveto{\pgfqpoint{2.797508in}{3.346504in}}{\pgfqpoint{2.799822in}{3.352091in}}{\pgfqpoint{2.799822in}{3.357915in}}%
\pgfpathcurveto{\pgfqpoint{2.799822in}{3.363739in}}{\pgfqpoint{2.797508in}{3.369325in}}{\pgfqpoint{2.793390in}{3.373443in}}%
\pgfpathcurveto{\pgfqpoint{2.789272in}{3.377561in}}{\pgfqpoint{2.783686in}{3.379875in}}{\pgfqpoint{2.777862in}{3.379875in}}%
\pgfpathcurveto{\pgfqpoint{2.772038in}{3.379875in}}{\pgfqpoint{2.766452in}{3.377561in}}{\pgfqpoint{2.762333in}{3.373443in}}%
\pgfpathcurveto{\pgfqpoint{2.758215in}{3.369325in}}{\pgfqpoint{2.755901in}{3.363739in}}{\pgfqpoint{2.755901in}{3.357915in}}%
\pgfpathcurveto{\pgfqpoint{2.755901in}{3.352091in}}{\pgfqpoint{2.758215in}{3.346504in}}{\pgfqpoint{2.762333in}{3.342386in}}%
\pgfpathcurveto{\pgfqpoint{2.766452in}{3.338268in}}{\pgfqpoint{2.772038in}{3.335954in}}{\pgfqpoint{2.777862in}{3.335954in}}%
\pgfpathlineto{\pgfqpoint{2.777862in}{3.335954in}}%
\pgfpathclose%
\pgfusepath{stroke,fill}%
\end{pgfscope}%
\begin{pgfscope}%
\pgfpathrectangle{\pgfqpoint{1.073501in}{0.880000in}}{\pgfqpoint{6.052998in}{6.160000in}}%
\pgfusepath{clip}%
\pgfsetbuttcap%
\pgfsetroundjoin%
\definecolor{currentfill}{rgb}{0.200000,0.800000,0.200000}%
\pgfsetfillcolor{currentfill}%
\pgfsetlinewidth{1.003750pt}%
\definecolor{currentstroke}{rgb}{0.200000,0.800000,0.200000}%
\pgfsetstrokecolor{currentstroke}%
\pgfsetdash{}{0pt}%
\pgfpathmoveto{\pgfqpoint{2.917060in}{3.302739in}}%
\pgfpathcurveto{\pgfqpoint{2.922884in}{3.302739in}}{\pgfqpoint{2.928470in}{3.305053in}}{\pgfqpoint{2.932589in}{3.309171in}}%
\pgfpathcurveto{\pgfqpoint{2.936707in}{3.313290in}}{\pgfqpoint{2.939021in}{3.318876in}}{\pgfqpoint{2.939021in}{3.324700in}}%
\pgfpathcurveto{\pgfqpoint{2.939021in}{3.330524in}}{\pgfqpoint{2.936707in}{3.336110in}}{\pgfqpoint{2.932589in}{3.340228in}}%
\pgfpathcurveto{\pgfqpoint{2.928470in}{3.344346in}}{\pgfqpoint{2.922884in}{3.346660in}}{\pgfqpoint{2.917060in}{3.346660in}}%
\pgfpathcurveto{\pgfqpoint{2.911236in}{3.346660in}}{\pgfqpoint{2.905650in}{3.344346in}}{\pgfqpoint{2.901532in}{3.340228in}}%
\pgfpathcurveto{\pgfqpoint{2.897414in}{3.336110in}}{\pgfqpoint{2.895100in}{3.330524in}}{\pgfqpoint{2.895100in}{3.324700in}}%
\pgfpathcurveto{\pgfqpoint{2.895100in}{3.318876in}}{\pgfqpoint{2.897414in}{3.313290in}}{\pgfqpoint{2.901532in}{3.309171in}}%
\pgfpathcurveto{\pgfqpoint{2.905650in}{3.305053in}}{\pgfqpoint{2.911236in}{3.302739in}}{\pgfqpoint{2.917060in}{3.302739in}}%
\pgfpathlineto{\pgfqpoint{2.917060in}{3.302739in}}%
\pgfpathclose%
\pgfusepath{stroke,fill}%
\end{pgfscope}%
\begin{pgfscope}%
\pgfpathrectangle{\pgfqpoint{1.073501in}{0.880000in}}{\pgfqpoint{6.052998in}{6.160000in}}%
\pgfusepath{clip}%
\pgfsetbuttcap%
\pgfsetroundjoin%
\definecolor{currentfill}{rgb}{0.200000,0.800000,0.200000}%
\pgfsetfillcolor{currentfill}%
\pgfsetlinewidth{1.003750pt}%
\definecolor{currentstroke}{rgb}{0.200000,0.800000,0.200000}%
\pgfsetstrokecolor{currentstroke}%
\pgfsetdash{}{0pt}%
\pgfpathmoveto{\pgfqpoint{2.959638in}{3.227719in}}%
\pgfpathcurveto{\pgfqpoint{2.965461in}{3.227719in}}{\pgfqpoint{2.971048in}{3.230033in}}{\pgfqpoint{2.975166in}{3.234151in}}%
\pgfpathcurveto{\pgfqpoint{2.979284in}{3.238269in}}{\pgfqpoint{2.981598in}{3.243855in}}{\pgfqpoint{2.981598in}{3.249679in}}%
\pgfpathcurveto{\pgfqpoint{2.981598in}{3.255503in}}{\pgfqpoint{2.979284in}{3.261089in}}{\pgfqpoint{2.975166in}{3.265207in}}%
\pgfpathcurveto{\pgfqpoint{2.971048in}{3.269325in}}{\pgfqpoint{2.965461in}{3.271639in}}{\pgfqpoint{2.959638in}{3.271639in}}%
\pgfpathcurveto{\pgfqpoint{2.953814in}{3.271639in}}{\pgfqpoint{2.948227in}{3.269325in}}{\pgfqpoint{2.944109in}{3.265207in}}%
\pgfpathcurveto{\pgfqpoint{2.939991in}{3.261089in}}{\pgfqpoint{2.937677in}{3.255503in}}{\pgfqpoint{2.937677in}{3.249679in}}%
\pgfpathcurveto{\pgfqpoint{2.937677in}{3.243855in}}{\pgfqpoint{2.939991in}{3.238269in}}{\pgfqpoint{2.944109in}{3.234151in}}%
\pgfpathcurveto{\pgfqpoint{2.948227in}{3.230033in}}{\pgfqpoint{2.953814in}{3.227719in}}{\pgfqpoint{2.959638in}{3.227719in}}%
\pgfpathlineto{\pgfqpoint{2.959638in}{3.227719in}}%
\pgfpathclose%
\pgfusepath{stroke,fill}%
\end{pgfscope}%
\begin{pgfscope}%
\pgfpathrectangle{\pgfqpoint{1.073501in}{0.880000in}}{\pgfqpoint{6.052998in}{6.160000in}}%
\pgfusepath{clip}%
\pgfsetbuttcap%
\pgfsetroundjoin%
\definecolor{currentfill}{rgb}{0.200000,0.800000,0.200000}%
\pgfsetfillcolor{currentfill}%
\pgfsetlinewidth{1.003750pt}%
\definecolor{currentstroke}{rgb}{0.200000,0.800000,0.200000}%
\pgfsetstrokecolor{currentstroke}%
\pgfsetdash{}{0pt}%
\pgfpathmoveto{\pgfqpoint{2.932812in}{3.104609in}}%
\pgfpathcurveto{\pgfqpoint{2.938636in}{3.104609in}}{\pgfqpoint{2.944222in}{3.106923in}}{\pgfqpoint{2.948340in}{3.111041in}}%
\pgfpathcurveto{\pgfqpoint{2.952458in}{3.115159in}}{\pgfqpoint{2.954772in}{3.120745in}}{\pgfqpoint{2.954772in}{3.126569in}}%
\pgfpathcurveto{\pgfqpoint{2.954772in}{3.132393in}}{\pgfqpoint{2.952458in}{3.137979in}}{\pgfqpoint{2.948340in}{3.142097in}}%
\pgfpathcurveto{\pgfqpoint{2.944222in}{3.146216in}}{\pgfqpoint{2.938636in}{3.148529in}}{\pgfqpoint{2.932812in}{3.148529in}}%
\pgfpathcurveto{\pgfqpoint{2.926988in}{3.148529in}}{\pgfqpoint{2.921402in}{3.146216in}}{\pgfqpoint{2.917284in}{3.142097in}}%
\pgfpathcurveto{\pgfqpoint{2.913166in}{3.137979in}}{\pgfqpoint{2.910852in}{3.132393in}}{\pgfqpoint{2.910852in}{3.126569in}}%
\pgfpathcurveto{\pgfqpoint{2.910852in}{3.120745in}}{\pgfqpoint{2.913166in}{3.115159in}}{\pgfqpoint{2.917284in}{3.111041in}}%
\pgfpathcurveto{\pgfqpoint{2.921402in}{3.106923in}}{\pgfqpoint{2.926988in}{3.104609in}}{\pgfqpoint{2.932812in}{3.104609in}}%
\pgfpathlineto{\pgfqpoint{2.932812in}{3.104609in}}%
\pgfpathclose%
\pgfusepath{stroke,fill}%
\end{pgfscope}%
\begin{pgfscope}%
\pgfpathrectangle{\pgfqpoint{1.073501in}{0.880000in}}{\pgfqpoint{6.052998in}{6.160000in}}%
\pgfusepath{clip}%
\pgfsetbuttcap%
\pgfsetroundjoin%
\definecolor{currentfill}{rgb}{0.200000,0.800000,0.200000}%
\pgfsetfillcolor{currentfill}%
\pgfsetlinewidth{1.003750pt}%
\definecolor{currentstroke}{rgb}{0.200000,0.800000,0.200000}%
\pgfsetstrokecolor{currentstroke}%
\pgfsetdash{}{0pt}%
\pgfpathmoveto{\pgfqpoint{2.833552in}{2.909705in}}%
\pgfpathcurveto{\pgfqpoint{2.839376in}{2.909705in}}{\pgfqpoint{2.844962in}{2.912019in}}{\pgfqpoint{2.849080in}{2.916137in}}%
\pgfpathcurveto{\pgfqpoint{2.853199in}{2.920255in}}{\pgfqpoint{2.855512in}{2.925841in}}{\pgfqpoint{2.855512in}{2.931665in}}%
\pgfpathcurveto{\pgfqpoint{2.855512in}{2.937489in}}{\pgfqpoint{2.853199in}{2.943075in}}{\pgfqpoint{2.849080in}{2.947193in}}%
\pgfpathcurveto{\pgfqpoint{2.844962in}{2.951312in}}{\pgfqpoint{2.839376in}{2.953625in}}{\pgfqpoint{2.833552in}{2.953625in}}%
\pgfpathcurveto{\pgfqpoint{2.827728in}{2.953625in}}{\pgfqpoint{2.822142in}{2.951312in}}{\pgfqpoint{2.818024in}{2.947193in}}%
\pgfpathcurveto{\pgfqpoint{2.813906in}{2.943075in}}{\pgfqpoint{2.811592in}{2.937489in}}{\pgfqpoint{2.811592in}{2.931665in}}%
\pgfpathcurveto{\pgfqpoint{2.811592in}{2.925841in}}{\pgfqpoint{2.813906in}{2.920255in}}{\pgfqpoint{2.818024in}{2.916137in}}%
\pgfpathcurveto{\pgfqpoint{2.822142in}{2.912019in}}{\pgfqpoint{2.827728in}{2.909705in}}{\pgfqpoint{2.833552in}{2.909705in}}%
\pgfpathlineto{\pgfqpoint{2.833552in}{2.909705in}}%
\pgfpathclose%
\pgfusepath{stroke,fill}%
\end{pgfscope}%
\begin{pgfscope}%
\pgfpathrectangle{\pgfqpoint{1.073501in}{0.880000in}}{\pgfqpoint{6.052998in}{6.160000in}}%
\pgfusepath{clip}%
\pgfsetbuttcap%
\pgfsetroundjoin%
\definecolor{currentfill}{rgb}{0.200000,0.800000,0.200000}%
\pgfsetfillcolor{currentfill}%
\pgfsetlinewidth{1.003750pt}%
\definecolor{currentstroke}{rgb}{0.200000,0.800000,0.200000}%
\pgfsetstrokecolor{currentstroke}%
\pgfsetdash{}{0pt}%
\pgfpathmoveto{\pgfqpoint{3.058216in}{2.970821in}}%
\pgfpathcurveto{\pgfqpoint{3.064040in}{2.970821in}}{\pgfqpoint{3.069627in}{2.973135in}}{\pgfqpoint{3.073745in}{2.977253in}}%
\pgfpathcurveto{\pgfqpoint{3.077863in}{2.981371in}}{\pgfqpoint{3.080177in}{2.986957in}}{\pgfqpoint{3.080177in}{2.992781in}}%
\pgfpathcurveto{\pgfqpoint{3.080177in}{2.998605in}}{\pgfqpoint{3.077863in}{3.004191in}}{\pgfqpoint{3.073745in}{3.008309in}}%
\pgfpathcurveto{\pgfqpoint{3.069627in}{3.012428in}}{\pgfqpoint{3.064040in}{3.014741in}}{\pgfqpoint{3.058216in}{3.014741in}}%
\pgfpathcurveto{\pgfqpoint{3.052393in}{3.014741in}}{\pgfqpoint{3.046806in}{3.012428in}}{\pgfqpoint{3.042688in}{3.008309in}}%
\pgfpathcurveto{\pgfqpoint{3.038570in}{3.004191in}}{\pgfqpoint{3.036256in}{2.998605in}}{\pgfqpoint{3.036256in}{2.992781in}}%
\pgfpathcurveto{\pgfqpoint{3.036256in}{2.986957in}}{\pgfqpoint{3.038570in}{2.981371in}}{\pgfqpoint{3.042688in}{2.977253in}}%
\pgfpathcurveto{\pgfqpoint{3.046806in}{2.973135in}}{\pgfqpoint{3.052393in}{2.970821in}}{\pgfqpoint{3.058216in}{2.970821in}}%
\pgfpathlineto{\pgfqpoint{3.058216in}{2.970821in}}%
\pgfpathclose%
\pgfusepath{stroke,fill}%
\end{pgfscope}%
\begin{pgfscope}%
\pgfpathrectangle{\pgfqpoint{1.073501in}{0.880000in}}{\pgfqpoint{6.052998in}{6.160000in}}%
\pgfusepath{clip}%
\pgfsetbuttcap%
\pgfsetroundjoin%
\definecolor{currentfill}{rgb}{0.200000,0.800000,0.200000}%
\pgfsetfillcolor{currentfill}%
\pgfsetlinewidth{1.003750pt}%
\definecolor{currentstroke}{rgb}{0.200000,0.800000,0.200000}%
\pgfsetstrokecolor{currentstroke}%
\pgfsetdash{}{0pt}%
\pgfpathmoveto{\pgfqpoint{3.147222in}{2.931755in}}%
\pgfpathcurveto{\pgfqpoint{3.153046in}{2.931755in}}{\pgfqpoint{3.158632in}{2.934069in}}{\pgfqpoint{3.162751in}{2.938187in}}%
\pgfpathcurveto{\pgfqpoint{3.166869in}{2.942305in}}{\pgfqpoint{3.169183in}{2.947891in}}{\pgfqpoint{3.169183in}{2.953715in}}%
\pgfpathcurveto{\pgfqpoint{3.169183in}{2.959539in}}{\pgfqpoint{3.166869in}{2.965125in}}{\pgfqpoint{3.162751in}{2.969243in}}%
\pgfpathcurveto{\pgfqpoint{3.158632in}{2.973361in}}{\pgfqpoint{3.153046in}{2.975675in}}{\pgfqpoint{3.147222in}{2.975675in}}%
\pgfpathcurveto{\pgfqpoint{3.141398in}{2.975675in}}{\pgfqpoint{3.135812in}{2.973361in}}{\pgfqpoint{3.131694in}{2.969243in}}%
\pgfpathcurveto{\pgfqpoint{3.127576in}{2.965125in}}{\pgfqpoint{3.125262in}{2.959539in}}{\pgfqpoint{3.125262in}{2.953715in}}%
\pgfpathcurveto{\pgfqpoint{3.125262in}{2.947891in}}{\pgfqpoint{3.127576in}{2.942305in}}{\pgfqpoint{3.131694in}{2.938187in}}%
\pgfpathcurveto{\pgfqpoint{3.135812in}{2.934069in}}{\pgfqpoint{3.141398in}{2.931755in}}{\pgfqpoint{3.147222in}{2.931755in}}%
\pgfpathlineto{\pgfqpoint{3.147222in}{2.931755in}}%
\pgfpathclose%
\pgfusepath{stroke,fill}%
\end{pgfscope}%
\begin{pgfscope}%
\pgfpathrectangle{\pgfqpoint{1.073501in}{0.880000in}}{\pgfqpoint{6.052998in}{6.160000in}}%
\pgfusepath{clip}%
\pgfsetbuttcap%
\pgfsetroundjoin%
\definecolor{currentfill}{rgb}{0.200000,0.800000,0.200000}%
\pgfsetfillcolor{currentfill}%
\pgfsetlinewidth{1.003750pt}%
\definecolor{currentstroke}{rgb}{0.200000,0.800000,0.200000}%
\pgfsetstrokecolor{currentstroke}%
\pgfsetdash{}{0pt}%
\pgfpathmoveto{\pgfqpoint{3.260758in}{2.927699in}}%
\pgfpathcurveto{\pgfqpoint{3.266582in}{2.927699in}}{\pgfqpoint{3.272168in}{2.930012in}}{\pgfqpoint{3.276286in}{2.934131in}}%
\pgfpathcurveto{\pgfqpoint{3.280405in}{2.938249in}}{\pgfqpoint{3.282718in}{2.943835in}}{\pgfqpoint{3.282718in}{2.949659in}}%
\pgfpathcurveto{\pgfqpoint{3.282718in}{2.955483in}}{\pgfqpoint{3.280405in}{2.961069in}}{\pgfqpoint{3.276286in}{2.965187in}}%
\pgfpathcurveto{\pgfqpoint{3.272168in}{2.969305in}}{\pgfqpoint{3.266582in}{2.971619in}}{\pgfqpoint{3.260758in}{2.971619in}}%
\pgfpathcurveto{\pgfqpoint{3.254934in}{2.971619in}}{\pgfqpoint{3.249348in}{2.969305in}}{\pgfqpoint{3.245230in}{2.965187in}}%
\pgfpathcurveto{\pgfqpoint{3.241112in}{2.961069in}}{\pgfqpoint{3.238798in}{2.955483in}}{\pgfqpoint{3.238798in}{2.949659in}}%
\pgfpathcurveto{\pgfqpoint{3.238798in}{2.943835in}}{\pgfqpoint{3.241112in}{2.938249in}}{\pgfqpoint{3.245230in}{2.934131in}}%
\pgfpathcurveto{\pgfqpoint{3.249348in}{2.930012in}}{\pgfqpoint{3.254934in}{2.927699in}}{\pgfqpoint{3.260758in}{2.927699in}}%
\pgfpathlineto{\pgfqpoint{3.260758in}{2.927699in}}%
\pgfpathclose%
\pgfusepath{stroke,fill}%
\end{pgfscope}%
\begin{pgfscope}%
\pgfpathrectangle{\pgfqpoint{1.073501in}{0.880000in}}{\pgfqpoint{6.052998in}{6.160000in}}%
\pgfusepath{clip}%
\pgfsetbuttcap%
\pgfsetroundjoin%
\definecolor{currentfill}{rgb}{0.200000,0.800000,0.200000}%
\pgfsetfillcolor{currentfill}%
\pgfsetlinewidth{1.003750pt}%
\definecolor{currentstroke}{rgb}{0.200000,0.800000,0.200000}%
\pgfsetstrokecolor{currentstroke}%
\pgfsetdash{}{0pt}%
\pgfpathmoveto{\pgfqpoint{3.247394in}{2.771828in}}%
\pgfpathcurveto{\pgfqpoint{3.253218in}{2.771828in}}{\pgfqpoint{3.258804in}{2.774142in}}{\pgfqpoint{3.262922in}{2.778260in}}%
\pgfpathcurveto{\pgfqpoint{3.267040in}{2.782379in}}{\pgfqpoint{3.269354in}{2.787965in}}{\pgfqpoint{3.269354in}{2.793789in}}%
\pgfpathcurveto{\pgfqpoint{3.269354in}{2.799613in}}{\pgfqpoint{3.267040in}{2.805199in}}{\pgfqpoint{3.262922in}{2.809317in}}%
\pgfpathcurveto{\pgfqpoint{3.258804in}{2.813435in}}{\pgfqpoint{3.253218in}{2.815749in}}{\pgfqpoint{3.247394in}{2.815749in}}%
\pgfpathcurveto{\pgfqpoint{3.241570in}{2.815749in}}{\pgfqpoint{3.235984in}{2.813435in}}{\pgfqpoint{3.231866in}{2.809317in}}%
\pgfpathcurveto{\pgfqpoint{3.227748in}{2.805199in}}{\pgfqpoint{3.225434in}{2.799613in}}{\pgfqpoint{3.225434in}{2.793789in}}%
\pgfpathcurveto{\pgfqpoint{3.225434in}{2.787965in}}{\pgfqpoint{3.227748in}{2.782379in}}{\pgfqpoint{3.231866in}{2.778260in}}%
\pgfpathcurveto{\pgfqpoint{3.235984in}{2.774142in}}{\pgfqpoint{3.241570in}{2.771828in}}{\pgfqpoint{3.247394in}{2.771828in}}%
\pgfpathlineto{\pgfqpoint{3.247394in}{2.771828in}}%
\pgfpathclose%
\pgfusepath{stroke,fill}%
\end{pgfscope}%
\begin{pgfscope}%
\pgfpathrectangle{\pgfqpoint{1.073501in}{0.880000in}}{\pgfqpoint{6.052998in}{6.160000in}}%
\pgfusepath{clip}%
\pgfsetbuttcap%
\pgfsetroundjoin%
\definecolor{currentfill}{rgb}{0.200000,0.800000,0.200000}%
\pgfsetfillcolor{currentfill}%
\pgfsetlinewidth{1.003750pt}%
\definecolor{currentstroke}{rgb}{0.200000,0.800000,0.200000}%
\pgfsetstrokecolor{currentstroke}%
\pgfsetdash{}{0pt}%
\pgfpathmoveto{\pgfqpoint{3.311871in}{2.702700in}}%
\pgfpathcurveto{\pgfqpoint{3.317695in}{2.702700in}}{\pgfqpoint{3.323281in}{2.705014in}}{\pgfqpoint{3.327399in}{2.709132in}}%
\pgfpathcurveto{\pgfqpoint{3.331517in}{2.713250in}}{\pgfqpoint{3.333831in}{2.718836in}}{\pgfqpoint{3.333831in}{2.724660in}}%
\pgfpathcurveto{\pgfqpoint{3.333831in}{2.730484in}}{\pgfqpoint{3.331517in}{2.736071in}}{\pgfqpoint{3.327399in}{2.740189in}}%
\pgfpathcurveto{\pgfqpoint{3.323281in}{2.744307in}}{\pgfqpoint{3.317695in}{2.746621in}}{\pgfqpoint{3.311871in}{2.746621in}}%
\pgfpathcurveto{\pgfqpoint{3.306047in}{2.746621in}}{\pgfqpoint{3.300461in}{2.744307in}}{\pgfqpoint{3.296343in}{2.740189in}}%
\pgfpathcurveto{\pgfqpoint{3.292225in}{2.736071in}}{\pgfqpoint{3.289911in}{2.730484in}}{\pgfqpoint{3.289911in}{2.724660in}}%
\pgfpathcurveto{\pgfqpoint{3.289911in}{2.718836in}}{\pgfqpoint{3.292225in}{2.713250in}}{\pgfqpoint{3.296343in}{2.709132in}}%
\pgfpathcurveto{\pgfqpoint{3.300461in}{2.705014in}}{\pgfqpoint{3.306047in}{2.702700in}}{\pgfqpoint{3.311871in}{2.702700in}}%
\pgfpathlineto{\pgfqpoint{3.311871in}{2.702700in}}%
\pgfpathclose%
\pgfusepath{stroke,fill}%
\end{pgfscope}%
\begin{pgfscope}%
\pgfpathrectangle{\pgfqpoint{1.073501in}{0.880000in}}{\pgfqpoint{6.052998in}{6.160000in}}%
\pgfusepath{clip}%
\pgfsetbuttcap%
\pgfsetroundjoin%
\definecolor{currentfill}{rgb}{0.200000,0.800000,0.200000}%
\pgfsetfillcolor{currentfill}%
\pgfsetlinewidth{1.003750pt}%
\definecolor{currentstroke}{rgb}{0.200000,0.800000,0.200000}%
\pgfsetstrokecolor{currentstroke}%
\pgfsetdash{}{0pt}%
\pgfpathmoveto{\pgfqpoint{3.421677in}{2.706543in}}%
\pgfpathcurveto{\pgfqpoint{3.427501in}{2.706543in}}{\pgfqpoint{3.433087in}{2.708857in}}{\pgfqpoint{3.437205in}{2.712976in}}%
\pgfpathcurveto{\pgfqpoint{3.441323in}{2.717094in}}{\pgfqpoint{3.443637in}{2.722680in}}{\pgfqpoint{3.443637in}{2.728504in}}%
\pgfpathcurveto{\pgfqpoint{3.443637in}{2.734328in}}{\pgfqpoint{3.441323in}{2.739914in}}{\pgfqpoint{3.437205in}{2.744032in}}%
\pgfpathcurveto{\pgfqpoint{3.433087in}{2.748150in}}{\pgfqpoint{3.427501in}{2.750464in}}{\pgfqpoint{3.421677in}{2.750464in}}%
\pgfpathcurveto{\pgfqpoint{3.415853in}{2.750464in}}{\pgfqpoint{3.410267in}{2.748150in}}{\pgfqpoint{3.406149in}{2.744032in}}%
\pgfpathcurveto{\pgfqpoint{3.402031in}{2.739914in}}{\pgfqpoint{3.399717in}{2.734328in}}{\pgfqpoint{3.399717in}{2.728504in}}%
\pgfpathcurveto{\pgfqpoint{3.399717in}{2.722680in}}{\pgfqpoint{3.402031in}{2.717094in}}{\pgfqpoint{3.406149in}{2.712976in}}%
\pgfpathcurveto{\pgfqpoint{3.410267in}{2.708857in}}{\pgfqpoint{3.415853in}{2.706543in}}{\pgfqpoint{3.421677in}{2.706543in}}%
\pgfpathlineto{\pgfqpoint{3.421677in}{2.706543in}}%
\pgfpathclose%
\pgfusepath{stroke,fill}%
\end{pgfscope}%
\begin{pgfscope}%
\pgfpathrectangle{\pgfqpoint{1.073501in}{0.880000in}}{\pgfqpoint{6.052998in}{6.160000in}}%
\pgfusepath{clip}%
\pgfsetbuttcap%
\pgfsetroundjoin%
\definecolor{currentfill}{rgb}{0.200000,0.800000,0.200000}%
\pgfsetfillcolor{currentfill}%
\pgfsetlinewidth{1.003750pt}%
\definecolor{currentstroke}{rgb}{0.200000,0.800000,0.200000}%
\pgfsetstrokecolor{currentstroke}%
\pgfsetdash{}{0pt}%
\pgfpathmoveto{\pgfqpoint{3.495502in}{2.654422in}}%
\pgfpathcurveto{\pgfqpoint{3.501326in}{2.654422in}}{\pgfqpoint{3.506912in}{2.656736in}}{\pgfqpoint{3.511030in}{2.660854in}}%
\pgfpathcurveto{\pgfqpoint{3.515148in}{2.664972in}}{\pgfqpoint{3.517462in}{2.670558in}}{\pgfqpoint{3.517462in}{2.676382in}}%
\pgfpathcurveto{\pgfqpoint{3.517462in}{2.682206in}}{\pgfqpoint{3.515148in}{2.687792in}}{\pgfqpoint{3.511030in}{2.691911in}}%
\pgfpathcurveto{\pgfqpoint{3.506912in}{2.696029in}}{\pgfqpoint{3.501326in}{2.698343in}}{\pgfqpoint{3.495502in}{2.698343in}}%
\pgfpathcurveto{\pgfqpoint{3.489678in}{2.698343in}}{\pgfqpoint{3.484092in}{2.696029in}}{\pgfqpoint{3.479974in}{2.691911in}}%
\pgfpathcurveto{\pgfqpoint{3.475856in}{2.687792in}}{\pgfqpoint{3.473542in}{2.682206in}}{\pgfqpoint{3.473542in}{2.676382in}}%
\pgfpathcurveto{\pgfqpoint{3.473542in}{2.670558in}}{\pgfqpoint{3.475856in}{2.664972in}}{\pgfqpoint{3.479974in}{2.660854in}}%
\pgfpathcurveto{\pgfqpoint{3.484092in}{2.656736in}}{\pgfqpoint{3.489678in}{2.654422in}}{\pgfqpoint{3.495502in}{2.654422in}}%
\pgfpathlineto{\pgfqpoint{3.495502in}{2.654422in}}%
\pgfpathclose%
\pgfusepath{stroke,fill}%
\end{pgfscope}%
\begin{pgfscope}%
\pgfpathrectangle{\pgfqpoint{1.073501in}{0.880000in}}{\pgfqpoint{6.052998in}{6.160000in}}%
\pgfusepath{clip}%
\pgfsetbuttcap%
\pgfsetroundjoin%
\definecolor{currentfill}{rgb}{0.200000,0.800000,0.200000}%
\pgfsetfillcolor{currentfill}%
\pgfsetlinewidth{1.003750pt}%
\definecolor{currentstroke}{rgb}{0.200000,0.800000,0.200000}%
\pgfsetstrokecolor{currentstroke}%
\pgfsetdash{}{0pt}%
\pgfpathmoveto{\pgfqpoint{3.580565in}{2.624561in}}%
\pgfpathcurveto{\pgfqpoint{3.586389in}{2.624561in}}{\pgfqpoint{3.591975in}{2.626874in}}{\pgfqpoint{3.596093in}{2.630993in}}%
\pgfpathcurveto{\pgfqpoint{3.600212in}{2.635111in}}{\pgfqpoint{3.602525in}{2.640697in}}{\pgfqpoint{3.602525in}{2.646521in}}%
\pgfpathcurveto{\pgfqpoint{3.602525in}{2.652345in}}{\pgfqpoint{3.600212in}{2.657931in}}{\pgfqpoint{3.596093in}{2.662049in}}%
\pgfpathcurveto{\pgfqpoint{3.591975in}{2.666167in}}{\pgfqpoint{3.586389in}{2.668481in}}{\pgfqpoint{3.580565in}{2.668481in}}%
\pgfpathcurveto{\pgfqpoint{3.574741in}{2.668481in}}{\pgfqpoint{3.569155in}{2.666167in}}{\pgfqpoint{3.565037in}{2.662049in}}%
\pgfpathcurveto{\pgfqpoint{3.560919in}{2.657931in}}{\pgfqpoint{3.558605in}{2.652345in}}{\pgfqpoint{3.558605in}{2.646521in}}%
\pgfpathcurveto{\pgfqpoint{3.558605in}{2.640697in}}{\pgfqpoint{3.560919in}{2.635111in}}{\pgfqpoint{3.565037in}{2.630993in}}%
\pgfpathcurveto{\pgfqpoint{3.569155in}{2.626874in}}{\pgfqpoint{3.574741in}{2.624561in}}{\pgfqpoint{3.580565in}{2.624561in}}%
\pgfpathlineto{\pgfqpoint{3.580565in}{2.624561in}}%
\pgfpathclose%
\pgfusepath{stroke,fill}%
\end{pgfscope}%
\begin{pgfscope}%
\pgfpathrectangle{\pgfqpoint{1.073501in}{0.880000in}}{\pgfqpoint{6.052998in}{6.160000in}}%
\pgfusepath{clip}%
\pgfsetbuttcap%
\pgfsetroundjoin%
\definecolor{currentfill}{rgb}{0.200000,0.800000,0.200000}%
\pgfsetfillcolor{currentfill}%
\pgfsetlinewidth{1.003750pt}%
\definecolor{currentstroke}{rgb}{0.200000,0.800000,0.200000}%
\pgfsetstrokecolor{currentstroke}%
\pgfsetdash{}{0pt}%
\pgfpathmoveto{\pgfqpoint{3.677877in}{2.632154in}}%
\pgfpathcurveto{\pgfqpoint{3.683701in}{2.632154in}}{\pgfqpoint{3.689287in}{2.634467in}}{\pgfqpoint{3.693405in}{2.638586in}}%
\pgfpathcurveto{\pgfqpoint{3.697524in}{2.642704in}}{\pgfqpoint{3.699837in}{2.648290in}}{\pgfqpoint{3.699837in}{2.654114in}}%
\pgfpathcurveto{\pgfqpoint{3.699837in}{2.659938in}}{\pgfqpoint{3.697524in}{2.665524in}}{\pgfqpoint{3.693405in}{2.669642in}}%
\pgfpathcurveto{\pgfqpoint{3.689287in}{2.673760in}}{\pgfqpoint{3.683701in}{2.676074in}}{\pgfqpoint{3.677877in}{2.676074in}}%
\pgfpathcurveto{\pgfqpoint{3.672053in}{2.676074in}}{\pgfqpoint{3.666467in}{2.673760in}}{\pgfqpoint{3.662349in}{2.669642in}}%
\pgfpathcurveto{\pgfqpoint{3.658231in}{2.665524in}}{\pgfqpoint{3.655917in}{2.659938in}}{\pgfqpoint{3.655917in}{2.654114in}}%
\pgfpathcurveto{\pgfqpoint{3.655917in}{2.648290in}}{\pgfqpoint{3.658231in}{2.642704in}}{\pgfqpoint{3.662349in}{2.638586in}}%
\pgfpathcurveto{\pgfqpoint{3.666467in}{2.634467in}}{\pgfqpoint{3.672053in}{2.632154in}}{\pgfqpoint{3.677877in}{2.632154in}}%
\pgfpathlineto{\pgfqpoint{3.677877in}{2.632154in}}%
\pgfpathclose%
\pgfusepath{stroke,fill}%
\end{pgfscope}%
\begin{pgfscope}%
\pgfpathrectangle{\pgfqpoint{1.073501in}{0.880000in}}{\pgfqpoint{6.052998in}{6.160000in}}%
\pgfusepath{clip}%
\pgfsetbuttcap%
\pgfsetroundjoin%
\definecolor{currentfill}{rgb}{0.200000,0.800000,0.200000}%
\pgfsetfillcolor{currentfill}%
\pgfsetlinewidth{1.003750pt}%
\definecolor{currentstroke}{rgb}{0.200000,0.800000,0.200000}%
\pgfsetstrokecolor{currentstroke}%
\pgfsetdash{}{0pt}%
\pgfpathmoveto{\pgfqpoint{3.760288in}{2.604415in}}%
\pgfpathcurveto{\pgfqpoint{3.766112in}{2.604415in}}{\pgfqpoint{3.771698in}{2.606729in}}{\pgfqpoint{3.775816in}{2.610847in}}%
\pgfpathcurveto{\pgfqpoint{3.779934in}{2.614965in}}{\pgfqpoint{3.782248in}{2.620551in}}{\pgfqpoint{3.782248in}{2.626375in}}%
\pgfpathcurveto{\pgfqpoint{3.782248in}{2.632199in}}{\pgfqpoint{3.779934in}{2.637785in}}{\pgfqpoint{3.775816in}{2.641903in}}%
\pgfpathcurveto{\pgfqpoint{3.771698in}{2.646021in}}{\pgfqpoint{3.766112in}{2.648335in}}{\pgfqpoint{3.760288in}{2.648335in}}%
\pgfpathcurveto{\pgfqpoint{3.754464in}{2.648335in}}{\pgfqpoint{3.748878in}{2.646021in}}{\pgfqpoint{3.744760in}{2.641903in}}%
\pgfpathcurveto{\pgfqpoint{3.740642in}{2.637785in}}{\pgfqpoint{3.738328in}{2.632199in}}{\pgfqpoint{3.738328in}{2.626375in}}%
\pgfpathcurveto{\pgfqpoint{3.738328in}{2.620551in}}{\pgfqpoint{3.740642in}{2.614965in}}{\pgfqpoint{3.744760in}{2.610847in}}%
\pgfpathcurveto{\pgfqpoint{3.748878in}{2.606729in}}{\pgfqpoint{3.754464in}{2.604415in}}{\pgfqpoint{3.760288in}{2.604415in}}%
\pgfpathlineto{\pgfqpoint{3.760288in}{2.604415in}}%
\pgfpathclose%
\pgfusepath{stroke,fill}%
\end{pgfscope}%
\begin{pgfscope}%
\pgfpathrectangle{\pgfqpoint{1.073501in}{0.880000in}}{\pgfqpoint{6.052998in}{6.160000in}}%
\pgfusepath{clip}%
\pgfsetbuttcap%
\pgfsetroundjoin%
\definecolor{currentfill}{rgb}{0.200000,0.800000,0.200000}%
\pgfsetfillcolor{currentfill}%
\pgfsetlinewidth{1.003750pt}%
\definecolor{currentstroke}{rgb}{0.200000,0.800000,0.200000}%
\pgfsetstrokecolor{currentstroke}%
\pgfsetdash{}{0pt}%
\pgfpathmoveto{\pgfqpoint{3.866674in}{2.688301in}}%
\pgfpathcurveto{\pgfqpoint{3.872498in}{2.688301in}}{\pgfqpoint{3.878084in}{2.690615in}}{\pgfqpoint{3.882203in}{2.694733in}}%
\pgfpathcurveto{\pgfqpoint{3.886321in}{2.698852in}}{\pgfqpoint{3.888635in}{2.704438in}}{\pgfqpoint{3.888635in}{2.710262in}}%
\pgfpathcurveto{\pgfqpoint{3.888635in}{2.716086in}}{\pgfqpoint{3.886321in}{2.721672in}}{\pgfqpoint{3.882203in}{2.725790in}}%
\pgfpathcurveto{\pgfqpoint{3.878084in}{2.729908in}}{\pgfqpoint{3.872498in}{2.732222in}}{\pgfqpoint{3.866674in}{2.732222in}}%
\pgfpathcurveto{\pgfqpoint{3.860850in}{2.732222in}}{\pgfqpoint{3.855264in}{2.729908in}}{\pgfqpoint{3.851146in}{2.725790in}}%
\pgfpathcurveto{\pgfqpoint{3.847028in}{2.721672in}}{\pgfqpoint{3.844714in}{2.716086in}}{\pgfqpoint{3.844714in}{2.710262in}}%
\pgfpathcurveto{\pgfqpoint{3.844714in}{2.704438in}}{\pgfqpoint{3.847028in}{2.698852in}}{\pgfqpoint{3.851146in}{2.694733in}}%
\pgfpathcurveto{\pgfqpoint{3.855264in}{2.690615in}}{\pgfqpoint{3.860850in}{2.688301in}}{\pgfqpoint{3.866674in}{2.688301in}}%
\pgfpathlineto{\pgfqpoint{3.866674in}{2.688301in}}%
\pgfpathclose%
\pgfusepath{stroke,fill}%
\end{pgfscope}%
\begin{pgfscope}%
\pgfpathrectangle{\pgfqpoint{1.073501in}{0.880000in}}{\pgfqpoint{6.052998in}{6.160000in}}%
\pgfusepath{clip}%
\pgfsetbuttcap%
\pgfsetroundjoin%
\definecolor{currentfill}{rgb}{0.200000,0.800000,0.200000}%
\pgfsetfillcolor{currentfill}%
\pgfsetlinewidth{1.003750pt}%
\definecolor{currentstroke}{rgb}{0.200000,0.800000,0.200000}%
\pgfsetstrokecolor{currentstroke}%
\pgfsetdash{}{0pt}%
\pgfpathmoveto{\pgfqpoint{3.931448in}{2.575368in}}%
\pgfpathcurveto{\pgfqpoint{3.937272in}{2.575368in}}{\pgfqpoint{3.942858in}{2.577682in}}{\pgfqpoint{3.946976in}{2.581800in}}%
\pgfpathcurveto{\pgfqpoint{3.951094in}{2.585918in}}{\pgfqpoint{3.953408in}{2.591504in}}{\pgfqpoint{3.953408in}{2.597328in}}%
\pgfpathcurveto{\pgfqpoint{3.953408in}{2.603152in}}{\pgfqpoint{3.951094in}{2.608738in}}{\pgfqpoint{3.946976in}{2.612856in}}%
\pgfpathcurveto{\pgfqpoint{3.942858in}{2.616975in}}{\pgfqpoint{3.937272in}{2.619288in}}{\pgfqpoint{3.931448in}{2.619288in}}%
\pgfpathcurveto{\pgfqpoint{3.925624in}{2.619288in}}{\pgfqpoint{3.920038in}{2.616975in}}{\pgfqpoint{3.915919in}{2.612856in}}%
\pgfpathcurveto{\pgfqpoint{3.911801in}{2.608738in}}{\pgfqpoint{3.909487in}{2.603152in}}{\pgfqpoint{3.909487in}{2.597328in}}%
\pgfpathcurveto{\pgfqpoint{3.909487in}{2.591504in}}{\pgfqpoint{3.911801in}{2.585918in}}{\pgfqpoint{3.915919in}{2.581800in}}%
\pgfpathcurveto{\pgfqpoint{3.920038in}{2.577682in}}{\pgfqpoint{3.925624in}{2.575368in}}{\pgfqpoint{3.931448in}{2.575368in}}%
\pgfpathlineto{\pgfqpoint{3.931448in}{2.575368in}}%
\pgfpathclose%
\pgfusepath{stroke,fill}%
\end{pgfscope}%
\begin{pgfscope}%
\pgfpathrectangle{\pgfqpoint{1.073501in}{0.880000in}}{\pgfqpoint{6.052998in}{6.160000in}}%
\pgfusepath{clip}%
\pgfsetbuttcap%
\pgfsetroundjoin%
\definecolor{currentfill}{rgb}{0.200000,0.800000,0.200000}%
\pgfsetfillcolor{currentfill}%
\pgfsetlinewidth{1.003750pt}%
\definecolor{currentstroke}{rgb}{0.200000,0.800000,0.200000}%
\pgfsetstrokecolor{currentstroke}%
\pgfsetdash{}{0pt}%
\pgfpathmoveto{\pgfqpoint{4.010523in}{2.478162in}}%
\pgfpathcurveto{\pgfqpoint{4.016346in}{2.478162in}}{\pgfqpoint{4.021933in}{2.480476in}}{\pgfqpoint{4.026051in}{2.484594in}}%
\pgfpathcurveto{\pgfqpoint{4.030169in}{2.488712in}}{\pgfqpoint{4.032483in}{2.494298in}}{\pgfqpoint{4.032483in}{2.500122in}}%
\pgfpathcurveto{\pgfqpoint{4.032483in}{2.505946in}}{\pgfqpoint{4.030169in}{2.511532in}}{\pgfqpoint{4.026051in}{2.515650in}}%
\pgfpathcurveto{\pgfqpoint{4.021933in}{2.519768in}}{\pgfqpoint{4.016346in}{2.522082in}}{\pgfqpoint{4.010523in}{2.522082in}}%
\pgfpathcurveto{\pgfqpoint{4.004699in}{2.522082in}}{\pgfqpoint{3.999112in}{2.519768in}}{\pgfqpoint{3.994994in}{2.515650in}}%
\pgfpathcurveto{\pgfqpoint{3.990876in}{2.511532in}}{\pgfqpoint{3.988562in}{2.505946in}}{\pgfqpoint{3.988562in}{2.500122in}}%
\pgfpathcurveto{\pgfqpoint{3.988562in}{2.494298in}}{\pgfqpoint{3.990876in}{2.488712in}}{\pgfqpoint{3.994994in}{2.484594in}}%
\pgfpathcurveto{\pgfqpoint{3.999112in}{2.480476in}}{\pgfqpoint{4.004699in}{2.478162in}}{\pgfqpoint{4.010523in}{2.478162in}}%
\pgfpathlineto{\pgfqpoint{4.010523in}{2.478162in}}%
\pgfpathclose%
\pgfusepath{stroke,fill}%
\end{pgfscope}%
\begin{pgfscope}%
\pgfpathrectangle{\pgfqpoint{1.073501in}{0.880000in}}{\pgfqpoint{6.052998in}{6.160000in}}%
\pgfusepath{clip}%
\pgfsetbuttcap%
\pgfsetroundjoin%
\definecolor{currentfill}{rgb}{0.200000,0.800000,0.200000}%
\pgfsetfillcolor{currentfill}%
\pgfsetlinewidth{1.003750pt}%
\definecolor{currentstroke}{rgb}{0.200000,0.800000,0.200000}%
\pgfsetstrokecolor{currentstroke}%
\pgfsetdash{}{0pt}%
\pgfpathmoveto{\pgfqpoint{4.103836in}{2.553650in}}%
\pgfpathcurveto{\pgfqpoint{4.109660in}{2.553650in}}{\pgfqpoint{4.115246in}{2.555964in}}{\pgfqpoint{4.119364in}{2.560082in}}%
\pgfpathcurveto{\pgfqpoint{4.123482in}{2.564200in}}{\pgfqpoint{4.125796in}{2.569786in}}{\pgfqpoint{4.125796in}{2.575610in}}%
\pgfpathcurveto{\pgfqpoint{4.125796in}{2.581434in}}{\pgfqpoint{4.123482in}{2.587020in}}{\pgfqpoint{4.119364in}{2.591138in}}%
\pgfpathcurveto{\pgfqpoint{4.115246in}{2.595257in}}{\pgfqpoint{4.109660in}{2.597570in}}{\pgfqpoint{4.103836in}{2.597570in}}%
\pgfpathcurveto{\pgfqpoint{4.098012in}{2.597570in}}{\pgfqpoint{4.092426in}{2.595257in}}{\pgfqpoint{4.088308in}{2.591138in}}%
\pgfpathcurveto{\pgfqpoint{4.084190in}{2.587020in}}{\pgfqpoint{4.081876in}{2.581434in}}{\pgfqpoint{4.081876in}{2.575610in}}%
\pgfpathcurveto{\pgfqpoint{4.081876in}{2.569786in}}{\pgfqpoint{4.084190in}{2.564200in}}{\pgfqpoint{4.088308in}{2.560082in}}%
\pgfpathcurveto{\pgfqpoint{4.092426in}{2.555964in}}{\pgfqpoint{4.098012in}{2.553650in}}{\pgfqpoint{4.103836in}{2.553650in}}%
\pgfpathlineto{\pgfqpoint{4.103836in}{2.553650in}}%
\pgfpathclose%
\pgfusepath{stroke,fill}%
\end{pgfscope}%
\begin{pgfscope}%
\pgfpathrectangle{\pgfqpoint{1.073501in}{0.880000in}}{\pgfqpoint{6.052998in}{6.160000in}}%
\pgfusepath{clip}%
\pgfsetbuttcap%
\pgfsetroundjoin%
\definecolor{currentfill}{rgb}{0.200000,0.800000,0.200000}%
\pgfsetfillcolor{currentfill}%
\pgfsetlinewidth{1.003750pt}%
\definecolor{currentstroke}{rgb}{0.200000,0.800000,0.200000}%
\pgfsetstrokecolor{currentstroke}%
\pgfsetdash{}{0pt}%
\pgfpathmoveto{\pgfqpoint{4.194784in}{2.473637in}}%
\pgfpathcurveto{\pgfqpoint{4.200607in}{2.473637in}}{\pgfqpoint{4.206194in}{2.475951in}}{\pgfqpoint{4.210312in}{2.480069in}}%
\pgfpathcurveto{\pgfqpoint{4.214430in}{2.484187in}}{\pgfqpoint{4.216744in}{2.489773in}}{\pgfqpoint{4.216744in}{2.495597in}}%
\pgfpathcurveto{\pgfqpoint{4.216744in}{2.501421in}}{\pgfqpoint{4.214430in}{2.507007in}}{\pgfqpoint{4.210312in}{2.511125in}}%
\pgfpathcurveto{\pgfqpoint{4.206194in}{2.515243in}}{\pgfqpoint{4.200607in}{2.517557in}}{\pgfqpoint{4.194784in}{2.517557in}}%
\pgfpathcurveto{\pgfqpoint{4.188960in}{2.517557in}}{\pgfqpoint{4.183373in}{2.515243in}}{\pgfqpoint{4.179255in}{2.511125in}}%
\pgfpathcurveto{\pgfqpoint{4.175137in}{2.507007in}}{\pgfqpoint{4.172823in}{2.501421in}}{\pgfqpoint{4.172823in}{2.495597in}}%
\pgfpathcurveto{\pgfqpoint{4.172823in}{2.489773in}}{\pgfqpoint{4.175137in}{2.484187in}}{\pgfqpoint{4.179255in}{2.480069in}}%
\pgfpathcurveto{\pgfqpoint{4.183373in}{2.475951in}}{\pgfqpoint{4.188960in}{2.473637in}}{\pgfqpoint{4.194784in}{2.473637in}}%
\pgfpathlineto{\pgfqpoint{4.194784in}{2.473637in}}%
\pgfpathclose%
\pgfusepath{stroke,fill}%
\end{pgfscope}%
\begin{pgfscope}%
\pgfpathrectangle{\pgfqpoint{1.073501in}{0.880000in}}{\pgfqpoint{6.052998in}{6.160000in}}%
\pgfusepath{clip}%
\pgfsetbuttcap%
\pgfsetroundjoin%
\definecolor{currentfill}{rgb}{0.200000,0.800000,0.200000}%
\pgfsetfillcolor{currentfill}%
\pgfsetlinewidth{1.003750pt}%
\definecolor{currentstroke}{rgb}{0.200000,0.800000,0.200000}%
\pgfsetstrokecolor{currentstroke}%
\pgfsetdash{}{0pt}%
\pgfpathmoveto{\pgfqpoint{4.272595in}{2.607833in}}%
\pgfpathcurveto{\pgfqpoint{4.278419in}{2.607833in}}{\pgfqpoint{4.284005in}{2.610146in}}{\pgfqpoint{4.288123in}{2.614265in}}%
\pgfpathcurveto{\pgfqpoint{4.292241in}{2.618383in}}{\pgfqpoint{4.294555in}{2.623969in}}{\pgfqpoint{4.294555in}{2.629793in}}%
\pgfpathcurveto{\pgfqpoint{4.294555in}{2.635617in}}{\pgfqpoint{4.292241in}{2.641203in}}{\pgfqpoint{4.288123in}{2.645321in}}%
\pgfpathcurveto{\pgfqpoint{4.284005in}{2.649439in}}{\pgfqpoint{4.278419in}{2.651753in}}{\pgfqpoint{4.272595in}{2.651753in}}%
\pgfpathcurveto{\pgfqpoint{4.266771in}{2.651753in}}{\pgfqpoint{4.261184in}{2.649439in}}{\pgfqpoint{4.257066in}{2.645321in}}%
\pgfpathcurveto{\pgfqpoint{4.252948in}{2.641203in}}{\pgfqpoint{4.250634in}{2.635617in}}{\pgfqpoint{4.250634in}{2.629793in}}%
\pgfpathcurveto{\pgfqpoint{4.250634in}{2.623969in}}{\pgfqpoint{4.252948in}{2.618383in}}{\pgfqpoint{4.257066in}{2.614265in}}%
\pgfpathcurveto{\pgfqpoint{4.261184in}{2.610146in}}{\pgfqpoint{4.266771in}{2.607833in}}{\pgfqpoint{4.272595in}{2.607833in}}%
\pgfpathlineto{\pgfqpoint{4.272595in}{2.607833in}}%
\pgfpathclose%
\pgfusepath{stroke,fill}%
\end{pgfscope}%
\begin{pgfscope}%
\pgfpathrectangle{\pgfqpoint{1.073501in}{0.880000in}}{\pgfqpoint{6.052998in}{6.160000in}}%
\pgfusepath{clip}%
\pgfsetbuttcap%
\pgfsetroundjoin%
\definecolor{currentfill}{rgb}{0.200000,0.800000,0.200000}%
\pgfsetfillcolor{currentfill}%
\pgfsetlinewidth{1.003750pt}%
\definecolor{currentstroke}{rgb}{0.200000,0.800000,0.200000}%
\pgfsetstrokecolor{currentstroke}%
\pgfsetdash{}{0pt}%
\pgfpathmoveto{\pgfqpoint{4.377566in}{2.496870in}}%
\pgfpathcurveto{\pgfqpoint{4.383390in}{2.496870in}}{\pgfqpoint{4.388976in}{2.499184in}}{\pgfqpoint{4.393094in}{2.503302in}}%
\pgfpathcurveto{\pgfqpoint{4.397212in}{2.507421in}}{\pgfqpoint{4.399526in}{2.513007in}}{\pgfqpoint{4.399526in}{2.518831in}}%
\pgfpathcurveto{\pgfqpoint{4.399526in}{2.524655in}}{\pgfqpoint{4.397212in}{2.530241in}}{\pgfqpoint{4.393094in}{2.534359in}}%
\pgfpathcurveto{\pgfqpoint{4.388976in}{2.538477in}}{\pgfqpoint{4.383390in}{2.540791in}}{\pgfqpoint{4.377566in}{2.540791in}}%
\pgfpathcurveto{\pgfqpoint{4.371742in}{2.540791in}}{\pgfqpoint{4.366156in}{2.538477in}}{\pgfqpoint{4.362038in}{2.534359in}}%
\pgfpathcurveto{\pgfqpoint{4.357920in}{2.530241in}}{\pgfqpoint{4.355606in}{2.524655in}}{\pgfqpoint{4.355606in}{2.518831in}}%
\pgfpathcurveto{\pgfqpoint{4.355606in}{2.513007in}}{\pgfqpoint{4.357920in}{2.507421in}}{\pgfqpoint{4.362038in}{2.503302in}}%
\pgfpathcurveto{\pgfqpoint{4.366156in}{2.499184in}}{\pgfqpoint{4.371742in}{2.496870in}}{\pgfqpoint{4.377566in}{2.496870in}}%
\pgfpathlineto{\pgfqpoint{4.377566in}{2.496870in}}%
\pgfpathclose%
\pgfusepath{stroke,fill}%
\end{pgfscope}%
\begin{pgfscope}%
\pgfpathrectangle{\pgfqpoint{1.073501in}{0.880000in}}{\pgfqpoint{6.052998in}{6.160000in}}%
\pgfusepath{clip}%
\pgfsetbuttcap%
\pgfsetroundjoin%
\definecolor{currentfill}{rgb}{0.200000,0.800000,0.200000}%
\pgfsetfillcolor{currentfill}%
\pgfsetlinewidth{1.003750pt}%
\definecolor{currentstroke}{rgb}{0.200000,0.800000,0.200000}%
\pgfsetstrokecolor{currentstroke}%
\pgfsetdash{}{0pt}%
\pgfpathmoveto{\pgfqpoint{4.453797in}{2.572979in}}%
\pgfpathcurveto{\pgfqpoint{4.459621in}{2.572979in}}{\pgfqpoint{4.465207in}{2.575293in}}{\pgfqpoint{4.469326in}{2.579411in}}%
\pgfpathcurveto{\pgfqpoint{4.473444in}{2.583529in}}{\pgfqpoint{4.475758in}{2.589115in}}{\pgfqpoint{4.475758in}{2.594939in}}%
\pgfpathcurveto{\pgfqpoint{4.475758in}{2.600763in}}{\pgfqpoint{4.473444in}{2.606349in}}{\pgfqpoint{4.469326in}{2.610467in}}%
\pgfpathcurveto{\pgfqpoint{4.465207in}{2.614585in}}{\pgfqpoint{4.459621in}{2.616899in}}{\pgfqpoint{4.453797in}{2.616899in}}%
\pgfpathcurveto{\pgfqpoint{4.447973in}{2.616899in}}{\pgfqpoint{4.442387in}{2.614585in}}{\pgfqpoint{4.438269in}{2.610467in}}%
\pgfpathcurveto{\pgfqpoint{4.434151in}{2.606349in}}{\pgfqpoint{4.431837in}{2.600763in}}{\pgfqpoint{4.431837in}{2.594939in}}%
\pgfpathcurveto{\pgfqpoint{4.431837in}{2.589115in}}{\pgfqpoint{4.434151in}{2.583529in}}{\pgfqpoint{4.438269in}{2.579411in}}%
\pgfpathcurveto{\pgfqpoint{4.442387in}{2.575293in}}{\pgfqpoint{4.447973in}{2.572979in}}{\pgfqpoint{4.453797in}{2.572979in}}%
\pgfpathlineto{\pgfqpoint{4.453797in}{2.572979in}}%
\pgfpathclose%
\pgfusepath{stroke,fill}%
\end{pgfscope}%
\begin{pgfscope}%
\pgfpathrectangle{\pgfqpoint{1.073501in}{0.880000in}}{\pgfqpoint{6.052998in}{6.160000in}}%
\pgfusepath{clip}%
\pgfsetbuttcap%
\pgfsetroundjoin%
\definecolor{currentfill}{rgb}{0.200000,0.800000,0.200000}%
\pgfsetfillcolor{currentfill}%
\pgfsetlinewidth{1.003750pt}%
\definecolor{currentstroke}{rgb}{0.200000,0.800000,0.200000}%
\pgfsetstrokecolor{currentstroke}%
\pgfsetdash{}{0pt}%
\pgfpathmoveto{\pgfqpoint{4.538180in}{2.598951in}}%
\pgfpathcurveto{\pgfqpoint{4.544004in}{2.598951in}}{\pgfqpoint{4.549590in}{2.601265in}}{\pgfqpoint{4.553709in}{2.605383in}}%
\pgfpathcurveto{\pgfqpoint{4.557827in}{2.609501in}}{\pgfqpoint{4.560141in}{2.615088in}}{\pgfqpoint{4.560141in}{2.620912in}}%
\pgfpathcurveto{\pgfqpoint{4.560141in}{2.626736in}}{\pgfqpoint{4.557827in}{2.632322in}}{\pgfqpoint{4.553709in}{2.636440in}}%
\pgfpathcurveto{\pgfqpoint{4.549590in}{2.640558in}}{\pgfqpoint{4.544004in}{2.642872in}}{\pgfqpoint{4.538180in}{2.642872in}}%
\pgfpathcurveto{\pgfqpoint{4.532356in}{2.642872in}}{\pgfqpoint{4.526770in}{2.640558in}}{\pgfqpoint{4.522652in}{2.636440in}}%
\pgfpathcurveto{\pgfqpoint{4.518534in}{2.632322in}}{\pgfqpoint{4.516220in}{2.626736in}}{\pgfqpoint{4.516220in}{2.620912in}}%
\pgfpathcurveto{\pgfqpoint{4.516220in}{2.615088in}}{\pgfqpoint{4.518534in}{2.609501in}}{\pgfqpoint{4.522652in}{2.605383in}}%
\pgfpathcurveto{\pgfqpoint{4.526770in}{2.601265in}}{\pgfqpoint{4.532356in}{2.598951in}}{\pgfqpoint{4.538180in}{2.598951in}}%
\pgfpathlineto{\pgfqpoint{4.538180in}{2.598951in}}%
\pgfpathclose%
\pgfusepath{stroke,fill}%
\end{pgfscope}%
\begin{pgfscope}%
\pgfpathrectangle{\pgfqpoint{1.073501in}{0.880000in}}{\pgfqpoint{6.052998in}{6.160000in}}%
\pgfusepath{clip}%
\pgfsetbuttcap%
\pgfsetroundjoin%
\definecolor{currentfill}{rgb}{0.200000,0.800000,0.200000}%
\pgfsetfillcolor{currentfill}%
\pgfsetlinewidth{1.003750pt}%
\definecolor{currentstroke}{rgb}{0.200000,0.800000,0.200000}%
\pgfsetstrokecolor{currentstroke}%
\pgfsetdash{}{0pt}%
\pgfpathmoveto{\pgfqpoint{4.663370in}{2.518203in}}%
\pgfpathcurveto{\pgfqpoint{4.669194in}{2.518203in}}{\pgfqpoint{4.674780in}{2.520517in}}{\pgfqpoint{4.678898in}{2.524635in}}%
\pgfpathcurveto{\pgfqpoint{4.683016in}{2.528753in}}{\pgfqpoint{4.685330in}{2.534339in}}{\pgfqpoint{4.685330in}{2.540163in}}%
\pgfpathcurveto{\pgfqpoint{4.685330in}{2.545987in}}{\pgfqpoint{4.683016in}{2.551573in}}{\pgfqpoint{4.678898in}{2.555692in}}%
\pgfpathcurveto{\pgfqpoint{4.674780in}{2.559810in}}{\pgfqpoint{4.669194in}{2.562124in}}{\pgfqpoint{4.663370in}{2.562124in}}%
\pgfpathcurveto{\pgfqpoint{4.657546in}{2.562124in}}{\pgfqpoint{4.651960in}{2.559810in}}{\pgfqpoint{4.647842in}{2.555692in}}%
\pgfpathcurveto{\pgfqpoint{4.643724in}{2.551573in}}{\pgfqpoint{4.641410in}{2.545987in}}{\pgfqpoint{4.641410in}{2.540163in}}%
\pgfpathcurveto{\pgfqpoint{4.641410in}{2.534339in}}{\pgfqpoint{4.643724in}{2.528753in}}{\pgfqpoint{4.647842in}{2.524635in}}%
\pgfpathcurveto{\pgfqpoint{4.651960in}{2.520517in}}{\pgfqpoint{4.657546in}{2.518203in}}{\pgfqpoint{4.663370in}{2.518203in}}%
\pgfpathlineto{\pgfqpoint{4.663370in}{2.518203in}}%
\pgfpathclose%
\pgfusepath{stroke,fill}%
\end{pgfscope}%
\begin{pgfscope}%
\pgfpathrectangle{\pgfqpoint{1.073501in}{0.880000in}}{\pgfqpoint{6.052998in}{6.160000in}}%
\pgfusepath{clip}%
\pgfsetbuttcap%
\pgfsetroundjoin%
\definecolor{currentfill}{rgb}{0.200000,0.800000,0.200000}%
\pgfsetfillcolor{currentfill}%
\pgfsetlinewidth{1.003750pt}%
\definecolor{currentstroke}{rgb}{0.200000,0.800000,0.200000}%
\pgfsetstrokecolor{currentstroke}%
\pgfsetdash{}{0pt}%
\pgfpathmoveto{\pgfqpoint{4.661010in}{2.753426in}}%
\pgfpathcurveto{\pgfqpoint{4.666834in}{2.753426in}}{\pgfqpoint{4.672420in}{2.755739in}}{\pgfqpoint{4.676538in}{2.759858in}}%
\pgfpathcurveto{\pgfqpoint{4.680656in}{2.763976in}}{\pgfqpoint{4.682970in}{2.769562in}}{\pgfqpoint{4.682970in}{2.775386in}}%
\pgfpathcurveto{\pgfqpoint{4.682970in}{2.781210in}}{\pgfqpoint{4.680656in}{2.786796in}}{\pgfqpoint{4.676538in}{2.790914in}}%
\pgfpathcurveto{\pgfqpoint{4.672420in}{2.795032in}}{\pgfqpoint{4.666834in}{2.797346in}}{\pgfqpoint{4.661010in}{2.797346in}}%
\pgfpathcurveto{\pgfqpoint{4.655186in}{2.797346in}}{\pgfqpoint{4.649600in}{2.795032in}}{\pgfqpoint{4.645481in}{2.790914in}}%
\pgfpathcurveto{\pgfqpoint{4.641363in}{2.786796in}}{\pgfqpoint{4.639049in}{2.781210in}}{\pgfqpoint{4.639049in}{2.775386in}}%
\pgfpathcurveto{\pgfqpoint{4.639049in}{2.769562in}}{\pgfqpoint{4.641363in}{2.763976in}}{\pgfqpoint{4.645481in}{2.759858in}}%
\pgfpathcurveto{\pgfqpoint{4.649600in}{2.755739in}}{\pgfqpoint{4.655186in}{2.753426in}}{\pgfqpoint{4.661010in}{2.753426in}}%
\pgfpathlineto{\pgfqpoint{4.661010in}{2.753426in}}%
\pgfpathclose%
\pgfusepath{stroke,fill}%
\end{pgfscope}%
\begin{pgfscope}%
\pgfpathrectangle{\pgfqpoint{1.073501in}{0.880000in}}{\pgfqpoint{6.052998in}{6.160000in}}%
\pgfusepath{clip}%
\pgfsetbuttcap%
\pgfsetroundjoin%
\definecolor{currentfill}{rgb}{0.200000,0.800000,0.200000}%
\pgfsetfillcolor{currentfill}%
\pgfsetlinewidth{1.003750pt}%
\definecolor{currentstroke}{rgb}{0.200000,0.800000,0.200000}%
\pgfsetstrokecolor{currentstroke}%
\pgfsetdash{}{0pt}%
\pgfpathmoveto{\pgfqpoint{4.815918in}{2.637387in}}%
\pgfpathcurveto{\pgfqpoint{4.821742in}{2.637387in}}{\pgfqpoint{4.827328in}{2.639701in}}{\pgfqpoint{4.831446in}{2.643819in}}%
\pgfpathcurveto{\pgfqpoint{4.835564in}{2.647937in}}{\pgfqpoint{4.837878in}{2.653524in}}{\pgfqpoint{4.837878in}{2.659348in}}%
\pgfpathcurveto{\pgfqpoint{4.837878in}{2.665172in}}{\pgfqpoint{4.835564in}{2.670758in}}{\pgfqpoint{4.831446in}{2.674876in}}%
\pgfpathcurveto{\pgfqpoint{4.827328in}{2.678994in}}{\pgfqpoint{4.821742in}{2.681308in}}{\pgfqpoint{4.815918in}{2.681308in}}%
\pgfpathcurveto{\pgfqpoint{4.810094in}{2.681308in}}{\pgfqpoint{4.804508in}{2.678994in}}{\pgfqpoint{4.800390in}{2.674876in}}%
\pgfpathcurveto{\pgfqpoint{4.796271in}{2.670758in}}{\pgfqpoint{4.793958in}{2.665172in}}{\pgfqpoint{4.793958in}{2.659348in}}%
\pgfpathcurveto{\pgfqpoint{4.793958in}{2.653524in}}{\pgfqpoint{4.796271in}{2.647937in}}{\pgfqpoint{4.800390in}{2.643819in}}%
\pgfpathcurveto{\pgfqpoint{4.804508in}{2.639701in}}{\pgfqpoint{4.810094in}{2.637387in}}{\pgfqpoint{4.815918in}{2.637387in}}%
\pgfpathlineto{\pgfqpoint{4.815918in}{2.637387in}}%
\pgfpathclose%
\pgfusepath{stroke,fill}%
\end{pgfscope}%
\begin{pgfscope}%
\pgfpathrectangle{\pgfqpoint{1.073501in}{0.880000in}}{\pgfqpoint{6.052998in}{6.160000in}}%
\pgfusepath{clip}%
\pgfsetbuttcap%
\pgfsetroundjoin%
\definecolor{currentfill}{rgb}{0.200000,0.800000,0.200000}%
\pgfsetfillcolor{currentfill}%
\pgfsetlinewidth{1.003750pt}%
\definecolor{currentstroke}{rgb}{0.200000,0.800000,0.200000}%
\pgfsetstrokecolor{currentstroke}%
\pgfsetdash{}{0pt}%
\pgfpathmoveto{\pgfqpoint{4.903722in}{2.671710in}}%
\pgfpathcurveto{\pgfqpoint{4.909546in}{2.671710in}}{\pgfqpoint{4.915132in}{2.674024in}}{\pgfqpoint{4.919250in}{2.678142in}}%
\pgfpathcurveto{\pgfqpoint{4.923369in}{2.682260in}}{\pgfqpoint{4.925682in}{2.687847in}}{\pgfqpoint{4.925682in}{2.693670in}}%
\pgfpathcurveto{\pgfqpoint{4.925682in}{2.699494in}}{\pgfqpoint{4.923369in}{2.705081in}}{\pgfqpoint{4.919250in}{2.709199in}}%
\pgfpathcurveto{\pgfqpoint{4.915132in}{2.713317in}}{\pgfqpoint{4.909546in}{2.715631in}}{\pgfqpoint{4.903722in}{2.715631in}}%
\pgfpathcurveto{\pgfqpoint{4.897898in}{2.715631in}}{\pgfqpoint{4.892312in}{2.713317in}}{\pgfqpoint{4.888194in}{2.709199in}}%
\pgfpathcurveto{\pgfqpoint{4.884076in}{2.705081in}}{\pgfqpoint{4.881762in}{2.699494in}}{\pgfqpoint{4.881762in}{2.693670in}}%
\pgfpathcurveto{\pgfqpoint{4.881762in}{2.687847in}}{\pgfqpoint{4.884076in}{2.682260in}}{\pgfqpoint{4.888194in}{2.678142in}}%
\pgfpathcurveto{\pgfqpoint{4.892312in}{2.674024in}}{\pgfqpoint{4.897898in}{2.671710in}}{\pgfqpoint{4.903722in}{2.671710in}}%
\pgfpathlineto{\pgfqpoint{4.903722in}{2.671710in}}%
\pgfpathclose%
\pgfusepath{stroke,fill}%
\end{pgfscope}%
\begin{pgfscope}%
\pgfpathrectangle{\pgfqpoint{1.073501in}{0.880000in}}{\pgfqpoint{6.052998in}{6.160000in}}%
\pgfusepath{clip}%
\pgfsetbuttcap%
\pgfsetroundjoin%
\definecolor{currentfill}{rgb}{0.200000,0.800000,0.200000}%
\pgfsetfillcolor{currentfill}%
\pgfsetlinewidth{1.003750pt}%
\definecolor{currentstroke}{rgb}{0.200000,0.800000,0.200000}%
\pgfsetstrokecolor{currentstroke}%
\pgfsetdash{}{0pt}%
\pgfpathmoveto{\pgfqpoint{4.943222in}{2.777613in}}%
\pgfpathcurveto{\pgfqpoint{4.949045in}{2.777613in}}{\pgfqpoint{4.954632in}{2.779927in}}{\pgfqpoint{4.958750in}{2.784045in}}%
\pgfpathcurveto{\pgfqpoint{4.962868in}{2.788163in}}{\pgfqpoint{4.965182in}{2.793750in}}{\pgfqpoint{4.965182in}{2.799574in}}%
\pgfpathcurveto{\pgfqpoint{4.965182in}{2.805398in}}{\pgfqpoint{4.962868in}{2.810984in}}{\pgfqpoint{4.958750in}{2.815102in}}%
\pgfpathcurveto{\pgfqpoint{4.954632in}{2.819220in}}{\pgfqpoint{4.949045in}{2.821534in}}{\pgfqpoint{4.943222in}{2.821534in}}%
\pgfpathcurveto{\pgfqpoint{4.937398in}{2.821534in}}{\pgfqpoint{4.931811in}{2.819220in}}{\pgfqpoint{4.927693in}{2.815102in}}%
\pgfpathcurveto{\pgfqpoint{4.923575in}{2.810984in}}{\pgfqpoint{4.921261in}{2.805398in}}{\pgfqpoint{4.921261in}{2.799574in}}%
\pgfpathcurveto{\pgfqpoint{4.921261in}{2.793750in}}{\pgfqpoint{4.923575in}{2.788163in}}{\pgfqpoint{4.927693in}{2.784045in}}%
\pgfpathcurveto{\pgfqpoint{4.931811in}{2.779927in}}{\pgfqpoint{4.937398in}{2.777613in}}{\pgfqpoint{4.943222in}{2.777613in}}%
\pgfpathlineto{\pgfqpoint{4.943222in}{2.777613in}}%
\pgfpathclose%
\pgfusepath{stroke,fill}%
\end{pgfscope}%
\begin{pgfscope}%
\pgfpathrectangle{\pgfqpoint{1.073501in}{0.880000in}}{\pgfqpoint{6.052998in}{6.160000in}}%
\pgfusepath{clip}%
\pgfsetbuttcap%
\pgfsetroundjoin%
\definecolor{currentfill}{rgb}{0.200000,0.800000,0.200000}%
\pgfsetfillcolor{currentfill}%
\pgfsetlinewidth{1.003750pt}%
\definecolor{currentstroke}{rgb}{0.200000,0.800000,0.200000}%
\pgfsetstrokecolor{currentstroke}%
\pgfsetdash{}{0pt}%
\pgfpathmoveto{\pgfqpoint{4.973172in}{2.882526in}}%
\pgfpathcurveto{\pgfqpoint{4.978996in}{2.882526in}}{\pgfqpoint{4.984582in}{2.884840in}}{\pgfqpoint{4.988700in}{2.888958in}}%
\pgfpathcurveto{\pgfqpoint{4.992818in}{2.893076in}}{\pgfqpoint{4.995132in}{2.898662in}}{\pgfqpoint{4.995132in}{2.904486in}}%
\pgfpathcurveto{\pgfqpoint{4.995132in}{2.910310in}}{\pgfqpoint{4.992818in}{2.915896in}}{\pgfqpoint{4.988700in}{2.920014in}}%
\pgfpathcurveto{\pgfqpoint{4.984582in}{2.924132in}}{\pgfqpoint{4.978996in}{2.926446in}}{\pgfqpoint{4.973172in}{2.926446in}}%
\pgfpathcurveto{\pgfqpoint{4.967348in}{2.926446in}}{\pgfqpoint{4.961761in}{2.924132in}}{\pgfqpoint{4.957643in}{2.920014in}}%
\pgfpathcurveto{\pgfqpoint{4.953525in}{2.915896in}}{\pgfqpoint{4.951211in}{2.910310in}}{\pgfqpoint{4.951211in}{2.904486in}}%
\pgfpathcurveto{\pgfqpoint{4.951211in}{2.898662in}}{\pgfqpoint{4.953525in}{2.893076in}}{\pgfqpoint{4.957643in}{2.888958in}}%
\pgfpathcurveto{\pgfqpoint{4.961761in}{2.884840in}}{\pgfqpoint{4.967348in}{2.882526in}}{\pgfqpoint{4.973172in}{2.882526in}}%
\pgfpathlineto{\pgfqpoint{4.973172in}{2.882526in}}%
\pgfpathclose%
\pgfusepath{stroke,fill}%
\end{pgfscope}%
\begin{pgfscope}%
\pgfpathrectangle{\pgfqpoint{1.073501in}{0.880000in}}{\pgfqpoint{6.052998in}{6.160000in}}%
\pgfusepath{clip}%
\pgfsetbuttcap%
\pgfsetroundjoin%
\definecolor{currentfill}{rgb}{0.200000,0.800000,0.200000}%
\pgfsetfillcolor{currentfill}%
\pgfsetlinewidth{1.003750pt}%
\definecolor{currentstroke}{rgb}{0.200000,0.800000,0.200000}%
\pgfsetstrokecolor{currentstroke}%
\pgfsetdash{}{0pt}%
\pgfpathmoveto{\pgfqpoint{5.097787in}{2.873255in}}%
\pgfpathcurveto{\pgfqpoint{5.103611in}{2.873255in}}{\pgfqpoint{5.109198in}{2.875569in}}{\pgfqpoint{5.113316in}{2.879687in}}%
\pgfpathcurveto{\pgfqpoint{5.117434in}{2.883806in}}{\pgfqpoint{5.119748in}{2.889392in}}{\pgfqpoint{5.119748in}{2.895216in}}%
\pgfpathcurveto{\pgfqpoint{5.119748in}{2.901040in}}{\pgfqpoint{5.117434in}{2.906626in}}{\pgfqpoint{5.113316in}{2.910744in}}%
\pgfpathcurveto{\pgfqpoint{5.109198in}{2.914862in}}{\pgfqpoint{5.103611in}{2.917176in}}{\pgfqpoint{5.097787in}{2.917176in}}%
\pgfpathcurveto{\pgfqpoint{5.091964in}{2.917176in}}{\pgfqpoint{5.086377in}{2.914862in}}{\pgfqpoint{5.082259in}{2.910744in}}%
\pgfpathcurveto{\pgfqpoint{5.078141in}{2.906626in}}{\pgfqpoint{5.075827in}{2.901040in}}{\pgfqpoint{5.075827in}{2.895216in}}%
\pgfpathcurveto{\pgfqpoint{5.075827in}{2.889392in}}{\pgfqpoint{5.078141in}{2.883806in}}{\pgfqpoint{5.082259in}{2.879687in}}%
\pgfpathcurveto{\pgfqpoint{5.086377in}{2.875569in}}{\pgfqpoint{5.091964in}{2.873255in}}{\pgfqpoint{5.097787in}{2.873255in}}%
\pgfpathlineto{\pgfqpoint{5.097787in}{2.873255in}}%
\pgfpathclose%
\pgfusepath{stroke,fill}%
\end{pgfscope}%
\begin{pgfscope}%
\pgfpathrectangle{\pgfqpoint{1.073501in}{0.880000in}}{\pgfqpoint{6.052998in}{6.160000in}}%
\pgfusepath{clip}%
\pgfsetbuttcap%
\pgfsetroundjoin%
\definecolor{currentfill}{rgb}{0.200000,0.800000,0.200000}%
\pgfsetfillcolor{currentfill}%
\pgfsetlinewidth{1.003750pt}%
\definecolor{currentstroke}{rgb}{0.200000,0.800000,0.200000}%
\pgfsetstrokecolor{currentstroke}%
\pgfsetdash{}{0pt}%
\pgfpathmoveto{\pgfqpoint{5.139632in}{2.958920in}}%
\pgfpathcurveto{\pgfqpoint{5.145456in}{2.958920in}}{\pgfqpoint{5.151043in}{2.961234in}}{\pgfqpoint{5.155161in}{2.965352in}}%
\pgfpathcurveto{\pgfqpoint{5.159279in}{2.969470in}}{\pgfqpoint{5.161593in}{2.975056in}}{\pgfqpoint{5.161593in}{2.980880in}}%
\pgfpathcurveto{\pgfqpoint{5.161593in}{2.986704in}}{\pgfqpoint{5.159279in}{2.992291in}}{\pgfqpoint{5.155161in}{2.996409in}}%
\pgfpathcurveto{\pgfqpoint{5.151043in}{3.000527in}}{\pgfqpoint{5.145456in}{3.002841in}}{\pgfqpoint{5.139632in}{3.002841in}}%
\pgfpathcurveto{\pgfqpoint{5.133808in}{3.002841in}}{\pgfqpoint{5.128222in}{3.000527in}}{\pgfqpoint{5.124104in}{2.996409in}}%
\pgfpathcurveto{\pgfqpoint{5.119986in}{2.992291in}}{\pgfqpoint{5.117672in}{2.986704in}}{\pgfqpoint{5.117672in}{2.980880in}}%
\pgfpathcurveto{\pgfqpoint{5.117672in}{2.975056in}}{\pgfqpoint{5.119986in}{2.969470in}}{\pgfqpoint{5.124104in}{2.965352in}}%
\pgfpathcurveto{\pgfqpoint{5.128222in}{2.961234in}}{\pgfqpoint{5.133808in}{2.958920in}}{\pgfqpoint{5.139632in}{2.958920in}}%
\pgfpathlineto{\pgfqpoint{5.139632in}{2.958920in}}%
\pgfpathclose%
\pgfusepath{stroke,fill}%
\end{pgfscope}%
\begin{pgfscope}%
\pgfpathrectangle{\pgfqpoint{1.073501in}{0.880000in}}{\pgfqpoint{6.052998in}{6.160000in}}%
\pgfusepath{clip}%
\pgfsetbuttcap%
\pgfsetroundjoin%
\definecolor{currentfill}{rgb}{0.200000,0.800000,0.200000}%
\pgfsetfillcolor{currentfill}%
\pgfsetlinewidth{1.003750pt}%
\definecolor{currentstroke}{rgb}{0.200000,0.800000,0.200000}%
\pgfsetstrokecolor{currentstroke}%
\pgfsetdash{}{0pt}%
\pgfpathmoveto{\pgfqpoint{5.267357in}{2.967749in}}%
\pgfpathcurveto{\pgfqpoint{5.273181in}{2.967749in}}{\pgfqpoint{5.278767in}{2.970063in}}{\pgfqpoint{5.282886in}{2.974181in}}%
\pgfpathcurveto{\pgfqpoint{5.287004in}{2.978299in}}{\pgfqpoint{5.289318in}{2.983885in}}{\pgfqpoint{5.289318in}{2.989709in}}%
\pgfpathcurveto{\pgfqpoint{5.289318in}{2.995533in}}{\pgfqpoint{5.287004in}{3.001119in}}{\pgfqpoint{5.282886in}{3.005237in}}%
\pgfpathcurveto{\pgfqpoint{5.278767in}{3.009356in}}{\pgfqpoint{5.273181in}{3.011669in}}{\pgfqpoint{5.267357in}{3.011669in}}%
\pgfpathcurveto{\pgfqpoint{5.261533in}{3.011669in}}{\pgfqpoint{5.255947in}{3.009356in}}{\pgfqpoint{5.251829in}{3.005237in}}%
\pgfpathcurveto{\pgfqpoint{5.247711in}{3.001119in}}{\pgfqpoint{5.245397in}{2.995533in}}{\pgfqpoint{5.245397in}{2.989709in}}%
\pgfpathcurveto{\pgfqpoint{5.245397in}{2.983885in}}{\pgfqpoint{5.247711in}{2.978299in}}{\pgfqpoint{5.251829in}{2.974181in}}%
\pgfpathcurveto{\pgfqpoint{5.255947in}{2.970063in}}{\pgfqpoint{5.261533in}{2.967749in}}{\pgfqpoint{5.267357in}{2.967749in}}%
\pgfpathlineto{\pgfqpoint{5.267357in}{2.967749in}}%
\pgfpathclose%
\pgfusepath{stroke,fill}%
\end{pgfscope}%
\begin{pgfscope}%
\pgfpathrectangle{\pgfqpoint{1.073501in}{0.880000in}}{\pgfqpoint{6.052998in}{6.160000in}}%
\pgfusepath{clip}%
\pgfsetbuttcap%
\pgfsetroundjoin%
\definecolor{currentfill}{rgb}{0.200000,0.800000,0.200000}%
\pgfsetfillcolor{currentfill}%
\pgfsetlinewidth{1.003750pt}%
\definecolor{currentstroke}{rgb}{0.200000,0.800000,0.200000}%
\pgfsetstrokecolor{currentstroke}%
\pgfsetdash{}{0pt}%
\pgfpathmoveto{\pgfqpoint{5.317918in}{3.047911in}}%
\pgfpathcurveto{\pgfqpoint{5.323742in}{3.047911in}}{\pgfqpoint{5.329328in}{3.050225in}}{\pgfqpoint{5.333446in}{3.054343in}}%
\pgfpathcurveto{\pgfqpoint{5.337564in}{3.058461in}}{\pgfqpoint{5.339878in}{3.064047in}}{\pgfqpoint{5.339878in}{3.069871in}}%
\pgfpathcurveto{\pgfqpoint{5.339878in}{3.075695in}}{\pgfqpoint{5.337564in}{3.081281in}}{\pgfqpoint{5.333446in}{3.085399in}}%
\pgfpathcurveto{\pgfqpoint{5.329328in}{3.089518in}}{\pgfqpoint{5.323742in}{3.091831in}}{\pgfqpoint{5.317918in}{3.091831in}}%
\pgfpathcurveto{\pgfqpoint{5.312094in}{3.091831in}}{\pgfqpoint{5.306508in}{3.089518in}}{\pgfqpoint{5.302390in}{3.085399in}}%
\pgfpathcurveto{\pgfqpoint{5.298272in}{3.081281in}}{\pgfqpoint{5.295958in}{3.075695in}}{\pgfqpoint{5.295958in}{3.069871in}}%
\pgfpathcurveto{\pgfqpoint{5.295958in}{3.064047in}}{\pgfqpoint{5.298272in}{3.058461in}}{\pgfqpoint{5.302390in}{3.054343in}}%
\pgfpathcurveto{\pgfqpoint{5.306508in}{3.050225in}}{\pgfqpoint{5.312094in}{3.047911in}}{\pgfqpoint{5.317918in}{3.047911in}}%
\pgfpathlineto{\pgfqpoint{5.317918in}{3.047911in}}%
\pgfpathclose%
\pgfusepath{stroke,fill}%
\end{pgfscope}%
\begin{pgfscope}%
\pgfpathrectangle{\pgfqpoint{1.073501in}{0.880000in}}{\pgfqpoint{6.052998in}{6.160000in}}%
\pgfusepath{clip}%
\pgfsetbuttcap%
\pgfsetroundjoin%
\definecolor{currentfill}{rgb}{0.200000,0.800000,0.200000}%
\pgfsetfillcolor{currentfill}%
\pgfsetlinewidth{1.003750pt}%
\definecolor{currentstroke}{rgb}{0.200000,0.800000,0.200000}%
\pgfsetstrokecolor{currentstroke}%
\pgfsetdash{}{0pt}%
\pgfpathmoveto{\pgfqpoint{5.327496in}{3.153384in}}%
\pgfpathcurveto{\pgfqpoint{5.333320in}{3.153384in}}{\pgfqpoint{5.338906in}{3.155698in}}{\pgfqpoint{5.343024in}{3.159816in}}%
\pgfpathcurveto{\pgfqpoint{5.347142in}{3.163934in}}{\pgfqpoint{5.349456in}{3.169520in}}{\pgfqpoint{5.349456in}{3.175344in}}%
\pgfpathcurveto{\pgfqpoint{5.349456in}{3.181168in}}{\pgfqpoint{5.347142in}{3.186754in}}{\pgfqpoint{5.343024in}{3.190872in}}%
\pgfpathcurveto{\pgfqpoint{5.338906in}{3.194991in}}{\pgfqpoint{5.333320in}{3.197304in}}{\pgfqpoint{5.327496in}{3.197304in}}%
\pgfpathcurveto{\pgfqpoint{5.321672in}{3.197304in}}{\pgfqpoint{5.316086in}{3.194991in}}{\pgfqpoint{5.311967in}{3.190872in}}%
\pgfpathcurveto{\pgfqpoint{5.307849in}{3.186754in}}{\pgfqpoint{5.305535in}{3.181168in}}{\pgfqpoint{5.305535in}{3.175344in}}%
\pgfpathcurveto{\pgfqpoint{5.305535in}{3.169520in}}{\pgfqpoint{5.307849in}{3.163934in}}{\pgfqpoint{5.311967in}{3.159816in}}%
\pgfpathcurveto{\pgfqpoint{5.316086in}{3.155698in}}{\pgfqpoint{5.321672in}{3.153384in}}{\pgfqpoint{5.327496in}{3.153384in}}%
\pgfpathlineto{\pgfqpoint{5.327496in}{3.153384in}}%
\pgfpathclose%
\pgfusepath{stroke,fill}%
\end{pgfscope}%
\begin{pgfscope}%
\pgfpathrectangle{\pgfqpoint{1.073501in}{0.880000in}}{\pgfqpoint{6.052998in}{6.160000in}}%
\pgfusepath{clip}%
\pgfsetbuttcap%
\pgfsetroundjoin%
\definecolor{currentfill}{rgb}{0.200000,0.800000,0.200000}%
\pgfsetfillcolor{currentfill}%
\pgfsetlinewidth{1.003750pt}%
\definecolor{currentstroke}{rgb}{0.200000,0.800000,0.200000}%
\pgfsetstrokecolor{currentstroke}%
\pgfsetdash{}{0pt}%
\pgfpathmoveto{\pgfqpoint{5.318272in}{3.262209in}}%
\pgfpathcurveto{\pgfqpoint{5.324096in}{3.262209in}}{\pgfqpoint{5.329683in}{3.264523in}}{\pgfqpoint{5.333801in}{3.268641in}}%
\pgfpathcurveto{\pgfqpoint{5.337919in}{3.272760in}}{\pgfqpoint{5.340233in}{3.278346in}}{\pgfqpoint{5.340233in}{3.284170in}}%
\pgfpathcurveto{\pgfqpoint{5.340233in}{3.289994in}}{\pgfqpoint{5.337919in}{3.295580in}}{\pgfqpoint{5.333801in}{3.299698in}}%
\pgfpathcurveto{\pgfqpoint{5.329683in}{3.303816in}}{\pgfqpoint{5.324096in}{3.306130in}}{\pgfqpoint{5.318272in}{3.306130in}}%
\pgfpathcurveto{\pgfqpoint{5.312448in}{3.306130in}}{\pgfqpoint{5.306862in}{3.303816in}}{\pgfqpoint{5.302744in}{3.299698in}}%
\pgfpathcurveto{\pgfqpoint{5.298626in}{3.295580in}}{\pgfqpoint{5.296312in}{3.289994in}}{\pgfqpoint{5.296312in}{3.284170in}}%
\pgfpathcurveto{\pgfqpoint{5.296312in}{3.278346in}}{\pgfqpoint{5.298626in}{3.272760in}}{\pgfqpoint{5.302744in}{3.268641in}}%
\pgfpathcurveto{\pgfqpoint{5.306862in}{3.264523in}}{\pgfqpoint{5.312448in}{3.262209in}}{\pgfqpoint{5.318272in}{3.262209in}}%
\pgfpathlineto{\pgfqpoint{5.318272in}{3.262209in}}%
\pgfpathclose%
\pgfusepath{stroke,fill}%
\end{pgfscope}%
\begin{pgfscope}%
\pgfpathrectangle{\pgfqpoint{1.073501in}{0.880000in}}{\pgfqpoint{6.052998in}{6.160000in}}%
\pgfusepath{clip}%
\pgfsetbuttcap%
\pgfsetroundjoin%
\definecolor{currentfill}{rgb}{0.200000,0.800000,0.200000}%
\pgfsetfillcolor{currentfill}%
\pgfsetlinewidth{1.003750pt}%
\definecolor{currentstroke}{rgb}{0.200000,0.800000,0.200000}%
\pgfsetstrokecolor{currentstroke}%
\pgfsetdash{}{0pt}%
\pgfpathmoveto{\pgfqpoint{5.409096in}{3.314847in}}%
\pgfpathcurveto{\pgfqpoint{5.414920in}{3.314847in}}{\pgfqpoint{5.420506in}{3.317161in}}{\pgfqpoint{5.424624in}{3.321279in}}%
\pgfpathcurveto{\pgfqpoint{5.428742in}{3.325397in}}{\pgfqpoint{5.431056in}{3.330983in}}{\pgfqpoint{5.431056in}{3.336807in}}%
\pgfpathcurveto{\pgfqpoint{5.431056in}{3.342631in}}{\pgfqpoint{5.428742in}{3.348217in}}{\pgfqpoint{5.424624in}{3.352336in}}%
\pgfpathcurveto{\pgfqpoint{5.420506in}{3.356454in}}{\pgfqpoint{5.414920in}{3.358768in}}{\pgfqpoint{5.409096in}{3.358768in}}%
\pgfpathcurveto{\pgfqpoint{5.403272in}{3.358768in}}{\pgfqpoint{5.397686in}{3.356454in}}{\pgfqpoint{5.393568in}{3.352336in}}%
\pgfpathcurveto{\pgfqpoint{5.389450in}{3.348217in}}{\pgfqpoint{5.387136in}{3.342631in}}{\pgfqpoint{5.387136in}{3.336807in}}%
\pgfpathcurveto{\pgfqpoint{5.387136in}{3.330983in}}{\pgfqpoint{5.389450in}{3.325397in}}{\pgfqpoint{5.393568in}{3.321279in}}%
\pgfpathcurveto{\pgfqpoint{5.397686in}{3.317161in}}{\pgfqpoint{5.403272in}{3.314847in}}{\pgfqpoint{5.409096in}{3.314847in}}%
\pgfpathlineto{\pgfqpoint{5.409096in}{3.314847in}}%
\pgfpathclose%
\pgfusepath{stroke,fill}%
\end{pgfscope}%
\begin{pgfscope}%
\pgfpathrectangle{\pgfqpoint{1.073501in}{0.880000in}}{\pgfqpoint{6.052998in}{6.160000in}}%
\pgfusepath{clip}%
\pgfsetbuttcap%
\pgfsetroundjoin%
\definecolor{currentfill}{rgb}{0.200000,0.800000,0.200000}%
\pgfsetfillcolor{currentfill}%
\pgfsetlinewidth{1.003750pt}%
\definecolor{currentstroke}{rgb}{0.200000,0.800000,0.200000}%
\pgfsetstrokecolor{currentstroke}%
\pgfsetdash{}{0pt}%
\pgfpathmoveto{\pgfqpoint{5.460098in}{3.391541in}}%
\pgfpathcurveto{\pgfqpoint{5.465922in}{3.391541in}}{\pgfqpoint{5.471509in}{3.393854in}}{\pgfqpoint{5.475627in}{3.397973in}}%
\pgfpathcurveto{\pgfqpoint{5.479745in}{3.402091in}}{\pgfqpoint{5.482059in}{3.407677in}}{\pgfqpoint{5.482059in}{3.413501in}}%
\pgfpathcurveto{\pgfqpoint{5.482059in}{3.419325in}}{\pgfqpoint{5.479745in}{3.424911in}}{\pgfqpoint{5.475627in}{3.429029in}}%
\pgfpathcurveto{\pgfqpoint{5.471509in}{3.433147in}}{\pgfqpoint{5.465922in}{3.435461in}}{\pgfqpoint{5.460098in}{3.435461in}}%
\pgfpathcurveto{\pgfqpoint{5.454275in}{3.435461in}}{\pgfqpoint{5.448688in}{3.433147in}}{\pgfqpoint{5.444570in}{3.429029in}}%
\pgfpathcurveto{\pgfqpoint{5.440452in}{3.424911in}}{\pgfqpoint{5.438138in}{3.419325in}}{\pgfqpoint{5.438138in}{3.413501in}}%
\pgfpathcurveto{\pgfqpoint{5.438138in}{3.407677in}}{\pgfqpoint{5.440452in}{3.402091in}}{\pgfqpoint{5.444570in}{3.397973in}}%
\pgfpathcurveto{\pgfqpoint{5.448688in}{3.393854in}}{\pgfqpoint{5.454275in}{3.391541in}}{\pgfqpoint{5.460098in}{3.391541in}}%
\pgfpathlineto{\pgfqpoint{5.460098in}{3.391541in}}%
\pgfpathclose%
\pgfusepath{stroke,fill}%
\end{pgfscope}%
\begin{pgfscope}%
\pgfpathrectangle{\pgfqpoint{1.073501in}{0.880000in}}{\pgfqpoint{6.052998in}{6.160000in}}%
\pgfusepath{clip}%
\pgfsetbuttcap%
\pgfsetroundjoin%
\definecolor{currentfill}{rgb}{0.200000,0.800000,0.200000}%
\pgfsetfillcolor{currentfill}%
\pgfsetlinewidth{1.003750pt}%
\definecolor{currentstroke}{rgb}{0.200000,0.800000,0.200000}%
\pgfsetstrokecolor{currentstroke}%
\pgfsetdash{}{0pt}%
\pgfpathmoveto{\pgfqpoint{5.475285in}{3.482513in}}%
\pgfpathcurveto{\pgfqpoint{5.481109in}{3.482513in}}{\pgfqpoint{5.486695in}{3.484827in}}{\pgfqpoint{5.490813in}{3.488945in}}%
\pgfpathcurveto{\pgfqpoint{5.494931in}{3.493063in}}{\pgfqpoint{5.497245in}{3.498650in}}{\pgfqpoint{5.497245in}{3.504474in}}%
\pgfpathcurveto{\pgfqpoint{5.497245in}{3.510298in}}{\pgfqpoint{5.494931in}{3.515884in}}{\pgfqpoint{5.490813in}{3.520002in}}%
\pgfpathcurveto{\pgfqpoint{5.486695in}{3.524120in}}{\pgfqpoint{5.481109in}{3.526434in}}{\pgfqpoint{5.475285in}{3.526434in}}%
\pgfpathcurveto{\pgfqpoint{5.469461in}{3.526434in}}{\pgfqpoint{5.463875in}{3.524120in}}{\pgfqpoint{5.459756in}{3.520002in}}%
\pgfpathcurveto{\pgfqpoint{5.455638in}{3.515884in}}{\pgfqpoint{5.453324in}{3.510298in}}{\pgfqpoint{5.453324in}{3.504474in}}%
\pgfpathcurveto{\pgfqpoint{5.453324in}{3.498650in}}{\pgfqpoint{5.455638in}{3.493063in}}{\pgfqpoint{5.459756in}{3.488945in}}%
\pgfpathcurveto{\pgfqpoint{5.463875in}{3.484827in}}{\pgfqpoint{5.469461in}{3.482513in}}{\pgfqpoint{5.475285in}{3.482513in}}%
\pgfpathlineto{\pgfqpoint{5.475285in}{3.482513in}}%
\pgfpathclose%
\pgfusepath{stroke,fill}%
\end{pgfscope}%
\begin{pgfscope}%
\pgfpathrectangle{\pgfqpoint{1.073501in}{0.880000in}}{\pgfqpoint{6.052998in}{6.160000in}}%
\pgfusepath{clip}%
\pgfsetbuttcap%
\pgfsetroundjoin%
\definecolor{currentfill}{rgb}{0.200000,0.800000,0.200000}%
\pgfsetfillcolor{currentfill}%
\pgfsetlinewidth{1.003750pt}%
\definecolor{currentstroke}{rgb}{0.200000,0.800000,0.200000}%
\pgfsetstrokecolor{currentstroke}%
\pgfsetdash{}{0pt}%
\pgfpathmoveto{\pgfqpoint{5.494584in}{3.570590in}}%
\pgfpathcurveto{\pgfqpoint{5.500408in}{3.570590in}}{\pgfqpoint{5.505994in}{3.572903in}}{\pgfqpoint{5.510112in}{3.577022in}}%
\pgfpathcurveto{\pgfqpoint{5.514230in}{3.581140in}}{\pgfqpoint{5.516544in}{3.586726in}}{\pgfqpoint{5.516544in}{3.592550in}}%
\pgfpathcurveto{\pgfqpoint{5.516544in}{3.598374in}}{\pgfqpoint{5.514230in}{3.603960in}}{\pgfqpoint{5.510112in}{3.608078in}}%
\pgfpathcurveto{\pgfqpoint{5.505994in}{3.612196in}}{\pgfqpoint{5.500408in}{3.614510in}}{\pgfqpoint{5.494584in}{3.614510in}}%
\pgfpathcurveto{\pgfqpoint{5.488760in}{3.614510in}}{\pgfqpoint{5.483174in}{3.612196in}}{\pgfqpoint{5.479056in}{3.608078in}}%
\pgfpathcurveto{\pgfqpoint{5.474938in}{3.603960in}}{\pgfqpoint{5.472624in}{3.598374in}}{\pgfqpoint{5.472624in}{3.592550in}}%
\pgfpathcurveto{\pgfqpoint{5.472624in}{3.586726in}}{\pgfqpoint{5.474938in}{3.581140in}}{\pgfqpoint{5.479056in}{3.577022in}}%
\pgfpathcurveto{\pgfqpoint{5.483174in}{3.572903in}}{\pgfqpoint{5.488760in}{3.570590in}}{\pgfqpoint{5.494584in}{3.570590in}}%
\pgfpathlineto{\pgfqpoint{5.494584in}{3.570590in}}%
\pgfpathclose%
\pgfusepath{stroke,fill}%
\end{pgfscope}%
\begin{pgfscope}%
\pgfpathrectangle{\pgfqpoint{1.073501in}{0.880000in}}{\pgfqpoint{6.052998in}{6.160000in}}%
\pgfusepath{clip}%
\pgfsetbuttcap%
\pgfsetroundjoin%
\definecolor{currentfill}{rgb}{0.200000,0.800000,0.200000}%
\pgfsetfillcolor{currentfill}%
\pgfsetlinewidth{1.003750pt}%
\definecolor{currentstroke}{rgb}{0.200000,0.800000,0.200000}%
\pgfsetstrokecolor{currentstroke}%
\pgfsetdash{}{0pt}%
\pgfpathmoveto{\pgfqpoint{5.492823in}{3.662281in}}%
\pgfpathcurveto{\pgfqpoint{5.498647in}{3.662281in}}{\pgfqpoint{5.504233in}{3.664595in}}{\pgfqpoint{5.508351in}{3.668713in}}%
\pgfpathcurveto{\pgfqpoint{5.512469in}{3.672831in}}{\pgfqpoint{5.514783in}{3.678417in}}{\pgfqpoint{5.514783in}{3.684241in}}%
\pgfpathcurveto{\pgfqpoint{5.514783in}{3.690065in}}{\pgfqpoint{5.512469in}{3.695651in}}{\pgfqpoint{5.508351in}{3.699769in}}%
\pgfpathcurveto{\pgfqpoint{5.504233in}{3.703887in}}{\pgfqpoint{5.498647in}{3.706201in}}{\pgfqpoint{5.492823in}{3.706201in}}%
\pgfpathcurveto{\pgfqpoint{5.486999in}{3.706201in}}{\pgfqpoint{5.481413in}{3.703887in}}{\pgfqpoint{5.477294in}{3.699769in}}%
\pgfpathcurveto{\pgfqpoint{5.473176in}{3.695651in}}{\pgfqpoint{5.470862in}{3.690065in}}{\pgfqpoint{5.470862in}{3.684241in}}%
\pgfpathcurveto{\pgfqpoint{5.470862in}{3.678417in}}{\pgfqpoint{5.473176in}{3.672831in}}{\pgfqpoint{5.477294in}{3.668713in}}%
\pgfpathcurveto{\pgfqpoint{5.481413in}{3.664595in}}{\pgfqpoint{5.486999in}{3.662281in}}{\pgfqpoint{5.492823in}{3.662281in}}%
\pgfpathlineto{\pgfqpoint{5.492823in}{3.662281in}}%
\pgfpathclose%
\pgfusepath{stroke,fill}%
\end{pgfscope}%
\begin{pgfscope}%
\pgfpathrectangle{\pgfqpoint{1.073501in}{0.880000in}}{\pgfqpoint{6.052998in}{6.160000in}}%
\pgfusepath{clip}%
\pgfsetbuttcap%
\pgfsetroundjoin%
\definecolor{currentfill}{rgb}{0.200000,0.800000,0.200000}%
\pgfsetfillcolor{currentfill}%
\pgfsetlinewidth{1.003750pt}%
\definecolor{currentstroke}{rgb}{0.200000,0.800000,0.200000}%
\pgfsetstrokecolor{currentstroke}%
\pgfsetdash{}{0pt}%
\pgfpathmoveto{\pgfqpoint{5.467791in}{3.754501in}}%
\pgfpathcurveto{\pgfqpoint{5.473615in}{3.754501in}}{\pgfqpoint{5.479201in}{3.756815in}}{\pgfqpoint{5.483319in}{3.760933in}}%
\pgfpathcurveto{\pgfqpoint{5.487437in}{3.765052in}}{\pgfqpoint{5.489751in}{3.770638in}}{\pgfqpoint{5.489751in}{3.776462in}}%
\pgfpathcurveto{\pgfqpoint{5.489751in}{3.782286in}}{\pgfqpoint{5.487437in}{3.787872in}}{\pgfqpoint{5.483319in}{3.791990in}}%
\pgfpathcurveto{\pgfqpoint{5.479201in}{3.796108in}}{\pgfqpoint{5.473615in}{3.798422in}}{\pgfqpoint{5.467791in}{3.798422in}}%
\pgfpathcurveto{\pgfqpoint{5.461967in}{3.798422in}}{\pgfqpoint{5.456381in}{3.796108in}}{\pgfqpoint{5.452262in}{3.791990in}}%
\pgfpathcurveto{\pgfqpoint{5.448144in}{3.787872in}}{\pgfqpoint{5.445830in}{3.782286in}}{\pgfqpoint{5.445830in}{3.776462in}}%
\pgfpathcurveto{\pgfqpoint{5.445830in}{3.770638in}}{\pgfqpoint{5.448144in}{3.765052in}}{\pgfqpoint{5.452262in}{3.760933in}}%
\pgfpathcurveto{\pgfqpoint{5.456381in}{3.756815in}}{\pgfqpoint{5.461967in}{3.754501in}}{\pgfqpoint{5.467791in}{3.754501in}}%
\pgfpathlineto{\pgfqpoint{5.467791in}{3.754501in}}%
\pgfpathclose%
\pgfusepath{stroke,fill}%
\end{pgfscope}%
\begin{pgfscope}%
\pgfpathrectangle{\pgfqpoint{1.073501in}{0.880000in}}{\pgfqpoint{6.052998in}{6.160000in}}%
\pgfusepath{clip}%
\pgfsetbuttcap%
\pgfsetroundjoin%
\definecolor{currentfill}{rgb}{0.200000,0.800000,0.200000}%
\pgfsetfillcolor{currentfill}%
\pgfsetlinewidth{1.003750pt}%
\definecolor{currentstroke}{rgb}{0.200000,0.800000,0.200000}%
\pgfsetstrokecolor{currentstroke}%
\pgfsetdash{}{0pt}%
\pgfpathmoveto{\pgfqpoint{5.537324in}{3.836072in}}%
\pgfpathcurveto{\pgfqpoint{5.543148in}{3.836072in}}{\pgfqpoint{5.548735in}{3.838386in}}{\pgfqpoint{5.552853in}{3.842504in}}%
\pgfpathcurveto{\pgfqpoint{5.556971in}{3.846623in}}{\pgfqpoint{5.559285in}{3.852209in}}{\pgfqpoint{5.559285in}{3.858033in}}%
\pgfpathcurveto{\pgfqpoint{5.559285in}{3.863857in}}{\pgfqpoint{5.556971in}{3.869443in}}{\pgfqpoint{5.552853in}{3.873561in}}%
\pgfpathcurveto{\pgfqpoint{5.548735in}{3.877679in}}{\pgfqpoint{5.543148in}{3.879993in}}{\pgfqpoint{5.537324in}{3.879993in}}%
\pgfpathcurveto{\pgfqpoint{5.531500in}{3.879993in}}{\pgfqpoint{5.525914in}{3.877679in}}{\pgfqpoint{5.521796in}{3.873561in}}%
\pgfpathcurveto{\pgfqpoint{5.517678in}{3.869443in}}{\pgfqpoint{5.515364in}{3.863857in}}{\pgfqpoint{5.515364in}{3.858033in}}%
\pgfpathcurveto{\pgfqpoint{5.515364in}{3.852209in}}{\pgfqpoint{5.517678in}{3.846623in}}{\pgfqpoint{5.521796in}{3.842504in}}%
\pgfpathcurveto{\pgfqpoint{5.525914in}{3.838386in}}{\pgfqpoint{5.531500in}{3.836072in}}{\pgfqpoint{5.537324in}{3.836072in}}%
\pgfpathlineto{\pgfqpoint{5.537324in}{3.836072in}}%
\pgfpathclose%
\pgfusepath{stroke,fill}%
\end{pgfscope}%
\begin{pgfscope}%
\pgfpathrectangle{\pgfqpoint{1.073501in}{0.880000in}}{\pgfqpoint{6.052998in}{6.160000in}}%
\pgfusepath{clip}%
\pgfsetbuttcap%
\pgfsetroundjoin%
\definecolor{currentfill}{rgb}{0.200000,0.800000,0.200000}%
\pgfsetfillcolor{currentfill}%
\pgfsetlinewidth{1.003750pt}%
\definecolor{currentstroke}{rgb}{0.200000,0.800000,0.200000}%
\pgfsetstrokecolor{currentstroke}%
\pgfsetdash{}{0pt}%
\pgfpathmoveto{\pgfqpoint{5.562371in}{3.925790in}}%
\pgfpathcurveto{\pgfqpoint{5.568195in}{3.925790in}}{\pgfqpoint{5.573781in}{3.928103in}}{\pgfqpoint{5.577899in}{3.932222in}}%
\pgfpathcurveto{\pgfqpoint{5.582017in}{3.936340in}}{\pgfqpoint{5.584331in}{3.941926in}}{\pgfqpoint{5.584331in}{3.947750in}}%
\pgfpathcurveto{\pgfqpoint{5.584331in}{3.953574in}}{\pgfqpoint{5.582017in}{3.959160in}}{\pgfqpoint{5.577899in}{3.963278in}}%
\pgfpathcurveto{\pgfqpoint{5.573781in}{3.967396in}}{\pgfqpoint{5.568195in}{3.969710in}}{\pgfqpoint{5.562371in}{3.969710in}}%
\pgfpathcurveto{\pgfqpoint{5.556547in}{3.969710in}}{\pgfqpoint{5.550961in}{3.967396in}}{\pgfqpoint{5.546843in}{3.963278in}}%
\pgfpathcurveto{\pgfqpoint{5.542725in}{3.959160in}}{\pgfqpoint{5.540411in}{3.953574in}}{\pgfqpoint{5.540411in}{3.947750in}}%
\pgfpathcurveto{\pgfqpoint{5.540411in}{3.941926in}}{\pgfqpoint{5.542725in}{3.936340in}}{\pgfqpoint{5.546843in}{3.932222in}}%
\pgfpathcurveto{\pgfqpoint{5.550961in}{3.928103in}}{\pgfqpoint{5.556547in}{3.925790in}}{\pgfqpoint{5.562371in}{3.925790in}}%
\pgfpathlineto{\pgfqpoint{5.562371in}{3.925790in}}%
\pgfpathclose%
\pgfusepath{stroke,fill}%
\end{pgfscope}%
\begin{pgfscope}%
\pgfpathrectangle{\pgfqpoint{1.073501in}{0.880000in}}{\pgfqpoint{6.052998in}{6.160000in}}%
\pgfusepath{clip}%
\pgfsetbuttcap%
\pgfsetmiterjoin%
\pgfsetlinewidth{1.003750pt}%
\definecolor{currentstroke}{rgb}{0.800000,0.200000,0.200000}%
\pgfsetstrokecolor{currentstroke}%
\pgfsetdash{}{0pt}%
\pgfpathmoveto{\pgfqpoint{4.149768in}{1.685521in}}%
\pgfpathcurveto{\pgfqpoint{4.747241in}{1.685521in}}{\pgfqpoint{5.320323in}{1.922899in}}{\pgfqpoint{5.742800in}{2.345376in}}%
\pgfpathcurveto{\pgfqpoint{6.165277in}{2.767853in}}{\pgfqpoint{6.402656in}{3.340935in}}{\pgfqpoint{6.402656in}{3.938408in}}%
\pgfpathcurveto{\pgfqpoint{6.402656in}{4.535881in}}{\pgfqpoint{6.165277in}{5.108963in}}{\pgfqpoint{5.742800in}{5.531440in}}%
\pgfpathcurveto{\pgfqpoint{5.320323in}{5.953917in}}{\pgfqpoint{4.747241in}{6.191296in}}{\pgfqpoint{4.149768in}{6.191296in}}%
\pgfpathcurveto{\pgfqpoint{3.552296in}{6.191296in}}{\pgfqpoint{2.979214in}{5.953917in}}{\pgfqpoint{2.556736in}{5.531440in}}%
\pgfpathcurveto{\pgfqpoint{2.134259in}{5.108963in}}{\pgfqpoint{1.896881in}{4.535881in}}{\pgfqpoint{1.896881in}{3.938408in}}%
\pgfpathcurveto{\pgfqpoint{1.896881in}{3.340935in}}{\pgfqpoint{2.134259in}{2.767853in}}{\pgfqpoint{2.556736in}{2.345376in}}%
\pgfpathcurveto{\pgfqpoint{2.979214in}{1.922899in}}{\pgfqpoint{3.552296in}{1.685521in}}{\pgfqpoint{4.149768in}{1.685521in}}%
\pgfpathlineto{\pgfqpoint{4.149768in}{1.685521in}}%
\pgfpathclose%
\pgfusepath{stroke}%
\end{pgfscope}%
\begin{pgfscope}%
\pgfpathrectangle{\pgfqpoint{1.073501in}{0.880000in}}{\pgfqpoint{6.052998in}{6.160000in}}%
\pgfusepath{clip}%
\pgfsetbuttcap%
\pgfsetroundjoin%
\definecolor{currentfill}{rgb}{0.000000,0.000000,0.000000}%
\pgfsetfillcolor{currentfill}%
\pgfsetlinewidth{1.003750pt}%
\definecolor{currentstroke}{rgb}{0.000000,0.000000,0.000000}%
\pgfsetstrokecolor{currentstroke}%
\pgfsetdash{}{0pt}%
\pgfsys@defobject{currentmarker}{\pgfqpoint{-0.021960in}{-0.021960in}}{\pgfqpoint{0.021960in}{0.021960in}}{%
\pgfpathmoveto{\pgfqpoint{0.000000in}{-0.021960in}}%
\pgfpathcurveto{\pgfqpoint{0.005824in}{-0.021960in}}{\pgfqpoint{0.011410in}{-0.019646in}}{\pgfqpoint{0.015528in}{-0.015528in}}%
\pgfpathcurveto{\pgfqpoint{0.019646in}{-0.011410in}}{\pgfqpoint{0.021960in}{-0.005824in}}{\pgfqpoint{0.021960in}{0.000000in}}%
\pgfpathcurveto{\pgfqpoint{0.021960in}{0.005824in}}{\pgfqpoint{0.019646in}{0.011410in}}{\pgfqpoint{0.015528in}{0.015528in}}%
\pgfpathcurveto{\pgfqpoint{0.011410in}{0.019646in}}{\pgfqpoint{0.005824in}{0.021960in}}{\pgfqpoint{0.000000in}{0.021960in}}%
\pgfpathcurveto{\pgfqpoint{-0.005824in}{0.021960in}}{\pgfqpoint{-0.011410in}{0.019646in}}{\pgfqpoint{-0.015528in}{0.015528in}}%
\pgfpathcurveto{\pgfqpoint{-0.019646in}{0.011410in}}{\pgfqpoint{-0.021960in}{0.005824in}}{\pgfqpoint{-0.021960in}{0.000000in}}%
\pgfpathcurveto{\pgfqpoint{-0.021960in}{-0.005824in}}{\pgfqpoint{-0.019646in}{-0.011410in}}{\pgfqpoint{-0.015528in}{-0.015528in}}%
\pgfpathcurveto{\pgfqpoint{-0.011410in}{-0.019646in}}{\pgfqpoint{-0.005824in}{-0.021960in}}{\pgfqpoint{0.000000in}{-0.021960in}}%
\pgfpathlineto{\pgfqpoint{0.000000in}{-0.021960in}}%
\pgfpathclose%
\pgfusepath{stroke,fill}%
}%
\begin{pgfscope}%
\pgfsys@transformshift{4.149768in}{3.938408in}%
\pgfsys@useobject{currentmarker}{}%
\end{pgfscope}%
\end{pgfscope}%
\begin{pgfscope}%
\pgfpathrectangle{\pgfqpoint{1.073501in}{0.880000in}}{\pgfqpoint{6.052998in}{6.160000in}}%
\pgfusepath{clip}%
\pgfsetbuttcap%
\pgfsetmiterjoin%
\pgfsetlinewidth{1.003750pt}%
\definecolor{currentstroke}{rgb}{0.200000,0.800000,0.200000}%
\pgfsetstrokecolor{currentstroke}%
\pgfsetdash{}{0pt}%
\pgfpathmoveto{\pgfqpoint{4.137842in}{2.527936in}}%
\pgfpathcurveto{\pgfqpoint{4.516412in}{2.527936in}}{\pgfqpoint{4.879527in}{2.678343in}}{\pgfqpoint{5.147217in}{2.946032in}}%
\pgfpathcurveto{\pgfqpoint{5.414906in}{3.213722in}}{\pgfqpoint{5.565313in}{3.576837in}}{\pgfqpoint{5.565313in}{3.955407in}}%
\pgfpathcurveto{\pgfqpoint{5.565313in}{4.333977in}}{\pgfqpoint{5.414906in}{4.697093in}}{\pgfqpoint{5.147217in}{4.964782in}}%
\pgfpathcurveto{\pgfqpoint{4.879527in}{5.232472in}}{\pgfqpoint{4.516412in}{5.382879in}}{\pgfqpoint{4.137842in}{5.382879in}}%
\pgfpathcurveto{\pgfqpoint{3.759272in}{5.382879in}}{\pgfqpoint{3.396156in}{5.232472in}}{\pgfqpoint{3.128467in}{4.964782in}}%
\pgfpathcurveto{\pgfqpoint{2.860777in}{4.697093in}}{\pgfqpoint{2.710370in}{4.333977in}}{\pgfqpoint{2.710370in}{3.955407in}}%
\pgfpathcurveto{\pgfqpoint{2.710370in}{3.576837in}}{\pgfqpoint{2.860777in}{3.213722in}}{\pgfqpoint{3.128467in}{2.946032in}}%
\pgfpathcurveto{\pgfqpoint{3.396156in}{2.678343in}}{\pgfqpoint{3.759272in}{2.527936in}}{\pgfqpoint{4.137842in}{2.527936in}}%
\pgfpathlineto{\pgfqpoint{4.137842in}{2.527936in}}%
\pgfpathclose%
\pgfusepath{stroke}%
\end{pgfscope}%
\begin{pgfscope}%
\pgfpathrectangle{\pgfqpoint{1.073501in}{0.880000in}}{\pgfqpoint{6.052998in}{6.160000in}}%
\pgfusepath{clip}%
\pgfsetbuttcap%
\pgfsetroundjoin%
\definecolor{currentfill}{rgb}{0.000000,0.000000,0.000000}%
\pgfsetfillcolor{currentfill}%
\pgfsetlinewidth{1.003750pt}%
\definecolor{currentstroke}{rgb}{0.000000,0.000000,0.000000}%
\pgfsetstrokecolor{currentstroke}%
\pgfsetdash{}{0pt}%
\pgfsys@defobject{currentmarker}{\pgfqpoint{-0.021960in}{-0.021960in}}{\pgfqpoint{0.021960in}{0.021960in}}{%
\pgfpathmoveto{\pgfqpoint{0.000000in}{-0.021960in}}%
\pgfpathcurveto{\pgfqpoint{0.005824in}{-0.021960in}}{\pgfqpoint{0.011410in}{-0.019646in}}{\pgfqpoint{0.015528in}{-0.015528in}}%
\pgfpathcurveto{\pgfqpoint{0.019646in}{-0.011410in}}{\pgfqpoint{0.021960in}{-0.005824in}}{\pgfqpoint{0.021960in}{0.000000in}}%
\pgfpathcurveto{\pgfqpoint{0.021960in}{0.005824in}}{\pgfqpoint{0.019646in}{0.011410in}}{\pgfqpoint{0.015528in}{0.015528in}}%
\pgfpathcurveto{\pgfqpoint{0.011410in}{0.019646in}}{\pgfqpoint{0.005824in}{0.021960in}}{\pgfqpoint{0.000000in}{0.021960in}}%
\pgfpathcurveto{\pgfqpoint{-0.005824in}{0.021960in}}{\pgfqpoint{-0.011410in}{0.019646in}}{\pgfqpoint{-0.015528in}{0.015528in}}%
\pgfpathcurveto{\pgfqpoint{-0.019646in}{0.011410in}}{\pgfqpoint{-0.021960in}{0.005824in}}{\pgfqpoint{-0.021960in}{0.000000in}}%
\pgfpathcurveto{\pgfqpoint{-0.021960in}{-0.005824in}}{\pgfqpoint{-0.019646in}{-0.011410in}}{\pgfqpoint{-0.015528in}{-0.015528in}}%
\pgfpathcurveto{\pgfqpoint{-0.011410in}{-0.019646in}}{\pgfqpoint{-0.005824in}{-0.021960in}}{\pgfqpoint{0.000000in}{-0.021960in}}%
\pgfpathlineto{\pgfqpoint{0.000000in}{-0.021960in}}%
\pgfpathclose%
\pgfusepath{stroke,fill}%
}%
\begin{pgfscope}%
\pgfsys@transformshift{4.137842in}{3.955407in}%
\pgfsys@useobject{currentmarker}{}%
\end{pgfscope}%
\end{pgfscope}%
\begin{pgfscope}%
\pgfpathrectangle{\pgfqpoint{1.073501in}{0.880000in}}{\pgfqpoint{6.052998in}{6.160000in}}%
\pgfusepath{clip}%
\pgfsetbuttcap%
\pgfsetmiterjoin%
\pgfsetlinewidth{1.003750pt}%
\definecolor{currentstroke}{rgb}{0.200000,0.200000,0.800000}%
\pgfsetstrokecolor{currentstroke}%
\pgfsetdash{}{0pt}%
\pgfpathmoveto{\pgfqpoint{4.156988in}{1.242134in}}%
\pgfpathcurveto{\pgfqpoint{4.874552in}{1.242134in}}{\pgfqpoint{5.562822in}{1.527225in}}{\pgfqpoint{6.070216in}{2.034619in}}%
\pgfpathcurveto{\pgfqpoint{6.577610in}{2.542013in}}{\pgfqpoint{6.862701in}{3.230283in}}{\pgfqpoint{6.862701in}{3.947846in}}%
\pgfpathcurveto{\pgfqpoint{6.862701in}{4.665409in}}{\pgfqpoint{6.577610in}{5.353680in}}{\pgfqpoint{6.070216in}{5.861074in}}%
\pgfpathcurveto{\pgfqpoint{5.562822in}{6.368467in}}{\pgfqpoint{4.874552in}{6.653558in}}{\pgfqpoint{4.156988in}{6.653558in}}%
\pgfpathcurveto{\pgfqpoint{3.439425in}{6.653558in}}{\pgfqpoint{2.751155in}{6.368467in}}{\pgfqpoint{2.243761in}{5.861074in}}%
\pgfpathcurveto{\pgfqpoint{1.736367in}{5.353680in}}{\pgfqpoint{1.451276in}{4.665409in}}{\pgfqpoint{1.451276in}{3.947846in}}%
\pgfpathcurveto{\pgfqpoint{1.451276in}{3.230283in}}{\pgfqpoint{1.736367in}{2.542013in}}{\pgfqpoint{2.243761in}{2.034619in}}%
\pgfpathcurveto{\pgfqpoint{2.751155in}{1.527225in}}{\pgfqpoint{3.439425in}{1.242134in}}{\pgfqpoint{4.156988in}{1.242134in}}%
\pgfpathlineto{\pgfqpoint{4.156988in}{1.242134in}}%
\pgfpathclose%
\pgfusepath{stroke}%
\end{pgfscope}%
\begin{pgfscope}%
\pgfpathrectangle{\pgfqpoint{1.073501in}{0.880000in}}{\pgfqpoint{6.052998in}{6.160000in}}%
\pgfusepath{clip}%
\pgfsetbuttcap%
\pgfsetroundjoin%
\definecolor{currentfill}{rgb}{0.000000,0.000000,0.000000}%
\pgfsetfillcolor{currentfill}%
\pgfsetlinewidth{1.003750pt}%
\definecolor{currentstroke}{rgb}{0.000000,0.000000,0.000000}%
\pgfsetstrokecolor{currentstroke}%
\pgfsetdash{}{0pt}%
\pgfsys@defobject{currentmarker}{\pgfqpoint{-0.021960in}{-0.021960in}}{\pgfqpoint{0.021960in}{0.021960in}}{%
\pgfpathmoveto{\pgfqpoint{0.000000in}{-0.021960in}}%
\pgfpathcurveto{\pgfqpoint{0.005824in}{-0.021960in}}{\pgfqpoint{0.011410in}{-0.019646in}}{\pgfqpoint{0.015528in}{-0.015528in}}%
\pgfpathcurveto{\pgfqpoint{0.019646in}{-0.011410in}}{\pgfqpoint{0.021960in}{-0.005824in}}{\pgfqpoint{0.021960in}{0.000000in}}%
\pgfpathcurveto{\pgfqpoint{0.021960in}{0.005824in}}{\pgfqpoint{0.019646in}{0.011410in}}{\pgfqpoint{0.015528in}{0.015528in}}%
\pgfpathcurveto{\pgfqpoint{0.011410in}{0.019646in}}{\pgfqpoint{0.005824in}{0.021960in}}{\pgfqpoint{0.000000in}{0.021960in}}%
\pgfpathcurveto{\pgfqpoint{-0.005824in}{0.021960in}}{\pgfqpoint{-0.011410in}{0.019646in}}{\pgfqpoint{-0.015528in}{0.015528in}}%
\pgfpathcurveto{\pgfqpoint{-0.019646in}{0.011410in}}{\pgfqpoint{-0.021960in}{0.005824in}}{\pgfqpoint{-0.021960in}{0.000000in}}%
\pgfpathcurveto{\pgfqpoint{-0.021960in}{-0.005824in}}{\pgfqpoint{-0.019646in}{-0.011410in}}{\pgfqpoint{-0.015528in}{-0.015528in}}%
\pgfpathcurveto{\pgfqpoint{-0.011410in}{-0.019646in}}{\pgfqpoint{-0.005824in}{-0.021960in}}{\pgfqpoint{0.000000in}{-0.021960in}}%
\pgfpathlineto{\pgfqpoint{0.000000in}{-0.021960in}}%
\pgfpathclose%
\pgfusepath{stroke,fill}%
}%
\begin{pgfscope}%
\pgfsys@transformshift{4.156988in}{3.947846in}%
\pgfsys@useobject{currentmarker}{}%
\end{pgfscope}%
\end{pgfscope}%
\begin{pgfscope}%
\pgfsetbuttcap%
\pgfsetroundjoin%
\definecolor{currentfill}{rgb}{0.000000,0.000000,0.000000}%
\pgfsetfillcolor{currentfill}%
\pgfsetlinewidth{0.803000pt}%
\definecolor{currentstroke}{rgb}{0.000000,0.000000,0.000000}%
\pgfsetstrokecolor{currentstroke}%
\pgfsetdash{}{0pt}%
\pgfsys@defobject{currentmarker}{\pgfqpoint{0.000000in}{-0.048611in}}{\pgfqpoint{0.000000in}{0.000000in}}{%
\pgfpathmoveto{\pgfqpoint{0.000000in}{0.000000in}}%
\pgfpathlineto{\pgfqpoint{0.000000in}{-0.048611in}}%
\pgfusepath{stroke,fill}%
}%
\begin{pgfscope}%
\pgfsys@transformshift{1.736134in}{0.880000in}%
\pgfsys@useobject{currentmarker}{}%
\end{pgfscope}%
\end{pgfscope}%
\begin{pgfscope}%
\definecolor{textcolor}{rgb}{0.000000,0.000000,0.000000}%
\pgfsetstrokecolor{textcolor}%
\pgfsetfillcolor{textcolor}%
\pgftext[x=1.736134in,y=0.782778in,,top]{\color{textcolor}{\sffamily\fontsize{10.000000}{12.000000}\selectfont\catcode`\^=\active\def^{\ifmmode\sp\else\^{}\fi}\catcode`\%=\active\def%{\%}\ensuremath{-}400}}%
\end{pgfscope}%
\begin{pgfscope}%
\pgfsetbuttcap%
\pgfsetroundjoin%
\definecolor{currentfill}{rgb}{0.000000,0.000000,0.000000}%
\pgfsetfillcolor{currentfill}%
\pgfsetlinewidth{0.803000pt}%
\definecolor{currentstroke}{rgb}{0.000000,0.000000,0.000000}%
\pgfsetstrokecolor{currentstroke}%
\pgfsetdash{}{0pt}%
\pgfsys@defobject{currentmarker}{\pgfqpoint{0.000000in}{-0.048611in}}{\pgfqpoint{0.000000in}{0.000000in}}{%
\pgfpathmoveto{\pgfqpoint{0.000000in}{0.000000in}}%
\pgfpathlineto{\pgfqpoint{0.000000in}{-0.048611in}}%
\pgfusepath{stroke,fill}%
}%
\begin{pgfscope}%
\pgfsys@transformshift{2.930871in}{0.880000in}%
\pgfsys@useobject{currentmarker}{}%
\end{pgfscope}%
\end{pgfscope}%
\begin{pgfscope}%
\definecolor{textcolor}{rgb}{0.000000,0.000000,0.000000}%
\pgfsetstrokecolor{textcolor}%
\pgfsetfillcolor{textcolor}%
\pgftext[x=2.930871in,y=0.782778in,,top]{\color{textcolor}{\sffamily\fontsize{10.000000}{12.000000}\selectfont\catcode`\^=\active\def^{\ifmmode\sp\else\^{}\fi}\catcode`\%=\active\def%{\%}\ensuremath{-}200}}%
\end{pgfscope}%
\begin{pgfscope}%
\pgfsetbuttcap%
\pgfsetroundjoin%
\definecolor{currentfill}{rgb}{0.000000,0.000000,0.000000}%
\pgfsetfillcolor{currentfill}%
\pgfsetlinewidth{0.803000pt}%
\definecolor{currentstroke}{rgb}{0.000000,0.000000,0.000000}%
\pgfsetstrokecolor{currentstroke}%
\pgfsetdash{}{0pt}%
\pgfsys@defobject{currentmarker}{\pgfqpoint{0.000000in}{-0.048611in}}{\pgfqpoint{0.000000in}{0.000000in}}{%
\pgfpathmoveto{\pgfqpoint{0.000000in}{0.000000in}}%
\pgfpathlineto{\pgfqpoint{0.000000in}{-0.048611in}}%
\pgfusepath{stroke,fill}%
}%
\begin{pgfscope}%
\pgfsys@transformshift{4.125609in}{0.880000in}%
\pgfsys@useobject{currentmarker}{}%
\end{pgfscope}%
\end{pgfscope}%
\begin{pgfscope}%
\definecolor{textcolor}{rgb}{0.000000,0.000000,0.000000}%
\pgfsetstrokecolor{textcolor}%
\pgfsetfillcolor{textcolor}%
\pgftext[x=4.125609in,y=0.782778in,,top]{\color{textcolor}{\sffamily\fontsize{10.000000}{12.000000}\selectfont\catcode`\^=\active\def^{\ifmmode\sp\else\^{}\fi}\catcode`\%=\active\def%{\%}0}}%
\end{pgfscope}%
\begin{pgfscope}%
\pgfsetbuttcap%
\pgfsetroundjoin%
\definecolor{currentfill}{rgb}{0.000000,0.000000,0.000000}%
\pgfsetfillcolor{currentfill}%
\pgfsetlinewidth{0.803000pt}%
\definecolor{currentstroke}{rgb}{0.000000,0.000000,0.000000}%
\pgfsetstrokecolor{currentstroke}%
\pgfsetdash{}{0pt}%
\pgfsys@defobject{currentmarker}{\pgfqpoint{0.000000in}{-0.048611in}}{\pgfqpoint{0.000000in}{0.000000in}}{%
\pgfpathmoveto{\pgfqpoint{0.000000in}{0.000000in}}%
\pgfpathlineto{\pgfqpoint{0.000000in}{-0.048611in}}%
\pgfusepath{stroke,fill}%
}%
\begin{pgfscope}%
\pgfsys@transformshift{5.320347in}{0.880000in}%
\pgfsys@useobject{currentmarker}{}%
\end{pgfscope}%
\end{pgfscope}%
\begin{pgfscope}%
\definecolor{textcolor}{rgb}{0.000000,0.000000,0.000000}%
\pgfsetstrokecolor{textcolor}%
\pgfsetfillcolor{textcolor}%
\pgftext[x=5.320347in,y=0.782778in,,top]{\color{textcolor}{\sffamily\fontsize{10.000000}{12.000000}\selectfont\catcode`\^=\active\def^{\ifmmode\sp\else\^{}\fi}\catcode`\%=\active\def%{\%}200}}%
\end{pgfscope}%
\begin{pgfscope}%
\pgfsetbuttcap%
\pgfsetroundjoin%
\definecolor{currentfill}{rgb}{0.000000,0.000000,0.000000}%
\pgfsetfillcolor{currentfill}%
\pgfsetlinewidth{0.803000pt}%
\definecolor{currentstroke}{rgb}{0.000000,0.000000,0.000000}%
\pgfsetstrokecolor{currentstroke}%
\pgfsetdash{}{0pt}%
\pgfsys@defobject{currentmarker}{\pgfqpoint{0.000000in}{-0.048611in}}{\pgfqpoint{0.000000in}{0.000000in}}{%
\pgfpathmoveto{\pgfqpoint{0.000000in}{0.000000in}}%
\pgfpathlineto{\pgfqpoint{0.000000in}{-0.048611in}}%
\pgfusepath{stroke,fill}%
}%
\begin{pgfscope}%
\pgfsys@transformshift{6.515084in}{0.880000in}%
\pgfsys@useobject{currentmarker}{}%
\end{pgfscope}%
\end{pgfscope}%
\begin{pgfscope}%
\definecolor{textcolor}{rgb}{0.000000,0.000000,0.000000}%
\pgfsetstrokecolor{textcolor}%
\pgfsetfillcolor{textcolor}%
\pgftext[x=6.515084in,y=0.782778in,,top]{\color{textcolor}{\sffamily\fontsize{10.000000}{12.000000}\selectfont\catcode`\^=\active\def^{\ifmmode\sp\else\^{}\fi}\catcode`\%=\active\def%{\%}400}}%
\end{pgfscope}%
\begin{pgfscope}%
\pgfsetbuttcap%
\pgfsetroundjoin%
\definecolor{currentfill}{rgb}{0.000000,0.000000,0.000000}%
\pgfsetfillcolor{currentfill}%
\pgfsetlinewidth{0.803000pt}%
\definecolor{currentstroke}{rgb}{0.000000,0.000000,0.000000}%
\pgfsetstrokecolor{currentstroke}%
\pgfsetdash{}{0pt}%
\pgfsys@defobject{currentmarker}{\pgfqpoint{-0.048611in}{0.000000in}}{\pgfqpoint{-0.000000in}{0.000000in}}{%
\pgfpathmoveto{\pgfqpoint{-0.000000in}{0.000000in}}%
\pgfpathlineto{\pgfqpoint{-0.048611in}{0.000000in}}%
\pgfusepath{stroke,fill}%
}%
\begin{pgfscope}%
\pgfsys@transformshift{1.073501in}{1.558274in}%
\pgfsys@useobject{currentmarker}{}%
\end{pgfscope}%
\end{pgfscope}%
\begin{pgfscope}%
\definecolor{textcolor}{rgb}{0.000000,0.000000,0.000000}%
\pgfsetstrokecolor{textcolor}%
\pgfsetfillcolor{textcolor}%
\pgftext[x=0.603157in, y=1.505513in, left, base]{\color{textcolor}{\sffamily\fontsize{10.000000}{12.000000}\selectfont\catcode`\^=\active\def^{\ifmmode\sp\else\^{}\fi}\catcode`\%=\active\def%{\%}\ensuremath{-}400}}%
\end{pgfscope}%
\begin{pgfscope}%
\pgfsetbuttcap%
\pgfsetroundjoin%
\definecolor{currentfill}{rgb}{0.000000,0.000000,0.000000}%
\pgfsetfillcolor{currentfill}%
\pgfsetlinewidth{0.803000pt}%
\definecolor{currentstroke}{rgb}{0.000000,0.000000,0.000000}%
\pgfsetstrokecolor{currentstroke}%
\pgfsetdash{}{0pt}%
\pgfsys@defobject{currentmarker}{\pgfqpoint{-0.048611in}{0.000000in}}{\pgfqpoint{-0.000000in}{0.000000in}}{%
\pgfpathmoveto{\pgfqpoint{-0.000000in}{0.000000in}}%
\pgfpathlineto{\pgfqpoint{-0.048611in}{0.000000in}}%
\pgfusepath{stroke,fill}%
}%
\begin{pgfscope}%
\pgfsys@transformshift{1.073501in}{2.753012in}%
\pgfsys@useobject{currentmarker}{}%
\end{pgfscope}%
\end{pgfscope}%
\begin{pgfscope}%
\definecolor{textcolor}{rgb}{0.000000,0.000000,0.000000}%
\pgfsetstrokecolor{textcolor}%
\pgfsetfillcolor{textcolor}%
\pgftext[x=0.603157in, y=2.700251in, left, base]{\color{textcolor}{\sffamily\fontsize{10.000000}{12.000000}\selectfont\catcode`\^=\active\def^{\ifmmode\sp\else\^{}\fi}\catcode`\%=\active\def%{\%}\ensuremath{-}200}}%
\end{pgfscope}%
\begin{pgfscope}%
\pgfsetbuttcap%
\pgfsetroundjoin%
\definecolor{currentfill}{rgb}{0.000000,0.000000,0.000000}%
\pgfsetfillcolor{currentfill}%
\pgfsetlinewidth{0.803000pt}%
\definecolor{currentstroke}{rgb}{0.000000,0.000000,0.000000}%
\pgfsetstrokecolor{currentstroke}%
\pgfsetdash{}{0pt}%
\pgfsys@defobject{currentmarker}{\pgfqpoint{-0.048611in}{0.000000in}}{\pgfqpoint{-0.000000in}{0.000000in}}{%
\pgfpathmoveto{\pgfqpoint{-0.000000in}{0.000000in}}%
\pgfpathlineto{\pgfqpoint{-0.048611in}{0.000000in}}%
\pgfusepath{stroke,fill}%
}%
\begin{pgfscope}%
\pgfsys@transformshift{1.073501in}{3.947750in}%
\pgfsys@useobject{currentmarker}{}%
\end{pgfscope}%
\end{pgfscope}%
\begin{pgfscope}%
\definecolor{textcolor}{rgb}{0.000000,0.000000,0.000000}%
\pgfsetstrokecolor{textcolor}%
\pgfsetfillcolor{textcolor}%
\pgftext[x=0.887913in, y=3.894988in, left, base]{\color{textcolor}{\sffamily\fontsize{10.000000}{12.000000}\selectfont\catcode`\^=\active\def^{\ifmmode\sp\else\^{}\fi}\catcode`\%=\active\def%{\%}0}}%
\end{pgfscope}%
\begin{pgfscope}%
\pgfsetbuttcap%
\pgfsetroundjoin%
\definecolor{currentfill}{rgb}{0.000000,0.000000,0.000000}%
\pgfsetfillcolor{currentfill}%
\pgfsetlinewidth{0.803000pt}%
\definecolor{currentstroke}{rgb}{0.000000,0.000000,0.000000}%
\pgfsetstrokecolor{currentstroke}%
\pgfsetdash{}{0pt}%
\pgfsys@defobject{currentmarker}{\pgfqpoint{-0.048611in}{0.000000in}}{\pgfqpoint{-0.000000in}{0.000000in}}{%
\pgfpathmoveto{\pgfqpoint{-0.000000in}{0.000000in}}%
\pgfpathlineto{\pgfqpoint{-0.048611in}{0.000000in}}%
\pgfusepath{stroke,fill}%
}%
\begin{pgfscope}%
\pgfsys@transformshift{1.073501in}{5.142488in}%
\pgfsys@useobject{currentmarker}{}%
\end{pgfscope}%
\end{pgfscope}%
\begin{pgfscope}%
\definecolor{textcolor}{rgb}{0.000000,0.000000,0.000000}%
\pgfsetstrokecolor{textcolor}%
\pgfsetfillcolor{textcolor}%
\pgftext[x=0.711183in, y=5.089726in, left, base]{\color{textcolor}{\sffamily\fontsize{10.000000}{12.000000}\selectfont\catcode`\^=\active\def^{\ifmmode\sp\else\^{}\fi}\catcode`\%=\active\def%{\%}200}}%
\end{pgfscope}%
\begin{pgfscope}%
\pgfsetbuttcap%
\pgfsetroundjoin%
\definecolor{currentfill}{rgb}{0.000000,0.000000,0.000000}%
\pgfsetfillcolor{currentfill}%
\pgfsetlinewidth{0.803000pt}%
\definecolor{currentstroke}{rgb}{0.000000,0.000000,0.000000}%
\pgfsetstrokecolor{currentstroke}%
\pgfsetdash{}{0pt}%
\pgfsys@defobject{currentmarker}{\pgfqpoint{-0.048611in}{0.000000in}}{\pgfqpoint{-0.000000in}{0.000000in}}{%
\pgfpathmoveto{\pgfqpoint{-0.000000in}{0.000000in}}%
\pgfpathlineto{\pgfqpoint{-0.048611in}{0.000000in}}%
\pgfusepath{stroke,fill}%
}%
\begin{pgfscope}%
\pgfsys@transformshift{1.073501in}{6.337225in}%
\pgfsys@useobject{currentmarker}{}%
\end{pgfscope}%
\end{pgfscope}%
\begin{pgfscope}%
\definecolor{textcolor}{rgb}{0.000000,0.000000,0.000000}%
\pgfsetstrokecolor{textcolor}%
\pgfsetfillcolor{textcolor}%
\pgftext[x=0.711183in, y=6.284464in, left, base]{\color{textcolor}{\sffamily\fontsize{10.000000}{12.000000}\selectfont\catcode`\^=\active\def^{\ifmmode\sp\else\^{}\fi}\catcode`\%=\active\def%{\%}400}}%
\end{pgfscope}%
\begin{pgfscope}%
\pgfsetrectcap%
\pgfsetmiterjoin%
\pgfsetlinewidth{0.803000pt}%
\definecolor{currentstroke}{rgb}{0.000000,0.000000,0.000000}%
\pgfsetstrokecolor{currentstroke}%
\pgfsetdash{}{0pt}%
\pgfpathmoveto{\pgfqpoint{1.073501in}{0.880000in}}%
\pgfpathlineto{\pgfqpoint{1.073501in}{7.040000in}}%
\pgfusepath{stroke}%
\end{pgfscope}%
\begin{pgfscope}%
\pgfsetrectcap%
\pgfsetmiterjoin%
\pgfsetlinewidth{0.803000pt}%
\definecolor{currentstroke}{rgb}{0.000000,0.000000,0.000000}%
\pgfsetstrokecolor{currentstroke}%
\pgfsetdash{}{0pt}%
\pgfpathmoveto{\pgfqpoint{7.126499in}{0.880000in}}%
\pgfpathlineto{\pgfqpoint{7.126499in}{7.040000in}}%
\pgfusepath{stroke}%
\end{pgfscope}%
\begin{pgfscope}%
\pgfsetrectcap%
\pgfsetmiterjoin%
\pgfsetlinewidth{0.803000pt}%
\definecolor{currentstroke}{rgb}{0.000000,0.000000,0.000000}%
\pgfsetstrokecolor{currentstroke}%
\pgfsetdash{}{0pt}%
\pgfpathmoveto{\pgfqpoint{1.073501in}{0.880000in}}%
\pgfpathlineto{\pgfqpoint{7.126499in}{0.880000in}}%
\pgfusepath{stroke}%
\end{pgfscope}%
\begin{pgfscope}%
\pgfsetrectcap%
\pgfsetmiterjoin%
\pgfsetlinewidth{0.803000pt}%
\definecolor{currentstroke}{rgb}{0.000000,0.000000,0.000000}%
\pgfsetstrokecolor{currentstroke}%
\pgfsetdash{}{0pt}%
\pgfpathmoveto{\pgfqpoint{1.073501in}{7.040000in}}%
\pgfpathlineto{\pgfqpoint{7.126499in}{7.040000in}}%
\pgfusepath{stroke}%
\end{pgfscope}%
\end{pgfpicture}%
\makeatother%
\endgroup%
}
    \label{fig:concentric_rings}
    \caption{Example of a dataset with 4 concentric rings, and background noise, and a good classification.}
\end{figure}

\begin{figure*}[!ht]
\centering
\begin{tabular}{rrrrrrrr}
    \hline
       Number of rings &   Ring noise &   Background noise &   Avg. Error &   Avg. Runtime &   Iterations &   Experiments &   Avg. Detected Noise \\
    \hline
                     2 &            0 &                  0 &      2.46091 &     0.00138317 &      4       &             3 &                0      \\
                     2 &           10 &                  0 &      8.34515 &     0.00144163 &      4.33333 &             3 &                0      \\
                     2 &           10 &                 20 &     27.6871  &     0.739527   &   3339.67    &             3 &               39.3333 \\
                     3 &            0 &                  0 &      2.77079 &     0.00386943 &      8       &             3 &                0      \\
                     3 &           10 &                  0 &      8.26071 &     0.00237887 &      5       &             3 &                0      \\
                     3 &           10 &                 10 &     31.2555  &     1.63057    &   5004.5     &             2 &               33.5    \\
                     4 &            0 &                  0 &      7.70461 &     0.0523678  &    102       &             3 &                0      \\
                     4 &           10 &                 10 &    308.983   &     4.96551    &  10000       &             4 &                0      \\
    \hline
\end{tabular}
\caption{Results of the general test with concentric rings.}
\end{figure*}
\begin{figure*}[!ht]
    \centering
    \begin{tabular}{rrrrrr}
        \hline
           Avg. Error &   Avg. Runtime &   Iterations &   Experiments &   Avg. Detected Noise &   Background noise \\
        \hline
              4.24471 &     0.00268757 &       7.25   &             4 &                    86 &                100 \\
             19.0933  &     0.0050642  &      10.3333 &             3 &                   160 &                200 \\
        \hline
        \end{tabular}
    \caption{Results of the needle in the haystack test.}
    \end{figure*}

It is important to mention that we found the hyperparameters to be sensible. For the sake of generality, we used consistent hyperparameters for the general experiments.


\subsubsection{Needle in the Haystack - Finding a ring in a lot of noise}

% figure
\begin{figure}[H]
    \centering
    \resizebox{0.9\linewidth}{!}{%% Creator: Matplotlib, PGF backend
%%
%% To include the figure in your LaTeX document, write
%%   \input{<filename>.pgf}
%%
%% Make sure the required packages are loaded in your preamble
%%   \usepackage{pgf}
%%
%% Also ensure that all the required font packages are loaded; for instance,
%% the lmodern package is sometimes necessary when using math font.
%%   \usepackage{lmodern}
%%
%% Figures using additional raster images can only be included by \input if
%% they are in the same directory as the main LaTeX file. For loading figures
%% from other directories you can use the `import` package
%%   \usepackage{import}
%%
%% and then include the figures with
%%   \import{<path to file>}{<filename>.pgf}
%%
%% Matplotlib used the following preamble
%%   \def\mathdefault#1{#1}
%%   \everymath=\expandafter{\the\everymath\displaystyle}
%%   
%%   \usepackage{fontspec}
%%   \setmainfont{DejaVuSerif.ttf}[Path=\detokenize{C:/Users/dagom/anaconda3/envs/pytorch/lib/site-packages/matplotlib/mpl-data/fonts/ttf/}]
%%   \setsansfont{DejaVuSans.ttf}[Path=\detokenize{C:/Users/dagom/anaconda3/envs/pytorch/lib/site-packages/matplotlib/mpl-data/fonts/ttf/}]
%%   \setmonofont{DejaVuSansMono.ttf}[Path=\detokenize{C:/Users/dagom/anaconda3/envs/pytorch/lib/site-packages/matplotlib/mpl-data/fonts/ttf/}]
%%   \makeatletter\@ifpackageloaded{underscore}{}{\usepackage[strings]{underscore}}\makeatother
%%
\begingroup%
\makeatletter%
\begin{pgfpicture}%
\pgfpathrectangle{\pgfpointorigin}{\pgfqpoint{8.000000in}{8.000000in}}%
\pgfusepath{use as bounding box, clip}%
\begin{pgfscope}%
\pgfsetbuttcap%
\pgfsetmiterjoin%
\definecolor{currentfill}{rgb}{1.000000,1.000000,1.000000}%
\pgfsetfillcolor{currentfill}%
\pgfsetlinewidth{0.000000pt}%
\definecolor{currentstroke}{rgb}{1.000000,1.000000,1.000000}%
\pgfsetstrokecolor{currentstroke}%
\pgfsetdash{}{0pt}%
\pgfpathmoveto{\pgfqpoint{0.000000in}{0.000000in}}%
\pgfpathlineto{\pgfqpoint{8.000000in}{0.000000in}}%
\pgfpathlineto{\pgfqpoint{8.000000in}{8.000000in}}%
\pgfpathlineto{\pgfqpoint{0.000000in}{8.000000in}}%
\pgfpathlineto{\pgfqpoint{0.000000in}{0.000000in}}%
\pgfpathclose%
\pgfusepath{fill}%
\end{pgfscope}%
\begin{pgfscope}%
\pgfsetbuttcap%
\pgfsetmiterjoin%
\definecolor{currentfill}{rgb}{1.000000,1.000000,1.000000}%
\pgfsetfillcolor{currentfill}%
\pgfsetlinewidth{0.000000pt}%
\definecolor{currentstroke}{rgb}{0.000000,0.000000,0.000000}%
\pgfsetstrokecolor{currentstroke}%
\pgfsetstrokeopacity{0.000000}%
\pgfsetdash{}{0pt}%
\pgfpathmoveto{\pgfqpoint{1.582361in}{0.880000in}}%
\pgfpathlineto{\pgfqpoint{6.617639in}{0.880000in}}%
\pgfpathlineto{\pgfqpoint{6.617639in}{7.040000in}}%
\pgfpathlineto{\pgfqpoint{1.582361in}{7.040000in}}%
\pgfpathlineto{\pgfqpoint{1.582361in}{0.880000in}}%
\pgfpathclose%
\pgfusepath{fill}%
\end{pgfscope}%
\begin{pgfscope}%
\pgfpathrectangle{\pgfqpoint{1.582361in}{0.880000in}}{\pgfqpoint{5.035278in}{6.160000in}}%
\pgfusepath{clip}%
\pgfsetbuttcap%
\pgfsetroundjoin%
\definecolor{currentfill}{rgb}{0.800000,0.200000,0.200000}%
\pgfsetfillcolor{currentfill}%
\pgfsetlinewidth{1.003750pt}%
\definecolor{currentstroke}{rgb}{0.800000,0.200000,0.200000}%
\pgfsetstrokecolor{currentstroke}%
\pgfsetdash{}{0pt}%
\pgfpathmoveto{\pgfqpoint{5.440932in}{5.321269in}}%
\pgfpathcurveto{\pgfqpoint{5.446756in}{5.321269in}}{\pgfqpoint{5.452343in}{5.323583in}}{\pgfqpoint{5.456461in}{5.327701in}}%
\pgfpathcurveto{\pgfqpoint{5.460579in}{5.331820in}}{\pgfqpoint{5.462893in}{5.337406in}}{\pgfqpoint{5.462893in}{5.343230in}}%
\pgfpathcurveto{\pgfqpoint{5.462893in}{5.349054in}}{\pgfqpoint{5.460579in}{5.354640in}}{\pgfqpoint{5.456461in}{5.358758in}}%
\pgfpathcurveto{\pgfqpoint{5.452343in}{5.362876in}}{\pgfqpoint{5.446756in}{5.365190in}}{\pgfqpoint{5.440932in}{5.365190in}}%
\pgfpathcurveto{\pgfqpoint{5.435109in}{5.365190in}}{\pgfqpoint{5.429522in}{5.362876in}}{\pgfqpoint{5.425404in}{5.358758in}}%
\pgfpathcurveto{\pgfqpoint{5.421286in}{5.354640in}}{\pgfqpoint{5.418972in}{5.349054in}}{\pgfqpoint{5.418972in}{5.343230in}}%
\pgfpathcurveto{\pgfqpoint{5.418972in}{5.337406in}}{\pgfqpoint{5.421286in}{5.331820in}}{\pgfqpoint{5.425404in}{5.327701in}}%
\pgfpathcurveto{\pgfqpoint{5.429522in}{5.323583in}}{\pgfqpoint{5.435109in}{5.321269in}}{\pgfqpoint{5.440932in}{5.321269in}}%
\pgfpathlineto{\pgfqpoint{5.440932in}{5.321269in}}%
\pgfpathclose%
\pgfusepath{stroke,fill}%
\end{pgfscope}%
\begin{pgfscope}%
\pgfpathrectangle{\pgfqpoint{1.582361in}{0.880000in}}{\pgfqpoint{5.035278in}{6.160000in}}%
\pgfusepath{clip}%
\pgfsetbuttcap%
\pgfsetroundjoin%
\definecolor{currentfill}{rgb}{0.800000,0.200000,0.200000}%
\pgfsetfillcolor{currentfill}%
\pgfsetlinewidth{1.003750pt}%
\definecolor{currentstroke}{rgb}{0.800000,0.200000,0.200000}%
\pgfsetstrokecolor{currentstroke}%
\pgfsetdash{}{0pt}%
\pgfpathmoveto{\pgfqpoint{5.440226in}{5.365996in}}%
\pgfpathcurveto{\pgfqpoint{5.446050in}{5.365996in}}{\pgfqpoint{5.451636in}{5.368310in}}{\pgfqpoint{5.455755in}{5.372428in}}%
\pgfpathcurveto{\pgfqpoint{5.459873in}{5.376546in}}{\pgfqpoint{5.462187in}{5.382133in}}{\pgfqpoint{5.462187in}{5.387956in}}%
\pgfpathcurveto{\pgfqpoint{5.462187in}{5.393780in}}{\pgfqpoint{5.459873in}{5.399367in}}{\pgfqpoint{5.455755in}{5.403485in}}%
\pgfpathcurveto{\pgfqpoint{5.451636in}{5.407603in}}{\pgfqpoint{5.446050in}{5.409917in}}{\pgfqpoint{5.440226in}{5.409917in}}%
\pgfpathcurveto{\pgfqpoint{5.434402in}{5.409917in}}{\pgfqpoint{5.428816in}{5.407603in}}{\pgfqpoint{5.424698in}{5.403485in}}%
\pgfpathcurveto{\pgfqpoint{5.420580in}{5.399367in}}{\pgfqpoint{5.418266in}{5.393780in}}{\pgfqpoint{5.418266in}{5.387956in}}%
\pgfpathcurveto{\pgfqpoint{5.418266in}{5.382133in}}{\pgfqpoint{5.420580in}{5.376546in}}{\pgfqpoint{5.424698in}{5.372428in}}%
\pgfpathcurveto{\pgfqpoint{5.428816in}{5.368310in}}{\pgfqpoint{5.434402in}{5.365996in}}{\pgfqpoint{5.440226in}{5.365996in}}%
\pgfpathlineto{\pgfqpoint{5.440226in}{5.365996in}}%
\pgfpathclose%
\pgfusepath{stroke,fill}%
\end{pgfscope}%
\begin{pgfscope}%
\pgfpathrectangle{\pgfqpoint{1.582361in}{0.880000in}}{\pgfqpoint{5.035278in}{6.160000in}}%
\pgfusepath{clip}%
\pgfsetbuttcap%
\pgfsetroundjoin%
\definecolor{currentfill}{rgb}{0.800000,0.200000,0.200000}%
\pgfsetfillcolor{currentfill}%
\pgfsetlinewidth{1.003750pt}%
\definecolor{currentstroke}{rgb}{0.800000,0.200000,0.200000}%
\pgfsetstrokecolor{currentstroke}%
\pgfsetdash{}{0pt}%
\pgfpathmoveto{\pgfqpoint{5.438109in}{5.410678in}}%
\pgfpathcurveto{\pgfqpoint{5.443932in}{5.410678in}}{\pgfqpoint{5.449519in}{5.412992in}}{\pgfqpoint{5.453637in}{5.417110in}}%
\pgfpathcurveto{\pgfqpoint{5.457755in}{5.421229in}}{\pgfqpoint{5.460069in}{5.426815in}}{\pgfqpoint{5.460069in}{5.432639in}}%
\pgfpathcurveto{\pgfqpoint{5.460069in}{5.438463in}}{\pgfqpoint{5.457755in}{5.444049in}}{\pgfqpoint{5.453637in}{5.448167in}}%
\pgfpathcurveto{\pgfqpoint{5.449519in}{5.452285in}}{\pgfqpoint{5.443932in}{5.454599in}}{\pgfqpoint{5.438109in}{5.454599in}}%
\pgfpathcurveto{\pgfqpoint{5.432285in}{5.454599in}}{\pgfqpoint{5.426698in}{5.452285in}}{\pgfqpoint{5.422580in}{5.448167in}}%
\pgfpathcurveto{\pgfqpoint{5.418462in}{5.444049in}}{\pgfqpoint{5.416148in}{5.438463in}}{\pgfqpoint{5.416148in}{5.432639in}}%
\pgfpathcurveto{\pgfqpoint{5.416148in}{5.426815in}}{\pgfqpoint{5.418462in}{5.421229in}}{\pgfqpoint{5.422580in}{5.417110in}}%
\pgfpathcurveto{\pgfqpoint{5.426698in}{5.412992in}}{\pgfqpoint{5.432285in}{5.410678in}}{\pgfqpoint{5.438109in}{5.410678in}}%
\pgfpathlineto{\pgfqpoint{5.438109in}{5.410678in}}%
\pgfpathclose%
\pgfusepath{stroke,fill}%
\end{pgfscope}%
\begin{pgfscope}%
\pgfpathrectangle{\pgfqpoint{1.582361in}{0.880000in}}{\pgfqpoint{5.035278in}{6.160000in}}%
\pgfusepath{clip}%
\pgfsetbuttcap%
\pgfsetroundjoin%
\definecolor{currentfill}{rgb}{0.800000,0.200000,0.200000}%
\pgfsetfillcolor{currentfill}%
\pgfsetlinewidth{1.003750pt}%
\definecolor{currentstroke}{rgb}{0.800000,0.200000,0.200000}%
\pgfsetstrokecolor{currentstroke}%
\pgfsetdash{}{0pt}%
\pgfpathmoveto{\pgfqpoint{5.434581in}{5.455271in}}%
\pgfpathcurveto{\pgfqpoint{5.440405in}{5.455271in}}{\pgfqpoint{5.445991in}{5.457585in}}{\pgfqpoint{5.450110in}{5.461703in}}%
\pgfpathcurveto{\pgfqpoint{5.454228in}{5.465822in}}{\pgfqpoint{5.456542in}{5.471408in}}{\pgfqpoint{5.456542in}{5.477232in}}%
\pgfpathcurveto{\pgfqpoint{5.456542in}{5.483056in}}{\pgfqpoint{5.454228in}{5.488642in}}{\pgfqpoint{5.450110in}{5.492760in}}%
\pgfpathcurveto{\pgfqpoint{5.445991in}{5.496878in}}{\pgfqpoint{5.440405in}{5.499192in}}{\pgfqpoint{5.434581in}{5.499192in}}%
\pgfpathcurveto{\pgfqpoint{5.428757in}{5.499192in}}{\pgfqpoint{5.423171in}{5.496878in}}{\pgfqpoint{5.419053in}{5.492760in}}%
\pgfpathcurveto{\pgfqpoint{5.414935in}{5.488642in}}{\pgfqpoint{5.412621in}{5.483056in}}{\pgfqpoint{5.412621in}{5.477232in}}%
\pgfpathcurveto{\pgfqpoint{5.412621in}{5.471408in}}{\pgfqpoint{5.414935in}{5.465822in}}{\pgfqpoint{5.419053in}{5.461703in}}%
\pgfpathcurveto{\pgfqpoint{5.423171in}{5.457585in}}{\pgfqpoint{5.428757in}{5.455271in}}{\pgfqpoint{5.434581in}{5.455271in}}%
\pgfpathlineto{\pgfqpoint{5.434581in}{5.455271in}}%
\pgfpathclose%
\pgfusepath{stroke,fill}%
\end{pgfscope}%
\begin{pgfscope}%
\pgfpathrectangle{\pgfqpoint{1.582361in}{0.880000in}}{\pgfqpoint{5.035278in}{6.160000in}}%
\pgfusepath{clip}%
\pgfsetbuttcap%
\pgfsetroundjoin%
\definecolor{currentfill}{rgb}{0.800000,0.200000,0.200000}%
\pgfsetfillcolor{currentfill}%
\pgfsetlinewidth{1.003750pt}%
\definecolor{currentstroke}{rgb}{0.800000,0.200000,0.200000}%
\pgfsetstrokecolor{currentstroke}%
\pgfsetdash{}{0pt}%
\pgfpathmoveto{\pgfqpoint{5.429648in}{5.499731in}}%
\pgfpathcurveto{\pgfqpoint{5.435472in}{5.499731in}}{\pgfqpoint{5.441058in}{5.502045in}}{\pgfqpoint{5.445176in}{5.506163in}}%
\pgfpathcurveto{\pgfqpoint{5.449294in}{5.510281in}}{\pgfqpoint{5.451608in}{5.515867in}}{\pgfqpoint{5.451608in}{5.521691in}}%
\pgfpathcurveto{\pgfqpoint{5.451608in}{5.527515in}}{\pgfqpoint{5.449294in}{5.533101in}}{\pgfqpoint{5.445176in}{5.537219in}}%
\pgfpathcurveto{\pgfqpoint{5.441058in}{5.541338in}}{\pgfqpoint{5.435472in}{5.543651in}}{\pgfqpoint{5.429648in}{5.543651in}}%
\pgfpathcurveto{\pgfqpoint{5.423824in}{5.543651in}}{\pgfqpoint{5.418238in}{5.541338in}}{\pgfqpoint{5.414120in}{5.537219in}}%
\pgfpathcurveto{\pgfqpoint{5.410002in}{5.533101in}}{\pgfqpoint{5.407688in}{5.527515in}}{\pgfqpoint{5.407688in}{5.521691in}}%
\pgfpathcurveto{\pgfqpoint{5.407688in}{5.515867in}}{\pgfqpoint{5.410002in}{5.510281in}}{\pgfqpoint{5.414120in}{5.506163in}}%
\pgfpathcurveto{\pgfqpoint{5.418238in}{5.502045in}}{\pgfqpoint{5.423824in}{5.499731in}}{\pgfqpoint{5.429648in}{5.499731in}}%
\pgfpathlineto{\pgfqpoint{5.429648in}{5.499731in}}%
\pgfpathclose%
\pgfusepath{stroke,fill}%
\end{pgfscope}%
\begin{pgfscope}%
\pgfpathrectangle{\pgfqpoint{1.582361in}{0.880000in}}{\pgfqpoint{5.035278in}{6.160000in}}%
\pgfusepath{clip}%
\pgfsetbuttcap%
\pgfsetroundjoin%
\definecolor{currentfill}{rgb}{0.800000,0.200000,0.200000}%
\pgfsetfillcolor{currentfill}%
\pgfsetlinewidth{1.003750pt}%
\definecolor{currentstroke}{rgb}{0.800000,0.200000,0.200000}%
\pgfsetstrokecolor{currentstroke}%
\pgfsetdash{}{0pt}%
\pgfpathmoveto{\pgfqpoint{5.423314in}{5.544013in}}%
\pgfpathcurveto{\pgfqpoint{5.429138in}{5.544013in}}{\pgfqpoint{5.434724in}{5.546326in}}{\pgfqpoint{5.438842in}{5.550445in}}%
\pgfpathcurveto{\pgfqpoint{5.442960in}{5.554563in}}{\pgfqpoint{5.445274in}{5.560149in}}{\pgfqpoint{5.445274in}{5.565973in}}%
\pgfpathcurveto{\pgfqpoint{5.445274in}{5.571797in}}{\pgfqpoint{5.442960in}{5.577383in}}{\pgfqpoint{5.438842in}{5.581501in}}%
\pgfpathcurveto{\pgfqpoint{5.434724in}{5.585619in}}{\pgfqpoint{5.429138in}{5.587933in}}{\pgfqpoint{5.423314in}{5.587933in}}%
\pgfpathcurveto{\pgfqpoint{5.417490in}{5.587933in}}{\pgfqpoint{5.411904in}{5.585619in}}{\pgfqpoint{5.407785in}{5.581501in}}%
\pgfpathcurveto{\pgfqpoint{5.403667in}{5.577383in}}{\pgfqpoint{5.401353in}{5.571797in}}{\pgfqpoint{5.401353in}{5.565973in}}%
\pgfpathcurveto{\pgfqpoint{5.401353in}{5.560149in}}{\pgfqpoint{5.403667in}{5.554563in}}{\pgfqpoint{5.407785in}{5.550445in}}%
\pgfpathcurveto{\pgfqpoint{5.411904in}{5.546326in}}{\pgfqpoint{5.417490in}{5.544013in}}{\pgfqpoint{5.423314in}{5.544013in}}%
\pgfpathlineto{\pgfqpoint{5.423314in}{5.544013in}}%
\pgfpathclose%
\pgfusepath{stroke,fill}%
\end{pgfscope}%
\begin{pgfscope}%
\pgfpathrectangle{\pgfqpoint{1.582361in}{0.880000in}}{\pgfqpoint{5.035278in}{6.160000in}}%
\pgfusepath{clip}%
\pgfsetbuttcap%
\pgfsetroundjoin%
\definecolor{currentfill}{rgb}{0.800000,0.200000,0.200000}%
\pgfsetfillcolor{currentfill}%
\pgfsetlinewidth{1.003750pt}%
\definecolor{currentstroke}{rgb}{0.800000,0.200000,0.200000}%
\pgfsetstrokecolor{currentstroke}%
\pgfsetdash{}{0pt}%
\pgfpathmoveto{\pgfqpoint{5.415585in}{5.588072in}}%
\pgfpathcurveto{\pgfqpoint{5.421409in}{5.588072in}}{\pgfqpoint{5.426995in}{5.590386in}}{\pgfqpoint{5.431113in}{5.594504in}}%
\pgfpathcurveto{\pgfqpoint{5.435231in}{5.598622in}}{\pgfqpoint{5.437545in}{5.604208in}}{\pgfqpoint{5.437545in}{5.610032in}}%
\pgfpathcurveto{\pgfqpoint{5.437545in}{5.615856in}}{\pgfqpoint{5.435231in}{5.621442in}}{\pgfqpoint{5.431113in}{5.625561in}}%
\pgfpathcurveto{\pgfqpoint{5.426995in}{5.629679in}}{\pgfqpoint{5.421409in}{5.631993in}}{\pgfqpoint{5.415585in}{5.631993in}}%
\pgfpathcurveto{\pgfqpoint{5.409761in}{5.631993in}}{\pgfqpoint{5.404175in}{5.629679in}}{\pgfqpoint{5.400056in}{5.625561in}}%
\pgfpathcurveto{\pgfqpoint{5.395938in}{5.621442in}}{\pgfqpoint{5.393624in}{5.615856in}}{\pgfqpoint{5.393624in}{5.610032in}}%
\pgfpathcurveto{\pgfqpoint{5.393624in}{5.604208in}}{\pgfqpoint{5.395938in}{5.598622in}}{\pgfqpoint{5.400056in}{5.594504in}}%
\pgfpathcurveto{\pgfqpoint{5.404175in}{5.590386in}}{\pgfqpoint{5.409761in}{5.588072in}}{\pgfqpoint{5.415585in}{5.588072in}}%
\pgfpathlineto{\pgfqpoint{5.415585in}{5.588072in}}%
\pgfpathclose%
\pgfusepath{stroke,fill}%
\end{pgfscope}%
\begin{pgfscope}%
\pgfpathrectangle{\pgfqpoint{1.582361in}{0.880000in}}{\pgfqpoint{5.035278in}{6.160000in}}%
\pgfusepath{clip}%
\pgfsetbuttcap%
\pgfsetroundjoin%
\definecolor{currentfill}{rgb}{0.800000,0.200000,0.200000}%
\pgfsetfillcolor{currentfill}%
\pgfsetlinewidth{1.003750pt}%
\definecolor{currentstroke}{rgb}{0.800000,0.200000,0.200000}%
\pgfsetstrokecolor{currentstroke}%
\pgfsetdash{}{0pt}%
\pgfpathmoveto{\pgfqpoint{5.406469in}{5.631866in}}%
\pgfpathcurveto{\pgfqpoint{5.412293in}{5.631866in}}{\pgfqpoint{5.417879in}{5.634180in}}{\pgfqpoint{5.421997in}{5.638298in}}%
\pgfpathcurveto{\pgfqpoint{5.426115in}{5.642416in}}{\pgfqpoint{5.428429in}{5.648002in}}{\pgfqpoint{5.428429in}{5.653826in}}%
\pgfpathcurveto{\pgfqpoint{5.428429in}{5.659650in}}{\pgfqpoint{5.426115in}{5.665236in}}{\pgfqpoint{5.421997in}{5.669354in}}%
\pgfpathcurveto{\pgfqpoint{5.417879in}{5.673472in}}{\pgfqpoint{5.412293in}{5.675786in}}{\pgfqpoint{5.406469in}{5.675786in}}%
\pgfpathcurveto{\pgfqpoint{5.400645in}{5.675786in}}{\pgfqpoint{5.395058in}{5.673472in}}{\pgfqpoint{5.390940in}{5.669354in}}%
\pgfpathcurveto{\pgfqpoint{5.386822in}{5.665236in}}{\pgfqpoint{5.384508in}{5.659650in}}{\pgfqpoint{5.384508in}{5.653826in}}%
\pgfpathcurveto{\pgfqpoint{5.384508in}{5.648002in}}{\pgfqpoint{5.386822in}{5.642416in}}{\pgfqpoint{5.390940in}{5.638298in}}%
\pgfpathcurveto{\pgfqpoint{5.395058in}{5.634180in}}{\pgfqpoint{5.400645in}{5.631866in}}{\pgfqpoint{5.406469in}{5.631866in}}%
\pgfpathlineto{\pgfqpoint{5.406469in}{5.631866in}}%
\pgfpathclose%
\pgfusepath{stroke,fill}%
\end{pgfscope}%
\begin{pgfscope}%
\pgfpathrectangle{\pgfqpoint{1.582361in}{0.880000in}}{\pgfqpoint{5.035278in}{6.160000in}}%
\pgfusepath{clip}%
\pgfsetbuttcap%
\pgfsetroundjoin%
\definecolor{currentfill}{rgb}{0.800000,0.200000,0.200000}%
\pgfsetfillcolor{currentfill}%
\pgfsetlinewidth{1.003750pt}%
\definecolor{currentstroke}{rgb}{0.800000,0.200000,0.200000}%
\pgfsetstrokecolor{currentstroke}%
\pgfsetdash{}{0pt}%
\pgfpathmoveto{\pgfqpoint{5.395975in}{5.675350in}}%
\pgfpathcurveto{\pgfqpoint{5.401798in}{5.675350in}}{\pgfqpoint{5.407385in}{5.677664in}}{\pgfqpoint{5.411503in}{5.681782in}}%
\pgfpathcurveto{\pgfqpoint{5.415621in}{5.685900in}}{\pgfqpoint{5.417935in}{5.691486in}}{\pgfqpoint{5.417935in}{5.697310in}}%
\pgfpathcurveto{\pgfqpoint{5.417935in}{5.703134in}}{\pgfqpoint{5.415621in}{5.708720in}}{\pgfqpoint{5.411503in}{5.712838in}}%
\pgfpathcurveto{\pgfqpoint{5.407385in}{5.716956in}}{\pgfqpoint{5.401798in}{5.719270in}}{\pgfqpoint{5.395975in}{5.719270in}}%
\pgfpathcurveto{\pgfqpoint{5.390151in}{5.719270in}}{\pgfqpoint{5.384564in}{5.716956in}}{\pgfqpoint{5.380446in}{5.712838in}}%
\pgfpathcurveto{\pgfqpoint{5.376328in}{5.708720in}}{\pgfqpoint{5.374014in}{5.703134in}}{\pgfqpoint{5.374014in}{5.697310in}}%
\pgfpathcurveto{\pgfqpoint{5.374014in}{5.691486in}}{\pgfqpoint{5.376328in}{5.685900in}}{\pgfqpoint{5.380446in}{5.681782in}}%
\pgfpathcurveto{\pgfqpoint{5.384564in}{5.677664in}}{\pgfqpoint{5.390151in}{5.675350in}}{\pgfqpoint{5.395975in}{5.675350in}}%
\pgfpathlineto{\pgfqpoint{5.395975in}{5.675350in}}%
\pgfpathclose%
\pgfusepath{stroke,fill}%
\end{pgfscope}%
\begin{pgfscope}%
\pgfpathrectangle{\pgfqpoint{1.582361in}{0.880000in}}{\pgfqpoint{5.035278in}{6.160000in}}%
\pgfusepath{clip}%
\pgfsetbuttcap%
\pgfsetroundjoin%
\definecolor{currentfill}{rgb}{0.800000,0.200000,0.200000}%
\pgfsetfillcolor{currentfill}%
\pgfsetlinewidth{1.003750pt}%
\definecolor{currentstroke}{rgb}{0.800000,0.200000,0.200000}%
\pgfsetstrokecolor{currentstroke}%
\pgfsetdash{}{0pt}%
\pgfpathmoveto{\pgfqpoint{5.384113in}{5.718481in}}%
\pgfpathcurveto{\pgfqpoint{5.389937in}{5.718481in}}{\pgfqpoint{5.395523in}{5.720795in}}{\pgfqpoint{5.399641in}{5.724913in}}%
\pgfpathcurveto{\pgfqpoint{5.403759in}{5.729031in}}{\pgfqpoint{5.406073in}{5.734617in}}{\pgfqpoint{5.406073in}{5.740441in}}%
\pgfpathcurveto{\pgfqpoint{5.406073in}{5.746265in}}{\pgfqpoint{5.403759in}{5.751851in}}{\pgfqpoint{5.399641in}{5.755969in}}%
\pgfpathcurveto{\pgfqpoint{5.395523in}{5.760087in}}{\pgfqpoint{5.389937in}{5.762401in}}{\pgfqpoint{5.384113in}{5.762401in}}%
\pgfpathcurveto{\pgfqpoint{5.378289in}{5.762401in}}{\pgfqpoint{5.372703in}{5.760087in}}{\pgfqpoint{5.368585in}{5.755969in}}%
\pgfpathcurveto{\pgfqpoint{5.364467in}{5.751851in}}{\pgfqpoint{5.362153in}{5.746265in}}{\pgfqpoint{5.362153in}{5.740441in}}%
\pgfpathcurveto{\pgfqpoint{5.362153in}{5.734617in}}{\pgfqpoint{5.364467in}{5.729031in}}{\pgfqpoint{5.368585in}{5.724913in}}%
\pgfpathcurveto{\pgfqpoint{5.372703in}{5.720795in}}{\pgfqpoint{5.378289in}{5.718481in}}{\pgfqpoint{5.384113in}{5.718481in}}%
\pgfpathlineto{\pgfqpoint{5.384113in}{5.718481in}}%
\pgfpathclose%
\pgfusepath{stroke,fill}%
\end{pgfscope}%
\begin{pgfscope}%
\pgfpathrectangle{\pgfqpoint{1.582361in}{0.880000in}}{\pgfqpoint{5.035278in}{6.160000in}}%
\pgfusepath{clip}%
\pgfsetbuttcap%
\pgfsetroundjoin%
\definecolor{currentfill}{rgb}{0.800000,0.200000,0.200000}%
\pgfsetfillcolor{currentfill}%
\pgfsetlinewidth{1.003750pt}%
\definecolor{currentstroke}{rgb}{0.800000,0.200000,0.200000}%
\pgfsetstrokecolor{currentstroke}%
\pgfsetdash{}{0pt}%
\pgfpathmoveto{\pgfqpoint{5.370896in}{5.761216in}}%
\pgfpathcurveto{\pgfqpoint{5.376720in}{5.761216in}}{\pgfqpoint{5.382306in}{5.763530in}}{\pgfqpoint{5.386424in}{5.767648in}}%
\pgfpathcurveto{\pgfqpoint{5.390542in}{5.771766in}}{\pgfqpoint{5.392856in}{5.777352in}}{\pgfqpoint{5.392856in}{5.783176in}}%
\pgfpathcurveto{\pgfqpoint{5.392856in}{5.789000in}}{\pgfqpoint{5.390542in}{5.794586in}}{\pgfqpoint{5.386424in}{5.798704in}}%
\pgfpathcurveto{\pgfqpoint{5.382306in}{5.802822in}}{\pgfqpoint{5.376720in}{5.805136in}}{\pgfqpoint{5.370896in}{5.805136in}}%
\pgfpathcurveto{\pgfqpoint{5.365072in}{5.805136in}}{\pgfqpoint{5.359486in}{5.802822in}}{\pgfqpoint{5.355368in}{5.798704in}}%
\pgfpathcurveto{\pgfqpoint{5.351249in}{5.794586in}}{\pgfqpoint{5.348936in}{5.789000in}}{\pgfqpoint{5.348936in}{5.783176in}}%
\pgfpathcurveto{\pgfqpoint{5.348936in}{5.777352in}}{\pgfqpoint{5.351249in}{5.771766in}}{\pgfqpoint{5.355368in}{5.767648in}}%
\pgfpathcurveto{\pgfqpoint{5.359486in}{5.763530in}}{\pgfqpoint{5.365072in}{5.761216in}}{\pgfqpoint{5.370896in}{5.761216in}}%
\pgfpathlineto{\pgfqpoint{5.370896in}{5.761216in}}%
\pgfpathclose%
\pgfusepath{stroke,fill}%
\end{pgfscope}%
\begin{pgfscope}%
\pgfpathrectangle{\pgfqpoint{1.582361in}{0.880000in}}{\pgfqpoint{5.035278in}{6.160000in}}%
\pgfusepath{clip}%
\pgfsetbuttcap%
\pgfsetroundjoin%
\definecolor{currentfill}{rgb}{0.800000,0.200000,0.200000}%
\pgfsetfillcolor{currentfill}%
\pgfsetlinewidth{1.003750pt}%
\definecolor{currentstroke}{rgb}{0.800000,0.200000,0.200000}%
\pgfsetstrokecolor{currentstroke}%
\pgfsetdash{}{0pt}%
\pgfpathmoveto{\pgfqpoint{5.356336in}{5.803512in}}%
\pgfpathcurveto{\pgfqpoint{5.362160in}{5.803512in}}{\pgfqpoint{5.367746in}{5.805826in}}{\pgfqpoint{5.371864in}{5.809944in}}%
\pgfpathcurveto{\pgfqpoint{5.375982in}{5.814063in}}{\pgfqpoint{5.378296in}{5.819649in}}{\pgfqpoint{5.378296in}{5.825473in}}%
\pgfpathcurveto{\pgfqpoint{5.378296in}{5.831297in}}{\pgfqpoint{5.375982in}{5.836883in}}{\pgfqpoint{5.371864in}{5.841001in}}%
\pgfpathcurveto{\pgfqpoint{5.367746in}{5.845119in}}{\pgfqpoint{5.362160in}{5.847433in}}{\pgfqpoint{5.356336in}{5.847433in}}%
\pgfpathcurveto{\pgfqpoint{5.350512in}{5.847433in}}{\pgfqpoint{5.344926in}{5.845119in}}{\pgfqpoint{5.340808in}{5.841001in}}%
\pgfpathcurveto{\pgfqpoint{5.336690in}{5.836883in}}{\pgfqpoint{5.334376in}{5.831297in}}{\pgfqpoint{5.334376in}{5.825473in}}%
\pgfpathcurveto{\pgfqpoint{5.334376in}{5.819649in}}{\pgfqpoint{5.336690in}{5.814063in}}{\pgfqpoint{5.340808in}{5.809944in}}%
\pgfpathcurveto{\pgfqpoint{5.344926in}{5.805826in}}{\pgfqpoint{5.350512in}{5.803512in}}{\pgfqpoint{5.356336in}{5.803512in}}%
\pgfpathlineto{\pgfqpoint{5.356336in}{5.803512in}}%
\pgfpathclose%
\pgfusepath{stroke,fill}%
\end{pgfscope}%
\begin{pgfscope}%
\pgfpathrectangle{\pgfqpoint{1.582361in}{0.880000in}}{\pgfqpoint{5.035278in}{6.160000in}}%
\pgfusepath{clip}%
\pgfsetbuttcap%
\pgfsetroundjoin%
\definecolor{currentfill}{rgb}{0.800000,0.200000,0.200000}%
\pgfsetfillcolor{currentfill}%
\pgfsetlinewidth{1.003750pt}%
\definecolor{currentstroke}{rgb}{0.800000,0.200000,0.200000}%
\pgfsetstrokecolor{currentstroke}%
\pgfsetdash{}{0pt}%
\pgfpathmoveto{\pgfqpoint{5.340448in}{5.845328in}}%
\pgfpathcurveto{\pgfqpoint{5.346272in}{5.845328in}}{\pgfqpoint{5.351858in}{5.847642in}}{\pgfqpoint{5.355977in}{5.851760in}}%
\pgfpathcurveto{\pgfqpoint{5.360095in}{5.855878in}}{\pgfqpoint{5.362409in}{5.861465in}}{\pgfqpoint{5.362409in}{5.867288in}}%
\pgfpathcurveto{\pgfqpoint{5.362409in}{5.873112in}}{\pgfqpoint{5.360095in}{5.878699in}}{\pgfqpoint{5.355977in}{5.882817in}}%
\pgfpathcurveto{\pgfqpoint{5.351858in}{5.886935in}}{\pgfqpoint{5.346272in}{5.889249in}}{\pgfqpoint{5.340448in}{5.889249in}}%
\pgfpathcurveto{\pgfqpoint{5.334624in}{5.889249in}}{\pgfqpoint{5.329038in}{5.886935in}}{\pgfqpoint{5.324920in}{5.882817in}}%
\pgfpathcurveto{\pgfqpoint{5.320802in}{5.878699in}}{\pgfqpoint{5.318488in}{5.873112in}}{\pgfqpoint{5.318488in}{5.867288in}}%
\pgfpathcurveto{\pgfqpoint{5.318488in}{5.861465in}}{\pgfqpoint{5.320802in}{5.855878in}}{\pgfqpoint{5.324920in}{5.851760in}}%
\pgfpathcurveto{\pgfqpoint{5.329038in}{5.847642in}}{\pgfqpoint{5.334624in}{5.845328in}}{\pgfqpoint{5.340448in}{5.845328in}}%
\pgfpathlineto{\pgfqpoint{5.340448in}{5.845328in}}%
\pgfpathclose%
\pgfusepath{stroke,fill}%
\end{pgfscope}%
\begin{pgfscope}%
\pgfpathrectangle{\pgfqpoint{1.582361in}{0.880000in}}{\pgfqpoint{5.035278in}{6.160000in}}%
\pgfusepath{clip}%
\pgfsetbuttcap%
\pgfsetroundjoin%
\definecolor{currentfill}{rgb}{0.800000,0.200000,0.200000}%
\pgfsetfillcolor{currentfill}%
\pgfsetlinewidth{1.003750pt}%
\definecolor{currentstroke}{rgb}{0.800000,0.200000,0.200000}%
\pgfsetstrokecolor{currentstroke}%
\pgfsetdash{}{0pt}%
\pgfpathmoveto{\pgfqpoint{5.323248in}{5.886622in}}%
\pgfpathcurveto{\pgfqpoint{5.329072in}{5.886622in}}{\pgfqpoint{5.334659in}{5.888936in}}{\pgfqpoint{5.338777in}{5.893054in}}%
\pgfpathcurveto{\pgfqpoint{5.342895in}{5.897172in}}{\pgfqpoint{5.345209in}{5.902758in}}{\pgfqpoint{5.345209in}{5.908582in}}%
\pgfpathcurveto{\pgfqpoint{5.345209in}{5.914406in}}{\pgfqpoint{5.342895in}{5.919992in}}{\pgfqpoint{5.338777in}{5.924110in}}%
\pgfpathcurveto{\pgfqpoint{5.334659in}{5.928228in}}{\pgfqpoint{5.329072in}{5.930542in}}{\pgfqpoint{5.323248in}{5.930542in}}%
\pgfpathcurveto{\pgfqpoint{5.317425in}{5.930542in}}{\pgfqpoint{5.311838in}{5.928228in}}{\pgfqpoint{5.307720in}{5.924110in}}%
\pgfpathcurveto{\pgfqpoint{5.303602in}{5.919992in}}{\pgfqpoint{5.301288in}{5.914406in}}{\pgfqpoint{5.301288in}{5.908582in}}%
\pgfpathcurveto{\pgfqpoint{5.301288in}{5.902758in}}{\pgfqpoint{5.303602in}{5.897172in}}{\pgfqpoint{5.307720in}{5.893054in}}%
\pgfpathcurveto{\pgfqpoint{5.311838in}{5.888936in}}{\pgfqpoint{5.317425in}{5.886622in}}{\pgfqpoint{5.323248in}{5.886622in}}%
\pgfpathlineto{\pgfqpoint{5.323248in}{5.886622in}}%
\pgfpathclose%
\pgfusepath{stroke,fill}%
\end{pgfscope}%
\begin{pgfscope}%
\pgfpathrectangle{\pgfqpoint{1.582361in}{0.880000in}}{\pgfqpoint{5.035278in}{6.160000in}}%
\pgfusepath{clip}%
\pgfsetbuttcap%
\pgfsetroundjoin%
\definecolor{currentfill}{rgb}{0.800000,0.200000,0.200000}%
\pgfsetfillcolor{currentfill}%
\pgfsetlinewidth{1.003750pt}%
\definecolor{currentstroke}{rgb}{0.800000,0.200000,0.200000}%
\pgfsetstrokecolor{currentstroke}%
\pgfsetdash{}{0pt}%
\pgfpathmoveto{\pgfqpoint{5.304754in}{5.927352in}}%
\pgfpathcurveto{\pgfqpoint{5.310578in}{5.927352in}}{\pgfqpoint{5.316164in}{5.929665in}}{\pgfqpoint{5.320282in}{5.933784in}}%
\pgfpathcurveto{\pgfqpoint{5.324400in}{5.937902in}}{\pgfqpoint{5.326714in}{5.943488in}}{\pgfqpoint{5.326714in}{5.949312in}}%
\pgfpathcurveto{\pgfqpoint{5.326714in}{5.955136in}}{\pgfqpoint{5.324400in}{5.960722in}}{\pgfqpoint{5.320282in}{5.964840in}}%
\pgfpathcurveto{\pgfqpoint{5.316164in}{5.968958in}}{\pgfqpoint{5.310578in}{5.971272in}}{\pgfqpoint{5.304754in}{5.971272in}}%
\pgfpathcurveto{\pgfqpoint{5.298930in}{5.971272in}}{\pgfqpoint{5.293344in}{5.968958in}}{\pgfqpoint{5.289225in}{5.964840in}}%
\pgfpathcurveto{\pgfqpoint{5.285107in}{5.960722in}}{\pgfqpoint{5.282793in}{5.955136in}}{\pgfqpoint{5.282793in}{5.949312in}}%
\pgfpathcurveto{\pgfqpoint{5.282793in}{5.943488in}}{\pgfqpoint{5.285107in}{5.937902in}}{\pgfqpoint{5.289225in}{5.933784in}}%
\pgfpathcurveto{\pgfqpoint{5.293344in}{5.929665in}}{\pgfqpoint{5.298930in}{5.927352in}}{\pgfqpoint{5.304754in}{5.927352in}}%
\pgfpathlineto{\pgfqpoint{5.304754in}{5.927352in}}%
\pgfpathclose%
\pgfusepath{stroke,fill}%
\end{pgfscope}%
\begin{pgfscope}%
\pgfpathrectangle{\pgfqpoint{1.582361in}{0.880000in}}{\pgfqpoint{5.035278in}{6.160000in}}%
\pgfusepath{clip}%
\pgfsetbuttcap%
\pgfsetroundjoin%
\definecolor{currentfill}{rgb}{0.800000,0.200000,0.200000}%
\pgfsetfillcolor{currentfill}%
\pgfsetlinewidth{1.003750pt}%
\definecolor{currentstroke}{rgb}{0.800000,0.200000,0.200000}%
\pgfsetstrokecolor{currentstroke}%
\pgfsetdash{}{0pt}%
\pgfpathmoveto{\pgfqpoint{5.284982in}{5.967477in}}%
\pgfpathcurveto{\pgfqpoint{5.290806in}{5.967477in}}{\pgfqpoint{5.296392in}{5.969791in}}{\pgfqpoint{5.300510in}{5.973909in}}%
\pgfpathcurveto{\pgfqpoint{5.304629in}{5.978027in}}{\pgfqpoint{5.306942in}{5.983614in}}{\pgfqpoint{5.306942in}{5.989438in}}%
\pgfpathcurveto{\pgfqpoint{5.306942in}{5.995261in}}{\pgfqpoint{5.304629in}{6.000848in}}{\pgfqpoint{5.300510in}{6.004966in}}%
\pgfpathcurveto{\pgfqpoint{5.296392in}{6.009084in}}{\pgfqpoint{5.290806in}{6.011398in}}{\pgfqpoint{5.284982in}{6.011398in}}%
\pgfpathcurveto{\pgfqpoint{5.279158in}{6.011398in}}{\pgfqpoint{5.273572in}{6.009084in}}{\pgfqpoint{5.269454in}{6.004966in}}%
\pgfpathcurveto{\pgfqpoint{5.265336in}{6.000848in}}{\pgfqpoint{5.263022in}{5.995261in}}{\pgfqpoint{5.263022in}{5.989438in}}%
\pgfpathcurveto{\pgfqpoint{5.263022in}{5.983614in}}{\pgfqpoint{5.265336in}{5.978027in}}{\pgfqpoint{5.269454in}{5.973909in}}%
\pgfpathcurveto{\pgfqpoint{5.273572in}{5.969791in}}{\pgfqpoint{5.279158in}{5.967477in}}{\pgfqpoint{5.284982in}{5.967477in}}%
\pgfpathlineto{\pgfqpoint{5.284982in}{5.967477in}}%
\pgfpathclose%
\pgfusepath{stroke,fill}%
\end{pgfscope}%
\begin{pgfscope}%
\pgfpathrectangle{\pgfqpoint{1.582361in}{0.880000in}}{\pgfqpoint{5.035278in}{6.160000in}}%
\pgfusepath{clip}%
\pgfsetbuttcap%
\pgfsetroundjoin%
\definecolor{currentfill}{rgb}{0.800000,0.200000,0.200000}%
\pgfsetfillcolor{currentfill}%
\pgfsetlinewidth{1.003750pt}%
\definecolor{currentstroke}{rgb}{0.800000,0.200000,0.200000}%
\pgfsetstrokecolor{currentstroke}%
\pgfsetdash{}{0pt}%
\pgfpathmoveto{\pgfqpoint{5.263954in}{6.006959in}}%
\pgfpathcurveto{\pgfqpoint{5.269778in}{6.006959in}}{\pgfqpoint{5.275364in}{6.009273in}}{\pgfqpoint{5.279482in}{6.013391in}}%
\pgfpathcurveto{\pgfqpoint{5.283600in}{6.017509in}}{\pgfqpoint{5.285914in}{6.023095in}}{\pgfqpoint{5.285914in}{6.028919in}}%
\pgfpathcurveto{\pgfqpoint{5.285914in}{6.034743in}}{\pgfqpoint{5.283600in}{6.040329in}}{\pgfqpoint{5.279482in}{6.044447in}}%
\pgfpathcurveto{\pgfqpoint{5.275364in}{6.048566in}}{\pgfqpoint{5.269778in}{6.050879in}}{\pgfqpoint{5.263954in}{6.050879in}}%
\pgfpathcurveto{\pgfqpoint{5.258130in}{6.050879in}}{\pgfqpoint{5.252544in}{6.048566in}}{\pgfqpoint{5.248426in}{6.044447in}}%
\pgfpathcurveto{\pgfqpoint{5.244308in}{6.040329in}}{\pgfqpoint{5.241994in}{6.034743in}}{\pgfqpoint{5.241994in}{6.028919in}}%
\pgfpathcurveto{\pgfqpoint{5.241994in}{6.023095in}}{\pgfqpoint{5.244308in}{6.017509in}}{\pgfqpoint{5.248426in}{6.013391in}}%
\pgfpathcurveto{\pgfqpoint{5.252544in}{6.009273in}}{\pgfqpoint{5.258130in}{6.006959in}}{\pgfqpoint{5.263954in}{6.006959in}}%
\pgfpathlineto{\pgfqpoint{5.263954in}{6.006959in}}%
\pgfpathclose%
\pgfusepath{stroke,fill}%
\end{pgfscope}%
\begin{pgfscope}%
\pgfpathrectangle{\pgfqpoint{1.582361in}{0.880000in}}{\pgfqpoint{5.035278in}{6.160000in}}%
\pgfusepath{clip}%
\pgfsetbuttcap%
\pgfsetroundjoin%
\definecolor{currentfill}{rgb}{0.800000,0.200000,0.200000}%
\pgfsetfillcolor{currentfill}%
\pgfsetlinewidth{1.003750pt}%
\definecolor{currentstroke}{rgb}{0.800000,0.200000,0.200000}%
\pgfsetstrokecolor{currentstroke}%
\pgfsetdash{}{0pt}%
\pgfpathmoveto{\pgfqpoint{5.241690in}{6.045757in}}%
\pgfpathcurveto{\pgfqpoint{5.247514in}{6.045757in}}{\pgfqpoint{5.253100in}{6.048071in}}{\pgfqpoint{5.257218in}{6.052189in}}%
\pgfpathcurveto{\pgfqpoint{5.261336in}{6.056307in}}{\pgfqpoint{5.263650in}{6.061893in}}{\pgfqpoint{5.263650in}{6.067717in}}%
\pgfpathcurveto{\pgfqpoint{5.263650in}{6.073541in}}{\pgfqpoint{5.261336in}{6.079127in}}{\pgfqpoint{5.257218in}{6.083245in}}%
\pgfpathcurveto{\pgfqpoint{5.253100in}{6.087364in}}{\pgfqpoint{5.247514in}{6.089677in}}{\pgfqpoint{5.241690in}{6.089677in}}%
\pgfpathcurveto{\pgfqpoint{5.235866in}{6.089677in}}{\pgfqpoint{5.230280in}{6.087364in}}{\pgfqpoint{5.226162in}{6.083245in}}%
\pgfpathcurveto{\pgfqpoint{5.222043in}{6.079127in}}{\pgfqpoint{5.219730in}{6.073541in}}{\pgfqpoint{5.219730in}{6.067717in}}%
\pgfpathcurveto{\pgfqpoint{5.219730in}{6.061893in}}{\pgfqpoint{5.222043in}{6.056307in}}{\pgfqpoint{5.226162in}{6.052189in}}%
\pgfpathcurveto{\pgfqpoint{5.230280in}{6.048071in}}{\pgfqpoint{5.235866in}{6.045757in}}{\pgfqpoint{5.241690in}{6.045757in}}%
\pgfpathlineto{\pgfqpoint{5.241690in}{6.045757in}}%
\pgfpathclose%
\pgfusepath{stroke,fill}%
\end{pgfscope}%
\begin{pgfscope}%
\pgfpathrectangle{\pgfqpoint{1.582361in}{0.880000in}}{\pgfqpoint{5.035278in}{6.160000in}}%
\pgfusepath{clip}%
\pgfsetbuttcap%
\pgfsetroundjoin%
\definecolor{currentfill}{rgb}{0.800000,0.200000,0.200000}%
\pgfsetfillcolor{currentfill}%
\pgfsetlinewidth{1.003750pt}%
\definecolor{currentstroke}{rgb}{0.800000,0.200000,0.200000}%
\pgfsetstrokecolor{currentstroke}%
\pgfsetdash{}{0pt}%
\pgfpathmoveto{\pgfqpoint{5.218212in}{6.083833in}}%
\pgfpathcurveto{\pgfqpoint{5.224036in}{6.083833in}}{\pgfqpoint{5.229622in}{6.086147in}}{\pgfqpoint{5.233740in}{6.090265in}}%
\pgfpathcurveto{\pgfqpoint{5.237858in}{6.094383in}}{\pgfqpoint{5.240172in}{6.099969in}}{\pgfqpoint{5.240172in}{6.105793in}}%
\pgfpathcurveto{\pgfqpoint{5.240172in}{6.111617in}}{\pgfqpoint{5.237858in}{6.117203in}}{\pgfqpoint{5.233740in}{6.121321in}}%
\pgfpathcurveto{\pgfqpoint{5.229622in}{6.125439in}}{\pgfqpoint{5.224036in}{6.127753in}}{\pgfqpoint{5.218212in}{6.127753in}}%
\pgfpathcurveto{\pgfqpoint{5.212388in}{6.127753in}}{\pgfqpoint{5.206802in}{6.125439in}}{\pgfqpoint{5.202684in}{6.121321in}}%
\pgfpathcurveto{\pgfqpoint{5.198566in}{6.117203in}}{\pgfqpoint{5.196252in}{6.111617in}}{\pgfqpoint{5.196252in}{6.105793in}}%
\pgfpathcurveto{\pgfqpoint{5.196252in}{6.099969in}}{\pgfqpoint{5.198566in}{6.094383in}}{\pgfqpoint{5.202684in}{6.090265in}}%
\pgfpathcurveto{\pgfqpoint{5.206802in}{6.086147in}}{\pgfqpoint{5.212388in}{6.083833in}}{\pgfqpoint{5.218212in}{6.083833in}}%
\pgfpathlineto{\pgfqpoint{5.218212in}{6.083833in}}%
\pgfpathclose%
\pgfusepath{stroke,fill}%
\end{pgfscope}%
\begin{pgfscope}%
\pgfpathrectangle{\pgfqpoint{1.582361in}{0.880000in}}{\pgfqpoint{5.035278in}{6.160000in}}%
\pgfusepath{clip}%
\pgfsetbuttcap%
\pgfsetroundjoin%
\definecolor{currentfill}{rgb}{0.800000,0.200000,0.200000}%
\pgfsetfillcolor{currentfill}%
\pgfsetlinewidth{1.003750pt}%
\definecolor{currentstroke}{rgb}{0.800000,0.200000,0.200000}%
\pgfsetstrokecolor{currentstroke}%
\pgfsetdash{}{0pt}%
\pgfpathmoveto{\pgfqpoint{5.193544in}{6.121149in}}%
\pgfpathcurveto{\pgfqpoint{5.199368in}{6.121149in}}{\pgfqpoint{5.204954in}{6.123462in}}{\pgfqpoint{5.209072in}{6.127581in}}%
\pgfpathcurveto{\pgfqpoint{5.213190in}{6.131699in}}{\pgfqpoint{5.215504in}{6.137285in}}{\pgfqpoint{5.215504in}{6.143109in}}%
\pgfpathcurveto{\pgfqpoint{5.215504in}{6.148933in}}{\pgfqpoint{5.213190in}{6.154519in}}{\pgfqpoint{5.209072in}{6.158637in}}%
\pgfpathcurveto{\pgfqpoint{5.204954in}{6.162755in}}{\pgfqpoint{5.199368in}{6.165069in}}{\pgfqpoint{5.193544in}{6.165069in}}%
\pgfpathcurveto{\pgfqpoint{5.187720in}{6.165069in}}{\pgfqpoint{5.182134in}{6.162755in}}{\pgfqpoint{5.178015in}{6.158637in}}%
\pgfpathcurveto{\pgfqpoint{5.173897in}{6.154519in}}{\pgfqpoint{5.171583in}{6.148933in}}{\pgfqpoint{5.171583in}{6.143109in}}%
\pgfpathcurveto{\pgfqpoint{5.171583in}{6.137285in}}{\pgfqpoint{5.173897in}{6.131699in}}{\pgfqpoint{5.178015in}{6.127581in}}%
\pgfpathcurveto{\pgfqpoint{5.182134in}{6.123462in}}{\pgfqpoint{5.187720in}{6.121149in}}{\pgfqpoint{5.193544in}{6.121149in}}%
\pgfpathlineto{\pgfqpoint{5.193544in}{6.121149in}}%
\pgfpathclose%
\pgfusepath{stroke,fill}%
\end{pgfscope}%
\begin{pgfscope}%
\pgfpathrectangle{\pgfqpoint{1.582361in}{0.880000in}}{\pgfqpoint{5.035278in}{6.160000in}}%
\pgfusepath{clip}%
\pgfsetbuttcap%
\pgfsetroundjoin%
\definecolor{currentfill}{rgb}{0.800000,0.200000,0.200000}%
\pgfsetfillcolor{currentfill}%
\pgfsetlinewidth{1.003750pt}%
\definecolor{currentstroke}{rgb}{0.800000,0.200000,0.200000}%
\pgfsetstrokecolor{currentstroke}%
\pgfsetdash{}{0pt}%
\pgfpathmoveto{\pgfqpoint{5.167710in}{6.157667in}}%
\pgfpathcurveto{\pgfqpoint{5.173534in}{6.157667in}}{\pgfqpoint{5.179120in}{6.159981in}}{\pgfqpoint{5.183238in}{6.164099in}}%
\pgfpathcurveto{\pgfqpoint{5.187356in}{6.168217in}}{\pgfqpoint{5.189670in}{6.173803in}}{\pgfqpoint{5.189670in}{6.179627in}}%
\pgfpathcurveto{\pgfqpoint{5.189670in}{6.185451in}}{\pgfqpoint{5.187356in}{6.191037in}}{\pgfqpoint{5.183238in}{6.195155in}}%
\pgfpathcurveto{\pgfqpoint{5.179120in}{6.199274in}}{\pgfqpoint{5.173534in}{6.201587in}}{\pgfqpoint{5.167710in}{6.201587in}}%
\pgfpathcurveto{\pgfqpoint{5.161886in}{6.201587in}}{\pgfqpoint{5.156300in}{6.199274in}}{\pgfqpoint{5.152182in}{6.195155in}}%
\pgfpathcurveto{\pgfqpoint{5.148063in}{6.191037in}}{\pgfqpoint{5.145750in}{6.185451in}}{\pgfqpoint{5.145750in}{6.179627in}}%
\pgfpathcurveto{\pgfqpoint{5.145750in}{6.173803in}}{\pgfqpoint{5.148063in}{6.168217in}}{\pgfqpoint{5.152182in}{6.164099in}}%
\pgfpathcurveto{\pgfqpoint{5.156300in}{6.159981in}}{\pgfqpoint{5.161886in}{6.157667in}}{\pgfqpoint{5.167710in}{6.157667in}}%
\pgfpathlineto{\pgfqpoint{5.167710in}{6.157667in}}%
\pgfpathclose%
\pgfusepath{stroke,fill}%
\end{pgfscope}%
\begin{pgfscope}%
\pgfpathrectangle{\pgfqpoint{1.582361in}{0.880000in}}{\pgfqpoint{5.035278in}{6.160000in}}%
\pgfusepath{clip}%
\pgfsetbuttcap%
\pgfsetroundjoin%
\definecolor{currentfill}{rgb}{0.800000,0.200000,0.200000}%
\pgfsetfillcolor{currentfill}%
\pgfsetlinewidth{1.003750pt}%
\definecolor{currentstroke}{rgb}{0.800000,0.200000,0.200000}%
\pgfsetstrokecolor{currentstroke}%
\pgfsetdash{}{0pt}%
\pgfpathmoveto{\pgfqpoint{5.140736in}{6.193352in}}%
\pgfpathcurveto{\pgfqpoint{5.146560in}{6.193352in}}{\pgfqpoint{5.152146in}{6.195666in}}{\pgfqpoint{5.156264in}{6.199784in}}%
\pgfpathcurveto{\pgfqpoint{5.160382in}{6.203902in}}{\pgfqpoint{5.162696in}{6.209488in}}{\pgfqpoint{5.162696in}{6.215312in}}%
\pgfpathcurveto{\pgfqpoint{5.162696in}{6.221136in}}{\pgfqpoint{5.160382in}{6.226722in}}{\pgfqpoint{5.156264in}{6.230840in}}%
\pgfpathcurveto{\pgfqpoint{5.152146in}{6.234958in}}{\pgfqpoint{5.146560in}{6.237272in}}{\pgfqpoint{5.140736in}{6.237272in}}%
\pgfpathcurveto{\pgfqpoint{5.134912in}{6.237272in}}{\pgfqpoint{5.129326in}{6.234958in}}{\pgfqpoint{5.125208in}{6.230840in}}%
\pgfpathcurveto{\pgfqpoint{5.121090in}{6.226722in}}{\pgfqpoint{5.118776in}{6.221136in}}{\pgfqpoint{5.118776in}{6.215312in}}%
\pgfpathcurveto{\pgfqpoint{5.118776in}{6.209488in}}{\pgfqpoint{5.121090in}{6.203902in}}{\pgfqpoint{5.125208in}{6.199784in}}%
\pgfpathcurveto{\pgfqpoint{5.129326in}{6.195666in}}{\pgfqpoint{5.134912in}{6.193352in}}{\pgfqpoint{5.140736in}{6.193352in}}%
\pgfpathlineto{\pgfqpoint{5.140736in}{6.193352in}}%
\pgfpathclose%
\pgfusepath{stroke,fill}%
\end{pgfscope}%
\begin{pgfscope}%
\pgfpathrectangle{\pgfqpoint{1.582361in}{0.880000in}}{\pgfqpoint{5.035278in}{6.160000in}}%
\pgfusepath{clip}%
\pgfsetbuttcap%
\pgfsetroundjoin%
\definecolor{currentfill}{rgb}{0.800000,0.200000,0.200000}%
\pgfsetfillcolor{currentfill}%
\pgfsetlinewidth{1.003750pt}%
\definecolor{currentstroke}{rgb}{0.800000,0.200000,0.200000}%
\pgfsetstrokecolor{currentstroke}%
\pgfsetdash{}{0pt}%
\pgfpathmoveto{\pgfqpoint{5.112649in}{6.228167in}}%
\pgfpathcurveto{\pgfqpoint{5.118473in}{6.228167in}}{\pgfqpoint{5.124059in}{6.230481in}}{\pgfqpoint{5.128177in}{6.234599in}}%
\pgfpathcurveto{\pgfqpoint{5.132296in}{6.238717in}}{\pgfqpoint{5.134609in}{6.244303in}}{\pgfqpoint{5.134609in}{6.250127in}}%
\pgfpathcurveto{\pgfqpoint{5.134609in}{6.255951in}}{\pgfqpoint{5.132296in}{6.261537in}}{\pgfqpoint{5.128177in}{6.265656in}}%
\pgfpathcurveto{\pgfqpoint{5.124059in}{6.269774in}}{\pgfqpoint{5.118473in}{6.272088in}}{\pgfqpoint{5.112649in}{6.272088in}}%
\pgfpathcurveto{\pgfqpoint{5.106825in}{6.272088in}}{\pgfqpoint{5.101239in}{6.269774in}}{\pgfqpoint{5.097121in}{6.265656in}}%
\pgfpathcurveto{\pgfqpoint{5.093003in}{6.261537in}}{\pgfqpoint{5.090689in}{6.255951in}}{\pgfqpoint{5.090689in}{6.250127in}}%
\pgfpathcurveto{\pgfqpoint{5.090689in}{6.244303in}}{\pgfqpoint{5.093003in}{6.238717in}}{\pgfqpoint{5.097121in}{6.234599in}}%
\pgfpathcurveto{\pgfqpoint{5.101239in}{6.230481in}}{\pgfqpoint{5.106825in}{6.228167in}}{\pgfqpoint{5.112649in}{6.228167in}}%
\pgfpathlineto{\pgfqpoint{5.112649in}{6.228167in}}%
\pgfpathclose%
\pgfusepath{stroke,fill}%
\end{pgfscope}%
\begin{pgfscope}%
\pgfpathrectangle{\pgfqpoint{1.582361in}{0.880000in}}{\pgfqpoint{5.035278in}{6.160000in}}%
\pgfusepath{clip}%
\pgfsetbuttcap%
\pgfsetroundjoin%
\definecolor{currentfill}{rgb}{0.800000,0.200000,0.200000}%
\pgfsetfillcolor{currentfill}%
\pgfsetlinewidth{1.003750pt}%
\definecolor{currentstroke}{rgb}{0.800000,0.200000,0.200000}%
\pgfsetstrokecolor{currentstroke}%
\pgfsetdash{}{0pt}%
\pgfpathmoveto{\pgfqpoint{5.083477in}{6.262078in}}%
\pgfpathcurveto{\pgfqpoint{5.089301in}{6.262078in}}{\pgfqpoint{5.094887in}{6.264392in}}{\pgfqpoint{5.099005in}{6.268510in}}%
\pgfpathcurveto{\pgfqpoint{5.103124in}{6.272628in}}{\pgfqpoint{5.105437in}{6.278215in}}{\pgfqpoint{5.105437in}{6.284039in}}%
\pgfpathcurveto{\pgfqpoint{5.105437in}{6.289863in}}{\pgfqpoint{5.103124in}{6.295449in}}{\pgfqpoint{5.099005in}{6.299567in}}%
\pgfpathcurveto{\pgfqpoint{5.094887in}{6.303685in}}{\pgfqpoint{5.089301in}{6.305999in}}{\pgfqpoint{5.083477in}{6.305999in}}%
\pgfpathcurveto{\pgfqpoint{5.077653in}{6.305999in}}{\pgfqpoint{5.072067in}{6.303685in}}{\pgfqpoint{5.067949in}{6.299567in}}%
\pgfpathcurveto{\pgfqpoint{5.063831in}{6.295449in}}{\pgfqpoint{5.061517in}{6.289863in}}{\pgfqpoint{5.061517in}{6.284039in}}%
\pgfpathcurveto{\pgfqpoint{5.061517in}{6.278215in}}{\pgfqpoint{5.063831in}{6.272628in}}{\pgfqpoint{5.067949in}{6.268510in}}%
\pgfpathcurveto{\pgfqpoint{5.072067in}{6.264392in}}{\pgfqpoint{5.077653in}{6.262078in}}{\pgfqpoint{5.083477in}{6.262078in}}%
\pgfpathlineto{\pgfqpoint{5.083477in}{6.262078in}}%
\pgfpathclose%
\pgfusepath{stroke,fill}%
\end{pgfscope}%
\begin{pgfscope}%
\pgfpathrectangle{\pgfqpoint{1.582361in}{0.880000in}}{\pgfqpoint{5.035278in}{6.160000in}}%
\pgfusepath{clip}%
\pgfsetbuttcap%
\pgfsetroundjoin%
\definecolor{currentfill}{rgb}{0.800000,0.200000,0.200000}%
\pgfsetfillcolor{currentfill}%
\pgfsetlinewidth{1.003750pt}%
\definecolor{currentstroke}{rgb}{0.800000,0.200000,0.200000}%
\pgfsetstrokecolor{currentstroke}%
\pgfsetdash{}{0pt}%
\pgfpathmoveto{\pgfqpoint{5.053249in}{6.295052in}}%
\pgfpathcurveto{\pgfqpoint{5.059073in}{6.295052in}}{\pgfqpoint{5.064659in}{6.297366in}}{\pgfqpoint{5.068777in}{6.301484in}}%
\pgfpathcurveto{\pgfqpoint{5.072896in}{6.305602in}}{\pgfqpoint{5.075209in}{6.311188in}}{\pgfqpoint{5.075209in}{6.317012in}}%
\pgfpathcurveto{\pgfqpoint{5.075209in}{6.322836in}}{\pgfqpoint{5.072896in}{6.328422in}}{\pgfqpoint{5.068777in}{6.332540in}}%
\pgfpathcurveto{\pgfqpoint{5.064659in}{6.336659in}}{\pgfqpoint{5.059073in}{6.338972in}}{\pgfqpoint{5.053249in}{6.338972in}}%
\pgfpathcurveto{\pgfqpoint{5.047425in}{6.338972in}}{\pgfqpoint{5.041839in}{6.336659in}}{\pgfqpoint{5.037721in}{6.332540in}}%
\pgfpathcurveto{\pgfqpoint{5.033603in}{6.328422in}}{\pgfqpoint{5.031289in}{6.322836in}}{\pgfqpoint{5.031289in}{6.317012in}}%
\pgfpathcurveto{\pgfqpoint{5.031289in}{6.311188in}}{\pgfqpoint{5.033603in}{6.305602in}}{\pgfqpoint{5.037721in}{6.301484in}}%
\pgfpathcurveto{\pgfqpoint{5.041839in}{6.297366in}}{\pgfqpoint{5.047425in}{6.295052in}}{\pgfqpoint{5.053249in}{6.295052in}}%
\pgfpathlineto{\pgfqpoint{5.053249in}{6.295052in}}%
\pgfpathclose%
\pgfusepath{stroke,fill}%
\end{pgfscope}%
\begin{pgfscope}%
\pgfpathrectangle{\pgfqpoint{1.582361in}{0.880000in}}{\pgfqpoint{5.035278in}{6.160000in}}%
\pgfusepath{clip}%
\pgfsetbuttcap%
\pgfsetroundjoin%
\definecolor{currentfill}{rgb}{0.800000,0.200000,0.200000}%
\pgfsetfillcolor{currentfill}%
\pgfsetlinewidth{1.003750pt}%
\definecolor{currentstroke}{rgb}{0.800000,0.200000,0.200000}%
\pgfsetstrokecolor{currentstroke}%
\pgfsetdash{}{0pt}%
\pgfpathmoveto{\pgfqpoint{5.021995in}{6.327055in}}%
\pgfpathcurveto{\pgfqpoint{5.027819in}{6.327055in}}{\pgfqpoint{5.033405in}{6.329369in}}{\pgfqpoint{5.037524in}{6.333487in}}%
\pgfpathcurveto{\pgfqpoint{5.041642in}{6.337605in}}{\pgfqpoint{5.043956in}{6.343191in}}{\pgfqpoint{5.043956in}{6.349015in}}%
\pgfpathcurveto{\pgfqpoint{5.043956in}{6.354839in}}{\pgfqpoint{5.041642in}{6.360425in}}{\pgfqpoint{5.037524in}{6.364543in}}%
\pgfpathcurveto{\pgfqpoint{5.033405in}{6.368661in}}{\pgfqpoint{5.027819in}{6.370975in}}{\pgfqpoint{5.021995in}{6.370975in}}%
\pgfpathcurveto{\pgfqpoint{5.016171in}{6.370975in}}{\pgfqpoint{5.010585in}{6.368661in}}{\pgfqpoint{5.006467in}{6.364543in}}%
\pgfpathcurveto{\pgfqpoint{5.002349in}{6.360425in}}{\pgfqpoint{5.000035in}{6.354839in}}{\pgfqpoint{5.000035in}{6.349015in}}%
\pgfpathcurveto{\pgfqpoint{5.000035in}{6.343191in}}{\pgfqpoint{5.002349in}{6.337605in}}{\pgfqpoint{5.006467in}{6.333487in}}%
\pgfpathcurveto{\pgfqpoint{5.010585in}{6.329369in}}{\pgfqpoint{5.016171in}{6.327055in}}{\pgfqpoint{5.021995in}{6.327055in}}%
\pgfpathlineto{\pgfqpoint{5.021995in}{6.327055in}}%
\pgfpathclose%
\pgfusepath{stroke,fill}%
\end{pgfscope}%
\begin{pgfscope}%
\pgfpathrectangle{\pgfqpoint{1.582361in}{0.880000in}}{\pgfqpoint{5.035278in}{6.160000in}}%
\pgfusepath{clip}%
\pgfsetbuttcap%
\pgfsetroundjoin%
\definecolor{currentfill}{rgb}{0.800000,0.200000,0.200000}%
\pgfsetfillcolor{currentfill}%
\pgfsetlinewidth{1.003750pt}%
\definecolor{currentstroke}{rgb}{0.800000,0.200000,0.200000}%
\pgfsetstrokecolor{currentstroke}%
\pgfsetdash{}{0pt}%
\pgfpathmoveto{\pgfqpoint{4.989747in}{6.358055in}}%
\pgfpathcurveto{\pgfqpoint{4.995571in}{6.358055in}}{\pgfqpoint{5.001157in}{6.360369in}}{\pgfqpoint{5.005275in}{6.364487in}}%
\pgfpathcurveto{\pgfqpoint{5.009393in}{6.368605in}}{\pgfqpoint{5.011707in}{6.374191in}}{\pgfqpoint{5.011707in}{6.380015in}}%
\pgfpathcurveto{\pgfqpoint{5.011707in}{6.385839in}}{\pgfqpoint{5.009393in}{6.391425in}}{\pgfqpoint{5.005275in}{6.395543in}}%
\pgfpathcurveto{\pgfqpoint{5.001157in}{6.399662in}}{\pgfqpoint{4.995571in}{6.401975in}}{\pgfqpoint{4.989747in}{6.401975in}}%
\pgfpathcurveto{\pgfqpoint{4.983923in}{6.401975in}}{\pgfqpoint{4.978337in}{6.399662in}}{\pgfqpoint{4.974219in}{6.395543in}}%
\pgfpathcurveto{\pgfqpoint{4.970100in}{6.391425in}}{\pgfqpoint{4.967787in}{6.385839in}}{\pgfqpoint{4.967787in}{6.380015in}}%
\pgfpathcurveto{\pgfqpoint{4.967787in}{6.374191in}}{\pgfqpoint{4.970100in}{6.368605in}}{\pgfqpoint{4.974219in}{6.364487in}}%
\pgfpathcurveto{\pgfqpoint{4.978337in}{6.360369in}}{\pgfqpoint{4.983923in}{6.358055in}}{\pgfqpoint{4.989747in}{6.358055in}}%
\pgfpathlineto{\pgfqpoint{4.989747in}{6.358055in}}%
\pgfpathclose%
\pgfusepath{stroke,fill}%
\end{pgfscope}%
\begin{pgfscope}%
\pgfpathrectangle{\pgfqpoint{1.582361in}{0.880000in}}{\pgfqpoint{5.035278in}{6.160000in}}%
\pgfusepath{clip}%
\pgfsetbuttcap%
\pgfsetroundjoin%
\definecolor{currentfill}{rgb}{0.800000,0.200000,0.200000}%
\pgfsetfillcolor{currentfill}%
\pgfsetlinewidth{1.003750pt}%
\definecolor{currentstroke}{rgb}{0.800000,0.200000,0.200000}%
\pgfsetstrokecolor{currentstroke}%
\pgfsetdash{}{0pt}%
\pgfpathmoveto{\pgfqpoint{4.956536in}{6.388022in}}%
\pgfpathcurveto{\pgfqpoint{4.962360in}{6.388022in}}{\pgfqpoint{4.967946in}{6.390336in}}{\pgfqpoint{4.972064in}{6.394454in}}%
\pgfpathcurveto{\pgfqpoint{4.976182in}{6.398572in}}{\pgfqpoint{4.978496in}{6.404158in}}{\pgfqpoint{4.978496in}{6.409982in}}%
\pgfpathcurveto{\pgfqpoint{4.978496in}{6.415806in}}{\pgfqpoint{4.976182in}{6.421392in}}{\pgfqpoint{4.972064in}{6.425510in}}%
\pgfpathcurveto{\pgfqpoint{4.967946in}{6.429628in}}{\pgfqpoint{4.962360in}{6.431942in}}{\pgfqpoint{4.956536in}{6.431942in}}%
\pgfpathcurveto{\pgfqpoint{4.950712in}{6.431942in}}{\pgfqpoint{4.945126in}{6.429628in}}{\pgfqpoint{4.941007in}{6.425510in}}%
\pgfpathcurveto{\pgfqpoint{4.936889in}{6.421392in}}{\pgfqpoint{4.934575in}{6.415806in}}{\pgfqpoint{4.934575in}{6.409982in}}%
\pgfpathcurveto{\pgfqpoint{4.934575in}{6.404158in}}{\pgfqpoint{4.936889in}{6.398572in}}{\pgfqpoint{4.941007in}{6.394454in}}%
\pgfpathcurveto{\pgfqpoint{4.945126in}{6.390336in}}{\pgfqpoint{4.950712in}{6.388022in}}{\pgfqpoint{4.956536in}{6.388022in}}%
\pgfpathlineto{\pgfqpoint{4.956536in}{6.388022in}}%
\pgfpathclose%
\pgfusepath{stroke,fill}%
\end{pgfscope}%
\begin{pgfscope}%
\pgfpathrectangle{\pgfqpoint{1.582361in}{0.880000in}}{\pgfqpoint{5.035278in}{6.160000in}}%
\pgfusepath{clip}%
\pgfsetbuttcap%
\pgfsetroundjoin%
\definecolor{currentfill}{rgb}{0.800000,0.200000,0.200000}%
\pgfsetfillcolor{currentfill}%
\pgfsetlinewidth{1.003750pt}%
\definecolor{currentstroke}{rgb}{0.800000,0.200000,0.200000}%
\pgfsetstrokecolor{currentstroke}%
\pgfsetdash{}{0pt}%
\pgfpathmoveto{\pgfqpoint{4.922395in}{6.416925in}}%
\pgfpathcurveto{\pgfqpoint{4.928219in}{6.416925in}}{\pgfqpoint{4.933805in}{6.419239in}}{\pgfqpoint{4.937923in}{6.423357in}}%
\pgfpathcurveto{\pgfqpoint{4.942042in}{6.427475in}}{\pgfqpoint{4.944355in}{6.433061in}}{\pgfqpoint{4.944355in}{6.438885in}}%
\pgfpathcurveto{\pgfqpoint{4.944355in}{6.444709in}}{\pgfqpoint{4.942042in}{6.450295in}}{\pgfqpoint{4.937923in}{6.454414in}}%
\pgfpathcurveto{\pgfqpoint{4.933805in}{6.458532in}}{\pgfqpoint{4.928219in}{6.460846in}}{\pgfqpoint{4.922395in}{6.460846in}}%
\pgfpathcurveto{\pgfqpoint{4.916571in}{6.460846in}}{\pgfqpoint{4.910985in}{6.458532in}}{\pgfqpoint{4.906867in}{6.454414in}}%
\pgfpathcurveto{\pgfqpoint{4.902749in}{6.450295in}}{\pgfqpoint{4.900435in}{6.444709in}}{\pgfqpoint{4.900435in}{6.438885in}}%
\pgfpathcurveto{\pgfqpoint{4.900435in}{6.433061in}}{\pgfqpoint{4.902749in}{6.427475in}}{\pgfqpoint{4.906867in}{6.423357in}}%
\pgfpathcurveto{\pgfqpoint{4.910985in}{6.419239in}}{\pgfqpoint{4.916571in}{6.416925in}}{\pgfqpoint{4.922395in}{6.416925in}}%
\pgfpathlineto{\pgfqpoint{4.922395in}{6.416925in}}%
\pgfpathclose%
\pgfusepath{stroke,fill}%
\end{pgfscope}%
\begin{pgfscope}%
\pgfpathrectangle{\pgfqpoint{1.582361in}{0.880000in}}{\pgfqpoint{5.035278in}{6.160000in}}%
\pgfusepath{clip}%
\pgfsetbuttcap%
\pgfsetroundjoin%
\definecolor{currentfill}{rgb}{0.800000,0.200000,0.200000}%
\pgfsetfillcolor{currentfill}%
\pgfsetlinewidth{1.003750pt}%
\definecolor{currentstroke}{rgb}{0.800000,0.200000,0.200000}%
\pgfsetstrokecolor{currentstroke}%
\pgfsetdash{}{0pt}%
\pgfpathmoveto{\pgfqpoint{4.887359in}{6.444736in}}%
\pgfpathcurveto{\pgfqpoint{4.893183in}{6.444736in}}{\pgfqpoint{4.898769in}{6.447050in}}{\pgfqpoint{4.902887in}{6.451168in}}%
\pgfpathcurveto{\pgfqpoint{4.907006in}{6.455286in}}{\pgfqpoint{4.909319in}{6.460873in}}{\pgfqpoint{4.909319in}{6.466697in}}%
\pgfpathcurveto{\pgfqpoint{4.909319in}{6.472521in}}{\pgfqpoint{4.907006in}{6.478107in}}{\pgfqpoint{4.902887in}{6.482225in}}%
\pgfpathcurveto{\pgfqpoint{4.898769in}{6.486343in}}{\pgfqpoint{4.893183in}{6.488657in}}{\pgfqpoint{4.887359in}{6.488657in}}%
\pgfpathcurveto{\pgfqpoint{4.881535in}{6.488657in}}{\pgfqpoint{4.875949in}{6.486343in}}{\pgfqpoint{4.871831in}{6.482225in}}%
\pgfpathcurveto{\pgfqpoint{4.867713in}{6.478107in}}{\pgfqpoint{4.865399in}{6.472521in}}{\pgfqpoint{4.865399in}{6.466697in}}%
\pgfpathcurveto{\pgfqpoint{4.865399in}{6.460873in}}{\pgfqpoint{4.867713in}{6.455286in}}{\pgfqpoint{4.871831in}{6.451168in}}%
\pgfpathcurveto{\pgfqpoint{4.875949in}{6.447050in}}{\pgfqpoint{4.881535in}{6.444736in}}{\pgfqpoint{4.887359in}{6.444736in}}%
\pgfpathlineto{\pgfqpoint{4.887359in}{6.444736in}}%
\pgfpathclose%
\pgfusepath{stroke,fill}%
\end{pgfscope}%
\begin{pgfscope}%
\pgfpathrectangle{\pgfqpoint{1.582361in}{0.880000in}}{\pgfqpoint{5.035278in}{6.160000in}}%
\pgfusepath{clip}%
\pgfsetbuttcap%
\pgfsetroundjoin%
\definecolor{currentfill}{rgb}{0.800000,0.200000,0.200000}%
\pgfsetfillcolor{currentfill}%
\pgfsetlinewidth{1.003750pt}%
\definecolor{currentstroke}{rgb}{0.800000,0.200000,0.200000}%
\pgfsetstrokecolor{currentstroke}%
\pgfsetdash{}{0pt}%
\pgfpathmoveto{\pgfqpoint{4.851463in}{6.471428in}}%
\pgfpathcurveto{\pgfqpoint{4.857287in}{6.471428in}}{\pgfqpoint{4.862873in}{6.473742in}}{\pgfqpoint{4.866991in}{6.477860in}}%
\pgfpathcurveto{\pgfqpoint{4.871109in}{6.481978in}}{\pgfqpoint{4.873423in}{6.487564in}}{\pgfqpoint{4.873423in}{6.493388in}}%
\pgfpathcurveto{\pgfqpoint{4.873423in}{6.499212in}}{\pgfqpoint{4.871109in}{6.504798in}}{\pgfqpoint{4.866991in}{6.508916in}}%
\pgfpathcurveto{\pgfqpoint{4.862873in}{6.513034in}}{\pgfqpoint{4.857287in}{6.515348in}}{\pgfqpoint{4.851463in}{6.515348in}}%
\pgfpathcurveto{\pgfqpoint{4.845639in}{6.515348in}}{\pgfqpoint{4.840053in}{6.513034in}}{\pgfqpoint{4.835934in}{6.508916in}}%
\pgfpathcurveto{\pgfqpoint{4.831816in}{6.504798in}}{\pgfqpoint{4.829502in}{6.499212in}}{\pgfqpoint{4.829502in}{6.493388in}}%
\pgfpathcurveto{\pgfqpoint{4.829502in}{6.487564in}}{\pgfqpoint{4.831816in}{6.481978in}}{\pgfqpoint{4.835934in}{6.477860in}}%
\pgfpathcurveto{\pgfqpoint{4.840053in}{6.473742in}}{\pgfqpoint{4.845639in}{6.471428in}}{\pgfqpoint{4.851463in}{6.471428in}}%
\pgfpathlineto{\pgfqpoint{4.851463in}{6.471428in}}%
\pgfpathclose%
\pgfusepath{stroke,fill}%
\end{pgfscope}%
\begin{pgfscope}%
\pgfpathrectangle{\pgfqpoint{1.582361in}{0.880000in}}{\pgfqpoint{5.035278in}{6.160000in}}%
\pgfusepath{clip}%
\pgfsetbuttcap%
\pgfsetroundjoin%
\definecolor{currentfill}{rgb}{0.800000,0.200000,0.200000}%
\pgfsetfillcolor{currentfill}%
\pgfsetlinewidth{1.003750pt}%
\definecolor{currentstroke}{rgb}{0.800000,0.200000,0.200000}%
\pgfsetstrokecolor{currentstroke}%
\pgfsetdash{}{0pt}%
\pgfpathmoveto{\pgfqpoint{4.814742in}{6.496972in}}%
\pgfpathcurveto{\pgfqpoint{4.820565in}{6.496972in}}{\pgfqpoint{4.826152in}{6.499286in}}{\pgfqpoint{4.830270in}{6.503404in}}%
\pgfpathcurveto{\pgfqpoint{4.834388in}{6.507523in}}{\pgfqpoint{4.836702in}{6.513109in}}{\pgfqpoint{4.836702in}{6.518933in}}%
\pgfpathcurveto{\pgfqpoint{4.836702in}{6.524757in}}{\pgfqpoint{4.834388in}{6.530343in}}{\pgfqpoint{4.830270in}{6.534461in}}%
\pgfpathcurveto{\pgfqpoint{4.826152in}{6.538579in}}{\pgfqpoint{4.820565in}{6.540893in}}{\pgfqpoint{4.814742in}{6.540893in}}%
\pgfpathcurveto{\pgfqpoint{4.808918in}{6.540893in}}{\pgfqpoint{4.803331in}{6.538579in}}{\pgfqpoint{4.799213in}{6.534461in}}%
\pgfpathcurveto{\pgfqpoint{4.795095in}{6.530343in}}{\pgfqpoint{4.792781in}{6.524757in}}{\pgfqpoint{4.792781in}{6.518933in}}%
\pgfpathcurveto{\pgfqpoint{4.792781in}{6.513109in}}{\pgfqpoint{4.795095in}{6.507523in}}{\pgfqpoint{4.799213in}{6.503404in}}%
\pgfpathcurveto{\pgfqpoint{4.803331in}{6.499286in}}{\pgfqpoint{4.808918in}{6.496972in}}{\pgfqpoint{4.814742in}{6.496972in}}%
\pgfpathlineto{\pgfqpoint{4.814742in}{6.496972in}}%
\pgfpathclose%
\pgfusepath{stroke,fill}%
\end{pgfscope}%
\begin{pgfscope}%
\pgfpathrectangle{\pgfqpoint{1.582361in}{0.880000in}}{\pgfqpoint{5.035278in}{6.160000in}}%
\pgfusepath{clip}%
\pgfsetbuttcap%
\pgfsetroundjoin%
\definecolor{currentfill}{rgb}{0.800000,0.200000,0.200000}%
\pgfsetfillcolor{currentfill}%
\pgfsetlinewidth{1.003750pt}%
\definecolor{currentstroke}{rgb}{0.800000,0.200000,0.200000}%
\pgfsetstrokecolor{currentstroke}%
\pgfsetdash{}{0pt}%
\pgfpathmoveto{\pgfqpoint{4.777232in}{6.521345in}}%
\pgfpathcurveto{\pgfqpoint{4.783056in}{6.521345in}}{\pgfqpoint{4.788642in}{6.523659in}}{\pgfqpoint{4.792760in}{6.527777in}}%
\pgfpathcurveto{\pgfqpoint{4.796879in}{6.531895in}}{\pgfqpoint{4.799192in}{6.537482in}}{\pgfqpoint{4.799192in}{6.543306in}}%
\pgfpathcurveto{\pgfqpoint{4.799192in}{6.549130in}}{\pgfqpoint{4.796879in}{6.554716in}}{\pgfqpoint{4.792760in}{6.558834in}}%
\pgfpathcurveto{\pgfqpoint{4.788642in}{6.562952in}}{\pgfqpoint{4.783056in}{6.565266in}}{\pgfqpoint{4.777232in}{6.565266in}}%
\pgfpathcurveto{\pgfqpoint{4.771408in}{6.565266in}}{\pgfqpoint{4.765822in}{6.562952in}}{\pgfqpoint{4.761704in}{6.558834in}}%
\pgfpathcurveto{\pgfqpoint{4.757586in}{6.554716in}}{\pgfqpoint{4.755272in}{6.549130in}}{\pgfqpoint{4.755272in}{6.543306in}}%
\pgfpathcurveto{\pgfqpoint{4.755272in}{6.537482in}}{\pgfqpoint{4.757586in}{6.531895in}}{\pgfqpoint{4.761704in}{6.527777in}}%
\pgfpathcurveto{\pgfqpoint{4.765822in}{6.523659in}}{\pgfqpoint{4.771408in}{6.521345in}}{\pgfqpoint{4.777232in}{6.521345in}}%
\pgfpathlineto{\pgfqpoint{4.777232in}{6.521345in}}%
\pgfpathclose%
\pgfusepath{stroke,fill}%
\end{pgfscope}%
\begin{pgfscope}%
\pgfpathrectangle{\pgfqpoint{1.582361in}{0.880000in}}{\pgfqpoint{5.035278in}{6.160000in}}%
\pgfusepath{clip}%
\pgfsetbuttcap%
\pgfsetroundjoin%
\definecolor{currentfill}{rgb}{0.800000,0.200000,0.200000}%
\pgfsetfillcolor{currentfill}%
\pgfsetlinewidth{1.003750pt}%
\definecolor{currentstroke}{rgb}{0.800000,0.200000,0.200000}%
\pgfsetstrokecolor{currentstroke}%
\pgfsetdash{}{0pt}%
\pgfpathmoveto{\pgfqpoint{4.738972in}{6.544522in}}%
\pgfpathcurveto{\pgfqpoint{4.744796in}{6.544522in}}{\pgfqpoint{4.750382in}{6.546836in}}{\pgfqpoint{4.754500in}{6.550954in}}%
\pgfpathcurveto{\pgfqpoint{4.758619in}{6.555072in}}{\pgfqpoint{4.760932in}{6.560658in}}{\pgfqpoint{4.760932in}{6.566482in}}%
\pgfpathcurveto{\pgfqpoint{4.760932in}{6.572306in}}{\pgfqpoint{4.758619in}{6.577892in}}{\pgfqpoint{4.754500in}{6.582010in}}%
\pgfpathcurveto{\pgfqpoint{4.750382in}{6.586129in}}{\pgfqpoint{4.744796in}{6.588442in}}{\pgfqpoint{4.738972in}{6.588442in}}%
\pgfpathcurveto{\pgfqpoint{4.733148in}{6.588442in}}{\pgfqpoint{4.727562in}{6.586129in}}{\pgfqpoint{4.723444in}{6.582010in}}%
\pgfpathcurveto{\pgfqpoint{4.719326in}{6.577892in}}{\pgfqpoint{4.717012in}{6.572306in}}{\pgfqpoint{4.717012in}{6.566482in}}%
\pgfpathcurveto{\pgfqpoint{4.717012in}{6.560658in}}{\pgfqpoint{4.719326in}{6.555072in}}{\pgfqpoint{4.723444in}{6.550954in}}%
\pgfpathcurveto{\pgfqpoint{4.727562in}{6.546836in}}{\pgfqpoint{4.733148in}{6.544522in}}{\pgfqpoint{4.738972in}{6.544522in}}%
\pgfpathlineto{\pgfqpoint{4.738972in}{6.544522in}}%
\pgfpathclose%
\pgfusepath{stroke,fill}%
\end{pgfscope}%
\begin{pgfscope}%
\pgfpathrectangle{\pgfqpoint{1.582361in}{0.880000in}}{\pgfqpoint{5.035278in}{6.160000in}}%
\pgfusepath{clip}%
\pgfsetbuttcap%
\pgfsetroundjoin%
\definecolor{currentfill}{rgb}{0.800000,0.200000,0.200000}%
\pgfsetfillcolor{currentfill}%
\pgfsetlinewidth{1.003750pt}%
\definecolor{currentstroke}{rgb}{0.800000,0.200000,0.200000}%
\pgfsetstrokecolor{currentstroke}%
\pgfsetdash{}{0pt}%
\pgfpathmoveto{\pgfqpoint{4.700000in}{6.566479in}}%
\pgfpathcurveto{\pgfqpoint{4.705824in}{6.566479in}}{\pgfqpoint{4.711410in}{6.568793in}}{\pgfqpoint{4.715528in}{6.572911in}}%
\pgfpathcurveto{\pgfqpoint{4.719646in}{6.577029in}}{\pgfqpoint{4.721960in}{6.582615in}}{\pgfqpoint{4.721960in}{6.588439in}}%
\pgfpathcurveto{\pgfqpoint{4.721960in}{6.594263in}}{\pgfqpoint{4.719646in}{6.599850in}}{\pgfqpoint{4.715528in}{6.603968in}}%
\pgfpathcurveto{\pgfqpoint{4.711410in}{6.608086in}}{\pgfqpoint{4.705824in}{6.610400in}}{\pgfqpoint{4.700000in}{6.610400in}}%
\pgfpathcurveto{\pgfqpoint{4.694176in}{6.610400in}}{\pgfqpoint{4.688590in}{6.608086in}}{\pgfqpoint{4.684471in}{6.603968in}}%
\pgfpathcurveto{\pgfqpoint{4.680353in}{6.599850in}}{\pgfqpoint{4.678039in}{6.594263in}}{\pgfqpoint{4.678039in}{6.588439in}}%
\pgfpathcurveto{\pgfqpoint{4.678039in}{6.582615in}}{\pgfqpoint{4.680353in}{6.577029in}}{\pgfqpoint{4.684471in}{6.572911in}}%
\pgfpathcurveto{\pgfqpoint{4.688590in}{6.568793in}}{\pgfqpoint{4.694176in}{6.566479in}}{\pgfqpoint{4.700000in}{6.566479in}}%
\pgfpathlineto{\pgfqpoint{4.700000in}{6.566479in}}%
\pgfpathclose%
\pgfusepath{stroke,fill}%
\end{pgfscope}%
\begin{pgfscope}%
\pgfpathrectangle{\pgfqpoint{1.582361in}{0.880000in}}{\pgfqpoint{5.035278in}{6.160000in}}%
\pgfusepath{clip}%
\pgfsetbuttcap%
\pgfsetroundjoin%
\definecolor{currentfill}{rgb}{0.800000,0.200000,0.200000}%
\pgfsetfillcolor{currentfill}%
\pgfsetlinewidth{1.003750pt}%
\definecolor{currentstroke}{rgb}{0.800000,0.200000,0.200000}%
\pgfsetstrokecolor{currentstroke}%
\pgfsetdash{}{0pt}%
\pgfpathmoveto{\pgfqpoint{4.660353in}{6.587195in}}%
\pgfpathcurveto{\pgfqpoint{4.666177in}{6.587195in}}{\pgfqpoint{4.671763in}{6.589509in}}{\pgfqpoint{4.675882in}{6.593627in}}%
\pgfpathcurveto{\pgfqpoint{4.680000in}{6.597745in}}{\pgfqpoint{4.682314in}{6.603331in}}{\pgfqpoint{4.682314in}{6.609155in}}%
\pgfpathcurveto{\pgfqpoint{4.682314in}{6.614979in}}{\pgfqpoint{4.680000in}{6.620565in}}{\pgfqpoint{4.675882in}{6.624684in}}%
\pgfpathcurveto{\pgfqpoint{4.671763in}{6.628802in}}{\pgfqpoint{4.666177in}{6.631116in}}{\pgfqpoint{4.660353in}{6.631116in}}%
\pgfpathcurveto{\pgfqpoint{4.654529in}{6.631116in}}{\pgfqpoint{4.648943in}{6.628802in}}{\pgfqpoint{4.644825in}{6.624684in}}%
\pgfpathcurveto{\pgfqpoint{4.640707in}{6.620565in}}{\pgfqpoint{4.638393in}{6.614979in}}{\pgfqpoint{4.638393in}{6.609155in}}%
\pgfpathcurveto{\pgfqpoint{4.638393in}{6.603331in}}{\pgfqpoint{4.640707in}{6.597745in}}{\pgfqpoint{4.644825in}{6.593627in}}%
\pgfpathcurveto{\pgfqpoint{4.648943in}{6.589509in}}{\pgfqpoint{4.654529in}{6.587195in}}{\pgfqpoint{4.660353in}{6.587195in}}%
\pgfpathlineto{\pgfqpoint{4.660353in}{6.587195in}}%
\pgfpathclose%
\pgfusepath{stroke,fill}%
\end{pgfscope}%
\begin{pgfscope}%
\pgfpathrectangle{\pgfqpoint{1.582361in}{0.880000in}}{\pgfqpoint{5.035278in}{6.160000in}}%
\pgfusepath{clip}%
\pgfsetbuttcap%
\pgfsetroundjoin%
\definecolor{currentfill}{rgb}{0.800000,0.200000,0.200000}%
\pgfsetfillcolor{currentfill}%
\pgfsetlinewidth{1.003750pt}%
\definecolor{currentstroke}{rgb}{0.800000,0.200000,0.200000}%
\pgfsetstrokecolor{currentstroke}%
\pgfsetdash{}{0pt}%
\pgfpathmoveto{\pgfqpoint{4.620073in}{6.606649in}}%
\pgfpathcurveto{\pgfqpoint{4.625897in}{6.606649in}}{\pgfqpoint{4.631483in}{6.608963in}}{\pgfqpoint{4.635601in}{6.613081in}}%
\pgfpathcurveto{\pgfqpoint{4.639719in}{6.617199in}}{\pgfqpoint{4.642033in}{6.622786in}}{\pgfqpoint{4.642033in}{6.628609in}}%
\pgfpathcurveto{\pgfqpoint{4.642033in}{6.634433in}}{\pgfqpoint{4.639719in}{6.640020in}}{\pgfqpoint{4.635601in}{6.644138in}}%
\pgfpathcurveto{\pgfqpoint{4.631483in}{6.648256in}}{\pgfqpoint{4.625897in}{6.650570in}}{\pgfqpoint{4.620073in}{6.650570in}}%
\pgfpathcurveto{\pgfqpoint{4.614249in}{6.650570in}}{\pgfqpoint{4.608663in}{6.648256in}}{\pgfqpoint{4.604544in}{6.644138in}}%
\pgfpathcurveto{\pgfqpoint{4.600426in}{6.640020in}}{\pgfqpoint{4.598112in}{6.634433in}}{\pgfqpoint{4.598112in}{6.628609in}}%
\pgfpathcurveto{\pgfqpoint{4.598112in}{6.622786in}}{\pgfqpoint{4.600426in}{6.617199in}}{\pgfqpoint{4.604544in}{6.613081in}}%
\pgfpathcurveto{\pgfqpoint{4.608663in}{6.608963in}}{\pgfqpoint{4.614249in}{6.606649in}}{\pgfqpoint{4.620073in}{6.606649in}}%
\pgfpathlineto{\pgfqpoint{4.620073in}{6.606649in}}%
\pgfpathclose%
\pgfusepath{stroke,fill}%
\end{pgfscope}%
\begin{pgfscope}%
\pgfpathrectangle{\pgfqpoint{1.582361in}{0.880000in}}{\pgfqpoint{5.035278in}{6.160000in}}%
\pgfusepath{clip}%
\pgfsetbuttcap%
\pgfsetroundjoin%
\definecolor{currentfill}{rgb}{0.800000,0.200000,0.200000}%
\pgfsetfillcolor{currentfill}%
\pgfsetlinewidth{1.003750pt}%
\definecolor{currentstroke}{rgb}{0.800000,0.200000,0.200000}%
\pgfsetstrokecolor{currentstroke}%
\pgfsetdash{}{0pt}%
\pgfpathmoveto{\pgfqpoint{4.579198in}{6.624822in}}%
\pgfpathcurveto{\pgfqpoint{4.585022in}{6.624822in}}{\pgfqpoint{4.590608in}{6.627136in}}{\pgfqpoint{4.594726in}{6.631254in}}%
\pgfpathcurveto{\pgfqpoint{4.598845in}{6.635372in}}{\pgfqpoint{4.601158in}{6.640958in}}{\pgfqpoint{4.601158in}{6.646782in}}%
\pgfpathcurveto{\pgfqpoint{4.601158in}{6.652606in}}{\pgfqpoint{4.598845in}{6.658192in}}{\pgfqpoint{4.594726in}{6.662310in}}%
\pgfpathcurveto{\pgfqpoint{4.590608in}{6.666429in}}{\pgfqpoint{4.585022in}{6.668742in}}{\pgfqpoint{4.579198in}{6.668742in}}%
\pgfpathcurveto{\pgfqpoint{4.573374in}{6.668742in}}{\pgfqpoint{4.567788in}{6.666429in}}{\pgfqpoint{4.563670in}{6.662310in}}%
\pgfpathcurveto{\pgfqpoint{4.559552in}{6.658192in}}{\pgfqpoint{4.557238in}{6.652606in}}{\pgfqpoint{4.557238in}{6.646782in}}%
\pgfpathcurveto{\pgfqpoint{4.557238in}{6.640958in}}{\pgfqpoint{4.559552in}{6.635372in}}{\pgfqpoint{4.563670in}{6.631254in}}%
\pgfpathcurveto{\pgfqpoint{4.567788in}{6.627136in}}{\pgfqpoint{4.573374in}{6.624822in}}{\pgfqpoint{4.579198in}{6.624822in}}%
\pgfpathlineto{\pgfqpoint{4.579198in}{6.624822in}}%
\pgfpathclose%
\pgfusepath{stroke,fill}%
\end{pgfscope}%
\begin{pgfscope}%
\pgfpathrectangle{\pgfqpoint{1.582361in}{0.880000in}}{\pgfqpoint{5.035278in}{6.160000in}}%
\pgfusepath{clip}%
\pgfsetbuttcap%
\pgfsetroundjoin%
\definecolor{currentfill}{rgb}{0.800000,0.200000,0.200000}%
\pgfsetfillcolor{currentfill}%
\pgfsetlinewidth{1.003750pt}%
\definecolor{currentstroke}{rgb}{0.800000,0.200000,0.200000}%
\pgfsetstrokecolor{currentstroke}%
\pgfsetdash{}{0pt}%
\pgfpathmoveto{\pgfqpoint{4.537770in}{6.641695in}}%
\pgfpathcurveto{\pgfqpoint{4.543594in}{6.641695in}}{\pgfqpoint{4.549180in}{6.644009in}}{\pgfqpoint{4.553298in}{6.648127in}}%
\pgfpathcurveto{\pgfqpoint{4.557417in}{6.652245in}}{\pgfqpoint{4.559730in}{6.657832in}}{\pgfqpoint{4.559730in}{6.663656in}}%
\pgfpathcurveto{\pgfqpoint{4.559730in}{6.669480in}}{\pgfqpoint{4.557417in}{6.675066in}}{\pgfqpoint{4.553298in}{6.679184in}}%
\pgfpathcurveto{\pgfqpoint{4.549180in}{6.683302in}}{\pgfqpoint{4.543594in}{6.685616in}}{\pgfqpoint{4.537770in}{6.685616in}}%
\pgfpathcurveto{\pgfqpoint{4.531946in}{6.685616in}}{\pgfqpoint{4.526360in}{6.683302in}}{\pgfqpoint{4.522242in}{6.679184in}}%
\pgfpathcurveto{\pgfqpoint{4.518124in}{6.675066in}}{\pgfqpoint{4.515810in}{6.669480in}}{\pgfqpoint{4.515810in}{6.663656in}}%
\pgfpathcurveto{\pgfqpoint{4.515810in}{6.657832in}}{\pgfqpoint{4.518124in}{6.652245in}}{\pgfqpoint{4.522242in}{6.648127in}}%
\pgfpathcurveto{\pgfqpoint{4.526360in}{6.644009in}}{\pgfqpoint{4.531946in}{6.641695in}}{\pgfqpoint{4.537770in}{6.641695in}}%
\pgfpathlineto{\pgfqpoint{4.537770in}{6.641695in}}%
\pgfpathclose%
\pgfusepath{stroke,fill}%
\end{pgfscope}%
\begin{pgfscope}%
\pgfpathrectangle{\pgfqpoint{1.582361in}{0.880000in}}{\pgfqpoint{5.035278in}{6.160000in}}%
\pgfusepath{clip}%
\pgfsetbuttcap%
\pgfsetroundjoin%
\definecolor{currentfill}{rgb}{0.800000,0.200000,0.200000}%
\pgfsetfillcolor{currentfill}%
\pgfsetlinewidth{1.003750pt}%
\definecolor{currentstroke}{rgb}{0.800000,0.200000,0.200000}%
\pgfsetstrokecolor{currentstroke}%
\pgfsetdash{}{0pt}%
\pgfpathmoveto{\pgfqpoint{4.495830in}{6.657252in}}%
\pgfpathcurveto{\pgfqpoint{4.501654in}{6.657252in}}{\pgfqpoint{4.507240in}{6.659566in}}{\pgfqpoint{4.511359in}{6.663684in}}%
\pgfpathcurveto{\pgfqpoint{4.515477in}{6.667803in}}{\pgfqpoint{4.517791in}{6.673389in}}{\pgfqpoint{4.517791in}{6.679213in}}%
\pgfpathcurveto{\pgfqpoint{4.517791in}{6.685037in}}{\pgfqpoint{4.515477in}{6.690623in}}{\pgfqpoint{4.511359in}{6.694741in}}%
\pgfpathcurveto{\pgfqpoint{4.507240in}{6.698859in}}{\pgfqpoint{4.501654in}{6.701173in}}{\pgfqpoint{4.495830in}{6.701173in}}%
\pgfpathcurveto{\pgfqpoint{4.490006in}{6.701173in}}{\pgfqpoint{4.484420in}{6.698859in}}{\pgfqpoint{4.480302in}{6.694741in}}%
\pgfpathcurveto{\pgfqpoint{4.476184in}{6.690623in}}{\pgfqpoint{4.473870in}{6.685037in}}{\pgfqpoint{4.473870in}{6.679213in}}%
\pgfpathcurveto{\pgfqpoint{4.473870in}{6.673389in}}{\pgfqpoint{4.476184in}{6.667803in}}{\pgfqpoint{4.480302in}{6.663684in}}%
\pgfpathcurveto{\pgfqpoint{4.484420in}{6.659566in}}{\pgfqpoint{4.490006in}{6.657252in}}{\pgfqpoint{4.495830in}{6.657252in}}%
\pgfpathlineto{\pgfqpoint{4.495830in}{6.657252in}}%
\pgfpathclose%
\pgfusepath{stroke,fill}%
\end{pgfscope}%
\begin{pgfscope}%
\pgfpathrectangle{\pgfqpoint{1.582361in}{0.880000in}}{\pgfqpoint{5.035278in}{6.160000in}}%
\pgfusepath{clip}%
\pgfsetbuttcap%
\pgfsetroundjoin%
\definecolor{currentfill}{rgb}{0.800000,0.200000,0.200000}%
\pgfsetfillcolor{currentfill}%
\pgfsetlinewidth{1.003750pt}%
\definecolor{currentstroke}{rgb}{0.800000,0.200000,0.200000}%
\pgfsetstrokecolor{currentstroke}%
\pgfsetdash{}{0pt}%
\pgfpathmoveto{\pgfqpoint{4.453420in}{6.671478in}}%
\pgfpathcurveto{\pgfqpoint{4.459244in}{6.671478in}}{\pgfqpoint{4.464830in}{6.673792in}}{\pgfqpoint{4.468948in}{6.677910in}}%
\pgfpathcurveto{\pgfqpoint{4.473067in}{6.682028in}}{\pgfqpoint{4.475380in}{6.687614in}}{\pgfqpoint{4.475380in}{6.693438in}}%
\pgfpathcurveto{\pgfqpoint{4.475380in}{6.699262in}}{\pgfqpoint{4.473067in}{6.704848in}}{\pgfqpoint{4.468948in}{6.708966in}}%
\pgfpathcurveto{\pgfqpoint{4.464830in}{6.713085in}}{\pgfqpoint{4.459244in}{6.715398in}}{\pgfqpoint{4.453420in}{6.715398in}}%
\pgfpathcurveto{\pgfqpoint{4.447596in}{6.715398in}}{\pgfqpoint{4.442010in}{6.713085in}}{\pgfqpoint{4.437892in}{6.708966in}}%
\pgfpathcurveto{\pgfqpoint{4.433774in}{6.704848in}}{\pgfqpoint{4.431460in}{6.699262in}}{\pgfqpoint{4.431460in}{6.693438in}}%
\pgfpathcurveto{\pgfqpoint{4.431460in}{6.687614in}}{\pgfqpoint{4.433774in}{6.682028in}}{\pgfqpoint{4.437892in}{6.677910in}}%
\pgfpathcurveto{\pgfqpoint{4.442010in}{6.673792in}}{\pgfqpoint{4.447596in}{6.671478in}}{\pgfqpoint{4.453420in}{6.671478in}}%
\pgfpathlineto{\pgfqpoint{4.453420in}{6.671478in}}%
\pgfpathclose%
\pgfusepath{stroke,fill}%
\end{pgfscope}%
\begin{pgfscope}%
\pgfpathrectangle{\pgfqpoint{1.582361in}{0.880000in}}{\pgfqpoint{5.035278in}{6.160000in}}%
\pgfusepath{clip}%
\pgfsetbuttcap%
\pgfsetroundjoin%
\definecolor{currentfill}{rgb}{0.800000,0.200000,0.200000}%
\pgfsetfillcolor{currentfill}%
\pgfsetlinewidth{1.003750pt}%
\definecolor{currentstroke}{rgb}{0.800000,0.200000,0.200000}%
\pgfsetstrokecolor{currentstroke}%
\pgfsetdash{}{0pt}%
\pgfpathmoveto{\pgfqpoint{4.410582in}{6.684357in}}%
\pgfpathcurveto{\pgfqpoint{4.416406in}{6.684357in}}{\pgfqpoint{4.421992in}{6.686671in}}{\pgfqpoint{4.426110in}{6.690789in}}%
\pgfpathcurveto{\pgfqpoint{4.430228in}{6.694908in}}{\pgfqpoint{4.432542in}{6.700494in}}{\pgfqpoint{4.432542in}{6.706318in}}%
\pgfpathcurveto{\pgfqpoint{4.432542in}{6.712142in}}{\pgfqpoint{4.430228in}{6.717728in}}{\pgfqpoint{4.426110in}{6.721846in}}%
\pgfpathcurveto{\pgfqpoint{4.421992in}{6.725964in}}{\pgfqpoint{4.416406in}{6.728278in}}{\pgfqpoint{4.410582in}{6.728278in}}%
\pgfpathcurveto{\pgfqpoint{4.404758in}{6.728278in}}{\pgfqpoint{4.399172in}{6.725964in}}{\pgfqpoint{4.395054in}{6.721846in}}%
\pgfpathcurveto{\pgfqpoint{4.390936in}{6.717728in}}{\pgfqpoint{4.388622in}{6.712142in}}{\pgfqpoint{4.388622in}{6.706318in}}%
\pgfpathcurveto{\pgfqpoint{4.388622in}{6.700494in}}{\pgfqpoint{4.390936in}{6.694908in}}{\pgfqpoint{4.395054in}{6.690789in}}%
\pgfpathcurveto{\pgfqpoint{4.399172in}{6.686671in}}{\pgfqpoint{4.404758in}{6.684357in}}{\pgfqpoint{4.410582in}{6.684357in}}%
\pgfpathlineto{\pgfqpoint{4.410582in}{6.684357in}}%
\pgfpathclose%
\pgfusepath{stroke,fill}%
\end{pgfscope}%
\begin{pgfscope}%
\pgfpathrectangle{\pgfqpoint{1.582361in}{0.880000in}}{\pgfqpoint{5.035278in}{6.160000in}}%
\pgfusepath{clip}%
\pgfsetbuttcap%
\pgfsetroundjoin%
\definecolor{currentfill}{rgb}{0.800000,0.200000,0.200000}%
\pgfsetfillcolor{currentfill}%
\pgfsetlinewidth{1.003750pt}%
\definecolor{currentstroke}{rgb}{0.800000,0.200000,0.200000}%
\pgfsetstrokecolor{currentstroke}%
\pgfsetdash{}{0pt}%
\pgfpathmoveto{\pgfqpoint{4.367359in}{6.695878in}}%
\pgfpathcurveto{\pgfqpoint{4.373183in}{6.695878in}}{\pgfqpoint{4.378769in}{6.698192in}}{\pgfqpoint{4.382887in}{6.702310in}}%
\pgfpathcurveto{\pgfqpoint{4.387005in}{6.706428in}}{\pgfqpoint{4.389319in}{6.712014in}}{\pgfqpoint{4.389319in}{6.717838in}}%
\pgfpathcurveto{\pgfqpoint{4.389319in}{6.723662in}}{\pgfqpoint{4.387005in}{6.729248in}}{\pgfqpoint{4.382887in}{6.733367in}}%
\pgfpathcurveto{\pgfqpoint{4.378769in}{6.737485in}}{\pgfqpoint{4.373183in}{6.739799in}}{\pgfqpoint{4.367359in}{6.739799in}}%
\pgfpathcurveto{\pgfqpoint{4.361535in}{6.739799in}}{\pgfqpoint{4.355949in}{6.737485in}}{\pgfqpoint{4.351830in}{6.733367in}}%
\pgfpathcurveto{\pgfqpoint{4.347712in}{6.729248in}}{\pgfqpoint{4.345398in}{6.723662in}}{\pgfqpoint{4.345398in}{6.717838in}}%
\pgfpathcurveto{\pgfqpoint{4.345398in}{6.712014in}}{\pgfqpoint{4.347712in}{6.706428in}}{\pgfqpoint{4.351830in}{6.702310in}}%
\pgfpathcurveto{\pgfqpoint{4.355949in}{6.698192in}}{\pgfqpoint{4.361535in}{6.695878in}}{\pgfqpoint{4.367359in}{6.695878in}}%
\pgfpathlineto{\pgfqpoint{4.367359in}{6.695878in}}%
\pgfpathclose%
\pgfusepath{stroke,fill}%
\end{pgfscope}%
\begin{pgfscope}%
\pgfpathrectangle{\pgfqpoint{1.582361in}{0.880000in}}{\pgfqpoint{5.035278in}{6.160000in}}%
\pgfusepath{clip}%
\pgfsetbuttcap%
\pgfsetroundjoin%
\definecolor{currentfill}{rgb}{0.800000,0.200000,0.200000}%
\pgfsetfillcolor{currentfill}%
\pgfsetlinewidth{1.003750pt}%
\definecolor{currentstroke}{rgb}{0.800000,0.200000,0.200000}%
\pgfsetstrokecolor{currentstroke}%
\pgfsetdash{}{0pt}%
\pgfpathmoveto{\pgfqpoint{4.323793in}{6.706029in}}%
\pgfpathcurveto{\pgfqpoint{4.329617in}{6.706029in}}{\pgfqpoint{4.335203in}{6.708342in}}{\pgfqpoint{4.339321in}{6.712461in}}%
\pgfpathcurveto{\pgfqpoint{4.343440in}{6.716579in}}{\pgfqpoint{4.345753in}{6.722165in}}{\pgfqpoint{4.345753in}{6.727989in}}%
\pgfpathcurveto{\pgfqpoint{4.345753in}{6.733813in}}{\pgfqpoint{4.343440in}{6.739399in}}{\pgfqpoint{4.339321in}{6.743517in}}%
\pgfpathcurveto{\pgfqpoint{4.335203in}{6.747635in}}{\pgfqpoint{4.329617in}{6.749949in}}{\pgfqpoint{4.323793in}{6.749949in}}%
\pgfpathcurveto{\pgfqpoint{4.317969in}{6.749949in}}{\pgfqpoint{4.312383in}{6.747635in}}{\pgfqpoint{4.308265in}{6.743517in}}%
\pgfpathcurveto{\pgfqpoint{4.304147in}{6.739399in}}{\pgfqpoint{4.301833in}{6.733813in}}{\pgfqpoint{4.301833in}{6.727989in}}%
\pgfpathcurveto{\pgfqpoint{4.301833in}{6.722165in}}{\pgfqpoint{4.304147in}{6.716579in}}{\pgfqpoint{4.308265in}{6.712461in}}%
\pgfpathcurveto{\pgfqpoint{4.312383in}{6.708342in}}{\pgfqpoint{4.317969in}{6.706029in}}{\pgfqpoint{4.323793in}{6.706029in}}%
\pgfpathlineto{\pgfqpoint{4.323793in}{6.706029in}}%
\pgfpathclose%
\pgfusepath{stroke,fill}%
\end{pgfscope}%
\begin{pgfscope}%
\pgfpathrectangle{\pgfqpoint{1.582361in}{0.880000in}}{\pgfqpoint{5.035278in}{6.160000in}}%
\pgfusepath{clip}%
\pgfsetbuttcap%
\pgfsetroundjoin%
\definecolor{currentfill}{rgb}{0.800000,0.200000,0.200000}%
\pgfsetfillcolor{currentfill}%
\pgfsetlinewidth{1.003750pt}%
\definecolor{currentstroke}{rgb}{0.800000,0.200000,0.200000}%
\pgfsetstrokecolor{currentstroke}%
\pgfsetdash{}{0pt}%
\pgfpathmoveto{\pgfqpoint{4.279929in}{6.714799in}}%
\pgfpathcurveto{\pgfqpoint{4.285753in}{6.714799in}}{\pgfqpoint{4.291339in}{6.717113in}}{\pgfqpoint{4.295457in}{6.721231in}}%
\pgfpathcurveto{\pgfqpoint{4.299575in}{6.725349in}}{\pgfqpoint{4.301889in}{6.730935in}}{\pgfqpoint{4.301889in}{6.736759in}}%
\pgfpathcurveto{\pgfqpoint{4.301889in}{6.742583in}}{\pgfqpoint{4.299575in}{6.748169in}}{\pgfqpoint{4.295457in}{6.752287in}}%
\pgfpathcurveto{\pgfqpoint{4.291339in}{6.756405in}}{\pgfqpoint{4.285753in}{6.758719in}}{\pgfqpoint{4.279929in}{6.758719in}}%
\pgfpathcurveto{\pgfqpoint{4.274105in}{6.758719in}}{\pgfqpoint{4.268519in}{6.756405in}}{\pgfqpoint{4.264401in}{6.752287in}}%
\pgfpathcurveto{\pgfqpoint{4.260283in}{6.748169in}}{\pgfqpoint{4.257969in}{6.742583in}}{\pgfqpoint{4.257969in}{6.736759in}}%
\pgfpathcurveto{\pgfqpoint{4.257969in}{6.730935in}}{\pgfqpoint{4.260283in}{6.725349in}}{\pgfqpoint{4.264401in}{6.721231in}}%
\pgfpathcurveto{\pgfqpoint{4.268519in}{6.717113in}}{\pgfqpoint{4.274105in}{6.714799in}}{\pgfqpoint{4.279929in}{6.714799in}}%
\pgfpathlineto{\pgfqpoint{4.279929in}{6.714799in}}%
\pgfpathclose%
\pgfusepath{stroke,fill}%
\end{pgfscope}%
\begin{pgfscope}%
\pgfpathrectangle{\pgfqpoint{1.582361in}{0.880000in}}{\pgfqpoint{5.035278in}{6.160000in}}%
\pgfusepath{clip}%
\pgfsetbuttcap%
\pgfsetroundjoin%
\definecolor{currentfill}{rgb}{0.800000,0.200000,0.200000}%
\pgfsetfillcolor{currentfill}%
\pgfsetlinewidth{1.003750pt}%
\definecolor{currentstroke}{rgb}{0.800000,0.200000,0.200000}%
\pgfsetstrokecolor{currentstroke}%
\pgfsetdash{}{0pt}%
\pgfpathmoveto{\pgfqpoint{4.235810in}{6.722180in}}%
\pgfpathcurveto{\pgfqpoint{4.241634in}{6.722180in}}{\pgfqpoint{4.247220in}{6.724494in}}{\pgfqpoint{4.251338in}{6.728612in}}%
\pgfpathcurveto{\pgfqpoint{4.255456in}{6.732730in}}{\pgfqpoint{4.257770in}{6.738316in}}{\pgfqpoint{4.257770in}{6.744140in}}%
\pgfpathcurveto{\pgfqpoint{4.257770in}{6.749964in}}{\pgfqpoint{4.255456in}{6.755550in}}{\pgfqpoint{4.251338in}{6.759668in}}%
\pgfpathcurveto{\pgfqpoint{4.247220in}{6.763786in}}{\pgfqpoint{4.241634in}{6.766100in}}{\pgfqpoint{4.235810in}{6.766100in}}%
\pgfpathcurveto{\pgfqpoint{4.229986in}{6.766100in}}{\pgfqpoint{4.224400in}{6.763786in}}{\pgfqpoint{4.220282in}{6.759668in}}%
\pgfpathcurveto{\pgfqpoint{4.216163in}{6.755550in}}{\pgfqpoint{4.213850in}{6.749964in}}{\pgfqpoint{4.213850in}{6.744140in}}%
\pgfpathcurveto{\pgfqpoint{4.213850in}{6.738316in}}{\pgfqpoint{4.216163in}{6.732730in}}{\pgfqpoint{4.220282in}{6.728612in}}%
\pgfpathcurveto{\pgfqpoint{4.224400in}{6.724494in}}{\pgfqpoint{4.229986in}{6.722180in}}{\pgfqpoint{4.235810in}{6.722180in}}%
\pgfpathlineto{\pgfqpoint{4.235810in}{6.722180in}}%
\pgfpathclose%
\pgfusepath{stroke,fill}%
\end{pgfscope}%
\begin{pgfscope}%
\pgfpathrectangle{\pgfqpoint{1.582361in}{0.880000in}}{\pgfqpoint{5.035278in}{6.160000in}}%
\pgfusepath{clip}%
\pgfsetbuttcap%
\pgfsetroundjoin%
\definecolor{currentfill}{rgb}{0.800000,0.200000,0.200000}%
\pgfsetfillcolor{currentfill}%
\pgfsetlinewidth{1.003750pt}%
\definecolor{currentstroke}{rgb}{0.800000,0.200000,0.200000}%
\pgfsetstrokecolor{currentstroke}%
\pgfsetdash{}{0pt}%
\pgfpathmoveto{\pgfqpoint{4.191480in}{6.728164in}}%
\pgfpathcurveto{\pgfqpoint{4.197304in}{6.728164in}}{\pgfqpoint{4.202890in}{6.730478in}}{\pgfqpoint{4.207008in}{6.734596in}}%
\pgfpathcurveto{\pgfqpoint{4.211126in}{6.738714in}}{\pgfqpoint{4.213440in}{6.744301in}}{\pgfqpoint{4.213440in}{6.750125in}}%
\pgfpathcurveto{\pgfqpoint{4.213440in}{6.755949in}}{\pgfqpoint{4.211126in}{6.761535in}}{\pgfqpoint{4.207008in}{6.765653in}}%
\pgfpathcurveto{\pgfqpoint{4.202890in}{6.769771in}}{\pgfqpoint{4.197304in}{6.772085in}}{\pgfqpoint{4.191480in}{6.772085in}}%
\pgfpathcurveto{\pgfqpoint{4.185656in}{6.772085in}}{\pgfqpoint{4.180069in}{6.769771in}}{\pgfqpoint{4.175951in}{6.765653in}}%
\pgfpathcurveto{\pgfqpoint{4.171833in}{6.761535in}}{\pgfqpoint{4.169519in}{6.755949in}}{\pgfqpoint{4.169519in}{6.750125in}}%
\pgfpathcurveto{\pgfqpoint{4.169519in}{6.744301in}}{\pgfqpoint{4.171833in}{6.738714in}}{\pgfqpoint{4.175951in}{6.734596in}}%
\pgfpathcurveto{\pgfqpoint{4.180069in}{6.730478in}}{\pgfqpoint{4.185656in}{6.728164in}}{\pgfqpoint{4.191480in}{6.728164in}}%
\pgfpathlineto{\pgfqpoint{4.191480in}{6.728164in}}%
\pgfpathclose%
\pgfusepath{stroke,fill}%
\end{pgfscope}%
\begin{pgfscope}%
\pgfpathrectangle{\pgfqpoint{1.582361in}{0.880000in}}{\pgfqpoint{5.035278in}{6.160000in}}%
\pgfusepath{clip}%
\pgfsetbuttcap%
\pgfsetroundjoin%
\definecolor{currentfill}{rgb}{0.800000,0.200000,0.200000}%
\pgfsetfillcolor{currentfill}%
\pgfsetlinewidth{1.003750pt}%
\definecolor{currentstroke}{rgb}{0.800000,0.200000,0.200000}%
\pgfsetstrokecolor{currentstroke}%
\pgfsetdash{}{0pt}%
\pgfpathmoveto{\pgfqpoint{4.146983in}{6.732746in}}%
\pgfpathcurveto{\pgfqpoint{4.152806in}{6.732746in}}{\pgfqpoint{4.158393in}{6.735060in}}{\pgfqpoint{4.162511in}{6.739178in}}%
\pgfpathcurveto{\pgfqpoint{4.166629in}{6.743297in}}{\pgfqpoint{4.168943in}{6.748883in}}{\pgfqpoint{4.168943in}{6.754707in}}%
\pgfpathcurveto{\pgfqpoint{4.168943in}{6.760531in}}{\pgfqpoint{4.166629in}{6.766117in}}{\pgfqpoint{4.162511in}{6.770235in}}%
\pgfpathcurveto{\pgfqpoint{4.158393in}{6.774353in}}{\pgfqpoint{4.152806in}{6.776667in}}{\pgfqpoint{4.146983in}{6.776667in}}%
\pgfpathcurveto{\pgfqpoint{4.141159in}{6.776667in}}{\pgfqpoint{4.135572in}{6.774353in}}{\pgfqpoint{4.131454in}{6.770235in}}%
\pgfpathcurveto{\pgfqpoint{4.127336in}{6.766117in}}{\pgfqpoint{4.125022in}{6.760531in}}{\pgfqpoint{4.125022in}{6.754707in}}%
\pgfpathcurveto{\pgfqpoint{4.125022in}{6.748883in}}{\pgfqpoint{4.127336in}{6.743297in}}{\pgfqpoint{4.131454in}{6.739178in}}%
\pgfpathcurveto{\pgfqpoint{4.135572in}{6.735060in}}{\pgfqpoint{4.141159in}{6.732746in}}{\pgfqpoint{4.146983in}{6.732746in}}%
\pgfpathlineto{\pgfqpoint{4.146983in}{6.732746in}}%
\pgfpathclose%
\pgfusepath{stroke,fill}%
\end{pgfscope}%
\begin{pgfscope}%
\pgfpathrectangle{\pgfqpoint{1.582361in}{0.880000in}}{\pgfqpoint{5.035278in}{6.160000in}}%
\pgfusepath{clip}%
\pgfsetbuttcap%
\pgfsetroundjoin%
\definecolor{currentfill}{rgb}{0.800000,0.200000,0.200000}%
\pgfsetfillcolor{currentfill}%
\pgfsetlinewidth{1.003750pt}%
\definecolor{currentstroke}{rgb}{0.800000,0.200000,0.200000}%
\pgfsetstrokecolor{currentstroke}%
\pgfsetdash{}{0pt}%
\pgfpathmoveto{\pgfqpoint{4.102363in}{6.735922in}}%
\pgfpathcurveto{\pgfqpoint{4.108187in}{6.735922in}}{\pgfqpoint{4.113773in}{6.738236in}}{\pgfqpoint{4.117891in}{6.742354in}}%
\pgfpathcurveto{\pgfqpoint{4.122009in}{6.746472in}}{\pgfqpoint{4.124323in}{6.752058in}}{\pgfqpoint{4.124323in}{6.757882in}}%
\pgfpathcurveto{\pgfqpoint{4.124323in}{6.763706in}}{\pgfqpoint{4.122009in}{6.769292in}}{\pgfqpoint{4.117891in}{6.773410in}}%
\pgfpathcurveto{\pgfqpoint{4.113773in}{6.777528in}}{\pgfqpoint{4.108187in}{6.779842in}}{\pgfqpoint{4.102363in}{6.779842in}}%
\pgfpathcurveto{\pgfqpoint{4.096539in}{6.779842in}}{\pgfqpoint{4.090953in}{6.777528in}}{\pgfqpoint{4.086835in}{6.773410in}}%
\pgfpathcurveto{\pgfqpoint{4.082717in}{6.769292in}}{\pgfqpoint{4.080403in}{6.763706in}}{\pgfqpoint{4.080403in}{6.757882in}}%
\pgfpathcurveto{\pgfqpoint{4.080403in}{6.752058in}}{\pgfqpoint{4.082717in}{6.746472in}}{\pgfqpoint{4.086835in}{6.742354in}}%
\pgfpathcurveto{\pgfqpoint{4.090953in}{6.738236in}}{\pgfqpoint{4.096539in}{6.735922in}}{\pgfqpoint{4.102363in}{6.735922in}}%
\pgfpathlineto{\pgfqpoint{4.102363in}{6.735922in}}%
\pgfpathclose%
\pgfusepath{stroke,fill}%
\end{pgfscope}%
\begin{pgfscope}%
\pgfpathrectangle{\pgfqpoint{1.582361in}{0.880000in}}{\pgfqpoint{5.035278in}{6.160000in}}%
\pgfusepath{clip}%
\pgfsetbuttcap%
\pgfsetroundjoin%
\definecolor{currentfill}{rgb}{0.800000,0.200000,0.200000}%
\pgfsetfillcolor{currentfill}%
\pgfsetlinewidth{1.003750pt}%
\definecolor{currentstroke}{rgb}{0.800000,0.200000,0.200000}%
\pgfsetstrokecolor{currentstroke}%
\pgfsetdash{}{0pt}%
\pgfpathmoveto{\pgfqpoint{4.057666in}{6.737687in}}%
\pgfpathcurveto{\pgfqpoint{4.063489in}{6.737687in}}{\pgfqpoint{4.069076in}{6.740001in}}{\pgfqpoint{4.073194in}{6.744119in}}%
\pgfpathcurveto{\pgfqpoint{4.077312in}{6.748237in}}{\pgfqpoint{4.079626in}{6.753823in}}{\pgfqpoint{4.079626in}{6.759647in}}%
\pgfpathcurveto{\pgfqpoint{4.079626in}{6.765471in}}{\pgfqpoint{4.077312in}{6.771057in}}{\pgfqpoint{4.073194in}{6.775175in}}%
\pgfpathcurveto{\pgfqpoint{4.069076in}{6.779293in}}{\pgfqpoint{4.063489in}{6.781607in}}{\pgfqpoint{4.057666in}{6.781607in}}%
\pgfpathcurveto{\pgfqpoint{4.051842in}{6.781607in}}{\pgfqpoint{4.046255in}{6.779293in}}{\pgfqpoint{4.042137in}{6.775175in}}%
\pgfpathcurveto{\pgfqpoint{4.038019in}{6.771057in}}{\pgfqpoint{4.035705in}{6.765471in}}{\pgfqpoint{4.035705in}{6.759647in}}%
\pgfpathcurveto{\pgfqpoint{4.035705in}{6.753823in}}{\pgfqpoint{4.038019in}{6.748237in}}{\pgfqpoint{4.042137in}{6.744119in}}%
\pgfpathcurveto{\pgfqpoint{4.046255in}{6.740001in}}{\pgfqpoint{4.051842in}{6.737687in}}{\pgfqpoint{4.057666in}{6.737687in}}%
\pgfpathlineto{\pgfqpoint{4.057666in}{6.737687in}}%
\pgfpathclose%
\pgfusepath{stroke,fill}%
\end{pgfscope}%
\begin{pgfscope}%
\pgfpathrectangle{\pgfqpoint{1.582361in}{0.880000in}}{\pgfqpoint{5.035278in}{6.160000in}}%
\pgfusepath{clip}%
\pgfsetbuttcap%
\pgfsetroundjoin%
\definecolor{currentfill}{rgb}{0.800000,0.200000,0.200000}%
\pgfsetfillcolor{currentfill}%
\pgfsetlinewidth{1.003750pt}%
\definecolor{currentstroke}{rgb}{0.800000,0.200000,0.200000}%
\pgfsetstrokecolor{currentstroke}%
\pgfsetdash{}{0pt}%
\pgfpathmoveto{\pgfqpoint{4.012935in}{6.738040in}}%
\pgfpathcurveto{\pgfqpoint{4.018758in}{6.738040in}}{\pgfqpoint{4.024345in}{6.740354in}}{\pgfqpoint{4.028463in}{6.744472in}}%
\pgfpathcurveto{\pgfqpoint{4.032581in}{6.748590in}}{\pgfqpoint{4.034895in}{6.754176in}}{\pgfqpoint{4.034895in}{6.760000in}}%
\pgfpathcurveto{\pgfqpoint{4.034895in}{6.765824in}}{\pgfqpoint{4.032581in}{6.771410in}}{\pgfqpoint{4.028463in}{6.775528in}}%
\pgfpathcurveto{\pgfqpoint{4.024345in}{6.779646in}}{\pgfqpoint{4.018758in}{6.781960in}}{\pgfqpoint{4.012935in}{6.781960in}}%
\pgfpathcurveto{\pgfqpoint{4.007111in}{6.781960in}}{\pgfqpoint{4.001524in}{6.779646in}}{\pgfqpoint{3.997406in}{6.775528in}}%
\pgfpathcurveto{\pgfqpoint{3.993288in}{6.771410in}}{\pgfqpoint{3.990974in}{6.765824in}}{\pgfqpoint{3.990974in}{6.760000in}}%
\pgfpathcurveto{\pgfqpoint{3.990974in}{6.754176in}}{\pgfqpoint{3.993288in}{6.748590in}}{\pgfqpoint{3.997406in}{6.744472in}}%
\pgfpathcurveto{\pgfqpoint{4.001524in}{6.740354in}}{\pgfqpoint{4.007111in}{6.738040in}}{\pgfqpoint{4.012935in}{6.738040in}}%
\pgfpathlineto{\pgfqpoint{4.012935in}{6.738040in}}%
\pgfpathclose%
\pgfusepath{stroke,fill}%
\end{pgfscope}%
\begin{pgfscope}%
\pgfpathrectangle{\pgfqpoint{1.582361in}{0.880000in}}{\pgfqpoint{5.035278in}{6.160000in}}%
\pgfusepath{clip}%
\pgfsetbuttcap%
\pgfsetroundjoin%
\definecolor{currentfill}{rgb}{0.800000,0.200000,0.200000}%
\pgfsetfillcolor{currentfill}%
\pgfsetlinewidth{1.003750pt}%
\definecolor{currentstroke}{rgb}{0.800000,0.200000,0.200000}%
\pgfsetstrokecolor{currentstroke}%
\pgfsetdash{}{0pt}%
\pgfpathmoveto{\pgfqpoint{3.968215in}{6.736981in}}%
\pgfpathcurveto{\pgfqpoint{3.974039in}{6.736981in}}{\pgfqpoint{3.979625in}{6.739294in}}{\pgfqpoint{3.983743in}{6.743413in}}%
\pgfpathcurveto{\pgfqpoint{3.987861in}{6.747531in}}{\pgfqpoint{3.990175in}{6.753117in}}{\pgfqpoint{3.990175in}{6.758941in}}%
\pgfpathcurveto{\pgfqpoint{3.990175in}{6.764765in}}{\pgfqpoint{3.987861in}{6.770351in}}{\pgfqpoint{3.983743in}{6.774469in}}%
\pgfpathcurveto{\pgfqpoint{3.979625in}{6.778587in}}{\pgfqpoint{3.974039in}{6.780901in}}{\pgfqpoint{3.968215in}{6.780901in}}%
\pgfpathcurveto{\pgfqpoint{3.962391in}{6.780901in}}{\pgfqpoint{3.956805in}{6.778587in}}{\pgfqpoint{3.952686in}{6.774469in}}%
\pgfpathcurveto{\pgfqpoint{3.948568in}{6.770351in}}{\pgfqpoint{3.946254in}{6.764765in}}{\pgfqpoint{3.946254in}{6.758941in}}%
\pgfpathcurveto{\pgfqpoint{3.946254in}{6.753117in}}{\pgfqpoint{3.948568in}{6.747531in}}{\pgfqpoint{3.952686in}{6.743413in}}%
\pgfpathcurveto{\pgfqpoint{3.956805in}{6.739294in}}{\pgfqpoint{3.962391in}{6.736981in}}{\pgfqpoint{3.968215in}{6.736981in}}%
\pgfpathlineto{\pgfqpoint{3.968215in}{6.736981in}}%
\pgfpathclose%
\pgfusepath{stroke,fill}%
\end{pgfscope}%
\begin{pgfscope}%
\pgfpathrectangle{\pgfqpoint{1.582361in}{0.880000in}}{\pgfqpoint{5.035278in}{6.160000in}}%
\pgfusepath{clip}%
\pgfsetbuttcap%
\pgfsetroundjoin%
\definecolor{currentfill}{rgb}{0.800000,0.200000,0.200000}%
\pgfsetfillcolor{currentfill}%
\pgfsetlinewidth{1.003750pt}%
\definecolor{currentstroke}{rgb}{0.800000,0.200000,0.200000}%
\pgfsetstrokecolor{currentstroke}%
\pgfsetdash{}{0pt}%
\pgfpathmoveto{\pgfqpoint{3.923551in}{6.734510in}}%
\pgfpathcurveto{\pgfqpoint{3.929375in}{6.734510in}}{\pgfqpoint{3.934961in}{6.736824in}}{\pgfqpoint{3.939079in}{6.740942in}}%
\pgfpathcurveto{\pgfqpoint{3.943197in}{6.745060in}}{\pgfqpoint{3.945511in}{6.750647in}}{\pgfqpoint{3.945511in}{6.756470in}}%
\pgfpathcurveto{\pgfqpoint{3.945511in}{6.762294in}}{\pgfqpoint{3.943197in}{6.767881in}}{\pgfqpoint{3.939079in}{6.771999in}}%
\pgfpathcurveto{\pgfqpoint{3.934961in}{6.776117in}}{\pgfqpoint{3.929375in}{6.778431in}}{\pgfqpoint{3.923551in}{6.778431in}}%
\pgfpathcurveto{\pgfqpoint{3.917727in}{6.778431in}}{\pgfqpoint{3.912141in}{6.776117in}}{\pgfqpoint{3.908022in}{6.771999in}}%
\pgfpathcurveto{\pgfqpoint{3.903904in}{6.767881in}}{\pgfqpoint{3.901590in}{6.762294in}}{\pgfqpoint{3.901590in}{6.756470in}}%
\pgfpathcurveto{\pgfqpoint{3.901590in}{6.750647in}}{\pgfqpoint{3.903904in}{6.745060in}}{\pgfqpoint{3.908022in}{6.740942in}}%
\pgfpathcurveto{\pgfqpoint{3.912141in}{6.736824in}}{\pgfqpoint{3.917727in}{6.734510in}}{\pgfqpoint{3.923551in}{6.734510in}}%
\pgfpathlineto{\pgfqpoint{3.923551in}{6.734510in}}%
\pgfpathclose%
\pgfusepath{stroke,fill}%
\end{pgfscope}%
\begin{pgfscope}%
\pgfpathrectangle{\pgfqpoint{1.582361in}{0.880000in}}{\pgfqpoint{5.035278in}{6.160000in}}%
\pgfusepath{clip}%
\pgfsetbuttcap%
\pgfsetroundjoin%
\definecolor{currentfill}{rgb}{0.800000,0.200000,0.200000}%
\pgfsetfillcolor{currentfill}%
\pgfsetlinewidth{1.003750pt}%
\definecolor{currentstroke}{rgb}{0.800000,0.200000,0.200000}%
\pgfsetstrokecolor{currentstroke}%
\pgfsetdash{}{0pt}%
\pgfpathmoveto{\pgfqpoint{3.878987in}{6.730631in}}%
\pgfpathcurveto{\pgfqpoint{3.884811in}{6.730631in}}{\pgfqpoint{3.890397in}{6.732945in}}{\pgfqpoint{3.894515in}{6.737063in}}%
\pgfpathcurveto{\pgfqpoint{3.898633in}{6.741181in}}{\pgfqpoint{3.900947in}{6.746767in}}{\pgfqpoint{3.900947in}{6.752591in}}%
\pgfpathcurveto{\pgfqpoint{3.900947in}{6.758415in}}{\pgfqpoint{3.898633in}{6.764001in}}{\pgfqpoint{3.894515in}{6.768120in}}%
\pgfpathcurveto{\pgfqpoint{3.890397in}{6.772238in}}{\pgfqpoint{3.884811in}{6.774552in}}{\pgfqpoint{3.878987in}{6.774552in}}%
\pgfpathcurveto{\pgfqpoint{3.873163in}{6.774552in}}{\pgfqpoint{3.867577in}{6.772238in}}{\pgfqpoint{3.863459in}{6.768120in}}%
\pgfpathcurveto{\pgfqpoint{3.859340in}{6.764001in}}{\pgfqpoint{3.857027in}{6.758415in}}{\pgfqpoint{3.857027in}{6.752591in}}%
\pgfpathcurveto{\pgfqpoint{3.857027in}{6.746767in}}{\pgfqpoint{3.859340in}{6.741181in}}{\pgfqpoint{3.863459in}{6.737063in}}%
\pgfpathcurveto{\pgfqpoint{3.867577in}{6.732945in}}{\pgfqpoint{3.873163in}{6.730631in}}{\pgfqpoint{3.878987in}{6.730631in}}%
\pgfpathlineto{\pgfqpoint{3.878987in}{6.730631in}}%
\pgfpathclose%
\pgfusepath{stroke,fill}%
\end{pgfscope}%
\begin{pgfscope}%
\pgfpathrectangle{\pgfqpoint{1.582361in}{0.880000in}}{\pgfqpoint{5.035278in}{6.160000in}}%
\pgfusepath{clip}%
\pgfsetbuttcap%
\pgfsetroundjoin%
\definecolor{currentfill}{rgb}{0.800000,0.200000,0.200000}%
\pgfsetfillcolor{currentfill}%
\pgfsetlinewidth{1.003750pt}%
\definecolor{currentstroke}{rgb}{0.800000,0.200000,0.200000}%
\pgfsetstrokecolor{currentstroke}%
\pgfsetdash{}{0pt}%
\pgfpathmoveto{\pgfqpoint{3.834568in}{6.725347in}}%
\pgfpathcurveto{\pgfqpoint{3.840392in}{6.725347in}}{\pgfqpoint{3.845978in}{6.727661in}}{\pgfqpoint{3.850096in}{6.731779in}}%
\pgfpathcurveto{\pgfqpoint{3.854214in}{6.735897in}}{\pgfqpoint{3.856528in}{6.741483in}}{\pgfqpoint{3.856528in}{6.747307in}}%
\pgfpathcurveto{\pgfqpoint{3.856528in}{6.753131in}}{\pgfqpoint{3.854214in}{6.758717in}}{\pgfqpoint{3.850096in}{6.762836in}}%
\pgfpathcurveto{\pgfqpoint{3.845978in}{6.766954in}}{\pgfqpoint{3.840392in}{6.769268in}}{\pgfqpoint{3.834568in}{6.769268in}}%
\pgfpathcurveto{\pgfqpoint{3.828744in}{6.769268in}}{\pgfqpoint{3.823158in}{6.766954in}}{\pgfqpoint{3.819039in}{6.762836in}}%
\pgfpathcurveto{\pgfqpoint{3.814921in}{6.758717in}}{\pgfqpoint{3.812607in}{6.753131in}}{\pgfqpoint{3.812607in}{6.747307in}}%
\pgfpathcurveto{\pgfqpoint{3.812607in}{6.741483in}}{\pgfqpoint{3.814921in}{6.735897in}}{\pgfqpoint{3.819039in}{6.731779in}}%
\pgfpathcurveto{\pgfqpoint{3.823158in}{6.727661in}}{\pgfqpoint{3.828744in}{6.725347in}}{\pgfqpoint{3.834568in}{6.725347in}}%
\pgfpathlineto{\pgfqpoint{3.834568in}{6.725347in}}%
\pgfpathclose%
\pgfusepath{stroke,fill}%
\end{pgfscope}%
\begin{pgfscope}%
\pgfpathrectangle{\pgfqpoint{1.582361in}{0.880000in}}{\pgfqpoint{5.035278in}{6.160000in}}%
\pgfusepath{clip}%
\pgfsetbuttcap%
\pgfsetroundjoin%
\definecolor{currentfill}{rgb}{0.800000,0.200000,0.200000}%
\pgfsetfillcolor{currentfill}%
\pgfsetlinewidth{1.003750pt}%
\definecolor{currentstroke}{rgb}{0.800000,0.200000,0.200000}%
\pgfsetstrokecolor{currentstroke}%
\pgfsetdash{}{0pt}%
\pgfpathmoveto{\pgfqpoint{3.790337in}{6.718663in}}%
\pgfpathcurveto{\pgfqpoint{3.796161in}{6.718663in}}{\pgfqpoint{3.801748in}{6.720977in}}{\pgfqpoint{3.805866in}{6.725095in}}%
\pgfpathcurveto{\pgfqpoint{3.809984in}{6.729214in}}{\pgfqpoint{3.812298in}{6.734800in}}{\pgfqpoint{3.812298in}{6.740624in}}%
\pgfpathcurveto{\pgfqpoint{3.812298in}{6.746448in}}{\pgfqpoint{3.809984in}{6.752034in}}{\pgfqpoint{3.805866in}{6.756152in}}%
\pgfpathcurveto{\pgfqpoint{3.801748in}{6.760270in}}{\pgfqpoint{3.796161in}{6.762584in}}{\pgfqpoint{3.790337in}{6.762584in}}%
\pgfpathcurveto{\pgfqpoint{3.784514in}{6.762584in}}{\pgfqpoint{3.778927in}{6.760270in}}{\pgfqpoint{3.774809in}{6.756152in}}%
\pgfpathcurveto{\pgfqpoint{3.770691in}{6.752034in}}{\pgfqpoint{3.768377in}{6.746448in}}{\pgfqpoint{3.768377in}{6.740624in}}%
\pgfpathcurveto{\pgfqpoint{3.768377in}{6.734800in}}{\pgfqpoint{3.770691in}{6.729214in}}{\pgfqpoint{3.774809in}{6.725095in}}%
\pgfpathcurveto{\pgfqpoint{3.778927in}{6.720977in}}{\pgfqpoint{3.784514in}{6.718663in}}{\pgfqpoint{3.790337in}{6.718663in}}%
\pgfpathlineto{\pgfqpoint{3.790337in}{6.718663in}}%
\pgfpathclose%
\pgfusepath{stroke,fill}%
\end{pgfscope}%
\begin{pgfscope}%
\pgfpathrectangle{\pgfqpoint{1.582361in}{0.880000in}}{\pgfqpoint{5.035278in}{6.160000in}}%
\pgfusepath{clip}%
\pgfsetbuttcap%
\pgfsetroundjoin%
\definecolor{currentfill}{rgb}{0.800000,0.200000,0.200000}%
\pgfsetfillcolor{currentfill}%
\pgfsetlinewidth{1.003750pt}%
\definecolor{currentstroke}{rgb}{0.800000,0.200000,0.200000}%
\pgfsetstrokecolor{currentstroke}%
\pgfsetdash{}{0pt}%
\pgfpathmoveto{\pgfqpoint{3.746340in}{6.710587in}}%
\pgfpathcurveto{\pgfqpoint{3.752164in}{6.710587in}}{\pgfqpoint{3.757750in}{6.712901in}}{\pgfqpoint{3.761869in}{6.717019in}}%
\pgfpathcurveto{\pgfqpoint{3.765987in}{6.721137in}}{\pgfqpoint{3.768301in}{6.726723in}}{\pgfqpoint{3.768301in}{6.732547in}}%
\pgfpathcurveto{\pgfqpoint{3.768301in}{6.738371in}}{\pgfqpoint{3.765987in}{6.743957in}}{\pgfqpoint{3.761869in}{6.748075in}}%
\pgfpathcurveto{\pgfqpoint{3.757750in}{6.752193in}}{\pgfqpoint{3.752164in}{6.754507in}}{\pgfqpoint{3.746340in}{6.754507in}}%
\pgfpathcurveto{\pgfqpoint{3.740516in}{6.754507in}}{\pgfqpoint{3.734930in}{6.752193in}}{\pgfqpoint{3.730812in}{6.748075in}}%
\pgfpathcurveto{\pgfqpoint{3.726694in}{6.743957in}}{\pgfqpoint{3.724380in}{6.738371in}}{\pgfqpoint{3.724380in}{6.732547in}}%
\pgfpathcurveto{\pgfqpoint{3.724380in}{6.726723in}}{\pgfqpoint{3.726694in}{6.721137in}}{\pgfqpoint{3.730812in}{6.717019in}}%
\pgfpathcurveto{\pgfqpoint{3.734930in}{6.712901in}}{\pgfqpoint{3.740516in}{6.710587in}}{\pgfqpoint{3.746340in}{6.710587in}}%
\pgfpathlineto{\pgfqpoint{3.746340in}{6.710587in}}%
\pgfpathclose%
\pgfusepath{stroke,fill}%
\end{pgfscope}%
\begin{pgfscope}%
\pgfpathrectangle{\pgfqpoint{1.582361in}{0.880000in}}{\pgfqpoint{5.035278in}{6.160000in}}%
\pgfusepath{clip}%
\pgfsetbuttcap%
\pgfsetroundjoin%
\definecolor{currentfill}{rgb}{0.800000,0.200000,0.200000}%
\pgfsetfillcolor{currentfill}%
\pgfsetlinewidth{1.003750pt}%
\definecolor{currentstroke}{rgb}{0.800000,0.200000,0.200000}%
\pgfsetstrokecolor{currentstroke}%
\pgfsetdash{}{0pt}%
\pgfpathmoveto{\pgfqpoint{3.702620in}{6.701125in}}%
\pgfpathcurveto{\pgfqpoint{3.708444in}{6.701125in}}{\pgfqpoint{3.714030in}{6.703439in}}{\pgfqpoint{3.718148in}{6.707557in}}%
\pgfpathcurveto{\pgfqpoint{3.722266in}{6.711675in}}{\pgfqpoint{3.724580in}{6.717262in}}{\pgfqpoint{3.724580in}{6.723086in}}%
\pgfpathcurveto{\pgfqpoint{3.724580in}{6.728909in}}{\pgfqpoint{3.722266in}{6.734496in}}{\pgfqpoint{3.718148in}{6.738614in}}%
\pgfpathcurveto{\pgfqpoint{3.714030in}{6.742732in}}{\pgfqpoint{3.708444in}{6.745046in}}{\pgfqpoint{3.702620in}{6.745046in}}%
\pgfpathcurveto{\pgfqpoint{3.696796in}{6.745046in}}{\pgfqpoint{3.691210in}{6.742732in}}{\pgfqpoint{3.687092in}{6.738614in}}%
\pgfpathcurveto{\pgfqpoint{3.682974in}{6.734496in}}{\pgfqpoint{3.680660in}{6.728909in}}{\pgfqpoint{3.680660in}{6.723086in}}%
\pgfpathcurveto{\pgfqpoint{3.680660in}{6.717262in}}{\pgfqpoint{3.682974in}{6.711675in}}{\pgfqpoint{3.687092in}{6.707557in}}%
\pgfpathcurveto{\pgfqpoint{3.691210in}{6.703439in}}{\pgfqpoint{3.696796in}{6.701125in}}{\pgfqpoint{3.702620in}{6.701125in}}%
\pgfpathlineto{\pgfqpoint{3.702620in}{6.701125in}}%
\pgfpathclose%
\pgfusepath{stroke,fill}%
\end{pgfscope}%
\begin{pgfscope}%
\pgfpathrectangle{\pgfqpoint{1.582361in}{0.880000in}}{\pgfqpoint{5.035278in}{6.160000in}}%
\pgfusepath{clip}%
\pgfsetbuttcap%
\pgfsetroundjoin%
\definecolor{currentfill}{rgb}{0.800000,0.200000,0.200000}%
\pgfsetfillcolor{currentfill}%
\pgfsetlinewidth{1.003750pt}%
\definecolor{currentstroke}{rgb}{0.800000,0.200000,0.200000}%
\pgfsetstrokecolor{currentstroke}%
\pgfsetdash{}{0pt}%
\pgfpathmoveto{\pgfqpoint{3.659220in}{6.690288in}}%
\pgfpathcurveto{\pgfqpoint{3.665044in}{6.690288in}}{\pgfqpoint{3.670630in}{6.692602in}}{\pgfqpoint{3.674748in}{6.696720in}}%
\pgfpathcurveto{\pgfqpoint{3.678867in}{6.700838in}}{\pgfqpoint{3.681180in}{6.706425in}}{\pgfqpoint{3.681180in}{6.712249in}}%
\pgfpathcurveto{\pgfqpoint{3.681180in}{6.718073in}}{\pgfqpoint{3.678867in}{6.723659in}}{\pgfqpoint{3.674748in}{6.727777in}}%
\pgfpathcurveto{\pgfqpoint{3.670630in}{6.731895in}}{\pgfqpoint{3.665044in}{6.734209in}}{\pgfqpoint{3.659220in}{6.734209in}}%
\pgfpathcurveto{\pgfqpoint{3.653396in}{6.734209in}}{\pgfqpoint{3.647810in}{6.731895in}}{\pgfqpoint{3.643692in}{6.727777in}}%
\pgfpathcurveto{\pgfqpoint{3.639574in}{6.723659in}}{\pgfqpoint{3.637260in}{6.718073in}}{\pgfqpoint{3.637260in}{6.712249in}}%
\pgfpathcurveto{\pgfqpoint{3.637260in}{6.706425in}}{\pgfqpoint{3.639574in}{6.700838in}}{\pgfqpoint{3.643692in}{6.696720in}}%
\pgfpathcurveto{\pgfqpoint{3.647810in}{6.692602in}}{\pgfqpoint{3.653396in}{6.690288in}}{\pgfqpoint{3.659220in}{6.690288in}}%
\pgfpathlineto{\pgfqpoint{3.659220in}{6.690288in}}%
\pgfpathclose%
\pgfusepath{stroke,fill}%
\end{pgfscope}%
\begin{pgfscope}%
\pgfpathrectangle{\pgfqpoint{1.582361in}{0.880000in}}{\pgfqpoint{5.035278in}{6.160000in}}%
\pgfusepath{clip}%
\pgfsetbuttcap%
\pgfsetroundjoin%
\definecolor{currentfill}{rgb}{0.800000,0.200000,0.200000}%
\pgfsetfillcolor{currentfill}%
\pgfsetlinewidth{1.003750pt}%
\definecolor{currentstroke}{rgb}{0.800000,0.200000,0.200000}%
\pgfsetstrokecolor{currentstroke}%
\pgfsetdash{}{0pt}%
\pgfpathmoveto{\pgfqpoint{3.616184in}{6.678087in}}%
\pgfpathcurveto{\pgfqpoint{3.622008in}{6.678087in}}{\pgfqpoint{3.627594in}{6.680401in}}{\pgfqpoint{3.631712in}{6.684519in}}%
\pgfpathcurveto{\pgfqpoint{3.635830in}{6.688637in}}{\pgfqpoint{3.638144in}{6.694223in}}{\pgfqpoint{3.638144in}{6.700047in}}%
\pgfpathcurveto{\pgfqpoint{3.638144in}{6.705871in}}{\pgfqpoint{3.635830in}{6.711457in}}{\pgfqpoint{3.631712in}{6.715575in}}%
\pgfpathcurveto{\pgfqpoint{3.627594in}{6.719693in}}{\pgfqpoint{3.622008in}{6.722007in}}{\pgfqpoint{3.616184in}{6.722007in}}%
\pgfpathcurveto{\pgfqpoint{3.610360in}{6.722007in}}{\pgfqpoint{3.604774in}{6.719693in}}{\pgfqpoint{3.600656in}{6.715575in}}%
\pgfpathcurveto{\pgfqpoint{3.596538in}{6.711457in}}{\pgfqpoint{3.594224in}{6.705871in}}{\pgfqpoint{3.594224in}{6.700047in}}%
\pgfpathcurveto{\pgfqpoint{3.594224in}{6.694223in}}{\pgfqpoint{3.596538in}{6.688637in}}{\pgfqpoint{3.600656in}{6.684519in}}%
\pgfpathcurveto{\pgfqpoint{3.604774in}{6.680401in}}{\pgfqpoint{3.610360in}{6.678087in}}{\pgfqpoint{3.616184in}{6.678087in}}%
\pgfpathlineto{\pgfqpoint{3.616184in}{6.678087in}}%
\pgfpathclose%
\pgfusepath{stroke,fill}%
\end{pgfscope}%
\begin{pgfscope}%
\pgfpathrectangle{\pgfqpoint{1.582361in}{0.880000in}}{\pgfqpoint{5.035278in}{6.160000in}}%
\pgfusepath{clip}%
\pgfsetbuttcap%
\pgfsetroundjoin%
\definecolor{currentfill}{rgb}{0.800000,0.200000,0.200000}%
\pgfsetfillcolor{currentfill}%
\pgfsetlinewidth{1.003750pt}%
\definecolor{currentstroke}{rgb}{0.800000,0.200000,0.200000}%
\pgfsetstrokecolor{currentstroke}%
\pgfsetdash{}{0pt}%
\pgfpathmoveto{\pgfqpoint{3.573555in}{6.664533in}}%
\pgfpathcurveto{\pgfqpoint{3.579379in}{6.664533in}}{\pgfqpoint{3.584965in}{6.666846in}}{\pgfqpoint{3.589083in}{6.670965in}}%
\pgfpathcurveto{\pgfqpoint{3.593201in}{6.675083in}}{\pgfqpoint{3.595515in}{6.680669in}}{\pgfqpoint{3.595515in}{6.686493in}}%
\pgfpathcurveto{\pgfqpoint{3.595515in}{6.692317in}}{\pgfqpoint{3.593201in}{6.697903in}}{\pgfqpoint{3.589083in}{6.702021in}}%
\pgfpathcurveto{\pgfqpoint{3.584965in}{6.706139in}}{\pgfqpoint{3.579379in}{6.708453in}}{\pgfqpoint{3.573555in}{6.708453in}}%
\pgfpathcurveto{\pgfqpoint{3.567731in}{6.708453in}}{\pgfqpoint{3.562145in}{6.706139in}}{\pgfqpoint{3.558026in}{6.702021in}}%
\pgfpathcurveto{\pgfqpoint{3.553908in}{6.697903in}}{\pgfqpoint{3.551594in}{6.692317in}}{\pgfqpoint{3.551594in}{6.686493in}}%
\pgfpathcurveto{\pgfqpoint{3.551594in}{6.680669in}}{\pgfqpoint{3.553908in}{6.675083in}}{\pgfqpoint{3.558026in}{6.670965in}}%
\pgfpathcurveto{\pgfqpoint{3.562145in}{6.666846in}}{\pgfqpoint{3.567731in}{6.664533in}}{\pgfqpoint{3.573555in}{6.664533in}}%
\pgfpathlineto{\pgfqpoint{3.573555in}{6.664533in}}%
\pgfpathclose%
\pgfusepath{stroke,fill}%
\end{pgfscope}%
\begin{pgfscope}%
\pgfpathrectangle{\pgfqpoint{1.582361in}{0.880000in}}{\pgfqpoint{5.035278in}{6.160000in}}%
\pgfusepath{clip}%
\pgfsetbuttcap%
\pgfsetroundjoin%
\definecolor{currentfill}{rgb}{0.800000,0.200000,0.200000}%
\pgfsetfillcolor{currentfill}%
\pgfsetlinewidth{1.003750pt}%
\definecolor{currentstroke}{rgb}{0.800000,0.200000,0.200000}%
\pgfsetstrokecolor{currentstroke}%
\pgfsetdash{}{0pt}%
\pgfpathmoveto{\pgfqpoint{3.531374in}{6.649639in}}%
\pgfpathcurveto{\pgfqpoint{3.537198in}{6.649639in}}{\pgfqpoint{3.542784in}{6.651953in}}{\pgfqpoint{3.546903in}{6.656071in}}%
\pgfpathcurveto{\pgfqpoint{3.551021in}{6.660190in}}{\pgfqpoint{3.553335in}{6.665776in}}{\pgfqpoint{3.553335in}{6.671600in}}%
\pgfpathcurveto{\pgfqpoint{3.553335in}{6.677424in}}{\pgfqpoint{3.551021in}{6.683010in}}{\pgfqpoint{3.546903in}{6.687128in}}%
\pgfpathcurveto{\pgfqpoint{3.542784in}{6.691246in}}{\pgfqpoint{3.537198in}{6.693560in}}{\pgfqpoint{3.531374in}{6.693560in}}%
\pgfpathcurveto{\pgfqpoint{3.525550in}{6.693560in}}{\pgfqpoint{3.519964in}{6.691246in}}{\pgfqpoint{3.515846in}{6.687128in}}%
\pgfpathcurveto{\pgfqpoint{3.511728in}{6.683010in}}{\pgfqpoint{3.509414in}{6.677424in}}{\pgfqpoint{3.509414in}{6.671600in}}%
\pgfpathcurveto{\pgfqpoint{3.509414in}{6.665776in}}{\pgfqpoint{3.511728in}{6.660190in}}{\pgfqpoint{3.515846in}{6.656071in}}%
\pgfpathcurveto{\pgfqpoint{3.519964in}{6.651953in}}{\pgfqpoint{3.525550in}{6.649639in}}{\pgfqpoint{3.531374in}{6.649639in}}%
\pgfpathlineto{\pgfqpoint{3.531374in}{6.649639in}}%
\pgfpathclose%
\pgfusepath{stroke,fill}%
\end{pgfscope}%
\begin{pgfscope}%
\pgfpathrectangle{\pgfqpoint{1.582361in}{0.880000in}}{\pgfqpoint{5.035278in}{6.160000in}}%
\pgfusepath{clip}%
\pgfsetbuttcap%
\pgfsetroundjoin%
\definecolor{currentfill}{rgb}{0.800000,0.200000,0.200000}%
\pgfsetfillcolor{currentfill}%
\pgfsetlinewidth{1.003750pt}%
\definecolor{currentstroke}{rgb}{0.800000,0.200000,0.200000}%
\pgfsetstrokecolor{currentstroke}%
\pgfsetdash{}{0pt}%
\pgfpathmoveto{\pgfqpoint{3.489685in}{6.633422in}}%
\pgfpathcurveto{\pgfqpoint{3.495509in}{6.633422in}}{\pgfqpoint{3.501095in}{6.635736in}}{\pgfqpoint{3.505213in}{6.639854in}}%
\pgfpathcurveto{\pgfqpoint{3.509332in}{6.643972in}}{\pgfqpoint{3.511645in}{6.649558in}}{\pgfqpoint{3.511645in}{6.655382in}}%
\pgfpathcurveto{\pgfqpoint{3.511645in}{6.661206in}}{\pgfqpoint{3.509332in}{6.666793in}}{\pgfqpoint{3.505213in}{6.670911in}}%
\pgfpathcurveto{\pgfqpoint{3.501095in}{6.675029in}}{\pgfqpoint{3.495509in}{6.677343in}}{\pgfqpoint{3.489685in}{6.677343in}}%
\pgfpathcurveto{\pgfqpoint{3.483861in}{6.677343in}}{\pgfqpoint{3.478275in}{6.675029in}}{\pgfqpoint{3.474157in}{6.670911in}}%
\pgfpathcurveto{\pgfqpoint{3.470039in}{6.666793in}}{\pgfqpoint{3.467725in}{6.661206in}}{\pgfqpoint{3.467725in}{6.655382in}}%
\pgfpathcurveto{\pgfqpoint{3.467725in}{6.649558in}}{\pgfqpoint{3.470039in}{6.643972in}}{\pgfqpoint{3.474157in}{6.639854in}}%
\pgfpathcurveto{\pgfqpoint{3.478275in}{6.635736in}}{\pgfqpoint{3.483861in}{6.633422in}}{\pgfqpoint{3.489685in}{6.633422in}}%
\pgfpathlineto{\pgfqpoint{3.489685in}{6.633422in}}%
\pgfpathclose%
\pgfusepath{stroke,fill}%
\end{pgfscope}%
\begin{pgfscope}%
\pgfpathrectangle{\pgfqpoint{1.582361in}{0.880000in}}{\pgfqpoint{5.035278in}{6.160000in}}%
\pgfusepath{clip}%
\pgfsetbuttcap%
\pgfsetroundjoin%
\definecolor{currentfill}{rgb}{0.800000,0.200000,0.200000}%
\pgfsetfillcolor{currentfill}%
\pgfsetlinewidth{1.003750pt}%
\definecolor{currentstroke}{rgb}{0.800000,0.200000,0.200000}%
\pgfsetstrokecolor{currentstroke}%
\pgfsetdash{}{0pt}%
\pgfpathmoveto{\pgfqpoint{3.448529in}{6.615897in}}%
\pgfpathcurveto{\pgfqpoint{3.454353in}{6.615897in}}{\pgfqpoint{3.459939in}{6.618211in}}{\pgfqpoint{3.464057in}{6.622329in}}%
\pgfpathcurveto{\pgfqpoint{3.468175in}{6.626447in}}{\pgfqpoint{3.470489in}{6.632033in}}{\pgfqpoint{3.470489in}{6.637857in}}%
\pgfpathcurveto{\pgfqpoint{3.470489in}{6.643681in}}{\pgfqpoint{3.468175in}{6.649267in}}{\pgfqpoint{3.464057in}{6.653385in}}%
\pgfpathcurveto{\pgfqpoint{3.459939in}{6.657504in}}{\pgfqpoint{3.454353in}{6.659817in}}{\pgfqpoint{3.448529in}{6.659817in}}%
\pgfpathcurveto{\pgfqpoint{3.442705in}{6.659817in}}{\pgfqpoint{3.437119in}{6.657504in}}{\pgfqpoint{3.433001in}{6.653385in}}%
\pgfpathcurveto{\pgfqpoint{3.428882in}{6.649267in}}{\pgfqpoint{3.426569in}{6.643681in}}{\pgfqpoint{3.426569in}{6.637857in}}%
\pgfpathcurveto{\pgfqpoint{3.426569in}{6.632033in}}{\pgfqpoint{3.428882in}{6.626447in}}{\pgfqpoint{3.433001in}{6.622329in}}%
\pgfpathcurveto{\pgfqpoint{3.437119in}{6.618211in}}{\pgfqpoint{3.442705in}{6.615897in}}{\pgfqpoint{3.448529in}{6.615897in}}%
\pgfpathlineto{\pgfqpoint{3.448529in}{6.615897in}}%
\pgfpathclose%
\pgfusepath{stroke,fill}%
\end{pgfscope}%
\begin{pgfscope}%
\pgfpathrectangle{\pgfqpoint{1.582361in}{0.880000in}}{\pgfqpoint{5.035278in}{6.160000in}}%
\pgfusepath{clip}%
\pgfsetbuttcap%
\pgfsetroundjoin%
\definecolor{currentfill}{rgb}{0.800000,0.200000,0.200000}%
\pgfsetfillcolor{currentfill}%
\pgfsetlinewidth{1.003750pt}%
\definecolor{currentstroke}{rgb}{0.800000,0.200000,0.200000}%
\pgfsetstrokecolor{currentstroke}%
\pgfsetdash{}{0pt}%
\pgfpathmoveto{\pgfqpoint{3.407946in}{6.597081in}}%
\pgfpathcurveto{\pgfqpoint{3.413770in}{6.597081in}}{\pgfqpoint{3.419356in}{6.599395in}}{\pgfqpoint{3.423474in}{6.603513in}}%
\pgfpathcurveto{\pgfqpoint{3.427593in}{6.607631in}}{\pgfqpoint{3.429906in}{6.613217in}}{\pgfqpoint{3.429906in}{6.619041in}}%
\pgfpathcurveto{\pgfqpoint{3.429906in}{6.624865in}}{\pgfqpoint{3.427593in}{6.630452in}}{\pgfqpoint{3.423474in}{6.634570in}}%
\pgfpathcurveto{\pgfqpoint{3.419356in}{6.638688in}}{\pgfqpoint{3.413770in}{6.641002in}}{\pgfqpoint{3.407946in}{6.641002in}}%
\pgfpathcurveto{\pgfqpoint{3.402122in}{6.641002in}}{\pgfqpoint{3.396536in}{6.638688in}}{\pgfqpoint{3.392418in}{6.634570in}}%
\pgfpathcurveto{\pgfqpoint{3.388300in}{6.630452in}}{\pgfqpoint{3.385986in}{6.624865in}}{\pgfqpoint{3.385986in}{6.619041in}}%
\pgfpathcurveto{\pgfqpoint{3.385986in}{6.613217in}}{\pgfqpoint{3.388300in}{6.607631in}}{\pgfqpoint{3.392418in}{6.603513in}}%
\pgfpathcurveto{\pgfqpoint{3.396536in}{6.599395in}}{\pgfqpoint{3.402122in}{6.597081in}}{\pgfqpoint{3.407946in}{6.597081in}}%
\pgfpathlineto{\pgfqpoint{3.407946in}{6.597081in}}%
\pgfpathclose%
\pgfusepath{stroke,fill}%
\end{pgfscope}%
\begin{pgfscope}%
\pgfpathrectangle{\pgfqpoint{1.582361in}{0.880000in}}{\pgfqpoint{5.035278in}{6.160000in}}%
\pgfusepath{clip}%
\pgfsetbuttcap%
\pgfsetroundjoin%
\definecolor{currentfill}{rgb}{0.800000,0.200000,0.200000}%
\pgfsetfillcolor{currentfill}%
\pgfsetlinewidth{1.003750pt}%
\definecolor{currentstroke}{rgb}{0.800000,0.200000,0.200000}%
\pgfsetstrokecolor{currentstroke}%
\pgfsetdash{}{0pt}%
\pgfpathmoveto{\pgfqpoint{3.367978in}{6.576994in}}%
\pgfpathcurveto{\pgfqpoint{3.373802in}{6.576994in}}{\pgfqpoint{3.379388in}{6.579307in}}{\pgfqpoint{3.383506in}{6.583426in}}%
\pgfpathcurveto{\pgfqpoint{3.387624in}{6.587544in}}{\pgfqpoint{3.389938in}{6.593130in}}{\pgfqpoint{3.389938in}{6.598954in}}%
\pgfpathcurveto{\pgfqpoint{3.389938in}{6.604778in}}{\pgfqpoint{3.387624in}{6.610364in}}{\pgfqpoint{3.383506in}{6.614482in}}%
\pgfpathcurveto{\pgfqpoint{3.379388in}{6.618600in}}{\pgfqpoint{3.373802in}{6.620914in}}{\pgfqpoint{3.367978in}{6.620914in}}%
\pgfpathcurveto{\pgfqpoint{3.362154in}{6.620914in}}{\pgfqpoint{3.356568in}{6.618600in}}{\pgfqpoint{3.352450in}{6.614482in}}%
\pgfpathcurveto{\pgfqpoint{3.348331in}{6.610364in}}{\pgfqpoint{3.346018in}{6.604778in}}{\pgfqpoint{3.346018in}{6.598954in}}%
\pgfpathcurveto{\pgfqpoint{3.346018in}{6.593130in}}{\pgfqpoint{3.348331in}{6.587544in}}{\pgfqpoint{3.352450in}{6.583426in}}%
\pgfpathcurveto{\pgfqpoint{3.356568in}{6.579307in}}{\pgfqpoint{3.362154in}{6.576994in}}{\pgfqpoint{3.367978in}{6.576994in}}%
\pgfpathlineto{\pgfqpoint{3.367978in}{6.576994in}}%
\pgfpathclose%
\pgfusepath{stroke,fill}%
\end{pgfscope}%
\begin{pgfscope}%
\pgfpathrectangle{\pgfqpoint{1.582361in}{0.880000in}}{\pgfqpoint{5.035278in}{6.160000in}}%
\pgfusepath{clip}%
\pgfsetbuttcap%
\pgfsetroundjoin%
\definecolor{currentfill}{rgb}{0.800000,0.200000,0.200000}%
\pgfsetfillcolor{currentfill}%
\pgfsetlinewidth{1.003750pt}%
\definecolor{currentstroke}{rgb}{0.800000,0.200000,0.200000}%
\pgfsetstrokecolor{currentstroke}%
\pgfsetdash{}{0pt}%
\pgfpathmoveto{\pgfqpoint{3.328663in}{6.555654in}}%
\pgfpathcurveto{\pgfqpoint{3.334487in}{6.555654in}}{\pgfqpoint{3.340074in}{6.557968in}}{\pgfqpoint{3.344192in}{6.562086in}}%
\pgfpathcurveto{\pgfqpoint{3.348310in}{6.566204in}}{\pgfqpoint{3.350624in}{6.571791in}}{\pgfqpoint{3.350624in}{6.577615in}}%
\pgfpathcurveto{\pgfqpoint{3.350624in}{6.583439in}}{\pgfqpoint{3.348310in}{6.589025in}}{\pgfqpoint{3.344192in}{6.593143in}}%
\pgfpathcurveto{\pgfqpoint{3.340074in}{6.597261in}}{\pgfqpoint{3.334487in}{6.599575in}}{\pgfqpoint{3.328663in}{6.599575in}}%
\pgfpathcurveto{\pgfqpoint{3.322839in}{6.599575in}}{\pgfqpoint{3.317253in}{6.597261in}}{\pgfqpoint{3.313135in}{6.593143in}}%
\pgfpathcurveto{\pgfqpoint{3.309017in}{6.589025in}}{\pgfqpoint{3.306703in}{6.583439in}}{\pgfqpoint{3.306703in}{6.577615in}}%
\pgfpathcurveto{\pgfqpoint{3.306703in}{6.571791in}}{\pgfqpoint{3.309017in}{6.566204in}}{\pgfqpoint{3.313135in}{6.562086in}}%
\pgfpathcurveto{\pgfqpoint{3.317253in}{6.557968in}}{\pgfqpoint{3.322839in}{6.555654in}}{\pgfqpoint{3.328663in}{6.555654in}}%
\pgfpathlineto{\pgfqpoint{3.328663in}{6.555654in}}%
\pgfpathclose%
\pgfusepath{stroke,fill}%
\end{pgfscope}%
\begin{pgfscope}%
\pgfpathrectangle{\pgfqpoint{1.582361in}{0.880000in}}{\pgfqpoint{5.035278in}{6.160000in}}%
\pgfusepath{clip}%
\pgfsetbuttcap%
\pgfsetroundjoin%
\definecolor{currentfill}{rgb}{0.800000,0.200000,0.200000}%
\pgfsetfillcolor{currentfill}%
\pgfsetlinewidth{1.003750pt}%
\definecolor{currentstroke}{rgb}{0.800000,0.200000,0.200000}%
\pgfsetstrokecolor{currentstroke}%
\pgfsetdash{}{0pt}%
\pgfpathmoveto{\pgfqpoint{3.290042in}{6.533085in}}%
\pgfpathcurveto{\pgfqpoint{3.295866in}{6.533085in}}{\pgfqpoint{3.301452in}{6.535398in}}{\pgfqpoint{3.305571in}{6.539517in}}%
\pgfpathcurveto{\pgfqpoint{3.309689in}{6.543635in}}{\pgfqpoint{3.312003in}{6.549221in}}{\pgfqpoint{3.312003in}{6.555045in}}%
\pgfpathcurveto{\pgfqpoint{3.312003in}{6.560869in}}{\pgfqpoint{3.309689in}{6.566455in}}{\pgfqpoint{3.305571in}{6.570573in}}%
\pgfpathcurveto{\pgfqpoint{3.301452in}{6.574691in}}{\pgfqpoint{3.295866in}{6.577005in}}{\pgfqpoint{3.290042in}{6.577005in}}%
\pgfpathcurveto{\pgfqpoint{3.284218in}{6.577005in}}{\pgfqpoint{3.278632in}{6.574691in}}{\pgfqpoint{3.274514in}{6.570573in}}%
\pgfpathcurveto{\pgfqpoint{3.270396in}{6.566455in}}{\pgfqpoint{3.268082in}{6.560869in}}{\pgfqpoint{3.268082in}{6.555045in}}%
\pgfpathcurveto{\pgfqpoint{3.268082in}{6.549221in}}{\pgfqpoint{3.270396in}{6.543635in}}{\pgfqpoint{3.274514in}{6.539517in}}%
\pgfpathcurveto{\pgfqpoint{3.278632in}{6.535398in}}{\pgfqpoint{3.284218in}{6.533085in}}{\pgfqpoint{3.290042in}{6.533085in}}%
\pgfpathlineto{\pgfqpoint{3.290042in}{6.533085in}}%
\pgfpathclose%
\pgfusepath{stroke,fill}%
\end{pgfscope}%
\begin{pgfscope}%
\pgfpathrectangle{\pgfqpoint{1.582361in}{0.880000in}}{\pgfqpoint{5.035278in}{6.160000in}}%
\pgfusepath{clip}%
\pgfsetbuttcap%
\pgfsetroundjoin%
\definecolor{currentfill}{rgb}{0.800000,0.200000,0.200000}%
\pgfsetfillcolor{currentfill}%
\pgfsetlinewidth{1.003750pt}%
\definecolor{currentstroke}{rgb}{0.800000,0.200000,0.200000}%
\pgfsetstrokecolor{currentstroke}%
\pgfsetdash{}{0pt}%
\pgfpathmoveto{\pgfqpoint{3.252153in}{6.509307in}}%
\pgfpathcurveto{\pgfqpoint{3.257977in}{6.509307in}}{\pgfqpoint{3.263563in}{6.511621in}}{\pgfqpoint{3.267681in}{6.515739in}}%
\pgfpathcurveto{\pgfqpoint{3.271799in}{6.519857in}}{\pgfqpoint{3.274113in}{6.525443in}}{\pgfqpoint{3.274113in}{6.531267in}}%
\pgfpathcurveto{\pgfqpoint{3.274113in}{6.537091in}}{\pgfqpoint{3.271799in}{6.542677in}}{\pgfqpoint{3.267681in}{6.546795in}}%
\pgfpathcurveto{\pgfqpoint{3.263563in}{6.550914in}}{\pgfqpoint{3.257977in}{6.553227in}}{\pgfqpoint{3.252153in}{6.553227in}}%
\pgfpathcurveto{\pgfqpoint{3.246329in}{6.553227in}}{\pgfqpoint{3.240743in}{6.550914in}}{\pgfqpoint{3.236625in}{6.546795in}}%
\pgfpathcurveto{\pgfqpoint{3.232507in}{6.542677in}}{\pgfqpoint{3.230193in}{6.537091in}}{\pgfqpoint{3.230193in}{6.531267in}}%
\pgfpathcurveto{\pgfqpoint{3.230193in}{6.525443in}}{\pgfqpoint{3.232507in}{6.519857in}}{\pgfqpoint{3.236625in}{6.515739in}}%
\pgfpathcurveto{\pgfqpoint{3.240743in}{6.511621in}}{\pgfqpoint{3.246329in}{6.509307in}}{\pgfqpoint{3.252153in}{6.509307in}}%
\pgfpathlineto{\pgfqpoint{3.252153in}{6.509307in}}%
\pgfpathclose%
\pgfusepath{stroke,fill}%
\end{pgfscope}%
\begin{pgfscope}%
\pgfpathrectangle{\pgfqpoint{1.582361in}{0.880000in}}{\pgfqpoint{5.035278in}{6.160000in}}%
\pgfusepath{clip}%
\pgfsetbuttcap%
\pgfsetroundjoin%
\definecolor{currentfill}{rgb}{0.800000,0.200000,0.200000}%
\pgfsetfillcolor{currentfill}%
\pgfsetlinewidth{1.003750pt}%
\definecolor{currentstroke}{rgb}{0.800000,0.200000,0.200000}%
\pgfsetstrokecolor{currentstroke}%
\pgfsetdash{}{0pt}%
\pgfpathmoveto{\pgfqpoint{3.215033in}{6.484345in}}%
\pgfpathcurveto{\pgfqpoint{3.220857in}{6.484345in}}{\pgfqpoint{3.226443in}{6.486659in}}{\pgfqpoint{3.230561in}{6.490777in}}%
\pgfpathcurveto{\pgfqpoint{3.234679in}{6.494895in}}{\pgfqpoint{3.236993in}{6.500481in}}{\pgfqpoint{3.236993in}{6.506305in}}%
\pgfpathcurveto{\pgfqpoint{3.236993in}{6.512129in}}{\pgfqpoint{3.234679in}{6.517715in}}{\pgfqpoint{3.230561in}{6.521833in}}%
\pgfpathcurveto{\pgfqpoint{3.226443in}{6.525952in}}{\pgfqpoint{3.220857in}{6.528266in}}{\pgfqpoint{3.215033in}{6.528266in}}%
\pgfpathcurveto{\pgfqpoint{3.209209in}{6.528266in}}{\pgfqpoint{3.203623in}{6.525952in}}{\pgfqpoint{3.199505in}{6.521833in}}%
\pgfpathcurveto{\pgfqpoint{3.195387in}{6.517715in}}{\pgfqpoint{3.193073in}{6.512129in}}{\pgfqpoint{3.193073in}{6.506305in}}%
\pgfpathcurveto{\pgfqpoint{3.193073in}{6.500481in}}{\pgfqpoint{3.195387in}{6.494895in}}{\pgfqpoint{3.199505in}{6.490777in}}%
\pgfpathcurveto{\pgfqpoint{3.203623in}{6.486659in}}{\pgfqpoint{3.209209in}{6.484345in}}{\pgfqpoint{3.215033in}{6.484345in}}%
\pgfpathlineto{\pgfqpoint{3.215033in}{6.484345in}}%
\pgfpathclose%
\pgfusepath{stroke,fill}%
\end{pgfscope}%
\begin{pgfscope}%
\pgfpathrectangle{\pgfqpoint{1.582361in}{0.880000in}}{\pgfqpoint{5.035278in}{6.160000in}}%
\pgfusepath{clip}%
\pgfsetbuttcap%
\pgfsetroundjoin%
\definecolor{currentfill}{rgb}{0.800000,0.200000,0.200000}%
\pgfsetfillcolor{currentfill}%
\pgfsetlinewidth{1.003750pt}%
\definecolor{currentstroke}{rgb}{0.800000,0.200000,0.200000}%
\pgfsetstrokecolor{currentstroke}%
\pgfsetdash{}{0pt}%
\pgfpathmoveto{\pgfqpoint{3.178720in}{6.458224in}}%
\pgfpathcurveto{\pgfqpoint{3.184544in}{6.458224in}}{\pgfqpoint{3.190130in}{6.460538in}}{\pgfqpoint{3.194248in}{6.464656in}}%
\pgfpathcurveto{\pgfqpoint{3.198366in}{6.468774in}}{\pgfqpoint{3.200680in}{6.474360in}}{\pgfqpoint{3.200680in}{6.480184in}}%
\pgfpathcurveto{\pgfqpoint{3.200680in}{6.486008in}}{\pgfqpoint{3.198366in}{6.491594in}}{\pgfqpoint{3.194248in}{6.495712in}}%
\pgfpathcurveto{\pgfqpoint{3.190130in}{6.499830in}}{\pgfqpoint{3.184544in}{6.502144in}}{\pgfqpoint{3.178720in}{6.502144in}}%
\pgfpathcurveto{\pgfqpoint{3.172896in}{6.502144in}}{\pgfqpoint{3.167310in}{6.499830in}}{\pgfqpoint{3.163191in}{6.495712in}}%
\pgfpathcurveto{\pgfqpoint{3.159073in}{6.491594in}}{\pgfqpoint{3.156759in}{6.486008in}}{\pgfqpoint{3.156759in}{6.480184in}}%
\pgfpathcurveto{\pgfqpoint{3.156759in}{6.474360in}}{\pgfqpoint{3.159073in}{6.468774in}}{\pgfqpoint{3.163191in}{6.464656in}}%
\pgfpathcurveto{\pgfqpoint{3.167310in}{6.460538in}}{\pgfqpoint{3.172896in}{6.458224in}}{\pgfqpoint{3.178720in}{6.458224in}}%
\pgfpathlineto{\pgfqpoint{3.178720in}{6.458224in}}%
\pgfpathclose%
\pgfusepath{stroke,fill}%
\end{pgfscope}%
\begin{pgfscope}%
\pgfpathrectangle{\pgfqpoint{1.582361in}{0.880000in}}{\pgfqpoint{5.035278in}{6.160000in}}%
\pgfusepath{clip}%
\pgfsetbuttcap%
\pgfsetroundjoin%
\definecolor{currentfill}{rgb}{0.800000,0.200000,0.200000}%
\pgfsetfillcolor{currentfill}%
\pgfsetlinewidth{1.003750pt}%
\definecolor{currentstroke}{rgb}{0.800000,0.200000,0.200000}%
\pgfsetstrokecolor{currentstroke}%
\pgfsetdash{}{0pt}%
\pgfpathmoveto{\pgfqpoint{3.143249in}{6.430969in}}%
\pgfpathcurveto{\pgfqpoint{3.149073in}{6.430969in}}{\pgfqpoint{3.154659in}{6.433283in}}{\pgfqpoint{3.158777in}{6.437401in}}%
\pgfpathcurveto{\pgfqpoint{3.162895in}{6.441519in}}{\pgfqpoint{3.165209in}{6.447105in}}{\pgfqpoint{3.165209in}{6.452929in}}%
\pgfpathcurveto{\pgfqpoint{3.165209in}{6.458753in}}{\pgfqpoint{3.162895in}{6.464339in}}{\pgfqpoint{3.158777in}{6.468457in}}%
\pgfpathcurveto{\pgfqpoint{3.154659in}{6.472576in}}{\pgfqpoint{3.149073in}{6.474890in}}{\pgfqpoint{3.143249in}{6.474890in}}%
\pgfpathcurveto{\pgfqpoint{3.137425in}{6.474890in}}{\pgfqpoint{3.131839in}{6.472576in}}{\pgfqpoint{3.127721in}{6.468457in}}%
\pgfpathcurveto{\pgfqpoint{3.123603in}{6.464339in}}{\pgfqpoint{3.121289in}{6.458753in}}{\pgfqpoint{3.121289in}{6.452929in}}%
\pgfpathcurveto{\pgfqpoint{3.121289in}{6.447105in}}{\pgfqpoint{3.123603in}{6.441519in}}{\pgfqpoint{3.127721in}{6.437401in}}%
\pgfpathcurveto{\pgfqpoint{3.131839in}{6.433283in}}{\pgfqpoint{3.137425in}{6.430969in}}{\pgfqpoint{3.143249in}{6.430969in}}%
\pgfpathlineto{\pgfqpoint{3.143249in}{6.430969in}}%
\pgfpathclose%
\pgfusepath{stroke,fill}%
\end{pgfscope}%
\begin{pgfscope}%
\pgfpathrectangle{\pgfqpoint{1.582361in}{0.880000in}}{\pgfqpoint{5.035278in}{6.160000in}}%
\pgfusepath{clip}%
\pgfsetbuttcap%
\pgfsetroundjoin%
\definecolor{currentfill}{rgb}{0.800000,0.200000,0.200000}%
\pgfsetfillcolor{currentfill}%
\pgfsetlinewidth{1.003750pt}%
\definecolor{currentstroke}{rgb}{0.800000,0.200000,0.200000}%
\pgfsetstrokecolor{currentstroke}%
\pgfsetdash{}{0pt}%
\pgfpathmoveto{\pgfqpoint{3.108656in}{6.402608in}}%
\pgfpathcurveto{\pgfqpoint{3.114480in}{6.402608in}}{\pgfqpoint{3.120067in}{6.404922in}}{\pgfqpoint{3.124185in}{6.409040in}}%
\pgfpathcurveto{\pgfqpoint{3.128303in}{6.413158in}}{\pgfqpoint{3.130617in}{6.418744in}}{\pgfqpoint{3.130617in}{6.424568in}}%
\pgfpathcurveto{\pgfqpoint{3.130617in}{6.430392in}}{\pgfqpoint{3.128303in}{6.435979in}}{\pgfqpoint{3.124185in}{6.440097in}}%
\pgfpathcurveto{\pgfqpoint{3.120067in}{6.444215in}}{\pgfqpoint{3.114480in}{6.446529in}}{\pgfqpoint{3.108656in}{6.446529in}}%
\pgfpathcurveto{\pgfqpoint{3.102833in}{6.446529in}}{\pgfqpoint{3.097246in}{6.444215in}}{\pgfqpoint{3.093128in}{6.440097in}}%
\pgfpathcurveto{\pgfqpoint{3.089010in}{6.435979in}}{\pgfqpoint{3.086696in}{6.430392in}}{\pgfqpoint{3.086696in}{6.424568in}}%
\pgfpathcurveto{\pgfqpoint{3.086696in}{6.418744in}}{\pgfqpoint{3.089010in}{6.413158in}}{\pgfqpoint{3.093128in}{6.409040in}}%
\pgfpathcurveto{\pgfqpoint{3.097246in}{6.404922in}}{\pgfqpoint{3.102833in}{6.402608in}}{\pgfqpoint{3.108656in}{6.402608in}}%
\pgfpathlineto{\pgfqpoint{3.108656in}{6.402608in}}%
\pgfpathclose%
\pgfusepath{stroke,fill}%
\end{pgfscope}%
\begin{pgfscope}%
\pgfpathrectangle{\pgfqpoint{1.582361in}{0.880000in}}{\pgfqpoint{5.035278in}{6.160000in}}%
\pgfusepath{clip}%
\pgfsetbuttcap%
\pgfsetroundjoin%
\definecolor{currentfill}{rgb}{0.800000,0.200000,0.200000}%
\pgfsetfillcolor{currentfill}%
\pgfsetlinewidth{1.003750pt}%
\definecolor{currentstroke}{rgb}{0.800000,0.200000,0.200000}%
\pgfsetstrokecolor{currentstroke}%
\pgfsetdash{}{0pt}%
\pgfpathmoveto{\pgfqpoint{3.074976in}{6.373169in}}%
\pgfpathcurveto{\pgfqpoint{3.080800in}{6.373169in}}{\pgfqpoint{3.086387in}{6.375483in}}{\pgfqpoint{3.090505in}{6.379601in}}%
\pgfpathcurveto{\pgfqpoint{3.094623in}{6.383720in}}{\pgfqpoint{3.096937in}{6.389306in}}{\pgfqpoint{3.096937in}{6.395130in}}%
\pgfpathcurveto{\pgfqpoint{3.096937in}{6.400954in}}{\pgfqpoint{3.094623in}{6.406540in}}{\pgfqpoint{3.090505in}{6.410658in}}%
\pgfpathcurveto{\pgfqpoint{3.086387in}{6.414776in}}{\pgfqpoint{3.080800in}{6.417090in}}{\pgfqpoint{3.074976in}{6.417090in}}%
\pgfpathcurveto{\pgfqpoint{3.069153in}{6.417090in}}{\pgfqpoint{3.063566in}{6.414776in}}{\pgfqpoint{3.059448in}{6.410658in}}%
\pgfpathcurveto{\pgfqpoint{3.055330in}{6.406540in}}{\pgfqpoint{3.053016in}{6.400954in}}{\pgfqpoint{3.053016in}{6.395130in}}%
\pgfpathcurveto{\pgfqpoint{3.053016in}{6.389306in}}{\pgfqpoint{3.055330in}{6.383720in}}{\pgfqpoint{3.059448in}{6.379601in}}%
\pgfpathcurveto{\pgfqpoint{3.063566in}{6.375483in}}{\pgfqpoint{3.069153in}{6.373169in}}{\pgfqpoint{3.074976in}{6.373169in}}%
\pgfpathlineto{\pgfqpoint{3.074976in}{6.373169in}}%
\pgfpathclose%
\pgfusepath{stroke,fill}%
\end{pgfscope}%
\begin{pgfscope}%
\pgfpathrectangle{\pgfqpoint{1.582361in}{0.880000in}}{\pgfqpoint{5.035278in}{6.160000in}}%
\pgfusepath{clip}%
\pgfsetbuttcap%
\pgfsetroundjoin%
\definecolor{currentfill}{rgb}{0.800000,0.200000,0.200000}%
\pgfsetfillcolor{currentfill}%
\pgfsetlinewidth{1.003750pt}%
\definecolor{currentstroke}{rgb}{0.800000,0.200000,0.200000}%
\pgfsetstrokecolor{currentstroke}%
\pgfsetdash{}{0pt}%
\pgfpathmoveto{\pgfqpoint{3.042243in}{6.342682in}}%
\pgfpathcurveto{\pgfqpoint{3.048066in}{6.342682in}}{\pgfqpoint{3.053653in}{6.344996in}}{\pgfqpoint{3.057771in}{6.349114in}}%
\pgfpathcurveto{\pgfqpoint{3.061889in}{6.353232in}}{\pgfqpoint{3.064203in}{6.358818in}}{\pgfqpoint{3.064203in}{6.364642in}}%
\pgfpathcurveto{\pgfqpoint{3.064203in}{6.370466in}}{\pgfqpoint{3.061889in}{6.376052in}}{\pgfqpoint{3.057771in}{6.380171in}}%
\pgfpathcurveto{\pgfqpoint{3.053653in}{6.384289in}}{\pgfqpoint{3.048066in}{6.386603in}}{\pgfqpoint{3.042243in}{6.386603in}}%
\pgfpathcurveto{\pgfqpoint{3.036419in}{6.386603in}}{\pgfqpoint{3.030832in}{6.384289in}}{\pgfqpoint{3.026714in}{6.380171in}}%
\pgfpathcurveto{\pgfqpoint{3.022596in}{6.376052in}}{\pgfqpoint{3.020282in}{6.370466in}}{\pgfqpoint{3.020282in}{6.364642in}}%
\pgfpathcurveto{\pgfqpoint{3.020282in}{6.358818in}}{\pgfqpoint{3.022596in}{6.353232in}}{\pgfqpoint{3.026714in}{6.349114in}}%
\pgfpathcurveto{\pgfqpoint{3.030832in}{6.344996in}}{\pgfqpoint{3.036419in}{6.342682in}}{\pgfqpoint{3.042243in}{6.342682in}}%
\pgfpathlineto{\pgfqpoint{3.042243in}{6.342682in}}%
\pgfpathclose%
\pgfusepath{stroke,fill}%
\end{pgfscope}%
\begin{pgfscope}%
\pgfpathrectangle{\pgfqpoint{1.582361in}{0.880000in}}{\pgfqpoint{5.035278in}{6.160000in}}%
\pgfusepath{clip}%
\pgfsetbuttcap%
\pgfsetroundjoin%
\definecolor{currentfill}{rgb}{0.800000,0.200000,0.200000}%
\pgfsetfillcolor{currentfill}%
\pgfsetlinewidth{1.003750pt}%
\definecolor{currentstroke}{rgb}{0.800000,0.200000,0.200000}%
\pgfsetstrokecolor{currentstroke}%
\pgfsetdash{}{0pt}%
\pgfpathmoveto{\pgfqpoint{3.010487in}{6.311177in}}%
\pgfpathcurveto{\pgfqpoint{3.016311in}{6.311177in}}{\pgfqpoint{3.021898in}{6.313491in}}{\pgfqpoint{3.026016in}{6.317609in}}%
\pgfpathcurveto{\pgfqpoint{3.030134in}{6.321727in}}{\pgfqpoint{3.032448in}{6.327313in}}{\pgfqpoint{3.032448in}{6.333137in}}%
\pgfpathcurveto{\pgfqpoint{3.032448in}{6.338961in}}{\pgfqpoint{3.030134in}{6.344547in}}{\pgfqpoint{3.026016in}{6.348665in}}%
\pgfpathcurveto{\pgfqpoint{3.021898in}{6.352783in}}{\pgfqpoint{3.016311in}{6.355097in}}{\pgfqpoint{3.010487in}{6.355097in}}%
\pgfpathcurveto{\pgfqpoint{3.004663in}{6.355097in}}{\pgfqpoint{2.999077in}{6.352783in}}{\pgfqpoint{2.994959in}{6.348665in}}%
\pgfpathcurveto{\pgfqpoint{2.990841in}{6.344547in}}{\pgfqpoint{2.988527in}{6.338961in}}{\pgfqpoint{2.988527in}{6.333137in}}%
\pgfpathcurveto{\pgfqpoint{2.988527in}{6.327313in}}{\pgfqpoint{2.990841in}{6.321727in}}{\pgfqpoint{2.994959in}{6.317609in}}%
\pgfpathcurveto{\pgfqpoint{2.999077in}{6.313491in}}{\pgfqpoint{3.004663in}{6.311177in}}{\pgfqpoint{3.010487in}{6.311177in}}%
\pgfpathlineto{\pgfqpoint{3.010487in}{6.311177in}}%
\pgfpathclose%
\pgfusepath{stroke,fill}%
\end{pgfscope}%
\begin{pgfscope}%
\pgfpathrectangle{\pgfqpoint{1.582361in}{0.880000in}}{\pgfqpoint{5.035278in}{6.160000in}}%
\pgfusepath{clip}%
\pgfsetbuttcap%
\pgfsetroundjoin%
\definecolor{currentfill}{rgb}{0.800000,0.200000,0.200000}%
\pgfsetfillcolor{currentfill}%
\pgfsetlinewidth{1.003750pt}%
\definecolor{currentstroke}{rgb}{0.800000,0.200000,0.200000}%
\pgfsetstrokecolor{currentstroke}%
\pgfsetdash{}{0pt}%
\pgfpathmoveto{\pgfqpoint{2.979743in}{6.278684in}}%
\pgfpathcurveto{\pgfqpoint{2.985567in}{6.278684in}}{\pgfqpoint{2.991153in}{6.280998in}}{\pgfqpoint{2.995271in}{6.285116in}}%
\pgfpathcurveto{\pgfqpoint{2.999389in}{6.289235in}}{\pgfqpoint{3.001703in}{6.294821in}}{\pgfqpoint{3.001703in}{6.300645in}}%
\pgfpathcurveto{\pgfqpoint{3.001703in}{6.306469in}}{\pgfqpoint{2.999389in}{6.312055in}}{\pgfqpoint{2.995271in}{6.316173in}}%
\pgfpathcurveto{\pgfqpoint{2.991153in}{6.320291in}}{\pgfqpoint{2.985567in}{6.322605in}}{\pgfqpoint{2.979743in}{6.322605in}}%
\pgfpathcurveto{\pgfqpoint{2.973919in}{6.322605in}}{\pgfqpoint{2.968333in}{6.320291in}}{\pgfqpoint{2.964214in}{6.316173in}}%
\pgfpathcurveto{\pgfqpoint{2.960096in}{6.312055in}}{\pgfqpoint{2.957782in}{6.306469in}}{\pgfqpoint{2.957782in}{6.300645in}}%
\pgfpathcurveto{\pgfqpoint{2.957782in}{6.294821in}}{\pgfqpoint{2.960096in}{6.289235in}}{\pgfqpoint{2.964214in}{6.285116in}}%
\pgfpathcurveto{\pgfqpoint{2.968333in}{6.280998in}}{\pgfqpoint{2.973919in}{6.278684in}}{\pgfqpoint{2.979743in}{6.278684in}}%
\pgfpathlineto{\pgfqpoint{2.979743in}{6.278684in}}%
\pgfpathclose%
\pgfusepath{stroke,fill}%
\end{pgfscope}%
\begin{pgfscope}%
\pgfpathrectangle{\pgfqpoint{1.582361in}{0.880000in}}{\pgfqpoint{5.035278in}{6.160000in}}%
\pgfusepath{clip}%
\pgfsetbuttcap%
\pgfsetroundjoin%
\definecolor{currentfill}{rgb}{0.800000,0.200000,0.200000}%
\pgfsetfillcolor{currentfill}%
\pgfsetlinewidth{1.003750pt}%
\definecolor{currentstroke}{rgb}{0.800000,0.200000,0.200000}%
\pgfsetstrokecolor{currentstroke}%
\pgfsetdash{}{0pt}%
\pgfpathmoveto{\pgfqpoint{2.950039in}{6.245238in}}%
\pgfpathcurveto{\pgfqpoint{2.955863in}{6.245238in}}{\pgfqpoint{2.961449in}{6.247552in}}{\pgfqpoint{2.965567in}{6.251670in}}%
\pgfpathcurveto{\pgfqpoint{2.969685in}{6.255788in}}{\pgfqpoint{2.971999in}{6.261374in}}{\pgfqpoint{2.971999in}{6.267198in}}%
\pgfpathcurveto{\pgfqpoint{2.971999in}{6.273022in}}{\pgfqpoint{2.969685in}{6.278608in}}{\pgfqpoint{2.965567in}{6.282726in}}%
\pgfpathcurveto{\pgfqpoint{2.961449in}{6.286844in}}{\pgfqpoint{2.955863in}{6.289158in}}{\pgfqpoint{2.950039in}{6.289158in}}%
\pgfpathcurveto{\pgfqpoint{2.944215in}{6.289158in}}{\pgfqpoint{2.938629in}{6.286844in}}{\pgfqpoint{2.934511in}{6.282726in}}%
\pgfpathcurveto{\pgfqpoint{2.930393in}{6.278608in}}{\pgfqpoint{2.928079in}{6.273022in}}{\pgfqpoint{2.928079in}{6.267198in}}%
\pgfpathcurveto{\pgfqpoint{2.928079in}{6.261374in}}{\pgfqpoint{2.930393in}{6.255788in}}{\pgfqpoint{2.934511in}{6.251670in}}%
\pgfpathcurveto{\pgfqpoint{2.938629in}{6.247552in}}{\pgfqpoint{2.944215in}{6.245238in}}{\pgfqpoint{2.950039in}{6.245238in}}%
\pgfpathlineto{\pgfqpoint{2.950039in}{6.245238in}}%
\pgfpathclose%
\pgfusepath{stroke,fill}%
\end{pgfscope}%
\begin{pgfscope}%
\pgfpathrectangle{\pgfqpoint{1.582361in}{0.880000in}}{\pgfqpoint{5.035278in}{6.160000in}}%
\pgfusepath{clip}%
\pgfsetbuttcap%
\pgfsetroundjoin%
\definecolor{currentfill}{rgb}{0.800000,0.200000,0.200000}%
\pgfsetfillcolor{currentfill}%
\pgfsetlinewidth{1.003750pt}%
\definecolor{currentstroke}{rgb}{0.800000,0.200000,0.200000}%
\pgfsetstrokecolor{currentstroke}%
\pgfsetdash{}{0pt}%
\pgfpathmoveto{\pgfqpoint{2.921406in}{6.210870in}}%
\pgfpathcurveto{\pgfqpoint{2.927230in}{6.210870in}}{\pgfqpoint{2.932816in}{6.213184in}}{\pgfqpoint{2.936934in}{6.217302in}}%
\pgfpathcurveto{\pgfqpoint{2.941052in}{6.221420in}}{\pgfqpoint{2.943366in}{6.227007in}}{\pgfqpoint{2.943366in}{6.232830in}}%
\pgfpathcurveto{\pgfqpoint{2.943366in}{6.238654in}}{\pgfqpoint{2.941052in}{6.244241in}}{\pgfqpoint{2.936934in}{6.248359in}}%
\pgfpathcurveto{\pgfqpoint{2.932816in}{6.252477in}}{\pgfqpoint{2.927230in}{6.254791in}}{\pgfqpoint{2.921406in}{6.254791in}}%
\pgfpathcurveto{\pgfqpoint{2.915582in}{6.254791in}}{\pgfqpoint{2.909996in}{6.252477in}}{\pgfqpoint{2.905878in}{6.248359in}}%
\pgfpathcurveto{\pgfqpoint{2.901760in}{6.244241in}}{\pgfqpoint{2.899446in}{6.238654in}}{\pgfqpoint{2.899446in}{6.232830in}}%
\pgfpathcurveto{\pgfqpoint{2.899446in}{6.227007in}}{\pgfqpoint{2.901760in}{6.221420in}}{\pgfqpoint{2.905878in}{6.217302in}}%
\pgfpathcurveto{\pgfqpoint{2.909996in}{6.213184in}}{\pgfqpoint{2.915582in}{6.210870in}}{\pgfqpoint{2.921406in}{6.210870in}}%
\pgfpathlineto{\pgfqpoint{2.921406in}{6.210870in}}%
\pgfpathclose%
\pgfusepath{stroke,fill}%
\end{pgfscope}%
\begin{pgfscope}%
\pgfpathrectangle{\pgfqpoint{1.582361in}{0.880000in}}{\pgfqpoint{5.035278in}{6.160000in}}%
\pgfusepath{clip}%
\pgfsetbuttcap%
\pgfsetroundjoin%
\definecolor{currentfill}{rgb}{0.800000,0.200000,0.200000}%
\pgfsetfillcolor{currentfill}%
\pgfsetlinewidth{1.003750pt}%
\definecolor{currentstroke}{rgb}{0.800000,0.200000,0.200000}%
\pgfsetstrokecolor{currentstroke}%
\pgfsetdash{}{0pt}%
\pgfpathmoveto{\pgfqpoint{2.893872in}{6.175616in}}%
\pgfpathcurveto{\pgfqpoint{2.899696in}{6.175616in}}{\pgfqpoint{2.905282in}{6.177930in}}{\pgfqpoint{2.909400in}{6.182048in}}%
\pgfpathcurveto{\pgfqpoint{2.913519in}{6.186166in}}{\pgfqpoint{2.915832in}{6.191752in}}{\pgfqpoint{2.915832in}{6.197576in}}%
\pgfpathcurveto{\pgfqpoint{2.915832in}{6.203400in}}{\pgfqpoint{2.913519in}{6.208986in}}{\pgfqpoint{2.909400in}{6.213104in}}%
\pgfpathcurveto{\pgfqpoint{2.905282in}{6.217222in}}{\pgfqpoint{2.899696in}{6.219536in}}{\pgfqpoint{2.893872in}{6.219536in}}%
\pgfpathcurveto{\pgfqpoint{2.888048in}{6.219536in}}{\pgfqpoint{2.882462in}{6.217222in}}{\pgfqpoint{2.878344in}{6.213104in}}%
\pgfpathcurveto{\pgfqpoint{2.874226in}{6.208986in}}{\pgfqpoint{2.871912in}{6.203400in}}{\pgfqpoint{2.871912in}{6.197576in}}%
\pgfpathcurveto{\pgfqpoint{2.871912in}{6.191752in}}{\pgfqpoint{2.874226in}{6.186166in}}{\pgfqpoint{2.878344in}{6.182048in}}%
\pgfpathcurveto{\pgfqpoint{2.882462in}{6.177930in}}{\pgfqpoint{2.888048in}{6.175616in}}{\pgfqpoint{2.893872in}{6.175616in}}%
\pgfpathlineto{\pgfqpoint{2.893872in}{6.175616in}}%
\pgfpathclose%
\pgfusepath{stroke,fill}%
\end{pgfscope}%
\begin{pgfscope}%
\pgfpathrectangle{\pgfqpoint{1.582361in}{0.880000in}}{\pgfqpoint{5.035278in}{6.160000in}}%
\pgfusepath{clip}%
\pgfsetbuttcap%
\pgfsetroundjoin%
\definecolor{currentfill}{rgb}{0.800000,0.200000,0.200000}%
\pgfsetfillcolor{currentfill}%
\pgfsetlinewidth{1.003750pt}%
\definecolor{currentstroke}{rgb}{0.800000,0.200000,0.200000}%
\pgfsetstrokecolor{currentstroke}%
\pgfsetdash{}{0pt}%
\pgfpathmoveto{\pgfqpoint{2.867465in}{6.139510in}}%
\pgfpathcurveto{\pgfqpoint{2.873289in}{6.139510in}}{\pgfqpoint{2.878875in}{6.141824in}}{\pgfqpoint{2.882993in}{6.145942in}}%
\pgfpathcurveto{\pgfqpoint{2.887111in}{6.150060in}}{\pgfqpoint{2.889425in}{6.155646in}}{\pgfqpoint{2.889425in}{6.161470in}}%
\pgfpathcurveto{\pgfqpoint{2.889425in}{6.167294in}}{\pgfqpoint{2.887111in}{6.172880in}}{\pgfqpoint{2.882993in}{6.176998in}}%
\pgfpathcurveto{\pgfqpoint{2.878875in}{6.181116in}}{\pgfqpoint{2.873289in}{6.183430in}}{\pgfqpoint{2.867465in}{6.183430in}}%
\pgfpathcurveto{\pgfqpoint{2.861641in}{6.183430in}}{\pgfqpoint{2.856055in}{6.181116in}}{\pgfqpoint{2.851937in}{6.176998in}}%
\pgfpathcurveto{\pgfqpoint{2.847819in}{6.172880in}}{\pgfqpoint{2.845505in}{6.167294in}}{\pgfqpoint{2.845505in}{6.161470in}}%
\pgfpathcurveto{\pgfqpoint{2.845505in}{6.155646in}}{\pgfqpoint{2.847819in}{6.150060in}}{\pgfqpoint{2.851937in}{6.145942in}}%
\pgfpathcurveto{\pgfqpoint{2.856055in}{6.141824in}}{\pgfqpoint{2.861641in}{6.139510in}}{\pgfqpoint{2.867465in}{6.139510in}}%
\pgfpathlineto{\pgfqpoint{2.867465in}{6.139510in}}%
\pgfpathclose%
\pgfusepath{stroke,fill}%
\end{pgfscope}%
\begin{pgfscope}%
\pgfpathrectangle{\pgfqpoint{1.582361in}{0.880000in}}{\pgfqpoint{5.035278in}{6.160000in}}%
\pgfusepath{clip}%
\pgfsetbuttcap%
\pgfsetroundjoin%
\definecolor{currentfill}{rgb}{0.800000,0.200000,0.200000}%
\pgfsetfillcolor{currentfill}%
\pgfsetlinewidth{1.003750pt}%
\definecolor{currentstroke}{rgb}{0.800000,0.200000,0.200000}%
\pgfsetstrokecolor{currentstroke}%
\pgfsetdash{}{0pt}%
\pgfpathmoveto{\pgfqpoint{2.842211in}{6.102588in}}%
\pgfpathcurveto{\pgfqpoint{2.848035in}{6.102588in}}{\pgfqpoint{2.853621in}{6.104902in}}{\pgfqpoint{2.857739in}{6.109020in}}%
\pgfpathcurveto{\pgfqpoint{2.861857in}{6.113138in}}{\pgfqpoint{2.864171in}{6.118724in}}{\pgfqpoint{2.864171in}{6.124548in}}%
\pgfpathcurveto{\pgfqpoint{2.864171in}{6.130372in}}{\pgfqpoint{2.861857in}{6.135958in}}{\pgfqpoint{2.857739in}{6.140077in}}%
\pgfpathcurveto{\pgfqpoint{2.853621in}{6.144195in}}{\pgfqpoint{2.848035in}{6.146509in}}{\pgfqpoint{2.842211in}{6.146509in}}%
\pgfpathcurveto{\pgfqpoint{2.836387in}{6.146509in}}{\pgfqpoint{2.830801in}{6.144195in}}{\pgfqpoint{2.826683in}{6.140077in}}%
\pgfpathcurveto{\pgfqpoint{2.822564in}{6.135958in}}{\pgfqpoint{2.820251in}{6.130372in}}{\pgfqpoint{2.820251in}{6.124548in}}%
\pgfpathcurveto{\pgfqpoint{2.820251in}{6.118724in}}{\pgfqpoint{2.822564in}{6.113138in}}{\pgfqpoint{2.826683in}{6.109020in}}%
\pgfpathcurveto{\pgfqpoint{2.830801in}{6.104902in}}{\pgfqpoint{2.836387in}{6.102588in}}{\pgfqpoint{2.842211in}{6.102588in}}%
\pgfpathlineto{\pgfqpoint{2.842211in}{6.102588in}}%
\pgfpathclose%
\pgfusepath{stroke,fill}%
\end{pgfscope}%
\begin{pgfscope}%
\pgfpathrectangle{\pgfqpoint{1.582361in}{0.880000in}}{\pgfqpoint{5.035278in}{6.160000in}}%
\pgfusepath{clip}%
\pgfsetbuttcap%
\pgfsetroundjoin%
\definecolor{currentfill}{rgb}{0.800000,0.200000,0.200000}%
\pgfsetfillcolor{currentfill}%
\pgfsetlinewidth{1.003750pt}%
\definecolor{currentstroke}{rgb}{0.800000,0.200000,0.200000}%
\pgfsetstrokecolor{currentstroke}%
\pgfsetdash{}{0pt}%
\pgfpathmoveto{\pgfqpoint{2.818135in}{6.064888in}}%
\pgfpathcurveto{\pgfqpoint{2.823959in}{6.064888in}}{\pgfqpoint{2.829545in}{6.067201in}}{\pgfqpoint{2.833663in}{6.071320in}}%
\pgfpathcurveto{\pgfqpoint{2.837781in}{6.075438in}}{\pgfqpoint{2.840095in}{6.081024in}}{\pgfqpoint{2.840095in}{6.086848in}}%
\pgfpathcurveto{\pgfqpoint{2.840095in}{6.092672in}}{\pgfqpoint{2.837781in}{6.098258in}}{\pgfqpoint{2.833663in}{6.102376in}}%
\pgfpathcurveto{\pgfqpoint{2.829545in}{6.106494in}}{\pgfqpoint{2.823959in}{6.108808in}}{\pgfqpoint{2.818135in}{6.108808in}}%
\pgfpathcurveto{\pgfqpoint{2.812311in}{6.108808in}}{\pgfqpoint{2.806725in}{6.106494in}}{\pgfqpoint{2.802607in}{6.102376in}}%
\pgfpathcurveto{\pgfqpoint{2.798488in}{6.098258in}}{\pgfqpoint{2.796175in}{6.092672in}}{\pgfqpoint{2.796175in}{6.086848in}}%
\pgfpathcurveto{\pgfqpoint{2.796175in}{6.081024in}}{\pgfqpoint{2.798488in}{6.075438in}}{\pgfqpoint{2.802607in}{6.071320in}}%
\pgfpathcurveto{\pgfqpoint{2.806725in}{6.067201in}}{\pgfqpoint{2.812311in}{6.064888in}}{\pgfqpoint{2.818135in}{6.064888in}}%
\pgfpathlineto{\pgfqpoint{2.818135in}{6.064888in}}%
\pgfpathclose%
\pgfusepath{stroke,fill}%
\end{pgfscope}%
\begin{pgfscope}%
\pgfpathrectangle{\pgfqpoint{1.582361in}{0.880000in}}{\pgfqpoint{5.035278in}{6.160000in}}%
\pgfusepath{clip}%
\pgfsetbuttcap%
\pgfsetroundjoin%
\definecolor{currentfill}{rgb}{0.800000,0.200000,0.200000}%
\pgfsetfillcolor{currentfill}%
\pgfsetlinewidth{1.003750pt}%
\definecolor{currentstroke}{rgb}{0.800000,0.200000,0.200000}%
\pgfsetstrokecolor{currentstroke}%
\pgfsetdash{}{0pt}%
\pgfpathmoveto{\pgfqpoint{2.795261in}{6.026446in}}%
\pgfpathcurveto{\pgfqpoint{2.801085in}{6.026446in}}{\pgfqpoint{2.806671in}{6.028760in}}{\pgfqpoint{2.810789in}{6.032878in}}%
\pgfpathcurveto{\pgfqpoint{2.814907in}{6.036996in}}{\pgfqpoint{2.817221in}{6.042582in}}{\pgfqpoint{2.817221in}{6.048406in}}%
\pgfpathcurveto{\pgfqpoint{2.817221in}{6.054230in}}{\pgfqpoint{2.814907in}{6.059816in}}{\pgfqpoint{2.810789in}{6.063934in}}%
\pgfpathcurveto{\pgfqpoint{2.806671in}{6.068052in}}{\pgfqpoint{2.801085in}{6.070366in}}{\pgfqpoint{2.795261in}{6.070366in}}%
\pgfpathcurveto{\pgfqpoint{2.789437in}{6.070366in}}{\pgfqpoint{2.783851in}{6.068052in}}{\pgfqpoint{2.779733in}{6.063934in}}%
\pgfpathcurveto{\pgfqpoint{2.775615in}{6.059816in}}{\pgfqpoint{2.773301in}{6.054230in}}{\pgfqpoint{2.773301in}{6.048406in}}%
\pgfpathcurveto{\pgfqpoint{2.773301in}{6.042582in}}{\pgfqpoint{2.775615in}{6.036996in}}{\pgfqpoint{2.779733in}{6.032878in}}%
\pgfpathcurveto{\pgfqpoint{2.783851in}{6.028760in}}{\pgfqpoint{2.789437in}{6.026446in}}{\pgfqpoint{2.795261in}{6.026446in}}%
\pgfpathlineto{\pgfqpoint{2.795261in}{6.026446in}}%
\pgfpathclose%
\pgfusepath{stroke,fill}%
\end{pgfscope}%
\begin{pgfscope}%
\pgfpathrectangle{\pgfqpoint{1.582361in}{0.880000in}}{\pgfqpoint{5.035278in}{6.160000in}}%
\pgfusepath{clip}%
\pgfsetbuttcap%
\pgfsetroundjoin%
\definecolor{currentfill}{rgb}{0.800000,0.200000,0.200000}%
\pgfsetfillcolor{currentfill}%
\pgfsetlinewidth{1.003750pt}%
\definecolor{currentstroke}{rgb}{0.800000,0.200000,0.200000}%
\pgfsetstrokecolor{currentstroke}%
\pgfsetdash{}{0pt}%
\pgfpathmoveto{\pgfqpoint{2.773612in}{5.987301in}}%
\pgfpathcurveto{\pgfqpoint{2.779436in}{5.987301in}}{\pgfqpoint{2.785022in}{5.989615in}}{\pgfqpoint{2.789140in}{5.993733in}}%
\pgfpathcurveto{\pgfqpoint{2.793258in}{5.997851in}}{\pgfqpoint{2.795572in}{6.003437in}}{\pgfqpoint{2.795572in}{6.009261in}}%
\pgfpathcurveto{\pgfqpoint{2.795572in}{6.015085in}}{\pgfqpoint{2.793258in}{6.020671in}}{\pgfqpoint{2.789140in}{6.024790in}}%
\pgfpathcurveto{\pgfqpoint{2.785022in}{6.028908in}}{\pgfqpoint{2.779436in}{6.031222in}}{\pgfqpoint{2.773612in}{6.031222in}}%
\pgfpathcurveto{\pgfqpoint{2.767788in}{6.031222in}}{\pgfqpoint{2.762202in}{6.028908in}}{\pgfqpoint{2.758084in}{6.024790in}}%
\pgfpathcurveto{\pgfqpoint{2.753966in}{6.020671in}}{\pgfqpoint{2.751652in}{6.015085in}}{\pgfqpoint{2.751652in}{6.009261in}}%
\pgfpathcurveto{\pgfqpoint{2.751652in}{6.003437in}}{\pgfqpoint{2.753966in}{5.997851in}}{\pgfqpoint{2.758084in}{5.993733in}}%
\pgfpathcurveto{\pgfqpoint{2.762202in}{5.989615in}}{\pgfqpoint{2.767788in}{5.987301in}}{\pgfqpoint{2.773612in}{5.987301in}}%
\pgfpathlineto{\pgfqpoint{2.773612in}{5.987301in}}%
\pgfpathclose%
\pgfusepath{stroke,fill}%
\end{pgfscope}%
\begin{pgfscope}%
\pgfpathrectangle{\pgfqpoint{1.582361in}{0.880000in}}{\pgfqpoint{5.035278in}{6.160000in}}%
\pgfusepath{clip}%
\pgfsetbuttcap%
\pgfsetroundjoin%
\definecolor{currentfill}{rgb}{0.800000,0.200000,0.200000}%
\pgfsetfillcolor{currentfill}%
\pgfsetlinewidth{1.003750pt}%
\definecolor{currentstroke}{rgb}{0.800000,0.200000,0.200000}%
\pgfsetstrokecolor{currentstroke}%
\pgfsetdash{}{0pt}%
\pgfpathmoveto{\pgfqpoint{2.753210in}{5.947492in}}%
\pgfpathcurveto{\pgfqpoint{2.759034in}{5.947492in}}{\pgfqpoint{2.764620in}{5.949806in}}{\pgfqpoint{2.768738in}{5.953924in}}%
\pgfpathcurveto{\pgfqpoint{2.772856in}{5.958043in}}{\pgfqpoint{2.775170in}{5.963629in}}{\pgfqpoint{2.775170in}{5.969453in}}%
\pgfpathcurveto{\pgfqpoint{2.775170in}{5.975277in}}{\pgfqpoint{2.772856in}{5.980863in}}{\pgfqpoint{2.768738in}{5.984981in}}%
\pgfpathcurveto{\pgfqpoint{2.764620in}{5.989099in}}{\pgfqpoint{2.759034in}{5.991413in}}{\pgfqpoint{2.753210in}{5.991413in}}%
\pgfpathcurveto{\pgfqpoint{2.747386in}{5.991413in}}{\pgfqpoint{2.741800in}{5.989099in}}{\pgfqpoint{2.737681in}{5.984981in}}%
\pgfpathcurveto{\pgfqpoint{2.733563in}{5.980863in}}{\pgfqpoint{2.731249in}{5.975277in}}{\pgfqpoint{2.731249in}{5.969453in}}%
\pgfpathcurveto{\pgfqpoint{2.731249in}{5.963629in}}{\pgfqpoint{2.733563in}{5.958043in}}{\pgfqpoint{2.737681in}{5.953924in}}%
\pgfpathcurveto{\pgfqpoint{2.741800in}{5.949806in}}{\pgfqpoint{2.747386in}{5.947492in}}{\pgfqpoint{2.753210in}{5.947492in}}%
\pgfpathlineto{\pgfqpoint{2.753210in}{5.947492in}}%
\pgfpathclose%
\pgfusepath{stroke,fill}%
\end{pgfscope}%
\begin{pgfscope}%
\pgfpathrectangle{\pgfqpoint{1.582361in}{0.880000in}}{\pgfqpoint{5.035278in}{6.160000in}}%
\pgfusepath{clip}%
\pgfsetbuttcap%
\pgfsetroundjoin%
\definecolor{currentfill}{rgb}{0.800000,0.200000,0.200000}%
\pgfsetfillcolor{currentfill}%
\pgfsetlinewidth{1.003750pt}%
\definecolor{currentstroke}{rgb}{0.800000,0.200000,0.200000}%
\pgfsetstrokecolor{currentstroke}%
\pgfsetdash{}{0pt}%
\pgfpathmoveto{\pgfqpoint{2.734074in}{5.907060in}}%
\pgfpathcurveto{\pgfqpoint{2.739898in}{5.907060in}}{\pgfqpoint{2.745484in}{5.909373in}}{\pgfqpoint{2.749602in}{5.913492in}}%
\pgfpathcurveto{\pgfqpoint{2.753721in}{5.917610in}}{\pgfqpoint{2.756034in}{5.923196in}}{\pgfqpoint{2.756034in}{5.929020in}}%
\pgfpathcurveto{\pgfqpoint{2.756034in}{5.934844in}}{\pgfqpoint{2.753721in}{5.940430in}}{\pgfqpoint{2.749602in}{5.944548in}}%
\pgfpathcurveto{\pgfqpoint{2.745484in}{5.948666in}}{\pgfqpoint{2.739898in}{5.950980in}}{\pgfqpoint{2.734074in}{5.950980in}}%
\pgfpathcurveto{\pgfqpoint{2.728250in}{5.950980in}}{\pgfqpoint{2.722664in}{5.948666in}}{\pgfqpoint{2.718546in}{5.944548in}}%
\pgfpathcurveto{\pgfqpoint{2.714428in}{5.940430in}}{\pgfqpoint{2.712114in}{5.934844in}}{\pgfqpoint{2.712114in}{5.929020in}}%
\pgfpathcurveto{\pgfqpoint{2.712114in}{5.923196in}}{\pgfqpoint{2.714428in}{5.917610in}}{\pgfqpoint{2.718546in}{5.913492in}}%
\pgfpathcurveto{\pgfqpoint{2.722664in}{5.909373in}}{\pgfqpoint{2.728250in}{5.907060in}}{\pgfqpoint{2.734074in}{5.907060in}}%
\pgfpathlineto{\pgfqpoint{2.734074in}{5.907060in}}%
\pgfpathclose%
\pgfusepath{stroke,fill}%
\end{pgfscope}%
\begin{pgfscope}%
\pgfpathrectangle{\pgfqpoint{1.582361in}{0.880000in}}{\pgfqpoint{5.035278in}{6.160000in}}%
\pgfusepath{clip}%
\pgfsetbuttcap%
\pgfsetroundjoin%
\definecolor{currentfill}{rgb}{0.800000,0.200000,0.200000}%
\pgfsetfillcolor{currentfill}%
\pgfsetlinewidth{1.003750pt}%
\definecolor{currentstroke}{rgb}{0.800000,0.200000,0.200000}%
\pgfsetstrokecolor{currentstroke}%
\pgfsetdash{}{0pt}%
\pgfpathmoveto{\pgfqpoint{2.716225in}{5.866043in}}%
\pgfpathcurveto{\pgfqpoint{2.722049in}{5.866043in}}{\pgfqpoint{2.727635in}{5.868357in}}{\pgfqpoint{2.731753in}{5.872475in}}%
\pgfpathcurveto{\pgfqpoint{2.735871in}{5.876593in}}{\pgfqpoint{2.738185in}{5.882179in}}{\pgfqpoint{2.738185in}{5.888003in}}%
\pgfpathcurveto{\pgfqpoint{2.738185in}{5.893827in}}{\pgfqpoint{2.735871in}{5.899413in}}{\pgfqpoint{2.731753in}{5.903531in}}%
\pgfpathcurveto{\pgfqpoint{2.727635in}{5.907649in}}{\pgfqpoint{2.722049in}{5.909963in}}{\pgfqpoint{2.716225in}{5.909963in}}%
\pgfpathcurveto{\pgfqpoint{2.710401in}{5.909963in}}{\pgfqpoint{2.704814in}{5.907649in}}{\pgfqpoint{2.700696in}{5.903531in}}%
\pgfpathcurveto{\pgfqpoint{2.696578in}{5.899413in}}{\pgfqpoint{2.694264in}{5.893827in}}{\pgfqpoint{2.694264in}{5.888003in}}%
\pgfpathcurveto{\pgfqpoint{2.694264in}{5.882179in}}{\pgfqpoint{2.696578in}{5.876593in}}{\pgfqpoint{2.700696in}{5.872475in}}%
\pgfpathcurveto{\pgfqpoint{2.704814in}{5.868357in}}{\pgfqpoint{2.710401in}{5.866043in}}{\pgfqpoint{2.716225in}{5.866043in}}%
\pgfpathlineto{\pgfqpoint{2.716225in}{5.866043in}}%
\pgfpathclose%
\pgfusepath{stroke,fill}%
\end{pgfscope}%
\begin{pgfscope}%
\pgfpathrectangle{\pgfqpoint{1.582361in}{0.880000in}}{\pgfqpoint{5.035278in}{6.160000in}}%
\pgfusepath{clip}%
\pgfsetbuttcap%
\pgfsetroundjoin%
\definecolor{currentfill}{rgb}{0.800000,0.200000,0.200000}%
\pgfsetfillcolor{currentfill}%
\pgfsetlinewidth{1.003750pt}%
\definecolor{currentstroke}{rgb}{0.800000,0.200000,0.200000}%
\pgfsetstrokecolor{currentstroke}%
\pgfsetdash{}{0pt}%
\pgfpathmoveto{\pgfqpoint{2.699679in}{5.824483in}}%
\pgfpathcurveto{\pgfqpoint{2.705503in}{5.824483in}}{\pgfqpoint{2.711089in}{5.826797in}}{\pgfqpoint{2.715207in}{5.830915in}}%
\pgfpathcurveto{\pgfqpoint{2.719325in}{5.835033in}}{\pgfqpoint{2.721639in}{5.840619in}}{\pgfqpoint{2.721639in}{5.846443in}}%
\pgfpathcurveto{\pgfqpoint{2.721639in}{5.852267in}}{\pgfqpoint{2.719325in}{5.857853in}}{\pgfqpoint{2.715207in}{5.861972in}}%
\pgfpathcurveto{\pgfqpoint{2.711089in}{5.866090in}}{\pgfqpoint{2.705503in}{5.868404in}}{\pgfqpoint{2.699679in}{5.868404in}}%
\pgfpathcurveto{\pgfqpoint{2.693855in}{5.868404in}}{\pgfqpoint{2.688269in}{5.866090in}}{\pgfqpoint{2.684150in}{5.861972in}}%
\pgfpathcurveto{\pgfqpoint{2.680032in}{5.857853in}}{\pgfqpoint{2.677718in}{5.852267in}}{\pgfqpoint{2.677718in}{5.846443in}}%
\pgfpathcurveto{\pgfqpoint{2.677718in}{5.840619in}}{\pgfqpoint{2.680032in}{5.835033in}}{\pgfqpoint{2.684150in}{5.830915in}}%
\pgfpathcurveto{\pgfqpoint{2.688269in}{5.826797in}}{\pgfqpoint{2.693855in}{5.824483in}}{\pgfqpoint{2.699679in}{5.824483in}}%
\pgfpathlineto{\pgfqpoint{2.699679in}{5.824483in}}%
\pgfpathclose%
\pgfusepath{stroke,fill}%
\end{pgfscope}%
\begin{pgfscope}%
\pgfpathrectangle{\pgfqpoint{1.582361in}{0.880000in}}{\pgfqpoint{5.035278in}{6.160000in}}%
\pgfusepath{clip}%
\pgfsetbuttcap%
\pgfsetroundjoin%
\definecolor{currentfill}{rgb}{0.800000,0.200000,0.200000}%
\pgfsetfillcolor{currentfill}%
\pgfsetlinewidth{1.003750pt}%
\definecolor{currentstroke}{rgb}{0.800000,0.200000,0.200000}%
\pgfsetstrokecolor{currentstroke}%
\pgfsetdash{}{0pt}%
\pgfpathmoveto{\pgfqpoint{2.684453in}{5.782422in}}%
\pgfpathcurveto{\pgfqpoint{2.690277in}{5.782422in}}{\pgfqpoint{2.695863in}{5.784735in}}{\pgfqpoint{2.699981in}{5.788854in}}%
\pgfpathcurveto{\pgfqpoint{2.704100in}{5.792972in}}{\pgfqpoint{2.706413in}{5.798558in}}{\pgfqpoint{2.706413in}{5.804382in}}%
\pgfpathcurveto{\pgfqpoint{2.706413in}{5.810206in}}{\pgfqpoint{2.704100in}{5.815792in}}{\pgfqpoint{2.699981in}{5.819910in}}%
\pgfpathcurveto{\pgfqpoint{2.695863in}{5.824028in}}{\pgfqpoint{2.690277in}{5.826342in}}{\pgfqpoint{2.684453in}{5.826342in}}%
\pgfpathcurveto{\pgfqpoint{2.678629in}{5.826342in}}{\pgfqpoint{2.673043in}{5.824028in}}{\pgfqpoint{2.668925in}{5.819910in}}%
\pgfpathcurveto{\pgfqpoint{2.664807in}{5.815792in}}{\pgfqpoint{2.662493in}{5.810206in}}{\pgfqpoint{2.662493in}{5.804382in}}%
\pgfpathcurveto{\pgfqpoint{2.662493in}{5.798558in}}{\pgfqpoint{2.664807in}{5.792972in}}{\pgfqpoint{2.668925in}{5.788854in}}%
\pgfpathcurveto{\pgfqpoint{2.673043in}{5.784735in}}{\pgfqpoint{2.678629in}{5.782422in}}{\pgfqpoint{2.684453in}{5.782422in}}%
\pgfpathlineto{\pgfqpoint{2.684453in}{5.782422in}}%
\pgfpathclose%
\pgfusepath{stroke,fill}%
\end{pgfscope}%
\begin{pgfscope}%
\pgfpathrectangle{\pgfqpoint{1.582361in}{0.880000in}}{\pgfqpoint{5.035278in}{6.160000in}}%
\pgfusepath{clip}%
\pgfsetbuttcap%
\pgfsetroundjoin%
\definecolor{currentfill}{rgb}{0.800000,0.200000,0.200000}%
\pgfsetfillcolor{currentfill}%
\pgfsetlinewidth{1.003750pt}%
\definecolor{currentstroke}{rgb}{0.800000,0.200000,0.200000}%
\pgfsetstrokecolor{currentstroke}%
\pgfsetdash{}{0pt}%
\pgfpathmoveto{\pgfqpoint{2.670563in}{5.739900in}}%
\pgfpathcurveto{\pgfqpoint{2.676387in}{5.739900in}}{\pgfqpoint{2.681973in}{5.742214in}}{\pgfqpoint{2.686091in}{5.746332in}}%
\pgfpathcurveto{\pgfqpoint{2.690209in}{5.750451in}}{\pgfqpoint{2.692523in}{5.756037in}}{\pgfqpoint{2.692523in}{5.761861in}}%
\pgfpathcurveto{\pgfqpoint{2.692523in}{5.767685in}}{\pgfqpoint{2.690209in}{5.773271in}}{\pgfqpoint{2.686091in}{5.777389in}}%
\pgfpathcurveto{\pgfqpoint{2.681973in}{5.781507in}}{\pgfqpoint{2.676387in}{5.783821in}}{\pgfqpoint{2.670563in}{5.783821in}}%
\pgfpathcurveto{\pgfqpoint{2.664739in}{5.783821in}}{\pgfqpoint{2.659153in}{5.781507in}}{\pgfqpoint{2.655035in}{5.777389in}}%
\pgfpathcurveto{\pgfqpoint{2.650917in}{5.773271in}}{\pgfqpoint{2.648603in}{5.767685in}}{\pgfqpoint{2.648603in}{5.761861in}}%
\pgfpathcurveto{\pgfqpoint{2.648603in}{5.756037in}}{\pgfqpoint{2.650917in}{5.750451in}}{\pgfqpoint{2.655035in}{5.746332in}}%
\pgfpathcurveto{\pgfqpoint{2.659153in}{5.742214in}}{\pgfqpoint{2.664739in}{5.739900in}}{\pgfqpoint{2.670563in}{5.739900in}}%
\pgfpathlineto{\pgfqpoint{2.670563in}{5.739900in}}%
\pgfpathclose%
\pgfusepath{stroke,fill}%
\end{pgfscope}%
\begin{pgfscope}%
\pgfpathrectangle{\pgfqpoint{1.582361in}{0.880000in}}{\pgfqpoint{5.035278in}{6.160000in}}%
\pgfusepath{clip}%
\pgfsetbuttcap%
\pgfsetroundjoin%
\definecolor{currentfill}{rgb}{0.800000,0.200000,0.200000}%
\pgfsetfillcolor{currentfill}%
\pgfsetlinewidth{1.003750pt}%
\definecolor{currentstroke}{rgb}{0.800000,0.200000,0.200000}%
\pgfsetstrokecolor{currentstroke}%
\pgfsetdash{}{0pt}%
\pgfpathmoveto{\pgfqpoint{2.658022in}{5.696962in}}%
\pgfpathcurveto{\pgfqpoint{2.663846in}{5.696962in}}{\pgfqpoint{2.669432in}{5.699276in}}{\pgfqpoint{2.673550in}{5.703394in}}%
\pgfpathcurveto{\pgfqpoint{2.677668in}{5.707512in}}{\pgfqpoint{2.679982in}{5.713098in}}{\pgfqpoint{2.679982in}{5.718922in}}%
\pgfpathcurveto{\pgfqpoint{2.679982in}{5.724746in}}{\pgfqpoint{2.677668in}{5.730332in}}{\pgfqpoint{2.673550in}{5.734451in}}%
\pgfpathcurveto{\pgfqpoint{2.669432in}{5.738569in}}{\pgfqpoint{2.663846in}{5.740883in}}{\pgfqpoint{2.658022in}{5.740883in}}%
\pgfpathcurveto{\pgfqpoint{2.652198in}{5.740883in}}{\pgfqpoint{2.646612in}{5.738569in}}{\pgfqpoint{2.642494in}{5.734451in}}%
\pgfpathcurveto{\pgfqpoint{2.638376in}{5.730332in}}{\pgfqpoint{2.636062in}{5.724746in}}{\pgfqpoint{2.636062in}{5.718922in}}%
\pgfpathcurveto{\pgfqpoint{2.636062in}{5.713098in}}{\pgfqpoint{2.638376in}{5.707512in}}{\pgfqpoint{2.642494in}{5.703394in}}%
\pgfpathcurveto{\pgfqpoint{2.646612in}{5.699276in}}{\pgfqpoint{2.652198in}{5.696962in}}{\pgfqpoint{2.658022in}{5.696962in}}%
\pgfpathlineto{\pgfqpoint{2.658022in}{5.696962in}}%
\pgfpathclose%
\pgfusepath{stroke,fill}%
\end{pgfscope}%
\begin{pgfscope}%
\pgfpathrectangle{\pgfqpoint{1.582361in}{0.880000in}}{\pgfqpoint{5.035278in}{6.160000in}}%
\pgfusepath{clip}%
\pgfsetbuttcap%
\pgfsetroundjoin%
\definecolor{currentfill}{rgb}{0.800000,0.200000,0.200000}%
\pgfsetfillcolor{currentfill}%
\pgfsetlinewidth{1.003750pt}%
\definecolor{currentstroke}{rgb}{0.800000,0.200000,0.200000}%
\pgfsetstrokecolor{currentstroke}%
\pgfsetdash{}{0pt}%
\pgfpathmoveto{\pgfqpoint{2.646843in}{5.653649in}}%
\pgfpathcurveto{\pgfqpoint{2.652667in}{5.653649in}}{\pgfqpoint{2.658253in}{5.655963in}}{\pgfqpoint{2.662371in}{5.660081in}}%
\pgfpathcurveto{\pgfqpoint{2.666489in}{5.664199in}}{\pgfqpoint{2.668803in}{5.669785in}}{\pgfqpoint{2.668803in}{5.675609in}}%
\pgfpathcurveto{\pgfqpoint{2.668803in}{5.681433in}}{\pgfqpoint{2.666489in}{5.687019in}}{\pgfqpoint{2.662371in}{5.691138in}}%
\pgfpathcurveto{\pgfqpoint{2.658253in}{5.695256in}}{\pgfqpoint{2.652667in}{5.697570in}}{\pgfqpoint{2.646843in}{5.697570in}}%
\pgfpathcurveto{\pgfqpoint{2.641019in}{5.697570in}}{\pgfqpoint{2.635433in}{5.695256in}}{\pgfqpoint{2.631315in}{5.691138in}}%
\pgfpathcurveto{\pgfqpoint{2.627196in}{5.687019in}}{\pgfqpoint{2.624883in}{5.681433in}}{\pgfqpoint{2.624883in}{5.675609in}}%
\pgfpathcurveto{\pgfqpoint{2.624883in}{5.669785in}}{\pgfqpoint{2.627196in}{5.664199in}}{\pgfqpoint{2.631315in}{5.660081in}}%
\pgfpathcurveto{\pgfqpoint{2.635433in}{5.655963in}}{\pgfqpoint{2.641019in}{5.653649in}}{\pgfqpoint{2.646843in}{5.653649in}}%
\pgfpathlineto{\pgfqpoint{2.646843in}{5.653649in}}%
\pgfpathclose%
\pgfusepath{stroke,fill}%
\end{pgfscope}%
\begin{pgfscope}%
\pgfpathrectangle{\pgfqpoint{1.582361in}{0.880000in}}{\pgfqpoint{5.035278in}{6.160000in}}%
\pgfusepath{clip}%
\pgfsetbuttcap%
\pgfsetroundjoin%
\definecolor{currentfill}{rgb}{0.800000,0.200000,0.200000}%
\pgfsetfillcolor{currentfill}%
\pgfsetlinewidth{1.003750pt}%
\definecolor{currentstroke}{rgb}{0.800000,0.200000,0.200000}%
\pgfsetstrokecolor{currentstroke}%
\pgfsetdash{}{0pt}%
\pgfpathmoveto{\pgfqpoint{2.637036in}{5.610005in}}%
\pgfpathcurveto{\pgfqpoint{2.642860in}{5.610005in}}{\pgfqpoint{2.648447in}{5.612319in}}{\pgfqpoint{2.652565in}{5.616437in}}%
\pgfpathcurveto{\pgfqpoint{2.656683in}{5.620555in}}{\pgfqpoint{2.658997in}{5.626141in}}{\pgfqpoint{2.658997in}{5.631965in}}%
\pgfpathcurveto{\pgfqpoint{2.658997in}{5.637789in}}{\pgfqpoint{2.656683in}{5.643375in}}{\pgfqpoint{2.652565in}{5.647493in}}%
\pgfpathcurveto{\pgfqpoint{2.648447in}{5.651612in}}{\pgfqpoint{2.642860in}{5.653925in}}{\pgfqpoint{2.637036in}{5.653925in}}%
\pgfpathcurveto{\pgfqpoint{2.631213in}{5.653925in}}{\pgfqpoint{2.625626in}{5.651612in}}{\pgfqpoint{2.621508in}{5.647493in}}%
\pgfpathcurveto{\pgfqpoint{2.617390in}{5.643375in}}{\pgfqpoint{2.615076in}{5.637789in}}{\pgfqpoint{2.615076in}{5.631965in}}%
\pgfpathcurveto{\pgfqpoint{2.615076in}{5.626141in}}{\pgfqpoint{2.617390in}{5.620555in}}{\pgfqpoint{2.621508in}{5.616437in}}%
\pgfpathcurveto{\pgfqpoint{2.625626in}{5.612319in}}{\pgfqpoint{2.631213in}{5.610005in}}{\pgfqpoint{2.637036in}{5.610005in}}%
\pgfpathlineto{\pgfqpoint{2.637036in}{5.610005in}}%
\pgfpathclose%
\pgfusepath{stroke,fill}%
\end{pgfscope}%
\begin{pgfscope}%
\pgfpathrectangle{\pgfqpoint{1.582361in}{0.880000in}}{\pgfqpoint{5.035278in}{6.160000in}}%
\pgfusepath{clip}%
\pgfsetbuttcap%
\pgfsetroundjoin%
\definecolor{currentfill}{rgb}{0.800000,0.200000,0.200000}%
\pgfsetfillcolor{currentfill}%
\pgfsetlinewidth{1.003750pt}%
\definecolor{currentstroke}{rgb}{0.800000,0.200000,0.200000}%
\pgfsetstrokecolor{currentstroke}%
\pgfsetdash{}{0pt}%
\pgfpathmoveto{\pgfqpoint{2.628613in}{5.566073in}}%
\pgfpathcurveto{\pgfqpoint{2.634437in}{5.566073in}}{\pgfqpoint{2.640023in}{5.568387in}}{\pgfqpoint{2.644141in}{5.572505in}}%
\pgfpathcurveto{\pgfqpoint{2.648259in}{5.576623in}}{\pgfqpoint{2.650573in}{5.582209in}}{\pgfqpoint{2.650573in}{5.588033in}}%
\pgfpathcurveto{\pgfqpoint{2.650573in}{5.593857in}}{\pgfqpoint{2.648259in}{5.599443in}}{\pgfqpoint{2.644141in}{5.603561in}}%
\pgfpathcurveto{\pgfqpoint{2.640023in}{5.607679in}}{\pgfqpoint{2.634437in}{5.609993in}}{\pgfqpoint{2.628613in}{5.609993in}}%
\pgfpathcurveto{\pgfqpoint{2.622789in}{5.609993in}}{\pgfqpoint{2.617203in}{5.607679in}}{\pgfqpoint{2.613085in}{5.603561in}}%
\pgfpathcurveto{\pgfqpoint{2.608966in}{5.599443in}}{\pgfqpoint{2.606653in}{5.593857in}}{\pgfqpoint{2.606653in}{5.588033in}}%
\pgfpathcurveto{\pgfqpoint{2.606653in}{5.582209in}}{\pgfqpoint{2.608966in}{5.576623in}}{\pgfqpoint{2.613085in}{5.572505in}}%
\pgfpathcurveto{\pgfqpoint{2.617203in}{5.568387in}}{\pgfqpoint{2.622789in}{5.566073in}}{\pgfqpoint{2.628613in}{5.566073in}}%
\pgfpathlineto{\pgfqpoint{2.628613in}{5.566073in}}%
\pgfpathclose%
\pgfusepath{stroke,fill}%
\end{pgfscope}%
\begin{pgfscope}%
\pgfpathrectangle{\pgfqpoint{1.582361in}{0.880000in}}{\pgfqpoint{5.035278in}{6.160000in}}%
\pgfusepath{clip}%
\pgfsetbuttcap%
\pgfsetroundjoin%
\definecolor{currentfill}{rgb}{0.800000,0.200000,0.200000}%
\pgfsetfillcolor{currentfill}%
\pgfsetlinewidth{1.003750pt}%
\definecolor{currentstroke}{rgb}{0.800000,0.200000,0.200000}%
\pgfsetstrokecolor{currentstroke}%
\pgfsetdash{}{0pt}%
\pgfpathmoveto{\pgfqpoint{2.621580in}{5.521897in}}%
\pgfpathcurveto{\pgfqpoint{2.627404in}{5.521897in}}{\pgfqpoint{2.632990in}{5.524211in}}{\pgfqpoint{2.637109in}{5.528329in}}%
\pgfpathcurveto{\pgfqpoint{2.641227in}{5.532447in}}{\pgfqpoint{2.643541in}{5.538033in}}{\pgfqpoint{2.643541in}{5.543857in}}%
\pgfpathcurveto{\pgfqpoint{2.643541in}{5.549681in}}{\pgfqpoint{2.641227in}{5.555267in}}{\pgfqpoint{2.637109in}{5.559385in}}%
\pgfpathcurveto{\pgfqpoint{2.632990in}{5.563503in}}{\pgfqpoint{2.627404in}{5.565817in}}{\pgfqpoint{2.621580in}{5.565817in}}%
\pgfpathcurveto{\pgfqpoint{2.615756in}{5.565817in}}{\pgfqpoint{2.610170in}{5.563503in}}{\pgfqpoint{2.606052in}{5.559385in}}%
\pgfpathcurveto{\pgfqpoint{2.601934in}{5.555267in}}{\pgfqpoint{2.599620in}{5.549681in}}{\pgfqpoint{2.599620in}{5.543857in}}%
\pgfpathcurveto{\pgfqpoint{2.599620in}{5.538033in}}{\pgfqpoint{2.601934in}{5.532447in}}{\pgfqpoint{2.606052in}{5.528329in}}%
\pgfpathcurveto{\pgfqpoint{2.610170in}{5.524211in}}{\pgfqpoint{2.615756in}{5.521897in}}{\pgfqpoint{2.621580in}{5.521897in}}%
\pgfpathlineto{\pgfqpoint{2.621580in}{5.521897in}}%
\pgfpathclose%
\pgfusepath{stroke,fill}%
\end{pgfscope}%
\begin{pgfscope}%
\pgfpathrectangle{\pgfqpoint{1.582361in}{0.880000in}}{\pgfqpoint{5.035278in}{6.160000in}}%
\pgfusepath{clip}%
\pgfsetbuttcap%
\pgfsetroundjoin%
\definecolor{currentfill}{rgb}{0.800000,0.200000,0.200000}%
\pgfsetfillcolor{currentfill}%
\pgfsetlinewidth{1.003750pt}%
\definecolor{currentstroke}{rgb}{0.800000,0.200000,0.200000}%
\pgfsetstrokecolor{currentstroke}%
\pgfsetdash{}{0pt}%
\pgfpathmoveto{\pgfqpoint{2.615946in}{5.477521in}}%
\pgfpathcurveto{\pgfqpoint{2.621770in}{5.477521in}}{\pgfqpoint{2.627356in}{5.479835in}}{\pgfqpoint{2.631474in}{5.483953in}}%
\pgfpathcurveto{\pgfqpoint{2.635592in}{5.488071in}}{\pgfqpoint{2.637906in}{5.493657in}}{\pgfqpoint{2.637906in}{5.499481in}}%
\pgfpathcurveto{\pgfqpoint{2.637906in}{5.505305in}}{\pgfqpoint{2.635592in}{5.510891in}}{\pgfqpoint{2.631474in}{5.515009in}}%
\pgfpathcurveto{\pgfqpoint{2.627356in}{5.519127in}}{\pgfqpoint{2.621770in}{5.521441in}}{\pgfqpoint{2.615946in}{5.521441in}}%
\pgfpathcurveto{\pgfqpoint{2.610122in}{5.521441in}}{\pgfqpoint{2.604536in}{5.519127in}}{\pgfqpoint{2.600418in}{5.515009in}}%
\pgfpathcurveto{\pgfqpoint{2.596299in}{5.510891in}}{\pgfqpoint{2.593986in}{5.505305in}}{\pgfqpoint{2.593986in}{5.499481in}}%
\pgfpathcurveto{\pgfqpoint{2.593986in}{5.493657in}}{\pgfqpoint{2.596299in}{5.488071in}}{\pgfqpoint{2.600418in}{5.483953in}}%
\pgfpathcurveto{\pgfqpoint{2.604536in}{5.479835in}}{\pgfqpoint{2.610122in}{5.477521in}}{\pgfqpoint{2.615946in}{5.477521in}}%
\pgfpathlineto{\pgfqpoint{2.615946in}{5.477521in}}%
\pgfpathclose%
\pgfusepath{stroke,fill}%
\end{pgfscope}%
\begin{pgfscope}%
\pgfpathrectangle{\pgfqpoint{1.582361in}{0.880000in}}{\pgfqpoint{5.035278in}{6.160000in}}%
\pgfusepath{clip}%
\pgfsetbuttcap%
\pgfsetroundjoin%
\definecolor{currentfill}{rgb}{0.800000,0.200000,0.200000}%
\pgfsetfillcolor{currentfill}%
\pgfsetlinewidth{1.003750pt}%
\definecolor{currentstroke}{rgb}{0.800000,0.200000,0.200000}%
\pgfsetstrokecolor{currentstroke}%
\pgfsetdash{}{0pt}%
\pgfpathmoveto{\pgfqpoint{2.611715in}{5.432989in}}%
\pgfpathcurveto{\pgfqpoint{2.617539in}{5.432989in}}{\pgfqpoint{2.623125in}{5.435303in}}{\pgfqpoint{2.627243in}{5.439421in}}%
\pgfpathcurveto{\pgfqpoint{2.631361in}{5.443539in}}{\pgfqpoint{2.633675in}{5.449125in}}{\pgfqpoint{2.633675in}{5.454949in}}%
\pgfpathcurveto{\pgfqpoint{2.633675in}{5.460773in}}{\pgfqpoint{2.631361in}{5.466359in}}{\pgfqpoint{2.627243in}{5.470477in}}%
\pgfpathcurveto{\pgfqpoint{2.623125in}{5.474595in}}{\pgfqpoint{2.617539in}{5.476909in}}{\pgfqpoint{2.611715in}{5.476909in}}%
\pgfpathcurveto{\pgfqpoint{2.605891in}{5.476909in}}{\pgfqpoint{2.600305in}{5.474595in}}{\pgfqpoint{2.596187in}{5.470477in}}%
\pgfpathcurveto{\pgfqpoint{2.592069in}{5.466359in}}{\pgfqpoint{2.589755in}{5.460773in}}{\pgfqpoint{2.589755in}{5.454949in}}%
\pgfpathcurveto{\pgfqpoint{2.589755in}{5.449125in}}{\pgfqpoint{2.592069in}{5.443539in}}{\pgfqpoint{2.596187in}{5.439421in}}%
\pgfpathcurveto{\pgfqpoint{2.600305in}{5.435303in}}{\pgfqpoint{2.605891in}{5.432989in}}{\pgfqpoint{2.611715in}{5.432989in}}%
\pgfpathlineto{\pgfqpoint{2.611715in}{5.432989in}}%
\pgfpathclose%
\pgfusepath{stroke,fill}%
\end{pgfscope}%
\begin{pgfscope}%
\pgfpathrectangle{\pgfqpoint{1.582361in}{0.880000in}}{\pgfqpoint{5.035278in}{6.160000in}}%
\pgfusepath{clip}%
\pgfsetbuttcap%
\pgfsetroundjoin%
\definecolor{currentfill}{rgb}{0.800000,0.200000,0.200000}%
\pgfsetfillcolor{currentfill}%
\pgfsetlinewidth{1.003750pt}%
\definecolor{currentstroke}{rgb}{0.800000,0.200000,0.200000}%
\pgfsetstrokecolor{currentstroke}%
\pgfsetdash{}{0pt}%
\pgfpathmoveto{\pgfqpoint{2.608892in}{5.388346in}}%
\pgfpathcurveto{\pgfqpoint{2.614716in}{5.388346in}}{\pgfqpoint{2.620302in}{5.390660in}}{\pgfqpoint{2.624420in}{5.394778in}}%
\pgfpathcurveto{\pgfqpoint{2.628539in}{5.398896in}}{\pgfqpoint{2.630852in}{5.404482in}}{\pgfqpoint{2.630852in}{5.410306in}}%
\pgfpathcurveto{\pgfqpoint{2.630852in}{5.416130in}}{\pgfqpoint{2.628539in}{5.421716in}}{\pgfqpoint{2.624420in}{5.425834in}}%
\pgfpathcurveto{\pgfqpoint{2.620302in}{5.429952in}}{\pgfqpoint{2.614716in}{5.432266in}}{\pgfqpoint{2.608892in}{5.432266in}}%
\pgfpathcurveto{\pgfqpoint{2.603068in}{5.432266in}}{\pgfqpoint{2.597482in}{5.429952in}}{\pgfqpoint{2.593364in}{5.425834in}}%
\pgfpathcurveto{\pgfqpoint{2.589246in}{5.421716in}}{\pgfqpoint{2.586932in}{5.416130in}}{\pgfqpoint{2.586932in}{5.410306in}}%
\pgfpathcurveto{\pgfqpoint{2.586932in}{5.404482in}}{\pgfqpoint{2.589246in}{5.398896in}}{\pgfqpoint{2.593364in}{5.394778in}}%
\pgfpathcurveto{\pgfqpoint{2.597482in}{5.390660in}}{\pgfqpoint{2.603068in}{5.388346in}}{\pgfqpoint{2.608892in}{5.388346in}}%
\pgfpathlineto{\pgfqpoint{2.608892in}{5.388346in}}%
\pgfpathclose%
\pgfusepath{stroke,fill}%
\end{pgfscope}%
\begin{pgfscope}%
\pgfpathrectangle{\pgfqpoint{1.582361in}{0.880000in}}{\pgfqpoint{5.035278in}{6.160000in}}%
\pgfusepath{clip}%
\pgfsetbuttcap%
\pgfsetroundjoin%
\definecolor{currentfill}{rgb}{0.800000,0.200000,0.200000}%
\pgfsetfillcolor{currentfill}%
\pgfsetlinewidth{1.003750pt}%
\definecolor{currentstroke}{rgb}{0.800000,0.200000,0.200000}%
\pgfsetstrokecolor{currentstroke}%
\pgfsetdash{}{0pt}%
\pgfpathmoveto{\pgfqpoint{2.607480in}{5.343636in}}%
\pgfpathcurveto{\pgfqpoint{2.613304in}{5.343636in}}{\pgfqpoint{2.618890in}{5.345949in}}{\pgfqpoint{2.623008in}{5.350068in}}%
\pgfpathcurveto{\pgfqpoint{2.627126in}{5.354186in}}{\pgfqpoint{2.629440in}{5.359772in}}{\pgfqpoint{2.629440in}{5.365596in}}%
\pgfpathcurveto{\pgfqpoint{2.629440in}{5.371420in}}{\pgfqpoint{2.627126in}{5.377006in}}{\pgfqpoint{2.623008in}{5.381124in}}%
\pgfpathcurveto{\pgfqpoint{2.618890in}{5.385242in}}{\pgfqpoint{2.613304in}{5.387556in}}{\pgfqpoint{2.607480in}{5.387556in}}%
\pgfpathcurveto{\pgfqpoint{2.601656in}{5.387556in}}{\pgfqpoint{2.596070in}{5.385242in}}{\pgfqpoint{2.591952in}{5.381124in}}%
\pgfpathcurveto{\pgfqpoint{2.587834in}{5.377006in}}{\pgfqpoint{2.585520in}{5.371420in}}{\pgfqpoint{2.585520in}{5.365596in}}%
\pgfpathcurveto{\pgfqpoint{2.585520in}{5.359772in}}{\pgfqpoint{2.587834in}{5.354186in}}{\pgfqpoint{2.591952in}{5.350068in}}%
\pgfpathcurveto{\pgfqpoint{2.596070in}{5.345949in}}{\pgfqpoint{2.601656in}{5.343636in}}{\pgfqpoint{2.607480in}{5.343636in}}%
\pgfpathlineto{\pgfqpoint{2.607480in}{5.343636in}}%
\pgfpathclose%
\pgfusepath{stroke,fill}%
\end{pgfscope}%
\begin{pgfscope}%
\pgfpathrectangle{\pgfqpoint{1.582361in}{0.880000in}}{\pgfqpoint{5.035278in}{6.160000in}}%
\pgfusepath{clip}%
\pgfsetbuttcap%
\pgfsetroundjoin%
\definecolor{currentfill}{rgb}{0.800000,0.200000,0.200000}%
\pgfsetfillcolor{currentfill}%
\pgfsetlinewidth{1.003750pt}%
\definecolor{currentstroke}{rgb}{0.800000,0.200000,0.200000}%
\pgfsetstrokecolor{currentstroke}%
\pgfsetdash{}{0pt}%
\pgfpathmoveto{\pgfqpoint{2.607480in}{5.298903in}}%
\pgfpathcurveto{\pgfqpoint{2.613304in}{5.298903in}}{\pgfqpoint{2.618890in}{5.301217in}}{\pgfqpoint{2.623008in}{5.305335in}}%
\pgfpathcurveto{\pgfqpoint{2.627126in}{5.309453in}}{\pgfqpoint{2.629440in}{5.315040in}}{\pgfqpoint{2.629440in}{5.320863in}}%
\pgfpathcurveto{\pgfqpoint{2.629440in}{5.326687in}}{\pgfqpoint{2.627126in}{5.332274in}}{\pgfqpoint{2.623008in}{5.336392in}}%
\pgfpathcurveto{\pgfqpoint{2.618890in}{5.340510in}}{\pgfqpoint{2.613304in}{5.342824in}}{\pgfqpoint{2.607480in}{5.342824in}}%
\pgfpathcurveto{\pgfqpoint{2.601656in}{5.342824in}}{\pgfqpoint{2.596070in}{5.340510in}}{\pgfqpoint{2.591952in}{5.336392in}}%
\pgfpathcurveto{\pgfqpoint{2.587834in}{5.332274in}}{\pgfqpoint{2.585520in}{5.326687in}}{\pgfqpoint{2.585520in}{5.320863in}}%
\pgfpathcurveto{\pgfqpoint{2.585520in}{5.315040in}}{\pgfqpoint{2.587834in}{5.309453in}}{\pgfqpoint{2.591952in}{5.305335in}}%
\pgfpathcurveto{\pgfqpoint{2.596070in}{5.301217in}}{\pgfqpoint{2.601656in}{5.298903in}}{\pgfqpoint{2.607480in}{5.298903in}}%
\pgfpathlineto{\pgfqpoint{2.607480in}{5.298903in}}%
\pgfpathclose%
\pgfusepath{stroke,fill}%
\end{pgfscope}%
\begin{pgfscope}%
\pgfpathrectangle{\pgfqpoint{1.582361in}{0.880000in}}{\pgfqpoint{5.035278in}{6.160000in}}%
\pgfusepath{clip}%
\pgfsetbuttcap%
\pgfsetroundjoin%
\definecolor{currentfill}{rgb}{0.800000,0.200000,0.200000}%
\pgfsetfillcolor{currentfill}%
\pgfsetlinewidth{1.003750pt}%
\definecolor{currentstroke}{rgb}{0.800000,0.200000,0.200000}%
\pgfsetstrokecolor{currentstroke}%
\pgfsetdash{}{0pt}%
\pgfpathmoveto{\pgfqpoint{2.608892in}{5.254193in}}%
\pgfpathcurveto{\pgfqpoint{2.614716in}{5.254193in}}{\pgfqpoint{2.620302in}{5.256507in}}{\pgfqpoint{2.624420in}{5.260625in}}%
\pgfpathcurveto{\pgfqpoint{2.628539in}{5.264743in}}{\pgfqpoint{2.630852in}{5.270330in}}{\pgfqpoint{2.630852in}{5.276153in}}%
\pgfpathcurveto{\pgfqpoint{2.630852in}{5.281977in}}{\pgfqpoint{2.628539in}{5.287564in}}{\pgfqpoint{2.624420in}{5.291682in}}%
\pgfpathcurveto{\pgfqpoint{2.620302in}{5.295800in}}{\pgfqpoint{2.614716in}{5.298114in}}{\pgfqpoint{2.608892in}{5.298114in}}%
\pgfpathcurveto{\pgfqpoint{2.603068in}{5.298114in}}{\pgfqpoint{2.597482in}{5.295800in}}{\pgfqpoint{2.593364in}{5.291682in}}%
\pgfpathcurveto{\pgfqpoint{2.589246in}{5.287564in}}{\pgfqpoint{2.586932in}{5.281977in}}{\pgfqpoint{2.586932in}{5.276153in}}%
\pgfpathcurveto{\pgfqpoint{2.586932in}{5.270330in}}{\pgfqpoint{2.589246in}{5.264743in}}{\pgfqpoint{2.593364in}{5.260625in}}%
\pgfpathcurveto{\pgfqpoint{2.597482in}{5.256507in}}{\pgfqpoint{2.603068in}{5.254193in}}{\pgfqpoint{2.608892in}{5.254193in}}%
\pgfpathlineto{\pgfqpoint{2.608892in}{5.254193in}}%
\pgfpathclose%
\pgfusepath{stroke,fill}%
\end{pgfscope}%
\begin{pgfscope}%
\pgfpathrectangle{\pgfqpoint{1.582361in}{0.880000in}}{\pgfqpoint{5.035278in}{6.160000in}}%
\pgfusepath{clip}%
\pgfsetbuttcap%
\pgfsetroundjoin%
\definecolor{currentfill}{rgb}{0.800000,0.200000,0.200000}%
\pgfsetfillcolor{currentfill}%
\pgfsetlinewidth{1.003750pt}%
\definecolor{currentstroke}{rgb}{0.800000,0.200000,0.200000}%
\pgfsetstrokecolor{currentstroke}%
\pgfsetdash{}{0pt}%
\pgfpathmoveto{\pgfqpoint{2.611715in}{5.209550in}}%
\pgfpathcurveto{\pgfqpoint{2.617539in}{5.209550in}}{\pgfqpoint{2.623125in}{5.211864in}}{\pgfqpoint{2.627243in}{5.215982in}}%
\pgfpathcurveto{\pgfqpoint{2.631361in}{5.220100in}}{\pgfqpoint{2.633675in}{5.225686in}}{\pgfqpoint{2.633675in}{5.231510in}}%
\pgfpathcurveto{\pgfqpoint{2.633675in}{5.237334in}}{\pgfqpoint{2.631361in}{5.242920in}}{\pgfqpoint{2.627243in}{5.247038in}}%
\pgfpathcurveto{\pgfqpoint{2.623125in}{5.251157in}}{\pgfqpoint{2.617539in}{5.253470in}}{\pgfqpoint{2.611715in}{5.253470in}}%
\pgfpathcurveto{\pgfqpoint{2.605891in}{5.253470in}}{\pgfqpoint{2.600305in}{5.251157in}}{\pgfqpoint{2.596187in}{5.247038in}}%
\pgfpathcurveto{\pgfqpoint{2.592069in}{5.242920in}}{\pgfqpoint{2.589755in}{5.237334in}}{\pgfqpoint{2.589755in}{5.231510in}}%
\pgfpathcurveto{\pgfqpoint{2.589755in}{5.225686in}}{\pgfqpoint{2.592069in}{5.220100in}}{\pgfqpoint{2.596187in}{5.215982in}}%
\pgfpathcurveto{\pgfqpoint{2.600305in}{5.211864in}}{\pgfqpoint{2.605891in}{5.209550in}}{\pgfqpoint{2.611715in}{5.209550in}}%
\pgfpathlineto{\pgfqpoint{2.611715in}{5.209550in}}%
\pgfpathclose%
\pgfusepath{stroke,fill}%
\end{pgfscope}%
\begin{pgfscope}%
\pgfpathrectangle{\pgfqpoint{1.582361in}{0.880000in}}{\pgfqpoint{5.035278in}{6.160000in}}%
\pgfusepath{clip}%
\pgfsetbuttcap%
\pgfsetroundjoin%
\definecolor{currentfill}{rgb}{0.800000,0.200000,0.200000}%
\pgfsetfillcolor{currentfill}%
\pgfsetlinewidth{1.003750pt}%
\definecolor{currentstroke}{rgb}{0.800000,0.200000,0.200000}%
\pgfsetstrokecolor{currentstroke}%
\pgfsetdash{}{0pt}%
\pgfpathmoveto{\pgfqpoint{2.615946in}{5.165018in}}%
\pgfpathcurveto{\pgfqpoint{2.621770in}{5.165018in}}{\pgfqpoint{2.627356in}{5.167332in}}{\pgfqpoint{2.631474in}{5.171450in}}%
\pgfpathcurveto{\pgfqpoint{2.635592in}{5.175568in}}{\pgfqpoint{2.637906in}{5.181154in}}{\pgfqpoint{2.637906in}{5.186978in}}%
\pgfpathcurveto{\pgfqpoint{2.637906in}{5.192802in}}{\pgfqpoint{2.635592in}{5.198389in}}{\pgfqpoint{2.631474in}{5.202507in}}%
\pgfpathcurveto{\pgfqpoint{2.627356in}{5.206625in}}{\pgfqpoint{2.621770in}{5.208939in}}{\pgfqpoint{2.615946in}{5.208939in}}%
\pgfpathcurveto{\pgfqpoint{2.610122in}{5.208939in}}{\pgfqpoint{2.604536in}{5.206625in}}{\pgfqpoint{2.600418in}{5.202507in}}%
\pgfpathcurveto{\pgfqpoint{2.596299in}{5.198389in}}{\pgfqpoint{2.593986in}{5.192802in}}{\pgfqpoint{2.593986in}{5.186978in}}%
\pgfpathcurveto{\pgfqpoint{2.593986in}{5.181154in}}{\pgfqpoint{2.596299in}{5.175568in}}{\pgfqpoint{2.600418in}{5.171450in}}%
\pgfpathcurveto{\pgfqpoint{2.604536in}{5.167332in}}{\pgfqpoint{2.610122in}{5.165018in}}{\pgfqpoint{2.615946in}{5.165018in}}%
\pgfpathlineto{\pgfqpoint{2.615946in}{5.165018in}}%
\pgfpathclose%
\pgfusepath{stroke,fill}%
\end{pgfscope}%
\begin{pgfscope}%
\pgfpathrectangle{\pgfqpoint{1.582361in}{0.880000in}}{\pgfqpoint{5.035278in}{6.160000in}}%
\pgfusepath{clip}%
\pgfsetbuttcap%
\pgfsetroundjoin%
\definecolor{currentfill}{rgb}{0.800000,0.200000,0.200000}%
\pgfsetfillcolor{currentfill}%
\pgfsetlinewidth{1.003750pt}%
\definecolor{currentstroke}{rgb}{0.800000,0.200000,0.200000}%
\pgfsetstrokecolor{currentstroke}%
\pgfsetdash{}{0pt}%
\pgfpathmoveto{\pgfqpoint{2.621580in}{5.120642in}}%
\pgfpathcurveto{\pgfqpoint{2.627404in}{5.120642in}}{\pgfqpoint{2.632990in}{5.122956in}}{\pgfqpoint{2.637109in}{5.127074in}}%
\pgfpathcurveto{\pgfqpoint{2.641227in}{5.131192in}}{\pgfqpoint{2.643541in}{5.136778in}}{\pgfqpoint{2.643541in}{5.142602in}}%
\pgfpathcurveto{\pgfqpoint{2.643541in}{5.148426in}}{\pgfqpoint{2.641227in}{5.154012in}}{\pgfqpoint{2.637109in}{5.158131in}}%
\pgfpathcurveto{\pgfqpoint{2.632990in}{5.162249in}}{\pgfqpoint{2.627404in}{5.164563in}}{\pgfqpoint{2.621580in}{5.164563in}}%
\pgfpathcurveto{\pgfqpoint{2.615756in}{5.164563in}}{\pgfqpoint{2.610170in}{5.162249in}}{\pgfqpoint{2.606052in}{5.158131in}}%
\pgfpathcurveto{\pgfqpoint{2.601934in}{5.154012in}}{\pgfqpoint{2.599620in}{5.148426in}}{\pgfqpoint{2.599620in}{5.142602in}}%
\pgfpathcurveto{\pgfqpoint{2.599620in}{5.136778in}}{\pgfqpoint{2.601934in}{5.131192in}}{\pgfqpoint{2.606052in}{5.127074in}}%
\pgfpathcurveto{\pgfqpoint{2.610170in}{5.122956in}}{\pgfqpoint{2.615756in}{5.120642in}}{\pgfqpoint{2.621580in}{5.120642in}}%
\pgfpathlineto{\pgfqpoint{2.621580in}{5.120642in}}%
\pgfpathclose%
\pgfusepath{stroke,fill}%
\end{pgfscope}%
\begin{pgfscope}%
\pgfpathrectangle{\pgfqpoint{1.582361in}{0.880000in}}{\pgfqpoint{5.035278in}{6.160000in}}%
\pgfusepath{clip}%
\pgfsetbuttcap%
\pgfsetroundjoin%
\definecolor{currentfill}{rgb}{0.800000,0.200000,0.200000}%
\pgfsetfillcolor{currentfill}%
\pgfsetlinewidth{1.003750pt}%
\definecolor{currentstroke}{rgb}{0.800000,0.200000,0.200000}%
\pgfsetstrokecolor{currentstroke}%
\pgfsetdash{}{0pt}%
\pgfpathmoveto{\pgfqpoint{2.628613in}{5.076466in}}%
\pgfpathcurveto{\pgfqpoint{2.634437in}{5.076466in}}{\pgfqpoint{2.640023in}{5.078780in}}{\pgfqpoint{2.644141in}{5.082898in}}%
\pgfpathcurveto{\pgfqpoint{2.648259in}{5.087016in}}{\pgfqpoint{2.650573in}{5.092602in}}{\pgfqpoint{2.650573in}{5.098426in}}%
\pgfpathcurveto{\pgfqpoint{2.650573in}{5.104250in}}{\pgfqpoint{2.648259in}{5.109836in}}{\pgfqpoint{2.644141in}{5.113954in}}%
\pgfpathcurveto{\pgfqpoint{2.640023in}{5.118073in}}{\pgfqpoint{2.634437in}{5.120387in}}{\pgfqpoint{2.628613in}{5.120387in}}%
\pgfpathcurveto{\pgfqpoint{2.622789in}{5.120387in}}{\pgfqpoint{2.617203in}{5.118073in}}{\pgfqpoint{2.613085in}{5.113954in}}%
\pgfpathcurveto{\pgfqpoint{2.608966in}{5.109836in}}{\pgfqpoint{2.606653in}{5.104250in}}{\pgfqpoint{2.606653in}{5.098426in}}%
\pgfpathcurveto{\pgfqpoint{2.606653in}{5.092602in}}{\pgfqpoint{2.608966in}{5.087016in}}{\pgfqpoint{2.613085in}{5.082898in}}%
\pgfpathcurveto{\pgfqpoint{2.617203in}{5.078780in}}{\pgfqpoint{2.622789in}{5.076466in}}{\pgfqpoint{2.628613in}{5.076466in}}%
\pgfpathlineto{\pgfqpoint{2.628613in}{5.076466in}}%
\pgfpathclose%
\pgfusepath{stroke,fill}%
\end{pgfscope}%
\begin{pgfscope}%
\pgfpathrectangle{\pgfqpoint{1.582361in}{0.880000in}}{\pgfqpoint{5.035278in}{6.160000in}}%
\pgfusepath{clip}%
\pgfsetbuttcap%
\pgfsetroundjoin%
\definecolor{currentfill}{rgb}{0.800000,0.200000,0.200000}%
\pgfsetfillcolor{currentfill}%
\pgfsetlinewidth{1.003750pt}%
\definecolor{currentstroke}{rgb}{0.800000,0.200000,0.200000}%
\pgfsetstrokecolor{currentstroke}%
\pgfsetdash{}{0pt}%
\pgfpathmoveto{\pgfqpoint{2.637036in}{5.032534in}}%
\pgfpathcurveto{\pgfqpoint{2.642860in}{5.032534in}}{\pgfqpoint{2.648447in}{5.034848in}}{\pgfqpoint{2.652565in}{5.038966in}}%
\pgfpathcurveto{\pgfqpoint{2.656683in}{5.043084in}}{\pgfqpoint{2.658997in}{5.048670in}}{\pgfqpoint{2.658997in}{5.054494in}}%
\pgfpathcurveto{\pgfqpoint{2.658997in}{5.060318in}}{\pgfqpoint{2.656683in}{5.065904in}}{\pgfqpoint{2.652565in}{5.070022in}}%
\pgfpathcurveto{\pgfqpoint{2.648447in}{5.074141in}}{\pgfqpoint{2.642860in}{5.076454in}}{\pgfqpoint{2.637036in}{5.076454in}}%
\pgfpathcurveto{\pgfqpoint{2.631213in}{5.076454in}}{\pgfqpoint{2.625626in}{5.074141in}}{\pgfqpoint{2.621508in}{5.070022in}}%
\pgfpathcurveto{\pgfqpoint{2.617390in}{5.065904in}}{\pgfqpoint{2.615076in}{5.060318in}}{\pgfqpoint{2.615076in}{5.054494in}}%
\pgfpathcurveto{\pgfqpoint{2.615076in}{5.048670in}}{\pgfqpoint{2.617390in}{5.043084in}}{\pgfqpoint{2.621508in}{5.038966in}}%
\pgfpathcurveto{\pgfqpoint{2.625626in}{5.034848in}}{\pgfqpoint{2.631213in}{5.032534in}}{\pgfqpoint{2.637036in}{5.032534in}}%
\pgfpathlineto{\pgfqpoint{2.637036in}{5.032534in}}%
\pgfpathclose%
\pgfusepath{stroke,fill}%
\end{pgfscope}%
\begin{pgfscope}%
\pgfpathrectangle{\pgfqpoint{1.582361in}{0.880000in}}{\pgfqpoint{5.035278in}{6.160000in}}%
\pgfusepath{clip}%
\pgfsetbuttcap%
\pgfsetroundjoin%
\definecolor{currentfill}{rgb}{0.800000,0.200000,0.200000}%
\pgfsetfillcolor{currentfill}%
\pgfsetlinewidth{1.003750pt}%
\definecolor{currentstroke}{rgb}{0.800000,0.200000,0.200000}%
\pgfsetstrokecolor{currentstroke}%
\pgfsetdash{}{0pt}%
\pgfpathmoveto{\pgfqpoint{2.646843in}{4.988890in}}%
\pgfpathcurveto{\pgfqpoint{2.652667in}{4.988890in}}{\pgfqpoint{2.658253in}{4.991204in}}{\pgfqpoint{2.662371in}{4.995322in}}%
\pgfpathcurveto{\pgfqpoint{2.666489in}{4.999440in}}{\pgfqpoint{2.668803in}{5.005026in}}{\pgfqpoint{2.668803in}{5.010850in}}%
\pgfpathcurveto{\pgfqpoint{2.668803in}{5.016674in}}{\pgfqpoint{2.666489in}{5.022260in}}{\pgfqpoint{2.662371in}{5.026378in}}%
\pgfpathcurveto{\pgfqpoint{2.658253in}{5.030496in}}{\pgfqpoint{2.652667in}{5.032810in}}{\pgfqpoint{2.646843in}{5.032810in}}%
\pgfpathcurveto{\pgfqpoint{2.641019in}{5.032810in}}{\pgfqpoint{2.635433in}{5.030496in}}{\pgfqpoint{2.631315in}{5.026378in}}%
\pgfpathcurveto{\pgfqpoint{2.627196in}{5.022260in}}{\pgfqpoint{2.624883in}{5.016674in}}{\pgfqpoint{2.624883in}{5.010850in}}%
\pgfpathcurveto{\pgfqpoint{2.624883in}{5.005026in}}{\pgfqpoint{2.627196in}{4.999440in}}{\pgfqpoint{2.631315in}{4.995322in}}%
\pgfpathcurveto{\pgfqpoint{2.635433in}{4.991204in}}{\pgfqpoint{2.641019in}{4.988890in}}{\pgfqpoint{2.646843in}{4.988890in}}%
\pgfpathlineto{\pgfqpoint{2.646843in}{4.988890in}}%
\pgfpathclose%
\pgfusepath{stroke,fill}%
\end{pgfscope}%
\begin{pgfscope}%
\pgfpathrectangle{\pgfqpoint{1.582361in}{0.880000in}}{\pgfqpoint{5.035278in}{6.160000in}}%
\pgfusepath{clip}%
\pgfsetbuttcap%
\pgfsetroundjoin%
\definecolor{currentfill}{rgb}{0.800000,0.200000,0.200000}%
\pgfsetfillcolor{currentfill}%
\pgfsetlinewidth{1.003750pt}%
\definecolor{currentstroke}{rgb}{0.800000,0.200000,0.200000}%
\pgfsetstrokecolor{currentstroke}%
\pgfsetdash{}{0pt}%
\pgfpathmoveto{\pgfqpoint{2.658022in}{4.945577in}}%
\pgfpathcurveto{\pgfqpoint{2.663846in}{4.945577in}}{\pgfqpoint{2.669432in}{4.947891in}}{\pgfqpoint{2.673550in}{4.952009in}}%
\pgfpathcurveto{\pgfqpoint{2.677668in}{4.956127in}}{\pgfqpoint{2.679982in}{4.961713in}}{\pgfqpoint{2.679982in}{4.967537in}}%
\pgfpathcurveto{\pgfqpoint{2.679982in}{4.973361in}}{\pgfqpoint{2.677668in}{4.978947in}}{\pgfqpoint{2.673550in}{4.983065in}}%
\pgfpathcurveto{\pgfqpoint{2.669432in}{4.987183in}}{\pgfqpoint{2.663846in}{4.989497in}}{\pgfqpoint{2.658022in}{4.989497in}}%
\pgfpathcurveto{\pgfqpoint{2.652198in}{4.989497in}}{\pgfqpoint{2.646612in}{4.987183in}}{\pgfqpoint{2.642494in}{4.983065in}}%
\pgfpathcurveto{\pgfqpoint{2.638376in}{4.978947in}}{\pgfqpoint{2.636062in}{4.973361in}}{\pgfqpoint{2.636062in}{4.967537in}}%
\pgfpathcurveto{\pgfqpoint{2.636062in}{4.961713in}}{\pgfqpoint{2.638376in}{4.956127in}}{\pgfqpoint{2.642494in}{4.952009in}}%
\pgfpathcurveto{\pgfqpoint{2.646612in}{4.947891in}}{\pgfqpoint{2.652198in}{4.945577in}}{\pgfqpoint{2.658022in}{4.945577in}}%
\pgfpathlineto{\pgfqpoint{2.658022in}{4.945577in}}%
\pgfpathclose%
\pgfusepath{stroke,fill}%
\end{pgfscope}%
\begin{pgfscope}%
\pgfpathrectangle{\pgfqpoint{1.582361in}{0.880000in}}{\pgfqpoint{5.035278in}{6.160000in}}%
\pgfusepath{clip}%
\pgfsetbuttcap%
\pgfsetroundjoin%
\definecolor{currentfill}{rgb}{0.800000,0.200000,0.200000}%
\pgfsetfillcolor{currentfill}%
\pgfsetlinewidth{1.003750pt}%
\definecolor{currentstroke}{rgb}{0.800000,0.200000,0.200000}%
\pgfsetstrokecolor{currentstroke}%
\pgfsetdash{}{0pt}%
\pgfpathmoveto{\pgfqpoint{2.670563in}{4.902638in}}%
\pgfpathcurveto{\pgfqpoint{2.676387in}{4.902638in}}{\pgfqpoint{2.681973in}{4.904952in}}{\pgfqpoint{2.686091in}{4.909070in}}%
\pgfpathcurveto{\pgfqpoint{2.690209in}{4.913188in}}{\pgfqpoint{2.692523in}{4.918775in}}{\pgfqpoint{2.692523in}{4.924599in}}%
\pgfpathcurveto{\pgfqpoint{2.692523in}{4.930423in}}{\pgfqpoint{2.690209in}{4.936009in}}{\pgfqpoint{2.686091in}{4.940127in}}%
\pgfpathcurveto{\pgfqpoint{2.681973in}{4.944245in}}{\pgfqpoint{2.676387in}{4.946559in}}{\pgfqpoint{2.670563in}{4.946559in}}%
\pgfpathcurveto{\pgfqpoint{2.664739in}{4.946559in}}{\pgfqpoint{2.659153in}{4.944245in}}{\pgfqpoint{2.655035in}{4.940127in}}%
\pgfpathcurveto{\pgfqpoint{2.650917in}{4.936009in}}{\pgfqpoint{2.648603in}{4.930423in}}{\pgfqpoint{2.648603in}{4.924599in}}%
\pgfpathcurveto{\pgfqpoint{2.648603in}{4.918775in}}{\pgfqpoint{2.650917in}{4.913188in}}{\pgfqpoint{2.655035in}{4.909070in}}%
\pgfpathcurveto{\pgfqpoint{2.659153in}{4.904952in}}{\pgfqpoint{2.664739in}{4.902638in}}{\pgfqpoint{2.670563in}{4.902638in}}%
\pgfpathlineto{\pgfqpoint{2.670563in}{4.902638in}}%
\pgfpathclose%
\pgfusepath{stroke,fill}%
\end{pgfscope}%
\begin{pgfscope}%
\pgfpathrectangle{\pgfqpoint{1.582361in}{0.880000in}}{\pgfqpoint{5.035278in}{6.160000in}}%
\pgfusepath{clip}%
\pgfsetbuttcap%
\pgfsetroundjoin%
\definecolor{currentfill}{rgb}{0.800000,0.200000,0.200000}%
\pgfsetfillcolor{currentfill}%
\pgfsetlinewidth{1.003750pt}%
\definecolor{currentstroke}{rgb}{0.800000,0.200000,0.200000}%
\pgfsetstrokecolor{currentstroke}%
\pgfsetdash{}{0pt}%
\pgfpathmoveto{\pgfqpoint{2.684453in}{4.860117in}}%
\pgfpathcurveto{\pgfqpoint{2.690277in}{4.860117in}}{\pgfqpoint{2.695863in}{4.862431in}}{\pgfqpoint{2.699981in}{4.866549in}}%
\pgfpathcurveto{\pgfqpoint{2.704100in}{4.870667in}}{\pgfqpoint{2.706413in}{4.876254in}}{\pgfqpoint{2.706413in}{4.882077in}}%
\pgfpathcurveto{\pgfqpoint{2.706413in}{4.887901in}}{\pgfqpoint{2.704100in}{4.893488in}}{\pgfqpoint{2.699981in}{4.897606in}}%
\pgfpathcurveto{\pgfqpoint{2.695863in}{4.901724in}}{\pgfqpoint{2.690277in}{4.904038in}}{\pgfqpoint{2.684453in}{4.904038in}}%
\pgfpathcurveto{\pgfqpoint{2.678629in}{4.904038in}}{\pgfqpoint{2.673043in}{4.901724in}}{\pgfqpoint{2.668925in}{4.897606in}}%
\pgfpathcurveto{\pgfqpoint{2.664807in}{4.893488in}}{\pgfqpoint{2.662493in}{4.887901in}}{\pgfqpoint{2.662493in}{4.882077in}}%
\pgfpathcurveto{\pgfqpoint{2.662493in}{4.876254in}}{\pgfqpoint{2.664807in}{4.870667in}}{\pgfqpoint{2.668925in}{4.866549in}}%
\pgfpathcurveto{\pgfqpoint{2.673043in}{4.862431in}}{\pgfqpoint{2.678629in}{4.860117in}}{\pgfqpoint{2.684453in}{4.860117in}}%
\pgfpathlineto{\pgfqpoint{2.684453in}{4.860117in}}%
\pgfpathclose%
\pgfusepath{stroke,fill}%
\end{pgfscope}%
\begin{pgfscope}%
\pgfpathrectangle{\pgfqpoint{1.582361in}{0.880000in}}{\pgfqpoint{5.035278in}{6.160000in}}%
\pgfusepath{clip}%
\pgfsetbuttcap%
\pgfsetroundjoin%
\definecolor{currentfill}{rgb}{0.800000,0.200000,0.200000}%
\pgfsetfillcolor{currentfill}%
\pgfsetlinewidth{1.003750pt}%
\definecolor{currentstroke}{rgb}{0.800000,0.200000,0.200000}%
\pgfsetstrokecolor{currentstroke}%
\pgfsetdash{}{0pt}%
\pgfpathmoveto{\pgfqpoint{2.699679in}{4.818056in}}%
\pgfpathcurveto{\pgfqpoint{2.705503in}{4.818056in}}{\pgfqpoint{2.711089in}{4.820370in}}{\pgfqpoint{2.715207in}{4.824488in}}%
\pgfpathcurveto{\pgfqpoint{2.719325in}{4.828606in}}{\pgfqpoint{2.721639in}{4.834192in}}{\pgfqpoint{2.721639in}{4.840016in}}%
\pgfpathcurveto{\pgfqpoint{2.721639in}{4.845840in}}{\pgfqpoint{2.719325in}{4.851426in}}{\pgfqpoint{2.715207in}{4.855544in}}%
\pgfpathcurveto{\pgfqpoint{2.711089in}{4.859662in}}{\pgfqpoint{2.705503in}{4.861976in}}{\pgfqpoint{2.699679in}{4.861976in}}%
\pgfpathcurveto{\pgfqpoint{2.693855in}{4.861976in}}{\pgfqpoint{2.688269in}{4.859662in}}{\pgfqpoint{2.684150in}{4.855544in}}%
\pgfpathcurveto{\pgfqpoint{2.680032in}{4.851426in}}{\pgfqpoint{2.677718in}{4.845840in}}{\pgfqpoint{2.677718in}{4.840016in}}%
\pgfpathcurveto{\pgfqpoint{2.677718in}{4.834192in}}{\pgfqpoint{2.680032in}{4.828606in}}{\pgfqpoint{2.684150in}{4.824488in}}%
\pgfpathcurveto{\pgfqpoint{2.688269in}{4.820370in}}{\pgfqpoint{2.693855in}{4.818056in}}{\pgfqpoint{2.699679in}{4.818056in}}%
\pgfpathlineto{\pgfqpoint{2.699679in}{4.818056in}}%
\pgfpathclose%
\pgfusepath{stroke,fill}%
\end{pgfscope}%
\begin{pgfscope}%
\pgfpathrectangle{\pgfqpoint{1.582361in}{0.880000in}}{\pgfqpoint{5.035278in}{6.160000in}}%
\pgfusepath{clip}%
\pgfsetbuttcap%
\pgfsetroundjoin%
\definecolor{currentfill}{rgb}{0.800000,0.200000,0.200000}%
\pgfsetfillcolor{currentfill}%
\pgfsetlinewidth{1.003750pt}%
\definecolor{currentstroke}{rgb}{0.800000,0.200000,0.200000}%
\pgfsetstrokecolor{currentstroke}%
\pgfsetdash{}{0pt}%
\pgfpathmoveto{\pgfqpoint{2.716225in}{4.776496in}}%
\pgfpathcurveto{\pgfqpoint{2.722049in}{4.776496in}}{\pgfqpoint{2.727635in}{4.778810in}}{\pgfqpoint{2.731753in}{4.782928in}}%
\pgfpathcurveto{\pgfqpoint{2.735871in}{4.787046in}}{\pgfqpoint{2.738185in}{4.792632in}}{\pgfqpoint{2.738185in}{4.798456in}}%
\pgfpathcurveto{\pgfqpoint{2.738185in}{4.804280in}}{\pgfqpoint{2.735871in}{4.809866in}}{\pgfqpoint{2.731753in}{4.813984in}}%
\pgfpathcurveto{\pgfqpoint{2.727635in}{4.818103in}}{\pgfqpoint{2.722049in}{4.820416in}}{\pgfqpoint{2.716225in}{4.820416in}}%
\pgfpathcurveto{\pgfqpoint{2.710401in}{4.820416in}}{\pgfqpoint{2.704814in}{4.818103in}}{\pgfqpoint{2.700696in}{4.813984in}}%
\pgfpathcurveto{\pgfqpoint{2.696578in}{4.809866in}}{\pgfqpoint{2.694264in}{4.804280in}}{\pgfqpoint{2.694264in}{4.798456in}}%
\pgfpathcurveto{\pgfqpoint{2.694264in}{4.792632in}}{\pgfqpoint{2.696578in}{4.787046in}}{\pgfqpoint{2.700696in}{4.782928in}}%
\pgfpathcurveto{\pgfqpoint{2.704814in}{4.778810in}}{\pgfqpoint{2.710401in}{4.776496in}}{\pgfqpoint{2.716225in}{4.776496in}}%
\pgfpathlineto{\pgfqpoint{2.716225in}{4.776496in}}%
\pgfpathclose%
\pgfusepath{stroke,fill}%
\end{pgfscope}%
\begin{pgfscope}%
\pgfpathrectangle{\pgfqpoint{1.582361in}{0.880000in}}{\pgfqpoint{5.035278in}{6.160000in}}%
\pgfusepath{clip}%
\pgfsetbuttcap%
\pgfsetroundjoin%
\definecolor{currentfill}{rgb}{0.800000,0.200000,0.200000}%
\pgfsetfillcolor{currentfill}%
\pgfsetlinewidth{1.003750pt}%
\definecolor{currentstroke}{rgb}{0.800000,0.200000,0.200000}%
\pgfsetstrokecolor{currentstroke}%
\pgfsetdash{}{0pt}%
\pgfpathmoveto{\pgfqpoint{2.734074in}{4.735479in}}%
\pgfpathcurveto{\pgfqpoint{2.739898in}{4.735479in}}{\pgfqpoint{2.745484in}{4.737793in}}{\pgfqpoint{2.749602in}{4.741911in}}%
\pgfpathcurveto{\pgfqpoint{2.753721in}{4.746029in}}{\pgfqpoint{2.756034in}{4.751616in}}{\pgfqpoint{2.756034in}{4.757439in}}%
\pgfpathcurveto{\pgfqpoint{2.756034in}{4.763263in}}{\pgfqpoint{2.753721in}{4.768850in}}{\pgfqpoint{2.749602in}{4.772968in}}%
\pgfpathcurveto{\pgfqpoint{2.745484in}{4.777086in}}{\pgfqpoint{2.739898in}{4.779400in}}{\pgfqpoint{2.734074in}{4.779400in}}%
\pgfpathcurveto{\pgfqpoint{2.728250in}{4.779400in}}{\pgfqpoint{2.722664in}{4.777086in}}{\pgfqpoint{2.718546in}{4.772968in}}%
\pgfpathcurveto{\pgfqpoint{2.714428in}{4.768850in}}{\pgfqpoint{2.712114in}{4.763263in}}{\pgfqpoint{2.712114in}{4.757439in}}%
\pgfpathcurveto{\pgfqpoint{2.712114in}{4.751616in}}{\pgfqpoint{2.714428in}{4.746029in}}{\pgfqpoint{2.718546in}{4.741911in}}%
\pgfpathcurveto{\pgfqpoint{2.722664in}{4.737793in}}{\pgfqpoint{2.728250in}{4.735479in}}{\pgfqpoint{2.734074in}{4.735479in}}%
\pgfpathlineto{\pgfqpoint{2.734074in}{4.735479in}}%
\pgfpathclose%
\pgfusepath{stroke,fill}%
\end{pgfscope}%
\begin{pgfscope}%
\pgfpathrectangle{\pgfqpoint{1.582361in}{0.880000in}}{\pgfqpoint{5.035278in}{6.160000in}}%
\pgfusepath{clip}%
\pgfsetbuttcap%
\pgfsetroundjoin%
\definecolor{currentfill}{rgb}{0.800000,0.200000,0.200000}%
\pgfsetfillcolor{currentfill}%
\pgfsetlinewidth{1.003750pt}%
\definecolor{currentstroke}{rgb}{0.800000,0.200000,0.200000}%
\pgfsetstrokecolor{currentstroke}%
\pgfsetdash{}{0pt}%
\pgfpathmoveto{\pgfqpoint{2.753210in}{4.695046in}}%
\pgfpathcurveto{\pgfqpoint{2.759034in}{4.695046in}}{\pgfqpoint{2.764620in}{4.697360in}}{\pgfqpoint{2.768738in}{4.701478in}}%
\pgfpathcurveto{\pgfqpoint{2.772856in}{4.705596in}}{\pgfqpoint{2.775170in}{4.711183in}}{\pgfqpoint{2.775170in}{4.717007in}}%
\pgfpathcurveto{\pgfqpoint{2.775170in}{4.722831in}}{\pgfqpoint{2.772856in}{4.728417in}}{\pgfqpoint{2.768738in}{4.732535in}}%
\pgfpathcurveto{\pgfqpoint{2.764620in}{4.736653in}}{\pgfqpoint{2.759034in}{4.738967in}}{\pgfqpoint{2.753210in}{4.738967in}}%
\pgfpathcurveto{\pgfqpoint{2.747386in}{4.738967in}}{\pgfqpoint{2.741800in}{4.736653in}}{\pgfqpoint{2.737681in}{4.732535in}}%
\pgfpathcurveto{\pgfqpoint{2.733563in}{4.728417in}}{\pgfqpoint{2.731249in}{4.722831in}}{\pgfqpoint{2.731249in}{4.717007in}}%
\pgfpathcurveto{\pgfqpoint{2.731249in}{4.711183in}}{\pgfqpoint{2.733563in}{4.705596in}}{\pgfqpoint{2.737681in}{4.701478in}}%
\pgfpathcurveto{\pgfqpoint{2.741800in}{4.697360in}}{\pgfqpoint{2.747386in}{4.695046in}}{\pgfqpoint{2.753210in}{4.695046in}}%
\pgfpathlineto{\pgfqpoint{2.753210in}{4.695046in}}%
\pgfpathclose%
\pgfusepath{stroke,fill}%
\end{pgfscope}%
\begin{pgfscope}%
\pgfpathrectangle{\pgfqpoint{1.582361in}{0.880000in}}{\pgfqpoint{5.035278in}{6.160000in}}%
\pgfusepath{clip}%
\pgfsetbuttcap%
\pgfsetroundjoin%
\definecolor{currentfill}{rgb}{0.800000,0.200000,0.200000}%
\pgfsetfillcolor{currentfill}%
\pgfsetlinewidth{1.003750pt}%
\definecolor{currentstroke}{rgb}{0.800000,0.200000,0.200000}%
\pgfsetstrokecolor{currentstroke}%
\pgfsetdash{}{0pt}%
\pgfpathmoveto{\pgfqpoint{2.773612in}{4.655238in}}%
\pgfpathcurveto{\pgfqpoint{2.779436in}{4.655238in}}{\pgfqpoint{2.785022in}{4.657552in}}{\pgfqpoint{2.789140in}{4.661670in}}%
\pgfpathcurveto{\pgfqpoint{2.793258in}{4.665788in}}{\pgfqpoint{2.795572in}{4.671374in}}{\pgfqpoint{2.795572in}{4.677198in}}%
\pgfpathcurveto{\pgfqpoint{2.795572in}{4.683022in}}{\pgfqpoint{2.793258in}{4.688608in}}{\pgfqpoint{2.789140in}{4.692726in}}%
\pgfpathcurveto{\pgfqpoint{2.785022in}{4.696844in}}{\pgfqpoint{2.779436in}{4.699158in}}{\pgfqpoint{2.773612in}{4.699158in}}%
\pgfpathcurveto{\pgfqpoint{2.767788in}{4.699158in}}{\pgfqpoint{2.762202in}{4.696844in}}{\pgfqpoint{2.758084in}{4.692726in}}%
\pgfpathcurveto{\pgfqpoint{2.753966in}{4.688608in}}{\pgfqpoint{2.751652in}{4.683022in}}{\pgfqpoint{2.751652in}{4.677198in}}%
\pgfpathcurveto{\pgfqpoint{2.751652in}{4.671374in}}{\pgfqpoint{2.753966in}{4.665788in}}{\pgfqpoint{2.758084in}{4.661670in}}%
\pgfpathcurveto{\pgfqpoint{2.762202in}{4.657552in}}{\pgfqpoint{2.767788in}{4.655238in}}{\pgfqpoint{2.773612in}{4.655238in}}%
\pgfpathlineto{\pgfqpoint{2.773612in}{4.655238in}}%
\pgfpathclose%
\pgfusepath{stroke,fill}%
\end{pgfscope}%
\begin{pgfscope}%
\pgfpathrectangle{\pgfqpoint{1.582361in}{0.880000in}}{\pgfqpoint{5.035278in}{6.160000in}}%
\pgfusepath{clip}%
\pgfsetbuttcap%
\pgfsetroundjoin%
\definecolor{currentfill}{rgb}{0.800000,0.200000,0.200000}%
\pgfsetfillcolor{currentfill}%
\pgfsetlinewidth{1.003750pt}%
\definecolor{currentstroke}{rgb}{0.800000,0.200000,0.200000}%
\pgfsetstrokecolor{currentstroke}%
\pgfsetdash{}{0pt}%
\pgfpathmoveto{\pgfqpoint{2.795261in}{4.616093in}}%
\pgfpathcurveto{\pgfqpoint{2.801085in}{4.616093in}}{\pgfqpoint{2.806671in}{4.618407in}}{\pgfqpoint{2.810789in}{4.622525in}}%
\pgfpathcurveto{\pgfqpoint{2.814907in}{4.626643in}}{\pgfqpoint{2.817221in}{4.632229in}}{\pgfqpoint{2.817221in}{4.638053in}}%
\pgfpathcurveto{\pgfqpoint{2.817221in}{4.643877in}}{\pgfqpoint{2.814907in}{4.649463in}}{\pgfqpoint{2.810789in}{4.653582in}}%
\pgfpathcurveto{\pgfqpoint{2.806671in}{4.657700in}}{\pgfqpoint{2.801085in}{4.660014in}}{\pgfqpoint{2.795261in}{4.660014in}}%
\pgfpathcurveto{\pgfqpoint{2.789437in}{4.660014in}}{\pgfqpoint{2.783851in}{4.657700in}}{\pgfqpoint{2.779733in}{4.653582in}}%
\pgfpathcurveto{\pgfqpoint{2.775615in}{4.649463in}}{\pgfqpoint{2.773301in}{4.643877in}}{\pgfqpoint{2.773301in}{4.638053in}}%
\pgfpathcurveto{\pgfqpoint{2.773301in}{4.632229in}}{\pgfqpoint{2.775615in}{4.626643in}}{\pgfqpoint{2.779733in}{4.622525in}}%
\pgfpathcurveto{\pgfqpoint{2.783851in}{4.618407in}}{\pgfqpoint{2.789437in}{4.616093in}}{\pgfqpoint{2.795261in}{4.616093in}}%
\pgfpathlineto{\pgfqpoint{2.795261in}{4.616093in}}%
\pgfpathclose%
\pgfusepath{stroke,fill}%
\end{pgfscope}%
\begin{pgfscope}%
\pgfpathrectangle{\pgfqpoint{1.582361in}{0.880000in}}{\pgfqpoint{5.035278in}{6.160000in}}%
\pgfusepath{clip}%
\pgfsetbuttcap%
\pgfsetroundjoin%
\definecolor{currentfill}{rgb}{0.800000,0.200000,0.200000}%
\pgfsetfillcolor{currentfill}%
\pgfsetlinewidth{1.003750pt}%
\definecolor{currentstroke}{rgb}{0.800000,0.200000,0.200000}%
\pgfsetstrokecolor{currentstroke}%
\pgfsetdash{}{0pt}%
\pgfpathmoveto{\pgfqpoint{2.818135in}{4.577651in}}%
\pgfpathcurveto{\pgfqpoint{2.823959in}{4.577651in}}{\pgfqpoint{2.829545in}{4.579965in}}{\pgfqpoint{2.833663in}{4.584083in}}%
\pgfpathcurveto{\pgfqpoint{2.837781in}{4.588201in}}{\pgfqpoint{2.840095in}{4.593788in}}{\pgfqpoint{2.840095in}{4.599612in}}%
\pgfpathcurveto{\pgfqpoint{2.840095in}{4.605435in}}{\pgfqpoint{2.837781in}{4.611022in}}{\pgfqpoint{2.833663in}{4.615140in}}%
\pgfpathcurveto{\pgfqpoint{2.829545in}{4.619258in}}{\pgfqpoint{2.823959in}{4.621572in}}{\pgfqpoint{2.818135in}{4.621572in}}%
\pgfpathcurveto{\pgfqpoint{2.812311in}{4.621572in}}{\pgfqpoint{2.806725in}{4.619258in}}{\pgfqpoint{2.802607in}{4.615140in}}%
\pgfpathcurveto{\pgfqpoint{2.798488in}{4.611022in}}{\pgfqpoint{2.796175in}{4.605435in}}{\pgfqpoint{2.796175in}{4.599612in}}%
\pgfpathcurveto{\pgfqpoint{2.796175in}{4.593788in}}{\pgfqpoint{2.798488in}{4.588201in}}{\pgfqpoint{2.802607in}{4.584083in}}%
\pgfpathcurveto{\pgfqpoint{2.806725in}{4.579965in}}{\pgfqpoint{2.812311in}{4.577651in}}{\pgfqpoint{2.818135in}{4.577651in}}%
\pgfpathlineto{\pgfqpoint{2.818135in}{4.577651in}}%
\pgfpathclose%
\pgfusepath{stroke,fill}%
\end{pgfscope}%
\begin{pgfscope}%
\pgfpathrectangle{\pgfqpoint{1.582361in}{0.880000in}}{\pgfqpoint{5.035278in}{6.160000in}}%
\pgfusepath{clip}%
\pgfsetbuttcap%
\pgfsetroundjoin%
\definecolor{currentfill}{rgb}{0.800000,0.200000,0.200000}%
\pgfsetfillcolor{currentfill}%
\pgfsetlinewidth{1.003750pt}%
\definecolor{currentstroke}{rgb}{0.800000,0.200000,0.200000}%
\pgfsetstrokecolor{currentstroke}%
\pgfsetdash{}{0pt}%
\pgfpathmoveto{\pgfqpoint{2.842211in}{4.539951in}}%
\pgfpathcurveto{\pgfqpoint{2.848035in}{4.539951in}}{\pgfqpoint{2.853621in}{4.542265in}}{\pgfqpoint{2.857739in}{4.546383in}}%
\pgfpathcurveto{\pgfqpoint{2.861857in}{4.550501in}}{\pgfqpoint{2.864171in}{4.556087in}}{\pgfqpoint{2.864171in}{4.561911in}}%
\pgfpathcurveto{\pgfqpoint{2.864171in}{4.567735in}}{\pgfqpoint{2.861857in}{4.573321in}}{\pgfqpoint{2.857739in}{4.577439in}}%
\pgfpathcurveto{\pgfqpoint{2.853621in}{4.581557in}}{\pgfqpoint{2.848035in}{4.583871in}}{\pgfqpoint{2.842211in}{4.583871in}}%
\pgfpathcurveto{\pgfqpoint{2.836387in}{4.583871in}}{\pgfqpoint{2.830801in}{4.581557in}}{\pgfqpoint{2.826683in}{4.577439in}}%
\pgfpathcurveto{\pgfqpoint{2.822564in}{4.573321in}}{\pgfqpoint{2.820251in}{4.567735in}}{\pgfqpoint{2.820251in}{4.561911in}}%
\pgfpathcurveto{\pgfqpoint{2.820251in}{4.556087in}}{\pgfqpoint{2.822564in}{4.550501in}}{\pgfqpoint{2.826683in}{4.546383in}}%
\pgfpathcurveto{\pgfqpoint{2.830801in}{4.542265in}}{\pgfqpoint{2.836387in}{4.539951in}}{\pgfqpoint{2.842211in}{4.539951in}}%
\pgfpathlineto{\pgfqpoint{2.842211in}{4.539951in}}%
\pgfpathclose%
\pgfusepath{stroke,fill}%
\end{pgfscope}%
\begin{pgfscope}%
\pgfpathrectangle{\pgfqpoint{1.582361in}{0.880000in}}{\pgfqpoint{5.035278in}{6.160000in}}%
\pgfusepath{clip}%
\pgfsetbuttcap%
\pgfsetroundjoin%
\definecolor{currentfill}{rgb}{0.800000,0.200000,0.200000}%
\pgfsetfillcolor{currentfill}%
\pgfsetlinewidth{1.003750pt}%
\definecolor{currentstroke}{rgb}{0.800000,0.200000,0.200000}%
\pgfsetstrokecolor{currentstroke}%
\pgfsetdash{}{0pt}%
\pgfpathmoveto{\pgfqpoint{2.867465in}{4.503029in}}%
\pgfpathcurveto{\pgfqpoint{2.873289in}{4.503029in}}{\pgfqpoint{2.878875in}{4.505343in}}{\pgfqpoint{2.882993in}{4.509461in}}%
\pgfpathcurveto{\pgfqpoint{2.887111in}{4.513579in}}{\pgfqpoint{2.889425in}{4.519165in}}{\pgfqpoint{2.889425in}{4.524989in}}%
\pgfpathcurveto{\pgfqpoint{2.889425in}{4.530813in}}{\pgfqpoint{2.887111in}{4.536399in}}{\pgfqpoint{2.882993in}{4.540518in}}%
\pgfpathcurveto{\pgfqpoint{2.878875in}{4.544636in}}{\pgfqpoint{2.873289in}{4.546950in}}{\pgfqpoint{2.867465in}{4.546950in}}%
\pgfpathcurveto{\pgfqpoint{2.861641in}{4.546950in}}{\pgfqpoint{2.856055in}{4.544636in}}{\pgfqpoint{2.851937in}{4.540518in}}%
\pgfpathcurveto{\pgfqpoint{2.847819in}{4.536399in}}{\pgfqpoint{2.845505in}{4.530813in}}{\pgfqpoint{2.845505in}{4.524989in}}%
\pgfpathcurveto{\pgfqpoint{2.845505in}{4.519165in}}{\pgfqpoint{2.847819in}{4.513579in}}{\pgfqpoint{2.851937in}{4.509461in}}%
\pgfpathcurveto{\pgfqpoint{2.856055in}{4.505343in}}{\pgfqpoint{2.861641in}{4.503029in}}{\pgfqpoint{2.867465in}{4.503029in}}%
\pgfpathlineto{\pgfqpoint{2.867465in}{4.503029in}}%
\pgfpathclose%
\pgfusepath{stroke,fill}%
\end{pgfscope}%
\begin{pgfscope}%
\pgfpathrectangle{\pgfqpoint{1.582361in}{0.880000in}}{\pgfqpoint{5.035278in}{6.160000in}}%
\pgfusepath{clip}%
\pgfsetbuttcap%
\pgfsetroundjoin%
\definecolor{currentfill}{rgb}{0.800000,0.200000,0.200000}%
\pgfsetfillcolor{currentfill}%
\pgfsetlinewidth{1.003750pt}%
\definecolor{currentstroke}{rgb}{0.800000,0.200000,0.200000}%
\pgfsetstrokecolor{currentstroke}%
\pgfsetdash{}{0pt}%
\pgfpathmoveto{\pgfqpoint{2.893872in}{4.466923in}}%
\pgfpathcurveto{\pgfqpoint{2.899696in}{4.466923in}}{\pgfqpoint{2.905282in}{4.469237in}}{\pgfqpoint{2.909400in}{4.473355in}}%
\pgfpathcurveto{\pgfqpoint{2.913519in}{4.477473in}}{\pgfqpoint{2.915832in}{4.483059in}}{\pgfqpoint{2.915832in}{4.488883in}}%
\pgfpathcurveto{\pgfqpoint{2.915832in}{4.494707in}}{\pgfqpoint{2.913519in}{4.500293in}}{\pgfqpoint{2.909400in}{4.504412in}}%
\pgfpathcurveto{\pgfqpoint{2.905282in}{4.508530in}}{\pgfqpoint{2.899696in}{4.510844in}}{\pgfqpoint{2.893872in}{4.510844in}}%
\pgfpathcurveto{\pgfqpoint{2.888048in}{4.510844in}}{\pgfqpoint{2.882462in}{4.508530in}}{\pgfqpoint{2.878344in}{4.504412in}}%
\pgfpathcurveto{\pgfqpoint{2.874226in}{4.500293in}}{\pgfqpoint{2.871912in}{4.494707in}}{\pgfqpoint{2.871912in}{4.488883in}}%
\pgfpathcurveto{\pgfqpoint{2.871912in}{4.483059in}}{\pgfqpoint{2.874226in}{4.477473in}}{\pgfqpoint{2.878344in}{4.473355in}}%
\pgfpathcurveto{\pgfqpoint{2.882462in}{4.469237in}}{\pgfqpoint{2.888048in}{4.466923in}}{\pgfqpoint{2.893872in}{4.466923in}}%
\pgfpathlineto{\pgfqpoint{2.893872in}{4.466923in}}%
\pgfpathclose%
\pgfusepath{stroke,fill}%
\end{pgfscope}%
\begin{pgfscope}%
\pgfpathrectangle{\pgfqpoint{1.582361in}{0.880000in}}{\pgfqpoint{5.035278in}{6.160000in}}%
\pgfusepath{clip}%
\pgfsetbuttcap%
\pgfsetroundjoin%
\definecolor{currentfill}{rgb}{0.800000,0.200000,0.200000}%
\pgfsetfillcolor{currentfill}%
\pgfsetlinewidth{1.003750pt}%
\definecolor{currentstroke}{rgb}{0.800000,0.200000,0.200000}%
\pgfsetstrokecolor{currentstroke}%
\pgfsetdash{}{0pt}%
\pgfpathmoveto{\pgfqpoint{2.921406in}{4.431669in}}%
\pgfpathcurveto{\pgfqpoint{2.927230in}{4.431669in}}{\pgfqpoint{2.932816in}{4.433983in}}{\pgfqpoint{2.936934in}{4.438101in}}%
\pgfpathcurveto{\pgfqpoint{2.941052in}{4.442219in}}{\pgfqpoint{2.943366in}{4.447805in}}{\pgfqpoint{2.943366in}{4.453629in}}%
\pgfpathcurveto{\pgfqpoint{2.943366in}{4.459453in}}{\pgfqpoint{2.941052in}{4.465039in}}{\pgfqpoint{2.936934in}{4.469157in}}%
\pgfpathcurveto{\pgfqpoint{2.932816in}{4.473275in}}{\pgfqpoint{2.927230in}{4.475589in}}{\pgfqpoint{2.921406in}{4.475589in}}%
\pgfpathcurveto{\pgfqpoint{2.915582in}{4.475589in}}{\pgfqpoint{2.909996in}{4.473275in}}{\pgfqpoint{2.905878in}{4.469157in}}%
\pgfpathcurveto{\pgfqpoint{2.901760in}{4.465039in}}{\pgfqpoint{2.899446in}{4.459453in}}{\pgfqpoint{2.899446in}{4.453629in}}%
\pgfpathcurveto{\pgfqpoint{2.899446in}{4.447805in}}{\pgfqpoint{2.901760in}{4.442219in}}{\pgfqpoint{2.905878in}{4.438101in}}%
\pgfpathcurveto{\pgfqpoint{2.909996in}{4.433983in}}{\pgfqpoint{2.915582in}{4.431669in}}{\pgfqpoint{2.921406in}{4.431669in}}%
\pgfpathlineto{\pgfqpoint{2.921406in}{4.431669in}}%
\pgfpathclose%
\pgfusepath{stroke,fill}%
\end{pgfscope}%
\begin{pgfscope}%
\pgfpathrectangle{\pgfqpoint{1.582361in}{0.880000in}}{\pgfqpoint{5.035278in}{6.160000in}}%
\pgfusepath{clip}%
\pgfsetbuttcap%
\pgfsetroundjoin%
\definecolor{currentfill}{rgb}{0.800000,0.200000,0.200000}%
\pgfsetfillcolor{currentfill}%
\pgfsetlinewidth{1.003750pt}%
\definecolor{currentstroke}{rgb}{0.800000,0.200000,0.200000}%
\pgfsetstrokecolor{currentstroke}%
\pgfsetdash{}{0pt}%
\pgfpathmoveto{\pgfqpoint{2.950039in}{4.397301in}}%
\pgfpathcurveto{\pgfqpoint{2.955863in}{4.397301in}}{\pgfqpoint{2.961449in}{4.399615in}}{\pgfqpoint{2.965567in}{4.403733in}}%
\pgfpathcurveto{\pgfqpoint{2.969685in}{4.407851in}}{\pgfqpoint{2.971999in}{4.413437in}}{\pgfqpoint{2.971999in}{4.419261in}}%
\pgfpathcurveto{\pgfqpoint{2.971999in}{4.425085in}}{\pgfqpoint{2.969685in}{4.430671in}}{\pgfqpoint{2.965567in}{4.434790in}}%
\pgfpathcurveto{\pgfqpoint{2.961449in}{4.438908in}}{\pgfqpoint{2.955863in}{4.441222in}}{\pgfqpoint{2.950039in}{4.441222in}}%
\pgfpathcurveto{\pgfqpoint{2.944215in}{4.441222in}}{\pgfqpoint{2.938629in}{4.438908in}}{\pgfqpoint{2.934511in}{4.434790in}}%
\pgfpathcurveto{\pgfqpoint{2.930393in}{4.430671in}}{\pgfqpoint{2.928079in}{4.425085in}}{\pgfqpoint{2.928079in}{4.419261in}}%
\pgfpathcurveto{\pgfqpoint{2.928079in}{4.413437in}}{\pgfqpoint{2.930393in}{4.407851in}}{\pgfqpoint{2.934511in}{4.403733in}}%
\pgfpathcurveto{\pgfqpoint{2.938629in}{4.399615in}}{\pgfqpoint{2.944215in}{4.397301in}}{\pgfqpoint{2.950039in}{4.397301in}}%
\pgfpathlineto{\pgfqpoint{2.950039in}{4.397301in}}%
\pgfpathclose%
\pgfusepath{stroke,fill}%
\end{pgfscope}%
\begin{pgfscope}%
\pgfpathrectangle{\pgfqpoint{1.582361in}{0.880000in}}{\pgfqpoint{5.035278in}{6.160000in}}%
\pgfusepath{clip}%
\pgfsetbuttcap%
\pgfsetroundjoin%
\definecolor{currentfill}{rgb}{0.800000,0.200000,0.200000}%
\pgfsetfillcolor{currentfill}%
\pgfsetlinewidth{1.003750pt}%
\definecolor{currentstroke}{rgb}{0.800000,0.200000,0.200000}%
\pgfsetstrokecolor{currentstroke}%
\pgfsetdash{}{0pt}%
\pgfpathmoveto{\pgfqpoint{2.979743in}{4.363854in}}%
\pgfpathcurveto{\pgfqpoint{2.985567in}{4.363854in}}{\pgfqpoint{2.991153in}{4.366168in}}{\pgfqpoint{2.995271in}{4.370286in}}%
\pgfpathcurveto{\pgfqpoint{2.999389in}{4.374405in}}{\pgfqpoint{3.001703in}{4.379991in}}{\pgfqpoint{3.001703in}{4.385815in}}%
\pgfpathcurveto{\pgfqpoint{3.001703in}{4.391639in}}{\pgfqpoint{2.999389in}{4.397225in}}{\pgfqpoint{2.995271in}{4.401343in}}%
\pgfpathcurveto{\pgfqpoint{2.991153in}{4.405461in}}{\pgfqpoint{2.985567in}{4.407775in}}{\pgfqpoint{2.979743in}{4.407775in}}%
\pgfpathcurveto{\pgfqpoint{2.973919in}{4.407775in}}{\pgfqpoint{2.968333in}{4.405461in}}{\pgfqpoint{2.964214in}{4.401343in}}%
\pgfpathcurveto{\pgfqpoint{2.960096in}{4.397225in}}{\pgfqpoint{2.957782in}{4.391639in}}{\pgfqpoint{2.957782in}{4.385815in}}%
\pgfpathcurveto{\pgfqpoint{2.957782in}{4.379991in}}{\pgfqpoint{2.960096in}{4.374405in}}{\pgfqpoint{2.964214in}{4.370286in}}%
\pgfpathcurveto{\pgfqpoint{2.968333in}{4.366168in}}{\pgfqpoint{2.973919in}{4.363854in}}{\pgfqpoint{2.979743in}{4.363854in}}%
\pgfpathlineto{\pgfqpoint{2.979743in}{4.363854in}}%
\pgfpathclose%
\pgfusepath{stroke,fill}%
\end{pgfscope}%
\begin{pgfscope}%
\pgfpathrectangle{\pgfqpoint{1.582361in}{0.880000in}}{\pgfqpoint{5.035278in}{6.160000in}}%
\pgfusepath{clip}%
\pgfsetbuttcap%
\pgfsetroundjoin%
\definecolor{currentfill}{rgb}{0.800000,0.200000,0.200000}%
\pgfsetfillcolor{currentfill}%
\pgfsetlinewidth{1.003750pt}%
\definecolor{currentstroke}{rgb}{0.800000,0.200000,0.200000}%
\pgfsetstrokecolor{currentstroke}%
\pgfsetdash{}{0pt}%
\pgfpathmoveto{\pgfqpoint{3.010487in}{4.331362in}}%
\pgfpathcurveto{\pgfqpoint{3.016311in}{4.331362in}}{\pgfqpoint{3.021898in}{4.333676in}}{\pgfqpoint{3.026016in}{4.337794in}}%
\pgfpathcurveto{\pgfqpoint{3.030134in}{4.341912in}}{\pgfqpoint{3.032448in}{4.347499in}}{\pgfqpoint{3.032448in}{4.353322in}}%
\pgfpathcurveto{\pgfqpoint{3.032448in}{4.359146in}}{\pgfqpoint{3.030134in}{4.364733in}}{\pgfqpoint{3.026016in}{4.368851in}}%
\pgfpathcurveto{\pgfqpoint{3.021898in}{4.372969in}}{\pgfqpoint{3.016311in}{4.375283in}}{\pgfqpoint{3.010487in}{4.375283in}}%
\pgfpathcurveto{\pgfqpoint{3.004663in}{4.375283in}}{\pgfqpoint{2.999077in}{4.372969in}}{\pgfqpoint{2.994959in}{4.368851in}}%
\pgfpathcurveto{\pgfqpoint{2.990841in}{4.364733in}}{\pgfqpoint{2.988527in}{4.359146in}}{\pgfqpoint{2.988527in}{4.353322in}}%
\pgfpathcurveto{\pgfqpoint{2.988527in}{4.347499in}}{\pgfqpoint{2.990841in}{4.341912in}}{\pgfqpoint{2.994959in}{4.337794in}}%
\pgfpathcurveto{\pgfqpoint{2.999077in}{4.333676in}}{\pgfqpoint{3.004663in}{4.331362in}}{\pgfqpoint{3.010487in}{4.331362in}}%
\pgfpathlineto{\pgfqpoint{3.010487in}{4.331362in}}%
\pgfpathclose%
\pgfusepath{stroke,fill}%
\end{pgfscope}%
\begin{pgfscope}%
\pgfpathrectangle{\pgfqpoint{1.582361in}{0.880000in}}{\pgfqpoint{5.035278in}{6.160000in}}%
\pgfusepath{clip}%
\pgfsetbuttcap%
\pgfsetroundjoin%
\definecolor{currentfill}{rgb}{0.800000,0.200000,0.200000}%
\pgfsetfillcolor{currentfill}%
\pgfsetlinewidth{1.003750pt}%
\definecolor{currentstroke}{rgb}{0.800000,0.200000,0.200000}%
\pgfsetstrokecolor{currentstroke}%
\pgfsetdash{}{0pt}%
\pgfpathmoveto{\pgfqpoint{3.042243in}{4.299857in}}%
\pgfpathcurveto{\pgfqpoint{3.048066in}{4.299857in}}{\pgfqpoint{3.053653in}{4.302171in}}{\pgfqpoint{3.057771in}{4.306289in}}%
\pgfpathcurveto{\pgfqpoint{3.061889in}{4.310407in}}{\pgfqpoint{3.064203in}{4.315993in}}{\pgfqpoint{3.064203in}{4.321817in}}%
\pgfpathcurveto{\pgfqpoint{3.064203in}{4.327641in}}{\pgfqpoint{3.061889in}{4.333227in}}{\pgfqpoint{3.057771in}{4.337345in}}%
\pgfpathcurveto{\pgfqpoint{3.053653in}{4.341463in}}{\pgfqpoint{3.048066in}{4.343777in}}{\pgfqpoint{3.042243in}{4.343777in}}%
\pgfpathcurveto{\pgfqpoint{3.036419in}{4.343777in}}{\pgfqpoint{3.030832in}{4.341463in}}{\pgfqpoint{3.026714in}{4.337345in}}%
\pgfpathcurveto{\pgfqpoint{3.022596in}{4.333227in}}{\pgfqpoint{3.020282in}{4.327641in}}{\pgfqpoint{3.020282in}{4.321817in}}%
\pgfpathcurveto{\pgfqpoint{3.020282in}{4.315993in}}{\pgfqpoint{3.022596in}{4.310407in}}{\pgfqpoint{3.026714in}{4.306289in}}%
\pgfpathcurveto{\pgfqpoint{3.030832in}{4.302171in}}{\pgfqpoint{3.036419in}{4.299857in}}{\pgfqpoint{3.042243in}{4.299857in}}%
\pgfpathlineto{\pgfqpoint{3.042243in}{4.299857in}}%
\pgfpathclose%
\pgfusepath{stroke,fill}%
\end{pgfscope}%
\begin{pgfscope}%
\pgfpathrectangle{\pgfqpoint{1.582361in}{0.880000in}}{\pgfqpoint{5.035278in}{6.160000in}}%
\pgfusepath{clip}%
\pgfsetbuttcap%
\pgfsetroundjoin%
\definecolor{currentfill}{rgb}{0.800000,0.200000,0.200000}%
\pgfsetfillcolor{currentfill}%
\pgfsetlinewidth{1.003750pt}%
\definecolor{currentstroke}{rgb}{0.800000,0.200000,0.200000}%
\pgfsetstrokecolor{currentstroke}%
\pgfsetdash{}{0pt}%
\pgfpathmoveto{\pgfqpoint{3.074976in}{4.269369in}}%
\pgfpathcurveto{\pgfqpoint{3.080800in}{4.269369in}}{\pgfqpoint{3.086387in}{4.271683in}}{\pgfqpoint{3.090505in}{4.275801in}}%
\pgfpathcurveto{\pgfqpoint{3.094623in}{4.279920in}}{\pgfqpoint{3.096937in}{4.285506in}}{\pgfqpoint{3.096937in}{4.291330in}}%
\pgfpathcurveto{\pgfqpoint{3.096937in}{4.297154in}}{\pgfqpoint{3.094623in}{4.302740in}}{\pgfqpoint{3.090505in}{4.306858in}}%
\pgfpathcurveto{\pgfqpoint{3.086387in}{4.310976in}}{\pgfqpoint{3.080800in}{4.313290in}}{\pgfqpoint{3.074976in}{4.313290in}}%
\pgfpathcurveto{\pgfqpoint{3.069153in}{4.313290in}}{\pgfqpoint{3.063566in}{4.310976in}}{\pgfqpoint{3.059448in}{4.306858in}}%
\pgfpathcurveto{\pgfqpoint{3.055330in}{4.302740in}}{\pgfqpoint{3.053016in}{4.297154in}}{\pgfqpoint{3.053016in}{4.291330in}}%
\pgfpathcurveto{\pgfqpoint{3.053016in}{4.285506in}}{\pgfqpoint{3.055330in}{4.279920in}}{\pgfqpoint{3.059448in}{4.275801in}}%
\pgfpathcurveto{\pgfqpoint{3.063566in}{4.271683in}}{\pgfqpoint{3.069153in}{4.269369in}}{\pgfqpoint{3.074976in}{4.269369in}}%
\pgfpathlineto{\pgfqpoint{3.074976in}{4.269369in}}%
\pgfpathclose%
\pgfusepath{stroke,fill}%
\end{pgfscope}%
\begin{pgfscope}%
\pgfpathrectangle{\pgfqpoint{1.582361in}{0.880000in}}{\pgfqpoint{5.035278in}{6.160000in}}%
\pgfusepath{clip}%
\pgfsetbuttcap%
\pgfsetroundjoin%
\definecolor{currentfill}{rgb}{0.800000,0.200000,0.200000}%
\pgfsetfillcolor{currentfill}%
\pgfsetlinewidth{1.003750pt}%
\definecolor{currentstroke}{rgb}{0.800000,0.200000,0.200000}%
\pgfsetstrokecolor{currentstroke}%
\pgfsetdash{}{0pt}%
\pgfpathmoveto{\pgfqpoint{3.108656in}{4.239931in}}%
\pgfpathcurveto{\pgfqpoint{3.114480in}{4.239931in}}{\pgfqpoint{3.120067in}{4.242245in}}{\pgfqpoint{3.124185in}{4.246363in}}%
\pgfpathcurveto{\pgfqpoint{3.128303in}{4.250481in}}{\pgfqpoint{3.130617in}{4.256067in}}{\pgfqpoint{3.130617in}{4.261891in}}%
\pgfpathcurveto{\pgfqpoint{3.130617in}{4.267715in}}{\pgfqpoint{3.128303in}{4.273301in}}{\pgfqpoint{3.124185in}{4.277419in}}%
\pgfpathcurveto{\pgfqpoint{3.120067in}{4.281537in}}{\pgfqpoint{3.114480in}{4.283851in}}{\pgfqpoint{3.108656in}{4.283851in}}%
\pgfpathcurveto{\pgfqpoint{3.102833in}{4.283851in}}{\pgfqpoint{3.097246in}{4.281537in}}{\pgfqpoint{3.093128in}{4.277419in}}%
\pgfpathcurveto{\pgfqpoint{3.089010in}{4.273301in}}{\pgfqpoint{3.086696in}{4.267715in}}{\pgfqpoint{3.086696in}{4.261891in}}%
\pgfpathcurveto{\pgfqpoint{3.086696in}{4.256067in}}{\pgfqpoint{3.089010in}{4.250481in}}{\pgfqpoint{3.093128in}{4.246363in}}%
\pgfpathcurveto{\pgfqpoint{3.097246in}{4.242245in}}{\pgfqpoint{3.102833in}{4.239931in}}{\pgfqpoint{3.108656in}{4.239931in}}%
\pgfpathlineto{\pgfqpoint{3.108656in}{4.239931in}}%
\pgfpathclose%
\pgfusepath{stroke,fill}%
\end{pgfscope}%
\begin{pgfscope}%
\pgfpathrectangle{\pgfqpoint{1.582361in}{0.880000in}}{\pgfqpoint{5.035278in}{6.160000in}}%
\pgfusepath{clip}%
\pgfsetbuttcap%
\pgfsetroundjoin%
\definecolor{currentfill}{rgb}{0.800000,0.200000,0.200000}%
\pgfsetfillcolor{currentfill}%
\pgfsetlinewidth{1.003750pt}%
\definecolor{currentstroke}{rgb}{0.800000,0.200000,0.200000}%
\pgfsetstrokecolor{currentstroke}%
\pgfsetdash{}{0pt}%
\pgfpathmoveto{\pgfqpoint{3.143249in}{4.211570in}}%
\pgfpathcurveto{\pgfqpoint{3.149073in}{4.211570in}}{\pgfqpoint{3.154659in}{4.213884in}}{\pgfqpoint{3.158777in}{4.218002in}}%
\pgfpathcurveto{\pgfqpoint{3.162895in}{4.222120in}}{\pgfqpoint{3.165209in}{4.227706in}}{\pgfqpoint{3.165209in}{4.233530in}}%
\pgfpathcurveto{\pgfqpoint{3.165209in}{4.239354in}}{\pgfqpoint{3.162895in}{4.244940in}}{\pgfqpoint{3.158777in}{4.249058in}}%
\pgfpathcurveto{\pgfqpoint{3.154659in}{4.253176in}}{\pgfqpoint{3.149073in}{4.255490in}}{\pgfqpoint{3.143249in}{4.255490in}}%
\pgfpathcurveto{\pgfqpoint{3.137425in}{4.255490in}}{\pgfqpoint{3.131839in}{4.253176in}}{\pgfqpoint{3.127721in}{4.249058in}}%
\pgfpathcurveto{\pgfqpoint{3.123603in}{4.244940in}}{\pgfqpoint{3.121289in}{4.239354in}}{\pgfqpoint{3.121289in}{4.233530in}}%
\pgfpathcurveto{\pgfqpoint{3.121289in}{4.227706in}}{\pgfqpoint{3.123603in}{4.222120in}}{\pgfqpoint{3.127721in}{4.218002in}}%
\pgfpathcurveto{\pgfqpoint{3.131839in}{4.213884in}}{\pgfqpoint{3.137425in}{4.211570in}}{\pgfqpoint{3.143249in}{4.211570in}}%
\pgfpathlineto{\pgfqpoint{3.143249in}{4.211570in}}%
\pgfpathclose%
\pgfusepath{stroke,fill}%
\end{pgfscope}%
\begin{pgfscope}%
\pgfpathrectangle{\pgfqpoint{1.582361in}{0.880000in}}{\pgfqpoint{5.035278in}{6.160000in}}%
\pgfusepath{clip}%
\pgfsetbuttcap%
\pgfsetroundjoin%
\definecolor{currentfill}{rgb}{0.800000,0.200000,0.200000}%
\pgfsetfillcolor{currentfill}%
\pgfsetlinewidth{1.003750pt}%
\definecolor{currentstroke}{rgb}{0.800000,0.200000,0.200000}%
\pgfsetstrokecolor{currentstroke}%
\pgfsetdash{}{0pt}%
\pgfpathmoveto{\pgfqpoint{3.178720in}{4.184315in}}%
\pgfpathcurveto{\pgfqpoint{3.184544in}{4.184315in}}{\pgfqpoint{3.190130in}{4.186629in}}{\pgfqpoint{3.194248in}{4.190747in}}%
\pgfpathcurveto{\pgfqpoint{3.198366in}{4.194865in}}{\pgfqpoint{3.200680in}{4.200451in}}{\pgfqpoint{3.200680in}{4.206275in}}%
\pgfpathcurveto{\pgfqpoint{3.200680in}{4.212099in}}{\pgfqpoint{3.198366in}{4.217686in}}{\pgfqpoint{3.194248in}{4.221804in}}%
\pgfpathcurveto{\pgfqpoint{3.190130in}{4.225922in}}{\pgfqpoint{3.184544in}{4.228236in}}{\pgfqpoint{3.178720in}{4.228236in}}%
\pgfpathcurveto{\pgfqpoint{3.172896in}{4.228236in}}{\pgfqpoint{3.167310in}{4.225922in}}{\pgfqpoint{3.163191in}{4.221804in}}%
\pgfpathcurveto{\pgfqpoint{3.159073in}{4.217686in}}{\pgfqpoint{3.156759in}{4.212099in}}{\pgfqpoint{3.156759in}{4.206275in}}%
\pgfpathcurveto{\pgfqpoint{3.156759in}{4.200451in}}{\pgfqpoint{3.159073in}{4.194865in}}{\pgfqpoint{3.163191in}{4.190747in}}%
\pgfpathcurveto{\pgfqpoint{3.167310in}{4.186629in}}{\pgfqpoint{3.172896in}{4.184315in}}{\pgfqpoint{3.178720in}{4.184315in}}%
\pgfpathlineto{\pgfqpoint{3.178720in}{4.184315in}}%
\pgfpathclose%
\pgfusepath{stroke,fill}%
\end{pgfscope}%
\begin{pgfscope}%
\pgfpathrectangle{\pgfqpoint{1.582361in}{0.880000in}}{\pgfqpoint{5.035278in}{6.160000in}}%
\pgfusepath{clip}%
\pgfsetbuttcap%
\pgfsetroundjoin%
\definecolor{currentfill}{rgb}{0.800000,0.200000,0.200000}%
\pgfsetfillcolor{currentfill}%
\pgfsetlinewidth{1.003750pt}%
\definecolor{currentstroke}{rgb}{0.800000,0.200000,0.200000}%
\pgfsetstrokecolor{currentstroke}%
\pgfsetdash{}{0pt}%
\pgfpathmoveto{\pgfqpoint{3.215033in}{4.158194in}}%
\pgfpathcurveto{\pgfqpoint{3.220857in}{4.158194in}}{\pgfqpoint{3.226443in}{4.160508in}}{\pgfqpoint{3.230561in}{4.164626in}}%
\pgfpathcurveto{\pgfqpoint{3.234679in}{4.168744in}}{\pgfqpoint{3.236993in}{4.174330in}}{\pgfqpoint{3.236993in}{4.180154in}}%
\pgfpathcurveto{\pgfqpoint{3.236993in}{4.185978in}}{\pgfqpoint{3.234679in}{4.191564in}}{\pgfqpoint{3.230561in}{4.195682in}}%
\pgfpathcurveto{\pgfqpoint{3.226443in}{4.199800in}}{\pgfqpoint{3.220857in}{4.202114in}}{\pgfqpoint{3.215033in}{4.202114in}}%
\pgfpathcurveto{\pgfqpoint{3.209209in}{4.202114in}}{\pgfqpoint{3.203623in}{4.199800in}}{\pgfqpoint{3.199505in}{4.195682in}}%
\pgfpathcurveto{\pgfqpoint{3.195387in}{4.191564in}}{\pgfqpoint{3.193073in}{4.185978in}}{\pgfqpoint{3.193073in}{4.180154in}}%
\pgfpathcurveto{\pgfqpoint{3.193073in}{4.174330in}}{\pgfqpoint{3.195387in}{4.168744in}}{\pgfqpoint{3.199505in}{4.164626in}}%
\pgfpathcurveto{\pgfqpoint{3.203623in}{4.160508in}}{\pgfqpoint{3.209209in}{4.158194in}}{\pgfqpoint{3.215033in}{4.158194in}}%
\pgfpathlineto{\pgfqpoint{3.215033in}{4.158194in}}%
\pgfpathclose%
\pgfusepath{stroke,fill}%
\end{pgfscope}%
\begin{pgfscope}%
\pgfpathrectangle{\pgfqpoint{1.582361in}{0.880000in}}{\pgfqpoint{5.035278in}{6.160000in}}%
\pgfusepath{clip}%
\pgfsetbuttcap%
\pgfsetroundjoin%
\definecolor{currentfill}{rgb}{0.800000,0.200000,0.200000}%
\pgfsetfillcolor{currentfill}%
\pgfsetlinewidth{1.003750pt}%
\definecolor{currentstroke}{rgb}{0.800000,0.200000,0.200000}%
\pgfsetstrokecolor{currentstroke}%
\pgfsetdash{}{0pt}%
\pgfpathmoveto{\pgfqpoint{3.252153in}{4.133232in}}%
\pgfpathcurveto{\pgfqpoint{3.257977in}{4.133232in}}{\pgfqpoint{3.263563in}{4.135546in}}{\pgfqpoint{3.267681in}{4.139664in}}%
\pgfpathcurveto{\pgfqpoint{3.271799in}{4.143782in}}{\pgfqpoint{3.274113in}{4.149368in}}{\pgfqpoint{3.274113in}{4.155192in}}%
\pgfpathcurveto{\pgfqpoint{3.274113in}{4.161016in}}{\pgfqpoint{3.271799in}{4.166602in}}{\pgfqpoint{3.267681in}{4.170720in}}%
\pgfpathcurveto{\pgfqpoint{3.263563in}{4.174839in}}{\pgfqpoint{3.257977in}{4.177152in}}{\pgfqpoint{3.252153in}{4.177152in}}%
\pgfpathcurveto{\pgfqpoint{3.246329in}{4.177152in}}{\pgfqpoint{3.240743in}{4.174839in}}{\pgfqpoint{3.236625in}{4.170720in}}%
\pgfpathcurveto{\pgfqpoint{3.232507in}{4.166602in}}{\pgfqpoint{3.230193in}{4.161016in}}{\pgfqpoint{3.230193in}{4.155192in}}%
\pgfpathcurveto{\pgfqpoint{3.230193in}{4.149368in}}{\pgfqpoint{3.232507in}{4.143782in}}{\pgfqpoint{3.236625in}{4.139664in}}%
\pgfpathcurveto{\pgfqpoint{3.240743in}{4.135546in}}{\pgfqpoint{3.246329in}{4.133232in}}{\pgfqpoint{3.252153in}{4.133232in}}%
\pgfpathlineto{\pgfqpoint{3.252153in}{4.133232in}}%
\pgfpathclose%
\pgfusepath{stroke,fill}%
\end{pgfscope}%
\begin{pgfscope}%
\pgfpathrectangle{\pgfqpoint{1.582361in}{0.880000in}}{\pgfqpoint{5.035278in}{6.160000in}}%
\pgfusepath{clip}%
\pgfsetbuttcap%
\pgfsetroundjoin%
\definecolor{currentfill}{rgb}{0.800000,0.200000,0.200000}%
\pgfsetfillcolor{currentfill}%
\pgfsetlinewidth{1.003750pt}%
\definecolor{currentstroke}{rgb}{0.800000,0.200000,0.200000}%
\pgfsetstrokecolor{currentstroke}%
\pgfsetdash{}{0pt}%
\pgfpathmoveto{\pgfqpoint{3.290042in}{4.109454in}}%
\pgfpathcurveto{\pgfqpoint{3.295866in}{4.109454in}}{\pgfqpoint{3.301452in}{4.111768in}}{\pgfqpoint{3.305571in}{4.115886in}}%
\pgfpathcurveto{\pgfqpoint{3.309689in}{4.120004in}}{\pgfqpoint{3.312003in}{4.125591in}}{\pgfqpoint{3.312003in}{4.131414in}}%
\pgfpathcurveto{\pgfqpoint{3.312003in}{4.137238in}}{\pgfqpoint{3.309689in}{4.142825in}}{\pgfqpoint{3.305571in}{4.146943in}}%
\pgfpathcurveto{\pgfqpoint{3.301452in}{4.151061in}}{\pgfqpoint{3.295866in}{4.153375in}}{\pgfqpoint{3.290042in}{4.153375in}}%
\pgfpathcurveto{\pgfqpoint{3.284218in}{4.153375in}}{\pgfqpoint{3.278632in}{4.151061in}}{\pgfqpoint{3.274514in}{4.146943in}}%
\pgfpathcurveto{\pgfqpoint{3.270396in}{4.142825in}}{\pgfqpoint{3.268082in}{4.137238in}}{\pgfqpoint{3.268082in}{4.131414in}}%
\pgfpathcurveto{\pgfqpoint{3.268082in}{4.125591in}}{\pgfqpoint{3.270396in}{4.120004in}}{\pgfqpoint{3.274514in}{4.115886in}}%
\pgfpathcurveto{\pgfqpoint{3.278632in}{4.111768in}}{\pgfqpoint{3.284218in}{4.109454in}}{\pgfqpoint{3.290042in}{4.109454in}}%
\pgfpathlineto{\pgfqpoint{3.290042in}{4.109454in}}%
\pgfpathclose%
\pgfusepath{stroke,fill}%
\end{pgfscope}%
\begin{pgfscope}%
\pgfpathrectangle{\pgfqpoint{1.582361in}{0.880000in}}{\pgfqpoint{5.035278in}{6.160000in}}%
\pgfusepath{clip}%
\pgfsetbuttcap%
\pgfsetroundjoin%
\definecolor{currentfill}{rgb}{0.800000,0.200000,0.200000}%
\pgfsetfillcolor{currentfill}%
\pgfsetlinewidth{1.003750pt}%
\definecolor{currentstroke}{rgb}{0.800000,0.200000,0.200000}%
\pgfsetstrokecolor{currentstroke}%
\pgfsetdash{}{0pt}%
\pgfpathmoveto{\pgfqpoint{3.328663in}{4.086884in}}%
\pgfpathcurveto{\pgfqpoint{3.334487in}{4.086884in}}{\pgfqpoint{3.340074in}{4.089198in}}{\pgfqpoint{3.344192in}{4.093316in}}%
\pgfpathcurveto{\pgfqpoint{3.348310in}{4.097435in}}{\pgfqpoint{3.350624in}{4.103021in}}{\pgfqpoint{3.350624in}{4.108845in}}%
\pgfpathcurveto{\pgfqpoint{3.350624in}{4.114669in}}{\pgfqpoint{3.348310in}{4.120255in}}{\pgfqpoint{3.344192in}{4.124373in}}%
\pgfpathcurveto{\pgfqpoint{3.340074in}{4.128491in}}{\pgfqpoint{3.334487in}{4.130805in}}{\pgfqpoint{3.328663in}{4.130805in}}%
\pgfpathcurveto{\pgfqpoint{3.322839in}{4.130805in}}{\pgfqpoint{3.317253in}{4.128491in}}{\pgfqpoint{3.313135in}{4.124373in}}%
\pgfpathcurveto{\pgfqpoint{3.309017in}{4.120255in}}{\pgfqpoint{3.306703in}{4.114669in}}{\pgfqpoint{3.306703in}{4.108845in}}%
\pgfpathcurveto{\pgfqpoint{3.306703in}{4.103021in}}{\pgfqpoint{3.309017in}{4.097435in}}{\pgfqpoint{3.313135in}{4.093316in}}%
\pgfpathcurveto{\pgfqpoint{3.317253in}{4.089198in}}{\pgfqpoint{3.322839in}{4.086884in}}{\pgfqpoint{3.328663in}{4.086884in}}%
\pgfpathlineto{\pgfqpoint{3.328663in}{4.086884in}}%
\pgfpathclose%
\pgfusepath{stroke,fill}%
\end{pgfscope}%
\begin{pgfscope}%
\pgfpathrectangle{\pgfqpoint{1.582361in}{0.880000in}}{\pgfqpoint{5.035278in}{6.160000in}}%
\pgfusepath{clip}%
\pgfsetbuttcap%
\pgfsetroundjoin%
\definecolor{currentfill}{rgb}{0.800000,0.200000,0.200000}%
\pgfsetfillcolor{currentfill}%
\pgfsetlinewidth{1.003750pt}%
\definecolor{currentstroke}{rgb}{0.800000,0.200000,0.200000}%
\pgfsetstrokecolor{currentstroke}%
\pgfsetdash{}{0pt}%
\pgfpathmoveto{\pgfqpoint{3.367978in}{4.065545in}}%
\pgfpathcurveto{\pgfqpoint{3.373802in}{4.065545in}}{\pgfqpoint{3.379388in}{4.067859in}}{\pgfqpoint{3.383506in}{4.071977in}}%
\pgfpathcurveto{\pgfqpoint{3.387624in}{4.076095in}}{\pgfqpoint{3.389938in}{4.081682in}}{\pgfqpoint{3.389938in}{4.087505in}}%
\pgfpathcurveto{\pgfqpoint{3.389938in}{4.093329in}}{\pgfqpoint{3.387624in}{4.098916in}}{\pgfqpoint{3.383506in}{4.103034in}}%
\pgfpathcurveto{\pgfqpoint{3.379388in}{4.107152in}}{\pgfqpoint{3.373802in}{4.109466in}}{\pgfqpoint{3.367978in}{4.109466in}}%
\pgfpathcurveto{\pgfqpoint{3.362154in}{4.109466in}}{\pgfqpoint{3.356568in}{4.107152in}}{\pgfqpoint{3.352450in}{4.103034in}}%
\pgfpathcurveto{\pgfqpoint{3.348331in}{4.098916in}}{\pgfqpoint{3.346018in}{4.093329in}}{\pgfqpoint{3.346018in}{4.087505in}}%
\pgfpathcurveto{\pgfqpoint{3.346018in}{4.081682in}}{\pgfqpoint{3.348331in}{4.076095in}}{\pgfqpoint{3.352450in}{4.071977in}}%
\pgfpathcurveto{\pgfqpoint{3.356568in}{4.067859in}}{\pgfqpoint{3.362154in}{4.065545in}}{\pgfqpoint{3.367978in}{4.065545in}}%
\pgfpathlineto{\pgfqpoint{3.367978in}{4.065545in}}%
\pgfpathclose%
\pgfusepath{stroke,fill}%
\end{pgfscope}%
\begin{pgfscope}%
\pgfpathrectangle{\pgfqpoint{1.582361in}{0.880000in}}{\pgfqpoint{5.035278in}{6.160000in}}%
\pgfusepath{clip}%
\pgfsetbuttcap%
\pgfsetroundjoin%
\definecolor{currentfill}{rgb}{0.800000,0.200000,0.200000}%
\pgfsetfillcolor{currentfill}%
\pgfsetlinewidth{1.003750pt}%
\definecolor{currentstroke}{rgb}{0.800000,0.200000,0.200000}%
\pgfsetstrokecolor{currentstroke}%
\pgfsetdash{}{0pt}%
\pgfpathmoveto{\pgfqpoint{3.407946in}{4.045458in}}%
\pgfpathcurveto{\pgfqpoint{3.413770in}{4.045458in}}{\pgfqpoint{3.419356in}{4.047772in}}{\pgfqpoint{3.423474in}{4.051890in}}%
\pgfpathcurveto{\pgfqpoint{3.427593in}{4.056008in}}{\pgfqpoint{3.429906in}{4.061594in}}{\pgfqpoint{3.429906in}{4.067418in}}%
\pgfpathcurveto{\pgfqpoint{3.429906in}{4.073242in}}{\pgfqpoint{3.427593in}{4.078828in}}{\pgfqpoint{3.423474in}{4.082946in}}%
\pgfpathcurveto{\pgfqpoint{3.419356in}{4.087064in}}{\pgfqpoint{3.413770in}{4.089378in}}{\pgfqpoint{3.407946in}{4.089378in}}%
\pgfpathcurveto{\pgfqpoint{3.402122in}{4.089378in}}{\pgfqpoint{3.396536in}{4.087064in}}{\pgfqpoint{3.392418in}{4.082946in}}%
\pgfpathcurveto{\pgfqpoint{3.388300in}{4.078828in}}{\pgfqpoint{3.385986in}{4.073242in}}{\pgfqpoint{3.385986in}{4.067418in}}%
\pgfpathcurveto{\pgfqpoint{3.385986in}{4.061594in}}{\pgfqpoint{3.388300in}{4.056008in}}{\pgfqpoint{3.392418in}{4.051890in}}%
\pgfpathcurveto{\pgfqpoint{3.396536in}{4.047772in}}{\pgfqpoint{3.402122in}{4.045458in}}{\pgfqpoint{3.407946in}{4.045458in}}%
\pgfpathlineto{\pgfqpoint{3.407946in}{4.045458in}}%
\pgfpathclose%
\pgfusepath{stroke,fill}%
\end{pgfscope}%
\begin{pgfscope}%
\pgfpathrectangle{\pgfqpoint{1.582361in}{0.880000in}}{\pgfqpoint{5.035278in}{6.160000in}}%
\pgfusepath{clip}%
\pgfsetbuttcap%
\pgfsetroundjoin%
\definecolor{currentfill}{rgb}{0.800000,0.200000,0.200000}%
\pgfsetfillcolor{currentfill}%
\pgfsetlinewidth{1.003750pt}%
\definecolor{currentstroke}{rgb}{0.800000,0.200000,0.200000}%
\pgfsetstrokecolor{currentstroke}%
\pgfsetdash{}{0pt}%
\pgfpathmoveto{\pgfqpoint{3.448529in}{4.026642in}}%
\pgfpathcurveto{\pgfqpoint{3.454353in}{4.026642in}}{\pgfqpoint{3.459939in}{4.028956in}}{\pgfqpoint{3.464057in}{4.033074in}}%
\pgfpathcurveto{\pgfqpoint{3.468175in}{4.037192in}}{\pgfqpoint{3.470489in}{4.042778in}}{\pgfqpoint{3.470489in}{4.048602in}}%
\pgfpathcurveto{\pgfqpoint{3.470489in}{4.054426in}}{\pgfqpoint{3.468175in}{4.060012in}}{\pgfqpoint{3.464057in}{4.064130in}}%
\pgfpathcurveto{\pgfqpoint{3.459939in}{4.068249in}}{\pgfqpoint{3.454353in}{4.070562in}}{\pgfqpoint{3.448529in}{4.070562in}}%
\pgfpathcurveto{\pgfqpoint{3.442705in}{4.070562in}}{\pgfqpoint{3.437119in}{4.068249in}}{\pgfqpoint{3.433001in}{4.064130in}}%
\pgfpathcurveto{\pgfqpoint{3.428882in}{4.060012in}}{\pgfqpoint{3.426569in}{4.054426in}}{\pgfqpoint{3.426569in}{4.048602in}}%
\pgfpathcurveto{\pgfqpoint{3.426569in}{4.042778in}}{\pgfqpoint{3.428882in}{4.037192in}}{\pgfqpoint{3.433001in}{4.033074in}}%
\pgfpathcurveto{\pgfqpoint{3.437119in}{4.028956in}}{\pgfqpoint{3.442705in}{4.026642in}}{\pgfqpoint{3.448529in}{4.026642in}}%
\pgfpathlineto{\pgfqpoint{3.448529in}{4.026642in}}%
\pgfpathclose%
\pgfusepath{stroke,fill}%
\end{pgfscope}%
\begin{pgfscope}%
\pgfpathrectangle{\pgfqpoint{1.582361in}{0.880000in}}{\pgfqpoint{5.035278in}{6.160000in}}%
\pgfusepath{clip}%
\pgfsetbuttcap%
\pgfsetroundjoin%
\definecolor{currentfill}{rgb}{0.800000,0.200000,0.200000}%
\pgfsetfillcolor{currentfill}%
\pgfsetlinewidth{1.003750pt}%
\definecolor{currentstroke}{rgb}{0.800000,0.200000,0.200000}%
\pgfsetstrokecolor{currentstroke}%
\pgfsetdash{}{0pt}%
\pgfpathmoveto{\pgfqpoint{3.489685in}{4.009117in}}%
\pgfpathcurveto{\pgfqpoint{3.495509in}{4.009117in}}{\pgfqpoint{3.501095in}{4.011431in}}{\pgfqpoint{3.505213in}{4.015549in}}%
\pgfpathcurveto{\pgfqpoint{3.509332in}{4.019667in}}{\pgfqpoint{3.511645in}{4.025253in}}{\pgfqpoint{3.511645in}{4.031077in}}%
\pgfpathcurveto{\pgfqpoint{3.511645in}{4.036901in}}{\pgfqpoint{3.509332in}{4.042487in}}{\pgfqpoint{3.505213in}{4.046605in}}%
\pgfpathcurveto{\pgfqpoint{3.501095in}{4.050723in}}{\pgfqpoint{3.495509in}{4.053037in}}{\pgfqpoint{3.489685in}{4.053037in}}%
\pgfpathcurveto{\pgfqpoint{3.483861in}{4.053037in}}{\pgfqpoint{3.478275in}{4.050723in}}{\pgfqpoint{3.474157in}{4.046605in}}%
\pgfpathcurveto{\pgfqpoint{3.470039in}{4.042487in}}{\pgfqpoint{3.467725in}{4.036901in}}{\pgfqpoint{3.467725in}{4.031077in}}%
\pgfpathcurveto{\pgfqpoint{3.467725in}{4.025253in}}{\pgfqpoint{3.470039in}{4.019667in}}{\pgfqpoint{3.474157in}{4.015549in}}%
\pgfpathcurveto{\pgfqpoint{3.478275in}{4.011431in}}{\pgfqpoint{3.483861in}{4.009117in}}{\pgfqpoint{3.489685in}{4.009117in}}%
\pgfpathlineto{\pgfqpoint{3.489685in}{4.009117in}}%
\pgfpathclose%
\pgfusepath{stroke,fill}%
\end{pgfscope}%
\begin{pgfscope}%
\pgfpathrectangle{\pgfqpoint{1.582361in}{0.880000in}}{\pgfqpoint{5.035278in}{6.160000in}}%
\pgfusepath{clip}%
\pgfsetbuttcap%
\pgfsetroundjoin%
\definecolor{currentfill}{rgb}{0.800000,0.200000,0.200000}%
\pgfsetfillcolor{currentfill}%
\pgfsetlinewidth{1.003750pt}%
\definecolor{currentstroke}{rgb}{0.800000,0.200000,0.200000}%
\pgfsetstrokecolor{currentstroke}%
\pgfsetdash{}{0pt}%
\pgfpathmoveto{\pgfqpoint{3.531374in}{3.992899in}}%
\pgfpathcurveto{\pgfqpoint{3.537198in}{3.992899in}}{\pgfqpoint{3.542784in}{3.995213in}}{\pgfqpoint{3.546903in}{3.999331in}}%
\pgfpathcurveto{\pgfqpoint{3.551021in}{4.003450in}}{\pgfqpoint{3.553335in}{4.009036in}}{\pgfqpoint{3.553335in}{4.014860in}}%
\pgfpathcurveto{\pgfqpoint{3.553335in}{4.020684in}}{\pgfqpoint{3.551021in}{4.026270in}}{\pgfqpoint{3.546903in}{4.030388in}}%
\pgfpathcurveto{\pgfqpoint{3.542784in}{4.034506in}}{\pgfqpoint{3.537198in}{4.036820in}}{\pgfqpoint{3.531374in}{4.036820in}}%
\pgfpathcurveto{\pgfqpoint{3.525550in}{4.036820in}}{\pgfqpoint{3.519964in}{4.034506in}}{\pgfqpoint{3.515846in}{4.030388in}}%
\pgfpathcurveto{\pgfqpoint{3.511728in}{4.026270in}}{\pgfqpoint{3.509414in}{4.020684in}}{\pgfqpoint{3.509414in}{4.014860in}}%
\pgfpathcurveto{\pgfqpoint{3.509414in}{4.009036in}}{\pgfqpoint{3.511728in}{4.003450in}}{\pgfqpoint{3.515846in}{3.999331in}}%
\pgfpathcurveto{\pgfqpoint{3.519964in}{3.995213in}}{\pgfqpoint{3.525550in}{3.992899in}}{\pgfqpoint{3.531374in}{3.992899in}}%
\pgfpathlineto{\pgfqpoint{3.531374in}{3.992899in}}%
\pgfpathclose%
\pgfusepath{stroke,fill}%
\end{pgfscope}%
\begin{pgfscope}%
\pgfpathrectangle{\pgfqpoint{1.582361in}{0.880000in}}{\pgfqpoint{5.035278in}{6.160000in}}%
\pgfusepath{clip}%
\pgfsetbuttcap%
\pgfsetroundjoin%
\definecolor{currentfill}{rgb}{0.800000,0.200000,0.200000}%
\pgfsetfillcolor{currentfill}%
\pgfsetlinewidth{1.003750pt}%
\definecolor{currentstroke}{rgb}{0.800000,0.200000,0.200000}%
\pgfsetstrokecolor{currentstroke}%
\pgfsetdash{}{0pt}%
\pgfpathmoveto{\pgfqpoint{3.573555in}{3.978006in}}%
\pgfpathcurveto{\pgfqpoint{3.579379in}{3.978006in}}{\pgfqpoint{3.584965in}{3.980320in}}{\pgfqpoint{3.589083in}{3.984438in}}%
\pgfpathcurveto{\pgfqpoint{3.593201in}{3.988556in}}{\pgfqpoint{3.595515in}{3.994143in}}{\pgfqpoint{3.595515in}{3.999966in}}%
\pgfpathcurveto{\pgfqpoint{3.595515in}{4.005790in}}{\pgfqpoint{3.593201in}{4.011377in}}{\pgfqpoint{3.589083in}{4.015495in}}%
\pgfpathcurveto{\pgfqpoint{3.584965in}{4.019613in}}{\pgfqpoint{3.579379in}{4.021927in}}{\pgfqpoint{3.573555in}{4.021927in}}%
\pgfpathcurveto{\pgfqpoint{3.567731in}{4.021927in}}{\pgfqpoint{3.562145in}{4.019613in}}{\pgfqpoint{3.558026in}{4.015495in}}%
\pgfpathcurveto{\pgfqpoint{3.553908in}{4.011377in}}{\pgfqpoint{3.551594in}{4.005790in}}{\pgfqpoint{3.551594in}{3.999966in}}%
\pgfpathcurveto{\pgfqpoint{3.551594in}{3.994143in}}{\pgfqpoint{3.553908in}{3.988556in}}{\pgfqpoint{3.558026in}{3.984438in}}%
\pgfpathcurveto{\pgfqpoint{3.562145in}{3.980320in}}{\pgfqpoint{3.567731in}{3.978006in}}{\pgfqpoint{3.573555in}{3.978006in}}%
\pgfpathlineto{\pgfqpoint{3.573555in}{3.978006in}}%
\pgfpathclose%
\pgfusepath{stroke,fill}%
\end{pgfscope}%
\begin{pgfscope}%
\pgfpathrectangle{\pgfqpoint{1.582361in}{0.880000in}}{\pgfqpoint{5.035278in}{6.160000in}}%
\pgfusepath{clip}%
\pgfsetbuttcap%
\pgfsetroundjoin%
\definecolor{currentfill}{rgb}{0.800000,0.200000,0.200000}%
\pgfsetfillcolor{currentfill}%
\pgfsetlinewidth{1.003750pt}%
\definecolor{currentstroke}{rgb}{0.800000,0.200000,0.200000}%
\pgfsetstrokecolor{currentstroke}%
\pgfsetdash{}{0pt}%
\pgfpathmoveto{\pgfqpoint{3.616184in}{3.964452in}}%
\pgfpathcurveto{\pgfqpoint{3.622008in}{3.964452in}}{\pgfqpoint{3.627594in}{3.966766in}}{\pgfqpoint{3.631712in}{3.970884in}}%
\pgfpathcurveto{\pgfqpoint{3.635830in}{3.975002in}}{\pgfqpoint{3.638144in}{3.980588in}}{\pgfqpoint{3.638144in}{3.986412in}}%
\pgfpathcurveto{\pgfqpoint{3.638144in}{3.992236in}}{\pgfqpoint{3.635830in}{3.997822in}}{\pgfqpoint{3.631712in}{4.001941in}}%
\pgfpathcurveto{\pgfqpoint{3.627594in}{4.006059in}}{\pgfqpoint{3.622008in}{4.008373in}}{\pgfqpoint{3.616184in}{4.008373in}}%
\pgfpathcurveto{\pgfqpoint{3.610360in}{4.008373in}}{\pgfqpoint{3.604774in}{4.006059in}}{\pgfqpoint{3.600656in}{4.001941in}}%
\pgfpathcurveto{\pgfqpoint{3.596538in}{3.997822in}}{\pgfqpoint{3.594224in}{3.992236in}}{\pgfqpoint{3.594224in}{3.986412in}}%
\pgfpathcurveto{\pgfqpoint{3.594224in}{3.980588in}}{\pgfqpoint{3.596538in}{3.975002in}}{\pgfqpoint{3.600656in}{3.970884in}}%
\pgfpathcurveto{\pgfqpoint{3.604774in}{3.966766in}}{\pgfqpoint{3.610360in}{3.964452in}}{\pgfqpoint{3.616184in}{3.964452in}}%
\pgfpathlineto{\pgfqpoint{3.616184in}{3.964452in}}%
\pgfpathclose%
\pgfusepath{stroke,fill}%
\end{pgfscope}%
\begin{pgfscope}%
\pgfpathrectangle{\pgfqpoint{1.582361in}{0.880000in}}{\pgfqpoint{5.035278in}{6.160000in}}%
\pgfusepath{clip}%
\pgfsetbuttcap%
\pgfsetroundjoin%
\definecolor{currentfill}{rgb}{0.800000,0.200000,0.200000}%
\pgfsetfillcolor{currentfill}%
\pgfsetlinewidth{1.003750pt}%
\definecolor{currentstroke}{rgb}{0.800000,0.200000,0.200000}%
\pgfsetstrokecolor{currentstroke}%
\pgfsetdash{}{0pt}%
\pgfpathmoveto{\pgfqpoint{3.659220in}{3.952250in}}%
\pgfpathcurveto{\pgfqpoint{3.665044in}{3.952250in}}{\pgfqpoint{3.670630in}{3.954564in}}{\pgfqpoint{3.674748in}{3.958682in}}%
\pgfpathcurveto{\pgfqpoint{3.678867in}{3.962801in}}{\pgfqpoint{3.681180in}{3.968387in}}{\pgfqpoint{3.681180in}{3.974211in}}%
\pgfpathcurveto{\pgfqpoint{3.681180in}{3.980035in}}{\pgfqpoint{3.678867in}{3.985621in}}{\pgfqpoint{3.674748in}{3.989739in}}%
\pgfpathcurveto{\pgfqpoint{3.670630in}{3.993857in}}{\pgfqpoint{3.665044in}{3.996171in}}{\pgfqpoint{3.659220in}{3.996171in}}%
\pgfpathcurveto{\pgfqpoint{3.653396in}{3.996171in}}{\pgfqpoint{3.647810in}{3.993857in}}{\pgfqpoint{3.643692in}{3.989739in}}%
\pgfpathcurveto{\pgfqpoint{3.639574in}{3.985621in}}{\pgfqpoint{3.637260in}{3.980035in}}{\pgfqpoint{3.637260in}{3.974211in}}%
\pgfpathcurveto{\pgfqpoint{3.637260in}{3.968387in}}{\pgfqpoint{3.639574in}{3.962801in}}{\pgfqpoint{3.643692in}{3.958682in}}%
\pgfpathcurveto{\pgfqpoint{3.647810in}{3.954564in}}{\pgfqpoint{3.653396in}{3.952250in}}{\pgfqpoint{3.659220in}{3.952250in}}%
\pgfpathlineto{\pgfqpoint{3.659220in}{3.952250in}}%
\pgfpathclose%
\pgfusepath{stroke,fill}%
\end{pgfscope}%
\begin{pgfscope}%
\pgfpathrectangle{\pgfqpoint{1.582361in}{0.880000in}}{\pgfqpoint{5.035278in}{6.160000in}}%
\pgfusepath{clip}%
\pgfsetbuttcap%
\pgfsetroundjoin%
\definecolor{currentfill}{rgb}{0.800000,0.200000,0.200000}%
\pgfsetfillcolor{currentfill}%
\pgfsetlinewidth{1.003750pt}%
\definecolor{currentstroke}{rgb}{0.800000,0.200000,0.200000}%
\pgfsetstrokecolor{currentstroke}%
\pgfsetdash{}{0pt}%
\pgfpathmoveto{\pgfqpoint{3.702620in}{3.941414in}}%
\pgfpathcurveto{\pgfqpoint{3.708444in}{3.941414in}}{\pgfqpoint{3.714030in}{3.943727in}}{\pgfqpoint{3.718148in}{3.947846in}}%
\pgfpathcurveto{\pgfqpoint{3.722266in}{3.951964in}}{\pgfqpoint{3.724580in}{3.957550in}}{\pgfqpoint{3.724580in}{3.963374in}}%
\pgfpathcurveto{\pgfqpoint{3.724580in}{3.969198in}}{\pgfqpoint{3.722266in}{3.974784in}}{\pgfqpoint{3.718148in}{3.978902in}}%
\pgfpathcurveto{\pgfqpoint{3.714030in}{3.983020in}}{\pgfqpoint{3.708444in}{3.985334in}}{\pgfqpoint{3.702620in}{3.985334in}}%
\pgfpathcurveto{\pgfqpoint{3.696796in}{3.985334in}}{\pgfqpoint{3.691210in}{3.983020in}}{\pgfqpoint{3.687092in}{3.978902in}}%
\pgfpathcurveto{\pgfqpoint{3.682974in}{3.974784in}}{\pgfqpoint{3.680660in}{3.969198in}}{\pgfqpoint{3.680660in}{3.963374in}}%
\pgfpathcurveto{\pgfqpoint{3.680660in}{3.957550in}}{\pgfqpoint{3.682974in}{3.951964in}}{\pgfqpoint{3.687092in}{3.947846in}}%
\pgfpathcurveto{\pgfqpoint{3.691210in}{3.943727in}}{\pgfqpoint{3.696796in}{3.941414in}}{\pgfqpoint{3.702620in}{3.941414in}}%
\pgfpathlineto{\pgfqpoint{3.702620in}{3.941414in}}%
\pgfpathclose%
\pgfusepath{stroke,fill}%
\end{pgfscope}%
\begin{pgfscope}%
\pgfpathrectangle{\pgfqpoint{1.582361in}{0.880000in}}{\pgfqpoint{5.035278in}{6.160000in}}%
\pgfusepath{clip}%
\pgfsetbuttcap%
\pgfsetroundjoin%
\definecolor{currentfill}{rgb}{0.800000,0.200000,0.200000}%
\pgfsetfillcolor{currentfill}%
\pgfsetlinewidth{1.003750pt}%
\definecolor{currentstroke}{rgb}{0.800000,0.200000,0.200000}%
\pgfsetstrokecolor{currentstroke}%
\pgfsetdash{}{0pt}%
\pgfpathmoveto{\pgfqpoint{3.746340in}{3.931952in}}%
\pgfpathcurveto{\pgfqpoint{3.752164in}{3.931952in}}{\pgfqpoint{3.757750in}{3.934266in}}{\pgfqpoint{3.761869in}{3.938384in}}%
\pgfpathcurveto{\pgfqpoint{3.765987in}{3.942502in}}{\pgfqpoint{3.768301in}{3.948088in}}{\pgfqpoint{3.768301in}{3.953912in}}%
\pgfpathcurveto{\pgfqpoint{3.768301in}{3.959736in}}{\pgfqpoint{3.765987in}{3.965322in}}{\pgfqpoint{3.761869in}{3.969441in}}%
\pgfpathcurveto{\pgfqpoint{3.757750in}{3.973559in}}{\pgfqpoint{3.752164in}{3.975873in}}{\pgfqpoint{3.746340in}{3.975873in}}%
\pgfpathcurveto{\pgfqpoint{3.740516in}{3.975873in}}{\pgfqpoint{3.734930in}{3.973559in}}{\pgfqpoint{3.730812in}{3.969441in}}%
\pgfpathcurveto{\pgfqpoint{3.726694in}{3.965322in}}{\pgfqpoint{3.724380in}{3.959736in}}{\pgfqpoint{3.724380in}{3.953912in}}%
\pgfpathcurveto{\pgfqpoint{3.724380in}{3.948088in}}{\pgfqpoint{3.726694in}{3.942502in}}{\pgfqpoint{3.730812in}{3.938384in}}%
\pgfpathcurveto{\pgfqpoint{3.734930in}{3.934266in}}{\pgfqpoint{3.740516in}{3.931952in}}{\pgfqpoint{3.746340in}{3.931952in}}%
\pgfpathlineto{\pgfqpoint{3.746340in}{3.931952in}}%
\pgfpathclose%
\pgfusepath{stroke,fill}%
\end{pgfscope}%
\begin{pgfscope}%
\pgfpathrectangle{\pgfqpoint{1.582361in}{0.880000in}}{\pgfqpoint{5.035278in}{6.160000in}}%
\pgfusepath{clip}%
\pgfsetbuttcap%
\pgfsetroundjoin%
\definecolor{currentfill}{rgb}{0.800000,0.200000,0.200000}%
\pgfsetfillcolor{currentfill}%
\pgfsetlinewidth{1.003750pt}%
\definecolor{currentstroke}{rgb}{0.800000,0.200000,0.200000}%
\pgfsetstrokecolor{currentstroke}%
\pgfsetdash{}{0pt}%
\pgfpathmoveto{\pgfqpoint{3.790337in}{3.923875in}}%
\pgfpathcurveto{\pgfqpoint{3.796161in}{3.923875in}}{\pgfqpoint{3.801748in}{3.926189in}}{\pgfqpoint{3.805866in}{3.930307in}}%
\pgfpathcurveto{\pgfqpoint{3.809984in}{3.934426in}}{\pgfqpoint{3.812298in}{3.940012in}}{\pgfqpoint{3.812298in}{3.945836in}}%
\pgfpathcurveto{\pgfqpoint{3.812298in}{3.951660in}}{\pgfqpoint{3.809984in}{3.957246in}}{\pgfqpoint{3.805866in}{3.961364in}}%
\pgfpathcurveto{\pgfqpoint{3.801748in}{3.965482in}}{\pgfqpoint{3.796161in}{3.967796in}}{\pgfqpoint{3.790337in}{3.967796in}}%
\pgfpathcurveto{\pgfqpoint{3.784514in}{3.967796in}}{\pgfqpoint{3.778927in}{3.965482in}}{\pgfqpoint{3.774809in}{3.961364in}}%
\pgfpathcurveto{\pgfqpoint{3.770691in}{3.957246in}}{\pgfqpoint{3.768377in}{3.951660in}}{\pgfqpoint{3.768377in}{3.945836in}}%
\pgfpathcurveto{\pgfqpoint{3.768377in}{3.940012in}}{\pgfqpoint{3.770691in}{3.934426in}}{\pgfqpoint{3.774809in}{3.930307in}}%
\pgfpathcurveto{\pgfqpoint{3.778927in}{3.926189in}}{\pgfqpoint{3.784514in}{3.923875in}}{\pgfqpoint{3.790337in}{3.923875in}}%
\pgfpathlineto{\pgfqpoint{3.790337in}{3.923875in}}%
\pgfpathclose%
\pgfusepath{stroke,fill}%
\end{pgfscope}%
\begin{pgfscope}%
\pgfpathrectangle{\pgfqpoint{1.582361in}{0.880000in}}{\pgfqpoint{5.035278in}{6.160000in}}%
\pgfusepath{clip}%
\pgfsetbuttcap%
\pgfsetroundjoin%
\definecolor{currentfill}{rgb}{0.800000,0.200000,0.200000}%
\pgfsetfillcolor{currentfill}%
\pgfsetlinewidth{1.003750pt}%
\definecolor{currentstroke}{rgb}{0.800000,0.200000,0.200000}%
\pgfsetstrokecolor{currentstroke}%
\pgfsetdash{}{0pt}%
\pgfpathmoveto{\pgfqpoint{3.834568in}{3.917192in}}%
\pgfpathcurveto{\pgfqpoint{3.840392in}{3.917192in}}{\pgfqpoint{3.845978in}{3.919506in}}{\pgfqpoint{3.850096in}{3.923624in}}%
\pgfpathcurveto{\pgfqpoint{3.854214in}{3.927742in}}{\pgfqpoint{3.856528in}{3.933328in}}{\pgfqpoint{3.856528in}{3.939152in}}%
\pgfpathcurveto{\pgfqpoint{3.856528in}{3.944976in}}{\pgfqpoint{3.854214in}{3.950562in}}{\pgfqpoint{3.850096in}{3.954680in}}%
\pgfpathcurveto{\pgfqpoint{3.845978in}{3.958798in}}{\pgfqpoint{3.840392in}{3.961112in}}{\pgfqpoint{3.834568in}{3.961112in}}%
\pgfpathcurveto{\pgfqpoint{3.828744in}{3.961112in}}{\pgfqpoint{3.823158in}{3.958798in}}{\pgfqpoint{3.819039in}{3.954680in}}%
\pgfpathcurveto{\pgfqpoint{3.814921in}{3.950562in}}{\pgfqpoint{3.812607in}{3.944976in}}{\pgfqpoint{3.812607in}{3.939152in}}%
\pgfpathcurveto{\pgfqpoint{3.812607in}{3.933328in}}{\pgfqpoint{3.814921in}{3.927742in}}{\pgfqpoint{3.819039in}{3.923624in}}%
\pgfpathcurveto{\pgfqpoint{3.823158in}{3.919506in}}{\pgfqpoint{3.828744in}{3.917192in}}{\pgfqpoint{3.834568in}{3.917192in}}%
\pgfpathlineto{\pgfqpoint{3.834568in}{3.917192in}}%
\pgfpathclose%
\pgfusepath{stroke,fill}%
\end{pgfscope}%
\begin{pgfscope}%
\pgfpathrectangle{\pgfqpoint{1.582361in}{0.880000in}}{\pgfqpoint{5.035278in}{6.160000in}}%
\pgfusepath{clip}%
\pgfsetbuttcap%
\pgfsetroundjoin%
\definecolor{currentfill}{rgb}{0.800000,0.200000,0.200000}%
\pgfsetfillcolor{currentfill}%
\pgfsetlinewidth{1.003750pt}%
\definecolor{currentstroke}{rgb}{0.800000,0.200000,0.200000}%
\pgfsetstrokecolor{currentstroke}%
\pgfsetdash{}{0pt}%
\pgfpathmoveto{\pgfqpoint{3.878987in}{3.911908in}}%
\pgfpathcurveto{\pgfqpoint{3.884811in}{3.911908in}}{\pgfqpoint{3.890397in}{3.914222in}}{\pgfqpoint{3.894515in}{3.918340in}}%
\pgfpathcurveto{\pgfqpoint{3.898633in}{3.922458in}}{\pgfqpoint{3.900947in}{3.928044in}}{\pgfqpoint{3.900947in}{3.933868in}}%
\pgfpathcurveto{\pgfqpoint{3.900947in}{3.939692in}}{\pgfqpoint{3.898633in}{3.945278in}}{\pgfqpoint{3.894515in}{3.949396in}}%
\pgfpathcurveto{\pgfqpoint{3.890397in}{3.953514in}}{\pgfqpoint{3.884811in}{3.955828in}}{\pgfqpoint{3.878987in}{3.955828in}}%
\pgfpathcurveto{\pgfqpoint{3.873163in}{3.955828in}}{\pgfqpoint{3.867577in}{3.953514in}}{\pgfqpoint{3.863459in}{3.949396in}}%
\pgfpathcurveto{\pgfqpoint{3.859340in}{3.945278in}}{\pgfqpoint{3.857027in}{3.939692in}}{\pgfqpoint{3.857027in}{3.933868in}}%
\pgfpathcurveto{\pgfqpoint{3.857027in}{3.928044in}}{\pgfqpoint{3.859340in}{3.922458in}}{\pgfqpoint{3.863459in}{3.918340in}}%
\pgfpathcurveto{\pgfqpoint{3.867577in}{3.914222in}}{\pgfqpoint{3.873163in}{3.911908in}}{\pgfqpoint{3.878987in}{3.911908in}}%
\pgfpathlineto{\pgfqpoint{3.878987in}{3.911908in}}%
\pgfpathclose%
\pgfusepath{stroke,fill}%
\end{pgfscope}%
\begin{pgfscope}%
\pgfpathrectangle{\pgfqpoint{1.582361in}{0.880000in}}{\pgfqpoint{5.035278in}{6.160000in}}%
\pgfusepath{clip}%
\pgfsetbuttcap%
\pgfsetroundjoin%
\definecolor{currentfill}{rgb}{0.800000,0.200000,0.200000}%
\pgfsetfillcolor{currentfill}%
\pgfsetlinewidth{1.003750pt}%
\definecolor{currentstroke}{rgb}{0.800000,0.200000,0.200000}%
\pgfsetstrokecolor{currentstroke}%
\pgfsetdash{}{0pt}%
\pgfpathmoveto{\pgfqpoint{3.923551in}{3.908029in}}%
\pgfpathcurveto{\pgfqpoint{3.929375in}{3.908029in}}{\pgfqpoint{3.934961in}{3.910343in}}{\pgfqpoint{3.939079in}{3.914461in}}%
\pgfpathcurveto{\pgfqpoint{3.943197in}{3.918579in}}{\pgfqpoint{3.945511in}{3.924165in}}{\pgfqpoint{3.945511in}{3.929989in}}%
\pgfpathcurveto{\pgfqpoint{3.945511in}{3.935813in}}{\pgfqpoint{3.943197in}{3.941399in}}{\pgfqpoint{3.939079in}{3.945517in}}%
\pgfpathcurveto{\pgfqpoint{3.934961in}{3.949635in}}{\pgfqpoint{3.929375in}{3.951949in}}{\pgfqpoint{3.923551in}{3.951949in}}%
\pgfpathcurveto{\pgfqpoint{3.917727in}{3.951949in}}{\pgfqpoint{3.912141in}{3.949635in}}{\pgfqpoint{3.908022in}{3.945517in}}%
\pgfpathcurveto{\pgfqpoint{3.903904in}{3.941399in}}{\pgfqpoint{3.901590in}{3.935813in}}{\pgfqpoint{3.901590in}{3.929989in}}%
\pgfpathcurveto{\pgfqpoint{3.901590in}{3.924165in}}{\pgfqpoint{3.903904in}{3.918579in}}{\pgfqpoint{3.908022in}{3.914461in}}%
\pgfpathcurveto{\pgfqpoint{3.912141in}{3.910343in}}{\pgfqpoint{3.917727in}{3.908029in}}{\pgfqpoint{3.923551in}{3.908029in}}%
\pgfpathlineto{\pgfqpoint{3.923551in}{3.908029in}}%
\pgfpathclose%
\pgfusepath{stroke,fill}%
\end{pgfscope}%
\begin{pgfscope}%
\pgfpathrectangle{\pgfqpoint{1.582361in}{0.880000in}}{\pgfqpoint{5.035278in}{6.160000in}}%
\pgfusepath{clip}%
\pgfsetbuttcap%
\pgfsetroundjoin%
\definecolor{currentfill}{rgb}{0.800000,0.200000,0.200000}%
\pgfsetfillcolor{currentfill}%
\pgfsetlinewidth{1.003750pt}%
\definecolor{currentstroke}{rgb}{0.800000,0.200000,0.200000}%
\pgfsetstrokecolor{currentstroke}%
\pgfsetdash{}{0pt}%
\pgfpathmoveto{\pgfqpoint{3.968215in}{3.905558in}}%
\pgfpathcurveto{\pgfqpoint{3.974039in}{3.905558in}}{\pgfqpoint{3.979625in}{3.907872in}}{\pgfqpoint{3.983743in}{3.911990in}}%
\pgfpathcurveto{\pgfqpoint{3.987861in}{3.916108in}}{\pgfqpoint{3.990175in}{3.921695in}}{\pgfqpoint{3.990175in}{3.927519in}}%
\pgfpathcurveto{\pgfqpoint{3.990175in}{3.933342in}}{\pgfqpoint{3.987861in}{3.938929in}}{\pgfqpoint{3.983743in}{3.943047in}}%
\pgfpathcurveto{\pgfqpoint{3.979625in}{3.947165in}}{\pgfqpoint{3.974039in}{3.949479in}}{\pgfqpoint{3.968215in}{3.949479in}}%
\pgfpathcurveto{\pgfqpoint{3.962391in}{3.949479in}}{\pgfqpoint{3.956805in}{3.947165in}}{\pgfqpoint{3.952686in}{3.943047in}}%
\pgfpathcurveto{\pgfqpoint{3.948568in}{3.938929in}}{\pgfqpoint{3.946254in}{3.933342in}}{\pgfqpoint{3.946254in}{3.927519in}}%
\pgfpathcurveto{\pgfqpoint{3.946254in}{3.921695in}}{\pgfqpoint{3.948568in}{3.916108in}}{\pgfqpoint{3.952686in}{3.911990in}}%
\pgfpathcurveto{\pgfqpoint{3.956805in}{3.907872in}}{\pgfqpoint{3.962391in}{3.905558in}}{\pgfqpoint{3.968215in}{3.905558in}}%
\pgfpathlineto{\pgfqpoint{3.968215in}{3.905558in}}%
\pgfpathclose%
\pgfusepath{stroke,fill}%
\end{pgfscope}%
\begin{pgfscope}%
\pgfpathrectangle{\pgfqpoint{1.582361in}{0.880000in}}{\pgfqpoint{5.035278in}{6.160000in}}%
\pgfusepath{clip}%
\pgfsetbuttcap%
\pgfsetroundjoin%
\definecolor{currentfill}{rgb}{0.800000,0.200000,0.200000}%
\pgfsetfillcolor{currentfill}%
\pgfsetlinewidth{1.003750pt}%
\definecolor{currentstroke}{rgb}{0.800000,0.200000,0.200000}%
\pgfsetstrokecolor{currentstroke}%
\pgfsetdash{}{0pt}%
\pgfpathmoveto{\pgfqpoint{4.012935in}{3.904499in}}%
\pgfpathcurveto{\pgfqpoint{4.018758in}{3.904499in}}{\pgfqpoint{4.024345in}{3.906813in}}{\pgfqpoint{4.028463in}{3.910931in}}%
\pgfpathcurveto{\pgfqpoint{4.032581in}{3.915049in}}{\pgfqpoint{4.034895in}{3.920635in}}{\pgfqpoint{4.034895in}{3.926459in}}%
\pgfpathcurveto{\pgfqpoint{4.034895in}{3.932283in}}{\pgfqpoint{4.032581in}{3.937869in}}{\pgfqpoint{4.028463in}{3.941988in}}%
\pgfpathcurveto{\pgfqpoint{4.024345in}{3.946106in}}{\pgfqpoint{4.018758in}{3.948420in}}{\pgfqpoint{4.012935in}{3.948420in}}%
\pgfpathcurveto{\pgfqpoint{4.007111in}{3.948420in}}{\pgfqpoint{4.001524in}{3.946106in}}{\pgfqpoint{3.997406in}{3.941988in}}%
\pgfpathcurveto{\pgfqpoint{3.993288in}{3.937869in}}{\pgfqpoint{3.990974in}{3.932283in}}{\pgfqpoint{3.990974in}{3.926459in}}%
\pgfpathcurveto{\pgfqpoint{3.990974in}{3.920635in}}{\pgfqpoint{3.993288in}{3.915049in}}{\pgfqpoint{3.997406in}{3.910931in}}%
\pgfpathcurveto{\pgfqpoint{4.001524in}{3.906813in}}{\pgfqpoint{4.007111in}{3.904499in}}{\pgfqpoint{4.012935in}{3.904499in}}%
\pgfpathlineto{\pgfqpoint{4.012935in}{3.904499in}}%
\pgfpathclose%
\pgfusepath{stroke,fill}%
\end{pgfscope}%
\begin{pgfscope}%
\pgfpathrectangle{\pgfqpoint{1.582361in}{0.880000in}}{\pgfqpoint{5.035278in}{6.160000in}}%
\pgfusepath{clip}%
\pgfsetbuttcap%
\pgfsetroundjoin%
\definecolor{currentfill}{rgb}{0.800000,0.200000,0.200000}%
\pgfsetfillcolor{currentfill}%
\pgfsetlinewidth{1.003750pt}%
\definecolor{currentstroke}{rgb}{0.800000,0.200000,0.200000}%
\pgfsetstrokecolor{currentstroke}%
\pgfsetdash{}{0pt}%
\pgfpathmoveto{\pgfqpoint{4.057666in}{3.904852in}}%
\pgfpathcurveto{\pgfqpoint{4.063489in}{3.904852in}}{\pgfqpoint{4.069076in}{3.907166in}}{\pgfqpoint{4.073194in}{3.911284in}}%
\pgfpathcurveto{\pgfqpoint{4.077312in}{3.915402in}}{\pgfqpoint{4.079626in}{3.920988in}}{\pgfqpoint{4.079626in}{3.926812in}}%
\pgfpathcurveto{\pgfqpoint{4.079626in}{3.932636in}}{\pgfqpoint{4.077312in}{3.938223in}}{\pgfqpoint{4.073194in}{3.942341in}}%
\pgfpathcurveto{\pgfqpoint{4.069076in}{3.946459in}}{\pgfqpoint{4.063489in}{3.948773in}}{\pgfqpoint{4.057666in}{3.948773in}}%
\pgfpathcurveto{\pgfqpoint{4.051842in}{3.948773in}}{\pgfqpoint{4.046255in}{3.946459in}}{\pgfqpoint{4.042137in}{3.942341in}}%
\pgfpathcurveto{\pgfqpoint{4.038019in}{3.938223in}}{\pgfqpoint{4.035705in}{3.932636in}}{\pgfqpoint{4.035705in}{3.926812in}}%
\pgfpathcurveto{\pgfqpoint{4.035705in}{3.920988in}}{\pgfqpoint{4.038019in}{3.915402in}}{\pgfqpoint{4.042137in}{3.911284in}}%
\pgfpathcurveto{\pgfqpoint{4.046255in}{3.907166in}}{\pgfqpoint{4.051842in}{3.904852in}}{\pgfqpoint{4.057666in}{3.904852in}}%
\pgfpathlineto{\pgfqpoint{4.057666in}{3.904852in}}%
\pgfpathclose%
\pgfusepath{stroke,fill}%
\end{pgfscope}%
\begin{pgfscope}%
\pgfpathrectangle{\pgfqpoint{1.582361in}{0.880000in}}{\pgfqpoint{5.035278in}{6.160000in}}%
\pgfusepath{clip}%
\pgfsetbuttcap%
\pgfsetroundjoin%
\definecolor{currentfill}{rgb}{0.800000,0.200000,0.200000}%
\pgfsetfillcolor{currentfill}%
\pgfsetlinewidth{1.003750pt}%
\definecolor{currentstroke}{rgb}{0.800000,0.200000,0.200000}%
\pgfsetstrokecolor{currentstroke}%
\pgfsetdash{}{0pt}%
\pgfpathmoveto{\pgfqpoint{4.102363in}{3.906617in}}%
\pgfpathcurveto{\pgfqpoint{4.108187in}{3.906617in}}{\pgfqpoint{4.113773in}{3.908931in}}{\pgfqpoint{4.117891in}{3.913049in}}%
\pgfpathcurveto{\pgfqpoint{4.122009in}{3.917167in}}{\pgfqpoint{4.124323in}{3.922753in}}{\pgfqpoint{4.124323in}{3.928577in}}%
\pgfpathcurveto{\pgfqpoint{4.124323in}{3.934401in}}{\pgfqpoint{4.122009in}{3.939988in}}{\pgfqpoint{4.117891in}{3.944106in}}%
\pgfpathcurveto{\pgfqpoint{4.113773in}{3.948224in}}{\pgfqpoint{4.108187in}{3.950538in}}{\pgfqpoint{4.102363in}{3.950538in}}%
\pgfpathcurveto{\pgfqpoint{4.096539in}{3.950538in}}{\pgfqpoint{4.090953in}{3.948224in}}{\pgfqpoint{4.086835in}{3.944106in}}%
\pgfpathcurveto{\pgfqpoint{4.082717in}{3.939988in}}{\pgfqpoint{4.080403in}{3.934401in}}{\pgfqpoint{4.080403in}{3.928577in}}%
\pgfpathcurveto{\pgfqpoint{4.080403in}{3.922753in}}{\pgfqpoint{4.082717in}{3.917167in}}{\pgfqpoint{4.086835in}{3.913049in}}%
\pgfpathcurveto{\pgfqpoint{4.090953in}{3.908931in}}{\pgfqpoint{4.096539in}{3.906617in}}{\pgfqpoint{4.102363in}{3.906617in}}%
\pgfpathlineto{\pgfqpoint{4.102363in}{3.906617in}}%
\pgfpathclose%
\pgfusepath{stroke,fill}%
\end{pgfscope}%
\begin{pgfscope}%
\pgfpathrectangle{\pgfqpoint{1.582361in}{0.880000in}}{\pgfqpoint{5.035278in}{6.160000in}}%
\pgfusepath{clip}%
\pgfsetbuttcap%
\pgfsetroundjoin%
\definecolor{currentfill}{rgb}{0.800000,0.200000,0.200000}%
\pgfsetfillcolor{currentfill}%
\pgfsetlinewidth{1.003750pt}%
\definecolor{currentstroke}{rgb}{0.800000,0.200000,0.200000}%
\pgfsetstrokecolor{currentstroke}%
\pgfsetdash{}{0pt}%
\pgfpathmoveto{\pgfqpoint{4.146983in}{3.909792in}}%
\pgfpathcurveto{\pgfqpoint{4.152806in}{3.909792in}}{\pgfqpoint{4.158393in}{3.912106in}}{\pgfqpoint{4.162511in}{3.916224in}}%
\pgfpathcurveto{\pgfqpoint{4.166629in}{3.920342in}}{\pgfqpoint{4.168943in}{3.925929in}}{\pgfqpoint{4.168943in}{3.931753in}}%
\pgfpathcurveto{\pgfqpoint{4.168943in}{3.937577in}}{\pgfqpoint{4.166629in}{3.943163in}}{\pgfqpoint{4.162511in}{3.947281in}}%
\pgfpathcurveto{\pgfqpoint{4.158393in}{3.951399in}}{\pgfqpoint{4.152806in}{3.953713in}}{\pgfqpoint{4.146983in}{3.953713in}}%
\pgfpathcurveto{\pgfqpoint{4.141159in}{3.953713in}}{\pgfqpoint{4.135572in}{3.951399in}}{\pgfqpoint{4.131454in}{3.947281in}}%
\pgfpathcurveto{\pgfqpoint{4.127336in}{3.943163in}}{\pgfqpoint{4.125022in}{3.937577in}}{\pgfqpoint{4.125022in}{3.931753in}}%
\pgfpathcurveto{\pgfqpoint{4.125022in}{3.925929in}}{\pgfqpoint{4.127336in}{3.920342in}}{\pgfqpoint{4.131454in}{3.916224in}}%
\pgfpathcurveto{\pgfqpoint{4.135572in}{3.912106in}}{\pgfqpoint{4.141159in}{3.909792in}}{\pgfqpoint{4.146983in}{3.909792in}}%
\pgfpathlineto{\pgfqpoint{4.146983in}{3.909792in}}%
\pgfpathclose%
\pgfusepath{stroke,fill}%
\end{pgfscope}%
\begin{pgfscope}%
\pgfpathrectangle{\pgfqpoint{1.582361in}{0.880000in}}{\pgfqpoint{5.035278in}{6.160000in}}%
\pgfusepath{clip}%
\pgfsetbuttcap%
\pgfsetroundjoin%
\definecolor{currentfill}{rgb}{0.800000,0.200000,0.200000}%
\pgfsetfillcolor{currentfill}%
\pgfsetlinewidth{1.003750pt}%
\definecolor{currentstroke}{rgb}{0.800000,0.200000,0.200000}%
\pgfsetstrokecolor{currentstroke}%
\pgfsetdash{}{0pt}%
\pgfpathmoveto{\pgfqpoint{4.191480in}{3.914374in}}%
\pgfpathcurveto{\pgfqpoint{4.197304in}{3.914374in}}{\pgfqpoint{4.202890in}{3.916688in}}{\pgfqpoint{4.207008in}{3.920806in}}%
\pgfpathcurveto{\pgfqpoint{4.211126in}{3.924925in}}{\pgfqpoint{4.213440in}{3.930511in}}{\pgfqpoint{4.213440in}{3.936335in}}%
\pgfpathcurveto{\pgfqpoint{4.213440in}{3.942159in}}{\pgfqpoint{4.211126in}{3.947745in}}{\pgfqpoint{4.207008in}{3.951863in}}%
\pgfpathcurveto{\pgfqpoint{4.202890in}{3.955981in}}{\pgfqpoint{4.197304in}{3.958295in}}{\pgfqpoint{4.191480in}{3.958295in}}%
\pgfpathcurveto{\pgfqpoint{4.185656in}{3.958295in}}{\pgfqpoint{4.180069in}{3.955981in}}{\pgfqpoint{4.175951in}{3.951863in}}%
\pgfpathcurveto{\pgfqpoint{4.171833in}{3.947745in}}{\pgfqpoint{4.169519in}{3.942159in}}{\pgfqpoint{4.169519in}{3.936335in}}%
\pgfpathcurveto{\pgfqpoint{4.169519in}{3.930511in}}{\pgfqpoint{4.171833in}{3.924925in}}{\pgfqpoint{4.175951in}{3.920806in}}%
\pgfpathcurveto{\pgfqpoint{4.180069in}{3.916688in}}{\pgfqpoint{4.185656in}{3.914374in}}{\pgfqpoint{4.191480in}{3.914374in}}%
\pgfpathlineto{\pgfqpoint{4.191480in}{3.914374in}}%
\pgfpathclose%
\pgfusepath{stroke,fill}%
\end{pgfscope}%
\begin{pgfscope}%
\pgfpathrectangle{\pgfqpoint{1.582361in}{0.880000in}}{\pgfqpoint{5.035278in}{6.160000in}}%
\pgfusepath{clip}%
\pgfsetbuttcap%
\pgfsetroundjoin%
\definecolor{currentfill}{rgb}{0.800000,0.200000,0.200000}%
\pgfsetfillcolor{currentfill}%
\pgfsetlinewidth{1.003750pt}%
\definecolor{currentstroke}{rgb}{0.800000,0.200000,0.200000}%
\pgfsetstrokecolor{currentstroke}%
\pgfsetdash{}{0pt}%
\pgfpathmoveto{\pgfqpoint{4.235810in}{3.920359in}}%
\pgfpathcurveto{\pgfqpoint{4.241634in}{3.920359in}}{\pgfqpoint{4.247220in}{3.922673in}}{\pgfqpoint{4.251338in}{3.926791in}}%
\pgfpathcurveto{\pgfqpoint{4.255456in}{3.930909in}}{\pgfqpoint{4.257770in}{3.936495in}}{\pgfqpoint{4.257770in}{3.942319in}}%
\pgfpathcurveto{\pgfqpoint{4.257770in}{3.948143in}}{\pgfqpoint{4.255456in}{3.953729in}}{\pgfqpoint{4.251338in}{3.957848in}}%
\pgfpathcurveto{\pgfqpoint{4.247220in}{3.961966in}}{\pgfqpoint{4.241634in}{3.964280in}}{\pgfqpoint{4.235810in}{3.964280in}}%
\pgfpathcurveto{\pgfqpoint{4.229986in}{3.964280in}}{\pgfqpoint{4.224400in}{3.961966in}}{\pgfqpoint{4.220282in}{3.957848in}}%
\pgfpathcurveto{\pgfqpoint{4.216163in}{3.953729in}}{\pgfqpoint{4.213850in}{3.948143in}}{\pgfqpoint{4.213850in}{3.942319in}}%
\pgfpathcurveto{\pgfqpoint{4.213850in}{3.936495in}}{\pgfqpoint{4.216163in}{3.930909in}}{\pgfqpoint{4.220282in}{3.926791in}}%
\pgfpathcurveto{\pgfqpoint{4.224400in}{3.922673in}}{\pgfqpoint{4.229986in}{3.920359in}}{\pgfqpoint{4.235810in}{3.920359in}}%
\pgfpathlineto{\pgfqpoint{4.235810in}{3.920359in}}%
\pgfpathclose%
\pgfusepath{stroke,fill}%
\end{pgfscope}%
\begin{pgfscope}%
\pgfpathrectangle{\pgfqpoint{1.582361in}{0.880000in}}{\pgfqpoint{5.035278in}{6.160000in}}%
\pgfusepath{clip}%
\pgfsetbuttcap%
\pgfsetroundjoin%
\definecolor{currentfill}{rgb}{0.800000,0.200000,0.200000}%
\pgfsetfillcolor{currentfill}%
\pgfsetlinewidth{1.003750pt}%
\definecolor{currentstroke}{rgb}{0.800000,0.200000,0.200000}%
\pgfsetstrokecolor{currentstroke}%
\pgfsetdash{}{0pt}%
\pgfpathmoveto{\pgfqpoint{4.279929in}{3.927740in}}%
\pgfpathcurveto{\pgfqpoint{4.285753in}{3.927740in}}{\pgfqpoint{4.291339in}{3.930054in}}{\pgfqpoint{4.295457in}{3.934172in}}%
\pgfpathcurveto{\pgfqpoint{4.299575in}{3.938290in}}{\pgfqpoint{4.301889in}{3.943876in}}{\pgfqpoint{4.301889in}{3.949700in}}%
\pgfpathcurveto{\pgfqpoint{4.301889in}{3.955524in}}{\pgfqpoint{4.299575in}{3.961110in}}{\pgfqpoint{4.295457in}{3.965229in}}%
\pgfpathcurveto{\pgfqpoint{4.291339in}{3.969347in}}{\pgfqpoint{4.285753in}{3.971661in}}{\pgfqpoint{4.279929in}{3.971661in}}%
\pgfpathcurveto{\pgfqpoint{4.274105in}{3.971661in}}{\pgfqpoint{4.268519in}{3.969347in}}{\pgfqpoint{4.264401in}{3.965229in}}%
\pgfpathcurveto{\pgfqpoint{4.260283in}{3.961110in}}{\pgfqpoint{4.257969in}{3.955524in}}{\pgfqpoint{4.257969in}{3.949700in}}%
\pgfpathcurveto{\pgfqpoint{4.257969in}{3.943876in}}{\pgfqpoint{4.260283in}{3.938290in}}{\pgfqpoint{4.264401in}{3.934172in}}%
\pgfpathcurveto{\pgfqpoint{4.268519in}{3.930054in}}{\pgfqpoint{4.274105in}{3.927740in}}{\pgfqpoint{4.279929in}{3.927740in}}%
\pgfpathlineto{\pgfqpoint{4.279929in}{3.927740in}}%
\pgfpathclose%
\pgfusepath{stroke,fill}%
\end{pgfscope}%
\begin{pgfscope}%
\pgfpathrectangle{\pgfqpoint{1.582361in}{0.880000in}}{\pgfqpoint{5.035278in}{6.160000in}}%
\pgfusepath{clip}%
\pgfsetbuttcap%
\pgfsetroundjoin%
\definecolor{currentfill}{rgb}{0.800000,0.200000,0.200000}%
\pgfsetfillcolor{currentfill}%
\pgfsetlinewidth{1.003750pt}%
\definecolor{currentstroke}{rgb}{0.800000,0.200000,0.200000}%
\pgfsetstrokecolor{currentstroke}%
\pgfsetdash{}{0pt}%
\pgfpathmoveto{\pgfqpoint{4.323793in}{3.936510in}}%
\pgfpathcurveto{\pgfqpoint{4.329617in}{3.936510in}}{\pgfqpoint{4.335203in}{3.938824in}}{\pgfqpoint{4.339321in}{3.942942in}}%
\pgfpathcurveto{\pgfqpoint{4.343440in}{3.947060in}}{\pgfqpoint{4.345753in}{3.952647in}}{\pgfqpoint{4.345753in}{3.958470in}}%
\pgfpathcurveto{\pgfqpoint{4.345753in}{3.964294in}}{\pgfqpoint{4.343440in}{3.969881in}}{\pgfqpoint{4.339321in}{3.973999in}}%
\pgfpathcurveto{\pgfqpoint{4.335203in}{3.978117in}}{\pgfqpoint{4.329617in}{3.980431in}}{\pgfqpoint{4.323793in}{3.980431in}}%
\pgfpathcurveto{\pgfqpoint{4.317969in}{3.980431in}}{\pgfqpoint{4.312383in}{3.978117in}}{\pgfqpoint{4.308265in}{3.973999in}}%
\pgfpathcurveto{\pgfqpoint{4.304147in}{3.969881in}}{\pgfqpoint{4.301833in}{3.964294in}}{\pgfqpoint{4.301833in}{3.958470in}}%
\pgfpathcurveto{\pgfqpoint{4.301833in}{3.952647in}}{\pgfqpoint{4.304147in}{3.947060in}}{\pgfqpoint{4.308265in}{3.942942in}}%
\pgfpathcurveto{\pgfqpoint{4.312383in}{3.938824in}}{\pgfqpoint{4.317969in}{3.936510in}}{\pgfqpoint{4.323793in}{3.936510in}}%
\pgfpathlineto{\pgfqpoint{4.323793in}{3.936510in}}%
\pgfpathclose%
\pgfusepath{stroke,fill}%
\end{pgfscope}%
\begin{pgfscope}%
\pgfpathrectangle{\pgfqpoint{1.582361in}{0.880000in}}{\pgfqpoint{5.035278in}{6.160000in}}%
\pgfusepath{clip}%
\pgfsetbuttcap%
\pgfsetroundjoin%
\definecolor{currentfill}{rgb}{0.800000,0.200000,0.200000}%
\pgfsetfillcolor{currentfill}%
\pgfsetlinewidth{1.003750pt}%
\definecolor{currentstroke}{rgb}{0.800000,0.200000,0.200000}%
\pgfsetstrokecolor{currentstroke}%
\pgfsetdash{}{0pt}%
\pgfpathmoveto{\pgfqpoint{4.367359in}{3.946661in}}%
\pgfpathcurveto{\pgfqpoint{4.373183in}{3.946661in}}{\pgfqpoint{4.378769in}{3.948975in}}{\pgfqpoint{4.382887in}{3.953093in}}%
\pgfpathcurveto{\pgfqpoint{4.387005in}{3.957211in}}{\pgfqpoint{4.389319in}{3.962797in}}{\pgfqpoint{4.389319in}{3.968621in}}%
\pgfpathcurveto{\pgfqpoint{4.389319in}{3.974445in}}{\pgfqpoint{4.387005in}{3.980031in}}{\pgfqpoint{4.382887in}{3.984149in}}%
\pgfpathcurveto{\pgfqpoint{4.378769in}{3.988267in}}{\pgfqpoint{4.373183in}{3.990581in}}{\pgfqpoint{4.367359in}{3.990581in}}%
\pgfpathcurveto{\pgfqpoint{4.361535in}{3.990581in}}{\pgfqpoint{4.355949in}{3.988267in}}{\pgfqpoint{4.351830in}{3.984149in}}%
\pgfpathcurveto{\pgfqpoint{4.347712in}{3.980031in}}{\pgfqpoint{4.345398in}{3.974445in}}{\pgfqpoint{4.345398in}{3.968621in}}%
\pgfpathcurveto{\pgfqpoint{4.345398in}{3.962797in}}{\pgfqpoint{4.347712in}{3.957211in}}{\pgfqpoint{4.351830in}{3.953093in}}%
\pgfpathcurveto{\pgfqpoint{4.355949in}{3.948975in}}{\pgfqpoint{4.361535in}{3.946661in}}{\pgfqpoint{4.367359in}{3.946661in}}%
\pgfpathlineto{\pgfqpoint{4.367359in}{3.946661in}}%
\pgfpathclose%
\pgfusepath{stroke,fill}%
\end{pgfscope}%
\begin{pgfscope}%
\pgfpathrectangle{\pgfqpoint{1.582361in}{0.880000in}}{\pgfqpoint{5.035278in}{6.160000in}}%
\pgfusepath{clip}%
\pgfsetbuttcap%
\pgfsetroundjoin%
\definecolor{currentfill}{rgb}{0.800000,0.200000,0.200000}%
\pgfsetfillcolor{currentfill}%
\pgfsetlinewidth{1.003750pt}%
\definecolor{currentstroke}{rgb}{0.800000,0.200000,0.200000}%
\pgfsetstrokecolor{currentstroke}%
\pgfsetdash{}{0pt}%
\pgfpathmoveto{\pgfqpoint{4.410582in}{3.958181in}}%
\pgfpathcurveto{\pgfqpoint{4.416406in}{3.958181in}}{\pgfqpoint{4.421992in}{3.960495in}}{\pgfqpoint{4.426110in}{3.964613in}}%
\pgfpathcurveto{\pgfqpoint{4.430228in}{3.968732in}}{\pgfqpoint{4.432542in}{3.974318in}}{\pgfqpoint{4.432542in}{3.980142in}}%
\pgfpathcurveto{\pgfqpoint{4.432542in}{3.985966in}}{\pgfqpoint{4.430228in}{3.991552in}}{\pgfqpoint{4.426110in}{3.995670in}}%
\pgfpathcurveto{\pgfqpoint{4.421992in}{3.999788in}}{\pgfqpoint{4.416406in}{4.002102in}}{\pgfqpoint{4.410582in}{4.002102in}}%
\pgfpathcurveto{\pgfqpoint{4.404758in}{4.002102in}}{\pgfqpoint{4.399172in}{3.999788in}}{\pgfqpoint{4.395054in}{3.995670in}}%
\pgfpathcurveto{\pgfqpoint{4.390936in}{3.991552in}}{\pgfqpoint{4.388622in}{3.985966in}}{\pgfqpoint{4.388622in}{3.980142in}}%
\pgfpathcurveto{\pgfqpoint{4.388622in}{3.974318in}}{\pgfqpoint{4.390936in}{3.968732in}}{\pgfqpoint{4.395054in}{3.964613in}}%
\pgfpathcurveto{\pgfqpoint{4.399172in}{3.960495in}}{\pgfqpoint{4.404758in}{3.958181in}}{\pgfqpoint{4.410582in}{3.958181in}}%
\pgfpathlineto{\pgfqpoint{4.410582in}{3.958181in}}%
\pgfpathclose%
\pgfusepath{stroke,fill}%
\end{pgfscope}%
\begin{pgfscope}%
\pgfpathrectangle{\pgfqpoint{1.582361in}{0.880000in}}{\pgfqpoint{5.035278in}{6.160000in}}%
\pgfusepath{clip}%
\pgfsetbuttcap%
\pgfsetroundjoin%
\definecolor{currentfill}{rgb}{0.800000,0.200000,0.200000}%
\pgfsetfillcolor{currentfill}%
\pgfsetlinewidth{1.003750pt}%
\definecolor{currentstroke}{rgb}{0.800000,0.200000,0.200000}%
\pgfsetstrokecolor{currentstroke}%
\pgfsetdash{}{0pt}%
\pgfpathmoveto{\pgfqpoint{4.453420in}{3.971061in}}%
\pgfpathcurveto{\pgfqpoint{4.459244in}{3.971061in}}{\pgfqpoint{4.464830in}{3.973375in}}{\pgfqpoint{4.468948in}{3.977493in}}%
\pgfpathcurveto{\pgfqpoint{4.473067in}{3.981611in}}{\pgfqpoint{4.475380in}{3.987197in}}{\pgfqpoint{4.475380in}{3.993021in}}%
\pgfpathcurveto{\pgfqpoint{4.475380in}{3.998845in}}{\pgfqpoint{4.473067in}{4.004431in}}{\pgfqpoint{4.468948in}{4.008549in}}%
\pgfpathcurveto{\pgfqpoint{4.464830in}{4.012668in}}{\pgfqpoint{4.459244in}{4.014981in}}{\pgfqpoint{4.453420in}{4.014981in}}%
\pgfpathcurveto{\pgfqpoint{4.447596in}{4.014981in}}{\pgfqpoint{4.442010in}{4.012668in}}{\pgfqpoint{4.437892in}{4.008549in}}%
\pgfpathcurveto{\pgfqpoint{4.433774in}{4.004431in}}{\pgfqpoint{4.431460in}{3.998845in}}{\pgfqpoint{4.431460in}{3.993021in}}%
\pgfpathcurveto{\pgfqpoint{4.431460in}{3.987197in}}{\pgfqpoint{4.433774in}{3.981611in}}{\pgfqpoint{4.437892in}{3.977493in}}%
\pgfpathcurveto{\pgfqpoint{4.442010in}{3.973375in}}{\pgfqpoint{4.447596in}{3.971061in}}{\pgfqpoint{4.453420in}{3.971061in}}%
\pgfpathlineto{\pgfqpoint{4.453420in}{3.971061in}}%
\pgfpathclose%
\pgfusepath{stroke,fill}%
\end{pgfscope}%
\begin{pgfscope}%
\pgfpathrectangle{\pgfqpoint{1.582361in}{0.880000in}}{\pgfqpoint{5.035278in}{6.160000in}}%
\pgfusepath{clip}%
\pgfsetbuttcap%
\pgfsetroundjoin%
\definecolor{currentfill}{rgb}{0.800000,0.200000,0.200000}%
\pgfsetfillcolor{currentfill}%
\pgfsetlinewidth{1.003750pt}%
\definecolor{currentstroke}{rgb}{0.800000,0.200000,0.200000}%
\pgfsetstrokecolor{currentstroke}%
\pgfsetdash{}{0pt}%
\pgfpathmoveto{\pgfqpoint{4.495830in}{3.985286in}}%
\pgfpathcurveto{\pgfqpoint{4.501654in}{3.985286in}}{\pgfqpoint{4.507240in}{3.987600in}}{\pgfqpoint{4.511359in}{3.991718in}}%
\pgfpathcurveto{\pgfqpoint{4.515477in}{3.995836in}}{\pgfqpoint{4.517791in}{4.001423in}}{\pgfqpoint{4.517791in}{4.007247in}}%
\pgfpathcurveto{\pgfqpoint{4.517791in}{4.013071in}}{\pgfqpoint{4.515477in}{4.018657in}}{\pgfqpoint{4.511359in}{4.022775in}}%
\pgfpathcurveto{\pgfqpoint{4.507240in}{4.026893in}}{\pgfqpoint{4.501654in}{4.029207in}}{\pgfqpoint{4.495830in}{4.029207in}}%
\pgfpathcurveto{\pgfqpoint{4.490006in}{4.029207in}}{\pgfqpoint{4.484420in}{4.026893in}}{\pgfqpoint{4.480302in}{4.022775in}}%
\pgfpathcurveto{\pgfqpoint{4.476184in}{4.018657in}}{\pgfqpoint{4.473870in}{4.013071in}}{\pgfqpoint{4.473870in}{4.007247in}}%
\pgfpathcurveto{\pgfqpoint{4.473870in}{4.001423in}}{\pgfqpoint{4.476184in}{3.995836in}}{\pgfqpoint{4.480302in}{3.991718in}}%
\pgfpathcurveto{\pgfqpoint{4.484420in}{3.987600in}}{\pgfqpoint{4.490006in}{3.985286in}}{\pgfqpoint{4.495830in}{3.985286in}}%
\pgfpathlineto{\pgfqpoint{4.495830in}{3.985286in}}%
\pgfpathclose%
\pgfusepath{stroke,fill}%
\end{pgfscope}%
\begin{pgfscope}%
\pgfpathrectangle{\pgfqpoint{1.582361in}{0.880000in}}{\pgfqpoint{5.035278in}{6.160000in}}%
\pgfusepath{clip}%
\pgfsetbuttcap%
\pgfsetroundjoin%
\definecolor{currentfill}{rgb}{0.800000,0.200000,0.200000}%
\pgfsetfillcolor{currentfill}%
\pgfsetlinewidth{1.003750pt}%
\definecolor{currentstroke}{rgb}{0.800000,0.200000,0.200000}%
\pgfsetstrokecolor{currentstroke}%
\pgfsetdash{}{0pt}%
\pgfpathmoveto{\pgfqpoint{4.537770in}{4.000843in}}%
\pgfpathcurveto{\pgfqpoint{4.543594in}{4.000843in}}{\pgfqpoint{4.549180in}{4.003157in}}{\pgfqpoint{4.553298in}{4.007275in}}%
\pgfpathcurveto{\pgfqpoint{4.557417in}{4.011394in}}{\pgfqpoint{4.559730in}{4.016980in}}{\pgfqpoint{4.559730in}{4.022804in}}%
\pgfpathcurveto{\pgfqpoint{4.559730in}{4.028628in}}{\pgfqpoint{4.557417in}{4.034214in}}{\pgfqpoint{4.553298in}{4.038332in}}%
\pgfpathcurveto{\pgfqpoint{4.549180in}{4.042450in}}{\pgfqpoint{4.543594in}{4.044764in}}{\pgfqpoint{4.537770in}{4.044764in}}%
\pgfpathcurveto{\pgfqpoint{4.531946in}{4.044764in}}{\pgfqpoint{4.526360in}{4.042450in}}{\pgfqpoint{4.522242in}{4.038332in}}%
\pgfpathcurveto{\pgfqpoint{4.518124in}{4.034214in}}{\pgfqpoint{4.515810in}{4.028628in}}{\pgfqpoint{4.515810in}{4.022804in}}%
\pgfpathcurveto{\pgfqpoint{4.515810in}{4.016980in}}{\pgfqpoint{4.518124in}{4.011394in}}{\pgfqpoint{4.522242in}{4.007275in}}%
\pgfpathcurveto{\pgfqpoint{4.526360in}{4.003157in}}{\pgfqpoint{4.531946in}{4.000843in}}{\pgfqpoint{4.537770in}{4.000843in}}%
\pgfpathlineto{\pgfqpoint{4.537770in}{4.000843in}}%
\pgfpathclose%
\pgfusepath{stroke,fill}%
\end{pgfscope}%
\begin{pgfscope}%
\pgfpathrectangle{\pgfqpoint{1.582361in}{0.880000in}}{\pgfqpoint{5.035278in}{6.160000in}}%
\pgfusepath{clip}%
\pgfsetbuttcap%
\pgfsetroundjoin%
\definecolor{currentfill}{rgb}{0.800000,0.200000,0.200000}%
\pgfsetfillcolor{currentfill}%
\pgfsetlinewidth{1.003750pt}%
\definecolor{currentstroke}{rgb}{0.800000,0.200000,0.200000}%
\pgfsetstrokecolor{currentstroke}%
\pgfsetdash{}{0pt}%
\pgfpathmoveto{\pgfqpoint{4.579198in}{4.017717in}}%
\pgfpathcurveto{\pgfqpoint{4.585022in}{4.017717in}}{\pgfqpoint{4.590608in}{4.020031in}}{\pgfqpoint{4.594726in}{4.024149in}}%
\pgfpathcurveto{\pgfqpoint{4.598845in}{4.028267in}}{\pgfqpoint{4.601158in}{4.033853in}}{\pgfqpoint{4.601158in}{4.039677in}}%
\pgfpathcurveto{\pgfqpoint{4.601158in}{4.045501in}}{\pgfqpoint{4.598845in}{4.051087in}}{\pgfqpoint{4.594726in}{4.055205in}}%
\pgfpathcurveto{\pgfqpoint{4.590608in}{4.059323in}}{\pgfqpoint{4.585022in}{4.061637in}}{\pgfqpoint{4.579198in}{4.061637in}}%
\pgfpathcurveto{\pgfqpoint{4.573374in}{4.061637in}}{\pgfqpoint{4.567788in}{4.059323in}}{\pgfqpoint{4.563670in}{4.055205in}}%
\pgfpathcurveto{\pgfqpoint{4.559552in}{4.051087in}}{\pgfqpoint{4.557238in}{4.045501in}}{\pgfqpoint{4.557238in}{4.039677in}}%
\pgfpathcurveto{\pgfqpoint{4.557238in}{4.033853in}}{\pgfqpoint{4.559552in}{4.028267in}}{\pgfqpoint{4.563670in}{4.024149in}}%
\pgfpathcurveto{\pgfqpoint{4.567788in}{4.020031in}}{\pgfqpoint{4.573374in}{4.017717in}}{\pgfqpoint{4.579198in}{4.017717in}}%
\pgfpathlineto{\pgfqpoint{4.579198in}{4.017717in}}%
\pgfpathclose%
\pgfusepath{stroke,fill}%
\end{pgfscope}%
\begin{pgfscope}%
\pgfpathrectangle{\pgfqpoint{1.582361in}{0.880000in}}{\pgfqpoint{5.035278in}{6.160000in}}%
\pgfusepath{clip}%
\pgfsetbuttcap%
\pgfsetroundjoin%
\definecolor{currentfill}{rgb}{0.800000,0.200000,0.200000}%
\pgfsetfillcolor{currentfill}%
\pgfsetlinewidth{1.003750pt}%
\definecolor{currentstroke}{rgb}{0.800000,0.200000,0.200000}%
\pgfsetstrokecolor{currentstroke}%
\pgfsetdash{}{0pt}%
\pgfpathmoveto{\pgfqpoint{4.620073in}{4.035890in}}%
\pgfpathcurveto{\pgfqpoint{4.625897in}{4.035890in}}{\pgfqpoint{4.631483in}{4.038203in}}{\pgfqpoint{4.635601in}{4.042322in}}%
\pgfpathcurveto{\pgfqpoint{4.639719in}{4.046440in}}{\pgfqpoint{4.642033in}{4.052026in}}{\pgfqpoint{4.642033in}{4.057850in}}%
\pgfpathcurveto{\pgfqpoint{4.642033in}{4.063674in}}{\pgfqpoint{4.639719in}{4.069260in}}{\pgfqpoint{4.635601in}{4.073378in}}%
\pgfpathcurveto{\pgfqpoint{4.631483in}{4.077496in}}{\pgfqpoint{4.625897in}{4.079810in}}{\pgfqpoint{4.620073in}{4.079810in}}%
\pgfpathcurveto{\pgfqpoint{4.614249in}{4.079810in}}{\pgfqpoint{4.608663in}{4.077496in}}{\pgfqpoint{4.604544in}{4.073378in}}%
\pgfpathcurveto{\pgfqpoint{4.600426in}{4.069260in}}{\pgfqpoint{4.598112in}{4.063674in}}{\pgfqpoint{4.598112in}{4.057850in}}%
\pgfpathcurveto{\pgfqpoint{4.598112in}{4.052026in}}{\pgfqpoint{4.600426in}{4.046440in}}{\pgfqpoint{4.604544in}{4.042322in}}%
\pgfpathcurveto{\pgfqpoint{4.608663in}{4.038203in}}{\pgfqpoint{4.614249in}{4.035890in}}{\pgfqpoint{4.620073in}{4.035890in}}%
\pgfpathlineto{\pgfqpoint{4.620073in}{4.035890in}}%
\pgfpathclose%
\pgfusepath{stroke,fill}%
\end{pgfscope}%
\begin{pgfscope}%
\pgfpathrectangle{\pgfqpoint{1.582361in}{0.880000in}}{\pgfqpoint{5.035278in}{6.160000in}}%
\pgfusepath{clip}%
\pgfsetbuttcap%
\pgfsetroundjoin%
\definecolor{currentfill}{rgb}{0.800000,0.200000,0.200000}%
\pgfsetfillcolor{currentfill}%
\pgfsetlinewidth{1.003750pt}%
\definecolor{currentstroke}{rgb}{0.800000,0.200000,0.200000}%
\pgfsetstrokecolor{currentstroke}%
\pgfsetdash{}{0pt}%
\pgfpathmoveto{\pgfqpoint{4.660353in}{4.055344in}}%
\pgfpathcurveto{\pgfqpoint{4.666177in}{4.055344in}}{\pgfqpoint{4.671763in}{4.057658in}}{\pgfqpoint{4.675882in}{4.061776in}}%
\pgfpathcurveto{\pgfqpoint{4.680000in}{4.065894in}}{\pgfqpoint{4.682314in}{4.071480in}}{\pgfqpoint{4.682314in}{4.077304in}}%
\pgfpathcurveto{\pgfqpoint{4.682314in}{4.083128in}}{\pgfqpoint{4.680000in}{4.088714in}}{\pgfqpoint{4.675882in}{4.092832in}}%
\pgfpathcurveto{\pgfqpoint{4.671763in}{4.096950in}}{\pgfqpoint{4.666177in}{4.099264in}}{\pgfqpoint{4.660353in}{4.099264in}}%
\pgfpathcurveto{\pgfqpoint{4.654529in}{4.099264in}}{\pgfqpoint{4.648943in}{4.096950in}}{\pgfqpoint{4.644825in}{4.092832in}}%
\pgfpathcurveto{\pgfqpoint{4.640707in}{4.088714in}}{\pgfqpoint{4.638393in}{4.083128in}}{\pgfqpoint{4.638393in}{4.077304in}}%
\pgfpathcurveto{\pgfqpoint{4.638393in}{4.071480in}}{\pgfqpoint{4.640707in}{4.065894in}}{\pgfqpoint{4.644825in}{4.061776in}}%
\pgfpathcurveto{\pgfqpoint{4.648943in}{4.057658in}}{\pgfqpoint{4.654529in}{4.055344in}}{\pgfqpoint{4.660353in}{4.055344in}}%
\pgfpathlineto{\pgfqpoint{4.660353in}{4.055344in}}%
\pgfpathclose%
\pgfusepath{stroke,fill}%
\end{pgfscope}%
\begin{pgfscope}%
\pgfpathrectangle{\pgfqpoint{1.582361in}{0.880000in}}{\pgfqpoint{5.035278in}{6.160000in}}%
\pgfusepath{clip}%
\pgfsetbuttcap%
\pgfsetroundjoin%
\definecolor{currentfill}{rgb}{0.800000,0.200000,0.200000}%
\pgfsetfillcolor{currentfill}%
\pgfsetlinewidth{1.003750pt}%
\definecolor{currentstroke}{rgb}{0.800000,0.200000,0.200000}%
\pgfsetstrokecolor{currentstroke}%
\pgfsetdash{}{0pt}%
\pgfpathmoveto{\pgfqpoint{4.700000in}{4.076060in}}%
\pgfpathcurveto{\pgfqpoint{4.705824in}{4.076060in}}{\pgfqpoint{4.711410in}{4.078374in}}{\pgfqpoint{4.715528in}{4.082492in}}%
\pgfpathcurveto{\pgfqpoint{4.719646in}{4.086610in}}{\pgfqpoint{4.721960in}{4.092196in}}{\pgfqpoint{4.721960in}{4.098020in}}%
\pgfpathcurveto{\pgfqpoint{4.721960in}{4.103844in}}{\pgfqpoint{4.719646in}{4.109430in}}{\pgfqpoint{4.715528in}{4.113548in}}%
\pgfpathcurveto{\pgfqpoint{4.711410in}{4.117666in}}{\pgfqpoint{4.705824in}{4.119980in}}{\pgfqpoint{4.700000in}{4.119980in}}%
\pgfpathcurveto{\pgfqpoint{4.694176in}{4.119980in}}{\pgfqpoint{4.688590in}{4.117666in}}{\pgfqpoint{4.684471in}{4.113548in}}%
\pgfpathcurveto{\pgfqpoint{4.680353in}{4.109430in}}{\pgfqpoint{4.678039in}{4.103844in}}{\pgfqpoint{4.678039in}{4.098020in}}%
\pgfpathcurveto{\pgfqpoint{4.678039in}{4.092196in}}{\pgfqpoint{4.680353in}{4.086610in}}{\pgfqpoint{4.684471in}{4.082492in}}%
\pgfpathcurveto{\pgfqpoint{4.688590in}{4.078374in}}{\pgfqpoint{4.694176in}{4.076060in}}{\pgfqpoint{4.700000in}{4.076060in}}%
\pgfpathlineto{\pgfqpoint{4.700000in}{4.076060in}}%
\pgfpathclose%
\pgfusepath{stroke,fill}%
\end{pgfscope}%
\begin{pgfscope}%
\pgfpathrectangle{\pgfqpoint{1.582361in}{0.880000in}}{\pgfqpoint{5.035278in}{6.160000in}}%
\pgfusepath{clip}%
\pgfsetbuttcap%
\pgfsetroundjoin%
\definecolor{currentfill}{rgb}{0.800000,0.200000,0.200000}%
\pgfsetfillcolor{currentfill}%
\pgfsetlinewidth{1.003750pt}%
\definecolor{currentstroke}{rgb}{0.800000,0.200000,0.200000}%
\pgfsetstrokecolor{currentstroke}%
\pgfsetdash{}{0pt}%
\pgfpathmoveto{\pgfqpoint{4.738972in}{4.098017in}}%
\pgfpathcurveto{\pgfqpoint{4.744796in}{4.098017in}}{\pgfqpoint{4.750382in}{4.100331in}}{\pgfqpoint{4.754500in}{4.104449in}}%
\pgfpathcurveto{\pgfqpoint{4.758619in}{4.108567in}}{\pgfqpoint{4.760932in}{4.114153in}}{\pgfqpoint{4.760932in}{4.119977in}}%
\pgfpathcurveto{\pgfqpoint{4.760932in}{4.125801in}}{\pgfqpoint{4.758619in}{4.131387in}}{\pgfqpoint{4.754500in}{4.135505in}}%
\pgfpathcurveto{\pgfqpoint{4.750382in}{4.139624in}}{\pgfqpoint{4.744796in}{4.141937in}}{\pgfqpoint{4.738972in}{4.141937in}}%
\pgfpathcurveto{\pgfqpoint{4.733148in}{4.141937in}}{\pgfqpoint{4.727562in}{4.139624in}}{\pgfqpoint{4.723444in}{4.135505in}}%
\pgfpathcurveto{\pgfqpoint{4.719326in}{4.131387in}}{\pgfqpoint{4.717012in}{4.125801in}}{\pgfqpoint{4.717012in}{4.119977in}}%
\pgfpathcurveto{\pgfqpoint{4.717012in}{4.114153in}}{\pgfqpoint{4.719326in}{4.108567in}}{\pgfqpoint{4.723444in}{4.104449in}}%
\pgfpathcurveto{\pgfqpoint{4.727562in}{4.100331in}}{\pgfqpoint{4.733148in}{4.098017in}}{\pgfqpoint{4.738972in}{4.098017in}}%
\pgfpathlineto{\pgfqpoint{4.738972in}{4.098017in}}%
\pgfpathclose%
\pgfusepath{stroke,fill}%
\end{pgfscope}%
\begin{pgfscope}%
\pgfpathrectangle{\pgfqpoint{1.582361in}{0.880000in}}{\pgfqpoint{5.035278in}{6.160000in}}%
\pgfusepath{clip}%
\pgfsetbuttcap%
\pgfsetroundjoin%
\definecolor{currentfill}{rgb}{0.800000,0.200000,0.200000}%
\pgfsetfillcolor{currentfill}%
\pgfsetlinewidth{1.003750pt}%
\definecolor{currentstroke}{rgb}{0.800000,0.200000,0.200000}%
\pgfsetstrokecolor{currentstroke}%
\pgfsetdash{}{0pt}%
\pgfpathmoveto{\pgfqpoint{4.777232in}{4.121193in}}%
\pgfpathcurveto{\pgfqpoint{4.783056in}{4.121193in}}{\pgfqpoint{4.788642in}{4.123507in}}{\pgfqpoint{4.792760in}{4.127626in}}%
\pgfpathcurveto{\pgfqpoint{4.796879in}{4.131744in}}{\pgfqpoint{4.799192in}{4.137330in}}{\pgfqpoint{4.799192in}{4.143154in}}%
\pgfpathcurveto{\pgfqpoint{4.799192in}{4.148978in}}{\pgfqpoint{4.796879in}{4.154564in}}{\pgfqpoint{4.792760in}{4.158682in}}%
\pgfpathcurveto{\pgfqpoint{4.788642in}{4.162800in}}{\pgfqpoint{4.783056in}{4.165114in}}{\pgfqpoint{4.777232in}{4.165114in}}%
\pgfpathcurveto{\pgfqpoint{4.771408in}{4.165114in}}{\pgfqpoint{4.765822in}{4.162800in}}{\pgfqpoint{4.761704in}{4.158682in}}%
\pgfpathcurveto{\pgfqpoint{4.757586in}{4.154564in}}{\pgfqpoint{4.755272in}{4.148978in}}{\pgfqpoint{4.755272in}{4.143154in}}%
\pgfpathcurveto{\pgfqpoint{4.755272in}{4.137330in}}{\pgfqpoint{4.757586in}{4.131744in}}{\pgfqpoint{4.761704in}{4.127626in}}%
\pgfpathcurveto{\pgfqpoint{4.765822in}{4.123507in}}{\pgfqpoint{4.771408in}{4.121193in}}{\pgfqpoint{4.777232in}{4.121193in}}%
\pgfpathlineto{\pgfqpoint{4.777232in}{4.121193in}}%
\pgfpathclose%
\pgfusepath{stroke,fill}%
\end{pgfscope}%
\begin{pgfscope}%
\pgfpathrectangle{\pgfqpoint{1.582361in}{0.880000in}}{\pgfqpoint{5.035278in}{6.160000in}}%
\pgfusepath{clip}%
\pgfsetbuttcap%
\pgfsetroundjoin%
\definecolor{currentfill}{rgb}{0.800000,0.200000,0.200000}%
\pgfsetfillcolor{currentfill}%
\pgfsetlinewidth{1.003750pt}%
\definecolor{currentstroke}{rgb}{0.800000,0.200000,0.200000}%
\pgfsetstrokecolor{currentstroke}%
\pgfsetdash{}{0pt}%
\pgfpathmoveto{\pgfqpoint{4.814742in}{4.145566in}}%
\pgfpathcurveto{\pgfqpoint{4.820565in}{4.145566in}}{\pgfqpoint{4.826152in}{4.147880in}}{\pgfqpoint{4.830270in}{4.151998in}}%
\pgfpathcurveto{\pgfqpoint{4.834388in}{4.156116in}}{\pgfqpoint{4.836702in}{4.161703in}}{\pgfqpoint{4.836702in}{4.167527in}}%
\pgfpathcurveto{\pgfqpoint{4.836702in}{4.173351in}}{\pgfqpoint{4.834388in}{4.178937in}}{\pgfqpoint{4.830270in}{4.183055in}}%
\pgfpathcurveto{\pgfqpoint{4.826152in}{4.187173in}}{\pgfqpoint{4.820565in}{4.189487in}}{\pgfqpoint{4.814742in}{4.189487in}}%
\pgfpathcurveto{\pgfqpoint{4.808918in}{4.189487in}}{\pgfqpoint{4.803331in}{4.187173in}}{\pgfqpoint{4.799213in}{4.183055in}}%
\pgfpathcurveto{\pgfqpoint{4.795095in}{4.178937in}}{\pgfqpoint{4.792781in}{4.173351in}}{\pgfqpoint{4.792781in}{4.167527in}}%
\pgfpathcurveto{\pgfqpoint{4.792781in}{4.161703in}}{\pgfqpoint{4.795095in}{4.156116in}}{\pgfqpoint{4.799213in}{4.151998in}}%
\pgfpathcurveto{\pgfqpoint{4.803331in}{4.147880in}}{\pgfqpoint{4.808918in}{4.145566in}}{\pgfqpoint{4.814742in}{4.145566in}}%
\pgfpathlineto{\pgfqpoint{4.814742in}{4.145566in}}%
\pgfpathclose%
\pgfusepath{stroke,fill}%
\end{pgfscope}%
\begin{pgfscope}%
\pgfpathrectangle{\pgfqpoint{1.582361in}{0.880000in}}{\pgfqpoint{5.035278in}{6.160000in}}%
\pgfusepath{clip}%
\pgfsetbuttcap%
\pgfsetroundjoin%
\definecolor{currentfill}{rgb}{0.800000,0.200000,0.200000}%
\pgfsetfillcolor{currentfill}%
\pgfsetlinewidth{1.003750pt}%
\definecolor{currentstroke}{rgb}{0.800000,0.200000,0.200000}%
\pgfsetstrokecolor{currentstroke}%
\pgfsetdash{}{0pt}%
\pgfpathmoveto{\pgfqpoint{4.851463in}{4.171111in}}%
\pgfpathcurveto{\pgfqpoint{4.857287in}{4.171111in}}{\pgfqpoint{4.862873in}{4.173425in}}{\pgfqpoint{4.866991in}{4.177543in}}%
\pgfpathcurveto{\pgfqpoint{4.871109in}{4.181661in}}{\pgfqpoint{4.873423in}{4.187247in}}{\pgfqpoint{4.873423in}{4.193071in}}%
\pgfpathcurveto{\pgfqpoint{4.873423in}{4.198895in}}{\pgfqpoint{4.871109in}{4.204482in}}{\pgfqpoint{4.866991in}{4.208600in}}%
\pgfpathcurveto{\pgfqpoint{4.862873in}{4.212718in}}{\pgfqpoint{4.857287in}{4.215032in}}{\pgfqpoint{4.851463in}{4.215032in}}%
\pgfpathcurveto{\pgfqpoint{4.845639in}{4.215032in}}{\pgfqpoint{4.840053in}{4.212718in}}{\pgfqpoint{4.835934in}{4.208600in}}%
\pgfpathcurveto{\pgfqpoint{4.831816in}{4.204482in}}{\pgfqpoint{4.829502in}{4.198895in}}{\pgfqpoint{4.829502in}{4.193071in}}%
\pgfpathcurveto{\pgfqpoint{4.829502in}{4.187247in}}{\pgfqpoint{4.831816in}{4.181661in}}{\pgfqpoint{4.835934in}{4.177543in}}%
\pgfpathcurveto{\pgfqpoint{4.840053in}{4.173425in}}{\pgfqpoint{4.845639in}{4.171111in}}{\pgfqpoint{4.851463in}{4.171111in}}%
\pgfpathlineto{\pgfqpoint{4.851463in}{4.171111in}}%
\pgfpathclose%
\pgfusepath{stroke,fill}%
\end{pgfscope}%
\begin{pgfscope}%
\pgfpathrectangle{\pgfqpoint{1.582361in}{0.880000in}}{\pgfqpoint{5.035278in}{6.160000in}}%
\pgfusepath{clip}%
\pgfsetbuttcap%
\pgfsetroundjoin%
\definecolor{currentfill}{rgb}{0.800000,0.200000,0.200000}%
\pgfsetfillcolor{currentfill}%
\pgfsetlinewidth{1.003750pt}%
\definecolor{currentstroke}{rgb}{0.800000,0.200000,0.200000}%
\pgfsetstrokecolor{currentstroke}%
\pgfsetdash{}{0pt}%
\pgfpathmoveto{\pgfqpoint{4.887359in}{4.197802in}}%
\pgfpathcurveto{\pgfqpoint{4.893183in}{4.197802in}}{\pgfqpoint{4.898769in}{4.200116in}}{\pgfqpoint{4.902887in}{4.204235in}}%
\pgfpathcurveto{\pgfqpoint{4.907006in}{4.208353in}}{\pgfqpoint{4.909319in}{4.213939in}}{\pgfqpoint{4.909319in}{4.219763in}}%
\pgfpathcurveto{\pgfqpoint{4.909319in}{4.225587in}}{\pgfqpoint{4.907006in}{4.231173in}}{\pgfqpoint{4.902887in}{4.235291in}}%
\pgfpathcurveto{\pgfqpoint{4.898769in}{4.239409in}}{\pgfqpoint{4.893183in}{4.241723in}}{\pgfqpoint{4.887359in}{4.241723in}}%
\pgfpathcurveto{\pgfqpoint{4.881535in}{4.241723in}}{\pgfqpoint{4.875949in}{4.239409in}}{\pgfqpoint{4.871831in}{4.235291in}}%
\pgfpathcurveto{\pgfqpoint{4.867713in}{4.231173in}}{\pgfqpoint{4.865399in}{4.225587in}}{\pgfqpoint{4.865399in}{4.219763in}}%
\pgfpathcurveto{\pgfqpoint{4.865399in}{4.213939in}}{\pgfqpoint{4.867713in}{4.208353in}}{\pgfqpoint{4.871831in}{4.204235in}}%
\pgfpathcurveto{\pgfqpoint{4.875949in}{4.200116in}}{\pgfqpoint{4.881535in}{4.197802in}}{\pgfqpoint{4.887359in}{4.197802in}}%
\pgfpathlineto{\pgfqpoint{4.887359in}{4.197802in}}%
\pgfpathclose%
\pgfusepath{stroke,fill}%
\end{pgfscope}%
\begin{pgfscope}%
\pgfpathrectangle{\pgfqpoint{1.582361in}{0.880000in}}{\pgfqpoint{5.035278in}{6.160000in}}%
\pgfusepath{clip}%
\pgfsetbuttcap%
\pgfsetroundjoin%
\definecolor{currentfill}{rgb}{0.800000,0.200000,0.200000}%
\pgfsetfillcolor{currentfill}%
\pgfsetlinewidth{1.003750pt}%
\definecolor{currentstroke}{rgb}{0.800000,0.200000,0.200000}%
\pgfsetstrokecolor{currentstroke}%
\pgfsetdash{}{0pt}%
\pgfpathmoveto{\pgfqpoint{4.922395in}{4.225614in}}%
\pgfpathcurveto{\pgfqpoint{4.928219in}{4.225614in}}{\pgfqpoint{4.933805in}{4.227928in}}{\pgfqpoint{4.937923in}{4.232046in}}%
\pgfpathcurveto{\pgfqpoint{4.942042in}{4.236164in}}{\pgfqpoint{4.944355in}{4.241750in}}{\pgfqpoint{4.944355in}{4.247574in}}%
\pgfpathcurveto{\pgfqpoint{4.944355in}{4.253398in}}{\pgfqpoint{4.942042in}{4.258984in}}{\pgfqpoint{4.937923in}{4.263102in}}%
\pgfpathcurveto{\pgfqpoint{4.933805in}{4.267220in}}{\pgfqpoint{4.928219in}{4.269534in}}{\pgfqpoint{4.922395in}{4.269534in}}%
\pgfpathcurveto{\pgfqpoint{4.916571in}{4.269534in}}{\pgfqpoint{4.910985in}{4.267220in}}{\pgfqpoint{4.906867in}{4.263102in}}%
\pgfpathcurveto{\pgfqpoint{4.902749in}{4.258984in}}{\pgfqpoint{4.900435in}{4.253398in}}{\pgfqpoint{4.900435in}{4.247574in}}%
\pgfpathcurveto{\pgfqpoint{4.900435in}{4.241750in}}{\pgfqpoint{4.902749in}{4.236164in}}{\pgfqpoint{4.906867in}{4.232046in}}%
\pgfpathcurveto{\pgfqpoint{4.910985in}{4.227928in}}{\pgfqpoint{4.916571in}{4.225614in}}{\pgfqpoint{4.922395in}{4.225614in}}%
\pgfpathlineto{\pgfqpoint{4.922395in}{4.225614in}}%
\pgfpathclose%
\pgfusepath{stroke,fill}%
\end{pgfscope}%
\begin{pgfscope}%
\pgfpathrectangle{\pgfqpoint{1.582361in}{0.880000in}}{\pgfqpoint{5.035278in}{6.160000in}}%
\pgfusepath{clip}%
\pgfsetbuttcap%
\pgfsetroundjoin%
\definecolor{currentfill}{rgb}{0.800000,0.200000,0.200000}%
\pgfsetfillcolor{currentfill}%
\pgfsetlinewidth{1.003750pt}%
\definecolor{currentstroke}{rgb}{0.800000,0.200000,0.200000}%
\pgfsetstrokecolor{currentstroke}%
\pgfsetdash{}{0pt}%
\pgfpathmoveto{\pgfqpoint{4.956536in}{4.254517in}}%
\pgfpathcurveto{\pgfqpoint{4.962360in}{4.254517in}}{\pgfqpoint{4.967946in}{4.256831in}}{\pgfqpoint{4.972064in}{4.260949in}}%
\pgfpathcurveto{\pgfqpoint{4.976182in}{4.265067in}}{\pgfqpoint{4.978496in}{4.270653in}}{\pgfqpoint{4.978496in}{4.276477in}}%
\pgfpathcurveto{\pgfqpoint{4.978496in}{4.282301in}}{\pgfqpoint{4.976182in}{4.287887in}}{\pgfqpoint{4.972064in}{4.292006in}}%
\pgfpathcurveto{\pgfqpoint{4.967946in}{4.296124in}}{\pgfqpoint{4.962360in}{4.298438in}}{\pgfqpoint{4.956536in}{4.298438in}}%
\pgfpathcurveto{\pgfqpoint{4.950712in}{4.298438in}}{\pgfqpoint{4.945126in}{4.296124in}}{\pgfqpoint{4.941007in}{4.292006in}}%
\pgfpathcurveto{\pgfqpoint{4.936889in}{4.287887in}}{\pgfqpoint{4.934575in}{4.282301in}}{\pgfqpoint{4.934575in}{4.276477in}}%
\pgfpathcurveto{\pgfqpoint{4.934575in}{4.270653in}}{\pgfqpoint{4.936889in}{4.265067in}}{\pgfqpoint{4.941007in}{4.260949in}}%
\pgfpathcurveto{\pgfqpoint{4.945126in}{4.256831in}}{\pgfqpoint{4.950712in}{4.254517in}}{\pgfqpoint{4.956536in}{4.254517in}}%
\pgfpathlineto{\pgfqpoint{4.956536in}{4.254517in}}%
\pgfpathclose%
\pgfusepath{stroke,fill}%
\end{pgfscope}%
\begin{pgfscope}%
\pgfpathrectangle{\pgfqpoint{1.582361in}{0.880000in}}{\pgfqpoint{5.035278in}{6.160000in}}%
\pgfusepath{clip}%
\pgfsetbuttcap%
\pgfsetroundjoin%
\definecolor{currentfill}{rgb}{0.800000,0.200000,0.200000}%
\pgfsetfillcolor{currentfill}%
\pgfsetlinewidth{1.003750pt}%
\definecolor{currentstroke}{rgb}{0.800000,0.200000,0.200000}%
\pgfsetstrokecolor{currentstroke}%
\pgfsetdash{}{0pt}%
\pgfpathmoveto{\pgfqpoint{4.989747in}{4.284484in}}%
\pgfpathcurveto{\pgfqpoint{4.995571in}{4.284484in}}{\pgfqpoint{5.001157in}{4.286798in}}{\pgfqpoint{5.005275in}{4.290916in}}%
\pgfpathcurveto{\pgfqpoint{5.009393in}{4.295034in}}{\pgfqpoint{5.011707in}{4.300620in}}{\pgfqpoint{5.011707in}{4.306444in}}%
\pgfpathcurveto{\pgfqpoint{5.011707in}{4.312268in}}{\pgfqpoint{5.009393in}{4.317854in}}{\pgfqpoint{5.005275in}{4.321972in}}%
\pgfpathcurveto{\pgfqpoint{5.001157in}{4.326091in}}{\pgfqpoint{4.995571in}{4.328404in}}{\pgfqpoint{4.989747in}{4.328404in}}%
\pgfpathcurveto{\pgfqpoint{4.983923in}{4.328404in}}{\pgfqpoint{4.978337in}{4.326091in}}{\pgfqpoint{4.974219in}{4.321972in}}%
\pgfpathcurveto{\pgfqpoint{4.970100in}{4.317854in}}{\pgfqpoint{4.967787in}{4.312268in}}{\pgfqpoint{4.967787in}{4.306444in}}%
\pgfpathcurveto{\pgfqpoint{4.967787in}{4.300620in}}{\pgfqpoint{4.970100in}{4.295034in}}{\pgfqpoint{4.974219in}{4.290916in}}%
\pgfpathcurveto{\pgfqpoint{4.978337in}{4.286798in}}{\pgfqpoint{4.983923in}{4.284484in}}{\pgfqpoint{4.989747in}{4.284484in}}%
\pgfpathlineto{\pgfqpoint{4.989747in}{4.284484in}}%
\pgfpathclose%
\pgfusepath{stroke,fill}%
\end{pgfscope}%
\begin{pgfscope}%
\pgfpathrectangle{\pgfqpoint{1.582361in}{0.880000in}}{\pgfqpoint{5.035278in}{6.160000in}}%
\pgfusepath{clip}%
\pgfsetbuttcap%
\pgfsetroundjoin%
\definecolor{currentfill}{rgb}{0.800000,0.200000,0.200000}%
\pgfsetfillcolor{currentfill}%
\pgfsetlinewidth{1.003750pt}%
\definecolor{currentstroke}{rgb}{0.800000,0.200000,0.200000}%
\pgfsetstrokecolor{currentstroke}%
\pgfsetdash{}{0pt}%
\pgfpathmoveto{\pgfqpoint{5.021995in}{4.315484in}}%
\pgfpathcurveto{\pgfqpoint{5.027819in}{4.315484in}}{\pgfqpoint{5.033405in}{4.317798in}}{\pgfqpoint{5.037524in}{4.321916in}}%
\pgfpathcurveto{\pgfqpoint{5.041642in}{4.326034in}}{\pgfqpoint{5.043956in}{4.331620in}}{\pgfqpoint{5.043956in}{4.337444in}}%
\pgfpathcurveto{\pgfqpoint{5.043956in}{4.343268in}}{\pgfqpoint{5.041642in}{4.348854in}}{\pgfqpoint{5.037524in}{4.352973in}}%
\pgfpathcurveto{\pgfqpoint{5.033405in}{4.357091in}}{\pgfqpoint{5.027819in}{4.359405in}}{\pgfqpoint{5.021995in}{4.359405in}}%
\pgfpathcurveto{\pgfqpoint{5.016171in}{4.359405in}}{\pgfqpoint{5.010585in}{4.357091in}}{\pgfqpoint{5.006467in}{4.352973in}}%
\pgfpathcurveto{\pgfqpoint{5.002349in}{4.348854in}}{\pgfqpoint{5.000035in}{4.343268in}}{\pgfqpoint{5.000035in}{4.337444in}}%
\pgfpathcurveto{\pgfqpoint{5.000035in}{4.331620in}}{\pgfqpoint{5.002349in}{4.326034in}}{\pgfqpoint{5.006467in}{4.321916in}}%
\pgfpathcurveto{\pgfqpoint{5.010585in}{4.317798in}}{\pgfqpoint{5.016171in}{4.315484in}}{\pgfqpoint{5.021995in}{4.315484in}}%
\pgfpathlineto{\pgfqpoint{5.021995in}{4.315484in}}%
\pgfpathclose%
\pgfusepath{stroke,fill}%
\end{pgfscope}%
\begin{pgfscope}%
\pgfpathrectangle{\pgfqpoint{1.582361in}{0.880000in}}{\pgfqpoint{5.035278in}{6.160000in}}%
\pgfusepath{clip}%
\pgfsetbuttcap%
\pgfsetroundjoin%
\definecolor{currentfill}{rgb}{0.800000,0.200000,0.200000}%
\pgfsetfillcolor{currentfill}%
\pgfsetlinewidth{1.003750pt}%
\definecolor{currentstroke}{rgb}{0.800000,0.200000,0.200000}%
\pgfsetstrokecolor{currentstroke}%
\pgfsetdash{}{0pt}%
\pgfpathmoveto{\pgfqpoint{5.053249in}{4.347487in}}%
\pgfpathcurveto{\pgfqpoint{5.059073in}{4.347487in}}{\pgfqpoint{5.064659in}{4.349801in}}{\pgfqpoint{5.068777in}{4.353919in}}%
\pgfpathcurveto{\pgfqpoint{5.072896in}{4.358037in}}{\pgfqpoint{5.075209in}{4.363623in}}{\pgfqpoint{5.075209in}{4.369447in}}%
\pgfpathcurveto{\pgfqpoint{5.075209in}{4.375271in}}{\pgfqpoint{5.072896in}{4.380857in}}{\pgfqpoint{5.068777in}{4.384975in}}%
\pgfpathcurveto{\pgfqpoint{5.064659in}{4.389094in}}{\pgfqpoint{5.059073in}{4.391407in}}{\pgfqpoint{5.053249in}{4.391407in}}%
\pgfpathcurveto{\pgfqpoint{5.047425in}{4.391407in}}{\pgfqpoint{5.041839in}{4.389094in}}{\pgfqpoint{5.037721in}{4.384975in}}%
\pgfpathcurveto{\pgfqpoint{5.033603in}{4.380857in}}{\pgfqpoint{5.031289in}{4.375271in}}{\pgfqpoint{5.031289in}{4.369447in}}%
\pgfpathcurveto{\pgfqpoint{5.031289in}{4.363623in}}{\pgfqpoint{5.033603in}{4.358037in}}{\pgfqpoint{5.037721in}{4.353919in}}%
\pgfpathcurveto{\pgfqpoint{5.041839in}{4.349801in}}{\pgfqpoint{5.047425in}{4.347487in}}{\pgfqpoint{5.053249in}{4.347487in}}%
\pgfpathlineto{\pgfqpoint{5.053249in}{4.347487in}}%
\pgfpathclose%
\pgfusepath{stroke,fill}%
\end{pgfscope}%
\begin{pgfscope}%
\pgfpathrectangle{\pgfqpoint{1.582361in}{0.880000in}}{\pgfqpoint{5.035278in}{6.160000in}}%
\pgfusepath{clip}%
\pgfsetbuttcap%
\pgfsetroundjoin%
\definecolor{currentfill}{rgb}{0.800000,0.200000,0.200000}%
\pgfsetfillcolor{currentfill}%
\pgfsetlinewidth{1.003750pt}%
\definecolor{currentstroke}{rgb}{0.800000,0.200000,0.200000}%
\pgfsetstrokecolor{currentstroke}%
\pgfsetdash{}{0pt}%
\pgfpathmoveto{\pgfqpoint{5.083477in}{4.380460in}}%
\pgfpathcurveto{\pgfqpoint{5.089301in}{4.380460in}}{\pgfqpoint{5.094887in}{4.382774in}}{\pgfqpoint{5.099005in}{4.386892in}}%
\pgfpathcurveto{\pgfqpoint{5.103124in}{4.391011in}}{\pgfqpoint{5.105437in}{4.396597in}}{\pgfqpoint{5.105437in}{4.402421in}}%
\pgfpathcurveto{\pgfqpoint{5.105437in}{4.408245in}}{\pgfqpoint{5.103124in}{4.413831in}}{\pgfqpoint{5.099005in}{4.417949in}}%
\pgfpathcurveto{\pgfqpoint{5.094887in}{4.422067in}}{\pgfqpoint{5.089301in}{4.424381in}}{\pgfqpoint{5.083477in}{4.424381in}}%
\pgfpathcurveto{\pgfqpoint{5.077653in}{4.424381in}}{\pgfqpoint{5.072067in}{4.422067in}}{\pgfqpoint{5.067949in}{4.417949in}}%
\pgfpathcurveto{\pgfqpoint{5.063831in}{4.413831in}}{\pgfqpoint{5.061517in}{4.408245in}}{\pgfqpoint{5.061517in}{4.402421in}}%
\pgfpathcurveto{\pgfqpoint{5.061517in}{4.396597in}}{\pgfqpoint{5.063831in}{4.391011in}}{\pgfqpoint{5.067949in}{4.386892in}}%
\pgfpathcurveto{\pgfqpoint{5.072067in}{4.382774in}}{\pgfqpoint{5.077653in}{4.380460in}}{\pgfqpoint{5.083477in}{4.380460in}}%
\pgfpathlineto{\pgfqpoint{5.083477in}{4.380460in}}%
\pgfpathclose%
\pgfusepath{stroke,fill}%
\end{pgfscope}%
\begin{pgfscope}%
\pgfpathrectangle{\pgfqpoint{1.582361in}{0.880000in}}{\pgfqpoint{5.035278in}{6.160000in}}%
\pgfusepath{clip}%
\pgfsetbuttcap%
\pgfsetroundjoin%
\definecolor{currentfill}{rgb}{0.800000,0.200000,0.200000}%
\pgfsetfillcolor{currentfill}%
\pgfsetlinewidth{1.003750pt}%
\definecolor{currentstroke}{rgb}{0.800000,0.200000,0.200000}%
\pgfsetstrokecolor{currentstroke}%
\pgfsetdash{}{0pt}%
\pgfpathmoveto{\pgfqpoint{5.112649in}{4.414372in}}%
\pgfpathcurveto{\pgfqpoint{5.118473in}{4.414372in}}{\pgfqpoint{5.124059in}{4.416686in}}{\pgfqpoint{5.128177in}{4.420804in}}%
\pgfpathcurveto{\pgfqpoint{5.132296in}{4.424922in}}{\pgfqpoint{5.134609in}{4.430508in}}{\pgfqpoint{5.134609in}{4.436332in}}%
\pgfpathcurveto{\pgfqpoint{5.134609in}{4.442156in}}{\pgfqpoint{5.132296in}{4.447742in}}{\pgfqpoint{5.128177in}{4.451860in}}%
\pgfpathcurveto{\pgfqpoint{5.124059in}{4.455978in}}{\pgfqpoint{5.118473in}{4.458292in}}{\pgfqpoint{5.112649in}{4.458292in}}%
\pgfpathcurveto{\pgfqpoint{5.106825in}{4.458292in}}{\pgfqpoint{5.101239in}{4.455978in}}{\pgfqpoint{5.097121in}{4.451860in}}%
\pgfpathcurveto{\pgfqpoint{5.093003in}{4.447742in}}{\pgfqpoint{5.090689in}{4.442156in}}{\pgfqpoint{5.090689in}{4.436332in}}%
\pgfpathcurveto{\pgfqpoint{5.090689in}{4.430508in}}{\pgfqpoint{5.093003in}{4.424922in}}{\pgfqpoint{5.097121in}{4.420804in}}%
\pgfpathcurveto{\pgfqpoint{5.101239in}{4.416686in}}{\pgfqpoint{5.106825in}{4.414372in}}{\pgfqpoint{5.112649in}{4.414372in}}%
\pgfpathlineto{\pgfqpoint{5.112649in}{4.414372in}}%
\pgfpathclose%
\pgfusepath{stroke,fill}%
\end{pgfscope}%
\begin{pgfscope}%
\pgfpathrectangle{\pgfqpoint{1.582361in}{0.880000in}}{\pgfqpoint{5.035278in}{6.160000in}}%
\pgfusepath{clip}%
\pgfsetbuttcap%
\pgfsetroundjoin%
\definecolor{currentfill}{rgb}{0.800000,0.200000,0.200000}%
\pgfsetfillcolor{currentfill}%
\pgfsetlinewidth{1.003750pt}%
\definecolor{currentstroke}{rgb}{0.800000,0.200000,0.200000}%
\pgfsetstrokecolor{currentstroke}%
\pgfsetdash{}{0pt}%
\pgfpathmoveto{\pgfqpoint{5.140736in}{4.449187in}}%
\pgfpathcurveto{\pgfqpoint{5.146560in}{4.449187in}}{\pgfqpoint{5.152146in}{4.451501in}}{\pgfqpoint{5.156264in}{4.455619in}}%
\pgfpathcurveto{\pgfqpoint{5.160382in}{4.459737in}}{\pgfqpoint{5.162696in}{4.465323in}}{\pgfqpoint{5.162696in}{4.471147in}}%
\pgfpathcurveto{\pgfqpoint{5.162696in}{4.476971in}}{\pgfqpoint{5.160382in}{4.482558in}}{\pgfqpoint{5.156264in}{4.486676in}}%
\pgfpathcurveto{\pgfqpoint{5.152146in}{4.490794in}}{\pgfqpoint{5.146560in}{4.493108in}}{\pgfqpoint{5.140736in}{4.493108in}}%
\pgfpathcurveto{\pgfqpoint{5.134912in}{4.493108in}}{\pgfqpoint{5.129326in}{4.490794in}}{\pgfqpoint{5.125208in}{4.486676in}}%
\pgfpathcurveto{\pgfqpoint{5.121090in}{4.482558in}}{\pgfqpoint{5.118776in}{4.476971in}}{\pgfqpoint{5.118776in}{4.471147in}}%
\pgfpathcurveto{\pgfqpoint{5.118776in}{4.465323in}}{\pgfqpoint{5.121090in}{4.459737in}}{\pgfqpoint{5.125208in}{4.455619in}}%
\pgfpathcurveto{\pgfqpoint{5.129326in}{4.451501in}}{\pgfqpoint{5.134912in}{4.449187in}}{\pgfqpoint{5.140736in}{4.449187in}}%
\pgfpathlineto{\pgfqpoint{5.140736in}{4.449187in}}%
\pgfpathclose%
\pgfusepath{stroke,fill}%
\end{pgfscope}%
\begin{pgfscope}%
\pgfpathrectangle{\pgfqpoint{1.582361in}{0.880000in}}{\pgfqpoint{5.035278in}{6.160000in}}%
\pgfusepath{clip}%
\pgfsetbuttcap%
\pgfsetroundjoin%
\definecolor{currentfill}{rgb}{0.800000,0.200000,0.200000}%
\pgfsetfillcolor{currentfill}%
\pgfsetlinewidth{1.003750pt}%
\definecolor{currentstroke}{rgb}{0.800000,0.200000,0.200000}%
\pgfsetstrokecolor{currentstroke}%
\pgfsetdash{}{0pt}%
\pgfpathmoveto{\pgfqpoint{5.167710in}{4.484872in}}%
\pgfpathcurveto{\pgfqpoint{5.173534in}{4.484872in}}{\pgfqpoint{5.179120in}{4.487186in}}{\pgfqpoint{5.183238in}{4.491304in}}%
\pgfpathcurveto{\pgfqpoint{5.187356in}{4.495422in}}{\pgfqpoint{5.189670in}{4.501008in}}{\pgfqpoint{5.189670in}{4.506832in}}%
\pgfpathcurveto{\pgfqpoint{5.189670in}{4.512656in}}{\pgfqpoint{5.187356in}{4.518242in}}{\pgfqpoint{5.183238in}{4.522360in}}%
\pgfpathcurveto{\pgfqpoint{5.179120in}{4.526478in}}{\pgfqpoint{5.173534in}{4.528792in}}{\pgfqpoint{5.167710in}{4.528792in}}%
\pgfpathcurveto{\pgfqpoint{5.161886in}{4.528792in}}{\pgfqpoint{5.156300in}{4.526478in}}{\pgfqpoint{5.152182in}{4.522360in}}%
\pgfpathcurveto{\pgfqpoint{5.148063in}{4.518242in}}{\pgfqpoint{5.145750in}{4.512656in}}{\pgfqpoint{5.145750in}{4.506832in}}%
\pgfpathcurveto{\pgfqpoint{5.145750in}{4.501008in}}{\pgfqpoint{5.148063in}{4.495422in}}{\pgfqpoint{5.152182in}{4.491304in}}%
\pgfpathcurveto{\pgfqpoint{5.156300in}{4.487186in}}{\pgfqpoint{5.161886in}{4.484872in}}{\pgfqpoint{5.167710in}{4.484872in}}%
\pgfpathlineto{\pgfqpoint{5.167710in}{4.484872in}}%
\pgfpathclose%
\pgfusepath{stroke,fill}%
\end{pgfscope}%
\begin{pgfscope}%
\pgfpathrectangle{\pgfqpoint{1.582361in}{0.880000in}}{\pgfqpoint{5.035278in}{6.160000in}}%
\pgfusepath{clip}%
\pgfsetbuttcap%
\pgfsetroundjoin%
\definecolor{currentfill}{rgb}{0.800000,0.200000,0.200000}%
\pgfsetfillcolor{currentfill}%
\pgfsetlinewidth{1.003750pt}%
\definecolor{currentstroke}{rgb}{0.800000,0.200000,0.200000}%
\pgfsetstrokecolor{currentstroke}%
\pgfsetdash{}{0pt}%
\pgfpathmoveto{\pgfqpoint{5.193544in}{4.521390in}}%
\pgfpathcurveto{\pgfqpoint{5.199368in}{4.521390in}}{\pgfqpoint{5.204954in}{4.523704in}}{\pgfqpoint{5.209072in}{4.527822in}}%
\pgfpathcurveto{\pgfqpoint{5.213190in}{4.531940in}}{\pgfqpoint{5.215504in}{4.537527in}}{\pgfqpoint{5.215504in}{4.543350in}}%
\pgfpathcurveto{\pgfqpoint{5.215504in}{4.549174in}}{\pgfqpoint{5.213190in}{4.554761in}}{\pgfqpoint{5.209072in}{4.558879in}}%
\pgfpathcurveto{\pgfqpoint{5.204954in}{4.562997in}}{\pgfqpoint{5.199368in}{4.565311in}}{\pgfqpoint{5.193544in}{4.565311in}}%
\pgfpathcurveto{\pgfqpoint{5.187720in}{4.565311in}}{\pgfqpoint{5.182134in}{4.562997in}}{\pgfqpoint{5.178015in}{4.558879in}}%
\pgfpathcurveto{\pgfqpoint{5.173897in}{4.554761in}}{\pgfqpoint{5.171583in}{4.549174in}}{\pgfqpoint{5.171583in}{4.543350in}}%
\pgfpathcurveto{\pgfqpoint{5.171583in}{4.537527in}}{\pgfqpoint{5.173897in}{4.531940in}}{\pgfqpoint{5.178015in}{4.527822in}}%
\pgfpathcurveto{\pgfqpoint{5.182134in}{4.523704in}}{\pgfqpoint{5.187720in}{4.521390in}}{\pgfqpoint{5.193544in}{4.521390in}}%
\pgfpathlineto{\pgfqpoint{5.193544in}{4.521390in}}%
\pgfpathclose%
\pgfusepath{stroke,fill}%
\end{pgfscope}%
\begin{pgfscope}%
\pgfpathrectangle{\pgfqpoint{1.582361in}{0.880000in}}{\pgfqpoint{5.035278in}{6.160000in}}%
\pgfusepath{clip}%
\pgfsetbuttcap%
\pgfsetroundjoin%
\definecolor{currentfill}{rgb}{0.800000,0.200000,0.200000}%
\pgfsetfillcolor{currentfill}%
\pgfsetlinewidth{1.003750pt}%
\definecolor{currentstroke}{rgb}{0.800000,0.200000,0.200000}%
\pgfsetstrokecolor{currentstroke}%
\pgfsetdash{}{0pt}%
\pgfpathmoveto{\pgfqpoint{5.218212in}{4.558706in}}%
\pgfpathcurveto{\pgfqpoint{5.224036in}{4.558706in}}{\pgfqpoint{5.229622in}{4.561020in}}{\pgfqpoint{5.233740in}{4.565138in}}%
\pgfpathcurveto{\pgfqpoint{5.237858in}{4.569256in}}{\pgfqpoint{5.240172in}{4.574842in}}{\pgfqpoint{5.240172in}{4.580666in}}%
\pgfpathcurveto{\pgfqpoint{5.240172in}{4.586490in}}{\pgfqpoint{5.237858in}{4.592076in}}{\pgfqpoint{5.233740in}{4.596194in}}%
\pgfpathcurveto{\pgfqpoint{5.229622in}{4.600313in}}{\pgfqpoint{5.224036in}{4.602627in}}{\pgfqpoint{5.218212in}{4.602627in}}%
\pgfpathcurveto{\pgfqpoint{5.212388in}{4.602627in}}{\pgfqpoint{5.206802in}{4.600313in}}{\pgfqpoint{5.202684in}{4.596194in}}%
\pgfpathcurveto{\pgfqpoint{5.198566in}{4.592076in}}{\pgfqpoint{5.196252in}{4.586490in}}{\pgfqpoint{5.196252in}{4.580666in}}%
\pgfpathcurveto{\pgfqpoint{5.196252in}{4.574842in}}{\pgfqpoint{5.198566in}{4.569256in}}{\pgfqpoint{5.202684in}{4.565138in}}%
\pgfpathcurveto{\pgfqpoint{5.206802in}{4.561020in}}{\pgfqpoint{5.212388in}{4.558706in}}{\pgfqpoint{5.218212in}{4.558706in}}%
\pgfpathlineto{\pgfqpoint{5.218212in}{4.558706in}}%
\pgfpathclose%
\pgfusepath{stroke,fill}%
\end{pgfscope}%
\begin{pgfscope}%
\pgfpathrectangle{\pgfqpoint{1.582361in}{0.880000in}}{\pgfqpoint{5.035278in}{6.160000in}}%
\pgfusepath{clip}%
\pgfsetbuttcap%
\pgfsetroundjoin%
\definecolor{currentfill}{rgb}{0.800000,0.200000,0.200000}%
\pgfsetfillcolor{currentfill}%
\pgfsetlinewidth{1.003750pt}%
\definecolor{currentstroke}{rgb}{0.800000,0.200000,0.200000}%
\pgfsetstrokecolor{currentstroke}%
\pgfsetdash{}{0pt}%
\pgfpathmoveto{\pgfqpoint{5.241690in}{4.596782in}}%
\pgfpathcurveto{\pgfqpoint{5.247514in}{4.596782in}}{\pgfqpoint{5.253100in}{4.599096in}}{\pgfqpoint{5.257218in}{4.603214in}}%
\pgfpathcurveto{\pgfqpoint{5.261336in}{4.607332in}}{\pgfqpoint{5.263650in}{4.612918in}}{\pgfqpoint{5.263650in}{4.618742in}}%
\pgfpathcurveto{\pgfqpoint{5.263650in}{4.624566in}}{\pgfqpoint{5.261336in}{4.630152in}}{\pgfqpoint{5.257218in}{4.634270in}}%
\pgfpathcurveto{\pgfqpoint{5.253100in}{4.638389in}}{\pgfqpoint{5.247514in}{4.640702in}}{\pgfqpoint{5.241690in}{4.640702in}}%
\pgfpathcurveto{\pgfqpoint{5.235866in}{4.640702in}}{\pgfqpoint{5.230280in}{4.638389in}}{\pgfqpoint{5.226162in}{4.634270in}}%
\pgfpathcurveto{\pgfqpoint{5.222043in}{4.630152in}}{\pgfqpoint{5.219730in}{4.624566in}}{\pgfqpoint{5.219730in}{4.618742in}}%
\pgfpathcurveto{\pgfqpoint{5.219730in}{4.612918in}}{\pgfqpoint{5.222043in}{4.607332in}}{\pgfqpoint{5.226162in}{4.603214in}}%
\pgfpathcurveto{\pgfqpoint{5.230280in}{4.599096in}}{\pgfqpoint{5.235866in}{4.596782in}}{\pgfqpoint{5.241690in}{4.596782in}}%
\pgfpathlineto{\pgfqpoint{5.241690in}{4.596782in}}%
\pgfpathclose%
\pgfusepath{stroke,fill}%
\end{pgfscope}%
\begin{pgfscope}%
\pgfpathrectangle{\pgfqpoint{1.582361in}{0.880000in}}{\pgfqpoint{5.035278in}{6.160000in}}%
\pgfusepath{clip}%
\pgfsetbuttcap%
\pgfsetroundjoin%
\definecolor{currentfill}{rgb}{0.800000,0.200000,0.200000}%
\pgfsetfillcolor{currentfill}%
\pgfsetlinewidth{1.003750pt}%
\definecolor{currentstroke}{rgb}{0.800000,0.200000,0.200000}%
\pgfsetstrokecolor{currentstroke}%
\pgfsetdash{}{0pt}%
\pgfpathmoveto{\pgfqpoint{5.263954in}{4.635580in}}%
\pgfpathcurveto{\pgfqpoint{5.269778in}{4.635580in}}{\pgfqpoint{5.275364in}{4.637894in}}{\pgfqpoint{5.279482in}{4.642012in}}%
\pgfpathcurveto{\pgfqpoint{5.283600in}{4.646130in}}{\pgfqpoint{5.285914in}{4.651716in}}{\pgfqpoint{5.285914in}{4.657540in}}%
\pgfpathcurveto{\pgfqpoint{5.285914in}{4.663364in}}{\pgfqpoint{5.283600in}{4.668950in}}{\pgfqpoint{5.279482in}{4.673068in}}%
\pgfpathcurveto{\pgfqpoint{5.275364in}{4.677187in}}{\pgfqpoint{5.269778in}{4.679500in}}{\pgfqpoint{5.263954in}{4.679500in}}%
\pgfpathcurveto{\pgfqpoint{5.258130in}{4.679500in}}{\pgfqpoint{5.252544in}{4.677187in}}{\pgfqpoint{5.248426in}{4.673068in}}%
\pgfpathcurveto{\pgfqpoint{5.244308in}{4.668950in}}{\pgfqpoint{5.241994in}{4.663364in}}{\pgfqpoint{5.241994in}{4.657540in}}%
\pgfpathcurveto{\pgfqpoint{5.241994in}{4.651716in}}{\pgfqpoint{5.244308in}{4.646130in}}{\pgfqpoint{5.248426in}{4.642012in}}%
\pgfpathcurveto{\pgfqpoint{5.252544in}{4.637894in}}{\pgfqpoint{5.258130in}{4.635580in}}{\pgfqpoint{5.263954in}{4.635580in}}%
\pgfpathlineto{\pgfqpoint{5.263954in}{4.635580in}}%
\pgfpathclose%
\pgfusepath{stroke,fill}%
\end{pgfscope}%
\begin{pgfscope}%
\pgfpathrectangle{\pgfqpoint{1.582361in}{0.880000in}}{\pgfqpoint{5.035278in}{6.160000in}}%
\pgfusepath{clip}%
\pgfsetbuttcap%
\pgfsetroundjoin%
\definecolor{currentfill}{rgb}{0.800000,0.200000,0.200000}%
\pgfsetfillcolor{currentfill}%
\pgfsetlinewidth{1.003750pt}%
\definecolor{currentstroke}{rgb}{0.800000,0.200000,0.200000}%
\pgfsetstrokecolor{currentstroke}%
\pgfsetdash{}{0pt}%
\pgfpathmoveto{\pgfqpoint{5.284982in}{4.675062in}}%
\pgfpathcurveto{\pgfqpoint{5.290806in}{4.675062in}}{\pgfqpoint{5.296392in}{4.677375in}}{\pgfqpoint{5.300510in}{4.681494in}}%
\pgfpathcurveto{\pgfqpoint{5.304629in}{4.685612in}}{\pgfqpoint{5.306942in}{4.691198in}}{\pgfqpoint{5.306942in}{4.697022in}}%
\pgfpathcurveto{\pgfqpoint{5.306942in}{4.702846in}}{\pgfqpoint{5.304629in}{4.708432in}}{\pgfqpoint{5.300510in}{4.712550in}}%
\pgfpathcurveto{\pgfqpoint{5.296392in}{4.716668in}}{\pgfqpoint{5.290806in}{4.718982in}}{\pgfqpoint{5.284982in}{4.718982in}}%
\pgfpathcurveto{\pgfqpoint{5.279158in}{4.718982in}}{\pgfqpoint{5.273572in}{4.716668in}}{\pgfqpoint{5.269454in}{4.712550in}}%
\pgfpathcurveto{\pgfqpoint{5.265336in}{4.708432in}}{\pgfqpoint{5.263022in}{4.702846in}}{\pgfqpoint{5.263022in}{4.697022in}}%
\pgfpathcurveto{\pgfqpoint{5.263022in}{4.691198in}}{\pgfqpoint{5.265336in}{4.685612in}}{\pgfqpoint{5.269454in}{4.681494in}}%
\pgfpathcurveto{\pgfqpoint{5.273572in}{4.677375in}}{\pgfqpoint{5.279158in}{4.675062in}}{\pgfqpoint{5.284982in}{4.675062in}}%
\pgfpathlineto{\pgfqpoint{5.284982in}{4.675062in}}%
\pgfpathclose%
\pgfusepath{stroke,fill}%
\end{pgfscope}%
\begin{pgfscope}%
\pgfpathrectangle{\pgfqpoint{1.582361in}{0.880000in}}{\pgfqpoint{5.035278in}{6.160000in}}%
\pgfusepath{clip}%
\pgfsetbuttcap%
\pgfsetroundjoin%
\definecolor{currentfill}{rgb}{0.800000,0.200000,0.200000}%
\pgfsetfillcolor{currentfill}%
\pgfsetlinewidth{1.003750pt}%
\definecolor{currentstroke}{rgb}{0.800000,0.200000,0.200000}%
\pgfsetstrokecolor{currentstroke}%
\pgfsetdash{}{0pt}%
\pgfpathmoveto{\pgfqpoint{5.304754in}{4.715187in}}%
\pgfpathcurveto{\pgfqpoint{5.310578in}{4.715187in}}{\pgfqpoint{5.316164in}{4.717501in}}{\pgfqpoint{5.320282in}{4.721619in}}%
\pgfpathcurveto{\pgfqpoint{5.324400in}{4.725737in}}{\pgfqpoint{5.326714in}{4.731324in}}{\pgfqpoint{5.326714in}{4.737148in}}%
\pgfpathcurveto{\pgfqpoint{5.326714in}{4.742971in}}{\pgfqpoint{5.324400in}{4.748558in}}{\pgfqpoint{5.320282in}{4.752676in}}%
\pgfpathcurveto{\pgfqpoint{5.316164in}{4.756794in}}{\pgfqpoint{5.310578in}{4.759108in}}{\pgfqpoint{5.304754in}{4.759108in}}%
\pgfpathcurveto{\pgfqpoint{5.298930in}{4.759108in}}{\pgfqpoint{5.293344in}{4.756794in}}{\pgfqpoint{5.289225in}{4.752676in}}%
\pgfpathcurveto{\pgfqpoint{5.285107in}{4.748558in}}{\pgfqpoint{5.282793in}{4.742971in}}{\pgfqpoint{5.282793in}{4.737148in}}%
\pgfpathcurveto{\pgfqpoint{5.282793in}{4.731324in}}{\pgfqpoint{5.285107in}{4.725737in}}{\pgfqpoint{5.289225in}{4.721619in}}%
\pgfpathcurveto{\pgfqpoint{5.293344in}{4.717501in}}{\pgfqpoint{5.298930in}{4.715187in}}{\pgfqpoint{5.304754in}{4.715187in}}%
\pgfpathlineto{\pgfqpoint{5.304754in}{4.715187in}}%
\pgfpathclose%
\pgfusepath{stroke,fill}%
\end{pgfscope}%
\begin{pgfscope}%
\pgfpathrectangle{\pgfqpoint{1.582361in}{0.880000in}}{\pgfqpoint{5.035278in}{6.160000in}}%
\pgfusepath{clip}%
\pgfsetbuttcap%
\pgfsetroundjoin%
\definecolor{currentfill}{rgb}{0.800000,0.200000,0.200000}%
\pgfsetfillcolor{currentfill}%
\pgfsetlinewidth{1.003750pt}%
\definecolor{currentstroke}{rgb}{0.800000,0.200000,0.200000}%
\pgfsetstrokecolor{currentstroke}%
\pgfsetdash{}{0pt}%
\pgfpathmoveto{\pgfqpoint{5.323248in}{4.755917in}}%
\pgfpathcurveto{\pgfqpoint{5.329072in}{4.755917in}}{\pgfqpoint{5.334659in}{4.758231in}}{\pgfqpoint{5.338777in}{4.762349in}}%
\pgfpathcurveto{\pgfqpoint{5.342895in}{4.766467in}}{\pgfqpoint{5.345209in}{4.772053in}}{\pgfqpoint{5.345209in}{4.777877in}}%
\pgfpathcurveto{\pgfqpoint{5.345209in}{4.783701in}}{\pgfqpoint{5.342895in}{4.789288in}}{\pgfqpoint{5.338777in}{4.793406in}}%
\pgfpathcurveto{\pgfqpoint{5.334659in}{4.797524in}}{\pgfqpoint{5.329072in}{4.799838in}}{\pgfqpoint{5.323248in}{4.799838in}}%
\pgfpathcurveto{\pgfqpoint{5.317425in}{4.799838in}}{\pgfqpoint{5.311838in}{4.797524in}}{\pgfqpoint{5.307720in}{4.793406in}}%
\pgfpathcurveto{\pgfqpoint{5.303602in}{4.789288in}}{\pgfqpoint{5.301288in}{4.783701in}}{\pgfqpoint{5.301288in}{4.777877in}}%
\pgfpathcurveto{\pgfqpoint{5.301288in}{4.772053in}}{\pgfqpoint{5.303602in}{4.766467in}}{\pgfqpoint{5.307720in}{4.762349in}}%
\pgfpathcurveto{\pgfqpoint{5.311838in}{4.758231in}}{\pgfqpoint{5.317425in}{4.755917in}}{\pgfqpoint{5.323248in}{4.755917in}}%
\pgfpathlineto{\pgfqpoint{5.323248in}{4.755917in}}%
\pgfpathclose%
\pgfusepath{stroke,fill}%
\end{pgfscope}%
\begin{pgfscope}%
\pgfpathrectangle{\pgfqpoint{1.582361in}{0.880000in}}{\pgfqpoint{5.035278in}{6.160000in}}%
\pgfusepath{clip}%
\pgfsetbuttcap%
\pgfsetroundjoin%
\definecolor{currentfill}{rgb}{0.800000,0.200000,0.200000}%
\pgfsetfillcolor{currentfill}%
\pgfsetlinewidth{1.003750pt}%
\definecolor{currentstroke}{rgb}{0.800000,0.200000,0.200000}%
\pgfsetstrokecolor{currentstroke}%
\pgfsetdash{}{0pt}%
\pgfpathmoveto{\pgfqpoint{5.340448in}{4.797211in}}%
\pgfpathcurveto{\pgfqpoint{5.346272in}{4.797211in}}{\pgfqpoint{5.351858in}{4.799524in}}{\pgfqpoint{5.355977in}{4.803643in}}%
\pgfpathcurveto{\pgfqpoint{5.360095in}{4.807761in}}{\pgfqpoint{5.362409in}{4.813347in}}{\pgfqpoint{5.362409in}{4.819171in}}%
\pgfpathcurveto{\pgfqpoint{5.362409in}{4.824995in}}{\pgfqpoint{5.360095in}{4.830581in}}{\pgfqpoint{5.355977in}{4.834699in}}%
\pgfpathcurveto{\pgfqpoint{5.351858in}{4.838817in}}{\pgfqpoint{5.346272in}{4.841131in}}{\pgfqpoint{5.340448in}{4.841131in}}%
\pgfpathcurveto{\pgfqpoint{5.334624in}{4.841131in}}{\pgfqpoint{5.329038in}{4.838817in}}{\pgfqpoint{5.324920in}{4.834699in}}%
\pgfpathcurveto{\pgfqpoint{5.320802in}{4.830581in}}{\pgfqpoint{5.318488in}{4.824995in}}{\pgfqpoint{5.318488in}{4.819171in}}%
\pgfpathcurveto{\pgfqpoint{5.318488in}{4.813347in}}{\pgfqpoint{5.320802in}{4.807761in}}{\pgfqpoint{5.324920in}{4.803643in}}%
\pgfpathcurveto{\pgfqpoint{5.329038in}{4.799524in}}{\pgfqpoint{5.334624in}{4.797211in}}{\pgfqpoint{5.340448in}{4.797211in}}%
\pgfpathlineto{\pgfqpoint{5.340448in}{4.797211in}}%
\pgfpathclose%
\pgfusepath{stroke,fill}%
\end{pgfscope}%
\begin{pgfscope}%
\pgfpathrectangle{\pgfqpoint{1.582361in}{0.880000in}}{\pgfqpoint{5.035278in}{6.160000in}}%
\pgfusepath{clip}%
\pgfsetbuttcap%
\pgfsetroundjoin%
\definecolor{currentfill}{rgb}{0.800000,0.200000,0.200000}%
\pgfsetfillcolor{currentfill}%
\pgfsetlinewidth{1.003750pt}%
\definecolor{currentstroke}{rgb}{0.800000,0.200000,0.200000}%
\pgfsetstrokecolor{currentstroke}%
\pgfsetdash{}{0pt}%
\pgfpathmoveto{\pgfqpoint{5.356336in}{4.839026in}}%
\pgfpathcurveto{\pgfqpoint{5.362160in}{4.839026in}}{\pgfqpoint{5.367746in}{4.841340in}}{\pgfqpoint{5.371864in}{4.845458in}}%
\pgfpathcurveto{\pgfqpoint{5.375982in}{4.849577in}}{\pgfqpoint{5.378296in}{4.855163in}}{\pgfqpoint{5.378296in}{4.860987in}}%
\pgfpathcurveto{\pgfqpoint{5.378296in}{4.866811in}}{\pgfqpoint{5.375982in}{4.872397in}}{\pgfqpoint{5.371864in}{4.876515in}}%
\pgfpathcurveto{\pgfqpoint{5.367746in}{4.880633in}}{\pgfqpoint{5.362160in}{4.882947in}}{\pgfqpoint{5.356336in}{4.882947in}}%
\pgfpathcurveto{\pgfqpoint{5.350512in}{4.882947in}}{\pgfqpoint{5.344926in}{4.880633in}}{\pgfqpoint{5.340808in}{4.876515in}}%
\pgfpathcurveto{\pgfqpoint{5.336690in}{4.872397in}}{\pgfqpoint{5.334376in}{4.866811in}}{\pgfqpoint{5.334376in}{4.860987in}}%
\pgfpathcurveto{\pgfqpoint{5.334376in}{4.855163in}}{\pgfqpoint{5.336690in}{4.849577in}}{\pgfqpoint{5.340808in}{4.845458in}}%
\pgfpathcurveto{\pgfqpoint{5.344926in}{4.841340in}}{\pgfqpoint{5.350512in}{4.839026in}}{\pgfqpoint{5.356336in}{4.839026in}}%
\pgfpathlineto{\pgfqpoint{5.356336in}{4.839026in}}%
\pgfpathclose%
\pgfusepath{stroke,fill}%
\end{pgfscope}%
\begin{pgfscope}%
\pgfpathrectangle{\pgfqpoint{1.582361in}{0.880000in}}{\pgfqpoint{5.035278in}{6.160000in}}%
\pgfusepath{clip}%
\pgfsetbuttcap%
\pgfsetroundjoin%
\definecolor{currentfill}{rgb}{0.800000,0.200000,0.200000}%
\pgfsetfillcolor{currentfill}%
\pgfsetlinewidth{1.003750pt}%
\definecolor{currentstroke}{rgb}{0.800000,0.200000,0.200000}%
\pgfsetstrokecolor{currentstroke}%
\pgfsetdash{}{0pt}%
\pgfpathmoveto{\pgfqpoint{5.370896in}{4.881323in}}%
\pgfpathcurveto{\pgfqpoint{5.376720in}{4.881323in}}{\pgfqpoint{5.382306in}{4.883637in}}{\pgfqpoint{5.386424in}{4.887755in}}%
\pgfpathcurveto{\pgfqpoint{5.390542in}{4.891873in}}{\pgfqpoint{5.392856in}{4.897459in}}{\pgfqpoint{5.392856in}{4.903283in}}%
\pgfpathcurveto{\pgfqpoint{5.392856in}{4.909107in}}{\pgfqpoint{5.390542in}{4.914693in}}{\pgfqpoint{5.386424in}{4.918811in}}%
\pgfpathcurveto{\pgfqpoint{5.382306in}{4.922930in}}{\pgfqpoint{5.376720in}{4.925243in}}{\pgfqpoint{5.370896in}{4.925243in}}%
\pgfpathcurveto{\pgfqpoint{5.365072in}{4.925243in}}{\pgfqpoint{5.359486in}{4.922930in}}{\pgfqpoint{5.355368in}{4.918811in}}%
\pgfpathcurveto{\pgfqpoint{5.351249in}{4.914693in}}{\pgfqpoint{5.348936in}{4.909107in}}{\pgfqpoint{5.348936in}{4.903283in}}%
\pgfpathcurveto{\pgfqpoint{5.348936in}{4.897459in}}{\pgfqpoint{5.351249in}{4.891873in}}{\pgfqpoint{5.355368in}{4.887755in}}%
\pgfpathcurveto{\pgfqpoint{5.359486in}{4.883637in}}{\pgfqpoint{5.365072in}{4.881323in}}{\pgfqpoint{5.370896in}{4.881323in}}%
\pgfpathlineto{\pgfqpoint{5.370896in}{4.881323in}}%
\pgfpathclose%
\pgfusepath{stroke,fill}%
\end{pgfscope}%
\begin{pgfscope}%
\pgfpathrectangle{\pgfqpoint{1.582361in}{0.880000in}}{\pgfqpoint{5.035278in}{6.160000in}}%
\pgfusepath{clip}%
\pgfsetbuttcap%
\pgfsetroundjoin%
\definecolor{currentfill}{rgb}{0.800000,0.200000,0.200000}%
\pgfsetfillcolor{currentfill}%
\pgfsetlinewidth{1.003750pt}%
\definecolor{currentstroke}{rgb}{0.800000,0.200000,0.200000}%
\pgfsetstrokecolor{currentstroke}%
\pgfsetdash{}{0pt}%
\pgfpathmoveto{\pgfqpoint{5.384113in}{4.924058in}}%
\pgfpathcurveto{\pgfqpoint{5.389937in}{4.924058in}}{\pgfqpoint{5.395523in}{4.926372in}}{\pgfqpoint{5.399641in}{4.930490in}}%
\pgfpathcurveto{\pgfqpoint{5.403759in}{4.934608in}}{\pgfqpoint{5.406073in}{4.940194in}}{\pgfqpoint{5.406073in}{4.946018in}}%
\pgfpathcurveto{\pgfqpoint{5.406073in}{4.951842in}}{\pgfqpoint{5.403759in}{4.957428in}}{\pgfqpoint{5.399641in}{4.961547in}}%
\pgfpathcurveto{\pgfqpoint{5.395523in}{4.965665in}}{\pgfqpoint{5.389937in}{4.967979in}}{\pgfqpoint{5.384113in}{4.967979in}}%
\pgfpathcurveto{\pgfqpoint{5.378289in}{4.967979in}}{\pgfqpoint{5.372703in}{4.965665in}}{\pgfqpoint{5.368585in}{4.961547in}}%
\pgfpathcurveto{\pgfqpoint{5.364467in}{4.957428in}}{\pgfqpoint{5.362153in}{4.951842in}}{\pgfqpoint{5.362153in}{4.946018in}}%
\pgfpathcurveto{\pgfqpoint{5.362153in}{4.940194in}}{\pgfqpoint{5.364467in}{4.934608in}}{\pgfqpoint{5.368585in}{4.930490in}}%
\pgfpathcurveto{\pgfqpoint{5.372703in}{4.926372in}}{\pgfqpoint{5.378289in}{4.924058in}}{\pgfqpoint{5.384113in}{4.924058in}}%
\pgfpathlineto{\pgfqpoint{5.384113in}{4.924058in}}%
\pgfpathclose%
\pgfusepath{stroke,fill}%
\end{pgfscope}%
\begin{pgfscope}%
\pgfpathrectangle{\pgfqpoint{1.582361in}{0.880000in}}{\pgfqpoint{5.035278in}{6.160000in}}%
\pgfusepath{clip}%
\pgfsetbuttcap%
\pgfsetroundjoin%
\definecolor{currentfill}{rgb}{0.800000,0.200000,0.200000}%
\pgfsetfillcolor{currentfill}%
\pgfsetlinewidth{1.003750pt}%
\definecolor{currentstroke}{rgb}{0.800000,0.200000,0.200000}%
\pgfsetstrokecolor{currentstroke}%
\pgfsetdash{}{0pt}%
\pgfpathmoveto{\pgfqpoint{5.395975in}{4.967189in}}%
\pgfpathcurveto{\pgfqpoint{5.401798in}{4.967189in}}{\pgfqpoint{5.407385in}{4.969503in}}{\pgfqpoint{5.411503in}{4.973621in}}%
\pgfpathcurveto{\pgfqpoint{5.415621in}{4.977739in}}{\pgfqpoint{5.417935in}{4.983325in}}{\pgfqpoint{5.417935in}{4.989149in}}%
\pgfpathcurveto{\pgfqpoint{5.417935in}{4.994973in}}{\pgfqpoint{5.415621in}{5.000559in}}{\pgfqpoint{5.411503in}{5.004678in}}%
\pgfpathcurveto{\pgfqpoint{5.407385in}{5.008796in}}{\pgfqpoint{5.401798in}{5.011110in}}{\pgfqpoint{5.395975in}{5.011110in}}%
\pgfpathcurveto{\pgfqpoint{5.390151in}{5.011110in}}{\pgfqpoint{5.384564in}{5.008796in}}{\pgfqpoint{5.380446in}{5.004678in}}%
\pgfpathcurveto{\pgfqpoint{5.376328in}{5.000559in}}{\pgfqpoint{5.374014in}{4.994973in}}{\pgfqpoint{5.374014in}{4.989149in}}%
\pgfpathcurveto{\pgfqpoint{5.374014in}{4.983325in}}{\pgfqpoint{5.376328in}{4.977739in}}{\pgfqpoint{5.380446in}{4.973621in}}%
\pgfpathcurveto{\pgfqpoint{5.384564in}{4.969503in}}{\pgfqpoint{5.390151in}{4.967189in}}{\pgfqpoint{5.395975in}{4.967189in}}%
\pgfpathlineto{\pgfqpoint{5.395975in}{4.967189in}}%
\pgfpathclose%
\pgfusepath{stroke,fill}%
\end{pgfscope}%
\begin{pgfscope}%
\pgfpathrectangle{\pgfqpoint{1.582361in}{0.880000in}}{\pgfqpoint{5.035278in}{6.160000in}}%
\pgfusepath{clip}%
\pgfsetbuttcap%
\pgfsetroundjoin%
\definecolor{currentfill}{rgb}{0.800000,0.200000,0.200000}%
\pgfsetfillcolor{currentfill}%
\pgfsetlinewidth{1.003750pt}%
\definecolor{currentstroke}{rgb}{0.800000,0.200000,0.200000}%
\pgfsetstrokecolor{currentstroke}%
\pgfsetdash{}{0pt}%
\pgfpathmoveto{\pgfqpoint{5.406469in}{5.010673in}}%
\pgfpathcurveto{\pgfqpoint{5.412293in}{5.010673in}}{\pgfqpoint{5.417879in}{5.012987in}}{\pgfqpoint{5.421997in}{5.017105in}}%
\pgfpathcurveto{\pgfqpoint{5.426115in}{5.021223in}}{\pgfqpoint{5.428429in}{5.026809in}}{\pgfqpoint{5.428429in}{5.032633in}}%
\pgfpathcurveto{\pgfqpoint{5.428429in}{5.038457in}}{\pgfqpoint{5.426115in}{5.044043in}}{\pgfqpoint{5.421997in}{5.048162in}}%
\pgfpathcurveto{\pgfqpoint{5.417879in}{5.052280in}}{\pgfqpoint{5.412293in}{5.054594in}}{\pgfqpoint{5.406469in}{5.054594in}}%
\pgfpathcurveto{\pgfqpoint{5.400645in}{5.054594in}}{\pgfqpoint{5.395058in}{5.052280in}}{\pgfqpoint{5.390940in}{5.048162in}}%
\pgfpathcurveto{\pgfqpoint{5.386822in}{5.044043in}}{\pgfqpoint{5.384508in}{5.038457in}}{\pgfqpoint{5.384508in}{5.032633in}}%
\pgfpathcurveto{\pgfqpoint{5.384508in}{5.026809in}}{\pgfqpoint{5.386822in}{5.021223in}}{\pgfqpoint{5.390940in}{5.017105in}}%
\pgfpathcurveto{\pgfqpoint{5.395058in}{5.012987in}}{\pgfqpoint{5.400645in}{5.010673in}}{\pgfqpoint{5.406469in}{5.010673in}}%
\pgfpathlineto{\pgfqpoint{5.406469in}{5.010673in}}%
\pgfpathclose%
\pgfusepath{stroke,fill}%
\end{pgfscope}%
\begin{pgfscope}%
\pgfpathrectangle{\pgfqpoint{1.582361in}{0.880000in}}{\pgfqpoint{5.035278in}{6.160000in}}%
\pgfusepath{clip}%
\pgfsetbuttcap%
\pgfsetroundjoin%
\definecolor{currentfill}{rgb}{0.800000,0.200000,0.200000}%
\pgfsetfillcolor{currentfill}%
\pgfsetlinewidth{1.003750pt}%
\definecolor{currentstroke}{rgb}{0.800000,0.200000,0.200000}%
\pgfsetstrokecolor{currentstroke}%
\pgfsetdash{}{0pt}%
\pgfpathmoveto{\pgfqpoint{5.415585in}{5.054467in}}%
\pgfpathcurveto{\pgfqpoint{5.421409in}{5.054467in}}{\pgfqpoint{5.426995in}{5.056781in}}{\pgfqpoint{5.431113in}{5.060899in}}%
\pgfpathcurveto{\pgfqpoint{5.435231in}{5.065017in}}{\pgfqpoint{5.437545in}{5.070603in}}{\pgfqpoint{5.437545in}{5.076427in}}%
\pgfpathcurveto{\pgfqpoint{5.437545in}{5.082251in}}{\pgfqpoint{5.435231in}{5.087837in}}{\pgfqpoint{5.431113in}{5.091955in}}%
\pgfpathcurveto{\pgfqpoint{5.426995in}{5.096073in}}{\pgfqpoint{5.421409in}{5.098387in}}{\pgfqpoint{5.415585in}{5.098387in}}%
\pgfpathcurveto{\pgfqpoint{5.409761in}{5.098387in}}{\pgfqpoint{5.404175in}{5.096073in}}{\pgfqpoint{5.400056in}{5.091955in}}%
\pgfpathcurveto{\pgfqpoint{5.395938in}{5.087837in}}{\pgfqpoint{5.393624in}{5.082251in}}{\pgfqpoint{5.393624in}{5.076427in}}%
\pgfpathcurveto{\pgfqpoint{5.393624in}{5.070603in}}{\pgfqpoint{5.395938in}{5.065017in}}{\pgfqpoint{5.400056in}{5.060899in}}%
\pgfpathcurveto{\pgfqpoint{5.404175in}{5.056781in}}{\pgfqpoint{5.409761in}{5.054467in}}{\pgfqpoint{5.415585in}{5.054467in}}%
\pgfpathlineto{\pgfqpoint{5.415585in}{5.054467in}}%
\pgfpathclose%
\pgfusepath{stroke,fill}%
\end{pgfscope}%
\begin{pgfscope}%
\pgfpathrectangle{\pgfqpoint{1.582361in}{0.880000in}}{\pgfqpoint{5.035278in}{6.160000in}}%
\pgfusepath{clip}%
\pgfsetbuttcap%
\pgfsetroundjoin%
\definecolor{currentfill}{rgb}{0.800000,0.200000,0.200000}%
\pgfsetfillcolor{currentfill}%
\pgfsetlinewidth{1.003750pt}%
\definecolor{currentstroke}{rgb}{0.800000,0.200000,0.200000}%
\pgfsetstrokecolor{currentstroke}%
\pgfsetdash{}{0pt}%
\pgfpathmoveto{\pgfqpoint{5.423314in}{5.098526in}}%
\pgfpathcurveto{\pgfqpoint{5.429138in}{5.098526in}}{\pgfqpoint{5.434724in}{5.100840in}}{\pgfqpoint{5.438842in}{5.104958in}}%
\pgfpathcurveto{\pgfqpoint{5.442960in}{5.109076in}}{\pgfqpoint{5.445274in}{5.114663in}}{\pgfqpoint{5.445274in}{5.120487in}}%
\pgfpathcurveto{\pgfqpoint{5.445274in}{5.126310in}}{\pgfqpoint{5.442960in}{5.131897in}}{\pgfqpoint{5.438842in}{5.136015in}}%
\pgfpathcurveto{\pgfqpoint{5.434724in}{5.140133in}}{\pgfqpoint{5.429138in}{5.142447in}}{\pgfqpoint{5.423314in}{5.142447in}}%
\pgfpathcurveto{\pgfqpoint{5.417490in}{5.142447in}}{\pgfqpoint{5.411904in}{5.140133in}}{\pgfqpoint{5.407785in}{5.136015in}}%
\pgfpathcurveto{\pgfqpoint{5.403667in}{5.131897in}}{\pgfqpoint{5.401353in}{5.126310in}}{\pgfqpoint{5.401353in}{5.120487in}}%
\pgfpathcurveto{\pgfqpoint{5.401353in}{5.114663in}}{\pgfqpoint{5.403667in}{5.109076in}}{\pgfqpoint{5.407785in}{5.104958in}}%
\pgfpathcurveto{\pgfqpoint{5.411904in}{5.100840in}}{\pgfqpoint{5.417490in}{5.098526in}}{\pgfqpoint{5.423314in}{5.098526in}}%
\pgfpathlineto{\pgfqpoint{5.423314in}{5.098526in}}%
\pgfpathclose%
\pgfusepath{stroke,fill}%
\end{pgfscope}%
\begin{pgfscope}%
\pgfpathrectangle{\pgfqpoint{1.582361in}{0.880000in}}{\pgfqpoint{5.035278in}{6.160000in}}%
\pgfusepath{clip}%
\pgfsetbuttcap%
\pgfsetroundjoin%
\definecolor{currentfill}{rgb}{0.800000,0.200000,0.200000}%
\pgfsetfillcolor{currentfill}%
\pgfsetlinewidth{1.003750pt}%
\definecolor{currentstroke}{rgb}{0.800000,0.200000,0.200000}%
\pgfsetstrokecolor{currentstroke}%
\pgfsetdash{}{0pt}%
\pgfpathmoveto{\pgfqpoint{5.429648in}{5.142808in}}%
\pgfpathcurveto{\pgfqpoint{5.435472in}{5.142808in}}{\pgfqpoint{5.441058in}{5.145122in}}{\pgfqpoint{5.445176in}{5.149240in}}%
\pgfpathcurveto{\pgfqpoint{5.449294in}{5.153358in}}{\pgfqpoint{5.451608in}{5.158944in}}{\pgfqpoint{5.451608in}{5.164768in}}%
\pgfpathcurveto{\pgfqpoint{5.451608in}{5.170592in}}{\pgfqpoint{5.449294in}{5.176178in}}{\pgfqpoint{5.445176in}{5.180296in}}%
\pgfpathcurveto{\pgfqpoint{5.441058in}{5.184415in}}{\pgfqpoint{5.435472in}{5.186728in}}{\pgfqpoint{5.429648in}{5.186728in}}%
\pgfpathcurveto{\pgfqpoint{5.423824in}{5.186728in}}{\pgfqpoint{5.418238in}{5.184415in}}{\pgfqpoint{5.414120in}{5.180296in}}%
\pgfpathcurveto{\pgfqpoint{5.410002in}{5.176178in}}{\pgfqpoint{5.407688in}{5.170592in}}{\pgfqpoint{5.407688in}{5.164768in}}%
\pgfpathcurveto{\pgfqpoint{5.407688in}{5.158944in}}{\pgfqpoint{5.410002in}{5.153358in}}{\pgfqpoint{5.414120in}{5.149240in}}%
\pgfpathcurveto{\pgfqpoint{5.418238in}{5.145122in}}{\pgfqpoint{5.423824in}{5.142808in}}{\pgfqpoint{5.429648in}{5.142808in}}%
\pgfpathlineto{\pgfqpoint{5.429648in}{5.142808in}}%
\pgfpathclose%
\pgfusepath{stroke,fill}%
\end{pgfscope}%
\begin{pgfscope}%
\pgfpathrectangle{\pgfqpoint{1.582361in}{0.880000in}}{\pgfqpoint{5.035278in}{6.160000in}}%
\pgfusepath{clip}%
\pgfsetbuttcap%
\pgfsetroundjoin%
\definecolor{currentfill}{rgb}{0.800000,0.200000,0.200000}%
\pgfsetfillcolor{currentfill}%
\pgfsetlinewidth{1.003750pt}%
\definecolor{currentstroke}{rgb}{0.800000,0.200000,0.200000}%
\pgfsetstrokecolor{currentstroke}%
\pgfsetdash{}{0pt}%
\pgfpathmoveto{\pgfqpoint{5.434581in}{5.187267in}}%
\pgfpathcurveto{\pgfqpoint{5.440405in}{5.187267in}}{\pgfqpoint{5.445991in}{5.189581in}}{\pgfqpoint{5.450110in}{5.193699in}}%
\pgfpathcurveto{\pgfqpoint{5.454228in}{5.197818in}}{\pgfqpoint{5.456542in}{5.203404in}}{\pgfqpoint{5.456542in}{5.209228in}}%
\pgfpathcurveto{\pgfqpoint{5.456542in}{5.215052in}}{\pgfqpoint{5.454228in}{5.220638in}}{\pgfqpoint{5.450110in}{5.224756in}}%
\pgfpathcurveto{\pgfqpoint{5.445991in}{5.228874in}}{\pgfqpoint{5.440405in}{5.231188in}}{\pgfqpoint{5.434581in}{5.231188in}}%
\pgfpathcurveto{\pgfqpoint{5.428757in}{5.231188in}}{\pgfqpoint{5.423171in}{5.228874in}}{\pgfqpoint{5.419053in}{5.224756in}}%
\pgfpathcurveto{\pgfqpoint{5.414935in}{5.220638in}}{\pgfqpoint{5.412621in}{5.215052in}}{\pgfqpoint{5.412621in}{5.209228in}}%
\pgfpathcurveto{\pgfqpoint{5.412621in}{5.203404in}}{\pgfqpoint{5.414935in}{5.197818in}}{\pgfqpoint{5.419053in}{5.193699in}}%
\pgfpathcurveto{\pgfqpoint{5.423171in}{5.189581in}}{\pgfqpoint{5.428757in}{5.187267in}}{\pgfqpoint{5.434581in}{5.187267in}}%
\pgfpathlineto{\pgfqpoint{5.434581in}{5.187267in}}%
\pgfpathclose%
\pgfusepath{stroke,fill}%
\end{pgfscope}%
\begin{pgfscope}%
\pgfpathrectangle{\pgfqpoint{1.582361in}{0.880000in}}{\pgfqpoint{5.035278in}{6.160000in}}%
\pgfusepath{clip}%
\pgfsetbuttcap%
\pgfsetroundjoin%
\definecolor{currentfill}{rgb}{0.800000,0.200000,0.200000}%
\pgfsetfillcolor{currentfill}%
\pgfsetlinewidth{1.003750pt}%
\definecolor{currentstroke}{rgb}{0.800000,0.200000,0.200000}%
\pgfsetstrokecolor{currentstroke}%
\pgfsetdash{}{0pt}%
\pgfpathmoveto{\pgfqpoint{5.438109in}{5.231860in}}%
\pgfpathcurveto{\pgfqpoint{5.443932in}{5.231860in}}{\pgfqpoint{5.449519in}{5.234174in}}{\pgfqpoint{5.453637in}{5.238292in}}%
\pgfpathcurveto{\pgfqpoint{5.457755in}{5.242411in}}{\pgfqpoint{5.460069in}{5.247997in}}{\pgfqpoint{5.460069in}{5.253821in}}%
\pgfpathcurveto{\pgfqpoint{5.460069in}{5.259645in}}{\pgfqpoint{5.457755in}{5.265231in}}{\pgfqpoint{5.453637in}{5.269349in}}%
\pgfpathcurveto{\pgfqpoint{5.449519in}{5.273467in}}{\pgfqpoint{5.443932in}{5.275781in}}{\pgfqpoint{5.438109in}{5.275781in}}%
\pgfpathcurveto{\pgfqpoint{5.432285in}{5.275781in}}{\pgfqpoint{5.426698in}{5.273467in}}{\pgfqpoint{5.422580in}{5.269349in}}%
\pgfpathcurveto{\pgfqpoint{5.418462in}{5.265231in}}{\pgfqpoint{5.416148in}{5.259645in}}{\pgfqpoint{5.416148in}{5.253821in}}%
\pgfpathcurveto{\pgfqpoint{5.416148in}{5.247997in}}{\pgfqpoint{5.418462in}{5.242411in}}{\pgfqpoint{5.422580in}{5.238292in}}%
\pgfpathcurveto{\pgfqpoint{5.426698in}{5.234174in}}{\pgfqpoint{5.432285in}{5.231860in}}{\pgfqpoint{5.438109in}{5.231860in}}%
\pgfpathlineto{\pgfqpoint{5.438109in}{5.231860in}}%
\pgfpathclose%
\pgfusepath{stroke,fill}%
\end{pgfscope}%
\begin{pgfscope}%
\pgfpathrectangle{\pgfqpoint{1.582361in}{0.880000in}}{\pgfqpoint{5.035278in}{6.160000in}}%
\pgfusepath{clip}%
\pgfsetbuttcap%
\pgfsetroundjoin%
\definecolor{currentfill}{rgb}{0.800000,0.200000,0.200000}%
\pgfsetfillcolor{currentfill}%
\pgfsetlinewidth{1.003750pt}%
\definecolor{currentstroke}{rgb}{0.800000,0.200000,0.200000}%
\pgfsetstrokecolor{currentstroke}%
\pgfsetdash{}{0pt}%
\pgfpathmoveto{\pgfqpoint{5.440226in}{5.276543in}}%
\pgfpathcurveto{\pgfqpoint{5.446050in}{5.276543in}}{\pgfqpoint{5.451636in}{5.278856in}}{\pgfqpoint{5.455755in}{5.282975in}}%
\pgfpathcurveto{\pgfqpoint{5.459873in}{5.287093in}}{\pgfqpoint{5.462187in}{5.292679in}}{\pgfqpoint{5.462187in}{5.298503in}}%
\pgfpathcurveto{\pgfqpoint{5.462187in}{5.304327in}}{\pgfqpoint{5.459873in}{5.309913in}}{\pgfqpoint{5.455755in}{5.314031in}}%
\pgfpathcurveto{\pgfqpoint{5.451636in}{5.318149in}}{\pgfqpoint{5.446050in}{5.320463in}}{\pgfqpoint{5.440226in}{5.320463in}}%
\pgfpathcurveto{\pgfqpoint{5.434402in}{5.320463in}}{\pgfqpoint{5.428816in}{5.318149in}}{\pgfqpoint{5.424698in}{5.314031in}}%
\pgfpathcurveto{\pgfqpoint{5.420580in}{5.309913in}}{\pgfqpoint{5.418266in}{5.304327in}}{\pgfqpoint{5.418266in}{5.298503in}}%
\pgfpathcurveto{\pgfqpoint{5.418266in}{5.292679in}}{\pgfqpoint{5.420580in}{5.287093in}}{\pgfqpoint{5.424698in}{5.282975in}}%
\pgfpathcurveto{\pgfqpoint{5.428816in}{5.278856in}}{\pgfqpoint{5.434402in}{5.276543in}}{\pgfqpoint{5.440226in}{5.276543in}}%
\pgfpathlineto{\pgfqpoint{5.440226in}{5.276543in}}%
\pgfpathclose%
\pgfusepath{stroke,fill}%
\end{pgfscope}%
\begin{pgfscope}%
\pgfpathrectangle{\pgfqpoint{1.582361in}{0.880000in}}{\pgfqpoint{5.035278in}{6.160000in}}%
\pgfusepath{clip}%
\pgfsetbuttcap%
\pgfsetroundjoin%
\definecolor{currentfill}{rgb}{0.800000,0.200000,0.200000}%
\pgfsetfillcolor{currentfill}%
\pgfsetlinewidth{1.003750pt}%
\definecolor{currentstroke}{rgb}{0.800000,0.200000,0.200000}%
\pgfsetstrokecolor{currentstroke}%
\pgfsetdash{}{0pt}%
\pgfpathmoveto{\pgfqpoint{5.440932in}{5.321269in}}%
\pgfpathcurveto{\pgfqpoint{5.446756in}{5.321269in}}{\pgfqpoint{5.452343in}{5.323583in}}{\pgfqpoint{5.456461in}{5.327701in}}%
\pgfpathcurveto{\pgfqpoint{5.460579in}{5.331820in}}{\pgfqpoint{5.462893in}{5.337406in}}{\pgfqpoint{5.462893in}{5.343230in}}%
\pgfpathcurveto{\pgfqpoint{5.462893in}{5.349054in}}{\pgfqpoint{5.460579in}{5.354640in}}{\pgfqpoint{5.456461in}{5.358758in}}%
\pgfpathcurveto{\pgfqpoint{5.452343in}{5.362876in}}{\pgfqpoint{5.446756in}{5.365190in}}{\pgfqpoint{5.440932in}{5.365190in}}%
\pgfpathcurveto{\pgfqpoint{5.435109in}{5.365190in}}{\pgfqpoint{5.429522in}{5.362876in}}{\pgfqpoint{5.425404in}{5.358758in}}%
\pgfpathcurveto{\pgfqpoint{5.421286in}{5.354640in}}{\pgfqpoint{5.418972in}{5.349054in}}{\pgfqpoint{5.418972in}{5.343230in}}%
\pgfpathcurveto{\pgfqpoint{5.418972in}{5.337406in}}{\pgfqpoint{5.421286in}{5.331820in}}{\pgfqpoint{5.425404in}{5.327701in}}%
\pgfpathcurveto{\pgfqpoint{5.429522in}{5.323583in}}{\pgfqpoint{5.435109in}{5.321269in}}{\pgfqpoint{5.440932in}{5.321269in}}%
\pgfpathlineto{\pgfqpoint{5.440932in}{5.321269in}}%
\pgfpathclose%
\pgfusepath{stroke,fill}%
\end{pgfscope}%
\begin{pgfscope}%
\pgfpathrectangle{\pgfqpoint{1.582361in}{0.880000in}}{\pgfqpoint{5.035278in}{6.160000in}}%
\pgfusepath{clip}%
\pgfsetbuttcap%
\pgfsetroundjoin%
\definecolor{currentfill}{rgb}{0.500000,0.000000,0.500000}%
\pgfsetfillcolor{currentfill}%
\pgfsetlinewidth{1.003750pt}%
\definecolor{currentstroke}{rgb}{0.500000,0.000000,0.500000}%
\pgfsetstrokecolor{currentstroke}%
\pgfsetdash{}{0pt}%
\pgfpathmoveto{\pgfqpoint{6.081070in}{4.526286in}}%
\pgfpathcurveto{\pgfqpoint{6.086894in}{4.526286in}}{\pgfqpoint{6.092480in}{4.528600in}}{\pgfqpoint{6.096598in}{4.532718in}}%
\pgfpathcurveto{\pgfqpoint{6.100716in}{4.536836in}}{\pgfqpoint{6.103030in}{4.542422in}}{\pgfqpoint{6.103030in}{4.548246in}}%
\pgfpathcurveto{\pgfqpoint{6.103030in}{4.554070in}}{\pgfqpoint{6.100716in}{4.559656in}}{\pgfqpoint{6.096598in}{4.563774in}}%
\pgfpathcurveto{\pgfqpoint{6.092480in}{4.567893in}}{\pgfqpoint{6.086894in}{4.570206in}}{\pgfqpoint{6.081070in}{4.570206in}}%
\pgfpathcurveto{\pgfqpoint{6.075246in}{4.570206in}}{\pgfqpoint{6.069660in}{4.567893in}}{\pgfqpoint{6.065541in}{4.563774in}}%
\pgfpathcurveto{\pgfqpoint{6.061423in}{4.559656in}}{\pgfqpoint{6.059109in}{4.554070in}}{\pgfqpoint{6.059109in}{4.548246in}}%
\pgfpathcurveto{\pgfqpoint{6.059109in}{4.542422in}}{\pgfqpoint{6.061423in}{4.536836in}}{\pgfqpoint{6.065541in}{4.532718in}}%
\pgfpathcurveto{\pgfqpoint{6.069660in}{4.528600in}}{\pgfqpoint{6.075246in}{4.526286in}}{\pgfqpoint{6.081070in}{4.526286in}}%
\pgfpathlineto{\pgfqpoint{6.081070in}{4.526286in}}%
\pgfpathclose%
\pgfusepath{stroke,fill}%
\end{pgfscope}%
\begin{pgfscope}%
\pgfpathrectangle{\pgfqpoint{1.582361in}{0.880000in}}{\pgfqpoint{5.035278in}{6.160000in}}%
\pgfusepath{clip}%
\pgfsetbuttcap%
\pgfsetroundjoin%
\definecolor{currentfill}{rgb}{0.500000,0.000000,0.500000}%
\pgfsetfillcolor{currentfill}%
\pgfsetlinewidth{1.003750pt}%
\definecolor{currentstroke}{rgb}{0.500000,0.000000,0.500000}%
\pgfsetstrokecolor{currentstroke}%
\pgfsetdash{}{0pt}%
\pgfpathmoveto{\pgfqpoint{3.698171in}{4.419939in}}%
\pgfpathcurveto{\pgfqpoint{3.703994in}{4.419939in}}{\pgfqpoint{3.709581in}{4.422253in}}{\pgfqpoint{3.713699in}{4.426371in}}%
\pgfpathcurveto{\pgfqpoint{3.717817in}{4.430489in}}{\pgfqpoint{3.720131in}{4.436076in}}{\pgfqpoint{3.720131in}{4.441900in}}%
\pgfpathcurveto{\pgfqpoint{3.720131in}{4.447723in}}{\pgfqpoint{3.717817in}{4.453310in}}{\pgfqpoint{3.713699in}{4.457428in}}%
\pgfpathcurveto{\pgfqpoint{3.709581in}{4.461546in}}{\pgfqpoint{3.703994in}{4.463860in}}{\pgfqpoint{3.698171in}{4.463860in}}%
\pgfpathcurveto{\pgfqpoint{3.692347in}{4.463860in}}{\pgfqpoint{3.686760in}{4.461546in}}{\pgfqpoint{3.682642in}{4.457428in}}%
\pgfpathcurveto{\pgfqpoint{3.678524in}{4.453310in}}{\pgfqpoint{3.676210in}{4.447723in}}{\pgfqpoint{3.676210in}{4.441900in}}%
\pgfpathcurveto{\pgfqpoint{3.676210in}{4.436076in}}{\pgfqpoint{3.678524in}{4.430489in}}{\pgfqpoint{3.682642in}{4.426371in}}%
\pgfpathcurveto{\pgfqpoint{3.686760in}{4.422253in}}{\pgfqpoint{3.692347in}{4.419939in}}{\pgfqpoint{3.698171in}{4.419939in}}%
\pgfpathlineto{\pgfqpoint{3.698171in}{4.419939in}}%
\pgfpathclose%
\pgfusepath{stroke,fill}%
\end{pgfscope}%
\begin{pgfscope}%
\pgfpathrectangle{\pgfqpoint{1.582361in}{0.880000in}}{\pgfqpoint{5.035278in}{6.160000in}}%
\pgfusepath{clip}%
\pgfsetbuttcap%
\pgfsetroundjoin%
\definecolor{currentfill}{rgb}{0.500000,0.000000,0.500000}%
\pgfsetfillcolor{currentfill}%
\pgfsetlinewidth{1.003750pt}%
\definecolor{currentstroke}{rgb}{0.500000,0.000000,0.500000}%
\pgfsetstrokecolor{currentstroke}%
\pgfsetdash{}{0pt}%
\pgfpathmoveto{\pgfqpoint{3.952140in}{3.042092in}}%
\pgfpathcurveto{\pgfqpoint{3.957964in}{3.042092in}}{\pgfqpoint{3.963550in}{3.044405in}}{\pgfqpoint{3.967668in}{3.048524in}}%
\pgfpathcurveto{\pgfqpoint{3.971786in}{3.052642in}}{\pgfqpoint{3.974100in}{3.058228in}}{\pgfqpoint{3.974100in}{3.064052in}}%
\pgfpathcurveto{\pgfqpoint{3.974100in}{3.069876in}}{\pgfqpoint{3.971786in}{3.075462in}}{\pgfqpoint{3.967668in}{3.079580in}}%
\pgfpathcurveto{\pgfqpoint{3.963550in}{3.083698in}}{\pgfqpoint{3.957964in}{3.086012in}}{\pgfqpoint{3.952140in}{3.086012in}}%
\pgfpathcurveto{\pgfqpoint{3.946316in}{3.086012in}}{\pgfqpoint{3.940730in}{3.083698in}}{\pgfqpoint{3.936611in}{3.079580in}}%
\pgfpathcurveto{\pgfqpoint{3.932493in}{3.075462in}}{\pgfqpoint{3.930179in}{3.069876in}}{\pgfqpoint{3.930179in}{3.064052in}}%
\pgfpathcurveto{\pgfqpoint{3.930179in}{3.058228in}}{\pgfqpoint{3.932493in}{3.052642in}}{\pgfqpoint{3.936611in}{3.048524in}}%
\pgfpathcurveto{\pgfqpoint{3.940730in}{3.044405in}}{\pgfqpoint{3.946316in}{3.042092in}}{\pgfqpoint{3.952140in}{3.042092in}}%
\pgfpathlineto{\pgfqpoint{3.952140in}{3.042092in}}%
\pgfpathclose%
\pgfusepath{stroke,fill}%
\end{pgfscope}%
\begin{pgfscope}%
\pgfpathrectangle{\pgfqpoint{1.582361in}{0.880000in}}{\pgfqpoint{5.035278in}{6.160000in}}%
\pgfusepath{clip}%
\pgfsetbuttcap%
\pgfsetroundjoin%
\definecolor{currentfill}{rgb}{0.500000,0.000000,0.500000}%
\pgfsetfillcolor{currentfill}%
\pgfsetlinewidth{1.003750pt}%
\definecolor{currentstroke}{rgb}{0.500000,0.000000,0.500000}%
\pgfsetstrokecolor{currentstroke}%
\pgfsetdash{}{0pt}%
\pgfpathmoveto{\pgfqpoint{2.103663in}{4.973454in}}%
\pgfpathcurveto{\pgfqpoint{2.109486in}{4.973454in}}{\pgfqpoint{2.115073in}{4.975768in}}{\pgfqpoint{2.119191in}{4.979886in}}%
\pgfpathcurveto{\pgfqpoint{2.123309in}{4.984004in}}{\pgfqpoint{2.125623in}{4.989590in}}{\pgfqpoint{2.125623in}{4.995414in}}%
\pgfpathcurveto{\pgfqpoint{2.125623in}{5.001238in}}{\pgfqpoint{2.123309in}{5.006824in}}{\pgfqpoint{2.119191in}{5.010943in}}%
\pgfpathcurveto{\pgfqpoint{2.115073in}{5.015061in}}{\pgfqpoint{2.109486in}{5.017375in}}{\pgfqpoint{2.103663in}{5.017375in}}%
\pgfpathcurveto{\pgfqpoint{2.097839in}{5.017375in}}{\pgfqpoint{2.092252in}{5.015061in}}{\pgfqpoint{2.088134in}{5.010943in}}%
\pgfpathcurveto{\pgfqpoint{2.084016in}{5.006824in}}{\pgfqpoint{2.081702in}{5.001238in}}{\pgfqpoint{2.081702in}{4.995414in}}%
\pgfpathcurveto{\pgfqpoint{2.081702in}{4.989590in}}{\pgfqpoint{2.084016in}{4.984004in}}{\pgfqpoint{2.088134in}{4.979886in}}%
\pgfpathcurveto{\pgfqpoint{2.092252in}{4.975768in}}{\pgfqpoint{2.097839in}{4.973454in}}{\pgfqpoint{2.103663in}{4.973454in}}%
\pgfpathlineto{\pgfqpoint{2.103663in}{4.973454in}}%
\pgfpathclose%
\pgfusepath{stroke,fill}%
\end{pgfscope}%
\begin{pgfscope}%
\pgfpathrectangle{\pgfqpoint{1.582361in}{0.880000in}}{\pgfqpoint{5.035278in}{6.160000in}}%
\pgfusepath{clip}%
\pgfsetbuttcap%
\pgfsetroundjoin%
\definecolor{currentfill}{rgb}{0.500000,0.000000,0.500000}%
\pgfsetfillcolor{currentfill}%
\pgfsetlinewidth{1.003750pt}%
\definecolor{currentstroke}{rgb}{0.500000,0.000000,0.500000}%
\pgfsetstrokecolor{currentstroke}%
\pgfsetdash{}{0pt}%
\pgfpathmoveto{\pgfqpoint{5.168560in}{2.954878in}}%
\pgfpathcurveto{\pgfqpoint{5.174384in}{2.954878in}}{\pgfqpoint{5.179970in}{2.957192in}}{\pgfqpoint{5.184088in}{2.961310in}}%
\pgfpathcurveto{\pgfqpoint{5.188206in}{2.965428in}}{\pgfqpoint{5.190520in}{2.971014in}}{\pgfqpoint{5.190520in}{2.976838in}}%
\pgfpathcurveto{\pgfqpoint{5.190520in}{2.982662in}}{\pgfqpoint{5.188206in}{2.988248in}}{\pgfqpoint{5.184088in}{2.992367in}}%
\pgfpathcurveto{\pgfqpoint{5.179970in}{2.996485in}}{\pgfqpoint{5.174384in}{2.998799in}}{\pgfqpoint{5.168560in}{2.998799in}}%
\pgfpathcurveto{\pgfqpoint{5.162736in}{2.998799in}}{\pgfqpoint{5.157150in}{2.996485in}}{\pgfqpoint{5.153032in}{2.992367in}}%
\pgfpathcurveto{\pgfqpoint{5.148914in}{2.988248in}}{\pgfqpoint{5.146600in}{2.982662in}}{\pgfqpoint{5.146600in}{2.976838in}}%
\pgfpathcurveto{\pgfqpoint{5.146600in}{2.971014in}}{\pgfqpoint{5.148914in}{2.965428in}}{\pgfqpoint{5.153032in}{2.961310in}}%
\pgfpathcurveto{\pgfqpoint{5.157150in}{2.957192in}}{\pgfqpoint{5.162736in}{2.954878in}}{\pgfqpoint{5.168560in}{2.954878in}}%
\pgfpathlineto{\pgfqpoint{5.168560in}{2.954878in}}%
\pgfpathclose%
\pgfusepath{stroke,fill}%
\end{pgfscope}%
\begin{pgfscope}%
\pgfpathrectangle{\pgfqpoint{1.582361in}{0.880000in}}{\pgfqpoint{5.035278in}{6.160000in}}%
\pgfusepath{clip}%
\pgfsetbuttcap%
\pgfsetroundjoin%
\definecolor{currentfill}{rgb}{0.500000,0.000000,0.500000}%
\pgfsetfillcolor{currentfill}%
\pgfsetlinewidth{1.003750pt}%
\definecolor{currentstroke}{rgb}{0.500000,0.000000,0.500000}%
\pgfsetstrokecolor{currentstroke}%
\pgfsetdash{}{0pt}%
\pgfpathmoveto{\pgfqpoint{6.260319in}{1.765404in}}%
\pgfpathcurveto{\pgfqpoint{6.266143in}{1.765404in}}{\pgfqpoint{6.271729in}{1.767718in}}{\pgfqpoint{6.275847in}{1.771836in}}%
\pgfpathcurveto{\pgfqpoint{6.279965in}{1.775954in}}{\pgfqpoint{6.282279in}{1.781541in}}{\pgfqpoint{6.282279in}{1.787364in}}%
\pgfpathcurveto{\pgfqpoint{6.282279in}{1.793188in}}{\pgfqpoint{6.279965in}{1.798775in}}{\pgfqpoint{6.275847in}{1.802893in}}%
\pgfpathcurveto{\pgfqpoint{6.271729in}{1.807011in}}{\pgfqpoint{6.266143in}{1.809325in}}{\pgfqpoint{6.260319in}{1.809325in}}%
\pgfpathcurveto{\pgfqpoint{6.254495in}{1.809325in}}{\pgfqpoint{6.248909in}{1.807011in}}{\pgfqpoint{6.244791in}{1.802893in}}%
\pgfpathcurveto{\pgfqpoint{6.240673in}{1.798775in}}{\pgfqpoint{6.238359in}{1.793188in}}{\pgfqpoint{6.238359in}{1.787364in}}%
\pgfpathcurveto{\pgfqpoint{6.238359in}{1.781541in}}{\pgfqpoint{6.240673in}{1.775954in}}{\pgfqpoint{6.244791in}{1.771836in}}%
\pgfpathcurveto{\pgfqpoint{6.248909in}{1.767718in}}{\pgfqpoint{6.254495in}{1.765404in}}{\pgfqpoint{6.260319in}{1.765404in}}%
\pgfpathlineto{\pgfqpoint{6.260319in}{1.765404in}}%
\pgfpathclose%
\pgfusepath{stroke,fill}%
\end{pgfscope}%
\begin{pgfscope}%
\pgfpathrectangle{\pgfqpoint{1.582361in}{0.880000in}}{\pgfqpoint{5.035278in}{6.160000in}}%
\pgfusepath{clip}%
\pgfsetbuttcap%
\pgfsetroundjoin%
\definecolor{currentfill}{rgb}{0.500000,0.000000,0.500000}%
\pgfsetfillcolor{currentfill}%
\pgfsetlinewidth{1.003750pt}%
\definecolor{currentstroke}{rgb}{0.500000,0.000000,0.500000}%
\pgfsetstrokecolor{currentstroke}%
\pgfsetdash{}{0pt}%
\pgfpathmoveto{\pgfqpoint{5.681199in}{2.214038in}}%
\pgfpathcurveto{\pgfqpoint{5.687023in}{2.214038in}}{\pgfqpoint{5.692610in}{2.216352in}}{\pgfqpoint{5.696728in}{2.220470in}}%
\pgfpathcurveto{\pgfqpoint{5.700846in}{2.224588in}}{\pgfqpoint{5.703160in}{2.230174in}}{\pgfqpoint{5.703160in}{2.235998in}}%
\pgfpathcurveto{\pgfqpoint{5.703160in}{2.241822in}}{\pgfqpoint{5.700846in}{2.247408in}}{\pgfqpoint{5.696728in}{2.251526in}}%
\pgfpathcurveto{\pgfqpoint{5.692610in}{2.255645in}}{\pgfqpoint{5.687023in}{2.257958in}}{\pgfqpoint{5.681199in}{2.257958in}}%
\pgfpathcurveto{\pgfqpoint{5.675376in}{2.257958in}}{\pgfqpoint{5.669789in}{2.255645in}}{\pgfqpoint{5.665671in}{2.251526in}}%
\pgfpathcurveto{\pgfqpoint{5.661553in}{2.247408in}}{\pgfqpoint{5.659239in}{2.241822in}}{\pgfqpoint{5.659239in}{2.235998in}}%
\pgfpathcurveto{\pgfqpoint{5.659239in}{2.230174in}}{\pgfqpoint{5.661553in}{2.224588in}}{\pgfqpoint{5.665671in}{2.220470in}}%
\pgfpathcurveto{\pgfqpoint{5.669789in}{2.216352in}}{\pgfqpoint{5.675376in}{2.214038in}}{\pgfqpoint{5.681199in}{2.214038in}}%
\pgfpathlineto{\pgfqpoint{5.681199in}{2.214038in}}%
\pgfpathclose%
\pgfusepath{stroke,fill}%
\end{pgfscope}%
\begin{pgfscope}%
\pgfpathrectangle{\pgfqpoint{1.582361in}{0.880000in}}{\pgfqpoint{5.035278in}{6.160000in}}%
\pgfusepath{clip}%
\pgfsetbuttcap%
\pgfsetroundjoin%
\definecolor{currentfill}{rgb}{0.500000,0.000000,0.500000}%
\pgfsetfillcolor{currentfill}%
\pgfsetlinewidth{1.003750pt}%
\definecolor{currentstroke}{rgb}{0.500000,0.000000,0.500000}%
\pgfsetstrokecolor{currentstroke}%
\pgfsetdash{}{0pt}%
\pgfpathmoveto{\pgfqpoint{4.984144in}{2.182742in}}%
\pgfpathcurveto{\pgfqpoint{4.989968in}{2.182742in}}{\pgfqpoint{4.995554in}{2.185056in}}{\pgfqpoint{4.999672in}{2.189174in}}%
\pgfpathcurveto{\pgfqpoint{5.003791in}{2.193292in}}{\pgfqpoint{5.006104in}{2.198878in}}{\pgfqpoint{5.006104in}{2.204702in}}%
\pgfpathcurveto{\pgfqpoint{5.006104in}{2.210526in}}{\pgfqpoint{5.003791in}{2.216112in}}{\pgfqpoint{4.999672in}{2.220230in}}%
\pgfpathcurveto{\pgfqpoint{4.995554in}{2.224348in}}{\pgfqpoint{4.989968in}{2.226662in}}{\pgfqpoint{4.984144in}{2.226662in}}%
\pgfpathcurveto{\pgfqpoint{4.978320in}{2.226662in}}{\pgfqpoint{4.972734in}{2.224348in}}{\pgfqpoint{4.968616in}{2.220230in}}%
\pgfpathcurveto{\pgfqpoint{4.964498in}{2.216112in}}{\pgfqpoint{4.962184in}{2.210526in}}{\pgfqpoint{4.962184in}{2.204702in}}%
\pgfpathcurveto{\pgfqpoint{4.962184in}{2.198878in}}{\pgfqpoint{4.964498in}{2.193292in}}{\pgfqpoint{4.968616in}{2.189174in}}%
\pgfpathcurveto{\pgfqpoint{4.972734in}{2.185056in}}{\pgfqpoint{4.978320in}{2.182742in}}{\pgfqpoint{4.984144in}{2.182742in}}%
\pgfpathlineto{\pgfqpoint{4.984144in}{2.182742in}}%
\pgfpathclose%
\pgfusepath{stroke,fill}%
\end{pgfscope}%
\begin{pgfscope}%
\pgfpathrectangle{\pgfqpoint{1.582361in}{0.880000in}}{\pgfqpoint{5.035278in}{6.160000in}}%
\pgfusepath{clip}%
\pgfsetbuttcap%
\pgfsetroundjoin%
\definecolor{currentfill}{rgb}{0.800000,0.200000,0.200000}%
\pgfsetfillcolor{currentfill}%
\pgfsetlinewidth{1.003750pt}%
\definecolor{currentstroke}{rgb}{0.800000,0.200000,0.200000}%
\pgfsetstrokecolor{currentstroke}%
\pgfsetdash{}{0pt}%
\pgfpathmoveto{\pgfqpoint{5.767779in}{4.802511in}}%
\pgfpathcurveto{\pgfqpoint{5.773603in}{4.802511in}}{\pgfqpoint{5.779189in}{4.804824in}}{\pgfqpoint{5.783307in}{4.808943in}}%
\pgfpathcurveto{\pgfqpoint{5.787425in}{4.813061in}}{\pgfqpoint{5.789739in}{4.818647in}}{\pgfqpoint{5.789739in}{4.824471in}}%
\pgfpathcurveto{\pgfqpoint{5.789739in}{4.830295in}}{\pgfqpoint{5.787425in}{4.835881in}}{\pgfqpoint{5.783307in}{4.839999in}}%
\pgfpathcurveto{\pgfqpoint{5.779189in}{4.844117in}}{\pgfqpoint{5.773603in}{4.846431in}}{\pgfqpoint{5.767779in}{4.846431in}}%
\pgfpathcurveto{\pgfqpoint{5.761955in}{4.846431in}}{\pgfqpoint{5.756368in}{4.844117in}}{\pgfqpoint{5.752250in}{4.839999in}}%
\pgfpathcurveto{\pgfqpoint{5.748132in}{4.835881in}}{\pgfqpoint{5.745818in}{4.830295in}}{\pgfqpoint{5.745818in}{4.824471in}}%
\pgfpathcurveto{\pgfqpoint{5.745818in}{4.818647in}}{\pgfqpoint{5.748132in}{4.813061in}}{\pgfqpoint{5.752250in}{4.808943in}}%
\pgfpathcurveto{\pgfqpoint{5.756368in}{4.804824in}}{\pgfqpoint{5.761955in}{4.802511in}}{\pgfqpoint{5.767779in}{4.802511in}}%
\pgfpathlineto{\pgfqpoint{5.767779in}{4.802511in}}%
\pgfpathclose%
\pgfusepath{stroke,fill}%
\end{pgfscope}%
\begin{pgfscope}%
\pgfpathrectangle{\pgfqpoint{1.582361in}{0.880000in}}{\pgfqpoint{5.035278in}{6.160000in}}%
\pgfusepath{clip}%
\pgfsetbuttcap%
\pgfsetroundjoin%
\definecolor{currentfill}{rgb}{0.500000,0.000000,0.500000}%
\pgfsetfillcolor{currentfill}%
\pgfsetlinewidth{1.003750pt}%
\definecolor{currentstroke}{rgb}{0.500000,0.000000,0.500000}%
\pgfsetstrokecolor{currentstroke}%
\pgfsetdash{}{0pt}%
\pgfpathmoveto{\pgfqpoint{3.310056in}{1.182383in}}%
\pgfpathcurveto{\pgfqpoint{3.315880in}{1.182383in}}{\pgfqpoint{3.321466in}{1.184697in}}{\pgfqpoint{3.325585in}{1.188815in}}%
\pgfpathcurveto{\pgfqpoint{3.329703in}{1.192933in}}{\pgfqpoint{3.332017in}{1.198520in}}{\pgfqpoint{3.332017in}{1.204343in}}%
\pgfpathcurveto{\pgfqpoint{3.332017in}{1.210167in}}{\pgfqpoint{3.329703in}{1.215754in}}{\pgfqpoint{3.325585in}{1.219872in}}%
\pgfpathcurveto{\pgfqpoint{3.321466in}{1.223990in}}{\pgfqpoint{3.315880in}{1.226304in}}{\pgfqpoint{3.310056in}{1.226304in}}%
\pgfpathcurveto{\pgfqpoint{3.304232in}{1.226304in}}{\pgfqpoint{3.298646in}{1.223990in}}{\pgfqpoint{3.294528in}{1.219872in}}%
\pgfpathcurveto{\pgfqpoint{3.290410in}{1.215754in}}{\pgfqpoint{3.288096in}{1.210167in}}{\pgfqpoint{3.288096in}{1.204343in}}%
\pgfpathcurveto{\pgfqpoint{3.288096in}{1.198520in}}{\pgfqpoint{3.290410in}{1.192933in}}{\pgfqpoint{3.294528in}{1.188815in}}%
\pgfpathcurveto{\pgfqpoint{3.298646in}{1.184697in}}{\pgfqpoint{3.304232in}{1.182383in}}{\pgfqpoint{3.310056in}{1.182383in}}%
\pgfpathlineto{\pgfqpoint{3.310056in}{1.182383in}}%
\pgfpathclose%
\pgfusepath{stroke,fill}%
\end{pgfscope}%
\begin{pgfscope}%
\pgfpathrectangle{\pgfqpoint{1.582361in}{0.880000in}}{\pgfqpoint{5.035278in}{6.160000in}}%
\pgfusepath{clip}%
\pgfsetbuttcap%
\pgfsetroundjoin%
\definecolor{currentfill}{rgb}{0.500000,0.000000,0.500000}%
\pgfsetfillcolor{currentfill}%
\pgfsetlinewidth{1.003750pt}%
\definecolor{currentstroke}{rgb}{0.500000,0.000000,0.500000}%
\pgfsetstrokecolor{currentstroke}%
\pgfsetdash{}{0pt}%
\pgfpathmoveto{\pgfqpoint{4.289396in}{4.660331in}}%
\pgfpathcurveto{\pgfqpoint{4.295220in}{4.660331in}}{\pgfqpoint{4.300806in}{4.662645in}}{\pgfqpoint{4.304924in}{4.666763in}}%
\pgfpathcurveto{\pgfqpoint{4.309042in}{4.670881in}}{\pgfqpoint{4.311356in}{4.676467in}}{\pgfqpoint{4.311356in}{4.682291in}}%
\pgfpathcurveto{\pgfqpoint{4.311356in}{4.688115in}}{\pgfqpoint{4.309042in}{4.693701in}}{\pgfqpoint{4.304924in}{4.697819in}}%
\pgfpathcurveto{\pgfqpoint{4.300806in}{4.701937in}}{\pgfqpoint{4.295220in}{4.704251in}}{\pgfqpoint{4.289396in}{4.704251in}}%
\pgfpathcurveto{\pgfqpoint{4.283572in}{4.704251in}}{\pgfqpoint{4.277986in}{4.701937in}}{\pgfqpoint{4.273868in}{4.697819in}}%
\pgfpathcurveto{\pgfqpoint{4.269749in}{4.693701in}}{\pgfqpoint{4.267436in}{4.688115in}}{\pgfqpoint{4.267436in}{4.682291in}}%
\pgfpathcurveto{\pgfqpoint{4.267436in}{4.676467in}}{\pgfqpoint{4.269749in}{4.670881in}}{\pgfqpoint{4.273868in}{4.666763in}}%
\pgfpathcurveto{\pgfqpoint{4.277986in}{4.662645in}}{\pgfqpoint{4.283572in}{4.660331in}}{\pgfqpoint{4.289396in}{4.660331in}}%
\pgfpathlineto{\pgfqpoint{4.289396in}{4.660331in}}%
\pgfpathclose%
\pgfusepath{stroke,fill}%
\end{pgfscope}%
\begin{pgfscope}%
\pgfpathrectangle{\pgfqpoint{1.582361in}{0.880000in}}{\pgfqpoint{5.035278in}{6.160000in}}%
\pgfusepath{clip}%
\pgfsetbuttcap%
\pgfsetroundjoin%
\definecolor{currentfill}{rgb}{0.500000,0.000000,0.500000}%
\pgfsetfillcolor{currentfill}%
\pgfsetlinewidth{1.003750pt}%
\definecolor{currentstroke}{rgb}{0.500000,0.000000,0.500000}%
\pgfsetstrokecolor{currentstroke}%
\pgfsetdash{}{0pt}%
\pgfpathmoveto{\pgfqpoint{2.276979in}{4.637315in}}%
\pgfpathcurveto{\pgfqpoint{2.282803in}{4.637315in}}{\pgfqpoint{2.288390in}{4.639629in}}{\pgfqpoint{2.292508in}{4.643747in}}%
\pgfpathcurveto{\pgfqpoint{2.296626in}{4.647865in}}{\pgfqpoint{2.298940in}{4.653452in}}{\pgfqpoint{2.298940in}{4.659276in}}%
\pgfpathcurveto{\pgfqpoint{2.298940in}{4.665100in}}{\pgfqpoint{2.296626in}{4.670686in}}{\pgfqpoint{2.292508in}{4.674804in}}%
\pgfpathcurveto{\pgfqpoint{2.288390in}{4.678922in}}{\pgfqpoint{2.282803in}{4.681236in}}{\pgfqpoint{2.276979in}{4.681236in}}%
\pgfpathcurveto{\pgfqpoint{2.271156in}{4.681236in}}{\pgfqpoint{2.265569in}{4.678922in}}{\pgfqpoint{2.261451in}{4.674804in}}%
\pgfpathcurveto{\pgfqpoint{2.257333in}{4.670686in}}{\pgfqpoint{2.255019in}{4.665100in}}{\pgfqpoint{2.255019in}{4.659276in}}%
\pgfpathcurveto{\pgfqpoint{2.255019in}{4.653452in}}{\pgfqpoint{2.257333in}{4.647865in}}{\pgfqpoint{2.261451in}{4.643747in}}%
\pgfpathcurveto{\pgfqpoint{2.265569in}{4.639629in}}{\pgfqpoint{2.271156in}{4.637315in}}{\pgfqpoint{2.276979in}{4.637315in}}%
\pgfpathlineto{\pgfqpoint{2.276979in}{4.637315in}}%
\pgfpathclose%
\pgfusepath{stroke,fill}%
\end{pgfscope}%
\begin{pgfscope}%
\pgfpathrectangle{\pgfqpoint{1.582361in}{0.880000in}}{\pgfqpoint{5.035278in}{6.160000in}}%
\pgfusepath{clip}%
\pgfsetbuttcap%
\pgfsetroundjoin%
\definecolor{currentfill}{rgb}{0.500000,0.000000,0.500000}%
\pgfsetfillcolor{currentfill}%
\pgfsetlinewidth{1.003750pt}%
\definecolor{currentstroke}{rgb}{0.500000,0.000000,0.500000}%
\pgfsetstrokecolor{currentstroke}%
\pgfsetdash{}{0pt}%
\pgfpathmoveto{\pgfqpoint{3.528119in}{5.380918in}}%
\pgfpathcurveto{\pgfqpoint{3.533943in}{5.380918in}}{\pgfqpoint{3.539529in}{5.383232in}}{\pgfqpoint{3.543647in}{5.387350in}}%
\pgfpathcurveto{\pgfqpoint{3.547766in}{5.391468in}}{\pgfqpoint{3.550080in}{5.397054in}}{\pgfqpoint{3.550080in}{5.402878in}}%
\pgfpathcurveto{\pgfqpoint{3.550080in}{5.408702in}}{\pgfqpoint{3.547766in}{5.414288in}}{\pgfqpoint{3.543647in}{5.418406in}}%
\pgfpathcurveto{\pgfqpoint{3.539529in}{5.422524in}}{\pgfqpoint{3.533943in}{5.424838in}}{\pgfqpoint{3.528119in}{5.424838in}}%
\pgfpathcurveto{\pgfqpoint{3.522295in}{5.424838in}}{\pgfqpoint{3.516709in}{5.422524in}}{\pgfqpoint{3.512591in}{5.418406in}}%
\pgfpathcurveto{\pgfqpoint{3.508473in}{5.414288in}}{\pgfqpoint{3.506159in}{5.408702in}}{\pgfqpoint{3.506159in}{5.402878in}}%
\pgfpathcurveto{\pgfqpoint{3.506159in}{5.397054in}}{\pgfqpoint{3.508473in}{5.391468in}}{\pgfqpoint{3.512591in}{5.387350in}}%
\pgfpathcurveto{\pgfqpoint{3.516709in}{5.383232in}}{\pgfqpoint{3.522295in}{5.380918in}}{\pgfqpoint{3.528119in}{5.380918in}}%
\pgfpathlineto{\pgfqpoint{3.528119in}{5.380918in}}%
\pgfpathclose%
\pgfusepath{stroke,fill}%
\end{pgfscope}%
\begin{pgfscope}%
\pgfpathrectangle{\pgfqpoint{1.582361in}{0.880000in}}{\pgfqpoint{5.035278in}{6.160000in}}%
\pgfusepath{clip}%
\pgfsetbuttcap%
\pgfsetroundjoin%
\definecolor{currentfill}{rgb}{0.800000,0.200000,0.200000}%
\pgfsetfillcolor{currentfill}%
\pgfsetlinewidth{1.003750pt}%
\definecolor{currentstroke}{rgb}{0.800000,0.200000,0.200000}%
\pgfsetstrokecolor{currentstroke}%
\pgfsetdash{}{0pt}%
\pgfpathmoveto{\pgfqpoint{2.887514in}{5.718875in}}%
\pgfpathcurveto{\pgfqpoint{2.893338in}{5.718875in}}{\pgfqpoint{2.898924in}{5.721188in}}{\pgfqpoint{2.903042in}{5.725307in}}%
\pgfpathcurveto{\pgfqpoint{2.907160in}{5.729425in}}{\pgfqpoint{2.909474in}{5.735011in}}{\pgfqpoint{2.909474in}{5.740835in}}%
\pgfpathcurveto{\pgfqpoint{2.909474in}{5.746659in}}{\pgfqpoint{2.907160in}{5.752245in}}{\pgfqpoint{2.903042in}{5.756363in}}%
\pgfpathcurveto{\pgfqpoint{2.898924in}{5.760481in}}{\pgfqpoint{2.893338in}{5.762795in}}{\pgfqpoint{2.887514in}{5.762795in}}%
\pgfpathcurveto{\pgfqpoint{2.881690in}{5.762795in}}{\pgfqpoint{2.876104in}{5.760481in}}{\pgfqpoint{2.871986in}{5.756363in}}%
\pgfpathcurveto{\pgfqpoint{2.867867in}{5.752245in}}{\pgfqpoint{2.865554in}{5.746659in}}{\pgfqpoint{2.865554in}{5.740835in}}%
\pgfpathcurveto{\pgfqpoint{2.865554in}{5.735011in}}{\pgfqpoint{2.867867in}{5.729425in}}{\pgfqpoint{2.871986in}{5.725307in}}%
\pgfpathcurveto{\pgfqpoint{2.876104in}{5.721188in}}{\pgfqpoint{2.881690in}{5.718875in}}{\pgfqpoint{2.887514in}{5.718875in}}%
\pgfpathlineto{\pgfqpoint{2.887514in}{5.718875in}}%
\pgfpathclose%
\pgfusepath{stroke,fill}%
\end{pgfscope}%
\begin{pgfscope}%
\pgfpathrectangle{\pgfqpoint{1.582361in}{0.880000in}}{\pgfqpoint{5.035278in}{6.160000in}}%
\pgfusepath{clip}%
\pgfsetbuttcap%
\pgfsetroundjoin%
\definecolor{currentfill}{rgb}{0.500000,0.000000,0.500000}%
\pgfsetfillcolor{currentfill}%
\pgfsetlinewidth{1.003750pt}%
\definecolor{currentstroke}{rgb}{0.500000,0.000000,0.500000}%
\pgfsetstrokecolor{currentstroke}%
\pgfsetdash{}{0pt}%
\pgfpathmoveto{\pgfqpoint{1.970338in}{3.526609in}}%
\pgfpathcurveto{\pgfqpoint{1.976162in}{3.526609in}}{\pgfqpoint{1.981749in}{3.528923in}}{\pgfqpoint{1.985867in}{3.533041in}}%
\pgfpathcurveto{\pgfqpoint{1.989985in}{3.537159in}}{\pgfqpoint{1.992299in}{3.542745in}}{\pgfqpoint{1.992299in}{3.548569in}}%
\pgfpathcurveto{\pgfqpoint{1.992299in}{3.554393in}}{\pgfqpoint{1.989985in}{3.559979in}}{\pgfqpoint{1.985867in}{3.564098in}}%
\pgfpathcurveto{\pgfqpoint{1.981749in}{3.568216in}}{\pgfqpoint{1.976162in}{3.570530in}}{\pgfqpoint{1.970338in}{3.570530in}}%
\pgfpathcurveto{\pgfqpoint{1.964515in}{3.570530in}}{\pgfqpoint{1.958928in}{3.568216in}}{\pgfqpoint{1.954810in}{3.564098in}}%
\pgfpathcurveto{\pgfqpoint{1.950692in}{3.559979in}}{\pgfqpoint{1.948378in}{3.554393in}}{\pgfqpoint{1.948378in}{3.548569in}}%
\pgfpathcurveto{\pgfqpoint{1.948378in}{3.542745in}}{\pgfqpoint{1.950692in}{3.537159in}}{\pgfqpoint{1.954810in}{3.533041in}}%
\pgfpathcurveto{\pgfqpoint{1.958928in}{3.528923in}}{\pgfqpoint{1.964515in}{3.526609in}}{\pgfqpoint{1.970338in}{3.526609in}}%
\pgfpathlineto{\pgfqpoint{1.970338in}{3.526609in}}%
\pgfpathclose%
\pgfusepath{stroke,fill}%
\end{pgfscope}%
\begin{pgfscope}%
\pgfpathrectangle{\pgfqpoint{1.582361in}{0.880000in}}{\pgfqpoint{5.035278in}{6.160000in}}%
\pgfusepath{clip}%
\pgfsetbuttcap%
\pgfsetroundjoin%
\definecolor{currentfill}{rgb}{0.500000,0.000000,0.500000}%
\pgfsetfillcolor{currentfill}%
\pgfsetlinewidth{1.003750pt}%
\definecolor{currentstroke}{rgb}{0.500000,0.000000,0.500000}%
\pgfsetstrokecolor{currentstroke}%
\pgfsetdash{}{0pt}%
\pgfpathmoveto{\pgfqpoint{2.195912in}{4.025198in}}%
\pgfpathcurveto{\pgfqpoint{2.201736in}{4.025198in}}{\pgfqpoint{2.207322in}{4.027512in}}{\pgfqpoint{2.211441in}{4.031630in}}%
\pgfpathcurveto{\pgfqpoint{2.215559in}{4.035748in}}{\pgfqpoint{2.217873in}{4.041334in}}{\pgfqpoint{2.217873in}{4.047158in}}%
\pgfpathcurveto{\pgfqpoint{2.217873in}{4.052982in}}{\pgfqpoint{2.215559in}{4.058568in}}{\pgfqpoint{2.211441in}{4.062686in}}%
\pgfpathcurveto{\pgfqpoint{2.207322in}{4.066804in}}{\pgfqpoint{2.201736in}{4.069118in}}{\pgfqpoint{2.195912in}{4.069118in}}%
\pgfpathcurveto{\pgfqpoint{2.190088in}{4.069118in}}{\pgfqpoint{2.184502in}{4.066804in}}{\pgfqpoint{2.180384in}{4.062686in}}%
\pgfpathcurveto{\pgfqpoint{2.176266in}{4.058568in}}{\pgfqpoint{2.173952in}{4.052982in}}{\pgfqpoint{2.173952in}{4.047158in}}%
\pgfpathcurveto{\pgfqpoint{2.173952in}{4.041334in}}{\pgfqpoint{2.176266in}{4.035748in}}{\pgfqpoint{2.180384in}{4.031630in}}%
\pgfpathcurveto{\pgfqpoint{2.184502in}{4.027512in}}{\pgfqpoint{2.190088in}{4.025198in}}{\pgfqpoint{2.195912in}{4.025198in}}%
\pgfpathlineto{\pgfqpoint{2.195912in}{4.025198in}}%
\pgfpathclose%
\pgfusepath{stroke,fill}%
\end{pgfscope}%
\begin{pgfscope}%
\pgfpathrectangle{\pgfqpoint{1.582361in}{0.880000in}}{\pgfqpoint{5.035278in}{6.160000in}}%
\pgfusepath{clip}%
\pgfsetbuttcap%
\pgfsetroundjoin%
\definecolor{currentfill}{rgb}{0.500000,0.000000,0.500000}%
\pgfsetfillcolor{currentfill}%
\pgfsetlinewidth{1.003750pt}%
\definecolor{currentstroke}{rgb}{0.500000,0.000000,0.500000}%
\pgfsetstrokecolor{currentstroke}%
\pgfsetdash{}{0pt}%
\pgfpathmoveto{\pgfqpoint{6.388763in}{3.163872in}}%
\pgfpathcurveto{\pgfqpoint{6.394587in}{3.163872in}}{\pgfqpoint{6.400173in}{3.166186in}}{\pgfqpoint{6.404291in}{3.170304in}}%
\pgfpathcurveto{\pgfqpoint{6.408409in}{3.174422in}}{\pgfqpoint{6.410723in}{3.180008in}}{\pgfqpoint{6.410723in}{3.185832in}}%
\pgfpathcurveto{\pgfqpoint{6.410723in}{3.191656in}}{\pgfqpoint{6.408409in}{3.197242in}}{\pgfqpoint{6.404291in}{3.201361in}}%
\pgfpathcurveto{\pgfqpoint{6.400173in}{3.205479in}}{\pgfqpoint{6.394587in}{3.207793in}}{\pgfqpoint{6.388763in}{3.207793in}}%
\pgfpathcurveto{\pgfqpoint{6.382939in}{3.207793in}}{\pgfqpoint{6.377353in}{3.205479in}}{\pgfqpoint{6.373235in}{3.201361in}}%
\pgfpathcurveto{\pgfqpoint{6.369116in}{3.197242in}}{\pgfqpoint{6.366802in}{3.191656in}}{\pgfqpoint{6.366802in}{3.185832in}}%
\pgfpathcurveto{\pgfqpoint{6.366802in}{3.180008in}}{\pgfqpoint{6.369116in}{3.174422in}}{\pgfqpoint{6.373235in}{3.170304in}}%
\pgfpathcurveto{\pgfqpoint{6.377353in}{3.166186in}}{\pgfqpoint{6.382939in}{3.163872in}}{\pgfqpoint{6.388763in}{3.163872in}}%
\pgfpathlineto{\pgfqpoint{6.388763in}{3.163872in}}%
\pgfpathclose%
\pgfusepath{stroke,fill}%
\end{pgfscope}%
\begin{pgfscope}%
\pgfpathrectangle{\pgfqpoint{1.582361in}{0.880000in}}{\pgfqpoint{5.035278in}{6.160000in}}%
\pgfusepath{clip}%
\pgfsetbuttcap%
\pgfsetroundjoin%
\definecolor{currentfill}{rgb}{0.500000,0.000000,0.500000}%
\pgfsetfillcolor{currentfill}%
\pgfsetlinewidth{1.003750pt}%
\definecolor{currentstroke}{rgb}{0.500000,0.000000,0.500000}%
\pgfsetstrokecolor{currentstroke}%
\pgfsetdash{}{0pt}%
\pgfpathmoveto{\pgfqpoint{6.093408in}{1.952141in}}%
\pgfpathcurveto{\pgfqpoint{6.099232in}{1.952141in}}{\pgfqpoint{6.104818in}{1.954455in}}{\pgfqpoint{6.108936in}{1.958573in}}%
\pgfpathcurveto{\pgfqpoint{6.113054in}{1.962691in}}{\pgfqpoint{6.115368in}{1.968278in}}{\pgfqpoint{6.115368in}{1.974102in}}%
\pgfpathcurveto{\pgfqpoint{6.115368in}{1.979925in}}{\pgfqpoint{6.113054in}{1.985512in}}{\pgfqpoint{6.108936in}{1.989630in}}%
\pgfpathcurveto{\pgfqpoint{6.104818in}{1.993748in}}{\pgfqpoint{6.099232in}{1.996062in}}{\pgfqpoint{6.093408in}{1.996062in}}%
\pgfpathcurveto{\pgfqpoint{6.087584in}{1.996062in}}{\pgfqpoint{6.081998in}{1.993748in}}{\pgfqpoint{6.077879in}{1.989630in}}%
\pgfpathcurveto{\pgfqpoint{6.073761in}{1.985512in}}{\pgfqpoint{6.071447in}{1.979925in}}{\pgfqpoint{6.071447in}{1.974102in}}%
\pgfpathcurveto{\pgfqpoint{6.071447in}{1.968278in}}{\pgfqpoint{6.073761in}{1.962691in}}{\pgfqpoint{6.077879in}{1.958573in}}%
\pgfpathcurveto{\pgfqpoint{6.081998in}{1.954455in}}{\pgfqpoint{6.087584in}{1.952141in}}{\pgfqpoint{6.093408in}{1.952141in}}%
\pgfpathlineto{\pgfqpoint{6.093408in}{1.952141in}}%
\pgfpathclose%
\pgfusepath{stroke,fill}%
\end{pgfscope}%
\begin{pgfscope}%
\pgfpathrectangle{\pgfqpoint{1.582361in}{0.880000in}}{\pgfqpoint{5.035278in}{6.160000in}}%
\pgfusepath{clip}%
\pgfsetbuttcap%
\pgfsetroundjoin%
\definecolor{currentfill}{rgb}{0.500000,0.000000,0.500000}%
\pgfsetfillcolor{currentfill}%
\pgfsetlinewidth{1.003750pt}%
\definecolor{currentstroke}{rgb}{0.500000,0.000000,0.500000}%
\pgfsetstrokecolor{currentstroke}%
\pgfsetdash{}{0pt}%
\pgfpathmoveto{\pgfqpoint{5.108532in}{3.092400in}}%
\pgfpathcurveto{\pgfqpoint{5.114355in}{3.092400in}}{\pgfqpoint{5.119942in}{3.094714in}}{\pgfqpoint{5.124060in}{3.098833in}}%
\pgfpathcurveto{\pgfqpoint{5.128178in}{3.102951in}}{\pgfqpoint{5.130492in}{3.108537in}}{\pgfqpoint{5.130492in}{3.114361in}}%
\pgfpathcurveto{\pgfqpoint{5.130492in}{3.120185in}}{\pgfqpoint{5.128178in}{3.125771in}}{\pgfqpoint{5.124060in}{3.129889in}}%
\pgfpathcurveto{\pgfqpoint{5.119942in}{3.134007in}}{\pgfqpoint{5.114355in}{3.136321in}}{\pgfqpoint{5.108532in}{3.136321in}}%
\pgfpathcurveto{\pgfqpoint{5.102708in}{3.136321in}}{\pgfqpoint{5.097121in}{3.134007in}}{\pgfqpoint{5.093003in}{3.129889in}}%
\pgfpathcurveto{\pgfqpoint{5.088885in}{3.125771in}}{\pgfqpoint{5.086571in}{3.120185in}}{\pgfqpoint{5.086571in}{3.114361in}}%
\pgfpathcurveto{\pgfqpoint{5.086571in}{3.108537in}}{\pgfqpoint{5.088885in}{3.102951in}}{\pgfqpoint{5.093003in}{3.098833in}}%
\pgfpathcurveto{\pgfqpoint{5.097121in}{3.094714in}}{\pgfqpoint{5.102708in}{3.092400in}}{\pgfqpoint{5.108532in}{3.092400in}}%
\pgfpathlineto{\pgfqpoint{5.108532in}{3.092400in}}%
\pgfpathclose%
\pgfusepath{stroke,fill}%
\end{pgfscope}%
\begin{pgfscope}%
\pgfpathrectangle{\pgfqpoint{1.582361in}{0.880000in}}{\pgfqpoint{5.035278in}{6.160000in}}%
\pgfusepath{clip}%
\pgfsetbuttcap%
\pgfsetroundjoin%
\definecolor{currentfill}{rgb}{0.500000,0.000000,0.500000}%
\pgfsetfillcolor{currentfill}%
\pgfsetlinewidth{1.003750pt}%
\definecolor{currentstroke}{rgb}{0.500000,0.000000,0.500000}%
\pgfsetstrokecolor{currentstroke}%
\pgfsetdash{}{0pt}%
\pgfpathmoveto{\pgfqpoint{3.882846in}{1.906678in}}%
\pgfpathcurveto{\pgfqpoint{3.888670in}{1.906678in}}{\pgfqpoint{3.894256in}{1.908992in}}{\pgfqpoint{3.898374in}{1.913110in}}%
\pgfpathcurveto{\pgfqpoint{3.902493in}{1.917228in}}{\pgfqpoint{3.904806in}{1.922815in}}{\pgfqpoint{3.904806in}{1.928639in}}%
\pgfpathcurveto{\pgfqpoint{3.904806in}{1.934463in}}{\pgfqpoint{3.902493in}{1.940049in}}{\pgfqpoint{3.898374in}{1.944167in}}%
\pgfpathcurveto{\pgfqpoint{3.894256in}{1.948285in}}{\pgfqpoint{3.888670in}{1.950599in}}{\pgfqpoint{3.882846in}{1.950599in}}%
\pgfpathcurveto{\pgfqpoint{3.877022in}{1.950599in}}{\pgfqpoint{3.871436in}{1.948285in}}{\pgfqpoint{3.867318in}{1.944167in}}%
\pgfpathcurveto{\pgfqpoint{3.863200in}{1.940049in}}{\pgfqpoint{3.860886in}{1.934463in}}{\pgfqpoint{3.860886in}{1.928639in}}%
\pgfpathcurveto{\pgfqpoint{3.860886in}{1.922815in}}{\pgfqpoint{3.863200in}{1.917228in}}{\pgfqpoint{3.867318in}{1.913110in}}%
\pgfpathcurveto{\pgfqpoint{3.871436in}{1.908992in}}{\pgfqpoint{3.877022in}{1.906678in}}{\pgfqpoint{3.882846in}{1.906678in}}%
\pgfpathlineto{\pgfqpoint{3.882846in}{1.906678in}}%
\pgfpathclose%
\pgfusepath{stroke,fill}%
\end{pgfscope}%
\begin{pgfscope}%
\pgfpathrectangle{\pgfqpoint{1.582361in}{0.880000in}}{\pgfqpoint{5.035278in}{6.160000in}}%
\pgfusepath{clip}%
\pgfsetbuttcap%
\pgfsetroundjoin%
\definecolor{currentfill}{rgb}{0.800000,0.200000,0.200000}%
\pgfsetfillcolor{currentfill}%
\pgfsetlinewidth{1.003750pt}%
\definecolor{currentstroke}{rgb}{0.800000,0.200000,0.200000}%
\pgfsetstrokecolor{currentstroke}%
\pgfsetdash{}{0pt}%
\pgfpathmoveto{\pgfqpoint{3.599702in}{4.177892in}}%
\pgfpathcurveto{\pgfqpoint{3.605526in}{4.177892in}}{\pgfqpoint{3.611112in}{4.180206in}}{\pgfqpoint{3.615230in}{4.184324in}}%
\pgfpathcurveto{\pgfqpoint{3.619348in}{4.188442in}}{\pgfqpoint{3.621662in}{4.194029in}}{\pgfqpoint{3.621662in}{4.199853in}}%
\pgfpathcurveto{\pgfqpoint{3.621662in}{4.205677in}}{\pgfqpoint{3.619348in}{4.211263in}}{\pgfqpoint{3.615230in}{4.215381in}}%
\pgfpathcurveto{\pgfqpoint{3.611112in}{4.219499in}}{\pgfqpoint{3.605526in}{4.221813in}}{\pgfqpoint{3.599702in}{4.221813in}}%
\pgfpathcurveto{\pgfqpoint{3.593878in}{4.221813in}}{\pgfqpoint{3.588292in}{4.219499in}}{\pgfqpoint{3.584174in}{4.215381in}}%
\pgfpathcurveto{\pgfqpoint{3.580056in}{4.211263in}}{\pgfqpoint{3.577742in}{4.205677in}}{\pgfqpoint{3.577742in}{4.199853in}}%
\pgfpathcurveto{\pgfqpoint{3.577742in}{4.194029in}}{\pgfqpoint{3.580056in}{4.188442in}}{\pgfqpoint{3.584174in}{4.184324in}}%
\pgfpathcurveto{\pgfqpoint{3.588292in}{4.180206in}}{\pgfqpoint{3.593878in}{4.177892in}}{\pgfqpoint{3.599702in}{4.177892in}}%
\pgfpathlineto{\pgfqpoint{3.599702in}{4.177892in}}%
\pgfpathclose%
\pgfusepath{stroke,fill}%
\end{pgfscope}%
\begin{pgfscope}%
\pgfpathrectangle{\pgfqpoint{1.582361in}{0.880000in}}{\pgfqpoint{5.035278in}{6.160000in}}%
\pgfusepath{clip}%
\pgfsetbuttcap%
\pgfsetroundjoin%
\definecolor{currentfill}{rgb}{0.500000,0.000000,0.500000}%
\pgfsetfillcolor{currentfill}%
\pgfsetlinewidth{1.003750pt}%
\definecolor{currentstroke}{rgb}{0.500000,0.000000,0.500000}%
\pgfsetstrokecolor{currentstroke}%
\pgfsetdash{}{0pt}%
\pgfpathmoveto{\pgfqpoint{4.899610in}{5.374417in}}%
\pgfpathcurveto{\pgfqpoint{4.905434in}{5.374417in}}{\pgfqpoint{4.911020in}{5.376731in}}{\pgfqpoint{4.915138in}{5.380849in}}%
\pgfpathcurveto{\pgfqpoint{4.919256in}{5.384967in}}{\pgfqpoint{4.921570in}{5.390553in}}{\pgfqpoint{4.921570in}{5.396377in}}%
\pgfpathcurveto{\pgfqpoint{4.921570in}{5.402201in}}{\pgfqpoint{4.919256in}{5.407787in}}{\pgfqpoint{4.915138in}{5.411906in}}%
\pgfpathcurveto{\pgfqpoint{4.911020in}{5.416024in}}{\pgfqpoint{4.905434in}{5.418338in}}{\pgfqpoint{4.899610in}{5.418338in}}%
\pgfpathcurveto{\pgfqpoint{4.893786in}{5.418338in}}{\pgfqpoint{4.888200in}{5.416024in}}{\pgfqpoint{4.884082in}{5.411906in}}%
\pgfpathcurveto{\pgfqpoint{4.879964in}{5.407787in}}{\pgfqpoint{4.877650in}{5.402201in}}{\pgfqpoint{4.877650in}{5.396377in}}%
\pgfpathcurveto{\pgfqpoint{4.877650in}{5.390553in}}{\pgfqpoint{4.879964in}{5.384967in}}{\pgfqpoint{4.884082in}{5.380849in}}%
\pgfpathcurveto{\pgfqpoint{4.888200in}{5.376731in}}{\pgfqpoint{4.893786in}{5.374417in}}{\pgfqpoint{4.899610in}{5.374417in}}%
\pgfpathlineto{\pgfqpoint{4.899610in}{5.374417in}}%
\pgfpathclose%
\pgfusepath{stroke,fill}%
\end{pgfscope}%
\begin{pgfscope}%
\pgfpathrectangle{\pgfqpoint{1.582361in}{0.880000in}}{\pgfqpoint{5.035278in}{6.160000in}}%
\pgfusepath{clip}%
\pgfsetbuttcap%
\pgfsetroundjoin%
\definecolor{currentfill}{rgb}{0.500000,0.000000,0.500000}%
\pgfsetfillcolor{currentfill}%
\pgfsetlinewidth{1.003750pt}%
\definecolor{currentstroke}{rgb}{0.500000,0.000000,0.500000}%
\pgfsetstrokecolor{currentstroke}%
\pgfsetdash{}{0pt}%
\pgfpathmoveto{\pgfqpoint{2.883488in}{2.491746in}}%
\pgfpathcurveto{\pgfqpoint{2.889312in}{2.491746in}}{\pgfqpoint{2.894898in}{2.494060in}}{\pgfqpoint{2.899017in}{2.498178in}}%
\pgfpathcurveto{\pgfqpoint{2.903135in}{2.502297in}}{\pgfqpoint{2.905449in}{2.507883in}}{\pgfqpoint{2.905449in}{2.513707in}}%
\pgfpathcurveto{\pgfqpoint{2.905449in}{2.519531in}}{\pgfqpoint{2.903135in}{2.525117in}}{\pgfqpoint{2.899017in}{2.529235in}}%
\pgfpathcurveto{\pgfqpoint{2.894898in}{2.533353in}}{\pgfqpoint{2.889312in}{2.535667in}}{\pgfqpoint{2.883488in}{2.535667in}}%
\pgfpathcurveto{\pgfqpoint{2.877664in}{2.535667in}}{\pgfqpoint{2.872078in}{2.533353in}}{\pgfqpoint{2.867960in}{2.529235in}}%
\pgfpathcurveto{\pgfqpoint{2.863842in}{2.525117in}}{\pgfqpoint{2.861528in}{2.519531in}}{\pgfqpoint{2.861528in}{2.513707in}}%
\pgfpathcurveto{\pgfqpoint{2.861528in}{2.507883in}}{\pgfqpoint{2.863842in}{2.502297in}}{\pgfqpoint{2.867960in}{2.498178in}}%
\pgfpathcurveto{\pgfqpoint{2.872078in}{2.494060in}}{\pgfqpoint{2.877664in}{2.491746in}}{\pgfqpoint{2.883488in}{2.491746in}}%
\pgfpathlineto{\pgfqpoint{2.883488in}{2.491746in}}%
\pgfpathclose%
\pgfusepath{stroke,fill}%
\end{pgfscope}%
\begin{pgfscope}%
\pgfpathrectangle{\pgfqpoint{1.582361in}{0.880000in}}{\pgfqpoint{5.035278in}{6.160000in}}%
\pgfusepath{clip}%
\pgfsetbuttcap%
\pgfsetroundjoin%
\definecolor{currentfill}{rgb}{0.500000,0.000000,0.500000}%
\pgfsetfillcolor{currentfill}%
\pgfsetlinewidth{1.003750pt}%
\definecolor{currentstroke}{rgb}{0.500000,0.000000,0.500000}%
\pgfsetstrokecolor{currentstroke}%
\pgfsetdash{}{0pt}%
\pgfpathmoveto{\pgfqpoint{3.014696in}{3.001174in}}%
\pgfpathcurveto{\pgfqpoint{3.020520in}{3.001174in}}{\pgfqpoint{3.026107in}{3.003488in}}{\pgfqpoint{3.030225in}{3.007606in}}%
\pgfpathcurveto{\pgfqpoint{3.034343in}{3.011724in}}{\pgfqpoint{3.036657in}{3.017310in}}{\pgfqpoint{3.036657in}{3.023134in}}%
\pgfpathcurveto{\pgfqpoint{3.036657in}{3.028958in}}{\pgfqpoint{3.034343in}{3.034544in}}{\pgfqpoint{3.030225in}{3.038663in}}%
\pgfpathcurveto{\pgfqpoint{3.026107in}{3.042781in}}{\pgfqpoint{3.020520in}{3.045095in}}{\pgfqpoint{3.014696in}{3.045095in}}%
\pgfpathcurveto{\pgfqpoint{3.008873in}{3.045095in}}{\pgfqpoint{3.003286in}{3.042781in}}{\pgfqpoint{2.999168in}{3.038663in}}%
\pgfpathcurveto{\pgfqpoint{2.995050in}{3.034544in}}{\pgfqpoint{2.992736in}{3.028958in}}{\pgfqpoint{2.992736in}{3.023134in}}%
\pgfpathcurveto{\pgfqpoint{2.992736in}{3.017310in}}{\pgfqpoint{2.995050in}{3.011724in}}{\pgfqpoint{2.999168in}{3.007606in}}%
\pgfpathcurveto{\pgfqpoint{3.003286in}{3.003488in}}{\pgfqpoint{3.008873in}{3.001174in}}{\pgfqpoint{3.014696in}{3.001174in}}%
\pgfpathlineto{\pgfqpoint{3.014696in}{3.001174in}}%
\pgfpathclose%
\pgfusepath{stroke,fill}%
\end{pgfscope}%
\begin{pgfscope}%
\pgfpathrectangle{\pgfqpoint{1.582361in}{0.880000in}}{\pgfqpoint{5.035278in}{6.160000in}}%
\pgfusepath{clip}%
\pgfsetbuttcap%
\pgfsetroundjoin%
\definecolor{currentfill}{rgb}{0.500000,0.000000,0.500000}%
\pgfsetfillcolor{currentfill}%
\pgfsetlinewidth{1.003750pt}%
\definecolor{currentstroke}{rgb}{0.500000,0.000000,0.500000}%
\pgfsetstrokecolor{currentstroke}%
\pgfsetdash{}{0pt}%
\pgfpathmoveto{\pgfqpoint{5.945780in}{2.489037in}}%
\pgfpathcurveto{\pgfqpoint{5.951604in}{2.489037in}}{\pgfqpoint{5.957190in}{2.491351in}}{\pgfqpoint{5.961308in}{2.495469in}}%
\pgfpathcurveto{\pgfqpoint{5.965426in}{2.499587in}}{\pgfqpoint{5.967740in}{2.505174in}}{\pgfqpoint{5.967740in}{2.510998in}}%
\pgfpathcurveto{\pgfqpoint{5.967740in}{2.516822in}}{\pgfqpoint{5.965426in}{2.522408in}}{\pgfqpoint{5.961308in}{2.526526in}}%
\pgfpathcurveto{\pgfqpoint{5.957190in}{2.530644in}}{\pgfqpoint{5.951604in}{2.532958in}}{\pgfqpoint{5.945780in}{2.532958in}}%
\pgfpathcurveto{\pgfqpoint{5.939956in}{2.532958in}}{\pgfqpoint{5.934370in}{2.530644in}}{\pgfqpoint{5.930252in}{2.526526in}}%
\pgfpathcurveto{\pgfqpoint{5.926134in}{2.522408in}}{\pgfqpoint{5.923820in}{2.516822in}}{\pgfqpoint{5.923820in}{2.510998in}}%
\pgfpathcurveto{\pgfqpoint{5.923820in}{2.505174in}}{\pgfqpoint{5.926134in}{2.499587in}}{\pgfqpoint{5.930252in}{2.495469in}}%
\pgfpathcurveto{\pgfqpoint{5.934370in}{2.491351in}}{\pgfqpoint{5.939956in}{2.489037in}}{\pgfqpoint{5.945780in}{2.489037in}}%
\pgfpathlineto{\pgfqpoint{5.945780in}{2.489037in}}%
\pgfpathclose%
\pgfusepath{stroke,fill}%
\end{pgfscope}%
\begin{pgfscope}%
\pgfpathrectangle{\pgfqpoint{1.582361in}{0.880000in}}{\pgfqpoint{5.035278in}{6.160000in}}%
\pgfusepath{clip}%
\pgfsetbuttcap%
\pgfsetroundjoin%
\definecolor{currentfill}{rgb}{0.500000,0.000000,0.500000}%
\pgfsetfillcolor{currentfill}%
\pgfsetlinewidth{1.003750pt}%
\definecolor{currentstroke}{rgb}{0.500000,0.000000,0.500000}%
\pgfsetstrokecolor{currentstroke}%
\pgfsetdash{}{0pt}%
\pgfpathmoveto{\pgfqpoint{2.203350in}{2.073715in}}%
\pgfpathcurveto{\pgfqpoint{2.209174in}{2.073715in}}{\pgfqpoint{2.214760in}{2.076029in}}{\pgfqpoint{2.218878in}{2.080147in}}%
\pgfpathcurveto{\pgfqpoint{2.222997in}{2.084265in}}{\pgfqpoint{2.225310in}{2.089852in}}{\pgfqpoint{2.225310in}{2.095676in}}%
\pgfpathcurveto{\pgfqpoint{2.225310in}{2.101500in}}{\pgfqpoint{2.222997in}{2.107086in}}{\pgfqpoint{2.218878in}{2.111204in}}%
\pgfpathcurveto{\pgfqpoint{2.214760in}{2.115322in}}{\pgfqpoint{2.209174in}{2.117636in}}{\pgfqpoint{2.203350in}{2.117636in}}%
\pgfpathcurveto{\pgfqpoint{2.197526in}{2.117636in}}{\pgfqpoint{2.191940in}{2.115322in}}{\pgfqpoint{2.187822in}{2.111204in}}%
\pgfpathcurveto{\pgfqpoint{2.183704in}{2.107086in}}{\pgfqpoint{2.181390in}{2.101500in}}{\pgfqpoint{2.181390in}{2.095676in}}%
\pgfpathcurveto{\pgfqpoint{2.181390in}{2.089852in}}{\pgfqpoint{2.183704in}{2.084265in}}{\pgfqpoint{2.187822in}{2.080147in}}%
\pgfpathcurveto{\pgfqpoint{2.191940in}{2.076029in}}{\pgfqpoint{2.197526in}{2.073715in}}{\pgfqpoint{2.203350in}{2.073715in}}%
\pgfpathlineto{\pgfqpoint{2.203350in}{2.073715in}}%
\pgfpathclose%
\pgfusepath{stroke,fill}%
\end{pgfscope}%
\begin{pgfscope}%
\pgfpathrectangle{\pgfqpoint{1.582361in}{0.880000in}}{\pgfqpoint{5.035278in}{6.160000in}}%
\pgfusepath{clip}%
\pgfsetbuttcap%
\pgfsetroundjoin%
\definecolor{currentfill}{rgb}{0.500000,0.000000,0.500000}%
\pgfsetfillcolor{currentfill}%
\pgfsetlinewidth{1.003750pt}%
\definecolor{currentstroke}{rgb}{0.500000,0.000000,0.500000}%
\pgfsetstrokecolor{currentstroke}%
\pgfsetdash{}{0pt}%
\pgfpathmoveto{\pgfqpoint{2.112240in}{2.208866in}}%
\pgfpathcurveto{\pgfqpoint{2.118064in}{2.208866in}}{\pgfqpoint{2.123651in}{2.211180in}}{\pgfqpoint{2.127769in}{2.215298in}}%
\pgfpathcurveto{\pgfqpoint{2.131887in}{2.219416in}}{\pgfqpoint{2.134201in}{2.225002in}}{\pgfqpoint{2.134201in}{2.230826in}}%
\pgfpathcurveto{\pgfqpoint{2.134201in}{2.236650in}}{\pgfqpoint{2.131887in}{2.242236in}}{\pgfqpoint{2.127769in}{2.246354in}}%
\pgfpathcurveto{\pgfqpoint{2.123651in}{2.250473in}}{\pgfqpoint{2.118064in}{2.252786in}}{\pgfqpoint{2.112240in}{2.252786in}}%
\pgfpathcurveto{\pgfqpoint{2.106416in}{2.252786in}}{\pgfqpoint{2.100830in}{2.250473in}}{\pgfqpoint{2.096712in}{2.246354in}}%
\pgfpathcurveto{\pgfqpoint{2.092594in}{2.242236in}}{\pgfqpoint{2.090280in}{2.236650in}}{\pgfqpoint{2.090280in}{2.230826in}}%
\pgfpathcurveto{\pgfqpoint{2.090280in}{2.225002in}}{\pgfqpoint{2.092594in}{2.219416in}}{\pgfqpoint{2.096712in}{2.215298in}}%
\pgfpathcurveto{\pgfqpoint{2.100830in}{2.211180in}}{\pgfqpoint{2.106416in}{2.208866in}}{\pgfqpoint{2.112240in}{2.208866in}}%
\pgfpathlineto{\pgfqpoint{2.112240in}{2.208866in}}%
\pgfpathclose%
\pgfusepath{stroke,fill}%
\end{pgfscope}%
\begin{pgfscope}%
\pgfpathrectangle{\pgfqpoint{1.582361in}{0.880000in}}{\pgfqpoint{5.035278in}{6.160000in}}%
\pgfusepath{clip}%
\pgfsetbuttcap%
\pgfsetroundjoin%
\definecolor{currentfill}{rgb}{0.500000,0.000000,0.500000}%
\pgfsetfillcolor{currentfill}%
\pgfsetlinewidth{1.003750pt}%
\definecolor{currentstroke}{rgb}{0.500000,0.000000,0.500000}%
\pgfsetstrokecolor{currentstroke}%
\pgfsetdash{}{0pt}%
\pgfpathmoveto{\pgfqpoint{3.812487in}{4.839559in}}%
\pgfpathcurveto{\pgfqpoint{3.818311in}{4.839559in}}{\pgfqpoint{3.823897in}{4.841873in}}{\pgfqpoint{3.828015in}{4.845991in}}%
\pgfpathcurveto{\pgfqpoint{3.832133in}{4.850109in}}{\pgfqpoint{3.834447in}{4.855695in}}{\pgfqpoint{3.834447in}{4.861519in}}%
\pgfpathcurveto{\pgfqpoint{3.834447in}{4.867343in}}{\pgfqpoint{3.832133in}{4.872929in}}{\pgfqpoint{3.828015in}{4.877047in}}%
\pgfpathcurveto{\pgfqpoint{3.823897in}{4.881165in}}{\pgfqpoint{3.818311in}{4.883479in}}{\pgfqpoint{3.812487in}{4.883479in}}%
\pgfpathcurveto{\pgfqpoint{3.806663in}{4.883479in}}{\pgfqpoint{3.801077in}{4.881165in}}{\pgfqpoint{3.796959in}{4.877047in}}%
\pgfpathcurveto{\pgfqpoint{3.792840in}{4.872929in}}{\pgfqpoint{3.790527in}{4.867343in}}{\pgfqpoint{3.790527in}{4.861519in}}%
\pgfpathcurveto{\pgfqpoint{3.790527in}{4.855695in}}{\pgfqpoint{3.792840in}{4.850109in}}{\pgfqpoint{3.796959in}{4.845991in}}%
\pgfpathcurveto{\pgfqpoint{3.801077in}{4.841873in}}{\pgfqpoint{3.806663in}{4.839559in}}{\pgfqpoint{3.812487in}{4.839559in}}%
\pgfpathlineto{\pgfqpoint{3.812487in}{4.839559in}}%
\pgfpathclose%
\pgfusepath{stroke,fill}%
\end{pgfscope}%
\begin{pgfscope}%
\pgfpathrectangle{\pgfqpoint{1.582361in}{0.880000in}}{\pgfqpoint{5.035278in}{6.160000in}}%
\pgfusepath{clip}%
\pgfsetbuttcap%
\pgfsetroundjoin%
\definecolor{currentfill}{rgb}{0.500000,0.000000,0.500000}%
\pgfsetfillcolor{currentfill}%
\pgfsetlinewidth{1.003750pt}%
\definecolor{currentstroke}{rgb}{0.500000,0.000000,0.500000}%
\pgfsetstrokecolor{currentstroke}%
\pgfsetdash{}{0pt}%
\pgfpathmoveto{\pgfqpoint{3.691106in}{5.753455in}}%
\pgfpathcurveto{\pgfqpoint{3.696930in}{5.753455in}}{\pgfqpoint{3.702516in}{5.755769in}}{\pgfqpoint{3.706635in}{5.759887in}}%
\pgfpathcurveto{\pgfqpoint{3.710753in}{5.764005in}}{\pgfqpoint{3.713067in}{5.769591in}}{\pgfqpoint{3.713067in}{5.775415in}}%
\pgfpathcurveto{\pgfqpoint{3.713067in}{5.781239in}}{\pgfqpoint{3.710753in}{5.786826in}}{\pgfqpoint{3.706635in}{5.790944in}}%
\pgfpathcurveto{\pgfqpoint{3.702516in}{5.795062in}}{\pgfqpoint{3.696930in}{5.797376in}}{\pgfqpoint{3.691106in}{5.797376in}}%
\pgfpathcurveto{\pgfqpoint{3.685282in}{5.797376in}}{\pgfqpoint{3.679696in}{5.795062in}}{\pgfqpoint{3.675578in}{5.790944in}}%
\pgfpathcurveto{\pgfqpoint{3.671460in}{5.786826in}}{\pgfqpoint{3.669146in}{5.781239in}}{\pgfqpoint{3.669146in}{5.775415in}}%
\pgfpathcurveto{\pgfqpoint{3.669146in}{5.769591in}}{\pgfqpoint{3.671460in}{5.764005in}}{\pgfqpoint{3.675578in}{5.759887in}}%
\pgfpathcurveto{\pgfqpoint{3.679696in}{5.755769in}}{\pgfqpoint{3.685282in}{5.753455in}}{\pgfqpoint{3.691106in}{5.753455in}}%
\pgfpathlineto{\pgfqpoint{3.691106in}{5.753455in}}%
\pgfpathclose%
\pgfusepath{stroke,fill}%
\end{pgfscope}%
\begin{pgfscope}%
\pgfpathrectangle{\pgfqpoint{1.582361in}{0.880000in}}{\pgfqpoint{5.035278in}{6.160000in}}%
\pgfusepath{clip}%
\pgfsetbuttcap%
\pgfsetroundjoin%
\definecolor{currentfill}{rgb}{0.500000,0.000000,0.500000}%
\pgfsetfillcolor{currentfill}%
\pgfsetlinewidth{1.003750pt}%
\definecolor{currentstroke}{rgb}{0.500000,0.000000,0.500000}%
\pgfsetstrokecolor{currentstroke}%
\pgfsetdash{}{0pt}%
\pgfpathmoveto{\pgfqpoint{3.675322in}{2.519189in}}%
\pgfpathcurveto{\pgfqpoint{3.681146in}{2.519189in}}{\pgfqpoint{3.686732in}{2.521503in}}{\pgfqpoint{3.690850in}{2.525621in}}%
\pgfpathcurveto{\pgfqpoint{3.694968in}{2.529739in}}{\pgfqpoint{3.697282in}{2.535326in}}{\pgfqpoint{3.697282in}{2.541150in}}%
\pgfpathcurveto{\pgfqpoint{3.697282in}{2.546974in}}{\pgfqpoint{3.694968in}{2.552560in}}{\pgfqpoint{3.690850in}{2.556678in}}%
\pgfpathcurveto{\pgfqpoint{3.686732in}{2.560796in}}{\pgfqpoint{3.681146in}{2.563110in}}{\pgfqpoint{3.675322in}{2.563110in}}%
\pgfpathcurveto{\pgfqpoint{3.669498in}{2.563110in}}{\pgfqpoint{3.663912in}{2.560796in}}{\pgfqpoint{3.659794in}{2.556678in}}%
\pgfpathcurveto{\pgfqpoint{3.655675in}{2.552560in}}{\pgfqpoint{3.653362in}{2.546974in}}{\pgfqpoint{3.653362in}{2.541150in}}%
\pgfpathcurveto{\pgfqpoint{3.653362in}{2.535326in}}{\pgfqpoint{3.655675in}{2.529739in}}{\pgfqpoint{3.659794in}{2.525621in}}%
\pgfpathcurveto{\pgfqpoint{3.663912in}{2.521503in}}{\pgfqpoint{3.669498in}{2.519189in}}{\pgfqpoint{3.675322in}{2.519189in}}%
\pgfpathlineto{\pgfqpoint{3.675322in}{2.519189in}}%
\pgfpathclose%
\pgfusepath{stroke,fill}%
\end{pgfscope}%
\begin{pgfscope}%
\pgfpathrectangle{\pgfqpoint{1.582361in}{0.880000in}}{\pgfqpoint{5.035278in}{6.160000in}}%
\pgfusepath{clip}%
\pgfsetbuttcap%
\pgfsetroundjoin%
\definecolor{currentfill}{rgb}{0.500000,0.000000,0.500000}%
\pgfsetfillcolor{currentfill}%
\pgfsetlinewidth{1.003750pt}%
\definecolor{currentstroke}{rgb}{0.500000,0.000000,0.500000}%
\pgfsetstrokecolor{currentstroke}%
\pgfsetdash{}{0pt}%
\pgfpathmoveto{\pgfqpoint{3.904776in}{2.404058in}}%
\pgfpathcurveto{\pgfqpoint{3.910600in}{2.404058in}}{\pgfqpoint{3.916186in}{2.406371in}}{\pgfqpoint{3.920304in}{2.410490in}}%
\pgfpathcurveto{\pgfqpoint{3.924422in}{2.414608in}}{\pgfqpoint{3.926736in}{2.420194in}}{\pgfqpoint{3.926736in}{2.426018in}}%
\pgfpathcurveto{\pgfqpoint{3.926736in}{2.431842in}}{\pgfqpoint{3.924422in}{2.437428in}}{\pgfqpoint{3.920304in}{2.441546in}}%
\pgfpathcurveto{\pgfqpoint{3.916186in}{2.445664in}}{\pgfqpoint{3.910600in}{2.447978in}}{\pgfqpoint{3.904776in}{2.447978in}}%
\pgfpathcurveto{\pgfqpoint{3.898952in}{2.447978in}}{\pgfqpoint{3.893366in}{2.445664in}}{\pgfqpoint{3.889247in}{2.441546in}}%
\pgfpathcurveto{\pgfqpoint{3.885129in}{2.437428in}}{\pgfqpoint{3.882815in}{2.431842in}}{\pgfqpoint{3.882815in}{2.426018in}}%
\pgfpathcurveto{\pgfqpoint{3.882815in}{2.420194in}}{\pgfqpoint{3.885129in}{2.414608in}}{\pgfqpoint{3.889247in}{2.410490in}}%
\pgfpathcurveto{\pgfqpoint{3.893366in}{2.406371in}}{\pgfqpoint{3.898952in}{2.404058in}}{\pgfqpoint{3.904776in}{2.404058in}}%
\pgfpathlineto{\pgfqpoint{3.904776in}{2.404058in}}%
\pgfpathclose%
\pgfusepath{stroke,fill}%
\end{pgfscope}%
\begin{pgfscope}%
\pgfpathrectangle{\pgfqpoint{1.582361in}{0.880000in}}{\pgfqpoint{5.035278in}{6.160000in}}%
\pgfusepath{clip}%
\pgfsetbuttcap%
\pgfsetroundjoin%
\definecolor{currentfill}{rgb}{0.500000,0.000000,0.500000}%
\pgfsetfillcolor{currentfill}%
\pgfsetlinewidth{1.003750pt}%
\definecolor{currentstroke}{rgb}{0.500000,0.000000,0.500000}%
\pgfsetstrokecolor{currentstroke}%
\pgfsetdash{}{0pt}%
\pgfpathmoveto{\pgfqpoint{3.251279in}{4.689654in}}%
\pgfpathcurveto{\pgfqpoint{3.257103in}{4.689654in}}{\pgfqpoint{3.262689in}{4.691968in}}{\pgfqpoint{3.266807in}{4.696086in}}%
\pgfpathcurveto{\pgfqpoint{3.270925in}{4.700204in}}{\pgfqpoint{3.273239in}{4.705790in}}{\pgfqpoint{3.273239in}{4.711614in}}%
\pgfpathcurveto{\pgfqpoint{3.273239in}{4.717438in}}{\pgfqpoint{3.270925in}{4.723024in}}{\pgfqpoint{3.266807in}{4.727143in}}%
\pgfpathcurveto{\pgfqpoint{3.262689in}{4.731261in}}{\pgfqpoint{3.257103in}{4.733575in}}{\pgfqpoint{3.251279in}{4.733575in}}%
\pgfpathcurveto{\pgfqpoint{3.245455in}{4.733575in}}{\pgfqpoint{3.239869in}{4.731261in}}{\pgfqpoint{3.235750in}{4.727143in}}%
\pgfpathcurveto{\pgfqpoint{3.231632in}{4.723024in}}{\pgfqpoint{3.229318in}{4.717438in}}{\pgfqpoint{3.229318in}{4.711614in}}%
\pgfpathcurveto{\pgfqpoint{3.229318in}{4.705790in}}{\pgfqpoint{3.231632in}{4.700204in}}{\pgfqpoint{3.235750in}{4.696086in}}%
\pgfpathcurveto{\pgfqpoint{3.239869in}{4.691968in}}{\pgfqpoint{3.245455in}{4.689654in}}{\pgfqpoint{3.251279in}{4.689654in}}%
\pgfpathlineto{\pgfqpoint{3.251279in}{4.689654in}}%
\pgfpathclose%
\pgfusepath{stroke,fill}%
\end{pgfscope}%
\begin{pgfscope}%
\pgfpathrectangle{\pgfqpoint{1.582361in}{0.880000in}}{\pgfqpoint{5.035278in}{6.160000in}}%
\pgfusepath{clip}%
\pgfsetbuttcap%
\pgfsetroundjoin%
\definecolor{currentfill}{rgb}{0.500000,0.000000,0.500000}%
\pgfsetfillcolor{currentfill}%
\pgfsetlinewidth{1.003750pt}%
\definecolor{currentstroke}{rgb}{0.500000,0.000000,0.500000}%
\pgfsetstrokecolor{currentstroke}%
\pgfsetdash{}{0pt}%
\pgfpathmoveto{\pgfqpoint{2.740596in}{3.978410in}}%
\pgfpathcurveto{\pgfqpoint{2.746420in}{3.978410in}}{\pgfqpoint{2.752007in}{3.980724in}}{\pgfqpoint{2.756125in}{3.984842in}}%
\pgfpathcurveto{\pgfqpoint{2.760243in}{3.988960in}}{\pgfqpoint{2.762557in}{3.994546in}}{\pgfqpoint{2.762557in}{4.000370in}}%
\pgfpathcurveto{\pgfqpoint{2.762557in}{4.006194in}}{\pgfqpoint{2.760243in}{4.011780in}}{\pgfqpoint{2.756125in}{4.015899in}}%
\pgfpathcurveto{\pgfqpoint{2.752007in}{4.020017in}}{\pgfqpoint{2.746420in}{4.022331in}}{\pgfqpoint{2.740596in}{4.022331in}}%
\pgfpathcurveto{\pgfqpoint{2.734773in}{4.022331in}}{\pgfqpoint{2.729186in}{4.020017in}}{\pgfqpoint{2.725068in}{4.015899in}}%
\pgfpathcurveto{\pgfqpoint{2.720950in}{4.011780in}}{\pgfqpoint{2.718636in}{4.006194in}}{\pgfqpoint{2.718636in}{4.000370in}}%
\pgfpathcurveto{\pgfqpoint{2.718636in}{3.994546in}}{\pgfqpoint{2.720950in}{3.988960in}}{\pgfqpoint{2.725068in}{3.984842in}}%
\pgfpathcurveto{\pgfqpoint{2.729186in}{3.980724in}}{\pgfqpoint{2.734773in}{3.978410in}}{\pgfqpoint{2.740596in}{3.978410in}}%
\pgfpathlineto{\pgfqpoint{2.740596in}{3.978410in}}%
\pgfpathclose%
\pgfusepath{stroke,fill}%
\end{pgfscope}%
\begin{pgfscope}%
\pgfpathrectangle{\pgfqpoint{1.582361in}{0.880000in}}{\pgfqpoint{5.035278in}{6.160000in}}%
\pgfusepath{clip}%
\pgfsetbuttcap%
\pgfsetroundjoin%
\definecolor{currentfill}{rgb}{0.500000,0.000000,0.500000}%
\pgfsetfillcolor{currentfill}%
\pgfsetlinewidth{1.003750pt}%
\definecolor{currentstroke}{rgb}{0.500000,0.000000,0.500000}%
\pgfsetstrokecolor{currentstroke}%
\pgfsetdash{}{0pt}%
\pgfpathmoveto{\pgfqpoint{2.106272in}{5.279488in}}%
\pgfpathcurveto{\pgfqpoint{2.112096in}{5.279488in}}{\pgfqpoint{2.117683in}{5.281802in}}{\pgfqpoint{2.121801in}{5.285920in}}%
\pgfpathcurveto{\pgfqpoint{2.125919in}{5.290038in}}{\pgfqpoint{2.128233in}{5.295624in}}{\pgfqpoint{2.128233in}{5.301448in}}%
\pgfpathcurveto{\pgfqpoint{2.128233in}{5.307272in}}{\pgfqpoint{2.125919in}{5.312858in}}{\pgfqpoint{2.121801in}{5.316977in}}%
\pgfpathcurveto{\pgfqpoint{2.117683in}{5.321095in}}{\pgfqpoint{2.112096in}{5.323409in}}{\pgfqpoint{2.106272in}{5.323409in}}%
\pgfpathcurveto{\pgfqpoint{2.100449in}{5.323409in}}{\pgfqpoint{2.094862in}{5.321095in}}{\pgfqpoint{2.090744in}{5.316977in}}%
\pgfpathcurveto{\pgfqpoint{2.086626in}{5.312858in}}{\pgfqpoint{2.084312in}{5.307272in}}{\pgfqpoint{2.084312in}{5.301448in}}%
\pgfpathcurveto{\pgfqpoint{2.084312in}{5.295624in}}{\pgfqpoint{2.086626in}{5.290038in}}{\pgfqpoint{2.090744in}{5.285920in}}%
\pgfpathcurveto{\pgfqpoint{2.094862in}{5.281802in}}{\pgfqpoint{2.100449in}{5.279488in}}{\pgfqpoint{2.106272in}{5.279488in}}%
\pgfpathlineto{\pgfqpoint{2.106272in}{5.279488in}}%
\pgfpathclose%
\pgfusepath{stroke,fill}%
\end{pgfscope}%
\begin{pgfscope}%
\pgfpathrectangle{\pgfqpoint{1.582361in}{0.880000in}}{\pgfqpoint{5.035278in}{6.160000in}}%
\pgfusepath{clip}%
\pgfsetbuttcap%
\pgfsetroundjoin%
\definecolor{currentfill}{rgb}{0.800000,0.200000,0.200000}%
\pgfsetfillcolor{currentfill}%
\pgfsetlinewidth{1.003750pt}%
\definecolor{currentstroke}{rgb}{0.800000,0.200000,0.200000}%
\pgfsetstrokecolor{currentstroke}%
\pgfsetdash{}{0pt}%
\pgfpathmoveto{\pgfqpoint{2.485712in}{5.222983in}}%
\pgfpathcurveto{\pgfqpoint{2.491536in}{5.222983in}}{\pgfqpoint{2.497122in}{5.225296in}}{\pgfqpoint{2.501240in}{5.229415in}}%
\pgfpathcurveto{\pgfqpoint{2.505358in}{5.233533in}}{\pgfqpoint{2.507672in}{5.239119in}}{\pgfqpoint{2.507672in}{5.244943in}}%
\pgfpathcurveto{\pgfqpoint{2.507672in}{5.250767in}}{\pgfqpoint{2.505358in}{5.256353in}}{\pgfqpoint{2.501240in}{5.260471in}}%
\pgfpathcurveto{\pgfqpoint{2.497122in}{5.264589in}}{\pgfqpoint{2.491536in}{5.266903in}}{\pgfqpoint{2.485712in}{5.266903in}}%
\pgfpathcurveto{\pgfqpoint{2.479888in}{5.266903in}}{\pgfqpoint{2.474302in}{5.264589in}}{\pgfqpoint{2.470184in}{5.260471in}}%
\pgfpathcurveto{\pgfqpoint{2.466065in}{5.256353in}}{\pgfqpoint{2.463752in}{5.250767in}}{\pgfqpoint{2.463752in}{5.244943in}}%
\pgfpathcurveto{\pgfqpoint{2.463752in}{5.239119in}}{\pgfqpoint{2.466065in}{5.233533in}}{\pgfqpoint{2.470184in}{5.229415in}}%
\pgfpathcurveto{\pgfqpoint{2.474302in}{5.225296in}}{\pgfqpoint{2.479888in}{5.222983in}}{\pgfqpoint{2.485712in}{5.222983in}}%
\pgfpathlineto{\pgfqpoint{2.485712in}{5.222983in}}%
\pgfpathclose%
\pgfusepath{stroke,fill}%
\end{pgfscope}%
\begin{pgfscope}%
\pgfpathrectangle{\pgfqpoint{1.582361in}{0.880000in}}{\pgfqpoint{5.035278in}{6.160000in}}%
\pgfusepath{clip}%
\pgfsetbuttcap%
\pgfsetroundjoin%
\definecolor{currentfill}{rgb}{0.500000,0.000000,0.500000}%
\pgfsetfillcolor{currentfill}%
\pgfsetlinewidth{1.003750pt}%
\definecolor{currentstroke}{rgb}{0.500000,0.000000,0.500000}%
\pgfsetstrokecolor{currentstroke}%
\pgfsetdash{}{0pt}%
\pgfpathmoveto{\pgfqpoint{1.915309in}{4.126435in}}%
\pgfpathcurveto{\pgfqpoint{1.921133in}{4.126435in}}{\pgfqpoint{1.926719in}{4.128749in}}{\pgfqpoint{1.930837in}{4.132867in}}%
\pgfpathcurveto{\pgfqpoint{1.934955in}{4.136985in}}{\pgfqpoint{1.937269in}{4.142572in}}{\pgfqpoint{1.937269in}{4.148396in}}%
\pgfpathcurveto{\pgfqpoint{1.937269in}{4.154219in}}{\pgfqpoint{1.934955in}{4.159806in}}{\pgfqpoint{1.930837in}{4.163924in}}%
\pgfpathcurveto{\pgfqpoint{1.926719in}{4.168042in}}{\pgfqpoint{1.921133in}{4.170356in}}{\pgfqpoint{1.915309in}{4.170356in}}%
\pgfpathcurveto{\pgfqpoint{1.909485in}{4.170356in}}{\pgfqpoint{1.903899in}{4.168042in}}{\pgfqpoint{1.899781in}{4.163924in}}%
\pgfpathcurveto{\pgfqpoint{1.895662in}{4.159806in}}{\pgfqpoint{1.893349in}{4.154219in}}{\pgfqpoint{1.893349in}{4.148396in}}%
\pgfpathcurveto{\pgfqpoint{1.893349in}{4.142572in}}{\pgfqpoint{1.895662in}{4.136985in}}{\pgfqpoint{1.899781in}{4.132867in}}%
\pgfpathcurveto{\pgfqpoint{1.903899in}{4.128749in}}{\pgfqpoint{1.909485in}{4.126435in}}{\pgfqpoint{1.915309in}{4.126435in}}%
\pgfpathlineto{\pgfqpoint{1.915309in}{4.126435in}}%
\pgfpathclose%
\pgfusepath{stroke,fill}%
\end{pgfscope}%
\begin{pgfscope}%
\pgfpathrectangle{\pgfqpoint{1.582361in}{0.880000in}}{\pgfqpoint{5.035278in}{6.160000in}}%
\pgfusepath{clip}%
\pgfsetbuttcap%
\pgfsetroundjoin%
\definecolor{currentfill}{rgb}{0.500000,0.000000,0.500000}%
\pgfsetfillcolor{currentfill}%
\pgfsetlinewidth{1.003750pt}%
\definecolor{currentstroke}{rgb}{0.500000,0.000000,0.500000}%
\pgfsetstrokecolor{currentstroke}%
\pgfsetdash{}{0pt}%
\pgfpathmoveto{\pgfqpoint{3.397317in}{1.138040in}}%
\pgfpathcurveto{\pgfqpoint{3.403140in}{1.138040in}}{\pgfqpoint{3.408727in}{1.140354in}}{\pgfqpoint{3.412845in}{1.144472in}}%
\pgfpathcurveto{\pgfqpoint{3.416963in}{1.148590in}}{\pgfqpoint{3.419277in}{1.154176in}}{\pgfqpoint{3.419277in}{1.160000in}}%
\pgfpathcurveto{\pgfqpoint{3.419277in}{1.165824in}}{\pgfqpoint{3.416963in}{1.171410in}}{\pgfqpoint{3.412845in}{1.175528in}}%
\pgfpathcurveto{\pgfqpoint{3.408727in}{1.179646in}}{\pgfqpoint{3.403140in}{1.181960in}}{\pgfqpoint{3.397317in}{1.181960in}}%
\pgfpathcurveto{\pgfqpoint{3.391493in}{1.181960in}}{\pgfqpoint{3.385906in}{1.179646in}}{\pgfqpoint{3.381788in}{1.175528in}}%
\pgfpathcurveto{\pgfqpoint{3.377670in}{1.171410in}}{\pgfqpoint{3.375356in}{1.165824in}}{\pgfqpoint{3.375356in}{1.160000in}}%
\pgfpathcurveto{\pgfqpoint{3.375356in}{1.154176in}}{\pgfqpoint{3.377670in}{1.148590in}}{\pgfqpoint{3.381788in}{1.144472in}}%
\pgfpathcurveto{\pgfqpoint{3.385906in}{1.140354in}}{\pgfqpoint{3.391493in}{1.138040in}}{\pgfqpoint{3.397317in}{1.138040in}}%
\pgfpathlineto{\pgfqpoint{3.397317in}{1.138040in}}%
\pgfpathclose%
\pgfusepath{stroke,fill}%
\end{pgfscope}%
\begin{pgfscope}%
\pgfpathrectangle{\pgfqpoint{1.582361in}{0.880000in}}{\pgfqpoint{5.035278in}{6.160000in}}%
\pgfusepath{clip}%
\pgfsetbuttcap%
\pgfsetroundjoin%
\definecolor{currentfill}{rgb}{0.500000,0.000000,0.500000}%
\pgfsetfillcolor{currentfill}%
\pgfsetlinewidth{1.003750pt}%
\definecolor{currentstroke}{rgb}{0.500000,0.000000,0.500000}%
\pgfsetstrokecolor{currentstroke}%
\pgfsetdash{}{0pt}%
\pgfpathmoveto{\pgfqpoint{1.863612in}{3.079124in}}%
\pgfpathcurveto{\pgfqpoint{1.869435in}{3.079124in}}{\pgfqpoint{1.875022in}{3.081438in}}{\pgfqpoint{1.879140in}{3.085556in}}%
\pgfpathcurveto{\pgfqpoint{1.883258in}{3.089674in}}{\pgfqpoint{1.885572in}{3.095261in}}{\pgfqpoint{1.885572in}{3.101085in}}%
\pgfpathcurveto{\pgfqpoint{1.885572in}{3.106908in}}{\pgfqpoint{1.883258in}{3.112495in}}{\pgfqpoint{1.879140in}{3.116613in}}%
\pgfpathcurveto{\pgfqpoint{1.875022in}{3.120731in}}{\pgfqpoint{1.869435in}{3.123045in}}{\pgfqpoint{1.863612in}{3.123045in}}%
\pgfpathcurveto{\pgfqpoint{1.857788in}{3.123045in}}{\pgfqpoint{1.852201in}{3.120731in}}{\pgfqpoint{1.848083in}{3.116613in}}%
\pgfpathcurveto{\pgfqpoint{1.843965in}{3.112495in}}{\pgfqpoint{1.841651in}{3.106908in}}{\pgfqpoint{1.841651in}{3.101085in}}%
\pgfpathcurveto{\pgfqpoint{1.841651in}{3.095261in}}{\pgfqpoint{1.843965in}{3.089674in}}{\pgfqpoint{1.848083in}{3.085556in}}%
\pgfpathcurveto{\pgfqpoint{1.852201in}{3.081438in}}{\pgfqpoint{1.857788in}{3.079124in}}{\pgfqpoint{1.863612in}{3.079124in}}%
\pgfpathlineto{\pgfqpoint{1.863612in}{3.079124in}}%
\pgfpathclose%
\pgfusepath{stroke,fill}%
\end{pgfscope}%
\begin{pgfscope}%
\pgfpathrectangle{\pgfqpoint{1.582361in}{0.880000in}}{\pgfqpoint{5.035278in}{6.160000in}}%
\pgfusepath{clip}%
\pgfsetbuttcap%
\pgfsetroundjoin%
\definecolor{currentfill}{rgb}{0.500000,0.000000,0.500000}%
\pgfsetfillcolor{currentfill}%
\pgfsetlinewidth{1.003750pt}%
\definecolor{currentstroke}{rgb}{0.500000,0.000000,0.500000}%
\pgfsetstrokecolor{currentstroke}%
\pgfsetdash{}{0pt}%
\pgfpathmoveto{\pgfqpoint{3.191628in}{4.733776in}}%
\pgfpathcurveto{\pgfqpoint{3.197452in}{4.733776in}}{\pgfqpoint{3.203038in}{4.736090in}}{\pgfqpoint{3.207157in}{4.740208in}}%
\pgfpathcurveto{\pgfqpoint{3.211275in}{4.744326in}}{\pgfqpoint{3.213589in}{4.749912in}}{\pgfqpoint{3.213589in}{4.755736in}}%
\pgfpathcurveto{\pgfqpoint{3.213589in}{4.761560in}}{\pgfqpoint{3.211275in}{4.767146in}}{\pgfqpoint{3.207157in}{4.771264in}}%
\pgfpathcurveto{\pgfqpoint{3.203038in}{4.775382in}}{\pgfqpoint{3.197452in}{4.777696in}}{\pgfqpoint{3.191628in}{4.777696in}}%
\pgfpathcurveto{\pgfqpoint{3.185804in}{4.777696in}}{\pgfqpoint{3.180218in}{4.775382in}}{\pgfqpoint{3.176100in}{4.771264in}}%
\pgfpathcurveto{\pgfqpoint{3.171982in}{4.767146in}}{\pgfqpoint{3.169668in}{4.761560in}}{\pgfqpoint{3.169668in}{4.755736in}}%
\pgfpathcurveto{\pgfqpoint{3.169668in}{4.749912in}}{\pgfqpoint{3.171982in}{4.744326in}}{\pgfqpoint{3.176100in}{4.740208in}}%
\pgfpathcurveto{\pgfqpoint{3.180218in}{4.736090in}}{\pgfqpoint{3.185804in}{4.733776in}}{\pgfqpoint{3.191628in}{4.733776in}}%
\pgfpathlineto{\pgfqpoint{3.191628in}{4.733776in}}%
\pgfpathclose%
\pgfusepath{stroke,fill}%
\end{pgfscope}%
\begin{pgfscope}%
\pgfpathrectangle{\pgfqpoint{1.582361in}{0.880000in}}{\pgfqpoint{5.035278in}{6.160000in}}%
\pgfusepath{clip}%
\pgfsetbuttcap%
\pgfsetroundjoin%
\definecolor{currentfill}{rgb}{0.500000,0.000000,0.500000}%
\pgfsetfillcolor{currentfill}%
\pgfsetlinewidth{1.003750pt}%
\definecolor{currentstroke}{rgb}{0.500000,0.000000,0.500000}%
\pgfsetstrokecolor{currentstroke}%
\pgfsetdash{}{0pt}%
\pgfpathmoveto{\pgfqpoint{2.749742in}{1.979499in}}%
\pgfpathcurveto{\pgfqpoint{2.755566in}{1.979499in}}{\pgfqpoint{2.761152in}{1.981813in}}{\pgfqpoint{2.765271in}{1.985931in}}%
\pgfpathcurveto{\pgfqpoint{2.769389in}{1.990049in}}{\pgfqpoint{2.771703in}{1.995635in}}{\pgfqpoint{2.771703in}{2.001459in}}%
\pgfpathcurveto{\pgfqpoint{2.771703in}{2.007283in}}{\pgfqpoint{2.769389in}{2.012869in}}{\pgfqpoint{2.765271in}{2.016987in}}%
\pgfpathcurveto{\pgfqpoint{2.761152in}{2.021106in}}{\pgfqpoint{2.755566in}{2.023419in}}{\pgfqpoint{2.749742in}{2.023419in}}%
\pgfpathcurveto{\pgfqpoint{2.743918in}{2.023419in}}{\pgfqpoint{2.738332in}{2.021106in}}{\pgfqpoint{2.734214in}{2.016987in}}%
\pgfpathcurveto{\pgfqpoint{2.730096in}{2.012869in}}{\pgfqpoint{2.727782in}{2.007283in}}{\pgfqpoint{2.727782in}{2.001459in}}%
\pgfpathcurveto{\pgfqpoint{2.727782in}{1.995635in}}{\pgfqpoint{2.730096in}{1.990049in}}{\pgfqpoint{2.734214in}{1.985931in}}%
\pgfpathcurveto{\pgfqpoint{2.738332in}{1.981813in}}{\pgfqpoint{2.743918in}{1.979499in}}{\pgfqpoint{2.749742in}{1.979499in}}%
\pgfpathlineto{\pgfqpoint{2.749742in}{1.979499in}}%
\pgfpathclose%
\pgfusepath{stroke,fill}%
\end{pgfscope}%
\begin{pgfscope}%
\pgfpathrectangle{\pgfqpoint{1.582361in}{0.880000in}}{\pgfqpoint{5.035278in}{6.160000in}}%
\pgfusepath{clip}%
\pgfsetbuttcap%
\pgfsetroundjoin%
\definecolor{currentfill}{rgb}{0.500000,0.000000,0.500000}%
\pgfsetfillcolor{currentfill}%
\pgfsetlinewidth{1.003750pt}%
\definecolor{currentstroke}{rgb}{0.500000,0.000000,0.500000}%
\pgfsetstrokecolor{currentstroke}%
\pgfsetdash{}{0pt}%
\pgfpathmoveto{\pgfqpoint{4.874313in}{3.077241in}}%
\pgfpathcurveto{\pgfqpoint{4.880137in}{3.077241in}}{\pgfqpoint{4.885723in}{3.079555in}}{\pgfqpoint{4.889841in}{3.083673in}}%
\pgfpathcurveto{\pgfqpoint{4.893959in}{3.087791in}}{\pgfqpoint{4.896273in}{3.093377in}}{\pgfqpoint{4.896273in}{3.099201in}}%
\pgfpathcurveto{\pgfqpoint{4.896273in}{3.105025in}}{\pgfqpoint{4.893959in}{3.110611in}}{\pgfqpoint{4.889841in}{3.114730in}}%
\pgfpathcurveto{\pgfqpoint{4.885723in}{3.118848in}}{\pgfqpoint{4.880137in}{3.121162in}}{\pgfqpoint{4.874313in}{3.121162in}}%
\pgfpathcurveto{\pgfqpoint{4.868489in}{3.121162in}}{\pgfqpoint{4.862903in}{3.118848in}}{\pgfqpoint{4.858785in}{3.114730in}}%
\pgfpathcurveto{\pgfqpoint{4.854667in}{3.110611in}}{\pgfqpoint{4.852353in}{3.105025in}}{\pgfqpoint{4.852353in}{3.099201in}}%
\pgfpathcurveto{\pgfqpoint{4.852353in}{3.093377in}}{\pgfqpoint{4.854667in}{3.087791in}}{\pgfqpoint{4.858785in}{3.083673in}}%
\pgfpathcurveto{\pgfqpoint{4.862903in}{3.079555in}}{\pgfqpoint{4.868489in}{3.077241in}}{\pgfqpoint{4.874313in}{3.077241in}}%
\pgfpathlineto{\pgfqpoint{4.874313in}{3.077241in}}%
\pgfpathclose%
\pgfusepath{stroke,fill}%
\end{pgfscope}%
\begin{pgfscope}%
\pgfpathrectangle{\pgfqpoint{1.582361in}{0.880000in}}{\pgfqpoint{5.035278in}{6.160000in}}%
\pgfusepath{clip}%
\pgfsetbuttcap%
\pgfsetroundjoin%
\definecolor{currentfill}{rgb}{0.800000,0.200000,0.200000}%
\pgfsetfillcolor{currentfill}%
\pgfsetlinewidth{1.003750pt}%
\definecolor{currentstroke}{rgb}{0.800000,0.200000,0.200000}%
\pgfsetstrokecolor{currentstroke}%
\pgfsetdash{}{0pt}%
\pgfpathmoveto{\pgfqpoint{4.772594in}{4.037235in}}%
\pgfpathcurveto{\pgfqpoint{4.778418in}{4.037235in}}{\pgfqpoint{4.784005in}{4.039549in}}{\pgfqpoint{4.788123in}{4.043667in}}%
\pgfpathcurveto{\pgfqpoint{4.792241in}{4.047785in}}{\pgfqpoint{4.794555in}{4.053371in}}{\pgfqpoint{4.794555in}{4.059195in}}%
\pgfpathcurveto{\pgfqpoint{4.794555in}{4.065019in}}{\pgfqpoint{4.792241in}{4.070605in}}{\pgfqpoint{4.788123in}{4.074724in}}%
\pgfpathcurveto{\pgfqpoint{4.784005in}{4.078842in}}{\pgfqpoint{4.778418in}{4.081156in}}{\pgfqpoint{4.772594in}{4.081156in}}%
\pgfpathcurveto{\pgfqpoint{4.766771in}{4.081156in}}{\pgfqpoint{4.761184in}{4.078842in}}{\pgfqpoint{4.757066in}{4.074724in}}%
\pgfpathcurveto{\pgfqpoint{4.752948in}{4.070605in}}{\pgfqpoint{4.750634in}{4.065019in}}{\pgfqpoint{4.750634in}{4.059195in}}%
\pgfpathcurveto{\pgfqpoint{4.750634in}{4.053371in}}{\pgfqpoint{4.752948in}{4.047785in}}{\pgfqpoint{4.757066in}{4.043667in}}%
\pgfpathcurveto{\pgfqpoint{4.761184in}{4.039549in}}{\pgfqpoint{4.766771in}{4.037235in}}{\pgfqpoint{4.772594in}{4.037235in}}%
\pgfpathlineto{\pgfqpoint{4.772594in}{4.037235in}}%
\pgfpathclose%
\pgfusepath{stroke,fill}%
\end{pgfscope}%
\begin{pgfscope}%
\pgfpathrectangle{\pgfqpoint{1.582361in}{0.880000in}}{\pgfqpoint{5.035278in}{6.160000in}}%
\pgfusepath{clip}%
\pgfsetbuttcap%
\pgfsetroundjoin%
\definecolor{currentfill}{rgb}{0.500000,0.000000,0.500000}%
\pgfsetfillcolor{currentfill}%
\pgfsetlinewidth{1.003750pt}%
\definecolor{currentstroke}{rgb}{0.500000,0.000000,0.500000}%
\pgfsetstrokecolor{currentstroke}%
\pgfsetdash{}{0pt}%
\pgfpathmoveto{\pgfqpoint{3.790650in}{1.758615in}}%
\pgfpathcurveto{\pgfqpoint{3.796474in}{1.758615in}}{\pgfqpoint{3.802060in}{1.760929in}}{\pgfqpoint{3.806178in}{1.765047in}}%
\pgfpathcurveto{\pgfqpoint{3.810296in}{1.769165in}}{\pgfqpoint{3.812610in}{1.774751in}}{\pgfqpoint{3.812610in}{1.780575in}}%
\pgfpathcurveto{\pgfqpoint{3.812610in}{1.786399in}}{\pgfqpoint{3.810296in}{1.791985in}}{\pgfqpoint{3.806178in}{1.796104in}}%
\pgfpathcurveto{\pgfqpoint{3.802060in}{1.800222in}}{\pgfqpoint{3.796474in}{1.802536in}}{\pgfqpoint{3.790650in}{1.802536in}}%
\pgfpathcurveto{\pgfqpoint{3.784826in}{1.802536in}}{\pgfqpoint{3.779240in}{1.800222in}}{\pgfqpoint{3.775121in}{1.796104in}}%
\pgfpathcurveto{\pgfqpoint{3.771003in}{1.791985in}}{\pgfqpoint{3.768689in}{1.786399in}}{\pgfqpoint{3.768689in}{1.780575in}}%
\pgfpathcurveto{\pgfqpoint{3.768689in}{1.774751in}}{\pgfqpoint{3.771003in}{1.769165in}}{\pgfqpoint{3.775121in}{1.765047in}}%
\pgfpathcurveto{\pgfqpoint{3.779240in}{1.760929in}}{\pgfqpoint{3.784826in}{1.758615in}}{\pgfqpoint{3.790650in}{1.758615in}}%
\pgfpathlineto{\pgfqpoint{3.790650in}{1.758615in}}%
\pgfpathclose%
\pgfusepath{stroke,fill}%
\end{pgfscope}%
\begin{pgfscope}%
\pgfpathrectangle{\pgfqpoint{1.582361in}{0.880000in}}{\pgfqpoint{5.035278in}{6.160000in}}%
\pgfusepath{clip}%
\pgfsetbuttcap%
\pgfsetroundjoin%
\definecolor{currentfill}{rgb}{0.500000,0.000000,0.500000}%
\pgfsetfillcolor{currentfill}%
\pgfsetlinewidth{1.003750pt}%
\definecolor{currentstroke}{rgb}{0.500000,0.000000,0.500000}%
\pgfsetstrokecolor{currentstroke}%
\pgfsetdash{}{0pt}%
\pgfpathmoveto{\pgfqpoint{5.108750in}{3.605163in}}%
\pgfpathcurveto{\pgfqpoint{5.114574in}{3.605163in}}{\pgfqpoint{5.120160in}{3.607477in}}{\pgfqpoint{5.124279in}{3.611595in}}%
\pgfpathcurveto{\pgfqpoint{5.128397in}{3.615713in}}{\pgfqpoint{5.130711in}{3.621300in}}{\pgfqpoint{5.130711in}{3.627124in}}%
\pgfpathcurveto{\pgfqpoint{5.130711in}{3.632947in}}{\pgfqpoint{5.128397in}{3.638534in}}{\pgfqpoint{5.124279in}{3.642652in}}%
\pgfpathcurveto{\pgfqpoint{5.120160in}{3.646770in}}{\pgfqpoint{5.114574in}{3.649084in}}{\pgfqpoint{5.108750in}{3.649084in}}%
\pgfpathcurveto{\pgfqpoint{5.102926in}{3.649084in}}{\pgfqpoint{5.097340in}{3.646770in}}{\pgfqpoint{5.093222in}{3.642652in}}%
\pgfpathcurveto{\pgfqpoint{5.089104in}{3.638534in}}{\pgfqpoint{5.086790in}{3.632947in}}{\pgfqpoint{5.086790in}{3.627124in}}%
\pgfpathcurveto{\pgfqpoint{5.086790in}{3.621300in}}{\pgfqpoint{5.089104in}{3.615713in}}{\pgfqpoint{5.093222in}{3.611595in}}%
\pgfpathcurveto{\pgfqpoint{5.097340in}{3.607477in}}{\pgfqpoint{5.102926in}{3.605163in}}{\pgfqpoint{5.108750in}{3.605163in}}%
\pgfpathlineto{\pgfqpoint{5.108750in}{3.605163in}}%
\pgfpathclose%
\pgfusepath{stroke,fill}%
\end{pgfscope}%
\begin{pgfscope}%
\pgfpathrectangle{\pgfqpoint{1.582361in}{0.880000in}}{\pgfqpoint{5.035278in}{6.160000in}}%
\pgfusepath{clip}%
\pgfsetbuttcap%
\pgfsetroundjoin%
\definecolor{currentfill}{rgb}{0.500000,0.000000,0.500000}%
\pgfsetfillcolor{currentfill}%
\pgfsetlinewidth{1.003750pt}%
\definecolor{currentstroke}{rgb}{0.500000,0.000000,0.500000}%
\pgfsetstrokecolor{currentstroke}%
\pgfsetdash{}{0pt}%
\pgfpathmoveto{\pgfqpoint{3.406902in}{1.680275in}}%
\pgfpathcurveto{\pgfqpoint{3.412726in}{1.680275in}}{\pgfqpoint{3.418312in}{1.682589in}}{\pgfqpoint{3.422430in}{1.686707in}}%
\pgfpathcurveto{\pgfqpoint{3.426548in}{1.690825in}}{\pgfqpoint{3.428862in}{1.696411in}}{\pgfqpoint{3.428862in}{1.702235in}}%
\pgfpathcurveto{\pgfqpoint{3.428862in}{1.708059in}}{\pgfqpoint{3.426548in}{1.713645in}}{\pgfqpoint{3.422430in}{1.717763in}}%
\pgfpathcurveto{\pgfqpoint{3.418312in}{1.721881in}}{\pgfqpoint{3.412726in}{1.724195in}}{\pgfqpoint{3.406902in}{1.724195in}}%
\pgfpathcurveto{\pgfqpoint{3.401078in}{1.724195in}}{\pgfqpoint{3.395492in}{1.721881in}}{\pgfqpoint{3.391374in}{1.717763in}}%
\pgfpathcurveto{\pgfqpoint{3.387256in}{1.713645in}}{\pgfqpoint{3.384942in}{1.708059in}}{\pgfqpoint{3.384942in}{1.702235in}}%
\pgfpathcurveto{\pgfqpoint{3.384942in}{1.696411in}}{\pgfqpoint{3.387256in}{1.690825in}}{\pgfqpoint{3.391374in}{1.686707in}}%
\pgfpathcurveto{\pgfqpoint{3.395492in}{1.682589in}}{\pgfqpoint{3.401078in}{1.680275in}}{\pgfqpoint{3.406902in}{1.680275in}}%
\pgfpathlineto{\pgfqpoint{3.406902in}{1.680275in}}%
\pgfpathclose%
\pgfusepath{stroke,fill}%
\end{pgfscope}%
\begin{pgfscope}%
\pgfpathrectangle{\pgfqpoint{1.582361in}{0.880000in}}{\pgfqpoint{5.035278in}{6.160000in}}%
\pgfusepath{clip}%
\pgfsetbuttcap%
\pgfsetroundjoin%
\definecolor{currentfill}{rgb}{0.800000,0.200000,0.200000}%
\pgfsetfillcolor{currentfill}%
\pgfsetlinewidth{1.003750pt}%
\definecolor{currentstroke}{rgb}{0.800000,0.200000,0.200000}%
\pgfsetstrokecolor{currentstroke}%
\pgfsetdash{}{0pt}%
\pgfpathmoveto{\pgfqpoint{2.464166in}{5.275333in}}%
\pgfpathcurveto{\pgfqpoint{2.469990in}{5.275333in}}{\pgfqpoint{2.475576in}{5.277647in}}{\pgfqpoint{2.479694in}{5.281765in}}%
\pgfpathcurveto{\pgfqpoint{2.483812in}{5.285883in}}{\pgfqpoint{2.486126in}{5.291469in}}{\pgfqpoint{2.486126in}{5.297293in}}%
\pgfpathcurveto{\pgfqpoint{2.486126in}{5.303117in}}{\pgfqpoint{2.483812in}{5.308704in}}{\pgfqpoint{2.479694in}{5.312822in}}%
\pgfpathcurveto{\pgfqpoint{2.475576in}{5.316940in}}{\pgfqpoint{2.469990in}{5.319254in}}{\pgfqpoint{2.464166in}{5.319254in}}%
\pgfpathcurveto{\pgfqpoint{2.458342in}{5.319254in}}{\pgfqpoint{2.452756in}{5.316940in}}{\pgfqpoint{2.448638in}{5.312822in}}%
\pgfpathcurveto{\pgfqpoint{2.444520in}{5.308704in}}{\pgfqpoint{2.442206in}{5.303117in}}{\pgfqpoint{2.442206in}{5.297293in}}%
\pgfpathcurveto{\pgfqpoint{2.442206in}{5.291469in}}{\pgfqpoint{2.444520in}{5.285883in}}{\pgfqpoint{2.448638in}{5.281765in}}%
\pgfpathcurveto{\pgfqpoint{2.452756in}{5.277647in}}{\pgfqpoint{2.458342in}{5.275333in}}{\pgfqpoint{2.464166in}{5.275333in}}%
\pgfpathlineto{\pgfqpoint{2.464166in}{5.275333in}}%
\pgfpathclose%
\pgfusepath{stroke,fill}%
\end{pgfscope}%
\begin{pgfscope}%
\pgfpathrectangle{\pgfqpoint{1.582361in}{0.880000in}}{\pgfqpoint{5.035278in}{6.160000in}}%
\pgfusepath{clip}%
\pgfsetbuttcap%
\pgfsetroundjoin%
\definecolor{currentfill}{rgb}{0.500000,0.000000,0.500000}%
\pgfsetfillcolor{currentfill}%
\pgfsetlinewidth{1.003750pt}%
\definecolor{currentstroke}{rgb}{0.500000,0.000000,0.500000}%
\pgfsetstrokecolor{currentstroke}%
\pgfsetdash{}{0pt}%
\pgfpathmoveto{\pgfqpoint{1.952463in}{3.648908in}}%
\pgfpathcurveto{\pgfqpoint{1.958286in}{3.648908in}}{\pgfqpoint{1.963873in}{3.651222in}}{\pgfqpoint{1.967991in}{3.655340in}}%
\pgfpathcurveto{\pgfqpoint{1.972109in}{3.659458in}}{\pgfqpoint{1.974423in}{3.665044in}}{\pgfqpoint{1.974423in}{3.670868in}}%
\pgfpathcurveto{\pgfqpoint{1.974423in}{3.676692in}}{\pgfqpoint{1.972109in}{3.682278in}}{\pgfqpoint{1.967991in}{3.686397in}}%
\pgfpathcurveto{\pgfqpoint{1.963873in}{3.690515in}}{\pgfqpoint{1.958286in}{3.692829in}}{\pgfqpoint{1.952463in}{3.692829in}}%
\pgfpathcurveto{\pgfqpoint{1.946639in}{3.692829in}}{\pgfqpoint{1.941052in}{3.690515in}}{\pgfqpoint{1.936934in}{3.686397in}}%
\pgfpathcurveto{\pgfqpoint{1.932816in}{3.682278in}}{\pgfqpoint{1.930502in}{3.676692in}}{\pgfqpoint{1.930502in}{3.670868in}}%
\pgfpathcurveto{\pgfqpoint{1.930502in}{3.665044in}}{\pgfqpoint{1.932816in}{3.659458in}}{\pgfqpoint{1.936934in}{3.655340in}}%
\pgfpathcurveto{\pgfqpoint{1.941052in}{3.651222in}}{\pgfqpoint{1.946639in}{3.648908in}}{\pgfqpoint{1.952463in}{3.648908in}}%
\pgfpathlineto{\pgfqpoint{1.952463in}{3.648908in}}%
\pgfpathclose%
\pgfusepath{stroke,fill}%
\end{pgfscope}%
\begin{pgfscope}%
\pgfpathrectangle{\pgfqpoint{1.582361in}{0.880000in}}{\pgfqpoint{5.035278in}{6.160000in}}%
\pgfusepath{clip}%
\pgfsetbuttcap%
\pgfsetroundjoin%
\definecolor{currentfill}{rgb}{0.500000,0.000000,0.500000}%
\pgfsetfillcolor{currentfill}%
\pgfsetlinewidth{1.003750pt}%
\definecolor{currentstroke}{rgb}{0.500000,0.000000,0.500000}%
\pgfsetstrokecolor{currentstroke}%
\pgfsetdash{}{0pt}%
\pgfpathmoveto{\pgfqpoint{1.972386in}{1.718885in}}%
\pgfpathcurveto{\pgfqpoint{1.978210in}{1.718885in}}{\pgfqpoint{1.983796in}{1.721199in}}{\pgfqpoint{1.987914in}{1.725317in}}%
\pgfpathcurveto{\pgfqpoint{1.992032in}{1.729435in}}{\pgfqpoint{1.994346in}{1.735021in}}{\pgfqpoint{1.994346in}{1.740845in}}%
\pgfpathcurveto{\pgfqpoint{1.994346in}{1.746669in}}{\pgfqpoint{1.992032in}{1.752255in}}{\pgfqpoint{1.987914in}{1.756373in}}%
\pgfpathcurveto{\pgfqpoint{1.983796in}{1.760491in}}{\pgfqpoint{1.978210in}{1.762805in}}{\pgfqpoint{1.972386in}{1.762805in}}%
\pgfpathcurveto{\pgfqpoint{1.966562in}{1.762805in}}{\pgfqpoint{1.960976in}{1.760491in}}{\pgfqpoint{1.956858in}{1.756373in}}%
\pgfpathcurveto{\pgfqpoint{1.952740in}{1.752255in}}{\pgfqpoint{1.950426in}{1.746669in}}{\pgfqpoint{1.950426in}{1.740845in}}%
\pgfpathcurveto{\pgfqpoint{1.950426in}{1.735021in}}{\pgfqpoint{1.952740in}{1.729435in}}{\pgfqpoint{1.956858in}{1.725317in}}%
\pgfpathcurveto{\pgfqpoint{1.960976in}{1.721199in}}{\pgfqpoint{1.966562in}{1.718885in}}{\pgfqpoint{1.972386in}{1.718885in}}%
\pgfpathlineto{\pgfqpoint{1.972386in}{1.718885in}}%
\pgfpathclose%
\pgfusepath{stroke,fill}%
\end{pgfscope}%
\begin{pgfscope}%
\pgfpathrectangle{\pgfqpoint{1.582361in}{0.880000in}}{\pgfqpoint{5.035278in}{6.160000in}}%
\pgfusepath{clip}%
\pgfsetbuttcap%
\pgfsetroundjoin%
\definecolor{currentfill}{rgb}{0.500000,0.000000,0.500000}%
\pgfsetfillcolor{currentfill}%
\pgfsetlinewidth{1.003750pt}%
\definecolor{currentstroke}{rgb}{0.500000,0.000000,0.500000}%
\pgfsetstrokecolor{currentstroke}%
\pgfsetdash{}{0pt}%
\pgfpathmoveto{\pgfqpoint{4.264403in}{4.606147in}}%
\pgfpathcurveto{\pgfqpoint{4.270227in}{4.606147in}}{\pgfqpoint{4.275813in}{4.608461in}}{\pgfqpoint{4.279931in}{4.612579in}}%
\pgfpathcurveto{\pgfqpoint{4.284049in}{4.616697in}}{\pgfqpoint{4.286363in}{4.622283in}}{\pgfqpoint{4.286363in}{4.628107in}}%
\pgfpathcurveto{\pgfqpoint{4.286363in}{4.633931in}}{\pgfqpoint{4.284049in}{4.639517in}}{\pgfqpoint{4.279931in}{4.643636in}}%
\pgfpathcurveto{\pgfqpoint{4.275813in}{4.647754in}}{\pgfqpoint{4.270227in}{4.650068in}}{\pgfqpoint{4.264403in}{4.650068in}}%
\pgfpathcurveto{\pgfqpoint{4.258579in}{4.650068in}}{\pgfqpoint{4.252993in}{4.647754in}}{\pgfqpoint{4.248874in}{4.643636in}}%
\pgfpathcurveto{\pgfqpoint{4.244756in}{4.639517in}}{\pgfqpoint{4.242442in}{4.633931in}}{\pgfqpoint{4.242442in}{4.628107in}}%
\pgfpathcurveto{\pgfqpoint{4.242442in}{4.622283in}}{\pgfqpoint{4.244756in}{4.616697in}}{\pgfqpoint{4.248874in}{4.612579in}}%
\pgfpathcurveto{\pgfqpoint{4.252993in}{4.608461in}}{\pgfqpoint{4.258579in}{4.606147in}}{\pgfqpoint{4.264403in}{4.606147in}}%
\pgfpathlineto{\pgfqpoint{4.264403in}{4.606147in}}%
\pgfpathclose%
\pgfusepath{stroke,fill}%
\end{pgfscope}%
\begin{pgfscope}%
\pgfpathrectangle{\pgfqpoint{1.582361in}{0.880000in}}{\pgfqpoint{5.035278in}{6.160000in}}%
\pgfusepath{clip}%
\pgfsetbuttcap%
\pgfsetroundjoin%
\definecolor{currentfill}{rgb}{0.500000,0.000000,0.500000}%
\pgfsetfillcolor{currentfill}%
\pgfsetlinewidth{1.003750pt}%
\definecolor{currentstroke}{rgb}{0.500000,0.000000,0.500000}%
\pgfsetstrokecolor{currentstroke}%
\pgfsetdash{}{0pt}%
\pgfpathmoveto{\pgfqpoint{5.569893in}{3.994592in}}%
\pgfpathcurveto{\pgfqpoint{5.575717in}{3.994592in}}{\pgfqpoint{5.581304in}{3.996906in}}{\pgfqpoint{5.585422in}{4.001024in}}%
\pgfpathcurveto{\pgfqpoint{5.589540in}{4.005142in}}{\pgfqpoint{5.591854in}{4.010728in}}{\pgfqpoint{5.591854in}{4.016552in}}%
\pgfpathcurveto{\pgfqpoint{5.591854in}{4.022376in}}{\pgfqpoint{5.589540in}{4.027962in}}{\pgfqpoint{5.585422in}{4.032080in}}%
\pgfpathcurveto{\pgfqpoint{5.581304in}{4.036198in}}{\pgfqpoint{5.575717in}{4.038512in}}{\pgfqpoint{5.569893in}{4.038512in}}%
\pgfpathcurveto{\pgfqpoint{5.564070in}{4.038512in}}{\pgfqpoint{5.558483in}{4.036198in}}{\pgfqpoint{5.554365in}{4.032080in}}%
\pgfpathcurveto{\pgfqpoint{5.550247in}{4.027962in}}{\pgfqpoint{5.547933in}{4.022376in}}{\pgfqpoint{5.547933in}{4.016552in}}%
\pgfpathcurveto{\pgfqpoint{5.547933in}{4.010728in}}{\pgfqpoint{5.550247in}{4.005142in}}{\pgfqpoint{5.554365in}{4.001024in}}%
\pgfpathcurveto{\pgfqpoint{5.558483in}{3.996906in}}{\pgfqpoint{5.564070in}{3.994592in}}{\pgfqpoint{5.569893in}{3.994592in}}%
\pgfpathlineto{\pgfqpoint{5.569893in}{3.994592in}}%
\pgfpathclose%
\pgfusepath{stroke,fill}%
\end{pgfscope}%
\begin{pgfscope}%
\pgfpathrectangle{\pgfqpoint{1.582361in}{0.880000in}}{\pgfqpoint{5.035278in}{6.160000in}}%
\pgfusepath{clip}%
\pgfsetbuttcap%
\pgfsetroundjoin%
\definecolor{currentfill}{rgb}{0.800000,0.200000,0.200000}%
\pgfsetfillcolor{currentfill}%
\pgfsetlinewidth{1.003750pt}%
\definecolor{currentstroke}{rgb}{0.800000,0.200000,0.200000}%
\pgfsetstrokecolor{currentstroke}%
\pgfsetdash{}{0pt}%
\pgfpathmoveto{\pgfqpoint{4.563669in}{3.953321in}}%
\pgfpathcurveto{\pgfqpoint{4.569493in}{3.953321in}}{\pgfqpoint{4.575079in}{3.955635in}}{\pgfqpoint{4.579197in}{3.959753in}}%
\pgfpathcurveto{\pgfqpoint{4.583315in}{3.963871in}}{\pgfqpoint{4.585629in}{3.969458in}}{\pgfqpoint{4.585629in}{3.975282in}}%
\pgfpathcurveto{\pgfqpoint{4.585629in}{3.981105in}}{\pgfqpoint{4.583315in}{3.986692in}}{\pgfqpoint{4.579197in}{3.990810in}}%
\pgfpathcurveto{\pgfqpoint{4.575079in}{3.994928in}}{\pgfqpoint{4.569493in}{3.997242in}}{\pgfqpoint{4.563669in}{3.997242in}}%
\pgfpathcurveto{\pgfqpoint{4.557845in}{3.997242in}}{\pgfqpoint{4.552259in}{3.994928in}}{\pgfqpoint{4.548141in}{3.990810in}}%
\pgfpathcurveto{\pgfqpoint{4.544023in}{3.986692in}}{\pgfqpoint{4.541709in}{3.981105in}}{\pgfqpoint{4.541709in}{3.975282in}}%
\pgfpathcurveto{\pgfqpoint{4.541709in}{3.969458in}}{\pgfqpoint{4.544023in}{3.963871in}}{\pgfqpoint{4.548141in}{3.959753in}}%
\pgfpathcurveto{\pgfqpoint{4.552259in}{3.955635in}}{\pgfqpoint{4.557845in}{3.953321in}}{\pgfqpoint{4.563669in}{3.953321in}}%
\pgfpathlineto{\pgfqpoint{4.563669in}{3.953321in}}%
\pgfpathclose%
\pgfusepath{stroke,fill}%
\end{pgfscope}%
\begin{pgfscope}%
\pgfpathrectangle{\pgfqpoint{1.582361in}{0.880000in}}{\pgfqpoint{5.035278in}{6.160000in}}%
\pgfusepath{clip}%
\pgfsetbuttcap%
\pgfsetroundjoin%
\definecolor{currentfill}{rgb}{0.500000,0.000000,0.500000}%
\pgfsetfillcolor{currentfill}%
\pgfsetlinewidth{1.003750pt}%
\definecolor{currentstroke}{rgb}{0.500000,0.000000,0.500000}%
\pgfsetstrokecolor{currentstroke}%
\pgfsetdash{}{0pt}%
\pgfpathmoveto{\pgfqpoint{3.335415in}{5.254162in}}%
\pgfpathcurveto{\pgfqpoint{3.341239in}{5.254162in}}{\pgfqpoint{3.346825in}{5.256476in}}{\pgfqpoint{3.350943in}{5.260594in}}%
\pgfpathcurveto{\pgfqpoint{3.355061in}{5.264712in}}{\pgfqpoint{3.357375in}{5.270298in}}{\pgfqpoint{3.357375in}{5.276122in}}%
\pgfpathcurveto{\pgfqpoint{3.357375in}{5.281946in}}{\pgfqpoint{3.355061in}{5.287532in}}{\pgfqpoint{3.350943in}{5.291650in}}%
\pgfpathcurveto{\pgfqpoint{3.346825in}{5.295768in}}{\pgfqpoint{3.341239in}{5.298082in}}{\pgfqpoint{3.335415in}{5.298082in}}%
\pgfpathcurveto{\pgfqpoint{3.329591in}{5.298082in}}{\pgfqpoint{3.324005in}{5.295768in}}{\pgfqpoint{3.319887in}{5.291650in}}%
\pgfpathcurveto{\pgfqpoint{3.315768in}{5.287532in}}{\pgfqpoint{3.313455in}{5.281946in}}{\pgfqpoint{3.313455in}{5.276122in}}%
\pgfpathcurveto{\pgfqpoint{3.313455in}{5.270298in}}{\pgfqpoint{3.315768in}{5.264712in}}{\pgfqpoint{3.319887in}{5.260594in}}%
\pgfpathcurveto{\pgfqpoint{3.324005in}{5.256476in}}{\pgfqpoint{3.329591in}{5.254162in}}{\pgfqpoint{3.335415in}{5.254162in}}%
\pgfpathlineto{\pgfqpoint{3.335415in}{5.254162in}}%
\pgfpathclose%
\pgfusepath{stroke,fill}%
\end{pgfscope}%
\begin{pgfscope}%
\pgfpathrectangle{\pgfqpoint{1.582361in}{0.880000in}}{\pgfqpoint{5.035278in}{6.160000in}}%
\pgfusepath{clip}%
\pgfsetbuttcap%
\pgfsetroundjoin%
\definecolor{currentfill}{rgb}{0.800000,0.200000,0.200000}%
\pgfsetfillcolor{currentfill}%
\pgfsetlinewidth{1.003750pt}%
\definecolor{currentstroke}{rgb}{0.800000,0.200000,0.200000}%
\pgfsetstrokecolor{currentstroke}%
\pgfsetdash{}{0pt}%
\pgfpathmoveto{\pgfqpoint{4.235873in}{3.707498in}}%
\pgfpathcurveto{\pgfqpoint{4.241697in}{3.707498in}}{\pgfqpoint{4.247283in}{3.709812in}}{\pgfqpoint{4.251402in}{3.713930in}}%
\pgfpathcurveto{\pgfqpoint{4.255520in}{3.718049in}}{\pgfqpoint{4.257834in}{3.723635in}}{\pgfqpoint{4.257834in}{3.729459in}}%
\pgfpathcurveto{\pgfqpoint{4.257834in}{3.735283in}}{\pgfqpoint{4.255520in}{3.740869in}}{\pgfqpoint{4.251402in}{3.744987in}}%
\pgfpathcurveto{\pgfqpoint{4.247283in}{3.749105in}}{\pgfqpoint{4.241697in}{3.751419in}}{\pgfqpoint{4.235873in}{3.751419in}}%
\pgfpathcurveto{\pgfqpoint{4.230049in}{3.751419in}}{\pgfqpoint{4.224463in}{3.749105in}}{\pgfqpoint{4.220345in}{3.744987in}}%
\pgfpathcurveto{\pgfqpoint{4.216227in}{3.740869in}}{\pgfqpoint{4.213913in}{3.735283in}}{\pgfqpoint{4.213913in}{3.729459in}}%
\pgfpathcurveto{\pgfqpoint{4.213913in}{3.723635in}}{\pgfqpoint{4.216227in}{3.718049in}}{\pgfqpoint{4.220345in}{3.713930in}}%
\pgfpathcurveto{\pgfqpoint{4.224463in}{3.709812in}}{\pgfqpoint{4.230049in}{3.707498in}}{\pgfqpoint{4.235873in}{3.707498in}}%
\pgfpathlineto{\pgfqpoint{4.235873in}{3.707498in}}%
\pgfpathclose%
\pgfusepath{stroke,fill}%
\end{pgfscope}%
\begin{pgfscope}%
\pgfpathrectangle{\pgfqpoint{1.582361in}{0.880000in}}{\pgfqpoint{5.035278in}{6.160000in}}%
\pgfusepath{clip}%
\pgfsetbuttcap%
\pgfsetroundjoin%
\definecolor{currentfill}{rgb}{0.800000,0.200000,0.200000}%
\pgfsetfillcolor{currentfill}%
\pgfsetlinewidth{1.003750pt}%
\definecolor{currentstroke}{rgb}{0.800000,0.200000,0.200000}%
\pgfsetstrokecolor{currentstroke}%
\pgfsetdash{}{0pt}%
\pgfpathmoveto{\pgfqpoint{5.767651in}{5.649807in}}%
\pgfpathcurveto{\pgfqpoint{5.773475in}{5.649807in}}{\pgfqpoint{5.779061in}{5.652120in}}{\pgfqpoint{5.783179in}{5.656239in}}%
\pgfpathcurveto{\pgfqpoint{5.787298in}{5.660357in}}{\pgfqpoint{5.789611in}{5.665943in}}{\pgfqpoint{5.789611in}{5.671767in}}%
\pgfpathcurveto{\pgfqpoint{5.789611in}{5.677591in}}{\pgfqpoint{5.787298in}{5.683177in}}{\pgfqpoint{5.783179in}{5.687295in}}%
\pgfpathcurveto{\pgfqpoint{5.779061in}{5.691413in}}{\pgfqpoint{5.773475in}{5.693727in}}{\pgfqpoint{5.767651in}{5.693727in}}%
\pgfpathcurveto{\pgfqpoint{5.761827in}{5.693727in}}{\pgfqpoint{5.756241in}{5.691413in}}{\pgfqpoint{5.752123in}{5.687295in}}%
\pgfpathcurveto{\pgfqpoint{5.748005in}{5.683177in}}{\pgfqpoint{5.745691in}{5.677591in}}{\pgfqpoint{5.745691in}{5.671767in}}%
\pgfpathcurveto{\pgfqpoint{5.745691in}{5.665943in}}{\pgfqpoint{5.748005in}{5.660357in}}{\pgfqpoint{5.752123in}{5.656239in}}%
\pgfpathcurveto{\pgfqpoint{5.756241in}{5.652120in}}{\pgfqpoint{5.761827in}{5.649807in}}{\pgfqpoint{5.767651in}{5.649807in}}%
\pgfpathlineto{\pgfqpoint{5.767651in}{5.649807in}}%
\pgfpathclose%
\pgfusepath{stroke,fill}%
\end{pgfscope}%
\begin{pgfscope}%
\pgfpathrectangle{\pgfqpoint{1.582361in}{0.880000in}}{\pgfqpoint{5.035278in}{6.160000in}}%
\pgfusepath{clip}%
\pgfsetbuttcap%
\pgfsetroundjoin%
\definecolor{currentfill}{rgb}{0.500000,0.000000,0.500000}%
\pgfsetfillcolor{currentfill}%
\pgfsetlinewidth{1.003750pt}%
\definecolor{currentstroke}{rgb}{0.500000,0.000000,0.500000}%
\pgfsetstrokecolor{currentstroke}%
\pgfsetdash{}{0pt}%
\pgfpathmoveto{\pgfqpoint{5.624008in}{2.363225in}}%
\pgfpathcurveto{\pgfqpoint{5.629832in}{2.363225in}}{\pgfqpoint{5.635418in}{2.365539in}}{\pgfqpoint{5.639536in}{2.369657in}}%
\pgfpathcurveto{\pgfqpoint{5.643655in}{2.373775in}}{\pgfqpoint{5.645968in}{2.379362in}}{\pgfqpoint{5.645968in}{2.385186in}}%
\pgfpathcurveto{\pgfqpoint{5.645968in}{2.391009in}}{\pgfqpoint{5.643655in}{2.396596in}}{\pgfqpoint{5.639536in}{2.400714in}}%
\pgfpathcurveto{\pgfqpoint{5.635418in}{2.404832in}}{\pgfqpoint{5.629832in}{2.407146in}}{\pgfqpoint{5.624008in}{2.407146in}}%
\pgfpathcurveto{\pgfqpoint{5.618184in}{2.407146in}}{\pgfqpoint{5.612598in}{2.404832in}}{\pgfqpoint{5.608480in}{2.400714in}}%
\pgfpathcurveto{\pgfqpoint{5.604362in}{2.396596in}}{\pgfqpoint{5.602048in}{2.391009in}}{\pgfqpoint{5.602048in}{2.385186in}}%
\pgfpathcurveto{\pgfqpoint{5.602048in}{2.379362in}}{\pgfqpoint{5.604362in}{2.373775in}}{\pgfqpoint{5.608480in}{2.369657in}}%
\pgfpathcurveto{\pgfqpoint{5.612598in}{2.365539in}}{\pgfqpoint{5.618184in}{2.363225in}}{\pgfqpoint{5.624008in}{2.363225in}}%
\pgfpathlineto{\pgfqpoint{5.624008in}{2.363225in}}%
\pgfpathclose%
\pgfusepath{stroke,fill}%
\end{pgfscope}%
\begin{pgfscope}%
\pgfpathrectangle{\pgfqpoint{1.582361in}{0.880000in}}{\pgfqpoint{5.035278in}{6.160000in}}%
\pgfusepath{clip}%
\pgfsetbuttcap%
\pgfsetroundjoin%
\definecolor{currentfill}{rgb}{0.500000,0.000000,0.500000}%
\pgfsetfillcolor{currentfill}%
\pgfsetlinewidth{1.003750pt}%
\definecolor{currentstroke}{rgb}{0.500000,0.000000,0.500000}%
\pgfsetstrokecolor{currentstroke}%
\pgfsetdash{}{0pt}%
\pgfpathmoveto{\pgfqpoint{4.686213in}{3.504180in}}%
\pgfpathcurveto{\pgfqpoint{4.692037in}{3.504180in}}{\pgfqpoint{4.697623in}{3.506494in}}{\pgfqpoint{4.701741in}{3.510612in}}%
\pgfpathcurveto{\pgfqpoint{4.705859in}{3.514730in}}{\pgfqpoint{4.708173in}{3.520316in}}{\pgfqpoint{4.708173in}{3.526140in}}%
\pgfpathcurveto{\pgfqpoint{4.708173in}{3.531964in}}{\pgfqpoint{4.705859in}{3.537550in}}{\pgfqpoint{4.701741in}{3.541668in}}%
\pgfpathcurveto{\pgfqpoint{4.697623in}{3.545786in}}{\pgfqpoint{4.692037in}{3.548100in}}{\pgfqpoint{4.686213in}{3.548100in}}%
\pgfpathcurveto{\pgfqpoint{4.680389in}{3.548100in}}{\pgfqpoint{4.674803in}{3.545786in}}{\pgfqpoint{4.670684in}{3.541668in}}%
\pgfpathcurveto{\pgfqpoint{4.666566in}{3.537550in}}{\pgfqpoint{4.664252in}{3.531964in}}{\pgfqpoint{4.664252in}{3.526140in}}%
\pgfpathcurveto{\pgfqpoint{4.664252in}{3.520316in}}{\pgfqpoint{4.666566in}{3.514730in}}{\pgfqpoint{4.670684in}{3.510612in}}%
\pgfpathcurveto{\pgfqpoint{4.674803in}{3.506494in}}{\pgfqpoint{4.680389in}{3.504180in}}{\pgfqpoint{4.686213in}{3.504180in}}%
\pgfpathlineto{\pgfqpoint{4.686213in}{3.504180in}}%
\pgfpathclose%
\pgfusepath{stroke,fill}%
\end{pgfscope}%
\begin{pgfscope}%
\pgfpathrectangle{\pgfqpoint{1.582361in}{0.880000in}}{\pgfqpoint{5.035278in}{6.160000in}}%
\pgfusepath{clip}%
\pgfsetbuttcap%
\pgfsetroundjoin%
\definecolor{currentfill}{rgb}{0.800000,0.200000,0.200000}%
\pgfsetfillcolor{currentfill}%
\pgfsetlinewidth{1.003750pt}%
\definecolor{currentstroke}{rgb}{0.800000,0.200000,0.200000}%
\pgfsetstrokecolor{currentstroke}%
\pgfsetdash{}{0pt}%
\pgfpathmoveto{\pgfqpoint{4.714221in}{4.494259in}}%
\pgfpathcurveto{\pgfqpoint{4.720045in}{4.494259in}}{\pgfqpoint{4.725631in}{4.496573in}}{\pgfqpoint{4.729750in}{4.500691in}}%
\pgfpathcurveto{\pgfqpoint{4.733868in}{4.504809in}}{\pgfqpoint{4.736182in}{4.510396in}}{\pgfqpoint{4.736182in}{4.516220in}}%
\pgfpathcurveto{\pgfqpoint{4.736182in}{4.522044in}}{\pgfqpoint{4.733868in}{4.527630in}}{\pgfqpoint{4.729750in}{4.531748in}}%
\pgfpathcurveto{\pgfqpoint{4.725631in}{4.535866in}}{\pgfqpoint{4.720045in}{4.538180in}}{\pgfqpoint{4.714221in}{4.538180in}}%
\pgfpathcurveto{\pgfqpoint{4.708397in}{4.538180in}}{\pgfqpoint{4.702811in}{4.535866in}}{\pgfqpoint{4.698693in}{4.531748in}}%
\pgfpathcurveto{\pgfqpoint{4.694575in}{4.527630in}}{\pgfqpoint{4.692261in}{4.522044in}}{\pgfqpoint{4.692261in}{4.516220in}}%
\pgfpathcurveto{\pgfqpoint{4.692261in}{4.510396in}}{\pgfqpoint{4.694575in}{4.504809in}}{\pgfqpoint{4.698693in}{4.500691in}}%
\pgfpathcurveto{\pgfqpoint{4.702811in}{4.496573in}}{\pgfqpoint{4.708397in}{4.494259in}}{\pgfqpoint{4.714221in}{4.494259in}}%
\pgfpathlineto{\pgfqpoint{4.714221in}{4.494259in}}%
\pgfpathclose%
\pgfusepath{stroke,fill}%
\end{pgfscope}%
\begin{pgfscope}%
\pgfpathrectangle{\pgfqpoint{1.582361in}{0.880000in}}{\pgfqpoint{5.035278in}{6.160000in}}%
\pgfusepath{clip}%
\pgfsetbuttcap%
\pgfsetroundjoin%
\definecolor{currentfill}{rgb}{0.500000,0.000000,0.500000}%
\pgfsetfillcolor{currentfill}%
\pgfsetlinewidth{1.003750pt}%
\definecolor{currentstroke}{rgb}{0.500000,0.000000,0.500000}%
\pgfsetstrokecolor{currentstroke}%
\pgfsetdash{}{0pt}%
\pgfpathmoveto{\pgfqpoint{3.932306in}{3.231744in}}%
\pgfpathcurveto{\pgfqpoint{3.938130in}{3.231744in}}{\pgfqpoint{3.943716in}{3.234058in}}{\pgfqpoint{3.947834in}{3.238176in}}%
\pgfpathcurveto{\pgfqpoint{3.951953in}{3.242294in}}{\pgfqpoint{3.954266in}{3.247881in}}{\pgfqpoint{3.954266in}{3.253705in}}%
\pgfpathcurveto{\pgfqpoint{3.954266in}{3.259528in}}{\pgfqpoint{3.951953in}{3.265115in}}{\pgfqpoint{3.947834in}{3.269233in}}%
\pgfpathcurveto{\pgfqpoint{3.943716in}{3.273351in}}{\pgfqpoint{3.938130in}{3.275665in}}{\pgfqpoint{3.932306in}{3.275665in}}%
\pgfpathcurveto{\pgfqpoint{3.926482in}{3.275665in}}{\pgfqpoint{3.920896in}{3.273351in}}{\pgfqpoint{3.916778in}{3.269233in}}%
\pgfpathcurveto{\pgfqpoint{3.912660in}{3.265115in}}{\pgfqpoint{3.910346in}{3.259528in}}{\pgfqpoint{3.910346in}{3.253705in}}%
\pgfpathcurveto{\pgfqpoint{3.910346in}{3.247881in}}{\pgfqpoint{3.912660in}{3.242294in}}{\pgfqpoint{3.916778in}{3.238176in}}%
\pgfpathcurveto{\pgfqpoint{3.920896in}{3.234058in}}{\pgfqpoint{3.926482in}{3.231744in}}{\pgfqpoint{3.932306in}{3.231744in}}%
\pgfpathlineto{\pgfqpoint{3.932306in}{3.231744in}}%
\pgfpathclose%
\pgfusepath{stroke,fill}%
\end{pgfscope}%
\begin{pgfscope}%
\pgfpathrectangle{\pgfqpoint{1.582361in}{0.880000in}}{\pgfqpoint{5.035278in}{6.160000in}}%
\pgfusepath{clip}%
\pgfsetbuttcap%
\pgfsetroundjoin%
\definecolor{currentfill}{rgb}{0.500000,0.000000,0.500000}%
\pgfsetfillcolor{currentfill}%
\pgfsetlinewidth{1.003750pt}%
\definecolor{currentstroke}{rgb}{0.500000,0.000000,0.500000}%
\pgfsetstrokecolor{currentstroke}%
\pgfsetdash{}{0pt}%
\pgfpathmoveto{\pgfqpoint{6.376325in}{5.484102in}}%
\pgfpathcurveto{\pgfqpoint{6.382149in}{5.484102in}}{\pgfqpoint{6.387735in}{5.486416in}}{\pgfqpoint{6.391853in}{5.490534in}}%
\pgfpathcurveto{\pgfqpoint{6.395972in}{5.494652in}}{\pgfqpoint{6.398285in}{5.500239in}}{\pgfqpoint{6.398285in}{5.506063in}}%
\pgfpathcurveto{\pgfqpoint{6.398285in}{5.511886in}}{\pgfqpoint{6.395972in}{5.517473in}}{\pgfqpoint{6.391853in}{5.521591in}}%
\pgfpathcurveto{\pgfqpoint{6.387735in}{5.525709in}}{\pgfqpoint{6.382149in}{5.528023in}}{\pgfqpoint{6.376325in}{5.528023in}}%
\pgfpathcurveto{\pgfqpoint{6.370501in}{5.528023in}}{\pgfqpoint{6.364915in}{5.525709in}}{\pgfqpoint{6.360797in}{5.521591in}}%
\pgfpathcurveto{\pgfqpoint{6.356679in}{5.517473in}}{\pgfqpoint{6.354365in}{5.511886in}}{\pgfqpoint{6.354365in}{5.506063in}}%
\pgfpathcurveto{\pgfqpoint{6.354365in}{5.500239in}}{\pgfqpoint{6.356679in}{5.494652in}}{\pgfqpoint{6.360797in}{5.490534in}}%
\pgfpathcurveto{\pgfqpoint{6.364915in}{5.486416in}}{\pgfqpoint{6.370501in}{5.484102in}}{\pgfqpoint{6.376325in}{5.484102in}}%
\pgfpathlineto{\pgfqpoint{6.376325in}{5.484102in}}%
\pgfpathclose%
\pgfusepath{stroke,fill}%
\end{pgfscope}%
\begin{pgfscope}%
\pgfpathrectangle{\pgfqpoint{1.582361in}{0.880000in}}{\pgfqpoint{5.035278in}{6.160000in}}%
\pgfusepath{clip}%
\pgfsetbuttcap%
\pgfsetroundjoin%
\definecolor{currentfill}{rgb}{0.500000,0.000000,0.500000}%
\pgfsetfillcolor{currentfill}%
\pgfsetlinewidth{1.003750pt}%
\definecolor{currentstroke}{rgb}{0.500000,0.000000,0.500000}%
\pgfsetstrokecolor{currentstroke}%
\pgfsetdash{}{0pt}%
\pgfpathmoveto{\pgfqpoint{5.061037in}{2.744834in}}%
\pgfpathcurveto{\pgfqpoint{5.066861in}{2.744834in}}{\pgfqpoint{5.072447in}{2.747148in}}{\pgfqpoint{5.076565in}{2.751266in}}%
\pgfpathcurveto{\pgfqpoint{5.080684in}{2.755384in}}{\pgfqpoint{5.082997in}{2.760970in}}{\pgfqpoint{5.082997in}{2.766794in}}%
\pgfpathcurveto{\pgfqpoint{5.082997in}{2.772618in}}{\pgfqpoint{5.080684in}{2.778204in}}{\pgfqpoint{5.076565in}{2.782322in}}%
\pgfpathcurveto{\pgfqpoint{5.072447in}{2.786441in}}{\pgfqpoint{5.066861in}{2.788755in}}{\pgfqpoint{5.061037in}{2.788755in}}%
\pgfpathcurveto{\pgfqpoint{5.055213in}{2.788755in}}{\pgfqpoint{5.049627in}{2.786441in}}{\pgfqpoint{5.045509in}{2.782322in}}%
\pgfpathcurveto{\pgfqpoint{5.041391in}{2.778204in}}{\pgfqpoint{5.039077in}{2.772618in}}{\pgfqpoint{5.039077in}{2.766794in}}%
\pgfpathcurveto{\pgfqpoint{5.039077in}{2.760970in}}{\pgfqpoint{5.041391in}{2.755384in}}{\pgfqpoint{5.045509in}{2.751266in}}%
\pgfpathcurveto{\pgfqpoint{5.049627in}{2.747148in}}{\pgfqpoint{5.055213in}{2.744834in}}{\pgfqpoint{5.061037in}{2.744834in}}%
\pgfpathlineto{\pgfqpoint{5.061037in}{2.744834in}}%
\pgfpathclose%
\pgfusepath{stroke,fill}%
\end{pgfscope}%
\begin{pgfscope}%
\pgfpathrectangle{\pgfqpoint{1.582361in}{0.880000in}}{\pgfqpoint{5.035278in}{6.160000in}}%
\pgfusepath{clip}%
\pgfsetbuttcap%
\pgfsetroundjoin%
\definecolor{currentfill}{rgb}{0.800000,0.200000,0.200000}%
\pgfsetfillcolor{currentfill}%
\pgfsetlinewidth{1.003750pt}%
\definecolor{currentstroke}{rgb}{0.800000,0.200000,0.200000}%
\pgfsetstrokecolor{currentstroke}%
\pgfsetdash{}{0pt}%
\pgfpathmoveto{\pgfqpoint{4.239532in}{3.652399in}}%
\pgfpathcurveto{\pgfqpoint{4.245356in}{3.652399in}}{\pgfqpoint{4.250942in}{3.654713in}}{\pgfqpoint{4.255060in}{3.658831in}}%
\pgfpathcurveto{\pgfqpoint{4.259178in}{3.662949in}}{\pgfqpoint{4.261492in}{3.668535in}}{\pgfqpoint{4.261492in}{3.674359in}}%
\pgfpathcurveto{\pgfqpoint{4.261492in}{3.680183in}}{\pgfqpoint{4.259178in}{3.685769in}}{\pgfqpoint{4.255060in}{3.689887in}}%
\pgfpathcurveto{\pgfqpoint{4.250942in}{3.694005in}}{\pgfqpoint{4.245356in}{3.696319in}}{\pgfqpoint{4.239532in}{3.696319in}}%
\pgfpathcurveto{\pgfqpoint{4.233708in}{3.696319in}}{\pgfqpoint{4.228122in}{3.694005in}}{\pgfqpoint{4.224003in}{3.689887in}}%
\pgfpathcurveto{\pgfqpoint{4.219885in}{3.685769in}}{\pgfqpoint{4.217571in}{3.680183in}}{\pgfqpoint{4.217571in}{3.674359in}}%
\pgfpathcurveto{\pgfqpoint{4.217571in}{3.668535in}}{\pgfqpoint{4.219885in}{3.662949in}}{\pgfqpoint{4.224003in}{3.658831in}}%
\pgfpathcurveto{\pgfqpoint{4.228122in}{3.654713in}}{\pgfqpoint{4.233708in}{3.652399in}}{\pgfqpoint{4.239532in}{3.652399in}}%
\pgfpathlineto{\pgfqpoint{4.239532in}{3.652399in}}%
\pgfpathclose%
\pgfusepath{stroke,fill}%
\end{pgfscope}%
\begin{pgfscope}%
\pgfpathrectangle{\pgfqpoint{1.582361in}{0.880000in}}{\pgfqpoint{5.035278in}{6.160000in}}%
\pgfusepath{clip}%
\pgfsetbuttcap%
\pgfsetroundjoin%
\definecolor{currentfill}{rgb}{0.500000,0.000000,0.500000}%
\pgfsetfillcolor{currentfill}%
\pgfsetlinewidth{1.003750pt}%
\definecolor{currentstroke}{rgb}{0.500000,0.000000,0.500000}%
\pgfsetstrokecolor{currentstroke}%
\pgfsetdash{}{0pt}%
\pgfpathmoveto{\pgfqpoint{5.946418in}{1.689620in}}%
\pgfpathcurveto{\pgfqpoint{5.952242in}{1.689620in}}{\pgfqpoint{5.957829in}{1.691934in}}{\pgfqpoint{5.961947in}{1.696052in}}%
\pgfpathcurveto{\pgfqpoint{5.966065in}{1.700171in}}{\pgfqpoint{5.968379in}{1.705757in}}{\pgfqpoint{5.968379in}{1.711581in}}%
\pgfpathcurveto{\pgfqpoint{5.968379in}{1.717405in}}{\pgfqpoint{5.966065in}{1.722991in}}{\pgfqpoint{5.961947in}{1.727109in}}%
\pgfpathcurveto{\pgfqpoint{5.957829in}{1.731227in}}{\pgfqpoint{5.952242in}{1.733541in}}{\pgfqpoint{5.946418in}{1.733541in}}%
\pgfpathcurveto{\pgfqpoint{5.940595in}{1.733541in}}{\pgfqpoint{5.935008in}{1.731227in}}{\pgfqpoint{5.930890in}{1.727109in}}%
\pgfpathcurveto{\pgfqpoint{5.926772in}{1.722991in}}{\pgfqpoint{5.924458in}{1.717405in}}{\pgfqpoint{5.924458in}{1.711581in}}%
\pgfpathcurveto{\pgfqpoint{5.924458in}{1.705757in}}{\pgfqpoint{5.926772in}{1.700171in}}{\pgfqpoint{5.930890in}{1.696052in}}%
\pgfpathcurveto{\pgfqpoint{5.935008in}{1.691934in}}{\pgfqpoint{5.940595in}{1.689620in}}{\pgfqpoint{5.946418in}{1.689620in}}%
\pgfpathlineto{\pgfqpoint{5.946418in}{1.689620in}}%
\pgfpathclose%
\pgfusepath{stroke,fill}%
\end{pgfscope}%
\begin{pgfscope}%
\pgfpathrectangle{\pgfqpoint{1.582361in}{0.880000in}}{\pgfqpoint{5.035278in}{6.160000in}}%
\pgfusepath{clip}%
\pgfsetbuttcap%
\pgfsetroundjoin%
\definecolor{currentfill}{rgb}{0.500000,0.000000,0.500000}%
\pgfsetfillcolor{currentfill}%
\pgfsetlinewidth{1.003750pt}%
\definecolor{currentstroke}{rgb}{0.500000,0.000000,0.500000}%
\pgfsetstrokecolor{currentstroke}%
\pgfsetdash{}{0pt}%
\pgfpathmoveto{\pgfqpoint{2.654472in}{3.434604in}}%
\pgfpathcurveto{\pgfqpoint{2.660296in}{3.434604in}}{\pgfqpoint{2.665882in}{3.436918in}}{\pgfqpoint{2.670000in}{3.441036in}}%
\pgfpathcurveto{\pgfqpoint{2.674118in}{3.445154in}}{\pgfqpoint{2.676432in}{3.450740in}}{\pgfqpoint{2.676432in}{3.456564in}}%
\pgfpathcurveto{\pgfqpoint{2.676432in}{3.462388in}}{\pgfqpoint{2.674118in}{3.467974in}}{\pgfqpoint{2.670000in}{3.472092in}}%
\pgfpathcurveto{\pgfqpoint{2.665882in}{3.476210in}}{\pgfqpoint{2.660296in}{3.478524in}}{\pgfqpoint{2.654472in}{3.478524in}}%
\pgfpathcurveto{\pgfqpoint{2.648648in}{3.478524in}}{\pgfqpoint{2.643062in}{3.476210in}}{\pgfqpoint{2.638944in}{3.472092in}}%
\pgfpathcurveto{\pgfqpoint{2.634826in}{3.467974in}}{\pgfqpoint{2.632512in}{3.462388in}}{\pgfqpoint{2.632512in}{3.456564in}}%
\pgfpathcurveto{\pgfqpoint{2.632512in}{3.450740in}}{\pgfqpoint{2.634826in}{3.445154in}}{\pgfqpoint{2.638944in}{3.441036in}}%
\pgfpathcurveto{\pgfqpoint{2.643062in}{3.436918in}}{\pgfqpoint{2.648648in}{3.434604in}}{\pgfqpoint{2.654472in}{3.434604in}}%
\pgfpathlineto{\pgfqpoint{2.654472in}{3.434604in}}%
\pgfpathclose%
\pgfusepath{stroke,fill}%
\end{pgfscope}%
\begin{pgfscope}%
\pgfpathrectangle{\pgfqpoint{1.582361in}{0.880000in}}{\pgfqpoint{5.035278in}{6.160000in}}%
\pgfusepath{clip}%
\pgfsetbuttcap%
\pgfsetroundjoin%
\definecolor{currentfill}{rgb}{0.500000,0.000000,0.500000}%
\pgfsetfillcolor{currentfill}%
\pgfsetlinewidth{1.003750pt}%
\definecolor{currentstroke}{rgb}{0.500000,0.000000,0.500000}%
\pgfsetstrokecolor{currentstroke}%
\pgfsetdash{}{0pt}%
\pgfpathmoveto{\pgfqpoint{2.115836in}{1.898683in}}%
\pgfpathcurveto{\pgfqpoint{2.121660in}{1.898683in}}{\pgfqpoint{2.127246in}{1.900997in}}{\pgfqpoint{2.131364in}{1.905115in}}%
\pgfpathcurveto{\pgfqpoint{2.135482in}{1.909233in}}{\pgfqpoint{2.137796in}{1.914819in}}{\pgfqpoint{2.137796in}{1.920643in}}%
\pgfpathcurveto{\pgfqpoint{2.137796in}{1.926467in}}{\pgfqpoint{2.135482in}{1.932053in}}{\pgfqpoint{2.131364in}{1.936171in}}%
\pgfpathcurveto{\pgfqpoint{2.127246in}{1.940289in}}{\pgfqpoint{2.121660in}{1.942603in}}{\pgfqpoint{2.115836in}{1.942603in}}%
\pgfpathcurveto{\pgfqpoint{2.110012in}{1.942603in}}{\pgfqpoint{2.104426in}{1.940289in}}{\pgfqpoint{2.100308in}{1.936171in}}%
\pgfpathcurveto{\pgfqpoint{2.096189in}{1.932053in}}{\pgfqpoint{2.093876in}{1.926467in}}{\pgfqpoint{2.093876in}{1.920643in}}%
\pgfpathcurveto{\pgfqpoint{2.093876in}{1.914819in}}{\pgfqpoint{2.096189in}{1.909233in}}{\pgfqpoint{2.100308in}{1.905115in}}%
\pgfpathcurveto{\pgfqpoint{2.104426in}{1.900997in}}{\pgfqpoint{2.110012in}{1.898683in}}{\pgfqpoint{2.115836in}{1.898683in}}%
\pgfpathlineto{\pgfqpoint{2.115836in}{1.898683in}}%
\pgfpathclose%
\pgfusepath{stroke,fill}%
\end{pgfscope}%
\begin{pgfscope}%
\pgfpathrectangle{\pgfqpoint{1.582361in}{0.880000in}}{\pgfqpoint{5.035278in}{6.160000in}}%
\pgfusepath{clip}%
\pgfsetbuttcap%
\pgfsetroundjoin%
\definecolor{currentfill}{rgb}{0.500000,0.000000,0.500000}%
\pgfsetfillcolor{currentfill}%
\pgfsetlinewidth{1.003750pt}%
\definecolor{currentstroke}{rgb}{0.500000,0.000000,0.500000}%
\pgfsetstrokecolor{currentstroke}%
\pgfsetdash{}{0pt}%
\pgfpathmoveto{\pgfqpoint{2.281959in}{4.509804in}}%
\pgfpathcurveto{\pgfqpoint{2.287783in}{4.509804in}}{\pgfqpoint{2.293369in}{4.512118in}}{\pgfqpoint{2.297487in}{4.516236in}}%
\pgfpathcurveto{\pgfqpoint{2.301605in}{4.520354in}}{\pgfqpoint{2.303919in}{4.525941in}}{\pgfqpoint{2.303919in}{4.531764in}}%
\pgfpathcurveto{\pgfqpoint{2.303919in}{4.537588in}}{\pgfqpoint{2.301605in}{4.543175in}}{\pgfqpoint{2.297487in}{4.547293in}}%
\pgfpathcurveto{\pgfqpoint{2.293369in}{4.551411in}}{\pgfqpoint{2.287783in}{4.553725in}}{\pgfqpoint{2.281959in}{4.553725in}}%
\pgfpathcurveto{\pgfqpoint{2.276135in}{4.553725in}}{\pgfqpoint{2.270549in}{4.551411in}}{\pgfqpoint{2.266431in}{4.547293in}}%
\pgfpathcurveto{\pgfqpoint{2.262312in}{4.543175in}}{\pgfqpoint{2.259999in}{4.537588in}}{\pgfqpoint{2.259999in}{4.531764in}}%
\pgfpathcurveto{\pgfqpoint{2.259999in}{4.525941in}}{\pgfqpoint{2.262312in}{4.520354in}}{\pgfqpoint{2.266431in}{4.516236in}}%
\pgfpathcurveto{\pgfqpoint{2.270549in}{4.512118in}}{\pgfqpoint{2.276135in}{4.509804in}}{\pgfqpoint{2.281959in}{4.509804in}}%
\pgfpathlineto{\pgfqpoint{2.281959in}{4.509804in}}%
\pgfpathclose%
\pgfusepath{stroke,fill}%
\end{pgfscope}%
\begin{pgfscope}%
\pgfpathrectangle{\pgfqpoint{1.582361in}{0.880000in}}{\pgfqpoint{5.035278in}{6.160000in}}%
\pgfusepath{clip}%
\pgfsetbuttcap%
\pgfsetroundjoin%
\definecolor{currentfill}{rgb}{0.500000,0.000000,0.500000}%
\pgfsetfillcolor{currentfill}%
\pgfsetlinewidth{1.003750pt}%
\definecolor{currentstroke}{rgb}{0.500000,0.000000,0.500000}%
\pgfsetstrokecolor{currentstroke}%
\pgfsetdash{}{0pt}%
\pgfpathmoveto{\pgfqpoint{6.138896in}{4.738590in}}%
\pgfpathcurveto{\pgfqpoint{6.144720in}{4.738590in}}{\pgfqpoint{6.150306in}{4.740904in}}{\pgfqpoint{6.154424in}{4.745022in}}%
\pgfpathcurveto{\pgfqpoint{6.158542in}{4.749140in}}{\pgfqpoint{6.160856in}{4.754726in}}{\pgfqpoint{6.160856in}{4.760550in}}%
\pgfpathcurveto{\pgfqpoint{6.160856in}{4.766374in}}{\pgfqpoint{6.158542in}{4.771960in}}{\pgfqpoint{6.154424in}{4.776079in}}%
\pgfpathcurveto{\pgfqpoint{6.150306in}{4.780197in}}{\pgfqpoint{6.144720in}{4.782511in}}{\pgfqpoint{6.138896in}{4.782511in}}%
\pgfpathcurveto{\pgfqpoint{6.133072in}{4.782511in}}{\pgfqpoint{6.127486in}{4.780197in}}{\pgfqpoint{6.123368in}{4.776079in}}%
\pgfpathcurveto{\pgfqpoint{6.119250in}{4.771960in}}{\pgfqpoint{6.116936in}{4.766374in}}{\pgfqpoint{6.116936in}{4.760550in}}%
\pgfpathcurveto{\pgfqpoint{6.116936in}{4.754726in}}{\pgfqpoint{6.119250in}{4.749140in}}{\pgfqpoint{6.123368in}{4.745022in}}%
\pgfpathcurveto{\pgfqpoint{6.127486in}{4.740904in}}{\pgfqpoint{6.133072in}{4.738590in}}{\pgfqpoint{6.138896in}{4.738590in}}%
\pgfpathlineto{\pgfqpoint{6.138896in}{4.738590in}}%
\pgfpathclose%
\pgfusepath{stroke,fill}%
\end{pgfscope}%
\begin{pgfscope}%
\pgfpathrectangle{\pgfqpoint{1.582361in}{0.880000in}}{\pgfqpoint{5.035278in}{6.160000in}}%
\pgfusepath{clip}%
\pgfsetbuttcap%
\pgfsetroundjoin%
\definecolor{currentfill}{rgb}{0.500000,0.000000,0.500000}%
\pgfsetfillcolor{currentfill}%
\pgfsetlinewidth{1.003750pt}%
\definecolor{currentstroke}{rgb}{0.500000,0.000000,0.500000}%
\pgfsetstrokecolor{currentstroke}%
\pgfsetdash{}{0pt}%
\pgfpathmoveto{\pgfqpoint{1.942274in}{2.765079in}}%
\pgfpathcurveto{\pgfqpoint{1.948098in}{2.765079in}}{\pgfqpoint{1.953684in}{2.767393in}}{\pgfqpoint{1.957802in}{2.771511in}}%
\pgfpathcurveto{\pgfqpoint{1.961920in}{2.775629in}}{\pgfqpoint{1.964234in}{2.781215in}}{\pgfqpoint{1.964234in}{2.787039in}}%
\pgfpathcurveto{\pgfqpoint{1.964234in}{2.792863in}}{\pgfqpoint{1.961920in}{2.798449in}}{\pgfqpoint{1.957802in}{2.802567in}}%
\pgfpathcurveto{\pgfqpoint{1.953684in}{2.806685in}}{\pgfqpoint{1.948098in}{2.808999in}}{\pgfqpoint{1.942274in}{2.808999in}}%
\pgfpathcurveto{\pgfqpoint{1.936450in}{2.808999in}}{\pgfqpoint{1.930864in}{2.806685in}}{\pgfqpoint{1.926745in}{2.802567in}}%
\pgfpathcurveto{\pgfqpoint{1.922627in}{2.798449in}}{\pgfqpoint{1.920313in}{2.792863in}}{\pgfqpoint{1.920313in}{2.787039in}}%
\pgfpathcurveto{\pgfqpoint{1.920313in}{2.781215in}}{\pgfqpoint{1.922627in}{2.775629in}}{\pgfqpoint{1.926745in}{2.771511in}}%
\pgfpathcurveto{\pgfqpoint{1.930864in}{2.767393in}}{\pgfqpoint{1.936450in}{2.765079in}}{\pgfqpoint{1.942274in}{2.765079in}}%
\pgfpathlineto{\pgfqpoint{1.942274in}{2.765079in}}%
\pgfpathclose%
\pgfusepath{stroke,fill}%
\end{pgfscope}%
\begin{pgfscope}%
\pgfpathrectangle{\pgfqpoint{1.582361in}{0.880000in}}{\pgfqpoint{5.035278in}{6.160000in}}%
\pgfusepath{clip}%
\pgfsetbuttcap%
\pgfsetroundjoin%
\definecolor{currentfill}{rgb}{0.500000,0.000000,0.500000}%
\pgfsetfillcolor{currentfill}%
\pgfsetlinewidth{1.003750pt}%
\definecolor{currentstroke}{rgb}{0.500000,0.000000,0.500000}%
\pgfsetstrokecolor{currentstroke}%
\pgfsetdash{}{0pt}%
\pgfpathmoveto{\pgfqpoint{4.627711in}{2.263527in}}%
\pgfpathcurveto{\pgfqpoint{4.633535in}{2.263527in}}{\pgfqpoint{4.639121in}{2.265841in}}{\pgfqpoint{4.643239in}{2.269959in}}%
\pgfpathcurveto{\pgfqpoint{4.647357in}{2.274077in}}{\pgfqpoint{4.649671in}{2.279663in}}{\pgfqpoint{4.649671in}{2.285487in}}%
\pgfpathcurveto{\pgfqpoint{4.649671in}{2.291311in}}{\pgfqpoint{4.647357in}{2.296897in}}{\pgfqpoint{4.643239in}{2.301015in}}%
\pgfpathcurveto{\pgfqpoint{4.639121in}{2.305134in}}{\pgfqpoint{4.633535in}{2.307447in}}{\pgfqpoint{4.627711in}{2.307447in}}%
\pgfpathcurveto{\pgfqpoint{4.621887in}{2.307447in}}{\pgfqpoint{4.616301in}{2.305134in}}{\pgfqpoint{4.612182in}{2.301015in}}%
\pgfpathcurveto{\pgfqpoint{4.608064in}{2.296897in}}{\pgfqpoint{4.605750in}{2.291311in}}{\pgfqpoint{4.605750in}{2.285487in}}%
\pgfpathcurveto{\pgfqpoint{4.605750in}{2.279663in}}{\pgfqpoint{4.608064in}{2.274077in}}{\pgfqpoint{4.612182in}{2.269959in}}%
\pgfpathcurveto{\pgfqpoint{4.616301in}{2.265841in}}{\pgfqpoint{4.621887in}{2.263527in}}{\pgfqpoint{4.627711in}{2.263527in}}%
\pgfpathlineto{\pgfqpoint{4.627711in}{2.263527in}}%
\pgfpathclose%
\pgfusepath{stroke,fill}%
\end{pgfscope}%
\begin{pgfscope}%
\pgfpathrectangle{\pgfqpoint{1.582361in}{0.880000in}}{\pgfqpoint{5.035278in}{6.160000in}}%
\pgfusepath{clip}%
\pgfsetbuttcap%
\pgfsetroundjoin%
\definecolor{currentfill}{rgb}{0.500000,0.000000,0.500000}%
\pgfsetfillcolor{currentfill}%
\pgfsetlinewidth{1.003750pt}%
\definecolor{currentstroke}{rgb}{0.500000,0.000000,0.500000}%
\pgfsetstrokecolor{currentstroke}%
\pgfsetdash{}{0pt}%
\pgfpathmoveto{\pgfqpoint{5.784438in}{1.620707in}}%
\pgfpathcurveto{\pgfqpoint{5.790262in}{1.620707in}}{\pgfqpoint{5.795848in}{1.623021in}}{\pgfqpoint{5.799966in}{1.627139in}}%
\pgfpathcurveto{\pgfqpoint{5.804084in}{1.631257in}}{\pgfqpoint{5.806398in}{1.636843in}}{\pgfqpoint{5.806398in}{1.642667in}}%
\pgfpathcurveto{\pgfqpoint{5.806398in}{1.648491in}}{\pgfqpoint{5.804084in}{1.654077in}}{\pgfqpoint{5.799966in}{1.658196in}}%
\pgfpathcurveto{\pgfqpoint{5.795848in}{1.662314in}}{\pgfqpoint{5.790262in}{1.664628in}}{\pgfqpoint{5.784438in}{1.664628in}}%
\pgfpathcurveto{\pgfqpoint{5.778614in}{1.664628in}}{\pgfqpoint{5.773028in}{1.662314in}}{\pgfqpoint{5.768910in}{1.658196in}}%
\pgfpathcurveto{\pgfqpoint{5.764792in}{1.654077in}}{\pgfqpoint{5.762478in}{1.648491in}}{\pgfqpoint{5.762478in}{1.642667in}}%
\pgfpathcurveto{\pgfqpoint{5.762478in}{1.636843in}}{\pgfqpoint{5.764792in}{1.631257in}}{\pgfqpoint{5.768910in}{1.627139in}}%
\pgfpathcurveto{\pgfqpoint{5.773028in}{1.623021in}}{\pgfqpoint{5.778614in}{1.620707in}}{\pgfqpoint{5.784438in}{1.620707in}}%
\pgfpathlineto{\pgfqpoint{5.784438in}{1.620707in}}%
\pgfpathclose%
\pgfusepath{stroke,fill}%
\end{pgfscope}%
\begin{pgfscope}%
\pgfpathrectangle{\pgfqpoint{1.582361in}{0.880000in}}{\pgfqpoint{5.035278in}{6.160000in}}%
\pgfusepath{clip}%
\pgfsetbuttcap%
\pgfsetroundjoin%
\definecolor{currentfill}{rgb}{0.500000,0.000000,0.500000}%
\pgfsetfillcolor{currentfill}%
\pgfsetlinewidth{1.003750pt}%
\definecolor{currentstroke}{rgb}{0.500000,0.000000,0.500000}%
\pgfsetstrokecolor{currentstroke}%
\pgfsetdash{}{0pt}%
\pgfpathmoveto{\pgfqpoint{4.092020in}{2.987411in}}%
\pgfpathcurveto{\pgfqpoint{4.097844in}{2.987411in}}{\pgfqpoint{4.103430in}{2.989725in}}{\pgfqpoint{4.107548in}{2.993843in}}%
\pgfpathcurveto{\pgfqpoint{4.111666in}{2.997961in}}{\pgfqpoint{4.113980in}{3.003548in}}{\pgfqpoint{4.113980in}{3.009371in}}%
\pgfpathcurveto{\pgfqpoint{4.113980in}{3.015195in}}{\pgfqpoint{4.111666in}{3.020782in}}{\pgfqpoint{4.107548in}{3.024900in}}%
\pgfpathcurveto{\pgfqpoint{4.103430in}{3.029018in}}{\pgfqpoint{4.097844in}{3.031332in}}{\pgfqpoint{4.092020in}{3.031332in}}%
\pgfpathcurveto{\pgfqpoint{4.086196in}{3.031332in}}{\pgfqpoint{4.080610in}{3.029018in}}{\pgfqpoint{4.076492in}{3.024900in}}%
\pgfpathcurveto{\pgfqpoint{4.072374in}{3.020782in}}{\pgfqpoint{4.070060in}{3.015195in}}{\pgfqpoint{4.070060in}{3.009371in}}%
\pgfpathcurveto{\pgfqpoint{4.070060in}{3.003548in}}{\pgfqpoint{4.072374in}{2.997961in}}{\pgfqpoint{4.076492in}{2.993843in}}%
\pgfpathcurveto{\pgfqpoint{4.080610in}{2.989725in}}{\pgfqpoint{4.086196in}{2.987411in}}{\pgfqpoint{4.092020in}{2.987411in}}%
\pgfpathlineto{\pgfqpoint{4.092020in}{2.987411in}}%
\pgfpathclose%
\pgfusepath{stroke,fill}%
\end{pgfscope}%
\begin{pgfscope}%
\pgfpathrectangle{\pgfqpoint{1.582361in}{0.880000in}}{\pgfqpoint{5.035278in}{6.160000in}}%
\pgfusepath{clip}%
\pgfsetbuttcap%
\pgfsetroundjoin%
\definecolor{currentfill}{rgb}{0.500000,0.000000,0.500000}%
\pgfsetfillcolor{currentfill}%
\pgfsetlinewidth{1.003750pt}%
\definecolor{currentstroke}{rgb}{0.500000,0.000000,0.500000}%
\pgfsetstrokecolor{currentstroke}%
\pgfsetdash{}{0pt}%
\pgfpathmoveto{\pgfqpoint{2.956637in}{2.434908in}}%
\pgfpathcurveto{\pgfqpoint{2.962461in}{2.434908in}}{\pgfqpoint{2.968047in}{2.437222in}}{\pgfqpoint{2.972166in}{2.441340in}}%
\pgfpathcurveto{\pgfqpoint{2.976284in}{2.445458in}}{\pgfqpoint{2.978598in}{2.451045in}}{\pgfqpoint{2.978598in}{2.456868in}}%
\pgfpathcurveto{\pgfqpoint{2.978598in}{2.462692in}}{\pgfqpoint{2.976284in}{2.468279in}}{\pgfqpoint{2.972166in}{2.472397in}}%
\pgfpathcurveto{\pgfqpoint{2.968047in}{2.476515in}}{\pgfqpoint{2.962461in}{2.478829in}}{\pgfqpoint{2.956637in}{2.478829in}}%
\pgfpathcurveto{\pgfqpoint{2.950813in}{2.478829in}}{\pgfqpoint{2.945227in}{2.476515in}}{\pgfqpoint{2.941109in}{2.472397in}}%
\pgfpathcurveto{\pgfqpoint{2.936991in}{2.468279in}}{\pgfqpoint{2.934677in}{2.462692in}}{\pgfqpoint{2.934677in}{2.456868in}}%
\pgfpathcurveto{\pgfqpoint{2.934677in}{2.451045in}}{\pgfqpoint{2.936991in}{2.445458in}}{\pgfqpoint{2.941109in}{2.441340in}}%
\pgfpathcurveto{\pgfqpoint{2.945227in}{2.437222in}}{\pgfqpoint{2.950813in}{2.434908in}}{\pgfqpoint{2.956637in}{2.434908in}}%
\pgfpathlineto{\pgfqpoint{2.956637in}{2.434908in}}%
\pgfpathclose%
\pgfusepath{stroke,fill}%
\end{pgfscope}%
\begin{pgfscope}%
\pgfpathrectangle{\pgfqpoint{1.582361in}{0.880000in}}{\pgfqpoint{5.035278in}{6.160000in}}%
\pgfusepath{clip}%
\pgfsetbuttcap%
\pgfsetroundjoin%
\definecolor{currentfill}{rgb}{0.500000,0.000000,0.500000}%
\pgfsetfillcolor{currentfill}%
\pgfsetlinewidth{1.003750pt}%
\definecolor{currentstroke}{rgb}{0.500000,0.000000,0.500000}%
\pgfsetstrokecolor{currentstroke}%
\pgfsetdash{}{0pt}%
\pgfpathmoveto{\pgfqpoint{5.344024in}{2.537943in}}%
\pgfpathcurveto{\pgfqpoint{5.349848in}{2.537943in}}{\pgfqpoint{5.355434in}{2.540257in}}{\pgfqpoint{5.359552in}{2.544375in}}%
\pgfpathcurveto{\pgfqpoint{5.363670in}{2.548494in}}{\pgfqpoint{5.365984in}{2.554080in}}{\pgfqpoint{5.365984in}{2.559904in}}%
\pgfpathcurveto{\pgfqpoint{5.365984in}{2.565728in}}{\pgfqpoint{5.363670in}{2.571314in}}{\pgfqpoint{5.359552in}{2.575432in}}%
\pgfpathcurveto{\pgfqpoint{5.355434in}{2.579550in}}{\pgfqpoint{5.349848in}{2.581864in}}{\pgfqpoint{5.344024in}{2.581864in}}%
\pgfpathcurveto{\pgfqpoint{5.338200in}{2.581864in}}{\pgfqpoint{5.332614in}{2.579550in}}{\pgfqpoint{5.328495in}{2.575432in}}%
\pgfpathcurveto{\pgfqpoint{5.324377in}{2.571314in}}{\pgfqpoint{5.322063in}{2.565728in}}{\pgfqpoint{5.322063in}{2.559904in}}%
\pgfpathcurveto{\pgfqpoint{5.322063in}{2.554080in}}{\pgfqpoint{5.324377in}{2.548494in}}{\pgfqpoint{5.328495in}{2.544375in}}%
\pgfpathcurveto{\pgfqpoint{5.332614in}{2.540257in}}{\pgfqpoint{5.338200in}{2.537943in}}{\pgfqpoint{5.344024in}{2.537943in}}%
\pgfpathlineto{\pgfqpoint{5.344024in}{2.537943in}}%
\pgfpathclose%
\pgfusepath{stroke,fill}%
\end{pgfscope}%
\begin{pgfscope}%
\pgfpathrectangle{\pgfqpoint{1.582361in}{0.880000in}}{\pgfqpoint{5.035278in}{6.160000in}}%
\pgfusepath{clip}%
\pgfsetbuttcap%
\pgfsetroundjoin%
\definecolor{currentfill}{rgb}{0.500000,0.000000,0.500000}%
\pgfsetfillcolor{currentfill}%
\pgfsetlinewidth{1.003750pt}%
\definecolor{currentstroke}{rgb}{0.500000,0.000000,0.500000}%
\pgfsetstrokecolor{currentstroke}%
\pgfsetdash{}{0pt}%
\pgfpathmoveto{\pgfqpoint{6.178747in}{5.695106in}}%
\pgfpathcurveto{\pgfqpoint{6.184571in}{5.695106in}}{\pgfqpoint{6.190157in}{5.697420in}}{\pgfqpoint{6.194276in}{5.701538in}}%
\pgfpathcurveto{\pgfqpoint{6.198394in}{5.705656in}}{\pgfqpoint{6.200708in}{5.711243in}}{\pgfqpoint{6.200708in}{5.717067in}}%
\pgfpathcurveto{\pgfqpoint{6.200708in}{5.722890in}}{\pgfqpoint{6.198394in}{5.728477in}}{\pgfqpoint{6.194276in}{5.732595in}}%
\pgfpathcurveto{\pgfqpoint{6.190157in}{5.736713in}}{\pgfqpoint{6.184571in}{5.739027in}}{\pgfqpoint{6.178747in}{5.739027in}}%
\pgfpathcurveto{\pgfqpoint{6.172923in}{5.739027in}}{\pgfqpoint{6.167337in}{5.736713in}}{\pgfqpoint{6.163219in}{5.732595in}}%
\pgfpathcurveto{\pgfqpoint{6.159101in}{5.728477in}}{\pgfqpoint{6.156787in}{5.722890in}}{\pgfqpoint{6.156787in}{5.717067in}}%
\pgfpathcurveto{\pgfqpoint{6.156787in}{5.711243in}}{\pgfqpoint{6.159101in}{5.705656in}}{\pgfqpoint{6.163219in}{5.701538in}}%
\pgfpathcurveto{\pgfqpoint{6.167337in}{5.697420in}}{\pgfqpoint{6.172923in}{5.695106in}}{\pgfqpoint{6.178747in}{5.695106in}}%
\pgfpathlineto{\pgfqpoint{6.178747in}{5.695106in}}%
\pgfpathclose%
\pgfusepath{stroke,fill}%
\end{pgfscope}%
\begin{pgfscope}%
\pgfpathrectangle{\pgfqpoint{1.582361in}{0.880000in}}{\pgfqpoint{5.035278in}{6.160000in}}%
\pgfusepath{clip}%
\pgfsetbuttcap%
\pgfsetroundjoin%
\definecolor{currentfill}{rgb}{0.500000,0.000000,0.500000}%
\pgfsetfillcolor{currentfill}%
\pgfsetlinewidth{1.003750pt}%
\definecolor{currentstroke}{rgb}{0.500000,0.000000,0.500000}%
\pgfsetstrokecolor{currentstroke}%
\pgfsetdash{}{0pt}%
\pgfpathmoveto{\pgfqpoint{2.416090in}{3.208492in}}%
\pgfpathcurveto{\pgfqpoint{2.421914in}{3.208492in}}{\pgfqpoint{2.427500in}{3.210806in}}{\pgfqpoint{2.431618in}{3.214924in}}%
\pgfpathcurveto{\pgfqpoint{2.435737in}{3.219042in}}{\pgfqpoint{2.438050in}{3.224628in}}{\pgfqpoint{2.438050in}{3.230452in}}%
\pgfpathcurveto{\pgfqpoint{2.438050in}{3.236276in}}{\pgfqpoint{2.435737in}{3.241862in}}{\pgfqpoint{2.431618in}{3.245980in}}%
\pgfpathcurveto{\pgfqpoint{2.427500in}{3.250098in}}{\pgfqpoint{2.421914in}{3.252412in}}{\pgfqpoint{2.416090in}{3.252412in}}%
\pgfpathcurveto{\pgfqpoint{2.410266in}{3.252412in}}{\pgfqpoint{2.404680in}{3.250098in}}{\pgfqpoint{2.400562in}{3.245980in}}%
\pgfpathcurveto{\pgfqpoint{2.396444in}{3.241862in}}{\pgfqpoint{2.394130in}{3.236276in}}{\pgfqpoint{2.394130in}{3.230452in}}%
\pgfpathcurveto{\pgfqpoint{2.394130in}{3.224628in}}{\pgfqpoint{2.396444in}{3.219042in}}{\pgfqpoint{2.400562in}{3.214924in}}%
\pgfpathcurveto{\pgfqpoint{2.404680in}{3.210806in}}{\pgfqpoint{2.410266in}{3.208492in}}{\pgfqpoint{2.416090in}{3.208492in}}%
\pgfpathlineto{\pgfqpoint{2.416090in}{3.208492in}}%
\pgfpathclose%
\pgfusepath{stroke,fill}%
\end{pgfscope}%
\begin{pgfscope}%
\pgfpathrectangle{\pgfqpoint{1.582361in}{0.880000in}}{\pgfqpoint{5.035278in}{6.160000in}}%
\pgfusepath{clip}%
\pgfsetbuttcap%
\pgfsetroundjoin%
\definecolor{currentfill}{rgb}{0.500000,0.000000,0.500000}%
\pgfsetfillcolor{currentfill}%
\pgfsetlinewidth{1.003750pt}%
\definecolor{currentstroke}{rgb}{0.500000,0.000000,0.500000}%
\pgfsetstrokecolor{currentstroke}%
\pgfsetdash{}{0pt}%
\pgfpathmoveto{\pgfqpoint{5.156996in}{2.944857in}}%
\pgfpathcurveto{\pgfqpoint{5.162820in}{2.944857in}}{\pgfqpoint{5.168406in}{2.947171in}}{\pgfqpoint{5.172524in}{2.951289in}}%
\pgfpathcurveto{\pgfqpoint{5.176643in}{2.955407in}}{\pgfqpoint{5.178956in}{2.960993in}}{\pgfqpoint{5.178956in}{2.966817in}}%
\pgfpathcurveto{\pgfqpoint{5.178956in}{2.972641in}}{\pgfqpoint{5.176643in}{2.978227in}}{\pgfqpoint{5.172524in}{2.982345in}}%
\pgfpathcurveto{\pgfqpoint{5.168406in}{2.986464in}}{\pgfqpoint{5.162820in}{2.988777in}}{\pgfqpoint{5.156996in}{2.988777in}}%
\pgfpathcurveto{\pgfqpoint{5.151172in}{2.988777in}}{\pgfqpoint{5.145586in}{2.986464in}}{\pgfqpoint{5.141468in}{2.982345in}}%
\pgfpathcurveto{\pgfqpoint{5.137350in}{2.978227in}}{\pgfqpoint{5.135036in}{2.972641in}}{\pgfqpoint{5.135036in}{2.966817in}}%
\pgfpathcurveto{\pgfqpoint{5.135036in}{2.960993in}}{\pgfqpoint{5.137350in}{2.955407in}}{\pgfqpoint{5.141468in}{2.951289in}}%
\pgfpathcurveto{\pgfqpoint{5.145586in}{2.947171in}}{\pgfqpoint{5.151172in}{2.944857in}}{\pgfqpoint{5.156996in}{2.944857in}}%
\pgfpathlineto{\pgfqpoint{5.156996in}{2.944857in}}%
\pgfpathclose%
\pgfusepath{stroke,fill}%
\end{pgfscope}%
\begin{pgfscope}%
\pgfpathrectangle{\pgfqpoint{1.582361in}{0.880000in}}{\pgfqpoint{5.035278in}{6.160000in}}%
\pgfusepath{clip}%
\pgfsetbuttcap%
\pgfsetroundjoin%
\definecolor{currentfill}{rgb}{0.500000,0.000000,0.500000}%
\pgfsetfillcolor{currentfill}%
\pgfsetlinewidth{1.003750pt}%
\definecolor{currentstroke}{rgb}{0.500000,0.000000,0.500000}%
\pgfsetstrokecolor{currentstroke}%
\pgfsetdash{}{0pt}%
\pgfpathmoveto{\pgfqpoint{5.237299in}{1.805000in}}%
\pgfpathcurveto{\pgfqpoint{5.243122in}{1.805000in}}{\pgfqpoint{5.248709in}{1.807314in}}{\pgfqpoint{5.252827in}{1.811432in}}%
\pgfpathcurveto{\pgfqpoint{5.256945in}{1.815550in}}{\pgfqpoint{5.259259in}{1.821136in}}{\pgfqpoint{5.259259in}{1.826960in}}%
\pgfpathcurveto{\pgfqpoint{5.259259in}{1.832784in}}{\pgfqpoint{5.256945in}{1.838370in}}{\pgfqpoint{5.252827in}{1.842488in}}%
\pgfpathcurveto{\pgfqpoint{5.248709in}{1.846606in}}{\pgfqpoint{5.243122in}{1.848920in}}{\pgfqpoint{5.237299in}{1.848920in}}%
\pgfpathcurveto{\pgfqpoint{5.231475in}{1.848920in}}{\pgfqpoint{5.225888in}{1.846606in}}{\pgfqpoint{5.221770in}{1.842488in}}%
\pgfpathcurveto{\pgfqpoint{5.217652in}{1.838370in}}{\pgfqpoint{5.215338in}{1.832784in}}{\pgfqpoint{5.215338in}{1.826960in}}%
\pgfpathcurveto{\pgfqpoint{5.215338in}{1.821136in}}{\pgfqpoint{5.217652in}{1.815550in}}{\pgfqpoint{5.221770in}{1.811432in}}%
\pgfpathcurveto{\pgfqpoint{5.225888in}{1.807314in}}{\pgfqpoint{5.231475in}{1.805000in}}{\pgfqpoint{5.237299in}{1.805000in}}%
\pgfpathlineto{\pgfqpoint{5.237299in}{1.805000in}}%
\pgfpathclose%
\pgfusepath{stroke,fill}%
\end{pgfscope}%
\begin{pgfscope}%
\pgfpathrectangle{\pgfqpoint{1.582361in}{0.880000in}}{\pgfqpoint{5.035278in}{6.160000in}}%
\pgfusepath{clip}%
\pgfsetbuttcap%
\pgfsetroundjoin%
\definecolor{currentfill}{rgb}{0.500000,0.000000,0.500000}%
\pgfsetfillcolor{currentfill}%
\pgfsetlinewidth{1.003750pt}%
\definecolor{currentstroke}{rgb}{0.500000,0.000000,0.500000}%
\pgfsetstrokecolor{currentstroke}%
\pgfsetdash{}{0pt}%
\pgfpathmoveto{\pgfqpoint{4.666446in}{1.146446in}}%
\pgfpathcurveto{\pgfqpoint{4.672270in}{1.146446in}}{\pgfqpoint{4.677856in}{1.148760in}}{\pgfqpoint{4.681975in}{1.152878in}}%
\pgfpathcurveto{\pgfqpoint{4.686093in}{1.156996in}}{\pgfqpoint{4.688407in}{1.162582in}}{\pgfqpoint{4.688407in}{1.168406in}}%
\pgfpathcurveto{\pgfqpoint{4.688407in}{1.174230in}}{\pgfqpoint{4.686093in}{1.179816in}}{\pgfqpoint{4.681975in}{1.183935in}}%
\pgfpathcurveto{\pgfqpoint{4.677856in}{1.188053in}}{\pgfqpoint{4.672270in}{1.190367in}}{\pgfqpoint{4.666446in}{1.190367in}}%
\pgfpathcurveto{\pgfqpoint{4.660622in}{1.190367in}}{\pgfqpoint{4.655036in}{1.188053in}}{\pgfqpoint{4.650918in}{1.183935in}}%
\pgfpathcurveto{\pgfqpoint{4.646800in}{1.179816in}}{\pgfqpoint{4.644486in}{1.174230in}}{\pgfqpoint{4.644486in}{1.168406in}}%
\pgfpathcurveto{\pgfqpoint{4.644486in}{1.162582in}}{\pgfqpoint{4.646800in}{1.156996in}}{\pgfqpoint{4.650918in}{1.152878in}}%
\pgfpathcurveto{\pgfqpoint{4.655036in}{1.148760in}}{\pgfqpoint{4.660622in}{1.146446in}}{\pgfqpoint{4.666446in}{1.146446in}}%
\pgfpathlineto{\pgfqpoint{4.666446in}{1.146446in}}%
\pgfpathclose%
\pgfusepath{stroke,fill}%
\end{pgfscope}%
\begin{pgfscope}%
\pgfpathrectangle{\pgfqpoint{1.582361in}{0.880000in}}{\pgfqpoint{5.035278in}{6.160000in}}%
\pgfusepath{clip}%
\pgfsetbuttcap%
\pgfsetroundjoin%
\definecolor{currentfill}{rgb}{0.500000,0.000000,0.500000}%
\pgfsetfillcolor{currentfill}%
\pgfsetlinewidth{1.003750pt}%
\definecolor{currentstroke}{rgb}{0.500000,0.000000,0.500000}%
\pgfsetstrokecolor{currentstroke}%
\pgfsetdash{}{0pt}%
\pgfpathmoveto{\pgfqpoint{3.174712in}{5.388611in}}%
\pgfpathcurveto{\pgfqpoint{3.180536in}{5.388611in}}{\pgfqpoint{3.186122in}{5.390925in}}{\pgfqpoint{3.190240in}{5.395043in}}%
\pgfpathcurveto{\pgfqpoint{3.194358in}{5.399161in}}{\pgfqpoint{3.196672in}{5.404747in}}{\pgfqpoint{3.196672in}{5.410571in}}%
\pgfpathcurveto{\pgfqpoint{3.196672in}{5.416395in}}{\pgfqpoint{3.194358in}{5.421981in}}{\pgfqpoint{3.190240in}{5.426099in}}%
\pgfpathcurveto{\pgfqpoint{3.186122in}{5.430217in}}{\pgfqpoint{3.180536in}{5.432531in}}{\pgfqpoint{3.174712in}{5.432531in}}%
\pgfpathcurveto{\pgfqpoint{3.168888in}{5.432531in}}{\pgfqpoint{3.163302in}{5.430217in}}{\pgfqpoint{3.159183in}{5.426099in}}%
\pgfpathcurveto{\pgfqpoint{3.155065in}{5.421981in}}{\pgfqpoint{3.152751in}{5.416395in}}{\pgfqpoint{3.152751in}{5.410571in}}%
\pgfpathcurveto{\pgfqpoint{3.152751in}{5.404747in}}{\pgfqpoint{3.155065in}{5.399161in}}{\pgfqpoint{3.159183in}{5.395043in}}%
\pgfpathcurveto{\pgfqpoint{3.163302in}{5.390925in}}{\pgfqpoint{3.168888in}{5.388611in}}{\pgfqpoint{3.174712in}{5.388611in}}%
\pgfpathlineto{\pgfqpoint{3.174712in}{5.388611in}}%
\pgfpathclose%
\pgfusepath{stroke,fill}%
\end{pgfscope}%
\begin{pgfscope}%
\pgfpathrectangle{\pgfqpoint{1.582361in}{0.880000in}}{\pgfqpoint{5.035278in}{6.160000in}}%
\pgfusepath{clip}%
\pgfsetbuttcap%
\pgfsetroundjoin%
\definecolor{currentfill}{rgb}{0.500000,0.000000,0.500000}%
\pgfsetfillcolor{currentfill}%
\pgfsetlinewidth{1.003750pt}%
\definecolor{currentstroke}{rgb}{0.500000,0.000000,0.500000}%
\pgfsetstrokecolor{currentstroke}%
\pgfsetdash{}{0pt}%
\pgfpathmoveto{\pgfqpoint{2.307186in}{1.709333in}}%
\pgfpathcurveto{\pgfqpoint{2.313010in}{1.709333in}}{\pgfqpoint{2.318596in}{1.711647in}}{\pgfqpoint{2.322715in}{1.715765in}}%
\pgfpathcurveto{\pgfqpoint{2.326833in}{1.719883in}}{\pgfqpoint{2.329147in}{1.725469in}}{\pgfqpoint{2.329147in}{1.731293in}}%
\pgfpathcurveto{\pgfqpoint{2.329147in}{1.737117in}}{\pgfqpoint{2.326833in}{1.742703in}}{\pgfqpoint{2.322715in}{1.746821in}}%
\pgfpathcurveto{\pgfqpoint{2.318596in}{1.750940in}}{\pgfqpoint{2.313010in}{1.753253in}}{\pgfqpoint{2.307186in}{1.753253in}}%
\pgfpathcurveto{\pgfqpoint{2.301362in}{1.753253in}}{\pgfqpoint{2.295776in}{1.750940in}}{\pgfqpoint{2.291658in}{1.746821in}}%
\pgfpathcurveto{\pgfqpoint{2.287540in}{1.742703in}}{\pgfqpoint{2.285226in}{1.737117in}}{\pgfqpoint{2.285226in}{1.731293in}}%
\pgfpathcurveto{\pgfqpoint{2.285226in}{1.725469in}}{\pgfqpoint{2.287540in}{1.719883in}}{\pgfqpoint{2.291658in}{1.715765in}}%
\pgfpathcurveto{\pgfqpoint{2.295776in}{1.711647in}}{\pgfqpoint{2.301362in}{1.709333in}}{\pgfqpoint{2.307186in}{1.709333in}}%
\pgfpathlineto{\pgfqpoint{2.307186in}{1.709333in}}%
\pgfpathclose%
\pgfusepath{stroke,fill}%
\end{pgfscope}%
\begin{pgfscope}%
\pgfpathrectangle{\pgfqpoint{1.582361in}{0.880000in}}{\pgfqpoint{5.035278in}{6.160000in}}%
\pgfusepath{clip}%
\pgfsetbuttcap%
\pgfsetroundjoin%
\definecolor{currentfill}{rgb}{0.500000,0.000000,0.500000}%
\pgfsetfillcolor{currentfill}%
\pgfsetlinewidth{1.003750pt}%
\definecolor{currentstroke}{rgb}{0.500000,0.000000,0.500000}%
\pgfsetstrokecolor{currentstroke}%
\pgfsetdash{}{0pt}%
\pgfpathmoveto{\pgfqpoint{1.922827in}{5.458318in}}%
\pgfpathcurveto{\pgfqpoint{1.928651in}{5.458318in}}{\pgfqpoint{1.934237in}{5.460632in}}{\pgfqpoint{1.938355in}{5.464750in}}%
\pgfpathcurveto{\pgfqpoint{1.942473in}{5.468868in}}{\pgfqpoint{1.944787in}{5.474454in}}{\pgfqpoint{1.944787in}{5.480278in}}%
\pgfpathcurveto{\pgfqpoint{1.944787in}{5.486102in}}{\pgfqpoint{1.942473in}{5.491688in}}{\pgfqpoint{1.938355in}{5.495807in}}%
\pgfpathcurveto{\pgfqpoint{1.934237in}{5.499925in}}{\pgfqpoint{1.928651in}{5.502239in}}{\pgfqpoint{1.922827in}{5.502239in}}%
\pgfpathcurveto{\pgfqpoint{1.917003in}{5.502239in}}{\pgfqpoint{1.911417in}{5.499925in}}{\pgfqpoint{1.907299in}{5.495807in}}%
\pgfpathcurveto{\pgfqpoint{1.903181in}{5.491688in}}{\pgfqpoint{1.900867in}{5.486102in}}{\pgfqpoint{1.900867in}{5.480278in}}%
\pgfpathcurveto{\pgfqpoint{1.900867in}{5.474454in}}{\pgfqpoint{1.903181in}{5.468868in}}{\pgfqpoint{1.907299in}{5.464750in}}%
\pgfpathcurveto{\pgfqpoint{1.911417in}{5.460632in}}{\pgfqpoint{1.917003in}{5.458318in}}{\pgfqpoint{1.922827in}{5.458318in}}%
\pgfpathlineto{\pgfqpoint{1.922827in}{5.458318in}}%
\pgfpathclose%
\pgfusepath{stroke,fill}%
\end{pgfscope}%
\begin{pgfscope}%
\pgfpathrectangle{\pgfqpoint{1.582361in}{0.880000in}}{\pgfqpoint{5.035278in}{6.160000in}}%
\pgfusepath{clip}%
\pgfsetbuttcap%
\pgfsetroundjoin%
\definecolor{currentfill}{rgb}{0.500000,0.000000,0.500000}%
\pgfsetfillcolor{currentfill}%
\pgfsetlinewidth{1.003750pt}%
\definecolor{currentstroke}{rgb}{0.500000,0.000000,0.500000}%
\pgfsetstrokecolor{currentstroke}%
\pgfsetdash{}{0pt}%
\pgfpathmoveto{\pgfqpoint{5.167718in}{3.804268in}}%
\pgfpathcurveto{\pgfqpoint{5.173542in}{3.804268in}}{\pgfqpoint{5.179128in}{3.806582in}}{\pgfqpoint{5.183246in}{3.810700in}}%
\pgfpathcurveto{\pgfqpoint{5.187364in}{3.814818in}}{\pgfqpoint{5.189678in}{3.820405in}}{\pgfqpoint{5.189678in}{3.826229in}}%
\pgfpathcurveto{\pgfqpoint{5.189678in}{3.832052in}}{\pgfqpoint{5.187364in}{3.837639in}}{\pgfqpoint{5.183246in}{3.841757in}}%
\pgfpathcurveto{\pgfqpoint{5.179128in}{3.845875in}}{\pgfqpoint{5.173542in}{3.848189in}}{\pgfqpoint{5.167718in}{3.848189in}}%
\pgfpathcurveto{\pgfqpoint{5.161894in}{3.848189in}}{\pgfqpoint{5.156308in}{3.845875in}}{\pgfqpoint{5.152190in}{3.841757in}}%
\pgfpathcurveto{\pgfqpoint{5.148072in}{3.837639in}}{\pgfqpoint{5.145758in}{3.832052in}}{\pgfqpoint{5.145758in}{3.826229in}}%
\pgfpathcurveto{\pgfqpoint{5.145758in}{3.820405in}}{\pgfqpoint{5.148072in}{3.814818in}}{\pgfqpoint{5.152190in}{3.810700in}}%
\pgfpathcurveto{\pgfqpoint{5.156308in}{3.806582in}}{\pgfqpoint{5.161894in}{3.804268in}}{\pgfqpoint{5.167718in}{3.804268in}}%
\pgfpathlineto{\pgfqpoint{5.167718in}{3.804268in}}%
\pgfpathclose%
\pgfusepath{stroke,fill}%
\end{pgfscope}%
\begin{pgfscope}%
\pgfpathrectangle{\pgfqpoint{1.582361in}{0.880000in}}{\pgfqpoint{5.035278in}{6.160000in}}%
\pgfusepath{clip}%
\pgfsetbuttcap%
\pgfsetroundjoin%
\definecolor{currentfill}{rgb}{0.500000,0.000000,0.500000}%
\pgfsetfillcolor{currentfill}%
\pgfsetlinewidth{1.003750pt}%
\definecolor{currentstroke}{rgb}{0.500000,0.000000,0.500000}%
\pgfsetstrokecolor{currentstroke}%
\pgfsetdash{}{0pt}%
\pgfpathmoveto{\pgfqpoint{4.629391in}{2.421489in}}%
\pgfpathcurveto{\pgfqpoint{4.635215in}{2.421489in}}{\pgfqpoint{4.640801in}{2.423803in}}{\pgfqpoint{4.644919in}{2.427921in}}%
\pgfpathcurveto{\pgfqpoint{4.649037in}{2.432040in}}{\pgfqpoint{4.651351in}{2.437626in}}{\pgfqpoint{4.651351in}{2.443450in}}%
\pgfpathcurveto{\pgfqpoint{4.651351in}{2.449274in}}{\pgfqpoint{4.649037in}{2.454860in}}{\pgfqpoint{4.644919in}{2.458978in}}%
\pgfpathcurveto{\pgfqpoint{4.640801in}{2.463096in}}{\pgfqpoint{4.635215in}{2.465410in}}{\pgfqpoint{4.629391in}{2.465410in}}%
\pgfpathcurveto{\pgfqpoint{4.623567in}{2.465410in}}{\pgfqpoint{4.617981in}{2.463096in}}{\pgfqpoint{4.613863in}{2.458978in}}%
\pgfpathcurveto{\pgfqpoint{4.609744in}{2.454860in}}{\pgfqpoint{4.607431in}{2.449274in}}{\pgfqpoint{4.607431in}{2.443450in}}%
\pgfpathcurveto{\pgfqpoint{4.607431in}{2.437626in}}{\pgfqpoint{4.609744in}{2.432040in}}{\pgfqpoint{4.613863in}{2.427921in}}%
\pgfpathcurveto{\pgfqpoint{4.617981in}{2.423803in}}{\pgfqpoint{4.623567in}{2.421489in}}{\pgfqpoint{4.629391in}{2.421489in}}%
\pgfpathlineto{\pgfqpoint{4.629391in}{2.421489in}}%
\pgfpathclose%
\pgfusepath{stroke,fill}%
\end{pgfscope}%
\begin{pgfscope}%
\pgfpathrectangle{\pgfqpoint{1.582361in}{0.880000in}}{\pgfqpoint{5.035278in}{6.160000in}}%
\pgfusepath{clip}%
\pgfsetbuttcap%
\pgfsetroundjoin%
\definecolor{currentfill}{rgb}{0.500000,0.000000,0.500000}%
\pgfsetfillcolor{currentfill}%
\pgfsetlinewidth{1.003750pt}%
\definecolor{currentstroke}{rgb}{0.500000,0.000000,0.500000}%
\pgfsetstrokecolor{currentstroke}%
\pgfsetdash{}{0pt}%
\pgfpathmoveto{\pgfqpoint{3.216456in}{1.449260in}}%
\pgfpathcurveto{\pgfqpoint{3.222280in}{1.449260in}}{\pgfqpoint{3.227867in}{1.451574in}}{\pgfqpoint{3.231985in}{1.455692in}}%
\pgfpathcurveto{\pgfqpoint{3.236103in}{1.459810in}}{\pgfqpoint{3.238417in}{1.465397in}}{\pgfqpoint{3.238417in}{1.471221in}}%
\pgfpathcurveto{\pgfqpoint{3.238417in}{1.477044in}}{\pgfqpoint{3.236103in}{1.482631in}}{\pgfqpoint{3.231985in}{1.486749in}}%
\pgfpathcurveto{\pgfqpoint{3.227867in}{1.490867in}}{\pgfqpoint{3.222280in}{1.493181in}}{\pgfqpoint{3.216456in}{1.493181in}}%
\pgfpathcurveto{\pgfqpoint{3.210633in}{1.493181in}}{\pgfqpoint{3.205046in}{1.490867in}}{\pgfqpoint{3.200928in}{1.486749in}}%
\pgfpathcurveto{\pgfqpoint{3.196810in}{1.482631in}}{\pgfqpoint{3.194496in}{1.477044in}}{\pgfqpoint{3.194496in}{1.471221in}}%
\pgfpathcurveto{\pgfqpoint{3.194496in}{1.465397in}}{\pgfqpoint{3.196810in}{1.459810in}}{\pgfqpoint{3.200928in}{1.455692in}}%
\pgfpathcurveto{\pgfqpoint{3.205046in}{1.451574in}}{\pgfqpoint{3.210633in}{1.449260in}}{\pgfqpoint{3.216456in}{1.449260in}}%
\pgfpathlineto{\pgfqpoint{3.216456in}{1.449260in}}%
\pgfpathclose%
\pgfusepath{stroke,fill}%
\end{pgfscope}%
\begin{pgfscope}%
\pgfpathrectangle{\pgfqpoint{1.582361in}{0.880000in}}{\pgfqpoint{5.035278in}{6.160000in}}%
\pgfusepath{clip}%
\pgfsetbuttcap%
\pgfsetroundjoin%
\definecolor{currentfill}{rgb}{0.500000,0.000000,0.500000}%
\pgfsetfillcolor{currentfill}%
\pgfsetlinewidth{1.003750pt}%
\definecolor{currentstroke}{rgb}{0.500000,0.000000,0.500000}%
\pgfsetstrokecolor{currentstroke}%
\pgfsetdash{}{0pt}%
\pgfpathmoveto{\pgfqpoint{2.870664in}{3.522238in}}%
\pgfpathcurveto{\pgfqpoint{2.876488in}{3.522238in}}{\pgfqpoint{2.882074in}{3.524552in}}{\pgfqpoint{2.886192in}{3.528670in}}%
\pgfpathcurveto{\pgfqpoint{2.890311in}{3.532788in}}{\pgfqpoint{2.892624in}{3.538374in}}{\pgfqpoint{2.892624in}{3.544198in}}%
\pgfpathcurveto{\pgfqpoint{2.892624in}{3.550022in}}{\pgfqpoint{2.890311in}{3.555608in}}{\pgfqpoint{2.886192in}{3.559726in}}%
\pgfpathcurveto{\pgfqpoint{2.882074in}{3.563845in}}{\pgfqpoint{2.876488in}{3.566158in}}{\pgfqpoint{2.870664in}{3.566158in}}%
\pgfpathcurveto{\pgfqpoint{2.864840in}{3.566158in}}{\pgfqpoint{2.859254in}{3.563845in}}{\pgfqpoint{2.855136in}{3.559726in}}%
\pgfpathcurveto{\pgfqpoint{2.851018in}{3.555608in}}{\pgfqpoint{2.848704in}{3.550022in}}{\pgfqpoint{2.848704in}{3.544198in}}%
\pgfpathcurveto{\pgfqpoint{2.848704in}{3.538374in}}{\pgfqpoint{2.851018in}{3.532788in}}{\pgfqpoint{2.855136in}{3.528670in}}%
\pgfpathcurveto{\pgfqpoint{2.859254in}{3.524552in}}{\pgfqpoint{2.864840in}{3.522238in}}{\pgfqpoint{2.870664in}{3.522238in}}%
\pgfpathlineto{\pgfqpoint{2.870664in}{3.522238in}}%
\pgfpathclose%
\pgfusepath{stroke,fill}%
\end{pgfscope}%
\begin{pgfscope}%
\pgfpathrectangle{\pgfqpoint{1.582361in}{0.880000in}}{\pgfqpoint{5.035278in}{6.160000in}}%
\pgfusepath{clip}%
\pgfsetbuttcap%
\pgfsetroundjoin%
\definecolor{currentfill}{rgb}{0.500000,0.000000,0.500000}%
\pgfsetfillcolor{currentfill}%
\pgfsetlinewidth{1.003750pt}%
\definecolor{currentstroke}{rgb}{0.500000,0.000000,0.500000}%
\pgfsetstrokecolor{currentstroke}%
\pgfsetdash{}{0pt}%
\pgfpathmoveto{\pgfqpoint{4.129074in}{5.484551in}}%
\pgfpathcurveto{\pgfqpoint{4.134898in}{5.484551in}}{\pgfqpoint{4.140484in}{5.486864in}}{\pgfqpoint{4.144603in}{5.490983in}}%
\pgfpathcurveto{\pgfqpoint{4.148721in}{5.495101in}}{\pgfqpoint{4.151035in}{5.500687in}}{\pgfqpoint{4.151035in}{5.506511in}}%
\pgfpathcurveto{\pgfqpoint{4.151035in}{5.512335in}}{\pgfqpoint{4.148721in}{5.517921in}}{\pgfqpoint{4.144603in}{5.522039in}}%
\pgfpathcurveto{\pgfqpoint{4.140484in}{5.526157in}}{\pgfqpoint{4.134898in}{5.528471in}}{\pgfqpoint{4.129074in}{5.528471in}}%
\pgfpathcurveto{\pgfqpoint{4.123250in}{5.528471in}}{\pgfqpoint{4.117664in}{5.526157in}}{\pgfqpoint{4.113546in}{5.522039in}}%
\pgfpathcurveto{\pgfqpoint{4.109428in}{5.517921in}}{\pgfqpoint{4.107114in}{5.512335in}}{\pgfqpoint{4.107114in}{5.506511in}}%
\pgfpathcurveto{\pgfqpoint{4.107114in}{5.500687in}}{\pgfqpoint{4.109428in}{5.495101in}}{\pgfqpoint{4.113546in}{5.490983in}}%
\pgfpathcurveto{\pgfqpoint{4.117664in}{5.486864in}}{\pgfqpoint{4.123250in}{5.484551in}}{\pgfqpoint{4.129074in}{5.484551in}}%
\pgfpathlineto{\pgfqpoint{4.129074in}{5.484551in}}%
\pgfpathclose%
\pgfusepath{stroke,fill}%
\end{pgfscope}%
\begin{pgfscope}%
\pgfpathrectangle{\pgfqpoint{1.582361in}{0.880000in}}{\pgfqpoint{5.035278in}{6.160000in}}%
\pgfusepath{clip}%
\pgfsetbuttcap%
\pgfsetroundjoin%
\definecolor{currentfill}{rgb}{0.500000,0.000000,0.500000}%
\pgfsetfillcolor{currentfill}%
\pgfsetlinewidth{1.003750pt}%
\definecolor{currentstroke}{rgb}{0.500000,0.000000,0.500000}%
\pgfsetstrokecolor{currentstroke}%
\pgfsetdash{}{0pt}%
\pgfpathmoveto{\pgfqpoint{1.811237in}{4.151983in}}%
\pgfpathcurveto{\pgfqpoint{1.817061in}{4.151983in}}{\pgfqpoint{1.822647in}{4.154297in}}{\pgfqpoint{1.826765in}{4.158415in}}%
\pgfpathcurveto{\pgfqpoint{1.830884in}{4.162533in}}{\pgfqpoint{1.833198in}{4.168119in}}{\pgfqpoint{1.833198in}{4.173943in}}%
\pgfpathcurveto{\pgfqpoint{1.833198in}{4.179767in}}{\pgfqpoint{1.830884in}{4.185353in}}{\pgfqpoint{1.826765in}{4.189471in}}%
\pgfpathcurveto{\pgfqpoint{1.822647in}{4.193590in}}{\pgfqpoint{1.817061in}{4.195903in}}{\pgfqpoint{1.811237in}{4.195903in}}%
\pgfpathcurveto{\pgfqpoint{1.805413in}{4.195903in}}{\pgfqpoint{1.799827in}{4.193590in}}{\pgfqpoint{1.795709in}{4.189471in}}%
\pgfpathcurveto{\pgfqpoint{1.791591in}{4.185353in}}{\pgfqpoint{1.789277in}{4.179767in}}{\pgfqpoint{1.789277in}{4.173943in}}%
\pgfpathcurveto{\pgfqpoint{1.789277in}{4.168119in}}{\pgfqpoint{1.791591in}{4.162533in}}{\pgfqpoint{1.795709in}{4.158415in}}%
\pgfpathcurveto{\pgfqpoint{1.799827in}{4.154297in}}{\pgfqpoint{1.805413in}{4.151983in}}{\pgfqpoint{1.811237in}{4.151983in}}%
\pgfpathlineto{\pgfqpoint{1.811237in}{4.151983in}}%
\pgfpathclose%
\pgfusepath{stroke,fill}%
\end{pgfscope}%
\begin{pgfscope}%
\pgfpathrectangle{\pgfqpoint{1.582361in}{0.880000in}}{\pgfqpoint{5.035278in}{6.160000in}}%
\pgfusepath{clip}%
\pgfsetbuttcap%
\pgfsetroundjoin%
\definecolor{currentfill}{rgb}{0.500000,0.000000,0.500000}%
\pgfsetfillcolor{currentfill}%
\pgfsetlinewidth{1.003750pt}%
\definecolor{currentstroke}{rgb}{0.500000,0.000000,0.500000}%
\pgfsetstrokecolor{currentstroke}%
\pgfsetdash{}{0pt}%
\pgfpathmoveto{\pgfqpoint{4.111868in}{1.483393in}}%
\pgfpathcurveto{\pgfqpoint{4.117692in}{1.483393in}}{\pgfqpoint{4.123278in}{1.485707in}}{\pgfqpoint{4.127396in}{1.489825in}}%
\pgfpathcurveto{\pgfqpoint{4.131515in}{1.493944in}}{\pgfqpoint{4.133828in}{1.499530in}}{\pgfqpoint{4.133828in}{1.505354in}}%
\pgfpathcurveto{\pgfqpoint{4.133828in}{1.511178in}}{\pgfqpoint{4.131515in}{1.516764in}}{\pgfqpoint{4.127396in}{1.520882in}}%
\pgfpathcurveto{\pgfqpoint{4.123278in}{1.525000in}}{\pgfqpoint{4.117692in}{1.527314in}}{\pgfqpoint{4.111868in}{1.527314in}}%
\pgfpathcurveto{\pgfqpoint{4.106044in}{1.527314in}}{\pgfqpoint{4.100458in}{1.525000in}}{\pgfqpoint{4.096340in}{1.520882in}}%
\pgfpathcurveto{\pgfqpoint{4.092222in}{1.516764in}}{\pgfqpoint{4.089908in}{1.511178in}}{\pgfqpoint{4.089908in}{1.505354in}}%
\pgfpathcurveto{\pgfqpoint{4.089908in}{1.499530in}}{\pgfqpoint{4.092222in}{1.493944in}}{\pgfqpoint{4.096340in}{1.489825in}}%
\pgfpathcurveto{\pgfqpoint{4.100458in}{1.485707in}}{\pgfqpoint{4.106044in}{1.483393in}}{\pgfqpoint{4.111868in}{1.483393in}}%
\pgfpathlineto{\pgfqpoint{4.111868in}{1.483393in}}%
\pgfpathclose%
\pgfusepath{stroke,fill}%
\end{pgfscope}%
\begin{pgfscope}%
\pgfpathrectangle{\pgfqpoint{1.582361in}{0.880000in}}{\pgfqpoint{5.035278in}{6.160000in}}%
\pgfusepath{clip}%
\pgfsetbuttcap%
\pgfsetroundjoin%
\definecolor{currentfill}{rgb}{0.500000,0.000000,0.500000}%
\pgfsetfillcolor{currentfill}%
\pgfsetlinewidth{1.003750pt}%
\definecolor{currentstroke}{rgb}{0.500000,0.000000,0.500000}%
\pgfsetstrokecolor{currentstroke}%
\pgfsetdash{}{0pt}%
\pgfpathmoveto{\pgfqpoint{3.953010in}{5.577312in}}%
\pgfpathcurveto{\pgfqpoint{3.958833in}{5.577312in}}{\pgfqpoint{3.964420in}{5.579626in}}{\pgfqpoint{3.968538in}{5.583744in}}%
\pgfpathcurveto{\pgfqpoint{3.972656in}{5.587862in}}{\pgfqpoint{3.974970in}{5.593448in}}{\pgfqpoint{3.974970in}{5.599272in}}%
\pgfpathcurveto{\pgfqpoint{3.974970in}{5.605096in}}{\pgfqpoint{3.972656in}{5.610682in}}{\pgfqpoint{3.968538in}{5.614801in}}%
\pgfpathcurveto{\pgfqpoint{3.964420in}{5.618919in}}{\pgfqpoint{3.958833in}{5.621233in}}{\pgfqpoint{3.953010in}{5.621233in}}%
\pgfpathcurveto{\pgfqpoint{3.947186in}{5.621233in}}{\pgfqpoint{3.941599in}{5.618919in}}{\pgfqpoint{3.937481in}{5.614801in}}%
\pgfpathcurveto{\pgfqpoint{3.933363in}{5.610682in}}{\pgfqpoint{3.931049in}{5.605096in}}{\pgfqpoint{3.931049in}{5.599272in}}%
\pgfpathcurveto{\pgfqpoint{3.931049in}{5.593448in}}{\pgfqpoint{3.933363in}{5.587862in}}{\pgfqpoint{3.937481in}{5.583744in}}%
\pgfpathcurveto{\pgfqpoint{3.941599in}{5.579626in}}{\pgfqpoint{3.947186in}{5.577312in}}{\pgfqpoint{3.953010in}{5.577312in}}%
\pgfpathlineto{\pgfqpoint{3.953010in}{5.577312in}}%
\pgfpathclose%
\pgfusepath{stroke,fill}%
\end{pgfscope}%
\begin{pgfscope}%
\pgfpathrectangle{\pgfqpoint{1.582361in}{0.880000in}}{\pgfqpoint{5.035278in}{6.160000in}}%
\pgfusepath{clip}%
\pgfsetbuttcap%
\pgfsetroundjoin%
\definecolor{currentfill}{rgb}{0.500000,0.000000,0.500000}%
\pgfsetfillcolor{currentfill}%
\pgfsetlinewidth{1.003750pt}%
\definecolor{currentstroke}{rgb}{0.500000,0.000000,0.500000}%
\pgfsetstrokecolor{currentstroke}%
\pgfsetdash{}{0pt}%
\pgfpathmoveto{\pgfqpoint{3.373307in}{2.182036in}}%
\pgfpathcurveto{\pgfqpoint{3.379131in}{2.182036in}}{\pgfqpoint{3.384717in}{2.184350in}}{\pgfqpoint{3.388835in}{2.188468in}}%
\pgfpathcurveto{\pgfqpoint{3.392953in}{2.192586in}}{\pgfqpoint{3.395267in}{2.198172in}}{\pgfqpoint{3.395267in}{2.203996in}}%
\pgfpathcurveto{\pgfqpoint{3.395267in}{2.209820in}}{\pgfqpoint{3.392953in}{2.215406in}}{\pgfqpoint{3.388835in}{2.219524in}}%
\pgfpathcurveto{\pgfqpoint{3.384717in}{2.223642in}}{\pgfqpoint{3.379131in}{2.225956in}}{\pgfqpoint{3.373307in}{2.225956in}}%
\pgfpathcurveto{\pgfqpoint{3.367483in}{2.225956in}}{\pgfqpoint{3.361897in}{2.223642in}}{\pgfqpoint{3.357779in}{2.219524in}}%
\pgfpathcurveto{\pgfqpoint{3.353660in}{2.215406in}}{\pgfqpoint{3.351347in}{2.209820in}}{\pgfqpoint{3.351347in}{2.203996in}}%
\pgfpathcurveto{\pgfqpoint{3.351347in}{2.198172in}}{\pgfqpoint{3.353660in}{2.192586in}}{\pgfqpoint{3.357779in}{2.188468in}}%
\pgfpathcurveto{\pgfqpoint{3.361897in}{2.184350in}}{\pgfqpoint{3.367483in}{2.182036in}}{\pgfqpoint{3.373307in}{2.182036in}}%
\pgfpathlineto{\pgfqpoint{3.373307in}{2.182036in}}%
\pgfpathclose%
\pgfusepath{stroke,fill}%
\end{pgfscope}%
\begin{pgfscope}%
\pgfpathrectangle{\pgfqpoint{1.582361in}{0.880000in}}{\pgfqpoint{5.035278in}{6.160000in}}%
\pgfusepath{clip}%
\pgfsetbuttcap%
\pgfsetroundjoin%
\definecolor{currentfill}{rgb}{0.500000,0.000000,0.500000}%
\pgfsetfillcolor{currentfill}%
\pgfsetlinewidth{1.003750pt}%
\definecolor{currentstroke}{rgb}{0.500000,0.000000,0.500000}%
\pgfsetstrokecolor{currentstroke}%
\pgfsetdash{}{0pt}%
\pgfpathmoveto{\pgfqpoint{2.946359in}{2.979713in}}%
\pgfpathcurveto{\pgfqpoint{2.952183in}{2.979713in}}{\pgfqpoint{2.957769in}{2.982027in}}{\pgfqpoint{2.961887in}{2.986145in}}%
\pgfpathcurveto{\pgfqpoint{2.966005in}{2.990263in}}{\pgfqpoint{2.968319in}{2.995849in}}{\pgfqpoint{2.968319in}{3.001673in}}%
\pgfpathcurveto{\pgfqpoint{2.968319in}{3.007497in}}{\pgfqpoint{2.966005in}{3.013084in}}{\pgfqpoint{2.961887in}{3.017202in}}%
\pgfpathcurveto{\pgfqpoint{2.957769in}{3.021320in}}{\pgfqpoint{2.952183in}{3.023634in}}{\pgfqpoint{2.946359in}{3.023634in}}%
\pgfpathcurveto{\pgfqpoint{2.940535in}{3.023634in}}{\pgfqpoint{2.934949in}{3.021320in}}{\pgfqpoint{2.930831in}{3.017202in}}%
\pgfpathcurveto{\pgfqpoint{2.926713in}{3.013084in}}{\pgfqpoint{2.924399in}{3.007497in}}{\pgfqpoint{2.924399in}{3.001673in}}%
\pgfpathcurveto{\pgfqpoint{2.924399in}{2.995849in}}{\pgfqpoint{2.926713in}{2.990263in}}{\pgfqpoint{2.930831in}{2.986145in}}%
\pgfpathcurveto{\pgfqpoint{2.934949in}{2.982027in}}{\pgfqpoint{2.940535in}{2.979713in}}{\pgfqpoint{2.946359in}{2.979713in}}%
\pgfpathlineto{\pgfqpoint{2.946359in}{2.979713in}}%
\pgfpathclose%
\pgfusepath{stroke,fill}%
\end{pgfscope}%
\begin{pgfscope}%
\pgfpathrectangle{\pgfqpoint{1.582361in}{0.880000in}}{\pgfqpoint{5.035278in}{6.160000in}}%
\pgfusepath{clip}%
\pgfsetbuttcap%
\pgfsetroundjoin%
\definecolor{currentfill}{rgb}{0.500000,0.000000,0.500000}%
\pgfsetfillcolor{currentfill}%
\pgfsetlinewidth{1.003750pt}%
\definecolor{currentstroke}{rgb}{0.500000,0.000000,0.500000}%
\pgfsetstrokecolor{currentstroke}%
\pgfsetdash{}{0pt}%
\pgfpathmoveto{\pgfqpoint{2.786100in}{3.707782in}}%
\pgfpathcurveto{\pgfqpoint{2.791924in}{3.707782in}}{\pgfqpoint{2.797510in}{3.710095in}}{\pgfqpoint{2.801629in}{3.714214in}}%
\pgfpathcurveto{\pgfqpoint{2.805747in}{3.718332in}}{\pgfqpoint{2.808061in}{3.723918in}}{\pgfqpoint{2.808061in}{3.729742in}}%
\pgfpathcurveto{\pgfqpoint{2.808061in}{3.735566in}}{\pgfqpoint{2.805747in}{3.741152in}}{\pgfqpoint{2.801629in}{3.745270in}}%
\pgfpathcurveto{\pgfqpoint{2.797510in}{3.749388in}}{\pgfqpoint{2.791924in}{3.751702in}}{\pgfqpoint{2.786100in}{3.751702in}}%
\pgfpathcurveto{\pgfqpoint{2.780276in}{3.751702in}}{\pgfqpoint{2.774690in}{3.749388in}}{\pgfqpoint{2.770572in}{3.745270in}}%
\pgfpathcurveto{\pgfqpoint{2.766454in}{3.741152in}}{\pgfqpoint{2.764140in}{3.735566in}}{\pgfqpoint{2.764140in}{3.729742in}}%
\pgfpathcurveto{\pgfqpoint{2.764140in}{3.723918in}}{\pgfqpoint{2.766454in}{3.718332in}}{\pgfqpoint{2.770572in}{3.714214in}}%
\pgfpathcurveto{\pgfqpoint{2.774690in}{3.710095in}}{\pgfqpoint{2.780276in}{3.707782in}}{\pgfqpoint{2.786100in}{3.707782in}}%
\pgfpathlineto{\pgfqpoint{2.786100in}{3.707782in}}%
\pgfpathclose%
\pgfusepath{stroke,fill}%
\end{pgfscope}%
\begin{pgfscope}%
\pgfpathrectangle{\pgfqpoint{1.582361in}{0.880000in}}{\pgfqpoint{5.035278in}{6.160000in}}%
\pgfusepath{clip}%
\pgfsetbuttcap%
\pgfsetroundjoin%
\definecolor{currentfill}{rgb}{0.500000,0.000000,0.500000}%
\pgfsetfillcolor{currentfill}%
\pgfsetlinewidth{1.003750pt}%
\definecolor{currentstroke}{rgb}{0.500000,0.000000,0.500000}%
\pgfsetstrokecolor{currentstroke}%
\pgfsetdash{}{0pt}%
\pgfpathmoveto{\pgfqpoint{3.129119in}{1.382031in}}%
\pgfpathcurveto{\pgfqpoint{3.134943in}{1.382031in}}{\pgfqpoint{3.140529in}{1.384344in}}{\pgfqpoint{3.144647in}{1.388463in}}%
\pgfpathcurveto{\pgfqpoint{3.148765in}{1.392581in}}{\pgfqpoint{3.151079in}{1.398167in}}{\pgfqpoint{3.151079in}{1.403991in}}%
\pgfpathcurveto{\pgfqpoint{3.151079in}{1.409815in}}{\pgfqpoint{3.148765in}{1.415401in}}{\pgfqpoint{3.144647in}{1.419519in}}%
\pgfpathcurveto{\pgfqpoint{3.140529in}{1.423637in}}{\pgfqpoint{3.134943in}{1.425951in}}{\pgfqpoint{3.129119in}{1.425951in}}%
\pgfpathcurveto{\pgfqpoint{3.123295in}{1.425951in}}{\pgfqpoint{3.117709in}{1.423637in}}{\pgfqpoint{3.113591in}{1.419519in}}%
\pgfpathcurveto{\pgfqpoint{3.109473in}{1.415401in}}{\pgfqpoint{3.107159in}{1.409815in}}{\pgfqpoint{3.107159in}{1.403991in}}%
\pgfpathcurveto{\pgfqpoint{3.107159in}{1.398167in}}{\pgfqpoint{3.109473in}{1.392581in}}{\pgfqpoint{3.113591in}{1.388463in}}%
\pgfpathcurveto{\pgfqpoint{3.117709in}{1.384344in}}{\pgfqpoint{3.123295in}{1.382031in}}{\pgfqpoint{3.129119in}{1.382031in}}%
\pgfpathlineto{\pgfqpoint{3.129119in}{1.382031in}}%
\pgfpathclose%
\pgfusepath{stroke,fill}%
\end{pgfscope}%
\begin{pgfscope}%
\pgfpathrectangle{\pgfqpoint{1.582361in}{0.880000in}}{\pgfqpoint{5.035278in}{6.160000in}}%
\pgfusepath{clip}%
\pgfsetbuttcap%
\pgfsetroundjoin%
\definecolor{currentfill}{rgb}{0.500000,0.000000,0.500000}%
\pgfsetfillcolor{currentfill}%
\pgfsetlinewidth{1.003750pt}%
\definecolor{currentstroke}{rgb}{0.500000,0.000000,0.500000}%
\pgfsetstrokecolor{currentstroke}%
\pgfsetdash{}{0pt}%
\pgfpathmoveto{\pgfqpoint{3.548046in}{4.651738in}}%
\pgfpathcurveto{\pgfqpoint{3.553870in}{4.651738in}}{\pgfqpoint{3.559456in}{4.654052in}}{\pgfqpoint{3.563574in}{4.658170in}}%
\pgfpathcurveto{\pgfqpoint{3.567692in}{4.662288in}}{\pgfqpoint{3.570006in}{4.667874in}}{\pgfqpoint{3.570006in}{4.673698in}}%
\pgfpathcurveto{\pgfqpoint{3.570006in}{4.679522in}}{\pgfqpoint{3.567692in}{4.685108in}}{\pgfqpoint{3.563574in}{4.689227in}}%
\pgfpathcurveto{\pgfqpoint{3.559456in}{4.693345in}}{\pgfqpoint{3.553870in}{4.695659in}}{\pgfqpoint{3.548046in}{4.695659in}}%
\pgfpathcurveto{\pgfqpoint{3.542222in}{4.695659in}}{\pgfqpoint{3.536636in}{4.693345in}}{\pgfqpoint{3.532518in}{4.689227in}}%
\pgfpathcurveto{\pgfqpoint{3.528399in}{4.685108in}}{\pgfqpoint{3.526086in}{4.679522in}}{\pgfqpoint{3.526086in}{4.673698in}}%
\pgfpathcurveto{\pgfqpoint{3.526086in}{4.667874in}}{\pgfqpoint{3.528399in}{4.662288in}}{\pgfqpoint{3.532518in}{4.658170in}}%
\pgfpathcurveto{\pgfqpoint{3.536636in}{4.654052in}}{\pgfqpoint{3.542222in}{4.651738in}}{\pgfqpoint{3.548046in}{4.651738in}}%
\pgfpathlineto{\pgfqpoint{3.548046in}{4.651738in}}%
\pgfpathclose%
\pgfusepath{stroke,fill}%
\end{pgfscope}%
\begin{pgfscope}%
\pgfpathrectangle{\pgfqpoint{1.582361in}{0.880000in}}{\pgfqpoint{5.035278in}{6.160000in}}%
\pgfusepath{clip}%
\pgfsetbuttcap%
\pgfsetroundjoin%
\definecolor{currentfill}{rgb}{0.500000,0.000000,0.500000}%
\pgfsetfillcolor{currentfill}%
\pgfsetlinewidth{1.003750pt}%
\definecolor{currentstroke}{rgb}{0.500000,0.000000,0.500000}%
\pgfsetstrokecolor{currentstroke}%
\pgfsetdash{}{0pt}%
\pgfpathmoveto{\pgfqpoint{4.219302in}{2.132100in}}%
\pgfpathcurveto{\pgfqpoint{4.225126in}{2.132100in}}{\pgfqpoint{4.230712in}{2.134414in}}{\pgfqpoint{4.234830in}{2.138532in}}%
\pgfpathcurveto{\pgfqpoint{4.238948in}{2.142651in}}{\pgfqpoint{4.241262in}{2.148237in}}{\pgfqpoint{4.241262in}{2.154061in}}%
\pgfpathcurveto{\pgfqpoint{4.241262in}{2.159885in}}{\pgfqpoint{4.238948in}{2.165471in}}{\pgfqpoint{4.234830in}{2.169589in}}%
\pgfpathcurveto{\pgfqpoint{4.230712in}{2.173707in}}{\pgfqpoint{4.225126in}{2.176021in}}{\pgfqpoint{4.219302in}{2.176021in}}%
\pgfpathcurveto{\pgfqpoint{4.213478in}{2.176021in}}{\pgfqpoint{4.207892in}{2.173707in}}{\pgfqpoint{4.203774in}{2.169589in}}%
\pgfpathcurveto{\pgfqpoint{4.199656in}{2.165471in}}{\pgfqpoint{4.197342in}{2.159885in}}{\pgfqpoint{4.197342in}{2.154061in}}%
\pgfpathcurveto{\pgfqpoint{4.197342in}{2.148237in}}{\pgfqpoint{4.199656in}{2.142651in}}{\pgfqpoint{4.203774in}{2.138532in}}%
\pgfpathcurveto{\pgfqpoint{4.207892in}{2.134414in}}{\pgfqpoint{4.213478in}{2.132100in}}{\pgfqpoint{4.219302in}{2.132100in}}%
\pgfpathlineto{\pgfqpoint{4.219302in}{2.132100in}}%
\pgfpathclose%
\pgfusepath{stroke,fill}%
\end{pgfscope}%
\begin{pgfscope}%
\pgfpathrectangle{\pgfqpoint{1.582361in}{0.880000in}}{\pgfqpoint{5.035278in}{6.160000in}}%
\pgfusepath{clip}%
\pgfsetbuttcap%
\pgfsetroundjoin%
\definecolor{currentfill}{rgb}{0.500000,0.000000,0.500000}%
\pgfsetfillcolor{currentfill}%
\pgfsetlinewidth{1.003750pt}%
\definecolor{currentstroke}{rgb}{0.500000,0.000000,0.500000}%
\pgfsetstrokecolor{currentstroke}%
\pgfsetdash{}{0pt}%
\pgfpathmoveto{\pgfqpoint{5.840021in}{5.717563in}}%
\pgfpathcurveto{\pgfqpoint{5.845845in}{5.717563in}}{\pgfqpoint{5.851431in}{5.719877in}}{\pgfqpoint{5.855549in}{5.723995in}}%
\pgfpathcurveto{\pgfqpoint{5.859667in}{5.728113in}}{\pgfqpoint{5.861981in}{5.733699in}}{\pgfqpoint{5.861981in}{5.739523in}}%
\pgfpathcurveto{\pgfqpoint{5.861981in}{5.745347in}}{\pgfqpoint{5.859667in}{5.750933in}}{\pgfqpoint{5.855549in}{5.755051in}}%
\pgfpathcurveto{\pgfqpoint{5.851431in}{5.759170in}}{\pgfqpoint{5.845845in}{5.761483in}}{\pgfqpoint{5.840021in}{5.761483in}}%
\pgfpathcurveto{\pgfqpoint{5.834197in}{5.761483in}}{\pgfqpoint{5.828611in}{5.759170in}}{\pgfqpoint{5.824492in}{5.755051in}}%
\pgfpathcurveto{\pgfqpoint{5.820374in}{5.750933in}}{\pgfqpoint{5.818060in}{5.745347in}}{\pgfqpoint{5.818060in}{5.739523in}}%
\pgfpathcurveto{\pgfqpoint{5.818060in}{5.733699in}}{\pgfqpoint{5.820374in}{5.728113in}}{\pgfqpoint{5.824492in}{5.723995in}}%
\pgfpathcurveto{\pgfqpoint{5.828611in}{5.719877in}}{\pgfqpoint{5.834197in}{5.717563in}}{\pgfqpoint{5.840021in}{5.717563in}}%
\pgfpathlineto{\pgfqpoint{5.840021in}{5.717563in}}%
\pgfpathclose%
\pgfusepath{stroke,fill}%
\end{pgfscope}%
\begin{pgfscope}%
\pgfpathrectangle{\pgfqpoint{1.582361in}{0.880000in}}{\pgfqpoint{5.035278in}{6.160000in}}%
\pgfusepath{clip}%
\pgfsetbuttcap%
\pgfsetroundjoin%
\definecolor{currentfill}{rgb}{0.500000,0.000000,0.500000}%
\pgfsetfillcolor{currentfill}%
\pgfsetlinewidth{1.003750pt}%
\definecolor{currentstroke}{rgb}{0.500000,0.000000,0.500000}%
\pgfsetstrokecolor{currentstroke}%
\pgfsetdash{}{0pt}%
\pgfpathmoveto{\pgfqpoint{1.848380in}{1.727521in}}%
\pgfpathcurveto{\pgfqpoint{1.854204in}{1.727521in}}{\pgfqpoint{1.859790in}{1.729835in}}{\pgfqpoint{1.863909in}{1.733953in}}%
\pgfpathcurveto{\pgfqpoint{1.868027in}{1.738072in}}{\pgfqpoint{1.870341in}{1.743658in}}{\pgfqpoint{1.870341in}{1.749482in}}%
\pgfpathcurveto{\pgfqpoint{1.870341in}{1.755306in}}{\pgfqpoint{1.868027in}{1.760892in}}{\pgfqpoint{1.863909in}{1.765010in}}%
\pgfpathcurveto{\pgfqpoint{1.859790in}{1.769128in}}{\pgfqpoint{1.854204in}{1.771442in}}{\pgfqpoint{1.848380in}{1.771442in}}%
\pgfpathcurveto{\pgfqpoint{1.842556in}{1.771442in}}{\pgfqpoint{1.836970in}{1.769128in}}{\pgfqpoint{1.832852in}{1.765010in}}%
\pgfpathcurveto{\pgfqpoint{1.828734in}{1.760892in}}{\pgfqpoint{1.826420in}{1.755306in}}{\pgfqpoint{1.826420in}{1.749482in}}%
\pgfpathcurveto{\pgfqpoint{1.826420in}{1.743658in}}{\pgfqpoint{1.828734in}{1.738072in}}{\pgfqpoint{1.832852in}{1.733953in}}%
\pgfpathcurveto{\pgfqpoint{1.836970in}{1.729835in}}{\pgfqpoint{1.842556in}{1.727521in}}{\pgfqpoint{1.848380in}{1.727521in}}%
\pgfpathlineto{\pgfqpoint{1.848380in}{1.727521in}}%
\pgfpathclose%
\pgfusepath{stroke,fill}%
\end{pgfscope}%
\begin{pgfscope}%
\pgfpathrectangle{\pgfqpoint{1.582361in}{0.880000in}}{\pgfqpoint{5.035278in}{6.160000in}}%
\pgfusepath{clip}%
\pgfsetbuttcap%
\pgfsetroundjoin%
\definecolor{currentfill}{rgb}{0.500000,0.000000,0.500000}%
\pgfsetfillcolor{currentfill}%
\pgfsetlinewidth{1.003750pt}%
\definecolor{currentstroke}{rgb}{0.500000,0.000000,0.500000}%
\pgfsetstrokecolor{currentstroke}%
\pgfsetdash{}{0pt}%
\pgfpathmoveto{\pgfqpoint{2.094655in}{3.697225in}}%
\pgfpathcurveto{\pgfqpoint{2.100478in}{3.697225in}}{\pgfqpoint{2.106065in}{3.699539in}}{\pgfqpoint{2.110183in}{3.703657in}}%
\pgfpathcurveto{\pgfqpoint{2.114301in}{3.707775in}}{\pgfqpoint{2.116615in}{3.713361in}}{\pgfqpoint{2.116615in}{3.719185in}}%
\pgfpathcurveto{\pgfqpoint{2.116615in}{3.725009in}}{\pgfqpoint{2.114301in}{3.730595in}}{\pgfqpoint{2.110183in}{3.734713in}}%
\pgfpathcurveto{\pgfqpoint{2.106065in}{3.738831in}}{\pgfqpoint{2.100478in}{3.741145in}}{\pgfqpoint{2.094655in}{3.741145in}}%
\pgfpathcurveto{\pgfqpoint{2.088831in}{3.741145in}}{\pgfqpoint{2.083244in}{3.738831in}}{\pgfqpoint{2.079126in}{3.734713in}}%
\pgfpathcurveto{\pgfqpoint{2.075008in}{3.730595in}}{\pgfqpoint{2.072694in}{3.725009in}}{\pgfqpoint{2.072694in}{3.719185in}}%
\pgfpathcurveto{\pgfqpoint{2.072694in}{3.713361in}}{\pgfqpoint{2.075008in}{3.707775in}}{\pgfqpoint{2.079126in}{3.703657in}}%
\pgfpathcurveto{\pgfqpoint{2.083244in}{3.699539in}}{\pgfqpoint{2.088831in}{3.697225in}}{\pgfqpoint{2.094655in}{3.697225in}}%
\pgfpathlineto{\pgfqpoint{2.094655in}{3.697225in}}%
\pgfpathclose%
\pgfusepath{stroke,fill}%
\end{pgfscope}%
\begin{pgfscope}%
\pgfpathrectangle{\pgfqpoint{1.582361in}{0.880000in}}{\pgfqpoint{5.035278in}{6.160000in}}%
\pgfusepath{clip}%
\pgfsetbuttcap%
\pgfsetroundjoin%
\definecolor{currentfill}{rgb}{0.800000,0.200000,0.200000}%
\pgfsetfillcolor{currentfill}%
\pgfsetlinewidth{1.003750pt}%
\definecolor{currentstroke}{rgb}{0.800000,0.200000,0.200000}%
\pgfsetstrokecolor{currentstroke}%
\pgfsetdash{}{0pt}%
\pgfpathmoveto{\pgfqpoint{5.804007in}{4.960227in}}%
\pgfpathcurveto{\pgfqpoint{5.809830in}{4.960227in}}{\pgfqpoint{5.815417in}{4.962541in}}{\pgfqpoint{5.819535in}{4.966659in}}%
\pgfpathcurveto{\pgfqpoint{5.823653in}{4.970777in}}{\pgfqpoint{5.825967in}{4.976363in}}{\pgfqpoint{5.825967in}{4.982187in}}%
\pgfpathcurveto{\pgfqpoint{5.825967in}{4.988011in}}{\pgfqpoint{5.823653in}{4.993597in}}{\pgfqpoint{5.819535in}{4.997716in}}%
\pgfpathcurveto{\pgfqpoint{5.815417in}{5.001834in}}{\pgfqpoint{5.809830in}{5.004148in}}{\pgfqpoint{5.804007in}{5.004148in}}%
\pgfpathcurveto{\pgfqpoint{5.798183in}{5.004148in}}{\pgfqpoint{5.792596in}{5.001834in}}{\pgfqpoint{5.788478in}{4.997716in}}%
\pgfpathcurveto{\pgfqpoint{5.784360in}{4.993597in}}{\pgfqpoint{5.782046in}{4.988011in}}{\pgfqpoint{5.782046in}{4.982187in}}%
\pgfpathcurveto{\pgfqpoint{5.782046in}{4.976363in}}{\pgfqpoint{5.784360in}{4.970777in}}{\pgfqpoint{5.788478in}{4.966659in}}%
\pgfpathcurveto{\pgfqpoint{5.792596in}{4.962541in}}{\pgfqpoint{5.798183in}{4.960227in}}{\pgfqpoint{5.804007in}{4.960227in}}%
\pgfpathlineto{\pgfqpoint{5.804007in}{4.960227in}}%
\pgfpathclose%
\pgfusepath{stroke,fill}%
\end{pgfscope}%
\begin{pgfscope}%
\pgfpathrectangle{\pgfqpoint{1.582361in}{0.880000in}}{\pgfqpoint{5.035278in}{6.160000in}}%
\pgfusepath{clip}%
\pgfsetbuttcap%
\pgfsetroundjoin%
\definecolor{currentfill}{rgb}{0.500000,0.000000,0.500000}%
\pgfsetfillcolor{currentfill}%
\pgfsetlinewidth{1.003750pt}%
\definecolor{currentstroke}{rgb}{0.500000,0.000000,0.500000}%
\pgfsetstrokecolor{currentstroke}%
\pgfsetdash{}{0pt}%
\pgfpathmoveto{\pgfqpoint{4.675914in}{2.973356in}}%
\pgfpathcurveto{\pgfqpoint{4.681738in}{2.973356in}}{\pgfqpoint{4.687324in}{2.975669in}}{\pgfqpoint{4.691442in}{2.979788in}}%
\pgfpathcurveto{\pgfqpoint{4.695560in}{2.983906in}}{\pgfqpoint{4.697874in}{2.989492in}}{\pgfqpoint{4.697874in}{2.995316in}}%
\pgfpathcurveto{\pgfqpoint{4.697874in}{3.001140in}}{\pgfqpoint{4.695560in}{3.006726in}}{\pgfqpoint{4.691442in}{3.010844in}}%
\pgfpathcurveto{\pgfqpoint{4.687324in}{3.014962in}}{\pgfqpoint{4.681738in}{3.017276in}}{\pgfqpoint{4.675914in}{3.017276in}}%
\pgfpathcurveto{\pgfqpoint{4.670090in}{3.017276in}}{\pgfqpoint{4.664504in}{3.014962in}}{\pgfqpoint{4.660386in}{3.010844in}}%
\pgfpathcurveto{\pgfqpoint{4.656268in}{3.006726in}}{\pgfqpoint{4.653954in}{3.001140in}}{\pgfqpoint{4.653954in}{2.995316in}}%
\pgfpathcurveto{\pgfqpoint{4.653954in}{2.989492in}}{\pgfqpoint{4.656268in}{2.983906in}}{\pgfqpoint{4.660386in}{2.979788in}}%
\pgfpathcurveto{\pgfqpoint{4.664504in}{2.975669in}}{\pgfqpoint{4.670090in}{2.973356in}}{\pgfqpoint{4.675914in}{2.973356in}}%
\pgfpathlineto{\pgfqpoint{4.675914in}{2.973356in}}%
\pgfpathclose%
\pgfusepath{stroke,fill}%
\end{pgfscope}%
\begin{pgfscope}%
\pgfpathrectangle{\pgfqpoint{1.582361in}{0.880000in}}{\pgfqpoint{5.035278in}{6.160000in}}%
\pgfusepath{clip}%
\pgfsetbuttcap%
\pgfsetroundjoin%
\definecolor{currentfill}{rgb}{0.500000,0.000000,0.500000}%
\pgfsetfillcolor{currentfill}%
\pgfsetlinewidth{1.003750pt}%
\definecolor{currentstroke}{rgb}{0.500000,0.000000,0.500000}%
\pgfsetstrokecolor{currentstroke}%
\pgfsetdash{}{0pt}%
\pgfpathmoveto{\pgfqpoint{1.895530in}{4.792788in}}%
\pgfpathcurveto{\pgfqpoint{1.901354in}{4.792788in}}{\pgfqpoint{1.906940in}{4.795102in}}{\pgfqpoint{1.911058in}{4.799220in}}%
\pgfpathcurveto{\pgfqpoint{1.915177in}{4.803338in}}{\pgfqpoint{1.917491in}{4.808924in}}{\pgfqpoint{1.917491in}{4.814748in}}%
\pgfpathcurveto{\pgfqpoint{1.917491in}{4.820572in}}{\pgfqpoint{1.915177in}{4.826158in}}{\pgfqpoint{1.911058in}{4.830276in}}%
\pgfpathcurveto{\pgfqpoint{1.906940in}{4.834395in}}{\pgfqpoint{1.901354in}{4.836708in}}{\pgfqpoint{1.895530in}{4.836708in}}%
\pgfpathcurveto{\pgfqpoint{1.889706in}{4.836708in}}{\pgfqpoint{1.884120in}{4.834395in}}{\pgfqpoint{1.880002in}{4.830276in}}%
\pgfpathcurveto{\pgfqpoint{1.875884in}{4.826158in}}{\pgfqpoint{1.873570in}{4.820572in}}{\pgfqpoint{1.873570in}{4.814748in}}%
\pgfpathcurveto{\pgfqpoint{1.873570in}{4.808924in}}{\pgfqpoint{1.875884in}{4.803338in}}{\pgfqpoint{1.880002in}{4.799220in}}%
\pgfpathcurveto{\pgfqpoint{1.884120in}{4.795102in}}{\pgfqpoint{1.889706in}{4.792788in}}{\pgfqpoint{1.895530in}{4.792788in}}%
\pgfpathlineto{\pgfqpoint{1.895530in}{4.792788in}}%
\pgfpathclose%
\pgfusepath{stroke,fill}%
\end{pgfscope}%
\begin{pgfscope}%
\pgfpathrectangle{\pgfqpoint{1.582361in}{0.880000in}}{\pgfqpoint{5.035278in}{6.160000in}}%
\pgfusepath{clip}%
\pgfsetbuttcap%
\pgfsetmiterjoin%
\pgfsetlinewidth{1.003750pt}%
\definecolor{currentstroke}{rgb}{0.800000,0.200000,0.200000}%
\pgfsetstrokecolor{currentstroke}%
\pgfsetdash{}{0pt}%
\pgfpathmoveto{\pgfqpoint{4.045013in}{3.882541in}}%
\pgfpathcurveto{\pgfqpoint{4.422131in}{3.882541in}}{\pgfqpoint{4.783854in}{4.032372in}}{\pgfqpoint{5.050516in}{4.299034in}}%
\pgfpathcurveto{\pgfqpoint{5.317179in}{4.565696in}}{\pgfqpoint{5.467009in}{4.927419in}}{\pgfqpoint{5.467009in}{5.304537in}}%
\pgfpathcurveto{\pgfqpoint{5.467009in}{5.681654in}}{\pgfqpoint{5.317179in}{6.043377in}}{\pgfqpoint{5.050516in}{6.310040in}}%
\pgfpathcurveto{\pgfqpoint{4.783854in}{6.576702in}}{\pgfqpoint{4.422131in}{6.726532in}}{\pgfqpoint{4.045013in}{6.726532in}}%
\pgfpathcurveto{\pgfqpoint{3.667896in}{6.726532in}}{\pgfqpoint{3.306173in}{6.576702in}}{\pgfqpoint{3.039511in}{6.310040in}}%
\pgfpathcurveto{\pgfqpoint{2.772848in}{6.043377in}}{\pgfqpoint{2.623018in}{5.681654in}}{\pgfqpoint{2.623018in}{5.304537in}}%
\pgfpathcurveto{\pgfqpoint{2.623018in}{4.927419in}}{\pgfqpoint{2.772848in}{4.565696in}}{\pgfqpoint{3.039511in}{4.299034in}}%
\pgfpathcurveto{\pgfqpoint{3.306173in}{4.032372in}}{\pgfqpoint{3.667896in}{3.882541in}}{\pgfqpoint{4.045013in}{3.882541in}}%
\pgfpathlineto{\pgfqpoint{4.045013in}{3.882541in}}%
\pgfpathclose%
\pgfusepath{stroke}%
\end{pgfscope}%
\begin{pgfscope}%
\pgfpathrectangle{\pgfqpoint{1.582361in}{0.880000in}}{\pgfqpoint{5.035278in}{6.160000in}}%
\pgfusepath{clip}%
\pgfsetbuttcap%
\pgfsetroundjoin%
\definecolor{currentfill}{rgb}{0.000000,0.000000,0.000000}%
\pgfsetfillcolor{currentfill}%
\pgfsetlinewidth{1.003750pt}%
\definecolor{currentstroke}{rgb}{0.000000,0.000000,0.000000}%
\pgfsetstrokecolor{currentstroke}%
\pgfsetdash{}{0pt}%
\pgfsys@defobject{currentmarker}{\pgfqpoint{-0.021960in}{-0.021960in}}{\pgfqpoint{0.021960in}{0.021960in}}{%
\pgfpathmoveto{\pgfqpoint{0.000000in}{-0.021960in}}%
\pgfpathcurveto{\pgfqpoint{0.005824in}{-0.021960in}}{\pgfqpoint{0.011410in}{-0.019646in}}{\pgfqpoint{0.015528in}{-0.015528in}}%
\pgfpathcurveto{\pgfqpoint{0.019646in}{-0.011410in}}{\pgfqpoint{0.021960in}{-0.005824in}}{\pgfqpoint{0.021960in}{0.000000in}}%
\pgfpathcurveto{\pgfqpoint{0.021960in}{0.005824in}}{\pgfqpoint{0.019646in}{0.011410in}}{\pgfqpoint{0.015528in}{0.015528in}}%
\pgfpathcurveto{\pgfqpoint{0.011410in}{0.019646in}}{\pgfqpoint{0.005824in}{0.021960in}}{\pgfqpoint{0.000000in}{0.021960in}}%
\pgfpathcurveto{\pgfqpoint{-0.005824in}{0.021960in}}{\pgfqpoint{-0.011410in}{0.019646in}}{\pgfqpoint{-0.015528in}{0.015528in}}%
\pgfpathcurveto{\pgfqpoint{-0.019646in}{0.011410in}}{\pgfqpoint{-0.021960in}{0.005824in}}{\pgfqpoint{-0.021960in}{0.000000in}}%
\pgfpathcurveto{\pgfqpoint{-0.021960in}{-0.005824in}}{\pgfqpoint{-0.019646in}{-0.011410in}}{\pgfqpoint{-0.015528in}{-0.015528in}}%
\pgfpathcurveto{\pgfqpoint{-0.011410in}{-0.019646in}}{\pgfqpoint{-0.005824in}{-0.021960in}}{\pgfqpoint{0.000000in}{-0.021960in}}%
\pgfpathlineto{\pgfqpoint{0.000000in}{-0.021960in}}%
\pgfpathclose%
\pgfusepath{stroke,fill}%
}%
\begin{pgfscope}%
\pgfsys@transformshift{4.045013in}{5.304537in}%
\pgfsys@useobject{currentmarker}{}%
\end{pgfscope}%
\end{pgfscope}%
\begin{pgfscope}%
\pgfsetbuttcap%
\pgfsetroundjoin%
\definecolor{currentfill}{rgb}{0.000000,0.000000,0.000000}%
\pgfsetfillcolor{currentfill}%
\pgfsetlinewidth{0.803000pt}%
\definecolor{currentstroke}{rgb}{0.000000,0.000000,0.000000}%
\pgfsetstrokecolor{currentstroke}%
\pgfsetdash{}{0pt}%
\pgfsys@defobject{currentmarker}{\pgfqpoint{0.000000in}{-0.048611in}}{\pgfqpoint{0.000000in}{0.000000in}}{%
\pgfpathmoveto{\pgfqpoint{0.000000in}{0.000000in}}%
\pgfpathlineto{\pgfqpoint{0.000000in}{-0.048611in}}%
\pgfusepath{stroke,fill}%
}%
\begin{pgfscope}%
\pgfsys@transformshift{1.760389in}{0.880000in}%
\pgfsys@useobject{currentmarker}{}%
\end{pgfscope}%
\end{pgfscope}%
\begin{pgfscope}%
\definecolor{textcolor}{rgb}{0.000000,0.000000,0.000000}%
\pgfsetstrokecolor{textcolor}%
\pgfsetfillcolor{textcolor}%
\pgftext[x=1.760389in,y=0.782778in,,top]{\color{textcolor}{\sffamily\fontsize{10.000000}{12.000000}\selectfont\catcode`\^=\active\def^{\ifmmode\sp\else\^{}\fi}\catcode`\%=\active\def%{\%}\ensuremath{-}600}}%
\end{pgfscope}%
\begin{pgfscope}%
\pgfsetbuttcap%
\pgfsetroundjoin%
\definecolor{currentfill}{rgb}{0.000000,0.000000,0.000000}%
\pgfsetfillcolor{currentfill}%
\pgfsetlinewidth{0.803000pt}%
\definecolor{currentstroke}{rgb}{0.000000,0.000000,0.000000}%
\pgfsetstrokecolor{currentstroke}%
\pgfsetdash{}{0pt}%
\pgfsys@defobject{currentmarker}{\pgfqpoint{0.000000in}{-0.048611in}}{\pgfqpoint{0.000000in}{0.000000in}}{%
\pgfpathmoveto{\pgfqpoint{0.000000in}{0.000000in}}%
\pgfpathlineto{\pgfqpoint{0.000000in}{-0.048611in}}%
\pgfusepath{stroke,fill}%
}%
\begin{pgfscope}%
\pgfsys@transformshift{2.534091in}{0.880000in}%
\pgfsys@useobject{currentmarker}{}%
\end{pgfscope}%
\end{pgfscope}%
\begin{pgfscope}%
\definecolor{textcolor}{rgb}{0.000000,0.000000,0.000000}%
\pgfsetstrokecolor{textcolor}%
\pgfsetfillcolor{textcolor}%
\pgftext[x=2.534091in,y=0.782778in,,top]{\color{textcolor}{\sffamily\fontsize{10.000000}{12.000000}\selectfont\catcode`\^=\active\def^{\ifmmode\sp\else\^{}\fi}\catcode`\%=\active\def%{\%}\ensuremath{-}400}}%
\end{pgfscope}%
\begin{pgfscope}%
\pgfsetbuttcap%
\pgfsetroundjoin%
\definecolor{currentfill}{rgb}{0.000000,0.000000,0.000000}%
\pgfsetfillcolor{currentfill}%
\pgfsetlinewidth{0.803000pt}%
\definecolor{currentstroke}{rgb}{0.000000,0.000000,0.000000}%
\pgfsetstrokecolor{currentstroke}%
\pgfsetdash{}{0pt}%
\pgfsys@defobject{currentmarker}{\pgfqpoint{0.000000in}{-0.048611in}}{\pgfqpoint{0.000000in}{0.000000in}}{%
\pgfpathmoveto{\pgfqpoint{0.000000in}{0.000000in}}%
\pgfpathlineto{\pgfqpoint{0.000000in}{-0.048611in}}%
\pgfusepath{stroke,fill}%
}%
\begin{pgfscope}%
\pgfsys@transformshift{3.307794in}{0.880000in}%
\pgfsys@useobject{currentmarker}{}%
\end{pgfscope}%
\end{pgfscope}%
\begin{pgfscope}%
\definecolor{textcolor}{rgb}{0.000000,0.000000,0.000000}%
\pgfsetstrokecolor{textcolor}%
\pgfsetfillcolor{textcolor}%
\pgftext[x=3.307794in,y=0.782778in,,top]{\color{textcolor}{\sffamily\fontsize{10.000000}{12.000000}\selectfont\catcode`\^=\active\def^{\ifmmode\sp\else\^{}\fi}\catcode`\%=\active\def%{\%}\ensuremath{-}200}}%
\end{pgfscope}%
\begin{pgfscope}%
\pgfsetbuttcap%
\pgfsetroundjoin%
\definecolor{currentfill}{rgb}{0.000000,0.000000,0.000000}%
\pgfsetfillcolor{currentfill}%
\pgfsetlinewidth{0.803000pt}%
\definecolor{currentstroke}{rgb}{0.000000,0.000000,0.000000}%
\pgfsetstrokecolor{currentstroke}%
\pgfsetdash{}{0pt}%
\pgfsys@defobject{currentmarker}{\pgfqpoint{0.000000in}{-0.048611in}}{\pgfqpoint{0.000000in}{0.000000in}}{%
\pgfpathmoveto{\pgfqpoint{0.000000in}{0.000000in}}%
\pgfpathlineto{\pgfqpoint{0.000000in}{-0.048611in}}%
\pgfusepath{stroke,fill}%
}%
\begin{pgfscope}%
\pgfsys@transformshift{4.081496in}{0.880000in}%
\pgfsys@useobject{currentmarker}{}%
\end{pgfscope}%
\end{pgfscope}%
\begin{pgfscope}%
\definecolor{textcolor}{rgb}{0.000000,0.000000,0.000000}%
\pgfsetstrokecolor{textcolor}%
\pgfsetfillcolor{textcolor}%
\pgftext[x=4.081496in,y=0.782778in,,top]{\color{textcolor}{\sffamily\fontsize{10.000000}{12.000000}\selectfont\catcode`\^=\active\def^{\ifmmode\sp\else\^{}\fi}\catcode`\%=\active\def%{\%}0}}%
\end{pgfscope}%
\begin{pgfscope}%
\pgfsetbuttcap%
\pgfsetroundjoin%
\definecolor{currentfill}{rgb}{0.000000,0.000000,0.000000}%
\pgfsetfillcolor{currentfill}%
\pgfsetlinewidth{0.803000pt}%
\definecolor{currentstroke}{rgb}{0.000000,0.000000,0.000000}%
\pgfsetstrokecolor{currentstroke}%
\pgfsetdash{}{0pt}%
\pgfsys@defobject{currentmarker}{\pgfqpoint{0.000000in}{-0.048611in}}{\pgfqpoint{0.000000in}{0.000000in}}{%
\pgfpathmoveto{\pgfqpoint{0.000000in}{0.000000in}}%
\pgfpathlineto{\pgfqpoint{0.000000in}{-0.048611in}}%
\pgfusepath{stroke,fill}%
}%
\begin{pgfscope}%
\pgfsys@transformshift{4.855199in}{0.880000in}%
\pgfsys@useobject{currentmarker}{}%
\end{pgfscope}%
\end{pgfscope}%
\begin{pgfscope}%
\definecolor{textcolor}{rgb}{0.000000,0.000000,0.000000}%
\pgfsetstrokecolor{textcolor}%
\pgfsetfillcolor{textcolor}%
\pgftext[x=4.855199in,y=0.782778in,,top]{\color{textcolor}{\sffamily\fontsize{10.000000}{12.000000}\selectfont\catcode`\^=\active\def^{\ifmmode\sp\else\^{}\fi}\catcode`\%=\active\def%{\%}200}}%
\end{pgfscope}%
\begin{pgfscope}%
\pgfsetbuttcap%
\pgfsetroundjoin%
\definecolor{currentfill}{rgb}{0.000000,0.000000,0.000000}%
\pgfsetfillcolor{currentfill}%
\pgfsetlinewidth{0.803000pt}%
\definecolor{currentstroke}{rgb}{0.000000,0.000000,0.000000}%
\pgfsetstrokecolor{currentstroke}%
\pgfsetdash{}{0pt}%
\pgfsys@defobject{currentmarker}{\pgfqpoint{0.000000in}{-0.048611in}}{\pgfqpoint{0.000000in}{0.000000in}}{%
\pgfpathmoveto{\pgfqpoint{0.000000in}{0.000000in}}%
\pgfpathlineto{\pgfqpoint{0.000000in}{-0.048611in}}%
\pgfusepath{stroke,fill}%
}%
\begin{pgfscope}%
\pgfsys@transformshift{5.628901in}{0.880000in}%
\pgfsys@useobject{currentmarker}{}%
\end{pgfscope}%
\end{pgfscope}%
\begin{pgfscope}%
\definecolor{textcolor}{rgb}{0.000000,0.000000,0.000000}%
\pgfsetstrokecolor{textcolor}%
\pgfsetfillcolor{textcolor}%
\pgftext[x=5.628901in,y=0.782778in,,top]{\color{textcolor}{\sffamily\fontsize{10.000000}{12.000000}\selectfont\catcode`\^=\active\def^{\ifmmode\sp\else\^{}\fi}\catcode`\%=\active\def%{\%}400}}%
\end{pgfscope}%
\begin{pgfscope}%
\pgfsetbuttcap%
\pgfsetroundjoin%
\definecolor{currentfill}{rgb}{0.000000,0.000000,0.000000}%
\pgfsetfillcolor{currentfill}%
\pgfsetlinewidth{0.803000pt}%
\definecolor{currentstroke}{rgb}{0.000000,0.000000,0.000000}%
\pgfsetstrokecolor{currentstroke}%
\pgfsetdash{}{0pt}%
\pgfsys@defobject{currentmarker}{\pgfqpoint{0.000000in}{-0.048611in}}{\pgfqpoint{0.000000in}{0.000000in}}{%
\pgfpathmoveto{\pgfqpoint{0.000000in}{0.000000in}}%
\pgfpathlineto{\pgfqpoint{0.000000in}{-0.048611in}}%
\pgfusepath{stroke,fill}%
}%
\begin{pgfscope}%
\pgfsys@transformshift{6.402604in}{0.880000in}%
\pgfsys@useobject{currentmarker}{}%
\end{pgfscope}%
\end{pgfscope}%
\begin{pgfscope}%
\definecolor{textcolor}{rgb}{0.000000,0.000000,0.000000}%
\pgfsetstrokecolor{textcolor}%
\pgfsetfillcolor{textcolor}%
\pgftext[x=6.402604in,y=0.782778in,,top]{\color{textcolor}{\sffamily\fontsize{10.000000}{12.000000}\selectfont\catcode`\^=\active\def^{\ifmmode\sp\else\^{}\fi}\catcode`\%=\active\def%{\%}600}}%
\end{pgfscope}%
\begin{pgfscope}%
\pgfsetbuttcap%
\pgfsetroundjoin%
\definecolor{currentfill}{rgb}{0.000000,0.000000,0.000000}%
\pgfsetfillcolor{currentfill}%
\pgfsetlinewidth{0.803000pt}%
\definecolor{currentstroke}{rgb}{0.000000,0.000000,0.000000}%
\pgfsetstrokecolor{currentstroke}%
\pgfsetdash{}{0pt}%
\pgfsys@defobject{currentmarker}{\pgfqpoint{-0.048611in}{0.000000in}}{\pgfqpoint{-0.000000in}{0.000000in}}{%
\pgfpathmoveto{\pgfqpoint{-0.000000in}{0.000000in}}%
\pgfpathlineto{\pgfqpoint{-0.048611in}{0.000000in}}%
\pgfusepath{stroke,fill}%
}%
\begin{pgfscope}%
\pgfsys@transformshift{1.582361in}{1.137466in}%
\pgfsys@useobject{currentmarker}{}%
\end{pgfscope}%
\end{pgfscope}%
\begin{pgfscope}%
\definecolor{textcolor}{rgb}{0.000000,0.000000,0.000000}%
\pgfsetstrokecolor{textcolor}%
\pgfsetfillcolor{textcolor}%
\pgftext[x=1.112018in, y=1.084705in, left, base]{\color{textcolor}{\sffamily\fontsize{10.000000}{12.000000}\selectfont\catcode`\^=\active\def^{\ifmmode\sp\else\^{}\fi}\catcode`\%=\active\def%{\%}\ensuremath{-}600}}%
\end{pgfscope}%
\begin{pgfscope}%
\pgfsetbuttcap%
\pgfsetroundjoin%
\definecolor{currentfill}{rgb}{0.000000,0.000000,0.000000}%
\pgfsetfillcolor{currentfill}%
\pgfsetlinewidth{0.803000pt}%
\definecolor{currentstroke}{rgb}{0.000000,0.000000,0.000000}%
\pgfsetstrokecolor{currentstroke}%
\pgfsetdash{}{0pt}%
\pgfsys@defobject{currentmarker}{\pgfqpoint{-0.048611in}{0.000000in}}{\pgfqpoint{-0.000000in}{0.000000in}}{%
\pgfpathmoveto{\pgfqpoint{-0.000000in}{0.000000in}}%
\pgfpathlineto{\pgfqpoint{-0.048611in}{0.000000in}}%
\pgfusepath{stroke,fill}%
}%
\begin{pgfscope}%
\pgfsys@transformshift{1.582361in}{1.911169in}%
\pgfsys@useobject{currentmarker}{}%
\end{pgfscope}%
\end{pgfscope}%
\begin{pgfscope}%
\definecolor{textcolor}{rgb}{0.000000,0.000000,0.000000}%
\pgfsetstrokecolor{textcolor}%
\pgfsetfillcolor{textcolor}%
\pgftext[x=1.112018in, y=1.858407in, left, base]{\color{textcolor}{\sffamily\fontsize{10.000000}{12.000000}\selectfont\catcode`\^=\active\def^{\ifmmode\sp\else\^{}\fi}\catcode`\%=\active\def%{\%}\ensuremath{-}400}}%
\end{pgfscope}%
\begin{pgfscope}%
\pgfsetbuttcap%
\pgfsetroundjoin%
\definecolor{currentfill}{rgb}{0.000000,0.000000,0.000000}%
\pgfsetfillcolor{currentfill}%
\pgfsetlinewidth{0.803000pt}%
\definecolor{currentstroke}{rgb}{0.000000,0.000000,0.000000}%
\pgfsetstrokecolor{currentstroke}%
\pgfsetdash{}{0pt}%
\pgfsys@defobject{currentmarker}{\pgfqpoint{-0.048611in}{0.000000in}}{\pgfqpoint{-0.000000in}{0.000000in}}{%
\pgfpathmoveto{\pgfqpoint{-0.000000in}{0.000000in}}%
\pgfpathlineto{\pgfqpoint{-0.048611in}{0.000000in}}%
\pgfusepath{stroke,fill}%
}%
\begin{pgfscope}%
\pgfsys@transformshift{1.582361in}{2.684871in}%
\pgfsys@useobject{currentmarker}{}%
\end{pgfscope}%
\end{pgfscope}%
\begin{pgfscope}%
\definecolor{textcolor}{rgb}{0.000000,0.000000,0.000000}%
\pgfsetstrokecolor{textcolor}%
\pgfsetfillcolor{textcolor}%
\pgftext[x=1.112018in, y=2.632110in, left, base]{\color{textcolor}{\sffamily\fontsize{10.000000}{12.000000}\selectfont\catcode`\^=\active\def^{\ifmmode\sp\else\^{}\fi}\catcode`\%=\active\def%{\%}\ensuremath{-}200}}%
\end{pgfscope}%
\begin{pgfscope}%
\pgfsetbuttcap%
\pgfsetroundjoin%
\definecolor{currentfill}{rgb}{0.000000,0.000000,0.000000}%
\pgfsetfillcolor{currentfill}%
\pgfsetlinewidth{0.803000pt}%
\definecolor{currentstroke}{rgb}{0.000000,0.000000,0.000000}%
\pgfsetstrokecolor{currentstroke}%
\pgfsetdash{}{0pt}%
\pgfsys@defobject{currentmarker}{\pgfqpoint{-0.048611in}{0.000000in}}{\pgfqpoint{-0.000000in}{0.000000in}}{%
\pgfpathmoveto{\pgfqpoint{-0.000000in}{0.000000in}}%
\pgfpathlineto{\pgfqpoint{-0.048611in}{0.000000in}}%
\pgfusepath{stroke,fill}%
}%
\begin{pgfscope}%
\pgfsys@transformshift{1.582361in}{3.458574in}%
\pgfsys@useobject{currentmarker}{}%
\end{pgfscope}%
\end{pgfscope}%
\begin{pgfscope}%
\definecolor{textcolor}{rgb}{0.000000,0.000000,0.000000}%
\pgfsetstrokecolor{textcolor}%
\pgfsetfillcolor{textcolor}%
\pgftext[x=1.396773in, y=3.405812in, left, base]{\color{textcolor}{\sffamily\fontsize{10.000000}{12.000000}\selectfont\catcode`\^=\active\def^{\ifmmode\sp\else\^{}\fi}\catcode`\%=\active\def%{\%}0}}%
\end{pgfscope}%
\begin{pgfscope}%
\pgfsetbuttcap%
\pgfsetroundjoin%
\definecolor{currentfill}{rgb}{0.000000,0.000000,0.000000}%
\pgfsetfillcolor{currentfill}%
\pgfsetlinewidth{0.803000pt}%
\definecolor{currentstroke}{rgb}{0.000000,0.000000,0.000000}%
\pgfsetstrokecolor{currentstroke}%
\pgfsetdash{}{0pt}%
\pgfsys@defobject{currentmarker}{\pgfqpoint{-0.048611in}{0.000000in}}{\pgfqpoint{-0.000000in}{0.000000in}}{%
\pgfpathmoveto{\pgfqpoint{-0.000000in}{0.000000in}}%
\pgfpathlineto{\pgfqpoint{-0.048611in}{0.000000in}}%
\pgfusepath{stroke,fill}%
}%
\begin{pgfscope}%
\pgfsys@transformshift{1.582361in}{4.232276in}%
\pgfsys@useobject{currentmarker}{}%
\end{pgfscope}%
\end{pgfscope}%
\begin{pgfscope}%
\definecolor{textcolor}{rgb}{0.000000,0.000000,0.000000}%
\pgfsetstrokecolor{textcolor}%
\pgfsetfillcolor{textcolor}%
\pgftext[x=1.220043in, y=4.179515in, left, base]{\color{textcolor}{\sffamily\fontsize{10.000000}{12.000000}\selectfont\catcode`\^=\active\def^{\ifmmode\sp\else\^{}\fi}\catcode`\%=\active\def%{\%}200}}%
\end{pgfscope}%
\begin{pgfscope}%
\pgfsetbuttcap%
\pgfsetroundjoin%
\definecolor{currentfill}{rgb}{0.000000,0.000000,0.000000}%
\pgfsetfillcolor{currentfill}%
\pgfsetlinewidth{0.803000pt}%
\definecolor{currentstroke}{rgb}{0.000000,0.000000,0.000000}%
\pgfsetstrokecolor{currentstroke}%
\pgfsetdash{}{0pt}%
\pgfsys@defobject{currentmarker}{\pgfqpoint{-0.048611in}{0.000000in}}{\pgfqpoint{-0.000000in}{0.000000in}}{%
\pgfpathmoveto{\pgfqpoint{-0.000000in}{0.000000in}}%
\pgfpathlineto{\pgfqpoint{-0.048611in}{0.000000in}}%
\pgfusepath{stroke,fill}%
}%
\begin{pgfscope}%
\pgfsys@transformshift{1.582361in}{5.005979in}%
\pgfsys@useobject{currentmarker}{}%
\end{pgfscope}%
\end{pgfscope}%
\begin{pgfscope}%
\definecolor{textcolor}{rgb}{0.000000,0.000000,0.000000}%
\pgfsetstrokecolor{textcolor}%
\pgfsetfillcolor{textcolor}%
\pgftext[x=1.220043in, y=4.953217in, left, base]{\color{textcolor}{\sffamily\fontsize{10.000000}{12.000000}\selectfont\catcode`\^=\active\def^{\ifmmode\sp\else\^{}\fi}\catcode`\%=\active\def%{\%}400}}%
\end{pgfscope}%
\begin{pgfscope}%
\pgfsetbuttcap%
\pgfsetroundjoin%
\definecolor{currentfill}{rgb}{0.000000,0.000000,0.000000}%
\pgfsetfillcolor{currentfill}%
\pgfsetlinewidth{0.803000pt}%
\definecolor{currentstroke}{rgb}{0.000000,0.000000,0.000000}%
\pgfsetstrokecolor{currentstroke}%
\pgfsetdash{}{0pt}%
\pgfsys@defobject{currentmarker}{\pgfqpoint{-0.048611in}{0.000000in}}{\pgfqpoint{-0.000000in}{0.000000in}}{%
\pgfpathmoveto{\pgfqpoint{-0.000000in}{0.000000in}}%
\pgfpathlineto{\pgfqpoint{-0.048611in}{0.000000in}}%
\pgfusepath{stroke,fill}%
}%
\begin{pgfscope}%
\pgfsys@transformshift{1.582361in}{5.779682in}%
\pgfsys@useobject{currentmarker}{}%
\end{pgfscope}%
\end{pgfscope}%
\begin{pgfscope}%
\definecolor{textcolor}{rgb}{0.000000,0.000000,0.000000}%
\pgfsetstrokecolor{textcolor}%
\pgfsetfillcolor{textcolor}%
\pgftext[x=1.220043in, y=5.726920in, left, base]{\color{textcolor}{\sffamily\fontsize{10.000000}{12.000000}\selectfont\catcode`\^=\active\def^{\ifmmode\sp\else\^{}\fi}\catcode`\%=\active\def%{\%}600}}%
\end{pgfscope}%
\begin{pgfscope}%
\pgfsetbuttcap%
\pgfsetroundjoin%
\definecolor{currentfill}{rgb}{0.000000,0.000000,0.000000}%
\pgfsetfillcolor{currentfill}%
\pgfsetlinewidth{0.803000pt}%
\definecolor{currentstroke}{rgb}{0.000000,0.000000,0.000000}%
\pgfsetstrokecolor{currentstroke}%
\pgfsetdash{}{0pt}%
\pgfsys@defobject{currentmarker}{\pgfqpoint{-0.048611in}{0.000000in}}{\pgfqpoint{-0.000000in}{0.000000in}}{%
\pgfpathmoveto{\pgfqpoint{-0.000000in}{0.000000in}}%
\pgfpathlineto{\pgfqpoint{-0.048611in}{0.000000in}}%
\pgfusepath{stroke,fill}%
}%
\begin{pgfscope}%
\pgfsys@transformshift{1.582361in}{6.553384in}%
\pgfsys@useobject{currentmarker}{}%
\end{pgfscope}%
\end{pgfscope}%
\begin{pgfscope}%
\definecolor{textcolor}{rgb}{0.000000,0.000000,0.000000}%
\pgfsetstrokecolor{textcolor}%
\pgfsetfillcolor{textcolor}%
\pgftext[x=1.220043in, y=6.500623in, left, base]{\color{textcolor}{\sffamily\fontsize{10.000000}{12.000000}\selectfont\catcode`\^=\active\def^{\ifmmode\sp\else\^{}\fi}\catcode`\%=\active\def%{\%}800}}%
\end{pgfscope}%
\begin{pgfscope}%
\pgfsetrectcap%
\pgfsetmiterjoin%
\pgfsetlinewidth{0.803000pt}%
\definecolor{currentstroke}{rgb}{0.000000,0.000000,0.000000}%
\pgfsetstrokecolor{currentstroke}%
\pgfsetdash{}{0pt}%
\pgfpathmoveto{\pgfqpoint{1.582361in}{0.880000in}}%
\pgfpathlineto{\pgfqpoint{1.582361in}{7.040000in}}%
\pgfusepath{stroke}%
\end{pgfscope}%
\begin{pgfscope}%
\pgfsetrectcap%
\pgfsetmiterjoin%
\pgfsetlinewidth{0.803000pt}%
\definecolor{currentstroke}{rgb}{0.000000,0.000000,0.000000}%
\pgfsetstrokecolor{currentstroke}%
\pgfsetdash{}{0pt}%
\pgfpathmoveto{\pgfqpoint{6.617639in}{0.880000in}}%
\pgfpathlineto{\pgfqpoint{6.617639in}{7.040000in}}%
\pgfusepath{stroke}%
\end{pgfscope}%
\begin{pgfscope}%
\pgfsetrectcap%
\pgfsetmiterjoin%
\pgfsetlinewidth{0.803000pt}%
\definecolor{currentstroke}{rgb}{0.000000,0.000000,0.000000}%
\pgfsetstrokecolor{currentstroke}%
\pgfsetdash{}{0pt}%
\pgfpathmoveto{\pgfqpoint{1.582361in}{0.880000in}}%
\pgfpathlineto{\pgfqpoint{6.617639in}{0.880000in}}%
\pgfusepath{stroke}%
\end{pgfscope}%
\begin{pgfscope}%
\pgfsetrectcap%
\pgfsetmiterjoin%
\pgfsetlinewidth{0.803000pt}%
\definecolor{currentstroke}{rgb}{0.000000,0.000000,0.000000}%
\pgfsetstrokecolor{currentstroke}%
\pgfsetdash{}{0pt}%
\pgfpathmoveto{\pgfqpoint{1.582361in}{7.040000in}}%
\pgfpathlineto{\pgfqpoint{6.617639in}{7.040000in}}%
\pgfusepath{stroke}%
\end{pgfscope}%
\end{pgfpicture}%
\makeatother%
\endgroup%
}
    \label{fig:needle}
    \caption{Example of a dataset with a ring in a lot of noise, and background noise, and a good classification.}
\end{figure}

Another task we were interested about was the algorithm ability to, in a dataset with a lot of noise, find a ring. We used the same generation method as the general excentric test,
but with 200 non-noisy samples for the ring and 100 and 200 background noise samples. We did not try more noise samples because then the ring would be almost irrecognizable.
The results can be seen in the following table



\bibliographystyle{IEEEtran}
\bibliography{references.bib}

\vspace{12pt}
\color{red}
\end{document}
